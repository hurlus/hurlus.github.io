%%%%%%%%%%%%%%%%%%%%%%%%%%%%%%%%%
% LaTeX model https://hurlus.fr %
%%%%%%%%%%%%%%%%%%%%%%%%%%%%%%%%%

% Needed before document class
\RequirePackage{pdftexcmds} % needed for tests expressions
\RequirePackage{fix-cm} % correct units

% Define mode
\def\mode{a4}

\newif\ifaiv % a4
\newif\ifav % a5
\newif\ifbooklet % booklet
\newif\ifcover % cover for booklet

\ifnum \strcmp{\mode}{cover}=0
  \covertrue
\else\ifnum \strcmp{\mode}{booklet}=0
  \booklettrue
\else\ifnum \strcmp{\mode}{a5}=0
  \avtrue
\else
  \aivtrue
\fi\fi\fi

\ifbooklet % do not enclose with {}
  \documentclass[french,twoside]{book} % ,notitlepage
  \usepackage[%
    papersize={105mm, 297mm},
    inner=12mm,
    outer=12mm,
    top=20mm,
    bottom=15mm,
    marginparsep=0pt,
  ]{geometry}
  \usepackage[fontsize=9.5pt]{scrextend} % for Roboto
\else\ifav
  \documentclass[french,twoside]{book} % ,notitlepage
  \usepackage[%
    a5paper,
    inner=25mm,
    outer=15mm,
    top=15mm,
    bottom=15mm,
    marginparsep=0pt,
  ]{geometry}
  \usepackage[fontsize=12pt]{scrextend}
\else% A4 2 cols
  \documentclass[twocolumn]{report}
  \usepackage[%
    a4paper,
    inner=15mm,
    outer=10mm,
    top=25mm,
    bottom=18mm,
    marginparsep=0pt,
  ]{geometry}
  \setlength{\columnsep}{20mm}
  \usepackage[fontsize=9.5pt]{scrextend}
\fi\fi

%%%%%%%%%%%%%%
% Alignments %
%%%%%%%%%%%%%%
% before teinte macros

\setlength{\arrayrulewidth}{0.2pt}
\setlength{\columnseprule}{\arrayrulewidth} % twocol
\setlength{\parskip}{0pt} % classical para with no margin
\setlength{\parindent}{1.5em}

%%%%%%%%%%
% Colors %
%%%%%%%%%%
% before Teinte macros

\usepackage[dvipsnames]{xcolor}
\definecolor{rubric}{HTML}{800000} % the tonic 0c71c3
\def\columnseprulecolor{\color{rubric}}
\colorlet{borderline}{rubric!30!} % definecolor need exact code
\definecolor{shadecolor}{gray}{0.95}
\definecolor{bghi}{gray}{0.5}

%%%%%%%%%%%%%%%%%
% Teinte macros %
%%%%%%%%%%%%%%%%%
%%%%%%%%%%%%%%%%%%%%%%%%%%%%%%%%%%%%%%%%%%%%%%%%%%%
% <TEI> generic (LaTeX names generated by Teinte) %
%%%%%%%%%%%%%%%%%%%%%%%%%%%%%%%%%%%%%%%%%%%%%%%%%%%
% This template is inserted in a specific design
% It is XeLaTeX and otf fonts

\makeatletter % <@@@


\usepackage{blindtext} % generate text for testing
\usepackage[strict]{changepage} % for modulo 4
\usepackage{contour} % rounding words
\usepackage[nodayofweek]{datetime}
% \usepackage{DejaVuSans} % seems buggy for sffont font for symbols
\usepackage{enumitem} % <list>
\usepackage{etoolbox} % patch commands
\usepackage{fancyvrb}
\usepackage{fancyhdr}
\usepackage{float}
\usepackage{fontspec} % XeLaTeX mandatory for fonts
\usepackage{footnote} % used to capture notes in minipage (ex: quote)
\usepackage{framed} % bordering correct with footnote hack
\usepackage{graphicx}
\usepackage{lettrine} % drop caps
\usepackage{lipsum} % generate text for testing
\usepackage[framemethod=tikz,]{mdframed} % maybe used for frame with footnotes inside
\usepackage{pdftexcmds} % needed for tests expressions
\usepackage{polyglossia} % non-break space french punct, bug Warning: "Failed to patch part"
\usepackage[%
  indentfirst=false,
  vskip=1em,
  noorphanfirst=true,
  noorphanafter=true,
  leftmargin=\parindent,
  rightmargin=0pt,
]{quoting}
\usepackage{ragged2e}
\usepackage{setspace} % \setstretch for <quote>
\usepackage{tabularx} % <table>
\usepackage[explicit]{titlesec} % wear titles, !NO implicit
\usepackage{tikz} % ornaments
\usepackage{tocloft} % styling tocs
\usepackage[fit]{truncate} % used im runing titles
\usepackage{unicode-math}
\usepackage[normalem]{ulem} % breakable \uline, normalem is absolutely necessary to keep \emph
\usepackage{verse} % <l>
\usepackage{xcolor} % named colors
\usepackage{xparse} % @ifundefined
\XeTeXdefaultencoding "iso-8859-1" % bad encoding of xstring
\usepackage{xstring} % string tests
\XeTeXdefaultencoding "utf-8"
\PassOptionsToPackage{hyphens}{url} % before hyperref, which load url package

% TOTEST
% \usepackage{hypcap} % links in caption ?
% \usepackage{marginnote}
% TESTED
% \usepackage{background} % doesn’t work with xetek
% \usepackage{bookmark} % prefers the hyperref hack \phantomsection
% \usepackage[color, leftbars]{changebar} % 2 cols doc, impossible to keep bar left
% \usepackage[utf8x]{inputenc} % inputenc package ignored with utf8 based engines
% \usepackage[sfdefault,medium]{inter} % no small caps
% \usepackage{firamath} % choose firasans instead, firamath unavailable in Ubuntu 21-04
% \usepackage{flushend} % bad for last notes, supposed flush end of columns
% \usepackage[stable]{footmisc} % BAD for complex notes https://texfaq.org/FAQ-ftnsect
% \usepackage{helvet} % not for XeLaTeX
% \usepackage{multicol} % not compatible with too much packages (longtable, framed, memoir…)
% \usepackage[default,oldstyle,scale=0.95]{opensans} % no small caps
% \usepackage{sectsty} % \chapterfont OBSOLETE
% \usepackage{soul} % \ul for underline, OBSOLETE with XeTeX
% \usepackage[breakable]{tcolorbox} % text styling gone, footnote hack not kept with breakable


% Metadata inserted by a program, from the TEI source, for title page and runing heads
\title{\textbf{ Cours de philosophie positive : première et deuxième leçons }}
\date{1830}
\author{Auguste Comte}
\def\elbibl{Auguste Comte. 1830. \emph{Cours de philosophie positive : première et deuxième leçons}}
\def\elsource{Auguste Comte, \emph{{\itshape Cours de philosophie positive}}, 1\textsuperscript{re} et 2\textsuperscript{e} leçons, 5\textsuperscript{e} éd., Paris, Société positiviste, 1894.}

% Default metas
\newcommand{\colorprovide}[2]{\@ifundefinedcolor{#1}{\colorlet{#1}{#2}}{}}
\colorprovide{rubric}{red}
\colorprovide{silver}{lightgray}
\@ifundefined{syms}{\newfontfamily\syms{DejaVu Sans}}{}
\newif\ifdev
\@ifundefined{elbibl}{% No meta defined, maybe dev mode
  \newcommand{\elbibl}{Titre court ?}
  \newcommand{\elbook}{Titre du livre source ?}
  \newcommand{\elabstract}{Résumé\par}
  \newcommand{\elurl}{http://oeuvres.github.io/elbook/2}
  \author{Éric Lœchien}
  \title{Un titre de test assez long pour vérifier le comportement d’une maquette}
  \date{1566}
  \devtrue
}{}
\let\eltitle\@title
\let\elauthor\@author
\let\eldate\@date


\defaultfontfeatures{
  % Mapping=tex-text, % no effect seen
  Scale=MatchLowercase,
  Ligatures={TeX,Common},
}


% generic typo commands
\newcommand{\astermono}{\medskip\centerline{\color{rubric}\large\selectfont{\syms ✻}}\medskip\par}%
\newcommand{\astertri}{\medskip\par\centerline{\color{rubric}\large\selectfont{\syms ✻\,✻\,✻}}\medskip\par}%
\newcommand{\asterism}{\bigskip\par\noindent\parbox{\linewidth}{\centering\color{rubric}\large{\syms ✻}\\{\syms ✻}\hskip 0.75em{\syms ✻}}\bigskip\par}%

% lists
\newlength{\listmod}
\setlength{\listmod}{\parindent}
\setlist{
  itemindent=!,
  listparindent=\listmod,
  labelsep=0.2\listmod,
  parsep=0pt,
  % topsep=0.2em, % default topsep is best
}
\setlist[itemize]{
  label=—,
  leftmargin=0pt,
  labelindent=1.2em,
  labelwidth=0pt,
}
\setlist[enumerate]{
  label={\bf\color{rubric}\arabic*.},
  labelindent=0.8\listmod,
  leftmargin=\listmod,
  labelwidth=0pt,
}
\newlist{listalpha}{enumerate}{1}
\setlist[listalpha]{
  label={\bf\color{rubric}\alph*.},
  leftmargin=0pt,
  labelindent=0.8\listmod,
  labelwidth=0pt,
}
\newcommand{\listhead}[1]{\hspace{-1\listmod}\emph{#1}}

\renewcommand{\hrulefill}{%
  \leavevmode\leaders\hrule height 0.2pt\hfill\kern\z@}

% General typo
\DeclareTextFontCommand{\textlarge}{\large}
\DeclareTextFontCommand{\textsmall}{\small}

% commands, inlines
\newcommand{\anchor}[1]{\Hy@raisedlink{\hypertarget{#1}{}}} % link to top of an anchor (not baseline)
\newcommand\abbr[1]{#1}
\newcommand{\autour}[1]{\tikz[baseline=(X.base)]\node [draw=rubric,thin,rectangle,inner sep=1.5pt, rounded corners=3pt] (X) {\color{rubric}#1};}
\newcommand\corr[1]{#1}
\newcommand{\ed}[1]{ {\color{silver}\sffamily\footnotesize (#1)} } % <milestone ed="1688"/>
\newcommand\expan[1]{#1}
\newcommand\foreign[1]{\emph{#1}}
\newcommand\gap[1]{#1}
\renewcommand{\LettrineFontHook}{\color{rubric}}
\newcommand{\initial}[2]{\lettrine[lines=2, loversize=0.3, lhang=0.3]{#1}{#2}}
\newcommand{\initialiv}[2]{%
  \let\oldLFH\LettrineFontHook
  % \renewcommand{\LettrineFontHook}{\color{rubric}\ttfamily}
  \IfSubStr{QJ’}{#1}{
    \lettrine[lines=4, lhang=0.2, loversize=-0.1, lraise=0.2]{\smash{#1}}{#2}
  }{\IfSubStr{É}{#1}{
    \lettrine[lines=4, lhang=0.2, loversize=-0, lraise=0]{\smash{#1}}{#2}
  }{\IfSubStr{ÀÂ}{#1}{
    \lettrine[lines=4, lhang=0.2, loversize=-0, lraise=0, slope=0.6em]{\smash{#1}}{#2}
  }{\IfSubStr{A}{#1}{
    \lettrine[lines=4, lhang=0.2, loversize=0.2, slope=0.6em]{\smash{#1}}{#2}
  }{\IfSubStr{V}{#1}{
    \lettrine[lines=4, lhang=0.2, loversize=0.2, slope=-0.5em]{\smash{#1}}{#2}
  }{
    \lettrine[lines=4, lhang=0.2, loversize=0.2]{\smash{#1}}{#2}
  }}}}}
  \let\LettrineFontHook\oldLFH
}
\newcommand{\labelchar}[1]{\textbf{\color{rubric} #1}}
\newcommand{\milestone}[1]{\autour{\footnotesize\color{rubric} #1}} % <milestone n="4"/>
\newcommand\name[1]{#1}
\newcommand\orig[1]{#1}
\newcommand\orgName[1]{#1}
\newcommand\persName[1]{#1}
\newcommand\placeName[1]{#1}
\newcommand{\pn}[1]{\IfSubStr{-—–¶}{#1}% <p n="3"/>
  {\noindent{\bfseries\color{rubric}   ¶  }}
  {{\footnotesize\autour{ #1}  }}}
\newcommand\reg{}
% \newcommand\ref{} % already defined
\newcommand\sic[1]{#1}
\newcommand\surname[1]{\textsc{#1}}
\newcommand\term[1]{\textbf{#1}}

\def\mednobreak{\ifdim\lastskip<\medskipamount
  \removelastskip\nopagebreak\medskip\fi}
\def\bignobreak{\ifdim\lastskip<\bigskipamount
  \removelastskip\nopagebreak\bigskip\fi}

% commands, blocks
\newcommand{\byline}[1]{\bigskip{\RaggedLeft{#1}\par}\bigskip}
\newcommand{\bibl}[1]{{\RaggedLeft{#1}\par\bigskip}}
\newcommand{\biblitem}[1]{{\noindent\hangindent=\parindent   #1\par}}
\newcommand{\dateline}[1]{\medskip{\RaggedLeft{#1}\par}\bigskip}
\newcommand{\labelblock}[1]{\medbreak{\noindent\color{rubric}\bfseries #1}\par\mednobreak}
\newcommand{\salute}[1]{\bigbreak{#1}\par\medbreak}
\newcommand{\signed}[1]{\bigbreak\filbreak{\raggedleft #1\par}\medskip}

% environments for blocks (some may become commands)
\newenvironment{borderbox}{}{} % framing content
\newenvironment{citbibl}{\ifvmode\hfill\fi}{\ifvmode\par\fi }
\newenvironment{docAuthor}{\ifvmode\vskip4pt\fontsize{16pt}{18pt}\selectfont\fi\itshape}{\ifvmode\par\fi }
\newenvironment{docDate}{}{\ifvmode\par\fi }
\newenvironment{docImprint}{\vskip6pt}{\ifvmode\par\fi }
\newenvironment{docTitle}{\vskip6pt\bfseries\fontsize{18pt}{22pt}\selectfont}{\par }
\newenvironment{msHead}{\vskip6pt}{\par}
\newenvironment{msItem}{\vskip6pt}{\par}
\newenvironment{titlePart}{}{\par }


% environments for block containers
\newenvironment{argument}{\itshape\parindent0pt}{\vskip1.5em}
\newenvironment{biblfree}{}{\ifvmode\par\fi }
\newenvironment{bibitemlist}[1]{%
  \list{\@biblabel{\@arabic\c@enumiv}}%
  {%
    \settowidth\labelwidth{\@biblabel{#1}}%
    \leftmargin\labelwidth
    \advance\leftmargin\labelsep
    \@openbib@code
    \usecounter{enumiv}%
    \let\p@enumiv\@empty
    \renewcommand\theenumiv{\@arabic\c@enumiv}%
  }
  \sloppy
  \clubpenalty4000
  \@clubpenalty \clubpenalty
  \widowpenalty4000%
  \sfcode`\.\@m
}%
{\def\@noitemerr
  {\@latex@warning{Empty `bibitemlist' environment}}%
\endlist}
\newenvironment{quoteblock}% may be used for ornaments
  {\begin{quoting}}
  {\end{quoting}}

% table () is preceded and finished by custom command
\newcommand{\tableopen}[1]{%
  \ifnum\strcmp{#1}{wide}=0{%
    \begin{center}
  }
  \else\ifnum\strcmp{#1}{long}=0{%
    \begin{center}
  }
  \else{%
    \begin{center}
  }
  \fi\fi
}
\newcommand{\tableclose}[1]{%
  \ifnum\strcmp{#1}{wide}=0{%
    \end{center}
  }
  \else\ifnum\strcmp{#1}{long}=0{%
    \end{center}
  }
  \else{%
    \end{center}
  }
  \fi\fi
}


% text structure
\newcommand\chapteropen{} % before chapter title
\newcommand\chaptercont{} % after title, argument, epigraph…
\newcommand\chapterclose{} % maybe useful for multicol settings
\setcounter{secnumdepth}{-2} % no counters for hierarchy titles
\setcounter{tocdepth}{5} % deep toc
\markright{\@title} % ???
\markboth{\@title}{\@author} % ???
\renewcommand\tableofcontents{\@starttoc{toc}}
% toclof format
% \renewcommand{\@tocrmarg}{0.1em} % Useless command?
% \renewcommand{\@pnumwidth}{0.5em} % {1.75em}
\renewcommand{\@cftmaketoctitle}{}
\setlength{\cftbeforesecskip}{\z@ \@plus.2\p@}
\renewcommand{\cftchapfont}{}
\renewcommand{\cftchapdotsep}{\cftdotsep}
\renewcommand{\cftchapleader}{\normalfont\cftdotfill{\cftchapdotsep}}
\renewcommand{\cftchappagefont}{\bfseries}
\setlength{\cftbeforechapskip}{0em \@plus\p@}
% \renewcommand{\cftsecfont}{\small\relax}
\renewcommand{\cftsecpagefont}{\normalfont}
% \renewcommand{\cftsubsecfont}{\small\relax}
\renewcommand{\cftsecdotsep}{\cftdotsep}
\renewcommand{\cftsecpagefont}{\normalfont}
\renewcommand{\cftsecleader}{\normalfont\cftdotfill{\cftsecdotsep}}
\setlength{\cftsecindent}{1em}
\setlength{\cftsubsecindent}{2em}
\setlength{\cftsubsubsecindent}{3em}
\setlength{\cftchapnumwidth}{1em}
\setlength{\cftsecnumwidth}{1em}
\setlength{\cftsubsecnumwidth}{1em}
\setlength{\cftsubsubsecnumwidth}{1em}

% footnotes
\newif\ifheading
\newcommand*{\fnmarkscale}{\ifheading 0.70 \else 1 \fi}
\renewcommand\footnoterule{\vspace*{0.3cm}\hrule height \arrayrulewidth width 3cm \vspace*{0.3cm}}
\setlength\footnotesep{1.5\footnotesep} % footnote separator
\renewcommand\@makefntext[1]{\parindent 1.5em \noindent \hb@xt@1.8em{\hss{\normalfont\@thefnmark . }}#1} % no superscipt in foot
\patchcmd{\@footnotetext}{\footnotesize}{\footnotesize\sffamily}{}{} % before scrextend, hyperref


%   see https://tex.stackexchange.com/a/34449/5049
\def\truncdiv#1#2{((#1-(#2-1)/2)/#2)}
\def\moduloop#1#2{(#1-\truncdiv{#1}{#2}*#2)}
\def\modulo#1#2{\number\numexpr\moduloop{#1}{#2}\relax}

% orphans and widows
\clubpenalty=9996
\widowpenalty=9999
\brokenpenalty=4991
\predisplaypenalty=10000
\postdisplaypenalty=1549
\displaywidowpenalty=1602
\hyphenpenalty=400
% Copied from Rahtz but not understood
\def\@pnumwidth{1.55em}
\def\@tocrmarg {2.55em}
\def\@dotsep{4.5}
\emergencystretch 3em
\hbadness=4000
\pretolerance=750
\tolerance=2000
\vbadness=4000
\def\Gin@extensions{.pdf,.png,.jpg,.mps,.tif}
% \renewcommand{\@cite}[1]{#1} % biblio

\usepackage{hyperref} % supposed to be the last one, :o) except for the ones to follow
\urlstyle{same} % after hyperref
\hypersetup{
  % pdftex, % no effect
  pdftitle={\elbibl},
  % pdfauthor={Your name here},
  % pdfsubject={Your subject here},
  % pdfkeywords={keyword1, keyword2},
  bookmarksnumbered=true,
  bookmarksopen=true,
  bookmarksopenlevel=1,
  pdfstartview=Fit,
  breaklinks=true, % avoid long links
  pdfpagemode=UseOutlines,    % pdf toc
  hyperfootnotes=true,
  colorlinks=false,
  pdfborder=0 0 0,
  % pdfpagelayout=TwoPageRight,
  % linktocpage=true, % NO, toc, link only on page no
}

\makeatother % /@@@>
%%%%%%%%%%%%%%
% </TEI> end %
%%%%%%%%%%%%%%


%%%%%%%%%%%%%
% footnotes %
%%%%%%%%%%%%%
\renewcommand{\thefootnote}{\bfseries\textcolor{rubric}{\arabic{footnote}}} % color for footnote marks

%%%%%%%%%
% Fonts %
%%%%%%%%%
\usepackage[]{roboto} % SmallCaps, Regular is a bit bold
% \linespread{0.90} % too compact, keep font natural
\newfontfamily\fontrun[]{Roboto Condensed Light} % condensed runing heads
\ifav
  \setmainfont[
    ItalicFont={Roboto Light Italic},
  ]{Roboto}
\else\ifbooklet
  \setmainfont[
    ItalicFont={Roboto Light Italic},
  ]{Roboto}
\else
\setmainfont[
  ItalicFont={Roboto Italic},
]{Roboto Light}
\fi\fi
\renewcommand{\LettrineFontHook}{\bfseries\color{rubric}}
% \renewenvironment{labelblock}{\begin{center}\bfseries\color{rubric}}{\end{center}}

%%%%%%%%
% MISC %
%%%%%%%%

\setdefaultlanguage[frenchpart=false]{french} % bug on part


\newenvironment{quotebar}{%
    \def\FrameCommand{{\color{rubric!10!}\vrule width 0.5em} \hspace{0.9em}}%
    \def\OuterFrameSep{\itemsep} % séparateur vertical
    \MakeFramed {\advance\hsize-\width \FrameRestore}
  }%
  {%
    \endMakeFramed
  }
\renewenvironment{quoteblock}% may be used for ornaments
  {%
    \savenotes
    \setstretch{0.9}
    \normalfont
    \begin{quotebar}
  }
  {%
    \end{quotebar}
    \spewnotes
  }


\renewcommand{\headrulewidth}{\arrayrulewidth}
\renewcommand{\headrule}{{\color{rubric}\hrule}}

% delicate tuning, image has produce line-height problems in title on 2 lines
\titleformat{name=\chapter} % command
  [display] % shape
  {\vspace{1.5em}\centering} % format
  {} % label
  {0pt} % separator between n
  {}
[{\color{rubric}\huge\textbf{#1}}\bigskip] % after code
% \titlespacing{command}{left spacing}{before spacing}{after spacing}[right]
\titlespacing*{\chapter}{0pt}{-2em}{0pt}[0pt]

\titleformat{name=\section}
  [block]{}{}{}{}
  [\vbox{\color{rubric}\large\raggedleft\textbf{#1}}]
\titlespacing{\section}{0pt}{0pt plus 4pt minus 2pt}{\baselineskip}

\titleformat{name=\subsection}
  [block]
  {}
  {} % \thesection
  {} % separator \arrayrulewidth
  {}
[\vbox{\large\textbf{#1}}]
% \titlespacing{\subsection}{0pt}{0pt plus 4pt minus 2pt}{\baselineskip}

\ifaiv
  \fancypagestyle{main}{%
    \fancyhf{}
    \setlength{\headheight}{1.5em}
    \fancyhead{} % reset head
    \fancyfoot{} % reset foot
    \fancyhead[L]{\truncate{0.45\headwidth}{\fontrun\elbibl}} % book ref
    \fancyhead[R]{\truncate{0.45\headwidth}{ \fontrun\nouppercase\leftmark}} % Chapter title
    \fancyhead[C]{\thepage}
  }
  \fancypagestyle{plain}{% apply to chapter
    \fancyhf{}% clear all header and footer fields
    \setlength{\headheight}{1.5em}
    \fancyhead[L]{\truncate{0.9\headwidth}{\fontrun\elbibl}}
    \fancyhead[R]{\thepage}
  }
\else
  \fancypagestyle{main}{%
    \fancyhf{}
    \setlength{\headheight}{1.5em}
    \fancyhead{} % reset head
    \fancyfoot{} % reset foot
    \fancyhead[RE]{\truncate{0.9\headwidth}{\fontrun\elbibl}} % book ref
    \fancyhead[LO]{\truncate{0.9\headwidth}{\fontrun\nouppercase\leftmark}} % Chapter title, \nouppercase needed
    \fancyhead[RO,LE]{\thepage}
  }
  \fancypagestyle{plain}{% apply to chapter
    \fancyhf{}% clear all header and footer fields
    \setlength{\headheight}{1.5em}
    \fancyhead[L]{\truncate{0.9\headwidth}{\fontrun\elbibl}}
    \fancyhead[R]{\thepage}
  }
\fi

\ifav % a5 only
  \titleclass{\section}{top}
\fi

\newcommand\chapo{{%
  \vspace*{-3em}
  \centering % no vskip ()
  {\Large\addfontfeature{LetterSpace=25}\bfseries{\elauthor}}\par
  \smallskip
  {\large\eldate}\par
  \bigskip
  {\Large\selectfont{\eltitle}}\par
  \bigskip
  {\color{rubric}\hline\par}
  \bigskip
  {\Large TEXTE LIBRE À PARTICPATION LIBRE\par}
  \centerline{\small\color{rubric} {hurlus.fr, tiré le \today}}\par
  \bigskip
}}

\newcommand\cover{{%
  \thispagestyle{empty}
  \centering
  {\LARGE\bfseries{\elauthor}}\par
  \bigskip
  {\Large\eldate}\par
  \bigskip
  \bigskip
  {\LARGE\selectfont{\eltitle}}\par
  \vfill\null
  {\color{rubric}\setlength{\arrayrulewidth}{2pt}\hline\par}
  \vfill\null
  {\Large TEXTE LIBRE À PARTICPATION LIBRE\par}
  \centerline{{\href{https://hurlus.fr}{\dotuline{hurlus.fr}}, tiré le \today}}\par
}}

\begin{document}
\pagestyle{empty}
\ifbooklet{
  \cover\newpage
  \thispagestyle{empty}\hbox{}\newpage
  \cover\newpage\noindent Les voyages de la brochure\par
  \bigskip
  \begin{tabularx}{\textwidth}{l|X|X}
    \textbf{Date} & \textbf{Lieu}& \textbf{Nom/pseudo} \\ \hline
    \rule{0pt}{25cm} &  &   \\
  \end{tabularx}
  \newpage
  \addtocounter{page}{-4}
}\fi

\thispagestyle{empty}
\ifaiv
  \twocolumn[\chapo]
\else
  \chapo
\fi
{\it\elabstract}
\bigskip
\makeatletter\@starttoc{toc}\makeatother % toc without new page
\bigskip

\pagestyle{main} % after style

  
\salute{À mes illustres amis}

\salute{M. Le baron Fourier}

\salute{Secrétaire perpétuel de l’Académie royale des sciences}

\salute{M. le professeur}

\salute{H. M. D. De Blainville}

\salute{Membre de l’académie royale des sciences}

\salute{En témoignage de ma respectueuse affection.}

\salute{Auguste Comte}
\section[{Avertissement de l’auteur}]{Avertissement de l’auteur}\phantomsection
\label{avertissement}\renewcommand{\leftmark}{Avertissement de l’auteur}

\noindent Ce cours, résultat général de tous mes travaux depuis ma sortie de l’École polytechnique en 1816, fut ouvert pour la première fois en avril 1826. Après un petit nombre de séances, une maladie grave m’empêcha, à cette époque, de poursuivre une entreprise encouragée, dès sa naissance, par les suffrages de plusieurs savants du premier ordre, parmi lesquels je pouvais citer dès lors MM. Alexandre de Humboldt, de Blainville, et Poinsot, membres de l’Académie des sciences, qui voulurent bien suivre avec un intérêt soutenu l’exposition de mes idées. J’ai refait ce cours en entier l’hiver dernier, à partir du 4 janvier 1829, devant un auditoire dont avaient bien voulu faire partie M. Fourier, secrétaire perpétuel de l’Académie des sciences, MM. de Blainville, Poinsot, Navier, membres de la même Académie, MM. les professeurs Broussais, Esquirol, Binet, etc., auxquels je dois ici témoigner publiquement ma reconnaissance pour la manière dont ils ont accueilli cette nouvelle tentative philosophique.\par
Après m’être assuré par de tels suffrages que ce cours pouvait utilement recevoir une plus grande publicité, j’ai cru devoir, à cette intention, l’exposer cet hiver à l’Athénée royal de Paris, où il vient d’être ouvert le 9 décembre.\par
Le plan est demeuré complètement le même ; seulement les convenances de cet établissement m’obligent à restreindre un peu les développements de mon cours. Ils se trouvent tout entiers dans la publication que je fais aujourd’hui de mes leçons, telles qu’elles ont eu lieu l’année dernière.\par
Pour compléter cette notice historique, il est convenable de faire observer, relativement à quelques-unes des idées fondamentales exposées dans ce cours, que je les avais présentées antérieurement dans la première partie d’un ouvrage intitulé : \emph{Système de politique positive} imprimée à cent exemplaires en mai 1822, et réimprimée ensuite en avril 1824, à un nombre d’exemplaires plus considérable. Cette première partie n’a point encore été formellement publiée, mais seulement communiquée par la voie de l’impression, à un grand nombre de savants et de philosophes européens. Elle ne sera mise définitivement en circulation qu’avec la seconde partie, que j’espère pouvoir faire paraître à la fin de l’année 1830.\par
J’ai cru nécessaire de constater ici la publicité effective de ce premier travail, parce que quelques idées, offrant une certaine analogie avec une partie des miennes, se trouvent exposées, sans aucune mention de mes recherches, dans divers ouvrages publiés postérieurement, surtout en ce qui concerne la rénovation des théories sociales. Quoique des esprits différents aient pu, sans aucune communication, comme le montre souvent l’histoire de l’esprit humain, arriver séparément à des conceptions analogues en s’occupant d’une même classe de travaux, je devais néanmoins insister sur l’antériorité réelle d’un ouvrage peu connu du public, afin qu’on ne suppose pas que j’ai puisé le germe de certaines idées dans des écrits qui sont, au contraire, plus récents.\par
Plusieurs personnes m’ayant déjà demandé quelques éclaircissements relativement au titre de ce cours, je crois utile d’indiquer ici, à ce sujet, une explication sommaire.\par
L’expression {\itshape philosophie positive} étant constamment employée, dans toute l’étendue de ce cours, suivant une acception rigoureusement invariable, il m’a paru superflu de la définir autrement que par l’usage uniforme que j’en ai toujours fait. La première leçon, en particulier, peut être regardée tout entière comme le développement de la définition exacte de ce que j’appelle la philosophie positive.\par
Je regrette néanmoins d’avoir été obligé d’adopter, à défaut de tout autre, un terme comme celui de {\itshape philosophie}, qui a été si abusivement employé dans une multitude d’acceptions diverses. Mais l’adjectif {\itshape positive} par lequel j’en modifie le sens me paraît suffire pour faire disparaître, même au premier abord, toute équivoque essentielle, chez ceux, du moins, qui en connaissent bien la valeur. Je me bornerai donc, dans cet {\itshape avertissement}, à déclarer que j’emploie le mot {\itshape philosophie} dans l’acception que lui donnaient les anciens, et particulièrement Aristote, comme désignant le système général des conceptions humaines ; et, en ajoutant le mot {\itshape positive}, j’annonce que je considère cette manière spéciale de philosopher qui consiste à envisager les théories, dans quelque ordre d’idées que ce soit, comme ayant pour objet la coordination des faits observés, ce qui constitue le troisième et dernier état de la philosophie générale, primitivement théologique et ensuite métaphysique, ainsi que je l’explique dès la première leçon.\par
Il y a, sans doute, beaucoup d’analogie entre ma philosophie positive et ce que les savants anglais entendent, depuis Newton surtout, par philosophie naturelle. Mais je n’ai pas dû choisir cette dernière dénomination, non plus que celle de philosophie des sciences, qui serait peut-être encore plus précise, parce que l’une et l’autre ne s’entendent pas encore de tous les ordres de phénomènes, tandis que la philosophie positive, dans laquelle je comprends l’étude des phénomènes sociaux aussi bien que de tous les autres, désigne une manière uniforme de raisonner applicable à tous les sujets sur lesquels l’esprit humain peut s’exercer. En outre, l’expression philosophie naturelle est usitée, en Angleterre, pour désigner l’ensemble des diverses sciences d’observation considérées jusque dans leurs spécialités les plus détaillées ; au lieu que, par philosophie positive, comparé à sciences positives, j’entends seulement l’étude propre des généralités des différentes sciences, conçues comme soumises à une méthode unique, et comme formant les différentes parties d’un plan général de recherches. Le terme que j’ai été conduit à construire est donc, à la fois, plus étendu et plus restreint que les dénominations, d’ailleurs analogues, quant au caractère fondamental des idées, qu’on pourrait, de prime abord, regarder comme équivalentes.\par

\dateline{Paris, le 18 décembre 1829.}

\chapteropen
\chapter[{Première leçon}]{Première leçon}\phantomsection
\label{leçon\_1}\renewcommand{\leftmark}{Première leçon}

\begin{center}\emph{Exposition du but de ce cours, ou considérations générales sur la nature et l’importance de la philosophie positive.}\end{center}

\chaptercont
\section[{I.}]{I.}
\noindent (1) L’objet de cette première leçon est d’exposer nettement le but du cours, c’est-à-dire de déterminer exactement l’esprit dans lequel seront considérées les diverses branches fondamentales de la philosophie naturelle, indiquées par le programme sommaire que je vous ai présenté.\par
(2) Sans doute, la nature de ce cours ne saurait être complètement appréciée, de manière à pouvoir s’en former une opinion définitive, que lorsque les diverses parties en auront été successivement développées. Tel est l’inconvénient ordinaire des définitions relatives à des systèmes d’idées très étendus, quand elles en précèdent l’exposition. Mais les généralités peuvent être conçues sous deux aspects, ou comme aperçu d’une doctrine à établir, ou comme résumé d’une doctrine établie. Si c’est seulement sous ce dernier point de vue qu’elles acquièrent toute leur valeur, elles n’en ont pas moins déjà, sous le premier, une extrême importance, en caractérisant dès l’origine le sujet à considérer. La circonscription générale du champ de nos recherches, tracée avec toute la sévérité possible, est, pour notre esprit, un préliminaire particulièrement indispensable dans une étude aussi vaste et jusqu’ici aussi peu déterminée que celle dont nous allons nous occuper. C’est afin d’obéir à cette nécessité logique, que je crois devoir vous indiquer, dès ce moment, la série des considérations fondamentales qui ont donné naissance à ce nouveau cours, et qui seront d’ailleurs spécialement développées, dans la suite, avec toute l’extension que réclame la haute importance de chacune d’elles.
\section[{II.}]{II.}
\noindent (1) Pour expliquer convenablement la véritable nature et le caractère propre de la philosophie positive, il est indispensable de jeter d’abord un coup d’œil général sur la marche progressive de l’esprit humain, envisagée dans son ensemble : car une conception quelconque ne peut être bien connue que par son histoire.\par
(2) En étudiant ainsi le développement total de l’intelligence humaine dans ses diverses sphères d’activité, depuis son premier essor le plus simple jusqu’à nos jours, je crois avoir découvert une grande loi fondamentale, à laquelle il est assujetti par une nécessité invariable, et qui me semble pouvoir être solidement établie, soit sur les preuves rationnelles fournies par la connaissance de notre organisation, soit sur les vérifications historiques résultant d’un examen attentif du passé. Cette loi consiste en ce que chacune de nos conceptions principales, chaque branche de nos connaissances, passe successivement par trois états théoriques différents : l’état théologique, ou fictif ; l’état métaphysique, ou abstrait ; l’état scientifique, ou positif. En d’autres termes, l’esprit humain, par sa nature, emploie successivement dans chacune de ses recherches trois méthodes de philosopher dont le caractère est essentiellement différent et même radicalement opposé — d’abord la méthode théologique, ensuite la méthode métaphysique et enfin la méthode positive. De là, trois sortes de philosophies, ou de systèmes généraux de conceptions sur l’ensemble des phénomènes, qui s’excluent mutuellement : la première est le point de départ nécessaire, de l’intelligence humaine ; la troisième, son état fixe et définitif ; la seconde est uniquement destinée à servir de transition.\par
(3) Dans l’état théologique, l’esprit humain, dirigeant essentiellement ses recherches vers la nature intime des êtres, les causes premières et finales de tous les effets qui le frappent, en un mot vers les connaissances absolues, se représente les phénomènes comme produits par l’action directe et continue d’agents surnaturels plus ou moins nombreux, dont l’intervention arbitraire explique toutes les anomalies apparentes de l’univers.\par
(4) Dans l’état métaphysique, qui n’est au fond qu’une simple modification générale du premier, les agents surnaturels sont remplacés par des forces abstraites, véritables entités (abstractions personnifiées) inhérentes aux divers êtres du monde, et conçues comme capables d’engendrer par elles-mêmes tous les phénomènes observés, dont l’explication consiste alors à assigner pour chacun l’entité correspondante.\par
(5) Enfin, dans l’état positif, l’esprit humain reconnaissant l’impossibilité d’obtenir des notions absolues, renonce à chercher l’origine et la destination de l’univers, et à connaître les causes intimes des phénomènes, pour s’attacher uniquement à découvrir, par l’usage bien combiné du raisonnement et de l’observation, leurs lois effectives, c’est-à-dire leurs relations invariables de succession et de similitude. L’explication des faits, réduite alors à ses termes réels, n’est plus désormais que la liaison établie entre les divers phénomènes particuliers et quelques faits généraux dont les progrès de la science tendent de plus en plus à diminuer le nombre.\par
(6) Le système théologique est parvenu à la plus haute perfection dont il soit susceptible quand il a substitué l’action providentielle d’un être unique au jeu varié des nombreuses divinités indépendantes qui avaient été imaginées primitivement. De même, le dernier terme du système métaphysique consiste à concevoir, au lieu de différentes entités particulières, une seule grande entité générale, la {\itshape nature}, envisagée comme la source unique de tous les phénomènes. Pareillement, la perfection du système positif, vers laquelle il tend sans cesse, quoiqu’il soit très probable qu’il ne doive jamais l’atteindre, serait de pouvoir se représenter tous les divers phénomènes observables comme des cas particuliers d’un seul fait général, tel que celui de la gravitation, par exemple.
\section[{III.}]{III.}
\noindent (1) Ce n’est pas ici le lieu de démontrer spécialement cette loi fondamentale du développement de l’esprit humain, et d’en déduire les conséquences les plus importantes. Nous en traiterons directement, avec toute l’extension convenable, dans la partie de ce cours relative à l’étude des phénomènes sociaux. Je ne la considère maintenant que pour déterminer avec précision le véritable caractère de la philosophie positive, par opposition aux deux autres philosophies qui ont successivement dominé, jusqu’à ces derniers siècles, tout notre système intellectuel. Quant à présent, afin de ne pas laisser entièrement sans démonstration une loi de cette importance, dont les applications se présenteront fréquemment dans toute l’étendue de ce cours, je dois me borner à une indication rapide des motifs généraux les plus sensibles qui peuvent en constater l’exactitude.\par
(2) En premier lieu, il suffit, ce me semble, d’énoncer une telle loi, pour que la justesse en soit immédiatement vérifiée par tous ceux qui ont quelque connaissance approfondie de l’histoire générale des sciences. Il n’en est pas une seule, en effet, parvenue aujourd’hui à l’état positif, que chacun ne puisse aisément se représenter, dans le passé, essentiellement composée d’abstractions métaphysiques, et, en remontant encore davantage, tout à fait dominée par les conceptions théologiques. Nous aurons même malheureusement plus d’une occasion formelle de reconnaître, dans les diverses parties de ce cours, que les sciences les plus perfectionnées conservent encore aujourd’hui quelques traces très sensibles de ces deux états primitifs.\par
(3) Cette révolution générale de l’esprit humain peut d’ailleurs être aisément constatée aujourd’hui, d’une manière très sensible, quoique indirecte, en considérant le développement de l’intelligence individuelle. Le point de départ étant nécessairement le même dans l’éducation de l’individu que dans celle de l’espèce, les diverses phases principales de la première doivent représenter les époques fondamentales de la seconde. Or, chacun de nous, en contemplant sa propre histoire, ne se souvient-il pas qu’il a été successivement, quant à ses notions les plus importantes, {\itshape théologien} dans son enfance, {\itshape métaphysicien} dans sa jeunesse, et {\itshape physicien} dans sa virilité ? Cette vérification est facile aujourd’hui pour tous les hommes au niveau de leur siècle.\par
(4) Mais outre l’observation directe, générale ou individuelle, qui prouve l’exactitude de cette loi, je dois surtout, dans cette indication sommaire, mentionner les considérations théoriques qui en font sentir la nécessité.\par
La plus importante de ces considérations, puisée dans la nature même du sujet, consiste dans le besoin à toute époque, d’une théorie quelconque pour lier les faits, combine avec l’impossibilité évidente, pour l’esprit humain à son origine, de se former des théories d’après les observations.\par
Tous les bons esprits répètent, depuis Bacon, qu’il n’y a de connaissances réelles que celles qui reposent sur des faits observés. Cette maxime fondamentale est évidemment incontestable, si on l’applique comme il convient à l’état viril de notre intelligence. Mais, en se reportant à la formation de nos connaissances, il n’en est pas moins certain que l’esprit humain, dans son état primitif, ne pouvait ni ne devait penser ainsi. Car si, d’un côté, toute théorie positive doit nécessairement être fondée sur des observations, il est également sensible, d’un autre côté, que, pour se livrer à l’observation, notre esprit a besoin d’une théorie quelconque. Si, en contemplant les phénomènes, nous ne les rattachions point immédiatement à quelques principes, non seulement il nous serait impossible de combiner ces observations isolées, et, par conséquent, d’en tirer aucun fruit, mais nous serions même entièrement incapables de les retenir, et, le plus souvent, les faits resteraient inaperçus sous nos yeux.\par
Ainsi, pressé entre la nécessité d’observer pour se former des théories réelles et la nécessité non moins impérieuse de se créer des théories quelconques pour se livrer à des observations suivies, l’esprit humain, à sa naissance, se trouverait enfermé dans un cercle vicieux dont il n’aurait jamais eu aucun moyen de sortir, s’il ne se fût heureusement ouvert une issue naturelle par le développement spontané des conceptions théologiques, qui ont présenté un point de ralliement à ses efforts, et fourni un aliment à son activité. Tel est, indépendamment des hautes considérations sociales qui s’y rattachent et que je ne dois pas même indiquer en ce moment, le motif fondamental qui démontre la nécessité logique du caractère purement théologique de la philosophie primitive.\par
(5) Cette nécessité devient encore plus sensible en ayant égard à la parfaite convenance de la philosophie théologique avec la nature propre des recherches sur lesquelles l’esprit humain dans son enfance concentre si éminemment toute son activité. Il est bien remarquable, en effet, que les questions les plus radicalement inaccessibles à nos moyens, la nature intime des êtres, l’origine et la fin de tous les phénomènes, soient précisément celles que notre intelligence se propose par-dessus tout dans cet état primitif, tous les problèmes vraiment solubles étant presque envisagés comme indignes de méditations sérieuses. On en conçoit aisément la raison ; car c’est l’expérience seule qui a pu nous fournir la mesure de nos forces ; et, si l’homme n’avait d’abord commencé par en avoir une opinion exagérée, elles n’eussent jamais pu acquérir tout le développement dont elles sont susceptibles. Ainsi l’exige notre organisation. Mais, quoi qu’il en soit, représentons-nous, autant que possible, cette disposition si universelle et si prononcée, et demandons-nous quel accueil aurait reçu à une telle époque, en la supposant formée, la philosophie positive, dont la plus haute ambition est de découvrir les lois des phénomènes, dont le premier caractère propre est précisément de regarder comme nécessairement interdits à la raison humaine tous ces sublimes mystères que la philosophie théologique explique, au contraire, avec une si admirable facilité jusque dans leurs moindres détails.\par
Il en est de même en considérant sous le point de vue pratique la nature des recherches qui occupent primitivement l’esprit humain. Sous ce rapport, elles offrent à l’homme l’attrait si énergique d’un empire illimité à exercer sur le monde extérieur, envisagé comme entièrement destiné à notre usage, et comme présentant dans tous ses phénomènes des relations intimes et continues avec notre existence. Or, ces espérances chimériques, ces idées exagérées de l’importance de l’homme dans l’univers, que fait naître la philosophie théologique, et que détruit sans retour la première influence de la philosophie positive, sont, à l’origine, un stimulant indispensable, sans lequel on ne pourrait certainement concevoir que l’esprit humain se fût déterminé primitivement à de pénibles travaux.\par
Nous sommes aujourd’hui tellement éloignés de ces dispositions premières, du moins quant à la plupart des phénomènes, que nous avons peine à nous représenter exactement la puissance et la nécessité de considérations semblables. La raison humaine est maintenant assez mûre pour que nous entreprenions de laborieuses recherches scientifiques, sans avoir en vue aucun but étranger capable d’agir fortement sur l’imagination, comme celui que se proposaient les astrologues ou les alchimistes. Notre activité intellectuelle est suffisamment excitée par le pur espoir de découvrir les lois des phénomènes, par le simple désir de confirmer ou d’infirmer une théorie. Mais il ne pouvait en être ainsi dans l’enfance de l’esprit humain. Sans les attrayantes chimères de l’astrologie, sans les énergiques déceptions de l’alchimie, par exemple, où aurions-nous puisé la constance et l’ardeur nécessaires pour recueillir les longues suites d’observations et d’expériences qui ont plus tard servi de fondement aux premières théories positives de l’une et l’autre classe de phénomènes ?\par
Cette condition de notre développement intellectuel a été vivement sentie depuis longtemps par Kepler, pour l’astronomie, et justement appréciée de nos jours par Berthollet, pour la chimie.\par
(6) On voit donc, par cet ensemble de considérations, que, si la philosophie positive est le véritable état définitif de l’intelligence humaine, celui vers lequel elle a toujours tendu de plus en plus, elle n’en a pas moins dû nécessairement {\itshape employer} d’abord, et pendant une longue suite de siècles, soit comme méthode, soit comme doctrines provisoires, la philosophie théologique ; philosophie dont le caractère est d’être spontanée, et, par cela même la seule possible à l’origine, la seule aussi qui pût offrir à notre esprit naissant un intérêt suffisant. Il est maintenant très facile de sentir que, pour passer de cette philosophie provisoire à la philosophie définitive, l’esprit humain a dû naturellement adopter, comme philosophie transitoire, les méthodes et les doctrines métaphysiques. Cette dernière considération est indispensable pour compléter l’aperçu général de la grande loi que j’ai indiquée.\par
On conçoit sans peine, en effet, que notre entendement, contraint à ne marcher que par degrés presque insensibles, ne pouvait passer brusquement, et sans intermédiaires, de la philosophie théologique à la philosophie positive. La théologie et la physique sont si profondément incompatibles, leurs conceptions ont un caractère si radicalement opposé, qu’avant de renoncer aux unes pour employer exclusivement les autres, l’intelligence humaine a dû se servir de conceptions intermédiaires, d’un caractère bâtard, propre, par cela même, à opérer graduellement la transition. Telle est la destination naturelle des conceptions métaphysiques : elles n’ont pas d’autre utilité réelle. En substituant, dans l’étude des phénomènes, à l’action surnaturelle directrice une entité correspondante et inséparable, quoique celle-ci ne fût d’abord conçue que comme une émanation de la première, l’homme s’est habitué peu à peu à ne considérer que les faits eux-mêmes, les notions de ces agents métaphysiques ayant été graduellement subtilisées au point de n’être plus, aux yeux de tout esprit droit, que les noms abstraits des phénomènes. Il est impossible d’imaginer par quel autre procédé notre entendement aurait pu passer des considérations franchement surnaturelles aux considérations purement naturelles, du régime théologique au régime positif.
\section[{IV.}]{IV.}
\noindent (1) Après avoir ainsi établi, autant que je puis le faire sans entrer dans une discussion spéciale qui serait déplacée en ce moment, la loi générale du développement de l’esprit humain, tel que je le conçois, il nous sera maintenant aisé de déterminer avec précision la nature propre de la philosophie positive : ce qui est l’objet essentiel de ce discours.\par
Nous voyons, par ce qui précède, que le caractère fondamental de la philosophie positive est de regarder tous les phénomènes comme assujettis à des lois naturelles invariables, dont la découverte précise et la réduction au moindre nombre possible sont le but de tous nos efforts, en considérant comme absolument inaccessible et vide de sens pour nous la recherche de ce qu’on appelle les causes, soit premières, soit finales. Il est inutile d’insister beaucoup sur un principe devenu maintenant aussi familier à tous ceux qui ont fait une étude un peu approfondie des sciences d’observation. Chacun sait, en effet, que, dans nos explications positives, même les plus parfaites, nous n’avons nullement la prétention d’exposer les causes génératrices des phénomènes puisque nous ne ferions jamais alors que reculer la difficulté, mais seulement d’analyser avec exactitude les circonstances de leur production, et de les rattacher les unes aux autres par des relations normales de succession et de similitude.\par
Ainsi, pour en citer l’exemple le plus admirable, nous disons que les phénomènes généraux de l’univers sont expliqués, autant qu’ils puissent l’être, par la loi de la gravitation newtonienne, parce que, d’un côté, cette belle théorie nous montre toute l’immense variété des faits astronomiques, comme n’étant qu’un seul et même fait envisage sous divers points de vue : la tendance constante de toutes les molécules les unes vers les autres en raison directe de leurs masses, et en raison inverse des carrés de leurs distances ; tandis que, d’un autre côté, ce fait général nous est présenté comme une simple extension d’un phénomène qui nous est éminemment familier, et que, par cela seul, nous regardons comme parfaitement connu, la pesanteur des corps à la surface de la terre. Quant à déterminer ce que sont en elles-mêmes cette attraction et cette pesanteur, quelles en sont les causes, ce sont des questions que nous regardons tous comme insolubles, qui ne sont plus du domaine de la philosophie positive, et que nous abandonnons avec raison à l’imagination des théologiens, ou aux subtilités des métaphysiciens. La preuve manifeste de l’impossibilité d’obtenir de telles solutions, c’est que, toutes les fois qu’on a cherché à dire à ce sujet quelque chose de vraiment rationnel, les plus grands esprits n’ont pu que définir ces deux principes l’un par l’autre, en disant, pour l’attraction, qu’elle n’est autre chose qu’une pesanteur, universelle, et ensuite, pour la pesanteur qu’elle consiste simplement dans l’attraction terrestre. De telles explications, qui font sourire quand on prétend à connaître la nature intime des choses et le mode de génération des phénomènes, sont cependant tout ce que nous pouvons obtenir de plus satisfaisant, en nous montrant comme identiques deux ordres de phénomènes qui ont été si longtemps regardés comme n’ayant aucun rapport entre eux. Aucun esprit juste ne cherche aujourd’hui à aller plus loin.\par
Il serait aisé de multiplier ces exemples, qui se présenteront en foule dans toute la durée de ce cours, puisque tel est maintenant l’esprit qui dirige exclusivement les grandes combinaisons intellectuelles. Pour en citer en ce moment un seul parmi les travaux contemporains, je choisirai la belle série de recherches de M. Fourier sur la théorie de la chaleur. Elle nous offre la vérification très sensible des remarques générales précédentes. En effet, dans ce travail, dont le caractère philosophique est si éminemment positif ; les lois les plus importantes et les plus précises des phénomènes thermologiques se trouvent dévoilées, sans que l’auteur se soit enquis une seule fois de la nature intime de la chaleur, sans qu’il ait mentionné, autrement que pour en indiquer le vide, la controverse si agitée entre les partisans de la matière calorifique et ceux qui font consister la chaleur dans les vibrations d’un éther universel. Et néanmoins les plus hautes questions, dont plusieurs n’avaient même jamais été posées sont traitées dans cet ouvrage, preuve palpable que l’esprit humain, sans se jeter dans des problèmes inabordables, et en se restreignant dans les recherches d’un ordre entièrement positif, peut y trouver un aliment inépuisable à son activité la plus profonde.
\section[{V.}]{V.}
\noindent (1) Après avoir caractérisé, aussi exactement qu’il m’est permis de faire dans cet aperçu général, l’esprit de la philosophie positive, que ce cours tout entier est destiné à développer, je dois maintenant examiner à quelle époque de sa formation elle est parvenue aujourd’hui, et ce qui reste à faire pour achever de la constituer.\par
À cet effet, il faut d’abord considérer que les différentes branches de nos connaissances n’ont pas dû parcourir d’une vitesse égale les trois grandes phases de leur développement indiquées ci-dessus, ni, par conséquent, arriver simultanément à l’état positif. Il existe, sous ce rapport, un ordre invariable et nécessaire, que nos divers genres de conceptions ont suivi et dû suivre dans leur progression, et dont la considération exacte est le complément indispensable de la loi fondamentale énoncée précédemment. Cet ordre sera le sujet spécial de la prochaine leçon. Qu’il nous suffise, quant à présent, de savoir qu’il est conforme à la nature diverse des phénomènes, et qu’il est déterminé par leur degré de généralité, de simplicité et d’indépendance réciproque, trois considérations qui, bien que distinctes, concourent au même but. Ainsi, les phénomènes astronomiques d’abord, comme étant les plus généraux, les plus simples et les plus indépendants de tous les autres, et successivement, par les mêmes raisons, les phénomènes de la physique terrestre proprement dite, ceux de la chimie, et enfin les phénomènes physiologiques, ont été ramenés à des théories positives.\par
(2) Il est impossible d’assigner l’origine précise de cette révolution ; car on n’en peut dire avec exactitude, comme de tous les autres grands événements humains, qu’elle s’est accomplie constamment et de plus en plus, particulièrement depuis les travaux d’Aristote et de l’école d’Alexandrie, et ensuite depuis l’introduction des sciences naturelles dans l’Europe occidentale par les Arabes. Cependant, vu qu’il convient de fixer une époque pour empêcher la divagation des idées, j’indiquerai celle du grand mouvement imprimé à l’esprit humain, il y a deux siècles, par l’action combinée des préceptes de Bacon, des conceptions de Descartes, et des découvertes de Galilée, comme le moment où l’esprit de la philosophie positive a commencé à se prononcer dans le monde en opposition évidente avec l’esprit théologique et métaphysique. C’est alors, en effet, que les conceptions positives se sont dégagées nettement de l’alliage superstitieux et scolastique qui déguisait plus ou moins le véritable caractère de tous les travaux antérieurs.\par
Depuis cette mémorable époque, le mouvement d’ascension de la philosophie positive, et le mouvement de décadence de la philosophie théologique et métaphysique, ont été extrêmement marqués. Ils se sont enfin tellement prononcés, qu’il est devenu impossible aujourd’hui, à tous les observateurs ayant conscience de leur siècle, de méconnaître la destination finale de l’intelligence humaine pour les études positives, ainsi que son éloignement désormais irrévocable pour ces vaines doctrines et pour ces méthodes provisoires qui ne pouvaient convenir qu’à son premier essor. Ainsi, cette révolution fondamentale s’accomplira nécessairement dans toute son étendue. Si donc il lui reste encore quelque grande conquête à faire, quelque branche principale du domaine intellectuel à envahir, on peut être certain que la transformation s’y opérera, comme elle s’est effectuée dans toutes les autres. Car il serait évidemment contradictoire de supposer que l’esprit humain, si disposé à l’unité de méthode, conservât indéfiniment, pour une seule classe de phénomènes, sa manière primitive de philosopher, lorsqu’une fois il est arrivé à adopter pour tout le reste une nouvelle marche philosophique d’un caractère absolument opposé.\par
(3) Tout se réduit donc à une simple question de fait la philosophie positive, qui, dans les deux derniers siècles, a pris graduellement une si grande extension, embrasse-t-elle aujourd’hui tous les ordres de phénomènes ? Il est évident que cela n’est point, et que, par conséquent, il reste encore une grande opération scientifique à exécuter pour donner à la philosophie positive ce caractère d’universalité indispensable à sa constitution définitive.\par
En effet, dans les quatre catégories principales de phénomènes naturels énumérées tout à l’heure, les phénomènes astronomiques, physiques, chimiques et physiologiques, on remarque une lacune essentielle relative aux phénomènes sociaux, qui, bien que compris implicitement parmi les phénomènes physiologiques, méritent, soit par leur importance, soit par les difficultés propres à leur étude, de former une catégorie distincte. Ce dernier ordre de conceptions, qui se rapporte aux phénomènes les plus particuliers, les plus compliqués et les plus dépendants de tous les autres a dû nécessairement, par cela seul, se perfectionner plus lentement que tous les précédents. Même sans avoir égard aux obstacles plus spéciaux que nous considérerons Plus tard. Quoi qu’il en soit, il est évident qu’il n’est point encore entré dans le domaine de la philosophie positive.\par
Les méthodes théologiques et métaphysiques qui, relativement à tous les autres genres de phénomènes, ne sont plus maintenant employées par personne, soit comme moyen d’investigation, soit même seulement comme moyen d’argumentation, sont encore, au contraire, exclusivement usitées, sous l’un et l’autre rapport, pour tout ce qui concerne les phénomènes sociaux, quoique leur insuffisance à cet égard soit déjà pleinement sentie par tous les bons esprits, lassés de ces vaines contestations interminables entre le droit divin et la souveraineté du peuple.\par
Voilà donc la grande, mais évidemment la seule lacune qu’il s’agit de combler pour achever de constituer la philosophie positive. Maintenant que l’esprit humain a fondé la physique céleste, la physique terrestre, soit mécanique, soit chimique ; la physique organique, soit végétale, soit animale, il lui reste à terminer le système des sciences d’observation en fondant la {\itshape physique sociale.} Tel est aujourd’hui, sous plusieurs rapports capitaux, le plus grand et le plus pressant besoin de notre intelligence : tel est, j’ose le dire, le premier but de ce cours, son but spécial.\par
(4) Les conceptions que je tenterai de présenter relativement à l’étude des phénomènes sociaux, et dont j’espère que ce discours laisse déjà entrevoir le germe, ne sauraient avoir pour objet de donner immédiatement à la physique sociale le même degré de perfection qu’aux branches antérieures de la philosophie naturelle, ce qui serait évidemment chimérique, puisque celles-ci offrent déjà entre elles à cet égard une extrême inégalité, d’ailleurs inévitable. Mais elles seront destinées à imprimer à cette dernière classe de nos connaissances ce caractère positif déjà pris par toutes les autres. Si cette condition est une fois réellement remplie, le système philosophique des modernes sera enfin fondé dans son ensemble ; car aucun phénomène observable ne saurait évidemment manquer de rentrer dans quelqu’une des cinq grandes catégories dès lors établies des phénomènes astronomiques, physiques, chimiques, physiologiques et sociaux. Toutes nos conceptions fondamentales étant devenues homogènes, la philosophie sera définitivement constituée à l’état positif ; sans jamais pouvoir changer de caractère, il ne lui restera qu’à se développer indéfiniment par les acquisitions toujours croissantes qui résulteront inévitablement de nouvelles observations ou de méditations plus profondes. Ayant acquis par là le caractère d’universalité qui lui manque encore, la philosophie positive deviendra capable de se substituer entièrement, avec toute sa supériorité naturelle, à la philosophie théologique et à la philosophie métaphysique, dont cette universalité est aujourd’hui la seule propriété réelle, et qui, privées d’un tel motif de préférence, n’auront plus pour nos successeurs qu’une existence historique.\par
(5) Le but spécial de ce cours étant ainsi exposé, il est aisé de comprendre son second but, son but général, ce qui en fait un cours de philosophie positive, et non pas seulement un cours de physique sociale.\par
En effet, la fondation de la physique sociale complétant enfin le système des sciences naturelles, il devient possible et même nécessaire de résumer les diverses connaissances acquises, parvenues alors à un état fixe et homogène, pour les coordonner en les présentant comme autant de branches d’un tronc unique, au lieu de continuer à les concevoir seulement comme autant de corps isolés. C’est à cette fin qu’avant de procéder à l’étude des phénomènes sociaux, je considérerai successivement, dans l’ordre encyclopédique annoncé plus haut, les différentes sciences positives déjà formées.\par
Il est superflu, je pense, d’avertir qu’il ne saurait être question ici d’une suite de cours spéciaux sur chacune des branches principales de la philosophie naturelle. Sans parler de la durée matérielle d’une entreprise semblable, il est clair qu’une pareille prétention serait insoutenable de ma part, et je crois pouvoir ajouter de la part de qui que ce soit, dans l’état actuel de l’éducation humaine. Bien au contraire, un cours de la nature de celui-ci exige, pour être convenablement entendu, une série préalable d’études spéciales sur les diverses sciences qui y seront envisagées. Sans cette condition, il est bien difficile de sentir et impossible de juger les réflexions philosophiques dont ces sciences seront les sujets. En un mot c’est un {\itshape Cours de philosophie} positive, et non de sciences positives, que je me propose de faire. Il s’agit uniquement ici de considérer chaque science fondamentale dans ses relations avec le système positif tout entier, et quant à l’esprit qui la caractérise, c’est-à-dire sous le double rapport, de ses méthodes essentielles et de ses résultats principaux. Le plus souvent même je devrai me borner à mentionner ces derniers d’après les connaissances spéciales, pour tâcher d’en apprécier l’importance.\par
Afin de résumer les idées relativement au double but de ce cours, je dois faire observer que les deux objets, l’un spécial, l’autre général, que je me propose, quoique distincts en eux-mêmes, sont nécessairement inséparables. Car, d’un côté, il serait impossible de concevoir un cours de philosophie positive sans la fondation de la physique sociale, puisqu’il manquerait alors un élément essentiel, et que, par cela seul, les conceptions ne sauraient avoir ce caractère de généralité qui doit en être le principal attribut, et qui distingue notre étude actuelle de la série des études spéciales. D’un autre côté, comment procéder avec sûreté à l’étude positive des phénomènes sociaux, si l’esprit n’est d’abord préparé par la considération approfondie des méthodes positives déjà jugées pour les phénomènes moins compliqués, et muni, en outre, de la connaissance des lois principales des phénomènes antérieurs, qui toutes influent, d’une manière plus ou moins directe, sur les faits sociaux ?\par
Bien que toutes les sciences fondamentales n’inspirent pas aux esprits vulgaires un égal intérêt, il n’en est aucune qui doive être négligée dans une étude comme celle que nous entreprenons. Quant à leur importance pour le bonheur de l’espèce humaine, toutes sont certainement équivalentes, lorsqu’on les envisage d’une manière approfondie. Celles, d’ailleurs, dont les résultats présentent, au premier abord, un moindre intérêt pratique, se recommandent éminemment, soit par la plus grande perfection de leurs méthodes, soit comme étant le fondement indispensable de toutes les autres. C’est une considération sur laquelle j’aurai spécialement occasion de revenir dans la prochaine leçon.\par
(6) Pour prévenir, autant que possible, toutes les fausses interprétations qu’il est légitime de craindre sur la nature d’un cours aussi nouveau que celui-ci, je dois ajouter sommairement aux explications précédentes quelques considérations directement relatives à cette universalité de connaissances spéciales, que des juges irréfléchis pourraient regarder comme la tendance de ce cours, et qui est envisagée à si juste raison comme tout à fait contraire au véritable esprit de la philosophie positive. Ces considérations auront d’ailleurs l’avantage plus important de présenter cet esprit sous un nouveau point de vue, propre à achever d’en éclaircir la notion générale.\par
Dans l’état primitif de nos connaissances il n’existe aucune division régulière parmi nos travaux intellectuels ; toutes les sciences sont cultivées simultanément par les mêmes esprits. Ce mode d’organisation des études humaines, d’abord inévitable et même indispensable, comme nous aurons lieu de le constater plus tard, change peu à peu, à mesure que les divers ordres de conceptions se développent. Par une loi dont la nécessité est évidente, chaque branche du système scientifique se sépare insensiblement du tronc, lorsqu’elle a pris assez d’accroissement pour comporter une culture isolée, c’est-à-dire quand elle est parvenue à ce point de pouvoir occuper à elle seule l’activité permanente de quelques intelligences. C’est à cette répartition des diverses sortes de recherches entre différents ordres de savants, que nous devons évidemment le développement si remarquable qu’a pris enfin de nos jours chaque classe distincte des connaissances humaines, et qui rend manifeste l’impossibilité, chez les modernes, de cette universalité de recherches spéciales, si facile et si commune dans les temps antiques. En un mot, la division du travail intellectuel perfectionnée de plus en plus, est un des attributs caractéristiques les plus importants de la philosophie positive.\par
Mais tout en reconnaissant les prodigieux résultats de cette division, tout en voyant désormais en elle la véritable base fondamentale de l’organisation générale du monde savant, il est impossible, d’un autre côté, de n’être pas frappé des inconvénients capitaux qu’elle engendre, dans son état actuel, par l’excessive particularité des idées qui occupent exclusivement chaque intelligence individuelle. Ce fâcheux effet est sans doute inévitable jusqu’à un certain point, comme inhérent au principe même de la division ; c’est-à-dire que, par aucune mesure quelconque, nous ne parviendrons jamais à égaler sous ce rapport les anciens, chez lesquels une telle supériorité ne tenait surtout qu’au peu de développement de leurs connaissances. Nous pouvons néanmoins, ce me semble, par des moyens convenables, éviter les plus pernicieux effets de la spécialité exagérée, sans nuire à l’influence vivifiante de la séparation des recherches. Il est urgent de s’en occuper sérieusement ; car ces inconvénients, qui, par leur nature, tendent à s’accroître sans cesse, commencent à devenir très sensibles. De l’aveu de tous, les divisions, établies pour la plus grande perfection de nos travaux entre les diverses branches de la philosophie naturelle, sont finalement artificielles. N’oublions Pas que, nonobstant cet aveu il est déjà bien petit dans le monde savant le nombre des intelligences embrassant dans leurs conceptions l’ensemble même d’une science unique, qui n’est cependant à son tour qu’une partie d’un grand tout. La plupart se bornent déjà entièrement à la considération isolée d’une section plus ou moins étendue d’une science déterminée, sans s’occuper beaucoup de la relation de ces travaux particuliers avec le système général des connaissances positives. Hâtons-nous de remédier au mal, avant qu’il soit devenu plus grave. Craignons que l’esprit humain ne finisse par se perdre dans les travaux de détail. Ne nous dissimulons pas que c’est là essentiellement le côté faible par lequel les partisans de la philosophie théologique et de la philosophie métaphysique peuvent encore attaquer avec quelque espoir de succès la philosophie positive.\par
Le véritable moyen d’arrêter l’influence délétère dont l’avenir intellectuel semble menace, par suite d’une trop grande spécialisation des recherches individuelles, ne saurait être, évidemment, de revenir à cette antique confusion des travaux, qui tendrait à faire rétrograder l’esprit humain, et qui est d’ailleurs aujourd’hui heureusement devenue impossible. Il consiste, au contraire, dans le perfectionnement de la division du travail elle-même. Il suffit, en effet, de faire de l’étude des généralités scientifiques une grande spécialité de plus. Qu’une classe nouvelle de savants préparés par une éducation convenable, sans se livrer à la culture spéciale d’aucune branche particulière de la philosophie naturelle, s’occupe uniquement, en considérant les diverses sciences positives dans leur état actuel, à déterminer exactement l’esprit de chacune d’elles, à découvrir leurs relations et leur enchaînement, à résumer, s’il est possible, tous leurs principes propres en un moindre nombre de principes communs, en se conformant sans cesse aux maximes fondamentales de la méthode positive. Qu’en même temps, les autres savants, avant de se livrer à leurs spécialités respectives, soient rendus aptes désormais, par une éducation portant sur l’ensemble des connaissances positives, à profiter immédiatement des lumières répandues par ces savants voués à l’étude des généralités, et réciproquement : à rectifier leurs résultats, état de choses dont les savants actuels se rapprochent visiblement de jour en jour. Ces deux grandes conditions une fois remplies, et il est évident qu’elles peuvent l’être, la division du travail dans les sciences sera poussée, sans aucun danger, aussi loin que le développement des divers ordres de connaissances l’exigera. Une classe distincte, incessamment contrôlée par toutes les autres, ayant pour fonction propre et permanente de lier chaque nouvelle découverte particulière au système général, on n’aura plus à craindre qu’une trop grande attention donnée aux détails empêche jamais d’apercevoir l’ensemble. En un mot, l’organisation moderne du monde savant sera dès lors complètement fondée, et n’aura qu’à se développer indéfiniment, en conservant toujours le même caractère.\par
Former ainsi de l’étude des généralités scientifiques une section distincte du grand travail intellectuel, c’est simplement étendre l’application du même principe de division qui a successivement séparé les diverses spécialités ; car, tant que les différentes sciences positives ont été peu développées, leurs relations mutuelles ne pouvaient avoir assez d’importance pour donner lieu, au moins d’une manière permanente, à une classe particulière de travaux, et en même temps la nécessité de cette nouvelle étude était bien moins urgente. Mais aujourd’hui chacune des sciences a pris séparément assez d’extension pour que l’examen de leurs rapports mutuels puisse donner lieu à des travaux suivis, en même temps que ce nouvel ordre d’études devient indispensable pour prévenir la dispersion des conceptions humaines.\par
Telle est la manière dont je conçois la destination de la philosophie positive dans le système général des sciences positives proprement dites. Tel est, du moins, le but de ce cours.
\section[{VI.}]{VI.}
\noindent (1) Maintenant que j’ai essayé de déterminer aussi exactement qu’il m’a été possible de le faire, dans ce premier aperçu, l’esprit général d’un cours de philosophie positive, je crois devoir, pour imprimer à ce tableau tout son caractère, signaler rapidement les principaux avantages généraux que peut avoir un tel travail, si les conditions essentielles en sont convenablement remplies, relativement aux progrès de l’esprit humain. Je réduirai ce dernier ordre de considérations à l’indication de quatre propriétés fondamentales.\par
Premièrement l’étude de la philosophie positive, en considérant les résultats de l’activité de nos facultés intellectuelles, nous fournit le seul vrai moyen rationnel de mettre en évidence les lois logiques de l’esprit humain, qui ont été recherchées jusqu’ici par des voies si peu propres à les dévoiler.\par
Pour expliquer convenablement ma pensée à cet égard, je dois d’abord rappeler une conception philosophique de la plus haute importance, exposée par de Blainville dans la belle introduction de ses \emph{Principes généraux d’anatomie comparée}. Elle consiste en ce que tout être actif, et spécialement tout être vivant, peut être étudié, dans tous ses phénomènes, sous deux rapports fondamentaux, sous le rapport statique et sous le rapport dynamique, c’est-à-dire comme apte à agir et comme agissant effectivement. Il est clair, en effet, que toutes les considérations qu’on pourra présenter rentreront nécessairement dans l’un ou l’autre mode. Appliquons cette lumineuse maxime fondamentale à l’étude des fonctions intellectuelles.\par
Si l’on envisage ces fonctions sous le point de vue statique, leur étude peut consister que dans la détermination des conditions organiques dont elles dépendent ; elle forme ainsi une partie essentielle de l’anatomie et de la physiologie. En les considérant sous le point de vue dynamique, tout se réduit à étudier la marche effective de l’esprit humain en exercice, par l’examen des procédés réellement employés pour les diverses connaissances exactes qu’il a déjà acquises, ce qui constitue essentiellement l’objet général de la philosophie positive, ainsi que je l’ai définie dans ce discours. En un mot, regardant toutes les théories scientifiques comme autant de grands faits logiques, c’est uniquement par l’observation approfondie de ces faits qu’on peut s’élever à la connaissance des lois logiques.\par
Telles sont évidemment les deux seules voies générales, complémentaires l’une et l’autre, par lesquelles on puisse arriver à quelques notions rationnelles véritables sur les phénomènes intellectuels. On voit que, sous aucun rapport, il n’y a place pour cette psychologie illusoire, dernière transformation de la théologie, qu’on tente si vainement de ranimer aujourd’hui, et qui, sans s’inquiéter ni de l’étude physiologique de nos organes intellectuels, ni de l’observation des procédés rationnels qui dirigent effectivement nos diverses recherches scientifiques, prétend arriver à la découverte des lois fondamentales de l’esprit humain, en le contemplant en lui-même, c’est-à-dire en faisant complètement abstraction et des causes et des effets.\par
La prépondérance de la philosophie positive est successivement devenue telle depuis Bacon ; elle a pris aujourd’hui, indirectement, un si grand ascendant sur les esprits même qui sont demeurés les plus étrangers à son immense développement, que les métaphysiciens livrés à l’étude de notre intelligence n’ont pu espérer de ralentir la décadence de leur prétendue science qu’en se ravisant pour présenter leurs doctrines comme étant aussi fondées sur l’observation des faits. À cette fin, ils ont imaginé, dans ces derniers temps, de distinguer, par une subtilité fort singulière, deux sortes d’observations d’égale importance, l’une extérieure, l’autre intérieure, et dont la dernière est uniquement destinée à l’étude des phénomènes intellectuels. Ce n’est point ici le lieu d’entrer dans la discussion spéciale de ce sophisme fondamental. Je dois me borner à indiquer la considération principale qui prouve clairement que cette prétendue contemplation directe de l’esprit par lui-même est une pure illusion.\par
On croyait, il y a encore peu de temps, avoir expliqué la vision, en disant que l’action lumineuse des corps détermine sur la rétine des tableaux représentatifs des formes et des couleurs extérieures. À cela les physiologistes ont objecté avec raison que, si c’était comme images qu’agissaient les impressions lumineuses, il faudrait un autre œil pour les regarder. N’en est-il pas encore plus fortement de même dans le cas présent ?\par
Il est sensible, en effet, que, par une nécessité invincible, l’esprit humain peut observer directement tous les phénomènes, excepté les siens propres. Car, par qui serait faite l’observation ? On conçoit, relativement aux phénomènes moraux, que l’homme puisse s’observer lui-même sous le rapport des passions qui l’animent, par cette raison anatomique, que les organes qui en sont le siège sont distincts de ceux destinés aux fonctions observatrices. Encore même que chacun ait eu occasion de faire sur lui de telles remarques, elles ne sauraient évidemment avoir jamais une grande importance scientifique, et le meilleur moyen de connaître les passions sera-t-il toujours de les observer en dehors ; car tout état de passion très prononcé, c’est-à-dire précisément celui qu’il serait le plus essentiel d’examiner, est nécessairement incompatible avec l’état d’observation. Mais, quant à observer de la même manière les phénomènes intellectuels pendant qu’ils s’exécutent, il, y a impossibilité manifeste. L’individu pensant ne saurait se partager en deux, dont l’un raisonnerait, tandis que l’autre regarderait raisonner. L’organe observé et l’organe observateur étant, dans ce cas, identiques, comment l’observation pourrait-elle avoir lieu ?\par
Cette prétendue méthode psychologique est donc radicalement nulle dans son principe. Aussi, considérons à quels procédés profondément contradictoires elle conduit immédiatement ! D’un côté, on vous recommande de vous isoler, autant que possible, de toute sensation extérieure, il faut surtout vous interdire tout travail intellectuel ; car, si vous étiez seulement occupés à faire le calcul le plus simple, que deviendrait l’observation {\itshape intérieure} ? D’un autre côté, après avoir, enfin, à force de précautions, atteint cet état parfait de sommeil intellectuel, vous devez vous occuper à contempler les opérations qui s’exécuteront dans votre esprit lorsqu’il ne s’y passera plus rien Nos descendants verront sans doute de telles prétentions transportées un jour sur la scène.\par
Les résultats d’une aussi étrange manière de procéder sont parfaitement conformes au principe. Depuis deux mille ans que les métaphysiciens cultivent ainsi la psychologie, ils n’ont pu encore convenir d’une seule proposition intelligible et solidement arrêtée. Ils sont, même aujourd’hui, partagés en une multitude d’écoles qui disputent sans cesse sur les premiers éléments de leurs doctrines. {\itshape L’observation intérieure} engendre presque autant d’opinions divergentes qu’il y a d’individus croyant s’y livrer.\par
Les véritables savants, les hommes voués aux études positives, en sont encore à demander vainement à ces psychologues de citer une seule découverte réelle, grande ou petite, qui soit due à cette méthode si vantée. Ce n’est pas à dire pour cela que tous leurs travaux aient été absolument sans aucun résultat relativement aux progrès généraux de nos connaissances, indépendamment du service éminent qu’ils ont rendu en soutenant l’activité de notre intelligence, à l’époque où elle ne pouvait avoir d’aliment plus substantiel. Mais on peut affirmer que tout ce qui, dans leurs écrits, ne consiste pas, suivant la judicieuse expression d’un illustre philosophe positif (M. Cuvier), en métaphores prises pour des raisonnements, et présente quelque notion véritable, au lieu de provenir de leur prétendue méthode, a été obtenu par des observations effectives sur la marche de l’esprit humain, auxquelles a dû donner naissance, de temps à autre, le développement des sciences. Encore même, ces notions si clairsemées, proclamées avec tant d’emphase, et qui ne sont dues qu’à l’infidélité des psychologues à leur prétendue méthode, se trouvent-elles le plus souvent ou tort exagérées, ou très incomplètes, et bien inférieures aux remarques déjà faites sans ostentation par les savants sur les procédés qu’ils emploient. Il serait aisé d’en citer des exemples frappants, si je ne craignais d’accorder ici trop d’extension à une telle discussion : voyez, entre autres, ce qui est arrivé pour la théorie des signes.\par
(2) Les considérations que je viens d’indiquer relativement à la science logique, sont encore plus manifestes, quand on les transporte à l’art logique.\par
En effet, lorsqu’il s’agit, non seulement de savoir ce que c’est que la méthode positive, mais d’en avoir une connaissance assez nette et assez profonde pour en pouvoir faire un usage effectif, c’est en action qu’il faut la considérer ; ce sont les diverses grandes applications déjà vérifiées que l’esprit humain en a faites qu’il convient d’étudier. En un mot, ce n’est évidemment que par l’examen philosophique des sciences qu’il est possible d’y parvenir. La méthode n’est pas susceptible d’être étudiée séparément des recherches où elle est employée ; ou, du moins, ce n’est là qu’une étude morte, incapable de féconder l’esprit qui s’y livre. Tout ce qu’on en peut dire de réel, quand on l’envisage abstraitement, se réduit à des généralités tellement vagues, qu’elles ne sauraient avoir aucune influence sur le régime intellectuel. Lorsqu’on a bien établi, en thèse logique, que toutes nos connaissances doivent être fondées sur l’observation, que nous devons procéder tantôt des faits aux principes, et tantôt des principes aux faits, et quelques autres aphorismes semblables, on connaît beaucoup moins nettement la méthode que celui qui a étudié, d’une manière un peu approfondie, une seule science positive, même sans intention philosophique. C’est pour avoir méconnu ce fait essentiel, que nos psychologues sont conduits à prendre leurs rêveries pour de la science, croyant comprendre la méthode positive pour avoir lu les préceptes de Bacon ou le discours de Descartes.\par
J’ignore si, plus tard, il deviendra possible de faire a priori un véritable cours de méthode tout à fait indépendant de l’étude philosophique des sciences ; mais je suis bien convaincu que cela est inexécutable aujourd’hui, les grands procédés logiques ne pouvant encore être expliqués avec la précision suffisante séparément de leurs applications. J’ose ajouter, en outre, que, lors même qu’une telle entreprise pourrait être réalisée dans la suite, ce qui, en effet, se laisse concevoir, ce ne serait jamais néanmoins que par l’étude des applications régulières des procédés scientifiques qu’on pourrait parvenir à se former un bon système d’habitudes intellectuelles ; ce qui est pourtant le but essentiel de l’étude de la méthode. Je n’ai pas besoin d’insister davantage en ce moment sur un sujet qui reviendra fréquemment dans toute la durée de ce cours, et à l’égard duquel je présenterai spécialement de nouvelles considérations dans la prochaine leçon.\par
Tel doit être le premier grand résultat direct de la philosophie positive, la manifestation par expérience des lois que suivent dans leur accomplissement nos fonctions intellectuelles, et, par suite, la connaissance précise des règles générales convenables pour procéder sûrement à la recherche de la vérité.\par
(3) Une seconde conséquence, non moins importante, et d’un intérêt bien plus pressant, qu’est nécessairement destiné à produire aujourd’hui l’établissement de la philosophie positive définie dans ce discours, c’est de présider à la refonte générale de notre système d’éducation.\par
En effet, déjà les bons esprits reconnaissent unanimement la nécessité de remplacer notre éducation européenne, encore essentiellement théologique, métaphysique et littéraire, par une éducation positive, conforme à l’esprit de notre époque, et adaptée aux besoins de la civilisation moderne. Les tentatives variées qui se sont multipliées de plus en plus depuis, un siècle, particulièrement dans ces derniers temps, pour répandre et pour augmenter sans cesse l’instruction positive, et auxquelles les divers gouvernements européens se sont toujours associés avec empressement quand ils n’en ont pas pris l’initiative, témoignent assez que, de toutes parts, se développe le sentiment spontané de cette nécessité. Mais, tout en secondant autant que possible ces utiles entreprises, on ne doit pas se dissimuler que, dans l’état présent de nos idées, elles ne sont nullement susceptibles d’atteindre leur but principal, la régénération fondamentale de l’éducation générale. Car la spécialité exclusive, l’isolement trop prononce, qui caractérisent encore notre manière de concevoir et de cultiver les sciences, influent nécessairement à un haut degré sur la manière de les exposer dans l’enseignement. Qu’un bon esprit veuille aujourd’hui étudier les principales branches de la philosophie naturelle, afin de se former un système général d’idées positives, il sera obligé d’étudier séparément chacune d’elles d’après le même mode et dans le même détail que s’il voulait devenir spécialement ou astronome, ou chimiste, etc. ; ce qui rend une telle éducation presque impossible et nécessairement fort imparfaite, même pour les plus hautes intelligences placées dans les circonstances les plus favorables. Une telle manière de procéder serait donc tout à fait chimérique, relativement à l’éducation générale. Et néanmoins celle-ci exige absolument un ensemble de conceptions positives sur toutes les grandes classes de phénomènes naturels. C’est un tel ensemble qui doit devenir désormais, sur une échelle plus ou moins étendue, même dans les masses populaires, la base permanente de toutes les combinaisons humaines ; qui doit, en un mot, constituer l’esprit général de nos descendants. Pour que la philosophie naturelle puisse achever la régénération, déjà si préparée, de notre système intellectuel, il est donc indispensable que les différentes sciences dont elle se compose, présentées à toutes les intelligences comme les diverses branches d’un tronc unique, soient réduites d’abord à ce qui constitue leur esprit, c’est-à-dire à leurs méthodes principales et à leurs résultats les plus importants. Ce n’est qu’ainsi que l’enseignement des sciences peut devenir, parmi nous, la base d’une nouvelle éducation générale vraiment rationnelle. Qu’ensuite à cette instruction fondamentale s’ajoutent les diverses études scientifiques spéciales, correspondantes aux diverses éducations spéciales qui doivent succéder à l’éducation générale, cela ne peut évidemment être mis en doute. Mais la considération essentielle que j’ai voulu indiquer ici consiste en ce que toutes ces spécialités, même péniblement accumulées, seraient nécessairement insuffisantes pour renouveler réellement le système de notre éducation, si elles ne reposaient sur la base préalable de cet enseignement général, résultat direct de la philosophie positive définie dans ce discours.\par
(4) Non seulement l’étude spéciale des généralités scientifiques est destinée à réorganiser l’éducation, mais elle doit aussi contribuer aux progrès particuliers des diverses sciences positives ; ce qui constitue la troisième propriété fondamentale que je me suis proposé de signaler.\par
En effet, les divisions que nous établissons entre nos sciences, sans être arbitraires, comme quelques-uns le croient, sont essentiellement artificielles. En réalité, le sujet de toutes nos recherches est un ; nous ne le partageons que dans la vue de séparer les difficultés pour les mieux résoudre. Il en résulte plus d’une fois que, contrairement à nos répartitions classiques, des questions importantes exigeraient une certaine combinaison de plusieurs points de vue spéciaux, qui ne peut guère avoir lieu dans la constitution actuelle du monde savant ; ce qui expose à laisser ces problèmes sans solution beaucoup plus longtemps qu’il ne serait nécessaire. Un tel inconvénient doit se présenter surtout pour les doctrines les plus essentielles de chaque science positive en particulier. On en peut citer aisément des exemples très marquants, que je signalerai soigneusement, à mesure que le développement naturel de ce cours nous les présentera.\par
J’en pourrais citer, dans le passé, un exemple éminemment mémorable, en considérant l’admirable conception de Descartes relative à la géométrie analytique. Cette découverte fondamentale, qui a changé la face de la science mathématique, et dans laquelle on doit voir le véritable germe de tous les grands progrès ultérieurs, qu’est-elle autre chose que le résultat d’un rapprochement établi entre deux sciences conçues jusqu’alors d’une manière isolée ? Mais l’observation sera plus décisive en la faisant porter sur des questions encore pendantes.\par
Je me bornerai ici à choisir, dans la chimie, la doctrine si importante des proportions définies. Certainement, la mémorable discussion élevée de nos jours, relativement au principe fondamental de cette théorie, ne saurait encore, quelles que soient les apparences, être regardée comme irrévocablement terminée. Car ce n’est pas là, ce me semble, une simple question de chimie. Je crois pouvoir avancer que, pour obtenir à cet égard une décision vraiment définitive, c’est-à-dire pour déterminer si nous devons regarder comme une loi de la nature que les molécules se combinent nécessairement en nombres fixes, il serait indispensable de réunir le point de vue chimique avec le point de vue physiologique. Ce qui l’indique, c’est que, de l’aveu même des illustres chimistes qui ont le plus puissamment contribué à la formation de cette doctrine, on peut dire tout au plus qu’elle se vérifie constamment dans la composition des corps inorganiques ; mais elle se trouve au moins aussi constamment en défaut dans les composés organiques, auxquels il semble jusqu’à présent tout à fait impossible de l’étendre. Or, avant d’ériger cette théorie en un principe réellement fondamental, ne faudra-t-il pas d’abord s’être rendu compte de cette immense exception ? Ne tiendrait-elle pas à ce même caractère général, propre à tous les corps organisés, qui fait que, dans aucun de leurs phénomènes, il n’y a lieu à concevoir des nombres invariables ? Quoi qu’il en soit, un ordre tout nouveau de considérations, appartenant également à la chimie et à la physiologie, est évidemment nécessaire pour décider finalement, d’une manière quelconque, cette grande question de philosophie naturelle.\par
Je crois convenable d’indiquer encore ici un second exemple de même nature, mais qui, se rapportant à un sujet de recherches bien plus particulier, est encore plus concluant pour montrer l’importance spéciale de la philosophie positive dans la solution des questions qui exigent la combinaison de plusieurs sciences. Je le prends aussi dans la chimie. Il s’agit de la question, encore indécise, qui consiste à déterminer si l’azote doit être regardé, dans l’état présent de nos connaissances, comme un corps simple ou comme un corps composé. Vous savez par quelles considérations purement chimiques l’illustre Berzélius est parvenu à balancer l’opinion de presque tous les chimistes actuels, relativement à la simplicité de ce gaz. Mais ce que je ne dois pas négliger de faire particulièrement remarquer, c’est l’influence exercée à ce sujet sur l’esprit de Berzélius, comme il en fait lui-même le précieux aveu, par cette observation physiologique, que les animaux qui se nourrissent de matières non azotées renferment dans la composition de leurs tissus tout autant d’azote que les animaux carnivores. Il est clair, en effet, d’après cela, que, pour décider réellement si l’azote est ou non un corps simple, il faudra nécessairement faire intervenir la physiologie, et combiner, avec les considérations chimiques proprement dites, une série de recherches neuves sur la relation entre la composition des corps vivants et leur mode d’alimentation.\par
Il serait maintenant superflu de multiplier davantage les exemples de ces problèmes de nature multiple, qui ne sauraient être résolus que par l’intime combinaison de plusieurs sciences cultivées aujourd’hui d’une manière tout à fait indépendante. Ceux que je viens de citer suffisent pour faire sentir, en général, l’importance de la fonction que doit remplir dans le perfectionnement de chaque, science naturelle en particulier la philosophie positive, immédiatement destinée à organiser d’une manière permanente de telles combinaisons, qui ne pourraient se former convenablement sans elle.\par
(5) Enfin, une quatrième et dernière propriété fondamentale que je dois faire remarquer dès ce moment dans ce que j’ai appelé la philosophie positive, et qui doit sans doute lui mériter plus que toute autre l’attention générale, puisqu’elle est aujourd’hui la plus importante pour la pratique, c’est qu’elle peut être considérée comme la seule base solide de la réorganisation sociale qui doit terminer l’état de crise dans lequel se trouvent depuis si longtemps les nations les plus civilisées. La dernière partie de ce cours sera spécialement consacrée à établir cette proposition, en la développant dans toute son étendue. Mais l’esquisse générale du grand tableau que j’ai entrepris d’indiquer dans ce discours manquerait d’un de ses éléments les plus caractéristiques, si je négligeais de signaler ici une considération aussi essentielle.\par
Quelques réflexions bien simples suffiront pour justifier ce qu’une telle qualification paraît d’abord présenter de trop ambitieux.\par
Ce n’est pas aux lecteurs de cet ouvrage que je croirai jamais devoir prouver que les idées gouvernent et bouleversent le monde, ou, en d’autres termes, que tout le mécanisme social repose finalement sur des opinions. Ils savent surtout que la grande crise politique et morale des sociétés actuelles tient, en dernière analyse, à l’anarchie intellectuelle. Notre mal le plus grave consiste, en effet, dans cette profonde divergence qui existe maintenant entre tous les esprits relativement à toutes les maximes fondamentales dont la fixité est la première condition d’un véritable ordre social. Tant que les intelligences individuelles n’auront pas adhéré par un assentiment unanime à un certain nombre d’idées générales capables de former une doctrine sociale commune, on ne peut se dissimuler que l’état des nations restera, de toute nécessité, essentiellement révolutionnaire, malgré tous les palliatifs politiques qui pourront être adoptés, et ne comportera réellement que des institutions provisoires. Il est également certain que, si cette réunion des esprits dans une même communion de principes peut une fois être obtenue, les institutions, convenables en découleront nécessairement, sans donner lieu à aucune secousse grave, le plus grand désordre étant déjà dissipé par ce seul fait. C’est donc là que doit se porter principalement l’attention de tous ceux qui sentent l’importance d’un état de choses vraiment normal.\par
Maintenant, du point de vue élevé où nous ont placés graduellement les diverses considérations indiquées dans ce discours, il est aisé à la fois et de caractériser nettement dans son intime profondeur l’état présent des sociétés, et d’en déduire par quelle voie on peut le changer essentiellement. En me rattachant à la loi fondamentale énoncée au commencement de ce discours, je crois pouvoir résumer exactement toutes les observations relatives à la situation actuelle de la société en disant simplement que le désordre actuel des intelligences tient, en dernière analyse, à l’emploi simultané des trois philosophies radicalement incompatibles : la philosophie théologique, la philosophie métaphysique et la philosophie positive. Il est clair, en effet, que, si l’une quelconque de ces trois philosophies obtenait en réalité une prépondérance universelle et complète, il y aurait un ordre social déterminé, tandis que le mal consiste surtout dans l’absence de toute véritable organisation. C’est la coexistence de ces trois philosophies opposées qui empêche absolument de s’entendre sur aucun point essentiel. Or, si cette manière de voir est exacte, il ne s’agit plus que de savoir laquelle des trois philosophies peut et doit prévaloir par la nature des choses ; tout homme sensé devra ensuite, quelles qu’aient pu être, avant l’analyse de la question, ses opinions particulières, s’efforcer de concourir à son triomphe. La recherche étant une fois réduite à ces termes simples, elle ne paraît pas devoir rester longtemps incertaine ; car il est évident, par toutes sortes de raisons dont j’ai indiqué dans ce discours quelques-unes des principales, que la philosophie positive est seule destinée à prévaloir selon le cours ordinaire des choses. Seule elle a été, depuis une longue suite de siècles, constamment en progrès, tandis que ses antagonistes ont été constamment en décadence. Que ce soit à tort ou à raison, peu importe ; le fait général est incontestable, et il suffit. On peut le déplorer, mais non le détruire, ni par conséquent le négliger, sous peine de ne se livrer qu’à des spéculations illusoires. Cette révolution générale de l’esprit humain est aujourd’hui presque entièrement accomplie : il ne reste plus, comme je l’ai expliqué, qu’à compléter la philosophie positive en y comprenant l’étude des phénomènes sociaux, et ensuite à la résumer en un seul corps de doctrine homogène. Quand ce double travail sera suffisamment avancé, le triomphe définitif de la philosophie positive aura lieu spontanément, et rétablira l’ordre dans la société. La préférence si prononcée que presque tous les esprits, depuis les plus élevés jusqu’aux plus vulgaires, accordent aujourd’hui aux connaissances positives sur les conceptions vagues et mystiques, présage assez l’accueil que recevra cette philosophie, lorsqu’elle aura acquis la seule qualité qui lui manque encore, un caractère de généralité convenable.\par
En résumé, la philosophie théologique et la philosophie métaphysique se disputent aujourd’hui la tâche, trop supérieure aux forces de l’une et de l’autre, de réorganiser la société ; c’est entre elles seules que subsiste encore la lutte, sous ce rapport. La philosophie positive n’est intervenue jusqu’ici dans la contestation que pour les critiquer toutes deux, et elle s’en est assez bien acquittée pour les discréditer entièrement. Mettons-la enfin en état de prendre un rôle actif, sans nous inquiéter plus longtemps de débats devenus inutiles. Complétant la vaste opération intellectuelle commencée par Bacon, par Descartes et par Galilée, construisons directement le système d’idées générales que cette philosophie est désormais destinée à faire indéfiniment prévaloir dans l’espèce humaine, et la crise révolutionnaire qui tourmente les peuples civilisés sera essentiellement terminée.\par
Tels sont les quatre points de vue principaux sous lesquels j’ai cru devoir indiquer dès ce moment l’influence salutaire de la philosophie positive, pour servir de complément essentiel à la définition générale que j’ai essayé d’en exposer
\section[{VII.}]{VII.}
\noindent (1) Avant de terminer, je désire appeler un instant l’attention sur une dernière réflexion qui me semble convenable pour éviter, autant que possible, qu’on se forme d’avance une opinion erronée de la nature de ce cours.\par
En assignant pour but à la philosophie positive de résumer en un seul corps de doctrine homogène l’ensemble des connaissances acquises, relativement aux différents ordres de phénomènes naturels, il était loin de ma pensée de vouloir procéder à l’étude générale de ces phénomènes en les considérant tous comme des effets divers d’un principe unique, comme assujettis à une seule et même loi. Quoique je doive traiter spécialement cette question dans la prochaine leçon, je crois devoir, dès à présent, en faire la déclaration, afin de prévenir les reproches très mal fondés que pourraient m’adresser ceux qui, sur un faux aperçu, classeraient ce cours parmi ces tentatives d’explication universelle qu’on voit éclore journellement de la part d’esprits entièrement étrangers aux méthodes et aux connaissances scientifiques. Il ne s’agit ici de rien de semblable ; et le développement de ce cours en fournira la preuve manifeste à tous ceux chez lesquels les éclaircissements contenus dans ce discours auraient pu laisser quelques doutes à cet égard.\par
(2) Dans ma profonde conviction personnelle, je considère ces entreprises d’explication universelle de tous les phénomènes par une loi unique comme éminemment chimériques, même quand elles sont tentées par les intelligences les plus compétentes. Je crois que les moyens de l’esprit humain sont trop faibles, et l’univers trop compliqué pour qu’une telle perfection scientifique soit jamais à notre portée, et je pense, d’ailleurs, qu’on se forme généralement une idée très exagérée des avantages qui en résulteraient nécessairement, si elle était possible. Dans tous les cas, il me semble évident que, vu l’état présent de nos connaissances, nous en sommes encore beaucoup trop loin pour que de telles tentatives puissent être raisonnables avant un laps de temps considérable. Car, si on pouvait espérer d’y parvenir, ce ne pourrait être, suivant moi, qu’en rattachant tous les phénomènes naturels à la loi positive la plus générale que nous connaissions, la loi de la gravitation, qui lie déjà tous les phénomènes astronomiques à une partie de ceux de la physique terrestre. Laplace a exposé effectivement une conception par laquelle on pourrait ne voir dans les phénomènes chimiques que de simples effets moléculaires de l’attraction newtonienne, modifiée par la figure et la position mutuelle des atomes. Mais, outre l’indétermination dans laquelle resterait probablement toujours cette conception, par l’absence des données essentielles relatives à la constitution intime des corps, il est presque certain que la difficulté de l’appliquer serait telle, qu’on serait obligé de maintenir, comme artificielle, la division aujourd’hui établie comme naturelle entre l’astronomie et la chimie. Aussi Laplace n’a-t-il présenté cette idée que comme un simple jeu philosophique, incapable d’exercer réellement aucune influence utile sur les progrès de la science chimique. Il y a plus, d’ailleurs ; car, même en supposant vaincue cette insurmontable difficulté, on n’aurait pas encore atteint à l’unité scientifique, puisqu’il faudrait ensuite tenter de rattacher à la même loi l’ensemble des phénomènes physiologiques ; ce qui, certes, ne serait pas la partie la moins difficile de l’entreprise. Et néanmoins, l’hypothèse que nous venons de parcourir serait, tout bien considéré, la plus favorable à cette unité si désirée.\par
(3) je n’ai pas besoin de plus grands détails pour achever de convaincre que le but de ce cours n’est nullement de présenter tous les phénomènes naturels comme étant au fond identiques, sauf la variété des circonstances. La philosophie positive serait sans doute plus parfaite s’il pouvait en être ainsi. Mais cette condition n’est nullement nécessaire à sa formation systématique, non plus qu’à la réalisation des grandes et heureuses conséquences que nous l’avons vue destinée à produire. Il n’y a d’unité indispensable pour cela que l’unité de méthode, laquelle peut et doit évidemment exister, et se trouve déjà établie en majeure partie. Quant à la doctrine, il n’est pas nécessaire qu’elle soit une ; il suffit qu’elle soit homogène. C’est donc sous le double point de vue de l’unité des méthodes et de l’homogénéité des doctrines que nous considérerons, dans ce cours, les différentes classes de théories positives. Tout en tendant à diminuer, le plus Possible, le nombre des lois générales nécessaires à l’explication positive des phénomènes naturels, ce qui est, en effet, le but philosophique de la science, nous regarderons comme téméraire d’aspirer jamais, même pour l’avenir le plus éloigné, à les réduire rigoureusement à une seule.\par
J’ai tenté, dans ce discours, de déterminer, aussi exactement qu’il a été en mon pouvoir, le but, l’esprit et l’influence de la philosophie positive. J’ai donc marqué le terme vers lequel ont toujours tendu et tendront sans cesse tous mes travaux, soit dans ce cours, soit de toute autre manière. Personne n’est plus profondément convaincu que moi de l’insuffisance de mes forces intellectuelles, fussent-elles même très supérieures à leur valeur réelle, pour répondre à une tâche aussi vaste et aussi élevée. Mais ce qui ne peut être fait ni par un seul esprit, ni en une seule vie, un seul peut le proposer nettement : telle est toute mon ambition.\par
Ayant exposé le véritable but de ce cours, c’est-à-dire fixé le point de vue sous lequel je considérerai les diverses branches principales de la philosophie naturelle, je compléterai, dans la leçon prochaine, ces prolégomènes généraux en passant à l’exposition du plan, c’est-à-dire à la détermination de l’ordre encyclopédique qu’il convient d’établir entre les diverses classes des phénomènes naturels, et par conséquent entre les sciences positives correspondantes.
\chapterclose


\chapteropen
\chapter[{Deuxième leçon}]{Deuxième leçon}\phantomsection
\label{leçon\_2}\renewcommand{\leftmark}{Deuxième leçon}

\begin{center}\emph{Exposition du plan de ce cours, ou considérations générales sur la hiérarchie des sciences positives.}\end{center}

\chaptercont
\section[{I.}]{I.}
\noindent Après avoir caractérisé aussi exactement que possible, dans la leçon précédente, les considérations à présenter dans Ce Cours sur toutes les branches principales de la philosophie naturelle, il faut déterminer maintenant le plan que nous devons suivre, c’est-à-dire la classification rationnelle la plus convenable à établir entre les différentes sciences positives fondamentales, pour les étudier successivement sous le point de vue que nous avons fixé. Cette seconde discussion générale est indispensable pour achever de faire connaître dès l’origine le véritable esprit de ce cours.\par
(1) On conçoit aisément d’abord qu’il ne s’agit pas ici de faire la critique, malheureusement trop facile, des nombreuses classifications qui ont été proposées successivement depuis deux siècles, pour le système général des connaissances humaines, envisagé dans toute son étendue. On est aujourd’hui bien convaincu que toutes les échelles encyclopédiques construites, comme celles de Bacon et de d’Alembert, d’après une distinction quelconque des diverses facultés de l’esprit humain, sont par cela seul radicalement vicieuses, même quand cette distinction n’est pas, comme il arrive souvent, plus subtile que réelle ; car, dans, chacune de ses sphères d’activité, notre entendement emploie simultanément toutes ses facultés principales. Quant à toutes les autres classifications proposées, il suffira d’observer que les différentes discussions élevées à ce sujet ont eu pour résultat définitif de montrer dans chacune des vices fondamentaux, tellement qu’aucune n’a pu obtenir un assentiment unanime, et qu’il existe à cet égard presque autant d’opinions que d’individus. Ces diverses tentatives ont même été, en général, si mal conçues, qu’il en est résulté involontairement, dans la plupart des bons esprits, une prévention défavorable contre toute entreprise de ce genre.\par
(2) Sans nous arrêter davantage sur un fait si bien constaté, il est plus essentiel d’en rechercher la cause. Or, on peut aisément s’expliquer la profonde imperfection de ces tentatives encyclopédiques, si souvent renouvelées jusqu’ici. Je n’ai pas besoin de faire observer que, depuis le discrédit général dans lequel sont tombés les travaux de cette nature par suite du peu de solidité des premiers projets, ces classifications ne sont conçues le plus souvent que par des esprits presque entièrement étrangers à la connaissance des objets à classer. Sans avoir égard à cette considération personnelle, il en est une beaucoup plus importante, puisée dans la nature même du sujet, et qui montre clairement pourquoi il n’a pas été possible jusqu’ici de s’élever à une conception encyclopédique véritablement satisfaisante. Elle consiste dans le défaut d’homogénéité qui a toujours existé jusqu’à ces derniers temps entre les différentes parties du système intellectuel, les unes étant successivement devenues positives, tandis que les autres restaient théologiques ou métaphysiques. Dans un état de choses aussi incohérent, il était évidemment impossible d’établir aucune classification rationnelle. Comment parvenir à disposer, dans un système unique, des conceptions aussi profondément contradictoires ? C’est une difficulté contre laquelle sont venus échouer nécessairement tous les classificateurs, sans qu’aucun l’ait aperçue distinctement. Il était bien sensible néanmoins, pour quiconque eût bien connu la véritable situation de l’esprit humain, qu’une telle entreprise était prématurée, et qu’elle ne pourrait être tentée avec succès que lorsque toutes nos conceptions principales seraient devenues positives.\par
(3) Cette condition fondamentale pouvant maintenant être regardée comme remplie, d’après les explications données dans la leçon précédente, il est dès lors possible de procéder à une disposition vraiment rationnelle et durable d’un système dont toutes les parties sont enfin devenues homogènes.\par
D’un autre côté, la théorie générale des classifications établie dans ces derniers temps par les travaux philosophiques des botanistes et des zoologistes permet d’espérer un succès réel dans un semblable travail, en nous offrant un guide certain par le véritable principe fondamental de l’art de classer, qui n’avait jamais été conçu distinctement jusqu’alors. Ce principe est une conséquence nécessaire de la seule application directe de la méthode positive à la question même des classifications, qui, comme toute autre, doit être traitée par observation, au lieu d’être résolue par des considérations a priori. Il consiste en ce que la classification doit ressortir de l’étude même des objets à classer, et être déterminée par les affinités réelles et l’enchaînement naturel qu’ils présentent, de telle sorte que cette classification soit elle-même l’expression du fait le plus général, manifesté par la comparaison approfondie des objets qu’elle embrasse.\par
Appliquant cette règle fondamentale au cas actuel, c’est donc d’après la dépendance mutuelle qui a lieu effectivement entre les diverses sciences positives, que nous devons procéder à leur classification ; et cette dépendance, pour être réelle, ne peut résulter que de celle des phénomènes correspondants.
\section[{II.}]{II.}
\noindent Mais, avant d’exécuter, dans un tel esprit d’observation, cette importante opération encyclopédique, il est indispensable, pour ne pas nous égarer dans un travail trop étendu, de circonscrire avec plus de précision que nous ne l’avons fait jusqu’ici le sujet propre de la classification proposée.\par
(1) Tous les travaux humains sont, ou de spéculation, ou d’action. Ainsi, la division la plus générale de nos connaissances réelles consiste à les distinguer en théoriques et pratiques. Si nous considérons d’abord cette première division, il est évident que c’est seulement des connaissances théoriques qu’il doit être question dans un cours de la nature de celui-ci ; car il ne s’agit point d’observer le système entier des notions humaines mais uniquement celui des conceptions fondamentales sur les divers ordres de phénomènes, qui fournissent une base solide à toutes nos autres combinaisons quelconques, et qui ne sont, à leur tour, fondées sur aucun système intellectuel antécédent. Or, dans un tel travail, c’est la spéculation qu’il faut considérer, et non l’application, si ce n’est en tant que celle-ci peut éclaircir la première. C’est là probablement ce qu’entendait Bacon, quoique fort imparfaitement, par cette {\itshape philosophie première} qu’il indique comme devant être extraite de l’ensemble des sciences, et qui a été si diversement et toujours si étrangement conçue par les métaphysiciens qui ont entrepris de commenter sa pensée.\par
(2) Sans doute, quand on envisage l’ensemble complet des travaux de tout genre de l’espèce humaine, on doit concevoir l’étude de la nature comme destinée à fournir la véritable base rationnelle de l’action de l’homme sur la nature, puisque la connaissance des lois des phénomènes, dont le résultat constant est de nous les faire prévoir, peut seule évidemment nous conduire, dans la vie active, à les modifier à notre avantage les uns par les autres. Nos moyens naturels et directs pour agir sur les corps qui nous entourent sont extrêmement faibles et tout à fait disproportionnés à nos besoins. Toutes les fois que nous parvenons à exercer une grande action, c’est seulement parce que la connaissance des lois naturelles nous permet d’introduire, parmi les circonstances déterminées sous l’influence desquelles s’accomplissent les divers phénomènes, quelques éléments modificateurs, qui, quelque faibles qu’ils soient en eux-mêmes, suffisent, dans certains cas, pour faire tourner à notre satisfaction les résultats définitifs de l’ensemble des causes extérieures. En résumé, {\itshape science, d’où prévoyance ; prévoyance, d’où action} : telle est la formule très simple qui exprime, d’une manière exacte, la relation générale de la {\itshape science} et de {\itshape l’art}, en prenant ces deux expressions dans leur acception totale.\par
Mais, malgré l’importance capitale de cette relation, qui ne doit jamais être méconnue, ce serait se former des sciences une idée bien imparfaite que de les concevoir seulement comme les bases des arts, et c’est à quoi malheureusement on n’est que trop enclin de nos jours. Quels que soient les immenses services rendus à {\itshape l’industrie} par les théories scientifiques, quoique, suivant l’énergique expression de Bacon, la puissance soit nécessairement proportionnée à la connaissance, nous ne devons pas oublier que les sciences ont, avant tout, une destination plus directe et plus élevée, celle de satisfaire au besoin fondamental qu’éprouve notre intelligence de connaître les lois des phénomènes. Pour sentir combien ce besoin est profond et impérieux, il suffit de penser un instant aux effets physiologiques de {\itshape l’étonnement}, et de considérer que la sensation la plus terrible que nous puissions éprouver est celle qui se produit toutes les fois qu’un phénomène nous semble s’accomplir contradictoirement aux lois naturelles qui nous sont familières. Ce besoin de disposer les faits dans un ordre que nous puissions concevoir avec facilité (ce qui est l’objet propre de toutes les théories scientifiques) est tellement inhérent à notre organisation, que, si nous ne parvenions pas à le satisfaire par des conceptions positives, nous retournerions inévitablement aux explications théologiques et métaphysiques auxquelles il a primitivement donné naissance, comme je l’ai exposé dans la dernière leçon.\par
(3) J’ai cru devoir signaler expressément dès ce moment une considération qui se reproduira fréquemment dans toute la suite de ce cours, afin d’indiquer la nécessité de se prémunir contre la trop grande influence des habitudes actuelles, qui tendent à empêcher qu’on se forme des idées justes et nobles de l’importance et de la destination des sciences. Si la puissance prépondérante de notre organisation ne corrigeait, même involontairement, dans l’esprit des savants, ce qu’il y a sous ce rapport d’incomplet et d’étroit dans la tendance générale de notre époque, l’intelligence humaine, réduite à ne s’occuper que de recherches susceptibles d’une utilité pratique immédiate, se trouverait par cela seul, comme l’a très justement remarqué Condorcet, tout à fait arrêtée dans ses progrès, même à l’égard de ces applications auxquelles on aurait imprudemment sacrifié les travaux purement spéculatifs ; car les applications les plus importantes dérivent constamment de théories formées dans une simple intention scientifique, et qui souvent ont été cultivées pendant plusieurs siècles sans produire aucun résultat pratique. On en peut citer un exemple bien remarquable dans les belles spéculations des géomètres grecs sur les sections coniques, qui, après une longue suite de générations, ont servi, en déterminant la rénovation de l’astronomie, à conduire finalement l’art de la navigation au degré de perfectionnement qu’il a atteint dans ces derniers temps, et auquel il ne serait jamais parvenu sans les travaux si purement théoriques d’Archimède et d’Apollonius ; tellement que Condorcet a pu dire avec raison à cet égard : \emph{« Le matelot, qu’une exacte observation de la longitude préserve du naufrage, doit la vie à une théorie conçue, deux mille ans auparavant, par des hommes de génie qui avaient en vue de simples spéculations géométriques. »}\par
Il est donc évident qu’après avoir conçu d’une manière générale l’étude de la nature comme servant de base rationnelle à l’action sur la nature, l’esprit humain doit procéder aux recherches théoriques, en faisant complètement abstraction de toute considération pratique ; car nos moyens pour découvrir la vérité sont tellement faibles que, si nous ne les concentrions pas exclusivement vers ce but, et si, en cherchant la vérité, nous nous imposions en même temps la condition étrangère d’y trouver une utilité pratique immédiate, il nous serait presque toujours impossible d’y parvenir.\par
(4) Quoi qu’il en soit, il est certain que l’ensemble de nos connaissances sur la nature, et celui des procédés que nous en déduisons pour la modifier à notre avantage, forment deux systèmes essentiellement distincts par eux-mêmes, qu’il est convenable de concevoir et de cultiver séparément. En outre, le premier système étant la base du second, c’est évidemment celui qu’il convient de considérer d’abord dans une étude méthodique, même quand on se proposerait d’embrasser la totalité des connaissances humaines, tant d’application que de spéculation. Ce système théorique me paraît devoir constituer exclusivement aujourd’hui le sujet d’un cours vraiment rationnel de philosophie positive ; c’est ainsi du moins que je le conçois., Sans doute, il serait possible d’imaginer un cours plus étendu, portant à la fois sur les généralités théoriques et sur les généralités pratiques. Mais je ne pense pas qu’une telle entreprise, même indépendamment de son étendue, puisse être convenablement tentée dans l’état présent de l’esprit humain. Elle me semble, en effet, exiger préalablement un travail très important et d’une nature toute particulière, qui n’a pas encore été fait, celui de former, d’après les théories scientifiques proprement dites, les conceptions spéciales destinées à servir de bases directes aux procédés généraux de la pratique.\par
Au degré de développement déjà atteint par notre intelligence, ce n’est pas immédiatement que les sciences s’appliquent aux arts, du moins dans les cas les plus parfaits il existe entre ces deux ordres d’idées un ordre moyen, qui, encore mal déterminé dans son caractère philosophique, est déjà plus sensible quand on considère la classe sociale qui s’en occupe spécialement. Entre les savants proprement dits et les directeurs effectifs des travaux productifs, il commence à se former de nos jours une classe intermédiaire, celle des {\itshape ingénieurs}, dont la destination spéciale est d’organiser les relations de la théorie et de la pratique. Sans avoir aucunement en vue le progrès des connaissances scientifiques, elle les considère dans leur état présent pour en déduire les applications industrielles dont elles sont susceptibles. Telle est du moins la tendance naturelle des choses, quoiqu’il y ait encore à cet égard beaucoup de confusion. Le corps de doctrine propre à cette classe nouvelle, et qui doit constituer les véritables théories directes des différents arts, pourrait sans doute donner lieu à des considérations philosophiques d’un grand intérêt et d’une importance réelle. Mais un travail qui les embrasserait conjointement avec celles fondées sur les sciences proprement dites, serait aujourd’hui tout à fait prématuré ; car ces doctrines intermédiaires entre la théorie pure et la pratique directe ne sont point encore formées ; il n’en existe jusqu’ici que quelques éléments imparfaits, relatifs aux sciences et aux arts les plus avancés, et qui permettent seulement de concevoir la nature et la possibilité de semblables travaux pour l’ensemble des opérations humaines. C’est ainsi, pour en citer l’exemple le plus important, qu’on doit envisager la belle conception de Monge, relativement à la géométrie descriptive, qui n’est réellement autre chose qu’une théorie générale des arts de construction. J’aurai soin d’indiquer successivement le petit nombre d’idées analogues déjà formées et d’en faire apprécier l’importance, à mesure que le développement naturel de ce cours les présentera. Mais il est clair que des conceptions jusqu’à présent aussi incomplètes ne doivent point entrer, comme partie essentielle, dans un cours de philosophie positive qui ne doit comprendre, autant que possible, que des doctrines ayant un caractère fixe et nettement déterminé.\par
(5) On concevra d’autant mieux la difficulté de construire ces doctrines intermédiaires que je viens d’indiquer, si l’on considère que chaque art dépend non seulement d’une certaine science correspondante, mais à la fois de plusieurs, tellement que les arts les plus importants empruntent des secours directs à presque toutes les diverses sciences principales. C’est ainsi que la véritable théorie de l’agriculture, pour me borner au cas le plus essentiel, exige une intime combinaison de connaissances physiologiques, chimiques, physiques et même astronomiques et mathématiques : il en est de même des beaux-arts. On aperçoit aisément, d’après cette considération, pourquoi ces théories n’ont pu encore être formées, puisqu’elles supposent le développement préalable de toutes les différentes sciences fondamentales. Il en résulte également un nouveau motif de ne pas comprendre un tel ordre d’idées dans un cours de philosophie positive, puisque, loin de pouvoir contribuer à la formation systématique de cette philosophie, les théories générales propres aux différents arts principaux doivent, au contraire, comme nous le voyons, être vraisemblablement plus tard une des conséquences les plus utiles de sa construction.\par
En résume, nous ne devons donc considérer dans ce cours que les théories scientifiques et nullement leurs applications. Mais, avant de procéder à la classification méthodique de ses différentes parties, il me reste à exposer, relativement aux sciences proprement dites, une distinction importante, qui achèvera de circonscrire nettement le sujet propre de l’étude que nous entreprenons.\par
(6) Il faut distinguer, par rapport à tous les ordres de phénomènes, deux genres de sciences naturelles : les unes abstraites, générales, ont pour objet la découverte des lois qui régissent les diverses classes de phénomènes, en considérant tous les cas qu’on peut concevoir ; les autres concrètes, particulières, descriptives, et qu’on désigne quelquefois sous le nom de sciences naturelles proprement dites, consistent dans l’application de ces lois à l’histoire effective des différents êtres existants. Les premières sont donc fondamentales, c’est sur elles seulement que porteront nos études dans ce cours ; les autres, quelle que soit leur importance propre, ne sont réellement que secondaires, et ne doivent point, par conséquent, faire partie d’un travail que son extrême étendue naturelle nous oblige à réduire au moindre développement possible.\par
La distinction précédente ne peut présenter aucune obscurité aux esprits qui ont quelque connaissance spéciale des différentes sciences positives, puisqu’elle est à peu près l’équivalent de Celle qu’on énonce ordinairement dans presque tous les traités scientifiques, en comparant la physique dogmatique à l’histoire naturelle proprement dite. Quelques exemples suffiront d’ailleurs pour rendre sensible cette division, dont l’importance n’est pas encore convenablement appréciée.\par
On pourra d’abord l’apercevoir très nettement en comparant, d’une part, la physiologie générale, et d’une autre part, la zoologie et la botanique proprement dites. Ce sont évidemment, en effet, deux travaux d’un caractère fort distinct, que d’étudier, en général, les lois de la vie, ou de déterminer le mode d’existence de chaque corps vivant, en particulier. Cette seconde étude, en outre, est nécessairement fondée sur la première.\par
Il en est de même de la chimie, par rapport à la minéralogie ; la première est évidemment la base rationnelle de la seconde. Dans la chimie, on considère toutes les combinaisons possibles des molécules, et dans toutes les circonstances imaginables ; dans la minéralogie, on considère seulement celles de ces combinaisons qui se trouvent réalisées dans la constitution effective du globe terrestre, et sous l’influence des seules circonstances qui lui sont propres. Ce qui montre clairement la différence du point de vue chimique et du point de vue minéralogiques quoique les deux sciences portent sur les mêmes objets, c’est que la plupart des faits envisagés dans la première n’ont qu’une existence artificielle, de telle manière qu’un corps, comme le chlore ou le potassium, pourra avoir une extrême importance en Chimie par l’étendue et l’énergie de ses affinités, tandis qu’il n’en aura presque aucune en minéralogie ; et réciproquement, un composé, tel que le granit ou le quartz, sur lequel porte la majeure partie des considérations minéralogiques, n’offrira, sous le rapport chimique, qu’un intérêt très médiocre.\par
Ce qui rend, en général, plus sensible encore la nécessité logique de cette distinction fondamentale entre les deux grandes sections de la philosophie naturelle, c’est que non seulement chaque section de la physique concrète suppose la culture préalable de la section correspondante de la physique abstraite, mais qu’elle exige même la connaissance des lois générales relatives à tous les ordres de phénomènes. Ainsi, par exemple, non seulement l’étude spéciale de la terre, considérée sous tous les points de vue qu’elle peut présenter effectivement, exige la connaissance préalable de la physique et de la chimie, mais elle ne peut être faite convenablement, sans y introduire, d’une part, les connaissances astronomiques, et même, d’une autre part, les connaissances physiologiques ; en sorte qu’elle tient au système entier des sciences fondamentales. Il en est de même de chacune des sciences naturelles proprement dites. C’est précisément pour ce motif que la {\itshape physique concrète} a fait jusqu’à présent si peu de progrès réels, car elle n’a pu commencer à être étudiée d’une manière vraiment rationnelle qu’après la {\itshape physique abstraite}, et lorsque toutes les diverses branches principales de celle-ci eurent pris leur caractère définitif, ce qui n’a eu lieu que de nos jours. Jusqu’alors on n’a pu recueillir à ce sujet que des matériaux plus ou moins incohérents, qui sont même encore fort incomplets. Les faits connus ne pourront être coordonnés de manière à former de véritables théories spéciales des différents êtres de l’univers, que lorsque la distinction fondamentale rappelée ci-dessus sera plus profondément sentie et plus régulièrement organisée, et que, par suite, les savants particulièrement livres à l’étude des sciences naturelles proprement dites auront reconnu la nécessité de fonder leurs recherches sur une connaissance approfondie de toutes les sciences fondamentales, condition qui est encore aujourd’hui fort loin d’être convenablement remplie.\par
L’examen de cette condition confirme nettement pourquoi nous devons, dans ce cours de philosophie positive, réduire nos considérations à l’étude des sciences générales, sans embrasser en même temps les sciences descriptives ou particulières. On voit naître ici, en effet, une nouvelle propriété essentielle de cette étude propre des généralités de physique abstraite ; c’est de fournir la base rationnelle d’une physique concrète vraiment systématique. Ainsi, dans l’état présent de l’esprit humain, il y aurait une sorte de contradiction à vouloir réunir, dans un seul et même cours, les deux ordres de sciences. On peut dire, de plus, que, quand même la physique concrète aurait déjà atteint le degré de perfectionnement de la physique abstraite, et que, par suite, il serait possible, dans un cours de philosophie positive, d’embrasser à la fois l’une et l’autre, il n’en faudrait pas moins évidemment commencer par la section abstraite, qui restera la base invariable de l’autre. Il est clair, d’ailleurs, que la seule étude des généralités des sciences fondamentales est assez vaste par elle-même, pour qu’il importe d’en écarter, autant que possible, toutes les considérations qui ne sont pas indispensables ; or, celles relatives aux sciences secondaires seront toujours, quoi qu’il arrive, d’un genre distinct. La philosophie des sciences fondamentales, présentant un système de conceptions positives sur tous nos ordres de connaissances réelles, suffit, par cela même, pour constituer cette {\itshape philosophie première} que cherchait Bacon, et qui, étant destinée à servir désormais de base permanente à toutes les spéculations humaines, doit être soigneusement réduite à la plus simple expression possible.\par
Je n’ai pas besoin d’insister davantage en ce moment sur une telle discussion, que j’aurai naturellement plusieurs occasions de reproduire dans les diverses parties de ce cours. L’explication précédente est assez développée pour motiver la manière dont j’ai circonscrit le sujet général de nos considérations.\par
Ainsi, en résultat de tout ce qui vient d’être exposé dans cette leçon, nous voyons : 1° que la science humaine se composant, dans son ensemble, de connaissances spéculatives et de connaissances d’application, c’est seulement des premières que nous devons nous occuper ici ; 2° que les connaissances théoriques ou les sciences proprement dites, se divisant en sciences générales et sciences particulières, nous devons ne considérer ici que le premier ordre, et nous borner à la physique abstraite, quelque intérêt que puisse nous présenter la physique concrète.
\section[{III.}]{III.}
\noindent Le sujet propre de ce cours étant par là exactement circonscrit, il est facile maintenant de procéder à une classification rationnelle vraiment satisfaisante des sciences fondamentales, ce qui constitue la question encyclopédique, objet de cette leçon.\par
(1) Il faut, avant tout, commencer par reconnaître que, quelque naturelle que puisse être une telle classification, elle renferme toujours nécessairement quelque chose, sinon d’arbitraire, du moins d’artificiel, de manière à présenter une imperfection véritable.\par
En effet, le but principal que l’on doit avoir en vue dans tout travail encyclopédique, c’est de disposer les sciences dans l’ordre de leur enchaînement naturel, en suivant leur dépendance mutuelle ; de telle sorte qu’on puisse les exposer successivement, sans jamais être entraîné dans le moindre cercle vicieux. Or, c’est une condition qu’il me paraît impossible d’accomplir d’une manière tout à fait rigoureuse. Qu’il me soit permis de donner ici quelque développement à cette réflexion, que je crois importante pour caractériser la véritable difficulté de la recherche qui nous occupe actuellement. Cette considération, d’ailleurs, me donnera lieu d’établir, relativement à l’exposition de nos connaissances, un principe général dont j’aurai plus tard à présenter de fréquentes applications.\par
(2) Toute science peut être exposée suivant deux marches essentiellement distinctes, dont tout autre mode d’exposition ne saurait être qu’une combinaison, la marche {\itshape historique} et la marche {\itshape dogmatique.}\par
Par le premier procédé, on expose successivement les connaissances dans le même ordre effectif suivant lequel l’esprit humain les a réellement obtenues, et en adoptant, autant que possible, les mêmes voies.\par
Par le second, on présente le système des idées tel qu’il pourrait être conçu aujourd’hui par un seul esprit, qui, placé au point de vue convenable, et pourvu des connaissances suffisantes, s’occuperait à refaire la science dans son ensemble.\par
Le premier mode est évidemment celui par lequel commence, de toute nécessité, l’étude de chaque science naissante ; car il présente cette propriété, de n’exiger, pour l’exposition des connaissances, aucun nouveau travail distinct de celui de leur formation, toute la didactique se réduisant alors à étudier successivement, dans l’ordre chronologique, les divers ouvrages originaux qui ont contribué aux progrès de la science.\par
Le mode dogmatique, supposant, au contraire, que tous ces travaux particuliers ont été refondus en un système général, pour être présentés suivant un ordre logique plus naturel, n’est applicable qu’à une science déjà parvenue à un assez haut degré de développement. Mais à mesure que la science fait des progrès, l’ordre historique d’exposition devient de plus en plus impraticable, par la trop longue suite d’intermédiaires qu’il obligerait l’esprit à parcourir ; tandis que l’ordre {\itshape dogmatique} devient de plus en plus possible, en même temps que nécessaire, parce que de nouvelles conceptions permettent de présenter les découvertes antérieures sous un point de vue plus direct.\par
C’est ainsi, par exemple, que l’éducation d’un géomètre de l’antiquité consistait simplement dans l’étude successive du très petit nombre de traités originaux produits jusqu’alors sur les diverses parties de la géométrie, ce qui se réduisait essentiellement aux écrits d’Archimède et d’Apollonius ; tandis qu’au contraire, un géomètre moderne a communément terminé son éducation, sans avoir lu un seul ouvrage original, excepté relativement aux découvertes les plus récentes, qu’on ne peut connaître que par ce moyen.\par
La tendance constante de l’esprit humain, quant à l’exposition des connaissances, est donc de substituer de plus en plus à l’ordre historique l’ordre dogmatique, qui peut seul convenir à l’état perfectionné de notre intelligence.\par
Le problème général de l’éducation intellectuelle consiste à faire parvenir, en peu d’années, un seul entendement, le plus souvent médiocre, au même point de développement qui a été atteint, dans une longue suite de siècles par un grand nombre de génies supérieurs appliquant successivement, pendant leur vie entière, toutes leurs forces à l’étude d’un même sujet. Il est clair, d’après cela, que, quoiqu’il soit infiniment plus facile et plus court d’apprendre que d’inventer, il serait certainement impossible d’atteindre le but proposé si l’on voulait assujettir chaque esprit individuel à passer successivement par les mêmes intermédiaires qu’a dû suivre nécessairement le génie collectif de l’espèce humaine. De là, l’indispensable besoin de l’ordre dogmatique, qui est surtout si sensible aujourd’hui pour les sciences les plus avancées, dont le mode ordinaire d’exposition ne présente plus presque aucune trace de la filiation effective de leurs détails.\par
(3) Il faut néanmoins ajouter, pour prévenir toute exagération, que tout mode réel d’exposition est, inévitablement, une certaine combinaison de l’ordre dogmatique avec l’ordre historique, dans laquelle seulement le premier doit dominer constamment et de plus en plus. L’ordre dogmatique ne peut, en effet, être suivi d’une manière tout à fait rigoureuse ; car, par cela même qu’il exige une nouvelle élaboration des connaissances acquises, il n’est point applicable, à chaque époque de la science, aux parties récemment formées dont l’étude ne comporte qu’un ordre essentiellement historique, lequel ne présente pas d’ailleurs, dans ce cas, les inconvénients principaux qui le font rejeter en général.\par
La seule imperfection fondamentale qu’on pourrait reprocher au mode dogmatique, c’est de laisser ignorer la manière dont se sont formées les diverses connaissances humaines ce qui, quoique distinct de l’acquisition même de ces connaissances, est, en soi du plus haut intérêt pour tout esprit philosophique. Cette considération aurait à mes yeux, beaucoup de poids, si elle était réellement un motif en faveur de l’ordre historique. Mais il est aisé de voir qu’il n’y a qu’une relation apparente entre étudier une science en suivant le mode dit {\itshape historique}, et connaître véritablement l’histoire effective de cette science.\par
En effet, non seulement les diverses parties de chaque science, qu’on est conduit à séparer dans l’ordre {\itshape dogmatique}, se sont, en réalité, développées simultanément et sous l’influence les unes des autres, ce qui tendrait à faire préférer l’ordre {\itshape historique ;} mais en considérant, dans son ensemble, le développement effectif de l’esprit humain, on voit de plus que les différentes sciences ont été, dans le fait, perfectionnées en même temps et mutuellement ; on voit même que les progrès des sciences et ceux des arts ont dépendu les uns des autres, par d’innombrables influences réciproques, et enfin que tous ont été étroitement liés au développement général de la société humaine. Ce vaste enchaînement est tellement réel, que souvent, pour concevoir la génération effective d’une théorie scientifique, l’esprit est conduit à considérer le perfectionnement de quelque art qui n’a avec elle aucune liaison rationnelle, ou même quelque progrès particulier dans l’organisation sociale, sans lequel cette découverte n’eût pu avoir lieu. Nous en verrons dans la suite de nombreux exemples. Il résulte donc de là que l’on ne peut connaître la véritable histoire de chaque science, c’est-à-dire la formation réelle des découvertes dont elle se compose, qu’en étudiant, d’une manière générale et directe, l’histoire de l’humanité. C’est pourquoi tous les documents recueillis jusqu’ici sur l’histoire des mathématiques, de l’astronomie, de la médecine, etc., quelque précieux qu’ils soient, ne peuvent être regardés que comme des matériaux.\par
Le prétendu ordre {\itshape historique} d’exposition, même quand il pourrait être suivi rigoureusement pour les détails de chaque science en particulier, serait déjà purement hypothétique et abstrait sous le rapport le plus important, en ce qu’il considérerait le développement de cette science comme isolé. Bien loin de mettre en évidence la véritable histoire de la science, il tendrait à en faire concevoir une opinion très fausse.\par
Ainsi, nous sommes certainement convaincus que la connaissance de l’histoire des sciences est de la plus haute importance. Je pense même qu’on ne connaît pas complètement une science tant qu’on n’en sait pas l’histoire. Mais cette étude doit être conçue comme entièrement séparée de l’étude propre et dogmatique de la science, sans laquelle même cette histoire ne serait pas intelligible. Nous considérerons donc avec beaucoup de soin l’histoire réelle des sciences fondamentales qui vont être le sujet de nos méditations ; mais ce sera seulement dans la dernière partie de ce cours, celle relative à l’étude des phénomènes sociaux, en traitant du développement général de l’humanité, dont l’histoire des sciences constitue la partie la plus importante, quoique jusqu’ici la plus négligée. Dans l’étude de chaque science les considérations historiques incidentes qui pourront se présenter auront un caractère nettement distinct, de manière à ne pas altérer la nature propre de notre travail principal.\par
(4) La discussion précédente, qui doit d’ailleurs, comme on le voit, être spécialement développée plus tard, tend à préciser davantage, en le présentant sous un nouveau point de vue, le véritable esprit de ce cours. Mais, surtout, il en résulte, relativement à la question actuelle, la détermination exacte des conditions qu’on doit s’imposer, et qu’on peut justement espérer de remplir dans la construction d’une échelle encyclopédique des diverses sciences fondamentales.\par
On voit, en effet, que, quelque parfaite qu’on pût la supposer, cette classification ne saurait jamais être rigoureusement conforme à l’enchaînement historique des sciences. Quoi qu’on fasse, on ne peut éviter entièrement de présenter comme antérieure telle science qui aura cependant besoin, sous quelques rapports particuliers plus ou moins importants, d’emprunter des notions à une autre science classée dans un rang postérieur. Il faut tâcher seulement qu’un tel inconvénient n’ait pas lieu relativement aux conceptions caractéristiques de chaque science, car alors la classification serait tout à fait vicieuse.\par
Ainsi, par exemple, il me semble incontestable que, dans le système général des sciences, l’astronomie doit être placée avant la physique proprement dite, et néanmoins plusieurs branches de celle-ci, surtout l’optique, sont indispensables à l’exposition complète de la première.\par
De tels défauts secondaires, qui sont strictement inévitables, ne sauraient prévaloir contre une classification qui remplirait d’ailleurs convenablement les conditions principales. Ils tiennent à ce qu’il y a nécessairement d’artificiel dans notre division du travail intellectuel.\par
Néanmoins, quoique, d’après les explications précédentes, nous ne devions pas prendre l’ordre historique pour base de notre classification, je ne dois pas négliger d’indiquer d’avance, comme une propriété essentielle de l’échelle encyclopédique que je vais proposer, sa conformité générale avec l’ensemble de l’histoire scientifique ; en ce sens, que, malgré la simultanéité réelle et continue du développement des différentes sciences, celles qui seront classées comme antérieures seront, en effet, plus anciennes et constamment plus avancées que celles présentées comme postérieures. C’est ce qui doit avoir lieu inévitablement si, en réalité, nous prenons, comme cela doit être, pour principe de classification, l’enchaînement logique naturel des diverses sciences, le point de départ de l’espèce ayant dû nécessairement être le même que celui de l’individu.\par
Pour achever de déterminer avec toute la précision possible la difficulté exacte de la question encyclopédique que nous avons à résoudre, je crois utile d’introduire une considération mathématique fort simple, qui résumera rigoureusement l’ensemble des raisonnements exposés jusqu’ici dans cette leçon. Voici en quoi elle consiste.\par
Nous nous proposons de classer les sciences fondamentales. Or nous verrons bientôt que, tout bien considéré, il n’est pas possible d’en distinguer moins de six ; la plupart des savants en admettraient même vraisemblablement un plus grand nombre. Cela posé, on sait que six objets comportent 720 dispositions différentes. Les sciences fondamentales pourraient donc donner lieu à 720 classifications distinctes, parmi lesquelles il s’agit de choisir la classification nécessairement unique qui satisfait le mieux aux principales conditions du problème. On voit que, malgré le grand nombre d’échelles encyclopédiques successivement proposées jusqu’à présent, la discussion n’a porté encore que sur une bien faible partie des dispositions possibles ; et néanmoins, je crois pouvoir dire, sans exagération, qu’en examinant chacune de ces 720 classifications, il n’en serait peut-être pas une seule en faveur de laquelle on ne pût faire valoir quelques motifs plausibles ; car, en observant les diverses dispositions qui ont été effectivement proposées, on remarque entre elles les plus extrêmes différences ; les sciences, qui sont placées par les uns à la tête du système encyclopédique, étant renvoyées par d’autres à l’extrémité opposée, et réciproquement. C’est donc dans ce choix d’un seul ordre vraiment rationnel, parmi le nombre très considérable des systèmes possibles, que consiste la difficulté précise de la question que nous avons posée.
\section[{IV.}]{IV.}
\noindent (1) Abordant maintenant d’une manière directe cette grande question, rappelons-nous d’abord que, pour obtenir une classification naturelle et positive des sciences fondamentales, c’est dans la comparaison des divers ordres de phénomènes dont elles ont pour objet de découvrir les lois que nous devons en chercher le principe. Ce que nous voulons déterminer, c’est la dépendance réelle des diverses études scientifiques. Or cette dépendance ne peut résulter que de celle des phénomènes correspondants.\par
En considérant sous ce point de vue tous les phénomènes observables, nous allons voir qu’il est possible de les classer en un petit nombre de catégories naturelles disposées d’une telle manière, que l’étude rationnelle de chaque catégorie soit fondée sur la connaissance des lois principales de la catégorie précédente, et devienne le fondement de l’étude de la suivante. Cet ordre est déterminé par le degré de simplicité, ou, ce qui revient au même, par le degré de généralité des phénomènes, d’où résulte leur dépendance successive, et, en conséquence, la facilité plus ou moins grande de leur étude.\par
Il est clair, en effet, {\itshape a priori}, que les phénomènes les plus simples, ceux qui se compliquent le moins des autres, sont nécessairement aussi les plus généraux ; car ce qui s’observe dans le plus grand nombre de cas est, par cela même, dégagé le plus possible des circonstances propres à chaque cas séparé. C’est donc par l’étude des phénomènes les plus généraux ou les plus simples qu’il faut commencer, en procédant ensuite successivement jusqu’aux phénomènes les plus particuliers ou les plus compliqués, si l’on veut concevoir la philosophie naturelle d’une manière vraiment méthodique ; car cet ordre de généralité ou de simplicité, déterminant nécessairement l’enchaînement rationnel des diverses sciences fondamentales par la dépendance successive de leurs phénomènes, fixe ainsi leur degré de facilité.\par
En même temps, par une considération auxiliaire que je crois important de noter ici, et qui converge exactement avec toutes les précédentes, les phénomènes les plus généraux ou les plus simples, se trouvant nécessairement les plus étrangers à l’homme, doivent, par cela même, être étudiés dans une disposition d’esprit plus calme, plus rationnelle, ce qui constitue un nouveau motif pour que les sciences correspondantes se développent plus rapidement.\par
(2) Ayant ainsi indiqué la règle fondamentale qui doit présider à la classification des sciences, je puis passer immédiatement à la construction de l’échelle encyclopédique d’après laquelle le plan de ce cours doit être déterminé, et que chacun pourra aisément apprécier à l’aide des considérations précédentes.\par
Une première contemplation de l’ensemble des phénomènes naturels nous porte à les diviser d’abord, conformément au principe que nous venons d’établir, en deux grandes classes principales, la première comprenant tous les phénomènes des corps bruts, la seconde tous ceux des corps organisés.\par
Ces derniers sont évidemment, en effet, plus compliqués et plus particuliers que les autres ; ils dépendent des précédents, qui au contraire, n’en dépendent nullement. De la nécessité de n’étudier les phénomènes physiologiques qu’après ceux des corps inorganiques. De quelque manière qu’on explique les différences de ces deux sortes d’êtres, il est certain qu’on observe dans les corps vivants tous les phénomènes, soit mécaniques, soit chimiques, qui ont lieu ans les corps bruts, plus un ordre tout spécial de phénomènes, les phénomènes vitaux proprement dits, ceux qui tiennent à {\itshape l’organisation.} Il ne s’agit pas ici d’examiner si les deux classes de corps sont ou ne sont pas de la même {\itshape nature}, question insoluble qu’on agite encore beaucoup trop de nos jours, par un reste d’influence des habitudes théologiques et métaphysiques ; une telle question n’est pas du domaine de la philosophie positive, qui fait formellement profession d’ignorer absolument la {\itshape nature} intime d’un corps quelconque. Mais il n’est nullement indispensable de considérer les corps bruts et les corps vivants comme étant d’une nature essentiellement différente, pour reconnaître la nécessité de la séparation de leurs études.\par
Sans doute, les idées ne sont pas encore suffisamment fixées sur la manière générale de concevoir les phénomènes des corps vivants. Mais, quelque parti qu’on puisse prendre à cet égard par suite des progrès ultérieurs de la philosophie naturelle, la classification que nous établissons n’en saurait être aucunement affectée. En effet, regardât-on comme démontré, ce que permet à peine d’entrevoir l’état présent de la physiologie, que les phénomènes physiologiques sont toujours de simples phénomènes mécaniques, électriques et chimiques, modifiés par la structure et la composition propres aux corps organisés, notre division fondamentale n’en subsisterait pas moins. Car il reste toujours vrai, même dans cette hypothèse, que les phénomènes généraux doivent être étudiés avant de procéder à l’examen des modifications spéciales qu’ils éprouvent dans certains êtres de l’univers, par suite d’une disposition particulière des molécules. Ainsi, la division, qui est aujourd’hui fondée dans la plupart des esprits éclairés sur la diversité des lois, est de nature à se maintenir indéfiniment à cause de la subordination des phénomènes et par suite des études, quelque rapprochement qu’on puisse jamais établir solidement entre les deux classes de corps.\par
Ce n’est pas ici le lieu de développer, dans ses diverses parties essentielles, la comparaison générale entre les corps bruts et les corps vivants, qui sera le sujet spécial d’un examen approfondi dans la section physiologique de ce cours. Il suffit, quant à présent, d’avoir reconnu, en principe, la nécessité logique de séparer la science relative aux premiers de celle relative aux seconds, et de ne procéder à l’étude de la {\itshape physique organique} qu’après avoir établi les lois générales de la {\itshape physique inorganique.}\par
(3) Passons maintenant à la détermination de la sous-division principale dont est susceptible, d’après la même règle, chacune de ces deux grandes moitiés de la philosophie naturelle.\par
Pour la {\itshape physique inorganique}, nous voyons d’abord, en nous conformant toujours à l’ordre de généralité et de dépendance des phénomènes, qu’elle doit être partagée en deux sections distinctes, suivant qu’elle considère les phénomènes généraux de l’univers, ou, en particulier, ceux que présentent les corps terrestres. D’où la physique céleste, ou l’astronomie, soit géométrique, soit mécanique ; et la physique terrestre. La nécessité de cette division est exactement semblable à celle de la précédente.\par
Les phénomènes astronomiques étant les plus généraux, les plus simples, les plus abstraits de tous, c’est évidemment par leur étude que doit commencer la philosophie naturelle, puisque les lois auxquelles ils sont assujettis influent sur celles de tous les autres phénomènes, dont elles-mêmes sont, au contraire, essentiellement indépendantes. Dans tous les phénomènes de la physique terrestre, on observe d’abord les effets généraux de la gravitation universelle, plus quelques autres effets qui leur sont propres, et qui modifient les premiers. Il s’ensuit que, lorsqu’on analyse le phénomène terrestre le plus simple, non seulement en prenant un phénomène chimique, mais en choisissant même un phénomène purement mécanique, on le trouve constamment plus composé que le phénomène céleste le plus compliqué. C’est ainsi, par exemple, que le simple mouvement d’un corps pesant, même quand il ne s’agit que d’un solide, présente réellement, lorsqu’on veut tenir compte de toutes les circonstances déterminantes, un sujet de recherches plus compliqué que la question astronomique la plus difficile. Une telle considération montre clairement combien il est indispensable de séparer nettement la physique céleste et la physique terrestre, et de ne procéder à l’étude de la seconde qu’après celle de la première, qui en est la base rationnelle.\par
(4) La physique terrestre, à son tour, se sous-divise, d’après le même principe, en deux portions très distinctes, selon qu’elle envisage les corps sous le point de vue mécanique, ou sous le point de vue chimique. D’où la physique proprement dite et la chimie. Celle-ci, pour être conçue d’une manière vraiment méthodique, suppose évidemment la connaissance préalable de l’autre. Car tous les phénomènes chimiques sont nécessairement plus compliqués que les phénomènes physiques ; ils en dépendent sans influer sur eux. Chacun sait, en effet, que toute action chimique est soumise d’abord à l’influence de la pesanteur, de la chaleur, de l’électricité, etc., et présente, en outre, quelque chose de propre qui modifie l’action des agents précédents. Cette Considération, qui montre évidemment la chimie comme ne pouvant marcher qu’après la physique, la présente en même temps comme une science distincte. Car, quelque opinion qu’on adopte relativement aux affinités chimiques, et, quand même on ne verrait en elles, ainsi qu’on peut le concevoir, que des modifications de la gravitation générale produites par la figure et par la disposition mutuelle des atomes, il demeurerait incontestable que la nécessité d’avoir continuellement égard à ces conditions spéciales ne permettrait point de traiter la chimie comme un simple appendice de la physique. On serait donc obligé, dans tous les cas, ne fût-ce que pour la facilité de l’étude, de maintenir la division et l’enchaînement que l’on regarde aujourd’hui comme tenant à l’hétérogénéité des phénomènes.\par
(5) Telle est donc la distribution rationnelle des principales branches de la science générale des corps bruts. Une division analogue s’établit, de la même manière, dans la science générale des corps organisés.\par
Tous les êtres vivants présentent deux ordres de phénomènes essentiellement distincts, ceux relatifs à l’individu, et ceux qui concernent l’espèce, surtout quand elle est sociable. C’est principalement par rapport à l’homme, que cette distinction est fondamentale. Le dernier ordre de phénomènes est évidemment plus compliqué et plus particulier que le premier ; il en dépend sans influer sur lui. De là, deux grandes sections dans la {\itshape physique organique}, la physiologie proprement dite, et la physique sociale, qui est fondée sur la première.\par
Dans tous les phénomènes sociaux, on observe d’abord l’influence des lois physiologiques de l’individu, et, en outre, quelque chose de particulier qui en modifie les effets, et qui tient à l’action des individus les uns sur les autres, singulièrement compliquée, dans l’espèce humaine, par l’action de chaque génération sur celle qui la suit. Il est donc évident que, pour étudier convenablement les phénomènes sociaux, il faut d’abord partir d’une connaissance approfondie des lois relatives à la vie individuelle. D’un autre côté, cette subordination nécessaire entre les deux études ne prescrit nullement, comme quelques physiologistes du premier ordre ont été portés à le croire, de voir dans la physique sociale un simple appendice de la physiologie. Quoique les phénomènes soient certainement homogènes, ils ne sont point identiques, et la séparation des deux sciences est d’une importance vraiment fondamentale. Car il serait impossible de traiter l’étude collective de l’espèce comme une pure déduction de l’étude de l’individu, puisque les conditions sociales, qui modifient l’action des lois physiologiques, sont précisément alors la considération la plus essentielle. Ainsi, la physique sociale doit être fondée sur un corps d’observations directes qui lui Soit propre, tout en ayant égard, comme il convient, à son intime relation nécessaire avec la physiologie proprement dite.\par
On pourrait aisément établir une symétrie parfaite entre la division de la physique organique et celle ci-dessus exposée pour la physique inorganique, en rappelant la distinction vulgaire de la physiologie proprement dite en végétale et animale. Il serait facile, en effet, de rattacher cette sous-division au principe de classification que nous avons constamment suivi, puisque les phénomènes de la vie animale se présentent, en général du moins, comme plus compliqués et plus spéciaux que ceux de la vie végétale. Mais la recherche de cette symétrie précise aurait quelque chose de puéril, si elle entraînait à méconnaître ou à exagérer les analogies réelles ou les différences effectives des phénomènes. Or il est certain que la distinction entre la physiologie végétale et la physiologie animale, qui a une grande importance dans ce que j’ai appelé la {\itshape physique concrète}, n’en a presque aucune dans la {\itshape physique abstraite}, la seule dont il s’agisse ici. La connaissance des lois générales de la vie, qui doit être à nos yeux le véritable objet de la physiologie, exige la considération simultanée de toute la série organique sans distinction de végétaux et d’animaux, distinction qui, d’ailleurs, s’efface de jour en jour, à mesure que les phénomènes sont étudiés d’une manière plus approfondie.\par
Nous persisterons donc à ne considérer qu’une seule division dans la physique organique, quoique nous ayons cru devoir en établir deux successives dans la physique inorganique.\par
(6) En résultat de cette discussion, la philosophie positive se trouve donc naturellement partagée en cinq sciences fondamentales, dont la succession est déterminée par une subordination nécessaire et invariable, fondée, indépendamment de toute opinion hypothétique, sur la simple comparaison approfondie des phénomènes correspondants ; c’est l’astronomie, la physique, la chimie, la physiologie et enfin la physique sociale. La première considère les phénomènes les plus généraux, les plus simples, les plus abstraits et les plus éloignés de l’humanité ; ils influent sur tous les autres, sans être influencés par eux. Les phénomènes considérés par la dernière sont, au contraire, les plus particuliers, les plus compliqués, les plus concrets, et les plus directement intéressants pour l’homme ; ils dépendent, plus ou moins, de tous les précédents, sans exercer sur eux aucune influence. Entre ces deux extrêmes, les degrés de spécialité, de complication et de personnalité des phénomènes vont graduellement en augmentant, ainsi que leur dépendance successives. Telle est l’intime relation générale que la véritable observation philosophique, convenablement employée, et non de vaines distinctions arbitraires, nous conduit à établir entre les diverses sciences fondamentales. Tel doit donc être le plan de ce cours.\par
Je n’ai pu ici qu’esquisser l’exposition des considérations principales sur lesquelles repose cette classification. Pour la concevoir complètement, il faudrait maintenant, après l’avoir envisagée d’un point de vue général, l’examiner relativement à chaque science fondamentale en particulier. C’est ce que nous ferons soigneusement en commençant l’étude spéciale de chaque partie de ce cours. La construction de cette échelle encyclopédique, reprise ainsi successivement en partant de chacune des cinq grandes sciences, lui fera acquérir plus d’exactitude, et surtout mettra pleinement en évidence sa solidité. Ces avantages seront d’autant plus sensibles, que nous verrons alors la distribution intérieure de chaque science s’établir naturellement d’après le même principe, ce qui présentera tout le système des connaissances humaines décomposé, jusque dans ses détails secondaires, d’après une considération unique constamment suivie, celle du degré d’abstraction plus ou moins grand des conceptions correspondantes. Mais des travaux de ce genre, outre qu’ils nous entraîneraient maintenant beaucoup trop loin, seraient certainement déplacés dans cette leçon, où notre esprit doit se maintenir au point de vue le plus général de la philosophie positive.
\section[{V.}]{V.}
\noindent Néanmoins, pour faire apprécier aussi complètement que possible, dès ce moment, l’importance de cette hiérarchie fondamentale, dont je ferai, dans toute la suite de ce cours, des applications continuelles, je dois signaler rapidement ici ses propriétés générales les plus essentielles.\par
(1) Il faut d’abord remarquer, comme une vérification très décisive de l’exactitude de cette classification, sa conformité essentielle avec la coordination, en quelque sorte spontanée, qui se trouve en effet implicitement admise par les savants livrés à l’étude des diverses branches de la philosophie naturelle.\par
C’est une condition ordinairement fort négligée par les constructeurs d’échelles encyclopédiques, que de présenter comme distinctes les sciences que la marche effective de l’esprit humain a conduit, sans dessein prémédité, à cultiver séparément, et d’établir entre elles une subordination conforme aux relations positives que manifeste leur développement journalier. Un tel accord est néanmoins évidemment le plus sûr indice d’une bonne classification ; car les divisions qui se sont introduites spontanément dans le système scientifique n’ont pu être déterminées que par le sentiment longtemps éprouvé des véritables besoins de l’esprit humain, sans qu’on ait pu être égaré par des généralités vicieuses.\par
Mais quoique la classification ci-dessus proposée remplisse entièrement cette condition, ce qu’il serait superflu de prouver, il n’en faudrait pas conclure que les habitudes généralement établies aujourd’hui par expérience chez les savants rendraient inutile le travail encyclopédique que nous venons d’exécuter. Elles ont seulement rendu possible une telle opération, qui présente la différence fondamentale d’une conception rationnelle à une classification purement empirique. Il s’en faut d’ailleurs que cette classification soit ordinairement conçue et surtout suivie avec toute la précision nécessaire, et que son importance soit convenablement appréciée ; il suffirait, pour s’en convaincre, de considérer les graves infractions qui sont commises tous les jours contre cette loi encyclopédique, au grand préjudice de l’esprit humain.\par
(2) Un second caractère très essentiel de notre classification, c’est d’être nécessairement conforme à l’ordre effectif du développement de la philosophie naturelle. C’est ce que vérifie tout ce qu’on sait de l’histoire des sciences, particulièrement dans les deux derniers siècles, où nous pouvons suivre leur marche avec plus d’exactitude.\par
On conçoit, en effet, que l’étude rationnelle de chaque science fondamentale, exigeant la culture préalable de toutes celles qui la précèdent dans notre hiérarchie encyclopédique, n’a pu faire de progrès réels et prendre son véritable caractère, qu’après un grand développement des sciences antérieures, relatives à des phénomènes plus généraux, plus abstraits, moins compliqués et indépendants des autres. C’est donc dans cet ordre que la progression, quoique simultanée, a dû avoir lieu.\par
Cette considération me semble d’une telle importance, que je ne crois pas possible de comprendre réellement, sans y avoir égard, l’histoire de l’esprit humain. La loi générale qui domine toute cette histoire, et que j’ai exposée dans la leçon précédente, ne peut être convenablement entendue, si on ne la combine point dans l’application avec la formule encyclopédique que nous venons d’établir. Car, c’est suivant l’ordre énoncé par cette formule que les différentes théories humaines ont atteint successivement d’abord l’état théologique, ensuite l’état métaphysique, et enfin l’état positif. Si l’on ne tient pas compte dans l’usage de la loi de cette progression nécessaire, on rencontrera souvent des difficultés qui paraîtront insurmontables, car il est clair que l’état théologique ou métaphysique de certaines théories fondamentales a dû temporairement coïncider et a quelquefois coïncidé en effet avec l’état positif de celles qui leur sont antérieures dans notre système encyclopédique, ce qui tend à jeter sur la vérification de la loi générale une obscurité qu’on ne peut dissiper que par la classification précédente.\par
(3) En troisième lieu, cette classification présente la propriété très remarquable de marquer exactement la perfection relative des différentes sciences, laquelle consiste essentiellement dans le degré de précision des connaissances et dans leur coordination plus ou moins intime.\par
Il est aisé de sentir, en effet, que plus des phénomènes sont généraux, simples et abstraits, moins ils dépendent des autres et plus les connaissances qui s’y rapportent peuvent être précises, en même temps que leur coordination peut être plus complète. Ainsi les phénomènes organiques ne comportent qu’une étude à la fois moins exacte et moins systématique que les phénomènes des corps bruts. De même dans la physique inorganique, les phénomènes célestes, vu leur plus grande généralité et leur indépendance de tous les autres, ont donné lieu à une science bien plus précise et beaucoup plus liée que celle des phénomènes terrestres.\par
Cette observation, qui est si frappante dans l’étude effective des sciences, et qui a souvent donné lieu à des espérances chimériques ou à d’injustes comparaisons, se trouve donc complètement expliquée par l’ordre encyclopédique que j’ai établi. J’aurai naturellement occasion de lui donner toute son extension dans la leçon prochaine, en montrant que la possibilité d’appliquer à l’étude des divers phénomènes l’analyse mathématique, ce qui est le moyen de procurer à cette étude le plus haut degré possible de précision et de coordination, se trouve exactement déterminée par le rang qu’occupent ces phénomènes dans mon échelle encyclopédique.\par
Je ne dois point passer à une autre considération sans mettre le lecteur en garde à ce sujet contre une erreur fort grave, et qui, bien que très grossière, est encore extrêmement commune. Elle consiste à confondre le degré de précision que comportent nos différentes connaissances avec leur degré de certitude, d’où est résulté le préjugé très dangereux que, le premier étant évidemment fort inégal, il en doit être ainsi du second. Aussi parle-t-on souvent encore, quoique moins que jadis, de l’inégale certitude des diverses sciences, ce qui tend directement à décourager la culture des sciences les plus difficiles. Il est clair, néanmoins, que la précision et la certitude sont deux qualités en elles-mêmes fort différentes. Une proposition tout à fait absurde peut être extrêmement précise, comme si l’on disait, par exemple, que la somme des angles d’un triangle est égale à trois angles droits ; et une proposition très certaine peut ne comporter qu’une précision fort médiocre, comme lorsqu’on affirme, par exemple, que tout homme mourra. Si, d’après l’explication précédente, les diverses sciences doivent nécessairement présenter une précision très inégale, il n’en est nullement ainsi de leur certitude. Chacune peut offrir des résultats aussi certains que ceux de toute autre, pourvu qu’elle sache renfermer ses conclusions dans le degré de précision que comportent les phénomènes correspondants, condition qui peut n’être pas toujours très facile à remplir. Dans une science quelconque, tout ce qui est simplement conjectural n’est que plus ou moins probable, et ce n’est pas là ce qui compose son domaine essentiel ; tout ce qui est positif, c’est-à-dire fondé sur des faits constatés., est certain : il n’y a pas de distinction à cet égard.\par
(4) Enfin, la propriété la plus intéressante de notre formule encyclopédique, à cause de l’importance et de la multiplicité des applications immédiates qu’on en peut faire, c’est de déterminer directement le véritable plan général d’une éducation scientifique entièrement rationnelle. C’est ce qui résulte sur-le-champ de la seule composition de la formule.\par
Il est sensible, en effet, qu’avant d’entreprendre l’étude méthodique de quelqu’une des sciences fondamentales, il faut nécessairement s’être préparé par l’examen de celles relatives aux phénomènes antérieurs dans notre échelle encyclopédique, puisque ceux-ci influent toujours d’une manière prépondérante sur ceux dont on se propose de connaître les lois. Cette considération est tellement frappante, que, malgré son extrême importance pratique, je n’ai pas besoin d’insister davantage en ce moment sur un principe qui, plus tard, se reproduira d’ailleurs inévitablement, par rapport à chaque science fondamentale. Je me bornerai seulement à faire observer que, s’il est éminemment applicable à l’éducation générale, il l’est aussi particulièrement à l’éducation spéciale des savants.\par
Ainsi, les physiciens qui n’ont pas d’abord étudié l’astronomie, au moins sous un point de vue général ; les chimistes qui, avant de s’occuper de leur science propre, n’ont pas étudié préalablement l’astronomie et ensuite la physique ; les physiologistes qui ne se sont pas préparés à leurs travaux spéciaux par une étude préliminaire de l’astronomie, de la physique et de la chimie, ont manqué à l’une des conditions fondamentales de leur développement intellectuel. Il en est encore plus évidemment de même pour les esprits qui veulent se livrer à l’étude positive des phénomènes sociaux, sans avoir d’abord acquis une connaissance générale de l’astronomie, de la physique, de la chimie et de la physiologie.\par
Comme de telles conditions sont bien rarement remplies de nos jours, et qu’aucune institution régulière n’est organisée pour les accomplir, nous pouvons dire qu’il n’existe pas encore, pour les savants, d’éducation vraiment rationnelle. Cette considération est, à mes yeux, d’une si grande importance, que je ne crains pas d’attribuer en partie à ce vice de nos éducations actuelles l’état d’imperfection extrême où nous voyons encore les sciences les plus difficiles, état véritablement inférieur à ce que prescrit en effet la nature plus compliquée des phénomènes correspondants. Relativement à l’éducation générale, cette condition est encore bien plus nécessaire. Je la crois tellement indispensable, que je regarde l’enseignement scientifique comme incapable de réaliser les résultats généraux les plus essentiels qu’il est destiné à produire dans la société pour la rénovation du système intellectuel, si les diverses branches principales de la philosophie naturelle ne sont pas étudiées dans l’ordre convenable. N’oublions pas que, dans presque toutes les intelligences, même les plus élevées, les idées restent ordinairement enchaînées suivant l’ordre de leur acquisition première ; et que, par conséquent, c’est un mal le plus souvent irrémédiable que de n’avoir pas commencé par le commencement. Chaque siècle ne compte qu’un bien petit nombre de penseurs capables, à l’époque de leur virilité., comme Bacon, Descartes et Leibnitz, de faire véritablement table rase pour reconstruire de fond en comble le système entier de leurs idées acquises.\par
L’importance de notre loi encyclopédique pour servir de base à l’éducation scientifique ne peut être convenablement appréciée qu’en la considérant aussi par rapport à la méthode, au lieu de l’envisager seulement, comme nous venons de le faire, relativement à la doctrine.\par
Sous ce nouveau point de vue, une exécution convenable du plan général d’études que nous avons déterminé doit avoir pour résultat nécessaire de nous procurer une connaissance parfaite de la méthode positive, qui ne pourrait être obtenue d’aucune autre manière.\par
En effet, les phénomènes naturels ayant été classés de telle sorte, que ceux qui sont réellement homogènes restent toujours compris dans une même étude, tandis que ceux qui ont été affectés à des études différentes sont effectivement hétérogènes, il doit nécessairement en résulter que la Méthode positive générale sera constamment modifiée d’une manière uniforme dans l’étendue d’une même science fondamentale, et qu’elle éprouvera sans cesse des modifications différentes et de plus en plus composées, en passant d’une science à une autre. Nous aurons donc ainsi la certitude de la considérer dans toutes les variétés réelles dont elle est susceptible, ce qui n’aurait pu avoir lieu, si nous avions adopté une formule encyclopédique qui ne remplît pas les conditions essentielles posées ci-dessus.\par
Cette nouvelle considération est d’une importance vraiment fondamentale ; car, si nous avons vu en général, dans la dernière leçon, qu’il est impossible de connaître la méthode positive, quand on veut l’étudier séparément de son emploi, nous devons ajouter aujourd’hui qu’on ne peut s’en former une idée nette et exacte qu’en étudiant successivement, et dans l’ordre convenable, son application à toutes les diverses classes principales des phénomènes naturels. Une seule science ne suffirait point pour atteindre ce but, même en la choisissant le plus judicieusement possible. Car, quoique la méthode soit essentiellement identique dans toutes, chaque science développe spécialement tel ou tel de ses procédés caractéristiques, dont l’influence, trop peu prononcée dans les autres sciences, demeurerait inaperçue. Ainsi, par exemple, dans certaines branches de la philosophie, c’est l’observation proprement dite ; dans d’autres, c’est l’expérience, et telle ou telle nature d’expériences, qui constitue le principal moyen d’exploration. De même, tel précepte général, qui fait partie intégrante de la méthode, a été fourni primitivement par une certaine science ; et, bien qu’il ait pu être ensuite transporté dans d’autres, c’est à sa source qu’il faut l’étudier pour le bien connaître ; comme, par exemple, la théorie des classifications.\par
En se bornant à l’étude d’une science unique, il faudrait sans doute choisir la plus parfaite pour avoir un sentiment plus profond de la méthode positive. Or, la plus parfaite étant en même temps la plus simple, on n’aurait ainsi qu’une connaissance bien incomplète de la méthode, puisqu’on n’apprendrait pas quelles modifications essentielles elle doit subir pour s’adapter à des phénomènes plus compliqués. Chaque science fondamentale a donc, sous ce rapport, des avantages qui lui sont propres ; ce qui prouve clairement la nécessité de les considérer toutes, sous peine de ne se former que des conceptions trop étroites et des habitudes insuffisantes. Cette considération devant se reproduire fréquemment dans la suite, il est inutile de la développer davantage en ce moment.\par
(5) Je dois néanmoins ici, toujours sous le rapport de la méthode, insister spécialement sur le besoin, pour la bien connaître, non seulement d’étudier philosophiquement toutes les diverses sciences fondamentales, mais de les étudier suivant l’ordre encyclopédique établi dans cette leçon. Que peut produire de rationnel, à moins d’une extrême supériorité naturelle, un esprit qui s’occupe de prime abord de l’étude des phénomènes les plus compliques, sans avoir préalablement appris à connaître, par l’examen des phénomènes les plus simples, ce que c’est qu’une loi, ce que c’est {\itshape qu’observer}, ce que c’est qu’une conception positive, ce que c’est même qu’un raisonnement suivi ? Telle est pourtant encore aujourd’hui la marche ordinaire de nos jeunes physiologistes, qui abordent immédiatement l’étude des corps vivants, sans avoir le plus souvent été préparés autrement que par une éducation préliminaire réduite à l’étude d’une ou de deux langues mortes, et n’ayant, tout au plus, qu’une connaissance très superficielle de la physique et de la chimie, connaissance presque nulle sous le rapport de la méthode, puisqu’elle n’a pas été obtenue communément d’une manière rationnelle, et en partant du véritable point de départ de la philosophie naturelle. On conçoit combien il importe de réformer un plan d’études aussi vicieux. De même, relativement aux phénomènes Sociaux, qui sont encore plus compliqués, ne serait-ce point avoir fait un grand pas vers le retour des sociétés modernes à un état vraiment normal, que d’avoir reconnu la nécessité logique de ne procéder à l’étude de ces phénomènes, qu’après avoir dressé successivement l’organe intellectuel par l’examen philosophique approfondi de tous les phénomènes antérieurs ? On peut même dire avec précision que c’est là toute la difficulté principale. Car il est peu de bons esprits qui ne soient convaincus aujourd’hui qu’il faut étudier les phénomènes sociaux d’après la méthode positive. Seulement, ceux qui s’occupent de cette étude, ne sachant pas et ne pouvant pas savoir exactement en quoi consiste cette méthode, faute de l’avoir examinée dans ses applications antérieures, cette maxime est jusqu’à présent demeurée stérile pour la rénovation des théories sociales, qui ne sont pas encore sorties de l’état théologique ou de l’état métaphysique, malgré les efforts des prétendus réformateurs positifs. Cette considération sera, plus tard, spécialement développée ; je dois ici me borner à l’indiquer, uniquement pour faire apercevoir toute la portée de la conception encyclopédique que j’ai proposée dans cette leçon.\par
Tels sont donc les quatre points de vue principaux sous lesquels j’ai dû m’attacher à faire ressortir l’importance générale de la classification rationnelle et positive, établie ci-dessus pour les sciences fondamentales.
\section[{VI.}]{VI.}
\noindent Afin de compléter l’exposition générale du plan de ce cours, il me reste maintenant à considérer une lacune immense et capitale, que j’ai laissée à dessein dans ma formule encyclopédique, et que le lecteur a sans doute déjà remarquée. En effet, nous n’avons point marqué dans notre système scientifique le rang de la science mathématique.\par
(1) Le motif de cette omission volontaire est dans l’importance même de cette science, si vaste et si fondamentale. Car la leçon prochaine sera entièrement consacrée à la détermination exacte de son véritable caractère général, et par suite à la fixation précise de son rang encyclopédique. Mais, pour ne pas laisser incomplet, sous un rapport aussi capital, le grand tableau que j’ai tâché d’esquisser dans cette leçon, je dois indiquer ici sommairement, par anticipation, les résultats généraux de l’examen que nous entreprendrons dans la leçon suivante.\par
Dans l’état actuel du développement de nos connaissances positives, il convient, je crois, de regarder la science mathématique, moins comme une partie constituante de la philosophie naturelle proprement dite, que comme étant, depuis Descartes et Newton, la vraie base fondamentale de toute cette philosophie, quoique, à parler exactement, elle soit à la fois l’une et l’autre. Aujourd’hui, en effet, la science mathématique est bien moins importante par les connaissances, très réelles et très précieuses néanmoins, qui la composent directement, que comme constituant l’instrument le plus puissant que l’esprit humain puisse employer dans la recherche des lois des phénomènes naturels.\par
Pour présenter à cet égard une conception parfaitement nette et rigoureusement exacte, nous verrons qu’il faut diviser la science mathématique en deux grandes sciences, dont le caractère est essentiellement distinct : la mathématique abstraite, ou le calcul, en prenant ce mot dans sa plus grande extension, et la mathématique concrète, qui se compose, d’une part de la géométrie générale, d’une autre part de la mécanique rationnelle. La partie concrète est nécessairement fondée sur la partie abstraite, et devient à son tour la base directe de toute la philosophie naturelle, en considérant, autant que possible, tous les phénomènes de l’univers comme géométriques ou comme mécaniques.\par
La partie abstraite est la seule qui soit purement instrumentale, n’étant autre chose qu’une immense extension admirable de la logique naturelle à un certain ordre de déductions. La géométrie et la mécanique doivent, au contraire, être envisagées comme de véritables sciences naturelles, fondées, ainsi que toutes les autres, sur l’observation, quoique, par l’extrême simplicité de leurs phénomènes, elles comportent un degré infiniment plus parfait de systématisation, qui a pu quelquefois faire méconnaître le caractère expérimental de leurs premiers principes. Mais ces deux sciences physiques ont cela de particulier, que, dans l’état présent de l’esprit humain, elles sont déjà et seront toujours davantage employées comme méthode beaucoup plus que comme doctrine directe.\par
(2) Il est, du reste, évident qu’en plaçant ainsi la science mathématique à la tête de la philosophie positive, nous ne faisons qu’étendre davantage l’application de ce même principe de classification, fondé sur la dépendance successive des sciences en résultat du degré d’abstraction de leurs phénomènes respectifs, qui nous a fourni la série encyclopédique, établie dans cette leçon. Nous ne faisons maintenant que restituer à cette série son véritable premier terme, dont l’importance propre exigeait un examen spécial plus développé. On voit, en effet, que les phénomènes géométriques et mécaniques sont, de tous, les plus généraux, les plus simples, les plus abstraits, les plus irréductibles et les plus indépendants de tous les autres, dont ils sont, au contraire, la base. On conçoit pareillement que leur étude est un préliminaire indispensable à celle de tous les autres ordres de phénomènes. C’est donc la science mathématique qui doit constituer le véritable point de départ de toute éducation scientifique rationnelle, soit générale, soit spéciale ce qui explique l’usage universel qui s’est établi depuis longtemps à ce sujet, d’une manière empirique, quoiqu’il n’ait eu primitivement d’autre cause que la plus grande ancienneté relative de la science mathématique. Je dois me borner en ce moment à une indication très rapide de ces diverses considérations qui vont être l’objet spécial de la leçon suivante.\par
Nous avons donc exactement déterminé dans cette leçon, non d’après de vaines spéculations arbitraires, mais en le regardant comme le sujet d’un véritable problème philosophique, le plan rationnel qui doit nous guider constamment dans l’étude de la philosophie positive. En résultat définitif, la mathématique, l’astronomie, la physique, la chimie, la physiologie et la physique sociale : telle est la formule encyclopédique qui, parmi le très grand nombre de classifications que comportent les six sciences fondamentales, est seule logiquement conforme à la hiérarchie naturelle et invariable des phénomènes. Je n’ai pas besoin de rappeler l’importance de ce résultat, que le lecteur doit se rendre éminemment familier, pour en faire dans toute l’étendue de ce cours une application continuelle.\par
La conséquence finale de cette leçon, exprimée sous la forme la plus simple, consiste donc dans l’explication et la justification du grand tableau synoptique placé au commencement de cet ouvrage, et dans la construction duquel je me suis efforcé de suivre, aussi rigoureusement que possible, pour la distribution intérieure de chaque science fondamentale, le même principe de classification qui vient de nous fournir la série générale des sciences.
\chapterclose

 


% at least one empty page at end (for booklet couv)
\ifbooklet
  \pagestyle{empty}
  \clearpage
  % 2 empty pages maybe needed for 4e cover
  \ifnum\modulo{\value{page}}{4}=0 \hbox{}\newpage\hbox{}\newpage\fi
  \ifnum\modulo{\value{page}}{4}=1 \hbox{}\newpage\hbox{}\newpage\fi


  \hbox{}\newpage
  \ifodd\value{page}\hbox{}\newpage\fi
  {\centering\color{rubric}\bfseries\noindent\large
    Hurlus ? Qu’est-ce.\par
    \bigskip
  }
  \noindent Des bouquinistes électroniques, pour du texte libre à participation libre,
  téléchargeable gratuitement sur \href{https://hurlus.fr}{\dotuline{hurlus.fr}}.\par
  \bigskip
  \noindent Cette brochure a été produite par des éditeurs bénévoles.
  Elle n’est pas faîte pour être possédée, mais pour être lue, et puis donnée.
  Que circule le texte !
  En page de garde, on peut ajouter une date, un lieu, un nom ; pour suivre le voyage des idées.
  \par

  Ce texte a été choisi parce qu’une personne l’a aimé,
  ou haï, elle a en tous cas pensé qu’il partipait à la formation de notre présent ;
  sans le souci de plaire, vendre, ou militer pour une cause.
  \par

  L’édition électronique est soigneuse, tant sur la technique
  que sur l’établissement du texte ; mais sans aucune prétention scolaire, au contraire.
  Le but est de s’adresser à tous, sans distinction de science ou de diplôme.
  Au plus direct ! (possible)
  \par

  Cet exemplaire en papier a été tiré sur une imprimante personnelle
   ou une photocopieuse. Tout le monde peut le faire.
  Il suffit de
  télécharger un fichier sur \href{https://hurlus.fr}{\dotuline{hurlus.fr}},
  d’imprimer, et agrafer ; puis de lire et donner.\par

  \bigskip

  \noindent PS : Les hurlus furent aussi des rebelles protestants qui cassaient les statues dans les églises catholiques. En 1566 démarra la révolte des gueux dans le pays de Lille. L’insurrection enflamma la région jusqu’à Anvers où les gueux de mer bloquèrent les bateaux espagnols.
  Ce fut une rare guerre de libération dont naquit un pays toujours libre : les Pays-Bas.
  En plat pays francophone, par contre, restèrent des bandes de huguenots, les hurlus, progressivement réprimés par la très catholique Espagne.
  Cette mémoire d’une défaite est éteinte, rallumons-la. Sortons les livres du culte universitaire, cherchons les idoles de l’époque, pour les briser.
\fi

\ifdev % autotext in dev mode
\fontname\font — \textsc{Les règles du jeu}\par
(\hyperref[utopie]{\underline{Lien}})\par
\noindent \initialiv{A}{lors là}\blindtext\par
\noindent \initialiv{À}{ la bonheur des dames}\blindtext\par
\noindent \initialiv{É}{tonnez-le}\blindtext\par
\noindent \initialiv{Q}{ualitativement}\blindtext\par
\noindent \initialiv{V}{aloriser}\blindtext\par
\Blindtext
\phantomsection
\label{utopie}
\Blinddocument
\fi
\end{document}
