%%%%%%%%%%%%%%%%%%%%%%%%%%%%%%%%%
% LaTeX model https://hurlus.fr %
%%%%%%%%%%%%%%%%%%%%%%%%%%%%%%%%%

% Needed before document class
\RequirePackage{pdftexcmds} % needed for tests expressions
\RequirePackage{fix-cm} % correct units

% Define mode
\def\mode{a4}

\newif\ifaiv % a4
\newif\ifav % a5
\newif\ifbooklet % booklet
\newif\ifcover % cover for booklet

\ifnum \strcmp{\mode}{cover}=0
  \covertrue
\else\ifnum \strcmp{\mode}{booklet}=0
  \booklettrue
\else\ifnum \strcmp{\mode}{a5}=0
  \avtrue
\else
  \aivtrue
\fi\fi\fi

\ifbooklet % do not enclose with {}
  \documentclass[french,twoside]{book} % ,notitlepage
  \usepackage[%
    papersize={105mm, 297mm},
    inner=12mm,
    outer=12mm,
    top=20mm,
    bottom=15mm,
    marginparsep=0pt,
  ]{geometry}
  \usepackage[fontsize=9.5pt]{scrextend} % for Roboto
\else\ifav
  \documentclass[french,twoside]{book} % ,notitlepage
  \usepackage[%
    a5paper,
    inner=25mm,
    outer=15mm,
    top=15mm,
    bottom=15mm,
    marginparsep=0pt,
  ]{geometry}
  \usepackage[fontsize=12pt]{scrextend}
\else% A4 2 cols
  \documentclass[twocolumn]{report}
  \usepackage[%
    a4paper,
    inner=15mm,
    outer=10mm,
    top=25mm,
    bottom=18mm,
    marginparsep=0pt,
  ]{geometry}
  \setlength{\columnsep}{20mm}
  \usepackage[fontsize=9.5pt]{scrextend}
\fi\fi

%%%%%%%%%%%%%%
% Alignments %
%%%%%%%%%%%%%%
% before teinte macros

\setlength{\arrayrulewidth}{0.2pt}
\setlength{\columnseprule}{\arrayrulewidth} % twocol
\setlength{\parskip}{0pt} % classical para with no margin
\setlength{\parindent}{1.5em}

%%%%%%%%%%
% Colors %
%%%%%%%%%%
% before Teinte macros

\usepackage[dvipsnames]{xcolor}
\definecolor{rubric}{HTML}{0c71c3} % the tonic
\def\columnseprulecolor{\color{rubric}}
\colorlet{borderline}{rubric!30!} % definecolor need exact code
\definecolor{shadecolor}{gray}{0.95}
\definecolor{bghi}{gray}{0.5}

%%%%%%%%%%%%%%%%%
% Teinte macros %
%%%%%%%%%%%%%%%%%
%%%%%%%%%%%%%%%%%%%%%%%%%%%%%%%%%%%%%%%%%%%%%%%%%%%
% <TEI> generic (LaTeX names generated by Teinte) %
%%%%%%%%%%%%%%%%%%%%%%%%%%%%%%%%%%%%%%%%%%%%%%%%%%%
% This template is inserted in a specific design
% It is XeLaTeX and otf fonts

\makeatletter % <@@@


\usepackage{blindtext} % generate text for testing
\usepackage{contour} % rounding words
\usepackage[nodayofweek]{datetime}
\usepackage{DejaVuSans} % font for symbols
\usepackage{enumitem} % <list>
\usepackage{etoolbox} % patch commands
\usepackage{fancyvrb}
\usepackage{fancyhdr}
\usepackage{fontspec} % XeLaTeX mandatory for fonts
\usepackage{footnote} % used to capture notes in minipage (ex: quote)
\usepackage{framed} % bordering correct with footnote hack
\usepackage{graphicx}
\usepackage{lettrine} % drop caps
\usepackage{lipsum} % generate text for testing
\usepackage[framemethod=tikz,]{mdframed} % maybe used for frame with footnotes inside
\usepackage{pdftexcmds} % needed for tests expressions
\usepackage{polyglossia} % non-break space french punct, bug Warning: "Failed to patch part"
\usepackage[%
  indentfirst=false,
  vskip=1em,
  noorphanfirst=true,
  noorphanafter=true,
  leftmargin=\parindent,
  rightmargin=0pt,
]{quoting}
\usepackage{ragged2e}
\usepackage{setspace}
\usepackage{tabularx} % <table>
\usepackage[explicit]{titlesec} % wear titles, !NO implicit
\usepackage{tikz} % ornaments
\usepackage{tocloft} % styling tocs
\usepackage[fit]{truncate} % used im runing titles
\usepackage{unicode-math}
\usepackage[normalem]{ulem} % breakable \uline, normalem is absolutely necessary to keep \emph
\usepackage{verse} % <l>
\usepackage{xcolor} % named colors
\usepackage{xparse} % @ifundefined
\XeTeXdefaultencoding "iso-8859-1" % bad encoding of xstring
\usepackage{xstring} % string tests
\XeTeXdefaultencoding "utf-8"
\PassOptionsToPackage{hyphens}{url} % before hyperref, which load url package
\usepackage{hyperref} % supposed to be the last one, :o) except for the ones to follow
\urlstyle{same} % after hyperref

% TOTEST
% \usepackage{hypcap} % links in caption ?
% \usepackage{marginnote}
% TESTED
% \usepackage{background} % doesn’t work with xetek
% \usepackage{bookmark} % prefers the hyperref hack \phantomsection
% \usepackage[color, leftbars]{changebar} % 2 cols doc, impossible to keep bar left
% \usepackage[utf8x]{inputenc} % inputenc package ignored with utf8 based engines
% \usepackage[sfdefault,medium]{inter} % no small caps
% \usepackage{firamath} % choose firasans instead, firamath unavailable in Ubuntu 21-04
% \usepackage{flushend} % bad for last notes, supposed flush end of columns
% \usepackage[stable]{footmisc} % BAD for complex notes https://texfaq.org/FAQ-ftnsect
% \usepackage{helvet} % not for XeLaTeX
% \usepackage{multicol} % not compatible with too much packages (longtable, framed, memoir…)
% \usepackage[default,oldstyle,scale=0.95]{opensans} % no small caps
% \usepackage{sectsty} % \chapterfont OBSOLETE
% \usepackage{soul} % \ul for underline, OBSOLETE with XeTeX
% \usepackage[breakable]{tcolorbox} % text styling gone, footnote hack not kept with breakable



% Metadata inserted by a program, from the TEI source, for title page and runing heads
\title{\textbf{ l’Ecclésiaste }}
\date{}
\author{}
\def\elbibl{\emph{l’Ecclésiaste}}

% Default metas
\newcommand{\colorprovide}[2]{\@ifundefinedcolor{#1}{\colorlet{#1}{#2}}{}}
\colorprovide{rubric}{red}
\colorprovide{silver}{Gray}
\@ifundefined{syms}{\newfontfamily\syms{DejaVu Sans}}{}
\newif\ifdev
\@ifundefined{elbibl}{% No meta defined, maybe dev mode
  \newcommand{\elbibl}{Titre court ?}
  \newcommand{\elbook}{Titre du livre source ?}
  \newcommand{\elabstract}{Résumé\par}
  \newcommand{\elurl}{http://oeuvres.github.io/elbook/2}
  \author{Éric Lœchien}
  \title{Un titre de test assez long pour vérifier le comportement d’une maquette}
  \date{1566}
  \devtrue
}{}
\let\eltitle\@title
\let\elauthor\@author
\let\eldate\@date


\defaultfontfeatures{
  % Mapping=tex-text, % no effect seen
  Scale=MatchLowercase,
  Ligatures={TeX,Common},
}

\@ifundefined{\columnseprulecolor}{%
    \patchcmd\@outputdblcol{% find
      \normalcolor\vrule
    }{% and replace by
      \columnseprulecolor\vrule
    }{% success
    }{% failure
      \@latex@warning{Patching \string\@outputdblcol\space failed}%
    }
}{}

\hypersetup{
  % pdftex, % no effect
  pdftitle={\elbibl},
  % pdfauthor={Your name here},
  % pdfsubject={Your subject here},
  % pdfkeywords={keyword1, keyword2},
  bookmarksnumbered=true,
  bookmarksopen=true,
  bookmarksopenlevel=1,
  pdfstartview=Fit,
  breaklinks=true, % avoid long links
  pdfpagemode=UseOutlines,    % pdf toc
  hyperfootnotes=true,
  colorlinks=false,
  pdfborder=0 0 0,
  % pdfpagelayout=TwoPageRight,
  % linktocpage=true, % NO, toc, link only on page no
}


% generic typo commands
\newcommand{\astermono}{\medskip\centerline{\color{rubric}\large\selectfont{\syms ✻}}\medskip\par}%
\newcommand{\astertri}{\medskip\par\centerline{\color{rubric}\large\selectfont{\syms ✻\,✻\,✻}}\medskip\par}%
\newcommand{\asterism}{\bigskip\par\noindent\parbox{\linewidth}{\centering\color{rubric}\large{\syms ✻}\\{\syms ✻}\hskip 0.75em{\syms ✻}}\bigskip\par}%

% lists
\newlength{\listmod}
\setlength{\listmod}{\parindent}
\setlist{
  itemindent=!,
  listparindent=\listmod,
  labelsep=0.2\listmod,
  parsep=0pt,
  % topsep=0.2em, % default topsep is best
}
\setlist[itemize]{
  label=—,
  leftmargin=0pt,
  labelindent=1.2em,
  labelwidth=0pt,
}
\setlist[enumerate]{
  label={\bf\color{rubric}\arabic*.},
  labelindent=0.8\listmod,
  leftmargin=\listmod,
  labelwidth=0pt,
}
\newlist{listalpha}{enumerate}{1}
\setlist[listalpha]{
  label={\bf\color{rubric}\alph*.},
  leftmargin=0pt,
  labelindent=0.8\listmod,
  labelwidth=0pt,
}
\newcommand{\listhead}[1]{\hspace{-1\listmod}\emph{#1}}

\renewcommand{\hrulefill}{%
  \leavevmode\leaders\hrule height 0.2pt\hfill\kern\z@}

% General typo
\DeclareTextFontCommand{\textlarge}{\large}
\DeclareTextFontCommand{\textsmall}{\small}


% commands, inlines
\newcommand{\anchor}[1]{\Hy@raisedlink{\hypertarget{#1}{}}} % link to top of an anchor (not baseline)
\newcommand\abbr[1]{#1}
\newcommand{\autour}[1]{\tikz[baseline=(X.base)]\node [draw=rubric,thin,rectangle,inner sep=1.5pt, rounded corners=3pt] (X) {\color{rubric}#1};}
\newcommand\corr[1]{#1}
\newcommand{\ed}[1]{ {\color{silver}\sffamily\footnotesize (#1)} } % <milestone ed="1688"/>
\newcommand\expan[1]{#1}
\newcommand\foreign[1]{\emph{#1}}
\newcommand\gap[1]{#1}
\renewcommand{\LettrineFontHook}{\color{rubric}}
\newcommand{\initial}[2]{\lettrine[lines=2, loversize=0.3, lhang=0.3]{#1}{#2}}
\newcommand{\initialiv}[2]{%
  \let\oldLFH\LettrineFontHook
  % \renewcommand{\LettrineFontHook}{\color{rubric}\ttfamily}
  \IfSubStr{QJ’}{#1}{
    \lettrine[lines=4, lhang=0.2, loversize=-0.1, lraise=0.2]{\smash{#1}}{#2}
  }{\IfSubStr{É}{#1}{
    \lettrine[lines=4, lhang=0.2, loversize=-0, lraise=0]{\smash{#1}}{#2}
  }{\IfSubStr{ÀÂ}{#1}{
    \lettrine[lines=4, lhang=0.2, loversize=-0, lraise=0, slope=0.6em]{\smash{#1}}{#2}
  }{\IfSubStr{A}{#1}{
    \lettrine[lines=4, lhang=0.2, loversize=0.2, slope=0.6em]{\smash{#1}}{#2}
  }{\IfSubStr{V}{#1}{
    \lettrine[lines=4, lhang=0.2, loversize=0.2, slope=-0.5em]{\smash{#1}}{#2}
  }{
    \lettrine[lines=4, lhang=0.2, loversize=0.2]{\smash{#1}}{#2}
  }}}}}
  \let\LettrineFontHook\oldLFH
}
\newcommand{\labelchar}[1]{\textbf{\color{rubric} #1}}
\newcommand{\milestone}[1]{\autour{\footnotesize\color{rubric} #1}} % <milestone n="4"/>
\newcommand\name[1]{#1}
\newcommand\orig[1]{#1}
\newcommand\orgName[1]{#1}
\newcommand\persName[1]{#1}
\newcommand\placeName[1]{#1}
\newcommand{\pn}[1]{\IfSubStr{-—–¶}{#1}% <p n="3"/>
  {\noindent{\bfseries\color{rubric}   ¶  }}
  {{\footnotesize\autour{ #1}  }}}
\newcommand\reg{}
% \newcommand\ref{} % already defined
\newcommand\sic[1]{#1}
\newcommand\surname[1]{\textsc{#1}}
\newcommand\term[1]{\textbf{#1}}

\def\mednobreak{\ifdim\lastskip<\medskipamount
  \removelastskip\nopagebreak\medskip\fi}
\def\bignobreak{\ifdim\lastskip<\bigskipamount
  \removelastskip\nopagebreak\bigskip\fi}

% commands, blocks
\newcommand{\byline}[1]{\bigskip{\RaggedLeft{#1}\par}\bigskip}
\newcommand{\bibl}[1]{{\RaggedLeft{#1}\par\bigskip}}
\newcommand{\biblitem}[1]{{\noindent\hangindent=\parindent   #1\par}}
\newcommand{\dateline}[1]{\medskip{\RaggedLeft{#1}\par}\bigskip}
\newcommand{\labelblock}[1]{\medbreak{\noindent\color{rubric}\bfseries #1}\par\mednobreak}
\newcommand{\salute}[1]{\bigbreak{#1}\par\medbreak}
\newcommand{\signed}[1]{\bigbreak\filbreak{\raggedleft #1\par}\medskip}

% environments for blocks (some may become commands)
\newenvironment{borderbox}{}{} % framing content
\newenvironment{citbibl}{\ifvmode\hfill\fi}{\ifvmode\par\fi }
\newenvironment{docAuthor}{\ifvmode\vskip4pt\fontsize{16pt}{18pt}\selectfont\fi\itshape}{\ifvmode\par\fi }
\newenvironment{docDate}{}{\ifvmode\par\fi }
\newenvironment{docImprint}{\vskip6pt}{\ifvmode\par\fi }
\newenvironment{docTitle}{\vskip6pt\bfseries\fontsize{18pt}{22pt}\selectfont}{\par }
\newenvironment{msHead}{\vskip6pt}{\par}
\newenvironment{msItem}{\vskip6pt}{\par}
\newenvironment{titlePart}{}{\par }


% environments for block containers
\newenvironment{argument}{\itshape\parindent0pt}{\vskip1.5em}
\newenvironment{biblfree}{}{\ifvmode\par\fi }
\newenvironment{bibitemlist}[1]{%
  \list{\@biblabel{\@arabic\c@enumiv}}%
  {%
    \settowidth\labelwidth{\@biblabel{#1}}%
    \leftmargin\labelwidth
    \advance\leftmargin\labelsep
    \@openbib@code
    \usecounter{enumiv}%
    \let\p@enumiv\@empty
    \renewcommand\theenumiv{\@arabic\c@enumiv}%
  }
  \sloppy
  \clubpenalty4000
  \@clubpenalty \clubpenalty
  \widowpenalty4000%
  \sfcode`\.\@m
}%
{\def\@noitemerr
  {\@latex@warning{Empty `bibitemlist' environment}}%
\endlist}
\newenvironment{quoteblock}% may be used for ornaments
  {\begin{quoting}}
  {\end{quoting}}

% table () is preceded and finished by custom command
\newcommand{\tableopen}[1]{%
  \ifnum\strcmp{#1}{wide}=0{%
    \begin{center}
  }
  \else\ifnum\strcmp{#1}{long}=0{%
    \begin{center}
  }
  \else{%
    \begin{center}
  }
  \fi\fi
}
\newcommand{\tableclose}[1]{%
  \ifnum\strcmp{#1}{wide}=0{%
    \end{center}
  }
  \else\ifnum\strcmp{#1}{long}=0{%
    \end{center}
  }
  \else{%
    \end{center}
  }
  \fi\fi
}


% text structure
\newcommand\chapteropen{} % before chapter title
\newcommand\chaptercont{} % after title, argument, epigraph…
\newcommand\chapterclose{} % maybe useful for multicol settings
\setcounter{secnumdepth}{-2} % no counters for hierarchy titles
\setcounter{tocdepth}{5} % deep toc
\markright{\@title} % ???
\markboth{\@title}{\@author} % ???
\renewcommand\tableofcontents{\@starttoc{toc}}
% toclof format
% \renewcommand{\@tocrmarg}{0.1em} % Useless command?
% \renewcommand{\@pnumwidth}{0.5em} % {1.75em}
\renewcommand{\@cftmaketoctitle}{}
\setlength{\cftbeforesecskip}{\z@ \@plus.2\p@}
\renewcommand{\cftchapfont}{}
\renewcommand{\cftchapdotsep}{\cftdotsep}
\renewcommand{\cftchapleader}{\normalfont\cftdotfill{\cftchapdotsep}}
\renewcommand{\cftchappagefont}{\bfseries}
\setlength{\cftbeforechapskip}{0em \@plus\p@}
% \renewcommand{\cftsecfont}{\small\relax}
\renewcommand{\cftsecpagefont}{\normalfont}
% \renewcommand{\cftsubsecfont}{\small\relax}
\renewcommand{\cftsecdotsep}{\cftdotsep}
\renewcommand{\cftsecpagefont}{\normalfont}
\renewcommand{\cftsecleader}{\normalfont\cftdotfill{\cftsecdotsep}}
\setlength{\cftsecindent}{1em}
\setlength{\cftsubsecindent}{2em}
\setlength{\cftsubsubsecindent}{3em}
\setlength{\cftchapnumwidth}{1em}
\setlength{\cftsecnumwidth}{1em}
\setlength{\cftsubsecnumwidth}{1em}
\setlength{\cftsubsubsecnumwidth}{1em}

% footnotes
\newif\ifheading
\newcommand*{\fnmarkscale}{\ifheading 0.70 \else 1 \fi}
\renewcommand\footnoterule{\vspace*{0.3cm}\hrule height \arrayrulewidth width 3cm \vspace*{0.3cm}}
\setlength\footnotesep{1.5\footnotesep} % footnote separator
\renewcommand\@makefntext[1]{\parindent 1.5em \noindent \hb@xt@1.8em{\hss{\normalfont\@thefnmark . }}#1} % no superscipt in foot


% orphans and widows
\clubpenalty=9996
\widowpenalty=9999
\brokenpenalty=4991
\predisplaypenalty=10000
\postdisplaypenalty=1549
\displaywidowpenalty=1602
\hyphenpenalty=400
% Copied from Rahtz but not understood
\def\@pnumwidth{1.55em}
\def\@tocrmarg {2.55em}
\def\@dotsep{4.5}
\emergencystretch 3em
\hbadness=4000
\pretolerance=750
\tolerance=2000
\vbadness=4000
\def\Gin@extensions{.pdf,.png,.jpg,.mps,.tif}
% \renewcommand{\@cite}[1]{#1} % biblio

\makeatother % /@@@>
%%%%%%%%%%%%%%
% </TEI> end %
%%%%%%%%%%%%%%


%%%%%%%%%%%%%
% footnotes %
%%%%%%%%%%%%%
\renewcommand{\thefootnote}{\bfseries\textcolor{rubric}{\arabic{footnote}}} % color for footnote marks

%%%%%%%%%
% Fonts %
%%%%%%%%%
\usepackage[]{roboto} % SmallCaps, Regular is a bit bold
% \linespread{0.90} % too compact, keep font natural
\newfontfamily\fontrun[]{Roboto Condensed Light} % condensed runing heads
\ifav
  \setmainfont[
    ItalicFont={Roboto Light Italic},
  ]{Roboto}
\else\ifbooklet
  \setmainfont[
    ItalicFont={Roboto Light Italic},
  ]{Roboto}
\else
\setmainfont[
  ItalicFont={Roboto Italic},
]{Roboto Light}
\fi\fi
\renewcommand{\LettrineFontHook}{\bfseries\color{rubric}}
% \renewenvironment{labelblock}{\begin{center}\bfseries\color{rubric}}{\end{center}}

%%%%%%%%
% MISC %
%%%%%%%%

\setdefaultlanguage[frenchpart=false]{french} % bug on part


\newenvironment{quotebar}{%
    \def\FrameCommand{{\color{rubric!10!}\vrule width 0.5em} \hspace{0.9em}}%
    \def\OuterFrameSep{\itemsep} % séparateur vertical
    \MakeFramed {\advance\hsize-\width \FrameRestore}
  }%
  {%
    \endMakeFramed
  }
\renewenvironment{quoteblock}% may be used for ornaments
  {%
    \savenotes
    \setstretch{0.9}
    \normalfont
    \begin{quotebar}
  }
  {%
    \end{quotebar}
    \spewnotes
  }


\renewcommand{\headrulewidth}{\arrayrulewidth}
\renewcommand{\headrule}{{\color{rubric}\hrule}}

% delicate tuning, image has produce line-height problems in title on 2 lines
\titleformat{name=\chapter} % command
  [display] % shape
  {\vspace{1.5em}\centering} % format
  {} % label
  {0pt} % separator between n
  {}
[{\color{rubric}\huge\textbf{#1}}\bigskip] % after code
% \titlespacing{command}{left spacing}{before spacing}{after spacing}[right]
\titlespacing*{\chapter}{0pt}{-2em}{0pt}[0pt]

\titleformat{name=\section}
  [block]{}{}{}{}
  [\vbox{\color{rubric}\large\raggedleft\textbf{#1}}]
\titlespacing{\section}{0pt}{0pt plus 4pt minus 2pt}{\baselineskip}

\titleformat{name=\subsection}
  [block]
  {}
  {} % \thesection
  {} % separator \arrayrulewidth
  {}
[\vbox{\large\textbf{#1}}]
% \titlespacing{\subsection}{0pt}{0pt plus 4pt minus 2pt}{\baselineskip}

\ifaiv
  \fancypagestyle{main}{%
    \fancyhf{}
    \setlength{\headheight}{1.5em}
    \fancyhead{} % reset head
    \fancyfoot{} % reset foot
    \fancyhead[L]{\truncate{0.45\headwidth}{\fontrun\elbibl}} % book ref
    \fancyhead[R]{\truncate{0.45\headwidth}{ \fontrun\nouppercase\leftmark}} % Chapter title
    \fancyhead[C]{\thepage}
  }
  \fancypagestyle{plain}{% apply to chapter
    \fancyhf{}% clear all header and footer fields
    \setlength{\headheight}{1.5em}
    \fancyhead[L]{\truncate{0.9\headwidth}{\fontrun\elbibl}}
    \fancyhead[R]{\thepage}
  }
\else
  \fancypagestyle{main}{%
    \fancyhf{}
    \setlength{\headheight}{1.5em}
    \fancyhead{} % reset head
    \fancyfoot{} % reset foot
    \fancyhead[RE]{\truncate{0.9\headwidth}{\fontrun\elbibl}} % book ref
    \fancyhead[LO]{\truncate{0.9\headwidth}{\fontrun\nouppercase\leftmark}} % Chapter title, \nouppercase needed
    \fancyhead[RO,LE]{\thepage}
  }
  \fancypagestyle{plain}{% apply to chapter
    \fancyhf{}% clear all header and footer fields
    \setlength{\headheight}{1.5em}
    \fancyhead[L]{\truncate{0.9\headwidth}{\fontrun\elbibl}}
    \fancyhead[R]{\thepage}
  }
\fi

\ifav % a5 only
  \titleclass{\section}{top}
\fi

\newcommand\chapo{{%
  \vspace*{-3em}
  \centering % no vskip ()
  {\Large\addfontfeature{LetterSpace=25}\bfseries{\elauthor}}\par
  \smallskip
  {\large\eldate}\par
  \bigskip
  {\Large\selectfont{\eltitle}}\par
  \bigskip
  {\color{rubric}\hline\par}
  \bigskip
  {\Large LIVRE LIBRE À PRIX LIBRE, DEMANDEZ AU COMPTOIR\par}
  \centerline{\small\color{rubric} {hurlus.fr, tiré le \today}}\par
  \bigskip
}}


\begin{document}
\pagestyle{empty}
\ifbooklet{
  \thispagestyle{empty}
  \centering
  {\LARGE\bfseries{\elauthor}}\par
  \bigskip
  {\Large\eldate}\par
  \bigskip
  \bigskip
  {\LARGE\selectfont{\eltitle}}\par
  \vfill\null
  {\color{rubric}\setlength{\arrayrulewidth}{2pt}\hline\par}
  \vfill\null
  {\Large LIVRE LIBRE À PRIX LIBRE, DEMANDEZ AU COMPTOIR\par}
  \centerline{\small{hurlus.fr, tiré le \today}}\par
  \newpage\null\thispagestyle{empty}\newpage
  \addtocounter{page}{-2}
}\fi

\thispagestyle{empty}
\ifaiv
  \twocolumn[\chapo]
\else
  \chapo
\fi
{\it\elabstract}
\bigskip
\makeatletter\@starttoc{toc}\makeatother % toc without new page
\bigskip

\pagestyle{main} % after style

  \section[Chapitre 1]{Chapitre 1}\renewcommand{\leftmark}{Chapitre 1}


\begin{verse}
Paroles de Qohéleth, fils de David, roi à Jérusalem.\\
Vanité des vanités, dit Qohéleth, \\
vanité des vanités, tout est vanité.\\!
\end{verse}

\labelblock{Prologue}


\begin{verse}
Quel profit y a-t-il pour l’homme \\
de tout le travail qu’il fait sous le soleil ?\\
Un âge s’en va, un autre vient, \\
et la terre subsiste toujours.\\
Le soleil se lève et le soleil se couche, \\
il aspire à ce lieu d’où il se lève.\\
Le vent va vers le midi et tourne vers le nord, \\
le vent tourne, tourne et s’en va, \\
et le vent reprend ses tours.\\
Tous les torrents vont vers la mer, \\
et la mer n’est pas remplie ; \\
vers le lieu où vont les torrents, \\
là-bas, ils s’en vont de nouveau.\\
Tous les mots sont usés, on ne peut plus les dire, \\
l’œil ne se contente pas de ce qu’il voit, \\
et l’oreille ne se remplit pas de ce qu’elle entend.\\
Ce qui a été, c’est ce qui sera, \\
ce qui s’est fait, c’est ce qui se fera : \\
rien de nouveau sous le soleil !\\
S’il est une chose dont on puisse dire : \\
« Voyez, c’est nouveau, cela ! » \\
– cela existe déjà depuis les siècles qui nous ont précédés.\\
Il n’y a aucun souvenir des temps anciens ; \\
quant aux suivants qui viendront, \\
il ne restera d’eux aucun souvenir \\
chez ceux qui viendront après.\\!
\end{verse}

\labelblock{Confession du roi Salomon}


\begin{verse}
Moi, Qohéleth, j’ai été roi sur Israël, à Jérusalem.\\
J’ai eu à cœur de chercher et d’explorer par la sagesse \\
tout ce qui se fait sous le ciel. \\
C’est une occupation de malheur que Dieu a donnée \\
aux fils d’Adam pour qu’ils s’y appliquent.\\
J’ai vu toutes les œuvres qui se font sous le soleil ; \\
mais voici que tout est vanité et poursuite de vent.\\
Ce qui est courbé, on ne peut le redresser, \\
ce qui fait défaut ne peut être compté.\\!Je me suis dit à moi-même : \\
« Voici que j’ai fait grandir et progresser la sagesse \\
plus que quiconque m’a précédé comme roi sur Jérusalem. » \\
J’ai fait l’expérience de beaucoup de sagesse et de science,\\
j’ai eu à cœur de connaître la sagesse \\
et de connaître la folie et la sottise ; \\
j’ai connu que cela aussi, c’est poursuite de vent.\\
Car en beaucoup de sagesse, il y a beaucoup d’affliction ; \\
qui augmente le savoir augmente la douleur.\\!
\end{verse}
\section[Chapitre 2]{Chapitre 2}\renewcommand{\leftmark}{Chapitre 2}


\begin{verse}
Je me suis dit en moi-même : \\
« Allons, que je t’éprouve par la joie, goûte au bonheur ! » \\
Et voici, cela aussi est vanité.\\
Du rire, j’ai dit : « C’est fou ! » \\
Et de la joie : « Qu’est-ce que cela fait ? »\\
J’ai délibéré en mon cœur \\
de traîner ma chair dans le vin \\
et tout en conduisant mon cœur avec sagesse, \\
de tenir à la sottise, \\
le temps de voir ce qu’il est bon pour les fils d’Adam \\
de faire sous le ciel \\
pendant les jours comptés de leur vie.\\!J’ai entrepris de grandes œuvres : \\
je me suis bâti des maisons, planté des vignes ;\\
je me suis fait des jardins et des vergers, \\
j’y ai planté toutes sortes d’arbres fruitiers ;\\
je me suis fait des bassins \\
pour arroser de leur eau une forêt de jeunes arbres.\\!J’ai acheté des esclaves et des servantes, j’ai eu des domestiques, \\
et aussi du gros et du petit bétail en abondance \\
plus que tous mes prédécesseurs à Jérusalem.\\
J’ai aussi amassé de l’argent et de l’or, \\
la fortune des rois et des Etats ; \\
je me suis procuré des chanteurs et des chanteuses \\
et, délices des fils d’Adam, une dame, des dames.\\!Je devins grand, je m’enrichis \\
plus que tous mes prédécesseurs à Jérusalem. \\
Cependant ma sagesse, elle, m’assistait.\\
Je n’ai rien refusé à mes yeux de ce qu’ils demandaient ; \\
je n’ai privé mon cœur d’aucune joie, \\
car mon cœur jouissait de tout mon travail : \\
c’était la part qui me revenait de tout mon travail.\\
Mais je me suis tourné vers toutes les œuvres \\
qu’avaient faites mes mains \\
et vers le travail que j’avais eu tant de mal à faire. \\
Eh bien ! tout cela est vanité et poursuite de vent, \\
on n’en a aucun profit sous le soleil.\\!
\end{verse}

\labelblock{Bilan décevant}

Je me suis aussi tourné, pour les considérer, \\
vers sagesse, folie et sottise. \\
Voyons ! que sera l’homme qui viendra après le roi ? \\
Ce qu’on aura déjà fait de lui !\\

\begin{verse}
Voici ce que j’ai vu : \\
On profite de la sagesse plus que de la sottise, \\
comme on profite de la lumière plus que des ténèbres.\\
Le sage a les yeux là où il faut, \\
l’insensé marche dans les ténèbres. \\
Mais je sais, moi, qu’à tous les deux \\
un même sort arrivera.\\!Alors, moi, je me dis en moi-même : \\
Ce qui arrive à l’insensé m’arrivera aussi, \\
pourquoi donc ai-je été si sage ? \\
Je me dis à moi-même que cela aussi est vanité.\\
Car il n’y a pas de souvenir du sage, \\
pas plus que de l’insensé, pour toujours. \\
Déjà dans les jours qui viennent, tout sera oublié : \\
Eh quoi ? le sage meurt comme l’insensé !\\!Donc, je déteste la vie, \\
car je trouve mauvais ce qui se fait sous le soleil : \\
tout est vanité et poursuite de vent.\\
Moi, je déteste tout le travail que j’ai fait sous le soleil \\
et que j’abandonnerai à l’homme qui me succédera.\\
Qui sait s’il sera sage ou insensé ? \\
Il sera maître de tout mon travail, \\
que j’aurai fait avec ma sagesse sous le soleil : \\
cela aussi est vanité.\\!J’en suis venu à me décourager \\
pour tout le travail que j’ai fait sous le soleil.\\
En effet, voici un homme qui a fait son travail \\
avec sagesse, science et succès : \\
C’est à un homme qui n’y a pas travaillé \\
qu’il donnera sa part. \\
Cela aussi est vanité et grand mal.\\
Oui, que reste-t-il pour cet homme \\
de tout son travail et de tout l’effort personnel \\
qu’il aura fait, lui, sous le soleil ?\\
Tous ses jours, en effet, ne sont que douleur, \\
et son occupation n’est qu’affliction ; \\
même la nuit, son cœur est sans repos : \\
cela aussi est vanité.\\
Rien de bon pour l’homme, sinon de manger et de boire, \\
de goûter le bonheur dans son travail. \\
J’ai vu, moi, que cela aussi vient de la main de Dieu.\\!
\end{verse}
\noindent \pn{25}« Car qui a de quoi manger, qui sait jouir, sinon moi ? »  \milestone{26} Oui, il donne à l’homme qui lui plaît sagesse, science et joie, mais au pécheur il donne comme occupation de rassembler et d’amasser, pour donner à celui qui plaît à Dieu. Cela aussi est vanité et poursuite de vent.
\section[Chapitre 3]{Chapitre 3}\renewcommand{\leftmark}{Chapitre 3}


\labelblock{Les temps et la durée}


\begin{verse}
Il y a un moment pour tout \\
et un temps pour chaque chose sous le ciel :\\
un temps pour enfanter et un temps pour mourir, \\
un temps pour planter et un temps pour arracher le plant,\\
un temps pour tuer et un temps pour guérir, \\
un temps pour saper et un temps pour bâtir,\\!un temps pour pleurer et un temps pour rire, \\
un temps pour se lamenter et un temps pour danser,\\
un temps pour jeter des pierres et un temps pour amasser des pierres, \\
un temps pour embrasser et un temps pour éviter d’embrasser,\\!un temps pour chercher et un temps pour perdre, \\
un temps pour garder et un temps pour jeter,\\
un temps pour déchirer et un temps pour coudre, \\
un temps pour se taire et un temps pour parler,\\
un temps pour aimer et un temps pour haïr, \\
un temps de guerre et un temps de paix.\\
Quel profit a l’artisan du travail qu’il fait ?\\!Je vois l’occupation que Dieu a donnée \\
aux fils d’Adam pour qu’ils s’y occupent.\\
Il fait toute chose belle en son temps ; \\
à leur cœur il donne même le sens de la durée \\
sans que l’homme puisse découvrir \\
l’œuvre que fait Dieu depuis le début jusqu’à la fin.\\!Je sais qu’il n’y a rien de bon pour lui \\
que de se réjouir et de se donner du bon temps durant sa vie.\\
Et puis, tout homme qui mange et boit \\
et goûte au bonheur en tout son travail, \\
cela, c’est un don de Dieu.\\
Je sais que tout ce que fait Dieu, cela durera toujours ; \\
il n’y a rien à y ajouter, ni rien à en retrancher, \\
et Dieu fait en sorte qu’on ait de la crainte devant sa face.\\!
\end{verse}
Ce qui est a déjà été, et ce qui sera a déjà été, \\
et Dieu va rechercher ce qui a disparu.\\

\labelblock{Justice et rétribution}


\begin{verse}
J’ai encore vu sous le soleil \\
qu’au siège du jugement, là était la méchanceté, \\
et qu’au siège de la justice, là était la méchanceté.\\
Je me suis dit en moi-même : \\
Dieu jugera le juste et le méchant, \\
car il y a là un temps \\
pour chaque chose et pour chaque action.\\!Je me suis dit en moi-même, \\
au sujet des fils d’Adam, \\
que Dieu veut les éprouver ; \\
alors on verra qu’en eux-mêmes, ils ne sont que des bêtes.\\
Car le sort des fils d’Adam, c’est le sort de la bête, \\
c’est un sort identique : \\
telle la mort de celle-ci, telle la mort de ceux-là ; \\
ils ont tous un souffle identique : \\
la supériorité de l’homme sur la bête est nulle, \\
car tout est vanité.\\!Tout va vers un lieu unique, \\
tout vient de la poussière \\
et tout retourne à la poussière.\\
Qui connaît le souffle des fils d’Adam \\
qui monte, lui, vers le haut, \\
tandis que le souffle des bêtes \\
descend vers le bas, vers la terre ?\\!
\end{verse}
Je vois qu’il n’y a rien de mieux pour l’homme \\
que de jouir de ses œuvres, car telle est sa part. \\
Qui en effet l’emmènera voir ce qui sera après lui ?\\
\section[Chapitre 4]{Chapitre 4}\renewcommand{\leftmark}{Chapitre 4}


\labelblock{Le sort des opprimés}

D’autre part, je vois toutes les oppressions \\
qui se pratiquent sous le soleil. \\
Regardez les pleurs des opprimés : \\
ils n’ont pas de consolateur ; \\
la force est du côté des oppresseurs : \\
ils n’ont pas de consolateur.\\

\begin{verse}
Et moi, de féliciter les morts qui sont déjà morts \\
plutôt que les vivants qui sont encore en vie.\\
Et plus heureux que les deux \\
celui qui n’a pas encore été, \\
puisqu’il n’a pas vu l’œuvre mauvaise \\
qui se pratique sous le soleil.\\!
\end{verse}

\labelblock{Le travail et ses risques}

Je vois, moi, que tout le travail, \\
tout le succès d’une œuvre, \\
c’est jalousie des uns envers les autres : \\
cela est aussi vanité et poursuite de vent.\\

\begin{verse}
L’insensé se croise les bras \\
et dévore sa propre chair :\\
Mieux vaut le creux de la main plein de repos \\
que deux poignées de travail, de poursuite de vent.\\!
\end{verse}

\labelblock{La solitude et ses inconvénients}


\begin{verse}
Par ailleurs je vois une vanité sous le soleil.\\
Voici un homme seul, sans compagnon, \\
n’ayant ni fils ni frère. \\
Pas de limite à tout son travail, \\
même ses yeux ne sont jamais rassasiés de richesses. \\
Alors, moi, je travaille, \\
je me prive de bonheur : c’est pour qui ? \\
Cela est aussi vanité, c’est une mauvaise affaire.\\!Deux hommes valent mieux qu’un seul, \\
car ils ont un bon salaire pour leur travail.\\
En effet, s’ils tombent, l’un relève l’autre. \\
Mais malheur à celui qui est seul ! \\
S’il tombe, il n’a pas de second pour le relever.\\
De plus, s’ils couchent à deux, ils ont chaud, \\
mais celui qui est seul, comment se réchauffera-t-il ?\\
Et si quelqu’un vient à bout de celui qui est seul, \\
deux lui tiendront tête ; \\
un fil triple ne rompt pas vite.\\!
\end{verse}

\labelblock{Le pouvoir politique et ses risques}


\begin{verse}
Mieux vaut un gamin indigent, mais sage, \\
qu’un roi vieux, mais insensé, qui ne sait plus se laisser conseiller.\\
Que ce garçon soit sorti de prison pour régner, \\
qu’il soit même né mendiant pour exercer sa royauté,\\
j’ai vu tous les vivants qui marchent sous le soleil \\
être du côté du gamin, du second, \\
celui qui surgit à la place de l’autre.\\
Pas de fin à tout ce peuple, à tous ceux dont il est le chef. \\
Toutefois la postérité pourrait bien ne pas s’en réjouir, \\
car cela aussi est vanité et poursuite de vent.\\!
\end{verse}

\labelblock{Le geste rituel et ses risques}

Surveille tes pas quand tu vas à la Maison de Dieu, \\
approche-toi pour écouter plutôt que pour offrir le sacrifice des insensés ; \\
car ils ne savent pas qu’ils font le mal.\\
\section[Chapitre 5]{Chapitre 5}\renewcommand{\leftmark}{Chapitre 5}


\begin{verse}
Que ta bouche ne se précipite pas \\
et que ton cœur ne se hâte pas \\
de proférer une parole devant Dieu. \\
Car Dieu est dans le ciel, et toi sur la terre. \\
Donc, que tes paroles soient peu nombreuses !\\
Car de l’abondance des occupations vient le rêve \\
et de l’abondance des paroles, les propos ineptes.\\!
\end{verse}
Si tu fais un vœu à Dieu, \\
ne tarde pas à l’accomplir. \\
Car il n’y a pas de faveur pour les insensés ; \\
le vœu que tu as fait, accomplis-le.\\

\begin{verse}
Mieux vaut pour toi ne pas faire de vœu \\
que faire un vœu et ne pas l’accomplir.\\
Ne laisse pas ta bouche te rendre coupable tout entier, \\
et ne va pas dire au messager de Dieu : « C’est une méprise. » \\
Pourquoi Dieu devrait-il s’irriter de tes propos \\
et ruiner l’œuvre de tes mains ?\\
Quand il y a abondance de rêves, de vanités, \\
et beaucoup de paroles, alors, crains Dieu.\\!
\end{verse}

\labelblock{L’inévitable autorité et ses abus}


\begin{verse}
Si, dans l’Etat, tu vois l’indigent opprimé, \\
le droit et la justice violés, \\
ne sois pas surpris de la chose ; \\
car au-dessus d’un grand personnage \\
veille un autre grand, \\
et au-dessus d’eux, il y a encore des grands.\\
Et à tous, la terre profite ; \\
le roi est tributaire de l’agriculture.\\!
\end{verse}

\labelblock{La richesse et ses risques}


\begin{verse}
Qui aime l’argent ne se rassasiera pas d’argent, \\
ni du revenu celui qui aime le luxe. \\
Cela est aussi vanité.\\
Avec l’abondance des biens abondent ceux qui les consomment, \\
et quel bénéfice pour le propriétaire, \\
sinon un spectacle pour les yeux ?\\
Doux est le sommeil de l’ouvrier, \\
qu’il ait mangé peu ou beaucoup ; \\
mais la satiété du riche, elle, ne le laisse pas dormir.\\!Il y a un mal affligeant que j’ai vu sous le soleil : \\
la richesse conservée par son propriétaire pour son malheur.\\
Cette richesse périt dans une mauvaise affaire ; \\
s’il engendre un fils, celui-ci n’a plus rien en main.\\
Comme il est sorti du sein de sa mère, \\
nu, il s’en retournera comme il était venu : \\
il n’a rien retiré de son travail \\
qu’il puisse emporter avec lui.\\
Et cela est aussi un mal affligeant \\
qu’il s’en aille ainsi qu’il était venu : \\
quel profit pour lui d’avoir travaillé pour du vent ?\\
De plus, il consume tous ses jours dans les ténèbres ; \\
il est grandement affligé, déprimé, irrité.\\
Ce que, moi, je reconnais comme bien, le voici : \\
il convient de manger et de boire, \\
de goûter le bonheur dans tout le travail \\
que l’homme fait sous le soleil, \\
pendant le nombre des jours de vie que Dieu lui donne, \\
car telle est sa part.\\
De plus, tout homme à qui Dieu donne richesse et ressources \\
et à qui Il a laissé la faculté d’en manger, \\
d’en prendre sa part et de jouir de son travail, \\
c’est là un don de Dieu ;\\
non, il ne songe guère aux jours de sa vie, \\
tant que Dieu le tient attentif à la joie de son cœur.\\!
\end{verse}
\section[Chapitre 6]{Chapitre 6}\renewcommand{\leftmark}{Chapitre 6}


\begin{verse}
Il y a un mal que j’ai vu sous le soleil, \\
et il est immense pour l’humanité.\\
Soit un homme à qui Dieu donne richesse, ressources et gloire, \\
à qui rien ne manque pour lui-même de tout ce qu’il désire, \\
mais à qui Dieu ne laisse pas la faculté d’en manger, \\
car c’est quelqu’un d’étranger qui le mange : \\
cela aussi est vanité et mal affligeant.\\!
\end{verse}

\labelblock{La longévité et ses déceptions}


\begin{verse}
Soit un homme qui engendre cent fois \\
et vit de nombreuses années, \\
mais qui, si nombreux soient les jours de ses années, \\
ne se rassasie pas de bonheur \\
et n’a même pas de sépulture. \\
Je dis : L’avorton vaut mieux que lui,\\
car c’est en vain qu’il est venu \\
et il s’en va dans les ténèbres, \\
et par les ténèbres son nom sera recouvert ;\\
il n’a même pas vu le soleil et ne l’a pas connu, \\
il a du repos plus que l’autre.\\
Même si celui-ci avait vécu deux fois mille ans, \\
il n’aurait pas goûté le bonheur. \\
N’est-ce pas vers un lieu unique que tout va ?\\!
\end{verse}

\labelblock{L’homme demeure insatisfait}


\begin{verse}
Tout le travail de l’homme est pour sa bouche, \\
et pourtant l’appétit n’est pas comblé.\\
En effet, qu’a de plus le sage que l’insensé, \\
qu’a le pauvre qui sait aller de l’avant face à la vie ?\\
Mieux vaut la vision des yeux que le mouvement de l’appétit : \\
cela est aussi vanité et poursuite de vent.\\!Ce qui a été a déjà reçu un nom \\
et on sait ce que c’est, l’homme ; \\
mais il ne peut entrer en procès \\
avec plus fort que lui.\\
Quand il y a des paroles en abondance, \\
elles font abonder la vanité : \\
qu’est-ce que l’homme a de plus ?\\!
\end{verse}
En effet, qui sait ce qui est le mieux pour l’homme pendant l’existence, \\
pendant les nombreux jours de sa vaine existence \\
qu’il passe comme une ombre ? \\
Qui indiquera donc à l’homme \\
ce qui sera après lui sous le soleil ?\\
\section[Chapitre 7]{Chapitre 7}\renewcommand{\leftmark}{Chapitre 7}


\labelblock{Relativité des biens}

Mieux vaut le renom que l’huile exquise, \\
et le jour de la mort que le jour de la naissance.\\

\begin{verse}
Mieux vaut aller à la maison de deuil \\
qu’à la maison du banquet ; \\
puisque c’est la fin de tout homme, \\
il faut que les vivants y appliquent leur cœur.\\
Mieux vaut le chagrin que le rire, \\
car sous un visage en peine, le cœur peut être heureux ;\\
le cœur des sages est dans la maison de deuil, \\
et le cœur des insensés, dans la maison de joie.\\!Mieux vaut écouter la semonce du sage, \\
qu’être homme à écouter la chanson des insensés.\\
Car, tel le pétillement des broussailles sous la marmite, \\
tel est le rire de l’insensé. \\
Mais cela aussi est vanité,\\
que l’oppression rende fou le sage \\
et qu’un présent perde le cœur.\\!Mieux vaut l’aboutissement d’une chose que ses prémices, \\
mieux vaut un esprit patient qu’un esprit prétentieux.\\
Que ton esprit ne se hâte pas de s’irriter, \\
car l’irritation vit au cœur des insensés.\\
Ne dis pas : Comment se fait-il \\
que les temps anciens aient été meilleurs que ceux-ci ? \\
Ce n’est pas la sagesse \\
qui te fait poser cette question.\\!La sagesse est bonne comme un héritage ; \\
elle profite à ceux qui voient le soleil :\\
Car être à l’ombre de la sagesse, \\
c’est être à l’ombre de l’argent, \\
et le profit du savoir, \\
c’est que la sagesse fait vivre ceux qui la possèdent.\\
Regarde l’œuvre de Dieu : \\
Qui donc pourra réparer ce qu’Il a courbé ?\\
Au jour du bonheur, sois heureux, \\
et au jour du malheur, regarde : \\
celui-ci autant que celui-là, Dieu les a faits \\
de façon que l’homme ne puisse rien découvrir \\
de ce qui sera après lui.\\!
\end{verse}

\labelblock{Justice et sagesse}


\begin{verse}
Dans ma vaine existence, j’ai tout vu : \\
un juste qui se perd par sa justice, \\
un méchant qui survit par sa malice.\\
Ne sois pas juste à l’excès, \\
ne te fais pas trop sage ; \\
pourquoi te détruire ?\\
Ne fais pas trop le méchant \\
et ne deviens pas insensé ; \\
pourquoi mourir avant ton temps ?\\!Il est bon que tu tiennes à ceci \\
sans laisser ta main lâcher cela. \\
Car celui qui craint Dieu \\
fera aboutir l’une et l’autre chose.\\
La sagesse rend le sage plus fort \\
que dix gouverneurs présents dans une ville.\\!Car aucun homme n’est assez juste sur terre \\
pour faire le bien sans pécher.\\
D’ailleurs à tous les propos qu’on profère, \\
ne prête pas attention : \\
ainsi, tu n’entendras pas ton serviteur te dénigrer,\\
car bien des fois, tu as eu conscience, \\
toi aussi, de dénigrer les autres.\\!
\end{verse}

\labelblock{Introuvable sagesse chez l’homme et la femme}


\begin{verse}
J’ai essayé tout cela avec sagesse, \\
je disais : Je serai un sage. \\
Mais elle est loin de ma portée.\\
Ce qui est venu à l’existence est lointain \\
et profond, profond ! Qui le découvrira ?\\
Moi, je m’appliquerai de tout cœur \\
à connaître, à explorer, à rechercher \\
la sagesse et la logique, \\
à connaître aussi que la méchanceté est une sottise, \\
une sottise affolante.\\!Et je trouve, moi, plus amère que la mort \\
une femme quand elle est un traquenard, \\
et son cœur un filet, ses mains des liens : \\
celui qui plaît à Dieu lui échappera, \\
mais le pécheur se laissera prendre par elle.\\
Voilà ce que j’ai trouvé, a dit Qohéleth, \\
en les voyant l’une après l’autre pour trouver une opinion.\\
J’en suis encore à chercher et n’ai pas trouvé : \\
Un homme sur mille, je l’ai trouvé, \\
mais une femme parmi elles toutes, \\
je ne l’ai pas trouvée.\\
Seulement, vois-tu ce que j’ai trouvé : \\
Dieu a fait l’homme droit, \\
mais eux ils ont cherché une foule de complications.\\!
\end{verse}
\section[Chapitre 8]{Chapitre 8}\renewcommand{\leftmark}{Chapitre 8}


\labelblock{Le sage face au pouvoir}

Qui est comme le sage \\
et sait interpréter cette parole : \\
« La sagesse d’un homme illumine son visage \\
et la dureté de son visage en est transformée » ?\\

\begin{verse}
Moi ! Observe l’ordre du roi, \\
et, à cause du serment divin,\\
ne te presse pas de t’écarter de lui, \\
ne t’obstine pas dans un mauvais cas. \\
car il fera tout ce qui lui plaira,\\
car la parole du roi est souveraine, \\
et qui lui dira : « Que fais-tu ? »\\!
\end{verse}

\labelblock{Le juste ignore l’heure du jugement}


\begin{verse}
« Celui qui observe le commandement \\
ne connaîtra rien de mauvais. \\
Le temps et le jugement, le cœur du sage les connaît. »\\
Oui, il y a pour chaque chose un temps et un jugement, \\
mais il y a un grand malheur pour l’homme :\\
il ne sait pas ce qui arrivera, \\
qui lui indiquera quand cela arrivera ?\\
Personne n’a de pouvoir sur le souffle vital \\
pour retenir ce souffle ; \\
personne n’a de pouvoir sur le jour de la mort ; \\
il n’y a pas de relâche dans le combat, \\
et la méchanceté ne sauve pas son homme.\\!
\end{verse}
Tout cela, je l’ai vu en portant mon attention \\
sur tout ce qui se fait sous le soleil, \\
au temps où l’homme a sur l’homme \\
le pouvoir de lui faire du mal.\\

\labelblock{La joie reste possible malgré l’absence de rétribution}


\begin{verse}
Ainsi, j’ai vu des méchants mis au tombeau ; \\
on allait et venait depuis le lieu saint \\
et on oubliait dans la ville comme ils avaient agi. \\
Cela aussi est vanité.\\
Parce que la sentence contre l’œuvre mauvaise \\
n’est pas vite exécutée, \\
le cœur des fils d’Adam est rempli de malfaisance.\\
Que le pécheur fasse le mal cent fois, \\
alors même il prolonge sa vie. \\
Je sais pourtant, moi aussi, \\
« qu’il y aura du bonheur pour ceux qui craignent Dieu, \\
parce qu’ils ont de la crainte devant sa face,\\
mais qu’il n’y aura pas de bonheur pour le méchant \\
et que, passant comme l’ombre, il ne prolongera pas ses jours, \\
parce qu’il est sans crainte devant la face de Dieu ».\\
Il est un fait, sur la terre, qui est vanité : \\
il est des justes qui sont traités selon le fait des méchants, \\
et des méchants qui sont traités selon le fait des justes. \\
J’ai déjà dit que cela est aussi vanité,\\
et je fais l’éloge de la joie ; \\
car il n’y a pour l’homme sous le soleil \\
rien de bon, sinon de manger, de boire, de se réjouir ; \\
et cela l’accompagne dans son travail \\
durant les jours d’existence \\
que Dieu lui donne sous le soleil.\\!
\end{verse}

\labelblock{La sagesse dépose son bilan de faillite}


\begin{verse}
Quand j’eus à cœur de connaître la sagesse \\
et de voir les occupations auxquelles on s’affaire sur terre, \\
– même si, le jour et la nuit, l’homme ne voit pas de ses yeux le sommeil –\\
alors j’ai vu toute l’œuvre de Dieu ; \\
l’homme ne peut découvrir l’œuvre qui se fait sous le soleil, \\
bien que l’homme travaille à la rechercher, mais sans la découvrir ; \\
et même si le sage affirme qu’il sait, \\
il ne peut la découvrir.\\!
\end{verse}
\section[Chapitre 9]{Chapitre 9}\renewcommand{\leftmark}{Chapitre 9}


\labelblock{Un même sort pour tous}


\begin{verse}
Oui, tout cela, je l’ai pris à cœur, \\
et voici tout ce que j’ai éprouvé : \\
c’est que les justes, les sages et leurs travaux \\
sont entre les mains de Dieu. \\
Ni l’amour, ni la haine, l’homme ne les connaît, \\
tout cela le devance ;\\
tout est pareil pour tous, \\
un sort identique échoit au juste et au méchant, \\
au bon et au pur comme à l’impur, \\
à celui qui sacrifie et à celui qui ne sacrifie pas ; \\
il en est du bon comme du pécheur, \\
de celui qui prête serment comme de celui qui craint de le faire.\\!
\end{verse}
C’est un mal dans tout ce qui se fait sous le soleil \\
qu’un sort identique pour tous ; \\
aussi le cœur des fils d’Adam est-il plein de malice, \\
la folie est dans leur cœur pendant leur vie, \\
et après…, on s’en va vers les morts.\\

\begin{verse}
En effet, qui sera préféré ? \\
Pour tous les vivants, il y a une chose certaine : \\
un chien vivant vaut mieux qu’un lion mort.\\
Car les vivants savent qu’ils mourront ; \\
mais les morts ne savent rien du tout ; \\
pour eux, il n’y a plus de rétribution, \\
puisque leur souvenir est oublié.\\
Leurs amours, leurs haines, leurs jalousies \\
ont déjà péri ; ils n’auront plus jamais de part \\
à tout ce qui se fait sous le soleil.\\!
\end{verse}

\labelblock{Jouir de la vie comme d’un don de Dieu}


\begin{verse}
Va, mange avec joie ton pain \\
et bois de bon cœur ton vin, \\
car déjà Dieu a agréé tes œuvres.\\
Que tes vêtements soient toujours blancs \\
et que l’huile ne manque pas sur ta tête !\\!Goûte la vie avec la femme que tu aimes \\
durant tous les jours de ta vaine existence, \\
puisque Dieu te donne sous le soleil tous tes jours vains ; \\
car c’est là ta part dans la vie \\
et dans le travail que tu fais sous le soleil.\\
Tout ce que ta main se trouve capable de faire, \\
fais-le par tes propres forces ; \\
car il n’y a ni œuvre, ni bilan, ni savoir, ni sagesse \\
dans le séjour des morts où tu t’en iras.\\

\end{verse}

\labelblock{Les contretemps imprévisibles}


\begin{verse}
Je vois encore sous le soleil \\
que la course n’appartient pas aux plus robustes, \\
ni la bataille aux plus forts, \\
ni le pain aux plus sages, \\
ni la richesse aux plus intelligents, \\
ni la faveur aux plus savants, \\
car à tous leur arrivent heur et malheur.\\
En effet, l’homme ne connaît pas plus son heure \\
que les poissons qui se font prendre au filet de malheur, \\
que les passereaux pris au piège. \\
Ainsi les fils d’Adam sont surpris par le malheur \\
quand il tombe sur eux à l’improviste.\\

\end{verse}

\labelblock{La sagesse méconnue}


\begin{verse}
J’ai encore vu sous le soleil, en fait de sagesse, \\
une chose importante à mes yeux.\\
Il y avait une petite ville, de peu d’habitants. \\
Un grand roi marcha contre elle, l’investit \\
et dressa contre elle de grandes embuscades.\\
Il s’y trouvait un homme indigent et sage ; \\
il sauva la ville par sa sagesse, \\
mais personne ne se souvint de cet indigent.\\
Alors je dis, moi : \\
mieux vaut la sagesse que la puissance, \\
mais la sagesse de l’indigent est méprisée \\
et ses paroles ne sont pas écoutées.\\
Les paroles des sages se font entendre dans le calme, \\
mieux que les cris d’un souverain parmi les insensés.\\
Mieux vaut la sagesse que des engins de combat, \\
mais un seul maladroit annule beaucoup de bien.\\!
\end{verse}
\section[Chapitre 10]{Chapitre 10}\renewcommand{\leftmark}{Chapitre 10}

Des mouches mortes infectent et font fermenter \\
l’huile du parfumeur. \\
Un peu de sottise pèse plus \\
que la sagesse, que la gloire.\\

\begin{verse}
L’esprit du sage va du bon côté, \\
mais l’esprit de l’insensé va gauchement.\\
Même en chemin, quand l’insensé s’avance, \\
l’esprit lui fait défaut ; \\
il fait dire à tout le monde qu’il est insensé.\\!
\end{verse}

\labelblock{Le sage préfère la stabilité}

Si l’humeur du chef s’élève contre toi, \\
n’abandonne pas ton poste, \\
car le sang-froid évite de grandes maladresses.\\

\begin{verse}
Il y a un mal que j’ai vu sous le soleil, \\
comme une méprise échappée au souverain :\\
la sottise élevée aux plus hautes situations, \\
et des riches demeurant dans l’abaissement ;\\
j’ai vu des esclaves sur des chevaux, \\
et des princes marcher à pied comme des esclaves.\\!
\end{verse}

\labelblock{Les risques de l’action}


\begin{verse}
Qui creuse une fosse tombe dedans, \\
qui sape un mur, un serpent le mord,\\
qui extrait des pierres peut se blesser avec, \\
qui fend du bois encourt un danger.\\
Si le fer est émoussé et qu’on n’en aiguise pas le tranchant, \\
il faut redoubler de forces ; \\
il y a profit à exercer comme il convient la sagesse.\\
Si le serpent mord faute d’être charmé, \\
pas de profit pour le charmeur.\\!
\end{verse}

\labelblock{Sagesse et sottise}


\begin{verse}
Ce que dit la bouche d’un sage plaît, \\
mais les lèvres de l’insensé le ravalent ;\\
le début de ses propos est sottise, \\
et la fin de ses propos, folie mauvaise.\\
L’insensé multiplie les paroles ; \\
l’homme ne sait plus ce qui arrivera : \\
qui lui indiquera ce qui arrivera après lui ?\\
Le travail de l’insensé l’épuise, \\
il ne sait même pas comment aller à la ville.\\!
\end{verse}

\labelblock{Le roi et le pouvoir}


\begin{verse}
Malheur à toi, pays dont le roi est un gamin \\
et dont les princes festoient dès le matin !\\
Heureux es-tu, pays dont le roi est de souche noble \\
et dont les princes festoient en temps voulu, \\
pour prendre des forces et non pour boire !\\
Avec deux bras paresseux, la poutre cède, \\
quand les mains se relâchent, il pleut dans la maison.\\
Pour se divertir, on fait un repas, \\
et le vin réjouit la vie, \\
et l’argent répond à tout.\\
Ne maudis pas le roi dans ton for intérieur, \\
ne maudis pas le riche même en ta chambre à coucher, \\
car l’oiseau du ciel en emporte le bruit, \\
et la bête ailée fera connaître ce qu’on dit.\\!
\end{verse}
\section[Chapitre 11]{Chapitre 11}\renewcommand{\leftmark}{Chapitre 11}


\labelblock{Trop de prudence nuit}


\begin{verse}
Lance ton pain à la surface des eaux, \\
car à la longue tu le retrouveras.\\
Donne une part à sept ou même à huit personnes, \\
car tu ne sais pas quel malheur peut arriver sur la terre.\\
Si les nuages se remplissent, \\
ils déversent la pluie sur la terre ; \\
qu’un arbre tombe au sud aussi bien qu’au nord, \\
à l’endroit où il est tombé, il reste.\\
Qui observe le vent ne sème pas, \\
qui regarde les nuages ne moissonne pas.\\
De même que tu ignores le cheminement du souffle vital, \\
comme celui de l’ossification dans le ventre d’une femme enceinte, \\
ainsi tu ne peux connaître l’œuvre de Dieu, \\
Lui qui fait toutes choses.\\
Le matin, sème ta semence, \\
et le soir, ne laisse pas de repos à ta main, \\
car tu ne sais pas, de l’une ou de l’autre activité, celle qui convient, ou si toutes deux sont également bonnes.\\!
\end{verse}

\labelblock{Jouir de la vie avec retenue}


\begin{verse}
Douce est la lumière, \\
c’est un plaisir pour les yeux de voir le soleil.\\
Si l’homme vit de nombreuses années, \\
qu’il se réjouisse en elles toutes, \\
mais qu’il se souvienne que les jours sombres sont nombreux, \\
que tout ce qui vient est vanité.\\!Réjouis-toi, jeune homme, dans ta jeunesse, \\
que ton cœur soit heureux aux jours de ton adolescence, \\
marche selon les voies de ton cœur \\
et selon la vision de tes yeux. \\
Mais sache que pour tout cela, \\
Dieu te fera comparaître en jugement.\\
Eloigne de ton cœur l’affliction, \\
écarte de ta chair le mal, \\
car la jeunesse et l’aurore de la vie sont vanité.\\!
\end{verse}
\section[Chapitre 12]{Chapitre 12}\renewcommand{\leftmark}{Chapitre 12}


\labelblock{La vieillesse et la mort}


\begin{verse}
Et souviens-toi de ton Créateur \\
aux jours de ton adolescence, \\
– avant que ne viennent les mauvais jours \\
et que n’arrivent les années dont tu diras : \\
« Je n’y ai aucun plaisir »,\\
– avant que ne s’assombrissent le soleil et la lumière \\
et la lune et les étoiles, \\
et que les nuages ne reviennent, puis la pluie,\\
au jour où tremblent les gardiens de la maison, \\
où se courbent les hommes vigoureux, \\
où s’arrêtent celles qui meulent, trop peu nombreuses, \\
où perdent leur éclat celles qui regardent par la fenêtre,\\
quand les battants se ferment sur la rue, \\
tandis que tombe la voix de la meule, \\
quand on se lève au chant de l’oiseau \\
et que les vocalises s’éteignent ;\\
alors, on a peur de la montée, \\
on a des frayeurs en chemin, \\
tandis que l’amandier est en fleur, \\
que la sauterelle s’alourdit \\
et que le fruit du câprier éclate ; \\
alors que l’homme s’en va vers sa maison d’éternité, \\
et que déjà les pleureuses rôdent dans la rue ;\\
– avant que ne se détache le fil argenté \\
et que la coupe d’or ne se brise, \\
que la jarre ne se casse à la fontaine \\
et qu’à la citerne la poulie ne se brise,\\
– avant que la poussière ne retourne à la terre, selon ce qu’elle était, \\
et que le souffle ne retourne à Dieu qui l’avait donné.\\!
\end{verse}

\labelblock{Appendice}

Vanité des vanités, a dit le Qohéleth, tout est vanité.\\

\begin{verse}
Ce qui ajoute à la sagesse de Qohéleth, \\
c’est qu’il a encore enseigné la science au peuple ; \\
il a pesé, examiné, ajusté un grand nombre de proverbes.\\
Qohéleth s’est appliqué à trouver des paroles plaisantes \\
dont la teneur exacte est ici transcrite : \\
ce sont les paroles authentiques.\\
Les paroles des sages sont comme des aiguillons, \\
les auteurs des recueils sont des jalons bien plantés ; \\
tel est le don d’un berger unique.\\
Garde-toi, mon fils, d’y ajouter : \\
à multiplier les livres, il n’y a pas de limites, \\
et à beaucoup étudier, le corps s’épuise.\\
Fin du discours : Tout a été entendu. \\
Crains Dieu et observe ses commandements, \\
car c’est là tout l’homme :\\!
\end{verse}
Dieu fera venir toute œuvre en jugement \\
sur tout ce qu’elle recèle de bon ou de mauvais.\\
\noindent 
 


% at least one empty page at end (for booklet couv)
\ifbooklet
  \newpage\null\thispagestyle{empty}\newpage
\fi

\ifdev % autotext in dev mode
\fontname\font — \textsc{Les règles du jeu}\par
(\hyperref[utopie]{\underline{Lien}})\par
\noindent \initialiv{A}{lors là}\blindtext\par
\noindent \initialiv{À}{ la bonheur des dames}\blindtext\par
\noindent \initialiv{É}{tonnez-le}\blindtext\par
\noindent \initialiv{Q}{ualitativement}\blindtext\par
\noindent \initialiv{V}{aloriser}\blindtext\par
\Blindtext
\phantomsection
\label{utopie}
\Blinddocument
\fi
\end{document}
