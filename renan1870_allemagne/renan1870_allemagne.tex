%%%%%%%%%%%%%%%%%%%%%%%%%%%%%%%%%
% LaTeX model https://hurlus.fr %
%%%%%%%%%%%%%%%%%%%%%%%%%%%%%%%%%

% Needed before document class
\RequirePackage{pdftexcmds} % needed for tests expressions
\RequirePackage{fix-cm} % correct units

% Define mode
\def\mode{a4}

\newif\ifaiv % a4
\newif\ifav % a5
\newif\ifbooklet % booklet
\newif\ifcover % cover for booklet

\ifnum \strcmp{\mode}{cover}=0
  \covertrue
\else\ifnum \strcmp{\mode}{booklet}=0
  \booklettrue
\else\ifnum \strcmp{\mode}{a5}=0
  \avtrue
\else
  \aivtrue
\fi\fi\fi

\ifbooklet % do not enclose with {}
  \documentclass[french,twoside]{book} % ,notitlepage
  \usepackage[%
    papersize={105mm, 297mm},
    inner=12mm,
    outer=12mm,
    top=20mm,
    bottom=15mm,
    marginparsep=0pt,
  ]{geometry}
  \usepackage[fontsize=9.5pt]{scrextend} % for Roboto
\else\ifav
  \documentclass[french,twoside]{book} % ,notitlepage
  \usepackage[%
    a5paper,
    inner=25mm,
    outer=15mm,
    top=15mm,
    bottom=15mm,
    marginparsep=0pt,
  ]{geometry}
  \usepackage[fontsize=12pt]{scrextend}
\else% A4 2 cols
  \documentclass[twocolumn]{report}
  \usepackage[%
    a4paper,
    inner=15mm,
    outer=10mm,
    top=25mm,
    bottom=18mm,
    marginparsep=0pt,
  ]{geometry}
  \setlength{\columnsep}{20mm}
  \usepackage[fontsize=9.5pt]{scrextend}
\fi\fi

%%%%%%%%%%%%%%
% Alignments %
%%%%%%%%%%%%%%
% before teinte macros

\setlength{\arrayrulewidth}{0.2pt}
\setlength{\columnseprule}{\arrayrulewidth} % twocol
\setlength{\parskip}{0pt} % classical para with no margin
\setlength{\parindent}{1.5em}

%%%%%%%%%%
% Colors %
%%%%%%%%%%
% before Teinte macros

\usepackage[dvipsnames]{xcolor}
\definecolor{rubric}{HTML}{800000} % the tonic 0c71c3
\def\columnseprulecolor{\color{rubric}}
\colorlet{borderline}{rubric!30!} % definecolor need exact code
\definecolor{shadecolor}{gray}{0.95}
\definecolor{bghi}{gray}{0.5}

%%%%%%%%%%%%%%%%%
% Teinte macros %
%%%%%%%%%%%%%%%%%
%%%%%%%%%%%%%%%%%%%%%%%%%%%%%%%%%%%%%%%%%%%%%%%%%%%
% <TEI> generic (LaTeX names generated by Teinte) %
%%%%%%%%%%%%%%%%%%%%%%%%%%%%%%%%%%%%%%%%%%%%%%%%%%%
% This template is inserted in a specific design
% It is XeLaTeX and otf fonts

\makeatletter % <@@@


\usepackage{blindtext} % generate text for testing
\usepackage[strict]{changepage} % for modulo 4
\usepackage{contour} % rounding words
\usepackage[nodayofweek]{datetime}
% \usepackage{DejaVuSans} % seems buggy for sffont font for symbols
\usepackage{enumitem} % <list>
\usepackage{etoolbox} % patch commands
\usepackage{fancyvrb}
\usepackage{fancyhdr}
\usepackage{float}
\usepackage{fontspec} % XeLaTeX mandatory for fonts
\usepackage{footnote} % used to capture notes in minipage (ex: quote)
\usepackage{framed} % bordering correct with footnote hack
\usepackage{graphicx}
\usepackage{lettrine} % drop caps
\usepackage{lipsum} % generate text for testing
\usepackage[framemethod=tikz,]{mdframed} % maybe used for frame with footnotes inside
\usepackage{pdftexcmds} % needed for tests expressions
\usepackage{polyglossia} % non-break space french punct, bug Warning: "Failed to patch part"
\usepackage[%
  indentfirst=false,
  vskip=1em,
  noorphanfirst=true,
  noorphanafter=true,
  leftmargin=\parindent,
  rightmargin=0pt,
]{quoting}
\usepackage{ragged2e}
\usepackage{setspace} % \setstretch for <quote>
\usepackage{tabularx} % <table>
\usepackage[explicit]{titlesec} % wear titles, !NO implicit
\usepackage{tikz} % ornaments
\usepackage{tocloft} % styling tocs
\usepackage[fit]{truncate} % used im runing titles
\usepackage{unicode-math}
\usepackage[normalem]{ulem} % breakable \uline, normalem is absolutely necessary to keep \emph
\usepackage{verse} % <l>
\usepackage{xcolor} % named colors
\usepackage{xparse} % @ifundefined
\XeTeXdefaultencoding "iso-8859-1" % bad encoding of xstring
\usepackage{xstring} % string tests
\XeTeXdefaultencoding "utf-8"
\PassOptionsToPackage{hyphens}{url} % before hyperref, which load url package

% TOTEST
% \usepackage{hypcap} % links in caption ?
% \usepackage{marginnote}
% TESTED
% \usepackage{background} % doesn’t work with xetek
% \usepackage{bookmark} % prefers the hyperref hack \phantomsection
% \usepackage[color, leftbars]{changebar} % 2 cols doc, impossible to keep bar left
% \usepackage[utf8x]{inputenc} % inputenc package ignored with utf8 based engines
% \usepackage[sfdefault,medium]{inter} % no small caps
% \usepackage{firamath} % choose firasans instead, firamath unavailable in Ubuntu 21-04
% \usepackage{flushend} % bad for last notes, supposed flush end of columns
% \usepackage[stable]{footmisc} % BAD for complex notes https://texfaq.org/FAQ-ftnsect
% \usepackage{helvet} % not for XeLaTeX
% \usepackage{multicol} % not compatible with too much packages (longtable, framed, memoir…)
% \usepackage[default,oldstyle,scale=0.95]{opensans} % no small caps
% \usepackage{sectsty} % \chapterfont OBSOLETE
% \usepackage{soul} % \ul for underline, OBSOLETE with XeTeX
% \usepackage[breakable]{tcolorbox} % text styling gone, footnote hack not kept with breakable


% Metadata inserted by a program, from the TEI source, for title page and runing heads
\title{\textbf{ Ernest Renan et l’Allemagne. }}
\date{1870}
\author{Renan, Ernest (1823-1892)}
\def\elbibl{Renan, Ernest (1823-1892). 1870. \emph{Ernest Renan et l’Allemagne.}}

% Default metas
\newcommand{\colorprovide}[2]{\@ifundefinedcolor{#1}{\colorlet{#1}{#2}}{}}
\colorprovide{rubric}{red}
\colorprovide{silver}{lightgray}
\@ifundefined{syms}{\newfontfamily\syms{DejaVu Sans}}{}
\newif\ifdev
\@ifundefined{elbibl}{% No meta defined, maybe dev mode
  \newcommand{\elbibl}{Titre court ?}
  \newcommand{\elbook}{Titre du livre source ?}
  \newcommand{\elabstract}{Résumé\par}
  \newcommand{\elurl}{http://oeuvres.github.io/elbook/2}
  \author{Éric Lœchien}
  \title{Un titre de test assez long pour vérifier le comportement d’une maquette}
  \date{1566}
  \devtrue
}{}
\let\eltitle\@title
\let\elauthor\@author
\let\eldate\@date


\defaultfontfeatures{
  % Mapping=tex-text, % no effect seen
  Scale=MatchLowercase,
  Ligatures={TeX,Common},
}


% generic typo commands
\newcommand{\astermono}{\medskip\centerline{\color{rubric}\large\selectfont{\syms ✻}}\medskip\par}%
\newcommand{\astertri}{\medskip\par\centerline{\color{rubric}\large\selectfont{\syms ✻\,✻\,✻}}\medskip\par}%
\newcommand{\asterism}{\bigskip\par\noindent\parbox{\linewidth}{\centering\color{rubric}\large{\syms ✻}\\{\syms ✻}\hskip 0.75em{\syms ✻}}\bigskip\par}%

% lists
\newlength{\listmod}
\setlength{\listmod}{\parindent}
\setlist{
  itemindent=!,
  listparindent=\listmod,
  labelsep=0.2\listmod,
  parsep=0pt,
  % topsep=0.2em, % default topsep is best
}
\setlist[itemize]{
  label=—,
  leftmargin=0pt,
  labelindent=1.2em,
  labelwidth=0pt,
}
\setlist[enumerate]{
  label={\bf\color{rubric}\arabic*.},
  labelindent=0.8\listmod,
  leftmargin=\listmod,
  labelwidth=0pt,
}
\newlist{listalpha}{enumerate}{1}
\setlist[listalpha]{
  label={\bf\color{rubric}\alph*.},
  leftmargin=0pt,
  labelindent=0.8\listmod,
  labelwidth=0pt,
}
\newcommand{\listhead}[1]{\hspace{-1\listmod}\emph{#1}}

\renewcommand{\hrulefill}{%
  \leavevmode\leaders\hrule height 0.2pt\hfill\kern\z@}

% General typo
\DeclareTextFontCommand{\textlarge}{\large}
\DeclareTextFontCommand{\textsmall}{\small}

% commands, inlines
\newcommand{\anchor}[1]{\Hy@raisedlink{\hypertarget{#1}{}}} % link to top of an anchor (not baseline)
\newcommand\abbr[1]{#1}
\newcommand{\autour}[1]{\tikz[baseline=(X.base)]\node [draw=rubric,thin,rectangle,inner sep=1.5pt, rounded corners=3pt] (X) {\color{rubric}#1};}
\newcommand\corr[1]{#1}
\newcommand{\ed}[1]{ {\color{silver}\sffamily\footnotesize (#1)} } % <milestone ed="1688"/>
\newcommand\expan[1]{#1}
\newcommand\foreign[1]{\emph{#1}}
\newcommand\gap[1]{#1}
\renewcommand{\LettrineFontHook}{\color{rubric}}
\newcommand{\initial}[2]{\lettrine[lines=2, loversize=0.3, lhang=0.3]{#1}{#2}}
\newcommand{\initialiv}[2]{%
  \let\oldLFH\LettrineFontHook
  % \renewcommand{\LettrineFontHook}{\color{rubric}\ttfamily}
  \IfSubStr{QJ’}{#1}{
    \lettrine[lines=4, lhang=0.2, loversize=-0.1, lraise=0.2]{\smash{#1}}{#2}
  }{\IfSubStr{É}{#1}{
    \lettrine[lines=4, lhang=0.2, loversize=-0, lraise=0]{\smash{#1}}{#2}
  }{\IfSubStr{ÀÂ}{#1}{
    \lettrine[lines=4, lhang=0.2, loversize=-0, lraise=0, slope=0.6em]{\smash{#1}}{#2}
  }{\IfSubStr{A}{#1}{
    \lettrine[lines=4, lhang=0.2, loversize=0.2, slope=0.6em]{\smash{#1}}{#2}
  }{\IfSubStr{V}{#1}{
    \lettrine[lines=4, lhang=0.2, loversize=0.2, slope=-0.5em]{\smash{#1}}{#2}
  }{
    \lettrine[lines=4, lhang=0.2, loversize=0.2]{\smash{#1}}{#2}
  }}}}}
  \let\LettrineFontHook\oldLFH
}
\newcommand{\labelchar}[1]{\textbf{\color{rubric} #1}}
\newcommand{\milestone}[1]{\autour{\footnotesize\color{rubric} #1}} % <milestone n="4"/>
\newcommand\name[1]{#1}
\newcommand\orig[1]{#1}
\newcommand\orgName[1]{#1}
\newcommand\persName[1]{#1}
\newcommand\placeName[1]{#1}
\newcommand{\pn}[1]{\IfSubStr{-—–¶}{#1}% <p n="3"/>
  {\noindent{\bfseries\color{rubric}   ¶  }}
  {{\footnotesize\autour{ #1}  }}}
\newcommand\reg{}
% \newcommand\ref{} % already defined
\newcommand\sic[1]{#1}
\newcommand\surname[1]{\textsc{#1}}
\newcommand\term[1]{\textbf{#1}}

\def\mednobreak{\ifdim\lastskip<\medskipamount
  \removelastskip\nopagebreak\medskip\fi}
\def\bignobreak{\ifdim\lastskip<\bigskipamount
  \removelastskip\nopagebreak\bigskip\fi}

% commands, blocks
\newcommand{\byline}[1]{\bigskip{\RaggedLeft{#1}\par}\bigskip}
\newcommand{\bibl}[1]{{\RaggedLeft{#1}\par\bigskip}}
\newcommand{\biblitem}[1]{{\noindent\hangindent=\parindent   #1\par}}
\newcommand{\dateline}[1]{\medskip{\RaggedLeft{#1}\par}\bigskip}
\newcommand{\labelblock}[1]{\medbreak{\noindent\color{rubric}\bfseries #1}\par\mednobreak}
\newcommand{\salute}[1]{\bigbreak{#1}\par\medbreak}
\newcommand{\signed}[1]{\bigbreak\filbreak{\raggedleft #1\par}\medskip}

% environments for blocks (some may become commands)
\newenvironment{borderbox}{}{} % framing content
\newenvironment{citbibl}{\ifvmode\hfill\fi}{\ifvmode\par\fi }
\newenvironment{docAuthor}{\ifvmode\vskip4pt\fontsize{16pt}{18pt}\selectfont\fi\itshape}{\ifvmode\par\fi }
\newenvironment{docDate}{}{\ifvmode\par\fi }
\newenvironment{docImprint}{\vskip6pt}{\ifvmode\par\fi }
\newenvironment{docTitle}{\vskip6pt\bfseries\fontsize{18pt}{22pt}\selectfont}{\par }
\newenvironment{msHead}{\vskip6pt}{\par}
\newenvironment{msItem}{\vskip6pt}{\par}
\newenvironment{titlePart}{}{\par }


% environments for block containers
\newenvironment{argument}{\itshape\parindent0pt}{\vskip1.5em}
\newenvironment{biblfree}{}{\ifvmode\par\fi }
\newenvironment{bibitemlist}[1]{%
  \list{\@biblabel{\@arabic\c@enumiv}}%
  {%
    \settowidth\labelwidth{\@biblabel{#1}}%
    \leftmargin\labelwidth
    \advance\leftmargin\labelsep
    \@openbib@code
    \usecounter{enumiv}%
    \let\p@enumiv\@empty
    \renewcommand\theenumiv{\@arabic\c@enumiv}%
  }
  \sloppy
  \clubpenalty4000
  \@clubpenalty \clubpenalty
  \widowpenalty4000%
  \sfcode`\.\@m
}%
{\def\@noitemerr
  {\@latex@warning{Empty `bibitemlist' environment}}%
\endlist}
\newenvironment{quoteblock}% may be used for ornaments
  {\begin{quoting}}
  {\end{quoting}}

% table () is preceded and finished by custom command
\newcommand{\tableopen}[1]{%
  \ifnum\strcmp{#1}{wide}=0{%
    \begin{center}
  }
  \else\ifnum\strcmp{#1}{long}=0{%
    \begin{center}
  }
  \else{%
    \begin{center}
  }
  \fi\fi
}
\newcommand{\tableclose}[1]{%
  \ifnum\strcmp{#1}{wide}=0{%
    \end{center}
  }
  \else\ifnum\strcmp{#1}{long}=0{%
    \end{center}
  }
  \else{%
    \end{center}
  }
  \fi\fi
}


% text structure
\newcommand\chapteropen{} % before chapter title
\newcommand\chaptercont{} % after title, argument, epigraph…
\newcommand\chapterclose{} % maybe useful for multicol settings
\setcounter{secnumdepth}{-2} % no counters for hierarchy titles
\setcounter{tocdepth}{5} % deep toc
\markright{\@title} % ???
\markboth{\@title}{\@author} % ???
\renewcommand\tableofcontents{\@starttoc{toc}}
% toclof format
% \renewcommand{\@tocrmarg}{0.1em} % Useless command?
% \renewcommand{\@pnumwidth}{0.5em} % {1.75em}
\renewcommand{\@cftmaketoctitle}{}
\setlength{\cftbeforesecskip}{\z@ \@plus.2\p@}
\renewcommand{\cftchapfont}{}
\renewcommand{\cftchapdotsep}{\cftdotsep}
\renewcommand{\cftchapleader}{\normalfont\cftdotfill{\cftchapdotsep}}
\renewcommand{\cftchappagefont}{\bfseries}
\setlength{\cftbeforechapskip}{0em \@plus\p@}
% \renewcommand{\cftsecfont}{\small\relax}
\renewcommand{\cftsecpagefont}{\normalfont}
% \renewcommand{\cftsubsecfont}{\small\relax}
\renewcommand{\cftsecdotsep}{\cftdotsep}
\renewcommand{\cftsecpagefont}{\normalfont}
\renewcommand{\cftsecleader}{\normalfont\cftdotfill{\cftsecdotsep}}
\setlength{\cftsecindent}{1em}
\setlength{\cftsubsecindent}{2em}
\setlength{\cftsubsubsecindent}{3em}
\setlength{\cftchapnumwidth}{1em}
\setlength{\cftsecnumwidth}{1em}
\setlength{\cftsubsecnumwidth}{1em}
\setlength{\cftsubsubsecnumwidth}{1em}

% footnotes
\newif\ifheading
\newcommand*{\fnmarkscale}{\ifheading 0.70 \else 1 \fi}
\renewcommand\footnoterule{\vspace*{0.3cm}\hrule height \arrayrulewidth width 3cm \vspace*{0.3cm}}
\setlength\footnotesep{1.5\footnotesep} % footnote separator
\renewcommand\@makefntext[1]{\parindent 1.5em \noindent \hb@xt@1.8em{\hss{\normalfont\@thefnmark . }}#1} % no superscipt in foot
\patchcmd{\@footnotetext}{\footnotesize}{\footnotesize\sffamily}{}{} % before scrextend, hyperref


%   see https://tex.stackexchange.com/a/34449/5049
\def\truncdiv#1#2{((#1-(#2-1)/2)/#2)}
\def\moduloop#1#2{(#1-\truncdiv{#1}{#2}*#2)}
\def\modulo#1#2{\number\numexpr\moduloop{#1}{#2}\relax}

% orphans and widows
\clubpenalty=9996
\widowpenalty=9999
\brokenpenalty=4991
\predisplaypenalty=10000
\postdisplaypenalty=1549
\displaywidowpenalty=1602
\hyphenpenalty=400
% Copied from Rahtz but not understood
\def\@pnumwidth{1.55em}
\def\@tocrmarg {2.55em}
\def\@dotsep{4.5}
\emergencystretch 3em
\hbadness=4000
\pretolerance=750
\tolerance=2000
\vbadness=4000
\def\Gin@extensions{.pdf,.png,.jpg,.mps,.tif}
% \renewcommand{\@cite}[1]{#1} % biblio

\usepackage{hyperref} % supposed to be the last one, :o) except for the ones to follow
\urlstyle{same} % after hyperref
\hypersetup{
  % pdftex, % no effect
  pdftitle={\elbibl},
  % pdfauthor={Your name here},
  % pdfsubject={Your subject here},
  % pdfkeywords={keyword1, keyword2},
  bookmarksnumbered=true,
  bookmarksopen=true,
  bookmarksopenlevel=1,
  pdfstartview=Fit,
  breaklinks=true, % avoid long links
  pdfpagemode=UseOutlines,    % pdf toc
  hyperfootnotes=true,
  colorlinks=false,
  pdfborder=0 0 0,
  % pdfpagelayout=TwoPageRight,
  % linktocpage=true, % NO, toc, link only on page no
}

\makeatother % /@@@>
%%%%%%%%%%%%%%
% </TEI> end %
%%%%%%%%%%%%%%


%%%%%%%%%%%%%
% footnotes %
%%%%%%%%%%%%%
\renewcommand{\thefootnote}{\bfseries\textcolor{rubric}{\arabic{footnote}}} % color for footnote marks

%%%%%%%%%
% Fonts %
%%%%%%%%%
\usepackage[]{roboto} % SmallCaps, Regular is a bit bold
% \linespread{0.90} % too compact, keep font natural
\newfontfamily\fontrun[]{Roboto Condensed Light} % condensed runing heads
\ifav
  \setmainfont[
    ItalicFont={Roboto Light Italic},
  ]{Roboto}
\else\ifbooklet
  \setmainfont[
    ItalicFont={Roboto Light Italic},
  ]{Roboto}
\else
\setmainfont[
  ItalicFont={Roboto Italic},
]{Roboto Light}
\fi\fi
\renewcommand{\LettrineFontHook}{\bfseries\color{rubric}}
% \renewenvironment{labelblock}{\begin{center}\bfseries\color{rubric}}{\end{center}}

%%%%%%%%
% MISC %
%%%%%%%%

\setdefaultlanguage[frenchpart=false]{french} % bug on part


\newenvironment{quotebar}{%
    \def\FrameCommand{{\color{rubric!10!}\vrule width 0.5em} \hspace{0.9em}}%
    \def\OuterFrameSep{\itemsep} % séparateur vertical
    \MakeFramed {\advance\hsize-\width \FrameRestore}
  }%
  {%
    \endMakeFramed
  }
\renewenvironment{quoteblock}% may be used for ornaments
  {%
    \savenotes
    \setstretch{0.9}
    \normalfont
    \begin{quotebar}
  }
  {%
    \end{quotebar}
    \spewnotes
  }


\renewcommand{\headrulewidth}{\arrayrulewidth}
\renewcommand{\headrule}{{\color{rubric}\hrule}}

% delicate tuning, image has produce line-height problems in title on 2 lines
\titleformat{name=\chapter} % command
  [display] % shape
  {\vspace{1.5em}\centering} % format
  {} % label
  {0pt} % separator between n
  {}
[{\color{rubric}\huge\textbf{#1}}\bigskip] % after code
% \titlespacing{command}{left spacing}{before spacing}{after spacing}[right]
\titlespacing*{\chapter}{0pt}{-2em}{0pt}[0pt]

\titleformat{name=\section}
  [block]{}{}{}{}
  [\vbox{\color{rubric}\large\raggedleft\textbf{#1}}]
\titlespacing{\section}{0pt}{0pt plus 4pt minus 2pt}{\baselineskip}

\titleformat{name=\subsection}
  [block]
  {}
  {} % \thesection
  {} % separator \arrayrulewidth
  {}
[\vbox{\large\textbf{#1}}]
% \titlespacing{\subsection}{0pt}{0pt plus 4pt minus 2pt}{\baselineskip}

\ifaiv
  \fancypagestyle{main}{%
    \fancyhf{}
    \setlength{\headheight}{1.5em}
    \fancyhead{} % reset head
    \fancyfoot{} % reset foot
    \fancyhead[L]{\truncate{0.45\headwidth}{\fontrun\elbibl}} % book ref
    \fancyhead[R]{\truncate{0.45\headwidth}{ \fontrun\nouppercase\leftmark}} % Chapter title
    \fancyhead[C]{\thepage}
  }
  \fancypagestyle{plain}{% apply to chapter
    \fancyhf{}% clear all header and footer fields
    \setlength{\headheight}{1.5em}
    \fancyhead[L]{\truncate{0.9\headwidth}{\fontrun\elbibl}}
    \fancyhead[R]{\thepage}
  }
\else
  \fancypagestyle{main}{%
    \fancyhf{}
    \setlength{\headheight}{1.5em}
    \fancyhead{} % reset head
    \fancyfoot{} % reset foot
    \fancyhead[RE]{\truncate{0.9\headwidth}{\fontrun\elbibl}} % book ref
    \fancyhead[LO]{\truncate{0.9\headwidth}{\fontrun\nouppercase\leftmark}} % Chapter title, \nouppercase needed
    \fancyhead[RO,LE]{\thepage}
  }
  \fancypagestyle{plain}{% apply to chapter
    \fancyhf{}% clear all header and footer fields
    \setlength{\headheight}{1.5em}
    \fancyhead[L]{\truncate{0.9\headwidth}{\fontrun\elbibl}}
    \fancyhead[R]{\thepage}
  }
\fi

\ifav % a5 only
  \titleclass{\section}{top}
\fi

\newcommand\chapo{{%
  \vspace*{-3em}
  \centering % no vskip ()
  {\Large\addfontfeature{LetterSpace=25}\bfseries{\elauthor}}\par
  \smallskip
  {\large\eldate}\par
  \bigskip
  {\Large\selectfont{\eltitle}}\par
  \bigskip
  {\color{rubric}\hline\par}
  \bigskip
  {\Large TEXTE LIBRE À PARTICPATION LIBRE\par}
  \centerline{\small\color{rubric} {hurlus.fr, tiré le \today}}\par
  \bigskip
}}

\newcommand\cover{{%
  \thispagestyle{empty}
  \centering
  {\LARGE\bfseries{\elauthor}}\par
  \bigskip
  {\Large\eldate}\par
  \bigskip
  \bigskip
  {\LARGE\selectfont{\eltitle}}\par
  \vfill\null
  {\color{rubric}\setlength{\arrayrulewidth}{2pt}\hline\par}
  \vfill\null
  {\Large TEXTE LIBRE À PARTICPATION LIBRE\par}
  \centerline{{\href{https://hurlus.fr}{\dotuline{hurlus.fr}}, tiré le \today}}\par
}}

\begin{document}
\pagestyle{empty}
\ifbooklet{
  \cover\newpage
  \thispagestyle{empty}\hbox{}\newpage
  \cover\newpage\noindent Les voyages de la brochure\par
  \bigskip
  \begin{tabularx}{\textwidth}{l|X|X}
    \textbf{Date} & \textbf{Lieu}& \textbf{Nom/pseudo} \\ \hline
    \rule{0pt}{25cm} &  &   \\
  \end{tabularx}
  \newpage
  \addtocounter{page}{-4}
}\fi

\thispagestyle{empty}
\ifaiv
  \twocolumn[\chapo]
\else
  \chapo
\fi
{\it\elabstract}
\bigskip
\makeatletter\@starttoc{toc}\makeatother % toc without new page
\bigskip

\pagestyle{main} % after style

  \section[{La guerre Franco-Allemande}]{La guerre Franco-Allemande}\renewcommand{\leftmark}{La guerre Franco-Allemande}

\noindent ({\itshape Revue des Deux Mondes})\par

\dateline{15 septembre 1870}
\noindent En commençant à écrire ces pages, j’ignore quel sera l’état du monde au moment où elles seront terminées. Il faudrait un esprit bien frivole pour chercher à démêler l’avenir quand le présent n’a pas une heure assurée. Il est permis cependant à ceux qu’une conception philosophique de la vie a élevés au-dessus, non certes du patriotisme, mais des erreurs qu’un patriotisme peu éclairé entraîne, d’essayer de découvrir quelque chose à travers l’épaisse fumée qui ne laisse voir à l’horizon que l’image de la mort.\par
J’ai toujours regardé la guerre entre la France et l’Allemagne comme le plus grand malheur qui pût arriver à la civilisation. Tous, nous acceptons hautement les devoirs de la patrie, ses justes susceptibilités, ses espérances ; tous, nous avons une pleine confiance dans les forces profondes du pays, dans cette élasticité qui déjà plus d’une fois a fait rebondir la France sous la pression de l’infortune ; mais supposons les espérances permises de beaucoup dépassées, la guerre commencée n’en aura pas moins été un immense malheur. Elle aura semé une haine violente entre les deux portions de la race européenne dont l’union importait le plus au progrès de l’esprit humain. La grande maîtresse de l’investigation savante, l’ingénieuse, vive et prompte initiatrice du monde, à toute fine et délicate pensée, sont brouillées pour longtemps, à jamais peut-être ; chacune d’elles s’enfoncera dans ses défauts, l’une devenant de plus en plus rude et grossière, l’autre de plus en plus superficielle et arriérée. L’harmonie intellectuelle, morale, politique de l’humanité est rompue ; une aigre dissonance se mêlera au concert de la société européenne pendant des siècles.\par
En effet, mettons de côté les États-Unis d’Amérique dont l’avenir, brillant sans doute, est encore obscur, et qui en tout cas occupent un rang secondaire dans le travail original de l’esprit humain, la grandeur intellectuelle et morale de l’Europe repose sur une triple alliance dont la rupture est un deuil pour le progrès, l’alliance entre la France, l’Allemagne et l’Angleterre. Unies, ces trois grandes forces conduiraient le monde et le conduiraient bien, entraînant nécessairement après elles les autres éléments, considérables encore, dont se compose le réseau européen ; elles traceraient surtout d’une façon impérieuse sa voie à une autre force qu’il ne faut ni exagérer ni trop rabaisser, la Russie. La Russie n’est un danger que si le reste de l’Europe l’abandonne à la fausse idée d’une originalité qu’elle n’a peut-être pas, et lui permet de réunir en un faisceau les peuplades barbares du centre de l’Asie, peuplades tout à fait impuissantes par elles-mêmes, mais capables de discipline et fort susceptibles, si l’on n’y prend garde, de se grouper autour d’un Gengis Khan moscovite. Les États-Unis ne sont un danger que si la division de l’Europe leur permet de se laisser aller aux fumées d’une jeunesse présomptueuse et à de vieux ressentiments contre la mère patrie. Avec l’union de la France, de l’Angleterre et de l’Allemagne, le vieux continent gardait son équilibre, maîtrisait puissamment le nouveau, tenait en tutelle ce vaste monde oriental auquel il serait malsain de laisser concevoir des espérances exagérées. — Ce n’était là qu’un rêve. Un jour a suffi pour renverser l’édifice où s’abritaient nos espérances, pour ouvrir le monde à tous les dangers, à toutes les convoitises, à toutes les brutalités.\par
Dans cette situation, dont nous ne sommes en rien responsables, le devoir de tout esprit philosophique est de faire taire son émotion et d’étudier, d’une pensée froide et claire, les causes du mal, pour tâcher d’entrevoir la manière dont il est possible de l’atténuer. La paix se fera entre la France et l’Allemagne. L’extermination n’a qu’un temps ; elle trouve sa fin, comme les maladies contagieuses, dans ses ravages mêmes, comme la flamme, dans la destruction de l’objet qui lui servait d’aliment. J’ai lu, je ne sais où, la parabole de deux frères qui, du temps de Caïn et d’Abel sans doute, en vinrent à se haïr et résolurent de se battre jusqu’à ce qu’ils ne fussent plus frères. Quand, épuisés, ils tombèrent tous deux sur le sol, ils se trouvèrent encore frères, voisins, tributaires du même puits, riverains du même ruisseau.\par
Qui fera la paix entre la France et l’Allemagne ? Dans quelles conditions se fera cette paix ? On risquerait fort de se tromper, si l’on voulait parler de la paix provisoire ou plutôt de l’armistice qui se conclura dans quelques semaines ou quelques mois. Nous ne parlons ici que du règlement de comptes qui interviendra un jour pour le bien du monde entre les deux grandes nations de l’Europe centrale. Pour se former une idée à cet égard, il faut d’abord bien connaître de quelle façon l’Allemagne est arrivée à concevoir l’idée de sa propre nationalité.\par
\subsection[{I}]{I}
\noindent La loi du développement historique de l’Allemagne ne ressemble en rien à celle de la France ; la destinée de l’Allemagne, au contraire, est à beaucoup d’égards semblable à celle de l’Italie. Fondatrice du vieil Empire romain, dépositaire jalouse de ses traditions, l’Italie n’a jamais pu devenir une nation comme les autres. Succédant à l’Empire romain, fondatrice du nouvel empire carlovingien, se prétendant dépositaire d’un pouvoir universel, d’un droit plus que national, l’Allemagne était arrivée jusqu’à ces dernières années sans être un peuple. L’Empire romain et la papauté, qui en fut la suite, avaient perdu l’Italie. L’empire carlovingien faillit perdre l’Allemagne. L’empereur germanique ne fut pas plus capable de faire l’unité de la nation allemande que le pape de faire celle de l’Italie. On n’est maître chez soi que quand on n’a aucune prétention à régner hors de chez soi. Tout pays qui arrive à exercer une primauté politique, intellectuelle, religieuse, sur les autres peuples, l’expie par la perte de son existence nationale durant des siècles.\par
Il n’en fut pas de même de la France. Dès le X\textsuperscript{e} siècle la France se retire bien nettement de l’empire. Les deux joyaux du monde occidental, la couronne impériale et la tiare papale, elle les perd pour son bonheur. À partir de la mort de Charles le Gros, l’empire devient exclusivement l’apanage des Allemands, aucun roi de France n’est plus empereur d’Occident. D’autre part, la papauté devient la propriété de l’Italie. La {\itshape Francia}, telle que l’avait faite le traité de Verdun, est privilégiée justement à cause de ce qui lui manque : elle n’a ni l’empire, ni la papauté, les deux choses universelles qui troublent perpétuellement le pays qui les possède dans l’œuvre de sa concrétion intime. Dès le X\textsuperscript{e} siècle, la {\itshape Francia} est toute nationale, et, en effet, dans la seconde moitié de ce siècle elle substitue au Carlovingien, lourd Allemand qui la défend mal, une famille encore germanique sans doute, mais bien réellement mariée avec le sol : la famille des ducs de France, qui a un domaine propre et non pas seulement, comme les Carlovingiens, un titre abstrait. Dès lors commence autour de Paris cette admirable marche du développement national, qui aboutit à Louis XIV, à la Révolution, et dont le XIX\textsuperscript{e} siècle pourra voir la contrepartie, par suite de la triste loi qui condamne les choses humaines à entrer dans la voie de la décadence et de la destruction dès qu’elles sont achevées.\par
L’idée de former une nationalité compacte n’avait jamais été, jusqu’à la Révolution française, l’idée de l’Allemagne. Cette grande race allemande porte bien plus loin que la France le goût des indépendances provinciales ; la chance de guerres que nous appellerions civiles entre des parties de la même famille nationale ne l’effraye pas. Elle ne veut pas de l’unité pour elle-même, elle la veut uniquement par crainte de l’étranger ; elle tient par-dessus tout à la liberté de ses divisions intérieures. Ce fut là ce qui lui permit de faire la plus belle chose des temps modernes : la réforme luthérienne, chose, selon nous, supérieure à la philosophie et à la Révolution, œuvres de la France, et qui ne le cède qu’à la Renaissance, œuvre de l’Italie ; mais on a toujours les défauts de ses qualités. Depuis la chute des Hohenstaufen, la politique générale de l’Allemagne fut indécise, faible, empreinte d’une sorte de gaucherie ; à la suite de la Guerre de Trente Ans, la conscience d’une patrie allemande existe à peine. La royauté française abusa de ce pitoyable état politique d’une grande race. Elle fit ce qu’elle n’avait jamais fait : elle sortit de son programme, qui était de ne s’assimiler que des pays de langue française ; elle s’empara de l’Alsace, terre allemande. Le temps a légitimé cette conquête, puisque l’Alsace a pris ensuite une part si brillante aux grandes œuvres communes de la France. Il y eut cependant dans ce fait, qui au XVII\textsuperscript{e} siècle ne choqua personne, le germe d’un grave embarras pour l’époque où l’idée des nationalités deviendrait maîtresse du monde, et ferait prendre, dans les questions de délimitation territoriale, la langue et la race pour {\itshape critérium} de légitimité.\par
La Révolution française fut, à vrai dire, le fait générateur de l’idée de l’unité allemande. La Révolution répondait en un sens au vœu des meilleurs esprits de l’Allemagne ; mais ils s’en dégoûtèrent vite. L’Allemagne resta légitimiste et féodale ; sa conduite ne fut qu’une série d’hésitations, de malentendus, de fautes. La conduite de la France fut d’une suprême inconséquence. Elle, qui élevait dans le monde le drapeau du droit national, viola, dans l’ivresse de ses victoires, toutes les nationalités. L’Allemagne fut foulée aux pieds des chevaux ; le génie allemand, qui se développait alors d’une façon si merveilleuse, fut méconnu ; sa valeur sérieuse ne fut pas comprise des esprits bornés qui formaient l’élite intellectuelle du temps de l’Empire ; la conduite de Napoléon à l’égard des pays germaniques fut un tissu d’étourderies. Ce grand capitaine, cet éminent organisateur, était dénué des principes les plus élémentaires en fait de politique extérieure. Son idée d’une domination universelle de la France était folle, puisqu’il est bien établi que toute tentative d’hégémonie d’une nation européenne provoque, par une réaction nécessaire, une coalition de tous les autres États, coalition dont l’Angleterre, gardienne de l’équilibre, est toujours le centre de formation.\footnote{Ceci n’est vrai que du passé. La vieille Angleterre, paraît-il, n’existe plus de nos jours (septembre !871).}\par
Une nation ne prend d’ordinaire la complète conscience d’elle-même que sous la pression de l’étranger. La France existait avant Jeanne d’Arc et Charles VII ; cependant, c’est sous le poids de la domination anglaise que le mot de {\itshape France} prend un accent particulier. Un {\itshape moi}, pour prendre le langage de la philosophie, se crée toujours en opposition avec un autre {\itshape moi.} La France fit de la sorte l’Allemagne comme nation. La plaie avait été trop visible. Une nation, dans la pleine floraison de son génie et au plus haut point de sa force morale, avait été livrée sans défense à un adversaire moins intelligent et moins moral par les misérables divisions de ses petits princes, et faute d’un drapeau central. L’Autriche, ensemble à peine allemand, introduisant dans le corps germanique une foule d’éléments non germaniques, trahissait sans cesse la cause allemande et en sacrifiait les intérêts à ses combinaisons dynastiques. Un point de renaissance parut alors : ce fut la Prusse de Frédéric. Formation récente dans le corps germanique, la Prusse en recélait toute la force effective. Par le fond de sa population, elle était plus slave que germanique ; mais ce n’était point là un inconvénient, tout au contraire. Ce sont presque toujours ainsi des pays mixtes et limitrophes qui font l’unité politique d’une race : qu’on se rappelle le rôle de la Macédoine en Grèce, du Piémont en Italie. La réaction de la Prusse contre l’oppression de l’empire français fut très belle. On sait comment le génie de Stein tira de l’abaissement même la condition de la force, et comment l’organisation de l’armée prussienne, point de départ de l’Allemagne nouvelle, fut la conséquence directe de la bataille d’Iéna. Avec sa présomption habituelle et son inintelligence de la race germanique, Napoléon ne vit rien de tout cela. La bataille de Leipzig fut le signal d’une résurrection. De ce jour-là, il fut clair qu’une puissance nouvelle de premier ordre (la Prusse, tenant en sa main le drapeau allemand) faisait son entrée dans le monde. Au fond, la Révolution et l’Empire n’avaient rien compris à l’Allemagne, comme l’Allemagne n’avait rien compris à la France. Les grands esprits germaniques avaient pu saluer avec enthousiasme l’œuvre de la Révolution, parce que les principes de ce mouvement à l’origine étaient les leurs, ou plutôt ceux du XVIII\textsuperscript{e} siècle tout entier ; mais cette basse démocratie terroriste, se transformant en despotisme militaire et en instrument d’asservissement pour tous les peuples, les remplit d’horreur. Par réaction, l’Allemagne éclairée se montra en quelque sorte affamée d’ancien régime. La Révolution française trouvait l’obstacle qui devait l’arrêter dans la féodalité organisée de la Prusse) de la Poméranie, du Holstein, c’est-à-dire dans le fonds de populations antidémocratiques au premier chef des bords de la Baltique, populations fidèles à la légitimité, acceptant d’être menées, bâtonnées, servant bien quand elles sont bien commandées, ayant à leur tête une petite noblesse de village, forte de toute la force que donnent les préjugés et l’esprit étroit. La vraie résistance continentale à la Révolution et à l’Empire vint de cette Vendée du Nord ; c’est là que le gentilhomme campagnard, chez nous couvert de ridicule par la haute noblesse, la cour, la bourgeoisie, le peuple même, prit sa revanche sur la démocratie française, et prépara sourdement, sans bruit, sans plébiscites, sans journaux, l’étonnante apparition qui depuis quelques années vient de se dérouler devant nous.\par
La nécessité qui sous la Restauration obligea la France à renoncer à toute ambition extérieure, la sage politique qui sous Louis-Philippe rassura l’Europe, éloignèrent quelque temps le danger que recélait pour la France, sortie de la Révolution, cette anti-France de la Baltique, qui est la négation totale de nos principes les plus arrêtés. À part quelques paroles imprudentes d’hommes d’État de médiocre portée et quelques mauvais vers d’un poète étourdi\footnote{Il faut dire qu’il ne faisait que répondre à une provocation venant d’Allemagne.}, la France de ce temps songea peu à l’Allemagne. L’activité était tournée vers l’intérieur et non vers les agrandissements du dehors. On avait mille fois raison. La France est assez grande ; sa mission ne consiste pas à s’adjoindre des pays étrangers, elle consiste à offrir chez elle un de ces brillants développements dont elle est si capable, à montrer la réalisation prospère du système démocratique qu’elle a proclamé, et dont la possibilité n’a pas été jusqu’ici bien prouvée. Qu’un pays de dix-sept ou dix-huit millions d’habitants, comme était autrefois la Prusse, joue le tout pour le tout, et sorte, même au prix des plus grands hasards, d’une situation qui le laissait flotter entre les grands et les petits États, cela est naturel ; mais un pays de trente ou quarante millions d’habitants a tout ce qu’il faut pour être une grande nation. Que les frontières de la France aient été assez mal faites en !815, cela est possible ; mais, si l’on excepte quelques mauvais contours du côté de la Sarre et du Palatinat, qui furent tracés, à ce qu’il semble, sous le coup de chétives préoccupations militaires, le reste me paraît bien. Les pays flamands sont plus germaniques que français ; les pays wallons ont été empêchés de s’agglutiner au conglomérat français par des aventures historiques qui n’ont rien de fortuit ; cela tint au profond esprit municipal qui rendit la royauté française insupportable à ces pays. Il en faut dire autant de Genève et de la Suisse romande ; on peut ajouter que grande est l’utilité de ces petits pays français, séparés politiquement de la France ; ils offrent un asile aux émigrés de nos dissensions intestines, et, en temps de despotisme, ils servent de refuge à une pensée libre. La Prusse rhénane et le Palatinat sont des pays autrefois celtiques, mais profondément germanisés depuis deux mille ans. Si l’on excepte quelques vallées séparées de la France en 1815 par des raisons de stratégie, la France n’a donc pas un pouce de terre à désirer. L’Angleterre et l’Écosse n’ont en surface que les deux cinquièmes de la France, et pourtant l’Angleterre est-elle obligée de songer à des conquêtes territoriales pour être grande ?\par
Le sort de l’année 1848 fut, en cette question comme en toutes les autres, de soulever des problèmes qu’elle ne put résoudre, et qui, au bout d’un ou deux ans, reçurent des solutions par des moyens diamétralement opposés à ceux que rêvèrent les partis alors dominants. La question de l’unité allemande fut posée avec éclat ; selon la mode du temps, on crut tout arranger par une assemblée constituante. Ces efforts aboutirent à un éclatant échec. Qu’on traite les hommes de 1848 d’utopistes, ou qu’on reproche aux masses de n’avoir pas été assez éclairées pour les suivre, il est sûr que les essais de cette année demeurèrent tous infructueux. Pendant dix ans, les problèmes sommeillèrent, le patriotisme allemand sembla porter le deuil ; mais déjà un homme disait à ceux qui voulaient l’écouter : « Ces problèmes ne se résolvent pas, comme vous croyez, par la libre adhésion des peuples ; ils se résolvent par le fer et le feu. »\par
L’empereur Napoléon III rompit la glace par la guerre d’Italie, ou plutôt par la conclusion de cette guerre, qui fut l’annexion à la France de la Savoie et de Nice. La première de ces deux annexions était assez naturelle ; de tous les pays de langue française non réunis à la France, la Savoie était le seul qui pût sans inconvénient nous être dévolu ; depuis que le duc de Savoie était devenu roi d’Italie, une telle dévolution était presque dans la force des choses. Et cependant cette annexion eut bien plus d’inconvénients que d’avantages. Elle interdit à la France ce qui fait sa vraie force, le droit d’alléguer une politique désintéressée et uniquement inspirée par l’amour des principes ; elle donna une idée exagérée des plans d’agrandissement de l’empereur Napoléon III, mécontenta l’Angleterre, éveilla les soupçons de l’Europe, provoqua les hardies initiatives de M. de Bismarck.\par
Il est clair que, s’il y eut jamais un mouvement légitime en histoire, c’est celui qui, depuis soixante ans, porte l’Allemagne à se former en une seule nation. Si quelqu’un en tout cas a le droit de s’en plaindre, ce n’est pas la France, puisque l’Allemagne n’a obéi à cette tendance qu’à notre exemple, et pour résister à l’oppression que la France fit peser sur elle au XVII\textsuperscript{e} siècle et sous l’Empire. La France, ayant renoncé au principe de la légitimité, qui ne voyait dans telle ou telle agglomération de provinces en royaume ou en empire que la conséquence des mariages, des héritages, des conquêtes d’une dynastie, ne peut connaître qu’un seul principe de délimitation en géographie politique : je veux dire le principe des nationalités, ou, ce qui revient au même, la libre volonté des peuples de vivre ensemble, prouvée par des faits sérieux et efficaces. Pourquoi refuser à l’Allemagne le droit de faire chez elle ce que nous avons fait chez nous, ce que nous avons aidé l’Italie à faire ? N’est-il pas évident qu’une race dure, chaste, forte et grave comme la race germanique, une race placée au premier rang par les dons et le travail de la pensée, une race peu portée vers le plaisir, tout entière livrée à ses rêves et aux jouissances de son imagination, voudrait jouer dans l’ordre des faits politiques un rôle proportionné à son importance intellectuelle ? Le titre d’une nationalité, ce sont des hommes de génie, « gloires nationales », qui donnent aux sentiments de tel ou tel peuple une forme originale, et fournissent la grande matière de l’esprit national, quelque chose à aimer, à admirer, à vanter en commun. Dante, Pétrarque, les grands artistes de la Renaissance ont été les vrais fondateurs de l’unité italienne. Goethe, Schiller, Kant, Herder, ont créé la patrie allemande. Vouloir s’opposer à une éclosion annoncée par tant de signes eût été aussi absurde que de vouloir s’opposer à la marée montante. Vouloir lui donner des conseils, lui tracer la manière dont nous eussions désiré qu’elle s’accomplît, était puéril. Ce mouvement s’accomplissait par défiance de nous ; lui indiquer une règle, c’était fournir à une conscience nationale, soupçonneuse et susceptible, un {\itshape critérium} sûr, et l’inviter clairement à faire le contre-pied de ce que nous lui demandions. Certes je suis le premier à reconnaître qu’à ce besoin d’unité de la nation allemande il se mêla d’étranges excès. Le patriote allemand, comme le patriote italien, ne se détache pas facilement du vieux rôle universel de sa patrie. Certains Italiens rêvent encore le {\itshape primato} ; un très grand nombre d’Allemands rattachent leurs aspirations au souvenir du Saint-Empire, exerçant sur tout le monde européen une sorte de suzeraineté. Or la première condition d’un esprit national est de renoncer à toute prétention de rôle universel, le rôle universel étant destructeur de la nationalité. Plus d’une fois le patriotisme allemand s’est montré de la sorte injuste et partial. Ce théoricien de l’unité allemande qui soutient que l’Allemagne doit reprendre partout les débris de son vieil empire refuse d’écouter aucune raison quand on lui parle d’abandonner un pays aussi purement slave que le grand-duché de Posen\footnote{La possession de Posen par la Prusse ne saurait en aucune manière être assimilée à la possession de l’Alsace par la France. L'Alsace est francisée et ne proteste plus contre son annexion, tandis que Posen n’est pas germanisé et proteste. Le parallèle de l’Alsace est la Silésie, province slave de race et de langue, mais suffisamment germanisée, et dont personne ne conteste plus la légitime propriété à la Prusse.}. Le vrai, c’est que le principe des nationalités doit être entendu d’une façon large, sans subtilités. L’histoire a tracé les frontières des nations d’une manière qui n’est pas toujours la plus naturelle ; chaque nation a du trop, du trop peu ; il faut se tenir à ce que l’histoire a fait et au vœu des provinces, pour éviter d’impossibles analyses, d’inextricables difficultés.\par
Si la pensée de l’unité allemande était légitime, il était légitime aussi que cette unité se fît par la Prusse. Les tentatives parlementaires de Francfort ayant échoué, il ne restait que l’hégémonie de l’Autriche ou de la Prusse. L’Autriche renferme trop de Slaves, elle est trop antipathique à l’Allemagne protestante, elle a trop manqué durant des siècles à ses devoirs de puissance dirigeante en Allemagne, pour qu’elle pût être de nouveau appelée à jouer un rôle de ce genre. Si jamais, au contraire, il y eut une vocation historique bien marquée, ce fut celle de la Prusse depuis Frédéric le Grand. Il ne pouvait échapper à un esprit sagace que la Prusse était le centre d’un tourbillon ethnique nouveau, qu’elle jouait pour la nationalité allemande du Nord le rôle du cœur dans l’embryon, sauf à être plus tard absorbée par l’Allemagne qu’elle aurait faite, comme nous voyons le Piémont absorbé par l’Italie. Un homme se trouva pour s’emparer de toutes ces tendances latentes, pour les représenter et leur donner avec une énergie sans égale une pleine réalisation.\par
M. de Bismarck voulut deux choses que le philosophe le plus sévère pourrait déclarer légitimes si, dans l’application, le peu scrupuleux homme d’État n’avait montré que pour lui la force est synonyme de légitimité : d’abord, chasser de la confédération germanique l’Autriche, corps plus qu’à demi étranger qui l’empêchait d’exister ; en second lieu, grouper autour de la Prusse les membres de la patrie allemande que les hasards de l’histoire avaient dispersés. M. de Bismarck vit-il au-delà ? Son point de vue nécessairement borné d’homme pratique lui permit-il de soupçonner qu’un jour la Prusse serait absorbée par l’Allemagne et disparaîtrait en quelque sorte dans sa victoire, comme Rome finit d’exister en tant que ville le jour où elle eut achevé son œuvre d’unification ? Je l’ignore, car M. de Bismarck ne s’est pas jusqu’ici offert à l’analyse ; il ne s’y offrira peut-être jamais. Une des questions qu’un esprit curieux se pose le plus souvent, en réfléchissant sur l’histoire contemporaine, est de savoir si M. de Bismarck est philosophe, s’il voit la vanité de ce qu’il fait, tout en y travaillant avec ardeur, ou bien si c’est un croyant en politique, s’il est dupe de son œuvre, comme tous les esprits absolus, et n’en voit pas la caducité. J’incline vers la première hypothèse, car il me paraît difficile qu’un esprit si complet ne soit pas critique, et ne mesure pas dans son action la plus ardente les limites et le côté faible de ses desseins. Quoi qu’il en soit, s’il voit dans l’avenir les impossibilités du parti qui consisterait à faire de l’Allemagne une Prusse agrandie, il se garde de le dire, car le fanatisme étroit du parti des hobereaux prussiens ne supporterait pas un moment la pensée que le but de ce qui se fait par la Prusse n’est pas de prussianiser toute l’Allemagne, et plus tard le monde entier, au nom d’une sorte de mysticisme politique dont on semble vouloir se réserver le secret.\par
Les plans de M. de Bismarck furent élaborés dans la confidence et avec l’adhésion de l’empereur Napoléon III, ainsi que du petit nombre de personnes qui étaient initiées à ses desseins. Il est injuste de faire de cela un reproche à l’empereur Napoléon. C’est la France qui a élevé dans le monde le drapeau des nationalités ; toute nationalité qui naît et grandit devrait naître et grandir avec les encouragements de la France, devenir pour elle une amie. La nationalité allemande étant une nécessité historique, la sagesse voulait qu’on ne se mît pas à la traverse. La bonne politique n’est pas de s’opposer à ce qui est inévitable ; la bonne politique est d’y servir et de s’en servir. Une grande Allemagne libérale, formée en pleine amitié avec la France, devenait une pièce capitale en Europe, et créait avec la France et l’Angleterre une invincible trinité, entraînant le monde, surtout la Russie, dans les voies du progrès par la raison. Il était donc souverainement désirable que l’unité allemande, venant à se réaliser, ne se fît pas malgré la France, qu’elle se fît, bien au contraire, avec notre assentiment. La France n’était pas obligée d’y contribuer, mais elle était obligée de ne pas s’y opposer ; il était même naturel de songer au bon vouloir de la jeune nation future, de se ménager de sa part quelque chose de ce sentiment profond que les États-Unis d’Amérique garderont encore longtemps à la France en souvenir de Lafayette. Était-il opportun de tirer profit des circonstances pour notre agrandissement territorial ? Non, en principe, puisque de tels agrandissements sont à peu près inutiles. En quoi la France est-elle plus grande depuis l’adjonction de Nice et de la Savoie ? Cependant, l’opinion publique superficielle attachant beaucoup de prix à ces agrandissements, on pouvait, à l’époque des tractations amicales, stipuler quelques cessions, portant sur des pays disposés à se réunir à la France, pourvu qu’il fût bien entendu que ces agrandissements n’étaient pas le but de la négociation, que l’unique but de celle-ci était l’amitié de la France et de l’Allemagne. Pour répondre aux taquineries des hommes d’État de l’opposition et satisfaire à certaines exigences des militaires qui ont sans doute leur fondement, on pouvait, par exemple, stipuler avant la guerre la cession du Luxembourg au cas qu’il y consentit et la rectification de la Sarre, auxquelles la Prusse eût probablement consenti alors. Je le répète, j’estime qu’il eût mieux valu ne rien demander : le Luxembourg ne nous eût pas apporté plus de force que la Savoie ou Nice. Quant aux contours stratégiques des frontières, combien une bonne politique eût été un meilleur rempart ! L’effet d’une bonne politique eût été que personne ne nous eût attaqués, ou que, si quelqu’un avait pris contre nous l’offensive, nous eussions été défendus par la sympathie de toute l’Europe. — Quoi qu’il en soit, on ne prit aucun parti : une indécision déplorable paralysa la plume de l’empereur Napoléon III, et Sadowa arriva sans que rien eût été convenu pour le lendemain. Cette bataille qui, si l’on avait suivi une politique consistante, aurait pu être une victoire pour la France, devint ainsi une défaite, et, huit jours après, le gouvernement français prenait le deuil de l’événement auquel il avait plus que personne contribué.\par
À ce moment, d’ailleurs, entrèrent en scène deux éléments qui n’avaient eu aucune part aux conversations de Biarritz : l’opinion française et l’opinion prussienne exaltées. M. de Bismarck n’est pas la Prusse ; en dehors de lui existe un parti fanatique, absolu, tout d’une pièce, avec lequel il doit compter. M. de Bismarck par sa naissance appartient à ce parti ; mais il n’en a pas les préjugés. Pour se rendre maître de l’esprit du roi, faire taire ses scrupules et dominer les conseils étroits qui l’entourent, M. de Bismarck est obligé à des sacrifices. Après la victoire de Sadowa, le parti fanatique se trouva plus puissant que jamais ; toute transaction devint impossible. Ce qui arrivait à l’empereur Napoléon III arrivera, je le crains, à plusieurs de ceux qui auront des relations avec la Prusse. Cet esprit intraitable, cette roideur de caractère, cette fierté exagérée seront la source de beaucoup de difficultés. — En France, l’empereur Napoléon III se montra également débordé par une certaine opinion. L’opposition fut, cette fois, ce qu’elle est trop souvent, superficielle et déclamatoire. Il était facile de montrer que la conduite du gouvernement avait été pleine d’imprévoyance et de tergiversations. Il est clair qu’à l’époque des ouvertures de M. de Bismarck, le gouvernement aurait dû ou refuser de l’écouter ou avoir un plan de conduite qu’il pût appuyer d’une bonne armée sur le Rhin ; mais ce n’était pas là une raison pour venir soutenir chaque année, ainsi que le faisait l’opposition, que la France avait été vaincue à Sadowa, ni surtout pour établir en doctrine que la frontière de la France devait être garnie de petits États faibles, ennemis les uns des autres. Pouvait-on inventer un moyen plus efficace pour les persuader d’être unis et forts ? M. Thiers contribua beaucoup par ses aveux à exciter l’opinion allemande, laquelle est persuadée que cet honorable homme d’État représente l’opinion dominante de la bourgeoisie française et ses instincts secrets.\par
Le règlement de la question du Luxembourg mit cette situation funeste dans tout son jour. Rien n’avait été convenu avant Sadowa entre la France et la Prusse : la Prusse n’éluda donc aucun engagement en refusant toute concession ; mais, si la modération avait été dans le caractère de la cour de Berlin, comment ne lui eût-elle pas conseillé de tenir compte de l’émotion de la France, de ne pas pousser son droit et ses avantages à l’extrême ? Le Luxembourg est un pays insignifiant, tout à fait hybride, ni allemand ni français, ou, si l’on veut, l’un et l’autre. Son annexion à la France, précédée d’un plébiscite, n’avait rien qui pût mécontenter l’Allemand le plus correct dans son patriotisme. La roideur systématique de la Prusse prouva qu’elle n’entendait garder aucun souvenir reconnaissant des tractations qui avaient précédé Sadowa, et que la France, malgré l’appui réel qu’elle lui avait prêté, était toujours pour elle l’éternelle ennemie. Du côté de la France, on avait amené ce résultat par une série de fautes ; on avait été si malavisé, qu’on n’avait même pas le droit de se plaindre. On avait voulu jouer au fin, on avait trouvé plus fin que soi. On avait fait comme celui qui, ayant dans son jeu des cartes excellentes, n’a pas pu se décider à les jeter sur table, les réservant toujours pour des coups qui ne viennent jamais.\par
Est-ce à dire, comme le pensent beaucoup de personnes, que, depuis 1866, la guerre entre la France et la Prusse fût inévitable ? Non certes. Quand on peut attendre, peu de choses sont inévitables ; or on pouvait gagner du temps. La mort du roi de Prusse, ce qu’on sait du caractère sage et modéré du prince et de la princesse de Prusse, pouvaient déplacer bien des choses. Le parti militaire féodal prussien, qui est l’une des grandes causes de danger pour la paix de l’Europe, semble destiné à céder avec le temps beaucoup de son ascendant à la bourgeoisie berlinoise, à l’esprit allemand, si large, si libre, et qui deviendra profondément libéral dès qu’il sera délivré de l’étreinte du casernement prussien. Je sais que les symptômes de ceci ne se montrent guère encore, que l’Allemagne, toujours un peu timide dans l’action, a été conquise par la Prusse, sans qu’aucun indice ait montré la Prusse disposée à se perdre dans l’Allemagne ; mais le temps n’est pas venu pour une telle évolution. Acceptée comme moyen de lutte contre la France, l’hégémonie prussienne ne faiblira que quand une pareille lutte n’aura plus de raison d’être. La force avec laquelle est lancé le mouvement allemand donnera lieu à des développements très rapides. Il n’y a plus aucune analogie en histoire, si l’Allemagne conquise ne conquiert la Prusse à son tour et ne l’absorbe. Il est inadmissible que la race allemande, si peu révolutionnaire qu’elle soit, ne triomphe pas du noyau prussien, quelque résistant qu’il puisse être. Le principe prussien, d’après lequel la base d’une nation est une armée, et la base de l’armée une petite noblesse, ne saurait être appliqué à l’Allemagne. L’Allemagne, Berlin même, a une bourgeoisie. La base de la vraie nation allemande sera, comme celle de toutes les nations modernes, une bourgeoisie riche. Le principe prussien a fait quelque chose de très fort, mais qui ne saurait durer au-delà du jour où la Prusse aura terminé son œuvre. Sparte eût cessé d’être Sparte, si elle eût fait l’unité de la Grèce. La constitution et les mœurs romaines disparurent dès que Rome devint maîtresse du monde ; à partir de ce jour-là, Rome se vit gouvernée par le monde, et ce ne fut que justice.\par
Chaque année eût ainsi apporté à l’état de choses sorti de Sadowa les plus profondes transformations. Une heure d’aberration a troublé toutes les espérances des bons esprits. Sans songer qu’une nation jeune, dans tout le feu de son développement, a d’immenses avantages sur une nation vieillie qui a déjà rempli son programme et atteint l’égalité, on s’est jeté dans le gouffre de gaieté de cœur. La présomption et l’ignorance des militaires, l’étourderie de nos diplomates, leur vanité, leur sotte foi dans l’Autriche, machine disloquée dont il y a peu de compte à tenir, l’absence de pondération sérieuse dans le gouvernement, les accès bizarres d’une volonté intermittente comme les réveils d’un Epiménide ont amené sur l’espèce humaine les plus grands malheurs qu’elle eût connus depuis cinquante-cinq ans. Un incident qu’une habile diplomatie eût aplani en quelques heures a suffi pour déchaîner l’enfer… Retenons nos malédictions ; il y a des moments où l’horrible réalité est la plus cruelle des imprécations.
\subsection[{II}]{II}
\noindent Qui a fait la guerre ? Nous l’avons dit, ce me semble. — Il faut se garder, dans ces sortes de questions, de ne voir que les causes immédiates et prochaines. Si l’on se bornait aux considérations restreintes d’un observateur inattentif, la France aurait tous les torts. Si l’on se place à un point de vue plus élevé, la responsabilité de l’horrible malheur qui a fondu sur l’humanité en cette funeste année doit être partagée. La Prusse a facilement dans ses manières d’agir quelque chose de dur, d’intéressé, de peu généreux. Sentant sa force, elle n’a fait aucune concession. Du moment que M. de Bismarck voulut exécuter ses grandes entreprises de concert avec la France, il devait accepter les conséquences de la politique qu’il avait choisie. M. de Bismarck n’était pas obligé de mettre l’empereur Napoléon III dans ses confidences ; mais, l’ayant fait, il était obligé d’avoir des égards pour l’empereur et les hommes d’État français, ainsi que pour une fraction de l’opinion qu’il fallait ménager. Le grand mal de la Prusse, c’est l’orgueil. Foyer puissant d’ancien régime, elle s’irrite de notre prospérité bourgeoise ; ses gentilshommes sont blessés de voir des roturiers, je ne dis pas plus riches qu’eux, mais exerçant comme eux la profession qui ailleurs est le privilège de la noblesse. La jalousie chez eux double l’orgueil. « Nous sommes une jeunesse pauvre, disent-ils, des cadets qui veulent se faire leur place dans le monde. » Une des causes qui ont produit M. de Bismarck a été la vanité blessée du diplomate abreuvé d’avanies par ses confrères autrichiens traitant la Prusse en parvenue. Le sentiment qui a créé la Prusse a été quelque chose d’analogue : l’homme sérieux, pauvre, intelligent, sans charme, supporte avec peine les succès de société d’un rival qui, tout en lui étant fort inférieur pour les qualités solides, fait figure dans le monde, règle la mode et réussit par des dédains aristocratiques à l’empêcher de si y faire accepter.\par
La France n’a pas été moins coupable. Les journaux ont été superficiels, le parti militaire s’est montré présomptueux et entêté, l’opposition n’a paru attentive qu’à la recherche d’une fausse popularité, blâmant le gouvernement s’il préparait la guerre, l’insultant s’il ne la faisait pas, parlant sans cesse de la honte de Sadowa et de la nécessité d’une revanche ; mais le grand mal a été l’excès du pouvoir personnel. La conversion à la monarchie parlementaire affectée depuis un an était si peu sérieuse, qu’un ministère tout entier, la Chambre, le Sénat ont cédé presque sans résistance à une pensée personnelle du souverain qui ne répondait nullement à leurs idées ni à leurs désirs.\par
Et maintenant, qui fera la paix ?… La pire conséquence de la guerre, c’est de rendre impuissants ceux qui ne l’ont pas voulue, et d’ouvrir un cercle fatal où le bon sens est qualifié de lâcheté, parfois de trahison. Nous parlerons avec franchise. Une seule force au monde sera capable de réparer le mal que l’orgueil féodal, le patriotisme exagéré, l’excès du pouvoir personnel, le peu de développement du gouvernement parlementaire sur le continent ont fait en cette circonstance à la civilisation.\par
Cette force, c’est l’Europe. L’Europe a un intérêt majeur à ce qu’aucune des deux nations ne soit ni trop victorieuse ni trop vaincue. La disparition de la France du nombre des grandes puissances serait la fin de l’équilibre européen. J’ose dire que l’Angleterre en particulier sentirait, le jour où un tel événement viendrait à se produire, les conditions de son existence toutes changées. La France est une des conditions de la prospérité de l’Angleterre. L’Angleterre, selon la grande loi qui veut que la race primitive d’un pays prenne à la longue le dessus sur toutes les invasions, devient chaque jour plus celtique et moins germanique ; dans la grande lutte des races, elle est avec nous ; l’alliance de la France et de l’Angleterre est fondée pour des siècles. Que l’Angleterre porte sa pensée du côté des États-Unis, de Constantinople, de l’Inde : elle verra qu’elle a besoin de la France et d’une France forte.\par
Il ne faut pas s’y tromper en effet : une France faible et humiliée ne saurait exister. Que la France perde l’Alsace et la Lorraine, et la France n’est plus. L’édifice est si compact, que l’enlèvement d’une ou deux grosses pierres le ferait crouler. L’histoire naturelle nous apprend que l’animal dont l’organisation est très centralisée ne souffre pas l’amputation d’un membre important ; on voit souvent un homme à qui l’on coupe une jambe mourir de phtisie ; de même la France atteinte dans ses parties principales verrait sa vie générale s’éteindre et ses organes du centre insuffisants pour renvoyer la vie jusqu’aux extrémités.\par
Qu’on ne rêve donc pas de concilier deux choses contradictoires : conserver la France et l’amoindrir. Il y a des ennemis absolus de la France qui croient que le but suprême de la politique contemporaine doit être d’étouffer une puissance qui, selon eux, représente le mal. Que ces fanatiques conseillent d’en finir avec l’ennemi qu’ils ont momentanément vaincu, rien de plus simple ; mais que ceux qui croient que le monde serait mutilé si la France disparaissait y prennent garde. Une France diminuée perdrait successivement toutes ses parties ; l’ensemble se disloquerait, le midi se séparerait ; l’œuvre séculaire des rois de France serait anéantie, et, je vous le jure, le jour où cela arriverait, personne n’aurait lieu de s’en réjouir. Plus tard, quand on voudrait former la grande coalition que provoque toute ambition démesurée, on regretterait en Europe de ne pas avoir été plus prévoyant. Deux grandes races sont en présence ; toutes deux ont fait de grandes choses, toutes deux ont une grande tâche à remplir en commun ; il ne faut pas que l’une d’elles soit mise en un état qui équivaille à sa destruction. Le monde sans la France serait aussi mutilé que le monde sans l’Allemagne ; ces grands organes de l’humanité ont chacun leur office : il importe de les maintenir pour l’accomplissement de leur mission diverse. Sans attribuer à l’esprit français le premier rôle dans l’histoire de l’esprit humain, on doit reconnaître qu’il y joue un rôle essentiel : le concert serait troublé si cette note y manquait. Or, si vous voulez que l’oiseau chante, ne touchez pas à son bocage. La France humiliée, vous n’aurez plus d’esprit français.\par
Une intervention de l’Europe assurant à l’Allemagne l’entière liberté de ses mouvements intérieurs, maintenant les limites fixées en 1815 et défendant à la France d’en rêver d’autres, laissant la France vaincue, mais fière dans son intégrité, la livrant au souvenir de ses fautes et la laissant se dégager en toute liberté et comme elle l’entendrait de l’étrange situation intérieure qu’elle s’est faite, telle est la solution que doivent, selon nous, désirer les amis de l’humanité et de la civilisation. Non seulement cette solution mettrait fin à l’horrible déchirement qui trouble en ce moment la famille européenne, elle renfermerait de plus le germe d’un pouvoir destiné à exercer sur l’avenir l’action la plus bienfaisante.\par
Comment, en effet, un effroyable événement comme celui qui laissera autour de l’année 1870 un souvenir de terreur a-t-il été possible ? Parce que les diverses nations européennes sont trop indépendantes les unes des autres et n’ont personne au-dessus d’elles, parce qu’il n’y a ni congrès, ni diète, ni tribunal amphictyonique qui soient supérieurs aux souverainetés nationales. Un tel établissement existe à l’état virtuel, puisque l’Europe, surtout depuis 1814, a fréquemment agi en nom collectif, appuyant ses résolutions de la menace d’une coalition ; mais ce pouvoir central n’a pas été assez fort pour empêcher des guerres terribles. Il faut qu’il le devienne. Le rêve des utopistes de la paix, un tribunal sans armée pour appuyer ses décisions, est une chimère ; personne ne lui obéira. D’un autre côté, l’opinion selon laquelle la paix ne serait assurée que le jour où une nation aurait sur les autres une supériorité incontestée est l’inverse de la vérité ; toute nation exerçant l’hégémonie prépare par cela seul sa ruine en amenant la coalition de tous contre elle. La paix ne peut être établie et maintenue que par l’intérêt commun de l’Europe, ou, si l’on aime mieux, par la ligue des neutres passant à une attitude comminatoire. La justice entre deux parties contendantes n’a aucune chance de triompher ; mais entre dix parties contendantes la justice l’emporte, car il n’y a qu’elle qui offre une base commune d’entente, un terrain commun. La force capable de maintenir contre le plus puissant des États une décision jugée utile au salut de la famille européenne réside donc uniquement dans le pouvoir d’intervention, de médiation, de coalition des divers États. Espérons que ce pouvoir, prenant des formes de plus en plus concrètes et régulières, amènera dans l’avenir un vrai congrès, périodique, sinon permanent, et sera le cœur d’États-Unis d’Europe liés entre eux par un pacte fédéral. Aucune nation alors n’aura le droit de s’appeler « la grande nation », mais il sera loisible à chacune d’être une grande nation, à condition que ce titre, elle l’attende des autres et ne prétende pas se le décerner. C’est à l’histoire qu’il appartiendra plus tard de spécifier ce que chaque peuple aura fait pour l’humanité et de désigner les pays qui, à certaines époques, ont pu avoir sur les autres certains genres de supériorité.\par
De la sorte, on peut espérer que la crise épouvantable où est engagée l’humanité trouvera un moment d’arrêt. Le lendemain du jour où la faux de la mort aura été arrêtée, que devra-t-on faire ? Attaquer énergiquement la cause du mal. La cause du mal a été un déplorable régime politique qui a fait dépendre l’existence d’une nation des présomptueuses vantardises de militaires bornés, des dépits et de la vanité blessée de diplomates inconsistants. Opposons à cela le régime parlementaire, un vrai gouvernement des parties sérieuses et modérées du pays, non la chimère démocratique du règne de la volonté populaire avec tous ses caprices, mais le règne de la volonté nationale, résultat des bons instincts du peuple savamment interprétés par des pensées réfléchies. Le pays n’a pas voulu la guerre ; il ne la voudra jamais ; il veut son développement intérieur, soit sous forme de richesse, soit sous forme de libertés publiques. Donnons à l’étranger le spectacle de la prospérité, de la liberté, du calme, de l’égalité bien entendue, et la France reprendra l’ascendant qu’elle a perdu par les imprudentes manifestations de ses militaires et de ses diplomates. La France a des principes qui, bien que critiquables et dangereux à quelques égards, sont faits pour séduire le monde, quand la France donne la première l’exemple du respect de ces principes ; qu’elle présente chez elle le modèle d’un État vraiment libéral, où les droits de chacun sont garantis, d’un État bienveillant pour les autres États, renonçant définitivement à l’idée d’agrandissement, et tous, loin de l’attaquer, s’efforceront de l’imiter.\par
Il y a, je le sais, dans le monde des foyers de fanatisme où le tempérament règne encore ; il y a en certains pays une noblesse militaire, ennemie-née de ces conceptions raisonnables, et qui rêve l’extermination de ce qui ne lui ressemble pas. L’élément féodal de la Prusse d’une part, la Russie de l’autre, sont à cet âge où l’on a l’âcreté du sang barbare, sans retour en arrière ni désillusion. La France et jusqu’à un certain point l’Angleterre ont atteint leur but. La Prusse et la Russie ne sont pas encore arrivées à ce moment où l’on possède ce que l’on a voulu, où l’on considère froidement ce pour quoi l’on a troublé le monde, et où l’on s’aperçoit que ce n’est rien, que tout ici-bas n’est qu’un épisode d’un rêve éternel, une ride à la surface d’un infini qui tour à tour nous produit et nous absorbe. Ces races neuves et violentes du Nord sont bien plus naïves ; elles sont dupes de leurs désirs ; entraînées par le but qu’elles se proposent, elles ressemblent au jeune homme qui s’imagine que, l’objet de sa passion une fois obtenu, il sera pleinement heureux. À cela se joint un trait de caractère, un sentiment que les plaines sablonneuses du nord de l’Allemagne paraissent toujours avoir inspiré, le sentiment des Vandales chastes devant les mœurs et le luxe de l’Empire romain, une sorte de fureur puritaine, la jalousie et la rage contre la vie facile de ceux qui jouissent. Cette humeur sombre et fanatique existe encore de nos jours. De tels « esprits mélancoliques », comme on disait autrefois, se croient chargés de venger la vertu, de redresser les nations corrompues. Pour ces exaltés, l’idée de l’empire allemand n’est pas celle d’une nationalité limitée, libre chez elle, ne s’occupant pas du reste du monde ; ce qu’ils veulent, c’est une action universelle de la race germanique, renouvelant et dominant l’Europe. C’est là une frénésie bien chimérique ; car supposons, pour plaire à ces esprits chagrins, la France anéantie, la Belgique, la Hollande, la Suisse écrasées, l’Angleterre passive et silencieuse ; que dire du grand spectre de l’avenir germanique, des Slaves, qui aspireront d’autant plus à se séparer du corps germanique que ce dernier s’individualisera davantage ? La conscience slave s’élève en proportion de la conscience germanique, et s’oppose à celle-ci comme un pôle contraire ; l’une crée l’autre. L’Allemand a droit comme tout le monde à une patrie ; pas plus que personne, il n’a droit à la domination. Il faut observer d’ailleurs que de telles visées fanatiques ne sont nullement le fait de l’Allemagne éclairée. La plus complète personnification de l’Allemagne, c’est Goethe. Quoi de moins prussien que Goethe ? Qu’on se figure ce grand homme à Berlin et le débordement de sarcasmes olympiens que lui eussent inspirés cette roideur sans grâce ni esprit, ce lourd mysticisme de guerriers pieux et de généraux craignant Dieu ! Une fois délivrées de la crainte de la France, ces populations fines de la Saxe, de la Souabe, se soustrairont à l’enrégimentation prussienne ; le Midi en particulier reprendra sa vie gaie, sereine, harmonieuse et libre.\par
Le moyen pour que cela arrive, c’est que nous ne nous en mêlions pas. Le grand facteur de la Prusse, c’est la France, ou, pour mieux dire, l’appréhension d’une ingérence de la France dans les affaires allemandes. Moins la France s’occupera de l’Allemagne, plus l’unité allemande sera compromise, car l’Allemagne ne veut l’unité que par mesure de précaution. La France est en ce sens toute la force de la Prusse. La Prusse (j’entends la Prusse militaire et féodale) aura été une crise, non un état permanent ; ce qui durera réellement, c’est l’Allemagne. La Prusse aura été l’énergique moyen employé par l’Allemagne pour se délivrer de la menace de la France bonapartiste. La réunion des forces allemandes dans la main de la Prusse n’est qu’un fait amené par une nécessité passagère. Le danger disparu, l’union disparaîtra, et l’Allemagne reviendra bientôt à ses instincts naturels. Le lendemain de sa victoire, la Prusse se trouvera ainsi en face d’une Europe hostile et d’une Allemagne reprenant son goût pour les autonomies particulières. C’est ce qui me fait dire avec assurance : la Prusse passera, l’Allemagne restera. Or l’Allemagne livrée à son propre génie sera une nation libérale, pacifique, démocratique même dans le sens légitime ; je crois que les sciences sociales lui devront des progrès remarquables, et que plusieurs idées qui chez nous ont revêtu le masque effrayant de la démocratie socialiste se produiront chez elle sous une forme bienfaisante et réalisable.\par
La plus grande faute que pourrait commettre l’école libérale au milieu des horreurs qui nous assiègent, ce serait de désespérer. L’avenir est à elle. Cette guerre, objet des malédictions futures, est arrivée parce qu’on s’est écarté des maximes libérales, maximes qui sont en même temps celles de la paix et de l’union des peuples. Le funeste désir d’une revanche, désir qui prolongerait indéfiniment l’extermination, sera écarté par un sage développement de la politique libérale. C’est une fausse idée que la France puisse imiter les institutions militaires prussiennes. L’état social de la France ne veut pas que tous les citoyens soient soldats, ni que ceux qui le sont le soient toujours. Pour maintenir une armée organisée à la prussienne, il faut une petite noblesse ; or nous n’avons pas de noblesse, et, si nous en avions une, le génie de la France ferait que nous en aurions plutôt une grande qu’une petite. La Prusse fonde sa force sur le développement de l’instruction primaire et sur l’identité de l’armée et de la nation. Le parti conservateur en France admet difficilement ces deux principes, et, à vrai dire, il n’est pas sûr que le pays en soit capable. La Prusse étant, comme dirait Plutarque, d’un tempérament plus vertueux que la France, peut porter des institutions qui, appliquées sans précautions, donneraient peut-être chez nous des fruits tout différents, et seraient une source de révolutions. La Prusse touche en cela le bénéfice de la grande abnégation politique et sociale de ses populations. En obligeant ses rivaux à soigner l’instruction primaire et à imiter sa {\itshape Landwehr} (innovations qui, dans des pays catholiques et révolutionnaires, seront probablement anarchiques), elle les force à un régime sain pour elle, malsain pour eux, comme le buveur qui fait boire à son partenaire un vin qui l’enivrera, tandis que lui gardera sa raison.\par
En résumé, l’immense majorité de l’espèce humaine a horreur de la guerre. Les idées vraiment chrétiennes de douceur, de justice, de bonté, conquièrent de plus en plus le monde. L’esprit belliqueux ne vit plus que chez les soldats de profession, dans les classes nobles du nord de l’Allemagne et en Russie. La démocratie ne veut pas, ne comprend pas la guerre. Le progrès de la démocratie sera la fin du règne de ces hommes de fer, survivants d’un autre âge, que notre siècle a vus avec terreur sortir des entrailles du vieux monde germanique. Quelle que soit l’issue de la guerre actuelle, ce parti sera vaincu en Allemagne. La démocratie lui a compté les jours. J’ai des appréhensions contre certaines tendances de la démocratie, et je les ai dites, il y a un an, avec sincérité ; mais certes, si la démocratie se borne à débarrasser l’espèce humaine de ceux qui, pour la satisfaction de leurs vanités et de leurs rancunes, font égorger des millions d’hommes, elle aura mon plein assentiment et ma reconnaissante sympathie.\par
Le principe des nationalités indépendantes n’est pas de nature, comme plusieurs le pensent, à délivrer l’espèce humaine du fléau de la guerre ; au contraire, j’ai toujours craint que le principe des nationalités, substitué au doux et paternel symbole de la légitimité, ne fît dégénérer les luttes des peuples en exterminations de race, et ne chassât du code du droit des gens ces tempéraments, ces civilités qu’admettaient les petites guerres politiques et dynastiques d’autrefois. On verra la fin de la guerre quand, au principe des nationalités, on joindra le principe qui en est le correctif, celui de la fédération européenne, supérieure à toutes les nationalités, ajoutons : quand les questions démocratiques, contrepartie des questions de politique pure et de diplomatie, reprendront leur importance. Qu’on se rappelle 1848 : le mouvement français se reproduisit en secousses simultanées dans toute l’Allemagne. Partout les chefs militaires surent étouffer les naïves aspirations d’alors ; mais qui sait si les pauvres gens que ces mêmes chefs militaires mènent aujourd’hui à l’égorgement n’arriveront pas à éclaircir leur conscience ? Des naturalistes allemands, qui ont la prétention d’appliquer leur science à la politique, soutiennent, avec une froideur qui voudrait avoir l’air d’être profonde, que la loi de la destruction des races et de la lutte pour la vie se retrouve dans l’histoire, que la race la plus forte chasse nécessairement la plus faible, et que la race germanique, étant plus forte que la race latine et la race slave, est appelée à les vaincre et à se les subordonner. Laissons passer cette dernière prétention, quoiqu’elle pût donner lieu à bien des réserves. N’objectons pas non plus à ces matérialistes transcendants que le droit, la justice, la morale, choses qui n’ont pas de sens dans le règne animal, sont des lois de l’humanité ; des esprits si dégagés des vieilles idées nous répondraient probablement par un sourire. Bornons-nous à une observation : les espèces animales ne se liguent pas entre elles. On n’a jamais vu deux ou trois espèces en danger d’être détruites former une coalition contre leur ennemi commun ; les bêtes d’une même contrée n’ont entre elles ni alliances ni congrès. Le principe fédératif, gardien de la justice, est la base de l’humanité. Là est la garantie des droits de tous ; il n’y a pas de peuple européen qui ne doive s’incliner devant un pareil tribunal. Cette grande race germanique, bien plus réellement grande que ne le veulent ses maladroits apologistes, aura certes dans l’avenir un haut titre de plus, si l’on peut dire que c’est sa puissante action qui aura introduit définitivement dans le droit européen un principe aussi essentiel. Toutes les grandes hégémonies militaires, celle de l’Espagne au XVI\textsuperscript{e} siècle, celle de la France sous Louis XIV, celle de la France sous Napoléon, ont abouti à un prompt épuisement. Que la Prusse y prenne garde, sa politique radicale peut l’engager dans une série de complications dont il ne lui soit plus loisible de se dégager ; un œil pénétrant verrait peut-être dès à présent le nœud déjà formé de la coalition future. Les sages amis de la Prusse lui disent tout bas, non comme menace, mais comme avertissement : {\itshape Vœ victoribus} !\par

\section[{Lettre adressée à M. Strauss}]{Lettre adressée à M. Strauss}\renewcommand{\leftmark}{Lettre adressée à M. Strauss}


\dateline{Le 13 septembre 1870}
\noindent Le dix-huit août 1870, parut, dans la {\itshape Gazette d’Augsbourg}, une lettre que M. Strauss me faisait l’honneur de m’adresser sur les événements du temps. Elle se terminait ainsi :\par
« Vous trouverez peut-être étrange aussi que ces lignes ne vous parviennent que par l’intermédiaire d’un journal. Certes, dans des temps moins agités, je me serais assuré tout d’abord de votre agrément ; mais, dans les circonstances actuelles, avant que ma demande fût parvenue dans vos mains, et votre réponse dans les miennes, le vrai moment aurait passé. Et j’estime d’ailleurs qu’il peut y avoir quelque utilité à ce que, dans cette crise, deux hommes appartenant aux deux nations rivales, indépendants l’un de l’autre et étrangers à tout esprit de parti, échangent leurs vues sans passion, mais en toute franchise, sur les causes et sur la portée de la lutte actuelle ; car les pages que je viens d’écrire n’auront complètement atteint leur but que si elles vous déterminent à un semblable exposé de sentiments, fait à votre point de vue. »\par
Je me rendis à cette invitation ; le 16 septembre 1870, parut dans le {\itshape Journal des Débats} la réponse que je vais reproduire. La veille avait paru dans le même journal la traduction de la lettre de M. Strauss.\par
Monsieur et savant maître,\par
Vos hautes et philosophiques paroles nous sont arrivées à travers ce déchaînement de l’enfer, comme un message de paix ; elles nous ont été d’une grande consolation, à moi surtout qui dois à l’Allemagne ce à quoi je tiens le plus, ma philosophie, je dirai presque ma religion. J’étais au séminaire Saint-sulpiciens vers 1843, quand je commençai à connaître l’Allemagne par Goethe et Herder. Je crus entrer dans un temple, et, à partir de ce moment, tout ce que j’avais tenu jusque-là pour une pompe digne de la Divinité me fit l’effet de fleurs de papier jaunies et fanées. Aussi, comme je vous l’ai écrit au premier moment des hostilités, cette guerre m’a rempli de douleur, d’abord à cause des épouvantables calamités qu’elle ne pouvait manquer d’entraîner, ensuite à cause des haines, des jugements erronés qu’elle répandra et du tort qu’elle fera aux progrès de la vérité. Le grand malheur du monde est que la France ne comprend pas l’Allemagne et que l’Allemagne ne comprend pas la France : ce malentendu ne fera que s’aggraver. On ne combat le fanatisme que par un fanatisme opposé ; après la guerre, nous nous trouverons en présence d’esprits rétrécis par la passion, qui admettront difficilement notre libre et large sérénité.\par
Vos idées sur l’histoire du développement de l’unité allemande sont d’une parfaite justesse. Au moment où j’ai reçu le numéro de la {\itshape Gazette d’Augsbourg} qui contenait votre belle lettre, j’étais justement occupé à écrire pour la {\itshape Revue des Deux Mondes} un article qui paraîtra ces jours-ci, et où j’exposais des vues identiques aux vôtres. Il est clair que, dès que l’on a rejeté le principe de la légitimité dynastique, il n’y a plus, pour donner une base aux délimitations territoriales des États, que le droit des nationalités, c’est-à-dire des groupes naturels déterminés par la race, l’histoire et la volonté des populations. Or, s’il y a une nationalité qui ait un droit évident d’exister en toute son indépendance, c’est assurément la nationalité allemande. L’Allemagne a le meilleur titre national, je veux dire un rôle historique de première importance, une âme, une littérature, des hommes de génie, une conception particulière des choses divines et humaines. L’Allemagne a fait la plus importante révolution des temps modernes, la Réforme ; en outre, depuis un siècle, l’Allemagne a produit un des plus beaux développements intellectuels qu’il y ait jamais eu, un développement qui a, si j’ose le dire, ajouté un degré de plus à l’esprit humain en profondeur et en étendue, si bien que ceux qui n’ont pas participé à cette culture nouvelle sont, à ceux qui l’ont traversée, comme celui qui ne connaît que les mathématiques élémentaires est à celui qui connaît le calcul différentiel.\par
Qu’une si grande force intellectuelle, jointe à tant de moralité et de sérieux, dût produire un mouvement politique correspondant, que la nation allemande fût appelée à prendre dans l’ordre extérieur, matériel et pratique, une importance proportionnée à celle qu’elle avait dans l’ordre de l’esprit, c’est ce qui était évident pour toute personne instruite, non aveuglée par la routine et les partis pris superficiels. Ce qui ajoutait à la légitimité des vœux de l’Allemagne, c’est que le besoin d’unité était chez elle une mesure de précaution justifiée par les déplorables folies du Premier Empire, folies que les Français éclairés réprouvent autant que les Allemands, mais contre le retour desquelles il était bon de se prémunir, certaines personnes relevant encore ces souvenirs avec beaucoup d’étourderie.\par
C’est vous dire qu’en 1866 (je parle ici au nom d’un petit groupe de vrais libéraux) nous accueillîmes avec une grande joie l’augure de la constitution d’une Allemagne à l’état de puissance de premier ordre. Ce n’est pas qu’il nous agréât plus qu’à vous de voir ce grand et heureux événement réalisé par l’armée prussienne. Vous avez montré mieux que personne combien il s’en faut que la Prusse soit l’Allemagne. Mais n’importe ; nous avions à cet égard une pensée que, je pense, vous partagez : c’est que l’unité allemande, après avoir été faite par la Prusse, absorberait la Prusse, conformément à cette loi générale que le levain disparaît dans la pâte qu’il a fait lever. À ce pédantisme rogue et jaloux qui nous déplaît parfois dans la Prusse, nous voyions ainsi se substituer peu à peu et succéder en définitive l’esprit allemand, avec sa merveilleuse largeur, ses poétiques et philosophiques aspirations. Ce qu’il y avait de peu sympathique à nos instincts libéraux dans un pays féodal, très médiocrement parlementaire, dominé par une petite noblesse entichée d’une orthodoxie étroite et pleine de préjugés, nous l’oubliions comme vous l’oubliez vous-même, pour ne voir dans un avenir ultérieur que l’Allemagne, c’est-à-dire une grande nation libérale, destinée à faire faire un pas décisif aux questions politiques, religieuses et sociales, et peut-être à réaliser ce que nous avons essayé en France, jusqu’ici sans y réussir : une organisation scientifique et rationnelle de l’État.\par
Comment ces rêves ont-ils été déçus ? Comment ont-ils fait place à la plus amère réalité ? J’ai expliqué mes idées sur ce point dans la {\itshape Revue} ; les voici en deux mots : On peut faire aussi grande que l’on voudra la part des fautes du gouvernement français, mais il serait injuste d’oublier ce qu’a eu de répréhensible à beaucoup d’égards la conduite du gouvernement prussien. Vous savez que les plans de M. de Bismarck furent communiqués en 1865 à l’empereur Napoléon III, lequel, en somme, y adhéra. Si cette adhésion vint de la conviction que l’unité de l’Allemagne était une nécessité historique, et qu’il était désirable que cette unité se fît avec la pleine amitié de la France, l’empereur Napoléon III eut mille fois raison. Il est à ma connaissance personnelle qu’un mois à peu près avant le commencement des hostilités de 1866, l’empereur Napoléon III croyait au succès de la Prusse, et même qu’il le désirait. Malheureusement, l’hésitation, le goût des actes successivement contradictoires perdirent l’empereur en cette occasion comme en plusieurs autres. La victoire de Sadowa éclata sans que rien fût convenu. Versatilité inconcevable ! Égaré par les rodomontades du parti militaire, troublé par les reproches de l’opposition, l’empereur se laissa entraîner à regarder comme une défaite le résultat qui aurait dû être pour lui une victoire, et qu’en tout cas il avait voulu et amené.\par
Si le succès justifie tout, le gouvernement prussien est complètement absous ; mais nous sommes philosophes, monsieur ; nous avons la naïveté de croire que celui qui a réussi peut avoir eu des torts. Le gouvernement prussien avait sollicité, accepté l’alliance secrète de l’empereur Napoléon III et de la France. Quoique rien n’eût été stipulé, il devait à l’empereur et à la France des marques de gratitude et de sympathie. Un de vos compatriotes, qui montre en ce moment contre la France plus de passion que je n’aime à en voir chez un galant homme, me disait, à l’époque dont il s’agit, que l’Allemagne devait à la France une grande reconnaissance pour la part réelle, quoique négative, que cette dernière avait prise à sa fondation. Conduit par un principe d’orgueil qui aura dans l’avenir de fâcheuses conséquences, le cabinet de Berlin ne l’entendit pas ainsi. Certes les agrandissements territoriaux, quand il s’agit d’une nation forte déjà de trente ou quarante millions d’hommes, ont peu d’importance ; l’acquisition de la Savoie et de Nice a été pour la France plus fâcheuse qu’utile. On peut regretter cependant que le gouvernement prussien n’ait pas fait céder la rigueur de ses prétentions dans l’affaire du Luxembourg. Le Luxembourg cède à la France, la France n’eût pas été plus grande ni l’Allemagne plus petite ; mais cette concession insignifiante eût suffi pour satisfaire l’opinion superficielle, qui en un pays de suffrage universel doit être ménagée, et eût permis au gouvernement français de masquer sa retraite. Dans le plus grand château des croisés qui existe encore en Syrie, le {\itshape Kalaat-el-hosn}, se voit, en beaux caractères du XII\textsuperscript{e} siècle, sur une pierre au milieu des ruines, l’inscription suivante, que la maison de Hohenzollern devrait faire graver sur l’écusson de tous ses châteaux :\par


\begin{verse}
Sit tibi copia,\\
Sit sapientia,\\
Formaque detur ;\\
Inquinat omnia\\
Inquinat omnia\\
Sola superbia\\
Si comitetur.\\
\end{verse}

\noindent Dans les causes éloignées de la guerre, un esprit impartial peut donc faire presque égale la part de reproches que méritent d’un côté le gouvernement de la France et d’un autre côté celui de la Prusse. Quant à la cause prochaine, à ce pitoyable incident diplomatique ou plutôt ce jeu cruel de vanités blessées qui, pour venger de chétives querelles de diplomates, a déchaîné tous les fléaux sur l’espèce humaine, vous savez ce que j’en pense. J’étais à Tromsoë, où le plus splendide paysage de neige des mers polaires me faisait rêver aux îles des Morts de nos ancêtres celtes et germains, quand j’appris cette horrible nouvelle ; je n’ai jamais maudit comme ce jour-là le sort fatal qui semble condamner notre malheureux pays à n’être jamais conduit que par l’ignorance, la présomption et l’ineptie.\par
Cette guerre, quoi qu’on en dise, n’était nullement inévitable. La France ne voulait en aucune façon la guerre. Il ne faut pas juger de ces choses par des déclamations de journaux et des criailleries de boulevard. La France est profondément pacifique ; ses préoccupations sont tournées vers l’exploitation des énormes sources de richesse qu’elle possède et vers les questions démocratiques et sociales. Le roi Louis-Philippe avait vu le vrai sur ce point avec beaucoup de bon sens. Il sentait que la France, avec son éternelle blessure, toujours près de se rouvrir (le manque d’une dynastie ou d’une constitution universellement acceptée), ne pouvait pas faire la Grande Guerre. Une nation qui a rempli son programme et atteint l’égalité ne saurait lutter avec des peuples jeunes, pleins d’illusions et dans tout le feu de leur développement. Croyez-moi, les uniques causes de la guerre sont la faiblesse de nos institutions constitutionnelles et les funestes conseils que des militaires présomptueux et bornés, des diplomates vaniteux ou ignorants ont donnés à l’empereur. Le plébiscite n’y est pour rien ; au contraire, cette étrange manifestation, qui montra que la dynastie napoléonienne avait poussé ses racines jusqu’aux entrailles mêmes du pays, devait faire croire que l’empereur s’éloignerait ensuite de plus en plus des allures d’un joueur désespéré. Un homme qui possède de grands biens territoriaux nous paraît devoir être moins porté à tenter le sort sur un coup de dé que celui dont la richesse est douteuse. En réalité, pour écarter les dangers de conflagration, il suffisait d’attendre. Que de questions, dans les affaires de cette pauvre espèce humaine, il faut résoudre en ne les résolvant pas ! Au bout de quelques années on est tout surpris que la question n’existe plus. Y eut-il jamais une haine nationale comme celle qui pendant six siècles a divisé la France et l’Angleterre ? Il y a vingt-cinq ans, sous Louis-Philippe, cette haine était encore assez forte ; presque tout le monde déclarait qu’elle ne pouvait finir que par la guerre ; elle a disparu comme par enchantement.\par
Naturellement, cher monsieur, les libéraux éclairés n’ont eu ici qu’un seul vœu depuis l’heure fatale : voir finir ce qui n’aurait pas dû commencer. La France a eu mille fois tort de paraître vouloir s’opposer aux évolutions intérieures de l’Allemagne ; mais l’Allemagne commettrait une faute non moins grave en voulant porter atteinte à l’intégrité de la France. Si l’on a pour but de détruire la France, rien de mieux conçu qu’un tel plan ; mutilée, la France rentrerait en convulsions, et périrait. Ceux qui pensent, comme quelques-uns de vos compatriotes, que la France doit être supprimée du nombre des peuples, sont conséquents en demandant son amoindrissement ; ils voient très bien que cet amoindrissement serait sa fin ; mais ceux qui croient comme vous que la France est nécessaire à l’harmonie du monde doivent peser les conséquences qu’entraînerait un démembrement. Je puis parler ici avec une sorte d’impartialité. Je me suis étudié toute ma vie à être bon patriote, ainsi qu’un honnête homme doit l’être, mais en même temps à me garder du patriotisme exagéré comme d’une cause d’erreur. Ma philosophie, d’ailleurs, est l’idéalisme ; ou je vois le bien, le beau, le vrai, là est ma patrie. C’est au nom des vrais intérêts éternels de l’idéal que je serais désolé que la France n’existât plus. La France est nécessaire comme protestation contre le pédantisme, le dogmatisme, le rigorisme étroit. Vous qui avez si bien compris Voltaire devez comprendre cela. Cette légèreté qu’on nous reproche est au fond sérieux et honnête. Prenez garde que, si notre tour d’esprit, avec ses qualités et ses défauts, disparaissait, la conscience humaine serait sûrement amoindrie. La variété est nécessaire, et le premier devoir de l’homme qui cherche d’un cœur vraiment pieux à entrer dans les desseins de la Divinité est de supporter, de respecter même les organes providentiels de la vie spirituelle de l’humanité qui lui sont le moins congénères et le moins sympathiques. Votre illustre Mommsen, dans une lettre qui nous a un peu attristés, comparait il y a quelques jours notre littérature aux eaux bourbeuses de la Seine, et cherchait à en préserver le monde comme d’un poison. Quoi ! cet austère savant connaît donc nos journaux burlesques et notre niais petit théâtre bouffon ! Soyez assuré qu’il y a encore, derrière la littérature charlatanesque et misérable qui a chez nous comme partout les succès de la foule, une France fort distinguée, différente de la France du XVII\textsuperscript{e} et du XVIII\textsuperscript{e} siècle, de même race cependant : d’abord un groupe d’hommes de la plus haute valeur et du sérieux le plus accompli, puis une société exquise, charmante et sérieuse à la fois, fine, tolérante, aimable, sachant tout sans avoir rien appris, devinant d’instinct le dernier résultat de toute philosophie. Prenez garde de froisser cela. La France, pays très mixte, offre cette particularité que certaines plantes germaniques y poussent souvent mieux que dans leur sol natal ; on pourrait le démontrer par des exemples de notre histoire littéraire du XII\textsuperscript{e} siècle, par les chansons de geste, la philosophie scolastique, l’architecture gothique. Vous semblez croire que la diffusion des saines idées germaniques serait facilitée par certaines mesures radicales, détrompez-vous cette propagande serait alors arrêtée net le pays s’enfoncerait avec rage dans ses routines nationales et ses défauts particuliers. — « Tant pis pour lui ! » diront vos exaltés. — « Tant pis pour l’humanité ! » ajouterai-je. La suppression ou l’atrophie d’un membre fait pâtir tout le corps.\par
L’heure est solennelle. Il y a en France deux courants d’opinion. Les uns raisonnent ainsi « Finissons cette odieuse partie au plus vite cédons tout, l’Alsace, la Lorraine ; signons la paix ; puis haine à mort, préparatifs sans trêve, alliance avec n’importe qui, complaisances sans bornes pour toutes les ambitions russes ; un seul but, un seul mobile à la vie, guerre d’extermination contre la race germanique. » D’autres disent : « Sauvons l’intégrité de la France, développons les institutions constitutionnelles, réparons nos fautes, non en rêvant de prendre notre revanche d’une guerre où nous avons été injustes agresseurs, mais en contractant avec l’Allemagne et l’Angleterre une alliance dont l’effet sera de conduire le monde dans les voies de la civilisation libérale. » L’Allemagne décidera laquelle des deux politiques suivra la France, et du même coup elle décidera de l’avenir de la civilisation.\par
Vos germanistes fougueux allèguent que l’Alsace est une terre germanique, injustement détachée de l’empire allemand. Remarquez que les nationalités sont toutes des « cotes mal taillées » ; si l’on se met à raisonner ainsi sur l’ethnographie de chaque canton, on ouvre la porte à des guerres sans fin. De belles provinces de langue française ne font pas partie de la France, et cela est très avantageux, même pour la France. Des pays slaves appartiennent à la Prusse. Ces anomalies servent beaucoup à la civilisation. La réunion de l’Alsace à la France, par exemple, est un des faits qui ont le plus contribué à la propagande du germanisme ; c’est par l’Alsace que les idées, les méthodes, les livres de l’Allemagne passent d’ordinaire pour arriver jusqu’à nous. Il est incontestable que, si on soumettait la question au peuple alsacien, une immense majorité se prononcerait pour rester unie à la France. Est-il digne de l’Allemagne de s’attacher de force une province rebelle, irritée, devenue irréconciliable, surtout depuis la destruction de Strasbourg ? L’esprit est vraiment parfois confondu de l’audace de vos hommes d’État. Le roi de Prusse paraît en train de s’imposer la lourde tâche de résoudre la question française, de donner et par conséquent de garantir un gouvernement à la France. Peut-on, de gaieté de cœur, rechercher un pareil fardeau ? Comment ne voit-on pas que la conséquence de cette politique serait d’occuper la France à perpétuité avec 3 ou 400,000 hommes ? L’Allemagne veut donc rivaliser avec l’Espagne du XVI\textsuperscript{e} siècle ? Et sa grande et haute culture intellectuelle, que deviendrait-elle à ce jeu-là ? Qu’elle prenne garde qu’un jour, quand on voudra désigner les années les plus glorieuses de la race germanique, on ne préfère à la période de sa domination militaire, marquée peut-être par un abaissement intellectuel et moral, les premières années de notre siècle, où, vaincue, humiliée extérieurement, elle créait pour le monde la plus haute révélation de la raison que l’humanité eût connue jusque-là !\par
On s’étonne que quelques-uns de vos meilleurs esprits ne voient pas cela et surtout qu’ils se montrent contraires à une intervention de l’Europe en ces questions. La paix ne peut, à ce qu’il semble, être conclue directement entre la France et l’Allemagne ; elle ne peut être l’ouvrage que de l’Europe, qui a blâmé la guerre et qui doit vouloir qu’aucun des membres de la famille européenne ne soit trop affaibli. Vous parlez à bon droit de garanties contre le retour de rêves malsains ; mais quelle garantie vaudrait celle de l’Europe, consacrant de nouveau les frontières actuelles et interdisant à qui que ce soit de songer à déplacer les bornes fixées par les anciens traités ? Toute autre solution laissera la porte ouverte à des vengeances sans fin. Que l’Europe fasse cela, et elle aura posé pour l’avenir le germe de la plus féconde institution, je veux dire d’une autorité centrale, sorte de congrès des États-Unis d’Europe, jugeant les nations, s’imposant à elles, et corrigeant le principe des nationalités par le principe de fédération. Jusqu’à nos jours, cette force centrale de la communauté européenne ne s’est guère montrée en exercice que dans des coalitions passagères contre le peuple qui aspirait à une domination universelle ; il serait bon qu’une sorte de coalition permanente et préventive se formât pour le maintien des grands intérêts communs, qui sont après tout ceux de la raison et de la civilisation.\par
Le principe de la fédération européenne peut ainsi offrir une base de médiation semblable à celle que l’Église offrait au moyen âge. On est parfois tenté de prêter un rôle analogue aux tendances démocratiques et à l’importance que prennent de nos jours les problèmes sociaux. Le mouvement de l’histoire contemporaine est une sorte de balancement entre les questions patriotiques, d’une part, les questions démocratiques et sociales, de l’autre. Ces derniers problèmes ont un côté de légitimité, et seront peut-être en un sens la grande pacification de l’avenir. Il est certain que le parti démocratique, malgré ses aberrations, agite des problèmes supérieurs à la patrie ; les sectaires de ce parti se donnent la main par-dessus toutes les divisions de nationalité, et professent une grande indifférence pour les questions de point d’honneur, qui touchent surtout la noblesse et les militaires. Les milliers de pauvres gens qui en ce moment s’entre-tuent pour une cause qu’ils ne comprennent qu’à demi ne se haïssent pas ; ils ont des besoins, des intérêts communs. Qu’un jour ils arrivent à s’entendre et à se donner la main malgré leurs chefs, c’est là un rêve sans doute ; on peut cependant entrevoir plus d’un biais par où la politique à outrance de la Prusse pourra servir à l’avènement d’idées qu’elle ne soupçonne pas. Il paraît difficile que cette fureur d’une poignée d’hommes, reste des vieilles aristocraties, mène longtemps à l’égorgement des masses de populations douces, arrivées à une conscience démocratique assez avancée et plus ou moins imbues d’idées économiques (pour eux saintes) dont le propre est justement de ne pas tenir compte des rivalités nationales.\par
Ah ! cher maître, que Jésus a bien fait de fonder le royaume de Dieu, un monde supérieur à la haine, à la jalousie, à l’orgueil, où le plus estimé est, non pas, comme dans les tristes temps que nous traversons, celui qui fait le plus de mal, celui qui frappe, tue, insulte, celui qui est le plus menteur, le plus déloyal, le plus mal élevé, le plus défiant, le plus perfide, le plus fécond en mauvais procédés, en idées diaboliques, le plus ferme à la pitié, au pardon, celui qui n’a nulle politesse, qui surprend son adversaire, lui joue les plus mauvais tours ; mais celui qui est le plus doux, le plus modeste, le plus éloigné de toute assurance, jactance et dureté, celui qui cède le pas à tout le monde, celui qui se regarde comme le dernier ! La guerre est un tissu de péchés, un état contre nature où l’on recommande de faire comme belle action ce qu’en tout autre temps on commande d’éviter comme vice ou défaut, où c’est un devoir de se réjouir du malheur d’autrui, où celui qui rendrait le bien pour le mal, qui pratiquerait les préceptes évangéliques de pardon des injures, de goût pour l’humiliation, serait absurde et même blâmable. Ce qui fait entrer dans la Walhalla est ce qui exclut du royaume de Dieu. Avez-vous remarqué que ni dans les huit béatitudes, ni dans le sermon sur la montagne, ni dans l’Évangile, ni dans toute la littérature chrétienne primitive, il n’y a pas un mot qui mette les vertus militaires parmi celles qui gagnent le royaume du ciel ?\par
Insistons sur ces grands enseignements de paix, qui échappent aux hommes dupes de leur orgueil, entraînés par leur éternel et si peu philosophique oubli de la mort. Personne n’a le droit de se désintéresser des désastres de son pays ; mais le philosophe comme le chrétien a toujours des motifs de vivre. Le royaume de Dieu ne connaît ni vainqueurs ni vaincus ; il consiste dans les joies du cœur, de l’esprit et de l’imagination, que le vaincu goûte plus que le vainqueur, s’il est plus élevé moralement et s’il a plus d’esprit. Votre grand Goethe, votre admirable Fichte ne nous ont-ils pas appris comment on peut mener une vie noble et par conséquent heureuse au milieu de l’abaissement extérieur de sa patrie ? Un motif, du reste, m’inspire un grand repos d’esprit : l’an dernier, lors des élections pour le Corps législatif, je m’offris aux suffrages des électeurs ; je ne fus pas choisi ; mes affiches se voient encore sur les murs des villages de Seine-et-Marne ; on y peut lire : « Pas de révolution, pas de guerre. Une guerre serait aussi funeste qu’une révolution. » Pour avoir la conscience tranquille dans des temps comme les nôtres, il faut pouvoir se dire qu’on n’a pas fui systématiquement la vie publique, pas plus qu’on ne l’a recherchée.\par
Conservez-moi toujours votre amitié, et croyez à mes sentiments les plus élevés.
\section[{Nouvelle lettre adressée à M. Strauss}]{Nouvelle lettre adressée à M. Strauss}\renewcommand{\leftmark}{Nouvelle lettre adressée à M. Strauss}


\dateline{Le 15 septembre 1871}
\noindent Monsieur et savant maître,\par
À la fin de la lettre que vous m’avez adressée par la {\itshape Gazette d’Augsbourg}, le 18 août 1870, vous m’invitiez à exposer mes vues sur la situation terrible créée par les derniers événements. Je le fis ; ma réponse à votre lettre parut dans le {\itshape Journal des Débats}, le 16 septembre ; la veille, avait été insérée dans le même journal la traduction de votre lettre, telle que nous l’avait envoyée votre excellent interprète français, M. Charles Ritter. Si vous voulez bien réfléchir à l’état de Paris à cette époque, vous reconnaîtrez peut-être que ce journal faisait en cela preuve d’un certain courage. Le siège commença le lendemain, et toute communication entre l’intérieur de Paris et le reste du monde se trouva interrompue pendant cinq mois.\par
Plusieurs jours après la conclusion de l’armistice au mois de février 1871, j’appris une nouvelle qui me surprit, c’est que, le 2 octobre 1870, vous aviez fait dans la {\itshape Gazette d’Augsbourg} une réponse à ma lettre du 16 septembre. Vous ne pensiez pas sans doute que le blocus prussien fût aussi rigoureux qu’il l’était ; car, si vous l’aviez su, il est peu probable que vous m’eussiez adressé une lettre publique que je ne pouvais lire et à laquelle je ne pouvais répondre. Le malentendu en ces matières délicates est facile ; il faut que la personne qu’on a interpellée puisse donner des explications et rectifier, s’il y a lieu, les opinions qu’on lui prête. Dans le cas dont il s’agit, la crainte d’un malentendu n’était pas chimérique. Entre bien des rectifications, en effet, que j’aurais à faire à votre réponse du 2 octobre, il en est une qui a de l’importance. Trompé par l’expression de « traités de 1814 » que nous employons souvent en France pour désigner l’ensemble des conventions qui fixèrent les limites de la France à la chute du Premier Empire, vous avez cru que je demandais après Sedan qu’on revînt sur les cessions de 1815, qu’on nous rendît Saarlouis et Landau. Je suis fâché d’avoir été présenté par vous au public allemand comme capable d’une telle absurdité. Il me semble que, s’il y a une pensée qui résulte clairement de ce que j’ai écrit sur cette funeste guerre, c’est qu’il fallait s’en tenir aux frontières nationales telles que l’histoire les avait fixées, que toute annexion de pays sans le vœu des populations était une faute et même un crime.\par
Une circonstance augmenta encore mon chagrin. Peu de jours après que j’eus connu l’existence de votre lettre du 2 octobre, j’appris que la {\itshape Gazette d’Augsbourg} n’avait pas inséré la traduction de ma lettre du 16 septembre, si bien que ce journal, après m’avoir invité par votre organe à entrer dans la discussion, après avoir vu le {\itshape Journal des Débats}, dont la position était autrement délicate que la sienne, insérer vos pages hautaines sous le coup de l’émeute populaire, refusait de porter au public allemand victorieux les humbles pages ou je réclamais pour ma patrie vaincue un peu de générosité et de pitié. Je sais que vous avez regretté ce procédé ; mais c’est ici que j’admire de quoi est capable votre patriotisme exalté ; car, au lieu de vous retirer d’un débat où la parole était refusée à votre adversaire, vous avez inséré quelques jours après dans cette même {\itshape Gazette d’Augsbourg} une réplique à la lettre que vous m’aviez fait écrire et que vous n’aviez pas eu le crédit de faire publier. Voilà, monsieur, où je vois bien la différence entre nos manières de comprendre la vie. La passion qui vous remplit et qui vous semble sainte est capable de vous arracher un acte pénible. Une de nos faiblesses, au contraire, à nous autres Français de la vieille école, est de croire que les délicatesses du galant homme passent avant tout devoir, avant toute passion, avant toute croyance, avant la patrie, avant la religion. Cela nous fait du tort ; car on ne nous rend pas toujours la pareille, et, comme tous les délicats, nous jouons le rôle de dupes au milieu d’un monde qui ne nous comprend plus.\par
Il est vrai que vous m’avez fait ensuite un honneur auquel je suis sensible comme je le dois. Vous avez traduit vous-même ma réponse et l’avez réunie dans une brochure à vos deux lettres.\footnote{Leipzig, Hirzel, 1870.} Vous avez voulu que cette brochure se vendît au profit d’un établissement d’invalides allemands. Dieu me garde de vous faire une chicane au point de vue de la propriété littéraire ! L’œuvre à laquelle vous m’avez fait contribuer est d’ailleurs une œuvre d’humanité, et, si ma chétive prose a pu procurer quelques cigares à ceux qui ont pillé ma petite maison de Sèvres, je vous remercie de m’avoir fourni l’occasion de conformer ma conduite à quelques-uns des préceptes de Jésus que je crois les plus authentiques. Mais remarquez encore ces nuances légères. Certainement, si vous m’aviez permis de publier un écrit de vous, jamais, au grand jamais, je n’aurais eu l’idée d’en faire une édition au profit de notre hôtel des Invalides. Le but vous entraîne ; la passion vous empêche de voir ces mièvreries de gens blasés que nous appelons le goût et le tact.\par
Il m’est arrivé depuis un an ce qui arrive toujours à ceux qui prêchent la modération en temps de crise. Les événements ainsi que l’immense majorité de l’opinion m’ont donné tort. Je ne puis vous dire cependant que je sois converti. Attendons dix ou quinze années ; ma conviction est que la partie éclairée de l’Allemagne reconnaîtra alors qu’en lui conseillant d’user doucement de sa victoire, je fus son meilleur ami. Je ne crois pas à la durée des choses menées à l’extrême, et je serais bien surpris si une foi aussi absolue en la vertu d’une race que celle que professent M. de Bismarck et M. de Moltke n’aboutissait pas à une déconvenue. L’Allemagne, en se livrant aux hommes d’État et aux hommes de guerre de la Prusse, a monté un cheval fringant, qui la mènera où elle ne veut pas. Vous jouez trop gros jeu. À quoi ressemble votre conduite ? Exactement à celle de la France à l’époque qu’on lui reproche le plus. En 1792, les puissances européennes provoquent la France ; la France bat les puissances, ce qui était bien son droit ; puis elle pousse ses victoires à outrance, en quoi elle avait tort. L’outrance est mauvaise ; l’orgueil est le seul vice qui soit puni en ce monde. Triompher est toujours une faute et en tout cas quelque chose de bien peu philosophique. {\itshape Debemur morti nos nostraque.}\par
Ne vous imaginez pas être plus que d’autres à l’abri de l’erreur. Depuis un an, vos journaux se sont montrés moins ignorants sans doute que les nôtres, mais tout aussi passionnés, tout aussi immoraux, tout aussi aveugles. Ils ne voient pas une montagne qui est devant leurs yeux, l’opposition toujours croissante de la conscience slave à la conscience germanique, opposition qui aboutira à une lutte effroyable. Ils ne voient pas qu’en détruisant le pôle nord d’une pile on détruit le pôle sud, que la solidarité française faisait la solidarité allemande, qu’en mourant la France se vengera et rendra le plus mauvais service à l’Allemagne. L’Allemagne, en d’autres termes, a fait la faute d’écraser son adversaire. Qui n’a pas d’antithèse n’a pas de raison d’être. S’il n’y avait plus d’orthodoxes, ni vous ni moi n’existerions ; nous serions en face d’un stupide matérialisme vulgaire, qui nous tuerait bien mieux que les théologiens. L’Allemagne s’est comportée avec la France comme si elle ne devait jamais avoir d’autre ennemi. Or le précepte du vieux sage {\itshape Ama tanquam osurus} doit aujourd’hui être retourné ; il faut haïr comme si l’on devait un jour être l’allié de celui qu’on hait ; on ne sait pas de qui on devra quelque jour rechercher l’amitié.\par
Il ne sert de rien de dire qu’il y a soixante et soixante-dix ans nous avons agi exactement de la même manière, qu’alors nous avons fait en Europe la guerre de pillage, de massacre et de conquête que nous reprochons aux Allemands de 1870. Ces méfaits du Premier Empire, nous les avons toujours blâmés ; ils sont l’œuvre d’une génération avec laquelle nous avons peu de chose de commun et dont la gloire n’est plus la nôtre. À tort évidemment, nous nous étions habitués à croire que le XIX\textsuperscript{e} siècle avait inauguré une ère de civilisation, de paix, d’industrie, de souveraineté des populations. « Comment, dit-on, traitez-vous de crimes et de hontes des cessions d’âmes auxquelles ont autrefois consenti des races aussi nobles que la vôtre et dont vous-mêmes avez profité ? » — Distinguons les dates. Le droit d’autrefois n’est pas le droit d’aujourd’hui. Le sentiment des nationalités n’a pas cent ans. Frédéric II n’était pas plus mauvais Allemand dans son dédain pour la langue et la littérature allemandes que Voltaire n’était mauvais Français en se réjouissant de l’issue de la bataille de Rosbach. Une cession de province n’était alors qu’une translation de biens immeubles d’un prince à un prince ; les peuples y restaient le plus souvent indifférents. Cette conscience des peuples, nous l’avons créée dans le monde par notre révolution ; nous l’avons donnée à ceux que nous avons combattus et souvent injustement combattus ; elle est notre dogme. Voilà pourquoi nous autres libéraux français étions pour les Vénitiens, pour les Milanais contre l’Autriche ; pour la Bohême, pour la Hongrie contre la centralisation viennoise ; pour la Pologne contre la Russie ; pour les Grecs et les Slaves de Turquie contre les Turcs. Il y avait protestation de la part de Milan, de Venise, de la Bohême, de la Hongrie, de la Pologne, des Grecs et des Slaves de Turquie, cela nous suffisait. Nous étions également pour les Romagnols contre le pape ou plutôt contre la contrainte étrangère qui les maintenait malgré eux sujets du pape ; car nous ne pouvions admettre qu’une population soit confisquée contre son gré au profit d’une idée religieuse qui prétend qu’elle a besoin d’un territoire pour vivre. Dans la guerre de la sécession d’Amérique beaucoup de bons esprits, tout en étant peu sympathiques aux États du Sud, ne purent se décider à leur dénier le droit de se retirer d’une association dont ils ne voulaient plus faire partie, du moment qu’ils eurent prouvé par de rudes sacrifices que leur volonté à cet égard était sérieuse.\par
Cette règle de politique n’a rien de profond ni de transcendant ; mais il faut se garder, à force d’érudition et de métaphysique, de n’être plus juste ni humain. La guerre sera sans fin, si l’on n’admet des prescriptions pour les violences du passé. La Lorraine a fait partie de l’empire germanique, sans aucun doute ; mais la Hollande, la Suisse, l’Italie même, jusqu’à Bénévent, et en remontant au-delà du traité de Verdun, la France entière, en y comprenant même la Catalogne, en ont aussi fait partie. — L’Alsace est maintenant un pays germanique de langue et de race ; mais, avant d’être envahie par la race germanique, l’Alsace était un pays celtique, ainsi qu’une partie de l’Allemagne du Sud. Nous ne concluons pas de là que l’Allemagne du Sud doive être française ; mais qu’on ne vienne pas non plus soutenir que, par droit ancien, Metz et Luxembourg doivent être allemands. Nul ne peut dire où cette archéologie s’arrêterait. Presque partout où les patriotes fougueux de l’Allemagne réclament un droit germanique, nous pourrions réclamer un droit celtique antérieur, et avant la période celtique, il y avait, dit-on, les allophyles, les Finnois, les Lapons ; et avant les Lapons il y eut les hommes des cavernes ; et avant les hommes des cavernes, il y eut les orangs-outangs. Avec cette philosophie de l’histoire, il n’y aura de légitime dans le monde que le droit des orangs-outangs, injustement dépossédés par la perfidie des civilisés.\par
Soyons moins absolus ; à côté du droit des morts, admettons pour une petite part le droit des vivants. Le traité de 843, pacte conclu entre trois chefs barbares qui assurément ne se préoccupèrent dans le partage que de leurs convenances personnelles, ne saurait être une base éternelle de droit national. Le mariage de Marie de Bourgogne avec Maximilien ne saurait s’imposer à jamais à la volonté des peuples. Il est impossible d’admettre que l’humanité soit liée pour des siècles indéfinis par les mariages, les batailles, les traités des créatures bornées, ignorantes, égoïstes, qui au moyen âge tenaient la tête des affaires de ce bas monde. Ceux de vos historiens, comme Ranke, Sybel, qui ne voient dans l’histoire que le tableau des ambitions princières et des intrigues diplomatiques, pour lesquels une province se résume en la dynastie, souvent étrangère, qui l’a possédée, sont aussi peu philosophes que la naïve école qui veut que la Révolution française ait marqué une ère absolument nouvelle dans l’histoire. Un moyen terme entre ces extrêmes nous paraît seul pratique. Certes nous repoussons comme une erreur de fait fondamentale l’égalité des individus humains et l’égalité des races ; les parties élevées de l’humanité doivent dominer les parties basses ; la société humaine est un édifice à plusieurs étages, où doit régner la douceur, la bonté (l’homme y est tenu même envers les animaux), non l’égalité. Mais les nations européennes telles que les a faites l’histoire sont les pairs d’un grand sénat où chaque membre est inviolable. L’Europe est une confédération d’États réunis par l’idée commune de la civilisation. L’individualité de chaque nation est constituée sans doute par la race, la langue, l’histoire, la religion, mais aussi par quelque chose de beaucoup plus tangible, par le consentement actuel, par la volonté qu’ont les différentes provinces d’un État de vivre ensemble. Avant la malheureuse annexion de Nice, pas un canton de la France ne voulait se séparer de la France ; cela suffisait pour qu’il y eût crime européen à démembrer la France, quoique la France ne soit une ni de langue ni de race. Au contraire, des parties de la Belgique et de la Suisse, et jusqu’à un certain point les îles de la Manche, quoique parlant français, ne désirent nullement appartenir à la France ; cela suffit pour qu’il fût criminel de chercher à les y annexer par la force. L’Alsace est allemande de langue et de race ; mais elle ne désire pas faire partie de l’État allemand ; cela tranche la question. On parle du droit de la France, du droit de l’Allemagne. Ces abstractions nous touchent beaucoup moins que le droit qu’ont les Alsaciens, êtres vivants en chair et en os, de n’obéir qu’à un pouvoir consenti par eux.\par
Ne blâmez donc pas notre école libérale française de regarder comme une sorte de droit divin le droit qu’ont les populations de n’être pas transférées sans leur consentement. Pour ceux qui comme nous n’admettent plus le principe dynastique qui fait consister l’unité d’un État dans les droits personnels du souverain, il n’y a plus d’autre droit des gens que celui-là. De même qu’une nation légitimiste se fait hacher pour sa dynastie, de même nous sommes obligés de faire les derniers sacrifices pour que ceux qui étaient liés à nous par un pacte de vie et de mort ne souffrent pas violence. Nous n’admettons pas les cessions d’âmes ; si les territoires à céder étaient déserts, rien de mieux ; mais les hommes qui les habitent sont des créatures libres, et notre devoir est de les faire respecter.\par
Notre politique, c’est la politique du droit des nations ; la vôtre, c’est la politique des races : nous croyons que la nôtre vaut mieux. La division trop accusée de l’humanité en races, outre qu’elle repose sur une erreur scientifique, très peu de pays possédant une race, vraiment pure, ne peut mener qu’à des guerres d’extermination, à des guerres « zoologiques », permettez-moi de le dire, analogues à celles que les diverses espèces de rongeurs ou de carnassiers se livrent pour la vie. Ce serait la fin de ce mélange fécond, composé d’éléments nombreux et tous nécessaires, qui s’appelle l’humanité. Vous avez levé dans le monde le drapeau de la politique ethnographique et archéologique en place de la politique libérale ; cette politique vous sera fatale. La philologie comparée, que vous avez créée et que vous avez transportée à tort sur le terrain de la politique, vous jouera de mauvais tours. Les Slaves s’y passionnent ; chaque maître d’école slave est pour vous un ennemi, un termite qui ruine votre maison. Comment pouvez-vous croire que les Slaves ne vous feront pas ce que vous faites aux autres, eux qui en toute chose marchent après vous, suivent vos traces pas pour pas ? Chaque affirmation du germanisme est une affirmation du slavisme ; chaque mouvement de concentration de votre part est un mouvement qui « précipite » le Slave, le dégage, le fait être séparément. Un coup d’œil sur les affaires d’Autriche montre cela avec évidence. Le Slave, dans cinquante ans, saura que c’est vous qui avez fait son nom synonyme d’« esclave » ; il verra cette longue exploitation historique de sa race par la vôtre, et le nombre des Slaves est double du vôtre, et le Slave, comme le dragon de l’Apocalypse, dont la queue balaye la troisième partie des étoiles, traînera un jour après lui le troupeau de l’Asie centrale, l’ancienne clientèle des Gengis Khan et des Tamerlan. Combien il eût mieux valu vous réserver pour ce jour-là l’appel à la raison, à la moralité, à des amitiés de principes ! Songez quel poids pèsera dans la balance du monde le jour où la Bohême, la Moravie, la Croatie, la Serbie, toutes les populations Slaves de l’Empire ottoman, sûrement destinées à l’affranchissement, races héroïques encore, toutes militaires et qui n’ont besoin que d’être commandées, se grouperont autour de ce grand conglomérat moscovite, qui englobe déjà dans une gangue slave tant d’éléments divers, et qui paraît bien le noyau désigné de la future unité slave, de même que la Macédoine, à peine grecque, le Piémont, à peine italien, la Prusse, à peine allemande, ont été le centre de formation de l’unité grecque, de l’unité italienne, de l’unité allemande. Et vous êtes trop sages pour compter sur la reconnaissance que vous doit la Russie. Une des causes secrètes de la mauvaise humeur de la Prusse contre nous est de nous devoir une partie de sa culture. Une des blessures des Russes sera un jour d’avoir été civilisés par les Allemands. Ils le nieront, mais se l’avoueront tout en le niant, et ce souvenir les exaspérera. L’académie de Saint-Pétersbourg en voudra un jour autant à celle de Berlin, pour avoir été tout allemande, que celle de Berlin nous en veut, pour avoir été autrefois à moitié française. Notre siècle est le siècle du triomphe du serf sur son maître : le Slave a été, et à quelques égards est encore, votre serf.\par
Or, le jour de la conquête slave, nous vaudrons plus que vous, de même qu’Athènes sous l’Empire romain eut un rôle brillant encore, tandis que Sparte n’en eut plus.\par
Défiez-vous donc de l’ethnographie, ou plutôt ne l’appliquez pas trop à la politique. Sous prétexte d’une étymologie germanique, vous prenez pour la Prusse tel village de Lorraine. Les noms de Vienne ({\itshape Vindobona}), de Worms ({\itshape Borbitomagus}), de Mayence ({\itshape Mogontiacum}) sont gaulois ; nous ne vous réclamerons jamais ces villes ; mais, si un jour les Slaves viennent revendiquer la Prusse proprement dite, la Poméranie, la Silésie, Berlin, par la raison que tous ces noms sont slaves, s’ils font sur l’Elbe et sur l’Oder ce que vous avez fait sur la Moselle, s’ils pointent sur la carte les villages obotrites ou vélatabes, qu’aurez-vous à dire ? Nation n’est pas synonyme de race. La petite Suisse, si solidement bâtie, compte trois langues, trois ou quatre races, deux religions. Une nation est une grande association séculaire (non pas éternelle) entre des provinces en partie congénères formant noyau, et autour desquelles se groupent d’autres provinces liées les unes aux autres par des intérêts communs ou par d’anciens faits acceptés et devenus des intérêts. L’Angleterre, qui est la plus parfaite des nations, est la plus mêlée, au point de vue de l’ethnographie et de l’histoire. Bretons purs, Bretons romanisés, Irlandais, Calédoniens, Anglo-Saxons, Danois, Normands purs, Normands francisés, tout s’y est confondu.\par
Et j’ose dire qu’aucune nation n’aura tant à souffrir de cette fausse manière de raisonner que l’Allemagne. Vous savez mieux que moi que ce qui marqua le grand règne de la race germanique dans le monde, du V\textsuperscript{e} au XI\textsuperscript{e} siècle, ce fut moins d’occuper à l’état de population compacte de vastes pays contigus que d’essaimer l’Europe et d’y introduire un nouveau principe d’autorité. Pendant que le germanisme était maître de tout l’Occident, la Germanie proprement dite avait peu de corps. Les Slaves venaient jusqu’à l’Elbe, le vieux fond gaulois persistait ; si bien que l’empire germanique n’était en partie qu’une féodalité germanique régnant sur un fond slave et gaulois. Prenez garde, en ce siècle de la résurrection des morts, il pourrait se passer d’étranges choses. Si l’Allemagne s’abandonne à un sentiment trop exclusivement national, elle verra se rétrécir d’autant la zone de son rayonnement moral. La Bohême, qui était à demi digérée par le germanisme, vous échappe, comme une proie déjà avalée par un serpent boa, qui ressusciterait dans l’œsophage du monstre et ferait des efforts désespérés pour en sortir. Je veux croire que la conscience slave est morte en Silésie ; mais vous n’assimilerez pas Posen. Ces opérations veulent être enlevées d’emblée, pendant que le patient dort ; s’il vient à se réveiller, on ne les reprend plus. Une suspicion universelle contre votre puissance d’assimilation, contre vos écoles, va se répandre. Un vaste effort pour écarter vos nationaux, que l’on envisagera comme les avant-coureurs de vos armées, sera pour longtemps à l’ordre du jour. L’infiltration silencieuse de vos émigrants dans les grandes villes, qui était devenue un des faits sociaux les plus importants et les plus bienfaisants de notre siècle, va être bien diminuée. L’Allemand, ayant dévoilé ses appétits conquérants, ne s’avancera plus qu’en conquérant. Sous l’extérieur le plus pacifique, on verra un ennemi cherchant à s’impatroniser chez autrui. Croyez-moi, ce que vous avez perdu est faiblement compensé par les cinq milliards que vous avez gagnés.\par
Chacun doit se défier de ce qu’il y a d’exclusif et d’absolu dans son esprit. Ne nous imaginons jamais avoir tellement raison que nos adversaires aient complètement tort. Le Père céleste fait lever son soleil avec une bienveillance égale sur les spectacles les plus divers. Ce que nous croyons mauvais est souvent utile et nécessaire. Pour moi, je m’irriterais d’un monde où tous mèneraient le même genre de vie que moi. Comme vous, je me suis imposé, en qualité d’ancien clerc, d’observer strictement la règle des mœurs ; mais je serais désolé qu’il n’y eût pas des gens du monde pour représenter une vie plus libre. Je ne suis pas riche ; mais je ne pourrais guère vivre dans une société où il n’y aurait pas de gens riches. Je ne suis pas catholique ; mais je suis bien aise qu’il y ait des catholiques, des sœurs de charité, des curés de campagne, des carmélites, et il dépendrait de moi de supprimer tout cela que je ne le ferais pas. De même, vous autres Allemands, supportez ce qui ne vous ressemble pas ; si tout le monde était fait à votre image, le monde serait peut-être un peu morne et ennuyeux ; vos femmes elles-mêmes supportent avec peine cette austérité trop virile. Cet univers est un spectacle qu’un dieu se donne à lui-même. Servons les intentions du grand chorège en contribuant à rendre le spectacle aussi brillant, aussi varié que possible.\par
Votre race germanique a toujours l’air de croire à la Walhalla ; mais la Walhalla ne sera jamais le royaume de Dieu. Avec cet éclat militaire, l’Allemagne risque de manquer sa vraie vocation. Reprenons tous ensemble les grands et vrais problèmes, les problèmes sociaux, qui se résument ainsi : trouver une organisation rationnelle et aussi juste que possible de l’humanité. Ces problèmes ont été posés par la France en 1789 et en 1848 ; mais en général celui qui pose les problèmes n’est pas celui qui les résout. La France les attaqua d’une façon trop simple ; elle crut avoir trouvé une issue par la démocratie pure, par le suffrage universel et par des rêves d’organisation communiste du travail. Les deux tentatives ont échoué, et ce double échec a été la cause de réactions fâcheuses, pour lesquelles il convient d’être indulgent, si l’on songe que l’initiative en pareille matière a bien quelque mérite. Attaquez à votre tour ces problèmes. Créez à l’homme en dehors de l’État et par delà la famille une association qui l’élève, le soutienne, le corrige, l’assiste, le rende heureux, ce que fut l’Église et ce qu’elle n’est plus. Réformez l’Église, ou substituez-y quelque chose. L’excès du patriotisme nuit à ces œuvres universelles dont la base est le mot de saint Paul : {\itshape Non est Judœus neque Grœcus}. C’est justement parce que vos grands hommes d’il y a quatre-vingts ans n’étaient pas trop patriotes qu’ils ouvrirent cette large voie, où nous sommes leurs disciples. Je crains que votre génération ultra-patriotique, en repoussant tout ce qui n’est pas germanique pur, ne se prépare un auditoire beaucoup plus restreint. Jésus et les fondateurs du christianisme n’étaient pas des Allemands. Saint Boniface, les Irlandais qui vous ont appris à écrire du temps des Carlovingiens, les Italiens, qui ont été deux ou trois fois nos maîtres à tous, n’étaient pas des Allemands. Votre Goethe reconnaissait devoir quelque chose à cette France « corrompue » de Voltaire, de Diderot. Laissons ces fanatismes étroits aux régions inférieures de l’opinion. Permettez-moi de vous le dire : vous avez déchu. Vous avez été plus étroitement patriotes que nous. Chez nous, quelques hommes supérieurs ont trouvé dans leur philosophie le calme et l’impartialité ; chez vous, je ne connais personne, en dehors du parti démocratique, qui n’ait été ébranlé dans la froideur de ses jugements, qui n’ait été une fois injuste, qui n’ait recommandé de faire dans l’ordre des relations nationales ce qui eût été une honte selon les principes de la morale privée.\par
Mais je m’arrête ; on est aujourd’hui trop naïf à parler de modération, de justice, de fraternité, de la reconnaissance et des égards que les peuples se doivent entre eux. La conduite que vous allez être forcés de tenir dans les provinces annexées malgré elles achèvera de vous démoraliser. Vous allez être obligés de donner un démenti à tous vos principes, de traiter en criminels des hommes que vous devrez estimer, des hommes qui n’auront fait autre chose que ce que vous fîtes si noblement après Iéna ; toutes les idées morales vont être perverties. Notre système d’équilibre et d’amphictyonie européenne va être renvoyé au pays des chimères ; nos thèses libérales vont devenir un jargon vieilli. Par le fait des hommes d’État prussiens, la France d’ici longtemps n’aura plus qu’un objectif : reconquérir les provinces perdues. Attiser la haine toujours croissante des Slaves contre les Allemands, favoriser le panslavisme, servir sans réserve toutes les ambitions russes, faire miroiter aux yeux du parti catholique répandu partout le rétablissement du pape à Rome ; à l’intérieur, s’abandonner au parti légitimiste et clérical de l’Ouest, qui seul possède un fanatisme intense, voilà la politique que commande une telle situation. C’est justement l’inverse de ce que nous avions rêvé. On ne sert pas tour à tour deux causes opposées ; ce n’est pas nous qui conseillerons la destruction de ce que nous avons aimé, qui donnerons un plan pour trafiquer habilement de la question romaine, qui deviendrons russes et papistes, qui recommanderons la défiance et la malveillance envers les étrangers ; mais que voulez-vous ! nous serions coupables, d’un autre côté, si nous cherchions, en conseillant encore des poursuites généreuses et désintéressées, à empêcher le pays d’écouter la voix de deux millions de Français qui réclament l’aide de leur ancienne patrie.\par
La France est en train de dire comme votre Herwegh : « Assez d’amour comme cela ; essayons maintenant de la haine. » Je ne la suivrai pas dans cette expérience nouvelle, où l’on peut, au reste, douter qu’elle réussisse ; la résolution que la France tient le moins est celle de haïr. En tout cas, la vie est trop courte pour qu’il soit sage de perdre son temps et d’user sa force à un jeu si misérable. J’ai travaillé dans mon humble sphère à l’amitié de la France et de l’Allemagne ; si c’est maintenant « le temps de cesser les baisers », comme dit l’Ecclésiaste, je me retire. Je ne conseillerai pas la haine, après avoir conseillé l’amour ; je me tairai. Âpre et orgueilleuse est cette vertu germanique, qui nous punit, comme Prométhée, de nos téméraires essais, de notre folle « philanthropie ». Mais nous pouvons dire avec le grand vaincu : « Jupiter, malgré tout son orgueil, ferait bien d’être humble. Maintenant, puisqu’il est vainqueur, qu’il trône à son aise, se fiant au bruit de son tonnerre et secouant dans sa main son dard au souffle de feu. Tout cela ne le préservera pas un jour de tomber ignominieusement d’une chute horrible. Je le vois se créer lui-même son ennemi, monstre très difficile à combattre, qui trouvera une flamme supérieure à la foudre, un bruit supérieur au tonnerre. Vaincu alors, il comprendra par son expérience combien il est différent de régner ou de servir. »\par
Croyez, monsieur et illustre maître, à mes sentiments les plus élevés.
 


% at least one empty page at end (for booklet couv)
\ifbooklet
  \pagestyle{empty}
  \clearpage
  % 2 empty pages maybe needed for 4e cover
  \ifnum\modulo{\value{page}}{4}=0 \hbox{}\newpage\hbox{}\newpage\fi
  \ifnum\modulo{\value{page}}{4}=1 \hbox{}\newpage\hbox{}\newpage\fi


  \hbox{}\newpage
  \ifodd\value{page}\hbox{}\newpage\fi
  {\centering\color{rubric}\bfseries\noindent\large
    Hurlus ? Qu’est-ce.\par
    \bigskip
  }
  \noindent Des bouquinistes électroniques, pour du texte libre à participation libre,
  téléchargeable gratuitement sur \href{https://hurlus.fr}{\dotuline{hurlus.fr}}.\par
  \bigskip
  \noindent Cette brochure a été produite par des éditeurs bénévoles.
  Elle n’est pas faîte pour être possédée, mais pour être lue, et puis donnée.
  Que circule le texte !
  En page de garde, on peut ajouter une date, un lieu, un nom ; pour suivre le voyage des idées.
  \par

  Ce texte a été choisi parce qu’une personne l’a aimé,
  ou haï, elle a en tous cas pensé qu’il partipait à la formation de notre présent ;
  sans le souci de plaire, vendre, ou militer pour une cause.
  \par

  L’édition électronique est soigneuse, tant sur la technique
  que sur l’établissement du texte ; mais sans aucune prétention scolaire, au contraire.
  Le but est de s’adresser à tous, sans distinction de science ou de diplôme.
  Au plus direct ! (possible)
  \par

  Cet exemplaire en papier a été tiré sur une imprimante personnelle
   ou une photocopieuse. Tout le monde peut le faire.
  Il suffit de
  télécharger un fichier sur \href{https://hurlus.fr}{\dotuline{hurlus.fr}},
  d’imprimer, et agrafer ; puis de lire et donner.\par

  \bigskip

  \noindent PS : Les hurlus furent aussi des rebelles protestants qui cassaient les statues dans les églises catholiques. En 1566 démarra la révolte des gueux dans le pays de Lille. L’insurrection enflamma la région jusqu’à Anvers où les gueux de mer bloquèrent les bateaux espagnols.
  Ce fut une rare guerre de libération dont naquit un pays toujours libre : les Pays-Bas.
  En plat pays francophone, par contre, restèrent des bandes de huguenots, les hurlus, progressivement réprimés par la très catholique Espagne.
  Cette mémoire d’une défaite est éteinte, rallumons-la. Sortons les livres du culte universitaire, cherchons les idoles de l’époque, pour les briser.
\fi

\ifdev % autotext in dev mode
\fontname\font — \textsc{Les règles du jeu}\par
(\hyperref[utopie]{\underline{Lien}})\par
\noindent \initialiv{A}{lors là}\blindtext\par
\noindent \initialiv{À}{ la bonheur des dames}\blindtext\par
\noindent \initialiv{É}{tonnez-le}\blindtext\par
\noindent \initialiv{Q}{ualitativement}\blindtext\par
\noindent \initialiv{V}{aloriser}\blindtext\par
\Blindtext
\phantomsection
\label{utopie}
\Blinddocument
\fi
\end{document}
