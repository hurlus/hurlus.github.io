%%%%%%%%%%%%%%%%%%%%%%%%%%%%%%%%%
% LaTeX model https://hurlus.fr %
%%%%%%%%%%%%%%%%%%%%%%%%%%%%%%%%%

% Needed before document class
\RequirePackage{pdftexcmds} % needed for tests expressions
\RequirePackage{fix-cm} % correct units

% Define mode
\def\mode{a4}

\newif\ifaiv % a4
\newif\ifav % a5
\newif\ifbooklet % booklet
\newif\ifcover % cover for booklet

\ifnum \strcmp{\mode}{cover}=0
  \covertrue
\else\ifnum \strcmp{\mode}{booklet}=0
  \booklettrue
\else\ifnum \strcmp{\mode}{a5}=0
  \avtrue
\else
  \aivtrue
\fi\fi\fi

\ifbooklet % do not enclose with {}
  \documentclass[french,twoside]{book} % ,notitlepage
  \usepackage[%
    papersize={105mm, 297mm},
    inner=12mm,
    outer=12mm,
    top=20mm,
    bottom=15mm,
    marginparsep=3pt,
    marginpar=7mm,
  ]{geometry}
  \usepackage[fontsize=9.5pt]{scrextend} % for Roboto
\else\ifav
  \documentclass[french,twoside]{book} % ,notitlepage
  \usepackage[%
    a5paper,
    inner=25mm,
    outer=15mm,
    top=15mm,
    bottom=15mm,
    marginparsep=0pt,
  ]{geometry}
  \usepackage[fontsize=12pt]{scrextend}
\else% A4 2 cols
  \documentclass[twocolumn]{report}
  \usepackage[%
    a4paper,
    inner=15mm,
    outer=10mm,
    top=25mm,
    bottom=18mm,
    marginparsep=0pt,
  ]{geometry}
  \setlength{\columnsep}{20mm}
  \usepackage[fontsize=9.5pt]{scrextend}
\fi\fi

%%%%%%%%%%%%%%
% Alignments %
%%%%%%%%%%%%%%
% before teinte macros

\setlength{\arrayrulewidth}{0.2pt}
\setlength{\columnseprule}{\arrayrulewidth} % twocol
\setlength{\parskip}{0pt} % 1pt allow better vertical justification
\setlength{\parindent}{1.5em}

%%%%%%%%%%
% Colors %
%%%%%%%%%%
% before Teinte macros

\usepackage[dvipsnames]{xcolor}
\definecolor{rubric}{HTML}{800000} % the tonic 0c71c3
\def\columnseprulecolor{\color{rubric}}
\colorlet{borderline}{rubric!30!} % definecolor need exact code
\definecolor{shadecolor}{gray}{0.95}
\definecolor{bghi}{gray}{0.5}

%%%%%%%%%%%%%%%%%
% Teinte macros %
%%%%%%%%%%%%%%%%%
%%%%%%%%%%%%%%%%%%%%%%%%%%%%%%%%%%%%%%%%%%%%%%%%%%%
% <TEI> generic (LaTeX names generated by Teinte) %
%%%%%%%%%%%%%%%%%%%%%%%%%%%%%%%%%%%%%%%%%%%%%%%%%%%
% This template is inserted in a specific design
% It is XeLaTeX and otf fonts

\makeatletter % <@@@

\usepackage{alphalph} % for alph couter z, aa, ab…
\usepackage{blindtext} % generate text for testing
\usepackage[strict]{changepage} % for modulo 4
\usepackage{contour} % rounding words
\usepackage[nodayofweek]{datetime}
\usepackage{enumitem} % <list>
\usepackage{etoolbox} % patch commands
\usepackage{fancyvrb}
\usepackage{fancyhdr}
\usepackage{float}
\usepackage{fontspec} % XeLaTeX mandatory for fonts
\usepackage{footnote} % used to capture notes in minipage (ex: quote)
\usepackage{framed} % bordering correct with footnote hack
\usepackage{graphicx}
\usepackage{lettrine} % drop caps
\usepackage{lipsum} % generate text for testing
\usepackage{manyfoot} % for parallel footnote numerotation
\usepackage[framemethod=tikz,]{mdframed} % maybe used for frame with footnotes inside
\usepackage[defaultlines=2,all]{nowidow} % at least 2 lines by par (works well!)
\usepackage{pdftexcmds} % needed for tests expressions
\usepackage{poetry} % <l>, bad for theater
\usepackage{polyglossia} % bug Warning: "Failed to patch part"
\usepackage[%
  indentfirst=false,
  vskip=1em,
  noorphanfirst=true,
  noorphanafter=true,
  leftmargin=\parindent,
  rightmargin=0pt,
]{quoting}
\usepackage{ragged2e}
\usepackage{setspace} % \setstretch for <quote>
\usepackage{tabularx} % <table>
\usepackage[explicit]{titlesec} % wear titles, !NO implicit
\usepackage{tikz} % ornaments
\usepackage{tocloft} % styling tocs
\usepackage[fit]{truncate} % used im runing titles
\usepackage{unicode-math}
\usepackage[normalem]{ulem} % breakable \uline, normalem is absolutely necessary to keep \emph
\usepackage{xcolor} % named colors
\usepackage{xparse} % @ifundefined
\XeTeXdefaultencoding "iso-8859-1" % bad encoding of xstring
\usepackage{xstring} % string tests
\XeTeXdefaultencoding "utf-8"

\defaultfontfeatures{
  % Mapping=tex-text, % no effect seen
  Scale=MatchLowercase,
  Ligatures={TeX,Common},
}
\newfontfamily\zhfont{Noto Sans CJK SC}

% Metadata inserted by a program, from the TEI source, for title page and runing heads
\title{\textbf{ Qu’est-ce qu’une nation ? }\par
}
\date{1882}
\author{Ernest Renan}
\def\elbibl{Ernest Renan. 1882. \emph{Qu’est-ce qu’une nation ?}}
\def\elabstract{%
 
\labelblock{L’Europe est-elle une nation ?}

 \noindent Renan est souvent cité, \emph{« Qu’est-ce qu’une nation ? […] un plébiscite de tous les jours »}, si bien qu’on croit le connaître à force d’en avoir entendu parler. On ne connaît pas une personne par ce qu’on en dit, il faut aller l’entendre soi-même. Ce discours a été prononcé en 1882, à Paris, après la défaite de la France de 1870, pendant que l’on chantait dans les écoles pour reprendre l’Alsace et la Lorraine. Ce savant, historien et philologue de bientôt 60 ans, voulait dissoudre les mauvaises raisons de partir en guerre la fleur au fusil, en commençant par demander aux alsaciens et aux lorrains, « de quel état veulent-ils faire partie ? » Le plébiscite pour une nation n’était pas alors qu’un symbole ; il s’agissait de demander aux allemands de réaliser un référendum loyal auprès des alsaciens et lorrains.\par
 Le plus intéressant n’est pas tant sa définition de la nation que l’énumération de tout ce qu’elle n’est pas (la deuxième partie).\par
 
\begin{quoteblock}
\noindent L’homme n’est esclave ni de sa race, ni de sa langue, ni de sa religion, ni du cours des fleuves, ni de la direction des chaînes de montagnes.\end{quoteblock}

 \noindent Il faudrait sans doute inviter quelques réactionnaires racistes ou religieux à se persuader de ce texte, mais les personnes qui le liront effectivement n’ont généralement plus besoin d’être convaincues de ces évidences pour nos jours. Alors à quoi bon le lire ?\par
 
\begin{quoteblock}
\noindent Voilà des points sur lesquels un esprit réfléchi tient à être fixé, pour se mettre d’accord avec lui-même. Les affaires du monde ne se règlent guère par ces sortes de raisonnements ; mais les hommes appliqués veulent porter en ces matières quelque raison et démêler les confusions où s’embrouillent les esprits superficiels.\end{quoteblock}

 \noindent C’est le discours d’un sage dans un tumulte politique. D’aucuns diront un bourgeois qui a la tranquillité matérielle, sans doute, mais faut-il donc que tous les gueux soient énervés ? L’humeur et le caractère ne dépendent pas de la richesse, nul n’est interdit de sagesse.\par
 La première récompense de cette lecture est d’entrer dans la compagnie d’un savant qui sait, mais surtout, qui réfléchit, et nous apprend à le faire. Il est saisissant que ses vues historiques ou anthropologiques, beaucoup moins informées que nous ne le sommes aujourd’hui, ne recèlent pas de défauts majeurs. Il rappelle que l’information ne dispense pas d’avoir du jugement. Avec beaucoup plus de faits, on peut raconter des histoires fausses, ou nuisibles. La longue méditation permet d’oser des hypothèses durables, avec une culture classique à peine plus large que les humanistes de la Renaissance.\par
 Sur l’histoire de l’Europe par exemple, Renan constate que les germains, puis ensuite les vikings, n’ont pas laissé leur langue en terre latine. Il se demande alors simplement, quelles langues parlait la mère de leurs enfants ? La génétique des populations actuelles, ainsi que des squelettes anciens, confirme largement le brassage matrimonial qu’est l’Europe depuis des millénaires. Chez les latins, comme les germains, contrairement à par exemple l’Empire turc d’alors ; l’Europe est exogame (on se marie pas entre cousins), monogame, et \emph{universelle} (on est “tous frères”, ou plutôt, notre système de parenté est simpliste). Un médiéviste comme Alain Guerreau a dès 1980 identifier cette \href{https://halshs.archives-ouvertes.fr/halshs-00418565}{\dotuline{singularité féodale}}\footnote{\href{https://halshs.archives-ouvertes.fr/halshs-00418565}{\url{https://halshs.archives-ouvertes.fr/halshs-00418565}}} (analysée maintenant à plus \href{http://www.gallimard.fr/Catalogue/GALLIMARD/NRF-Essais/L-origine-des-systemes-familiaux}{\dotuline{large échelle}}\footnote{\href{http://www.gallimard.fr/Catalogue/GALLIMARD/NRF-Essais/L-origine-des-systemes-familiaux}{\url{http://www.gallimard.fr/Catalogue/GALLIMARD/NRF-Essais/L-origine-des-systemes-familiaux}}} par Emmanuel Todd).\par
 
\begin{quoteblock}
\noindent […] le droit primordial des races est étroit et plein de danger pour le véritable progrès […] Par leurs facultés diverses, souvent opposées, les nations servent à l’œuvre commune de la civilisation […]\end{quoteblock}

 \noindent Avec son siècle, il possède peut-être un secret que nous avons perdu pour raisonner juste avec des faits douteux et des séries incomplètes ; il a un principe selon lequel hiérarchiser l’essentiel et l’accessoire. Son principe peut être faux, il n’en a pas moins une grande force d’organisation. Par exemple, un jésuite missionnaire comme Mateo Ricci a bien mieux compris les chinois que Marco Polo, car vouloir les convaincre des mystères étranges du christianisme l’a obligé à entrer dans leurs conceptions religieuses. Ainsi, nous ne pouvons pas reprendre à Renan sa religion du progrès et de la mission civilisatrice de l’Europe sans l’examen critique de la colonisation par exemple, mais nous pouvons regretter de ne pas avoir un tel principe, afin d’avoir autant d’ambition que lui dans ses vues. C’est peut-être notre travail du moment que de retrouver notre raison dans l’histoire.\par
 
\begin{quoteblock}
\noindent Les nations ne sont pas quelque chose d’éternel. Elles ont commencé, elles finiront. La confédération européenne, probablement, les remplacera.\end{quoteblock}

 \noindent Renan avait l’optimisme à l’échelle de l’histoire, bien au-delà de 1914 et 1939, mais sans doute aussi, au-delà de notre temps ; il espérait la fin des guerres franco-allemande fratricides et une Europe fédérale (ou plutôt, une confédération de nations, dont certaines pouvaient être fédérales ; mais pas une fédération de régions). Est-ce que l’union européenne actuelle serait une nation selon lui ? Si nous avons une mémoire commune, ce n’est pas celle de héros fondateurs, mais plutôt de nos guerres fratricides. L’Europe est-elle un plébiscite ? Elle n’est plus l’expression de de la volonté générale depuis la manipulation du référendum de 2005, mais le projet d’une classe instruite sans égard pour les autres. Renan enfonce même un pieu vicieux.\par
 
\begin{quoteblock}
\noindent Les intérêts suffisent-ils à faire une nation ? je ne le crois pas. […] un \foreign{Zollverein} n’est pas une patrie.\end{quoteblock}

 \noindent Le \foreign{Zollverein} est l’union douanière allemande qui alla de 1834 jusqu’à l’unification en 1871. L'union douanière fut la première mission de la Communauté européenne, elle a commencé il y a plus de 50 ans (1968). Mais l’Allemagne ne s’est vraiment unie qu’après les victoires contre l’Autriche (1866), et contre la France (1870). Quel ennemi doit trouver l’Europe pour faire patrie ? La démographie n’est plus à faire mourir ses jeunes pour la nation, bien heureusement, mais il est à craindre qu’il faudrait encore une convulsion de l’histoire pour que des nations européennes, et probablement pas 25, prennent conscience d’un destin commun et fassent vraiment nation (au sens de Renan).
 \newpage

}
\def\elsource{\href{https://archive.org/details/questcequunenat00renagoog/page/n8/mode/1up}{\dotuline{Archives.org}}\footnote{\href{https://archive.org/details/questcequunenat00renagoog/page/n8/mode/1up}{\url{https://archive.org/details/questcequunenat00renagoog/page/n8/mode/1up}}}}
\def\eltitlepage{%
{\centering\parindent0pt
  {\LARGE\addfontfeature{LetterSpace=25}\bfseries Ernest Renan\par}\bigskip
  {\Large 1882\par}\bigskip
  {\LARGE
\bigskip\textbf{Qu’est-ce qu’une nation ?}\par

  }
}

}

% Default metas
\newcommand{\colorprovide}[2]{\@ifundefinedcolor{#1}{\colorlet{#1}{#2}}{}}
\colorprovide{rubric}{red}
\colorprovide{silver}{lightgray}
\@ifundefined{syms}{\newfontfamily\syms{DejaVu Sans}}{}
\newif\ifdev
\@ifundefined{elbibl}{% No meta defined, maybe dev mode
  \newcommand{\elbibl}{Titre court ?}
  \newcommand{\elbook}{Titre du livre source ?}
  \newcommand{\elabstract}{Résumé\par}
  \newcommand{\elurl}{http://oeuvres.github.io/elbook/2}
  \author{Éric Lœchien}
  \title{Un titre de test assez long pour vérifier le comportement d’une maquette}
  \date{1566}
  \devtrue
}{}
\let\eltitle\@title
\let\elauthor\@author
\let\eldate\@date




% generic typo commands
\newcommand{\astermono}{\medskip\centerline{\color{rubric}\large\selectfont{\syms ✻}}\medskip\par}%
\newcommand{\astertri}{\medskip\par\centerline{\color{rubric}\large\selectfont{\syms ✻\,✻\,✻}}\medskip\par}%
\newcommand{\asterism}{\bigskip\par\noindent\parbox{\linewidth}{\centering\color{rubric}\large{\syms ✻}\\{\syms ✻}\hskip 0.75em{\syms ✻}}\bigskip\par}%

% lists
\newlength{\listmod}
\setlength{\listmod}{\parindent}
\setlist{
  itemindent=!,
  listparindent=\listmod,
  labelsep=0.2\listmod,
  parsep=0pt,
  % topsep=0.2em, % default topsep is best
}
\setlist[itemize]{
  label=—,
  leftmargin=0pt,
  labelindent=1.2em,
  labelwidth=0pt,
}
\setlist[enumerate]{
  label={\arabic*°},
  labelindent=0.8\listmod,
  leftmargin=\listmod,
  labelwidth=0pt,
}
% list for big items
\newlist{decbig}{enumerate}{1}
\setlist[decbig]{
  label={\bf\color{rubric}\arabic*.},
  labelindent=0.8\listmod,
  leftmargin=\listmod,
  labelwidth=0pt,
}
\newlist{listalpha}{enumerate}{1}
\setlist[listalpha]{
  label={\bf\color{rubric}\alph*.},
  leftmargin=0pt,
  labelindent=0.8\listmod,
  labelwidth=0pt,
}
\newcommand{\listhead}[1]{\hspace{-1\listmod}\emph{#1}}

\renewcommand{\hrulefill}{%
  \leavevmode\leaders\hrule height 0.2pt\hfill\kern\z@}

% General typo
\DeclareTextFontCommand{\textlarge}{\large}
\DeclareTextFontCommand{\textsmall}{\small}

% commands, inlines
\newcommand{\anchor}[1]{\Hy@raisedlink{\hypertarget{#1}{}}} % link to top of an anchor (not baseline)
\newcommand\abbr[1]{#1}
\newcommand{\autour}[1]{\tikz[baseline=(X.base)]\node [draw=rubric,thin,rectangle,inner sep=1.5pt, rounded corners=3pt] (X) {\color{rubric}#1};}
\newcommand\corr[1]{#1}
\newcommand{\ed}[1]{ {\color{silver}\sffamily\footnotesize (#1)} } % <milestone ed="1688"/>
\newcommand\expan[1]{#1}
\newcommand\foreign[1]{\emph{#1}}
\newcommand\gap[1]{#1}
\renewcommand{\LettrineFontHook}{\color{rubric}}
\newcommand{\initial}[2]{\lettrine[lines=2, loversize=0.3, lhang=0.3]{#1}{#2}}
\newcommand{\initialiv}[2]{%
  \let\oldLFH\LettrineFontHook
  % \renewcommand{\LettrineFontHook}{\color{rubric}\ttfamily}
  \IfSubStr{QJ’}{#1}{
    \lettrine[lines=4, lhang=0.2, loversize=-0.1, lraise=0.2]{\smash{#1}}{#2}
  }{\IfSubStr{É}{#1}{
    \lettrine[lines=4, lhang=0.2, loversize=-0, lraise=0]{\smash{#1}}{#2}
  }{\IfSubStr{ÀÂ}{#1}{
    \lettrine[lines=4, lhang=0.2, loversize=-0, lraise=0, slope=0.6em]{\smash{#1}}{#2}
  }{\IfSubStr{A}{#1}{
    \lettrine[lines=4, lhang=0.2, loversize=0.2, slope=0.6em]{\smash{#1}}{#2}
  }{\IfSubStr{V}{#1}{
    \lettrine[lines=4, lhang=0.2, loversize=0.2, slope=-0.5em]{\smash{#1}}{#2}
  }{
    \lettrine[lines=4, lhang=0.2, loversize=0.2]{\smash{#1}}{#2}
  }}}}}
  \let\LettrineFontHook\oldLFH
}
\newcommand{\labelchar}[1]{\textbf{\color{rubric} #1}}
\newcommand{\lnatt}[1]{\reversemarginpar\marginpar[\sffamily\scriptsize #1]{}}
\newcommand{\milestone}[1]{\autour{\footnotesize\color{rubric} #1}} % <milestone n="4"/>
\newcommand\name[1]{#1}
\newcommand\orig[1]{#1}
\newcommand\orgName[1]{#1}
\newcommand\persName[1]{#1}
\newcommand\placeName[1]{#1}
\newcommand{\pn}[1]{\IfSubStr{-—–¶}{#1}% <p n="3"/>
  {\noindent{\bfseries\color{rubric}   ¶  }}
  {{\footnotesize\autour{#1}}}}
\newcommand\reg{}
% \newcommand\ref{} % already defined
\newcommand\sic[1]{#1}
\newcommand\surname[1]{\textsc{#1}}
\newcommand\term[1]{\textbf{#1}}
\newcommand\zh[1]{{\zhfont #1}}


\def\mednobreak{\ifdim\lastskip<\medskipamount
  \removelastskip\nopagebreak\medskip\fi}
\def\bignobreak{\ifdim\lastskip<\bigskipamount
  \removelastskip\nopagebreak\bigskip\fi}

% commands, blocks

\newcommand{\byline}[1]{\bigskip{\RaggedLeft{#1}\par}\bigskip}

\newcommand{\bibl}[1]{{\smallskip\setlength{\RaggedLeftLeftskip}{2em plus \leftskip}\RaggedLeft\normalsize\normalfont #1\par\medskip}}}
\newcommand{\biblitem}[1]{{\noindent\hangindent=\parindent   #1\par}}
\newcommand{\castItem}[1]{{\noindent\hangindent=\parindent #1\par}}
\newcommand{\dateline}[1]{\medskip{\RaggedLeft{#1}\par}\bigskip}
\newcommand{\docAuthor}[1]{{\large\bigskip #1 \par\bigskip}}
\newcommand{\docDate}[1]{#1 \ifvmode\par\fi }
\newcommand{\docImprint}[1]{\ifvmode\medskip\fi #1 \ifvmode\par\fi }
\newcommand{\labelblock}[1]{\medbreak{\noindent\color{rubric}\bfseries #1}\par\mednobreak}
\newcommand{\salute}[1]{\bigbreak{#1}\par\medbreak}
\newcommand{\signed}[1]{\medskip{\RaggedLeft #1\par}\bigbreak} % supposed bottom
\newcommand{\speaker}[1]{\medskip{\Centering\sffamily #1\par\nopagebreak}} % supposed bottom
\newcommand{\stagescene}[1]{{\Centering\sffamily #1\par}\bigskip}
\newcommand{\stagesp}[1]{\begingroup\leftskip\parindent\noindent\it\sffamily #1\par\endgroup} % left margin, better than list envs
\newcommand{\spl}[1]{\noindent\hangindent=2\parindent  #1\par} % sp/l
\newcommand{\trailer}[1]{{\Centering\bigskip #1\par}} % sp/l

% environments for blocks (some may become commands)
\newenvironment{borderbox}{}{} % framing content
\newenvironment{citbibl}{\ifvmode\hfill\fi}{\ifvmode\par\fi }
\newenvironment{msHead}{\vskip6pt}{\par}
\newenvironment{msItem}{\vskip6pt}{\par}


% environments for block containers
\newenvironment{argument}{\itshape\parindent0pt}{\bigskip}
\newenvironment{biblfree}{}{\ifvmode\par\fi }
\newenvironment{bibitemlist}[1]{%
  \list{\@biblabel{\@arabic\c@enumiv}}%
  {%
    \settowidth\labelwidth{\@biblabel{#1}}%
    \leftmargin\labelwidth
    \advance\leftmargin\labelsep
    \@openbib@code
    \usecounter{enumiv}%
    \let\p@enumiv\@empty
    \renewcommand\theenumiv{\@arabic\c@enumiv}%
  }
  \sloppy
  \clubpenalty4000
  \@clubpenalty \clubpenalty
  \widowpenalty4000%
  \sfcode`\.\@m
}%
{\def\@noitemerr
  {\@latex@warning{Empty `bibitemlist' environment}}%
\endlist}
\newenvironment{docTitle}{\LARGE\bigskip\bfseries\onehalfspacing}{\bigskip}
\newenvironment{epigraph}{\leftskip1.5\parindent \sffamily\large}{\bigskip}
\newenvironment{quoteblock}% may be used for ornaments
  {\begin{quoting}}
  {\end{quoting}}
\newenvironment{titlePage}
  {\Centering}
  {}






% table () is preceded and finished by custom command
\newcommand{\tableopen}[1]{%
  \ifnum\strcmp{#1}{wide}=0{%
    \begin{center}
  }
  \else\ifnum\strcmp{#1}{long}=0{%
    \begin{center}
  }
  \else{%
    \begin{center}
  }
  \fi\fi
}
\newcommand{\tableclose}[1]{%
  \ifnum\strcmp{#1}{wide}=0{%
    \end{center}
  }
  \else\ifnum\strcmp{#1}{long}=0{%
    \end{center}
  }
  \else{%
    \end{center}
  }
  \fi\fi
}


% text structure
\newcommand\chapteropen{} % before chapter title
\newcommand\chaptercont{} % after title, argument, epigraph…
\newcommand\chapterclose{} % maybe useful for multicol settings
\setcounter{secnumdepth}{-2} % no counters for hierarchy titles
\setcounter{tocdepth}{5} % deep toc
\renewcommand\tableofcontents{\@starttoc{toc}}
% toclof format
% \renewcommand{\@tocrmarg}{0.1em} % Useless command?
% \renewcommand{\@pnumwidth}{0.5em} % {1.75em}
\renewcommand{\@cftmaketoctitle}{}
\setlength{\cftbeforesecskip}{\z@ \@plus.2\p@}
\renewcommand{\cftchapfont}{}
\renewcommand{\cftchapdotsep}{\cftdotsep}
\renewcommand{\cftchapleader}{\normalfont\cftdotfill{\cftchapdotsep}}
\renewcommand{\cftchappagefont}{\bfseries}
\setlength{\cftbeforechapskip}{0em \@plus\p@}
% \renewcommand{\cftsecfont}{\small\relax}
\renewcommand{\cftsecpagefont}{\normalfont}
% \renewcommand{\cftsubsecfont}{\small\relax}
\renewcommand{\cftsecdotsep}{\cftdotsep}
\renewcommand{\cftsecpagefont}{\normalfont}
\renewcommand{\cftsecleader}{\normalfont\cftdotfill{\cftsecdotsep}}
\setlength{\cftsecindent}{1em}
\setlength{\cftsubsecindent}{2em}
\setlength{\cftsubsubsecindent}{3em}
\setlength{\cftchapnumwidth}{1em}
\setlength{\cftsecnumwidth}{1em}
\setlength{\cftsubsecnumwidth}{1em}
\setlength{\cftsubsubsecnumwidth}{1em}

% footnotes
\newif\ifheading
\newcommand*{\fnmarkscale}{\ifheading 0.70 \else 1 \fi}
\renewcommand\footnoterule{\vspace*{0.3cm}\hrule height \arrayrulewidth width 3cm \vspace*{0.3cm}}
\setlength\footnotesep{1.5\footnotesep} % footnote separator
\renewcommand\@makefntext[1]{\parindent 1.5em \noindent \hb@xt@1.8em{\hss{\normalfont\@thefnmark . }}#1} % no superscipt in foot
\patchcmd{\@footnotetext}{\footnotesize}{\footnotesize\sffamily}{}{} % before scrextend, hyperref
\DeclareNewFootnote{A}[alph] % for editor notes
\renewcommand*{\thefootnoteA}{\alphalph{\value{footnoteA}}} % z, aa, ab…

% poem
\setlength{\poembotskip}{0pt}
\setlength{\poemtopskip}{0pt}
\setlength{\poemindent}{0pt}
\poemlinenumsfalse

%   see https://tex.stackexchange.com/a/34449/5049
\def\truncdiv#1#2{((#1-(#2-1)/2)/#2)}
\def\moduloop#1#2{(#1-\truncdiv{#1}{#2}*#2)}
\def\modulo#1#2{\number\numexpr\moduloop{#1}{#2}\relax}

% orphans and widows, nowidow package in test
% from memoir package
\clubpenalty=9996
\widowpenalty=9999
\brokenpenalty=4991
\predisplaypenalty=10000
\postdisplaypenalty=1549
\displaywidowpenalty=1602
\hyphenpenalty=400
% report h or v overfull ?
\hbadness=4000
\vbadness=4000
% good to avoid lines too wide
\emergencystretch 3em
\pretolerance=750
\tolerance=2000
\def\Gin@extensions{.pdf,.png,.jpg,.mps,.tif}

\PassOptionsToPackage{hyphens}{url} % before hyperref and biblatex, which load url package
\usepackage{hyperref} % supposed to be the last one, :o) except for the ones to follow
\hypersetup{
  % pdftex, % no effect
  pdftitle={\elbibl},
  % pdfauthor={Your name here},
  % pdfsubject={Your subject here},
  % pdfkeywords={keyword1, keyword2},
  bookmarksnumbered=true,
  bookmarksopen=true,
  bookmarksopenlevel=1,
  pdfstartview=Fit,
  breaklinks=true, % avoid long links, overrided by url package
  pdfpagemode=UseOutlines,    % pdf toc
  hyperfootnotes=true,
  colorlinks=false,
  pdfborder=0 0 0,
  % pdfpagelayout=TwoPageRight,
  % linktocpage=true, % NO, toc, link only on page no
}
\urlstyle{same} % after hyperref



\makeatother % /@@@>
%%%%%%%%%%%%%%
% </TEI> end %
%%%%%%%%%%%%%%


%%%%%%%%%%%%%
% footnotes %
%%%%%%%%%%%%%
\renewcommand{\thefootnote}{\bfseries\textcolor{rubric}{\arabic{footnote}}} % color for footnote marks

%%%%%%%%%
% Fonts %
%%%%%%%%%
\usepackage[]{roboto} % SmallCaps, Regular is a bit bold
% \linespread{0.90} % too compact, keep font natural
\newfontfamily\fontrun[]{Roboto Condensed Light} % condensed runing heads
\ifav
  \setmainfont[
    ItalicFont={Roboto Light Italic},
  ]{Roboto}
\else\ifbooklet
  \setmainfont[
    ItalicFont={Roboto Light Italic},
  ]{Roboto}
\else
\setmainfont[
  ItalicFont={Roboto Italic},
]{Roboto Light}
\fi\fi
\renewcommand{\LettrineFontHook}{\bfseries\color{rubric}}
% \renewenvironment{labelblock}{\begin{center}\bfseries\color{rubric}}{\end{center}}

%%%%%%%%
% MISC %
%%%%%%%%

\setdefaultlanguage[frenchpart=false]{french} % bug on part


\newenvironment{quotebar}{%
    \def\FrameCommand{{\color{rubric!10!}\vrule width 0.5em} \hspace{0.9em}}%
    \def\OuterFrameSep{0pt} % séparateur vertical
    \MakeFramed {\advance\hsize-\width \FrameRestore}
  }%
  {%
    \endMakeFramed
  }
\renewenvironment{quoteblock}% may be used for ornaments
  {%
    \savenotes
    \setstretch{0.9}
    \normalfont
    \begin{quotebar}
  }
  {%
    \end{quotebar}
    \spewnotes
  }


\renewcommand{\headrulewidth}{\arrayrulewidth}
\renewcommand{\headrule}{{\color{rubric}\hrule}}
\renewcommand{\lnatt}[1]{\marginpar{\sffamily\scriptsize #1}}

% delicate tuning, image has produce line-height problems in title on 2 lines
\titleformat{name=\chapter} % command
  [display] % shape
  {\vspace{1.5em}\centering} % format
  {} % label
  {0pt} % separator between n
  {}
[{\color{rubric}\huge\textbf{#1}}\bigskip] % after code
% \titlespacing{command}{left spacing}{before spacing}{after spacing}[right]
\titlespacing*{\chapter}{0pt}{-2em}{0pt}[0pt]

\titleformat{name=\section}
  [display]{}{}{}{}
  [\vbox{\color{rubric}\large\raggedleft\textbf{#1}}]
\titlespacing{\section}{0pt}{0pt plus 4pt minus 2pt}{\baselineskip}

\titleformat{name=\subsection}
  [block]
  {}
  {} % \thesection
  {} % separator \arrayrulewidth
  {}
[\vbox{\large\textbf{#1}}]
% \titlespacing{\subsection}{0pt}{0pt plus 4pt minus 2pt}{\baselineskip}

\ifaiv
  \fancypagestyle{main}{%
    \fancyhf{}
    \setlength{\headheight}{1.5em}
    \fancyhead{} % reset head
    \fancyfoot{} % reset foot
    \fancyhead[L]{\truncate{0.45\headwidth}{\fontrun\elbibl}} % book ref
    \fancyhead[R]{\truncate{0.45\headwidth}{ \fontrun\nouppercase\leftmark}} % Chapter title
    \fancyhead[C]{\thepage}
  }
  \fancypagestyle{plain}{% apply to chapter
    \fancyhf{}% clear all header and footer fields
    \setlength{\headheight}{1.5em}
    \fancyhead[L]{\truncate{0.9\headwidth}{\fontrun\elbibl}}
    \fancyhead[R]{\thepage}
  }
\else
  \fancypagestyle{main}{%
    \fancyhf{}
    \setlength{\headheight}{1.5em}
    \fancyhead{} % reset head
    \fancyfoot{} % reset foot
    \fancyhead[RE]{\truncate{0.9\headwidth}{\fontrun\elbibl}} % book ref
    \fancyhead[LO]{\truncate{0.9\headwidth}{\fontrun\nouppercase\leftmark}} % Chapter title, \nouppercase needed
    \fancyhead[RO,LE]{\thepage}
  }
  \fancypagestyle{plain}{% apply to chapter
    \fancyhf{}% clear all header and footer fields
    \setlength{\headheight}{1.5em}
    \fancyhead[L]{\truncate{0.9\headwidth}{\fontrun\elbibl}}
    \fancyhead[R]{\thepage}
  }
\fi

\ifav % a5 only
  \titleclass{\section}{top}
\fi

\newcommand\chapo{{%
  \vspace*{-3em}
  \centering\parindent0pt % no vskip ()
  \eltitlepage
  \bigskip
  {\color{rubric}\hline\par}
  \bigskip
  {\Large TEXTE LIBRE À PARTICIPATIONS LIBRES\par}
  \centerline{\small\color{rubric} {\href{https://hurlus.fr}{\dotuline{hurlus.fr}}}, tiré le \today}\par
  \bigskip
}}

\newcommand\cover{{%
  \thispagestyle{empty}
  \centering\parindent0pt
  \eltitlepage
  \vfill\null
  {\color{rubric}\setlength{\arrayrulewidth}{2pt}\hline\par}
  \vfill\null
  {\Large TEXTE LIBRE À PARTICIPATIONS LIBRES\par}
  \centerline{\href{https://hurlus.fr}{\dotuline{hurlus.fr}}, tiré le \today}\par
}}

\begin{document}
\pagestyle{empty}
\ifbooklet{
  \cover\newpage
  \thispagestyle{empty}\hbox{}\newpage
  \cover\newpage\noindent Les voyages de la brochure\par
  \bigskip
  \begin{tabularx}{\textwidth}{l|X|X}
    \textbf{Date} & \textbf{Lieu}& \textbf{Nom/pseudo} \\ \hline
    \rule{0pt}{25cm} &  &   \\
  \end{tabularx}
  \newpage
  \addtocounter{page}{-4}
}\fi

\thispagestyle{empty}
\ifaiv
  \twocolumn[\chapo]
\else
  \chapo
\fi
{\it\elabstract}
\bigskip
\makeatletter\@starttoc{toc}\makeatother % toc without new page
\bigskip

\pagestyle{main} % after style
\setcounter{footnote}{0}
\setcounter{footnoteA}{0}
  
\section[{[Préface]}]{[Préface]}
\renewcommand{\leftmark}{[Préface]}

\noindent Je me propose d’analyser avec vous une idée, claire en apparence, mais qui prête aux plus dangereux malentendus. Les formes de la société humaine sont des plus variées. Les grandes agglomérations d’hommes à la façon de la {\placeName Chine}, de l’{\placeName Égypte}, de la plus ancienne {\placeName Babylonie} ; — la tribu à la façon des {\orgName Hébreux}, des {\orgName Arabes} ; — la cité à la façon d’{\placeName Athènes} et de {\placeName Sparte} ; — les réunions de pays divers à la manière de l’{\orgName Empire carlovingien} ; — les communautés sans patrie, maintenues par le lien religieux, comme sont celles des israélites, des parsis ; — les nations comme la {\placeName France}, l’{\placeName Angleterre} et la plupart des modernes {\placeName autonomies européennes} ; — les confédérations à la façon de la {\placeName Suisse}, de l’{\placeName Amérique} ; — des parentés comme celles que la race, ou plutôt la langue, établit entre les différentes branches de Germains, les différentes branches de Slaves ; — voilà des modes de groupements qui tous existent, ou bien ont existé, et qu’on ne saurait confondre les uns avec les autres sans les plus sérieux inconvénients. À l’époque de la Révolution française, on croyait que les institutions de petites villes indépendantes, telles que {\placeName Sparte} et {\placeName Rome}, pouvaient s’appliquer à nos grandes nations de trente à quarante millions d’âmes. De nos jours, on commet une erreur plus grave : on confond la race avec la nation, et l’on attribue à des groupes ethnographiques ou plutôt linguistiques une souveraineté analogue à celle des peuples réellement existants. Tâchons d’arriver à quelque précision en ces questions difficiles, où la moindre confusion sur le sens des mots, à l’origine du raisonnement, peut produire à la fin les plus funestes erreurs. Ce que nous allons faire est délicat ; c’est presque de la vivisection ; nous allons traiter les vivants comme d’ordinaire on traite les morts. Nous y mettrons la froideur, l’impartialité la plus absolue.

\chapteropen

\chapter[{I. [L’héritage romain]}]{I. [L’héritage romain]}
\renewcommand{\leftmark}{I. [L’héritage romain]}


\chaptercont
\noindent Depuis la fin de l’{\orgName Empire romain}, ou, mieux, depuis la dislocation de l’{\orgName Empire de Charlemagne}, l’{\placeName Europe occidentale} nous apparaît divisée en nations, dont quelques-unes, à certaines époques, ont cherché à exercer une hégémonie sur les autres, sans jamais y réussir d’une manière durable. Ce que n’ont pu {\persName Charles-Quint}, {\persName Louis XIV}, {\persName Napoléon 1\textsuperscript{er}}, personne probablement ne le pourra dans l’avenir. L’établissement d’un nouvel {\orgName Empire romain} ou d’un nouvel {\orgName Empire de Charlemagne} est devenu une impossibilité. La division de l’{\orgName Europe} est trop grande pour qu’une tentative de domination universelle ne provoque pas très vite une coalition qui fasse rentrer la nation ambitieuse dans ses bornes naturelles. Une sorte d’équilibre est établi pour longtemps. La {\orgName France}, l’{\orgName Angleterre}, l’{\orgName Allemagne}, la {\orgName Russie} seront encore, dans des centaines d’années, et malgré les aventures qu’elles auront courues, des individualités historiques, les pièces essentielles d’un damier, dont les cases varient sans cesse d’importance et de grandeur, mais ne se confondent jamais tout à fait.\par
Les nations, entendues de cette manière, sont quelque chose d’assez nouveau dans l’histoire. L’antiquité ne les connut pas ; l’{\orgName Égypte}, la {\orgName Chine}, l’antique {\orgName Chaldée} ne furent à aucun degré des nations. C’étaient des troupeaux menés par un fils du Soleil, ou un fils du Ciel. Il n’y eut pas de citoyens égyptiens, pas plus qu’il n’y a de citoyens chinois. L’antiquité classique eut des républiques et des royautés municipales, des confédérations de républiques locales, des empires ; elle n’eut guère la nation au sens où nous la comprenons. {\orgName Athènes}, {\orgName Sparte}, {\orgName Sidon}, {\orgName Tyr} sont de petits centres d’admirable patriotisme ; mais ce sont des cités avec un territoire relativement restreint. La {\orgName Gaule}, l’{\orgName Espagne}, l’{\orgName Italie}, avant leur absorption dans l’{\orgName Empire romain}, étaient des ensembles de peuplades, souvent liguées entre elles, mais sans institutions centrales, sans dynasties. L’{\orgName Empire assyrien}, l’{\orgName Empire persan}, l’{\orgName Empire d’Alexandre} ne furent pas non plus des patries. Il n’y eut jamais de patriotes assyriens ; l’{\orgName Empire persan} fut une vaste féodalité. Pas une nation ne rattache ses origines à la colossale aventure d’{\persName Alexandre}, qui fut cependant si riche en conséquences pour l’histoire générale de la civilisation.\par
L’{\orgName Empire romain} fut bien plus près d’être une patrie. En retour de l’immense bienfait de la cessation des guerres, la domination romaine, d’abord si dure, fut bien vite aimée. Ce fut une grande association, synonyme d’ordre, de paix et de civilisation. Dans les derniers temps de l’{\orgName Empire}, il y eut, chez les âmes élevées, chez les évêques éclairés, chez les lettrés, un vrai sentiment de « la paix romaine », opposée au chaos menaçant de la barbarie. Mais un empire, douze fois grand comme la {\placeName France} actuelle, ne saurait former un État dans l’acception moderne. La scission de l’{\placeName Orient} et de l’{\placeName Occident} était inévitable. Les essais d’un {\orgName empire gaulois}, au III\textsuperscript{e} siècle, ne réussirent pas. C’est l’invasion germanique qui introduisît dans le monde le principe qui, plus tard, a servi de base à l’existence des nationalités.\par
Que firent les {\orgName peuples germaniques}, en effet, depuis leurs grandes invasions du V\textsuperscript{e} siècle jusqu’aux dernières conquêtes normandes au X\textsuperscript{e} ? Ils changèrent peu le fond des races ; mais ils imposèrent des dynasties et une aristocratie militaire à des parties plus ou moins considérables de l’ancien {\orgName Empire d’Occident}, lesquelles prirent le nom de leurs envahisseurs. De là une {\placeName France}, une {\placeName Burgondie}, une {\placeName Lombardie} ; plus tard, une {\placeName Normandie}. La rapide prépondérance que prit l’{\orgName empire franc} refait un moment l’unité de l’{\orgName Occident} ; mais cet empire se brise irrémédiablement vers le milieu du IX\textsuperscript{e} siècle ; le traité de Verdun trace des divisions immuables en principe, et dès lors la {\orgName France}, l’{\orgName Allemagne}, l’{\orgName Angleterre}, l’{\orgName Italie}, l’{\orgName Espagne} s’acheminent, par des voies souvent détournées et à travers mille aventures, à leur pleine existence nationale, telle que nous la voyons s’épanouir aujourd’hui. Qu’est-ce qui caractérise, en effet, ces différents États ? C’est la fusion des populations qui les composent. Dans les pays que nous venons d’énumérer, rien d’analogue à ce que vous trouverez en {\placeName Turquie}, où le Turc, le Slave, le Grec, l’Arménien, l’Arabe, le Syrien, le Kurde sont aussi distincts aujourd’hui qu’au jour de la conquête. Deux circonstances essentielles contribuèrent à ce résultat. D’abord le fait que les {\orgName peuples germaniques} adoptèrent le christianisme dès qu’ils eurent des contacts un peu suivis avec les {\orgName peuples grecs et latins}. Quand le vainqueur et le vaincu sont de la même religion, ou plutôt, quand le vainqueur adopte la religion du vaincu, le système turc, la distinction absolue des hommes d’après la religion, ne peut plus se produire. La seconde circonstance fut, de la part des conquérants, l’oubli de leur propre langue. Les petits-fils de {\persName Clovis}, d’{\persName Alaric}, de {\persName Gondebaud}, d’{\persName Alboin}, de {\persName Rollon}, parlaient déjà roman. Ce fait était lui-même la conséquence d’une autre particularité importante : c’est que les {\orgName Francs}, les {\orgName Burgondes}, les {\orgName Goths}, les {\orgName Lombards}, les {\orgName Normands} avaient très peu de femmes de leur race avec eux. Pendant plusieurs générations, les chefs ne se marient qu’avec des femmes germaines ; mais leurs concubines sont latines, les nourrices des enfants sont latines ; toute la tribu épouse des femmes latines ; ce qui fit que la {\itshape lingua francica}, la {\itshape lingua gothica} n’eurent depuis l’établissement des {\orgName Francs} et des {\orgName Goths} en {\placeName terres romaines}, que de très courtes destinées. Il n’en fut pas ainsi en {\placeName Angleterre} ; car l’invasion anglo-saxonne avait sans doute des femmes avec elle ; la {\orgName population bretonne} s’enfuit et d’ailleurs, le latin n’était plus, ou même, ne fut jamais dominant dans la {\placeName Bretagne}. Si on eût généralement parlé gaulois dans la {\placeName Gaule}, au V\textsuperscript{e} siècle, {\persName Clovis} et les siens n’eussent pas abandonné le germanique pour le gaulois.\par
De là ce résultat capital que, malgré l’extrême violence des mœurs des envahisseurs germains, le moule qu’ils imposèrent devint, avec les siècles, le moule même de la nation. {\placeName France} devint très légitimement le nom d’un pays où il n’était entré qu’une imperceptible minorité de {\orgName Francs}. Au X\textsuperscript{e} siècle, dans les premières chansons de geste, qui sont un miroir si parfait de l’esprit du temps, tous les habitants de la {\placeName France} sont des Français. L’idée d’une différence de races dans la population de la {\placeName France}, si évidente chez {\persName Grégoire de Tours}, ne se présente à aucun degré chez les écrivains et les poètes français postérieurs à {\persName Hugues Capet}. La différence du noble et du vilain est aussi accentuée que possible ; mais la différence de l’un à l’autre n’est en rien une différence ethnique ; c’est une différence de courage, d’habitudes et d’éducation transmise héréditairement ; l’idée que l’origine de tout cela soit une conquête ne vient à personne. Le faux système d’après lequel la noblesse dut son origine à un privilège conféré par le roi pour de grands services rendus à la nation, si bien que tout noble est un anobli, ce système est établi comme un dogme dès le XIII\textsuperscript{e} siècle. La même chose se passa à la suite de presque toutes les conquêtes normandes. Au bout d’une ou deux générations, les envahisseurs normands ne se distinguaient plus du reste de la population ; leur influence n’en avait pas moins été profonde ; ils avaient donné au pays conquis une noblesse, des habitudes militaires, un patriotisme qu’il n’avait pas auparavant.\par
L’oubli, et je dirai même l’erreur historique, sont un facteur essentiel de la création d’une nation, et c’est ainsi que le progrès des études historiques est souvent pour la nationalité un danger. L’investigation historique, en effet, remet en lumière les faits de violence qui se sont passés à l’origine de toutes les formations politiques, même de celles dont les conséquences ont été le plus bienfaisantes. L’unité se fait toujours brutalement ; la réunion de la {\placeName France du Nord} et de la {\placeName France du Midi} a été le résultat d’une extermination et d’une terreur continuée pendant près d’un siècle. Le {\persName roi de France}, qui est, si j’ose le dire, le type idéal d’un cristallisateur séculaire ; le {\persName roi de France}, qui a fait la plus parfaite unité nationale qu’il y ait ; le {\persName roi de France}, vu de trop près, a perdu son prestige ; la nation qu’il avait formée l’a maudit, et, aujourd’hui, il n’y a que les esprits cultivés qui sachent ce qu’il valait et ce qu’il a fait.\par
C’est par le contraste que ces grandes lois de l’histoire de l’{\placeName Europe occidentale} deviennent sensibles. Dans l’entreprise que le {\persName roi de France}, en partie par sa tyrannie, en partie par sa justice, a si admirablement menée à terme, beaucoup de pays ont échoué. Sous la couronne de saint Étienne, les {\orgName Magyars} et les {\orgName Slaves} sont restés aussi distincts qu’ils l’étaient il y a huit cents ans. Loin de fondre les éléments divers de ses domaines, la {\orgName maison de Habsbourg} les a tenus distincts et souvent opposés les uns aux autres. En {\placeName Bohême}, l’élément tchèque et l’élément allemand sont superposés comme l’huile et l’eau dans un verre. La politique turque de la séparation des nationalités d’après la religion eu de bien plus graves conséquences : elle a causé la ruine de l’{\orgName Orient}. Prenez une ville comme {\placeName Salonique} ou {\placeName Smyrne}, vous y trouverez cinq ou six communautés dont chacune a ses souvenirs et qui n’ont entre elles presque rien en commun. Or l’essence d’une nation est que tous les individus aient beaucoup de choses en commun, et aussi que tous aient oublié bien des choses. Aucun citoyen français ne sait s’il est Burgonde, Alain, Taïfale, Visigoth ; tout citoyen français doit avoir oublié la Saint-Barthélemy, les massacres du {\placeName Midi} au XIII\textsuperscript{e} siècle. Il n’y a pas en {\placeName France} dix familles qui puissent fournir la preuve d’une origine franque, et encore une telle preuve serait-elle essentiellement défectueuse, par suite de mille croisements inconnus qui peuvent déranger tous les systèmes des généalogistes.\par
La nation moderne est donc un résultat historique amené par une série de faits convergeant dans le même sens. Tantôt l’unité a été réalisée par une dynastie, comme c’est le cas pour la {\placeName France} ; tantôt elle l’a été par la volonté directe des provinces, comme c’est le cas pour la {\placeName Hollande}, la {\placeName Suisse}, la {\placeName Belgique} ; tantôt par un esprit général, tardivement vainqueur des caprices de la féodalité, comme c’est le cas pour l’{\placeName Italie} et l’{\placeName Allemagne}. Toujours une profonde raison d’être a présidé à ces formations. Les principes, en pareils cas, se font jour par les surprises les plus inattendues. Nous avons vu, de nos jours, l’{\orgName Italie} unifiée par ses défaites, et la {\orgName Turquie} démolie par ses victoires. Chaque défaite avançait les affaires de l’{\orgName Italie} ; chaque victoire perdait la {\orgName Turquie} ; car l’{\orgName Italie} est une nation, et la {\orgName Turquie}, hors de l’{\placeName Asie Mineure}, n’en est pas une. C’est la gloire de la {\orgName France} d’avoir, par la Révolution française, proclamé qu’une nation existe par elle-même. Nous ne devons pas trouver mauvais qu’on nous imite. Le principe des nations est le nôtre. Mais qu’est-ce donc qu’une nation ? Pourquoi la {\placeName Hollande} est-elle une nation, tandis que le {\placeName Hanovre} ou le {\placeName grand-duché de Parme} n’en sont pas une ? Comment la {\placeName France} persiste-t-elle à être une nation, quand le principe qui l’a créée a disparu ? Comment la {\placeName Suisse}, qui a trois langues, deux religions, trois ou quatre races, est-elle une nation, quand la {\placeName Toscane}, par exemple, qui est si homogène, n’en est pas une ? Pourquoi l’{\placeName Autriche} est-elle un État et non pas une nation ? En quoi le principe des nationalités diffère-t-il du principe des races ? Voilà des points sur lesquels un esprit réfléchi tient à être fixé, pour se mettre d’accord avec lui-même. Les affaires du monde ne se règlent guère par ces sortes de raisonnements ; mais les hommes appliqués veulent porter en ces matières quelque raison et démêler les confusions où s’embrouillent les esprits superficiels.
\chapterclose


\chapteropen

\chapter[{II. [Ce qu’une nation n’est pas]}]{II. [Ce qu’une nation n’est pas]}
\renewcommand{\leftmark}{II. [Ce qu’une nation n’est pas]}


\chaptercont
\noindent À entendre certains théoriciens politiques, une nation est avant tout une dynastie, représentant une ancienne conquête, conquête acceptée d’abord, puis oubliée par la masse du peuple. Selon les politiques dont je parle, le groupement de provinces effectué par une dynastie, par ses guerres, par ses mariages, par ses traités, finit avec la dynastie qui l’a formé. Il est très vrai que la plupart des nations modernes ont été faites par une famille d’origine féodale, qui a contracté mariage avec le sol et qui a été en quelque sorte un noyau de centralisation. Les limites de la {\placeName France} en 1789 n’avaient rien de naturel ni de nécessaire. La large zone que la {\orgName maison capétienne} avait ajoutée à l’étroite lisière du traité de Verdun fut bien l’acquisition personnelle de cette maison. À l’époque où furent faites les annexions, on n’avait l’idée ni des limites naturelles, ni du droit des nations, ni de la volonté des provinces. La réunion de l’{\placeName Angleterre}, de l’{\placeName Irlande} et de l’{\placeName Écosse} fut de même un fait dynastique. L’{\placeName Italie} n’a tardé si longtemps à être une nation que parce que, parmi ses nombreuses maisons régnantes, aucune, avant notre siècle, ne se fit le centre de l’unité. Chose étrange, c’est à l’obscure île de {\placeName Sardaigne}, terre à peine italienne, qu’elle a pris un titre royal \footnote{La {\orgName maison de Savoie} ne doit son titre royal qu’à la possession de la {\placeName Sardaigne} (1720).}. La {\placeName Hollande}, qui s’est créée elle-même, par un acte d’héroïque résolution, a néanmoins contracté un mariage intime avec la {\orgName maison d’Orange}, et elle courrait de vrais dangers le jour où cette union serait compromise.\par
Une telle loi, cependant, est-elle absolue ? Non, sans doute. La {\placeName Suisse} et les {\placeName États-Unis}, qui se sont formés comme des conglomérats d’additions successives, n’ont aucune base dynastique. je ne discuterai pas la question en ce qui concerne la {\placeName France}. Il faudrait avoir le secret de l’avenir. Disons seulement que cette grande royauté française avait été si hautement nationale, que, le lendemain de sa chute, la nation a pu tenir sans elle. Et puis le XVIII\textsuperscript{e} siècle avait changé toute chose. L’homme était revenu, après des siècles d’abaissement, à l’esprit antique, au respect de lui-même, à l’idée de ses droits. Les mots de patrie et de citoyen avaient repris leur sens. Ainsi a pu s’accomplir l’opération la plus hardie qui ait été pratiquée dans l’histoire, opération que l’on peut comparer à ce que serait, en physiologie, la tentative de faire vivre en son identité première un corps à qui l’on aurait enlevé le cerveau et le cœur.\par
Il faut donc admettre qu’une nation peut exister sans principe dynastique, et même que des nations qui ont été formées par des dynasties peuvent se séparer de cette dynastie sans pour cela cesser d’exister, Le vieux principe qui ne tient compte que du droit des princes ne saurait plus être maintenu ; outre le droit dynastique, il y a le droit national. Ce droit national, sur quel critérium le fonder ? à quel signe le connaître ? de quel fait tangible le faire dériver ?\par

\section[{1. — De la race, disent plusieurs avec assurance.}]{1. — De la race, disent plusieurs avec assurance.}

\noindent Les divisions artificielles, résultant de la féodalité, des mariages princiers, des congrès de diplomates, sont caduques. Ce qui reste ferme et fixe, c’est la race des populations. Voilà ce qui constitue un droit, une légitimité. La famille germanique, par exemple, selon la théorie que j’expose, a le droit de reprendre les membres épars du germanisme, même quand ces membres ne demandent pas à se rejoindre. Le droit du germanisme sur telle province est plus fort que le droit des habitants de cette province sur eux-mêmes. On crée ainsi une sorte de droit primordial analogue à celui des rois de droit divin ; au principe des nations on substitue celui de l’ethnographie. C’est là une très grande erreur, qui, si elle devenait dominante, perdrait la civilisation européenne. Autant le principe des nations est juste et légitime, autant celui du droit primordial des races est étroit et plein de danger pour le véritable progrès.\par
Dans la tribu et la cité antiques, le fait de la race avait, nous le reconnaissons, une importance de premier ordre. La tribu et la cité antiques n’étaient qu’une extension de la famille. À {\placeName Sparte}, à {\placeName Athènes}, tous les citoyens étaient parents à des degrés plus ou moins rapprochés. Il en était de même chez les {\orgName Beni-Israël} ; il en est encore ainsi dans les {\orgName tribus arabes}. D’{\placeName Athènes}, de {\placeName Sparte}, de la {\orgName tribu israélite}, transportons-nous dans l’{\orgName Empire romain}. La situation est tout autre. Formée d’abord par la violence, puis maintenue par l’intérêt, cette grande agglomération de villes, de provinces absolument différentes, porte à l’idée de race le coup le plus grave. Le christianisme, avec son caractère universel et absolu, travaille plus efficacement encore dans le même sens. Il contracte avec l’{\orgName Empire romain} une alliance intime, et, par l’effet de ces deux incomparables agents d’unification, la raison ethnographique est écartée du gouvernement des choses humaines pour des siècles.\par
L’invasion des {\orgName barbares} fut, malgré les apparences, un pas de plus dans cette voie. Les découpures de {\placeName royaumes barbares} n’ont rien d’ethnographique ; elles sont réglées par la force ou le caprice des envahisseurs. La race des populations qu’ils subordonnaient était pour eux la chose la plus indifférente. {\persName Charlemagne} refit à sa manière ce que {\orgName Rome} avait déjà fait : un empire unique composé des races les plus diverses ; les auteurs du traité de Verdun, en traçant imperturbablement leurs deux grandes lignes du nord au sud, n’eurent pas le moindre souci de la race des gens qui se trouvaient à droite ou à gauche. Les mouvements de frontière qui s’opérèrent dans la suite du moyen âge furent aussi en dehors de toute tendance ethnographique. Si la politique suivie de la maison capétienne est arrivée à grouper à peu près, sous le nom de {\placeName France}, les territoires de l’ancienne {\placeName Gaule}, ce n’est pas là un effet de la tendance qu’auraient eue ces pays à se rejoindre à leurs congénères. Le {\placeName Dauphiné}, la {\placeName Bresse}, la {\placeName Provence}, la {\placeName Franche-Comté} ne se souvenaient plus d’une origine commune. Toute conscience gauloise avait péri dès le II\textsuperscript{e} siècle de notre ère, et ce n’est que par une vue d’érudition que, de nos jours, on a retrouvé rétrospectivement l’individualité du caractère gaulois.\par
La considération ethnographique n’a donc été pour rien dans la constitution des nations modernes. La {\orgName France} est celtique, ibérique, germanique. L’{\orgName Allemagne} est germanique, celtique et slave. L’{\orgName Italie} est le pays où l’ethnographie est la plus embarrassée. {\orgName Gaulois}, {\orgName Étrusques}, {\orgName Pélasges}, {\orgName Grecs}, sans parler de bien d’autres éléments, s’y croisent dans un indéchiffrable mélange. Les {\orgName îles Britanniques}, dans leur ensemble, offrent un mélange de sang celtique et germain dont les proportions sont singulièrement difficiles à définir.\par
La vérité est qu’il n’y a pas de race pure et que faire reposer la politique sur l’analyse ethnographique, c’est la faire porter sur une chimère. Les plus nobles pays, l’{\placeName Angleterre}, la {\placeName France}, l’{\placeName Italie}, sont ceux où le sang est le plus mêlé. L’{\placeName Allemagne} fait-elle à cet égard une exception ? Est-elle un pays germanique pur ? Quelle illusion ! Tout le {\placeName Sud} a été gaulois. Tout l’{\placeName Est}, à partir de l’{\placeName Elbe}, est slave. Et les parties que l’on prétend réellement pures le sont-elles en effet ? Nous touchons ici à un des problèmes sur lesquels il importe le plus de se faire des idées claires et de prévenir les malentendus.\par
Les discussions sur les races sont interminables, parce que le mot race est pris par les historiens philologues et par les anthropologistes physiologistes dans deux sens tout à fait différents \footnote{Ce point a été développé dans une conférence dont on peut lire l’analyse dans le bulletin de l’Association scientifique de {\placeName France}, 10 mars 1878 : \emph{Des services rendus aux Sciences historiques par la Philologie}.}. Pour les anthropologistes, la race a le même sens qu’en zoologie ; elle indique une descendance réelle, une parenté par le sang. Or l’étude des langues et de l’histoire ne conduit pas aux mêmes divisions que la physiologie. Les mots de brachycéphales, de dolichocéphales n’ont pas de place en histoire ni en philologie. Dans le groupe humain qui créa les langues et la discipline aryennes, il y avait déjà des brachycéphales et des dolichocéphales. Il en faut dire autant du groupe primitif qui créa les langues et l’institution dites sémitiques. En d’autres termes, les origines zoologiques de l’humanité sont énormément antérieures aux origines de la culture, de la civilisation, du langage. Les groupes aryen primitif, sémitique primitif, touranien primitif n’avaient aucune unité physiologique. Ces groupements sont des faits historiques qui ont eu lieu à une certaine époque, mettons il y a quinze ou vingt mille ans, tandis que l’origine zoologique de l’humanité se perd dans les ténèbres incalculables. Ce qu’on appelle philologiquement et historiquement la race germanique est sûrement une famille bien distincte dans l’espèce humaine. Mais est-ce là une famille au sens anthropologique ? Non, assurément. L’apparition de l’individualité germanique dans l’histoire ne se fait que très peu de siècles avant {\persName Jésus-Christ}. Apparemment les {\orgName Germains} ne sont pas sortis de terre à cette époque. Avant cela, fondus avec les {\orgName Slaves} dans la grande masse indistincte des {\orgName Scythes}, ils n’avaient pas leur individualité à part. Un Anglais est bien un type dans l’ensemble de l’humanité. Or le type de ce qu’on appelle très improprement la race anglo-saxonne \footnote{Les éléments germaniques ne sont pas beaucoup plus considérables dans le {\placeName Royaume-Uni} qu’ils ne l’étaient dans la {\placeName France}, à l’époque où elle possédait l’{\placeName Alsace} et {\placeName Metz}. La langue germanique a dominé dans les {\placeName îles Britanniques}, uniquement parce que le latin n’y avait pas entièrement remplacé les idiomes celtiques, ainsi que cela eut lieu dans les {\placeName Gaules}.} n’est ni le Breton du temps de {\persName César}, ni l’Anglo-Saxon de {\persName Hengist}, ni le Danois de {\persName Knut}, ni le Normand de {\persName Guillaume le Conquérant} ; c’est la résultante de tout cela. Le Français n’est ni un Gaulois, ni un Franc, ni un Burgonde. Il est ce qui est sorti de la grande chaudière où, sous la présidence du {\persName roi de France}, ont fermenté ensemble les éléments les plus divers. Un habitant de {\placeName Jersey} ou de {\placeName Guernesey} ne diffère en rien, pour les origines, de la {\orgName population normande} de la côte voisine. Au XI\textsuperscript{e} siècle, l’œil le plus pénétrant n’eût pas saisi des deux côtés du canal la plus légère différence. D’insignifiantes circonstances font que {\persName Philippe Auguste} ne prend pas ces îles avec le reste de la {\placeName Normandie}. Séparées les unes des autres depuis près de sept cents ans, les deux populations sont devenues non seulement étrangères les unes aux autres, mais tout à fait dissemblables. La race, comme nous l’entendons, nous autres, historiens, est donc quelque chose qui se fait et se défait. L’étude de la race est capitale pour le savant qui s’occupe de l’histoire de l’humanité. Elle n’a pas d’application en politique. La conscience instinctive qui a présidé à la confection de la carte d’{\placeName Europe} n’a tenu aucun compte de la race, et les premières nations de l’{\placeName Europe} sont des nations de sang essentiellement mélangé.\par
Le fait de la race, capital à l’origine, va donc toujours perdant de son importance. L’histoire humaine diffère essentiellement de la zoologie. La race n’y est pas tout, comme chez les rongeurs ou les félins, et on n’a pas le droit d’aller par le monde tâter le crâne des gens, puis les prendre à la gorge en leur disant : « Tu es notre sang ; tu nous appartiens ! » En dehors des caractères anthropologiques, il y a la raison, la justice, le vrai, le beau, qui sont les mêmes pour tous. Tenez, cette politique ethnographique n’est pas sûre. Vous l’exploitez aujourd’hui contre les autres ; puis vous la voyez se tourner contre vous-mêmes. Est-il certain que les {\orgName Allemands}, qui ont élevé si haut le drapeau de l’ethnographie, ne verront pas les {\orgName Slaves} venir analyser, à leur tour, les noms des villages de la {\placeName Saxe} et de la {\placeName Lusace}, rechercher les traces des {\orgName Wiltzes} ou des {\orgName Obotrites}, et demander compte des massacres et des ventes en masse que les {\orgName Othons} firent de leurs aïeux ? Pour tous il est bon de savoir oublier.\par
J’aime beaucoup l’ethnographie ; c’est une science d’un rare intérêt ; mais, comme je la veux libre, je la veux sans application politique. En ethnographie, comme dans toutes les études, les systèmes changent ; c’est la condition du progrès. Les limites des États suivraient les fluctuations de la science. Le patriotisme dépendrait d’une dissertation plus ou moins paradoxale. On viendrait dire au patriote : « Vous vous trompiez ; vous versiez votre sang pour telle cause ; vous croyiez être Celte ; non, vous êtes Germain. » Puis, dix ans après, on viendra vous dire que vous êtes Slave. Pour ne pas fausser la science, dispensons-la de donner un avis dans ces problèmes, où sont engagés tant d’intérêts. Soyez sûrs que, si on la charge de fournir des éléments à la diplomatie, on la surprendra bien des fois en flagrant délit de complaisance. Elle a mieux à faire : demandons-lui tout simplement la vérité.

\section[{2. — Ce que nous venons de dire de la race, il faut le dire de la langue.}]{2. — Ce que nous venons de dire de la race, il faut le dire de la langue.}

\noindent La langue invite à se réunir ; elle n’y force pas. Les {\orgName États-Unis} et l’{\orgName Angleterre}, l’{\orgName Amérique espagnole} et l’{\orgName Espagne} parlent la même langue et ne forment pas une seule nation. Au contraire, la {\orgName Suisse}, si bien faite, puisqu’elle a été faite par l’assentiment de ses différentes parties, compte trois ou quatre langues. Il y a dans l’homme quelque chose de supérieur à la langue : c’est la volonté. La volonté de la {\orgName Suisse} d’être unie, malgré la variété de ces idiomes, est un fait bien plus important qu’une similitude souvent obtenue par des vexations.\par
Un fait honorable pour la {\orgName France}, c’est qu’elle n’a jamais cherché à obtenir l’unité de la langue par des mesures de coercition. Ne peut-on pas avoir les mêmes sentiments et les mêmes pensées, aimer les mêmes choses en des langages différents ? Nous parlions tout à l’heure de l’inconvénient qu’il y aurait à faire dépendre la politique internationale de l’ethnographie. Il n’y en aurait pas moins à la faire dépendre de la philologie comparée. Laissons à ces intéressantes études l’entière liberté de leurs discussions ; ne les mêlons pas à ce qui en altérerait la sérénité. L’importance politique qu’on attache aux langues vient de ce qu’on les regarde comme des signes de race. Rien de plus faux. La {\orgName Prusse}, où l’on ne parle plus qu’allemand, parlait slave il y a quelques siècles ; le {\orgName pays de Galles} parle anglais ; la {\orgName Gaule} et l’{\orgName Espagne} parlent l’idiome primitif d’{\orgName Albe la Longue} ; l’{\orgName Égypte} parle arabe ; les exemples sont innombrables. Même aux origines, la similitude de langue n’entraînait pas la similitude de race. Prenons la {\orgName tribu proto-aryenne} ou proto-sémite ; il s’y trouvait des esclaves, qui parlaient la même langue que leurs maîtres ; or l’esclave était alors bien souvent d’une race différente de celle de son maître. Répétons-le : ces divisions de langues indo-européennes, sémitiques et autres, créées avec une si admirable sagacité par la philologie comparée, ne coïncident pas avec les divisions de l’anthropologie. Les langues sont des formations historiques, qui indiquent peu de choses sur le sang de ceux qui les parlent, et qui, en tout cas, ne sauraient enchaîner la liberté humaine quand il s’agit de déterminer la famille avec laquelle on s’unit pour la vie et pour la mort.\par
Cette considération exclusive de la langue a, comme l’attention trop forte donnée à la race, ses dangers, ses inconvénients. Quand on y met de l’exagération, on se renferme dans une culture déterminée, tenue pour nationale ; on se limite, on se claquemure. On quitte le grand air qu’on respire dans le vaste champ de l’humanité pour s’enfermer dans des conventicules de compatriotes. Rien de plus mauvais pour l’esprit ; rien de plus fâcheux pour la civilisation. N’abandonnons pas ce principe fondamental, que l’homme est un être raisonnable et moral, avant d’être parqué dans telle ou telle langue, avant d’être un membre de telle ou telle race, un adhérent de telle ou telle culture. Avant la culture française, la culture allemande, la culture italienne, il y a la culture humaine. Voyez les grands hommes de la Renaissance ; ils n’étaient ni Français, ni Italiens, ni Allemands. Ils avaient retrouvé, par leur commerce avec l’antiquité, le secret de l’éducation véritable de l’esprit humain, et ils s’y dévouaient corps et âme. Comme ils firent bien !

\section[{3. — La religion ne saurait non plus offrir une base suffisante à l’établissement d’une nationalité moderne.}]{3. — La religion ne saurait non plus offrir une base suffisante à l’établissement d’une nationalité moderne.}

\noindent À l’origine, la religion tenait à l’existence même du groupe social. Le groupe social était une extension de la famille. La religion, les rites étaient des rites de famille. La religion d’{\placeName Athènes}, c’était le culte d’{\placeName Athènes} même, de ses fondateurs mythiques, de ses lois, de ses usages. Elle n’impliquait aucune théologie dogmatique. Cette religion était, dans toute la force du terme, une religion d’État. On n’était pas Athénien si on refusait de la pratiquer. C’était au fond le culte de l’Acropole personnifiée. Jurer sur l’autel d’{\persName Aglaure} \footnote{{\persName Aglaure}, c’est l’Acropole elle-même, qui s’est dévouée pour sauver la patrie.}, c’était prêter le serment de mourir pour la patrie. Cette religion était l’équivalent de ce qu’est chez nous l’acte de tirer au sort, ou le culte du drapeau. Refuser de participer à un tel culte était comme serait dans nos sociétés modernes refuser le service militaire. C’était déclarer qu’on n’était pas Athénien. D’un autre côté, il est clair qu’un tel culte n’avait pas de sens pour celui qui n’était pas d’{\placeName Athènes} ; aussi n’exerçait-on aucun prosélytisme pour forcer des étrangers à l’accepter ; les esclaves d’{\placeName Athènes} ne le pratiquaient pas. Il en fut de même dans quelques petites républiques du moyen âge. On n’était pas bon Vénitien si l’on ne jurait point par {\persName saint Marc} ; on n’était pas bon Amalfitain si l’on ne mettait pas {\persName saint André} au-dessus de tous les autres saints du paradis. Dans ces petites sociétés, ce qui a été plus tard persécution, tyrannie, était légitime et tirait aussi peu à conséquence que le fait chez nous de souhaiter la fête au père de famille et de lui adresser des vœux au premier jour de l’an.\par
Ce qui était vrai à {\placeName Sparte}, à {\placeName Athènes}, ne l’était déjà plus dans les royaumes sortis de la conquête d’{\persName Alexandre}, ne l’était surtout plus dans l’{\placeName Empire romain}. Les persécutions d’{\persName Antiochus Épiphane} pour amener l’{\placeName Orient} au culte de {\persName Jupiter Olympien}, celles de l’{\placeName Empire romain} pour maintenir une prétendue religion d’État furent une faute, un crime, une véritable absurdité. De nos jours, la situation est parfaitement claire. Il n’y a plus de masses croyant d’une manière uniforme. Chacun croit et pratique à sa guise, ce qu’il peut, comme il veut. Il n’y a plus de religion d’État ; on peut être Français, Anglais, Allemand, en étant catholique, protestant, israélite, en ne pratiquant aucun culte. La religion est devenue chose individuelle ; elle regarde la conscience de chacun. La division des nations en catholiques, protestantes, n’existe plus. La religion, qui, il y a cinquante-deux ans, était un élément si considérable dans la formation de la {\placeName Belgique}, garde toute son importance dans le for intérieur de chacun ; mais elle est sortie presque entièrement des raisons qui tracent les limites des peuples.

\section[{4. — La communauté des intérêts est assurément un lien puissant entre les hommes.}]{4. — La communauté des intérêts est assurément un lien puissant entre les hommes.}

\noindent Les intérêts, cependant, suffisent-ils à faire une nation ? je ne le crois pas. La communauté des intérêts fait les traités de commerce. Il y a dans la nationalité un côté de sentiment ; elle est âme et corps à la fois ; un \foreign{Zollverein\footnote{« union douanière allemande » (de 1834 à l’unification allemande en 1871)}} n’est pas une patrie.

\section[{V. — La géographie, ce qu’on appelle les frontières naturelles, a certainement une part considérable dans la division des nations.}]{V. — La géographie, ce qu’on appelle les frontières naturelles, a certainement une part considérable dans la division des nations.}

\noindent La géographie est un des facteurs essentiels de l’histoire. Les rivières ont conduit les races ; les montagnes les ont arrêtées. Les premières ont favorisé, les secondes ont limité les mouvements historiques. Peut-on dire cependant, comme le croient certains partis, que les limites d’une nation sont écrites sur la carte et que cette nation a le droit de s’adjuger ce qui est nécessaire pour arrondir certains contours, pour atteindre telle montagne, telle rivière, à laquelle on prête une sorte de faculté limitante à priori ? je ne connais pas de doctrine plus arbitraire ni plus funeste. Avec cela, on justifie toutes les violences. Et, d’abord, sont-ce les montagnes ou bien sont-ce les rivières qui forment ces prétendues frontières naturelles ? Il est incontestable que les montagnes séparent ; mais les fleuves réunissent plutôt. Et puis toutes les montagnes ne sauraient découper des États. Quelles sont celles qui séparent et celles qui ne séparent pas ? De {\placeName Biarritz} à {\placeName Tornea}, il n’y a pas une embouchure de fleuve qui ait plus qu’une autre un caractère bornal. Si l’histoire l’avait voulu, la {\placeName Loire}, la {\placeName Seine}, la {\placeName Meuse}, l’{\placeName Elbe}, l’{\placeName Oder} auraient, autant que le {\placeName Rhin}, ce caractère de frontière naturelle qui a fait commettre tant d’infractions au droit fondamental, qui est la volonté des hommes. On parle de raisons stratégiques. Rien n’est absolu ; il est clair que bien des concessions doivent être faites à la nécessité. Mais il ne faut pas que ces concessions aillent trop loin. Autrement, tout le monde réclamera ses convenances militaires, et ce sera la guerre sans fin. Non, ce n’est pas la terre plus que la race qui fait une nation. La terre fournit le substratum, le champ de la lutte et du travail ; l’homme fournit l’âme. L’homme est tout dans la formation de cette chose sacrée qu’on appelle un peuple. Rien de matériel n’y suffit. Une nation est un principe spirituel, résultant des complications profondes de l’histoire, une famille spirituelle, non un groupe déterminé par la configuration du sol.\par
Nous venons de voir ce qui ne suffit pas à créer un tel principe spirituel : la race, la langue, les intérêts, l’affinité religieuse, la géographie, les nécessités militaires. Que faut-il donc en plus ? Par suite de ce qui a été dit antérieurement, je n’aurai pas désormais à retenir bien longtemps votre attention.
\chapterclose


\chapteropen

\chapter[{III. [Ce qu’une nation est]}]{III. [Ce qu’une nation est]}
\renewcommand{\leftmark}{III. [Ce qu’une nation est]}


\chaptercont
\noindent Une nation est une âme, un principe spirituel. Deux choses qui, à vrai dire, n’en font qu’une, constituent cette âme, ce principe spirituel. L’une est dans le passé, l’autre dans le présent. L’une est la possession en commun d’un riche legs de souvenirs ; l’autre est le consentement actuel, le désir de vivre ensemble, la volonté de continuer à faire valoir l’héritage qu’on a reçu indivis. L’homme, Messieurs, ne s’improvise pas. La nation, comme l’individu, est l’aboutissant d’un long passé d’efforts, de sacrifices et de dévouements. Le culte des ancêtres est de tous le plus légitime ; les ancêtres nous ont faits ce que nous sommes. Un passé héroïque, des grands hommes, de la gloire (j’entends de la véritable), voilà le capital social sur lequel on assied une idée nationale. Avoir des gloires communes dans le passé, une volonté commune dans le présent ; avoir fait de grandes choses ensemble, vouloir en faire encore, voilà les conditions essentielles pour être un peuple. On aime en proportion des sacrifices qu’on a consentis, des maux qu’on a soufferts. On aime la maison qu’on a bâtie et qu’on transmet. Le chant spartiate : « Nous sommes ce que vous fûtes ; nous serons ce que vous êtes » est dans sa simplicité l’hymne abrégé de toute patrie.\par
Dans le passé, un héritage de gloire et de regrets à partager, dans l’avenir un même programme à réaliser ; avoir souffert joui, espéré ensemble, voilà ce qui vaut mieux que des douanes communes et des frontières conformes aux idées stratégiques ; voilà ce que l’on comprend malgré les diversités de race et de langue. Je disais tout à l’heure : « avoir souffert ensemble » ; oui, la souffrance en commun unit plus que la joie. En fait de souvenirs nationaux, les deuils valent mieux que les triomphes, car ils imposent des devoirs, ils commandent l’effort en commun.\par
Une nation est donc une grande solidarité, constituée par le sentiment des sacrifices qu’on a faits et de ceux qu’on est disposé à faire encore. Elle suppose un passé ; elle se résume pourtant dans le présent par un fait tangible : le consentement, le désir clairement exprimé de continuer la vie commune. L’existence d’une nation est (pardonnez-moi cette métaphore) un plébiscite de tous les jours, comme l’existence de l’individu est une affirmation perpétuelle de vie. Oh ! je le sais, cela est moins métaphysique que le droit divin, moins brutal que le droit prétendu historique. Dans l’ordre d’idées que je vous soumets, une nation n’a pas plus qu’un roi le droit de dire à une province : « Tu m’appartiens, je te prends. » Une province, pour nous, ce sont ses habitants ; si quelqu’un en cette affaire a droit d’être consulté, c’est l’habitant. Une nation n’a jamais un véritable intérêt à s’annexer ou à retenir un pays malgré lui. Le vœu des nations est, en définitive, le seul critérium légitime, celui auquel il faut toujours en revenir.\par
Nous avons chassé de la politique les abstractions métaphysiques et théologiques. Que reste-t-il, après cela ? Il reste l’homme, ses désirs, ses besoins. La sécession, me direz-vous, et, à la longue, l’émiettement des nations sont la conséquence d’un système qui met ces vieux organismes à la merci de volontés souvent peu éclairées. Il est clair qu’en pareille matière aucun principe ne doit être poussé à l’excès. Les vérités de cet ordre ne sont applicables que dans leur ensemble et d’une façon très générale. Les volontés humaines changent ; mais qu’est-ce qui ne change pas ici-bas ? Les nations ne sont pas quelque chose d’éternel. Elles ont commencé, elles finiront. La confédération européenne, probablement, les remplacera. Mais telle n’est pas la loi du siècle où nous vivons. À l’heure présente, l’existence des nations est bonne, nécessaire même. Leur existence est la garantie de la liberté, qui serait perdue si le monde n’avait qu’une loi et qu’un maître.\par
Par leurs facultés diverses, souvent opposées, les nations servent à l’œuvre commune de la civilisation ; toutes apportent une note à ce grand concert de l’humanité, qui, en somme, est la plus haute réalité idéale que nous atteignions. Isolées, elles ont leurs parties faibles. Je me dis souvent qu’un individu qui aurait les défauts tenus chez les nations pour des qualités, qui se nourrirait de vaine gloire ; qui serait à ce point jaloux, égoïste, querelleur ; qui ne pourrait rien supporter sans dégainer, serait le plus insupportable des hommes. Mais toutes ces dissonances de détail disparaissent dans l’ensemble. Pauvre humanité, que tu as souffert ! que d’épreuves t’attendent encore ! Puisse l’esprit de sagesse te guider pour te préserver des innombrables dangers dont ta route est semée !\par
Je me résume, Messieurs. L’homme n’est esclave ni de sa race, ni de sa langue, ni de sa religion, ni du cours des fleuves, ni de la direction des chaînes de montagnes. Une grande agrégation d’hommes, saine d’esprit et chaude de cœur, crée une conscience morale qui s’appelle une nation. Tant que cette conscience morale prouve sa force par les sacrifices qu’exige l’abdication de l’individu au profit d’une communauté, elle est légitime, elle a le droit d’exister. Si des doutes s’élèvent sur ses frontières, consultez les populations disputées. Elles ont bien le droit d’avoir un avis dans la question. Voilà qui fera sourire les transcendants de la politique, ces infaillibles qui passent leur vie à se tromper et qui, du haut de leurs principes supérieurs, prennent en pitié notre terre à terre. « Consulter les populations, fi donc ! quelle naïveté ! Voilà bien ces chétives idées françaises qui prétendent remplacer la diplomatie et la guerre par des moyens \& une simplicité enfantine. » — Attendons, Messieurs ; laissons passer le règne des transcendants ; sachons subir le dédain des forts. Peut-être, après bien des tâtonnements infructueux, reviendra-t-on à nos modestes solutions empiriques. Le moyen d’avoir raison dans l’avenir est, à certaines heures, de savoir se résigner à être démodé.
\chapterclose

 


% at least one empty page at end (for booklet couv)
\ifbooklet
  \pagestyle{empty}
  \clearpage
  % 2 empty pages maybe needed for 4e cover
  \ifnum\modulo{\value{page}}{4}=0 \hbox{}\newpage\hbox{}\newpage\fi
  \ifnum\modulo{\value{page}}{4}=1 \hbox{}\newpage\hbox{}\newpage\fi


  \hbox{}\newpage
  \ifodd\value{page}\hbox{}\newpage\fi
  {\centering\color{rubric}\bfseries\noindent\large
    Hurlus ? Qu’est-ce.\par
    \bigskip
  }
  \noindent Des bouquinistes électroniques, pour du texte libre à participations libres,
  téléchargeable gratuitement sur \href{https://hurlus.fr}{\dotuline{hurlus.fr}}.\par
  \bigskip
  \noindent Cette brochure a été produite par des éditeurs bénévoles.
  Elle n’est pas faite pour être possédée, mais pour être lue, et puis donnée.
  En page de garde, on peut ajouter une date, un lieu, un nom ;
  comme une fiche de bibliothèque en papier,
  pour suivre le voyage du texte. Qui sait, un jour, il vous reviendra ?
  \par

  Ce texte a été choisi parce qu’une personne l’a aimé,
  ou haï, elle a pensé qu’il partipait à la formation de notre présent ;
  sans le souci de plaire, vendre, ou militer pour une cause.
  \par

  L’édition électronique est soigneuse, tant sur la technique
  que sur l’établissement du texte ; mais sans aucune prétention scolaire, au contraire.
  Le but est de s’adresser à tous, sans distinction de science ou de diplôme.
  \par

  Cet exemplaire en papier a été tiré sur une imprimante personnelle
   ou une photocopieuse. Tout le monde peut le faire.
  Il suffit de
  télécharger un fichier sur \href{https://hurlus.fr}{\dotuline{hurlus.fr}},
  d’imprimer, et agrafer (puis lire et donner).\par

  \bigskip

  \noindent PS : Les hurlus furent aussi des rebelles protestants qui cassaient les statues dans les églises catholiques. En 1566 démarra la révolte des gueux dans le pays de Lille. L’insurrection enflamma la région jusqu’à Anvers où les gueux de mer bloquèrent les bateaux espagnols.
  Ce fut une rare guerre de libération dont naquit un pays toujours libre : les Pays-Bas.
  En plat pays francophone, par contre, restèrent des bandes de huguenots, les hurlus, progressivement réprimés par la très catholique Espagne.
  Cette mémoire d’une défaite est éteinte, rallumons-la. Sortons les livres du culte universitaire, débusquons les idoles de l’époque, pour les démonter.
\fi

\end{document}
