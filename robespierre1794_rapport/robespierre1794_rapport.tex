%%%%%%%%%%%%%%%%%%%%%%%%%%%%%%%%%
% LaTeX model https://hurlus.fr %
%%%%%%%%%%%%%%%%%%%%%%%%%%%%%%%%%

% Needed before document class
\RequirePackage{pdftexcmds} % needed for tests expressions
\RequirePackage{fix-cm} % correct units

% Define mode
\def\mode{a4}

\newif\ifaiv % a4
\newif\ifav % a5
\newif\ifbooklet % booklet
\newif\ifcover % cover for booklet

\ifnum \strcmp{\mode}{cover}=0
  \covertrue
\else\ifnum \strcmp{\mode}{booklet}=0
  \booklettrue
\else\ifnum \strcmp{\mode}{a5}=0
  \avtrue
\else
  \aivtrue
\fi\fi\fi

\ifbooklet % do not enclose with {}
  \documentclass[french,twoside]{book} % ,notitlepage
  \usepackage[%
    papersize={105mm, 297mm},
    inner=12mm,
    outer=12mm,
    top=20mm,
    bottom=15mm,
    marginparsep=0pt,
  ]{geometry}
  \usepackage[fontsize=9.5pt]{scrextend} % for Roboto
\else\ifav
  \documentclass[french,twoside]{book} % ,notitlepage
  \usepackage[%
    a5paper,
    inner=25mm,
    outer=15mm,
    top=15mm,
    bottom=15mm,
    marginparsep=0pt,
  ]{geometry}
  \usepackage[fontsize=12pt]{scrextend}
\else% A4 2 cols
  \documentclass[twocolumn]{report}
  \usepackage[%
    a4paper,
    inner=15mm,
    outer=10mm,
    top=25mm,
    bottom=18mm,
    marginparsep=0pt,
  ]{geometry}
  \setlength{\columnsep}{20mm}
  \usepackage[fontsize=9.5pt]{scrextend}
\fi\fi

%%%%%%%%%%%%%%
% Alignments %
%%%%%%%%%%%%%%
% before teinte macros

\setlength{\arrayrulewidth}{0.2pt}
\setlength{\columnseprule}{\arrayrulewidth} % twocol
\setlength{\parskip}{0pt} % classical para with no margin
\setlength{\parindent}{1.5em}

%%%%%%%%%%
% Colors %
%%%%%%%%%%
% before Teinte macros

\usepackage[dvipsnames]{xcolor}
\definecolor{rubric}{HTML}{800000} % the tonic 0c71c3
\def\columnseprulecolor{\color{rubric}}
\colorlet{borderline}{rubric!30!} % definecolor need exact code
\definecolor{shadecolor}{gray}{0.95}
\definecolor{bghi}{gray}{0.5}

%%%%%%%%%%%%%%%%%
% Teinte macros %
%%%%%%%%%%%%%%%%%
%%%%%%%%%%%%%%%%%%%%%%%%%%%%%%%%%%%%%%%%%%%%%%%%%%%
% <TEI> generic (LaTeX names generated by Teinte) %
%%%%%%%%%%%%%%%%%%%%%%%%%%%%%%%%%%%%%%%%%%%%%%%%%%%
% This template is inserted in a specific design
% It is XeLaTeX and otf fonts

\makeatletter % <@@@


\usepackage{blindtext} % generate text for testing
\usepackage[strict]{changepage} % for modulo 4
\usepackage{contour} % rounding words
\usepackage[nodayofweek]{datetime}
% \usepackage{DejaVuSans} % seems buggy for sffont font for symbols
\usepackage{enumitem} % <list>
\usepackage{etoolbox} % patch commands
\usepackage{fancyvrb}
\usepackage{fancyhdr}
\usepackage{float}
\usepackage{fontspec} % XeLaTeX mandatory for fonts
\usepackage{footnote} % used to capture notes in minipage (ex: quote)
\usepackage{framed} % bordering correct with footnote hack
\usepackage{graphicx}
\usepackage{lettrine} % drop caps
\usepackage{lipsum} % generate text for testing
\usepackage[framemethod=tikz,]{mdframed} % maybe used for frame with footnotes inside
\usepackage{pdftexcmds} % needed for tests expressions
\usepackage{polyglossia} % non-break space french punct, bug Warning: "Failed to patch part"
\usepackage[%
  indentfirst=false,
  vskip=1em,
  noorphanfirst=true,
  noorphanafter=true,
  leftmargin=\parindent,
  rightmargin=0pt,
]{quoting}
\usepackage{ragged2e}
\usepackage{setspace} % \setstretch for <quote>
\usepackage{tabularx} % <table>
\usepackage[explicit]{titlesec} % wear titles, !NO implicit
\usepackage{tikz} % ornaments
\usepackage{tocloft} % styling tocs
\usepackage[fit]{truncate} % used im runing titles
\usepackage{unicode-math}
\usepackage[normalem]{ulem} % breakable \uline, normalem is absolutely necessary to keep \emph
\usepackage{verse} % <l>
\usepackage{xcolor} % named colors
\usepackage{xparse} % @ifundefined
\XeTeXdefaultencoding "iso-8859-1" % bad encoding of xstring
\usepackage{xstring} % string tests
\XeTeXdefaultencoding "utf-8"
\PassOptionsToPackage{hyphens}{url} % before hyperref, which load url package

% TOTEST
% \usepackage{hypcap} % links in caption ?
% \usepackage{marginnote}
% TESTED
% \usepackage{background} % doesn’t work with xetek
% \usepackage{bookmark} % prefers the hyperref hack \phantomsection
% \usepackage[color, leftbars]{changebar} % 2 cols doc, impossible to keep bar left
% \usepackage[utf8x]{inputenc} % inputenc package ignored with utf8 based engines
% \usepackage[sfdefault,medium]{inter} % no small caps
% \usepackage{firamath} % choose firasans instead, firamath unavailable in Ubuntu 21-04
% \usepackage{flushend} % bad for last notes, supposed flush end of columns
% \usepackage[stable]{footmisc} % BAD for complex notes https://texfaq.org/FAQ-ftnsect
% \usepackage{helvet} % not for XeLaTeX
% \usepackage{multicol} % not compatible with too much packages (longtable, framed, memoir…)
% \usepackage[default,oldstyle,scale=0.95]{opensans} % no small caps
% \usepackage{sectsty} % \chapterfont OBSOLETE
% \usepackage{soul} % \ul for underline, OBSOLETE with XeTeX
% \usepackage[breakable]{tcolorbox} % text styling gone, footnote hack not kept with breakable


% Metadata inserted by a program, from the TEI source, for title page and runing heads
\title{\textbf{ Rapport du 7 mai 1794 }}
\date{1794}
\author{Robespierre, Maximilien (1758-1794)}
\def\elbibl{Robespierre, Maximilien (1758-1794). 1794. \emph{Rapport du 7 mai 1794}}
\def\elsource{Maximilien Robespierre, « Rapport fait au nom du Comité de salut public, par Maximilien Robespierre, sur les rapports des idées religieuses et morales avec les principes républicains, et sur les fêtes nationales. Séance du 18 floréal, l’an second de la République française une et indivisible. Imprimé par ordre de la Convention nationale (18 floréal an II — 7 mai 1794) », in {\itshape Discours par Maximilien Robespierre (17 avril 1792-27 juillet 1794)}, texte en français moderne établi par Charles Vellay. HTML : \href{http://www.gutenberg.org/files/29887/29887-h/29887-h.htm}{\dotuline{Gutenberg}}\footnote{\href{http://www.gutenberg.org/files/29887/29887-h/29887-h.htm}{\url{http://www.gutenberg.org/files/29887/29887-h/29887-h.htm}}}.}

% Default metas
\newcommand{\colorprovide}[2]{\@ifundefinedcolor{#1}{\colorlet{#1}{#2}}{}}
\colorprovide{rubric}{red}
\colorprovide{silver}{lightgray}
\@ifundefined{syms}{\newfontfamily\syms{DejaVu Sans}}{}
\newif\ifdev
\@ifundefined{elbibl}{% No meta defined, maybe dev mode
  \newcommand{\elbibl}{Titre court ?}
  \newcommand{\elbook}{Titre du livre source ?}
  \newcommand{\elabstract}{Résumé\par}
  \newcommand{\elurl}{http://oeuvres.github.io/elbook/2}
  \author{Éric Lœchien}
  \title{Un titre de test assez long pour vérifier le comportement d’une maquette}
  \date{1566}
  \devtrue
}{}
\let\eltitle\@title
\let\elauthor\@author
\let\eldate\@date


\defaultfontfeatures{
  % Mapping=tex-text, % no effect seen
  Scale=MatchLowercase,
  Ligatures={TeX,Common},
}


% generic typo commands
\newcommand{\astermono}{\medskip\centerline{\color{rubric}\large\selectfont{\syms ✻}}\medskip\par}%
\newcommand{\astertri}{\medskip\par\centerline{\color{rubric}\large\selectfont{\syms ✻\,✻\,✻}}\medskip\par}%
\newcommand{\asterism}{\bigskip\par\noindent\parbox{\linewidth}{\centering\color{rubric}\large{\syms ✻}\\{\syms ✻}\hskip 0.75em{\syms ✻}}\bigskip\par}%

% lists
\newlength{\listmod}
\setlength{\listmod}{\parindent}
\setlist{
  itemindent=!,
  listparindent=\listmod,
  labelsep=0.2\listmod,
  parsep=0pt,
  % topsep=0.2em, % default topsep is best
}
\setlist[itemize]{
  label=—,
  leftmargin=0pt,
  labelindent=1.2em,
  labelwidth=0pt,
}
\setlist[enumerate]{
  label={\bf\color{rubric}\arabic*.},
  labelindent=0.8\listmod,
  leftmargin=\listmod,
  labelwidth=0pt,
}
\newlist{listalpha}{enumerate}{1}
\setlist[listalpha]{
  label={\bf\color{rubric}\alph*.},
  leftmargin=0pt,
  labelindent=0.8\listmod,
  labelwidth=0pt,
}
\newcommand{\listhead}[1]{\hspace{-1\listmod}\emph{#1}}

\renewcommand{\hrulefill}{%
  \leavevmode\leaders\hrule height 0.2pt\hfill\kern\z@}

% General typo
\DeclareTextFontCommand{\textlarge}{\large}
\DeclareTextFontCommand{\textsmall}{\small}

% commands, inlines
\newcommand{\anchor}[1]{\Hy@raisedlink{\hypertarget{#1}{}}} % link to top of an anchor (not baseline)
\newcommand\abbr[1]{#1}
\newcommand{\autour}[1]{\tikz[baseline=(X.base)]\node [draw=rubric,thin,rectangle,inner sep=1.5pt, rounded corners=3pt] (X) {\color{rubric}#1};}
\newcommand\corr[1]{#1}
\newcommand{\ed}[1]{ {\color{silver}\sffamily\footnotesize (#1)} } % <milestone ed="1688"/>
\newcommand\expan[1]{#1}
\newcommand\foreign[1]{\emph{#1}}
\newcommand\gap[1]{#1}
\renewcommand{\LettrineFontHook}{\color{rubric}}
\newcommand{\initial}[2]{\lettrine[lines=2, loversize=0.3, lhang=0.3]{#1}{#2}}
\newcommand{\initialiv}[2]{%
  \let\oldLFH\LettrineFontHook
  % \renewcommand{\LettrineFontHook}{\color{rubric}\ttfamily}
  \IfSubStr{QJ’}{#1}{
    \lettrine[lines=4, lhang=0.2, loversize=-0.1, lraise=0.2]{\smash{#1}}{#2}
  }{\IfSubStr{É}{#1}{
    \lettrine[lines=4, lhang=0.2, loversize=-0, lraise=0]{\smash{#1}}{#2}
  }{\IfSubStr{ÀÂ}{#1}{
    \lettrine[lines=4, lhang=0.2, loversize=-0, lraise=0, slope=0.6em]{\smash{#1}}{#2}
  }{\IfSubStr{A}{#1}{
    \lettrine[lines=4, lhang=0.2, loversize=0.2, slope=0.6em]{\smash{#1}}{#2}
  }{\IfSubStr{V}{#1}{
    \lettrine[lines=4, lhang=0.2, loversize=0.2, slope=-0.5em]{\smash{#1}}{#2}
  }{
    \lettrine[lines=4, lhang=0.2, loversize=0.2]{\smash{#1}}{#2}
  }}}}}
  \let\LettrineFontHook\oldLFH
}
\newcommand{\labelchar}[1]{\textbf{\color{rubric} #1}}
\newcommand{\milestone}[1]{\autour{\footnotesize\color{rubric} #1}} % <milestone n="4"/>
\newcommand\name[1]{#1}
\newcommand\orig[1]{#1}
\newcommand\orgName[1]{#1}
\newcommand\persName[1]{#1}
\newcommand\placeName[1]{#1}
\newcommand{\pn}[1]{\IfSubStr{-—–¶}{#1}% <p n="3"/>
  {\noindent{\bfseries\color{rubric}   ¶  }}
  {{\footnotesize\autour{ #1}  }}}
\newcommand\reg{}
% \newcommand\ref{} % already defined
\newcommand\sic[1]{#1}
\newcommand\surname[1]{\textsc{#1}}
\newcommand\term[1]{\textbf{#1}}

\def\mednobreak{\ifdim\lastskip<\medskipamount
  \removelastskip\nopagebreak\medskip\fi}
\def\bignobreak{\ifdim\lastskip<\bigskipamount
  \removelastskip\nopagebreak\bigskip\fi}

% commands, blocks
\newcommand{\byline}[1]{\bigskip{\RaggedLeft{#1}\par}\bigskip}
\newcommand{\bibl}[1]{{\RaggedLeft{#1}\par\bigskip}}
\newcommand{\biblitem}[1]{{\noindent\hangindent=\parindent   #1\par}}
\newcommand{\dateline}[1]{\medskip{\RaggedLeft{#1}\par}\bigskip}
\newcommand{\labelblock}[1]{\medbreak{\noindent\color{rubric}\bfseries #1}\par\mednobreak}
\newcommand{\salute}[1]{\bigbreak{#1}\par\medbreak}
\newcommand{\signed}[1]{\bigbreak\filbreak{\raggedleft #1\par}\medskip}

% environments for blocks (some may become commands)
\newenvironment{borderbox}{}{} % framing content
\newenvironment{citbibl}{\ifvmode\hfill\fi}{\ifvmode\par\fi }
\newenvironment{docAuthor}{\ifvmode\vskip4pt\fontsize{16pt}{18pt}\selectfont\fi\itshape}{\ifvmode\par\fi }
\newenvironment{docDate}{}{\ifvmode\par\fi }
\newenvironment{docImprint}{\vskip6pt}{\ifvmode\par\fi }
\newenvironment{docTitle}{\vskip6pt\bfseries\fontsize{18pt}{22pt}\selectfont}{\par }
\newenvironment{msHead}{\vskip6pt}{\par}
\newenvironment{msItem}{\vskip6pt}{\par}
\newenvironment{titlePart}{}{\par }


% environments for block containers
\newenvironment{argument}{\itshape\parindent0pt}{\vskip1.5em}
\newenvironment{biblfree}{}{\ifvmode\par\fi }
\newenvironment{bibitemlist}[1]{%
  \list{\@biblabel{\@arabic\c@enumiv}}%
  {%
    \settowidth\labelwidth{\@biblabel{#1}}%
    \leftmargin\labelwidth
    \advance\leftmargin\labelsep
    \@openbib@code
    \usecounter{enumiv}%
    \let\p@enumiv\@empty
    \renewcommand\theenumiv{\@arabic\c@enumiv}%
  }
  \sloppy
  \clubpenalty4000
  \@clubpenalty \clubpenalty
  \widowpenalty4000%
  \sfcode`\.\@m
}%
{\def\@noitemerr
  {\@latex@warning{Empty `bibitemlist' environment}}%
\endlist}
\newenvironment{quoteblock}% may be used for ornaments
  {\begin{quoting}}
  {\end{quoting}}

% table () is preceded and finished by custom command
\newcommand{\tableopen}[1]{%
  \ifnum\strcmp{#1}{wide}=0{%
    \begin{center}
  }
  \else\ifnum\strcmp{#1}{long}=0{%
    \begin{center}
  }
  \else{%
    \begin{center}
  }
  \fi\fi
}
\newcommand{\tableclose}[1]{%
  \ifnum\strcmp{#1}{wide}=0{%
    \end{center}
  }
  \else\ifnum\strcmp{#1}{long}=0{%
    \end{center}
  }
  \else{%
    \end{center}
  }
  \fi\fi
}


% text structure
\newcommand\chapteropen{} % before chapter title
\newcommand\chaptercont{} % after title, argument, epigraph…
\newcommand\chapterclose{} % maybe useful for multicol settings
\setcounter{secnumdepth}{-2} % no counters for hierarchy titles
\setcounter{tocdepth}{5} % deep toc
\markright{\@title} % ???
\markboth{\@title}{\@author} % ???
\renewcommand\tableofcontents{\@starttoc{toc}}
% toclof format
% \renewcommand{\@tocrmarg}{0.1em} % Useless command?
% \renewcommand{\@pnumwidth}{0.5em} % {1.75em}
\renewcommand{\@cftmaketoctitle}{}
\setlength{\cftbeforesecskip}{\z@ \@plus.2\p@}
\renewcommand{\cftchapfont}{}
\renewcommand{\cftchapdotsep}{\cftdotsep}
\renewcommand{\cftchapleader}{\normalfont\cftdotfill{\cftchapdotsep}}
\renewcommand{\cftchappagefont}{\bfseries}
\setlength{\cftbeforechapskip}{0em \@plus\p@}
% \renewcommand{\cftsecfont}{\small\relax}
\renewcommand{\cftsecpagefont}{\normalfont}
% \renewcommand{\cftsubsecfont}{\small\relax}
\renewcommand{\cftsecdotsep}{\cftdotsep}
\renewcommand{\cftsecpagefont}{\normalfont}
\renewcommand{\cftsecleader}{\normalfont\cftdotfill{\cftsecdotsep}}
\setlength{\cftsecindent}{1em}
\setlength{\cftsubsecindent}{2em}
\setlength{\cftsubsubsecindent}{3em}
\setlength{\cftchapnumwidth}{1em}
\setlength{\cftsecnumwidth}{1em}
\setlength{\cftsubsecnumwidth}{1em}
\setlength{\cftsubsubsecnumwidth}{1em}

% footnotes
\newif\ifheading
\newcommand*{\fnmarkscale}{\ifheading 0.70 \else 1 \fi}
\renewcommand\footnoterule{\vspace*{0.3cm}\hrule height \arrayrulewidth width 3cm \vspace*{0.3cm}}
\setlength\footnotesep{1.5\footnotesep} % footnote separator
\renewcommand\@makefntext[1]{\parindent 1.5em \noindent \hb@xt@1.8em{\hss{\normalfont\@thefnmark . }}#1} % no superscipt in foot
\patchcmd{\@footnotetext}{\footnotesize}{\footnotesize\sffamily}{}{} % before scrextend, hyperref


%   see https://tex.stackexchange.com/a/34449/5049
\def\truncdiv#1#2{((#1-(#2-1)/2)/#2)}
\def\moduloop#1#2{(#1-\truncdiv{#1}{#2}*#2)}
\def\modulo#1#2{\number\numexpr\moduloop{#1}{#2}\relax}

% orphans and widows
\clubpenalty=9996
\widowpenalty=9999
\brokenpenalty=4991
\predisplaypenalty=10000
\postdisplaypenalty=1549
\displaywidowpenalty=1602
\hyphenpenalty=400
% Copied from Rahtz but not understood
\def\@pnumwidth{1.55em}
\def\@tocrmarg {2.55em}
\def\@dotsep{4.5}
\emergencystretch 3em
\hbadness=4000
\pretolerance=750
\tolerance=2000
\vbadness=4000
\def\Gin@extensions{.pdf,.png,.jpg,.mps,.tif}
% \renewcommand{\@cite}[1]{#1} % biblio

\usepackage{hyperref} % supposed to be the last one, :o) except for the ones to follow
\urlstyle{same} % after hyperref
\hypersetup{
  % pdftex, % no effect
  pdftitle={\elbibl},
  % pdfauthor={Your name here},
  % pdfsubject={Your subject here},
  % pdfkeywords={keyword1, keyword2},
  bookmarksnumbered=true,
  bookmarksopen=true,
  bookmarksopenlevel=1,
  pdfstartview=Fit,
  breaklinks=true, % avoid long links
  pdfpagemode=UseOutlines,    % pdf toc
  hyperfootnotes=true,
  colorlinks=false,
  pdfborder=0 0 0,
  % pdfpagelayout=TwoPageRight,
  % linktocpage=true, % NO, toc, link only on page no
}

\makeatother % /@@@>
%%%%%%%%%%%%%%
% </TEI> end %
%%%%%%%%%%%%%%


%%%%%%%%%%%%%
% footnotes %
%%%%%%%%%%%%%
\renewcommand{\thefootnote}{\bfseries\textcolor{rubric}{\arabic{footnote}}} % color for footnote marks

%%%%%%%%%
% Fonts %
%%%%%%%%%
\usepackage[]{roboto} % SmallCaps, Regular is a bit bold
% \linespread{0.90} % too compact, keep font natural
\newfontfamily\fontrun[]{Roboto Condensed Light} % condensed runing heads
\ifav
  \setmainfont[
    ItalicFont={Roboto Light Italic},
  ]{Roboto}
\else\ifbooklet
  \setmainfont[
    ItalicFont={Roboto Light Italic},
  ]{Roboto}
\else
\setmainfont[
  ItalicFont={Roboto Italic},
]{Roboto Light}
\fi\fi
\renewcommand{\LettrineFontHook}{\bfseries\color{rubric}}
% \renewenvironment{labelblock}{\begin{center}\bfseries\color{rubric}}{\end{center}}

%%%%%%%%
% MISC %
%%%%%%%%

\setdefaultlanguage[frenchpart=false]{french} % bug on part


\newenvironment{quotebar}{%
    \def\FrameCommand{{\color{rubric!10!}\vrule width 0.5em} \hspace{0.9em}}%
    \def\OuterFrameSep{\itemsep} % séparateur vertical
    \MakeFramed {\advance\hsize-\width \FrameRestore}
  }%
  {%
    \endMakeFramed
  }
\renewenvironment{quoteblock}% may be used for ornaments
  {%
    \savenotes
    \setstretch{0.9}
    \normalfont
    \begin{quotebar}
  }
  {%
    \end{quotebar}
    \spewnotes
  }


\renewcommand{\headrulewidth}{\arrayrulewidth}
\renewcommand{\headrule}{{\color{rubric}\hrule}}

% delicate tuning, image has produce line-height problems in title on 2 lines
\titleformat{name=\chapter} % command
  [display] % shape
  {\vspace{1.5em}\centering} % format
  {} % label
  {0pt} % separator between n
  {}
[{\color{rubric}\huge\textbf{#1}}\bigskip] % after code
% \titlespacing{command}{left spacing}{before spacing}{after spacing}[right]
\titlespacing*{\chapter}{0pt}{-2em}{0pt}[0pt]

\titleformat{name=\section}
  [block]{}{}{}{}
  [\vbox{\color{rubric}\large\raggedleft\textbf{#1}}]
\titlespacing{\section}{0pt}{0pt plus 4pt minus 2pt}{\baselineskip}

\titleformat{name=\subsection}
  [block]
  {}
  {} % \thesection
  {} % separator \arrayrulewidth
  {}
[\vbox{\large\textbf{#1}}]
% \titlespacing{\subsection}{0pt}{0pt plus 4pt minus 2pt}{\baselineskip}

\ifaiv
  \fancypagestyle{main}{%
    \fancyhf{}
    \setlength{\headheight}{1.5em}
    \fancyhead{} % reset head
    \fancyfoot{} % reset foot
    \fancyhead[L]{\truncate{0.45\headwidth}{\fontrun\elbibl}} % book ref
    \fancyhead[R]{\truncate{0.45\headwidth}{ \fontrun\nouppercase\leftmark}} % Chapter title
    \fancyhead[C]{\thepage}
  }
  \fancypagestyle{plain}{% apply to chapter
    \fancyhf{}% clear all header and footer fields
    \setlength{\headheight}{1.5em}
    \fancyhead[L]{\truncate{0.9\headwidth}{\fontrun\elbibl}}
    \fancyhead[R]{\thepage}
  }
\else
  \fancypagestyle{main}{%
    \fancyhf{}
    \setlength{\headheight}{1.5em}
    \fancyhead{} % reset head
    \fancyfoot{} % reset foot
    \fancyhead[RE]{\truncate{0.9\headwidth}{\fontrun\elbibl}} % book ref
    \fancyhead[LO]{\truncate{0.9\headwidth}{\fontrun\nouppercase\leftmark}} % Chapter title, \nouppercase needed
    \fancyhead[RO,LE]{\thepage}
  }
  \fancypagestyle{plain}{% apply to chapter
    \fancyhf{}% clear all header and footer fields
    \setlength{\headheight}{1.5em}
    \fancyhead[L]{\truncate{0.9\headwidth}{\fontrun\elbibl}}
    \fancyhead[R]{\thepage}
  }
\fi

\ifav % a5 only
  \titleclass{\section}{top}
\fi

\newcommand\chapo{{%
  \vspace*{-3em}
  \centering % no vskip ()
  {\Large\addfontfeature{LetterSpace=25}\bfseries{\elauthor}}\par
  \smallskip
  {\large\eldate}\par
  \bigskip
  {\Large\selectfont{\eltitle}}\par
  \bigskip
  {\color{rubric}\hline\par}
  \bigskip
  {\Large TEXTE LIBRE À PARTICPATION LIBRE\par}
  \centerline{\small\color{rubric} {hurlus.fr, tiré le \today}}\par
  \bigskip
}}

\newcommand\cover{{%
  \thispagestyle{empty}
  \centering
  {\LARGE\bfseries{\elauthor}}\par
  \bigskip
  {\Large\eldate}\par
  \bigskip
  \bigskip
  {\LARGE\selectfont{\eltitle}}\par
  \vfill\null
  {\color{rubric}\setlength{\arrayrulewidth}{2pt}\hline\par}
  \vfill\null
  {\Large TEXTE LIBRE À PARTICPATION LIBRE\par}
  \centerline{{\href{https://hurlus.fr}{\dotuline{hurlus.fr}}, tiré le \today}}\par
}}

\begin{document}
\pagestyle{empty}
\ifbooklet{
  \cover\newpage
  \thispagestyle{empty}\hbox{}\newpage
  \cover\newpage\noindent Les voyages de la brochure\par
  \bigskip
  \begin{tabularx}{\textwidth}{l|X|X}
    \textbf{Date} & \textbf{Lieu}& \textbf{Nom/pseudo} \\ \hline
    \rule{0pt}{25cm} &  &   \\
  \end{tabularx}
  \newpage
  \addtocounter{page}{-4}
}\fi

\thispagestyle{empty}
\ifaiv
  \twocolumn[\chapo]
\else
  \chapo
\fi
{\it\elabstract}
\bigskip
\makeatletter\@starttoc{toc}\makeatother % toc without new page
\bigskip

\pagestyle{main} % after style

  \chapter[{Rapport fait au nom du Comité de salut public, par Maximilien Robespierre, sur les rapports des idées religieuses et morales avec les principes républicains, et sur les fêtes nationales}]{Rapport fait au nom du Comité de salut public, par Maximilien Robespierre, sur les rapports des idées religieuses et morales avec les principes républicains, et sur les fêtes nationales}
\noindent Citoyens,\par
C’est dans la prospérité que les peuples, ainsi que les particuliers, doivent, pour ainsi dire, se recueillir pour écouter, dans le silence des passions, la voix de la sagesse. Le moment où le bruit de nos victoires retentit dans l’univers est donc celui où les législateurs de la République française doivent veiller, avec une nouvelle sollicitude, sur eux-mêmes et sur la patrie, et affermir les principes sur lesquels doivent reposer la stabilité et la félicité de la République. Nous venons aujourd’hui soumettre à votre méditation des vérités profondes qui importent au bonheur des hommes, et vous proposer des mesures qui en découlent naturellement.\par
Le monde moral, beaucoup plus encore que le monde physique, semble plein de contrastes et d’énigmes. La nature nous dit que l’homme est né pour la liberté, et l’expérience des siècles nous montre l’homme esclave. Ses droits sont écrits dans son cœur, et son humiliation dans l’histoire. Le genre humain respecte Caton, et se courbe sous le joug de César. La postérité honore la vertu de Brutus ; mais elle ne la permet que dans l’histoire ancienne. Les siècles et la terre sont le partage du crime et de la tyrannie ; la liberté et la vertu se sont à peine reposées un instant sur quelques points du globe. Sparte brille comme un éclair dans des ténèbres immenses……\par
Ne dis pas cependant, ô Brutus, que la vertu est un fantôme ! Et vous, fondateurs de la République française, gardez-vous de désespérer de l’humanité, ou de douter un moment du succès de votre grande entreprise !\par
Le monde a changé, il doit changer encore. Qu’y a-t-il de commun entre ce qui est et ce qui fut ? Les nations civilisées ont succédé aux sauvages errants dans les déserts ; les moissons fertiles ont pris la place des forêts antiques qui couvraient le globe. Un monde a paru au-delà des bornes du monde ; les habitants de la terre ont ajouté les mers à leur domaine immense ; l’homme a conquis la foudre et conjuré celle du ciel. Comparez le langage imparfait des hiéroglyphes avec les miracles de l’imprimerie ; rapprochez le voyage des Argonautes de celui de La Pérouse ; mesurez la distance entre les observations astronomiques des mages de l’Asie et les découvertes de Newton, ou bien entre l’ébauche tracée par la main de Dibutade et les tableaux de David.\par
Tout a changé dans l’ordre physique ; tout doit changer dans l’ordre moral et politique. La moitié de la révolution du monde est déjà faite ; l’autre moitié doit s’accomplir.\par
La raison de l’homme ressemble encore au globe qu’il habite ; la moitié en est plongée dans les ténèbres, quand l’autre est éclairée. Les peuples de l’Europe ont fait des progrès étonnants dans ce qu’on appelle les arts et les sciences, et ils semblent dans l’ignorance des premières notions de la morale publique. Ils connaissent tout, excepté leurs droits et leurs devoirs. D’où vient ce mélange de génie et de stupidité ? De ce que, pour chercher à se rendre habile dans les arts, il ne faut que suivre ses passions, tandis que, pour défendre ses droits et respecter ceux d’autrui, il faut les vaincre. Il en est une autre raison : c’est que les rois qui font le destin de la terre ne craignent ni les grands géomètres, ni les grands peintres, ni les grands poètes, et qu’ils redoutent les philosophes rigides et les défenseurs de l’humanité.\par
Cependant le genre humain est dans un état violent qui ne peut être durable. La raison humaine marche depuis longtemps contre les trônes, à pas lents, et par des routes détournées, mais sûres. Le génie menace le despotisme alors même qu’il semble le caresser ; il n’est plus guère défendu que par l’habitude et par la terreur, et surtout par l’appui que lui prête la ligue des riches et de tous les oppresseurs subalternes qu’épouvante le caractère imposant de la révolution française.\par
Le peuple français semble avoir devancé de deux mille ans le reste de l’espèce humaine ; on serait tenté même de le regarder, au milieu d’elle, comme une espèce différente. L’Europe est à genoux devant les ombres des tyrans que nous punissons.\par
En Europe, un laboureur, un artisan est un animal dressé pour les plaisirs d’un noble ; en France, les nobles cherchent à se transformer en laboureurs et en artisans, et ne peuvent pas même obtenir cet honneur.\par
L’Europe ne conçoit pas qu’on puisse vivre sans rois, sans nobles ; et nous, que l’on puisse vivre avec eux.\par
L’Europe prodigue son sang pour river les chaînes de l’humanité, et nous pour les briser.\par
Nos sublimes voisins entretiennent gravement l’univers de la santé du roi, de ses divertissements, de ses voyages ; ils veulent absolument apprendre à la postérité à quelle heure il a dîné, à quel moment il est revenu de la chasse, quelle est la terre heureuse qui, à chaque instant du jour, eut l’honneur d’être foulée par ses pieds augustes, quels sont les noms des esclaves privilégiés qui ont paru en sa présence, au lever, au coucher du soleil.\par
Nous lui apprendrons, nous, les noms et les vertus des héros morts en combattant pour la liberté ; nous loi apprendrons dans quelle terre les derniers satellites des tyrans ont mordu la poussière ; nous lui apprendrons à quelle heure a sonné le trépas des oppresseurs du monde.\par
Oui, cette terre délicieuse que nous habitons, et que la nature caresse avec prédilection, est faite pour être le domaine de la liberté et du bonheur ; ce peuple sensible et fier est vraiment né pour la gloire et pour la vertu. Ô ma patrie ! si le destin m’avait fait naître dans une contrée étrangère et lointaine, j’aurais adressé au ciel des vœux continuels pour ta prospérité ; j’aurais versé des larmes d’attendrissement au récit de tes combats et de tes vertus ; mon âme attentive aurait suivi avec une inquiète ardeur tous les mouvements de ta glorieuse révolution ; j’aurais envié le sort de tes citoyens, j’aurais envié celui de tes représentants. Je suis Français, je suis l’un de tes représentants… Ô peuple sublime ! reçois le sacrifice de tout mon être ; heureux celui qui est né au milieu de toi ! plus heureux celui qui peut mourir pour ton bonheur !\par
Ô vous, à qui il a confié ses intérêts et sa puissance, que ne pouvez-vous pas avec lui et pour lui ! Oui, vous pouvez montrer au monde le spectacle nouveau de la démocratie affermie dans un vaste empire. Ceux qui, dans l’enfance du droit public, et du sein de la servitude, ont balbutié des maximes contraires, prévoyaient-ils les prodiges opérés depuis un an ? Ce qui vous reste à faire est-il plus difficile que ce que vous avez fait ? Quels sont les politiques qui peuvent vous servir de précepteurs ou de modèles ? Ne faut-il pas que vous fassiez précisément tout le contraire de ce qui a été fait avant vous ? L’art de gouverner a été jusqu’à nos jours l’art de tromper et de corrompre les hommes : il ne doit être que celui de les éclairer et de les rendre meilleurs.\par
Il y a deux sortes d’égoïsme : l’un, vil, cruel, qui isole l’homme de ses semblables, qui cherche un bien-être exclusif acheté par la misère d’autrui ; l’autre, généreux, bienfaisant, qui confond notre bonheur dans le bonheur de tous, qui attache notre gloire à celle de la patrie. Le premier fait les oppresseurs et les tyrans ; le second, les défenseurs de l’humanité. Suivons son impulsion salutaire : chérissons le repos acheté par de glorieux travaux ; ne craignons point la mort qui les couronne, et nous consoliderons le bonheur de notre patrie et même le nôtre.\par
Le vice et la vertu font les destins de la terre : ce sont les deux génies opposés qui se la disputent. La source de l’un et de l’autre est dans les passions de l’homme. Selon la direction qui est donnée à ses passions, l’homme s’élève jusqu’aux cieux ou s’enfonce dans des abîmes fangeux. Or, le but de toutes les institutions sociales, c’est de les diriger vers la justice, qui est à la fois le bonheur public et le bonheur privé.\par
Le fondement unique de la société civile, c’est la morale ! Toutes les associations qui nous font la guerre reposent sur le crime : ce ne sont aux yeux de la vérité que des hordes de sauvages policés et de brigands disciplinés. À quoi se réduit donc cette science mystérieuse de la politique et de la législation ? À mettre dans les lois et dans l’administration les vérités morales reléguées dans les livres des philosophes, et à appliquer à la conduite des peuples les notions triviales de probité que chacun est forcé d’adopter pour sa conduite privée, c’est-à-dire à employer autant d’habileté à faire régner la justice que les gouvernements en ont mis jusqu’ici à être injustes impunément ou avec bienséance.\par
Aussi, voyez combien d’art les rois et leurs complices ont épuisé pour échapper à l’application de ces principes, et pour obscurcir toutes les notions du juste et de l’injuste ! Qu’il était exquis, le bon sens de ce pirate qui répondit à Alexandre : « On m’appelle brigand, parce que je n’ai qu’un navire ; et toi, parce que tu as une flotte, on t’appelle conquérant ! » Avec quelle impudeur ils font des lois contre le vol, lorsqu’ils envahissent la fortune publique ! On condamne en leur nom les assassins, et ils assassinent des millions d’hommes par la guerre et par la misère. Sous la monarchie, les vertus domestiques ne sont que des ridicules : mais les vertus publiques sont des crimes ; la seule vertu est d’être l’instrument docile des crimes du prince, le seul honneur est d’être aussi méchant que lui. Sous la monarchie, il est permis d’aimer sa famille, mais non la patrie. Il est honorable de défendre ses amis, mais non les opprimés. La probité de la monarchie respecte toutes les propriétés, excepté celle du pauvre ; elle protège tous les droits, excepté ceux du peuple.\par
Voici un article du code de la monarchie :\par
« Tu ne voleras pas, à moins que tu ne sois le roi, ou que tu n’aies obtenu un privilège du roi ; tu n’assassineras pas, à moins que tu ne fasses périr, d’un seul coup, plusieurs milliers d’hommes. »\par
Vous connaissez ce mot ingénu du cardinal de Richelieu, écrit dans son testament politique, que les rois doivent s’abstenir avec grand soin de se servir des gens de probité, parce qu’ils ne peuvent en tirer parti. Plus de deux mille ans auparavant, il y avait sur les bords du Pont-Euxin un petit roi qui professait la même doctrine d’une manière encore plus énergique. Ses favoris avaient fait mourir quelques-uns de ses amis par de fausses accusations. Il s’en aperçut ; un jour que l’un d’eux portait devant lui une nouvelle délation : « Je te ferais mourir, lui dit-il, si des scélérats tels que toi n’étaient pas nécessaires aux despotes. » On assure que ce prince était un des meilleurs qui aient existé.\par
Mais c’est en Angleterre que le machiavélisme a poussé cette doctrine royale au plus haut degré de perfection.\par
Je ne doute pas qu’il y ait beaucoup de marchands à Londres qui se piquent de quelque bonne foi dans les affaires de leur négoce ; mais il y a à parier que ces honnêtes gens trouvent tout naturel que les membres du parlement britannique vendent publiquement au roi George leur conscience et les droits du peuple, comme ils vendent eux-mêmes les productions de leurs manufactures.\par
Pitt déroule aux yeux de ce parlement la liste de ses bassesses et de ses forfaits : « tant pour la trahison, tant pour les assassinats des représentants du peuple et des patriotes, tant pour la calomnie, tant pour la famine, tant pour la corruption, tant pour la fabrication de la fausse monnaie » ; le sénat écoute avec un sang-froid admirable, et approuve le tout avec soumission.\par
En vain, la voix d’un seul homme s’élève avec l’indignation de la vertu contre tant d’infamies ; le ministre avoue ingénument qu’il ne comprend rien à des maximes si nouvelles pour lui, et le sénat rejette la motion.\par
Stanhope, ne demande point acte à tes indignes collègues de ton opposition à leurs crimes ; la postérité te le donnera, et leur censure est pour toi le plus beau titre à l’estime de ton siècle même.\par
Que conclure de tout ce que je viens de dire ? Que l’immoralité est la base du despotisme, comme la vertu est l’essence de la République.\par
La révolution, qui tend à l’établir, n’est que le passage du règne du crime à celui de la justice ; de là les efforts continuels des rois ligués contre nous et de tous les conspirateurs pour perpétuer chez nous les préjugés et les vices de la monarchie.\par
Tout ce qui regrettait l’ancien régime, tout ce qui ne s’était lancé dans la carrière de la révolution que pour arriver à un changement de dynastie, s’est appliqué, dès le commencement, à arrêter les progrès de la morale publique ; car quelle différence y avait-il entre les amis de d’Orléans ou d’York et ceux de Louis XVI, si ce n’est, de la part des premiers, peut-être un plus haut degré de lâcheté et d’hypocrisie ?\par
Les chefs des factions qui partagèrent les deux premières législatures, trop lâches pour croire à la République, trop corrompus pour la vouloir, ne cessèrent de conspirer pour effacer du cœur des hommes les principes éternels que leur propre politique les avait d’abord obligés de proclamer. La conjuration se déguisait alors sous la couleur de ce perfide modérantisme qui, protégeant le crime et tuant la vertu, nous ramenait par un chemin oblique et sûr à la tyrannie.\par
Quand l’énergie républicaine eut confondu ce lâche système et fondé la démocratie, l’aristocratie et l’étranger formèrent le plan de tout outrer et de tout corrompre. Ils se cachèrent sous les formes de la démocratie, pour la déshonorer par des travers aussi funestes que ridicules, et pour l’étouffer dans son berceau.\par
On attaqua la liberté en même temps par le modérantisme et par la fureur. Dans ce choc de deux factions opposées en apparence, mais dont les chefs étaient unis par des nœuds secrets, l’opinion publique était dissoute, la représentation avilie, le peuple nul ; et la révolution ne semblait être qu’un combat ridicule pour décider à quels fripons resterait le pouvoir de déchirer et de vendre la patrie.\par
La marche des chefs de parti qui semblaient les plus divisés fut toujours à peu près la même. Leur principal caractère fut une profonde hypocrisie.\par
Lafayette invoquait la Constitution pour relever la puissance royale. Dumouriez invoquait la Constitution pour protéger la faction girondine contre la Convention nationale. Au mois d’août 1792, Brissot et les Girondins voulaient faire de la Constitution un bouclier, pour parer le coup qui menaçait le trône. Au mois de janvier suivant, les mêmes conspirateurs réclamaient la souveraineté du peuple pour arracher la royauté à l’opprobre de l’échafaud, et pour allumer la guerre civile dans les assemblées sectionnaires. Hébert et ses complices réclamaient la souveraineté du peuple pour égorger la Convention nationale et anéantir le gouvernement républicain.\par
Brissot et les Girondins avaient voulu armer les riches contre le peuple ; la faction d’Hébert, en protégeant l’aristocratie, caressait le peuple pour l’opprimer par lui-même.\par
Danton, le plus dangereux des ennemis de la patrie, s’il n’en avait été le plus lâche ; Danton, ménageant tous les crimes, lié à tous les complots, promettant aux scélérats sa protection, aux patriotes sa fidélité, habile à expliquer ses trahisons par des prétextes de bien public, à justifier ses vices par ses défauts prétendus, faisait inculper par ses amis, d’une manière insignifiante ou favorable, les conspirateurs près de consommer la ruine de la République, pour avoir occasion de les défendre lui-même, transigeait avec Brissot, correspondait avec Ronsin, encourageait Hébert, et s’arrangeait à tout événement pour profiter également de leur chute ou de leur succès, et pour rallier tous les ennemis de la liberté contre le gouvernement républicain.\par
C’est surtout dans ces derniers temps que l’on vit se développer dans toute son étendue l’affreux système, ourdi par nos ennemis, de corrompre la morale publique. Pour mieux y réussir, ils s’en étaient eux-mêmes établis les professeurs ; ils allaient tout flétrir, tout confondre, par un mélange odieux de la pureté de nos principes avec la corruption de leurs cœurs.\par
Tous les fripons avaient usurpé une espèce de sacerdoce politique, et rangeaient dans la classe des profanes les fidèles représentants du peuple et tous les patriotes. On tremblait alors de proposer une idée juste ; ils avaient interdit au patriotisme l’usage du bon sens : il y eut un moment où il était défendu de s’opposer à la ruine de la patrie, sous peine de passer pour mauvais citoyen : le patriotisme n’était plus qu’un travestissement ridicule ou l’audace de déclamer contre la Convention. Grâce à cette subversion des idées révolutionnaires, l’aristocratie, absoute de tous ses crimes, tramait très patriotiquement le massacre des représentants du peuple et la résurrection de la royauté ; gorgés des trésors de la tyrannie, les conjurés prêchaient la pauvreté ; affamés d’or et de domination, ils prêchaient l’égalité avec insolence pour la faire haïr ; la liberté était pour eux l’indépendance du crime ; la révolution, un trafic ; le peuple, un instrument ; la patrie, une proie. Le peu de bien même qu’ils s’efforçaient de faire était un stratagème perfide pour nous faire plus aisément des maux irréparables. S’ils se montraient quelquefois sévères, c’était pour acquérir le droit de favoriser les ennemis de la liberté, et de proscrire ses amis. Couverts de tous les crimes, ils exigeaient des patriotes, non seulement l’infaillibilité, mais la garantie de tous les caprices de la fortune, afin que personne n’osât plus servir la patrie. Ils tonnaient contre l’agiotage et partageaient avec les agioteurs la fortune publique ; ils parlaient contre la tyrannie, pour mieux servir les tyrans. Les tyrans de l’Europe accusaient, par leur organe, la Convention nationale de tyrannie. On ne pouvait pas proposer au peuple de rétablir la royauté, ils voulaient le pousser à détruire son propre gouvernement ; on ne pouvait pas lui dire qu’il devait appeler ses ennemis, on lui disait qu’il fallait chasser ses défenseurs ; on ne pouvait pas lui dire de poser les armes, on le décourageait par de fausses nouvelles ; on comptait pour rien ses succès, et on exagérait ses échecs avec une coupable malignité.\par
On ne pouvait pas lui dire : Le fils du tyran, ou un autre Bourbon, ou bien l’un des fils du roi George, te rendrait heureux ; mais on lui disait : Tu es malheureux. On lui traçait le tableau de la disette qu’ils cherchaient eux-mêmes à amener ; on lui disait que les œufs, que le sucre n’étaient pas abondants. On ne lui disait pas que sa liberté valait quelque chose ; que l’humiliation de ses oppresseurs et tous les autres effets de la révolution n’étaient pas des biens méprisables ; qu’il combattait encore ; que la ruine de ses ennemis pouvait seule assurer son bonheur... ; mais il sentait tout cela. Enfin, ils ne pouvaient pas asservir le peuple français par la force ni par son propre consentement ; ils cherchaient à l’enchaîner par la subversion, par la révolte, par la corruption des mœurs.\par
Ils ont érigé l’immoralité, non seulement en système, mais en religion ; ils ont cherché à éteindre tous les sentiments religieux de la nature par leurs exemples, autant que par leurs préceptes. Le méchant voudrait dans son cœur qu’il ne restât pas sur la terre un seul homme de bien, afin de n’y plus rencontrer un seul accusateur, et de pouvoir y respirer en paix. Ceux-ci allèrent chercher dans les esprits et dans les cœurs tout ce qui sert d’appui à la morale, pour l’en arracher, et pour y étouffer l’accusateur invisible que la nature y a caché.\par
Les tyrans, satisfaits de l’audace de leurs émissaires, s’empressèrent d’étaler aux yeux de leurs sujets les extravagances qu’ils avaient achetées ; et, feignant de croire que c’était là le peuple français, ils semblèrent leur dire : « Que gagneriez-vous à secouer notre joug ? vous le voyez, les républicains ne valent pas mieux que nous. » Les tyrans ennemis de la France avaient ordonné un plan qui devait, si leurs espérances avaient été parfaitement remplies, embraser tout à coup notre République et élever une barrière insurmontable entre elle et les autres peuples ; les conjurés l’exécutèrent. Les mêmes fourbes qui avaient invoqué la souveraineté du peuple pour égorger la Convention nationale, alléguèrent la haine de la superstition pour nous donner la guerre civile et l’athéisme.\par
Que voulaient-ils, ceux qui, au sein des conspirations dont nous étions environnés, au milieu des embarras d’une telle guerre, au moment où les torches de la discorde civile fumaient encore, attaquèrent tout à coup tous les cultes par la violence, pour s’ériger eux-mêmes en apôtres fougueux du néant et en missionnaires fanatiques de l’athéisme ? Quel était le motif de cette grande opération tramée dans les ténèbres de la nuit, à l’insu de la Convention nationale, par des prêtres, par des étrangers et par des conspirateurs ? Était-ce l’amour de la patrie ? La patrie leur a déjà infligé le supplice des traîtres. Était-ce la haine des prêtres ? Les prêtres étaient leurs amis. Était-ce l’horreur du fanatisme ? C’était le seul moyen de lui offrir des armes. Était-ce le désir de hâter le triomphe de la Raison ? Mais on ne cessait de l’outrager par des violences absurdes et par des extravagances concertées pour la rendre odieuse : on ne semblait la reléguer dans les temples que pour la bannir de la République.\par
On servait la cause des rois ligués contre nous, des rois qui avaient eux-mêmes annoncé d’avance ces événements, et qui s’en prévalaient avec succès pour exciter contre nous le fanatisme des peuples par des manifestes et par des prières publiques. Il faut voir avec quelle sainte colère M. Pitt nous oppose ces faits, et avec quel soin le petit nombre d’hommes intègres qui existe au parlement d’Angleterre les rejette sur quelques hommes méprisables, désavoués et punis par vous.\par
Cependant, tandis que ceux-ci remplissaient leur mission, le peuple anglais jeûnait pour expier les péchés payés par M. Pitt, et les bourgeois de Londres portaient le deuil du culte catholique, comme ils avaient porté celui du roi Capet et de la reine Antoinette.\par
Admirable politique du ministre de George, qui faisait insulter l’Être suprême par ses émissaires, et voulait le venger par les baïonnettes anglaises et autrichiennes ! J’aime beaucoup la piété des rois, et je crois fermement à la religion de M. Pitt. Il est certain du moins qu’il a trouvé de bons amis en France ; car, suivant tous les calculs de la prudence humaine, l’intrigue dont je parle devait allumer un incendie rapide dans toute la République, et lui susciter de nouveaux ennemis au dehors.\par
Heureusement, le génie du peuple français, sa passion inaltérable pour la liberté, la sagesse avec laquelle vous avez averti les patriotes de bonne foi qui pouvaient être entraînés par l’exemple dangereux des inventeurs hypocrites de cette machination, enfin le soin qu’ont pris les prêtres eux-mêmes de désabuser le peuple sur leur propre compte, toutes ces causes ont prévenu la plus grande partie des inconvénients que les conspirateurs en attendaient. C’est à vous de faire cesser les autres, et de mettre à profit, s’il est possible, la perversité même de nos ennemis, pour assurer le triomphe des principes et de la liberté.\par
Ne consultez que le bien de la patrie et les intérêts de l’humanité. Toute institution, toute doctrine qui console et qui élève les âmes doit être accueillie ; rejetez toutes celles qui tendent à les dégrader et à les corrompre. Ranimez, exaltez tous les sentiments généreux et toutes les grandes idées morales qu’on a voulu éteindre ; rapprochez par le charme de l’amitié et par le lien de la vertu les hommes qu’on a voulu diviser. Qui donc t’a donné la mission d’annoncer au peuple que la Divinité n’existe pas, ô toi qui te passionnes pour cette aride doctrine, et qui ne te passionnas jamais pour la patrie ? Quel avantage trouves-tu à persuader à l’homme qu’une force aveugle préside à ses destinées et frappe au hasard le crime et la vertu, que son âme n’est qu’un souffle léger qui s’éteint aux portes du tombeau ?\par
L’idée de son néant lui inspirera-t-elle des sentiments plus purs et plus élevés que celle de son immortalité ? Lui inspirera-t-elle plus de respect pour ses semblables et pour lui-même, plus de dévouement pour la patrie, plus d’audace à braver la tyrannie, plus de mépris pour la mort ou pour la volupté ? Vous qui regrettez un ami vertueux, vous aimez à penser que la plus belle partie de lui-même a échappé au trépas ! Vous qui pleurez sur le cercueil d’un fils ou d’une épouse, êtes-vous consolé par celui qui vous dit qu’il ne reste plus d’eux qu’une vile poussière ? Malheureux qui expirez sous les coups d’un assassin, votre dernier soupir est un appel à la justice éternelle ! L’innocence sur l’échafaud fait pâlir le tyran sur son char de triomphe : aurait-elle cet ascendant, si le tombeau égalait l’oppresseur et l’opprimé ? Malheureux sophiste ! de quel droit viens-tu arracher à l’innocence le sceptre de la raison, pour le remettre dans les mains du crime, jeter un voile funèbre sur la nature, désespérer le malheur, réjouir le vice, attrister la vertu, dégrader l’humanité ? Plus un homme est doué de sensibilité et de génie, plus il s’attache aux idées qui agrandissent son être et qui élèvent son cœur ; et la doctrine des hommes de cette trempe devient celle de l’univers. Eh ! comment ces idées ne seraient-elles point des vérités ? Je ne conçois pas du moins comment la nature aurait pu suggérer à l’homme des fictions plus utiles que toutes les réalités ; et si l’existence de Dieu, si l’immortalité de l’âme n’étaient que des songes, elles seraient encore la plus belle de toutes les conceptions de l’esprit humain.\par
Je n’ai pas besoin d’observer qu’il ne s’agit pas ici de faire le procès à aucune opinion philosophique en particulier, ni de contester que tel philosophe peut être vertueux, quelles que soient ses opinions, et même en dépit d’elles, par la force d’un naturel heureux ou d’une raison supérieure. Il s’agit de considérer seulement l’athéisme comme national, et lié à un système de conspiration contre la République.\par
Eh ! que vous importent à vous, législateurs, les hypothèses diverses par lesquelles certains philosophes expliquent les phénomènes de la nature ? Vous pouvez abandonner tous ces objets à leurs disputes éternelles : ce n’est ni comme métaphysiciens, ni comme théologiens, que vous devez les envisager. Aux yeux du législateur, tout ce qui est utile au monde et bon dans la pratique, est la vérité.\par
L’idée de l’Être suprême et de l’immortalité de l’âme est un rappel continuel à la justice ; elle est donc sociale et républicaine. La Nature a mis dans l’homme le sentiment du plaisir et de la douleur qui le force à fuir les objets physiques qui lui sont nuisibles, et à chercher ceux qui lui conviennent. Le chef-d’œuvre de la société serait de créer en lui, pour les choses morales, un instinct rapide qui, sans le secours tardif du raisonnement, le portât à faire le bien et à éviter le mal ; car la raison particulière de chaque homme, égarée par ses passions, n’est souvent qu’un sophiste qui plaide leur cause, et l’autorité de l’homme peut toujours être attaquée par l’amour-propre de l’homme. Or, ce qui produit ou remplace cet instinct précieux, ce qui supplée à l’insuffisance de l’autorité humaine, c’est le sentiment religieux qu’imprime dans les âmes l’idée d’une sanction donnée aux préceptes de la morale par une puissance supérieure à l’homme. Aussi je ne sache pas qu’aucun législateur se soit jamais avisé de nationaliser l’athéisme ; je sais que les plus sages mêmes d’entre eux se sont permis de mêler à la vérité quelques fictions, soit pour frapper l’imagination des peuples ignorants, soit pour les attacher plus fortement à leurs institutions. Lycurgue et Solon eurent recours à l’autorité des oracles ; et Socrate lui-même, pour accréditer la vérité parmi ses concitoyens, se crut obligé de leur persuader qu’elle lui était inspirée par un génie familier.\par
Vous ne conclurez pas de là sans doute qu’il faille tromper les hommes pour les instruire, mais seulement que vous êtes heureux de vivre dans un siècle et dans un pays dont les lumières ne vous laissent d’autre tâche à remplir que de rappeler les hommes à la nature et à la vérité.\par
Vous vous garderez bien de briser le lien sacré qui les unit à l’auteur de leur être. Il suffit même que cette opinion ait régné chez un peuple, pour qu’il soit dangereux de la détruire. Car les motifs des devoirs et les bases de la moralité s’étant nécessairement liés à celte idée, l’effacer, c’est démoraliser le peuple. Il résulte du même principe qu’on ne doit jamais attaquer un culte établi qu’avec prudence et avec une certaine délicatesse, de peur qu’un changement subit et violent ne paraisse une atteinte portée à la morale, et une dispense de la probité même. Au reste, celui qui peut remplacer la Divinité dans le système de la vie sociale est à mes yeux un prodige de génie ; celui qui, sans l’avoir remplacée, ne songe qu’à la bannir de l’esprit des hommes, me paraît un prodige de stupidité ou de perversité.\par
Qu’est-ce que les conjurés avaient mis à la place de ce qu’ils détruisaient ? Rien, si ce n’est le chaos, le vide et la violence. Ils méprisaient trop le peuple pour prendre la peine de le persuader ; au lieu de l’éclairer, ils ne voulaient que l’irriter, l’effaroucher ou le dépraver.\par
Si les principes que j’ai développés jusqu’ici sont des erreurs, je me trompe du moins avec tout ce que le monde révère : prenons ici les leçons de l’histoire. Remarquez, je vous prie, comment les hommes qui ont influé sur la destinée des États furent déterminés vers l’un ou l’autre des deux systèmes opposés par leur caractère personnel et par la nature même de leurs vues politiques. Voyez-vous avec quel art profond César, plaidant dans le sénat romain en faveur des complices de Catilina, s’égare dans une digression contre le dogme de l’immortalité de l’âme, tant ces idées lui paraissent propres à éteindre dans le cœur des juges l’énergie de la vertu, tant la cause du crime lui paraît liée à celle de l’athéisme. Cicéron, au contraire, invoquait contre les traîtres et le glaive des lois et la foudre des dieux. Socrate mourant entretient ses amis de l’immortalité de l’âme. Léonidas aux Thermopyles, soupant avec ses compagnons d’armes, au moment d’exécuter le dessein le plus héroïque que la vertu humaine ait jamais conçu, les invite pour le lendemain à un autre banquet dans une vie nouvelle. Il y a loin de Socrate à Chaumette, et de Léonidas au Père Duchesne. Un grand homme, un véritable héros s’estime trop lui-même pour se complaire dans l’idée de son anéantissement. Un scélérat, méprisable à ses propres yeux, horrible à ceux d’autrui, sent que la nature ne peut lui faire de plus beau présent que le néant.\par
Caton ne balança point entre Épicure et Zénon. Brutus et les illustres conjurés qui partagèrent ses périls et sa gloire appartenaient aussi à cette secte sublime de stoïciens, qui eut des idées si hautes de la dignité de l’homme, qui poussa si loin l’enthousiasme de la vertu, et qui n’outra que l’héroïsme. Le stoïcisme enfanta des émules de Brutus et de Caton jusque dans les siècles affreux qui suivirent la perte de la liberté romaine. Le stoïcisme sauva l’honneur de la nature humaine dégradée par les vices des successeurs de César et surtout par la patience des peuples. La secte épicurienne revendiquait sans doute tous les scélérats qui opprimèrent leur patrie, et tous les lâches qui la laissèrent opprimer. Aussi, quoique le philosophe dont elle portait le nom ne fût pas personnellement un homme méprisable, les principes de son système, interprétés par la corruption, amenèrent des conséquences si funestes que l’antiquité elle-même la flétrit par la dénomination de {\itshape troupeau d’Épicure} ; et comme dans tous les temps le cœur humain est au fond le même, et que le même instinct ou le même système politique a commandé aux hommes la même marche, il sera facile d’appliquer les observations que je viens de faire, au moment actuel, et même au temps qui a précédé immédiatement notre révolution. Il est bon de jeter un coup d’œil sur ce temps, ne fût-ce que pour pouvoir expliquer une partie des phénomènes qui ont éclaté depuis.\par
Dès longtemps les observateurs éclairés pouvaient apercevoir quelques symptômes de la révolution actuelle. Tous les événements importants y tendaient ; les causes mêmes des particuliers susceptibles de quelque éclat s’attachaient à une intrigue politique. Les hommes de lettres renommés, en vertu de leur influence sur l’opinion, commençaient à en obtenir quelqu’une dans les affaires. Les plus ambitieux avaient formé dès lors une espèce de coalition qui augmentait leur importance ; ils semblaient s’être partagés en deux sectes, dont l’une défendait bêtement le clergé et le despotisme. La plus puissante et la plus illustre était celle qui fut connue sous le nom d’encyclopédistes. Elle renfermait quelques hommes estimables et un plus grand nombre de charlatans ambitieux. Plusieurs de ses chefs étaient devenus des personnages considérables dans l’État : quiconque ignorerait son influence et sa politique n’aurait pas une idée complète de la préface de notre révolution. Cette secte, en matière de politique, resta toujours au-dessous des droits du peuple ; en matière de morale, elle alla beaucoup au-delà de la destruction des préjugés religieux. Ses coryphées déclamaient quelquefois contre le despotisme, et ils étaient pensionnés par les despotes ; ils faisaient tantôt des livres contre la cour, et tantôt des dédicaces aux rois, des discours pour les courtisans, et des madrigaux pour les courtisanes ; ils étaient fiers dans leurs écrits, et rampants dans les antichambres. Cette secte propagea avec beaucoup de zèle l’opinion du matérialisme, qui prévalut parmi les grands et parmi les beaux esprits. On lui doit en grande partie cette espèce de philosophie pratique qui, réduisant l’égoïsme en système, regarde la société humaine comme une guerre de ruse, le succès comme la règle du juste et de l’injuste, la probité comme une affaire de goût ou de bienséance, le monde comme le patrimoine des fripons adroits. J’ai dit que ses coryphées étaient ambitieux ; les agitations gui annonçaient un grand changement dans l’ordre politique des choses avaient pu étendre leurs vues. On a remarqué que plusieurs d’entre eux avaient des liaisons intimes avec la maison d’Orléans, et la Constitution anglaise était, suivant eux, le chef-d’œuvre de la politique et le maximum du bonheur social.\par
Parmi ceux qui, du temps dont je parle, se signalèrent dans la carrière des lettres et de la philosophie, un homme, par l’élévation de son âme et par la grandeur de son caractère, se montra digne du ministère de précepteur du genre humain. Il attaqua la tyrannie avec franchise ; il parla avec enthousiasme de la divinité ; son éloquence mâle et probe peignit en traits de flamme les charmes de la vertu ; elle défendit ces dogmes consolateurs que la raison donne pour appui au cœur humain ; la pureté de sa doctrine, puisée dans la nature et dans la haine profonde du vice, autant que son mépris invincible pour les sophistes intrigants qui usurpaient le nom de philosophes, lui attira la haine et la persécution de ses rivaux et de ses faux amis. Ah ! s’il avait été témoin de cette révolution dont il fut le précurseur et qui l’a porté au Panthéon, qui peut douter que son âme généreuse eût embrassé avec transport la cause de la justice et de l’égalité ? Mais qu’ont fait pour elle ses lâches adversaires ? Ils ont combattu la révolution, dès le moment qu’ils ont craint qu’elle n’élevât le peuple au-dessus de toutes les vanités particulières ; les uns ont employé leur esprit à frelater les principes républicains et à corrompre l’opinion publique ; ils se sont prostitués aux factions, et surtout au parti d’Orléans ; les autres se sont renfermés dans une lâche neutralité. Les hommes de lettres en général se sont déshonorés dans cette révolution ; et, à la honte éternelle de l’esprit, la raison du peuple en a fait seule tous les frais.\par
Hommes petits et vains, rougissez, s’il est possible. Les prodiges qui ont immortalisé cette époque de l’histoire humaine ont été opérés sans vous et malgré vous ; le bon sens sans intrigue, et le génie sans instruction, ont porté la France à ce degré d’élévation qui épouvante votre bassesse et qui écrase votre nullité. Tel artisan s’est montré habile dans la connaissance des droits de l’homme, quand tel faiseur de livres, presque républicain en 1788, défendait stupidement la cause des rois en 1793. Tel laboureur répandait la lumière de la philosophie dans les campagnes, quand l’académicien Condorcet, jadis grand géomètre, dit-on, au jugement des littérateurs, et grand littérateur au dire des géomètres, depuis conspirateur timide, méprisé de tous les partis, travaillait sans cesse à l’obscurcir par le perfide fatras de ses rapsodies mercenaires.\par
Vous avez déjà été frappés, sans doute, de la tendresse avec laquelle tant d’hommes qui ont trahi leur patrie ont caressé les opinions sinistres que je combats. Que de rapprochements curieux peuvent s’offrir encore à vos esprits ! Nous avons entendu, qui croirait à cet excès d’impudeur ? nous avons entendu dans une société populaire le traître Guadet dénoncer un citoyen pour avoir prononcé le nom de la Providence. Nous avons entendu, quelque temps après, Hébert en accuser un autre pour avoir écrit contre l’athéisme. N’est-ce pas Vergniaud et Gensonné qui, en votre présence même, et à votre tribune, pérorèrent avec chaleur pour bannir du préambule de la Constitution le nom de l’Être suprême que vous y avez placé ? Danton, qui souriait de pitié aux mots de vertu, de gloire, de postérité ; Danton, dont le système était d’avilir ce qui peut élever l’âme ; Danton, qui était froid et muet dans les plus grands dangers de la liberté, parla après eux avec beaucoup de véhémence en faveur de la même opinion. D’où vient ce singulier accord de principe entre tant d’hommes qui paraissaient divisés ? Faut-il l’attribuer simplement au soin que prenaient les déserteurs de la cause du peuple, de chercher à couvrir leur défection par une affectation de zèle contre ce qu’ils appelaient les préjugés religieux, comme s’ils avaient voulu compenser leur indulgence pour l’aristocratie et la tyrannie par la guerre qu’ils déclaraient à la Divinité ?\par
Non, la conduite de ces personnages artificieux tenait sans doute à des vues politiques plus profondes ; ils sentaient que, pour détruire la liberté, il fallait favoriser par tous les moyens tout ce qui tend à justifier l’égoïsme, à dessécher le cœur et à effacer l’idée de ce beau moral, qui est la seule règle sur laquelle la raison publique juge les défenseurs et les ennemis de l’humanité. Ils embrassaient avec transport un système qui, confondant la destinée des bons et des méchants, ne laisse entre eux d’autre différence que les faveurs incertaines de la fortune, ni d’autre arbitre que le droit du plus fort ou du plus rusé.\par
Vous tendez à un but bien différent ; vous suivrez donc une politique contraire. Mais ne craignons-nous pas de réveiller le fanatisme et de donner un avantage à l’aristocratie ? Non : si nous adoptons le parti que la sagesse indique, il nous sera facile d’éviter cet écueil.\par
Ennemis du peuple, qui que vous soyez, jamais la Convention nationale ne favorisera votre perversité. Aristocrates, de quelques dehors spécieux que vous vouliez vous couvrir aujourd’hui, en vain chercheriez-vous à vous prévaloir de notre censure contre les auteurs d’une trame criminelle, pour accuser les patriotes sincères que la seule haine du fanatisme peut avoir entraînés à des démarches indiscrètes. Vous n’avez pas le droit d’accuser ; et la justice nationale, dans ces orages excités par les factions, sait discerner les erreurs des conspirations : elle saisira, d’une main sûre, tous les intrigants pervers, et ne frappera pas un seul homme de bien.\par
Fanatiques, n’espérez rien de nous. Rappeler les hommes au culte pur de l’Être suprême, c’est porter un coup mortel au fanatisme. Toutes les fictions disparaissent devant la Vérité et toutes les folies tombent devant la Raison. Sans contrainte, sans persécution, toutes les sectes doivent se confondre d’elles-mêmes dans la religion universelle de la Nature. Nous vous conseillerons donc de maintenir les principes que vous avez manifestés jusqu’ici. Que la liberté des cultes soit respectée, pour le triomphe même de la raison ; mais qu’elle ne trouble point l’ordre public, et qu’elle ne devienne point un moyen de conspiration. Si la malveillance contre-révolutionnaire se cachait sous ce prétexte, réprimez-la ; et reposez-vous du reste sur la puissance des principes et sur la force même des choses.\par
Prêtres ambitieux, n’attendez donc pas que nous travaillions à rétablir votre empire ; une telle entreprise serait même au-dessus de notre puissance. Vous vous êtes tués vous-mêmes, et on ne revient pas plus à la vie morale qu’à l’existence physique.\par
Et, d’ailleurs, qu’y a-t-il entre les prêtres et Dieu ? Les prêtres sont à la morale ce que les charlatans sont à la médecine. Combien le Dieu de la nature est différent du Dieu des prêtres ! Il ne connaît rien de si ressemblant à l’athéisme que les religions qu’ils ont faites. À force de défigurer l’Être suprême, ils l’ont anéanti autant qu’il était en eux ; ils en ont fait tantôt un globe de feu, tantôt un bœuf, tantôt un arbre, tantôt un homme, tantôt un roi. Les prêtres ont créé Dieu à leur image : ils l’ont fait jaloux, capricieux, avide, cruel, implacable. Ils l’ont traité comme jadis les maires du palais traitèrent les descendants de Clovis, pour régner sous son nom et se mettre à sa place. Ils l’ont relégué dans le ciel comme dans un palais, et ne l’ont appelé sur la terre que pour demander à leur profit des dîmes, des richesses, des honneurs, des plaisirs et de la puissance. Le véritable prêtre de l’Être suprême, c’est la Nature ; son temple, l’univers ; son culte, la vertu ; ses fêtes, la joie d’un grand peuple rassemblé sous ses yeux pour resserrer les doux nœuds de la fraternité universelle, et pour lui présenter l’hommage des cœurs sensibles et purs.\par
Prêtres, par quel titre avez-vous prouvé votre mission ? Avez-vous été plus justes, plus modestes, plus amis de la vérité que les autres hommes ? Avez-vous chéri l’égalité, défendu les droits des peuples, abhorré le despotisme et abattu la tyrannie ? C’est vous qui avez dit aux rois : {\itshape Vous êtes les images de Dieu sur la terre ; c’est de lui seul que vous tenez votre puissance}. Et les rois vous ont répondu : {\itshape Oui, vous êtes vraiment les envoyés de Dieu ; unissons-nous pour partager les dépouilles et les adorations des mortels}. Le sceptre et l’encensoir ont conspiré pour déshonorer le ciel et pour usurper la terre.\par
Laissons les prêtres, et retournons à la divinité. Attachons la morale à des bases éternelles et sacrées ; inspirons à l’homme ce respect religieux pour l’homme, ce sentiment profond de ses devoirs, qui est la seule garantie du bonheur social ; nourrissons-le par toutes nos institutions ; que l’éducation publique soit surtout dirigée vers ce but. Vous lui imprimerez sans doute un grand caractère, analogue à la nature de notre gouvernement et à la sublimité des destinées de la République. Vous sentirez la nécessité de la rendre commune et égale pour tous les Français. Il ne s’agit plus de former des {\itshape messieurs}, mais des citoyens : la patrie a seule droit d’élever ses enfants ; elle ne peut confier ce dépôt à l’orgueil des familles, ni aux préjugés des particuliers, aliments éternels de l’aristocratie et d’un fédéralisme domestique, qui rétrécit les âmes en les isolant, et détruit, avec l’égalité, tous les fondements de l’ordre social. Mais ce grand objet est étranger à la discussion actuelle.\par
Il est cependant une sorte d’institution qui doit être considérée comme une partie essentielle de l’éducation publique, et qui appartient nécessairement au sujet de ce rapport : je veux parler des fêtes nationales.\par
Rassemblez les hommes, vous les rendrez meilleurs ; car les hommes rassemblés chercheront à se plaire, et ils ne pourront se plaire que par les choses qui les rendent estimables. Donnez à leur réunion un grand motif moral et politique, et l’amour des choses honnêtes entrera avec le plaisir dans tous les cœurs ; car les hommes ne se voient pas sans plaisir.\par
L’homme est le plus grand objet qui soit dans la nature ; et le plus magnifique de tous les spectacles, c’est celui d’un grand peuple assemblé. On ne parle jamais sans enthousiasme des fêtes nationales de la Grèce : cependant elles n’avaient guère pour objet que des jeux où brillaient la force du corps, l’adresse, ou tout au plus le talent des poètes et des orateurs. Mais la Grèce était là ; on voyait un spectacle plus grand que les jeux : c’étaient les spectateurs eux-mêmes ; c’était le peuple vainqueur de l’Asie, que les vertus républicaines avaient élevé quelquefois au-dessus de l’humanité ; on voyait les grands hommes qui avaient sauvé et illustré la patrie : les pères montraient à leurs fils Miltiade, Aristide, Épaminondas, Timoléon, dont la seule présence était une leçon vivante de magnanimité, de justice et de patriotisme.\par
Combien il serait facile au peuple français de donner à ces assemblées un objet plus étendu et un plus grand caractère ! Un système de fêtes nationales bien entendu serait à la fois le plus doux lien de fraternité et le plus puissant moyen de régénération.\par
Ayez des fêtes générales et plus solennelles pour toute la République ; ayez des fêtes particulières et pour chaque lieu, qui soient des jours de repos, et qui remplacent ce que les circonstances ont détruit.\par
Que toutes tendent à réveiller les sentiments généreux qui font le charme et l’ornement de la vie humaine, l’enthousiasme de la liberté, l’amour de la patrie, le respect des lois. Que la mémoire des tyrans et des traîtres y soit vouée à l’exécration ; que celle des héros de la liberté et des bienfaiteurs de l’humanité y reçoive le juste tribut de la reconnaissance publique ; qu’elles puisent leur intérêt et leurs noms même dans les événements immortels de notre révolution, et dans les objets les plus sacrés et les plus chers au cœur de l’homme ; qu’elles soient embellies et distinguées par les emblèmes analogues à leur objet particulier. Invitons à nos fêtes, et la nature, et toutes les vertus ; que toutes soient célébrées sous les auspices de l’Être suprême ; qu’elles lui soient consacrées ; qu’elles s’ouvrent et qu’elles finissent par un hommage à sa puissance et à sa bonté.\par
Tu donneras ton nom sacré à l’une de nos plus belles fêtes, ô toi, fille de la Nature, mère du bonheur et de la gloire, toi seule légitime souveraine du monde, détrônée par le crime, toi à qui le peuple français a rendu ton empire, et qui lui donnes en échange une patrie et des mœurs, auguste Liberté ! tu partageras nos sacrifices avec ta compagne immortelle, la douce et sainte Egalite. Nous fêterons l’Humanité, l’Humanité avilie et foulée aux pieds par les ennemis de la République française. Ce sera un beau jour que celui où nous célébrerons la fête du genre humain ; c’est le banquet fraternel et sacré, où, du sein de la victoire, le peuple français invitera la famille immense dont seul il défend l’honneur et les imprescriptibles droits. Nous célébrerons aussi tous les grands hommes, de quelque temps et de quelque pays que ce soit, qui ont affranchi leur patrie du joug des tyrans, et qui ont fondé la liberté par de sages lois. Vous ne serez point oubliés, illustres martyrs de la République française ! Vous ne serez point oubliés, héros morts en combattant pour elle ! Qui pourrait oublier les héros de ma patrie ? La France leur doit la liberté, l’univers leur devra la sienne. Que l’univers célèbre bientôt leur gloire en jouissant de leurs bienfaits ! Combien de traits héroïques confondus dans la foule des grandes actions que la liberté a comme prodiguées parmi nous ! Combien de noms dignes d’être inscrits dans les fastes de l’histoire demeurent ensevelis dans l’obscurité ! Mânes inconnus et révérés, si vous échappez à la célébrité, vous n’échapperez point à notre tendre reconnaissance.\par
Qu’ils tremblent, tous les tyrans armés contre la liberté, s’il en existe encore alors ! Qu’ils tremblent le jour où les Français viendront sur vos tombeaux jurer de vous imiter ! Jeunes Français, entendez-vous l’immortel Bara qui, du sein du Panthéon, vous appelle à la gloire ? Venez répandre des fleurs sur sa tombe sacrée. Bara, enfant héroïque, tu nourrissais ta mère et tu mourus pour ta patrie ! Bara, tu as déjà reçu le prix de ton héroïsme ; la patrie a adopté ta mère ; la patrie, étouffant les factions criminelles, va s’élever triomphante sur les ruines des vices et des trônes. Ô Bara, tu n’as pas trouvé de modèle dans l’antiquité, mais tu as trouvé parmi nous des émules de ta vertu.\par
Par quelle fatalité ou par quelle ingratitude a-t-on laissé dans l’oubli un héros plus jeune encore et digne des hommages de la postérité ? Les Marseillais rebelles, rassemblés sur les bords de la Durance, se préparaient à passer cette rivière pour aller égorger les patriotes faibles et désarmés de ces malheureuses contrées ; une troupe peu nombreuse de républicains, réunis de l’autre côté, ne voyait d’autre ressource que de couper les câbles des pontons qui étaient au pouvoir de leurs ennemis : mais tenter une telle entreprise en présence des bataillons nombreux qui couvraient l’autre rive, et à la portée de leurs fusils, paraissait une entreprise chimérique aux plus hardis. Tout à coup un enfant de treize ans s’élance sur une hache ; il vole au bord du fleuve, et frappe le câble de toute sa force. Plusieurs décharges de mousqueterie sont dirigées contre lui ; il continue de frapper à coups redoublés ; enfin, il est atteint d’un coup mortel ; il s’écrie : {\itshape Je meurs, cela m’est égal ; c’est pour la liberté}. Il tombe ; il est mort… Respectable enfant, que la patrie s’enorgueillisse de t’avoir donné le jour ! Avec quel orgueil la Grèce et Rome auraient honoré ta mémoire, si elles avaient produit un héros tel que toi !\par
Citoyens, portons en pompe ses cendres au temple de la gloire ; que la République en deuil les arrose de larmes amères ! Non, ne le pleurons pas ; imitons-le, vengeons-le par la ruine de tous les ennemis de notre République\footnote{Le nom de ce héros est Agricol Viala. Il faut apprendre ici à la République entière deux traits d’une nature bien différente. Quand la mère du jeune Viala apprit la mort de son fils, sa douleur fut aussi profonde qu’elle était juste. Mais, lui dit-on, il est mort pour la patrie ! {\itshape Ah ! c’est vrai}, dit-elle, {\itshape il est mort pour la patrie}. Et ses larmes se séchèrent. L’autre fait, c’est que les Marseillais rebelles, ayant passé la Durance, eurent la lâcheté d’insulter aux restes du jeune héros, et jetèrent son corps dans les flots. ({\itshape Note de Robespierre}.)}.\par
Toutes les vertus se disputent le droit de présider à nos fêtes. Instituons la fête de la Gloire, non de celle qui ravage et opprime le monde, mais de celle qui l’affranchit, qui l’éclaire et qui le console ; de celle qui, après la patrie, est la première idole des cœurs généreux. Instituons une fête plus touchante : la fête du Malheur. Les esclaves adorent la fortune et le pouvoir ; nous, honorons le malheur, le malheur que l’humanité ne peut entièrement bannir de la terre, mais qu’elle console et soulage avec respect. Tu obtiendras aussi cet hommage, ô toi qui jadis unissais les héros et les sages, toi qui multiplies les forces des amis de la patrie, et dont les méchants, liés par le crime, ne connurent jamais que le simulacre imposteur, divine Amitié, tu retrouveras chez les Français républicains ta puissance et tes autels.\par
Pourquoi ne rendrions-nous pas le même honneur au pudique et généreux amour, à la foi conjugale, à la tendresse paternelle, à la piété filiale ? Nos fêtes, sans doute, ne seront ni sans intérêt, ni sans éclat. Vous y serez, braves défenseurs de la patrie, que décorent de glorieuses cicatrices. Vous y serez, vénérables vieillards, que le bonheur préparé à votre postérité doit consoler d’une longue vie passée sous le despotisme. Vous y serez, tendres élèves de la Patrie, qui croissez pour étendre sa gloire et pour recueillir le fruit de ses travaux.\par
Vous y serez, jeunes citoyennes, à qui la victoire doit ramener bientôt des frères et des amants dignes de vous. Vous y serez, mères de famille, dont les époux et les fils élèvent des trophées à la République avec les débris des trônes. Ô femmes françaises, chérissez la liberté achetée au prix de leur sang ; servez-vous de votre empire pour étendre celui de la vertu républicaine ! Ô femmes françaises, vous êtes dignes de l’amour et du respect de la terre ! Qu’avez-vous à envier aux femmes de Sparte ? Comme elles, vous avez donné le jour à des héros ; comme elles, vous les avez dévoués, avec un abandon sublime, à la Patrie.\par
Malheur à celui qui cherche à éteindre ce sublime enthousiasme, et à étouffer, par de désolantes doctrines, cet instinct moral du peuple, qui est le principe de toutes les grandes actions ! C’est à vous, représentants du peuple, qu’il appartient de faire triompher les vérités que nous venons de développer. Bravez les clameurs insensées de l’ignorance présomptueuse ou de la perversité hypocrite. Quelle est donc la dépravation dont nous étions environnés, s’il nous a fallu du courage pour les proclamer ? La postérité pourra-t-elle croire que les factions vaincues avaient porté l’audace jusqu’à nous accuser de modérantisme et d’aristocratie, pour avoir rappelé l’idée de la divinité et de la morale ? Croira-t-elle qu’on ait osé dire, jusque dans cette enceinte, que nous avions par là reculé la raison humaine de plusieurs siècles ? Ils invoquaient la raison, les monstres qui aiguisaient contre vous leurs poignards sacrilèges !\par
Tous ceux qui défendaient vos principes et votre dignité devaient être aussi sans doute les objets de leur fureur. Ne nous étonnons pas si tous les scélérats ligués contre vous semblent vouloir nous préparer la ciguë. Mais, avant de la boire, nous sauverons la patrie. Le vaisseau qui porte la fortune de la République n’est pas destiné à faire naufrage ; il vogue sous vos auspices, et les tempêtes seront forcées à le respecter.\par
Asseyez-vous donc tranquillement sur les bases immuables de la justice, et ravivez la morale publique. Tonnez sur la tête des coupables, et lancez la foudre sur tous vos ennemis. Quel est l’insolent qui, après avoir rampé aux pieds d’un roi, ose insulter à la majesté du peuple français dans la personne de ses représentants ? Commandez à la victoire, mais replongez surtout le vice dans le néant. Les ennemis de la République sont tous les hommes corrompus.\par
Le patriote n’est autre chose qu’un homme probe et magnanime dans toute la force de ce terme. C’est peu d’anéantir les rois, il faut faire respecter à tous les peuples le caractère du peuple français. C’est en vain que nous porterions au bout de l’univers la renommée de nos armes, si toutes les passions déchirent impunément le sein de la patrie. Défions-nous de l’ivresse même des succès. Soyons terribles dans les revers, modestes dans nos triomphes, et fixons au milieu de nous la paix et le bonheur par la sagesse et par la morale. Voilà le véritable but de nos travaux ; voilà la tâche la plus héroïque et la plus difficile. Nous croyons concourir à ce but, en vous proposant le décret suivant :\par
\par
DÉCRET\par
 {\itshape Article Premier.} \par
Le Peuple français reconnaît l’existence de l’Être suprême, et l’immortalité de l’âme.\par
II.\par
Il reconnaît que le culte digne de l’Être suprême est la pratique des devoirs de l’homme.\par
III.\par
Il met au premier rang de ces devoirs de détester la mauvaise foi et la tyrannie, de punir les tyrans et les traîtres, de secourir les malheureux, de respecter les faibles, de défendre les opprimés, de faire aux autres tout le bien qu’on peut, et de n’être injuste envers personne.\par
IV.\par
Il sera institué des fêtes pour rappeler l’homme à la pensée de la Divinité et à la dignité de son être.\par
V.\par
Elles emprunteront leurs noms des événements glorieux de notre Révolution, des vertus les plus chères et les plus utiles à l’homme, des plus grands bienfaits de la nature.\par
VI.\par
La République française célébrera tous les ans les fêtes du 14 juillet 1789, du 10 août 1792, du 21 janvier 1793, du 31 mai 1793.\par
VII.\par
Elle célébrera, aux jours de décadi, les fêtes dont l’énumération suit :\par
À l’Être suprême et à la Nature. Au Genre humain. Au Peuple français. Aux bienfaiteurs de l’humanité. Aux Martyrs de la liberté. À la Liberté et à l’Égalité. À la République. À la liberté du monde. À l’amour de la patrie. À la haine des tyrans et des traîtres. À la Vérité. À la Justice. À la Pudeur. À la Gloire et à l’Immortalité. À l’Amitié. À la Frugalité. Au Courage. À la Bonne Foi. À l’Héroïsme. Au Désintéressement. Au Stoïcisme. À l’Amour. À la Foi conjugale. À l’Amour paternel. À la Tendresse maternelle. À la Piété filiale. À l’Enfance. À la Jeunesse. À l’Âge viril. À la Vieillesse. Au Malheur. À l’Agriculture. À l’Industrie. À nos Aïeux. À la Postérité. Au Bonheur.\par
VIII.\par
Les Comités de salut public et d’instruction publique sont chargés de présenter un plan d’organisation de ces fêtes.\par
IX.\par
La Convention nationale appelle tous les talents dignes de servir la cause de l’humanité à l’honneur de concourir à leur établissement par des hymnes et des chants civiques, et par tous les moyens qui peuvent contribuer à leur embellissement et à leur utilité.\par
X.\par
Le Comité de salut public distinguera les ouvrages qui lui paraîtront les plus propres à remplir cet objet, et récompensera leurs auteurs.\par
XI.\par
La liberté des cultes est maintenue conformément au décret du 18 frimaire.\par
XII.\par
Tout rassemblement aristocratique et contraire à l’ordre public sera réprimé.\par
XIII.\par
En cas de troubles, dont un culte quelconque serait l’occasion ou le motif, ceux qui les exciteraient par des prédications fanatiques ou par des insinuations contre-révolutionnaires, ceux qui les provoqueraient par des violences injustes et gratuites, seront également punis selon la rigueur des lois.\par
XIV.\par
Il sera fait un rapport particulier sur les dispositions de détail \phantomsection
\label{\_GoBack}relatives au présent décret.\par
XV.\par
Il sera célébré, le 20 prairial prochain, une fête nationale en l’honneur de l’Être suprême.
 


% at least one empty page at end (for booklet couv)
\ifbooklet
  \pagestyle{empty}
  \clearpage
  % 2 empty pages maybe needed for 4e cover
  \ifnum\modulo{\value{page}}{4}=0 \hbox{}\newpage\hbox{}\newpage\fi
  \ifnum\modulo{\value{page}}{4}=1 \hbox{}\newpage\hbox{}\newpage\fi


  \hbox{}\newpage
  \ifodd\value{page}\hbox{}\newpage\fi
  {\centering\color{rubric}\bfseries\noindent\large
    Hurlus ? Qu’est-ce.\par
    \bigskip
  }
  \noindent Des bouquinistes électroniques, pour du texte libre à participation libre,
  téléchargeable gratuitement sur \href{https://hurlus.fr}{\dotuline{hurlus.fr}}.\par
  \bigskip
  \noindent Cette brochure a été produite par des éditeurs bénévoles.
  Elle n’est pas faîte pour être possédée, mais pour être lue, et puis donnée.
  Que circule le texte !
  En page de garde, on peut ajouter une date, un lieu, un nom ; pour suivre le voyage des idées.
  \par

  Ce texte a été choisi parce qu’une personne l’a aimé,
  ou haï, elle a en tous cas pensé qu’il partipait à la formation de notre présent ;
  sans le souci de plaire, vendre, ou militer pour une cause.
  \par

  L’édition électronique est soigneuse, tant sur la technique
  que sur l’établissement du texte ; mais sans aucune prétention scolaire, au contraire.
  Le but est de s’adresser à tous, sans distinction de science ou de diplôme.
  Au plus direct ! (possible)
  \par

  Cet exemplaire en papier a été tiré sur une imprimante personnelle
   ou une photocopieuse. Tout le monde peut le faire.
  Il suffit de
  télécharger un fichier sur \href{https://hurlus.fr}{\dotuline{hurlus.fr}},
  d’imprimer, et agrafer ; puis de lire et donner.\par

  \bigskip

  \noindent PS : Les hurlus furent aussi des rebelles protestants qui cassaient les statues dans les églises catholiques. En 1566 démarra la révolte des gueux dans le pays de Lille. L’insurrection enflamma la région jusqu’à Anvers où les gueux de mer bloquèrent les bateaux espagnols.
  Ce fut une rare guerre de libération dont naquit un pays toujours libre : les Pays-Bas.
  En plat pays francophone, par contre, restèrent des bandes de huguenots, les hurlus, progressivement réprimés par la très catholique Espagne.
  Cette mémoire d’une défaite est éteinte, rallumons-la. Sortons les livres du culte universitaire, cherchons les idoles de l’époque, pour les briser.
\fi

\ifdev % autotext in dev mode
\fontname\font — \textsc{Les règles du jeu}\par
(\hyperref[utopie]{\underline{Lien}})\par
\noindent \initialiv{A}{lors là}\blindtext\par
\noindent \initialiv{À}{ la bonheur des dames}\blindtext\par
\noindent \initialiv{É}{tonnez-le}\blindtext\par
\noindent \initialiv{Q}{ualitativement}\blindtext\par
\noindent \initialiv{V}{aloriser}\blindtext\par
\Blindtext
\phantomsection
\label{utopie}
\Blinddocument
\fi
\end{document}
