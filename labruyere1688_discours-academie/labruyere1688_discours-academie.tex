%%%%%%%%%%%%%%%%%%%%%%%%%%%%%%%%%
% LaTeX model https://hurlus.fr %
%%%%%%%%%%%%%%%%%%%%%%%%%%%%%%%%%

% Needed before document class
\RequirePackage{pdftexcmds} % needed for tests expressions
\RequirePackage{fix-cm} % correct units

% Define mode
\def\mode{a4}

\newif\ifaiv % a4
\newif\ifav % a5
\newif\ifbooklet % booklet
\newif\ifcover % cover for booklet

\ifnum \strcmp{\mode}{cover}=0
  \covertrue
\else\ifnum \strcmp{\mode}{booklet}=0
  \booklettrue
\else\ifnum \strcmp{\mode}{a5}=0
  \avtrue
\else
  \aivtrue
\fi\fi\fi

\ifbooklet % do not enclose with {}
  \documentclass[french,twoside]{book} % ,notitlepage
  \usepackage[%
    papersize={105mm, 297mm},
    inner=12mm,
    outer=12mm,
    top=20mm,
    bottom=15mm,
    marginparsep=0pt,
  ]{geometry}
  \usepackage[fontsize=9.5pt]{scrextend} % for Roboto
\else\ifav
  \documentclass[french,twoside]{book} % ,notitlepage
  \usepackage[%
    a5paper,
    inner=25mm,
    outer=15mm,
    top=15mm,
    bottom=15mm,
    marginparsep=0pt,
  ]{geometry}
  \usepackage[fontsize=12pt]{scrextend}
\else% A4 2 cols
  \documentclass[twocolumn]{report}
  \usepackage[%
    a4paper,
    inner=15mm,
    outer=10mm,
    top=25mm,
    bottom=18mm,
    marginparsep=0pt,
  ]{geometry}
  \setlength{\columnsep}{20mm}
  \usepackage[fontsize=9.5pt]{scrextend}
\fi\fi

%%%%%%%%%%%%%%
% Alignments %
%%%%%%%%%%%%%%
% before teinte macros

\setlength{\arrayrulewidth}{0.2pt}
\setlength{\columnseprule}{\arrayrulewidth} % twocol
\setlength{\parskip}{0pt} % classical para with no margin
\setlength{\parindent}{1.5em}

%%%%%%%%%%
% Colors %
%%%%%%%%%%
% before Teinte macros

\usepackage[dvipsnames]{xcolor}
\definecolor{rubric}{HTML}{800000} % the tonic 0c71c3
\def\columnseprulecolor{\color{rubric}}
\colorlet{borderline}{rubric!30!} % definecolor need exact code
\definecolor{shadecolor}{gray}{0.95}
\definecolor{bghi}{gray}{0.5}

%%%%%%%%%%%%%%%%%
% Teinte macros %
%%%%%%%%%%%%%%%%%
%%%%%%%%%%%%%%%%%%%%%%%%%%%%%%%%%%%%%%%%%%%%%%%%%%%
% <TEI> generic (LaTeX names generated by Teinte) %
%%%%%%%%%%%%%%%%%%%%%%%%%%%%%%%%%%%%%%%%%%%%%%%%%%%
% This template is inserted in a specific design
% It is XeLaTeX and otf fonts

\makeatletter % <@@@


\usepackage{blindtext} % generate text for testing
\usepackage[strict]{changepage} % for modulo 4
\usepackage{contour} % rounding words
\usepackage[nodayofweek]{datetime}
% \usepackage{DejaVuSans} % seems buggy for sffont font for symbols
\usepackage{enumitem} % <list>
\usepackage{etoolbox} % patch commands
\usepackage{fancyvrb}
\usepackage{fancyhdr}
\usepackage{float}
\usepackage{fontspec} % XeLaTeX mandatory for fonts
\usepackage{footnote} % used to capture notes in minipage (ex: quote)
\usepackage{framed} % bordering correct with footnote hack
\usepackage{graphicx}
\usepackage{lettrine} % drop caps
\usepackage{lipsum} % generate text for testing
\usepackage[framemethod=tikz,]{mdframed} % maybe used for frame with footnotes inside
\usepackage{pdftexcmds} % needed for tests expressions
\usepackage{polyglossia} % non-break space french punct, bug Warning: "Failed to patch part"
\usepackage[%
  indentfirst=false,
  vskip=1em,
  noorphanfirst=true,
  noorphanafter=true,
  leftmargin=\parindent,
  rightmargin=0pt,
]{quoting}
\usepackage{ragged2e}
\usepackage{setspace} % \setstretch for <quote>
\usepackage{tabularx} % <table>
\usepackage[explicit]{titlesec} % wear titles, !NO implicit
\usepackage{tikz} % ornaments
\usepackage{tocloft} % styling tocs
\usepackage[fit]{truncate} % used im runing titles
\usepackage{unicode-math}
\usepackage[normalem]{ulem} % breakable \uline, normalem is absolutely necessary to keep \emph
\usepackage{verse} % <l>
\usepackage{xcolor} % named colors
\usepackage{xparse} % @ifundefined
\XeTeXdefaultencoding "iso-8859-1" % bad encoding of xstring
\usepackage{xstring} % string tests
\XeTeXdefaultencoding "utf-8"
\PassOptionsToPackage{hyphens}{url} % before hyperref, which load url package

% TOTEST
% \usepackage{hypcap} % links in caption ?
% \usepackage{marginnote}
% TESTED
% \usepackage{background} % doesn’t work with xetek
% \usepackage{bookmark} % prefers the hyperref hack \phantomsection
% \usepackage[color, leftbars]{changebar} % 2 cols doc, impossible to keep bar left
% \usepackage[utf8x]{inputenc} % inputenc package ignored with utf8 based engines
% \usepackage[sfdefault,medium]{inter} % no small caps
% \usepackage{firamath} % choose firasans instead, firamath unavailable in Ubuntu 21-04
% \usepackage{flushend} % bad for last notes, supposed flush end of columns
% \usepackage[stable]{footmisc} % BAD for complex notes https://texfaq.org/FAQ-ftnsect
% \usepackage{helvet} % not for XeLaTeX
% \usepackage{multicol} % not compatible with too much packages (longtable, framed, memoir…)
% \usepackage[default,oldstyle,scale=0.95]{opensans} % no small caps
% \usepackage{sectsty} % \chapterfont OBSOLETE
% \usepackage{soul} % \ul for underline, OBSOLETE with XeTeX
% \usepackage[breakable]{tcolorbox} % text styling gone, footnote hack not kept with breakable


% Metadata inserted by a program, from the TEI source, for title page and runing heads
\title{\textbf{ Discours de réception à l’académie française }}
\date{1688}
\author{La Bruyère, Jean de}
\def\elbibl{La Bruyère, Jean de. 1688. \emph{Discours de réception à l’académie française}}

% Default metas
\newcommand{\colorprovide}[2]{\@ifundefinedcolor{#1}{\colorlet{#1}{#2}}{}}
\colorprovide{rubric}{red}
\colorprovide{silver}{lightgray}
\@ifundefined{syms}{\newfontfamily\syms{DejaVu Sans}}{}
\newif\ifdev
\@ifundefined{elbibl}{% No meta defined, maybe dev mode
  \newcommand{\elbibl}{Titre court ?}
  \newcommand{\elbook}{Titre du livre source ?}
  \newcommand{\elabstract}{Résumé\par}
  \newcommand{\elurl}{http://oeuvres.github.io/elbook/2}
  \author{Éric Lœchien}
  \title{Un titre de test assez long pour vérifier le comportement d’une maquette}
  \date{1566}
  \devtrue
}{}
\let\eltitle\@title
\let\elauthor\@author
\let\eldate\@date


\defaultfontfeatures{
  % Mapping=tex-text, % no effect seen
  Scale=MatchLowercase,
  Ligatures={TeX,Common},
}


% generic typo commands
\newcommand{\astermono}{\medskip\centerline{\color{rubric}\large\selectfont{\syms ✻}}\medskip\par}%
\newcommand{\astertri}{\medskip\par\centerline{\color{rubric}\large\selectfont{\syms ✻\,✻\,✻}}\medskip\par}%
\newcommand{\asterism}{\bigskip\par\noindent\parbox{\linewidth}{\centering\color{rubric}\large{\syms ✻}\\{\syms ✻}\hskip 0.75em{\syms ✻}}\bigskip\par}%

% lists
\newlength{\listmod}
\setlength{\listmod}{\parindent}
\setlist{
  itemindent=!,
  listparindent=\listmod,
  labelsep=0.2\listmod,
  parsep=0pt,
  % topsep=0.2em, % default topsep is best
}
\setlist[itemize]{
  label=—,
  leftmargin=0pt,
  labelindent=1.2em,
  labelwidth=0pt,
}
\setlist[enumerate]{
  label={\bf\color{rubric}\arabic*.},
  labelindent=0.8\listmod,
  leftmargin=\listmod,
  labelwidth=0pt,
}
\newlist{listalpha}{enumerate}{1}
\setlist[listalpha]{
  label={\bf\color{rubric}\alph*.},
  leftmargin=0pt,
  labelindent=0.8\listmod,
  labelwidth=0pt,
}
\newcommand{\listhead}[1]{\hspace{-1\listmod}\emph{#1}}

\renewcommand{\hrulefill}{%
  \leavevmode\leaders\hrule height 0.2pt\hfill\kern\z@}

% General typo
\DeclareTextFontCommand{\textlarge}{\large}
\DeclareTextFontCommand{\textsmall}{\small}

% commands, inlines
\newcommand{\anchor}[1]{\Hy@raisedlink{\hypertarget{#1}{}}} % link to top of an anchor (not baseline)
\newcommand\abbr[1]{#1}
\newcommand{\autour}[1]{\tikz[baseline=(X.base)]\node [draw=rubric,thin,rectangle,inner sep=1.5pt, rounded corners=3pt] (X) {\color{rubric}#1};}
\newcommand\corr[1]{#1}
\newcommand{\ed}[1]{ {\color{silver}\sffamily\footnotesize (#1)} } % <milestone ed="1688"/>
\newcommand\expan[1]{#1}
\newcommand\foreign[1]{\emph{#1}}
\newcommand\gap[1]{#1}
\renewcommand{\LettrineFontHook}{\color{rubric}}
\newcommand{\initial}[2]{\lettrine[lines=2, loversize=0.3, lhang=0.3]{#1}{#2}}
\newcommand{\initialiv}[2]{%
  \let\oldLFH\LettrineFontHook
  % \renewcommand{\LettrineFontHook}{\color{rubric}\ttfamily}
  \IfSubStr{QJ’}{#1}{
    \lettrine[lines=4, lhang=0.2, loversize=-0.1, lraise=0.2]{\smash{#1}}{#2}
  }{\IfSubStr{É}{#1}{
    \lettrine[lines=4, lhang=0.2, loversize=-0, lraise=0]{\smash{#1}}{#2}
  }{\IfSubStr{ÀÂ}{#1}{
    \lettrine[lines=4, lhang=0.2, loversize=-0, lraise=0, slope=0.6em]{\smash{#1}}{#2}
  }{\IfSubStr{A}{#1}{
    \lettrine[lines=4, lhang=0.2, loversize=0.2, slope=0.6em]{\smash{#1}}{#2}
  }{\IfSubStr{V}{#1}{
    \lettrine[lines=4, lhang=0.2, loversize=0.2, slope=-0.5em]{\smash{#1}}{#2}
  }{
    \lettrine[lines=4, lhang=0.2, loversize=0.2]{\smash{#1}}{#2}
  }}}}}
  \let\LettrineFontHook\oldLFH
}
\newcommand{\labelchar}[1]{\textbf{\color{rubric} #1}}
\newcommand{\milestone}[1]{\autour{\footnotesize\color{rubric} #1}} % <milestone n="4"/>
\newcommand\name[1]{#1}
\newcommand\orig[1]{#1}
\newcommand\orgName[1]{#1}
\newcommand\persName[1]{#1}
\newcommand\placeName[1]{#1}
\newcommand{\pn}[1]{\IfSubStr{-—–¶}{#1}% <p n="3"/>
  {\noindent{\bfseries\color{rubric}   ¶  }}
  {{\footnotesize\autour{ #1}  }}}
\newcommand\reg{}
% \newcommand\ref{} % already defined
\newcommand\sic[1]{#1}
\newcommand\surname[1]{\textsc{#1}}
\newcommand\term[1]{\textbf{#1}}

\def\mednobreak{\ifdim\lastskip<\medskipamount
  \removelastskip\nopagebreak\medskip\fi}
\def\bignobreak{\ifdim\lastskip<\bigskipamount
  \removelastskip\nopagebreak\bigskip\fi}

% commands, blocks
\newcommand{\byline}[1]{\bigskip{\RaggedLeft{#1}\par}\bigskip}
\newcommand{\bibl}[1]{{\RaggedLeft{#1}\par\bigskip}}
\newcommand{\biblitem}[1]{{\noindent\hangindent=\parindent   #1\par}}
\newcommand{\dateline}[1]{\medskip{\RaggedLeft{#1}\par}\bigskip}
\newcommand{\labelblock}[1]{\medbreak{\noindent\color{rubric}\bfseries #1}\par\mednobreak}
\newcommand{\salute}[1]{\bigbreak{#1}\par\medbreak}
\newcommand{\signed}[1]{\bigbreak\filbreak{\raggedleft #1\par}\medskip}

% environments for blocks (some may become commands)
\newenvironment{borderbox}{}{} % framing content
\newenvironment{citbibl}{\ifvmode\hfill\fi}{\ifvmode\par\fi }
\newenvironment{docAuthor}{\ifvmode\vskip4pt\fontsize{16pt}{18pt}\selectfont\fi\itshape}{\ifvmode\par\fi }
\newenvironment{docDate}{}{\ifvmode\par\fi }
\newenvironment{docImprint}{\vskip6pt}{\ifvmode\par\fi }
\newenvironment{docTitle}{\vskip6pt\bfseries\fontsize{18pt}{22pt}\selectfont}{\par }
\newenvironment{msHead}{\vskip6pt}{\par}
\newenvironment{msItem}{\vskip6pt}{\par}
\newenvironment{titlePart}{}{\par }


% environments for block containers
\newenvironment{argument}{\itshape\parindent0pt}{\vskip1.5em}
\newenvironment{biblfree}{}{\ifvmode\par\fi }
\newenvironment{bibitemlist}[1]{%
  \list{\@biblabel{\@arabic\c@enumiv}}%
  {%
    \settowidth\labelwidth{\@biblabel{#1}}%
    \leftmargin\labelwidth
    \advance\leftmargin\labelsep
    \@openbib@code
    \usecounter{enumiv}%
    \let\p@enumiv\@empty
    \renewcommand\theenumiv{\@arabic\c@enumiv}%
  }
  \sloppy
  \clubpenalty4000
  \@clubpenalty \clubpenalty
  \widowpenalty4000%
  \sfcode`\.\@m
}%
{\def\@noitemerr
  {\@latex@warning{Empty `bibitemlist' environment}}%
\endlist}
\newenvironment{quoteblock}% may be used for ornaments
  {\begin{quoting}}
  {\end{quoting}}

% table () is preceded and finished by custom command
\newcommand{\tableopen}[1]{%
  \ifnum\strcmp{#1}{wide}=0{%
    \begin{center}
  }
  \else\ifnum\strcmp{#1}{long}=0{%
    \begin{center}
  }
  \else{%
    \begin{center}
  }
  \fi\fi
}
\newcommand{\tableclose}[1]{%
  \ifnum\strcmp{#1}{wide}=0{%
    \end{center}
  }
  \else\ifnum\strcmp{#1}{long}=0{%
    \end{center}
  }
  \else{%
    \end{center}
  }
  \fi\fi
}


% text structure
\newcommand\chapteropen{} % before chapter title
\newcommand\chaptercont{} % after title, argument, epigraph…
\newcommand\chapterclose{} % maybe useful for multicol settings
\setcounter{secnumdepth}{-2} % no counters for hierarchy titles
\setcounter{tocdepth}{5} % deep toc
\markright{\@title} % ???
\markboth{\@title}{\@author} % ???
\renewcommand\tableofcontents{\@starttoc{toc}}
% toclof format
% \renewcommand{\@tocrmarg}{0.1em} % Useless command?
% \renewcommand{\@pnumwidth}{0.5em} % {1.75em}
\renewcommand{\@cftmaketoctitle}{}
\setlength{\cftbeforesecskip}{\z@ \@plus.2\p@}
\renewcommand{\cftchapfont}{}
\renewcommand{\cftchapdotsep}{\cftdotsep}
\renewcommand{\cftchapleader}{\normalfont\cftdotfill{\cftchapdotsep}}
\renewcommand{\cftchappagefont}{\bfseries}
\setlength{\cftbeforechapskip}{0em \@plus\p@}
% \renewcommand{\cftsecfont}{\small\relax}
\renewcommand{\cftsecpagefont}{\normalfont}
% \renewcommand{\cftsubsecfont}{\small\relax}
\renewcommand{\cftsecdotsep}{\cftdotsep}
\renewcommand{\cftsecpagefont}{\normalfont}
\renewcommand{\cftsecleader}{\normalfont\cftdotfill{\cftsecdotsep}}
\setlength{\cftsecindent}{1em}
\setlength{\cftsubsecindent}{2em}
\setlength{\cftsubsubsecindent}{3em}
\setlength{\cftchapnumwidth}{1em}
\setlength{\cftsecnumwidth}{1em}
\setlength{\cftsubsecnumwidth}{1em}
\setlength{\cftsubsubsecnumwidth}{1em}

% footnotes
\newif\ifheading
\newcommand*{\fnmarkscale}{\ifheading 0.70 \else 1 \fi}
\renewcommand\footnoterule{\vspace*{0.3cm}\hrule height \arrayrulewidth width 3cm \vspace*{0.3cm}}
\setlength\footnotesep{1.5\footnotesep} % footnote separator
\renewcommand\@makefntext[1]{\parindent 1.5em \noindent \hb@xt@1.8em{\hss{\normalfont\@thefnmark . }}#1} % no superscipt in foot
\patchcmd{\@footnotetext}{\footnotesize}{\footnotesize\sffamily}{}{} % before scrextend, hyperref


%   see https://tex.stackexchange.com/a/34449/5049
\def\truncdiv#1#2{((#1-(#2-1)/2)/#2)}
\def\moduloop#1#2{(#1-\truncdiv{#1}{#2}*#2)}
\def\modulo#1#2{\number\numexpr\moduloop{#1}{#2}\relax}

% orphans and widows
\clubpenalty=9996
\widowpenalty=9999
\brokenpenalty=4991
\predisplaypenalty=10000
\postdisplaypenalty=1549
\displaywidowpenalty=1602
\hyphenpenalty=400
% Copied from Rahtz but not understood
\def\@pnumwidth{1.55em}
\def\@tocrmarg {2.55em}
\def\@dotsep{4.5}
\emergencystretch 3em
\hbadness=4000
\pretolerance=750
\tolerance=2000
\vbadness=4000
\def\Gin@extensions{.pdf,.png,.jpg,.mps,.tif}
% \renewcommand{\@cite}[1]{#1} % biblio

\usepackage{hyperref} % supposed to be the last one, :o) except for the ones to follow
\urlstyle{same} % after hyperref
\hypersetup{
  % pdftex, % no effect
  pdftitle={\elbibl},
  % pdfauthor={Your name here},
  % pdfsubject={Your subject here},
  % pdfkeywords={keyword1, keyword2},
  bookmarksnumbered=true,
  bookmarksopen=true,
  bookmarksopenlevel=1,
  pdfstartview=Fit,
  breaklinks=true, % avoid long links
  pdfpagemode=UseOutlines,    % pdf toc
  hyperfootnotes=true,
  colorlinks=false,
  pdfborder=0 0 0,
  % pdfpagelayout=TwoPageRight,
  % linktocpage=true, % NO, toc, link only on page no
}

\makeatother % /@@@>
%%%%%%%%%%%%%%
% </TEI> end %
%%%%%%%%%%%%%%


%%%%%%%%%%%%%
% footnotes %
%%%%%%%%%%%%%
\renewcommand{\thefootnote}{\bfseries\textcolor{rubric}{\arabic{footnote}}} % color for footnote marks

%%%%%%%%%
% Fonts %
%%%%%%%%%
\usepackage[]{roboto} % SmallCaps, Regular is a bit bold
% \linespread{0.90} % too compact, keep font natural
\newfontfamily\fontrun[]{Roboto Condensed Light} % condensed runing heads
\ifav
  \setmainfont[
    ItalicFont={Roboto Light Italic},
  ]{Roboto}
\else\ifbooklet
  \setmainfont[
    ItalicFont={Roboto Light Italic},
  ]{Roboto}
\else
\setmainfont[
  ItalicFont={Roboto Italic},
]{Roboto Light}
\fi\fi
\renewcommand{\LettrineFontHook}{\bfseries\color{rubric}}
% \renewenvironment{labelblock}{\begin{center}\bfseries\color{rubric}}{\end{center}}

%%%%%%%%
% MISC %
%%%%%%%%

\setdefaultlanguage[frenchpart=false]{french} % bug on part


\newenvironment{quotebar}{%
    \def\FrameCommand{{\color{rubric!10!}\vrule width 0.5em} \hspace{0.9em}}%
    \def\OuterFrameSep{\itemsep} % séparateur vertical
    \MakeFramed {\advance\hsize-\width \FrameRestore}
  }%
  {%
    \endMakeFramed
  }
\renewenvironment{quoteblock}% may be used for ornaments
  {%
    \savenotes
    \setstretch{0.9}
    \normalfont
    \begin{quotebar}
  }
  {%
    \end{quotebar}
    \spewnotes
  }


\renewcommand{\headrulewidth}{\arrayrulewidth}
\renewcommand{\headrule}{{\color{rubric}\hrule}}

% delicate tuning, image has produce line-height problems in title on 2 lines
\titleformat{name=\chapter} % command
  [display] % shape
  {\vspace{1.5em}\centering} % format
  {} % label
  {0pt} % separator between n
  {}
[{\color{rubric}\huge\textbf{#1}}\bigskip] % after code
% \titlespacing{command}{left spacing}{before spacing}{after spacing}[right]
\titlespacing*{\chapter}{0pt}{-2em}{0pt}[0pt]

\titleformat{name=\section}
  [block]{}{}{}{}
  [\vbox{\color{rubric}\large\raggedleft\textbf{#1}}]
\titlespacing{\section}{0pt}{0pt plus 4pt minus 2pt}{\baselineskip}

\titleformat{name=\subsection}
  [block]
  {}
  {} % \thesection
  {} % separator \arrayrulewidth
  {}
[\vbox{\large\textbf{#1}}]
% \titlespacing{\subsection}{0pt}{0pt plus 4pt minus 2pt}{\baselineskip}

\ifaiv
  \fancypagestyle{main}{%
    \fancyhf{}
    \setlength{\headheight}{1.5em}
    \fancyhead{} % reset head
    \fancyfoot{} % reset foot
    \fancyhead[L]{\truncate{0.45\headwidth}{\fontrun\elbibl}} % book ref
    \fancyhead[R]{\truncate{0.45\headwidth}{ \fontrun\nouppercase\leftmark}} % Chapter title
    \fancyhead[C]{\thepage}
  }
  \fancypagestyle{plain}{% apply to chapter
    \fancyhf{}% clear all header and footer fields
    \setlength{\headheight}{1.5em}
    \fancyhead[L]{\truncate{0.9\headwidth}{\fontrun\elbibl}}
    \fancyhead[R]{\thepage}
  }
\else
  \fancypagestyle{main}{%
    \fancyhf{}
    \setlength{\headheight}{1.5em}
    \fancyhead{} % reset head
    \fancyfoot{} % reset foot
    \fancyhead[RE]{\truncate{0.9\headwidth}{\fontrun\elbibl}} % book ref
    \fancyhead[LO]{\truncate{0.9\headwidth}{\fontrun\nouppercase\leftmark}} % Chapter title, \nouppercase needed
    \fancyhead[RO,LE]{\thepage}
  }
  \fancypagestyle{plain}{% apply to chapter
    \fancyhf{}% clear all header and footer fields
    \setlength{\headheight}{1.5em}
    \fancyhead[L]{\truncate{0.9\headwidth}{\fontrun\elbibl}}
    \fancyhead[R]{\thepage}
  }
\fi

\ifav % a5 only
  \titleclass{\section}{top}
\fi

\newcommand\chapo{{%
  \vspace*{-3em}
  \centering % no vskip ()
  {\Large\addfontfeature{LetterSpace=25}\bfseries{\elauthor}}\par
  \smallskip
  {\large\eldate}\par
  \bigskip
  {\Large\selectfont{\eltitle}}\par
  \bigskip
  {\color{rubric}\hline\par}
  \bigskip
  {\Large TEXTE LIBRE À PARTICPATION LIBRE\par}
  \centerline{\small\color{rubric} {hurlus.fr, tiré le \today}}\par
  \bigskip
}}

\newcommand\cover{{%
  \thispagestyle{empty}
  \centering
  {\LARGE\bfseries{\elauthor}}\par
  \bigskip
  {\Large\eldate}\par
  \bigskip
  \bigskip
  {\LARGE\selectfont{\eltitle}}\par
  \vfill\null
  {\color{rubric}\setlength{\arrayrulewidth}{2pt}\hline\par}
  \vfill\null
  {\Large TEXTE LIBRE À PARTICPATION LIBRE\par}
  \centerline{{\href{https://hurlus.fr}{\dotuline{hurlus.fr}}, tiré le \today}}\par
}}

\begin{document}
\pagestyle{empty}
\ifbooklet{
  \cover\newpage
  \thispagestyle{empty}\hbox{}\newpage
  \cover\newpage\noindent Les voyages de la brochure\par
  \bigskip
  \begin{tabularx}{\textwidth}{l|X|X}
    \textbf{Date} & \textbf{Lieu}& \textbf{Nom/pseudo} \\ \hline
    \rule{0pt}{25cm} &  &   \\
  \end{tabularx}
  \newpage
  \addtocounter{page}{-4}
}\fi

\thispagestyle{empty}
\ifaiv
  \twocolumn[\chapo]
\else
  \chapo
\fi
{\it\elabstract}
\bigskip
\makeatletter\@starttoc{toc}\makeatother % toc without new page
\bigskip

\pagestyle{main} % after style

  
\chapteropen
\chapter[{Préface}]{Préface}\phantomsection
\label{acad-preface}\renewcommand{\leftmark}{Préface}


\chaptercont
\noindent Ceux qui, interrogés sur le discours que je fis à l’Académie française, le jour que j’eus l’honneur d’y être reçu, ont dit sèchement que j’avais fait des caractères, croyant le blâmer, en ont donné l’idée la plus avantageuse que je pouvais moi-même désirer ; car le public ayant approuvé ce genre d’écrire où je me suis appliqué depuis quelques années, c’était le prévenir en ma faveur que de faire une telle réponse. Il ne restait plus que de savoir si je n’aurais pas dû renoncer aux caractères dans le discours dont il s’agissait ; et cette question s’évanouit dès qu’on sait que l’usage a prévalu qu’un nouvel académicien compose celui qu’il doit prononcer, le jour de sa réception, de l’éloge du Roi, de ceux du cardinal de Richelieu, du chancelier Seguier, de la personne à qui il succède, et de l’Académie française. De ces cinq éloges, il y en a quatre de personnels ; or je demande à mes censeurs qu’ils me posent si bien la différence qu’il y a des éloges personnels aux caractères qui louent, que je la puisse sentir, et avouer ma faute. Si, chargé de faire quelque autre harangue, je retombe encore dans des peintures, c’est alors qu’on pourra écouter leur critique, et peut-être me condamner ; je dis peut-être, puisque les caractères, ou du moins les images des choses et des personnes, sont inévitables dans l’oraison, que tout écrivain est peintre, et tout excellent écrivain excellent peintre.\par
J'avoue que j’ai ajouté à ces tableaux, qui étaient de commande, les louanges de chacun des hommes illustres qui composent l’Académie française ; et ils ont dû me le pardonner, s’ils ont fait attention qu’autant pour ménager leur pudeur que pour éviter les caractères, je me suis abstenu de toucher à leurs personnes, pour ne parler que de leurs ouvrages, dont j’ai fait des éloges publics plus ou moins étendus, selon que les sujets qu’ils y ont traités pouvaient l’exiger.—J'ai loué des académiciens encore vivants, disent quelques-uns.—Il est vrai ; mais je les ai loués tous : qui d’entre eux aurait une raison de se plaindre ?— C'est une coutume toute nouvelle, ajoutent-ils, et qui n’avait point encore eu d’exemple.—Je veux en convenir, et que j’ai pris soin de m’écarter des lieux communs et des phrases proverbiales usées depuis si longtemps, pour avoir servi à un nombre infini de pareils discours depuis la naissance de l’Académie française. M'était-il donc si difficile de faire entrer Rome et Athènes, le Lycée et le Portique, dans l’éloge de cette savante compagnie ? Être au comble de ses voeux de se voir académicien ; protester que ce jour où l’on jouit pour la première fois d’un si rare bonheur est le jour le plus beau de sa vie ; douter si cet honneur qu’on vient de recevoir est une chose vraie ou qu’on ait songée ; espérer de puiser désormais à la source les plus pures eaux de l’éloquence française ; n’avoir accepté, n’avoir désiré une telle place que pour profiter des lumières de tant de personnes si éclairées ; promettre que tout indigne de leur choix qu’on se reconnaît, on s’efforcera de s’en rendre digne : cent autres formules de pareils compliments sont-elles si rares et si peu connues que je n’eusse pu les trouver, les placer, et en mériter des applaudissements ?\par
Parce donc que j’ai cru que, quoi que l’envie et l’injustice publient de l’Académie française, quoi qu’elles veuillent dire de son âge d’or et de sa décadence, elle n’a jamais, depuis son établissement, rassemblé un si grand nombre de personnages illustres pour toutes sortes de talents et en tout genre d’érudition, qu’il est facile aujourd’hui d’y en remarquer ; et que dans cette prévention où je suis, je n’ai pas espéré que cette Compagnie pût être une autre fois plus belle à peindre, ni prise dans un jour plus favorable, et que je me suis servi de l’occasion, ai-je rien fait qui doive m’attirer les moindres reproches ? Cicéron a pu louer impunément Brutus, César, Pompée, Marcellus, qui étaient vivants, qui étaient présents : il les a loués plusieurs fois ; il les a loués seuls dans le sénat, souvent en présence de leurs ennemis, toujours devant une compagnie jalouse de leur mérite, et qui avait bien d’autres délicatesses de politique sur la vertu des grands hommes que n’en saurait avoir l’Académie française. J'ai loué les académiciens, je les ai loués tous, et ce n’a pas été impunément : que me serait-il arrivé si je les avais blâmés tous ?\par
Je viens d’entendre, a dit Théobalde, une grande vilaine harangue qui m’a fait bâiller vingt fois, et qui m’a ennuyé à la mort. Voilà ce qu’il a dit, et voilà ensuite ce qu’il a fait, lui et peu d’autres qui ont cru devoir entrer dans les mêmes intérêts. Ils partirent pour la cour le lendemain de la prononciation de ma harangue ; ils allèrent de maisons en maisons ; ils dirent aux personnes auprès de qui ils ont accès que je leur avais balbutié la veille un discours où il n’y avait ni style ni sens commun, qui était rempli d’extravagances, et une vraie satire. Revenus à Paris, ils se cantonnèrent en divers quartiers, où ils répandirent tant de venin contre moi, s’acharnèrent si fort à diffamer cette harangue, soit dans leurs conversations, soit dans les lettres qu’ils écrivirent à leurs amis dans les provinces, en dirent tant de mal, et le persuadèrent si fortement à qui ne l’avait pas entendue, qu’ils crurent pouvoir insinuer au public, ou que les Caractères faits de la même main étaient mauvais, ou que s’ils étaient bons, je n’en étais pas l’auteur, mais qu’une femme de mes amies m’avait fourni ce qu’il y avait de plus supportable. Ils prononcèrent aussi que je n’étais pas capable de faire rien de suivi, pas même la moindre préface : tant ils estimaient impraticable à un homme même qui est dans l’habitude de penser, et d’écrire ce qu’il pense, l’art de lier ses pensées et de faire des transitions.\par
Ils firent plus : violant les lois de l’Académie française, qui défend aux académiciens d’écrire ou de faire écrire contre leurs confrères, ils lâchèrent sur moi deux auteurs associés à une même gazette ; ils les animèrent, non pas à publier contre moi une satire fine et ingénieuse, ouvrage trop au-dessous des uns et des autres, facile à manier, et dont les moindres esprits se trouvent capables, mais à me dire de ces injures grossières et personnelles, si difficiles à rencontrer, si pénibles à prononcer ou à écrire, surtout à des gens à qui je veux croire qu’il reste encore quelque pudeur et quelque soin de leur réputation.\par
Et en vérité je ne doute point que le public ne soit enfin étourdi et fatigué d’entendre, depuis quelques années, de vieux corbeaux croasser autour de ceux qui, d’un vol libre et d’une plume légère, se sont élevés à quelque gloire par leurs écrits. Ces oiseaux lugubres semblent, par leurs cris continuels, leur vouloir imputer le décri universel où tombe nécessairement tout ce qu’ils exposent au grand jour de l’impression : comme si on était cause qu’ils manquent de force et d’haleine, ou qu’on dût être responsable de cette médiocrité répandue sur leurs ouvrages. S'il s’imprime un livre de moeurs assez mal digéré pour tomber de soi-même et ne pas exciter leur jalousie, ils le louent volontiers, et plus volontiers encore ils n’en parlent point ; mais s’il est tel que le monde en parle, ils l’attaquent avec furie. Prose, vers, tout est sujet à leur censure, tout est en proie à une haine implacable, qu’ils ont conçue contre ce qui ose paraître dans quelque perfection, et avec les signes d’une approbation publique. On ne sait plus quelle morale leur fournir qui leur agrée : il faudra leur rendre celle de la Serre ou de des Marets, et s’ils en sont crus, revenir au Pédagogue chrétien et à la Cour sainte. Il paraît une nouvelle satire écrite contre les vices en général, qui, d’un vers fort et d’un style d’airain, enfonce ses traits contre l’avarice, l’excès du jeu, la chicane, la mollesse, l’ordure et l’hypocrisie, où personne n’est nommé ni désigné, où nulle femme vertueuse ne peut ni ne doit se reconnaître ; un Bourdaloue en chaire ne fait point de peintures du crime ni plus vives ni plus innocentes : il n’importe, c’est médisance, c’est calomnie. Voilà depuis quelque temps leur unique ton, celui qu’ils emploient contre les ouvrages de moeurs qui réussissent : ils y prennent tout littéralement, ils les lisent comme une histoire, ils n’y entendent ni la poésie ni la figure ; ainsi ils les condamnent ; ils y trouvent des endroits faibles : il y en a dans Homère, dans Pindare, dans Virgile et dans Horace ; où n’y en a-t-il point ? si ce n’est peut-être dans leurs écrits. Bernin n’a pas manié le marbre ni traité toutes ses figures d’une égale force ; mais on ne laisse pas de voir, dans ce qu’il a moins heureusement rencontré, de certains traits si achevés, tout proche de quelques autres qui le sont moins, qu’ils découvrent aisément l’excellence de l’ouvrier : si c’est un cheval, les crins sont tournés d’une main hardie, ils voltigent et semblent être le jouet du vent ; l’oeil est ardent, les naseaux soufflent le feu et la vie ; un ciseau de maître s’y retrouve en mille endroits ; il n’est pas donné à ses copistes ni à ses envieux d’arriver à de telles fautes par leurs chefs-d’oeuvre : l’on voit bien que c’est quelque chose de manqué par un habile homme, et une faute de Praxitèle.\par
Mais qui sont ceux qui, si tendres et si scrupuleux, ne peuvent même supporter que, sans blesser et sans nommer les vicieux, on se déclare contre le vice ? sont-ce des chartreux et des solitaires ? sont-ce les jésuites, hommes pieux et éclairés ? sont-ce ces hommes religieux qui habitent en France les cloîtres et les abbayes ? Tous au contraire lisent ces sortes d’ouvrages, et en particulier, et en public, à leurs récréations ; ils en inspirent la lecture à leurs pensionnaires, à leurs élèves ; ils en dépeuplent les boutiques, ils les conservent dans leurs bibliothèques. N'ont-ils pas les premiers reconnu le plan et l’économie du livre des Caractères ? N'ont-ils pas observé que de seize chapitres qui le composent, il y en a quinze qui, s’attachant à découvrir le faux et le ridicule qui se rencontrent dans les objets des passions et des attachements humains, ne tendent qu’à ruiner tous les obstacles qui affaiblissent d’abord, et qui éteignent ensuite dans tous les hommes la connaissance de Dieu ; qu’ainsi ils ne sont que des préparations au seizième et dernier chapitre, où l’athéisme est attaqué, et peut-être confondu ; où les preuves de Dieu, une partie du moins de celles que les faibles hommes sont capables de recevoir dans leur esprit, sont apportées ; où la providence de Dieu est défendue contre l’insulte et les plaintes des libertins ? Qui sont donc ceux qui osent répéter contre un ouvrage si sérieux et si utile ce continuel refrain : C'est médisance, c’est calomnie ? Il faut les nommer : ce sont des poètes ; mais quels poètes ? Des auteurs d’hymnes sacrés ou des traducteurs de psaumes, des Godeaux ou des Corneilles ? Non, mais des faiseurs de stances et d’élégies amoureuses, de ces beaux esprits qui tournent un sonnet sur une absence ou sur un retour, qui font une épigramme sur une belle gorge, et un madrigal sur une jouissance. Voilà ceux qui, par délicatesse de conscience, ne souffrent qu’impatiemment qu’en ménageant les particuliers avec toutes les précautions que la prudence peut suggérer, j’essaye, dans mon livre des Moeurs, de décrier, s’il est possible, tous les vices du coeur et de l’esprit, de rendre l’homme raisonnable et plus proche de devenir chrétien. Tels ont été les Théobaldes, ou ceux du moins qui travaillent sous eux et dans leur atelier.\par
Ils sont encore allés plus loin ; car palliant d’une politique zélée le chagrin de ne se sentir pas à leur gré si bien loués et si longtemps que chacun des autres académiciens, ils ont osé faire des applications délicates et dangereuses de l’endroit de ma harangue où, m’exposant seul à prendre le parti de toute la littérature contre leurs plus irréconciliables ennemis, gens pécunieux, que l’excès d’argent ou qu’une fortune faite par de certaines voies, jointe à la faveur des grands, qu’elle leur attire nécessairement, mène jusqu’à une froide insolence, je leur fais à la vérité à tous une vive apostrophe, mais qu’il n’est pas permis de détourner de dessus eux pour la rejeter sur un seul, et sur tout autre.\par
Ainsi en usent à mon égard, excités peut-être par les Théobaldes, ceux qui, se persuadant qu’un auteur écrit seulement pour les amuser par la satire, et point du tout pour les instruire par une saine morale, au lieu de prendre pour eux et de faire servir à la correction de leurs moeurs les divers traits qui sont semés dans un ouvrage, s’appliquent à découvrir, s’ils le peuvent, quels de leurs amis ou de leurs ennemis ces traits peuvent regarder, négligent dans un livre tout ce qui n’est que remarques solides ou sérieuses réflexions, quoique en si grand nombre qu’elles le composent presque tout entier, pour ne s’arrêter qu’aux peintures ou aux caractères ; et après les avoir expliqués à leur manière et en avoir cru trouver les originaux, donnent au public de longues listes, ou, comme ils les appellent, des clefs : fausses clefs, et qui leur sont aussi inutiles qu’elles sont injurieuses aux personnes dont les noms s’y voient déchiffrés, et à l’écrivain qui en est la cause, quoique innocente.\par
J'avais pris la précaution de protester dans une préface contre tous ces interprétations, que quelque connaissance que j’ai des hommes m’avait fait prévoir, jusqu’à hésiter quelque temps si je devais rendre mon livre public, et à balancer entre le désir d’être utile à ma patrie par mes écrits, et la crainte de fournir à quelques-uns de quoi exercer leur malignité. Mais puisque j’ai eu la faiblesse de publier ces Caractères, quelle digue élèverai-je contre ce déluge d’explications qui inonde la ville, et qui bientôt va gagner la cour ? Dirai-je sérieusement, et protesterai-je avec d’horribles serments, que je ne suis ni auteur ni complice de ces clefs qui courent ; que je n’en ai donné aucune ; que mes plus familiers amis savent que je les leur ai toutes refusées ; que les personnes les plus accréditées de la cour ont désespéré d’avoir mon secret ? N'est-ce pas la même chose que si je me tourmentais beaucoup à soutenir que je ne suis pas un malhonnête homme, un homme sans pudeur, sans moeurs, sans conscience, tel enfin que les gazetiers dont je viens de parler ont voulu me représenter dans leur libelle diffamatoire ?\par
Mais d’ailleurs comment aurais-je donné ces sortes de clefs, si je n’ai pu moi-même les forger telles qu’elles sont et que je les ai vues ? Étant presque toutes différentes entre elles, quel moyen de les faire servir à une même entrée, je veux dire à l’intelligence de mes Remarques ? Nommant des personnes de la cour et de la ville à qui je n’ai jamais parlé, que je ne connais point, peuvent-elles partir de moi et être distribuées de ma main ? Aurais-je donné celles qui se fabriquent à Romorentin, à Mortaigne et à Belesme, dont les différentes applications sont à la baillive, à la femme de l’assesseur, au président de l’Élection, au prévôt de la maréchaussée et au prévôt de la collégiale ? Les noms y sont fort bien marqués ; mais ils ne m’aident pas davantage à connaître les personnes. Qu'on me permette ici une vanité sur mon ouvrage : je suis presque disposé à croire qu’il faut que mes peintures expriment bien l’homme en général, puisqu’elles ressemblent à tant de particuliers, et que chacun y croit voir ceux de sa ville ou de sa province. J'ai peint à la vérité d’après nature, mais je n’ai pas toujours songé à peindre celui-ci ou celle-là dans mon livre des Moeurs. Je ne me suis point loué au public pour faire des portraits qui ne fussent que vrais et ressemblants, de peur que quelquefois ils ne fussent pas croyables, et ne parussent feints ou imaginés. Me rendant plus difficile, je suis allé plus loin : j’ai pris un trait d’un côté et un trait d’un autre ; et de ces divers traits qui pouvaient convenir à une même personne, j’en ai fait des peintures vraisemblables, cherchant moins à réjouir les lecteurs par le caractère, ou comme le disent les mécontents, par la satire de quelqu’un, qu’à leur proposer des défauts à éviter et des modèles à suivre.\par
Il me semble donc que je dois être moins blâmé que plaint de ceux qui par hasard verraient leurs noms écrits dans ces insolentes listes, que je désavoue et que je condamne autant qu’elles le méritent. J'ose même attendre d’eux cette justice, que sans s’arrêter à un auteur moral qui n’a eu nulle intention de les offenser par son ouvrage, ils passeront jusqu’aux interprètes, dont la noirceur est inexcusable. Je dis en effet ce que je dis, et nullement ce qu’on assure que j’ai voulu dire ; et je réponds encore moins de ce qu’on me fait dire, et que je ne dis point. Je nomme nettement les personnes que je veux nommer, toujours dans la vue de louer vertu ou leur mérite ; j’écris leurs noms en lettres capitales, afin qu’on les voie de loin, et que le lecteur ne coure pas risque de les manquer. Si j’avais voulu mettre des noms véritables aux peintures moins obligeantes, je me serais épargné le travail d’emprunter les noms de l’ancienne histoire, d’employer des lettres initiales, qui n’ont qu’une signification vaine et incertaine, de trouver enfin mille tours et mille faux-fuyants pour dépayser ceux qui me lisent, et les dégoûter des applications. Voilà la conduite que j’ai tenue dans la composition des Caractères.\par
Sur ce qui concerne la harangue, qui a paru longue et ennuyeuse au chef des mécontents, je ne sais en effet pourquoi j’ai tenté de faire de ce remerciement à l’Académie française un discours oratoire qui eût quelque force et quelque étendue. De zélés académiciens m’avaient déjà frayé ce chemin ; mais ils se sont trouvés en petit nombre ; et leur zèle pour l’honneur et pour la réputation de l’Académie n’a eu que peu d’imitateurs. Je pouvais suivre l’exemple de ceux qui, postulant une place dans cette compagnie sans avoir jamais rien écrit, quoiqu’ils sachent écrire, annoncent dédaigneusement, la veille de leur réception, qu’ils n’ont que deux mots à dire et qu’un moment à parler, quoique capables de parler longtemps et de parler bien.\par
J'ai pensé au contraire qu’ainsi que nul artisan n’est agrégé à aucune société, ni n’a ses lettres de maîtrise sans faire son chef-d’oeuvre, de même et avec encore plus de bienséance, un homme associé à un corps qui ne s’est soutenu et ne peut jamais se soutenir que par l’éloquence, se trouvait engagé à faire, en y entrant, un effort en ce genre, qui le fît aux yeux de tous paraître digne du choix dont il venait de l’honorer. Il me semblait encore que puisque l’éloquence profane ne paraissait plus régner au barreau, d’où elle a été bannie par la nécessité de l’expédition, et qu’elle ne devait plus être admise dans la chaire, où elle n’a été que trop soufferte, le seul asile qui pouvait lui rester était l’Académie française ; et qu’il n’y avait rien de plus naturel, ni qui pût rendre cette Compagnie plus célèbre, que si, au sujet des réceptions de nouveaux académiciens, elle savait quelquefois attirer la cour et la ville à ses assemblées, par la curiosité d’y entendre des pièces d’éloquence d’une juste étendue, faites de main de maîtres, et dont la profession est d’exceller dans la science de la parole.\par
Si je n’ai pas atteint mon but, qui était de prononcer un discours éloquent, il me paraît du moins que je me suis disculpé de l’avoir fait trop long de quelques minutes ; car si d’ailleurs Paris, à qui on l’avait promis mauvais, satirique et insensé, s’est plaint qu’on lui avait manqué de parole ; si Marly, où la curiosité de l’entendre s’était répandue, n’a point retenti d’applaudissements que la cour ait donnés à la critique qu’on en avait faite ; s’il a su franchir Chantilly, écueil des mauvais ouvrages ; si l’Académie française, à qui j’avais appelé comme au juge souverain de ces sortes de pièces, étant assemblée extraordinairement, a adopté celle-ci, l’a fait imprimer par son libraire, l’a mise dans ses archives ; si elle n’était pas en effet composée d’un style affecté, dur et interrompu, ni chargée de louanges fades et outrées, telles qu’on les lit dans les prologues d’opéras, et dans tant d’épîtres dédicatoires, il ne faut plus s’étonner qu’elle ait ennuyé Théobalde. Je vois les temps, le public me permettra de le dire, où ce ne sera pas assez de l’approbation qu’il aura donnée à un ouvrage pour en faire la réputation, et que pour y mettre le dernier sceau, il sera nécessaire que de certaines gens le désapprouvent, qu’ils y aient bâillé.\par
Car voudraient-ils, présentement qu’ils ont reconnu que cette harangue a moins mal réussi dans le public qu’ils ne l’avaient espéré, qu’ils savent que deux libraires ont plaidé à qui l’imprimerait, voudraient-ils désavouer leur goût et le jugement qu’ils en ont porté dans les premiers jours qu’elle fut prononcée ? Me permettraient-ils de publier, ou seulement de soupçonner, une tout autre raison de l’âpre censure qu’ils en firent, que la persuasion où ils étaient qu’elle la méritait ? On sait que cet homme, d’un nom et d’un mérite si distingué, avec qui j’eus l’honneur d’être reçu à l’Académie française, prié, sollicité, persécuté de consentir à l’impression de sa harangue, par ceux mêmes qui voulaient supprimer la mienne et en éteindre la mémoire, leur résista toujours avec fermeté. Il leur dit qu’il ne pouvait ni ne devait approuver une distinction si odieuse qu’ils voulaient faire entre lui et moi ; que la préférence qu’ils donnaient à son discours avec cette affectation et cet empressement qu’ils lui marquaient, bien loin de l’obliger, comme ils pouvaient le croire, lui faisait au contraire une véritable peine ; que deux discours également innocents, prononcés dans le même jour, devaient être imprimés dans le même temps. Il s’expliqua ensuite obligeamment, en public et en particulier, sur le violent chagrin qu’il ressentait de ce que les deux auteurs de la gazette que j’ai cités avaient fait servir les louanges qu’il leur avait plu de lui donner à un dessein formé de médire de moi, de mon discours et de mes Caractères ; et il me fit, sur cette satire injurieuse, des explications et des excuses qu’il ne me devait point. Si donc on voulait inférer de cette conduite des Théobaldes, qu’ils ont cru faussement avoir besoin de comparaisons et d’une harangue folle et décriée pour relever celle de mon collègue, ils doivent répondre, pour se laver de ce soupçon qui les déshonore, qu’ils ne sont ni courtisans, ni dévoués à la faveur, ni intéressés, ni adulateurs ; qu’au contraire ils sont sincères, et qu’ils ont dit naïvement ce qu’ils pensaient du plan, du style et des expressions de mon remerciement à l’Académie française. Mais on ne manquera pas d’insister et de leur dire que le jugement de la cour et de la ville, des grands et du peuple, lui a été favorable. Qu'importe ? Ils répliqueront avec confiance que le public a son goût, et qu’ils ont le leur : réponse qui ferme la bouche et qui termine tout différend. Il est vrai qu’elle m’éloigne de plus en plus de vouloir leur plaire par aucun de mes écrits ; car si j’ai un peu de santé avec quelques années de vie, je n’aurai plus d’autre ambition que celle de rendre, par des soins assidus et par de bons conseils, mes ouvrages tels qu’ils puissent toujours partager les Théobaldes et le public.
\chapterclose


\chapteropen
\chapter[{Discours prononcé dans l’académie française le lundi quinzième juin 1693}]{Discours prononcé dans l’académie française le lundi quinzième juin 1693}\phantomsection
\label{acad-discours}\renewcommand{\leftmark}{Discours prononcé dans l’académie française le lundi quinzième juin 1693}


\chaptercont
\noindent Messieurs,\par
Il serait difficile d’avoir l’honneur de se trouver au milieu de vous, d’avoir devant ses yeux l’Académie française, d’avoir lu l’histoire de son établissement, sans penser d’abord à celui à qui elle en est redevable, et sans se persuader qu’il n’y a rien de plus naturel, et qui doive moins vous déplaire, que d’entamer ce tissu de louanges qu’exigent le devoir et la coutume, par quelques traits où ce grand cardinal soit reconnaissable, et qui en renouvellent la mémoire.\par
Ce n’est point un personnage qu’il soit facile de rendre ni d’exprimer par de belles paroles ou par de riches figures, par ces discours moins faits pour relever le mérite de celui que l’on veut peindre, que pour montrer tout le feu et toute la vivacité de l’orateur. Suivez le règne de Louis le Juste : c’est la vie du cardinal de Richelieu, c’est son éloge et celui du prince qui l’a mis en oeuvre. Que pourrais-je ajouter à des faits encore récents et si mémorables ? Ouvrez son Testament politique, digérez cet ouvrage : c’est la peinture de son esprit ; son âme tout entière s’y développe ; l’on y découvre le secret de sa conduite et de ses actions ; l’on y trouve la source et la vraisemblance de tant et de si grands événements qui ont paru sous son administration : l’on y voit sans peine qu’un homme qui pense si virilement et si juste a pu agir sûrement et avec succès, et que celui qui a achevé de si grandes choses, ou n’a jamais écrit, ou a dû écrire comme il a fait.\par
Génie fort et supérieur, il a su tout le fond et tout le mystère du gouvernement ; il a connu le beau et le sublime du ministère ; il a respecté l’étranger, ménagé les couronnes, connu le poids de leur alliance ; il a opposé des alliés à des ennemis ; il a veillé aux intérêts du dehors, à ceux du dedans. Il n’a oublié que les siens : une vie laborieuse et languissante, souvent exposée, a été le prix d’une si haute vertu ; dépositaire des trésors de son maître, comblé de ses bienfaits, ordonnateur, dispensateur de ses finances, on ne saurait dire qu’il est mort riche.\par
Le croirait-on, Messieurs ? cette âme sérieuse et austère, formidable aux ennemis de l’État, inexorable aux factieux, plongée dans la négociation, occupée tantôt à affaiblir le parti de l’hérésie, tantôt à déconcerter une ligue, et tantôt à méditer une conquête, a trouvé le loisir d’être savante, a goûté les belles-lettres et ceux qui en faisaient profession. Comparez-vous, si vous l’osez, au grand Richelieu, hommes dévoués à la fortune, qui, par le succès de vos affaires particulières, vous jugez dignes que l’on vous confie les affaires publiques ; qui vous donnez pour des génies heureux et pour de bonnes têtes ; qui dites que vous ne savez rien, que vous n’avez jamais lu, que vous ne lirez point, ou pour marquer l’inutilité des sciences, ou pour paraître ne devoir rien aux autres, mais puiser tout de votre fonds. Apprenez que le cardinal de Richelieu a su, qu’il a lu : je ne dis pas qu’il n’a point eu d’éloignement pour les gens de lettres, mais qu’il les a aimés, caressés, favorisés, qu’il leur a ménagé des privilèges, qu’il leur destinait des pensions, qu’il les a réunis en une Compagnie célèbre, qu’il en a fait l’Académie française. Oui, hommes riches et ambitieux, contempteurs de la vertu, et de toute association qui ne roule pas sur les établissements et sur l’intérêt, celle-ci est une des pensées de ce grand ministre, né homme d’État, dévoué à l’État, esprit solide, éminent, capable dans ce qu’il faisait des motifs les plus relevés et qui tendaient au bien public comme à la gloire de la monarchie ; incapable de concevoir jamais rien qui ne fût digne de lui, du prince qu’il servait, de la France, à qui il avait consacré ses méditations et ses veilles.\par
Il savait quelle est la force et l’utilité de l’éloquence, la puissance de la parole qui aide la raison et la fait valoir, qui insinue aux hommes la justice et la probité, qui porte dans le coeur du soldat l’intrépidité et l’audace, qui calme les émotions populaires, qui excite à leurs devoirs les compagnies entières ou la multitude. Il n’ignorait pas quels sont les fruits de l’histoire et de la poésie, quelle est la nécessité de la grammaire, la base et le fondement des autres sciences ; et que pour conduire ces choses à un degré de perfection qui les rendît avantageuses à la République, il fallait dresser le plan d’une compagnie où la vertu seule fût admise, le mérite placé, l’esprit et le savoir rassemblés par des suffrages. N'allons pas plus loin : voilà, Messieurs, vos principes et votre règle, dont je ne suis qu’une exception.\par
Rappelez en votre mémoire, la comparaison ne vous sera pas injurieuse, rappelez ce grand et premier concile où les Pères qui le composaient étaient remarquables chacun par quelques membres mutilés, ou par les cicatrices qui leur étaient restées des fureurs de la persécution ; ils semblaient tenir de leurs plaies le droit de s’asseoir dans cette assemblée générale de toute l’Église : il n’y avait aucun de vos illustres prédécesseurs qu’on ne s’empressât de voir, qu’on ne montrât dans les places, qu’on ne désignât par quelque ouvrage fameux qui lui avait fait un grand nom, et qui lui donnait rang dans cette Académie naissante qu’ils avaient comme fondée. Tels étaient ces grands artisans de la parole, ces premiers maîtres de l’éloquence française ; tels vous êtes, Messieurs, qui ne cédez ni en savoir ni en mérite à nul de ceux qui vous ont précédés.\par
L'un, aussi correct dans sa langue que s’il l’avait apprise par règles et par principes, aussi élégant dans les langues étrangères que si elles lui étaient naturelles, en quelque idiome qu’il compose, semble toujours parler celui de son pays : il a entrepris, il a fini une pénible traduction, que le plus bel esprit pourrait avouer, et que le plus pieux personnage devrait désirer d’avoir faite.\par
L'autre fait revivre Virgile parmi nous, transmet dans notre langue les grâces et les richesses de la latine, fait des romans qui ont une fin, en bannit le prolixe et l’incroyable, pour y substituer le vraisemblable et le naturel.\par
Un autre, plus égal que Marot et plus poète que Voiture, a le jeu, le tour, et la naïveté de tous les deux ; il instruit en badinant, persuade aux hommes la vertu par l’organe des bêtes, élève les petits sujets jusqu’au sublime : homme unique dans son genre d’écrire ; toujours original soit qu’il invente, soit qu’il traduise ; qui a été au delà de ses modèles, modèle lui-même difficile à imiter.\par
Celui-ci passe Juvénal, atteint Horace, semble créer les pensées d’autrui et se rendre propre tout ce qu’il manie ; il a dans ce qu’il emprunte des autres toutes les grâces de la nouveauté et tout le mérite de l’invention. Ses vers, forts et harmonieux, faits de génie, quoique travaillés avec art, pleins de traits et de poésie, seront lus encore quand la langue aura vieilli, en seront les derniers débris : on y remarque une critique sûre, judicieuse et innocente, s’il est permis du moins de dire de ce qui est mauvais qu’il est mauvais.\par
Cet autre vient après un homme loué, applaudi, admiré, dont les vers volent en tous lieux et passent en proverbe, qui prime, qui règne sur la scène, qui s’est emparé de tout le théâtre. Il ne l’en dépossède pas, il est vrai ; mais il s’y établit avec lui : le monde s’accoutume à en voir faire la comparaison. Quelques-uns ne souffrent pas que Corneille, le grand Corneille, lui soit préféré ; quelques autres, qu’il lui soit égalé : ils en appellent à l’autre siècle ; ils attendent la fin de quelques vieillards qui, touchés indifféremment de tout ce qui rappelle leurs premières années, n’aiment peut-être dans OEdipe que le souvenir de leur jeunesse.\par
Que dirai-je de ce personnage qui a fait parler si longtemps une envieuse critique et qui l’a fait taire ; qu’on admire malgré soi, qui accable par le grand nombre et par l’éminence de ses talents ? Orateur, historien, théologien, philosophe, d’une rare érudition, d’une plus rare éloquence, soit dans ses entretiens, soit dans ses écrits, soit dans la chaire ; un défenseur de la religion, une lumière de l’Église, parlons d’avance le langage de la postérité, un Père de l’Église. Que n’est-il point ? Nommez, Messieurs, une vertu qui ne soit pas la sienne.\par
Toucherai-je aussi votre dernier choix, si digne de vous ? Quelles choses vous furent dites dans la place où je me trouve ! Je m’en souviens ; et après ce que vous avez entendu, comment osé-je parler ? comment daignez-vous m’entendre ? Avouons-le, on sent la force et l’ascendant de ce rare esprit, soit qu’il prêche de génie et sans préparation, soit qu’il prononce un discours étudié et oratoire, soit qu’il explique ses pensées dans la conversation : toujours maître de l’oreille et du coeur de ceux qui l’écoutent, il ne leur permet pas d’envier ni tant d’élévation, ni tant de facilité, de délicatesse, de politesse. On est assez heureux de l’entendre, de sentir ce qu’il dit, et comme il le dit ; on doit être content de soi, si l’on emporte ses réflexions et si l’on en profite. Quelle grande acquisition avez-vous faite en cet homme illustre ! À qui m’associez-vous !\par
Je voudrais, Messieurs, moins pressé par le temps et par les bienséances qui mettent des bornes à ce discours, pouvoir louer chacun de ceux qui composent cette Académie par des endroits encore plus marqués et par de plus vives expressions. Toutes les sortes de talents que l’on voit répandus parmi les hommes se trouvent partagés entre vous. Veut-on de diserts orateurs, qui aient semé dans la chaire toutes les fleurs de l’éloquence, qui, avec une saine morale, aient employé tous les tours et toutes les finesses de la langue, qui plaisent par un beau choix de paroles, qui fassent aimer les solennités, les temples, qui y fassent courir ? qu’on ne les cherche pas ailleurs, ils sont parmi vous. Admire-t-on une vaste et profonde littérature qui aille fouiller dans les archives de l’antiquité pour en retirer des choses ensevelies dans l’oubli, échappées aux esprits les plus curieux, ignorées des autres hommes ; une mémoire, une méthode, une précision à ne pouvoir dans ces recherches s’égarer d’une seule année, quelquefois d’un seul jour sur tant de siècles ? cette doctrine admirable, vous la possédez ; elle est du moins en quelques-uns de ceux qui forment cette savante assemblée. Si l’on est curieux du don des langues, joint au double talent de savoir avec exactitude les choses anciennes, et de narrer celles qui sont nouvelles avec autant de simplicité que de vérité, des qualités si rares ne vous manquent pas et sont réunies en un même sujet. Si l’on cherche des hommes habiles, pleins d’esprit et d’expérience, qui, par le privilège de leurs emplois, fassent parler le Prince avec dignité et avec justesse ; d’autres qui placent heureusement et avec succès, dans les négociations les plus délicates, les talents qu’ils ont de bien parler et de bien écrire ; d’autres encore qui prêtent leurs soins et leur vigilance aux affaires publiques, après les avoir employés aux judiciaires, toujours avec une égale réputation : tous se trouvent au milieu de vous, et je souffre à ne les pas nommer.\par
Si vous aimez le savoir joint à l’éloquence, vous n’attendrez pas longtemps : réservez seulement toute votre attention pour celui qui parlera après moi. Que vous manque-t-il enfin ? vous avez des écrivains habiles en l’une et en l’autre oraison ; des poètes en tout genre de poésies, soit morales, soit chrétiennes, soit héroïques, soit galantes et enjouées ; des imitateurs des anciens ; des critiques austères ; des esprits fins, délicats, subtils, ingénieux, propres à briller dans les conversations et dans les cercles. Encore une fois, à quels hommes, à quels grands sujets m’associez-vous !\par
Mais avec qui daignez-vous aujourd’hui me recevoir ? Après qui vous fais-je ce public remerciement ? Il ne doit pas néanmoins, cet homme si louable et si modeste, appréhender que je le loue : si proche de moi, il aurait autant de facilité que de disposition à m’interrompre. Je vous demanderai plus volontiers : À qui me faites-vous succéder ? À un homme QUI AVAIT DE LA VERTU.\par
Quelquefois, Messieurs, il arrive que ceux qui vous doivent les louanges des illustres morts dont ils remplissent la place, hésitent, partagés entre plusieurs choses qui méritent également qu’on les relève. Vous aviez choisi en M. l’abbé de la Chambre un homme si pieux, si tendre, si charitable, si louable par le coeur, qui avait des moeurs si sages et si chrétiennes, qui était si touché de religion, si attaché à ses devoirs, qu’une de ses moindres qualités était de bien écrire. De solides vertus, qu’on voudrait célébrer, font passer légèrement sur son érudition ou sur son éloquence ; on estime encore plus sa vie et sa conduite que ses ouvrages. Je préférerais en effet de prononcer le discours funèbre de celui à qui je succède, plutôt que de me borner à un simple éloge de son esprit. Le mérite en lui n’était pas une chose acquise, mais un patrimoine, un bien héréditaire, si du moins il en faut juger par le choix de celui qui avait livré son coeur, sa confiance, toute sa personne, à cette famille, qui l’avait rendue comme votre alliée, puisqu’on peut dire qu’il l’avait adoptée, et qu’il l’avait mise avec l’Académie française sous sa protection.\par
Je parle du chancelier Seguier. On s’en souvient comme de l’un des plus grands magistrats que la France ait nourris depuis ses commencements. Il a laissé à douter en quoi il excellait davantage, ou dans les belles-lettres, ou dans les affaires ; il est vrai du moins, et on en convient, qu’il surpassait en l’un et en l’autre tous ceux de son temps. Homme grave et familier, profond dans les délibérations, quoique doux et facile dans le commence, il a eu naturellement ce que tant d’autres veulent avoir et ne se donnent pas, ce qu’on n’a point par l’étude et par l’affectation, par les mots graves ou sentencieux, ce qui est plus rare que la science, et peut-être que la probité, je veux dire de la dignité. Il ne la devait point à l’éminence de son poste ; au contraire, il l’a anobli : il a été grand et accrédité sans ministère, et on ne voit pas que ceux qui ont su tout réunir en leurs personnes l’aient effacé.\par
Vous le perdîtes il y a quelques années, ce grand protecteur. Vous jetâtes la vue autour de vous, vous promenâtes vos yeux sur tous ceux qui s’offraient et qui se trouvaient honorés de vous recevoir ; mais le sentiment de votre perte fut tel, que dans les efforts que vous fîtes pour la réparer, vous osâtes penser à celui qui seul pouvait vous la faire oublier et la tourner à votre gloire. Avec quelle bonté, avec quelle humanité ce magnanime prince vous a-t-il reçus ! N'en soyons pas surpris, c’est son caractère : le même, Messieurs, que l’on voit éclater dans toutes les actions de sa belle vie, mais que les surprenantes révolutions arrivées dans un royaume voisin et allié de la France ont mis dans le plus beau jour qu’il pouvait jamais recevoir.\par
Quelle facilité est la nôtre pour perdre tout d’un coup le sentiment et la mémoire des choses dont nous nous sommes vus le plus fortement imprimés ! Souvenons-nous de ces jours tristes que nous avons passés dans l’agitation et dans le trouble, curieux, incertains quelle fortune auraient courue un grand roi, une grande reine, le prince leur fils, famille auguste, mais malheureuse, que la piété et la religion avaient poussée jusqu’aux dernières épreuves de l’adversité. Hélas ! avaient-ils péri sur la mer ou par les mains de leurs ennemis ? Nous ne le savions pas : on s’interrogeait, on se promettait réciproquement les premières nouvelles qui viendraient sur un événement si lamentable. Ce n’était plus une affaire publique, mais domestique ; on n’en dormait plus, on s’éveillait les uns les autres pour s’annoncer ce qu’on en avait appris. Et quand ces personnes royales, à qui l’on prenait tant d’intérêt, eussent pu échapper à la mer ou à leur patrie, était-ce assez ? ne fallait-il pas une terre étrangère où ils pussent aborder, un roi également bon et puissant qui pût et qui voulût les recevoir ? Je l’ai vue, cette réception, spectacle tendre s’il en fut jamais ! On y versait des larmes d’admiration et de joie. Ce prince n’a pas plus de grâce, lorsqu’à la tête de ses camps et de ses armées, il foudroie une ville qui lui résiste, ou qu’il dissipe les troupes ennemies du seul bruit de son approche.\par
S'il soutient cette longue guerre, n’en doutons pas, c’est pour nous donner une paix heureuse, c’est pour l’avoir à des conditions qui soient justes et qui fassent honneur à la nation ; qui ôtent pour toujours à l’ennemi l’espérance de nous troubler par de nouvelles hostilités. Que d’autres publient, exaltent ce que ce grand roi a exécuté, ou par lui-même, ou par ses capitaines, durant le cours de ces mouvements dont toute l’Europe est ébranlée : ils ont un sujet vaste et qui les exercera longtemps. Que d’autres augurent, s’ils le peuvent, ce qu’il veut achever dans cette campagne. Je ne parle que de son coeur, que de la pureté et de la droiture de ses intentions : elles sont connues, elles lui échappent. On le félicite sur des titres d’honneur dont il vient de gratifier quelques grands de son État : que dit-il ? qu’il ne peut être content quand tous ne le sont pas, et qu’il lui est impossible que tous le soient comme il le voudrait. Il sait, Messieurs, que la fortune d’un roi est de prendre des villes, de gagner des batailles, de reculer ses frontières, d’être craint de ses ennemis ; mais que la gloire du souverain consiste à être aimé de ses peuples, en avoir le coeur, et par le coeur tout ce qu’ils possèdent. Provinces éloignées, provinces voisines, ce prince humain et bienfaisant, que les peintres et les statuaires nous défigurent, vous tend les bras, vous regarde avec des yeux tendres et pleins de douceur ; c’est là son attitude : il veut voir vos habitants, vos bergers danser au son d’une flûte champêtre sous les saules et les peupliers, y mêler leurs voix rustiques, et chanter les louanges de celui qui, avec la paix et les fruits de la paix, leur aura rendu la joie et la sérénité.\par
C'est pour arriver à ce comble de ses souhaits, la félicité commune, qu’il se livre aux travaux et aux fatigues d’une guerre pénible, qu’il essuie l’inclémence du ciel et des saisons, qu’il expose sa personne, qu’il risque une vie heureuse : voilà son secret et les vues qui le font agir ; on les pénètre, on les discerne par les seules qualités de ceux qui sont en place, et qui l’aident de leurs conseils. Je ménage leur modestie : qu’ils me permettent seulement de remarquer qu’on ne devine point les projets de ce sage prince ; qu’on devine, au contraire, qu’on nomme les personnes qu’il va placer, et qu’il ne fait que confirmer la voix du peuple dans le choix qu’il fait de ses ministres. Il ne se décharge pas entièrement sur eux du poids de ses affaires ; lui-même, si je l’ose dire, il est son principal ministre. Toujours appliqué à nos besoins, il n’y a pour lui ni temps de relâche ni heures privilégiées : déjà la nuit s’avance, les gardes sont relevées aux avenues de son palais, les astres brillent au ciel et font leur course ; toute la nature repose, privée du jour, ensevelie dans les ombres ; nous reposons aussi, tandis que ce roi, retiré dans son balustre, veille seul sur nous et sur tout l’État. Tel est, Messieurs, le protecteur que vous vous êtes procuré, celui de ses peuples.\par
Vous m’avez admis dans une Compagnie illustrée par une si haute protection. Je ne le dissimule pas, j’ai assez estimé cette distinction pour désirer de l’avoir dans toute sa fleur et dans toute son intégrité, je veux dire de la devoir à votre seul choix ; et j’ai mis votre choix à tel prix, que je n’ai pas osé en blesser, pas même en effleurer la liberté, par une importune sollicitation. J'avais d’ailleurs une juste défiance de moi-même, je sentais de la répugnance à demander d’être préféré à d’autres qui pouvaient être choisis. J'avais cru entrevoir, Messieurs, une chose que je ne devais avoir aucune peine à croire, que vos inclinations se tournaient ailleurs, sur un sujet digne, sur un homme rempli de vertus, d’esprit et de connaissances, qui était tel avant le poste de confiance qu’il occupe, et qui serait tel encore s’il ne l’occupait plus. Je me sens touché, non de sa déférence, je sais celle que je lui dois, mais de l’amitié qu’il m’a témoignée, jusques à s’oublier en ma faveur. Un père mène son fils à un spectacle : la foule y est grande, la porte est assiégée ; il est haut et robuste, il fend la presse ; et comme il est près d’entrer, il pousse son fils devant lui, qui sans cette précaution, ou n’entrerait point, ou entrerait tard. Cette démarche d’avoir supplié quelques-uns de vous, comme il a fait, de détourner vers moi leurs suffrages, qui pouvaient si justement aller à lui, elle est rare, puisque dans ces circonstances elle est unique, et elle ne diminue rien de ma reconnaissance envers vous, puisque vos voix seules, toujours libres et arbitraires, donnent une place dans l’Académie française.\par
Vous me l’avez accordée, Messieurs, et de si bonne grâce, avec un consentement si unanime, que je la dois et la veux tenir de votre seule magnificence. Il n’y a ni poste, ni crédit, ni richesses, ni titres, ni autorité, ni faveur qui aient pu vous plier à faire ce choix : je n’ai rien de toutes ces choses, tout me manque. Un ouvrage qui a eu quelque succès par sa singularité, et dont les fausses, je dis les fausses et malignes applications pouvaient me nuire auprès des personnes moins équitables et moins éclairées que vous, a été toute la médiation que j’ai employée, et que vous avez reçue. Quel moyen de me repentir jamais d’avoir écrit ?
\chapterclose

 


% at least one empty page at end (for booklet couv)
\ifbooklet
  \pagestyle{empty}
  \clearpage
  % 2 empty pages maybe needed for 4e cover
  \ifnum\modulo{\value{page}}{4}=0 \hbox{}\newpage\hbox{}\newpage\fi
  \ifnum\modulo{\value{page}}{4}=1 \hbox{}\newpage\hbox{}\newpage\fi


  \hbox{}\newpage
  \ifodd\value{page}\hbox{}\newpage\fi
  {\centering\color{rubric}\bfseries\noindent\large
    Hurlus ? Qu’est-ce.\par
    \bigskip
  }
  \noindent Des bouquinistes électroniques, pour du texte libre à participation libre,
  téléchargeable gratuitement sur \href{https://hurlus.fr}{\dotuline{hurlus.fr}}.\par
  \bigskip
  \noindent Cette brochure a été produite par des éditeurs bénévoles.
  Elle n’est pas faîte pour être possédée, mais pour être lue, et puis donnée.
  Que circule le texte !
  En page de garde, on peut ajouter une date, un lieu, un nom ; pour suivre le voyage des idées.
  \par

  Ce texte a été choisi parce qu’une personne l’a aimé,
  ou haï, elle a en tous cas pensé qu’il partipait à la formation de notre présent ;
  sans le souci de plaire, vendre, ou militer pour une cause.
  \par

  L’édition électronique est soigneuse, tant sur la technique
  que sur l’établissement du texte ; mais sans aucune prétention scolaire, au contraire.
  Le but est de s’adresser à tous, sans distinction de science ou de diplôme.
  Au plus direct ! (possible)
  \par

  Cet exemplaire en papier a été tiré sur une imprimante personnelle
   ou une photocopieuse. Tout le monde peut le faire.
  Il suffit de
  télécharger un fichier sur \href{https://hurlus.fr}{\dotuline{hurlus.fr}},
  d’imprimer, et agrafer ; puis de lire et donner.\par

  \bigskip

  \noindent PS : Les hurlus furent aussi des rebelles protestants qui cassaient les statues dans les églises catholiques. En 1566 démarra la révolte des gueux dans le pays de Lille. L’insurrection enflamma la région jusqu’à Anvers où les gueux de mer bloquèrent les bateaux espagnols.
  Ce fut une rare guerre de libération dont naquit un pays toujours libre : les Pays-Bas.
  En plat pays francophone, par contre, restèrent des bandes de huguenots, les hurlus, progressivement réprimés par la très catholique Espagne.
  Cette mémoire d’une défaite est éteinte, rallumons-la. Sortons les livres du culte universitaire, cherchons les idoles de l’époque, pour les briser.
\fi

\ifdev % autotext in dev mode
\fontname\font — \textsc{Les règles du jeu}\par
(\hyperref[utopie]{\underline{Lien}})\par
\noindent \initialiv{A}{lors là}\blindtext\par
\noindent \initialiv{À}{ la bonheur des dames}\blindtext\par
\noindent \initialiv{É}{tonnez-le}\blindtext\par
\noindent \initialiv{Q}{ualitativement}\blindtext\par
\noindent \initialiv{V}{aloriser}\blindtext\par
\Blindtext
\phantomsection
\label{utopie}
\Blinddocument
\fi
\end{document}
