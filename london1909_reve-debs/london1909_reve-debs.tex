%%%%%%%%%%%%%%%%%%%%%%%%%%%%%%%%%
% LaTeX model https://hurlus.fr %
%%%%%%%%%%%%%%%%%%%%%%%%%%%%%%%%%

% Needed before document class
\RequirePackage{pdftexcmds} % needed for tests expressions
\RequirePackage{fix-cm} % correct units

% Define mode
\def\mode{a4}

\newif\ifaiv % a4
\newif\ifav % a5
\newif\ifbooklet % booklet
\newif\ifcover % cover for booklet

\ifnum \strcmp{\mode}{cover}=0
  \covertrue
\else\ifnum \strcmp{\mode}{booklet}=0
  \booklettrue
\else\ifnum \strcmp{\mode}{a5}=0
  \avtrue
\else
  \aivtrue
\fi\fi\fi

\ifbooklet % do not enclose with {}
  \documentclass[french,twoside]{book} % ,notitlepage
  \usepackage[%
    papersize={105mm, 297mm},
    inner=12mm,
    outer=12mm,
    top=20mm,
    bottom=15mm,
    marginparsep=0pt,
  ]{geometry}
  \usepackage[fontsize=9.5pt]{scrextend} % for Roboto
\else\ifav
  \documentclass[french,twoside]{book} % ,notitlepage
  \usepackage[%
    a5paper,
    inner=25mm,
    outer=15mm,
    top=15mm,
    bottom=15mm,
    marginparsep=0pt,
  ]{geometry}
  \usepackage[fontsize=12pt]{scrextend}
\else% A4 2 cols
  \documentclass[twocolumn]{report}
  \usepackage[%
    a4paper,
    inner=15mm,
    outer=10mm,
    top=25mm,
    bottom=18mm,
    marginparsep=0pt,
  ]{geometry}
  \setlength{\columnsep}{20mm}
  \usepackage[fontsize=9.5pt]{scrextend}
\fi\fi

%%%%%%%%%%%%%%
% Alignments %
%%%%%%%%%%%%%%
% before teinte macros

\setlength{\arrayrulewidth}{0.2pt}
\setlength{\columnseprule}{\arrayrulewidth} % twocol
\setlength{\parskip}{0pt} % classical para with no margin
\setlength{\parindent}{1.5em}

%%%%%%%%%%
% Colors %
%%%%%%%%%%
% before Teinte macros

\usepackage[dvipsnames]{xcolor}
\definecolor{rubric}{HTML}{800000} % the tonic 0c71c3
\def\columnseprulecolor{\color{rubric}}
\colorlet{borderline}{rubric!30!} % definecolor need exact code
\definecolor{shadecolor}{gray}{0.95}
\definecolor{bghi}{gray}{0.5}

%%%%%%%%%%%%%%%%%
% Teinte macros %
%%%%%%%%%%%%%%%%%
%%%%%%%%%%%%%%%%%%%%%%%%%%%%%%%%%%%%%%%%%%%%%%%%%%%
% <TEI> generic (LaTeX names generated by Teinte) %
%%%%%%%%%%%%%%%%%%%%%%%%%%%%%%%%%%%%%%%%%%%%%%%%%%%
% This template is inserted in a specific design
% It is XeLaTeX and otf fonts

\makeatletter % <@@@


\usepackage{blindtext} % generate text for testing
\usepackage[strict]{changepage} % for modulo 4
\usepackage{contour} % rounding words
\usepackage[nodayofweek]{datetime}
% \usepackage{DejaVuSans} % seems buggy for sffont font for symbols
\usepackage{enumitem} % <list>
\usepackage{etoolbox} % patch commands
\usepackage{fancyvrb}
\usepackage{fancyhdr}
\usepackage{float}
\usepackage{fontspec} % XeLaTeX mandatory for fonts
\usepackage{footnote} % used to capture notes in minipage (ex: quote)
\usepackage{framed} % bordering correct with footnote hack
\usepackage{graphicx}
\usepackage{lettrine} % drop caps
\usepackage{lipsum} % generate text for testing
\usepackage[framemethod=tikz,]{mdframed} % maybe used for frame with footnotes inside
\usepackage{pdftexcmds} % needed for tests expressions
\usepackage{polyglossia} % non-break space french punct, bug Warning: "Failed to patch part"
\usepackage[%
  indentfirst=false,
  vskip=1em,
  noorphanfirst=true,
  noorphanafter=true,
  leftmargin=\parindent,
  rightmargin=0pt,
]{quoting}
\usepackage{ragged2e}
\usepackage{setspace} % \setstretch for <quote>
\usepackage{tabularx} % <table>
\usepackage[explicit]{titlesec} % wear titles, !NO implicit
\usepackage{tikz} % ornaments
\usepackage{tocloft} % styling tocs
\usepackage[fit]{truncate} % used im runing titles
\usepackage{unicode-math}
\usepackage[normalem]{ulem} % breakable \uline, normalem is absolutely necessary to keep \emph
\usepackage{verse} % <l>
\usepackage{xcolor} % named colors
\usepackage{xparse} % @ifundefined
\XeTeXdefaultencoding "iso-8859-1" % bad encoding of xstring
\usepackage{xstring} % string tests
\XeTeXdefaultencoding "utf-8"
\PassOptionsToPackage{hyphens}{url} % before hyperref, which load url package

% TOTEST
% \usepackage{hypcap} % links in caption ?
% \usepackage{marginnote}
% TESTED
% \usepackage{background} % doesn’t work with xetek
% \usepackage{bookmark} % prefers the hyperref hack \phantomsection
% \usepackage[color, leftbars]{changebar} % 2 cols doc, impossible to keep bar left
% \usepackage[utf8x]{inputenc} % inputenc package ignored with utf8 based engines
% \usepackage[sfdefault,medium]{inter} % no small caps
% \usepackage{firamath} % choose firasans instead, firamath unavailable in Ubuntu 21-04
% \usepackage{flushend} % bad for last notes, supposed flush end of columns
% \usepackage[stable]{footmisc} % BAD for complex notes https://texfaq.org/FAQ-ftnsect
% \usepackage{helvet} % not for XeLaTeX
% \usepackage{multicol} % not compatible with too much packages (longtable, framed, memoir…)
% \usepackage[default,oldstyle,scale=0.95]{opensans} % no small caps
% \usepackage{sectsty} % \chapterfont OBSOLETE
% \usepackage{soul} % \ul for underline, OBSOLETE with XeTeX
% \usepackage[breakable]{tcolorbox} % text styling gone, footnote hack not kept with breakable


% Metadata inserted by a program, from the TEI source, for title page and runing heads
\title{\textbf{ Le rêve de Debs }}
\date{1909}
\author{London, Jack}
\def\elbibl{London, Jack. 1909. \emph{Le rêve de Debs}}
\def\elsource{\href{https://fr.wikisource.org/wiki/Le_Rêve_de_Debs}{\dotuline{Wikisource}}\footnote{\href{https://fr.wikisource.org/wiki/Le_Rêve_de_Debs}{\url{https://fr.wikisource.org/wiki/Le_Rêve_de_Debs}}}}

% Default metas
\newcommand{\colorprovide}[2]{\@ifundefinedcolor{#1}{\colorlet{#1}{#2}}{}}
\colorprovide{rubric}{red}
\colorprovide{silver}{lightgray}
\@ifundefined{syms}{\newfontfamily\syms{DejaVu Sans}}{}
\newif\ifdev
\@ifundefined{elbibl}{% No meta defined, maybe dev mode
  \newcommand{\elbibl}{Titre court ?}
  \newcommand{\elbook}{Titre du livre source ?}
  \newcommand{\elabstract}{Résumé\par}
  \newcommand{\elurl}{http://oeuvres.github.io/elbook/2}
  \author{Éric Lœchien}
  \title{Un titre de test assez long pour vérifier le comportement d’une maquette}
  \date{1566}
  \devtrue
}{}
\let\eltitle\@title
\let\elauthor\@author
\let\eldate\@date


\defaultfontfeatures{
  % Mapping=tex-text, % no effect seen
  Scale=MatchLowercase,
  Ligatures={TeX,Common},
}


% generic typo commands
\newcommand{\astermono}{\medskip\centerline{\color{rubric}\large\selectfont{\syms ✻}}\medskip\par}%
\newcommand{\astertri}{\medskip\par\centerline{\color{rubric}\large\selectfont{\syms ✻\,✻\,✻}}\medskip\par}%
\newcommand{\asterism}{\bigskip\par\noindent\parbox{\linewidth}{\centering\color{rubric}\large{\syms ✻}\\{\syms ✻}\hskip 0.75em{\syms ✻}}\bigskip\par}%

% lists
\newlength{\listmod}
\setlength{\listmod}{\parindent}
\setlist{
  itemindent=!,
  listparindent=\listmod,
  labelsep=0.2\listmod,
  parsep=0pt,
  % topsep=0.2em, % default topsep is best
}
\setlist[itemize]{
  label=—,
  leftmargin=0pt,
  labelindent=1.2em,
  labelwidth=0pt,
}
\setlist[enumerate]{
  label={\bf\color{rubric}\arabic*.},
  labelindent=0.8\listmod,
  leftmargin=\listmod,
  labelwidth=0pt,
}
\newlist{listalpha}{enumerate}{1}
\setlist[listalpha]{
  label={\bf\color{rubric}\alph*.},
  leftmargin=0pt,
  labelindent=0.8\listmod,
  labelwidth=0pt,
}
\newcommand{\listhead}[1]{\hspace{-1\listmod}\emph{#1}}

\renewcommand{\hrulefill}{%
  \leavevmode\leaders\hrule height 0.2pt\hfill\kern\z@}

% General typo
\DeclareTextFontCommand{\textlarge}{\large}
\DeclareTextFontCommand{\textsmall}{\small}

% commands, inlines
\newcommand{\anchor}[1]{\Hy@raisedlink{\hypertarget{#1}{}}} % link to top of an anchor (not baseline)
\newcommand\abbr[1]{#1}
\newcommand{\autour}[1]{\tikz[baseline=(X.base)]\node [draw=rubric,thin,rectangle,inner sep=1.5pt, rounded corners=3pt] (X) {\color{rubric}#1};}
\newcommand\corr[1]{#1}
\newcommand{\ed}[1]{ {\color{silver}\sffamily\footnotesize (#1)} } % <milestone ed="1688"/>
\newcommand\expan[1]{#1}
\newcommand\foreign[1]{\emph{#1}}
\newcommand\gap[1]{#1}
\renewcommand{\LettrineFontHook}{\color{rubric}}
\newcommand{\initial}[2]{\lettrine[lines=2, loversize=0.3, lhang=0.3]{#1}{#2}}
\newcommand{\initialiv}[2]{%
  \let\oldLFH\LettrineFontHook
  % \renewcommand{\LettrineFontHook}{\color{rubric}\ttfamily}
  \IfSubStr{QJ’}{#1}{
    \lettrine[lines=4, lhang=0.2, loversize=-0.1, lraise=0.2]{\smash{#1}}{#2}
  }{\IfSubStr{É}{#1}{
    \lettrine[lines=4, lhang=0.2, loversize=-0, lraise=0]{\smash{#1}}{#2}
  }{\IfSubStr{ÀÂ}{#1}{
    \lettrine[lines=4, lhang=0.2, loversize=-0, lraise=0, slope=0.6em]{\smash{#1}}{#2}
  }{\IfSubStr{A}{#1}{
    \lettrine[lines=4, lhang=0.2, loversize=0.2, slope=0.6em]{\smash{#1}}{#2}
  }{\IfSubStr{V}{#1}{
    \lettrine[lines=4, lhang=0.2, loversize=0.2, slope=-0.5em]{\smash{#1}}{#2}
  }{
    \lettrine[lines=4, lhang=0.2, loversize=0.2]{\smash{#1}}{#2}
  }}}}}
  \let\LettrineFontHook\oldLFH
}
\newcommand{\labelchar}[1]{\textbf{\color{rubric} #1}}
\newcommand{\milestone}[1]{\autour{\footnotesize\color{rubric} #1}} % <milestone n="4"/>
\newcommand\name[1]{#1}
\newcommand\orig[1]{#1}
\newcommand\orgName[1]{#1}
\newcommand\persName[1]{#1}
\newcommand\placeName[1]{#1}
\newcommand{\pn}[1]{\IfSubStr{-—–¶}{#1}% <p n="3"/>
  {\noindent{\bfseries\color{rubric}   ¶  }}
  {{\footnotesize\autour{ #1}  }}}
\newcommand\reg{}
% \newcommand\ref{} % already defined
\newcommand\sic[1]{#1}
\newcommand\surname[1]{\textsc{#1}}
\newcommand\term[1]{\textbf{#1}}

\def\mednobreak{\ifdim\lastskip<\medskipamount
  \removelastskip\nopagebreak\medskip\fi}
\def\bignobreak{\ifdim\lastskip<\bigskipamount
  \removelastskip\nopagebreak\bigskip\fi}

% commands, blocks
\newcommand{\byline}[1]{\bigskip{\RaggedLeft{#1}\par}\bigskip}
\newcommand{\bibl}[1]{{\RaggedLeft{#1}\par\bigskip}}
\newcommand{\biblitem}[1]{{\noindent\hangindent=\parindent   #1\par}}
\newcommand{\dateline}[1]{\medskip{\RaggedLeft{#1}\par}\bigskip}
\newcommand{\labelblock}[1]{\medbreak{\noindent\color{rubric}\bfseries #1}\par\mednobreak}
\newcommand{\salute}[1]{\bigbreak{#1}\par\medbreak}
\newcommand{\signed}[1]{\bigbreak\filbreak{\raggedleft #1\par}\medskip}

% environments for blocks (some may become commands)
\newenvironment{borderbox}{}{} % framing content
\newenvironment{citbibl}{\ifvmode\hfill\fi}{\ifvmode\par\fi }
\newenvironment{docAuthor}{\ifvmode\vskip4pt\fontsize{16pt}{18pt}\selectfont\fi\itshape}{\ifvmode\par\fi }
\newenvironment{docDate}{}{\ifvmode\par\fi }
\newenvironment{docImprint}{\vskip6pt}{\ifvmode\par\fi }
\newenvironment{docTitle}{\vskip6pt\bfseries\fontsize{18pt}{22pt}\selectfont}{\par }
\newenvironment{msHead}{\vskip6pt}{\par}
\newenvironment{msItem}{\vskip6pt}{\par}
\newenvironment{titlePart}{}{\par }


% environments for block containers
\newenvironment{argument}{\itshape\parindent0pt}{\vskip1.5em}
\newenvironment{biblfree}{}{\ifvmode\par\fi }
\newenvironment{bibitemlist}[1]{%
  \list{\@biblabel{\@arabic\c@enumiv}}%
  {%
    \settowidth\labelwidth{\@biblabel{#1}}%
    \leftmargin\labelwidth
    \advance\leftmargin\labelsep
    \@openbib@code
    \usecounter{enumiv}%
    \let\p@enumiv\@empty
    \renewcommand\theenumiv{\@arabic\c@enumiv}%
  }
  \sloppy
  \clubpenalty4000
  \@clubpenalty \clubpenalty
  \widowpenalty4000%
  \sfcode`\.\@m
}%
{\def\@noitemerr
  {\@latex@warning{Empty `bibitemlist' environment}}%
\endlist}
\newenvironment{quoteblock}% may be used for ornaments
  {\begin{quoting}}
  {\end{quoting}}

% table () is preceded and finished by custom command
\newcommand{\tableopen}[1]{%
  \ifnum\strcmp{#1}{wide}=0{%
    \begin{center}
  }
  \else\ifnum\strcmp{#1}{long}=0{%
    \begin{center}
  }
  \else{%
    \begin{center}
  }
  \fi\fi
}
\newcommand{\tableclose}[1]{%
  \ifnum\strcmp{#1}{wide}=0{%
    \end{center}
  }
  \else\ifnum\strcmp{#1}{long}=0{%
    \end{center}
  }
  \else{%
    \end{center}
  }
  \fi\fi
}


% text structure
\newcommand\chapteropen{} % before chapter title
\newcommand\chaptercont{} % after title, argument, epigraph…
\newcommand\chapterclose{} % maybe useful for multicol settings
\setcounter{secnumdepth}{-2} % no counters for hierarchy titles
\setcounter{tocdepth}{5} % deep toc
\markright{\@title} % ???
\markboth{\@title}{\@author} % ???
\renewcommand\tableofcontents{\@starttoc{toc}}
% toclof format
% \renewcommand{\@tocrmarg}{0.1em} % Useless command?
% \renewcommand{\@pnumwidth}{0.5em} % {1.75em}
\renewcommand{\@cftmaketoctitle}{}
\setlength{\cftbeforesecskip}{\z@ \@plus.2\p@}
\renewcommand{\cftchapfont}{}
\renewcommand{\cftchapdotsep}{\cftdotsep}
\renewcommand{\cftchapleader}{\normalfont\cftdotfill{\cftchapdotsep}}
\renewcommand{\cftchappagefont}{\bfseries}
\setlength{\cftbeforechapskip}{0em \@plus\p@}
% \renewcommand{\cftsecfont}{\small\relax}
\renewcommand{\cftsecpagefont}{\normalfont}
% \renewcommand{\cftsubsecfont}{\small\relax}
\renewcommand{\cftsecdotsep}{\cftdotsep}
\renewcommand{\cftsecpagefont}{\normalfont}
\renewcommand{\cftsecleader}{\normalfont\cftdotfill{\cftsecdotsep}}
\setlength{\cftsecindent}{1em}
\setlength{\cftsubsecindent}{2em}
\setlength{\cftsubsubsecindent}{3em}
\setlength{\cftchapnumwidth}{1em}
\setlength{\cftsecnumwidth}{1em}
\setlength{\cftsubsecnumwidth}{1em}
\setlength{\cftsubsubsecnumwidth}{1em}

% footnotes
\newif\ifheading
\newcommand*{\fnmarkscale}{\ifheading 0.70 \else 1 \fi}
\renewcommand\footnoterule{\vspace*{0.3cm}\hrule height \arrayrulewidth width 3cm \vspace*{0.3cm}}
\setlength\footnotesep{1.5\footnotesep} % footnote separator
\renewcommand\@makefntext[1]{\parindent 1.5em \noindent \hb@xt@1.8em{\hss{\normalfont\@thefnmark . }}#1} % no superscipt in foot
\patchcmd{\@footnotetext}{\footnotesize}{\footnotesize\sffamily}{}{} % before scrextend, hyperref


%   see https://tex.stackexchange.com/a/34449/5049
\def\truncdiv#1#2{((#1-(#2-1)/2)/#2)}
\def\moduloop#1#2{(#1-\truncdiv{#1}{#2}*#2)}
\def\modulo#1#2{\number\numexpr\moduloop{#1}{#2}\relax}

% orphans and widows
\clubpenalty=9996
\widowpenalty=9999
\brokenpenalty=4991
\predisplaypenalty=10000
\postdisplaypenalty=1549
\displaywidowpenalty=1602
\hyphenpenalty=400
% Copied from Rahtz but not understood
\def\@pnumwidth{1.55em}
\def\@tocrmarg {2.55em}
\def\@dotsep{4.5}
\emergencystretch 3em
\hbadness=4000
\pretolerance=750
\tolerance=2000
\vbadness=4000
\def\Gin@extensions{.pdf,.png,.jpg,.mps,.tif}
% \renewcommand{\@cite}[1]{#1} % biblio

\usepackage{hyperref} % supposed to be the last one, :o) except for the ones to follow
\urlstyle{same} % after hyperref
\hypersetup{
  % pdftex, % no effect
  pdftitle={\elbibl},
  % pdfauthor={Your name here},
  % pdfsubject={Your subject here},
  % pdfkeywords={keyword1, keyword2},
  bookmarksnumbered=true,
  bookmarksopen=true,
  bookmarksopenlevel=1,
  pdfstartview=Fit,
  breaklinks=true, % avoid long links
  pdfpagemode=UseOutlines,    % pdf toc
  hyperfootnotes=true,
  colorlinks=false,
  pdfborder=0 0 0,
  % pdfpagelayout=TwoPageRight,
  % linktocpage=true, % NO, toc, link only on page no
}

\makeatother % /@@@>
%%%%%%%%%%%%%%
% </TEI> end %
%%%%%%%%%%%%%%


%%%%%%%%%%%%%
% footnotes %
%%%%%%%%%%%%%
\renewcommand{\thefootnote}{\bfseries\textcolor{rubric}{\arabic{footnote}}} % color for footnote marks

%%%%%%%%%
% Fonts %
%%%%%%%%%
\usepackage[]{roboto} % SmallCaps, Regular is a bit bold
% \linespread{0.90} % too compact, keep font natural
\newfontfamily\fontrun[]{Roboto Condensed Light} % condensed runing heads
\ifav
  \setmainfont[
    ItalicFont={Roboto Light Italic},
  ]{Roboto}
\else\ifbooklet
  \setmainfont[
    ItalicFont={Roboto Light Italic},
  ]{Roboto}
\else
\setmainfont[
  ItalicFont={Roboto Italic},
]{Roboto Light}
\fi\fi
\renewcommand{\LettrineFontHook}{\bfseries\color{rubric}}
% \renewenvironment{labelblock}{\begin{center}\bfseries\color{rubric}}{\end{center}}

%%%%%%%%
% MISC %
%%%%%%%%

\setdefaultlanguage[frenchpart=false]{french} % bug on part


\newenvironment{quotebar}{%
    \def\FrameCommand{{\color{rubric!10!}\vrule width 0.5em} \hspace{0.9em}}%
    \def\OuterFrameSep{\itemsep} % séparateur vertical
    \MakeFramed {\advance\hsize-\width \FrameRestore}
  }%
  {%
    \endMakeFramed
  }
\renewenvironment{quoteblock}% may be used for ornaments
  {%
    \savenotes
    \setstretch{0.9}
    \normalfont
    \begin{quotebar}
  }
  {%
    \end{quotebar}
    \spewnotes
  }


\renewcommand{\headrulewidth}{\arrayrulewidth}
\renewcommand{\headrule}{{\color{rubric}\hrule}}

% delicate tuning, image has produce line-height problems in title on 2 lines
\titleformat{name=\chapter} % command
  [display] % shape
  {\vspace{1.5em}\centering} % format
  {} % label
  {0pt} % separator between n
  {}
[{\color{rubric}\huge\textbf{#1}}\bigskip] % after code
% \titlespacing{command}{left spacing}{before spacing}{after spacing}[right]
\titlespacing*{\chapter}{0pt}{-2em}{0pt}[0pt]

\titleformat{name=\section}
  [block]{}{}{}{}
  [\vbox{\color{rubric}\large\raggedleft\textbf{#1}}]
\titlespacing{\section}{0pt}{0pt plus 4pt minus 2pt}{\baselineskip}

\titleformat{name=\subsection}
  [block]
  {}
  {} % \thesection
  {} % separator \arrayrulewidth
  {}
[\vbox{\large\textbf{#1}}]
% \titlespacing{\subsection}{0pt}{0pt plus 4pt minus 2pt}{\baselineskip}

\ifaiv
  \fancypagestyle{main}{%
    \fancyhf{}
    \setlength{\headheight}{1.5em}
    \fancyhead{} % reset head
    \fancyfoot{} % reset foot
    \fancyhead[L]{\truncate{0.45\headwidth}{\fontrun\elbibl}} % book ref
    \fancyhead[R]{\truncate{0.45\headwidth}{ \fontrun\nouppercase\leftmark}} % Chapter title
    \fancyhead[C]{\thepage}
  }
  \fancypagestyle{plain}{% apply to chapter
    \fancyhf{}% clear all header and footer fields
    \setlength{\headheight}{1.5em}
    \fancyhead[L]{\truncate{0.9\headwidth}{\fontrun\elbibl}}
    \fancyhead[R]{\thepage}
  }
\else
  \fancypagestyle{main}{%
    \fancyhf{}
    \setlength{\headheight}{1.5em}
    \fancyhead{} % reset head
    \fancyfoot{} % reset foot
    \fancyhead[RE]{\truncate{0.9\headwidth}{\fontrun\elbibl}} % book ref
    \fancyhead[LO]{\truncate{0.9\headwidth}{\fontrun\nouppercase\leftmark}} % Chapter title, \nouppercase needed
    \fancyhead[RO,LE]{\thepage}
  }
  \fancypagestyle{plain}{% apply to chapter
    \fancyhf{}% clear all header and footer fields
    \setlength{\headheight}{1.5em}
    \fancyhead[L]{\truncate{0.9\headwidth}{\fontrun\elbibl}}
    \fancyhead[R]{\thepage}
  }
\fi

\ifav % a5 only
  \titleclass{\section}{top}
\fi

\newcommand\chapo{{%
  \vspace*{-3em}
  \centering % no vskip ()
  {\Large\addfontfeature{LetterSpace=25}\bfseries{\elauthor}}\par
  \smallskip
  {\large\eldate}\par
  \bigskip
  {\Large\selectfont{\eltitle}}\par
  \bigskip
  {\color{rubric}\hline\par}
  \bigskip
  {\Large TEXTE LIBRE À PARTICIPATIONS LIBRES\par}
  \centerline{\small\color{rubric} {hurlus.fr, tiré le \today}}\par
  \bigskip
}}

\newcommand\cover{{%
  \thispagestyle{empty}
  \centering
  {\LARGE\bfseries{\elauthor}}\par
  \bigskip
  {\Large\eldate}\par
  \bigskip
  \bigskip
  {\LARGE\selectfont{\eltitle}}\par
  \vfill\null
  {\color{rubric}\setlength{\arrayrulewidth}{2pt}\hline\par}
  \vfill\null
  {\Large TEXTE LIBRE À PARTICIPATIONS LIBRES\par}
  \centerline{{\href{https://hurlus.fr}{\dotuline{hurlus.fr}}, tiré le \today}}\par
}}

\begin{document}
\pagestyle{empty}
\ifbooklet{
  \cover\newpage
  \thispagestyle{empty}\hbox{}\newpage
  \cover\newpage\noindent Les voyages de la brochure\par
  \bigskip
  \begin{tabularx}{\textwidth}{l|X|X}
    \textbf{Date} & \textbf{Lieu}& \textbf{Nom/pseudo} \\ \hline
    \rule{0pt}{25cm} &  &   \\
  \end{tabularx}
  \newpage
  \addtocounter{page}{-4}
}\fi

\thispagestyle{empty}
\ifaiv
  \twocolumn[\chapo]
\else
  \chapo
\fi
{\it\elabstract}
\bigskip
\makeatletter\@starttoc{toc}\makeatother % toc without new page
\bigskip

\pagestyle{main} % after style

  \noindent \textbf{É}veillé une bonne heure plus tôt que d’habitude, je demeurais les yeux grands ouverts, méditant sur cette anomalie. Quelque chose, je ne sais quoi, devait aller de travers. Je pressentais qu’un événement terrible se passait ou se préparait : mais de quelle nature ? J’essayai de m’orienter. Je me souvenais que, lors du grand tremblement de terre de 1906, nombre de gens prétendaient s’être éveillés quelques instants avant la première secousse et avoir éprouvé dans l’intervalle une crainte inexplicable. San-Francisco allait-il subir un nouveau cataclysme ?\par
Je restai allongé une minute, engourdi dans l’attente, sans voir osciller ni crouler les murailles. Tout était plongé dans le silence et voilà l’explication de mon étonnement. Il me manquait le brouhaha de la grande cité vivante. D’ordinaire, à cette heure-là, les tramways se succédaient dans ma rue à une moyenne de trois minutes : et au cours des dix minutes suivantes, pas un ne se fit entendre. J’en conclus que les employés pouvaient être en grève ou que le courant faisait défaut par suite d’un accident. Mais non ! Le silence était trop profond : je n’entendais ni cahots, ni grincements de roues, ni piétinements de sabots sur les pavés en pente.\par
Je touchai le bouton à la tête de mon lit et tentai de percevoir le bruit de la sonnette, sachant pourtant qu’il ne pouvait me parvenir à travers les trois étages inférieurs. Mais la sonnerie fonctionnait car, au bout de quelques minutes, Brown entra avec le plateau et un journal du matin. Encore qu’il conservât son impassibilité coutumière, je surpris dans ses yeux une lueur d’inquiétude et remarquai également l’absence de lait sur le plateau.\par
— La crèmerie n’a pas fait de livraison aujourd’hui, déclara-t-il, la boulangerie non plus.\par
Je reportai mes regards sur le plateau : pas de croissants français, rien que des tranches de pain de seigle rassis de la veille, le plus détestable aliment que je connaisse.\par
— On ne m’a rien apporté, Monsieur, continua Brown en guise d’explication ; mais je l’interrompis.\par
— Le journal ?\par
— Oui, monsieur, c’est la seule chose qui ait été distribuée, et pour la dernière fois. Demain, les journaux ne paraîtront pas : c’est annoncé dans le journal. Dois-je envoyer chercher une boîte de lait condensé ?\par
Je fis un signe négatif, acceptai le café noir et dépliai la feuille. Les en-têtes expliquaient toute la situation, et exagéraient même, car les articles débordaient d’un pessimisme ridicule. Une grève générale, disaient-ils, venait d’être déclarée dans tous les États-Unis ; et le journal exprimait les craintes les plus alarmantes à propos de l’approvisionnement des grandes villes.\par
Je le parcourus à la hâte, esquivant les détails et me souvenant des troubles travaillistes du passé. Depuis une génération, la grève générale était le dada des travailleurs organisés : songe issu du cerveau de Debs, un des grands chefs de la classe ouvrière une trentaine d’années auparavant.\par
Je me rappelais même avoir pondu pour une revue, vers la fin de mes études, une étude sur ce sujet, intitulé : « Le Rêve de Debs ». Je dois l’avouer, j’avais traité la question comme une chimère et rien de plus. Le temps et le monde continuant à rouler, la fédération Américaine du Travail avait disparu, Gompers également, de même que Debs avec toutes ses théories révolutionnaires. Cependant le rêve avait persisté, et voici enfin qu’il se réalisait.\par
Puis, en lisant, je me mis à rire des sombres pronostics du journal. Témoin de la déconfiture du Travail Organisé à la suite de tant de conflits, je savais à quoi m’en tenir. L’affaire serait réglée au bout de quelques jours. Il s’agissait cette fois d’une grève nationale, et le gouvernement ne tarderait pas à la briser.\par
Je jetai le journal et m’habillai. Une promenade dans les rues de San-Francisco serait certainement intéressante alors qu’aucune roue n’y tournait et que la ville entière prenait des vacances forcées.\par
— Je vous demande pardon, Monsieur, me dit Brown en me présentant mon étui à cigares, mais Harmmed demande à vous voir avant que vous sortiez.\par
— Faites-le monter tout de suite, répondis-je.\par
Harmmed était le maître d’hôtel. À première vue, je le devinai en proie à une émotion contenue. Il entra tout de suite en matière.\par
— Que faire, Monsieur ? Il faudra des provisions, et les livreurs sont en grève. Et l’électricité est coupée… les électriciens doivent se croiser les bras, eux aussi.\par
— Les boutiques sont-elles ouvertes ? demandai-je.\par
— Les petites seulement, Monsieur. Les employés des maisons de détail sont partis et les grands magasins ne peuvent ouvrir : mais les boutiquiers et leur famille servent eux-mêmes les clients.\par
— En ce cas, prenez l’auto, dis-je, et allez faire une tournée d’emplettes. Achetez en quantité suffisante tout ce dont vous aurez besoin. Rapportez un paquet de bougies… non, une demi-douzaine de paquets. Et quand vous aurez fini, dites à Harrison d’amener la voiture me prendre au club… pas plus tard qu’à onze heures.\par
Harmmed hocha gravement la tête.\par
— Monsieur Harrison est en grève avec le Syndicat des chauffeurs, et moi-même je ne sais pas conduire.\par
— Tiens ! tiens ! tiens ! Eh bien ! Quand \emph{Monsieur} Harrison reviendra par ici, vous lui direz qu’il peut chercher une situation ailleurs.\par
— Bien, Monsieur.\par
— Vous n’appartenez pas, au moins, au Syndicat des maîtres d’hôtel, Harmmed ?\par
— Non, Monsieur, fut la réponse. Et même si j’y appartenais, je ne lâcherais pas mon patron dans une crise comme celle-ci. Non, Monsieur, je…\par
— À la bonne heure, je vous remercie. Maintenant, tenez-vous prêt à m’accompagner. Je prendrai le volant moi-même et nous recueillerons assez de provisions pour soutenir un siège.\par
Un beau premier mai, comme d’habitude. Pas un nuage au ciel : un air calme et tiède, presque embaumé. Beaucoup d’automobiles dehors, mais conduites par leurs propriétaires. Rues populeuses mais tranquilles. La classe ouvrière, endimanchée, prenait l’air et observait les effets de la grève.\par
Tout cela semblait une aventure si extraordinaire et néanmoins si paisible que j’y prenais moi-même un certain plaisir. Mes nerfs frémissaient d’une légère animation. Je croisai Miss Chickering, au volant de sa légère auto : elle fit un demi-tour et me rattrapa au coin.\par
— Bonjour, Monsieur Corf ! Oh ! dites-moi savez-vous où je pourrais me procurer des bougies ? Je viens d’entrer dans une demi-douzaine de boutiques, et il n’en reste plus. N’est-ce pas effarant ?\par
Mais l’éclat de ses yeux démentait ses paroles. Comme nous autres, elle se divertissait prodigieusement. Quelle aventure que cette exploration à la recherche de chandelles !\par
Il nous fallut traverser la ville et descendre dans le quartier ouvrier, au sud de Market Street, pour trouver de petites épiceries où il restât quelques objets à vendre.\par
Il nous fallut traverser la ville et Miss Chickering pensait qu’un paquet suffirait, mais je la persuadai d’en prendre quatre. La place ne manquait pas dans ma voiture, et j’y déposai une douzaine de paquets. On ne pouvait prédire quand se terminerait cette grève. Je remplis l’auto de sacs de farine, de levure, de conserves et de toutes les provisions conseillées par ce brave Harmmed, qui se démenait et gloussait comme un vieux coq.\par
La note dominante en ce premier jour de grève était l’absence de toute appréhension sérieuse. On faisait des gorges chaudes de l’annonce insérée dans les journaux du matin par la Confédération du Travail déclarant qu’elle tiendrait un mois ou un trimestre s’il le fallait. Nous aurions dû cependant le deviner dès ce premier jour en constatant que la classe ouvrière ne prenait aucune part à la ruée vers les provisions. Naturellement : depuis des semaines et des mois, elle avait adroitement et discrètement accumulé des réserves. Voilà pourquoi on nous laissait nettoyer les petites épiceries du quartier populeux.\par
Ce fut seulement dans l’après-midi, en arrivant au club, que je commençai de ressentir les premières alarmes. Plus d’olives pour les cocktails, et le service procédait par à-coups. La plupart des membres paraissaient irrités, quelques-uns tracassés.\par
Un tumulte de voix accueillit mon entrée. Le général Folsom, sa vaste panse étalée sur une banquette dans l’embrasure de la fenêtre du fumoir, se défendait contre une demi-douzaine de messieurs excités qui l’exhortaient à faire quelque chose.\par
— Que voulez-vous que je fasse de plus ? disait-il. Je ne reçois aucun ordre de Washington. Que l’un de vous envoie un télégramme et j’exécuterai ce que l’on me commandera. Mais je ne vois pas les mesures qu’on pourrait prendre. J’ai pris l’initiative ce matin, dès que j’eus appris la déclaration de grève, d’alerter les trois mille hommes de la garnison. Ils montent la garde auprès des banques, de la Monnaie, de la grande poste et de tous les monuments publics. Nul désordre ne s’est produit. Vous ne pouvez tout de même pas vous attendre à ce que je fasse tirer sur des gens qui se baladent paisiblement avec femmes et enfants dans leurs frusques du dimanche !\par
— Je voudrais bien savoir ce qui se passe à Wall Street, entendis-je remarquer par Jimmy Wombold.\par
Et j’imaginai facilement son inquiétude, car il se trouvait fortement engagé dans la grosse affaire du Consolidated Western.\par
— Dites donc, Corf ! m’interpella Atkinson au passage, votre auto marche-t-elle comme il faut ?\par
— Oui, répondis-je. Qu’est-il donc arrivé à la vôtre ?\par
— Elle est démolie. Tous les garages sont fermés. Et ma femme se trouve en panne quelque part de l’autre côté de la baie, à Truckee, je crois. Pas moyen de lui envoyer un télégramme à prix d’argent ou d’or. Elle devait arriver ce soir. Peut-être meurt-elle de faim. Prêtez-moi donc votre bagnole !\par
— Vous ne pourriez lui faire traverser la baie, objecta Halstead. Les bacs ne fonctionnent plus. Tenez, voici Rollinson. Eh ! Rollinson, écoutez un peu. Atkinson voudrait faire traverser la baie à une automobile. Sa femme est en détresse, dans le train, à Truckee. Ne pourriez-vous amener la \emph{Lurlette} de Tiburon pour transborder sa voiture ?\par
La \emph{Lurlette} était un yacht de deux cents tonneaux, gréé en goélette.\par
Rollinson hocha la tête.\par
— Vous ne trouveriez pas un seul débardeur pour embarquer l’automobile, même si je pouvais faire venir la \emph{Lurlette}, ce qui, d’ailleurs, m’est impossible, attendu que l’équipage fait partie du Syndicat des Gens de Mer, en grève comme tous les autres.\par
— Mais ma femme va mourir de faim ! gémit Atkinson.\par
À l’autre bout du fumoir, je tombai sur un groupe qui se démenait autour de Bertie Messener. Et Bertie les excitait à sa façon, cyniquement froide. Bertie se souciait peu de la grève, comme de tout le reste, d’ailleurs. C’était un blasé, du moins en ce qui concernait toutes les choses propres de la vie : son côté immonde n’exerçait aucun attrait sur lui. Ayant hérité de son père et de son oncle d’une vingtaine de millions de dollars placés dans des affaires de tout repos, il n’avait de sa vie accompli le moindre travail producteur, se contentant de courir le monde, de tout voir et de tout faire, sauf se marier. En dépit des attaques résolues de plusieurs centaines de mamans ambitieuses, considéré comme de bonne prise depuis plusieurs années, il ne se laissait pas capturer. Outre sa fortune, il possédait la jeunesse, la beauté et, comme je l’ai dit, la propreté morale. Superbe athlète, ce jeune dieu blond, mariage à part, faisait tout dans la perfection, ne se préoccupait de rien, ne couvait ni ambition ni passions, ni désir même d’exécuter des actes dont il s’acquittait mieux que personne.\par
— C’est une sédition ! criait un des hommes du groupe.\par
Et les autres de surenchérir, de parler de révolte, de révolution, d’anarchie même.\par

\astermono

\noindent \textbf{J}e ne vois rien de pareil, déclara Bertie. Je me suis promené toute la matinée dans les rues. Partout régnait un ordre parfait. Je n’ai jamais vu une population si respectueuse de la loi. Inutile d’employer de grands mots. Vous vous trompez sur toute la ligne. La grève est ce qu’elle prétend être : une grève générale, et c’est à vous de jouer, Messieurs !\par
— Oh ! nous jouerons comme il convient, répartit Garfield, un des millionnaires intéressés dans les moyens de transport. Nous montrerons à cette racaille où est sa place… tas de brutes ! Attendez seulement que le gouvernement intervienne !\par
— Mais où est votre gouvernement ? intervint Bertie. Il pourrait être tout aussi bien au fond de la mer en ce qui vous touche personnellement. Vous ignorez ce qui se passe à Washington. Vous ne savez même pas si vous avez ou non un gouvernement.\par
— Ne vous tracassez pas de cela, aboya Garfield.\par
— Je vous jure que je ne m’en tracasse pas le moins du monde, répondit Bertie avec un sourire languissant. Il me semble bien que ce soit vous qui vous tracassiez. Regardez-vous dans la glace, Garfield.\par
Garfield ne s’y regarda pas, mais il y aurait vu un monsieur, très excité, avec des cheveux gris de fer ébouriffés, une face congestionnée, une bouche maussade, vindicative et des yeux flamboyants.\par
— Ce n’est pas juste, je vous le déclare ! dit le petit Hanover, et je devinai, à son intonation, qu’il l’avait déjà déclaré à plusieurs reprises.\par
— Eh bien ! ceci dépasse les bornes, Hanover ! répliqua Bertie. Vous finissez par me fatiguer, tous, tant que vous êtes ! Vous employez à la fois des ouvriers syndiqués et non syndiqués. Vous me rebattez les oreilles avec votre liberté du travail et le droit pour l’homme de travailler. Depuis des années, tel est le refrain de vos harangues. Les travailleurs ne commettent aucun crime en organisant cette grève générale : ils ne violent aucune loi divine ni humaine. Cessez de geindre, Hanover. Depuis trop longtemps vous trompez le peuple en lui faisant sonner aux oreilles le droit de l’homme à travailler… ou à ne pas travailler ; vous ne sauriez en esquiver les conséquences. C’est une basse supercherie. Vous avez opprimé la classe ouvrière en lui serrant la vis ; maintenant elle vous tient et la serre à son tour, voilà tout, et vous jetez les hauts cris.\par
Tous les hommes du groupe éclatèrent en clameurs indignées, affirmant n’avoir jamais serré la vis aux travailleurs.\par
— Parfaitement, Monsieur ! hurlait Garfield. Nous avons fait notre devoir pour l’ouvrier. Au lieu de lui serrer la vis, nous lui offrions un moyen de vivre. Nous lui fournissions du travail. Que deviendrait-il sans nous ?\par
— Il ne s’en porterait que mieux, ricana Bertie. Vous avez maté l’ouvrier toutes les fois que vous en trouviez l’occasion et vous vous dérangiez au besoin pour la faire naître.\par
— Non ! non ! Ce n’est pas vrai ! hurla le chœur.\par
— Il y eut la grève des charretiers, ici même, à San-Francisco, poursuivit imperturbablement Bertie. L’Association des patrons a précipité cette grève. Vous ne l’ignorez pas. Et vous savez que je suis au courant, car je me trouvais ici même bien placé pour entendre les nouvelles et connaître les dessous de la lutte. D’abord, vous avez provoqué la grève, puis vous avez acheté le maire et le chef de la police pour la faire briser. C’était un spectacle intéressant de vous voir, Messieurs les philanthropes, abattre les charretiers et leur serrer la vis.\par
« Attendez, je n’ai pas fini. C’est l’an dernier seulement que les votes des travailleurs du Colorado ont élu un gouvernement. Celui-ci n’a jamais siégé, et vous en connaissez la raison : vous savez comment s’y prirent les philanthropes et capitalistes du Colorado. Ce fut un superbe exemple de votre manière d’écraser les travailleurs. Sur des accusations de meurtre forgées, de toutes pièces, vous avez gardé en prison, pendant trois ans, le président du Syndicat des mineurs du sud-ouest, et vous en avez profité pour démolir ce syndicat. Beau tour de vis, hein ? Vous l’avez renouvelé lorsque, pour la troisième fois, l’impôt progressif fut déclaré contraire à la constitution, et encore une fois en étouffant la loi de huit heures.\par
« Mais voici le comble de ces procédés éhontés : vous vous rappelez votre façon de vous y prendre en achetant Farburg, dernier président de l’ancienne Fédération américaine du Travail ? Il était votre créature, ou celle de tous les syndicats patronaux, ce qui revient au même. Vous avez fomenté la grande grève contre l’emploi des jaunes. Elle fut trahie par Farburg. Vous avez gagné la partie et la vieille Fédération du Travail est tombée en miettes. C’est vous-mêmes qui l’avez démolie, et vous en avez subi le contre-coup, car sur ses ruines se dressa la Ligue Internationale des Travailleurs\footnote{\emph{Internationnal League of Workers.}}, l’organisation la plus vaste et la plus solide des États-Unis. Vous êtes responsables de son existence et de la grève générale actuelle. En détruisant toutes les anciennes fédérations, vous avez contribué à la naissance de la I. L. W., qui vient de déclarer la grève générale, toujours sur le principe de la porte fermée aux syndiqués. Et vous osez me regarder en face et me dire que vous n’avez jamais fait le moindre mal aux travailleurs ? Allons donc !\par
Aucune dénégation ne s’éleva cette fois. Garfield s’écria en manière de défense :\par
— Nous n’avons rien fait que nous n’y eussions été contraints, si nous voulions gagner la partie.\par
— Il n’est point question de cela, répondit Bertie. Ce dont je me plains, c’est de vous entendre jeter les hauts cris maintenant que la situation se retourne contre vous. Combien de grèves avez-vous gagnées en réduisant les ouvriers à la famine ? Eh bien ! les ouvriers ont trouvé un moyen analogue de vous soumettre. S’ils ne peuvent y réussir qu’en vous affamant, vous crèverez de faim, voilà tout !\par
— Permettez-moi de vous dire que dans le passé, vous avez profité vousmême de tours de vis en question, insinua Brentwood, un des plus vils renards de la corporation des avocats. Le receleur ne vaut pas mieux que le voleur, ricana-t-il. Vous n’avez pas mis la main au pressoir, mais vous avez pris votre part de la cuvée.\par
— Ceci est tout à fait en dehors de notre affaire, Brentwood, dit Bertie d’une voix traînante. Vous vous montrez aussi médiocre que Hanover en empiétant sur la question morale. Je n’ai pas dit que telle ou telle chose était bonne ou mauvaise. Toute la société est pourrie, je le sais : et si je regimbe, c’est en vous entendant brailler maintenant que vous voilà par terre et que le monde ouvrier vous serre la vis. Naturellement, j’ai participé aux bénéfices que je vous reproche, et grâce à vous, Messieurs, sans mettre la main personnellement à la répugnante besogne. Vous vous en êtes chargés pour moi, non pas, vous pouvez m’en croire, que je sois plus vertueux que vous, mais parce que mon bon père et ses divers frères m’ont laissé assez de fortune pour payer les corvées abjectes.\par
— Si vous voulez insinuer que…, commença Brentwood en s’échauffant.\par
— Un instant, ne vous hérissez pas ! interrompit Bertie avec insolence. Inutile de jouer les hypocrites dans cette caverne de voleurs. Les attitudes empesées et superbes produisent leur effet dans les journaux, dans les patronages et écoles du dimanche : cela fait partie du jeu : mais, pour l’amour du ciel, ne recourons pas à ces moyens-là entre nous. Vous savez que je connais aussi bien que vous les tripotages accomplis dans la grève du bâtiment à l’automne dernier et les personnages qui ont fourni l’argent, ceux qui ont exécuté le travail et ceux qui en ont profité. – Brentwood devint cramoisi. Mais nous sommes tous barbouillés avec le même pinceau, et ce que nous avons de mieux à faire est de laisser de côté la moralité. Je le répète, jouez votre jeu, jouez-le jusqu’au bout, mais ne glapissez pas quand vous recevez des atouts.\par
Quand je quittai le groupe, Bertie, lancé sur une nouvelle piste, s’amusait à les torturer en leur peignant les aspects les plus sombres de la situation, l’insuffisance des provisions qui se faisait déjà sentir, et en leur demandant comment ils comptaient y remédier. Sur le point de sortir, je le retrouvai au vestiaire et le ramenai chez lui dans mon automobile.\par
— Un fameux coup, cette grève générale, me dit-il, tandis que nous dévalions dans les rues encombrées d’une foule tranquille. C’est un coup assourdissant. La classe ouvrière nous a surpris assoupis et nous a frappés à l’endroit le plus sensible, à l’estomac. Je vais quitter San-Francisco, Corf. Croyez-moi, faites-en autant. Filez à la campagne, n’importe où. Vous vous en trouverez bien. Achetez des vivres et réfugiez-vous sous une tente ou dans une cabane. Ici, nos pareils se trouveront bientôt réduits à la famine.\par
Je ne soupçonnais guère à quel point Bertie Messener raisonnait juste. Je le prenais pour un alarmiste. Pour moi, je me contentais de rester là à suivre la farce. Après l’avoir quitté, au lieu de rentrer droit à la maison, je me mis en quête de nouvelles provisions et fus surpris d’apprendre que les petites épiceries où j’étais allé le matin n’avaient plus rien à vendre. J’étendis mon cercle de recherches jusqu’au Potrero, où j’eus la chance de trouver un autre paquet de bougies, deux sacs de farine de froment, deux sacs de farine de seigle (pour les domestiques), une caisse de maïs et une autre de conserves de tomates. Tout semblait présager une disette temporaire, et je me félicitai d’avoir amassé une assez bonne réserve de comestibles.\par

\astermono

\noindent Le lendemain matin je pris mon café noir au lit comme d’habitude et, plus encore que le lait, mon journal me fit défaut. Cette absence de nouvelles sur ce qui se passait dans le monde m’affectait profondément. Au club, je n’appris pas grand-chose. Rider avait traversé l’estuaire d’Oakland sur son canot et Halstead était allé à San José et était revenu à San-Francisco. Ils annoncèrent que la situation était la même qu’à San-Francisco. La grève paralysait tout. Les gens du monde avaient raflé tous les approvisionnements des boutiques. Partout régnait un ordre parfait. Mais que se passait-il dans le reste du pays… à Chicago, à New York, à Washington ? Tout s’y passait probablement comme à San-Francisco, fut notre conclusion : mais l’absence de renseignements certains était irritante.\par
Le général Folsom nous communiqua quelques nouvelles. On avait essayé de mettre des télégraphistes militaires dans les bureaux de poste, mais les fils avaient été coupés dans toutes les directions. C’était jusqu’à présent le seul acte illégal commis par les grévistes, un acte pré-concerté, d’après l’opinion du général. S’étant mis en rapport par sans-fil avec le poste militaire de Bénicia, il avait appris qu’en ce moment même des patrouilles de soldats surveillaient les lignes télégraphiques jusqu’à Sacramento. Une fois même on y avait reçu un appel de cette dernière ville, mais aussitôt, quelque part, les fils furent coupés de nouveau. Le général pensait que dans tout le pays les autorités tentaient des efforts analogues, mais il ne se risquait pas à prédire si ces efforts réussiraient. Ce coupage de fils le tracassait : il ne pouvait s’empêcher d’y voir un détail important d’un plan prémédité par les travailleurs. Il regrettait en outre que le gouvernement n’eût pas réalisé depuis longtemps son projet d’établir un relai de postes sans-fil.\par
Les jours s’écoulèrent avec une monotonie désespérante. Il n’arrivait rien, et l’émotion commençait à s’émousser. La foule n’encombrait plus les rues. Les ouvriers ne venaient plus dans la ville haute pour voir comment nous prenions la grève, et il y roulait moins d’automobiles. Les garages et boutiques de réparations étant fermés, toute voiture en panne restait au rancart. Le manchon d’embrayage de la mienne s’était brisé, je ne pus la faire réparer à n’importe quel prix. Comme les autres, j’allai désormais à pied.\par
San-Francisco semblait mort, et nous ignorions ce qui se passait ailleurs, mais de cette ignorance nous pouvions déduire que le reste du pays restait plongé dans le même engourdissement. De temps en temps, la ville s’émaillait des proclamations de l’organisation du Travail, imprimées depuis plusieurs mois et témoignant du soin avec lequel la I. L. W. s’était préparée à la grève. Tous les détails en avaient été élaborés longtemps d’avance. Jusqu’ici aucune violence ne s’était produite, sauf, de la part des soldats, l’abattage à coups de fusils de quelques coupeurs de fils télégraphiques, mais la population des quartiers pauvres mourait de faim et commençait à s’agiter de façon inquiétante.\par
Les hommes d’affaires, les millionnaires et les classes libérales tenaient des réunions et adoptaient des résolutions, mais ne trouvaient aucun moyen de les publier, ni même de les faire imprimer. Un des résultats de ces meetings, cependant, fut de persuader le général Folsom de prendre militairement possession des maisons de gros et de tous les magasins de farines, grains et vivres de toute sorte. Il n’était que temps, car la pénurie sévissait dans les maisons des riches, et les distributions de vivres s’imposaient. Les figures de mes serviteurs s’allongeaient, et je fus stupéfait du trou creusé par eux dans mon tas de provisions. De fait, comme je le conjecturai plus tard, chacun d’eux me volait et mettait de côté un stock de provisions pour son propre compte.\par
Mais avec le rationnement surgirent de nouveaux ennuis. Il n’y avait à San-Francisco qu’une quantité de vivres limitée et qui ne durerait pas longtemps.\par
Nous savions que les ouvriers organisés possédaient leurs approvisionnements à eux ; néanmoins toute la classe ouvrière vint faire queue pour les distributions.\par
Il en résulta que les provisions confisquées par le général Folsom diminuèrent avec une rapidité vertigineuse. Comment les soldats pouvaient-ils distinguer entre un homme de la classe moyenne aux habits râpés, un membre de la I. L. W. ou un habitant des quartiers pauvres de la ville ? Ils n’auraient dû donner qu’à ces premiers ou derniers, mais ils ne connaissaient pas tous les membres de la I. L. W., encore moins leurs femmes, fils ou filles.\par
Avec l’aide de certains patrons, quelques syndiqués furent expulsés des files : mais cela n’avançait à rien. Pour empirer les choses, les remorqueurs du gouvernement qui transportaient les approvisionnements militaires de Mare Island à Angel Island ne trouvèrent plus rien à prendre dans le premier de ces dépôts : désormais les soldats reçurent leurs rations sur les approvisionnements confisqués et furent servis les premiers.\par
La fin s’annonçait, et la violence ne tarda pas à se manifester. La loi et l’ordre disparurent, et tout d’abord, il faut l’avouer, parmi les miséreux et les riches, tandis que les travailleurs organisés se maintenaient dans l’ordre : cela leur était facile, car les vivres ne leur manquaient pas.\par
Certain après-midi, je trouvai au club Halstead et Brentwood en train de comploter dans un coin. Ils m’initièrent à l’aventure projetée. L’automobile de Brentwood fonctionnait encore, et ils allaient voler du bétail. Halstead s’était muni d’un grand couteau et d’un couperet de boucherie.\par
Nous gagnâmes les environs de la ville. Des vaches paissaient par-ci, par-là, mais toujours sous la garde de leurs propriétaires. Nous continuâmes notre expédition en suivant la lisière de la cité vers l’est, et sur les hauteurs voisines de \emph{Hunter’s Point} nous rencontrâmes une vache gardée par une fillette. Nous ne perdîmes pas de temps en préliminaires, et la petite se sauva en hurlant pendant que nous égorgions la vache. Je passe les détails peu ragoûtants, car nous ignorions la façon de nous y prendre, et ce fut un fameux gâchis !\par
Au beau milieu de cette opération activée par la crainte, nous entendîmes des cris et vîmes accourir une bande d’hommes. Nous prîmes la fuite en abandonnant notre proie. À notre grande surprise, personne ne nous poursuivit et, tournant la tête, nous aperçûmes les intrus en train de découper fiévreusement la bête. Ils jouaient le même jeu que nous.\par
Pensant qu’il y en aurait assez pour tous, nous rebroussâmes chemin. La scène qui suivit défie toute description. Nous luttions et nous nous démenions comme des sauvages. Brentwood, je m’en souviens, avait l’air d’une parfaite brute, grognant, happant les morceaux et menaçant de tuer si on ne nous laissait pas une part suffisante.\par
Nous tenions déjà notre part lorsqu’une nouvelle irruption se produisit sur la scène. Nous avions affaire, cette fois, aux redoutables agents volontaires de la I. L. W., que la fillette était allée chercher. Au nombre d’une vingtaine, ils étaient armés de fouets et de casse-têtes. La petite fille, les joues mondées de larmes, tressautait de colère en criant : « Allez-y ! Tapez dessus ! Ce mannequin à lunettes, tenez, c’est lui qui a fait le coup ! Cassez-lui la figure ! »\par
Le mannequin à lunettes, c’était moi, et j’eus la figure suffisamment démolie, bien qu’ayant pris la précaution d’enlever mes verres séance tenante. Nous reçûmes une volée carabinée et nous nous dispersâmes dans toutes les directions. Brentwood, Halstead et moi courûmes vers l’auto. Brentwood saignait du nez et la joue de Halstead portait une marque de fouet sanguinolente.\par
Lorsque, échappés à la poursuite, nous rejoignîmes l’automobile, nous aperçûmes un petit veau qui se cachait derrière. Brentwood nous recommanda d’avancer prudemment, et lui-même se mit à ramper comme un loup ou un tigre. Couteau et couperet étaient restés dans la bagarre, mais Brentwood avait toujours ses mains, et il se roula par terre avec la pauvre bête, jusqu’à ce qu’il l’eût étranglée. Nous jetâmes la carcasse dans la voiture, étendîmes une couverture dessus et reprîmes la direction de la ville.\par
Mais nos déboires ne faisaient que commencer. Un pneu éclata. Nous n’avions aucun moyen de le remplacer et le crépuscule tombait. Il fallut abandonner l’auto. Brentwood allait devant, soufflant et trébuchant, portant sur ses épaules le veau avec sa couverture. Nous nous relayions sous le faix et nous sentions à bout de forces : en outre, nous nous étions égarés. Pour comble, nous rencontrâmes une bande de vauriens. Ils n’appartenaient pas à la I. L. W., et je crois qu’ils étaient aussi affamés que nous. En tout cas, ils s’emparèrent du veau et nous octroyèrent la raclée.\par
Pendant tout le reste du parcours, Brentwood se démena comme un aliéné, dont il avait l’air avec ses vêtements en loques, son nez enflé et ses yeux au beurre noir.\par
Désormais, il ne fut plus question de vols de bétail. Le général Folsom envoya ses troupiers rafler toutes les bêtes, et ses soldats, aidés par la milice, consommèrent la plus grande partie de la viande. On ne saurait blâmer le général. Sa consigne était de maintenir l’ordre et la légalité, et il les maintenait grâce à l’armée, qu’il devait nourrir avant tout.\par
Ce fut vers cette époque que se produisit la grande panique. Les classes aisées donnèrent l’exemple de la fuite, et la contagion gagna les faubourgs, dont les habitants quittèrent la ville dans une ruée éperdue.\par
Cet exode n’était pas pour déplaire au général Folsom. On estime que 200 000 âmes désertèrent San-Francisco : autant de bouches en moins à nourrir.\par
Je me rappelle bien cette journée. Le matin j’avais mangé une croûte de pain. Pendant la moitié de l’après-midi, j’avais fait queue, et à la tombée de la nuit, fatigué et lamentable, je rapportai à la maison un litre de riz et une tranche de lard. Brown vint m’ouvrir la porte, avec un visage morne et terrifié. Il m’informa de la fuite de tout mon personnel.\par

\astermono

\noindent Il demeurait seul. Touché de sa fidélité, et apprenant qu’il n’avait rien mangé de la journée, je partageai mes vivres avec lui. Nous fîmes cuire la moitié du riz et du lard, réservant le reste pour le lendemain.\par
J’allai me coucher sur ma faim et m’agitai toute la nuit sans pouvoir dormir. Je découvris au matin que Brown m’avait abandonné et, pis encore, avait emporté le reste du riz et du lard.\par
Au club, je trouvai une poignée d’hommes bien misérables. Plus de service du tout : le dernier serviteur venait de partir. Je remarquai la disparition de l’argenterie, et je devinai son destin. Le personnel ne l’avait pas volée, pour la bonne raison, je présume, que les membres en avaient déjà disposé par une méthode très simple. Au sud de Market Street, dans les logements ouvriers, les ménagères avaient fourni en échange des repas substantiels.\par
Je retournai chez moi. Effectivement, mon argenterie s’était aussi envolée, à l’exception d’une cruche massive, que j’enveloppai de papier et emportai au sud de Market Street.\par
Après un bon repas, je me sentis mieux et revins au club pour voir si la situation se modifiait. Hanover, Collins et Dakon en sortaient. Ils me dirent qu’il n’y avait plus personne à l’intérieur et m’invitèrent à me joindre à eux. Ils allaient quitter la ville sur les chevaux de Dakon ; celui-ci mit le dernier à ma disposition.\par
Dakon possédait quatre superbes chevaux d’attelage auxquels il tenait, et le général Folsom l’avait prévenu confidentiellement que le lendemain matin tous les chevaux demeurant dans la ville seraient réquisitionnés et abattus comme viande de boucherie. Il n’en restait guère, car on en avait lâché des milliers dans la campagne, dès les premiers jours de la disparition du foin et des grains.\par
Birdall, qui possédait de gros intérêts dans une entreprise de charrois, en avait mis trois cents en liberté ; estimés en moyenne à cinq cents dollars pièce, ils représentaient une perte sèche de cent cinquante mille dollars. Il espérait en retrouver la plupart après la grève, mais en définitive, il n’en revit pas un seul. Les fuyards de San Francisco les avaient tous mangés. D’ailleurs, on commençait déjà à abattre les chevaux et mulets de l’armée.\par
Heureusement, Dakon possédait dans son écurie une bonne provision de foin et de grain. Nous réussîmes à nous procurer quatre selles et trouvâmes les animaux en bonne forme, bien qu’ils ne fussent pas habitués à être montés.\par
Nous quittâmes Union Square pour nous engager dans les quartiers des théâtres, hôtels et grands magasins. Les rues étaient désertes. Par-ci, par-là nous rencontrions des automobiles en panne, abandonnées sur place. Aucun signe de vie, sauf quelques agents de police et les soldats de garde devant les banques et monuments publics.\par
Une seule fois nous fîmes halte pour lire la proclamation qu’un syndiqué était en train de coller. « Nous avons strictement maintenu l’ordre, disait l’affiche, et nous le maintiendrons jusqu’au bout. La fin viendra quand nos demandes auront reçu satisfaction, et elles la recevront quand nos employeurs auront été soumis par la famine, comme nous-mêmes l’avons été plus d’une fois. »\par
— Ce sont les propres termes de Messener, remarqua Collins. Et pour ma part, je suis tout disposé à me soumettre, mais ils se gardent de m’en fournir la moindre occasion. Je n’ai pas pris un bon repas depuis une éternité. Je me demande quel goût peut avoir la viande de cheval ?\par
Nous nous arrêtâmes devant une autre proclamation : « Quand nous croirons nos employeurs disposés à se soumettre, nous ouvrirons les bureaux télégraphiques et nous mettrons en communication avec les associations de patrons des États-Unis. Mais nous ne laisserons passer que les dépêches ayant trait aux conditions de paix. »\par
Au-delà de Market Street, nous entrions dans le quartier ouvrier. Ici, plus de rues désertes. Les syndiqués s’appuyaient aux chambranles de leurs portes ou causaient en petits groupes. Des enfants jouaient, joyeux et bien nourris, et de plantureuses ménagères bavardaient assises sur le seuil des portes. Tous nous jetaient des regards amusés. Des gosses nous couraient après, en criant : – « Eh ! m’sieu ! ça va, l’appétit ? – Une femme qui allaitait son bébé, cria à Dakon : « Eh ! bouffi ! veux-tu un gueuleton en échange de ton canasson… jambon et frites, gelée de groseilles, pain blanc, beurre et deux tasses de café ? »\par
— Avez-vous remarqué, me demanda Hanover, que depuis ces derniers jours il ne reste pas un seul chien errant dans les rues ?\par
Je l’avais remarqué sans y songer autrement. Il était grand temps de quitter cette malheureuse ville. Nous arrivâmes enfin à San Bruno Road, que nous devions suivre vers le sud. Je possédais une maison de campagne à Menlo, et c’est là que nous allions. Mais nous ne devions pas tarder à nous apercevoir que la campagne était encore plus dénuée et plus dangereuse que la ville. Dans celle-ci, soldats et syndiqués maintenaient l’ordre, mais la campagne était livrée à l’anarchie. Deux cent mille êtres avaient fui de San-Francisco, et nous trouvâmes de nombreuses preuves que leur passage avait produit le même effet que celui d’une nuée de sauterelles.\par
Ils avaient tout balayé, commettant partout vols et violences. De temps en temps, nous passions devant des cadavres étendus au bord de la route ou devant les ruines noircies de fermes incendiées. Les barrières avaient été abattues, les moissons foulées aux pieds, les légumes arrachés, les animaux de basse-cour égorgés par les hordes faméliques.\par
À l’écart des routes, quelques fermiers s’étaient défendus à coups de fusils de chasse et de revolvers, et tenaient bon encore. Ils nous criaient de passer au large et refusaient de parlementer avec nous. Toutes ces violences et destructions étaient l’œuvre des miséreux et des gens de classe supérieure. Les syndiqués, pourvus d’abondantes provisions, demeuraient tranquilles dans leurs maisons urbaines.\par
Nous ne devions guère tarder à recevoir des preuves concrètes de cette situation désespérée. Tout à coup, nous entendîmes sur notre droite des cris et des coups de feu, tandis que des balles sifflaient à dangereuse proximité. Des craquements se produisirent dans un fourré et un superbe cheval noir de trait traversa la route et disparut devant nous. À peine avions-nous eu le temps de remarquer qu’il saignait et boitait. Il était poursuivi par trois soldats, et la chasse continua dans les terrains boisés à notre gauche. Nous entendions les soldats se héler mutuellement. Un quatrième apparut sur la droite au bord de la route. Il boitait. Il s’assit sur un rocher et épongea son visage en sueur.\par
— Des miliciens, murmura Dakon. Des déserteurs.\par
L’homme grimaça un sourire et nous demanda une allumette.\par
En réponse à Dakon qui l’interrogeait sur les événements, il nous informa que les miliciens désertaient en masse. – Rien à bouffer, expliqua-t-il. On donne toute la mangeaille aux soldats réguliers.\par
Il nous apprit aussi qu’on avait libéré les prisonniers militaires de l’île Alcatraz, parce qu’on ne pouvait plus les nourrir.\par
Jamais je n’oublierai le spectacle qui plus loin s’offrit à nos yeux de façon soudaine à un tournant de la route. Les branches des arbres se rejoignaient au-dessus de nos têtes et le soleil filtrait au travers. Des papillons voletaient et le ramage d’une alouette nous parvenait des champs. Et au milieu du chemin était arrêtée une puissante automobile d’excursion. Plusieurs cadavres gisaient dans la voiture et autour. L’histoire se racontait d’elle-même. Les voyageurs, fuyant la ville, avaient été attaqués et arrachés de leurs sièges par une bande de faubouriens, d’apaches. L’affaire datait de moins de vingt-quatre heures. Des boîtes de viande et de fruit récemment ouvertes expliquaient le mobile de l’attaque. Dakon examina les corps.\par
— Je m’en doutais, déclara-t-il. J’ai voyagé dans cette voiture. Ce sont les Perriton, toute la famille. Prenons bien garde à nous-mêmes désormais.\par
— Mais nous n’avons pas de provisions pouvant provoquer cette attaque, observais-je.\par
Dakon montra du doigt sa monture, et je compris.\par
Au début de la journée, le cheval de Dakon avait perdu un fer. Le sabot s’était fendu, et vers midi la pauvre bête boitait. Dakon refusa de la monter plus longtemps, et aussi de l’abandonner. Sur ses instances, nous poursuivîmes notre route. Il nous rejoindrait à ma maison de campagne en conduisant son cheval par la bride. Nous ne devions plus le revoir, et aucun de nous ne sut jamais comment il était mort.\par
Vers une heure, nous arrivâmes à Menlo, ou plutôt à l’emplacement de cette ville, car elle était en ruines. De tous côtés gisaient des cadavres. Le quartier des affaires et une partie de celui des villas avaient été dévastés par l’incendie. Quelques hôtels particuliers restaient debout, mais quand nous fîmes mine d’approcher, on nous tira dessus. Nous rencontrâmes une femme en train de fouiller dans les ruines fumantes de sa maison.\par
Millionnaires et pauvres bougres, après avoir combattu côte à côte pour s’emparer des victuailles, s’étaient battus entre eux pour le partage. La ville de Palo Alto et l’Université de Stanford avaient subi le même sort, nous dit-on. Devant nous se trouvait un territoire désolé, dévasté ; et nous poussâmes un soupir de soulagement en détournant nos chevaux vers la route menant à ma propriété. Elle se trouvait à trois ou quatre kilomètres vers l’ouest, dissimulée parmi les premiers contreforts de la montagne.\par
Nous devions constater en avançant que la dévastation ne s’était pas bornée aux grandes artères. L’avant-garde de la ruée avait suivi les routes et mis à sac les petites villes en passant. Mais les suivants s’étaient déversés en éventail pour balayer la campagne. Ma maison, construite en béton, maçonnerie et tuiles, avait échappé à l’incendie, mais non au pillage. Dans le moulin à vent nous trouvâmes le cadavre du jardinier, entouré d’une litière de cartouches vides. Il s’était vaillamment défendu. Mais les deux aides italiens avaient disparu, ainsi que la concierge et son mari. Il ne restait plus rien de vivant : veaux, poulains, volaille et bétail de race, tout s’était évanoui. La cuisine et les cheminées, dont on s’était servi, offraient un aspect lamentable, et le nombre des foyers en plein air dénonçait qu’une multitude avait soupé et passé la nuit à cet endroit. Les gens avaient emporté ce qu’ils ne pouvaient manger. Il ne restait pas un morceau à nous mettre sous la dent.\par
Au petit jour, après une nuit passée à attendre en vain Dakon, nous repoussâmes à coups de revolver une demi-douzaine de maraudeurs. Il fallut tuer un des chevaux prêtés par notre ami et cacher la viande que nous ne pouvions consommer immédiatement.\par
Dans l’après-midi, Collins alla faire une promenade et ne revint pas. Ce fut pour Hanover la goutte qui fait déborder le vase. Il voulait fuir immédiatement, et j’eus grand-peine à le persuader d’attendre au lendemain.\par
Pour ma part, convaincu que la fin de la grève générale approchait, je pris la résolution de regagner San-Francisco. Nous nous séparâmes donc au matin. Hanover continuant vers le sud avec cinquante livres de viande de cheval ficelées sur sa selle, et moi-même pareillement chargé, retournant vers le nord.\par
Le petit Hanover devait s’en tirer sain et sauf, mais je sais que jusqu’à la fin de ses jours, il ennuiera tout le monde du récit de ses aventures.\par
De nouveau, sur la grand’route, je parvins jusqu’à Belmont, où trois miliciens me volèrent ma provision de viande. D’après eux, la situation ne se modifierait guère, sauf de mal en pis. Les grévistes avaient caché d’abondantes provisions et pouvaient tenir pendant des mois encore. Je réussis à avancer jusqu’à Baden. Là, mon cheval fut enlevé par une douzaine d’hommes, dont deux agents de police de San-Francisco, les autres appartenant à l’armée régulière. Mauvais présage : la situation devait être désespérée dès lors que les soldats commençaient à déserter. Quand je repris ma route à pied, ils avaient déjà allumé le feu, et le dernier des chevaux de Dakon gisait abattu.\par
Pour comble de malchance, je me foulai une cheville au moment d’atteindre le quartier sud de San-Francisco. Je passai toute cette nuit-là dans un hangar, grelottant de froid et brûlant de fièvre. J’y restai deux jours, trop malade pour bouger et, le troisième, après m’être improvisé une espèce de béquille, chancelant, étourdi et affaibli par ce jeûne prolongé, je me traînai vers la ville.\par
En entrant dans la ville, je me souvins de la famille ouvrière où j’avais troqué ma cruche d’argent, et la faim m’attira dans cette direction. Le crépuscule tombait lorsque j’y arrivai. Je fis le tour par l’allée et grimpai les marches de derrière sur lesquelles je tombai en faiblesse. Je réussis cependant, en allongeant ma béquille, à frapper à la porte.\par
Puis je dus m’évanouir, car je repris mes sens dans la cuisine. On m’avait mouillé le visage, et quelqu’un me versait du whisky dans la gorge. Je toussai et balbutiai, essayant d’expliquer que je n’avais plus de cruches d’argent, mais qu’ils ne perdraient rien par la suite s’ils voulaient seulement me donner quelque chose à manger. La ménagère m’interrompit :\par
— Mais, mon pauvre homme, vous ne savez donc pas la nouvelle ? La fin de la grève a été déclarée cet après-midi. Naturellement, nous allons vous restaurer.\par
Elle s’affaira, ouvrit une boîte de lard et se prépara à le faire frire.\par
— Donnez-m’en un peu tel quel, s’il vous plaît, demandai-je, et je me mis à dévorer du lard cru sur une tranche de pain, tandis que le mari m’expliquait que les demandes du Syndicat avaient été accordées. Le télégraphe recommençait à fonctionner depuis le début de l’après-midi, et partout dans le pays les associations patronales avaient cédé. Il ne restait plus de patrons à San Francisco, mais le général Folsom avait parlé en leur nom. Les trains et les vapeurs reprendraient leur service le lendemain matin, et l’ordre se rétablirait à bref délai.\par
Telle fut la fin de la grève générale. Je ne souhaite pas en voir une autre. C’était pire qu’une guerre. La grève générale est chose cruelle et immorale, et le cerveau humain devrait être capable de faire marcher l’industrie de façon plus rationnelle.\par
J’ai toujours Harrison pour chauffeur. D’après les conditions de l’I. L. W., tous ses membres ont dû être réinstallés dans leurs anciens emplois. Brown n’a jamais reparu, mais tous mes autres serviteurs sont revenus chez moi. Je n’ai pas eu le cœur de les congédier. Les pauvres diables devaient franchir, eux aussi, une dure impasse, lorsqu’ils se sont sauvés avec mon argenterie.\par
Et maintenant je ne puis les renvoyer, car la I. L. W. les a tous enrôlés. La tyrannie du travail organisé dépasse les bornes de la patience humaine.\par
Il faut faire quelque chose.
 


% at least one empty page at end (for booklet couv)
\ifbooklet
  \pagestyle{empty}
  \clearpage
  % 2 empty pages maybe needed for 4e cover
  \ifnum\modulo{\value{page}}{4}=0 \hbox{}\newpage\hbox{}\newpage\fi
  \ifnum\modulo{\value{page}}{4}=1 \hbox{}\newpage\hbox{}\newpage\fi


  \hbox{}\newpage
  \ifodd\value{page}\hbox{}\newpage\fi
  {\centering\color{rubric}\bfseries\noindent\large
    Hurlus ? Qu’est-ce.\par
    \bigskip
  }
  \noindent Des bouquinistes électroniques, pour du texte libre à participation libre,
  téléchargeable gratuitement sur \href{https://hurlus.fr}{\dotuline{hurlus.fr}}.\par
  \bigskip
  \noindent Cette brochure a été produite par des éditeurs bénévoles.
  Elle n’est pas faîte pour être possédée, mais pour être lue, et puis donnée.
  Que circule le texte !
  En page de garde, on peut ajouter une date, un lieu, un nom ; pour suivre le voyage des idées.
  \par

  Ce texte a été choisi parce qu’une personne l’a aimé,
  ou haï, elle a en tous cas pensé qu’il partipait à la formation de notre présent ;
  sans le souci de plaire, vendre, ou militer pour une cause.
  \par

  L’édition électronique est soigneuse, tant sur la technique
  que sur l’établissement du texte ; mais sans aucune prétention scolaire, au contraire.
  Le but est de s’adresser à tous, sans distinction de science ou de diplôme.
  Au plus direct ! (possible)
  \par

  Cet exemplaire en papier a été tiré sur une imprimante personnelle
   ou une photocopieuse. Tout le monde peut le faire.
  Il suffit de
  télécharger un fichier sur \href{https://hurlus.fr}{\dotuline{hurlus.fr}},
  d’imprimer, et agrafer ; puis de lire et donner.\par

  \bigskip

  \noindent PS : Les hurlus furent aussi des rebelles protestants qui cassaient les statues dans les églises catholiques. En 1566 démarra la révolte des gueux dans le pays de Lille. L’insurrection enflamma la région jusqu’à Anvers où les gueux de mer bloquèrent les bateaux espagnols.
  Ce fut une rare guerre de libération dont naquit un pays toujours libre : les Pays-Bas.
  En plat pays francophone, par contre, restèrent des bandes de huguenots, les hurlus, progressivement réprimés par la très catholique Espagne.
  Cette mémoire d’une défaite est éteinte, rallumons-la. Sortons les livres du culte universitaire, débusquons les idoles de l’époque, pour les démonter.
\fi

\ifdev % autotext in dev mode
\fontname\font — \textsc{Les règles du jeu}\par
(\hyperref[utopie]{\underline{Lien}})\par
\noindent \initialiv{A}{lors là}\blindtext\par
\noindent \initialiv{À}{ la bonheur des dames}\blindtext\par
\noindent \initialiv{É}{tonnez-le}\blindtext\par
\noindent \initialiv{Q}{ualitativement}\blindtext\par
\noindent \initialiv{V}{aloriser}\blindtext\par
\Blindtext
\phantomsection
\label{utopie}
\Blinddocument
\fi
\end{document}
