%%%%%%%%%%%%%%%%%%%%%%%%%%%%%%%%%
% LaTeX model https://hurlus.fr %
%%%%%%%%%%%%%%%%%%%%%%%%%%%%%%%%%

% Needed before document class
\RequirePackage{pdftexcmds} % needed for tests expressions
\RequirePackage{fix-cm} % correct units

% Define mode
\def\mode{a4}

\newif\ifaiv % a4
\newif\ifav % a5
\newif\ifbooklet % booklet
\newif\ifcover % cover for booklet

\ifnum \strcmp{\mode}{cover}=0
  \covertrue
\else\ifnum \strcmp{\mode}{booklet}=0
  \booklettrue
\else\ifnum \strcmp{\mode}{a5}=0
  \avtrue
\else
  \aivtrue
\fi\fi\fi

\ifbooklet % do not enclose with {}
  \documentclass[french,twoside]{book} % ,notitlepage
  \usepackage[%
    papersize={105mm, 297mm},
    inner=12mm,
    outer=12mm,
    top=20mm,
    bottom=15mm,
    marginparsep=0pt,
  ]{geometry}
  \usepackage[fontsize=9.5pt]{scrextend} % for Roboto
\else\ifav
  \documentclass[french,twoside]{book} % ,notitlepage
  \usepackage[%
    a5paper,
    inner=25mm,
    outer=15mm,
    top=15mm,
    bottom=15mm,
    marginparsep=0pt,
  ]{geometry}
  \usepackage[fontsize=12pt]{scrextend}
\else% A4 2 cols
  \documentclass[twocolumn]{report}
  \usepackage[%
    a4paper,
    inner=15mm,
    outer=10mm,
    top=25mm,
    bottom=18mm,
    marginparsep=0pt,
  ]{geometry}
  \setlength{\columnsep}{20mm}
  \usepackage[fontsize=9.5pt]{scrextend}
\fi\fi

%%%%%%%%%%%%%%
% Alignments %
%%%%%%%%%%%%%%
% before teinte macros

\setlength{\arrayrulewidth}{0.2pt}
\setlength{\columnseprule}{\arrayrulewidth} % twocol
\setlength{\parskip}{0pt} % classical para with no margin
\setlength{\parindent}{1.5em}

%%%%%%%%%%
% Colors %
%%%%%%%%%%
% before Teinte macros

\usepackage[dvipsnames]{xcolor}
\definecolor{rubric}{HTML}{0c71c3} % the tonic
\def\columnseprulecolor{\color{rubric}}
\colorlet{borderline}{rubric!30!} % definecolor need exact code
\definecolor{shadecolor}{gray}{0.95}
\definecolor{bghi}{gray}{0.5}

%%%%%%%%%%%%%%%%%
% Teinte macros %
%%%%%%%%%%%%%%%%%
%%%%%%%%%%%%%%%%%%%%%%%%%%%%%%%%%%%%%%%%%%%%%%%%%%%
% <TEI> generic (LaTeX names generated by Teinte) %
%%%%%%%%%%%%%%%%%%%%%%%%%%%%%%%%%%%%%%%%%%%%%%%%%%%
% This template is inserted in a specific design
% It is XeLaTeX and otf fonts

\makeatletter % <@@@


\usepackage{blindtext} % generate text for testing
\usepackage{contour} % rounding words
\usepackage[nodayofweek]{datetime}
\usepackage{DejaVuSans} % font for symbols
\usepackage{enumitem} % <list>
\usepackage{etoolbox} % patch commands
\usepackage{fancyvrb}
\usepackage{fancyhdr}
\usepackage{fontspec} % XeLaTeX mandatory for fonts
\usepackage{footnote} % used to capture notes in minipage (ex: quote)
\usepackage{framed} % bordering correct with footnote hack
\usepackage{graphicx}
\usepackage{lettrine} % drop caps
\usepackage{lipsum} % generate text for testing
\usepackage[framemethod=tikz,]{mdframed} % maybe used for frame with footnotes inside
\usepackage{pdftexcmds} % needed for tests expressions
\usepackage{polyglossia} % non-break space french punct, bug Warning: "Failed to patch part"
\usepackage[%
  indentfirst=false,
  vskip=1em,
  noorphanfirst=true,
  noorphanafter=true,
  leftmargin=\parindent,
  rightmargin=0pt,
]{quoting}
\usepackage{ragged2e}
\usepackage{setspace}
\usepackage{tabularx} % <table>
\usepackage[explicit]{titlesec} % wear titles, !NO implicit
\usepackage{tikz} % ornaments
\usepackage{tocloft} % styling tocs
\usepackage[fit]{truncate} % used im runing titles
\usepackage{unicode-math}
\usepackage[normalem]{ulem} % breakable \uline, normalem is absolutely necessary to keep \emph
\usepackage{verse} % <l>
\usepackage{xcolor} % named colors
\usepackage{xparse} % @ifundefined
\XeTeXdefaultencoding "iso-8859-1" % bad encoding of xstring
\usepackage{xstring} % string tests
\XeTeXdefaultencoding "utf-8"
\PassOptionsToPackage{hyphens}{url} % before hyperref, which load url package
\usepackage{hyperref} % supposed to be the last one, :o) except for the ones to follow
\urlstyle{same} % after hyperref

% TOTEST
% \usepackage{hypcap} % links in caption ?
% \usepackage{marginnote}
% TESTED
% \usepackage{background} % doesn’t work with xetek
% \usepackage{bookmark} % prefers the hyperref hack \phantomsection
% \usepackage[color, leftbars]{changebar} % 2 cols doc, impossible to keep bar left
% \usepackage[utf8x]{inputenc} % inputenc package ignored with utf8 based engines
% \usepackage[sfdefault,medium]{inter} % no small caps
% \usepackage{firamath} % choose firasans instead, firamath unavailable in Ubuntu 21-04
% \usepackage{flushend} % bad for last notes, supposed flush end of columns
% \usepackage[stable]{footmisc} % BAD for complex notes https://texfaq.org/FAQ-ftnsect
% \usepackage{helvet} % not for XeLaTeX
% \usepackage{multicol} % not compatible with too much packages (longtable, framed, memoir…)
% \usepackage[default,oldstyle,scale=0.95]{opensans} % no small caps
% \usepackage{sectsty} % \chapterfont OBSOLETE
% \usepackage{soul} % \ul for underline, OBSOLETE with XeTeX
% \usepackage[breakable]{tcolorbox} % text styling gone, footnote hack not kept with breakable



% Metadata inserted by a program, from the TEI source, for title page and runing heads
\title{\textbf{ Michel Kohlhaas }}
\date{1810}
\author{Kleist, Heinrich}
\def\elbibl{Kleist, Heinrich. 1810. \emph{Michel Kohlhaas}}
\def\elabstract{%
 
\labelblock{Préface d’un Hurlu.}

 \noindent Voici un roman, c’est plutôt rare aux Hurlus, qui plus est traduit de l’Allemand. Je me suis permis cette entorse parce qu’il parle de Droits, de Justice, mais pas encore d’Égalité. L’histoire de Michel Kohlhaas a lieu en territoire Germanique. Elle se déroule à une époque de montée en puissance de la Bourgeoisie. En effet, à ce moment de bascule, celle-ci tient à se montrer moralement exemplaire pour asseoir sa légitimité. Il va sans dire, qu’à l’exception de quelques-uns, la vieille aristocratie est corrompue, vénale et de ce fait à punir et pas encore à abattre.\par
 Dès la dixième ligne : le décor et l’intrigue est posée\par
 
\begin{quoteblock}
 \noindent Il partit un jour de chez lui avec une troupe de chevaux, tous beaux, gras et bien nourris. En cheminant, il calculait le profit qu’il comptait retirer de son marché, et l’usage qu’il en ferait ; une barrière, placée au travers de la route, et qu’il n’avait encore jamais vue, vint le tirer de ses méditations. C’était en face d’un château seigneurial de la juridiction saxonne.\par
 Il fut obligé de s’arrêter, quoique la pluie tombât à torrent, et il appela le gardien, qui montra bientôt à la fenêtre un visage rébarbatif.\par
 Le marchand le pria de vouloir bien venir lui ouvrir.\par
 « Qu’y a-t-il de nouveau ici ? » demanda-t-il au gardien, qui sortit de la maison après un assez long délai.\par
 — Privilège seigneurial du gentilhomme Wenzel de Tronka, répondit le douanier en ouvrant la barrière.\par
 — Quoi ! dit Kohlhaas ; et il regardait tourner la clef dans la serrure toute neuve.
 \end{quoteblock}

 \noindent Ce livre est inspiré de faits réels — Hans Kohlhase — cependant, sa fin est pour le moins, caricaturale, mais là n’est pas l’intérêt qu’il peut susciter.\par
 Le but de cette publication est de provoquer la réflexion autour des notions de \emph{Justice} et des \emph{Droits} mais également d’\emph{Égalité}, finalement les piliers maîtres de la \emph{République}. Celles-ci sont trop souvent fantasmées et invoquées plutôt qu’étudiées.\par
 Ce texte est à mettre en perspective avec \emph{Les idées politiques} de Maurras et \emph{Vers la révision} (du procès de Dreyfus) de Clémenceau tous deux en passe d’être publiés aux Hurlus.\par
 

\signed{Xavier Damay}
 
}
\def\elsource{ \href{https://www.ebooksgratuits.com/details.php?book=2734}{\dotuline{https://www.ebooksgratuits.com/details.php?book=2734}}\footnote{\href{https://www.ebooksgratuits.com/details.php?book=2734}{\url{https://www.ebooksgratuits.com/details.php?book=2734}}} }

% Default metas
\newcommand{\colorprovide}[2]{\@ifundefinedcolor{#1}{\colorlet{#1}{#2}}{}}
\colorprovide{rubric}{red}
\colorprovide{silver}{Gray}
\@ifundefined{syms}{\newfontfamily\syms{DejaVu Sans}}{}
\newif\ifdev
\@ifundefined{elbibl}{% No meta defined, maybe dev mode
  \newcommand{\elbibl}{Titre court ?}
  \newcommand{\elbook}{Titre du livre source ?}
  \newcommand{\elabstract}{Résumé\par}
  \newcommand{\elurl}{http://oeuvres.github.io/elbook/2}
  \author{Éric Lœchien}
  \title{Un titre de test assez long pour vérifier le comportement d’une maquette}
  \date{1566}
  \devtrue
}{}
\let\eltitle\@title
\let\elauthor\@author
\let\eldate\@date


\defaultfontfeatures{
  % Mapping=tex-text, % no effect seen
  Scale=MatchLowercase,
  Ligatures={TeX,Common},
}

\@ifundefined{\columnseprulecolor}{%
    \patchcmd\@outputdblcol{% find
      \normalcolor\vrule
    }{% and replace by
      \columnseprulecolor\vrule
    }{% success
    }{% failure
      \@latex@warning{Patching \string\@outputdblcol\space failed}%
    }
}{}

\hypersetup{
  % pdftex, % no effect
  pdftitle={\elbibl},
  % pdfauthor={Your name here},
  % pdfsubject={Your subject here},
  % pdfkeywords={keyword1, keyword2},
  bookmarksnumbered=true,
  bookmarksopen=true,
  bookmarksopenlevel=1,
  pdfstartview=Fit,
  breaklinks=true, % avoid long links
  pdfpagemode=UseOutlines,    % pdf toc
  hyperfootnotes=true,
  colorlinks=false,
  pdfborder=0 0 0,
  % pdfpagelayout=TwoPageRight,
  % linktocpage=true, % NO, toc, link only on page no
}


% generic typo commands
\newcommand{\astermono}{\medskip\centerline{\color{rubric}\large\selectfont{\syms ✻}}\medskip\par}%
\newcommand{\astertri}{\medskip\par\centerline{\color{rubric}\large\selectfont{\syms ✻\,✻\,✻}}\medskip\par}%
\newcommand{\asterism}{\bigskip\par\noindent\parbox{\linewidth}{\centering\color{rubric}\large{\syms ✻}\\{\syms ✻}\hskip 0.75em{\syms ✻}}\bigskip\par}%

% lists
\newlength{\listmod}
\setlength{\listmod}{\parindent}
\setlist{
  itemindent=!,
  listparindent=\listmod,
  labelsep=0.2\listmod,
  parsep=0pt,
  % topsep=0.2em, % default topsep is best
}
\setlist[itemize]{
  label=—,
  leftmargin=0pt,
  labelindent=1.2em,
  labelwidth=0pt,
}
\setlist[enumerate]{
  label={\bf\color{rubric}\arabic*.},
  labelindent=0.8\listmod,
  leftmargin=\listmod,
  labelwidth=0pt,
}
\newlist{listalpha}{enumerate}{1}
\setlist[listalpha]{
  label={\bf\color{rubric}\alph*.},
  leftmargin=0pt,
  labelindent=0.8\listmod,
  labelwidth=0pt,
}
\newcommand{\listhead}[1]{\hspace{-1\listmod}\emph{#1}}

\renewcommand{\hrulefill}{%
  \leavevmode\leaders\hrule height 0.2pt\hfill\kern\z@}

% General typo
\DeclareTextFontCommand{\textlarge}{\large}
\DeclareTextFontCommand{\textsmall}{\small}


% commands, inlines
\newcommand{\anchor}[1]{\Hy@raisedlink{\hypertarget{#1}{}}} % link to top of an anchor (not baseline)
\newcommand\abbr[1]{#1}
\newcommand{\autour}[1]{\tikz[baseline=(X.base)]\node [draw=rubric,thin,rectangle,inner sep=1.5pt, rounded corners=3pt] (X) {\color{rubric}#1};}
\newcommand\corr[1]{#1}
\newcommand{\ed}[1]{ {\color{silver}\sffamily\footnotesize (#1)} } % <milestone ed="1688"/>
\newcommand\expan[1]{#1}
\newcommand\foreign[1]{\emph{#1}}
\newcommand\gap[1]{#1}
\renewcommand{\LettrineFontHook}{\color{rubric}}
\newcommand{\initial}[2]{\lettrine[lines=2, loversize=0.3, lhang=0.3]{#1}{#2}}
\newcommand{\initialiv}[2]{%
  \let\oldLFH\LettrineFontHook
  % \renewcommand{\LettrineFontHook}{\color{rubric}\ttfamily}
  \IfSubStr{QJ’}{#1}{
    \lettrine[lines=4, lhang=0.2, loversize=-0.1, lraise=0.2]{\smash{#1}}{#2}
  }{\IfSubStr{É}{#1}{
    \lettrine[lines=4, lhang=0.2, loversize=-0, lraise=0]{\smash{#1}}{#2}
  }{\IfSubStr{ÀÂ}{#1}{
    \lettrine[lines=4, lhang=0.2, loversize=-0, lraise=0, slope=0.6em]{\smash{#1}}{#2}
  }{\IfSubStr{A}{#1}{
    \lettrine[lines=4, lhang=0.2, loversize=0.2, slope=0.6em]{\smash{#1}}{#2}
  }{\IfSubStr{V}{#1}{
    \lettrine[lines=4, lhang=0.2, loversize=0.2, slope=-0.5em]{\smash{#1}}{#2}
  }{
    \lettrine[lines=4, lhang=0.2, loversize=0.2]{\smash{#1}}{#2}
  }}}}}
  \let\LettrineFontHook\oldLFH
}
\newcommand{\labelchar}[1]{\textbf{\color{rubric} #1}}
\newcommand{\milestone}[1]{\autour{\footnotesize\color{rubric} #1}} % <milestone n="4"/>
\newcommand\name[1]{#1}
\newcommand\orig[1]{#1}
\newcommand\orgName[1]{#1}
\newcommand\persName[1]{#1}
\newcommand\placeName[1]{#1}
\newcommand{\pn}[1]{\IfSubStr{-—–¶}{#1}% <p n="3"/>
  {\noindent{\bfseries\color{rubric}   ¶  }}
  {{\footnotesize\autour{ #1}  }}}
\newcommand\reg{}
% \newcommand\ref{} % already defined
\newcommand\sic[1]{#1}
\newcommand\surname[1]{\textsc{#1}}
\newcommand\term[1]{\textbf{#1}}

\def\mednobreak{\ifdim\lastskip<\medskipamount
  \removelastskip\nopagebreak\medskip\fi}
\def\bignobreak{\ifdim\lastskip<\bigskipamount
  \removelastskip\nopagebreak\bigskip\fi}

% commands, blocks
\newcommand{\byline}[1]{\bigskip{\RaggedLeft{#1}\par}\bigskip}
\newcommand{\bibl}[1]{{\RaggedLeft{#1}\par\bigskip}}
\newcommand{\biblitem}[1]{{\noindent\hangindent=\parindent   #1\par}}
\newcommand{\dateline}[1]{\medskip{\RaggedLeft{#1}\par}\bigskip}
\newcommand{\labelblock}[1]{\medbreak{\noindent\color{rubric}\bfseries #1}\par\mednobreak}
\newcommand{\salute}[1]{\bigbreak{#1}\par\medbreak}
\newcommand{\signed}[1]{\bigbreak\filbreak{\raggedleft #1\par}\medskip}

% environments for blocks (some may become commands)
\newenvironment{borderbox}{}{} % framing content
\newenvironment{citbibl}{\ifvmode\hfill\fi}{\ifvmode\par\fi }
\newenvironment{docAuthor}{\ifvmode\vskip4pt\fontsize{16pt}{18pt}\selectfont\fi\itshape}{\ifvmode\par\fi }
\newenvironment{docDate}{}{\ifvmode\par\fi }
\newenvironment{docImprint}{\vskip6pt}{\ifvmode\par\fi }
\newenvironment{docTitle}{\vskip6pt\bfseries\fontsize{18pt}{22pt}\selectfont}{\par }
\newenvironment{msHead}{\vskip6pt}{\par}
\newenvironment{msItem}{\vskip6pt}{\par}
\newenvironment{titlePart}{}{\par }


% environments for block containers
\newenvironment{argument}{\itshape\parindent0pt}{\vskip1.5em}
\newenvironment{biblfree}{}{\ifvmode\par\fi }
\newenvironment{bibitemlist}[1]{%
  \list{\@biblabel{\@arabic\c@enumiv}}%
  {%
    \settowidth\labelwidth{\@biblabel{#1}}%
    \leftmargin\labelwidth
    \advance\leftmargin\labelsep
    \@openbib@code
    \usecounter{enumiv}%
    \let\p@enumiv\@empty
    \renewcommand\theenumiv{\@arabic\c@enumiv}%
  }
  \sloppy
  \clubpenalty4000
  \@clubpenalty \clubpenalty
  \widowpenalty4000%
  \sfcode`\.\@m
}%
{\def\@noitemerr
  {\@latex@warning{Empty `bibitemlist' environment}}%
\endlist}
\newenvironment{quoteblock}% may be used for ornaments
  {\begin{quoting}}
  {\end{quoting}}

% table () is preceded and finished by custom command
\newcommand{\tableopen}[1]{%
  \ifnum\strcmp{#1}{wide}=0{%
    \begin{center}
  }
  \else\ifnum\strcmp{#1}{long}=0{%
    \begin{center}
  }
  \else{%
    \begin{center}
  }
  \fi\fi
}
\newcommand{\tableclose}[1]{%
  \ifnum\strcmp{#1}{wide}=0{%
    \end{center}
  }
  \else\ifnum\strcmp{#1}{long}=0{%
    \end{center}
  }
  \else{%
    \end{center}
  }
  \fi\fi
}


% text structure
\newcommand\chapteropen{} % before chapter title
\newcommand\chaptercont{} % after title, argument, epigraph…
\newcommand\chapterclose{} % maybe useful for multicol settings
\setcounter{secnumdepth}{-2} % no counters for hierarchy titles
\setcounter{tocdepth}{5} % deep toc
\markright{\@title} % ???
\markboth{\@title}{\@author} % ???
\renewcommand\tableofcontents{\@starttoc{toc}}
% toclof format
% \renewcommand{\@tocrmarg}{0.1em} % Useless command?
% \renewcommand{\@pnumwidth}{0.5em} % {1.75em}
\renewcommand{\@cftmaketoctitle}{}
\setlength{\cftbeforesecskip}{\z@ \@plus.2\p@}
\renewcommand{\cftchapfont}{}
\renewcommand{\cftchapdotsep}{\cftdotsep}
\renewcommand{\cftchapleader}{\normalfont\cftdotfill{\cftchapdotsep}}
\renewcommand{\cftchappagefont}{\bfseries}
\setlength{\cftbeforechapskip}{0em \@plus\p@}
% \renewcommand{\cftsecfont}{\small\relax}
\renewcommand{\cftsecpagefont}{\normalfont}
% \renewcommand{\cftsubsecfont}{\small\relax}
\renewcommand{\cftsecdotsep}{\cftdotsep}
\renewcommand{\cftsecpagefont}{\normalfont}
\renewcommand{\cftsecleader}{\normalfont\cftdotfill{\cftsecdotsep}}
\setlength{\cftsecindent}{1em}
\setlength{\cftsubsecindent}{2em}
\setlength{\cftsubsubsecindent}{3em}
\setlength{\cftchapnumwidth}{1em}
\setlength{\cftsecnumwidth}{1em}
\setlength{\cftsubsecnumwidth}{1em}
\setlength{\cftsubsubsecnumwidth}{1em}

% footnotes
\newif\ifheading
\newcommand*{\fnmarkscale}{\ifheading 0.70 \else 1 \fi}
\renewcommand\footnoterule{\vspace*{0.3cm}\hrule height \arrayrulewidth width 3cm \vspace*{0.3cm}}
\setlength\footnotesep{1.5\footnotesep} % footnote separator
\renewcommand\@makefntext[1]{\parindent 1.5em \noindent \hb@xt@1.8em{\hss{\normalfont\@thefnmark . }}#1} % no superscipt in foot


% orphans and widows
\clubpenalty=9996
\widowpenalty=9999
\brokenpenalty=4991
\predisplaypenalty=10000
\postdisplaypenalty=1549
\displaywidowpenalty=1602
\hyphenpenalty=400
% Copied from Rahtz but not understood
\def\@pnumwidth{1.55em}
\def\@tocrmarg {2.55em}
\def\@dotsep{4.5}
\emergencystretch 3em
\hbadness=4000
\pretolerance=750
\tolerance=2000
\vbadness=4000
\def\Gin@extensions{.pdf,.png,.jpg,.mps,.tif}
% \renewcommand{\@cite}[1]{#1} % biblio

\makeatother % /@@@>
%%%%%%%%%%%%%%
% </TEI> end %
%%%%%%%%%%%%%%


%%%%%%%%%%%%%
% footnotes %
%%%%%%%%%%%%%
\renewcommand{\thefootnote}{\bfseries\textcolor{rubric}{\arabic{footnote}}} % color for footnote marks

%%%%%%%%%
% Fonts %
%%%%%%%%%
\usepackage[]{roboto} % SmallCaps, Regular is a bit bold
% \linespread{0.90} % too compact, keep font natural
\newfontfamily\fontrun[]{Roboto Condensed Light} % condensed runing heads
\ifav
  \setmainfont[
    ItalicFont={Roboto Light Italic},
  ]{Roboto}
\else\ifbooklet
  \setmainfont[
    ItalicFont={Roboto Light Italic},
  ]{Roboto}
\else
\setmainfont[
  ItalicFont={Roboto Italic},
]{Roboto Light}
\fi\fi
\renewcommand{\LettrineFontHook}{\bfseries\color{rubric}}
% \renewenvironment{labelblock}{\begin{center}\bfseries\color{rubric}}{\end{center}}

%%%%%%%%
% MISC %
%%%%%%%%

\setdefaultlanguage[frenchpart=false]{french} % bug on part


\newenvironment{quotebar}{%
    \def\FrameCommand{{\color{rubric!10!}\vrule width 0.5em} \hspace{0.9em}}%
    \def\OuterFrameSep{\itemsep} % séparateur vertical
    \MakeFramed {\advance\hsize-\width \FrameRestore}
  }%
  {%
    \endMakeFramed
  }
\renewenvironment{quoteblock}% may be used for ornaments
  {%
    \savenotes
    \setstretch{0.9}
    \normalfont
    \begin{quotebar}
  }
  {%
    \end{quotebar}
    \spewnotes
  }


\renewcommand{\headrulewidth}{\arrayrulewidth}
\renewcommand{\headrule}{{\color{rubric}\hrule}}

% delicate tuning, image has produce line-height problems in title on 2 lines
\titleformat{name=\chapter} % command
  [display] % shape
  {\vspace{1.5em}\centering} % format
  {} % label
  {0pt} % separator between n
  {}
[{\color{rubric}\huge\textbf{#1}}\bigskip] % after code
% \titlespacing{command}{left spacing}{before spacing}{after spacing}[right]
\titlespacing*{\chapter}{0pt}{-2em}{0pt}[0pt]

\titleformat{name=\section}
  [block]{}{}{}{}
  [\vbox{\color{rubric}\large\raggedleft\textbf{#1}}]
\titlespacing{\section}{0pt}{0pt plus 4pt minus 2pt}{\baselineskip}

\titleformat{name=\subsection}
  [block]
  {}
  {} % \thesection
  {} % separator \arrayrulewidth
  {}
[\vbox{\large\textbf{#1}}]
% \titlespacing{\subsection}{0pt}{0pt plus 4pt minus 2pt}{\baselineskip}

\ifaiv
  \fancypagestyle{main}{%
    \fancyhf{}
    \setlength{\headheight}{1.5em}
    \fancyhead{} % reset head
    \fancyfoot{} % reset foot
    \fancyhead[L]{\truncate{0.45\headwidth}{\fontrun\elbibl}} % book ref
    \fancyhead[R]{\truncate{0.45\headwidth}{ \fontrun\nouppercase\leftmark}} % Chapter title
    \fancyhead[C]{\thepage}
  }
  \fancypagestyle{plain}{% apply to chapter
    \fancyhf{}% clear all header and footer fields
    \setlength{\headheight}{1.5em}
    \fancyhead[L]{\truncate{0.9\headwidth}{\fontrun\elbibl}}
    \fancyhead[R]{\thepage}
  }
\else
  \fancypagestyle{main}{%
    \fancyhf{}
    \setlength{\headheight}{1.5em}
    \fancyhead{} % reset head
    \fancyfoot{} % reset foot
    \fancyhead[RE]{\truncate{0.9\headwidth}{\fontrun\elbibl}} % book ref
    \fancyhead[LO]{\truncate{0.9\headwidth}{\fontrun\nouppercase\leftmark}} % Chapter title, \nouppercase needed
    \fancyhead[RO,LE]{\thepage}
  }
  \fancypagestyle{plain}{% apply to chapter
    \fancyhf{}% clear all header and footer fields
    \setlength{\headheight}{1.5em}
    \fancyhead[L]{\truncate{0.9\headwidth}{\fontrun\elbibl}}
    \fancyhead[R]{\thepage}
  }
\fi

\ifav % a5 only
  \titleclass{\section}{top}
\fi

\newcommand\chapo{{%
  \vspace*{-3em}
  \centering % no vskip ()
  {\Large\addfontfeature{LetterSpace=25}\bfseries{\elauthor}}\par
  \smallskip
  {\large\eldate}\par
  \bigskip
  {\Large\selectfont{\eltitle}}\par
  \bigskip
  {\color{rubric}\hline\par}
  \bigskip
  {\Large LIVRE LIBRE À PRIX LIBRE, DEMANDEZ AU COMPTOIR\par}
  \centerline{\small\color{rubric} {hurlus.fr, tiré le \today}}\par
  \bigskip
}}


\begin{document}
\pagestyle{empty}
\ifbooklet{
  \thispagestyle{empty}
  \centering
  {\LARGE\bfseries{\elauthor}}\par
  \bigskip
  {\Large\eldate}\par
  \bigskip
  \bigskip
  {\LARGE\selectfont{\eltitle}}\par
  \vfill\null
  {\color{rubric}\setlength{\arrayrulewidth}{2pt}\hline\par}
  \vfill\null
  {\Large LIVRE LIBRE À PRIX LIBRE, DEMANDEZ AU COMPTOIR\par}
  \centerline{\small{hurlus.fr, tiré le \today}}\par
  \newpage\null\thispagestyle{empty}\newpage
  \addtocounter{page}{-2}
}\fi

\thispagestyle{empty}
\ifaiv
  \twocolumn[\chapo]
\else
  \chapo
\fi
{\it\elabstract}
\bigskip
\makeatletter\@starttoc{toc}\makeatother % toc without new page
\bigskip

\pagestyle{main} % after style

  
\chapteropen
\chapter[Chapitre premier]{Chapitre premier}\renewcommand{\leftmark}{Chapitre premier}


\chaptercont
\noindent Sur les bords du Hasel vivait, au milieu du XVI\textsuperscript{e} siècle, un marchand de chevaux, nommé Michel Kohlhaas. Il était fils d’un maître d’école, et son nom rappelle encore aujourd’hui l’un des hommes les plus justes, et en même temps l’un des plus criminels de son siècle.\par
Cet homme extraordinaire passa jusqu’à sa trentième année pour modèle du bon bourgeois. Il possédait, dans un petit village qui porte son nom, une ferme où il vivait paisiblement du gain de son commerce, élevant dans la crainte de Dieu et dans l’amour du travail et de la vertu les enfants que sa femme lui donnait chaque année. Il n’était pas un de ses voisins qui n’eût à se louer de sa bienfaisance ou de sa probité, et le monde eût dû bénir son nom, s’il n’avait poussé jusqu’à l’excès une de ses belles vertus. Le sentiment profond de la justice en fit un brigand et un meurtrier.\par
Il partit un jour de chez lui avec une troupe de chevaux, tous beaux, gras et bien nourris. En cheminant, il calculait le profit qu’il comptait retirer de son marché, et l’usage qu’il en ferait ; une barrière, placée au travers de la route, et qu’il n’avait encore jamais vue, vint le tirer de ses méditations. C’était en face d’un château seigneurial de la juridiction saxonne.\par
Il fut obligé de s’arrêter, quoique la pluie tombât à torrent, et il appela le gardien, qui montra bientôt à la fenêtre un visage rébarbatif.\par
Le marchand le pria de vouloir bien venir lui ouvrir.\par
« Qu’y a-t-il de nouveau ici ? » demanda-t-il au gardien, qui sortit de la maison après un assez long délai.\par
— Privilège seigneurial du gentilhomme Wenzel de Tronka, répondit le douanier en ouvrant la barrière.\par
— Quoi ! dit Kohlhaas ; et il regardait tourner la clef dans la serrure toute neuve.\par
« Le vieux seigneur est-il mort ?\par
— Oui, il est mort d’apoplexie, répondit le douanier en soulevant la barrière.\par
— Hé ! tant-pis, reprit Kohlhaas ; c’était un bien digne homme ; il s’intéressait au commerce, et il aidait volontiers les marchands qui pouvaient avoir besoin de ses secours ; c’est lui qui fit bâtir la chaussée qui mène au village, parce qu’une de mes juments s’y était cassé la jambe. »\par
Eh bien ! que dois-je payer ? »\par
Puis il tira avec peine de dessous son manteau agité par le vent la pièce de monnaie que réclamait le douanier.\par
« Voilà, mon vieux » ; et, jurant contre la rigueur de la saison, il ajouta :\par
« Il eût mieux valu pour vous et pour moi que l’arbre qui a servi à faire cette barrière fût resté dans la forêt. » En parlant ainsi, il se remit en marche ; mais à peine était-il sous la barrière, qu’une voix lui cria de la tour :\par
« Halte là, maquignon ! » et il vit le châtelain ouvrir une fenêtre et lui faire signe de s’arrêter.\par
« Qu’y a-t-il donc encore ? » se demanda-t-il à lui-même en arrêtant ses chevaux.\par
Le châtelain accourut, achevant de boutonner sa veste sur son large ventre, et, tout en jurant contre le froid et la pluie, il demanda à Kohlhaas son passeport.\par
« Mon passeport ! dit celui-ci, je n’en ai point. » Alors le châtelain, le regardant de travers, lui apprit qu’aucun marchand ne pouvait passer des chevaux sur la frontière sans une autorisation légale. Kohlhaas protesta qu’il avait passé dix-sept fois la frontière sans rien de semblable ; qu’il connaissait parfaitement les réglements du pays sur son commerce ; et que sans doute il y avait là-dedans une erreur à laquelle il le priait de réfléchir, sans l’arrêter plus longtemps, sa course du jour devant être encore très longue. Mais le châtain déclara qu’il ne passerait point ainsi pour la dix-huitième fois, parce que les réglements avaient changé, et qu’il devait livrer son passeport, ou retourner le chercher. Le maquignon, que cette vexation commençait à aigrir, descendit de cheval ; après avoir réfléchi un instant, il dit qu’il voulait parler au seigneur de Tronka ; puis il entra au château, suivi du châtelain, qui murmurait entre ses dents et le mesurait d’un air de mépris.\par
Il se trouva que le jeune seigneur était à boire avec quelques joyeux amis, et qu’un rire éclatant retentissait au milieu d’eux, lorsque Kohlhaas s’approcha pour exposer son affaire.\par
Les chevaliers se turent à l’arrivée de l’étranger ; mais à peine celui-ci eut-il décliné sa profession, que toute la bande s’écria : « Des chevaux ! des chevaux ! Où sont-ils ? » et chacun courut aux fenêtres ; puis, avec le consentement du seigneur, ils descendirent tous à la cour, où le domestique de Kohlhaas était entré avec les chevaux.\par
La pluie avait cessé ; le châtelain, l’intendant et les valets du château étaient déjà rassemblés autour de ces magnifiques animaux, et contemplaient avec admiration la crinière fournie de l’un, la queue flottante de l’autre, la douceur et la beauté de tous. L’on s’accorda à déclarer qu’il ne s’en trouvait pas de comparables dans tout le pays.\par
Kohlhaas répondit gaîment que le mérite des chevaux était loin d’égaler celui des cavaliers qui devaient les monter ; et il offrit à ces seigneurs de les leur vendre.\par
Le gentilhomme, enchanté d’un magnifique coursier bai, en demanda le prix, ainsi que celui de deux chevaux noirs, dont l’intendant assurait avoir un grand besoin pour les travaux de la maison. Mais lorsque Kohlhaas déclara quelle somme il comptait en retirer, tous les chevaliers se récrièrent, et le gentilhomme lui dit qu’il pouvait aller chercher la Table ronde et visiter le roi Arthur, s’il voulait vendre ses chevaux à ce prix.\par
Kohlhaas, qui avait surpris des regards d’intelligence entre le châtelain et l’intendant, et qui se sentait le cœur oppressé d’un triste pressentiment, fit tous ses efforts pour conclure le marché.\par
« Monseigneur, dit-il, j’ai payé, il y a six mois, vingt-cinq écus d’or de ces chevaux ; si vous les voulez à trente, je vous les cède. »\par
Deux cavaliers qui étaient près du gentilhomme l’assurèrent que les chevaux valaient bien cela ; mais comme il n’avait nulle envie de débourser tant d’argent, il éluda le marché, et Kohlhaas, ayant dit qu’il espérait avoir plus de succès à son prochain voyage, salua les chevaliers et prit les rênes de ses chevaux pour s’éloigner. Mais le châtelain, sortant de la foule et arrêtant le maquignon, lui dit avec rudesse qu’il savait bien qu’il ne pouvait passer sans passeport.\par
Kohlhaas se tourna vers le gentilhomme, et lui demanda s’il était vrai qu’il voulût par un acte si arbitraire mettre un obstacle à son commerce.\par
« Oui, Kohlhaas, répondit celui-ci d’un air incertain, tu dois livrer ton passeport ; parle au châtelain, puis continue ta route. »\par
Kohlhaas expliqua alors qu’il n’avait point voulu se mettre en contravention avec le nouveau règlement qu’il ne connaissait pas, et il pria le seigneur de Tronka de vouloir bien le laisser passer en faveur de son ignorance, lui promettant de demander un passeport à la chancellerie de Dresde, et de le livrer à son retour.\par
« Eh bien, dit le gentilhomme, pénétré du froid piquant de l’orage qui recommençait à gronder, qu’on laisse passer ce drôle. Venez » dit-il aux chevaliers ; et il fit un pas pour rentrer au château.\par
Mais le châtelain l’arrêtant, lui fit observer que cet homme devrait au moins laisser un gage, une sûreté jusqu’à la délivrance de son passeport, et l’intendant murmura dans sa barbe qu’il fallait garder comme otages les deux chevaux noirs.\par
« Assurément, dit le châtelain, c’est le plus simple moyen, et une fois qu’il aura livré son passeport, il pourra les reprendre. »\par
Kohlhaas chercha à en rappeler d’une décision si rigoureuse ; il dit au gentilhomme, dont tous les membres débiles tremblaient de froid, qu’il le frustrait ainsi de la vente de deux chevaux. Mais un violent coup de vent ayant jeté une bouffée de pluie et de grêle contre la porte du château, le gentilhomme, pour en finir, dit au marchand que s’il ne voulait laisser ses chevaux il ne passerait point la barrière et il rentra.\par
Michel Kohlhaas, voyant bien qu’il n’y avait pas d’autre parti à prendre, se décida à céder à la force. Dételant les deux beaux coursiers noirs, il les conduisit dans une écurie que lui indiqua le châtelain, puis remettant de l’argent à son domestique, il lui ordonna de rester pour garder les chevaux, et d’en avoir le plus grand soin jusqu’à son retour.\par
Il continua son chemin avec le reste de sa troupe vers Leipzig, où il voulait arriver pour la messe, de plus en plus incrédule à l’égard du nouveau réglement sur l’entrée des chevaux en Saxe.\par
Arrivé à Dresde, où il possédait une maison et des écuries, parce que c’était ordinairement de là qu’il se rendait dans les grands marchés, il courut à la chancellerie, et il apprit des conseillers, qu’il connaissait presque tous, ce que son propre jugement lui avait fait deviner, que toute cette histoire n’était qu’un tissu de faussetés. Sur sa demande, ils lui donnèrent un acte qui prouvait la nullité du prétendu règlement.\par
Le bon marchand riait en lui-même de la plaisanterie du petit gentilhomme dont il ne pouvait comprendre le but. Au bout de deux semaines, ayant vendu à sa satisfaction tous ses chevaux, il reprit la route de Tronkenbourg, sans autre sentiment d’amertume que celui qu’inspirent à tout homme les misères de la vie.\par
Le châtelain, auquel il remit l’attestation, ne fit aucune remarque ; il répondit seulement à la réclamation que Kohlhaas faisait de ses chevaux, qu’il pouvait entrer pour les prendre.\par
À peine dans la cour, le pauvre Kohlhaas eut le chagrin d’apprendre que son domestique avait été chassé de Tronkenbourg pour ses impertinences ; mais le jeune homme qui lui donnait cette nouvelle ne sut point lui dire ce qui avait causé cet événement, ni par qui les chevaux avaient été soignés depuis. Ouvrant une écurie, il y fit entrer Kohlhaas, dont le cœur était plein d’une vague inquiétude.\par
Quelle fut la surprise du marchand, lorsqu’au lieu de ses deux coursiers, gras, beaux et fringants, il ne vit qu’une couple de haridelles maigres, exténuées, dont les os pouvaient se compter, et dont les crinières embrouillées et malpropres tombaient en désordre ! Vrai tableau de la plus affreuse misère ! Le cœur du sensible Kohlhaas fut pénétré de douleur à cette vue, et il se brisa lorsqu’il entendit ces pauvres animaux hennir faiblement à son approche.\par
« Qu’est-il donc arrivé à ces malheureuses bêtes ? » demanda-t-il au jeune homme qui était resté près de lui.\par
Celui-ci l’assura qu’il ne leur était advenu aucun mal, qu’ils avaient été bien nourris et bien soignés, mais que, vu la grande abondance de la récolte et le manque de bêtes de somme, on les avait fait un peu travailler à la rentrée de la moisson.\par
Kohlhaas jura contre cet acte inouï de barbarie ; cependant, réprimant la vivacité de sa colère, il fit mine de vouloir quitter aussitôt ce repaire de brigands, lorsque le châtelain, attiré par cette conversation, s’approcha, et demanda de quoi il s’agissait.\par
« De quoi il s’agit ! repartit Kohlhaas vivement ; qui est-ce qui a permis au gentilhomme de Tronka et à ses gens de se servir de mes chevaux pour les travaux de la terre ? Y a-t-il de la justice à les avoir réduits en cet état, ajouta-t-il, en donnant un coup de fouet aux bêtes, qui furent trop faibles pour se lever.\par
— Voyez donc ce manant, répondit le châtelain en le regardant avec hauteur : comme s’il ne devrait pas plutôt remercier le ciel de ce que ses rosses vivent encore, de ce que l’on a bien voulu en prendre soin depuis que son domestique est parti, et leur fournir une partie de la paille qu’elles ont aidés à recueillir. » Puis il jura que s’il répliquait un seul mot, il appellerait les chiens qui le forceraient bien à le laisser en repos.\par
Le maquignon fit violence à son cœur, qui lui criait de rouler dans la boue ce gros ventre, et de donner du pied dans ce visage de cuivre ; son sentiment de la justice, qui ressemblait à un trébuchet, l’emporta sur sa colère.\par
Il n’était pas encore bien certain au fond du cœur que son adversaire fût dans son tort ; écoutant sans mot dire ses paroles offensantes, il rentra dans l’écurie, et considérant tristement ces pauvres bêtes, il demanda d’une voix basse pourquoi son domestique avait été renvoyé.\par
« Parce qu’il a été un impertinent, et qu’il a voulu s’opposer à un changement d’écurie devenu nécessaire par l’arrivée de deux cavaliers à Tronkenbourg. »\par
Kohlhaas aurait donné la valeur de ses chevaux pour avoir là son domestique, et pouvoir opposer son récit à celui de l’énorme châtelain.\par
Il réfléchissait à ce qu’il y avait à faire dans sa triste situation, lorsque la scène changea tout-à-coup. Le gentilhomme de Tronka, revenant de la chasse s’élança, dans la cour avec une suite nombreuse de cavaliers, de valets et de chiens. Il demanda qui était cet homme et ce qu’il voulait ; et le châtelain, prenant la parole au milieu des aboiements répétés de la meute contre l’étranger, raconta de la manière la plus méprisante que c’était Michel Kohlhaas le maquignon qui ne voulait pas reconnaître ses bêtes, et se mettait en rébellion parce qu’elles avaient un peu servi.\par
« Non, s’écria Kohlhaas, ce ne sont point là les chevaux qui valaient trente écus d’or ; je veux avoir mes chevaux gras et bien portants, tels que je les ai laissés ! »\par
Le gentilhomme, dont le visage s’était couvert d’une pâleur momentanée, descendit de cheval.\par
« Si le chien ne veut pas reprendre ses bêtes, dit-il froidement, qu’il les laisse. Venez, Gunther, ajouta-t-il, venez, Hans ; qu’on nous apporte du vin ! » Et il entra au château avec les chevaliers ses amis.\par
Michel Kohlhaas dit qu’il préférait appeler l’écorcheur ou laisser mourir de faim ces pauvres bêtes, plutôt que de les emmener à Kohlhaasenbruck ; et remontant sur son coursier, il partit en déclarant qu’il saurait se faire rendre justice.\par
Il reprenait à toute bride la route de Dresde, lorsque, réfléchissant à la plainte que l’on portait au château contre son domestique, il changea de direction et se rendit à sa ferme de Kohlhaasenbruck, pour y entendre, comme cela lui semblait juste et raisonnable, la déposition de cet homme.\par
Un sentiment déjà connu pour l’ordre et la justice dans toutes les choses de ce monde, faisait qu’il aurait regardé la perte de ses chevaux et toutes les offenses qu’il venait de recevoir, comme la suite naturelle de la faute que le châtelain reprochait à son domestique ; d’un autre côté, un sentiment aussi fort, et qui jetait de nouvelles racines à mesure qu’il cheminait, et qu’il entendait, partout où il s’arrêtait, raconter des actes de violence exercés contre tous les voyageurs à Tronkenbourg, lui faisait envisager comme un devoir, si tout cet événement n’était, ainsi qu’il le paraissait, qu’une escroquerie concertée d’avance, de demander satisfaction de cette injure, non seulement pour son propre repos, mais pour la sûreté future de tous ses concitoyens.\par
Arrivé à Kohlhaasenbruck, dès qu’il eut embrassé Lisbeth, sa femme chérie, et ses enfants qui sautaient autour de lui, il s’informa de Herse, le maître valet.\par
« Il est ici, répondit Lisbeth ; ce pauvre infortuné est revenu, il y a environ quinze jours, dans l’état le plus pitoyable et pouvant à peine se soutenir. Nous le fîmes mettre au lit, où il cracha beaucoup de sang ; il répondit à nos nombreuses questions par une histoire que personne ne pouvait comprendre. Il prétendait avoir été laissé par toi à Tronkenbourg, d’où il avait été forcé, par des traitements inouïs, de fuir sans pouvoir prendre avec lui les chevaux confiés à ses soins.\par
— Hem ! dit Kohlhaas, en posant son manteau, est-il guéri maintenant ?\par
— Oui, Michel, il est guéri du crachement de sang. Je voulus envoyer aussitôt un autre valet à Tronkenbourg pour le remplacer auprès des chevaux, car ce pauvre Herse s’est toujours montré si vrai et si fidèle que je n’ai pas douté un seul instant de la sincérité de son récit ; mais il me conjura de n’envoyer personne dans ce nid de brigands, et d’abandonner les bêtes à leur destin plutôt que de leur sacrifier un homme.\par
— Garde-t-il encore le lit ? demanda Kohlhaas en se débarrassant de sa cravate.\par
— Non, il peut se promener dans le jardin depuis quelques jours. Tu verras, mon cher Michel, qu’il est pleinement dans son droit, et qu’il a été victime d’une des plus horribles violences que l’on se soit encore permises à Tronkenbourg contre les étrangers.\par
— C’est ce que je veux examiner ; appelle-le, Lisbeth. »\par
En parlant ainsi, Kohlhaas s’assit gravement dans un fauteuil, et la bonne femme, toute joyeuse de le voir si modéré, courût chercher le domestique.\par
« Qu’as-tu fait à Tronkenbourg ? demanda Michel à celui-ci au moment où il entrait suivi de Lisbeth ; je ne suis point content de toi. »\par
Le domestique, dont le visage pâle se couvrit d’une vive rougeur, se tut quelques instants, puis il dit :\par
« Vous avez raison, mon maître, car, touché par les cris d’un enfant, j’ai jeté dans l’Elbe la mèche soufrée que j’avais prise, par une inspiration du ciel, pour mettre le feu à cette caverne de voleurs dont j’étais chassé.\par
— Mais pourquoi as-tu été chassé de Tronkenbourg ?\par
— Par la plus horrible violence, mon maître ; » et il essuya la sueur qui coulait de son front : « parce que je ne voulais pas consentir à ce que l’on fît travailler vos chevaux, et que je dis qu’ils étaient trop jeunes et n’avaient point été accoutumés à cela… »\par
Ici Kohlhaas l’interrompit et lui fit observer, en cherchant à cacher son trouble qu’il n’avait pas dit toute la vérité, puisqu’il savait bien que les chevaux avaient été attelés quelquefois au commencement du dernier printemps.\par
« Tu aurais dû, ajouta-t-il te montrer plus complaisant au château dont tu étais l’hôte en quelque sorte, et consentir à aider à la rentrée de la moisson.\par
— Et c’est aussi ce que j’ai fait, mon maître. Je pensais qu’après tout cela ne tuerait pas les chevaux, et le troisième jour ils rentrèrent trois chars de blé.\par
— Ils ne m’ont pas parlé de cela, Herse » s’écria Michel, dont le cœur se gonflait d’indignation et il baissa les yeux vers la terre.\par
Herse l’assura que les choses s’étaient bien passées ainsi. « Mon manque de complaisance, ajouta-t-il, consiste à n’avoir pas voulu suivre le conseil du châtelain et de l’intendant, qui me disaient de nourrir les chevaux avec le maigre foin de la commune, et de garder pour moi l’argent que vous m’aviez remis ; ce à quoi je répondis en leur tournant le dos.\par
— Mais tu n’as donc pas été chassé ?\par
— Plût à Dieu ! s’écria Herse, ce serait un crime de moins contre le ciel. Sur le soir du même jour, les chevaux de deux jeunes cavaliers qui venaient d’arriver à Tronkenbourg furent amenés dans l’écurie ; on en fit sortir les miens, et comme je demandais au châtelain où je devais les loger, il m’indiqua une étable à cochons, formée de quelques planches soutenues par des pieux, et adossées au mur du château.\par
— Peut-être n’en avait-elle que l’apparence, Herse, et n’était-ce point une étable à cochons.\par
— Je vous demande pardon, mon maître, c’en était une véritable, et les pourceaux y étaient encore au milieu de l’ordure la plus fétide.\par
— Mais sans doute il n’y avait pas d’autre place pour abriter les chevaux, et ceux des cavaliers avaient en effet quelque droit à être les mieux servis.\par
— La place était rare, il est vrai, reprit le domestique d’une voix éteinte ; il y avait alors au château sept cavaliers avec leurs chevaux. Cependant si vous eussiez été là, vous les auriez bien tous fait entrer dans l’écurie. Je dis que je voulais aller chercher une écurie dans le village, mais le châtelain prétendit que les chevaux ne devaient pas sortir du château.\par
« Hem, que répondis-tu à cela ?\par
— Comme l’intendant m’assura que les chevaliers n’étaient venus que pour la nuit, ce qui était faux, car j’appris le lendemain qu’ils devaient rester plusieurs semaines, je m’établis dans l’étable.\par
— Et ! tu ne la trouvas pas si mauvaise que tu l’avais d’abord supposé ?\par
— Non, parce que j’eus soin de la nettoyer et de donner quelqu’argent à la fille de basse-cour pour l’engager à mettre ailleurs ses cochons. Pour que les chevaux pussent se tenir debout pendant le jour, j’ôtais les planches qui leur servaient de couvert la nuit ; c’était une pitié que de voir ces pauvres bêtes allonger le col au-dessus des pieux, et ouvrir les naseaux avec inquiétude, comme si elles soupiraient après leur écurie de Kohlhaasenbruck.\par
— Mais alors, Herse, pourquoi as-tu été chassé ?\par
— Parce qu’il était impossible de compléter la ruine des chevaux tant que je restais là. Un jour que je les menais boire, le châtelain, l’intendant, les valets, se précipitèrent comme des possédés à ma poursuite, et lorsque je demandai à cette troupe furieuse ce qu’elle me voulait, le châtelain saisit la bride des chevaux, et me demanda où j’allais les conduire ?\par
« À l’abreuvoir, répliquai-je.\par
» À l’abreuvoir ! coquin ; je veux t’apprendre à t’aller abreuver sur la route de Kohlhaasenbruck, » et me tirant par la jambe, il me fit tomber de cheval tout étendu dans la boue. « Mort et tonnerre ! m’écriai-je, comment pouvez-vous me soupçonner ? N’ai-je pas laissé dans l’écurie les selles des chevaux et toutes mes hardes ». Tandis que le châtelain faisait rentrer mes chevaux, les domestiques se mirent à me battre à coup de fouets et de bâtons, jusqu’à ce que je tombasse presque mort devant la porte.\par
« Chiens de voleurs, que voulez-vous faire de mes chevaux ? » m’écriai-je en me relevant. Mais, pour toute réponse, le châtelain détachant les chiens de chasse, les excita contre moi ; j’arrachai une branche d’arbre pour me défendre, et j’en étendis trois morts à mes côtés ; alors un coup de sifflet rappela les autres dans la cour, la porte se ferma, et je tombai privé de sentiment sur la grande route.\par
— N’avais-tu point l’intention de t’échapper, Herse ? » dit Kohlhaas, pâle, tremblant, en lui lançant un regard scrutateur ; et comme le domestique, au lieu de répondre, regardait à terre, tandis que son visage se couvrait d’une ardente rougeur :\par
« Avoue-le-moi, ajouta son maître, tu n’aimais pas à être dans cette étable à cochons, et tu pensais que tu serais mieux dans l’écurie de Kohlhaasenbruck ?\par
— Ciel et tonnerre ! s’écria Herse, n’avais-je pas laissé dans l’étable mon linge et les harnais des chevaux ? Si j’avais eu l’intention de fuir, n’aurais-je pas pris sur moi trois écus d’or qui sont restés dans un mouchoir derrière la crèche ! Enfer et diable ! si vous me parlez ainsi, je saurai retrouver une mèche soufrée.\par
— Paix, paix, dit le marchand, je n’ai pas voulu t’offenser ; je crois mot pour mot tout ce que tu viens de me dire, et je jurerais de la vérité de ton récit s’il le fallait. Je regrette que tu aies tant souffert pour mon service. Va te mettre au lit, pauvre Herse, et fais-toi donner une bouteille de vin pour te consoler. Je te ferai rendre justice. »\par
Kohlhaas écrivit la note de ce que le domestique avait laissé dans l’étable, et le renvoya après lui avoir serré affectueusement la main.\par
Il raconta ensuite à Lisbeth tous les détails de son aventure, et lui déclara qu’il était décidé à réclamer la protection de la justice. Il eut le plaisir de voir qu’elle l’y encourageait de tout son cœur, et qu’elle était prête à supporter toutes les dépenses d’un procès ; car, disait-elle, c’est une œuvre de miséricorde que de mettre un terme aux violences qui se commettent à Tronkenbourg. »\par
Michel l’appela sa courageuse femme, et passa ce jour et le suivant à se réjouir avec elle et ses enfants, puis il partit pour porter sa plainte devant les juges de Dresde.
\chapterclose


\chapteropen
\chapter[Chapitre II]{Chapitre II}\renewcommand{\leftmark}{Chapitre II}


\chaptercont
\noindent Arrivé à la capitale, Kohlhaas composa, avec l’aide d’un homme de loi de sa connaissance, une plainte dans laquelle il fit le récit détaillé de la violence exercée par le gentilhomme de Tronka contre lui et son domestique Herse, et des dommages soufferts par tous deux. La circonstance que les chevaux avaient été retenus injustement au château, indépendamment de toutes les autres, semblait devoir assurer au marchand le prompt redressement du tort qui lui avait été fait. Pendant son séjour à Dresde, il ne manqua point d’amis qui lui promirent de prendre chaudement ses intérêts. Son commerce étendu et sa parfaite probité lui avaient gagné la bienveillance des hommes les plus distingués du pays.\par
Il mangea plusieurs fois chez son avocat, et après lui avoir remis une somme d’argent destinée aux frais de la procédure, il revint, entièrement tranquille sur le succès de son affaire, auprès de sa femme, à Kohlhaasenbruck.\par
Cependant des mois s’écoulèrent, et la fin de l’année arriva, sans qu’il reçût aucune nouvelle de sa plainte, pendante devant les tribunaux. Après avoir fait plusieurs démarches inutiles auprès de son avocat, celui-ci lui écrivit que sa plainte avait été annulée par de \emph{puissantes insinuations}, le gentilhomme de Tronka étant allié aux seigneurs Hinz et Kunz de Tronka, dont l’un était chambellan, l’autre grand échanson de l’électeur de Saxe.\par
Il lui conseillait de faire chercher ses chevaux à Tronkenbourg, et de renoncer à toutes poursuites juridiques, lui donnant à entendre que le gentilhomme, qui se trouvait en ce moment à la résidence, paraissait avoir ordonné à ses gens de les lui livrer ; et il terminait en le priant, dans le cas où il ne se contenterait pas ainsi, de vouloir bien lui épargner toute nouvelle intervention dans cette affaire.\par
Kohlhaas était depuis quelques jours à Brandenbourg. Le commandant de la ville, Henri de Geusau, dans l’arrondissement duquel se trouvait Kohlhaasenbruck, s’occupait à cette époque de plusieurs établissements de charité, et entre autres il cherchait à mettre à profit, pour le soulagement des incurables, une source minérale que l’on venait de découvrir dans un village voisin. Michel Kohlhaas, qui le connaissait pour lui avoir quelquefois vendu des chevaux, obtint de lui la permission d’essayer l’efficacité des bains sur le pauvre Herse, qui, depuis ses aventures à Tronkenbourg, était resté affligé d’un grand mal de poitrine.\par
Le commandant était auprès de la baignoire où Michel avait fait placer Herse, lorsque le marchand reçut la lettre de l’avocat, que sa femme lui envoyait ; il remarqua, tout en causant avec le médecin, que Kohlhaas laissait tomber une larme sur le papier qu’il venait de lire, et s’approchant de lui avec bienveillance, il lui demanda la cause de son chagrin.\par
Le marchand, pour toute réponse, lui tendit la lettre ; lorsque le commandant eut appris l’horrible injustice exercée à Tronkenbourg contre le pauvre Herse, qui devait en rester malade toute sa vie, il frappa sur l’épaule de Kohlhaas, et lui dit qu’il ne fallait point se décourager et qu’il l’aiderait de tout son pouvoir.\par
Il le fit venir chez lui, lui conseilla d’écrire un court récit de l’événement et de l’adresser à l’électeur de Brandenbourg, en y joignant la lettre de l’avocat, lui promettant de les lui faire parvenir avec d’autres papiers qu’il avait à lui envoyer. Il assura que cette démarche suffirait pour dévoiler les artifices du gentilhomme de Tronka et lui faire obtenir pleine justice.\par
Kohlhaas, vivement réjoui, le remercia de cette preuve de bienveillance et lui dit qu’il regrettait seulement de ne s’être pas d’abord adressé à la cour de Berlin ; puis étant entré dans le cabinet du commandant, il écrivit sa plainte qu’il lui laissa, et s’en retourna bien rassuré à Kohlhaasenbruck.\par
Il eut cependant le chagrin d’apprendre, quelques semaines après, d’un juge qui se rendait à Postdam par l’ordre du commandant, que le prince électeur avait remis son affaire entre les mains de son chancelier le comte de Kallheim, qui, au lieu de s’occuper immédiatement de la poursuite et de la punition du gentilhomme de Tronka, avait fait prendre des informations préalables auprès de la cour de Dresde.\par
Le juge ne put rien répondre de satisfaisant à cette question de Kohlhaas :\par
« Pourquoi procéder ainsi ? »\par
Il parut pressé de continuer sa route ; mais, par quelques mots qu’il laissa échapper, le marchand apprit que le comte de Kallheim était allié à la maison de Tronka.\par
Kohlhaas, qui ne trouvait plus aucun plaisir ni dans son commerce ni dans sa ferme, ni même auprès de sa femme et de ses enfants, passa le mois suivant dans une pénible attente ; ses dernières espérances furent détruites par le retour de Herse qui lui apportait de Brandenbourg un rescrit accompagné d’une lettre du commandant. Celui-ci marquait à Michel son chagrin de n’avoir pu rien faire pour la réussite de sa cause, et lui conseillait de faire reprendre ses chevaux à Tronkenbourg et d’en rester là. Il lui envoyait la déclaration de la cour à son égard. Elle portait que le tribunal de Dresde avait déclaré sa plainte inutile, puisque le seigneur de Tronka ne lui contestait nullement le droit de venir prendre ses chevaux à Tronkenbourg, ou de lui indiquer le lieu où il devait les lui renvoyer. Dans tous les cas, il était invité à ne plus importuner les tribunaux de telles niaiseries.\par
Kohlhaas, qui n’avait que faire de ses chevaux et dont le chagrin eût été égal, s’il se fut agi d’une couple de chiens, Kohlhaas frémit de rage à la lecture de cet acte.\par
À chaque bruit qu’il entendait, il regardait vers la porte cochère avec la plus pénible anxiété qui eût encore agité son cœur, craignant par-dessus tout de voir les gens du seigneur de Tronkenbourg venir lui offrir quelque dédommagement pour la maigreur et la misère de ses chevaux. C’était le seul cas dans lequel il ne fût pas certain de se rendre maître du sentiment qui s’emparerait de son âme si bien formée par l’expérience de la vie.\par
Mais il apprit bientôt par un de ses amis qui venait de Tronkenbourg, que ses chevaux étaient employés comme tous ceux du château au labeur des champs. À cette nouvelle, qui constatait le désordre de la société, il éprouva une joie secrète de retrouver son âme en harmonie avec l’ordre et la justice.\par
Il fit venir chez lui le bailli, son voisin, qui désirait depuis longtemps augmenter ses possessions par l’acquisition des terres qui les entouraient, et il lui demanda ce qu’il donnerait de ses propriétés brandenbourgeoises et saxonnes, de sa maison, de sa ferme et de ses terres.\par
Lisbeth pâlit à ces mots, et se détournant, elle jeta sur son plus jeune enfant qui jouait derrière elle un regard où se peignit la mort.\par
Le bailli demanda à Michel, en le regardant avec beaucoup de surprise, pour quelle raison il se décidait tout-à-coup à une résolution si étrange.\par
Celui-ci répondit avec une fausse gaîté que la pensée de vendre sa ferme n’était pas nouvelle, puisqu’ils en avaient souvent parlé ensemble, qu’il ne faisait qu’y ajouter la maison de Dresde ; qu’enfin il était prêt, s’il voulait en faire l’estimation, à dresser le contrat de vente. Il ajouta avec un rire forcé, que Kohlhaasenbruck n’était pas le monde, et qu’en prévoyant il pouvait désirer de mettre ordre à ses affaires, son âme lui disant qu’il était destiné à de grandes choses dont on entendrait bientôt parler.\par
Alors le bailli, posant sur la table sa canne et son chapeau qu’il avait jusque-là tenus entre ses genoux, prit la feuille de papier que le marchand lui présentait, et Kohlhaas, se rapprochant de lui, lui expliqua que c’était un contrat éventuel à l’échéance de quatre semaines, qu’il n’y manquait plus que les sommes et les signatures, et il le pria de nouveau de vouloir bien lui faire une offre, ajoutant qu’il était pressé de conclure.\par
Lisbeth, le cœur plein de tristesse, allait et venait dans la chambre pour cacher le trouble qui l’agitait.\par
Le bailli ayant objecté qu’il ne pouvait estimer la maison de Dresde qu’il n’avait jamais vue, Kohlhaas dit qu’il la lui céderait pour cent écus d’or, la moitié du prix qu’elle lui avait coûté. Son voisin, après avoir relu une seconde fois le contrat, séduit par cette manière facile de stipuler et presque décidé, demanda si les chevaux entraient dans le marché.\par
Kohlhaas répondit que son intention était de les garder, ainsi que les armes qui se trouvaient dans le magasin.\par
Alors le bailli prit la plume, et après avoir renouvelé une offre qu’il avait déjà faite autrefois à Kohlhaas, il parcourut le papier, et écrivit l’engagement d’un prêt de cent écus d’or sur les hypothèques du fond de Dresde, qu’il ne voulait point regarder comme acheté jusqu’à deux mois, pendant lesquels Kohlhaas serait le maître de le reprendre, s’il se repentait de son marché.\par
Le marchand, touché de ce procédé lui serra les mains avec beaucoup de reconnaissance ; et après être convenus que le quart du prix serait payé comptant et le reste au bout de trois mois sur la banque de Hambourg, le marchand fit apporter du vin pour boire au succès de sa négociation. Il dit à la servante qui apportait la bouteille de faire seller son cheval, parce qu’il voulait aller à la ville ; puis il se mit à parler des Turcs et des Polonais qui étaient alors en guerre, et entraîna son voisin dans mille conjectures politiques : après avoir bu encore un coup à la réussite de ses projets, le bailli se retira.\par
Dès qu’il eut quitté la chambre, Lisbeth tombant aux genoux de Michel, s’écria : « Si tu me portes dans ton cœur, ainsi que les enfants que je t’ai donnés, si nous n’en sommes pas déjà rejetés, pour quelque raison à moi inconnue, dis-moi ce que signifie cette étrange résolution.\par
— Chère Lisbeth, dit Kohlhaas, pour ne point t’affliger, je t’ai caché la déclaration du tribunal dans laquelle il est dit que ma plainte contre le gentilhomme de Tronka n’est qu’une niaiserie. Il y a sans doute un malentendu là-dedans, et j’ai pris la détermination d’aller moi-même demander justice.\par
— Mais pourquoi vendre ta maison ? dit Lisbeth en se relevant.\par
— Ma chère amie, dit Kohlhaas en la pressant tendrement contre son sein, puis-je rester dans un pays qui ne veut pas soutenir mon droit, où je suis traité comme un chien que l’on repousse du pied. Je suis certain que tu penses comme moi.\par
— Sais-tu si l’on ne veut pas te rendre justice, Michel ? Si tu t’approchais humblement du prince, ta supplique à la main, qui te dit qu’il te repousserait sans vouloir t’entendre ?\par
— Eh bien, ma chère femme, si ma crainte est sans fondement, je suis encore à temps de reprendre ma maison. Le prince est juste, je le sais, et si j’ai le bonheur de parvenir jusqu’à lui, je ne doute pas d’obtenir satisfaction et de revenir dans peu de jours auprès de toi pour ne plus te quitter. Mais il est toujours prudent de se préparer au pire. Je désire donc que tu t’éloignes pour quelque temps, si cela se peut, et que tu te rendes avec nos enfants chez ta cousine à Schwérin.\par
— Quoi ! s’écria Lisbeth, je dois aller à Schwérin sur la frontière avec mes enfants ! » Et le saisissement l’empêcha d’en dire davantage.\par
« Sans doute, reprit Kohlhaas, et dès à présent, car la démarche que je médite ne veut aucun retard.\par
— Oh ! je te comprends, tu n’as besoin que d’armes et de chevaux, tout le reste deviendra ce qu’il pourra ; » et à ces mots, elle se laissa tomber en pleurant sur une chaise.\par
« Chère Lisbeth, lui dit Kohlhaas avec tristesse, que fais-tu ! Dieu m’a béni dans ma femme et dans mes enfants : devrais-je aujourd’hui pour la première fois désirer qu’il en eût été autrement ? » Puis il s’assit à côté de Lisbeth, qui, rougissant à ce reproche, se jeta tout confuse dans ses bras.\par
« Dis-moi, continua-t-il en jouant avec les boucles de cheveux qui tombaient sur son front, que dois-je faire ? Faut-il que j’aille à Tronkenbourg redemander mes chevaux au gentilhomme ? »\par
Lisbeth n’osa dire oui ; elle secoua la tête en pleurant, et s’attachant fortement à lui, elle couvrit sa poitrine d’ardents baisers.\par
« Si tu sens, s’écria Kohlhaas, que je dois me faire rendre justice pour continuer ensuite mon paisible commerce, accorde-moi aussi la liberté de choisir mes moyens. »\par
Puis, se levant, il ordonna au domestique qui venait lui dire que son cheval était sellé, de se préparer à conduire sa femme dès le lendemain à Schwérin.\par
« Il me vient une idée, s’écria Lisbeth en essuyant ses larmes, et en s’approchant de la table où Kohlhaas s’était mis à écrire ; permets que j’aille moi-même à Berlin présenter ta supplique au prince électeur. »\par
Kohlhaas, vivement touché de cette marque de tendresse, la prit de nouveau dans ses bras.\par
« Chère amie, lui dit-il, c’est impossible : le prince est tellement entouré qu’il est très difficile de l’approcher. »\par
Lisbeth lui assura qu’il était plus facile à une femme trouver accès auprès de lui.\par
« Donne-moi ta supplique, ajouta-t-elle, et si tu ne demandes que de la voir entre ses mains, je te le promets, elle y parviendra. »\par
Kohlhaas, qui connaissait déjà le courage et la prudence de sa femme, lui demanda comment elle comptait s’y prendre. Elle répondit, en rougissant et les yeux baissés, que le castellan du château avait prétendu à sa main lors de son service à Schwérin ; qu’il s’était marié depuis ; mais qu’il ne l’avait jamais oubliée, et qu’elle était sûre de réussir, soit pour cette raison, soit pour d’autres encore qu’il serait trop long d’énumérer ici.\par
Kohlhaas, l’embrassant avec beaucoup de joie, dit qu’il acceptait son offre, et qu’elle n’avait qu’à se rendre au château. Le même jour, il la fit partir pour Berlin, dans une bonne calèche, avec son domestique Sternbald.
\chapterclose


\chapteropen
\chapter[Chapitre III]{Chapitre III}\renewcommand{\leftmark}{Chapitre III}


\chaptercont
\noindent Le voyage de Lisbeth fut la plus malheureuse de toutes les démarches inutiles faites par Kohlhaas dans cette affaire ; car peu de jours après, Sternbald entra dans la cour de Kohlhaasenbruck, conduisant au pas la voiture dans laquelle sa maîtresse était étendue, presque mourante d’une blessure dans la poitrine. Kohlhaas apprit du fidèle Sternbald que le castellan ne s’étant pas trouvé chez lui, ils avaient été obligés de descendre dans un hôtel tout voisin du château. Le lendemain Lisbeth avait quitté la maison, lui ordonnant de garder les chevaux, et le soir elle était rentrée dans cet état. Il paraissait qu’elle avait voulu s’approcher du prince, et que, sans l’ordre de celui-ci, et par le zèle grossier d’un des gardes qui l’entouraient, elle avait reçu un coup de lance dans la poitrine. C’était ainsi du moins que les gens qui l’avaient rapportée le soir avaient expliqué les choses ; car pour elle, elle n’avait pu parler, à cause du sang qui lui sortait de la bouche. Sternbald ajouta que son intention avait été de partir aussitôt à cheval pour venir lui apprendre ce malheureux accident, mais que sa maîtresse avait insisté, malgré les représentations du chirurgien, pour être ramenée sans délai.\par
Kohlhaas la porta sur un lit, où elle reprit ses sens pour quelques jours. Mais il chercha vainement à apprendre d’elle ce qui lui était arrivé ; elle restait l’œil fixe et la bouche close à toutes ses questions. Ce ne fut qu’un instant avant sa mort qu’elle sembla recouvrer la mémoire. Elle se tourna tout-à-coup vers le ministre luthérien, qui lisait l’Évangile à côté de son lit, et prenant la Bible de ses mains, elle se mit à la feuilleter rapidement, comme si elle y cherchait quelque chose ; puis montrant à Kohlhaas le verset suivant : « Pardonne à tes ennemis ; fais du bien à ceux qui te maudissent, etc. », elle lui serra la main, le regarda avec tendresse, et expira.\par
« Que Dieu ne me pardonne jamais si je pardonne au gentilhomme ! » pensa Kohlhaas ; puis après avoir fermé les yeux de sa femme chérie en versant un torrent de larmes amères, il sortit de la chambre.\par
Prenant les cent écus d’or que son voisin lui avait déjà remis sur la propriété de Dresde, il alla faire préparer pour Lisbeth une tombe aussi riche que celle d’une princesse. Le cercueil de chêne était doublé de métal, et garni de coussins de soie, ornés de galons d’or ; la fosse, de huit brassées de profondeur, fut creusée sous ses yeux, tandis qu’il se promenait à l’entour avec ses pauvres petits enfants.\par
Lorsque le jour de l’enterrement fut arrivé, on transporta le corps, blanc comme la neige, dans une salle tendue de noir ; le ministre prononça sur le cercueil un discours touchant ; ensuite on le plaça sur un char, et on le porta en terre. Kohlhaas, après avoir congédié les amis qui étaient venus rendre les derniers devoirs au cadavre de Lisbeth, alla se jeter encore une fois sur ce lit vide maintenant, et prononça de nouveau le serment de la vengeance.\par
Rentré dans son cabinet, il écrivit un acte par lequel il sommait le gentilhomme de Tronka, en vertu de son droit naturel, de ramener en personne ses chevaux à Kohlhaasenbruck, sous le délai de trois jours, et le condamnait à les nourrir dans leur écurie jusque ce qu’ils fussent redevenus aussi gras qu’ils étaient lorsqu’il les avait laissés à Tronkenbourg.\par
Il envoya ce billet ; et les trois jours s’étant écoulés sans qu’il eût reçu aucun message de Tronkenbourg, il appela Herse, lui dit ce qu’il avait écrit au seigneur, et lui demanda s’il voulait l’accompagner au château, pour enseigner au gentilhomme à s’acquitter de son devoir.\par
Herse le comprit aussitôt, et jetant son bonnet en l’air, il s’écria : « Partons, mon maître, partons aujourd’hui même. »\par
Kohlhaas ayant terminé la vente de sa ferme et fait partir ses enfants pour la frontière, appela le reste de ses domestiques qui étaient au nombre de sept, tous d’une fidélité éprouvée ; il les arma, et sur le soir il partit à leur tête pour Tronkenbourg.\par
Au commencement de la troisième nuit ils pénétrèrent dans la cour, et après avoir mis le feu aux dépendances du château, Herse se précipita dans la tour, et tomba à l’improviste sur le châtelain et surintendant qui, à moitié déshabillés, étaient établis autour d’une table de jeu.\par
Le gentilhomme Wenzel de Tronka, qui était précisément à rire avec ses jeunes amis de la sommation que lui avait envoyée le marchand, n’entendit pas plutôt résonner sa voix, qu’effrayé comme si l’ange de la justice fût descendu du ciel, il pâlit, et se levant, il s’enfuit en criant à ses amis « Sauvez-vous ! »\par
Kohlhaas, repoussant tous ceux qui voulaient s’opposer à son passage, entra en demandant le gentilhomme, et voyant que l’on cherchait à lui cacher une porte qui conduisait dans une autre aile du château, il s’y précipita.\par
Après avoir parcouru tout le bâtiment sans trouver le gentilhomme, Kohlhaas courut dans la cour. Le feu avait gagné, et le château se trouvait entouré d’une épaisse fumée et de flammes ardentes s’élevant jusqu’aux nues. Sternbald, aidé de trois valets, venait de jeter par la fenêtre les cadavres ensanglantés du châtelain et de l’intendant, aux cris de triomphe de Herse et aux plaintes confuses des femmes et des enfants de ces misérables.\par
Un jeune garçon de Tronkenbourg, voyant le feu prêt à atteindre les écuries, se hâtait d’y courir pour en faire sortir les chevaux du gentilhomme, lorsque Kohlhaas, se mettant sur son chemin, lui arracha la clef qu’il jeta par-dessus les murs, et le força, aux acclamations et à la risée de tous ses domestiques, de sauver les deux haridelles dont le couvert était déjà la proie des flammes. Le jeune homme, se retirant avec peine des ruines fumantes, présenta les bêtes à Kohlhaas, qui, le repoussant avec un violent coup de pied, alla s’asseoir sans rien dire à la porte du château, où il attendit le point du jour.\par
Lorsqu’il parut, le château n’offrait plus que des ruines, et personne ne s’y trouvait que Kohlhaas et ses sept domestiques.\par
Le marchand, accablé de tristesse, alla chercher dans les environs quelques renseignements sur le gentilhomme. Il revint plus calme, ayant appris qu’il y avait non loin de Tronkenbourg un couvent de femmes nommé Erlabrunn, sur les bords du Mulde, dont l’abbesse, Antonie de Tronka, était connue dans tout le pays pour une sainte. Il lui parut très vraisemblable que le gentilhomme s’y était réfugié.\par
Montant à l’appartement du château (la tour n’avait point souffert de l’incendie), il y écrivit un mandat où il sommait tout parent ou ami qui aurait caché le gentilhomme Wenzel de Tronka de le lui livrer sous peine de mort et de pillage. Il répandit aussitôt cette déclaration dans le pays, puis il en remit une copie à l’un de ses domestiques nommé Waldmann, pour qu’il la portât à dame Antonie.\par
Il prit à son service quelques-uns des hommes de Tronkenbourg qui avaient à se plaindre de leur maître, les arma, comme fantassins, de poignards et d’arbalètes, et les exerça à marcher derrière les cavaliers. Après leur avoir distribué de l’argent, il s’assit sur les ruines du château pour se reposer un instant de ses douloureuses fatigues.\par
Vers midi, Herse vint lui confirmer ce que son cœur lui avait fait pressentir, savoir, que le gentilhomme avait trouvé un asile à Erlabrunn chez sa tante la vieille dame Antonie.\par
Il paraissait qu’il avait fui du château par un escalier secret conduisant jusqu’au bord de l’Elbe.\par
Kohlhaas soupira à ce récit ; sa troupe s’étant mise en marche, arriva avant trois heures à Erlabrunn, armée de flambeaux pour mettre le feu au couvent.\par
Un sombre orage murmurait dans le lointain.\par
Waldmann, qui vint à la rencontre de son maître, lui ayant dit qu’il avait remis le mandat à l’abbesse, Kohlhaas ordonna à ses gens de sonner la cloche du couvent ; alors l’abbesse portant un crucifix d’argent, descendit la rampe, suivie de toutes ses nonnes, et vint se jeter aux pieds du cheval du maquignon.\par
Celui-ci demanda durement où était caché le gentilhomme.\par
« Il est à Wittemberg, honnête Kohlhaas, » répondit-elle d’une voix tremblante.\par
Kohlhaas, retombant dans la torture d’une vengeance non accomplie, allait ordonner à sa troupe d’avancer et de mettre le feu, lorsque la foudre, tonnant avec violence, vint arrêter la voix.\par
« N’avez-vous pas reçu mon mandat ? demanda-t-il à l’abbesse.\par
— Oui, répondit la dame d’une voix presqu’inintelligible, à présent même, trois heures après le départ de mon neveu : aussi vrai que Dieu existe. »\par
Waldmann répondit au sombre regard de son maître que c’était la pure vérité, les mauvais chemins l’ayant empêché d’arriver plus tôt.\par
Une effroyable averse vint en ce moment éteindre les mèches destructives, et Kohlhaas sentit en même temps se calmer dans son triste cœur toute colère contre de pauvres femmes. Saluant dame Antonie, il tourna le dos au couvent en s’écriant : « Suivez-moi, mes frères, le gentilhomme est à Wittemberg ; » et donnant de l’éperon, ils furent bientôt à une grande distance.\par
Au point du jour, ils entrèrent dans une auberge sur la grande route, où, vu la fatigue des chevaux, il fallait s’arrêter quelques heures. Kohlhaas réfléchissant qu’il lui serait impossible d’attaquer avec dix hommes une place comme Wittemberg, écrivit un second mandat dans lequel, après un court récit de tout ce qui lui était arrivé dans le pays, il invitait tout bon chrétien, sous la promesse d’une paie et d’un riche butin, à prendre parti dans sa guerre contre le gentilhomme de Tronka.\par
Le fanatisme et le mécontentement autant que l’attrait du gain lui attirèrent bientôt une bande de misérables que la paix avec la Pologne avait laissés sans ressources, en sorte qu’il se trouvait à la tête de trente hommes lorsqu’il arriva sur la rive droite de l’Elbe dans l’intention de réduire en cendres la ville de Wittemberg.\par
Il se retira avec sa troupe sous un vieux hangar ruiné dans la solitude d’une sombre forêt, et lorsqu’il eut appris par un de ses hommes, qu’il avait envoyé déguisé à la ville, que son mandat y était déjà connu, il partit avec ses gens pour mettre le feu aux faubourgs, pendant que les habitants de Wittemberg étaient encore livrés au sommeil.\par
Tandis que ses soldats, profitant du trouble que causait l’incendie, se livraient au pillage, il placarda contre la porte d’une église une déclaration contenant que lui, Kohlhaas avait allumé l’incendie, et que si on ne lui livrait aussitôt le gentilhomme, il le propagerait dans toute la ville jusqu’à ce qu’il n’y restât pas une muraille derrière laquelle il pût se cacher.\par
L’effroi des habitants fut inexprimable. Le feu, qui dans une courte nuit d’été n’avait heureusement brûlé que dix-neuf maisons, dans le nombre desquelles était une église, ne fut pas plutôt éteint, que le vieux préfet Otto de Gorgas envoya un bataillon de cinquante hommes à la poursuite de l’incendiaire.\par
Mais le chef de ce bataillon, nommé Gerstenberg prit si mal ses mesures, que l’expédition de Kohlhaas, loin d’en être arrêtée, en acquit une plus grande gloire.\par
Ce premier, divisant ses forces comme il le pensait pour cerner Kohlhaas et le faire prisonnier, fut lui-même cerné et battu, de telle sorte que le soir du jour suivant il ne restait pas un seul des hommes auxquels la ville avait confié sa défense.\par
Kohlhaas mit de nouveau le feu aux faubourgs, qui cette fois furent réduits en cendres, puis il afficha encore son mandat jusque sur l’hôtel de ville, y ajoutant le récit du sort que venait d’éprouver le capitaine de Gerstenberg.\par
Le préfet, indigné de l’audace de cet homme, se mit lui-même avec plusieurs cavaliers à la tête de deux cents hommes, et, après avoir donné au gentilhomme une garde qui devait le préserver de la fureur du peuple, jusqu’à ce que l’on pût le faire évader, il sortit le jour de Saint-Gervais pour repousser la fureur de l’hydre menaçante qui désolait le pays.\par
Mais le maquignon fut assez prudent pour éviter une rencontre ; il fit semblant de fuir, jusqu’à ce qu’il eût attiré le préfet à quelques milles de la ville, faisant courir le bruit qu’il se jetait sur le Brandenbourg ; puis, se retournant subitement, il revint en toute hâte à Wittemberg, et y mit le feu pour la troisième fois.\par
L’incendie, poussé par le vent du nord, se propagea avec une inconcevable rapidité ; en moins de trois heures, quarante-trois maisons, deux églises, plusieurs couvents, des écoles et la préfecture furent réduits en cendres.\par
Le préfet, apprenant le piège dans lequel il était tombé, retourna au point du jour à marche forcée sur la ville, qu’il trouva dans le plus grand désordre.\par
Le peuple, armé de poutres et de haches, rassemblé devant l’habitation du gentilhomme, demandait avec des cris de rage qu’on le fît sortir de la ville.\par
Les bourgmaistres, en habits de cérémonie, à la tête de toute la magistrature s’étaient transportés sur la place, et cherchaient vainement à obtenir que l’on attendît tranquillement le retour d’un messager qu’ils avaient envoyé à Dresde pour y demander les ordres de la cour à l’égard du gentilhomme. La populace furieuse, ne tenant nul compte de leurs discours, allait suivre le conseil des plus violents, qui était d’assiéger et de démolir la maison qui renfermait le gentilhomme, lorsqu’Otto de Gorgas rentra à la tête de sa troupe.\par
Ce digne homme, dont le seul aspect inspirait toujours au peuple la confiance et le respect, réussit à le calmer en lui montrant deux des complices de Kohlhaas, qu’il ramenait chargés de chaînes, et en lui faisant espérer de tenir bientôt leur chef entre ses mains.\par
Ensuite il pénétra dans l’appartement du gentilhomme, qu’il trouva tombant d’évanouissement en évanouissement entre les bras de deux médecins qui le rappelaient à la vie avec des essences et des cordiaux. Sentant bien que ce n’était pas le moment de lui faire des reproches, il se contenta de jeter sur lui un regard de mépris, en lui disant de s’habiller et de le suivre pour sa sûreté dans le palais de justice.\par
Lorsqu’on eut revêtu le gentilhomme d’un pourpoint et d’un casque, en ayant soin de laisser sa poitrine découverte, à cause des fréquents étouffements dont il était saisi, il parut dans la rue, appuyé sur le bras du comte de Gerschau, son beau-frère. Le peuple, contenu avec peine par les gens d’armes, l’accabla de mille imprécations, il le nomma le fléau du pays, la malédiction de la ville de Wittemberg, et le malheur de la Saxe.\par
Cependant, après un trajet pénible au milieu des décombres de l’incendie, il atteignit sans accident le palais de justice, où il fut enfermé dans une tour occupée par une forte garde.\par
Le retour du messager avec la résolution de la cour vint mettre le préfet dans un nouvel embarras ; la bourgeoisie de Dresde ayant adressé à celle-ci une pressante supplique, elle ne voulait aucunement le gentilhomme, de crainte d’attirer dans la capitale la guerre et l’incendie. Elle ordonnait au préfet de le laisser où il était, l’avertissant qu’elle allait envoyer contre Kohlhaas, pour venger la bonne ville de Wittemberg, une troupe de cinq cents hommes, sous le commandement du prince Frédéric de Meissen.\par
Le préfet vit bien que cette résolution n’était pas de nature à contenter le peuple ; car la guerre que le maquignon faisait dans les ténèbres, avec la poix, la paille et l’allumette, aurait rendu vaine une force plus considérable que celle du prince de Meissen. Il se décida à cacher la résolution de la cour, et se contenta de faire publier la croisade du prince contre Kohlhaas.\par
Le lendemain au point du jour une voiture fermée sortit du palais de justice, accompagnée de quatre cavaliers bien armés, qui prirent la route de Leipzig, et le bruit se répandit qu’ils conduisaient le gentilhomme à Pleissembourg. Le peuple, satisfait à son égard, se réunit en foule à la troupe du prince de Meissen.\par
Cependant Kohlhaas se trouvait dans la plus étrange situation, à la tête d’une centaine d’hommes, menacé d’un côté par le préfet, de l’autre par le prince.\par
Sa troupe étant bien armée et d’une vaillance éprouvée, il se décida à marcher courageusement au-devant de ce double orage. Le soir du même jour, il attaqua le prince de Meissen dans une rencontre nocturne à Muhlberg, où il eut le chagrin de voir tomber le fidèle Herse, qui combattait à ses côtés. Aigri par cette perte, Kohlhaas redoubla de vaillance, et au bout de trois heures de combat, le prince était hors d’état de rallier sa troupe, soit à cause du désordre qui y régnait, soit à cause de ses blessures.\par
Après cette première action, Kohlhaas se retourna sur le préfet, qu’il attaqua en plein jour et en rase campagne. La perte fut égale des deux côtés ; mais il est probable que le préfet eut été défait s’il n’eût profité de la nuit pour retourner à Wittemberg avec le reste de sa troupe.\par
Cinq jours après, Kohlhaas était devant Leipzig, mettant le feu à trois points de la ville. Il se nommait, dans le mandat qu’il répandit en cette occasion, un représentant de l’archange Michel, qui était venu, armé du glaive de la justice, livrer une guerre à feu et à sang à l’injustice et à la fausseté des hommes. Du château de Lutzen, dont il s’était emparé, il appelait le peuple à se joindre à lui pour réformer le monde, et cette feuille était signée avec une sorte d’égarement : « Du siège de notre gouvernement provisoire, le château de Lutzen. » Le bonheur des habitants voulut qu’une pluie abondante vînt déjouer les projets de l’incendiaire, et qu’il n’y eût de brûlé qu’une boutique voisine du château de Pleissembourg.\par
L’effroi et le trouble s’emparèrent de tous les cœurs, lorsqu’on apprit à Leipzig que le gentilhomme devait se trouver au château de Pleissembourg, et que l’on vit s’approcher une bande de deux cents cavaliers envoyés à sa recherche.\par
Ce fut en vain que les magistrats firent répandre dans les environs la déclaration que le gentilhomme n’était point à Pleissembourg ; le maquignon, dans de semblables feuilles, soutint qu’il s’y trouvait, et que, s’il était vrai qu’il n’y fût pas, il ne s’éloignerait que lorsqu’on lui aurait nommé le lieu de sa retraite.\par
Le prince de Saxe, instruit de la détresse où se trouvait sa ville de Leipzig, déclara qu’il allait marcher en personne contre Kohlhaas, à la tête de deux mille hommes. Il fit reprocher sévèrement au préfet Otto de Gorgas le perfide mensonge par lequel il avait rejeté sur Leipzig toute la fureur de l’incendiaire.\par
Mais il serait impossible de décrire le trouble de toute la Saxe et principalement de la résidence, lorsqu’on y eut connaissance d’une nouvelle affiche où l’on déclarait à Kohlhaas, que le gentilhomme Wenzel s’était enfui à Dresde chez ses cousins Hinz et Kunz de Tronka.
\chapterclose


\chapteropen
\chapter[Chapitre IV]{Chapitre IV}\renewcommand{\leftmark}{Chapitre IV}


\chaptercont
\noindent Ce fut alors que le docteur Martin Luther, voulant essayer d’user de son influence pour ramener à l’ordre cet homme extraordinaire, fit afficher dans les villes et les villages de l’électorat, le placard suivant, adressé à Kohlhaas.\par
« Kohlhaas, toi qui te donnes comme envoyé du ciel pour manier le glaive de la justice, téméraire ! aveuglé par la passion de la vengeance, tu t’es couvert de crimes et d’injustices ! Parce que dans une affaire de peu d’importance la cour a refusé de soutenir ton droit, tu te saisis du fer et du feu, et, semblable au loup furieux du désert, tu te jettes avec rage sur les paisibles contrées de ton prince ! Toi qui fais la guerre d’une manière pleine d’astuce et de perfidie, penses-tu, pécheur, être épargné devant le tribunal de ton Dieu, quand le jour viendra où seront examinés les cœurs ?\par
» Comment pourras-tu dire que la justice te fut refusée, toi, homme haineux, qui, séduit par l’attrait de la vengeance, t’es rebuté d’un premier refus, afin de pouvoir te livrer à toute ta fureur ?\par
» Est-ce donc un banc de conseillers et d’avocats qui ont refusé ta plainte que tu prends pour ton maître ?\par
» Et si je te disais, insensé, que ton prince ne sait rien de ton affaire ; que dis-je ? que le prince contre lequel tu te déclares ne connaît pas même ton nom ; que si au jour du jugement tu parais devant Dieu avec la pensée de te plaindre de lui, il pourra répondre : « Seigneur, je ne fis aucun mal à cet homme ; son existence me fut même étrangère ! »\par
» Sache donc que l’épée que tu portes est celle du brigand et de l’assassin ; tu es un rebelle et non point un serviteur de Dieu. Ce que tu mérites sur la terre, c’est la roue, et dans l’éternité, la damnation que Dieu réserve aux malfaiteurs et aux impies. »\par
« MARTIN LUTHER »\par
Le soir, lorsque Sternbald et Waldmann rentrèrent au château, ils virent avec une grande surprise ce placard affiché sur la porte cochère. Ils n’en parlèrent point à Kohlhaas, pensant qu’il ne tarderait pas à l’apercevoir. Mais celui-ci, concentré en lui-même et frappé d’une noire mélancolie, ne sortait que rarement, à la tombée de la nuit, pour donner des ordres rapides. Ce ne fut donc qu’au bout de quelques jours, sortant en grande cérémonie pour faire exécuter deux hommes qui s’étaient rendus coupables de pillage malgré sa défense, il remarqua cette feuille et la lut d’un bout à l’autre.\par
Il serait impossible d’exprimer ce que son âme éprouva en voyant la signature de l’homme qu’il aimait et qu’il respectait le plus au monde. Une vive rougeur couvrit son visage, il parut profondément touché, et se tournant vers ses domestiques, il ordonna à Waldmann de seller son cheval et à Sternbald de le suivre au château.\par
Cette courte exhortation avait suffi pour le retirer de son iniquité ; il dit à Sternbald qu’une affaire de la plus grande importance l’appelant à Wittemberg, il lui laissait pendant son absence, qui devait durer trois jours, le commandement du château de Lützen et de sa troupe. Puis prenant le costume d’un paysan de la Thuringe, il partit.\par
Arrivé à Wittemberg, il descendit dans un hôtel où il attendit la nuit ; lorsqu’elle fut venue, enveloppé de son manteau et armé de deux pistolets il se rendit chez Luther, et entra dans sa chambre sans se faire annoncer.\par
Celui-ci, qui était assis à une table couverte de manuscrits et plongé dans de savantes méditations, surpris de voir un homme d’une tournure étrange entrer brusquement et fermer la porte à clef, lui demanda qui il était et ce qu’il voulait. Mais Kohlhaas, tenant respectueusement son chapeau à la main, n’eut pas plutôt prononcé son nom avec le pressentiment de l’horreur qu’il allait causer, que Luther s’écria : « Sors d’ici, ton haleine est la peste, tout ton être est plein d’iniquité ! » et se levant, il courut à la sonnette.\par
Kohlhaas, sans reculer d’un pas et sortant l’un de ses pistolets de sa ceinture, lui dit : « Seigneur, si vous agitez la sonnette, cette arme va m’étendre mort à vos pieds. Asseyez-vous, et daignez m’écouter ; vous ne sauriez être plus en sûreté parmi les anges qui veillent sur vous, que vous ne l’êtes auprès de moi.\par
— Kohlhaas, dit Luther en reprenant sa place, que me veux-tu ?\par
— Je veux changer l’opinion que vous avez de moi. Je veux vous prouver que je ne suis point un homme injuste. Vous m’avez dit que mon prince ne connaissait point mon affaire ; eh bien ! procurez-moi un sauf-conduit, et je vais à Dresde la lui exposer.\par
— Scélérat ! s’écria Luther, qui donc t’a donné le droit de poursuivre partout le gentilhomme de Tronka, et, parce que tu ne le trouvais point dans son château de Tronkenbourg, de ravager sans pitié tout le pays qui le protège ?\par
— Personne, digne seigneur. Une dure réponse que je reçus de la cour de Dresde m’a séduit et égaré. J’en conviens, la guerre que j’ai entreprise contre la société est un crime, si, comme vous m’en avez donné l’assurance, je ne suis point rejeté par elle.\par
— Rejeté ! répéta Luther en le fixant avec surprise. Quelle folie s’empare de ton esprit ? Qui aurait pu te rejeter de la société ? Qui a jamais vu, en aucun cas, un homme repoussé par elle ?\par
— J’appelle rejeté, répondit Kohlhaas en joignant les mains, l’homme à qui les lois refusent leur protection. J’ai besoin de cette protection pour la réussite d’un commerce honnête ; c’est elle qui me permet de vivre en paix, dans mon pays ; mais si elle m’est refusée, je deviens semblable au sauvage furieux, et je puis sans crime m’armer, contre la société qui rompt avec moi, de la massue qui seule peut me protéger.\par
— Qui t’a refusé la protection des lois ? Ne t’ai-je pas écrit que la supplique que tu as adressée au monarque lui était restée inconnue ? Si des juges, si des conseillers refusent, sans l’en informer, de rendre justice à qui elle est due, et s’ils exposent ainsi son saint nom au mépris, quel autre que Dieu a le droit de lui demander compte de son mauvais choix ? Est-ce à toi, criminel ! est-ce à toi à le condamner ?\par
— Eh bien ! dit Kohlhaas, s’il est vrai que le prince ne m’a point rejeté, je rentre dans la société qu’il protège. Je vous le demande encore, procurez-moi un sauf-conduit pour Dresde ; je licencie la troupe que j’ai laissée à Lutzen, et je porte de nouveau ma plainte devant le tribunal. »\par
Luther garda le silence quelques instants ; son visage était sévère. Il ne pouvait souffrir l’orgueilleuse position dans laquelle se plaçait cet homme extraordinaire. Il lui demanda enfin ce qu’il voulait du tribunal de Dresde.\par
« Punition du gentilhomme selon la loi, répondit Kohlhaas ; restitution de mes chevaux dans leur état antérieur, et remboursement des dommages soufferts par moi et par mon valet Herse, mort à Muhlberg.\par
— Remboursement des dommages ! s’écria Luther. Par Juifs et Chrétiens, ta propre vengeance ne t’a-t-elle pas indemnisé bien au-delà de tes dommages ?\par
— Dieu me préserve de demander plus qu’il n’est juste. Ma maison et ma ferme et le bien-être que je possédais, je ne les redemande point, pas davantage que le prix de la sépulture de ma femme. Mais la pauvre vieille mère de Herse doit recevoir la valeur des objets laissés à Tronkenbourg par son fils, et le dommage que j’ai éprouvé en manquant la vente de mes chevaux doit être raisonnablement estimé par la cour.\par
— Insensé ! homme coupable et incompréhensible ! Après que ton épée t’a vengé de la manière la plus sanglante que l’on puisse imaginer, comment oses-tu exiger la réparation d’un tort si minime ?\par
— Seigneur, répliqua doucement Kohlhaas, tandis qu’une larme roulait sur sa joue, il m’en a coûté ma femme ; je veux montrer au monde que ma chère Lisbeth ne se mêla point d’une chose injuste. Permettez que j’agisse selon mon désir en ceci ; en toute autre chose je me conformerai à votre volonté.\par
— Considère, Kohlhaas, combien il eût mieux valu t’adresser au prince avant d’agir comme un furieux ; il t’aurait pleinement satisfait, je n’en doute pas, et si cela n’était point arrivé, n’aurais-tu pas mieux fait encore de pardonner au gentilhomme pour l’amour de ton Sauveur, et de reprendre tes chevaux pour les rétablir dans ton écurie à Kohlhaasenbruck.\par
— C’est possible, répondit Kohlhaas en faisant quelques pas dans la chambre ; il se peut que j’eusse fait comme vous dites, si j’avais su que le sang de ma femme devait couler. Mais à présent que cette affaire m’a tant coûté, elle doit être poussée à bout, et le gentilhomme sera contraint à restaurer mes chevaux. »\par
Après un instant de réflexion, Luther dit qu’il écrirait au prince électeur à son sujet ; qu’en attendant, il lui recommandait de se tenir tranquille à son château de Lutzen, où il apprendrait par un nouveau placard si le prince lui accordait une amnistie. « Cependant, ajouta-t-il, pendant que Kohlhaas s’inclinait pour lui baiser la main, il est possible que le prince te refuse cette grâce, car je sais qu’il prépare des troupes pour te surprendre à Lutzen. »\par
À ces mots, il se leva pour le congédier ; mais Kohlhaas mettant un genou en terre dit qu’il avait encore une grâce à lui demander, c’était de vouloir bien, sans de plus longues préparations, lui accorder le bienfait de la sainte cène.\par
« Oui, dit Luther en lui jetant un regard scrutateur, je le veux. Tu sais que notre Seigneur, dont tu demandes le corps et le sang, pardonnait à ses ennemis, veux-tu pardonner de même au gentilhomme, reprendre tes chevaux et retourner à Kohlhaasenbruck ?\par
— Digne seigneur, s’écria Kohlhaas en rougissant et en saisissant la main de Luther, notre divin Sauveur ne pardonna pas à tous ses ennemis. Demandez-moi de pardonner au prince, au châtelain, à l’intendant, aux seigneurs Hinz et Kunz de Tronka, à tous ceux enfin qui m’ont nui dans cette affaire ; mais pour que je puisse pardonner au gentilhomme, il faut d’abord qu’il ait restauré mes chevaux. »\par
À ces mots, Luther lui tournant le dos avec dédain tira la sonnette pour qu’un domestique vînt éclairer Kohlhaas, et il se remit à son bureau.\par
Le marchand, confus et les yeux baissés, ouvrit la porte fermée en dedans, que le domestique cherchait vainement à forcer.\par
Luther jetant un regard de côté sur Kohlhaas dit au domestique de l’éclairer, et celui-ci se plaçant devant la porte entrouverte, attendit qu’il sortît.\par
« Ainsi, mon seigneur, dit timidement Michel en faisant tourner son chapeau entre ses deux mains, vous me refusez le bienfait de la réconciliation ?\par
— Avec ton Dieu, oui, répondit sèchement Luther ; avec ton prince, c’est une épreuve que je tenterai, comme je te l’ai promis ; » puis il fit signe au domestique de reconduire aussitôt l’étranger. Kohlhaas, posant ses deux mains sur sa poitrine avec l’expression du plus amer chagrin, sortit de la chambre et disparut.\par
Le lendemain Luther adressa au prince électeur une lettre où, après avoir jeté un coup d’œil amer sur les seigneurs Hinz et Kunz de Tronka, qui avaient comme tout le monde, en étant instruits, rejeté la plainte du maquignon contre leur cousin, il faisait considérer au prince dont il connaissait toute la générosité, qu’il n’y avait rien de mieux à faire dans de si malheureuses circonstances, que d’accorder au maquignon une amnistie, qui lui permettrait de renouveler sa plainte devant les tribunaux.\par
L’électeur de Saxe reçut cette lettre en présence du prince Christiern de Meissen, généralissime du royaume, oncle du prince Frédéric, qui avait été blessé à Mühlberg ; du grand chancelier du tribunal, le comte de Wrede ; du comte de Kallheim, président de la chancellerie, et des deux gentilshommes Kunz et Hinz de Tronka, le premier chambellan, le second grand échanson, les amis de jeunesse et les favoris du monarque.\par
Le chambellan Kunz qui avait, en qualité de conseiller privé de la correspondance, la faculté de se servir des armes et du nom du prince, prit le premier la parole. Après avoir parlé de la plainte du maquignon, qu’il avoua n’avoir point prise en considération, la regardant comme une bagatelle de peu d’importance, il en vint à l’état actuel des choses. Il observa que ni les lois célestes, ni les lois humaines n’avaient pu permettre au maquignon de se faire droit lui-même d’une manière si horrible ; il peignit d’une part l’éclat qu’une négociation avec lui jetterait sur sa tête damnée, et de l’autre l’ignominie qui en résulterait pour la personne sacrée du prince. Cela lui parut si insupportable, que dans le feu de son zèle il prétendit qu’il aimerait mieux que le désir de cet enragé rebelle fût accompli, et voir son cousin forcé à remplir le rôle de palefrenier dans l’écurie de Kohlhaasenbruck, plutôt que de souffrir que le prince acceptât la proposition de Luther.\par
Le grand chancelier du tribunal comte de Wrede, se tournant à demi vers lui, exprima un vif regret qu’il n’eût pas montré dès l’origine de cette affaire cette vive sollicitude pour la gloire de son maître. Il ajouta qu’il était d’avis que le prince fît usage de son pouvoir pour réparer publiquement l’injustice commise contre le maquignon, considérant que cette seule démarche pourrait calmer le peuple et délivrer le pays des nouveaux malheurs que lui faisaient craindre les forces croissantes de l’incendiaire.\par
Le prince de Meissen, sur l’invitation que lui fit l’électeur de donner son avis, dit en s’adressant au grand chancelier, qu’il était rempli de respect pour l’opinion qu’il venait d’énoncer ; que cependant, tout en voulant accorder à Kohlhaas le droit qui lui avait été injustement refusé, il ne pensait pas que le mal fait par lui à Wittemberg, à Leipsick et en d’autres lieux encore, dût rester impuni. La paix et l’ordre établi avaient été tellement troublés par cet homme, qu’il serait bien difficile, avec quelque connaissance en droit, de pouvoir le justifier et l’absoudre. C’est pour cela, continua-t-il, qu’il se rangeait à l’opinion du chambellan : il trouvait qu’il n’y avait rien de mieux à faire qu’à marcher contre Lutzen, pour s’y saisir de Kohlhaas.\par
Le chambellan, prenant deux chaises pour lui et l’électeur, dit, en s’avançant dans la chambre d’un air affable, qu’il se réjouissait qu’un homme d’un si grand mérite et de tant d’esprit se trouvât du même sentiment que lui dans une affaire aussi importante. Mais l’électeur, tenant la chaise qu’il lui présentait sans s’y asseoir, rassura qu’il n’avait aucune raison de se réjouir, parce qu’avant d’employer ce moyen, il se croirait obligé de lui intenter un procès au nom de l’État pour le mauvais usage qu’il avait fait du pouvoir ; « car, s’écria-t-il, avant de punir Kohlhaas, n’est-il pas de toute justice de prononcer une sentence contre celui qui a mis l’épée entre ses mains ? »\par
Mais, voyant qu’il avait affligé le chambellan, il se retira en rougissant vers la fenêtre.\par
Le comte de Kallheim, après une pause embarrassante pour les deux parties, fit observer que l’on pourrait, avec le même droit, faire un procès au prince Frédéric qui avait marché contre Kohlhaas, et que de cette manière on ne sortirait pas du cercle magique dans lequel on se trouvait.\par
L’échanson Hinz de Tronka, s’approchant de la table, déclara qu’il ne comprenait pas que des hommes d’une si haute sagesse se trouvassent embarrassés sur le choix d’une détermination qui semblait si simple. Le maquignon avait, à sa connaissance, promis de renvoyer sa troupe, si on lui accordait un sauf-conduit ; mais il ne s’ensuivait point que l’on dût lui accorder une amnistie pour les atrocités dont il s’était chargé, deux choses que le docteur Luther ainsi que le prince ne devaient point confondre. « Si sa plainte contre le gentilhomme a été rejetée, ajouta-t-il en posant l’index sur le bout de son nez, cela n’excuse ni ses meurtres, ni ses brigandages. »\par
Ce sage stratagème satisfit également tous les assistants, et il méritait certainement l’approbation du monde et de la postérité.\par
L’électeur, voyant que le prince ainsi que le chambellan ne répondaient à ce discours que par un regard d’approbation, leva la séance en disant qu’il examinerait lui-même jusqu’au prochain conseil les différentes opinions qui venaient d’être débattues.\par
Il paraît que la mesure préliminaire dont le prince avait parlé, étant trop cruelle à son cœur sensible à l’amitié, lui ôta tout désir d’entreprendre l’expédition préparée contre Kohlhaas. Il se tint, au contraire, à l’opinion plus modérée du grand chancelier, comte de Wrede, qui lui fit judicieusement observer que l’armée de Kohlhaas, de quatre cents hommes, ne tarderait pas à tripler, vu le mécontentement général causé par l’injustice et la dureté du chambellan.\par
Se décidant à suivre le conseil de Luther, l’électeur remit toute la conduite du procès qui allait avoir lieu, entre les mains du comte de Wrede.\par
Peu de jours après, on vit paraître l’édit suivant :\par
« Moi, etc., etc., prince électeur de Saxe, en considération de la prière du docteur Martin Luther, j’accorde à Michel Kohlhaas, marchand de chevaux du Brandenbourg, un sauf-conduit pour Dresde, sous la condition qu’il posera les armes d’ici à trois jours et licenciera sa troupe. Dans le cas où il refuserait de profiter de cette grâce pour venir présenter sa plainte devant la cour, il sera poursuivi et puni avec toute la rigueur des lois, pour avoir entrepris de se venger lui-même ; dans le cas contraire, il obtiendra complète amnistie pour lui et pour tous les complices de ses violences. »\par
Kohlhaas n’eut pas plutôt connaissance de cet édit, qu’il congédia ses gens, donnant à chacun de l’argent et des directions. Il laissa tout ce qu’il avait en armes et en équipage de guerre dans le château de Lutzen, comme propriété de l’État, et après avoir remis à Waldmann une lettre adressée à son voisin de Kohlhaasenbruck, pour tenter de racheter sa ferme, et envoyé Sternbald à Schwérin chercher ses enfants, qu’il voulait avoir auprès de lui, il se rendit à Dresde, emportant, en papier, le peu d’argent qui lui restait.
\chapterclose


\chapteropen
\chapter[Chapitre V]{Chapitre V}\renewcommand{\leftmark}{Chapitre V}


\chaptercont
\noindent Le jour commençait à paraître sur les créneaux de la ville, où tout reposait encore, lorsque Kohlhaas frappa à la porte de sa propriété dans le faubourg de Dresde.\par
Grâce à la complaisance de son voisin le bailli, elle lui appartenait encore.\par
Au bout de quelques heures il pria le vieux Thomas, régisseur de la maison, d’aller dire au prince de Meissen que lui Kohlhaas, maquignon, était arrivé.\par
Le prince, se rendant aussitôt à son invitation, arriva accompagné de sa suite et d’une foule nombreuse de curieux. Car la nouvelle s’était déjà répandue que l’ange exterminateur, qui portait partout le fer et le feu, venait d’entrer dans les murs de Dresde.\par
Après avoir pénétré jusqu’à la chambre où Kohlhaas, à demi vêtu, était occupé à déjeûner, le prince lui demanda s’il était le marchand de chevaux.\par
« Oui, » dit Kohlhaas en lui présentant son portefeuille ; et il ajouta qu’il avait congédié sa troupe, et qu’il était venu à Dresde d’après la permission du prince, pour y porter sa plainte contre le gentilhomme de Tronka.\par
Le prince, jetant sur lui un regard pénétrant, le considéra de la tête aux pieds, puis il parcourut les papiers contenus dans le portefeuille, se faisant expliquer ce que signifiaient divers actes, signés du château de Lutzen ; il lui fit ensuite des questions sur ses enfants, sur sa fortune, sur le genre de vie qu’il comptait mener à l’avenir ; et s’étant assuré par toutes ses réponses que l’on n’avait plus rien à craindre de lui, il lui rendit son portefeuille, et lui dit que son procès commencerait dès qu’il aurait parlé au grand chancelier du tribunal, le comte de Wrede. « Pour le moment, ajouta-t-il en s’approchant de la fenêtre, et en regardant la foule qui s’était assemblée devant la maison, je vais te laisser une garde ; tu en as besoin pour ta sûreté ici, aussi bien que pour t’accompagner lorsque tu sortiras.\par
— Mais, dit Kohlhaas d’un air incertain, me donnez-vous votre parole de la supprimer dès que j’en exprimerai le désir ? »\par
Le prince répondit que cela allait sans dire ; et, lui présentant trois de ses lansquenets, il leur dit que l’homme auprès duquel il les laissait était libre, et que leur devoir était de le protéger contre les insultes du peuple. Puis, saluant Kohlhaas, il s’éloigna.\par
Vers midi, Kohlhaas, accompagné de ses trois lansquenets, et suivi d’une foule innombrable qui, le voyant bien gardé, n’osait lui faire aucun mal, se rendit chez le chancelier du tribunal. Celui-ci, après l’avoir introduit avec beaucoup de bonté dans sa chambre d’audience, s’entretint avec lui pendant deux heures de tout ce qui s’était passé depuis l’origine de sa dispute avec le gentilhomme jusqu’à ce jour, puis il l’adressa, pour la rédaction de sa plainte, à l’un des plus célèbres avocats de la ville.\par
Cependant le gentilhomme, sommé de venir répondre à la plainte portée contre lui par Michel Kohlhaas, fut tiré de sa prison de Wittemberg, et ne tarda pas à arriver chez ses cousins Hinz et Kunz, où il fut reçu avec la plus grande amertume et le plus profond mépris. Ils le nommèrent un misérable et un indigne, qui avait apporté la honte sur toute sa famille, et le prévinrent qu’il perdrait immanquablement son procès, et qu’ils lui conseillaient de se préparer à remplir ses devoirs de palefrenier.\par
Le gentilhomme répondit d’une voix faible et tremblante qu’il était le plus malheureux des hommes ; il jura n’avoir rien su de toute cette affaire, que le châtelain et l’intendant avaient seuls conduite ; et, se jetant sur une chaise, il les pria de ne point l’accabler de reproches inutiles, qui ne servaient qu’à rendre ses maux encore plus insupportables.\par
Le lendemain, les seigneurs de Tronka envoyèrent chez les fermiers de Tronkenbourg pour avoir des nouvelles des chevaux oubliés depuis l’incendie du château. Mais tout ce qu’ils purent apprendre des habitants des environs, fut qu’un valet avait été contraint à les sauver des flammes par l’incendiaire lui-même. La vieille intendante goutteuse, qui s’était enfuie à Meissen, assura que le domestique était sorti des frontières avec les chevaux, le lendemain de cet horrible jour. Des hommes de Dresde, qui avaient passé à Wildsruf quelques jours après l’incendie, dirent qu’ils y avaient rencontré un garçon avec deux chevaux éthiques qui, ne pouvant aller plus loin, avaient été vendus à un berger. Un messager, envoyé aussitôt à Wildsruf, rapporta la nouvelle que le berger les avait déjà revendus on ne savait à qui, et que le bruit courait même qu’ils étaient morts et enterrés à la voirie de Wildsruf.\par
On comprend aisément que c’était la chose que pouvaient le plus désirer les seigneurs de Tronka, qui avaient craint (leur cousin se trouvant sans écurie) que les chevaux ne fussent mis dans une des leurs pour y être restaurés.\par
Ils désirèrent avoir une certitude entière à cet égard ; c’est pourquoi le gentilhomme Wenzel de Tronka adressa, comme seigneur féodal et justicier, une lettre au juge de Wildsruf, où il donnait la description exacte des chevaux de Kohlhaas, et lui ordonnait de les chercher dans le village, et, s’ils s’y trouvaient encore, de les faire conduire chez le chambellan Kunz à Dresde.\par
Peu de jours après, un homme arriva sur la place du marché, traînant derrière sa charrette deux chevaux maigres et exténués. Le malheur du gentilhomme, et encore plus celui de Kohlhaas, voulut que ce fussent les chevaux de ce dernier, qui étaient tombés entre les mains de l’écorcheur de Dobbeln.\par
Les seigneurs de Tronka, instruits de l’arrivée de cet homme, se rendirent sur la place du marché, suivis de plusieurs cavaliers.\par
Le gentilhomme eut à peine aperçu les chevaux, qu’il dit, d’un air troublé, que ce n’étaient pas ceux de Kohlhaas. Mais le seigneur Kunz, jetant sur lui un regard plein de colère, s’avança vers l’écorcheur, et ouvrant son manteau pour lui laisser voir ses ordres et sa dignité, lui demanda si c’étaient là les chevaux qui avaient été vendus par le berger de Wildsruf.\par
L’écorcheur, très occupé à donner à boire au cheval gras et robuste qui était attelé à la charrette, répondit sans se déranger :\par
« Les noirs qui sont attachés là-derrière, je les ai achetés à un gardeur de pourceaux ; » puis, se baissant pour reprendre le seau qu’il avait posé devant sa bête, il ajouta que le maire de Wildsruf lui avait ordonné de les amener chez le seigneur Kunz de Tronka.\par
À ces mots, il se releva, et répandit dans la rue toute l’eau qui restait dans le seau.\par
Le chambellan, voyant que les manières de cet homme excitaient la risée du peuple, lui dit qu’il était lui-même le seigneur Kunz de Tronka, et que les chevaux qu’il avait amenés devaient, après avoir été sauvés de l’incendie de Tronkenbourg, avoir été vendus à un berger de Wildsruf, duquel les tenait sans doute le marchand de pourceaux.\par
Le rustre, replaçant le seau sur sa charrette, répondit qu’il remettrait les chevaux contre l’argent qui lui avait été promis ; que, du reste, il ne savait rien de ce qui s’était passé auparavant, ni si le marchand de cochons les tenait de Pierre, de Paul, ou du berger de Wildsruf ; qu’il lui suffisait de savoir qu’il ne les avait pas volés ; et enfilant son fouet dans sa ceinture, il se dirigea vers un cabaret voisin. Le chambellan, qui pensait bien que ces chevaux ne pouvaient être que ceux par qui le diable était entré dans la Saxe, retint l’écorcheur, et somma son cousin de s’expliquer. Celui-ci dit en tremblant de tous ses membres, que le plus prudent serait d’acheter les chevaux, qu’ils fussent ou non ceux de Kohlhaas. Le seigneur Kunz, maudissant le père et la mère qui l’avaient engendré, se tourna vers la foule, tout à fait incertain sur ce qu’il devait faire. Trop orgueilleux pour quitter la place où il voyait bien que le peuple n’attendait que son départ pour rire de lui, il appela le baron de Wenk, un de ses amis, qui passait dans la rue, et le pria de se rendre aussitôt chez le comte de Wrede, pour le prier d’amener Kohlhaas sur la place du marché.\par
Kohlhaas était précisément en conférence avec le comte de Wrede, lorsque le baron entra dans le cabinet du chancelier. Celui-ci, mettant de côté les papiers qu’il examinait, se leva d’un air impatient. Le baron lui exposa la situation dans laquelle se trouvaient les seigneurs de Tronka, et dit que l’écorcheur de Dobbeln était arrivé avec des chevaux dans un état si déplorable, que le gentilhomme ne pouvait les reconnaître pour ceux du marchand. « Ayez donc la bonté, ajouta-t-il, de faire prendre le maquignon chez lui, pour qu’il soit conduit sur la place du marché. »\par
Le grand chancelier, ôtant ses lunettes, répondit au baron qu’il était doublement, dans l’erreur ; premièrement, s’il croyait qu’il n’y eût pas d’autre moyen de se tirer d’embarras que l’inspection oculaire de Kohlhaas, et secondement, s’il se figurait que lui, grand chancelier, se croirait obligé de faire conduire Kohlhaas partout où ce serait le bon plaisir du gentilhomme. Puis, lui présentant le maquignon, qui s’était retiré à l’écart, il le pria de lui faire sa commission en personne.\par
Kohlhaas, sans rien laisser voir de ce qui se passait dans son âme, dit qu’il était prêt à le suivre ; et s’approchant de la table, devant laquelle le chancelier avait repris sa place, il rassembla ses papiers dans son portefeuille, tandis que le baron le considérait en ouvrant de grands yeux. Ensuite ils se rendirent, accompagnés des trois lansquenets, sur la place en question.\par
Le chambellan, qui avait avec peine conservé son sang-froid en présence du peuple, s’avança vers eux dès qu’il les aperçut, et demanda à Kohlhaas, en lui montrant la charrette, si c’étaient là ses chevaux.\par
Le marchand, après avoir tiré son chapeau devant le seigneur qu’il ne connaissait pas, jeta les yeux sur les pauvres bêtes qui, la tête basse, les jambes faibles et tremblantes, regardaient tristement et sans le manger, le foin qui était devant elles.\par
« Ce sont bien mes chevaux, dit-il ; puis, saluant encore une fois, il se mêla à la foule. Le chambellan, s’approchant d’un pas fier vers l’écorcheur, lui jeta une bourse, qu’il releva sans cesser de se gratter la tête avec un vieux peigne de plomb. Le seigneur Kunz appela l’un de ses valets, et lui ordonna de détacher les bêtes et de les emmener chez lui. Celui-ci, à l’appel de son maître, sortit d’une bande d’amis et de parents qu’il avait trouvés dans la foule ; mais à peine avait-il saisi le licol, que maître Himbold, son cousin, vint le prendre par le bras, et l’entraînant loin de la charrette, s’écria qu’il ne devait point toucher à ces bêtes éthiques. Il s’approcha ensuite du chambellan, qui était resté muet d’indignation, et il lui dit qu’il pouvait chercher une autre personne pour lui rendre ce service. Le chambellan, écumant de rage, se jeta sur maître Himbold, et le saisissant à la gorge, lui demanda de quel droit il empêchait ses valets de remplir leur devoir.\par
« Noble seigneur, répondit le maître en faisant un effort qui le délivra des mains du chambellan, un garçon de vingt ans est en âge de savoir ce qu’il doit faire, sans que personne ait besoin de l’influencer. Demandez-lui s’il veut seulement toucher les chevaux attachés à cette charrette. S’il le veut après ce que je lui ai dit, ainsi soit-il ; mais, à mon avis, il fera bien de les faire écorcher au plus tôt. »\par
À ces mots, le chambellan, se tournant avec dignité vers son valet, lui demanda s’il était décidé à suivre ses ordres, et à conduire les chevaux jusqu’à ses écuries. Le jeune homme, murmurant quelques invectives contre ces bêtes du diable, tourna le dos à son maître, qui, transporté de colère, le poursuivit dans la foule, lui arracha les armoiries de sa maison qu’il portait à son chapeau, et le chassa, à coups de plat de sabre, de son service et de la place. Maître Himbold, s’élançant sur le chambellan, le renversa. En vain le gentilhomme Wenzel, tout en cherchant à s’échapper de la mêlée, cria-t-il aux chevaliers de secourir son cousin ; avant qu’ils eussent fait un pas pour cela, le peuple était acharné sur le seigneur Kunz, qui ne dut la vie qu’à l’arrivée fortuite d’une bande d’archers. L’officier, après avoir dispersé la foule, arrêta maître Himbold, qui fut conduit en prison, tandis que le chambellan, couvert de sang, fut emporté au château par deux amis.\par
C’est ainsi qu’un malheureux destin semblait attaché à toutes les tentatives justes et raisonnables que faisait Kohlhaas pour obtenir le droit qui lui avait été refusé.\par
L’écorcheur de Dobbeln ayant fini son affaire, et ne voulant pas s’arrêter davantage, attacha les chevaux à une borne, où ils restèrent exposés aux railleries de tous les bandits et des garçons de rues jusqu’à \emph{ce} que la police ayant trouvé convenable de s’en occuper, les fit prendre par un écorcheur de la ville.\par
Il paraissait tout à fait invraisemblable que les chevaux pussent jamais être remis en état de rentrer à l’écurie de Kohlhaasenbruck, et supposé que cela eût été possible, il en serait résulté une si grande honte pour la famille du gentilhomme, qui était une des premières et des plus nobles de l’État, qu’il semblait beaucoup plus sage d’offrir à Kohlhaas une indemnité en argent. Celui-ci n’attendait plus que les ouvertures du gentilhomme ou de ses parents pour lui accorder pardon et oubli de tout ce qui s’était passé.\par
Mais c’était précisément pour faire ces ouvertures qu’il en coûtait à l’orgueil des chevaliers de Tronka.\par
Le chambellan, encore aigri par les blessures qu’il avait reçues, se plaignit au prince de ce qu’après avoir exposé sa vie pour faire aller les choses selon ses vœux, il se voyait encore obligé de sacrifier son honneur en s’abaissant jusqu’à la prière devant un homme qui n’avait attiré sur sa famille que honte et que ruine.
\chapterclose


\chapteropen
\chapter[Chapitre VI]{Chapitre VI}\renewcommand{\leftmark}{Chapitre VI}


\chaptercont
\noindent C’est là qu’en étaient les choses à Dresde, lorsqu’un orage se formant à Lutzen, vint fondre sur la tête du malheureux Kohlhaas, et ranimer les espérances des seigneurs de Tronka, qui résolurent d’en profiter pour le perdre.\par
Jean Nagelschmidt, l’un des hommes réunis par Kohlhaas et congédiés par lui à l’apparition de l’amnistie, avait trouvé bon de rassembler de nouveau, sur les frontières de la Bohème, une partie de ses anciens camarades, et de faire pour son propre compte le métier que lui avait enseigné le maquignon. Ce misérable, pour imposer aux coquins qui se joignaient à lui et donner plus d’éclat à ses brigandages, se disait le défenseur de Kohlhaas. Il prétendait que l’amnistie promise à ceux qui retourneraient tranquillement dans leurs foyers n’avait point été tenue, et que Kohlhaas lui-même, par la plus lâche perfidie, avait été arrêté dès son arrivée à Dresde et mis entre les mains d’une garde. Dans des placards semblables à ceux de Kohlhaas, il invitait les chrétiens à venir se joindre à lui pour veiller à l’exécution de l’amnistie promise par le prince. Dans le fait, il ne s’intéressait nullement au sort de Kohlhaas, et tout cela n’était qu’un prétexte pour pouvoir se livrer de nouveau au désordre et au pillage.\par
Les chevaliers ne purent cacher leur joie à la pensée de la nouvelle face qu’allait prendre toute cette affaire. Ils répandirent le bruit que Nagelschmidt avait pris les armes d’accord avec Kohlhaas, que celui-ci, après un faux semblant de soumission, avait caché sa troupe dans les forêts des environs, où elle n’attendait qu’un signal de lui pour en sortir de nouveau avec le fer et le feu.\par
Le prince Christiern de Meissen, très mécontent de cette tournure des choses, voulut avoir un entretien avec Kohlhaas.\par
Il le fit demander, et le marchand s’y rendit avec ses deux fils que Sternbald lui avait ramenés du Mecklembourg.\par
Après lui avoir fait quelques questions sur l’âge et le nom de ses enfants, le prince s’ouvrit à lui sur la rébellion de son ancien domestique Nagelschmidt, et lui présentant le mandat de cet homme, il lui demanda ce qu’il avait à dire pour sa justification.\par
Quelque vif et profond effroi qu’il ressentît à la vue de ce papier, le maquignon eut cependant peu de peine à se justifier devant un homme aussi juste que le prince, des préventions qui l’accusaient. Quelques papiers qu’il avait sur lui, lui prouvèrent aussitôt l’invraisemblance de son accord avec Nagelschmidt, qu’il avait résolu de faire pendre à Lutzen pour le punir de son insubordination, lorsque l’amnistie avait paru. Ils s’étaient quittés ennemis mortels.\par
Kohlhaas, sur l’invitation du prince, écrivit sous ses yeux une lettre à Nagelschmidt pour lui représenter toute l’indignité de sa rébellion, et l’avertir qu’en réunissant de nouveau ses gens après la publication de l’amnistie, il avait attiré sur lui toute la colère des lois.\par
Le prince ayant rassuré Kohlhaas en lui rappelant que tant qu’il serait à Dresde l’amnistie ne pouvait être rompue, il baisa encore une fois les enfants, leur donna quelques fruits qui étaient sur la table, serra la main du marchand et le salua.\par
Tous les efforts du grand chancelier pour terminer ce procès avant que quelque nouvelle charge contre Kohlhaas vînt aggraver sa cause, furent paralysés par ceux des seigneurs de Tronka, dont le but était au contraire de le traîner en longueur. Renonçant à l’aveu muet de la faute qu’ils avaient opposé jusqu’alors à l’accusation, ils commencèrent à la nier entièrement. Ils prétendirent, tantôt que les chevaux de Kohlhaas avaient été retenus au château sans le consentement du gentilhomme, par la seule volonté du châtelain et de l’intendant, tantôt qu’ils avaient été attaqués d’une violente maladie, peu après leur établissement à Tronkenbourg, et enfin ils produisirent un édit par lequel, douze ans auparavant, le passage des chevaux du Brandenbourg en Saxe avait été momentanément défendu à cause d’une maladie du bétail. Par ce document, clair comme le jour, la compétence du gentilhomme pour arrêter les chevaux sur la frontière se trouvait pleinement établie.\par
Kohlhaas ayant reçu de son digne voisin de Kohlhaasenbruck la permission de reprendre sa ferme, sous la condition d’un petit dédommagement, imagina de se servir du prétexte que sa présence était nécessaire pour terminer cet arrangement, afin de forcer ses juges à prendre une décision et à prononcer sur son destin.\par
Il se rendit chez le grand chancelier, et, lui montrant la lettre de son voisin, il dit que sa présence ne paraissant pas nécessaire à Dresde dans ce moment, il désirait aller passer à Kohlhaasenbruck huit ou dix jours, au bout desquels il promettait d’être de retour.\par
Le grand chancelier, prévoyant tout le tort que pourrait lui faire une absence de quelques jours, dans de pareilles circonstances, lui répondit d’un air mécontent que sa présence était plus nécessaire que jamais pour que le jugement ne se prononçât pas en faveur de ses adversaires.\par
Mais Kohlhaas l’ayant assuré qu’il avait une entière confiance en son avocat, et renouvelant sa demande, le grand chancelier, après une pause, lui dit qu’il n’avait qu’à demander un permis au prince de Meissen.\par
Kohlhaas, qui lisait dans le cœur du grand chancelier, s’affermit toujours davantage dans sa résolution, et se plaçant à sa table, il écrivit au prince de Meissen, comme chef du gubernium, pour obtenir de lui la permission de se rendre, pour quelques jours, à Kohlhaasenbruck.\par
Il reçut du baron Siegfried de Wenk, qui remplaçait le prince de Meissen au gubernium, pendant son séjour dans ses terres, la réponse que son désir serait exposé à son altesse le prince électeur, dont on lui ferait connaître la volonté.\par
Kohlhaas, dont le cœur commençait à battre avec inquiétude, attendit quelques jours la décision du prince ; une semaine s’étant écoulée tout entière sans qu’il reçût aucun message de la cour, il se décida à renouveler sa demande.\par
Mais quelle fut sa surprise lorsque, le soir du jour suivant, après avoir vainement attendu la réponse désirée, il vit de sa fenêtre que sa garde avait abandonné le pavillon qui lui avait été assigné pour demeure.\par
Thomas, qu’il appela pour lui demander ce que cela signifiait, répondit en soupirant :\par
« Monsieur, tout ne va pas comme cela devrait aller. Les lansquenets sont plus nombreux aujourd’hui ; ils se sont dispersés tout à l’entour de la maison. Il y en a deux, armés de lances, à la porte de la rue, deux à celle du jardin, et trois autres se sont établis dans l’antichambre, où ils prétendent passer la nuit. »\par
Kohlhaas devint pâle comme la mort. Il entendit au même instant un cliquetis d’armes, qui lui prouva la vérité de ce que venait de lui dire le vieux Thomas.\par
Quelque peu d’envie qu’il eût de dormir, il se mit au lit, où sa résolution fut bientôt prise pour le lendemain. Rien ne lui déplaisait plus, dans le gouvernement, que l’apparence de justice sous laquelle l’amnistie était rompue indignement. S’il était réellement prisonnier, ce qui semblait hors de doute, il voulait savoir pourquoi.\par
Le lendemain matin, il fit atteler sa voiture, disant qu’il voulait aller dîner à Lokwitz, chez un de ses anciens amis, qu’il avait rencontré à Dresde, et qui l’avait invité à le visiter avec ses enfants.\par
Les lansquenets, voyant ses préparatifs, envoyèrent secrètement un des leurs à la ville, et peu de minutes après, Kohlhaas, qui paraissait tout occupé de l’habillement de ses enfants, remarqua quelques archers, qui entrèrent avec leur officier dans la maison en face de la sienne.\par
Il fit approcher sa voiture, y plaça ses deux fils, et après avoir consolé ses petites filles, auxquelles il avait ordonné de rester avec la fille du vieux Thomas, il y monta lui-même, en disant aux lansquenets qu’ils n’avaient pas besoin de l’accompagner. Mais à peine était-il assis, que l’officier des archers, sortant avec sa suite de la maison où il était entré, vint lui demander où il allait, et pourquoi il ne se faisait pas suivre de la garde que lui avait donnée le prince de Meissen.\par
Kohlhaas répondit en souriant qu’il allait chez un ami, dans la maison duquel il serait parfaitement en sûreté, et qu’il voulait profiter de la liberté que le prince lui avait accordée de ne plus se servir de la garde, dès qu’il le trouverait convenable.\par
L’officier prétendit que le baron de Wenk, qui était en ce moment le chef de la police, lui avait ordonné de le faire garder soigneusement, et il le pria, puisqu’il ne voulait pas se faire accompagner de ses lansquenets dans sa partie de plaisir, de venir avec lui au gubernium pour éclaircir le malentendu qui, sans doute, causait ce conflit.\par
Kohlhaas, impatient de voir enfin son sort se décider, lui dit qu’il était prêt à le suivre. Le cœur vivement ému, il fit rentrer les enfants, et se rendit avec l’officier au gubernium.\par
Le baron de Wenk se trouvait en ce moment même en conférence avec un des hommes de la bande de Nagelschmidt que l’on venait d’arrêter. Dès qu’il aperçut Kohlhaas, il lui demanda durement ce qu’il voulait ; et celui-ci lui exposa humblement le désir qu’il avait d’aller à Lokwitz sans être accompagné des lansquenets. Le baron, changeant de couleur, lui répondit qu’il ferait bien de se tenir tranquille dans sa maison, et de renoncer à la bonne chère qu’il comptait faire chez son ami.\par
Puis se tournant vers l’officier, il lui rappela qu’il avait reçu l’ordre de veiller sur cet homme, et de ne le laisser sortir que sous la garde de six lansquenets armés.\par
« Quoi ! s’écria Kohlhaas, suis-je donc prisonnier, et dois-je croire que l’amnistie qui m’a été accordée à la face du monde soit si indignement violée ?\par
— Oui, oui, oui, » lui répondit le baron d’un air emporté ; puis lui tournant le dos, il retourna auprès de l’homme de Nagelschmidt.\par
Kohlhaas quitta la salle rempli de tristesse ; car il voyait bien qu’il venait de perdre le dernier espoir de se sauver par la fuite. Cependant il se félicitait intérieurement de se voir délivré de l’obligation de rester fidèle aux articles de l’amnistie.\par
Nagelschmidt, vivement repoussé de tous les côtés dans les vallées de l’Erzgebirge, prêt à succomber, et privé de tout secours, tenta d’intéresser Kohlhaas à son destin.\par
Étant instruit de la manière dont il était traité à la cour, il pensa qu’il ne lui serait pas difficile de l’engager à changer l’inimitié qui régnait entre eux en une nouvelle alliance.\par
Il lui envoya, par un de ses hommes, une lettre à peine lisible, où il lui offrait, à condition qu’il viendrait à Altembourg se remettre à la tête de sa troupe, tous les moyens de s’échapper de Dresde ; il lui promettait d’être à l’avenir plus soumis qu’auparavant, et de lui donner la première preuve de sa fidélité en venant lui-même l’arracher de sa prison.\par
Le jeune homme, porteur de cette lettre, eut le malheur d’être attaqué d’une fièvre violente, dans un village voisin de Dresde. Pendant le cours de sa maladie, la lettre tomba entre les mains des gens qui le secouraient, et dès qu’il fut rétabli, il fut arrêté et conduit par la garde au gubernium.\par
Le baron de Wenk, instruit de cette circonstance, se rendit chez le prince électeur, où se trouvaient réunis les seigneurs Kunz et Hinz de Tronka et le président de la chancellerie, comte de Kallheim. Ces messieurs furent de l’avis qu’il fallait aussitôt arrêter Kohlhaas, et porter contre lui une grave accusation pour ses secrètes relations avec Nagelschmidt, considérant que la lettre écrite par ce dernier ne pouvait avoir été que la suite d’une alliance antérieure avec le maquignon.\par
Le prince électeur se refusait encore à rompre l’amnistie accordée par lui à Kohlhaas ; il lui semblait au contraire que cette lettre établissait la probabilité qu’il n’existait aucune alliance entre lui et Nagelschmidt, et il résolut qu’avant de rien entreprendre on lui ferait remettre la lettre, et que l’on déciderait de son sort d’après sa réponse.\par
Le lendemain le jeune homme fut tiré de sa prison, conduit au gubernium, où le baron lui remit sa lettre ; et sous la promesse de le soustraire au châtiment qui l’attendait, il lui ordonna de la porter à Kohlhaas, comme si rien n’était arrivé.\par
Le maquignon, qui, quelques jours auparavant, aurait pour toute réponse livré le messager entre les mains des lansquenets, aigri maintenant par l’injustice du prince qui l’avait fait prisonnier, et persuadé qu’il était perdu sans retour, regarda le jeune homme avec tristesse, et lui demanda de revenir au bout de quelques heures ; puis il écrivit à Nagelschmidt qu’il acceptait sa proposition de reprendre le commandement de sa troupe, qu’il le priait en conséquence de lui envoyer une voiture et deux chevaux dans la ville neuve de Dresde, et deux cavaliers hardis et courageux pour l’aider à se débarrasser de ses lansquenets, dans le cas où il ne pourrait les gagner ; qu’il refusait du reste sa présence à Dresde, la regardant comme inutile et dangereuse. Il joignit à ce billet un rouleau de vingt couronnes d’or pour l’indemniser de ses frais.\par
Le messager étant revenu vers le soir, il lui remit le tout, en le priant de bien remplir son message.\par
Tout à fait indifférent à la rébellion de Nagelschmidt, son intention était de se rendre à Hambourg avec ses cinq enfants, de s’y embarquer pour le Levant, pour les Indes orientales, ou pour toute autre contrée où le soleil luirait sur des hommes différents de ceux qu’il connaissait ; car l’affaire de ses chevaux avait rempli son âme d’amertume et de haine contre l’humanité.\par
À peine la réponse de Kohlhaas fut-elle parvenue au château, qu’il fut arrêté par un ordre émané du cabinet du prince électeur, chargé de lourdes chaînes, et conduit dans la tour. Interrogé sur sa lettre à Nagelschmidt, il ne put nier qu’il l’avait écrite, et n’ayant rien à dire pour sa justification, il fut condamné à la marque et aux galères.\par
Ce fut à cette époque que l’électeur de Brandenbourg, animé du désir de sauver Kohlhaas, adressa à la cour de Saxe un édit par lequel il réclamait son sujet, le maquignon de Kohlhaasenbruck.\par
Le vaillant capitaine Henri de Geusau l’avait instruit depuis peu de l’histoire de cet homme extraordinaire, et de la faute dont s’était rendu coupable son archichancelier, le comte de Kallheim. Le prince, indigné de la complicité de ce parent du gentilhomme, l’avait aussitôt disgracié et remplacé par Henri de Geusau, qu’il chargea du soin de secourir Kohlhaas.\par
Celui-ci, rempli de pitié pour le malheureux auquel il s’était toujours intéressé, résolut d’employer tous ses moyens à le sauver.\par
Il demanda, au nom de son prince et des lois divines et humaines, qu’on lui livrât Kohlhaas pour qu’il fût puni des forfaits dont il était accusé, selon les lois du Brandenbourg ; de plus, il réclamait la permission d’envoyer à la cour de Dresde un procureur qui plaiderait de nouveau, au nom de Kohlhaas, et lui ferait obtenir justice pour la malheureuse affaire des chevaux.\par
Après un premier refus, l’archi chancelier de Geusau déclara que son prince saurait soutenir ses droits, que Kohlhaasenbruck était sur le territoire brandenbourgeois, et que la sentence exécutée contre l’un de ses sujets serait regardée comme une atteinte aux droits des nations.\par
L’électeur de Saxe, effrayé par la nouvelle de l’alliance que la couronne de Pologne venait de former contre lui avec la cour de Berlin, trouva prudent, ainsi que le chambellan Kunz et le prince Christiern, de consentir à ce que demandait Henri de Geusau.\par
Kohlhaas fut cédé à la cour de Berlin qui, après s’être informée de l’accusation portée contre le maquignon, résolut d’en appeler à l’empereur, et lui envoya pour cela une relation détaillée de la guerre de Kohlhaas dans la Saxe, et de la rupture indigne de l’amnistie qui lui avait été accordée.\par
Huit jours après, le maquignon partit de Dresde avec ses cinq enfants, escorté par le chevalier Frédéric de Malzahn, que l’électeur de Brandenbourg lui avait envoyé avec six chevaliers.
\chapterclose


\chapteropen
\chapter[Chapitre VII]{Chapitre VII}\renewcommand{\leftmark}{Chapitre VII}


\chaptercont
\noindent Le comte Aloyse de Kallheim, possesseur d’une vaste propriété sur les frontières de la Saxe, avait invité son gracieux seigneur à venir honorer de sa présence une grande partie de chasse à laquelle devait assister toute la cour. Des tentes dressées sur le penchant d’une colline au bord de la route de Dahne offraient un abri contre l’ardeur du soleil à la brillante société qui s’y réunissait pour se reposer des fatigues de la chasse, et pour y savourer, au son joyeux de mille instruments, les douceurs d’un repas champêtre.\par
Le prince électeur, la poitrine à demi découverte, et le chapeau orné d’une branche verte, selon la mode des chasseurs, était nonchalamment assis à côté de dame Héloïse, la femme du chambellan Hanz, qui quelques années auparavant avait été l’objet de ses premières amours.\par
« Buvons à la santé du malheureux qui passe sur la grande route, quel qu’il puisse être, » dit-il à la noble dame en lui présentant une coupe, et lui montrant la voiture escortée de cavaliers qui passait lentement le long des tentes.\par
Dame Héloïse, jetant sur lui un regard plein d’admiration et de respect, se leva pour répondre à son invitation, lorsque le comte de Kallheim s’approcha d’un air embarrassé, et dit en balbutiant que l’homme qui passait en voiture n’était autre que Michel Kohlhaas. Tout le monde fut étonné, parce que l’on savait qu’il avait quitté Dresde six jours auparavant.\par
Le chambellan se hâta de renverser sa coupe sur la terre, et le prince posa la sienne en rougissant.\par
Le chevalier de Malzahn ayant salué avec respect la compagnie qu’il ne connaissait pas, les convives reprirent le cours de leurs plaisirs, sans s’inquiéter davantage de l’infortuné maquignon, dont le voyage avait été si fort prolongé par la maladie d’un de ses enfants.\par
Vers le soir, toute la société s’étant dispersée pour jouir du spectacle d’un cerf aux abois, dame Héloïse, appuyée sur le bras du prince, s’égara jusqu’à la chaumière où Kohlhaas et son escorte s’étaient arrêtés pour la nuit. Dame Héloïse, très curieuse de connaître cet homme extraordinaire, entraîna le prince en l’assurant qu’il était méconnaissable dans ses habits de chasse. Celui-ci, incapable de résister à ses instances, enfonça son chapeau sur ses yeux, et disant avec amour : « Folie, tu gouvernes le monde, et ton trône est la bouche d’une belle femme, » il entra avec elle dans la maison.\par
Kohlhaas, assis sur un tas de paille, le dos appuyé contre la muraille, tenait son enfant malade dans ses bras, et lui donnait à manger, lorsque la noble dame, s’approchant, lui adressa plusieurs questions, auxquelles il répondit d’une manière brève, mais satisfaisante.\par
Le prince, qui ne savait que lui dire, ayant remarqué un petit étui de plomb suspendu à son cou par un cordon de soie, lui demanda ce qu’il contenait.\par
« Cet étui, dit Kohlhaas, renferme un petit billet cacheté que je reçus d’une manière bien étrange, il y a environ six mois, lorsqu’après avoir quitté Kohlhaasenbruck pour marcher à la recherche du gentilhomme qui m’a fait tant de mal, comme vous le savez peut-être, je passai à Juterbok. Le prince électeur de Saxe et le prince de Brandenbourg s’y trouvaient réunis. Un soir qu’ils se promenaient dans la ville pour jouir de la vue de la foire qui avait lieu en ce moment, ils virent une magicienne montée sur une banquette, prédisant l’avenir au peuple qui l’entourait. Ils lui demandèrent en plaisantant si elle n’avait rien à leur annoncer. J’étais trop loin pour entendre ce qui fut dit entre eux, et je montai sur un banc qui se trouvait derrière moi, moins par curiosité que pour faire place à ceux qui me poussaient.\par
» À peine fus-je dans cette position, qui m’exposait entièrement à la vue de cette femme, qu’elle descendit de sa banquette, s’élança vers moi au travers de la foule, et me remit ce petit billet cacheté, me disant que c’était une amulette que je devais conserver soigneusement, parce qu’elle me sauverait la vie.\par
» C’est sûrement à elle que je dois de n’avoir point péri à Dresde, et peut-être me préservera-t-elle encore à Berlin. »\par
À ces mots, le prince s’assit en pâlissant, et dame Héloïse lui demandant ce qu’il avait, il ne put répondre, et tomba sans connaissance avant qu’elle eût le temps de s’élancer à son côté et de le soutenir dans ses bras.\par
Des chasseurs le relevèrent et le mirent sur un lit. Le trouble fut à son comble lorsque le chambellan, qu’on avait envoyé chercher, après avoir fait toutes les tentatives pour le rappeler à la vie, dit qu’il semblait frappé de la foudre.\par
Il le fit transporter à pas lents jusqu’à la maison du comte de Kallheim, et le médecin, arrivé le lendemain matin, déclara qu’il avait tous les symptômes d’une fièvre nerveuse.\par
Dès qu’il fut mieux, sa première question concerna Kohlhaas. Le chambellan, se méprenant sur son sentiment, lui serra la main avec affection, et lui assura qu’il pouvait être parfaitement tranquille, cet homme devant être déjà hors de la Saxe ; puis il lui demanda ce qu’avait pu lui dire Kohlhaas pour le jeter dans cet état.\par
Le prince lui parla de l’étui que portait le maquignon, et lui assura qu’il était la seule cause de tout son mal. Puis il le supplia, en lui saisissant la main, de lui faire avoir cet objet, dont la possession était pour lui de la plus grande importance.\par
Le chambellan, ne comprenant rien au désir de son maître, dit qu’il n’y avait aucun moyen de s’en emparer, Kohlhaas n’étant probablement plus en Saxe. Puis voyant que le prince se cachait avec désespoir dans ses coussins, il lui demanda ce que contenait cet étui et par quel hasard il en avait eu connaissance. Le prince, blessé de la froideur du chambellan, ne lui répondit point, et, les yeux fixés sur le mouchoir de poche qu’il tenait à la main, il lui ordonna d’appeler un jeune chasseur dont il s’était déjà souvent servi pour des commissions délicates.\par
Exposant à ce jeune homme toute l’importance qu’il attachait à la possession de l’étui de Kohlhaas, il lui demanda s’il voulait gagner un droit éternel à sa reconnaissance en cherchant à s’en rendre maître avant que Kohlhaas eût atteint Berlin.\par
Le chasseur, sans se laisser effrayer par la singularité de cette commission, l’assura qu’il était entièrement dévoué à son service.\par
Le prince lui remit une attestation de sa main par laquelle il offrait à Kohlhaas la liberté et la vie s’il voulait lui livrer le billet que contenait l’étui de plomb.\par
Ayant eu le bonheur d’atteindre Kohlhaas dans un village voisin de la frontière où il s’était arrêté pour dîner, le jeune homme trouva le moyen de s’introduire auprès de lui et de lui faire part des propositions du prince. Mais le maquignon, qui connaissait maintenant le nom et le rang du seigneur qui s’était trouvé mal à la vue de son amulette et à l’ouïe de son récit, répondit, avec beaucoup de calme, qu’il ne tenait plus à la vie et qu’il préférait garder le billet. « Le prince a pu me faire marcher à l’échafaud, ajouta\emph{}-t-il, maintenant je puis à mon tour lui causer du chagrin, et j’en jouis. »\par
L’état du prince, à cette nouvelle, empira tellement, que le médecin désespéra de sauver ses jours. Cependant, grâce à la force de sa constitution, il se trouva au bout de quelques semaines convalescent et en état d’être conduit à Dresde.\par
Dès qu’il fut arrivé dans sa capitale, il fit appeler le prince Christiern de Meissen et lui demanda où en était l’affaire du maquignon. Celui-ci lui répondit que le conseiller Eibenmayer était parti pour Vienne, selon ses ordres, dès l’arrivée du savant avocat que l’électeur de Brandenbourg avait envoyé à Dresde pour attaquer le gentilhomme au nom de Kohlhaas ; et comme le prince montra du mécontentement de ce que l’on eût suivi ses ordres si ponctuellement il ajouta que le conseiller s’était empressé d’accuser Kohlhaas, devant la cour de Vienne, d’avoir troublé la paix du royaume, afin de prévenir la condamnation qui était près d’accabler le gentilhomme de Tronka.\par
L’électeur, se tournant pour cacher au prince Christiern ce qui se passait dans son âme, avoua qu’il n’avait rien à redire à cela ; et après lui avoir demandé avec indifférence ce qui s’était passé dans la ville pendant son absence, il le congédia.\par
Le même jour il écrivit à l’empereur une lettre particulière pour le supplier de la manière la plus persuasive, pour des raisons qu’il lui dirait plus tard, de vouloir bien lui faire la grâce d’ajourner le procès de Kohlhaas.\par
L’empereur lui répondit que le changement survenu dans ses désirs l’étonnait au-delà de toute expression ; mais que le maquignon étant cité au tribunal de l’empire comme perturbateur de l’ordre établi, lui, qui en était le chef, l’avait déclaré digne de toute la sévérité des lois, et qu’il venait d’envoyer l’assesseur de la cour, Franz Muller, à Berlin, pour faire accomplir son jugement.\par
Cette lettre abattit entièrement le courage du prince, et il perdit tout espoir en recevant la nouvelle que Kohlhaas avait été condamné à mourir sur l’échafaud. Ne pouvant supporter l’idée de perdre à jamais cet homme, il écrivit au prince de Brandenbourg qu’il ne comprenait pas que le maquignon fût condamné à mort. Il l’assurait que, malgré la sévérité avec laquelle il avait été traité en Saxe, il n’avait jamais eu l’intention de le faire mourir, et qu’il serait inconsolable si la faveur qu’il croyait lui avoir accordée en consentant à ce qu’il fût jugé à Berlin, le conduisait à un sort plus funeste.\par
L’électeur de Brandenbourg lui répondit que l’intervention de l’empereur dans cette affaire ne lui permettait plus d’adoucir le sort de Kohlhaas, et que les progrès de Nagelschmidt, dont les forces augmentaient chaque jour, en menaçant le Brandenbourg, rendaient nécessaire et désirable un acte de sévérité contre l’infortuné maquignon.\par
Le prince, accablé des soucis et du chagrin que lui causait toute cette affaire, tomba de nouveau malade. Le chambellan étant venu le voir, se jeta à ses genoux, et le pria, par tout ce qu’il avait de plus sacré et de plus cher, de lui ouvrir son cœur et de lui confier ce que contenait le billet qu’il désirait tant avoir. L’électeur lui dit de fermer la porte à clef, de s’asseoir sur son lit ; puis, saisissant sa main, qu’il pressa sur son cœur en soupirant, il commença en ces termes :\par
« Ta femme t’a sûrement déjà raconté que, le troisième jour de ma réunion à Juterbok avec le prince électeur de Brandenbourg, nous y rencontrâmes une prophétesse, et que le prince, étourdi comme il est de son naturel, avait aussitôt résolu de consulter cette femme, dans le but d’anéantir, en présence de tout le peuple, la réputation dont elle jouissait. Il lui demanda de lui indiquer, à l’instar de la sibylle romaine, quelque signe de la vérité de ses prédictions.\par
» Après nous avoir mesurés rapidement de la tête aux pieds, elle lui répondit hardiment que le signe auquel il reconnaîtrait la vérité de ses paroles serait la rencontre que nous ferions, en quittant la place, du chevreuil que le fils du jardinier élevait dans le parc du château. Tu dois savoir que cet animal, destiné à la table de la cour, était élevé dans la partie la plus retirée du parc, enfermé par plus d’une porte, et tout à fait dans l’impossibilité de paraître sur la place du marché. Cependant, pour être plus sûr encore de dévoiler ses mensonges, le prince, après m’avoir consulté, envoya au château pour ordonner que le chevreuil fût tué sur le champ, et préparé pour le repas du jour suivant ; puis, se tournant vers la femme, devant laquelle il avait donné ses ordres tout haut, il lui dit : « Voyons maintenant ce que tu as à me prédire. »\par
» La devineresse, regardant dans une de ses mains avec beaucoup d’attention, prononça, d’un air solennel, les paroles suivantes : « Noble prince, ta grâce doit régner longtemps, ta maison se couvrir de gloire, et ta postérité, grande et noble, s’élever à plus de puissance que tous les princes et les seigneurs du monde. »\par
» Le prince, après avoir considéré, tout pensif, les traits de cette femme, me dit à demi-voix qu’il se repentait d’avoir commandé la mort du chevreuil, et tandis que les chevaliers de sa suite, poussant des cris de joie, faisaient pleuvoir l’argent dans une cassette que la sibylle tenait ouverte devant elle, il lui demanda, en lui présentant une pièce d’or, si elle avait à me prédire un aussi beau destin. Au lieu de répondre, elle plaça sa main sur sa figure, pour se préserver du soleil, comme si elle en était incommodée ; elle me regarda, et lorsque je lui eus renouvelé la question du prince, et que je lui eus dit en plaisantant qu’elle paraissait n’avoir rien de bon à m’apprendre :\par
« Non, me dit-elle à l’oreille, d’un ton plein de mystère.\par
— Quoi ! m’écriai-je tout troublé, en faisant deux pas vers cette figure, dont le regard froid et sans vie ressemblait à celui d’une statue de marbre ; de quel côté ma maison est-elle menacée ? »\par
» La sibylle, prenant un morceau de charbon et un petit papier à la main, me dit qu’elle allait y écrire le nom du dernier prince de ma maison, le nombre d’années qu’elle devait encore conserver sa puissance, et le nom de celui qui l’en déposséderait par la force des armes.\par
» Ayant fait cela en présence de toute la foule, elle cacheta le billet, et lorsque je voulus m’en saisir avec toute l’impatience et la curiosité que tu peux imaginer : « Non, mon seigneur, me dit-elle en repoussant ma main, je vais le remettre à cet homme qui porte un plumet à son chapeau, et qui est debout sur un banc devant l’église ; » et avant que je pusse comprendre quelques paroles qu’elle ajouta, elle se mêla à la foule, sans que je pusse voir ce qu’elle faisait.\par
» Dans cet instant, et pour ma consolation, le messager du prince vint l’avertir que le chevreuil était tué, et qu’il l’avait vu emporter dans la cuisine par deux chasseurs. Le prince, me prenant par le bras, me fit prendre le chemin de la maison, en m’assurant que cette femme n’avait dit que des folies indignes de l’argent que nous y avions perdu.\par
» Mais quel fut notre saisissement lorsqu’un cri, s’élevant sur la place, nous fit tourner la tête, et que nous vîmes un énorme chien, traînant après lui le chevreuil tué, qu’il avait dérobé dans la cuisine du château. Épouvanté par les cris des cuisiniers qui le poursuivaient, il déposa sa proie à nos pieds, et s’enfuit.\par
» La foudre tombant devant moi ne m’eût pas plus anéanti que la vue de cet animal, qui constatait la vérité de tout ce qu’avait prédit la sibylle. Mon premier soin, dès que je me trouvai seul, fut de chercher partout l’homme au plumet ; mais toutes les recherches que je fis faire restèrent inutiles, et ce n’est que dans la chaumière de Dahne que j’ai retrouvé mon homme. »\par
Alors, lâchant la main du chambellan, le prince essuya la sueur de son front, et tomba, accablé de douleur, sur ses coussins.\par
Le chambellan, qui jugea tout-à-fait inutile d’opposer son jugement à celui du prince, lui conseilla de chercher un moyen de se rendre maître du billet, puis d’abandonner l’homme à son destin. Le prince, désespéré, l’assura qu’il ne savait plus qu’imaginer.\par
Le chambellan était obligé de se rendre à Berlin pour la succession de l’oncle de sa femme, l’archi-chancelier comte de Kallheim. Il promit au prince de faire une dernière tentative auprès de Kohlhaas ; mais, au bout de quelques jours, il lui fit savoir que toutes ses peines étaient perdues ; qu’il ne fallait plus songer à posséder jamais le billet, à moins qu’il n’y eût quelque moyen de s’en emparer après l’exécution de Kohlhaas, qui devait avoir lieu le lundi des Rameaux.\par
À cette nouvelle, le prince, qui, pour calmer son chagrin, avait fait venir deux célèbres astrologues, espérant trouver quelque sujet de consolation dans leurs horoscopes, dont l’explication n’avait fait qu’ajouter à ses craintes celle d’une guerre prochaine avec la Pologne ; le prince, dis-je, navré d’un désespoir insupportable à son âme, usée par tant d’inquiétudes mortelles, passa deux jours enfermé dans sa chambre, dégoûté de la vie, refusant toute nourriture ; ensuite, ayant fait dire au gubernium qu’il se rendait à la chasse chez le prince de Dessau, il quitta Dresde.\par
Mais on apprit que le prince de Dessau était malade, et que son excellence n’y avait point paru.
\chapterclose


\chapteropen
\chapter[Chapitre VIII]{Chapitre VIII}\renewcommand{\leftmark}{Chapitre VIII}


\chaptercont
\noindent Lorsque l’infortuné Kohlhaas eut entendu sa sentence de mort, on lui rendit ses papiers. S’occupant alors de mettre ordre à ses affaires par un testament, il les adressa à son honnête voisin de Kohlhaasenbruck, qu’il nomma tuteur de ses enfants.\par
Il jouit d’un calme et d’un bonheur inexprimables pendant les jours qui précédèrent sa mort. Sa prison ayant été ouverte par l’ordre spécial du prince, tous ses amis vinrent le visiter, et le théologien Jacob Freising, envoyé à lui par Luther avec une lettre de celui-ci, lui donna la communion qu’il avait si ardemment désirée.\par
Enfin, le lundi des Rameaux arriva sans que l’on reçût la grâce de Kohlhaas, quoique tout le peuple l’attendît, de la part de l’empereur.\par
Il sortit de sa prison, accompagné d’une forte garde, portant ses deux petits garçons entre ses bras, conduit par le théologien Jacob Freising, et entouré de ses amis, qui se pressaient pour lui serrer encore une fois la main en signe d’adieu. Lorsqu’il arriva sur la place de l’exécution, l’électeur de Brandenbourg s’y trouvait au milieu de toute sa cour. À la droite de Henri de Geusau était le procureur de l’Empire, Franz Muller, une copie de la sentence de mort à la main ; à sa gauche, le procureur du Brandenbourg, Antoine Zauner, avec la sentence qu’il avait fait prononcer à Dresde contre le gentilhomme de Tronka. Au milieu du cercle ouvert que formait le peuple, on voyait un héraut tenant par la bride deux beaux coursiers trépignant d’impatience ; c’étaient les chevaux de Kohlhaas, que le gentilhomme, en vertu de sa condamnation, avait été forcé de reprendre des mains de l’écorcheur, et de rétablir dans une écurie bâtie sur la place du marché de Dresde à cet effet.\par
« Kohlhaas, lui dit le prince au moment où il arrivait, voici le jour où justice te sera rendue ; regarde, voici les chevaux que tu avais laissés à Tronkenbourg ; voici les écus d’or de ton valet. Michel Kohlhaas, es-tu content ? »\par
Le maquignon, après avoir lu la conclusion du tribunal de Dresde, que lui présentait le conseiller Zauner, posa ses deux enfants par terre, et étant arrivé à l’article qui condamnait le gentilhomme Wenzel de Tronka à deux ans de prison, emporté par le sentiment puissant qui le dominait, il posa ses deux mains en croix sur sa poitrine, et se jeta aux genoux du prince, qu’il embrassa. Puis se relevant, il pressa sur son cœur la main de Henri de Geusau, et lui assura que le vœu le plus cher de son cœur était accompli sur la terre ; puis s’approchant des chevaux, il les caressa, et dit au chancelier qu’il les léguait à ses deux fils, Henri et Léopold.\par
Le chancelier Henri de Geusau l’assura que toutes ses volontés seraient accomplies ; et lui ayant demandé s’il n’avait rien à disposer en faveur de la mère de Herse, Kohlhaas l’appela. Lorsqu’elle fut sortie de la foule, il lui remit les pièces d’or qui avaient appartenu à son fils, et en outre la somme d’argent qui lui avait été assignée comme dédommagement de l’obstacle mis à son commerce par le gentilhomme.\par
« Maintenant, s’écria le prince, Michel Kohlhaas, marchand de chevaux, prépare-toi à donner satisfaction à Sa Majesté l’empereur, de la guerre que tu as allumée dans ses États. »\par
Kohlhaas, se découvrant la tête, dit qu’il était tout préparé. Embrassant encore une fois ses enfants en versant des larmes silencieuses, il les remit à son digne voisin de Kohlhaasenbruck, et marcha vers l’échafaud.\par
Il ôta lui-même sa cravate ; puis jetant un regard perçant sur la foule, il ouvrit son petit étui de plomb, prit le billet, le décacheta, et après l’avoir lu, jetant encore une fois les yeux sur un homme qui portait un panache bleu et blanc et qui commençait à se livrer au plus doux espoir, il mit le papier dans sa bouche et l’avala. L’homme au panache, poussant un cri, tomba évanoui, et la tête de Kohlhaas, tranchée d’un coup de sabre, roula sur le pavé au même instant.\par
La foule qui couvrait la place s’ébranla de toutes parts. Au milieu du tumulte général, on remarqua quelques chevaliers emportant entre leurs bras le prince de Saxe sans connaissance. Il était revêtu d’un déguisement à l’aide duquel il avait assisté incognito à l’exécution.\par
Ici finit l’histoire de Michel Kohlhaas ; son corps fut accompagné au cercueil par le peuple touché de compassion. L’électeur dit à Henri de Geusau qu’il voulait que les deux fils de Kohlhaas fussent élevés parmi ses pages. Le prince de Saxe, après avoir, non sans peine, recouvré ses sens, retourna à Dresde, épuisé de corps et d’âme. Ceux qui désirent en savoir davantage sur son compte pourront puiser de plus amples détails dans l’histoire de ce temps-là.
\chapterclose

 


% at least one empty page at end (for booklet couv)
\ifbooklet
  \newpage\null\thispagestyle{empty}\newpage
\fi

\ifdev % autotext in dev mode
\fontname\font — \textsc{Les règles du jeu}\par
(\hyperref[utopie]{\underline{Lien}})\par
\noindent \initialiv{A}{lors là}\blindtext\par
\noindent \initialiv{À}{ la bonheur des dames}\blindtext\par
\noindent \initialiv{É}{tonnez-le}\blindtext\par
\noindent \initialiv{Q}{ualitativement}\blindtext\par
\noindent \initialiv{V}{aloriser}\blindtext\par
\Blindtext
\phantomsection
\label{utopie}
\Blinddocument
\fi
\end{document}
