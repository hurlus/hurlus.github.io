%%%%%%%%%%%%%%%%%%%%%%%%%%%%%%%%%
% LaTeX model https://hurlus.fr %
%%%%%%%%%%%%%%%%%%%%%%%%%%%%%%%%%

% Needed before document class
\RequirePackage{pdftexcmds} % needed for tests expressions
\RequirePackage{fix-cm} % correct units

% Define mode
\def\mode{a4}

\newif\ifaiv % a4
\newif\ifav % a5
\newif\ifbooklet % booklet
\newif\ifcover % cover for booklet

\ifnum \strcmp{\mode}{cover}=0
  \covertrue
\else\ifnum \strcmp{\mode}{booklet}=0
  \booklettrue
\else\ifnum \strcmp{\mode}{a5}=0
  \avtrue
\else
  \aivtrue
\fi\fi\fi

\ifbooklet % do not enclose with {}
  \documentclass[french,twoside]{book} % ,notitlepage
  \usepackage[%
    papersize={105mm, 297mm},
    inner=12mm,
    outer=12mm,
    top=20mm,
    bottom=15mm,
    marginparsep=0pt,
  ]{geometry}
  \usepackage[fontsize=9.5pt]{scrextend} % for Roboto
\else\ifav
  \documentclass[french,twoside]{book} % ,notitlepage
  \usepackage[%
    a5paper,
    inner=25mm,
    outer=15mm,
    top=15mm,
    bottom=15mm,
    marginparsep=0pt,
  ]{geometry}
  \usepackage[fontsize=12pt]{scrextend}
\else% A4 2 cols
  \documentclass[twocolumn]{report}
  \usepackage[%
    a4paper,
    inner=15mm,
    outer=10mm,
    top=25mm,
    bottom=18mm,
    marginparsep=0pt,
  ]{geometry}
  \setlength{\columnsep}{20mm}
  \usepackage[fontsize=9.5pt]{scrextend}
\fi\fi

%%%%%%%%%%%%%%
% Alignments %
%%%%%%%%%%%%%%
% before teinte macros

\setlength{\arrayrulewidth}{0.2pt}
\setlength{\columnseprule}{\arrayrulewidth} % twocol
\setlength{\parskip}{0pt} % classical para with no margin
\setlength{\parindent}{1.5em}

%%%%%%%%%%
% Colors %
%%%%%%%%%%
% before Teinte macros

\usepackage[dvipsnames]{xcolor}
\definecolor{rubric}{HTML}{800000} % the tonic 0c71c3
\def\columnseprulecolor{\color{rubric}}
\colorlet{borderline}{rubric!30!} % definecolor need exact code
\definecolor{shadecolor}{gray}{0.95}
\definecolor{bghi}{gray}{0.5}

%%%%%%%%%%%%%%%%%
% Teinte macros %
%%%%%%%%%%%%%%%%%
%%%%%%%%%%%%%%%%%%%%%%%%%%%%%%%%%%%%%%%%%%%%%%%%%%%
% <TEI> generic (LaTeX names generated by Teinte) %
%%%%%%%%%%%%%%%%%%%%%%%%%%%%%%%%%%%%%%%%%%%%%%%%%%%
% This template is inserted in a specific design
% It is XeLaTeX and otf fonts

\makeatletter % <@@@


\usepackage{blindtext} % generate text for testing
\usepackage[strict]{changepage} % for modulo 4
\usepackage{contour} % rounding words
\usepackage[nodayofweek]{datetime}
% \usepackage{DejaVuSans} % seems buggy for sffont font for symbols
\usepackage{enumitem} % <list>
\usepackage{etoolbox} % patch commands
\usepackage{fancyvrb}
\usepackage{fancyhdr}
\usepackage{float}
\usepackage{fontspec} % XeLaTeX mandatory for fonts
\usepackage{footnote} % used to capture notes in minipage (ex: quote)
\usepackage{framed} % bordering correct with footnote hack
\usepackage{graphicx}
\usepackage{lettrine} % drop caps
\usepackage{lipsum} % generate text for testing
\usepackage[framemethod=tikz,]{mdframed} % maybe used for frame with footnotes inside
\usepackage{pdftexcmds} % needed for tests expressions
\usepackage{polyglossia} % non-break space french punct, bug Warning: "Failed to patch part"
\usepackage[%
  indentfirst=false,
  vskip=1em,
  noorphanfirst=true,
  noorphanafter=true,
  leftmargin=\parindent,
  rightmargin=0pt,
]{quoting}
\usepackage{ragged2e}
\usepackage{setspace} % \setstretch for <quote>
\usepackage{tabularx} % <table>
\usepackage[explicit]{titlesec} % wear titles, !NO implicit
\usepackage{tikz} % ornaments
\usepackage{tocloft} % styling tocs
\usepackage[fit]{truncate} % used im runing titles
\usepackage{unicode-math}
\usepackage[normalem]{ulem} % breakable \uline, normalem is absolutely necessary to keep \emph
\usepackage{verse} % <l>
\usepackage{xcolor} % named colors
\usepackage{xparse} % @ifundefined
\XeTeXdefaultencoding "iso-8859-1" % bad encoding of xstring
\usepackage{xstring} % string tests
\XeTeXdefaultencoding "utf-8"
\PassOptionsToPackage{hyphens}{url} % before hyperref, which load url package

% TOTEST
% \usepackage{hypcap} % links in caption ?
% \usepackage{marginnote}
% TESTED
% \usepackage{background} % doesn’t work with xetek
% \usepackage{bookmark} % prefers the hyperref hack \phantomsection
% \usepackage[color, leftbars]{changebar} % 2 cols doc, impossible to keep bar left
% \usepackage[utf8x]{inputenc} % inputenc package ignored with utf8 based engines
% \usepackage[sfdefault,medium]{inter} % no small caps
% \usepackage{firamath} % choose firasans instead, firamath unavailable in Ubuntu 21-04
% \usepackage{flushend} % bad for last notes, supposed flush end of columns
% \usepackage[stable]{footmisc} % BAD for complex notes https://texfaq.org/FAQ-ftnsect
% \usepackage{helvet} % not for XeLaTeX
% \usepackage{multicol} % not compatible with too much packages (longtable, framed, memoir…)
% \usepackage[default,oldstyle,scale=0.95]{opensans} % no small caps
% \usepackage{sectsty} % \chapterfont OBSOLETE
% \usepackage{soul} % \ul for underline, OBSOLETE with XeTeX
% \usepackage[breakable]{tcolorbox} % text styling gone, footnote hack not kept with breakable


% Metadata inserted by a program, from the TEI source, for title page and runing heads
\title{\textbf{ La France sous les deniers Capétiens (1223-1328) }}
\date{1939}
\author{Marc Bloch}
\def\elbibl{Marc Bloch. 1939. \emph{La France sous les deniers Capétiens (1223-1328)}}
\def\elsource{Marc Bloch, \emph{La France sous les deniers Capétiens (1223-1328)}}

% Default metas
\newcommand{\colorprovide}[2]{\@ifundefinedcolor{#1}{\colorlet{#1}{#2}}{}}
\colorprovide{rubric}{red}
\colorprovide{silver}{lightgray}
\@ifundefined{syms}{\newfontfamily\syms{DejaVu Sans}}{}
\newif\ifdev
\@ifundefined{elbibl}{% No meta defined, maybe dev mode
  \newcommand{\elbibl}{Titre court ?}
  \newcommand{\elbook}{Titre du livre source ?}
  \newcommand{\elabstract}{Résumé\par}
  \newcommand{\elurl}{http://oeuvres.github.io/elbook/2}
  \author{Éric Lœchien}
  \title{Un titre de test assez long pour vérifier le comportement d’une maquette}
  \date{1566}
  \devtrue
}{}
\let\eltitle\@title
\let\elauthor\@author
\let\eldate\@date


\defaultfontfeatures{
  % Mapping=tex-text, % no effect seen
  Scale=MatchLowercase,
  Ligatures={TeX,Common},
}


% generic typo commands
\newcommand{\astermono}{\medskip\centerline{\color{rubric}\large\selectfont{\syms ✻}}\medskip\par}%
\newcommand{\astertri}{\medskip\par\centerline{\color{rubric}\large\selectfont{\syms ✻\,✻\,✻}}\medskip\par}%
\newcommand{\asterism}{\bigskip\par\noindent\parbox{\linewidth}{\centering\color{rubric}\large{\syms ✻}\\{\syms ✻}\hskip 0.75em{\syms ✻}}\bigskip\par}%

% lists
\newlength{\listmod}
\setlength{\listmod}{\parindent}
\setlist{
  itemindent=!,
  listparindent=\listmod,
  labelsep=0.2\listmod,
  parsep=0pt,
  % topsep=0.2em, % default topsep is best
}
\setlist[itemize]{
  label=—,
  leftmargin=0pt,
  labelindent=1.2em,
  labelwidth=0pt,
}
\setlist[enumerate]{
  label={\bf\color{rubric}\arabic*.},
  labelindent=0.8\listmod,
  leftmargin=\listmod,
  labelwidth=0pt,
}
\newlist{listalpha}{enumerate}{1}
\setlist[listalpha]{
  label={\bf\color{rubric}\alph*.},
  leftmargin=0pt,
  labelindent=0.8\listmod,
  labelwidth=0pt,
}
\newcommand{\listhead}[1]{\hspace{-1\listmod}\emph{#1}}

\renewcommand{\hrulefill}{%
  \leavevmode\leaders\hrule height 0.2pt\hfill\kern\z@}

% General typo
\DeclareTextFontCommand{\textlarge}{\large}
\DeclareTextFontCommand{\textsmall}{\small}

% commands, inlines
\newcommand{\anchor}[1]{\Hy@raisedlink{\hypertarget{#1}{}}} % link to top of an anchor (not baseline)
\newcommand\abbr[1]{#1}
\newcommand{\autour}[1]{\tikz[baseline=(X.base)]\node [draw=rubric,thin,rectangle,inner sep=1.5pt, rounded corners=3pt] (X) {\color{rubric}#1};}
\newcommand\corr[1]{#1}
\newcommand{\ed}[1]{ {\color{silver}\sffamily\footnotesize (#1)} } % <milestone ed="1688"/>
\newcommand\expan[1]{#1}
\newcommand\foreign[1]{\emph{#1}}
\newcommand\gap[1]{#1}
\renewcommand{\LettrineFontHook}{\color{rubric}}
\newcommand{\initial}[2]{\lettrine[lines=2, loversize=0.3, lhang=0.3]{#1}{#2}}
\newcommand{\initialiv}[2]{%
  \let\oldLFH\LettrineFontHook
  % \renewcommand{\LettrineFontHook}{\color{rubric}\ttfamily}
  \IfSubStr{QJ’}{#1}{
    \lettrine[lines=4, lhang=0.2, loversize=-0.1, lraise=0.2]{\smash{#1}}{#2}
  }{\IfSubStr{É}{#1}{
    \lettrine[lines=4, lhang=0.2, loversize=-0, lraise=0]{\smash{#1}}{#2}
  }{\IfSubStr{ÀÂ}{#1}{
    \lettrine[lines=4, lhang=0.2, loversize=-0, lraise=0, slope=0.6em]{\smash{#1}}{#2}
  }{\IfSubStr{A}{#1}{
    \lettrine[lines=4, lhang=0.2, loversize=0.2, slope=0.6em]{\smash{#1}}{#2}
  }{\IfSubStr{V}{#1}{
    \lettrine[lines=4, lhang=0.2, loversize=0.2, slope=-0.5em]{\smash{#1}}{#2}
  }{
    \lettrine[lines=4, lhang=0.2, loversize=0.2]{\smash{#1}}{#2}
  }}}}}
  \let\LettrineFontHook\oldLFH
}
\newcommand{\labelchar}[1]{\textbf{\color{rubric} #1}}
\newcommand{\milestone}[1]{\autour{\footnotesize\color{rubric} #1}} % <milestone n="4"/>
\newcommand\name[1]{#1}
\newcommand\orig[1]{#1}
\newcommand\orgName[1]{#1}
\newcommand\persName[1]{#1}
\newcommand\placeName[1]{#1}
\newcommand{\pn}[1]{\IfSubStr{-—–¶}{#1}% <p n="3"/>
  {\noindent{\bfseries\color{rubric}   ¶  }}
  {{\footnotesize\autour{ #1}  }}}
\newcommand\reg{}
% \newcommand\ref{} % already defined
\newcommand\sic[1]{#1}
\newcommand\surname[1]{\textsc{#1}}
\newcommand\term[1]{\textbf{#1}}

\def\mednobreak{\ifdim\lastskip<\medskipamount
  \removelastskip\nopagebreak\medskip\fi}
\def\bignobreak{\ifdim\lastskip<\bigskipamount
  \removelastskip\nopagebreak\bigskip\fi}

% commands, blocks
\newcommand{\byline}[1]{\bigskip{\RaggedLeft{#1}\par}\bigskip}
\newcommand{\bibl}[1]{{\RaggedLeft{#1}\par\bigskip}}
\newcommand{\biblitem}[1]{{\noindent\hangindent=\parindent   #1\par}}
\newcommand{\dateline}[1]{\medskip{\RaggedLeft{#1}\par}\bigskip}
\newcommand{\labelblock}[1]{\medbreak{\noindent\color{rubric}\bfseries #1}\par\mednobreak}
\newcommand{\salute}[1]{\bigbreak{#1}\par\medbreak}
\newcommand{\signed}[1]{\bigbreak\filbreak{\raggedleft #1\par}\medskip}

% environments for blocks (some may become commands)
\newenvironment{borderbox}{}{} % framing content
\newenvironment{citbibl}{\ifvmode\hfill\fi}{\ifvmode\par\fi }
\newenvironment{docAuthor}{\ifvmode\vskip4pt\fontsize{16pt}{18pt}\selectfont\fi\itshape}{\ifvmode\par\fi }
\newenvironment{docDate}{}{\ifvmode\par\fi }
\newenvironment{docImprint}{\vskip6pt}{\ifvmode\par\fi }
\newenvironment{docTitle}{\vskip6pt\bfseries\fontsize{18pt}{22pt}\selectfont}{\par }
\newenvironment{msHead}{\vskip6pt}{\par}
\newenvironment{msItem}{\vskip6pt}{\par}
\newenvironment{titlePart}{}{\par }


% environments for block containers
\newenvironment{argument}{\itshape\parindent0pt}{\vskip1.5em}
\newenvironment{biblfree}{}{\ifvmode\par\fi }
\newenvironment{bibitemlist}[1]{%
  \list{\@biblabel{\@arabic\c@enumiv}}%
  {%
    \settowidth\labelwidth{\@biblabel{#1}}%
    \leftmargin\labelwidth
    \advance\leftmargin\labelsep
    \@openbib@code
    \usecounter{enumiv}%
    \let\p@enumiv\@empty
    \renewcommand\theenumiv{\@arabic\c@enumiv}%
  }
  \sloppy
  \clubpenalty4000
  \@clubpenalty \clubpenalty
  \widowpenalty4000%
  \sfcode`\.\@m
}%
{\def\@noitemerr
  {\@latex@warning{Empty `bibitemlist' environment}}%
\endlist}
\newenvironment{quoteblock}% may be used for ornaments
  {\begin{quoting}}
  {\end{quoting}}

% table () is preceded and finished by custom command
\newcommand{\tableopen}[1]{%
  \ifnum\strcmp{#1}{wide}=0{%
    \begin{center}
  }
  \else\ifnum\strcmp{#1}{long}=0{%
    \begin{center}
  }
  \else{%
    \begin{center}
  }
  \fi\fi
}
\newcommand{\tableclose}[1]{%
  \ifnum\strcmp{#1}{wide}=0{%
    \end{center}
  }
  \else\ifnum\strcmp{#1}{long}=0{%
    \end{center}
  }
  \else{%
    \end{center}
  }
  \fi\fi
}


% text structure
\newcommand\chapteropen{} % before chapter title
\newcommand\chaptercont{} % after title, argument, epigraph…
\newcommand\chapterclose{} % maybe useful for multicol settings
\setcounter{secnumdepth}{-2} % no counters for hierarchy titles
\setcounter{tocdepth}{5} % deep toc
\markright{\@title} % ???
\markboth{\@title}{\@author} % ???
\renewcommand\tableofcontents{\@starttoc{toc}}
% toclof format
% \renewcommand{\@tocrmarg}{0.1em} % Useless command?
% \renewcommand{\@pnumwidth}{0.5em} % {1.75em}
\renewcommand{\@cftmaketoctitle}{}
\setlength{\cftbeforesecskip}{\z@ \@plus.2\p@}
\renewcommand{\cftchapfont}{}
\renewcommand{\cftchapdotsep}{\cftdotsep}
\renewcommand{\cftchapleader}{\normalfont\cftdotfill{\cftchapdotsep}}
\renewcommand{\cftchappagefont}{\bfseries}
\setlength{\cftbeforechapskip}{0em \@plus\p@}
% \renewcommand{\cftsecfont}{\small\relax}
\renewcommand{\cftsecpagefont}{\normalfont}
% \renewcommand{\cftsubsecfont}{\small\relax}
\renewcommand{\cftsecdotsep}{\cftdotsep}
\renewcommand{\cftsecpagefont}{\normalfont}
\renewcommand{\cftsecleader}{\normalfont\cftdotfill{\cftsecdotsep}}
\setlength{\cftsecindent}{1em}
\setlength{\cftsubsecindent}{2em}
\setlength{\cftsubsubsecindent}{3em}
\setlength{\cftchapnumwidth}{1em}
\setlength{\cftsecnumwidth}{1em}
\setlength{\cftsubsecnumwidth}{1em}
\setlength{\cftsubsubsecnumwidth}{1em}

% footnotes
\newif\ifheading
\newcommand*{\fnmarkscale}{\ifheading 0.70 \else 1 \fi}
\renewcommand\footnoterule{\vspace*{0.3cm}\hrule height \arrayrulewidth width 3cm \vspace*{0.3cm}}
\setlength\footnotesep{1.5\footnotesep} % footnote separator
\renewcommand\@makefntext[1]{\parindent 1.5em \noindent \hb@xt@1.8em{\hss{\normalfont\@thefnmark . }}#1} % no superscipt in foot
\patchcmd{\@footnotetext}{\footnotesize}{\footnotesize\sffamily}{}{} % before scrextend, hyperref


%   see https://tex.stackexchange.com/a/34449/5049
\def\truncdiv#1#2{((#1-(#2-1)/2)/#2)}
\def\moduloop#1#2{(#1-\truncdiv{#1}{#2}*#2)}
\def\modulo#1#2{\number\numexpr\moduloop{#1}{#2}\relax}

% orphans and widows
\clubpenalty=9996
\widowpenalty=9999
\brokenpenalty=4991
\predisplaypenalty=10000
\postdisplaypenalty=1549
\displaywidowpenalty=1602
\hyphenpenalty=400
% Copied from Rahtz but not understood
\def\@pnumwidth{1.55em}
\def\@tocrmarg {2.55em}
\def\@dotsep{4.5}
\emergencystretch 3em
\hbadness=4000
\pretolerance=750
\tolerance=2000
\vbadness=4000
\def\Gin@extensions{.pdf,.png,.jpg,.mps,.tif}
% \renewcommand{\@cite}[1]{#1} % biblio

\usepackage{hyperref} % supposed to be the last one, :o) except for the ones to follow
\urlstyle{same} % after hyperref
\hypersetup{
  % pdftex, % no effect
  pdftitle={\elbibl},
  % pdfauthor={Your name here},
  % pdfsubject={Your subject here},
  % pdfkeywords={keyword1, keyword2},
  bookmarksnumbered=true,
  bookmarksopen=true,
  bookmarksopenlevel=1,
  pdfstartview=Fit,
  breaklinks=true, % avoid long links
  pdfpagemode=UseOutlines,    % pdf toc
  hyperfootnotes=true,
  colorlinks=false,
  pdfborder=0 0 0,
  % pdfpagelayout=TwoPageRight,
  % linktocpage=true, % NO, toc, link only on page no
}

\makeatother % /@@@>
%%%%%%%%%%%%%%
% </TEI> end %
%%%%%%%%%%%%%%


%%%%%%%%%%%%%
% footnotes %
%%%%%%%%%%%%%
\renewcommand{\thefootnote}{\bfseries\textcolor{rubric}{\arabic{footnote}}} % color for footnote marks

%%%%%%%%%
% Fonts %
%%%%%%%%%
\usepackage[]{roboto} % SmallCaps, Regular is a bit bold
% \linespread{0.90} % too compact, keep font natural
\newfontfamily\fontrun[]{Roboto Condensed Light} % condensed runing heads
\ifav
  \setmainfont[
    ItalicFont={Roboto Light Italic},
  ]{Roboto}
\else\ifbooklet
  \setmainfont[
    ItalicFont={Roboto Light Italic},
  ]{Roboto}
\else
\setmainfont[
  ItalicFont={Roboto Italic},
]{Roboto Light}
\fi\fi
\renewcommand{\LettrineFontHook}{\bfseries\color{rubric}}
% \renewenvironment{labelblock}{\begin{center}\bfseries\color{rubric}}{\end{center}}

%%%%%%%%
% MISC %
%%%%%%%%

\setdefaultlanguage[frenchpart=false]{french} % bug on part


\newenvironment{quotebar}{%
    \def\FrameCommand{{\color{rubric!10!}\vrule width 0.5em} \hspace{0.9em}}%
    \def\OuterFrameSep{\itemsep} % séparateur vertical
    \MakeFramed {\advance\hsize-\width \FrameRestore}
  }%
  {%
    \endMakeFramed
  }
\renewenvironment{quoteblock}% may be used for ornaments
  {%
    \savenotes
    \setstretch{0.9}
    \normalfont
    \begin{quotebar}
  }
  {%
    \end{quotebar}
    \spewnotes
  }


\renewcommand{\headrulewidth}{\arrayrulewidth}
\renewcommand{\headrule}{{\color{rubric}\hrule}}

% delicate tuning, image has produce line-height problems in title on 2 lines
\titleformat{name=\chapter} % command
  [display] % shape
  {\vspace{1.5em}\centering} % format
  {} % label
  {0pt} % separator between n
  {}
[{\color{rubric}\huge\textbf{#1}}\bigskip] % after code
% \titlespacing{command}{left spacing}{before spacing}{after spacing}[right]
\titlespacing*{\chapter}{0pt}{-2em}{0pt}[0pt]

\titleformat{name=\section}
  [block]{}{}{}{}
  [\vbox{\color{rubric}\large\raggedleft\textbf{#1}}]
\titlespacing{\section}{0pt}{0pt plus 4pt minus 2pt}{\baselineskip}

\titleformat{name=\subsection}
  [block]
  {}
  {} % \thesection
  {} % separator \arrayrulewidth
  {}
[\vbox{\large\textbf{#1}}]
% \titlespacing{\subsection}{0pt}{0pt plus 4pt minus 2pt}{\baselineskip}

\ifaiv
  \fancypagestyle{main}{%
    \fancyhf{}
    \setlength{\headheight}{1.5em}
    \fancyhead{} % reset head
    \fancyfoot{} % reset foot
    \fancyhead[L]{\truncate{0.45\headwidth}{\fontrun\elbibl}} % book ref
    \fancyhead[R]{\truncate{0.45\headwidth}{ \fontrun\nouppercase\leftmark}} % Chapter title
    \fancyhead[C]{\thepage}
  }
  \fancypagestyle{plain}{% apply to chapter
    \fancyhf{}% clear all header and footer fields
    \setlength{\headheight}{1.5em}
    \fancyhead[L]{\truncate{0.9\headwidth}{\fontrun\elbibl}}
    \fancyhead[R]{\thepage}
  }
\else
  \fancypagestyle{main}{%
    \fancyhf{}
    \setlength{\headheight}{1.5em}
    \fancyhead{} % reset head
    \fancyfoot{} % reset foot
    \fancyhead[RE]{\truncate{0.9\headwidth}{\fontrun\elbibl}} % book ref
    \fancyhead[LO]{\truncate{0.9\headwidth}{\fontrun\nouppercase\leftmark}} % Chapter title, \nouppercase needed
    \fancyhead[RO,LE]{\thepage}
  }
  \fancypagestyle{plain}{% apply to chapter
    \fancyhf{}% clear all header and footer fields
    \setlength{\headheight}{1.5em}
    \fancyhead[L]{\truncate{0.9\headwidth}{\fontrun\elbibl}}
    \fancyhead[R]{\thepage}
  }
\fi

\ifav % a5 only
  \titleclass{\section}{top}
\fi

\newcommand\chapo{{%
  \vspace*{-3em}
  \centering % no vskip ()
  {\Large\addfontfeature{LetterSpace=25}\bfseries{\elauthor}}\par
  \smallskip
  {\large\eldate}\par
  \bigskip
  {\Large\selectfont{\eltitle}}\par
  \bigskip
  {\color{rubric}\hline\par}
  \bigskip
  {\Large TEXTE LIBRE À PARTICPATION LIBRE\par}
  \centerline{\small\color{rubric} {hurlus.fr, tiré le \today}}\par
  \bigskip
}}

\newcommand\cover{{%
  \thispagestyle{empty}
  \centering
  {\LARGE\bfseries{\elauthor}}\par
  \bigskip
  {\Large\eldate}\par
  \bigskip
  \bigskip
  {\LARGE\selectfont{\eltitle}}\par
  \vfill\null
  {\color{rubric}\setlength{\arrayrulewidth}{2pt}\hline\par}
  \vfill\null
  {\Large TEXTE LIBRE À PARTICPATION LIBRE\par}
  \centerline{{\href{https://hurlus.fr}{\dotuline{hurlus.fr}}, tiré le \today}}\par
}}

\begin{document}
\pagestyle{empty}
\ifbooklet{
  \cover\newpage
  \thispagestyle{empty}\hbox{}\newpage
  \cover\newpage\noindent Les voyages de la brochure\par
  \bigskip
  \begin{tabularx}{\textwidth}{l|X|X}
    \textbf{Date} & \textbf{Lieu}& \textbf{Nom/pseudo} \\ \hline
    \rule{0pt}{25cm} &  &   \\
  \end{tabularx}
  \newpage
  \addtocounter{page}{-4}
}\fi

\thispagestyle{empty}
\ifaiv
  \twocolumn[\chapo]
\else
  \chapo
\fi
{\it\elabstract}
\bigskip
\makeatletter\@starttoc{toc}\makeatother % toc without new page
\bigskip

\pagestyle{main} % after style

  
\chapteropen
\chapter[{1. Introduction, Critique des sources}]{\textsc{1. }Introduction, Critique des sources}\phantomsection
\label{c01}\renewcommand{\leftmark}{\textsc{1. }Introduction, Critique des sources}


\chaptercont
\noindent Sujet du cours. Esprit du cours.\par
Lectures essentielles :\par
\biblitem{J. CALMETTE, {\itshape Le Monde féodal}, coll. Clio, Paris, 1934.}
\biblitem{LAVISSE, {\itshape Histoire de France}, t. III 1, par Luchaire (pour Louis VIII) ; t. III 2 par Langlois.}
\biblitem{ HALPHEN, coll. Peuples et civilisations, t. VI, {\itshape L’essor de l’Europe}, et VII 1, {\itshape La fin du Moyen Age}, Paris, 1932.}
\biblitem{PETIT-DUTAILLIS, {\itshape La monarchie féodale en France et en Angleterre, coll.} Évolution de l’Humanité, Paris, 1933.}
\biblitem{PIRENNE, FOCILLON, COHEN, {\itshape La civilisation occidentale au Moyen Age du XI\textsuperscript{e} au milieu du XIII\textsuperscript{e} siècle}, coll. Hist. générale du Moyen Age, t. VIII, Paris, 1934.}
\biblitem{PIRENNE, {\itshape Histoire de Belgique}, t. I, 5 éd., 1929.}
\noindent Fixation des cadres chronologiques par règne.\par

\begin{itemize}[itemsep=0pt,]
\item Louis VIII (14 juillet 1223-8 novembre 1226).
\item Saint Louis — avec Blanche de Castille (d. 21 ou 27 nov. 1252) — , Saint Louis étant né le 25 avril 1214, d. 25 avril 1270.
\item Philippe III, d. 5 octobre 1285.
\item Philippe le Bel, d. 30 novembre 1314.
\item Louis X, d. 5 juin 1316.
\item Philippe V, d. 1322.
\item Charles IV, d. 1\textsuperscript{er} février 1328.
\end{itemize}


\labelblock{Première question : comment connaissons-nous ?}

\section[{A. Documents d’archives }]{A. Documents d’archives \protect\footnotemark }\phantomsection
\label{c01a}
\footnotetext{ A. MOLINIER, {\itshape Les Sources de l’Histoire de France}, t. III, Paris, 1903.}
\noindent  \phantomsection
\label{p2} En règle générale, assez nombreux. Les XIII\textsuperscript{e}-XIV\textsuperscript{e}  siècles ont beaucoup écrit.\par

\begin{enumerate}[itemsep=0pt,]
\item Parce que l’instruction était plus répandue, notamment chez les laïcs.
\item Parce que le droit était moins purement formaliste. Époque d’ordre relatif, il a conservé et il n’y a pas eu, depuis, d’immenses catastrophes globales.
\end{enumerate}

\noindent Les archives des églises. Multiplication des cartulaires (qui eut d’ailleurs pour résultat de faire disparaître les originaux).\par
Elles ne sont plus les seules. Apparition des archives laïques. Archives de la Royauté reconstituées sous Philippe Auguste après Fréteval, 1194, par des copies, et déposées au Palais \footnote{ \noindent a) En décembre 1231 ({\itshape Layettes du Trésor des Chartes}, t. V, n° 360), saint Louis ordonne au concierge du Palais (qui était en même temps garde des Archives) de fournir à un personnage de l’entourage royal, maître Jean de la Cour, copie d’un traité conclu entre Louis VIII et le comte Ferrand de Flandre.\par
 b) Sous Philippe le Long (F. Delaborde, Introduction au t. V des {\itshape Layettes du Trésor des Chartes}, p. XLIII), Pierre d’Étampes, clerc du roi, a la garde des archives où il accomplit de gros travaux de copie et d’inventorisation. Nous possédons une lettre que lui fit écrire le bouteiller Henri de Sully : Ce soir même à l’heure des Vêpres, apportez à notre hôtel du Palais les privilèges de la cour de Rome qui autorisent le Roi à \emph{« faire prenre clers et les faire tenir sans encourre sentences »} et ce que vous trouverez du fait de Narbonne. Car le conseil s’assemblera à cette heure (Arch. Nat., J 476, n° 1 \textsuperscript{14}).
 }. Dès Philippe Auguste, s’est prise l’habitude des registres. Ce sont, à l’origine, des cartulaires : leur utilisation se marque bien par les faits suivants. Parmi les registres composés sous Philippe Auguste, le plus commode parce que le mieux classé était celui qu’après plusieurs tâtonnements le garde des sceaux, Guérin, avait fait compiler en 1220 \footnote{Arch. nat., JJ 26.}. On y écrira jusqu’à Philippe III des actes nouveaux sur les pages blanches. Quand saint Louis partit pour la Croisade, il dut le laisser à la Régente, mais il en fit faire une copie emportée à la croisade qui fut naturellement tenue au courant \footnote{Bibl. nat., ms. lat. 9778.}. Ils devinrent peu à peu des livres d’enregistrement. L’évolution s’achève sous Philippe le Bel. Cependant les méthodes de la chancellerie sont encore assez imparfaites. Surtout l’habitude des commissions déléguées fait que beaucoup de documents échappent à l’enregistrement.\par
 \phantomsection
\label{p3} Parallèlement, le Parlement. Brève histoire des {\itshape Olim} qui nous montre de même le passage du spicilège à l’enregistrement.\par
En même temps, les archives des grandes principautés se constituent (encore fort mal étudiées aujourd’hui). Exemples : les Cartulaires des comtes de Toulouse (Raymond VII), des comtes de Flandre etc. Même des petits seigneurs. Archives urbaines.\par
Nous savons donc pas mal. Mais il est intéressant de marquer ce que nous ne savons pas, ou très peu.\par

\begin{enumerate}[itemsep=0pt,]
\item Nous avons très peu de lettres privées, ou ces lettres sont pour la plupart des documents officiels. Nous ne savons donc à peu près rien que d’officiel, de déformé.
\item Les archives nous ont conservé surtout des documents originaux d’une valeur juridique durable. Par exemple, les contrats à censive perpétuelle plutôt que les autres à bail temporaire. Pour la même raison, nous sommes assez pauvres en comptes encore qu’il y en ait (belle série des Comptes de Saint-Denis depuis 1261). S’aggrave pour la Royauté de l’incendie de la Chambre des Comptes (26 ou 27 octobre 1737).
\end{enumerate}

\section[{B. Sources narratives}]{B. Sources narratives}\phantomsection
\label{c01b}
\noindent Observations préliminaires. Le goût de l’histoire. Son expansion chez les laïcs et surtout l’espèce d’indépendance intellectuelle acquise par ceux-ci, se traduisent par l’intrusion du français dans l’histoire (déjà au XII\textsuperscript{e} siècle, sous forme versifiée, désormais en prose).\par
J’examinerai simplement ici les principaux genres historiques :\par
\bigbreak
\noindent \labelchar{a)} Les chroniques universelles. Leur intérêt pour les parties anciennes et pour les parties récentes. Un type : Aubry de Trois Fontaines infiniment utile pour la Champagne (la chronique s’arrête à 1241). Encore, le {\itshape Speculum historiale} de Vincent de Beauvais (rédaction vers 1254 ; rapprocher du {\itshape Speculum naturale} et du {\itshape Speculum doctrinale).}\par
\bigbreak
\noindent \labelchar{b)} Les essais d’histoire nationale : essais privés, souvent à l’usage des laïques. Par exemple dès la première moitié du siècle, la {\itshape Chronique rimée des rois de France} du Tournaisien Ph. Mouskés ou la compilation saintongeaise qui porte le titre naïf de {\itshape Tote l’histoire de France.}\par
\bigbreak
\noindent \labelchar{c)} Les essais d’historiographie monarchique. Saint-Denis. Les premiers essais remontent au siècle précédent. C’est le recueil latin exécuté entre 1120 et 1131 qui forme le célèbre manuscrit de la Bibliothèque nationale (lat. 5925). Un autre recueil plus complet fut exécuté après la mort de Philippe Auguste. Avec des textes anciens (Einhart par exemple), il contenait, pour le règne de Philippe Auguste, la chronique du moine clunisien Rigord, continuée par Guillaume le Breton. Le travail historiographique ne fut pas continué sous Louis VIII (les compilations par où on comblera ensuite la lacune sont postérieures), ni même sous saint Louis. Mais, sous  \phantomsection
\label{p4} Philippe III, un moine de Saint-Denis, Guillaume de Nangis, composa une histoire de Louis IX, sous Philippe IV, une histoire de Philippe III. Ses confrères continuèrent son œuvre pour les règnes suivants. Par ailleurs, Philippe IV avait commandé à l’abbé de Saint-Denis une grande histoire de France qui, exécutée par frère Ives, fut remise à Philippe V en 1317, manuscrit richement orné que nous avons encore.\par
D’autre part, le français. Sous saint Louis, un moine de Saint-Denis, Primat, traduit le recueil du manuscrit latin 5925. Ce sera le noyau des \emph{Grandes Chroniques} continué par une traduction de Guillaume de Nangis et de ses continuateurs.\par
\bigbreak
\noindent \labelchar{d)} Les chroniques locales. Moins nombreuses que par le passé et non plus uniquement monastiques. Elles glissent aux rumeurs. Par exemple, celle de Geoffroi de Paris (1300-1316).\par
\bigbreak
\noindent \labelchar{e)} Les biographies.\par

\begin{enumerate}[itemsep=0pt,]
\item[]\listhead{Il faut distinguer deux catégories :}
\item Les autobiographies qui sont en même temps le récit des grandes choses auxquelles le personnage a été mêlé. Le premier exemple avait été donné au siècle précédent par des ouvrages inspirés des confessions (Guibert de Nogent, {\itshape De vita sua).} Désormais, nous trouvons l’habitude autobiographique entre les mains des laïques et qui écrivent en français : Villehardouin (après 1207) et Robert de Clary (vers 1216).
\item Les biographies. Elles ont été sans doute plus nombreuses qu’il n’en subsiste. Voyez celle de Guillaume le Maréchal (en vers) — mort en 1219 — conservée par un seul manuscrit ; il y a dû y avoir des cas analogues. Mais le seul roi sur lequel il nous reste des documents biographiques était saint Louis. Ici, il s’agit d’hagiographie.
\end{enumerate}

\bigbreak
\noindent \labelchar{f)} Hagiographie. Saint Louis. La canonisation. Trois enquêtes 1273-1280, 1281-1283, 1288-1297 (canonisation au mois d’août).\par
Perdues sauf quelques fragments.\par

\begin{enumerate}[itemsep=0pt,]
\item[]\listhead{Il faut y suppléer par les vies :}
\item Le confesseur même de saint Louis, Geoffroi de Beaulieu, un dominicain qui écrivit entre 1272 et 1276 à la prière du pape Grégoire X.
\item Le chapelain du roi, Guillaume de Chartres, qui écrivit (avant 1282) pour compléter Geoffroi de Beaulieu.
\item Le plus intéressant, Guillaume de Saint-Pathus, frère mineur, confesseur de la reine Marguerite qui écrit en 1302 ou 1303, sur la requête d’une fille du roi. Le grand intérêt est qu’il avait reçu communication d’une copie ou d’un abrégé de la deuxième enquête. Nous ne possédons qu’un texte en français, dont on ne peut dire s’il a été ou non précédé d’un texte en latin (ce qui est probable), ou s’il a été ou non traduit par l’auteur lui-même.
\end{enumerate}

\noindent Joinville et saint Louis \footnote{ Il existe plusieurs éditions de l’œuvre de Joinville. L’édition Natalis de Wailly, sous le titre : {\itshape Jean}, {\itshape sire de Joinville, Histoire de saint Louis, Credo et Lettre à Louis X. Texte original accompagné d’une traduction}, Paris, 1874, comporte, comme le titre l’indique, une traduction.}. Joinville appartenait à la haute noblesse champenoise ; il était né en 1225, il était donc de dix ans  \phantomsection
\label{p5} le cadet de saint Louis ; il n’était pas à l’origine le vassal direct de saint Louis ; il prit part à la croisade de 1248 ; et l’on connaît le passage célèbre \footnote{{\itshape Le livre des saintes paroles et des bons faits de notre saint roi Louis, transcrit en français moderne} par André Mary, Paris, 1928, p. 45.}.\par
Il se lie avec saint Louis qui l’attache à son service après l’arrivée de l’armée en Syrie et le prend en amitié. Il reçut de lui un fief de bourse, rente de 200 livres tournois, par un acte que nous avons encore. Après le retour en France, il ne vécut pas à sa cour, attaché d’ailleurs beaucoup plus étroitement au comte de Champagne qu’au roi, mais il vécut, semble-t-il, assez souvent à Paris. Il refuse de suivre saint Louis à sa dernière croisade qu’il blâme ; mais reste avec lui en termes affectueux \footnote{{\itshape Ibid}., p. 269.}. Il est un des témoins du procès de canonisation ; il écrit, ou mieux dicte, son ouvrage, bien après la mort du grand roi, vers 1305. Il vécut très vieux, jusqu’en 1317. Un italien qui visita la France vers 1310, Fr. Barberini, s’informant des habitudes de la société contemporaine ne crut pouvoir mieux faire que de s’adresser à Joinville et le consulta sur des problèmes tels que celui-ci : quand deux hommes de même condition sont assis à côté l’un de l’autre et mangent à la même écuelle, qui doit {\itshape trancher} pour l’autre (s’il n’est pas d’écuyer tranchant) ? réponse : celui qui a le couteau à sa droite. Nous savons aussi par Barberini que le vieux Joinville chez le comte de Champagne, son maître (dont il était sénéchal, c’est-à-dire chargé de la surveillance de son hôtel) fit une scène à un jeune écuyer qui, avant de \emph{trancher}, avait négligé de se laver les mains. C’était un homme courtois ! \par
Joinville et saint Louis. L’intimité \footnote{{\itshape Ibid}., p. 156-157. }.\par
Saint Louis et Joinville.\par
Lavement des pieds \footnote{{\itshape Ibid}., p. 9.}.\par
Joinville. Son récit relatif au fameux conseil d’Acre, où il se montre comme presque seul à conseiller au roi de rester en Syrie, est extrêmement controuvé, à peu près démenti par des documents contemporains (lettre écrite par saint Louis lui-même depuis Acre, et document connu sous le nom de lettre de Jean Sarrasin \footnote{ Voir G. PARIS, {\itshape Histoire littéraire de la France}, t. XXXII, Paris, 1898, p. 328-329. G. Paris est bien ennuyé de cette histoire-là et il se donne beaucoup de mal pour : p. 328 : \emph{« concilier... les conclusions qu’imposent des documents d’une authenticité inattaquable et la confiance due à un des plus beaux morceaux de notre littérature historique, lequel perdrait presque toute sa valeur s’il ne devait être considéré que comme un produit de l’amnésie sénile et de la vanité masquée sous un appel aux plus nobles inspirations de l’honneur et de la conscience »}.}).\par
Y joindre bien d’autres sources indirectes : littéraires notamment. Nous aurons à y revenir.
\section[{Conclusion}]{Conclusion}
\noindent Les faits les plus saillants sont faciles à connaître. La critique des sources relativement aisée. Les phénomènes de masse difficiles à établir, vu l’absence de statistique. Les individus mieux connus que par le passé ; mais beaucoup encore nous échappent.
\chapterclose


\chapteropen
\chapter[{2. Le cadre}]{\textsc{2. }Le cadre}\phantomsection
\label{c02}\renewcommand{\leftmark}{\textsc{2. }Le cadre}


\chaptercont
\section[{A. Les limites}]{A. Les limites}\phantomsection
\label{c02a}
\noindent  \phantomsection
\label{p7} Le problème : France d’aujourd’hui ? Royaume de France d’alors ?\par
Limites du royaume de France. À l’avènement de Louis VIII, ce sont encore celles du royaume de Charles le Chauve, à bien peu près. La grande extension est de plus tard. Mais elle a déjà un peu commencé.\par

\begin{enumerate}[itemsep=0pt,]
\item[]\listhead{Ces limites sont :}
\item L’Escaut jusqu’à un peu en amont de Tournai (au XIII\textsuperscript{e} siècle, l’Empire revendique le Pays de Waes à tort). Primitivement, l’Escaut était borné jusqu’à Bouchain. Mais l’Ostrevant, sur la rive gauche, ayant été occupé par les comtes de Hainaut au XI\textsuperscript{e} siècle, on perdit le souvenir de l’appartenance au royaume.
\item De là, la limite fait une boucle autour du Cambrésis, recoupe l’Escaut peu en aval de sa source, puis formée par des frontières de comtés, se dirige vers la Meuse qu’elle coupe en aval du confluent de la Semois.
\item Mais elle ne suivait le fleuve que sur quelques kilomètres. Sans contestations possibles, les comtés qui bordaient la Meuse du {\itshape pagus Castricius} aux environs de Mézières, jusqu’au Bassigny étaient d’Empire. De même, des comtés voisins du fleuve sans le toucher, comme le Barrois et une partie au moins de l’Ornois. Donc, la frontière était éloignée du fleuve de quelques lieues : elle courait, en particulier dans l’Argonne, où elle était peu claire. On s’accordait en général à la fixer à la Bienne.
\item De limite de comté en limite de comté, la frontière atteint la Saône (en aval de Port-sur-Saône) et longe en général le fleuve. Il y avait eu à l’origine des exceptions : car la frontière naturelle ne longe la Saône que là où celle-ci était une frontière de comtés. Le comté de Chalon qui est dans le royaume s’étend notablement sur la rive gauche, mais au cours des temps, la rive gauche était devenue fief d’Empire.
\item Au sud du comté de Mâcon, la frontière du royaume de Charles le Chauve abandonnait résolument la ligne Saône-Rhône pour ne la retrouver qu’au Petit Rhône, au delta, limite du comté de Nîmes. En effet, les comtés bordures de la rive droite — et même à quelque distance du fleuve — étaient à Lothaire et depuis à l’Empire : Lyonnais, Forez, Vivarais, Uzège. Mais la situation de ce côté n’était pas entière. L’Uzège avait fait partie de la constellation aux mains de la  \phantomsection
\label{p8} maison de Toulouse. Depuis le traité de liquidation de la croisade des Albigeois (1229), il avait passé dans le domaine direct du roi. On avait tout à fait oublié son appartenance à l’Empire. Au temps de Louis VII, le comte de Forez avait été amené à faire hommage au roi de France au moins pour plusieurs de ses châteaux. Les liens avec l’Empire avaient été par là pratiquement rompus. Enfin, probablement sous Philippe Auguste, le comte de Valentinois avait fait hommage au roi de la partie de son comté située sur la rive droite. Ainsi naissance de la théorie des quatre rivières. Mais Lyon et Viviers...
\item Au Sud des Pyrénées, la marche d’Espagne est encore théoriquement un fief français. Exemple de cela : Catalogne, Cerdagne, Roussillon. Revenu depuis 1162 au royaume d’Aragon, sa qualité de fief n’est plus mentionnée pour rien. En 1258, saint Louis renoncera à ses droits de seigneur de fief.
\end{enumerate}

\noindent Noter le caractère de cette frontière. Frontière de royaume. Ne coïncide ni avec les frontières d’évêché ou de province ecclésiastique, ni avec une frontière de seigneurie, ni avec une frontière de fief. Les comtes de Flandre étaient vassaux de l’Empire. Ceux de Champagne aussi, même pour quelques châteaux situés dans le royaume. L’archevêque de Lyon était fidèle du roi pour l’abbaye de Savigny et, depuis Philippe Auguste, le péage de Givors. Les comtes de Mâcon, vassaux des ducs de Bourgogne, l’étaient du roi pour quelques châteaux, des empereurs pour des possessions de la rive gauche. Tendance à établir des frontières claires, fluviales, une sorte de rapport entre les frontières d’État et féodales (très sensible sous saint Louis).
\section[{B. Quel est le nombre des hommes qui vivent dans ces limites ?}]{B. Quel est le nombre des hommes qui vivent dans ces limites ?}\phantomsection
\label{c02b}
\noindent (ou dans les limites légèrement accrues de 1328 ; hélas ! la précision n’est pas capitale).\par
Nous le savons mal. Et — ce qui est important — les gens du temps le savaient mal. On a tenté des évaluations. La plus récente est celle de Lot \footnote{F. LOT, {\itshape L’État des paroisses et des feux de 1328}, dans {\itshape Bibl. Éc. Chartes}, 1929, t. XC, p. 51-107 et p. 256-315.}. Lot s’appuie sur un état des paroisses et des feux dressé en 1328 par sénéchaussées ou bailliages. Naturellement, il faut interpoler : certaines principautés n’ont pas été recensées par les agents royaux.\par

\begin{enumerate}[itemsep=0pt,]
\item[]\listhead{Lot suppose :}
\item que tout feu égale une famille,
\item de 5 personnes (sauf à Paris où il en compte 3).
\end{enumerate}

\noindent Il trouve dans la région recensée 12 213 000. D’où, par extension, dans la France, entre 16 à 17 millions, sans les villes, et avec les villes, 19 à 20 millions. Il y a là dedans beaucoup d’hypothèses.\par

\begin{enumerate}[itemsep=0pt,]
\item[]\listhead{Je crois :}
\item que Lot compte pour le feu un chiffre un peu faible ;
\item qu’il n’a pas tenu compte que, parmi les régions non recensées, se trouvaient deux des régions les plus peuplées du royaume, Normandie et Flandre.
\end{enumerate}

\noindent Ses chiffres peuvent du moins fixer un ordre de grandeur. Disons, en chiffre rond, 20 millions. La France, aujourd’hui, est un pays de 40 millions. La superficie du royaume en 1328 était à peu près les 7/9 de celle d’aujourd’hui.\par

\begin{enumerate}[itemsep=0pt,]
\item[]\listhead{En gros :}
\item la population totale était de peu supérieure à la moitié de celle d’aujourd’hui (France pour France) ;
\item un peu supérieure aux 5/8, territoire égal.
\end{enumerate}

 \phantomsection
\label{p9}
\begin{enumerate}[itemsep=0pt,]
\item[]\listhead{Explication de ces différences :}
\item il n’y a pas de grandes villes. En 1328, Paris avec environ 200 000 habitants est très grand. Bruges, 35 000, Rouen peut-être 40 000. Les autres ... ;
\item la proportion de la surface réellement cultivée est faible — en raison des systèmes d’assolement.
\end{enumerate}

\noindent L’agriculture telle qu’on la pratiquait alors était une grande dévoratrice d’espace. Là où on avait semé, on n’obtenait que d’assez maigres moissons. Jamais, surtout, le finage tout entier ne donnait de récoltes. Les régimes d’assolement les plus perfectionnés exigeaient que, chaque année, un tiers ou une moitié des labours demeurassent en jachère. Plus fréquemment encore, l’alternance manquait de régularité.\par
Plus intéressantes peut-être que les chiffres, approximatifs, sont les considérations historiques que voici.\par
Il n’y a aucun doute que la première partie du XIII\textsuperscript{e} siècle n’ait coïncidé avec un moment d’intense peuplement, commencé au siècle précédent.\par

\begin{enumerate}[itemsep=0pt,]
\item[]\listhead{Connu :}
\item dans les villes par les témoignages directs, notamment paroisses, enceintes ;
\item dans les campagnes par a) les fondations de villes neuves ; b) le défrichement en général.
\end{enumerate}

\noindent Nous décrirons ailleurs comment, de 1050 à 1250 ou environ — et surtout depuis l’an 1200 — de nombreux villages tout neufs se créèrent à la fois sur les confins du monde latino-germanique — plateaux hispaniques, plaine du Nord, au-delà de l’Elbe — et au cœur même du vieux pays, où de toutes parts forêts ou friches se trouaient de labours ; comment aussi, autour des anciens villages, le terroir cultivé alla durant cette même période en s’accroissant. Exemples de villes neuves : les archevêques de Rouen, ayant acquis en 1197 la forêt d’Aliermont, y créent de nombreux villages. En 1224-1225, le chapitre de Reims et le comte de Champagne s’associent pour fonder Florent, en Argonne. Voici une modeste ville neuve : Bonlieu. Initiative due aux habitants : en 1310 les habitants de Gardomon offrent de fonder Réalville (Tarn-et-Garonne), achètent les terrains, s’engagent à fortifier la ville et demandent en revanche une charte de coutumes. Froideville, au bord de l’Orge, 1224.\par
En 1227 et 1228, une série d’actes nous montrent de nombreuses {\itshape novales} dans les villages appartenant à l’abbaye de Saint-Denis. Dans les revenus du domaine royal en Normandie, figurent sous saint Louis de nombreux essarts.\par
Dans la seconde moitié du siècle et au début du XIV\textsuperscript{e}, le mouvement semble se ralentir, sans recul.\par

\begin{listalpha}[itemsep=0pt,]
\item[]\listhead{Or, voyons ce que, pratiquement, cela signifie :}
\item Il y a plus de terre cultivée. Plus d’hommes aussi à vrai dire. Il est difficile de calculer le rapport. Mais les rendements aussi sont meilleurs. Et il semble bien hors de doute que la proportion des denrées agricoles aux hommes ait augmenté et, par conséquent, la possibilité d’une main-d’œuvre industrielle.
\item Il y a plus de sécurité que précédemment (moins qu’aujourd’hui).
\item Il y a des contacts plus aisés entre les hommes. Mais ici il faut voir de plus près.
\end{listalpha}

\section[{C. Comment les hommes communiquent-ils entre eux ?}]{C. Comment les hommes communiquent-ils entre eux ?}\phantomsection
\label{c02c}
\noindent  \phantomsection
\label{p10} Au sujet des possibilités naturelles de communication, il convient de noter qu’elles sont certainement meilleures en 1223 qu’un siècle auparavant et qu’elles vont, durant le siècle qui suit, en s’améliorant.\par

\begin{enumerate}[itemsep=0pt,]
\item[]\listhead{Cela pour les raisons qui suivent :}
\item La sécurité est plus grande : a) parce que les hommes sont plus proches ; b) les pouvoirs plus forts.
\item En ce qui concerne le transport des marchandises, transformations de l’attelage \footnote{ LEFEBVRE DES NOETTES, {\itshape L’attelage du cheval à travers les âges}, 2 vol., Paris, 1931.}.
\item Les pouvoirs publics — rois, princes, villes — ont pris en mains certains travaux publics. Guère, semble-t-il, les routes elles-mêmes. Mais constructions de ponts (très fréquentés depuis le XI\textsuperscript{e} siècle) ; mesures propres à faciliter la navigation sur les cours d’eau.
\end{enumerate}

\noindent Mais il ne faut pas exagérer. Si la situation n’est guère différente, en gros, en 1328, de ce qu’elle sera sous François 1\textsuperscript{er}, et même au début du règne de Louis XIV, elle est extrêmement différente, ne disons même pas de celle du XX\textsuperscript{e} siècle, mais de celle de l’époque de la Révolution.\par

\begin{enumerate}[itemsep=0pt,]
\item[]\listhead{Précisons :}
\item Les routes sont uniformément mauvaises. Des pistes. On y voyage à cheval, à cause des boues, etc.
\item Il n’y a nulle part d’entreprises de transport : des voituriers de métier, voilà tout.
\item Les distances horaires. Moyenne : 30 à 40 km par jour.
\end{enumerate}


\begin{enumerate}[itemsep=0pt,]
\item[]\listhead{Caractère de la circulation :}
\item Les hommes circulent beaucoup — plus en un certain sens que dans des civilisations de liaisons plus faciles. Ex. : les chefs, les fonctionnaires, les marchands, les jongleurs, les clercs, les pèlerins, les migrations paysannes. Mais en circulant beaucoup, ils sont moins en contact régulier.
\item Il n’existe pas de service régulier de nouvelles. Quelques détails si possible sur les arrivées de nouvelles.
\item La circulation des grosses marchandises est redevenue possible. Mais elle n’atteint pas encore le fin fond des campagnes.
\end{enumerate}

\noindent Exemple de difficultés créées par la distance, et comment elles ont été exagérées : Philippe Auguste et le Poitou. Vers 1205 \footnote{ LÉOPOLD DELISLE, {\itshape Catalogue des actes de Philippe Auguste}, Paris, 1856, n° 966, p. 222 et p. 510-511.}, Philippe Auguste envoya un émissaire à un baron poitevin, Raoul de Lusignan. Il lui demandait de gérer pendant cinq ans le domaine royal en Poitou et de lui livrer durant ce temps, ses terres et forteresses de Normandie (où Raoul possédait notamment le canton d’Eu). \emph{Le Poitou}, devait ajouter l’envoyé, \emph{« est si lointain que le roi ne peut y aller ni y envoyer comme il le faudrait »}.
\chapterclose


\chapteropen
\chapter[{3. Le gouvernement des hommes }]{\textsc{3. }Le gouvernement des hommes \protect\footnotemark }\phantomsection
\label{c03}\renewcommand{\leftmark}{\textsc{3. }Le gouvernement des hommes }

\footnotetext{ \noindent Bibliographie :\par
 \noindent\textbf{}\par
\biblitem{CH. PETIT-DUTAILLIS, {\itshape Étude sur la vie et le règne de Louis VIII}, Paris, 1894 ;}
\biblitem{CH. V. LANGLOIS, {\itshape Le règne de Philippe III le Hardi}, Paris, 1887 ;}
\biblitem{P. LEHUGEUR, {\itshape Le règne de Philippe le Long}, Paris, 1897 et {\itshape Philippe le Long, roi de France. Le mécanisme du gouvernement}, Paris 1931 ;}
 \noindent Sur Philippe le Bel, livre en préparation de Kienast extrait de l’{\itshape Historische Zeitschr}, t. 148, 1933, sous le titre \emph{Der fr. Staat in dreizehnten Jahrhundert}.\par
 \noindent\textbf{}\par
\biblitem{A. LUCHAIRE, {\itshape Manuel des institutions françaises : période des Capétiens directs}, Paris, 1892 (dépassé) ;}
\biblitem{A. ESMEIN, {\itshape Cours élémentaire d’histoire du droit français}, 14\textsuperscript{e} éd., 1921 ;}
\biblitem{E. CHÉNON, {\itshape Histoire générale du droit français}, t. I, 1926, t. II, 1929.}
\biblitem{P. VIOLLET, {\itshape Histoire des institutions politiques et administratives de la France}, 3 vol., Paris 1890-1903 ;}
\biblitem{Quelques indications très sommaires et parfois sujettes à caution dans G. DUPONT-FERRIER, {\itshape La formation de l’État français et l’unité française}, Paris, 1929 (Coll. A. Colin).}
 }

\chaptercont
\section[{A. Les cadres du gouvernement}]{A. Les cadres du gouvernement}\phantomsection
\label{c03a}
\noindent  \phantomsection
\label{p11} Nous allons parler, dans ce qui va suivre, presque constamment de roi, de royaume et d’État. C’étaient là de grandes réalités. Mais il est bon de rappeler tout d’abord que, dans le gouvernement des hommes, elles n’étaient pas les seules, à beaucoup près, qui comptassent. En dehors de toute dialectique juridique, prenons les trois obligations par où se reconnaissent le plus aisément aujourd’hui les liens d’un homme envers un État : le service militaire, l’impôt, la justice. Au XIII\textsuperscript{e} siècle, le tenancier d’une seigneurie doit l’impôt et l’ost à son seigneur, et a celui-ci pour juge ordinaire (ou, si son seigneur n’est pas haut justicier, un autre seigneur qui l’est). Le vassal militaire doit la taille, le service, à son seigneur de fief et, pour certaines causes du moins, dépend de sa cour. Le bourgeois d’une ville peut avoir certaines de ses obligations envers un seigneur. Mais aussi envers sa ville. Enfin si cependant au-dessus du seigneur et de la ville, il est habituel qu’un pouvoir supérieur s’élève qui exige lui aussi l’impôt, le service et rende la justice, souvent ce pouvoir n’est pas directement celui du roi. Entre lui et beaucoup des villes et seigneuries, s’interposent les principautés territoriales, héritières des comtés carolingiens, formées en général de groupes de comtés,  \phantomsection
\label{p12} auxquels se sont annexées toutes sortes de droits d’autres origines. La royauté pourtant a pu étendre, plus ou moins efficacement, son autorité militaire, financière et judiciaire aux sujets des seigneuries et des principautés, aux membres des groupes vassaliques, aux bourgeois des villes. Mais il importe de bien se représenter — d’autant qu’à ce sujet certains livres peuvent tromper — que sous Louis VIII la vie, pour beaucoup de Français, se déroule encore sans qu’ils aient affaire, si ce n’est exceptionnellement, à ce que nous appellerions le pouvoir de l’État. Sous Philippe le Bel et ses fils, ces occasions sont incontestablement plus fréquentes ; le contact pourtant n’est encore qu’intermittent.\par
Seigneurie, ville, principautés apparaîtront plusieurs fois au cours de notre exposé. Mais cette fois, suivant un ordre inverse de celui de la présente partie de ce cours, nous prendrons notre point de vue d’en haut : depuis le roi.
\section[{B. La royauté et les rois}]{B. La royauté et les rois}\phantomsection
\label{c03b}
\subsection[{1° La succession}]{1° La succession}\phantomsection
\label{c03b1}
\noindent Il n’y a plus, au XIII\textsuperscript{e} siècle, de problème dynastique. L’hérédité est si bien assurée que le premier souverain dont nous ayons à nous occuper, Louis VIII, est aussi le premier qui n’ait pas été associé à la royauté du vivant de son père. Preuve de la force qu’avait prise le sentiment de la légitimité. Mathieu Paris prête aux envoyés de saint Louis, refusant pour le frère du roi, Robert d’Artois, la couronne impériale, offerte par Grégoire IX, le propos suivant : \emph{« Credimus enim, dominum nostrum regem Galliae quem linea regii sanguinis provexit ad sceptra Francorum regenda, excellenciorem esse aliquo imperatore, quem sola provehit electio voluntaria ; sufficit domino comiti Roberto fratrem esse tanti regis \footnote{ MATHIEU PARIS, {\itshape Chronica majora}, éd. F. LIEBERMANN, {\itshape Monum. Germaniae}, {\itshape Scriptores}, t. XXVIII, p. 181.} »}. Les rois, de Louis VIII à Louis X, s’étant succédé de père en fils, il n’y a pas eu, jusqu’en 1316, de problème successoral. Seulement, à la mort de Louis VIII, une minorité, qui fut troublée, mais ne provoqua aucune tentative pour détrôner l’héritier légitime. Des difficultés ne naquirent qu’à la mort de Louis X.\par
Louis X mourut à Vincennes, le 5 juin 1316. Il laissait une fille, Jeanne. En outre, la reine Clémence était enceinte. Qu’allait-il se passer si l’enfant à naître était une fille ? La partie se jouait entre le frère du roi, Philippe, comte de Poitiers, et le duc de Bourgogne, Eudes, qui, étant le frère de la première femme de Louis X et par suite l’oncle de Jeanne, née de ce mariage, se considérait comme chargé de défendre les intérêts de cet enfant. Philippe, accouru de Lyon, s’empara du Louvre par surprise et prit en mains le gouvernement — c’est la première fois qu’apparaît le titre de régent — et l’on s’accorda à laisser les choses en état jusqu’à la naissance. La reine, le 15 novembre, accoucha d’un fils Jean (1\textsuperscript{er}) ; mais l’enfant mourut cinq jours après sa naissance. Philippe semble avoir hésité, mais en décembre, il prit le titre de roi et se fit sacrer à Reims dès le 9 janvier 1317, en l’absence des plus hauts barons et de son frère même. Une résistance s’esquissait. Depuis la fin du règne de Philippe  \phantomsection
\label{p13} le Bel, l’opposition au pouvoir royal se manifestait volontiers sous la forme de ligues provinciales qui réunissaient dans un territoire donné les nobles et parfois quelques bonnes villes. Le même procédé fut naturellement employé par les partisans de \emph{Madame Jeanne}. Des réunions de cette sorte eurent lieu dans le duché de Bourgogne par les soins du duc, et en Champagne (Louis X, comme ses frères, étant issu du mariage de l’héritière de la maison de Champagne avec Philippe le Bel, sa fille pouvait compter dans cette province sur la force d’un sentiment de légitimité dynastique).\par

\begin{itemize}[itemsep=0pt,]
\item[]\listhead{Par ailleurs, Philippe fit appel à l’opinion :}
\item en convoquant à Paris une assemblée de nobles, de prélats, de bourgeois de la ville et de docteurs de l’Université, par laquelle il se fit approuver ;
\item en envoyant dans tout le royaume des commissaires chargés de recruter des adhésions et surtout d’empêcher les sujets de s’associer aux ligues.
\end{itemize}

\noindent En fait, la guerre se borna à quelques escarmouches des forces royales avec le comte de Nevers. Le duc de Bourgogne se laissa acheter par un mariage avec la fille de Philippe V qui, héritière par sa mère de l’Artois et de la comté de Bourgogne, lui apportait l’expectative de ces beaux fiefs. Même la Navarre et la Champagne restaient, du moins à titre provisoire, à Philippe.\par
Philippe lui-même mourut le 2 ou le 3 janvier 1322, sans enfant mâle. Le troisième fils de Philippe le Bel, le comte de la Marche, Charles, succéda cette fois sans difficulté, gardant toujours Navarre et Champagne.\par
Lorsqu’il mourut lui-même, le 1\textsuperscript{er} février 1328, n’ayant que des filles, et que la reine enceinte à sa mort, eut le 1­\textsuperscript{er} avril accouché d’une fille, le point qui avait décidé en 1317 parut si bien acquis que personne ne songea à revendiquer l’héritage pour les filles. La seule question qui se posa, fut de savoir qui l’on devait préférer, des parents en ligne masculine — dont le plus proche était le cousin germain du roi, Philippe de Valois — ou des parents en ligne féminine, plus proches que ce dernier (le petit-fils de Philippe V par la duchesse de Bourgogne, sa fille, ou le neveu des trois derniers rois, par leur sœur, Édouard d’Angleterre). Par une conséquence naturelle de la décision prise en 1317, la ligne masculine l’emporta. Et cela sans beaucoup de difficultés. Edouard III ne devait prendre le titre de roi de France que bien plus tard, le 25 janvier 1340 \footnote{ P. VIOLLET, {\itshape Comment les femmes ont été exclues en France de la succession à la couronne, Mém. Acad. Inscript}., t. XXXIV, 2 (1893).}.\par
Quelques problèmes se posent à propos de ces événements.\par
\bigbreak
\noindent \labelchar{1)} Pourquoi se décida-t-on contre les filles ? Chacun sait que, plus tard, on devait en ce sens invoquer la \emph{Loi salique}. En effet, celle-ci (LIX, 5) exclut les femmes des successions immobilières. Mais le rapprochement avec la règle des successions royales — qui d’ailleurs ne s’imposait pas — ne fut fait que plus tard (pour la première fois à notre connaissance, en 1358, par un moine de Saint-Denis, Robert Lescot). Peut-être çà et là s’était-on avisé, dans les milieux de juristes du XIII\textsuperscript{e} siècle, de rappeler la Loi salique pour justifier les restrictions à la succession à la ligne féminine \footnote{ J.-J. HISELY, {\itshape Histoire du comté de Gruyère}, Lausanne, 1851, t. I, p. 341. {\itshape (Mémoires et documents publiés par la Société d’histoire de la Suisse romande}, t. IX).}. Mais c’étaient là des fantaisies savantes et nous n’avons aucune preuve que le mot ait été prononcé en 1317, ni en 1322, ni en 1328.\par
 \phantomsection
\label{p14} Quelles ont donc été les raisons ? En un sens, la décision est surprenante. Car l’hérédité féminine des grands fiefs était depuis longtemps acceptée. Les barons de 1317 avaient certainement entendu parler des comtesses Jeanne et Marguerite de Flandre qui, de 1202 à 1280, avaient successivement gouverné le comté. Ils voyaient parmi eux la comtesse d’Artois, Mahaut, dont les droits aux dépens de son neveu Robert, fils de son frère, avaient été reconnus par le Parlement en 1309. Bien mieux, le comte de Poitou lui-même avait obtenu de Louis X que, contrairement aux prescriptions de Philippe le Bel, son apanage passerait aux filles, à défaut d’hoirs mâles. Ainsi, nous voyons du moins que le droit à la couronne semblait chose {\itshape sui generis.} À dire vrai, il n’est pas certain qu’on ait beaucoup songé, en 1317, à une décision de principe. Une seule chronique — Continuateur de Guillaume de Nangis — nous dit qu’à l’assemblée de Paris on ait expressément exclu les femmes de la succession. On envisagea bien plutôt un problème de circonstances. Les fidèles de la royauté redoutaient une minorité, et d’une fille. Ils se rallièrent au représentant adulte de la dynastie.\par
\bigbreak
\noindent \labelchar{2)} Mais comment cette attitude fut-elle si aisément acceptée de beaucoup de personnages qui n’avaient pas intérêt à une royauté forte ? C’est que, comme on le verra, l’opposition à la royauté, si elle subsistait toujours assez vive, avait depuis le début du siècle changé d’aspect. On acceptait le pouvoir royal comme un fait et, plus ou moins consciemment, comme une nécessité. On cherchait simplement à le réformer et à le diminuer par le dedans. C’est tout le sens du mouvement des ligues.\par
Les événements de 1317 à 1322 sont donc la meilleure preuve de l’affermissement de la monarchie.
\subsection[{2° Les rois}]{2° Les rois}\phantomsection
\label{c03b2}
\noindent Ces rois qui se sont ainsi succédé paisiblement, quels étaient-ils ? Et de quelle manière ont-ils contribué en personne à l’affermissement de leur pouvoir ? Grosse question, et difficilement soluble. Prenons-les un à un.\par
Louis VIII (14 juillet 1223-8 novembre 1226) semble avoir paru à ses contemporains une figure assez pâle à côté du personnage haut en couleur qu’était son père, et frappé surtout par ses différences avec lui. Nous entrevoyons, à travers les propos conventionnels des chroniqueurs, un homme de santé médiocre, très pieux (le premier des saints de la dynastie), assez soigneusement instruit, très pénétré de la grandeur de sa race et de sa mission.\par
À la mort de Louis VIII, son fils, né en 1214, avait 12 ans. Le véritable roi, pendant bien des années, fut la reine mère, Blanche de Castille. Nous savons qu’elle était énergique, extraordinairement impérieuse, violente même à l’occasion, et très pieuse.\par
Louis IX est le seul des rois de France de ce temps que nous connaissons bien. Parce qu’il passa pour un saint, l’était en effet, et que sa personnalité, incontestablement très forte, a frappé ses contemporains. Encore convient-il — comme M. Petit-Dutaillis l’a marqué avec finesse — de faire la place chez lui d’une évolution très nette vers une piété et un mysticisme grandissants. Nous savons qu’il était, au moins dans sa jeunesse, un beau chevalier. \emph{« Le roi, dit Salimbene qui l’a vu de près, était mince et grêle, maigre comme  \phantomsection
\label{p15} il convient et de grande taille, avec un visage d’ange et une face pleine de grâce \footnote{ SALIMBENE, {\itshape Cronica}, éd. Holder-Egger, {\itshape Monumenta Germaniae, Scriptores}, t. XXXII, p. 222.} »}. Il était brave, mais médiocre chef militaire. Sur sa piété, son goût des propos à la fois sérieux et enjoués, tout a été dit, d’après Joinville et ses autres biographes, sur son empire sur lui-même, bien rare en ce temps et dans sa race, et qui lui permettra de dompter un caractère originellement fort emporté. Vers la fin de sa vie, on le voit de plus en plus austère, la santé ruinée par l’ascétisme, volontiers distant avec les siens. La croisade d’Égypte avait été mal préparée. Celle de Tunis fut une invraisemblable folie. Son règne avait été marqué par des traits où s’accusa l’empreinte de sa forte personnalité. Tout en lui n’était pas populaire. Il s’entoura volontiers de clercs, notamment de Mendiants, ce qui provoquait le mécontentement de l’opinion, dont Rutebeuf fut l’interprète, et qui s’exprima un jour par la bouche d’une femme qui était venue plaider en sa cour. Guillaume de Saint Pathus :\par

\begin{quoteblock}
\noindent « ... et une foiz quant le Parlement seoit à Paris et li benoiez rois fust descendu de sa chambre, la dite femme qui fu el pié des degrez li dist « Fi ! fi ! Deusses tu estre roi de France ! Mout miex fust que un autre fust roi que tu ; car tu es roy tant seulement des Freres Meneurs et des Freres Preecheurs et des prestres et des clers. Grant domage est que tu es roy de France, et c’est grant merveille que tu n’es bouté hors du roiaume \footnote{ GUILLAUME DE SAINT-PATHUS, {\itshape Vie de saint Louis}, éd. H. F. Delaborde, coll. Picard, Paris, 1899, p. 118.} ». Le roi dit qu’en effet il était indigne et lui fit donner de l’argent.\end{quoteblock}

\noindent Plusieurs décisions qu’il promulgua sous l’influence de l’Église mécontentèrent vivement la noblesse et ne purent être rigoureusement maintenues par ses successeurs : telle l’interdiction du duel judiciaire, celle des guerres privées et du port d’armes, les mesures qu’il prit contre les tournois. Il fut le roi de l’Inquisition, et il y a contre lui l’indice d’une hostilité assez vive des populations du Midi. Il lui arrive, dans les villes, d’intervenir assez vigoureusement contre les libertés urbaines et contre les révoltes des communes (voir Beauvais). Il était certainement profondément pénétré du sentiment de l’autorité royale. Pris pour arbitre entre le roi d’Angleterre et ses barons, la sentence qu’il rendit à Amiens le 24 janvier 1264 fut de tous points favorable au premier. Mais il conféra à la royauté un gros prestige, et tout n’était pas vaine illusion dans le souvenir qu’on devait garder plus tard du temps du bon roi saint Louis :\par

\begin{itemize}[itemsep=0pt,]
\item par sa sainteté personnelle et son évident souci du bien commun ;
\item parce qu’il fut toujours le maître chez lui et qu’on ne vit à sa cour ni favoris ni révolutions de palais ;
\item en vertu d’une certaine modération dans la pratique du gouvernement et dans la fiscalité. Il vendit la liberté à ses serfs, mais à un taux raisonnable (5\% des biens dans la châtellenie de Pierrefonds ; 10\% des meubles à Paray). On verra Philippe le Bel prétendre la vendre aux serfs du Languedoc pour le tiers de leur fortune !
\item parce que, si pénétré qu’il fût de son autorité, il la considéra toujours comme divine et bornée par le droit, c’est-à-dire par la coutume. Quelques textes sont à cet égard significatifs.
\end{itemize}

\noindent Celui-ci notamment : le roi, un jour, écoutait un sermon dans un cimetière. Dans une taverne voisine, des buveurs faisaient grand bruit, si bien qu’on avait peine à  \phantomsection
\label{p16} entendre le prédicateur. L’idée vint naturellement au roi d’envoyer ses sergents imposer le silence. Mais avant de le faire, il demanda à qui était la justice du lieu ; et ce fut seulement après qu’on lui eût répondu qu’elle lui appartenait, qu’il se décida à ce simple geste...\par
Avec Philippe III (25 avril 1270-5 octobre 1285) s’ouvre une période qui durera jusqu’à la fin de la dynastie capétienne : Philippe le Bel (qui meurt le 30 novembre 1314) et ses trois fils, dont nous avons vu les noms et les dates. Nous n’avons sur eux que peu de documents personnels — guère que des on-dit de contemporains plus ou moins mal informés. Aussi les connaissons-nous fort mal. Nous savons qu’ils parurent à leurs contemporains d’une belle prestance ; mais aussi qu’ils étaient probablement d’une santé délicate, car ils moururent tous, ou, comme Philippe le Bel, avant la cinquantaine, ou fort jeunes ; qu’ils étaient bons chevaliers, amateurs de tournois et de chasses ; très pieux par ailleurs, tous, uniformément, même Philippe le Bel qui fit ou laissa souffleter un pape. Nous voyons surtout que, dépourvus tout à fait de la forte personnalité d’un Philippe Auguste et d’un saint Louis, ils ont été dominés très étroitement par leur entourage. Celui-ci était loin d’être uni et les drames au palais ont joué un grand rôle dans la vie de ces souverains. Mais quelles que fussent ces discordes, la mentalité de ces hauts fonctionnaires royaux était, en gros, uniforme. Les règnes des derniers Capétiens ont été le règne d’un état-major.\par
Dans tous ces rois, d’ailleurs, un trait commun : le sens profond de la valeur de leur autorité et de leur mission quasi religieuse. Ils y ont été formés. Car ils ont tous été soigneusement instruits. Leurs maîtres leur ont parlé d’histoire et leur ont fait lire des livres d’histoire. Devenus adultes, ils se font composer des œuvres de cette espèce, comme cette grande compilation à la fois de l’histoire de Saint-Denis et de l’histoire de France que Philippe IV commanda à frère Ives. Ils y apprennent l’histoire de Charlemagne dont ils prétendent descendre depuis Philippe Auguste, en tout cas depuis Louis VIII (auquel un clerc, Gille de Paris, dédia un long poème intitulé le {\itshape Carolinus} et destiné à lui proposer l’exemple de son illustre aïeul) ; celle de leur dynastie, la grandeur d’une race qui commence à tirer de ses vieux souvenirs la prétention à une monarchie quasi-universelle. À partir de Philippe III s’y ajoute l’histoire de saint Louis, l’orgueil de descendre d’un saint. En 1298, nous voyons, par les Journaux du Trésor, le roi faire payer 20 livres parisis à un certain maître Pierre de la Croix (par ailleurs inconnu) pour avoir compilé une histoire de saint Louis. Il y a, dans les rois et autour d’eux, une idée et des sentiments monarchiques qu’il faut chercher à dégager.
\subsection[{3° L’idée monarchique}]{3° L’idée monarchique}\phantomsection
\label{c03b3}
\noindent Ce n’est pas uniquement une idée rationnelle. La conception que les hommes de ce temps se font des institutions politiques n’est pas plus pure d’éléments religieux et mystiques que leur conception du monde en général.\par
Que le roi fût un personnage sacré, que sa fonction mettait à part du monde des simples laïques, c’était, au XIII\textsuperscript{e} siècle, une idée déjà vieille. Mais une idée qui conservait encore toute sa force. Sans doute l’orthodoxie de la cour de France, l’influence de la notion beaucoup plus forte, depuis la réforme grégorienne, de la séparation  \phantomsection
\label{p17} du profane et du sacré empêchent qu’elle ne s’exprime avec autant de vivacité que par exemple en Allemagne, au temps de la querelle des Investitures, ou même qu’elle ne le fera plus tard, au temps du Grand Schisme. Elle subsiste cependant. Les docteurs l’expriment prudemment. \emph{« Les rois, écrit le cardinal français Jean le Moine, au temps de Philippe le Bel, qui sont oints, ne tiennent pas, à ce qu’il semble, le rôle de purs laïques ; ils le dépassent. »} Le peuple certainement pensait plus gros.\par
Ce caractère, on vient de voir que Jean le Moine l’attribue au sacre, ou plus précisément à l’onction. La thèse officielle était pourtant que le sacre n’était pas indispensable à l’exercice de la dignité royale. Le publiciste Jean de Paris, sous Philippe le Bel, se prononce expressément en ce sens, et il est significatif qu’à partir de Philippe III, la chancellerie ait pris l’habitude de dater les années de règne d’après l’avènement, non d’après le sacre. Mais l’idée populaire était plus simple. Une anecdote qui courait Paris vers 1314 et que nous a conservée le chroniqueur Jean de Saint-Victor atteste que l’on estimait communément ne pouvoir donner le nom de roi à l’héritier légitime qu’après le sacre \footnote{ JEAN DE SAINT VICTOR, dans {\itshape Recueil des Historiens de la France}, t. 21, p. 661 : \emph{« ... Ille enim quem tu regem Franciae reputas non est unctus adhuc nec coronatus, et ante hoc non debet rex nominari »}.}. De toutes façons d’ailleurs, celui-ci était une cérémonie dont on ne pouvait se passer. On le traitait vulgairement de \emph{sacrement}.\par
Il avait lieu, toujours, à Reims. Et l’on y procédait le plus tôt qu’on pouvait après l’avènement. La cérémonie était double : remise des insignes (dont le principal était la couronne) et onction. Ceci était le plus important. Les rois de France et d’Angleterre étaient oints non pas d’une huile bénie, comme les prêtres, mais d’un chrême (huile mélangée de baume), comme un évêque ; et oints sur la tête, comme les évêques. Mais le roi de France n’était pas oint d’un chrême quelconque. Depuis le IX\textsuperscript{e} siècle on tient pour certain que Clovis a été oint à Reims (il n’y avait, en fait, été que baptisé, et pour cause, le sacre étant une innovation carolingienne) et l’on raconte que, le jour du sacre, le prêtre qui apportait les saintes huiles, s’étant trouvé empêché par la foule d’arriver à temps, une colombe descendue du ciel avait apporté à saint Rémi une ampoule pleine d’un chrême miraculeux. L’ampoule était conservée à Reims et, depuis le XIII\textsuperscript{e} siècle au moins, on croyait communément que, bien qu’à chaque sacre on puisât quelques gouttes, le niveau du liquide ne baissait jamais. Les rois de France tiraient une grande gloire de ce sacre céleste, dont le privilège leur était particulier. Au XIII\textsuperscript{e} siècle, dans sa vie de saint Rémi, le poète Richier nous dit qu’ailleurs les rois doivent \emph{« lor ontions acheter en la mercerie »} ; en France seulement il en va autrement :\par

\begin{verse}
« Qu’onques coçons ne regratiers\\
N’i gaaingna denier à vendre\\
L’oncion... \footnote{ RICHIER, {\itshape La vie de saint Rémi}, 8143 et 8146-8, éd. W.N. Bolderston, Londres, 1912, p. 335.} »\\!
\end{verse}
\noindent Le caractère sacré des rois et celui, plus particulièrement accentué encore, du roi de France, se marquent par un autre trait encore. Comme Nogaret et Plaisians le disent dans un mémoire  \phantomsection
\label{p18} justificatif de 1310, \emph{« Dieu par ses mains opère en faveur des malades d’évidents miracles »}. Entendez que son toucher guérit les écrouelles. La tradition est solidement implantée depuis les premiers Capétiens (probablement depuis Robert le Pieux). Le pouvoir guérisseur, que les rois de France ne partagent qu’avec ceux d’Angleterre, n’est pas seulement souvent invoqué par les polémistes au service de la royauté capétienne. Il assure au roi médecin une vaste popularité. Comme les malades venus de loin recevaient une aumône et la recevaient seuls, des établissements des comptes de l’Hôtel nous ont conservé pour 1307 et 1308 des indications sur la provenance de ceux qui pour ces deux années (où la cour ne dépassa pas, au Sud, Poitiers) avaient accompli un voyage tant soit peu lointain. On les voit venir des régions éloignées : du Midi, de Bordeaux au Plantagenet, de grands fiefs, comme la Bretagne et la Bourgogne ; de l’Empire (Metz notamment et, dans le royaume d’Arles, Lausanne, la Savoie et Tarascon) ; d’Espagne ; en Italie, des villes lombardes, de Bologne, de Toscane et d’Ombrie. Cela, quatre ou cinq ans après Anagni ! Vraiment, il n’y a pas lieu de s’étonner que, vers 1230, un poète ait écrit des quatre fils de Louis VIII : \emph{« De saint lui sont venu »} ou que, sous Philippe le Bel, un écrivain italien, fidèle de la papauté cependant, Egidio Colonna, dédiant un ouvrage à Philippe le Bel, ait rédigé sa dédicace en ces mots : \emph{« Ex regia ac sanctissima prosapia oriundo... domino Philippo »}.\par
Ces sentiments sont d’une portée capitale. Mais naturellement ils n’épuisent pas l’idée qu’on pouvait se faire du pouvoir royal. Et il n’empêche que le roi n’ait à la fois des fonctions et des devoirs. L’idée de royauté absolue n’est pas médiévale.\par
Les obligations du roi sont essentiellement de donner la paix (c’est-à-dire l’ordre intérieur), de rendre bonne et miséricordieuse justice, enfin de protéger l’Église. Elles s’expriment traditionnellement dans le serment du sacre. Depuis 1226, le caractère ecclésiastique est accentué par la promesse de poursuivre l’hérésie.\par
L’hérédité même de la monarchie n’est pas liée à son caractère sacré : voir Byzance. Si fort qu’en soit le sentiment, toute trace d’élection n’a pas disparu au rite du sacre. Deux évêques demandent au peuple son assentiment. Acclamations. Mais depuis Louis IX (1226), cela n’a plus lieu qu’{\itshape après} la cérémonie.\par
Enfin, l’idée commune est que le roi ne peut prendre des décisions graves qu’après avoir pris conseil. \emph{« Tout soit il ainsi que li rois puist fere nouveaus establissemens, il doit mout prendre garde qu’il les face par resnable cause et pour le commun pourfit et sur grant conseil »}, écrit Beaumanoir \footnote{ BEAUMANOIR, {\itshape Coutumes de Beauvaisis}, éd. Salmon, Paris, 1900, t. II, p. 264.}.\par
Nous verrons comment ces idées se sont traduites dans les faits.\par
Le problème du patriotisme. Citations.
\section[{C. Les appétits territoriaux de la royauté}]{C. Les appétits territoriaux de la royauté}\phantomsection
\label{c03c}
\subsection[{1° Définitions}]{1° Définitions}\phantomsection
\label{c03c1}
\noindent Le roi était le roi de France, le roi de tout le royaume. Mais selon les régions, son pouvoir était plus ou moins direct. Il l’était au maximum là où il était lui-même le seigneur de la terre, percevant le cens, rendant toutes les formes de justice. Puis suivait une série de degrés. La forme la moins directe et la moins efficace de son pouvoir était celle qu’il exerçait dans les grandes principautés territoriales, où il trouvait devant lui un duc ou comte possesseur de tous les anciens pouvoirs comtaux, protecteur des églises, seigneur de fief de tous les petits et moyens seigneurs. D’où le double sens que les historiens — et peut-être même les hommes du temps — ont donné au mot de {\itshape domaine royal} : a) les terres dont le roi est — ne disons pas le propriétaire — du moins le seigneur direct et parfois même l’exploitant ; b) au sens étroit, la région du royaume où le roi n’a entre lui et ses sujets de tous rangs — seigneurs justiciers compris — l’intermédiaire d’aucun prince territorial.\par
Or l’œuvre de la royauté a été double.\par

\begin{enumerate}[itemsep=0pt,]
\item Elle a travaillé à imposer dans tout le royaume son autorité à tout le monde : hauts barons, grands princes de la terre. C’est ce que nous verrons plus tard.
\item Mais elle a également cherché à dominer le plus directement la terre elle-même, éliminant grands princes et même petits seigneurs. C’est à proprement parler l’œuvre de rassemblement territorial que nous allons étudier présentement \footnote{ Consulter à ce sujet : A. LONGNON, {\itshape La formation de l’unité française}, Paris, 1922. Titre trompeur ; recueil de renseignements commodes. — Y joindre l’{\itshape Atlas historique} du même auteur, publié à partir de 1885. Avec le \emph{texte explicatif des planches}.}.
\end{enumerate}

\subsection[{2° Les petites annexions}]{2° Les petites annexions}\phantomsection
\label{c03c2}
\noindent Dans cette œuvre de rassemblement, on peut distinguer deux parties : les annexions de grande envergure et les tentatives d’annexion (que nous énumérerons tout à l’heure) : — et la multitude des petites annexions, dont il faut tout d’abord dire un mot.\par
Ces petites annexions ne doivent pas être négligées. Elles ont été un des instruments les plus importants de l’autorité royale. Le roi ici acquiert une terre, là une justice, ailleurs un château, point d’appui militaire important, ou encore l’avouerie sur une église. Parfois encore un comté. Parfois même une simple mouvance. Montpellier était divisé entre deux seigneuries : Montpellieret, à l’évêque de Maguelonne ; Montpellier, à un seigneur propre qui était, à la fin du XIII\textsuperscript{e} siècle, le roi de Majorque dom Jaime, cadet de la maison d’Aragon. Mais cette seigneurie même était tenue en fief de l’évêque. En 1293, celui-ci vend à Philippe le Bel Montpellieret et son droit sur Montpellier.\par
Par exemple, Louis VIII n’a pas seulement consolidé la domination royale sur le Poitou, enlevé aux Plantagenets, ou sur une partie du Midi toulousain. Il a également acquis le comté du Perche, en Picardie les châteaux de Doullens et de Montreuil, la seigneurie  \phantomsection
\label{p20} d’Avesnes le Comte, des droits sur la ville de Saint-Riquier ; en Normandie, la seigneurie d’Aubigny en Cotentin ; en Anjou la seigneurie de Beaufort-en-Vallée ; les seigneuries de Neuville en Beine et de Remigni ; il s’est fait reconnaître le droit de mettre des garnisons dans tous les châteaux du Ponthieu. Philippe III, outre l’acquisition du comté de Guînes dans le Nord et celle du port de Harfleur dont l’intérêt saute aux yeux, a également donné au domaine (au sens étroit) des châteaux et des tours en grand nombre, en Picardie, en Normandie, dans l’Ile de France et en Berry. Langlois l’a dit justement à propos de ce roi, et il faut le répéter pour les autres : \emph{« chacune de ces opérations est insignifiante en elle-même, mais en s’additionnant, elles acquièrent une importance extrême »}.\par
Quels procédés ? Voici au moins les principaux.\par
\bigbreak
\noindent \labelchar{a)} {\itshape L’achat.} Le roi est riche. Et beaucoup de seigneurs ne le sont pas. Par exemple, telle est l’histoire de l’acquisition du comté de Guînes. Arnoux III était un brillant chevalier, fort prodigue. Il fut réduit à la misère. Il vendit quelques châteaux à son voisin, le comte d’Artois \footnote{ Sur la vente au comte d’Artois, {\itshape Inventaire des Archives du Pas-de-Calais}, série A, p. 43, col. 1.}. Puis au roi, en 1281, son comté même, moyennant une grosse somme payée comptant, une pension viagère et le paiement de ses dettes, qui étaient immenses. Ainsi s’exprime l’acte de vente lui-même, où le comte parle sans ambages de sa \emph{pauvreté} et de la crainte où il eût été (sans cette vente) de finir par \emph{mendier... honteusement}. Le fils du comte essaya de revendiquer son droit de retrait. Le Parlement le débouta, on ne sait pour quels motifs. Mais en 1295, les héritiers obtinrent de rentrer en possession.\par
Parfois, le roi pouvait acheter sans rien débourser, simplement en cédant certains de ses droits. Par exemple, en 1224, Louis VIII acquit des religieux de Homblières les terres de Neuville en Beine et de Remigni, seulement en exemptant ces moines de certains droits dont le plus important était l’ost. Ces achats ont été {\itshape très} nombreux. On peut leur joindre les échanges, où se marque une politique adroite d’arrondissement.\par
\bigbreak
\noindent \labelchar{b)} {\itshape La déshérence.} Le roi a droit aux héritages tombés en déshérence : tel le comté du Perche sous Louis VIII.\par
\bigbreak
\noindent \labelchar{c)} {\itshape La confiscation.} L’exemple le plus illustre est, bien entendu, le déshéritement de Jean sans Terre. Mais ce n’est pas le seul. Par exemple, le Ponthieu avait été confisqué — pour trahison de l’héritière et de sa mère — sous Philippe Auguste. Les acquisitions de Louis VIII dans cette région sont le résultat d’une transaction par laquelle furent rendus à la comtesse une partie seulement des terres et droits paternels.\par
\bigbreak
\noindent \labelchar{d)} {\itshape Les pariages.} On appelle ainsi un accord entre un seigneur — qui, le plus souvent, mais non toujours, est une église — et le roi. Moyennant protection, le seigneur associe le roi à ses droits sur une terre donnée. Ces actes ont été très nombreux. Philippe III, à notre connaissance, n’en a pas conclu moins de quatorze. Souvent l’objet du pariage était la fondation d’une ville neuve d’un commun accord : par exemple, en 1279, Beaumont de Lomagne fondée en pariage avec l’abbaye de Grandselve. Certains de ces pariages servaient avant tout des fins politiques. Par exemple :\par
En 1226 : pariage de Louis VIII avec les moines bénédictins de  \phantomsection
\label{p21} Saint-André sur une colline qui dominait Villeneuve-lès-Avignon (qui voulait échapper par là à la ville voisine). Le roi perçoit la moitié des droits de justice sur la ville de Saint-André ; il recevra le serment de fidélité des habitants ; surtout, il pourra y élever une forteresse et y tenir garnison.\par
En 1273 : pariage avec l’abbaye de Montfaucon en Argonne, sur la frontière contestée de l’Est et, juridiquement, en terre d’Empire.\par
Le \emph{Pariage} de Mende, conclu en février 1307 entre l’évêque et Philippe le Bel, à la suite de longues difficultés. Le Gévaudan (dont l’évêque revendiquait le titre de comte) est divisé en trois parties : terre épiscopale ; terre royale ; terre commune soumise au pariage. L’acte permettait à l’évêque d’asseoir une autorité contestée par le baronnage local.\par
Le pariage du Puy, conclu le 31 mars 1305, par où l’évêque associe le roi à son autorité sur la ville épiscopale.\par
Le pariage, instrument de pénétration.
\subsection[{3° Les pertes}]{3° Les pertes}\phantomsection
\label{c03c3}
\noindent L’inverse du tableau. Les diminutions du domaine. Il faut joindre ici les annexions petites ou grandes.\par
Les pertes ont pris divers aspects.\par
\bigbreak
\noindent \labelchar{a)} {\itshape Le simple grignotage.} D’où — très important, depuis Philippe le Bel — la recherche des droits usurpés. Par exemple, en 1288, le Parlement constate que \emph{« ceux qui, tenant dans la sénéchaussée de Beaucaire des terres en cens du roi, usurpent au-delà de ce qui leur appartient et refusent d’exhiber l’acte de donation, de peur de laisser constater leur fraude »}, devront présenter leurs titres.\par
\bigbreak
\noindent \labelchar{b)} {\itshape Les dons.} Extrêmement fréquents depuis le règne des favoris, sous Philippe le Bel et ses fils. Nous possédons l’édifiant cartulaire d’Enguerrand de Marigny. \emph{Quant aux Chambly}, écrit Langlois, \emph{« l’ensemble des chartes... que Philippe III et Philippe IV leur ont accordées est si considérable que le texte en fournirait aisément la matière de plusieurs volumes in-8° »}. Remède : les confiscations. Avec parfois exécution, ou même sans, comme celle dont fut l’objet, sous Charles IV, le seigneur de Sully très en faveur sous Philippe V. Il y eut même des essais de réglementation depuis 1318 et une grande enquête en restitution, ouverte en 1321 et confiée à un commissaire qui fit en effet rendre gorge à quelques personnages plus ou moins comblés sous les règnes précédents.\par
\bigbreak
\noindent \labelchar{c)} {\itshape Les apanages.} On désignait ainsi en droit privé les terres remises au cadet. Les rois faisaient de même. Mais les apanages ne prirent une véritable ampleur que depuis Louis VIII. Celui-ci, au moment de son testament, avait cinq fils. Le dernier devait être clerc. L’aîné, naturellement, roi. Aux trois autres, il légua des portions importantes des acquisitions récentes : à Robert l’Artois, à Jean l’Anjou et le Maine, à Alfonse le Poitou et l’Auvergne. Ainsi fut fait avec substitution à Jean, mort en bas âge, d’un fils né depuis, Charles (le cinquième fils, destiné à la cléricature, mourut également). Il est assez difficile de voir l’idée : \emph{« ne possit inter eos discordia suboriri »} dit le testament. Faut-il croire aussi : difficulté d’administrer ; idée que les acquisitions récentes sont vraiment des acquisitions dont le roi peut disposer plus aisément que des propres ; raisons personnelles (Blanche de Castille) ? En tout cas, les  \phantomsection
\label{p22} successeurs semblent avoir senti un danger. Les apanages furent beaucoup plus modestes sous saint Louis et Philippe III. Ceux de Philippe IV plus étendus (surtout le Poitou) et il compléta celui de son frère Charles en Valois.\par
Les apanages sont (depuis Louis VIII) transmissibles seulement en ligne directe.\par
Un hasard a limité le danger. Des grands apanages, seuls subsistent encore, en 1328, l’Artois et l’Anjou (ce dernier passé par mariage à Charles de Valois).
\subsection[{4° Les grandes opérations}]{4° Les grandes opérations}\phantomsection
\label{c03c4}
\subsubsection[{a) La liquidation de l’héritage angevin}]{a) La liquidation de l’héritage angevin}

\begin{itemize}[itemsep=0pt,]
\item[]\listhead{Les événements antérieurs doivent être rappelés en deux mots :}
\item conquête de 1066,
\item mariage de 1152,
\item confiscation de 1202 et la conquête de 1202-1204.
\end{itemize}

\noindent À l’avènement de Louis VIII, la situation était la suivante : Jean sans Terre, mort en 1216, était remplacé par Henri III. Il y avait trêve (qui devait expirer à Pâques 1224). La Normandie, le Maine, l’Anjou, la Touraine, étaient solidement occupés. En Bretagne, si longtemps fief des Plantagenets, Philippe Auguste avait fait épouser l’héritière Alix à un Capétien, Pierre Mauclerc, descendant de Louis le Gros, moyennant hommage lige au roi de France. Le Poitou, en revanche, avait été en grande partie reconquis par les soldats des Plantagenets. Le Périgord et le Limousin étaient disputés. La Guyenne intacte.\par
La guerre ayant repris à la date prévue, Louis VIII, grâce à une alliance avec Hugues de Lusignan, comte de la Marche, conquit le Poitou où force lui fut de faire à son allié de sérieuses concessions. Mais il échoua absolument dans sa tentative pour prendre pied en Guyenne.\par
La lutte ouverte sous Philippe Auguste, reprise ainsi sous le règne suivant, devait — coupée de trêves — se poursuivre encore pendant toute la première partie du règne de saint Louis \footnote{ Pour la période 1259-1328 : E. DÉPREZ, {\itshape Les préliminaires de la guerre de Cent Ans}, Paris, 1902, chap. I. {\itshape (Bibl. des écoles françaises d’Athènes et de Rome}, fasc. 86).}. Mais en 1242, la tentative d’Henri III (soutenu cette fois par les Lusignan dont le chef avait épousé la mère du roi anglais) pour reconquérir le Poitou, avait abouti à un échec complet, et il avait de graves embarras internes. Les deux rois s’accordèrent, le 28 mai 1258, par le traité de Paris.\par
Le traité fixait une frontière nette. Le roi renonçait à tout ce qu’il pouvait avoir — sauf encore mouvances féodales — dans le Limousin, le Périgord. Il reconnaissait l’expectative de la Saintonge au sud de la Charente et de l’Agenais, au cas où Alfonse de Poitiers mourrait sans héritiers directs. En revanche, Henri III renonçait absolument à la Normandie, l’Anjou, la Touraine, le Maine, le Poitou. Enfin, il se reconnaissait l’homme lige du roi pour ses possessions continentales. Enfin, saint Louis lui devait payer pendant deux ans de quoi entretenir 500 chevaliers.\par
 \phantomsection
\label{p23} Le traité a été pour saint Louis avant tout un avantage féodal, \emph{« il n’estoit pas mon home, si en entre en mon houmage \footnote{JOINVILLE, éd. Natalis de Wailly, chap. CXXXVII, p. 458.} »}.\par
En fait, il ne termine rien. Ses stipulations furent en somme correctement observées à la mort d’Alfonse de Poitiers.\par

\begin{listalpha}[itemsep=0pt,]
\item Clause de sûreté : le roi d’Angleterre doit faire jurer à ses vassaux immédiats en Guyenne de ne prêter main-forte qu’au roi de France, au cas où le duc de Guyenne violerait le traité.
\item Rien sur le problème judiciaire.
\item L’Agenais a été remis en 1279 ; la Saintonge en 1286 (l’Aunis fut racheté par le roi de France). Le traité de Paris prescrivait qu’une enquête serait faite pour savoir si le Quercy faisait partie de la dot de Jeanne d’Angleterre, femme de Raimond VI, et fille de Henri II, et ne fut jamais rendu.
\item L’Agenais gardé par le roi de France en 1325.
\end{listalpha}

\noindent Mais les litiges territoriaux — qui en l’état des frontières étaient inévitables — n’étaient pas le plus grave. La grosse difficulté naissait des droits de seigneur de fief du roi (noter l’acquisition du Ponthieu en 1279, par héritage), notamment sous leur forme judiciaire. Depuis 1273, les appels des barons du Midi à la cour de France se multipliaient. En 1293, fait plus grave, c’est le roi d’Angleterre lui-même qui fait cela (à la suite de querelles de matelots). La guerre s’ensuivit, entraînant les coalitions qui vont devenir classiques : de l’Angleterre avec la Flandre, de la France avec l’Écosse. Elle se termina par une paix de {\itshape statu quo} en 1303. Elle reprit en 1324, à la suite d’incidents relatifs à une bastide de Guyenne ; se termina en 1325. Le roi de France gardait l’Agenais, Bazas et La Réole. Une nouvelle paix de la part du roi de France permit certaines restitutions qui ne furent pas faites. Grignotages ! \par
La guerre de Cent Ans devait sortir de cela.
\subsubsection[{b) Les tentatives vers la Flandre}]{b) Les tentatives vers la Flandre}
\noindent Le grand fief flamand. Sa richesse. Les tentatives sous Louis VI pour imposer un comte de son choix avaient échoué. Mais, sous Philippe Auguste, le roi — grâce à son mariage avec une nièce du comte Philippe d’Alsace et surtout à l’âpre adresse de sa politique — avait su réaliser à son profit un démembrement de la grande principauté. Toute la \emph{terre d’Artois} avait été réunie à la couronne — avec Arras la grande ville de finance, et tout un front de mer sur la Manche. Ajoutez qu’en plein cœur du comté se trouvait située la ville épiscopale de Tournai dont l’évêque était le chef ecclésiastique de la plus grande partie du comté lui-même. Or, l’église de Tournai était église royale et Philippe Auguste, en accordant une commune aux habitants — qui sur son invitation avaient fortifié leur ville — avait achevé de s’assurer là un point d’appui.\par
Depuis le règne de Philippe Auguste et pour de longues années, la prépondérance française s’affirme nettement dans le comté. Les comtes ou comtesses avaient un hôtel à Paris, pensionnaient des personnes de l’entourage royal et des hommes de loi du Parlement. Eux-mêmes résidaient souvent à la cour. Une intervention de saint Louis dans une querelle de famille devait marquer cette suprématie et achever de la consolider.\par
 \phantomsection
\label{p24} La maison qui, depuis 1191, possédait la Flandre tenait en même temps, dans les Pays-Bas, un grand fief d’Empire : le Hainaut. Elle était représentée depuis 1244 par une femme, la comtesse Marguerite, qui avait elle-même succédé à une autre femme, sa sœur Jeanne, fille comme elle du comte Baudouin IX, empereur de Constantinople et mort captif des Bulgares. Or, Marguerite avait eu une vie matrimoniale agitée. Elle avait d’abord épousé — âgée de 10 ans ! — un baron hennuyer, Bouchard d’Avesnes. Mais ce mariage pouvait être tenu comme nul, Bouchard ayant naguère reçu les ordres, et il fut en fait déclaré tel par l’Église sur l’invitation de la comtesse Jeanne, qu’avaient irritée les revendications de Bouchard sur l’héritage du feu comte, père de sa femme. Longtemps Marguerite resta fidèle à son mari. Mais, en 1222, Marguerite avait abandonné celui-ci et Jeanne lui fit épouser l’année suivante un chevalier champenois, Guillaume de Dampierre. Quand Marguerite hérita du comté, ce double mariage posa un grave problème. Car de l’une et de l’autre union, des fils étaient nés, Jean d’ »  Avesnes », déclaré bâtard par le pape, mais que l’empereur tenait pour légitime et qui revendiquait son aînesse ; Guillaume de Dampierre le Jeune, qui arguait de sa légitimité. On se décida à soumettre la question à l’arbitrage du roi de France. Celui-ci, en 1246, se décida pour une solution transactionnelle : l’héritage serait partagé ; Jean d’Avesnes aurait le Hainaut, Guillaume de Dampierre, la Flandre. Solution assez difficile à fonder en droit, mais qui répondait à la fois au désir de conciliation, au souci de faire coïncider les limites des principautés avec celles des États, France et Empire, et, enfin, à l’intérêt de la royauté capétienne, puisqu’elle diminuait un grand pouvoir princier. Jean d’Avesnes refusa de s’incliner. Il devint un ennemi acharné du roi de France comme de son frère utérin Gui de Dampierre, comte depuis 1278 (son frère aîné Guillaume n’ayant pas vécu). Mais les efforts de Jean et, après lui, de son fils pour lancer contre les Capétiens une grande coalition de l’Empire et des princes lotharingiens échouèrent lamentablement. Hostilité persistante des deux maisons.\par
Cependant la puissance de Gui de Dampierre qui semblait, vers le début du règne de Philippe IV, le plus puissant des princes des Pays-Bas se trouva menacée par des troubles internes et ceux-ci offrirent à la royauté française une nouvelle occasion de réaliser ses vieilles ambitions.\par
Au cours du XIII\textsuperscript{e} siècle, comme nous l’avons vu, les grandes villes flamandes étaient parvenues à soustraire leurs échevinages à l’autorité du comte. La ville avait désormais ses magistrats à elle. Mais ils ne représentaient que le patriciat. Or, depuis le milieu du siècle surtout, les gens des métiers, opprimés politiquement et économiquement par les grands bourgeois, s’agitaient. À Bruges et à Ypres, en 1280, la révolte éclata brutale. Dans ces troubles, le comte trouva une occasion de rétablir son pouvoir sur les administrations urbaines. Tout en s’attachant à donner quelque satisfaction au commun qui s’était adressé à lui, les mesures qu’il prit eurent surtout pour effet d’établir sur les autorités des villes un contrôle analogue à celui qui existait en France. Alors les patriciats, menacés à la fois par le comte et le commun, se tournèrent vers le roi.\par
Les agents royaux voyaient dans ce rôle une trop heureuse occasion. Déjà, en 1275, le Parlement était intervenu, avec une certaine modération d’ailleurs, entre les Trente-neuf de Gand et le comte. En 1289, Philippe le Bel fit plus. Il plaça des garnisons à  \phantomsection
\label{p25} Gand, puis à Bruges et Douai. La bannière fleurdelysée flottait sur les beffrois. Les patriciens furent les \emph{Leliærts}.\par
Alors le comte se tourne naturellement, lui, vers le Plantagenet. Une première tentative de mariage entre la famille comtale et celle d’Angleterre n’aboutit, en 1274, qu’à faire emprisonner pendant quelque temps le comte au Louvre. Première humiliation suivie, en 1296, d’une seconde : Gui s’étant refusé de rendre Valenciennes au comte de Hainaut qui venait de s’allier avec le roi, il fut privé de son comté par la cour du roi, et ne se le vit restituer que moins la ville de Gand. En 1297, Gui se décida à refuser la fidélité à son seigneur de fief — dans les formes, par une longue lettre de défi — et s’allia au roi d’Angleterre. Mais son armée n’était pas de taille et la défaite flamande fut consommée par l’abandon d’Edouard 1\textsuperscript{er}. En 1300, le comté fut définitivement occupé. Le comte, ses fils et ses barons furent emprisonnés dans les châteaux du Capétien.\par
Mais le gouverneur, Jacques de Châtillon, auquel Philippe le Bel avait remis le comté, administra durement, uniquement au profit de la noblesse et du patriciat et sans même ménager les intérêts économiques des villes. Il y eut rapidement des émeutes où l’on vit paraître au premier rang les tisserands : à Bruges à deux reprises, en 1301, puis à Gand. Enfin, dans la nuit du 17 au 18 mai 1302, les troupes de Châtillon qui étaient rentrées victorieuses à Bruges, y furent massacrées avec plusieurs patriciens. Châtillon parvint à grand peine à s’enfuir. Ce fut \emph{le vendredi de Bruges}. Un tisserand brugeois, Pierre de Coninck et un prince de la famille comtale, Guillaume de Juliers, prirent la tête du mouvement, auquel s’unirent presque toutes les autres villes, sauf Gand, et beaucoup de paysans de la région maritime surtout. Une armée, composée surtout d’infanterie, fut formée. Sous les murs de Courtrai le 11 juillet 1302, elle triompha de la chevalerie française de Robert d’Artois. Un des principaux conseillers de Philippe le Bel, Pierre Flote, succomba dans la bataille, avec Robert.\par
Ce fut pour le roi et son entourage une grave humiliation. Mais quelque difficiles que fussent les campagnes dans les boues de Flandre, coupées de ravins et de fossés, au milieu d’un pays hostile où l’approvisionnement des armées posait de redoutables problèmes, la force des Capétiens, en hommes et en argent, était très grande. Et l’acharnement du roi ne se démentit pas. Après la bataille indécise de Mons en Pévèle (1304), où le roi fut un moment en danger, le comte, que Philippe le Bel avait au cours d’une trêve remis en liberté, se décida à traiter. La paix définitive fut conclue, après sa mort, par son fils Robert de Béthune, à Athis-sur-Orge, en 1305.\par
Elle fut dure. Dans ses formes : serment de fidélité au roi prêté à nouveau par le comte et comportant confiscation de sa terre, s’il le viole ; serment d’observer la paix prêté par tous les Flamands et renouvelé en certaines occasions ; envoi en pèlerinage de 300 bourgeois de Bruges. Dans ses clauses financières : lourde amende. Militaires : ordre de raser toutes les forteresses des villes. Territoriales : jusqu’à l’accomplissement intégral, le roi garde en gage les châtellenies de Lille, Douai et Béthune. En 1312, Robert cité devant les pairs pour diverses preuves d’infidélité ( ? ) dut (moyennant une concession sur les clauses pécuniaires) transformer cette cession provisoire en une cession définitive. Par ce nouveau démembrement, la Flandre \emph{wallone} devenait française.\par
 \phantomsection
\label{p26} Cependant il ne se résignait pas. Les clauses de destruction des forteresses et les clauses financières ne furent jamais exécutées. La guerre reprit en 1315, à la suite du refus d’hommage au nouveau roi, Louis X. Elle se termina, en 1320, par une paix de statu quo. La question de Flandre était encore ouverte. À dire vrai, elle prit une allure nouvelle lorsqu’en 1322, Robert de Béthune eut pour successeur son petit-fils Louis de Nevers qui, élevé à la cour de France, entouré de conseillers français, sera un très fidèle vassal du roi. La révolte des paysans de la Flandre occidentale, en 1324, fut provoquée par la tentative pour lever les impôts exigés par les indemnités d’Athis, rappelées en 1320, et par les exactions des nobles de retour après la paix. Elle fut soutenue par les métiers de Bruges et d’Ypres. Le roi fit lancer l’interdit sur les révoltés. À la mort de Charles le Bel, le comte était à Paris, sollicitant l’intervention de l’ost royal. Il devait l’obtenir de Philippe VI, dès après l’avènement (Cassel, 23 août 1328). Plus tard, la question flamande sera une des origines de la guerre de Cent Ans, mais sous la forme d’une révolte des villes contre le comte, allié au Valois.\par
En somme, succès partiel de la royauté. Pénétration vers le Nord. Mais annexion de la région la plus riche impossible.
\subsubsection[{c) Le Midi}]{c) Le Midi}
\noindent Pour comprendre les événements, il faut remonter à la croisade dite des Albigeois (depuis 1209). Philippe Auguste avait laissé faire en contrôlant. En 1215 il avait autorisé le futur Louis VIII, qui avait fait vœu. de croisade, à une chevauchée en Languedoc. En 1216 il avait accepté l’hommage de Simon pour les terres conquises. Mais le 25 juin 1218, ce dernier fut tué sous les murs de Toulouse révoltée. Alors le pape Honorius III confirma au fils de Simon, Amaury, les terres de son père et demanda pour lui l’aide de Philippe Auguste. Ce dernier, par crainte de se voir supplanté dans ce rôle de chef de croisade par un grand vassal, Thibaut de Champagne, envoya une nouvelle armée avec Louis. L’ost royal mit à sac Marmande, mais ne put prendre Toulouse et s’en retourna. Le jeune comte, Raimond VII, se mit à reconquérir peu à peu son comté. Louis VIII, devenu roi sur ces entrefaites, ne demandait pas mieux que d’intervenir : mais à condition qu’Amaury lui cédât ses droits et que la papauté lui concédât une dîme sur les églises du royaume. Ce qu’il obtint, en 1226, à la suite de négociations où Rome et la cour de France avaient essayé de se jouer l’une l’autre.\par
Alors, le roi se mit en branle, passant, par la rive gauche du Rhône, en terre d’Empire. Il prit au passage Avignon, dont le marquis de Provence — c’est-à-dire le comte de Toulouse — et le comte de Provence, d’orthodoxie douteuse, étaient co-seigneurs. Il soumit presque pacifiquement le duché de Narbonne et une partie du Toulousain. Mais l’hiver arriva avant qu’il pût rien tenter contre Toulouse ; l’ost repartit pour le Nord et le roi mourut sur le chemin du retour.\par
Le gouvernement de Blanche de Castille était trop faible pour pousser la conquête jusqu’au bout. D’autre part, Raimond VII ne se sentait pas capable de reprendre tout son héritage. Un accord intervint qui fut le traité de Paris, d’avril 1229. Raimond, réconcilié avec l’Église, abandonna au roi le duché de Narbonne (où furent établies les sénéchaussées de Nîmes et de Beaucaire) et l’Albigeois méridional, c’est-à-dire le pays entre le Tarn et l’Agout. Comme, en  \phantomsection
\label{p27} même temps, deux seigneuries constituées par des croisés, celles de Castres et de Mirepoix, passaient sous la mouvance directe du roi, tout le passage était entre ses mains. Le reste de l’héritage restait à Raimond, soit, en gros : le Toulousain, la moitié Nord de l’Albigeois, le Rouergue, une partie du Quercy et l’Agenais, comme fiefs du royaume. En outre, le marquisat de Provence, fief d’Empire, fut cédé au pape. Mais en 1234, Grégoire IX le rendit à Raimond, sur les instances de saint Louis (la partie de la Provence située au Nord du Rhône). Importance des acquisitions de la couronne. Difficultés d’établissement. La Méditerranée, élément nouveau dans la politique française.\par
Mais ce n’est pas tout. Le traité de Paris comptait une autre clause encore. Raimond VII prenait l’engagement de marier sa fille Jeanne à un des frères du roi (qui fut en fait Alfonse de Poitiers). À sa mort, de toute façon, Jeanne devait hériter du Toulousain. Le reste à ses autres enfants, s’il en avait. Sinon, à Jeanne. En sorte que le comte Raimond étant devenu veuf, une des grandes préoccupations de la politique française fut de mettre obstacle à ses tentatives de remariage. Ce à quoi elle réussit. À sa mort, en 1249, Alfonse de Poitiers devint comte de Toulouse et — possédant en même temps le Poitou et l’Auvergne — le plus puissant baron de France. De loin, car il résida toujours à Paris ou dans ses châteaux des environs, il administra durement et habilement ses terres. Ainsi jusqu’à la mort d’Alfonse qui survint au retour de la croisade de Tunis, en 1271, pendant 22 ans par conséquent, tout ce qui dans le Midi n’était pas capétien (Guyenne à part) appartint à un prince capétien ou fut placé sous sa mouvance.\par
Il y a plus. Jeanne étant morte quelques jours après son mari, Philippe le Hardi put mettre la main sur la succession, écartant et les prétentions de Charles d’Anjou sur l’apanage, et celles d’un parent sur le reste. Simplement, Philippe le Hardi céda au pape le marquisat de Provence.\par
Ainsi tout le Midi était passé sous la domination royale. À cela s’ajouta — à la fois par héritage et rachat — l’acquisition, en 1293, de la Bigorre. Les possessions directes des Capétiens bordaient l’Espagne. Nous allons voir dans un instant qu’elles y pénétraient même.
\subsubsection[{d) L’héritage champenois}]{d) L’héritage champenois}
\noindent Depuis la ruine de la maison de Toulouse, et devant les graves difficultés intérieures de la Flandre, les comtes de Champagne de la maison de Blois étaient les plus hauts barons de France. À la mort du roi de Navarre Sanche VII, en 1234, le comte Thibaut IV, son neveu, fut reconnu pour roi de ce petit royaume. Ainsi la maison de Blois obtenait une de ces couronnes qu’elle avait tant de fois poursuivies. Durant la minorité de saint Louis, Thibaut avait été un vassal quinteux, tantôt ennemi, tantôt allié. On le disait amoureux de la reine mère. Ses deux fils, dont l’aîné était gendre de saint Louis, et qui lui succédèrent tour à tour, furent de fidèles vassaux. Lorsque le second d’entre eux, Henri III, mourut en 1274, il ne laissait qu’une fille. Elle fut fiancée au second fils du roi de France, Philippe. Comme ce dernier, par la mort de son aîné en 1270, devint l’héritier présomptif, ce mariage eut pour effet une importante réunion. La Navarre était un petit royaume turbulent et exotique, qui fut surtout une source d’embarras.\par

\begin{listalpha}[itemsep=0pt,]
\item[]\listhead{Mais la Champagne :}
\item proximité de Paris ;
\item les foires ;
\item la frontière de l’Empire.
\end{listalpha}

\subsection[{5° La france en europe}]{5° La france en europe}\phantomsection
\label{c03c5}

\labelblock{— Aspect culturel.}

\noindent Le rayonnement de la culture française s’est manifesté sous plusieurs formes : l’expansion de l’art. L’art gothique est certainement d’origine française. Il s’est répandu hors de France, soit par transport direct d’artistes (Villard de Honnecourt travailla en Hongrie), soit par simple imitation (par exemple à Cantorbery, imitation de Sens).\par
Le prestige de Paris comme centre d’études. Pour le clerc colonais Alexandre de Roes qui écrit en 1281 son traité {\itshape De Translatione imperii}, il n’est pas douteux que si l’{\itshape imperium} doit appartenir à l’Allemagne, la prééminence dans le {\itshape studium} a été donnée comme une sorte de compensation à la France. Il s’agit avant tout de l’Université de Paris. Remarquer d’ailleurs qu’elle est non seulement fréquentée, mais illustrée par des étrangers. Ferment d’universalisme intellectuel.\par
Le prestige de la littérature dont témoignent les traducteurs ou adaptateurs en allemand, par exemple, commencé dès le XII\textsuperscript{e} siècle avec la {\itshape Chanson de Roland}, continué au XIII\textsuperscript{e} avec des œuvres telles que le {\itshape Tristan} de Gottfried de Strasbourg (vers 1210), le {\itshape Parzival}, à peu près contemporain, de Wolfram von Eschenbach, et, placée plus ou moins directement sous ce signe, toute la floraison de romans courtois qui remplit le XIII\textsuperscript{e} siècle. De même en Italie où toute la légende carolingienne française est adaptée et amplifiée. Fait notable, c’est le landgrave de Thuringe qui fit connaître à Wolfram la légende de Guillaume d’Orange et lui prêta le poème français.\par
La langue : En Allemagne, surtout dans les Pays-Bas. Nous savons que le comte Florent V de Hollande (1256-1296) avait appris le français à l’école. Le Brabançon Adenet le Roi écrit vers 1275 dans son poème de {\itshape Berthe aux Grands Pieds} :\par


\begin{verse}
« Avoit une coutume enz el Tyois pays\\
Que tout li grant seignor, li comte et li marchis\\
Avoient en tour aus gent françoise tous dis\\
Pour apprendre françois leur filles et leur fis. »\\
\end{verse}

\noindent Une foule de mots français d’origine chevaleresque ont pénétré en Allemagne. En Italie, c’est en français que Martin du Canal, en 1267, écrit sa {\itshape Chronique de Venise} parce que, dit-il, cette langue a cours dans le monde entier et est la plus agréable à lire et à entendre ; comme, en 1298, le Vénitien Marco Polo, le récit de ses voyages.\par
Enfin, réputation de chevalerie. Wolfram d’Eschenbach, au début du siècle, dit que la France est :\par

\begin{quoteblock}
\noindent « der echten riterschafte land »,\end{quoteblock}

\noindent et bien plus tard Ottokar, qui déteste Philippe le Bel, ne peut mieux  \phantomsection
\label{p29} louer la bravoure des Hongrois au Marchfeld qu’en disant qu’ils se conduisent aussi bien \emph{« que s’ils avaient appris à se battre en France »}.\par
Il y a recul en Orient par suite de la destruction des colonies françaises de Syrie (prise de Saint-Jean-d’Acre, 1291), de l’Empire latin de Byzance (1261, Constantinople), le passage progressif sous l’influence vénitienne ou génoise des principautés latines des Balkans, la consolidation enfin ou la naissance des littératures nationales.\par

\labelblock{— Aspect politique.}

\noindent Il ne faut pas confondre ce rayonnement culturel avec l’influence politique. Résistances. Résistances à l’expansion conquérante, surtout vers la fin de la période. Les vers que Dante, Purgatoire, XX, 43, met dans la bouche de Hugues Capet (au sujet duquel il recueille la légende d’après laquelle il était \emph{« Figliuol fui d’un beccajo di Parigi »}).\par


\begin{verse}
« Io fui radice della mala pianta\\
Che la terra cristiana tutta aduggia\\
Si che buon frutto rado se ne schianta. »\\
\end{verse}

\subsubsection[{a) Affaires d’Espagne}]{a) Affaires d’Espagne}
\noindent Louis IX n’avait de ce côté aucune ambition. Fidèle à une politique de raison, qui s’adaptait aux nouvelles formations d’État, il renonça même, en 1258, à toute seigneurie de fief sur l’ancienne Marche d’Espagne, Catalogne et Roussillon, dont le comte, un peu plus d’un siècle auparavant, était devenu roi d’Aragon (1150).\par
Philippe III fut moins sage. L’affaire de Navarre, une querelle dynastique en Castille (où les neveux du roi de France avaient été dépossédés de la succession au profit de leur oncle) l’amena à une guerre avec la Castille, guerre qui, d’ailleurs, tourna court. Puis l’affaire angevine l’entraîna dans une autre aventure espagnole. Comme nous le verrons, le frère de saint Louis, Charles d’Anjou, était devenu roi de Sicile (c’est-à-dire de la Sicile et de toute l’Italie péninsulaire du Sud). Cela, par la victoire qu’il avait remportée contre le fils de Frédéric II, Manfred. Or, la fille de Manfred avait épousé Pierre III d’Aragon. À la faveur d’une révolte des Siciliens, en 1281, Pierre III occupa l’île et se fit couronner à Palerme. L’influence de Charles d’Anjou, soutenu par la reine Marie de Brabant, était alors toute puissante à la cour de France. Aidé par le pape Martin IV, qui était français, il s’efforça d’obtenir du roi de France une \emph{croisade} contre l’Aragon ; le roi excommunié, ses sujets déliés par le pape de leur serment de fidélité, un fils de Philippe III devait prendre le royaume. Comme naguère Charles en Sicile. Il semble y avoir eu autour de Philippe de sages résistances. Mais le roi se laissa entraîner. Une grosse armée fut réunie qui, après avoir pénétré en Aragon, dut se replier. Philippe mourut au cours de la retraite.\par
Philippe IV liquida l’affaire par une paix de statu quo. Le roi d’Aragon, en 1295, renonça à la Sicile, mais son fils qui la gouvernait s’y maintint en dépit du pape et des Angevins. Par ailleurs, satisfaits de garder la Navarre, ni Philippe IV, ni ses fils ne témoignèrent d’aucune ambition outre-pyrénéenne.
\subsubsection[{b) L’Empire .}]{b) L’Empire \protect\footnotemark \textsc{.}}
\footnotetext{ Bibliographie sommaire. F KERN, {\itshape Die Anfänge der franzäsischen Ausdehnungspolitik bis zum Jahre 1308}, 1910. Les travaux de Gaston ZELLER, {\itshape La réunion de Metz à la France}, t. I, Strasbourg, 1926. Introduction — {\itshape La France et l’Allemagne depuis dix siècles}, Paris, 1932 (coll. A. Colin). {\itshape Les rois de France candidats à l’Empire}, dans \href{http://gallica.bnf.fr/document?O=N018271}{\dotuline{{\itshape Revue historique} [http://gallica.bnf.fr/document?O=N018271]}}, t. 173,1934, p. 273-311, et 497-534. — E. BERGER, {\itshape Saint Louis et Innocent IV}, Paris, 1893 (en introduction aux Registres d’Innocent IV, t. I, Paris, 1884). — G. LIZERAND, {\itshape Philippe le Bel et l’Empire au temps de Rodolphe de Habsbourg (1285-1291)}, dans \href{http://gallica.bnf.fr/document?O=N018238}{\dotuline{{\itshape Revue historique} [http://gallica.bnf.fr/document?O=N018238]}}, 1923, t. 142, pp. 161-191. — P. FOURNIER, {\itshape Le royaume d’Arles et de Vienne (1138-1378). Étude sur la formation territoriale de la France dans l’Est et le Sud-Est}, Paris, 1891.}
 \phantomsection
\label{p30}
\begin{itemize}[itemsep=0pt,]
\item[]\listhead{Comment se posait le problème :}
\item L’Empire et ses prétentions à la monarchie universelle.
\item L’attitude à prendre entre le Sacerdoce et l’Empire.
\item Les ambitions territoriales.
\end{itemize}


\labelblock{— {\itshape Première période} : le temps de Frédéric II et la liquidation de son héritage.}

\noindent Les problèmes de Frédéric II. Ses États. Sa lutte avec la Papauté depuis 1227. Apreté de la lutte ; effort de propagande (comme le prouve la ligue anticléricale des barons de France, en 1246).\par

\begin{enumerate}[itemsep=0pt,]
\item[]\listhead{L’attitude de saint Louis :}
\item piété,
\item mais alliance traditionnelle avec les Hohenstaufen,
\item surtout idée de la grandeur de l’autorité royale, et peu de sympathie pour une Papauté trop militante ; idée de la croisade ; enfin nécessité de tenir compte de l’opinion qui, dans le baronnat et même dans le clergé, n’était pas particulièrement favorable au pape.
\end{enumerate}

\noindent Mathieu Paris raconte qu’un prêtre de Paris ayant reçu, en 1245, l’ordre de promulguer la sentence d’excommunication contre Frédéric II, dit : \emph{« J’en ignore la cause. Mais ce que je n’ignore pas, c’est la haine inexorable qui divise les deux adversaires. Je sais aussi que l’un d’eux fait tort à l’autre. Lequel ? Je ne sais. Mais c’est celui-là que j’excommunie, et j’absous celui qui subit l’injure, si funeste à la Chrétienté. »} Donc, neutralité, avec un souci de dignité. Elle s’est manifestée dans les épisodes suivants.\par
En 1240, Louis IX refuse l’offre de la couronne impériale faite par Grégoire IX à Robert d’Artois. En 1241, Grégoire ayant convoqué un concile à Rome, quelques prélats français qui s’y rendaient par mer furent faits prisonniers en route par les Impériaux. Louis IX réclama fermement et obtint leur libération.\par
Lorsque Innocent IV s’enfuit d’Italie, il fut presque invité à s’installer dans le royaume. Mais il s’établit à Lyon, sur sa frontière. Lorsqu’on crut que l’Empereur allait venir l’enlever à Lyon, saint Louis qui était alors en Bourgogne, semble avoir pris des mesures pour le repousser. Comme Frédéric fut retenu en Italie par la révolte de Parme, l’incident n’eut pas de suite.\par
Jamais saint Louis ne cessa les rapports avec l’Empire excommunié.\par
Mais, en 1250, la mort de Frédéric II ouvrit une crise singulièrement grave. Saint Louis était à ce moment en Syrie d’où il ne revint que quatre ans plus tard. En son absence, Innocent IV, acharné contre la race de vipère des Hohenstaufen, continua, non sans choquer beaucoup de monde, la lutte contre les deux fils de  \phantomsection
\label{p31} l’Empereur, le roi des Romains, Conrad II et le bâtard Manfred \footnote{Adam de la Halle a dit de lui : \emph{« Biaus chevaliers et grans et sage fu Manfrois. De toutes bonnes tèches entechiés et cortois. Et ne lui faloit riens fors que seulement fois. »}} qui, après la mort de Conrad, survenue en 1254, se rendit maître, aux dépens de son neveu Conradin, du royaume de Sicile qui devint pour les papes successifs (Innocent était mort en décembre 1254) le point brûlant et devait amener l’intervention d’un prince français, Charles d’Anjou \footnote{ E. JORDAN, {\itshape Les origines de la domination angevine en Italie}, Paris, 1909.}.\par
Ce personnage qui devait remplir son temps du bruit de son nom, était le dernier des fils survivants de Louis VIII : une personnalité très forte. Il partageait la piété profonde de saint Louis. Salimbene le vit, dans une église de France, s’attarder à prier après même le roi qui l’attendit. Épris de vie chevaleresque, dédaigneux volontiers des habitudes monacales de son frère, sans son humour non plus, on disait qu’il ne riait jamais. Mais il était entre autres d’une ambition dévorante, et volontiers impérieux. Bien qu’il fût encore jeune (né en 1226), il avait déjà devant lui, lorsqu’il fut question de lui pour le royaume de Sicile, un passé agité. Comte d’Anjou et de Maine par son apanage, son frère lui avait fait épouser, en 1246, l’héritière du plus grand fief du royaume d’Arles, Béatrice de Provence (il héritait d’ailleurs du comté au détriment de ses belles-sœurs plus âgées, respectivement mariées à saint Louis, à Henri III et à un prince anglais ; l’aînée, la reine Marguerite, ne lui pardonna jamais). Tout en cherchant à établir fortement sa domination sur des noblesses turbulentes et des républiques urbaines, il profita de l’absence de son frère pour se lancer dans une autre aventure, la conquête du Hainaut, cédé par la comtesse Marguerite de Flandre, en haine de son fils Jean d’Avesnes. Le retour du roi mit fin à cette tentative de spoliation caractérisée.\par
Mais il avait déjà été à ce moment — en 1253 — l’objet des sollicitations du pape Innocent IV qui lui offrit la Sicile, fief pontifical. À ce moment, l’aventure lui parut impossible. Mais un nouveau pape, Urbain IV, qui était Champenois, reprit la question, dès 1262, après avoir d’abord sondé, pour un de ses fils, Louis IX, qui refusa. À la candidature de Charles, Louis IX, non sans répugnance, se laissa gagner. Manfred, après tout, était un usurpateur et la Sicile, un fief susceptible de commise. En 1265, l’armée angevine — où suivait un grand nombre de chevaliers français — s’ébranla. Charles, après avoir passé par Rome dont il avait été fait sénateur, envahit le royaume ; en 1266, la bataille de Bénévent, dans laquelle périt Manfred, assura sa conquête. Deux ans après, il triompha de l’invasion de Conradin.\par
Ainsi, la dynastie capétienne se trouvait entraînée dans les complications italiennes. Nous avons vu comment, par contre-coup, elles jetèrent Philippe III en Espagne. Après la mort de l’empereur guelfe, Richard de Cornouailles, en 1272, la même influence devait lancer Philippe III dans une candidature à l’Empire qui ne fut qu’esquissée (élection de Rodolphe de Habsbourg, 1\textsuperscript{er} octobre 1273).\par

\labelblock{— {\itshape Deuxième période} : l’expansion française dans l’Empire.}

\noindent Les conditions de fait ; force des Capétiens. Leur influence sur la cour romaine. Depuis 1305, début de la Papauté avignonnaise.  \phantomsection
\label{p32} En face, faiblesse de l’Empire (Rodolphe, 1273-1291 — Adolphe de Nassau jusqu’à sa défaite, en 1298, par Albert qui régna jusqu’en 1308. Puis Henri VII, 1308-1313, et Louis de Bavière, 1314-1347). L’héritage champenois.\par
L’état d’esprit. Ni les rois des Romains ou empereurs, ni leur entourage n’ont abandonné les idées anciennes. Si Rodolphe de Habsbourg, pas plus que ses deux successeurs immédiats, n’a pu se faire couronner, il n’en a pas moins accordé beaucoup d’attention aux affaires d’Italie, même du royaume d’Arles et de Lotharingie ou Lorraine. Témoin le traité {\itshape De praerogativa imperii} (1281) d’A. de Roes ; en 1311, le {\itshape De Monarchia} de Dante. Henri VII et Louis de Bavière ont repris la politique Staufen. Mais forcément les ambitions de fait sont obligées de se limiter à des buts immédiats : l’Allemagne, l’Italie, plutôt que les marches occidentales.\par
En face, une monarchie autour de laquelle a grandi le sentiment à la fois de l’indépendance et celui de la mission.\par
— {\itshape Le problème de la monarchie impériale universelle} :\par
En un sens, il est moins aigu en raison du changement de thème de la propagande de Frédéric II (1211-1250) : thème de la solidarité des rois. Néanmoins, la royauté capétienne est sur ses gardes, non sans raison. Bien qu’Innocent III ait incidemment, dans une décrétale célèbre (de 1202 ou 1205), déclaré que le roi de France \emph{« superiorem in temporalibus minime cognoscat »}, la théorie des civilistes et souvent des canonistes reste fidèle à la vieille notion. Brouillé avec Philippe le Bel et réconcilié avec le roi des Romains, Albert d’Autriche, Boniface VIII, en 1302, affirma solennellement la subordination, {\itshape de jure}, du roi vis-à-vis de l’empereur. Henri VII proclamait encore cette thèse en 1312, dans une circulaire adressée aux souverains de l’Europe lors de son couronnement et irritait Philippe IV, lorsqu’il lui écrivit en mettant son nom à lui, Empereur, avant celui du roi. Philippe protesta, dans sa réponse, de son indépendance et reproduisit — en l’inversant à son profit — le modèle de l’adresse. Déjà au temps de saint Louis et Philippe III les coutumes françaises mettent l’accent sur le fait que le roi ne tient que de Dieu. Puis vient, dans l’entourage de Philippe le Bel, la formule : \emph{« Rex est imperator in regno suo »}.\par
Prestige miraculeux de la royauté française (sainte Ampoule, écrouelles). Il se retrouve contenu dans toutes les polémiques, jusque dans cette {\itshape Quaestio in utramque partem} où se rencontre aussi la mention du {\itshape Rex imperator} et qui eut l’honneur d’une copie sur les registres de la Chancellerie. L’éloquence sacrée répandait ces idées : sermon du dominicain Guillaume de Sanqueville, prononcé vers 1300, où l’on voit à la fois railler l’Empire (em-pire), proclamer l’indépendance du royaume et insister sur la guérison des écrouelles. Voir ce qui a été dit des guérisons par Philippe IV. En outre, prestige de fait.\par
Ces idées s’expriment avec beaucoup de vigueur dans l’entourage de Philippe le Bel et de ses fils. Toute une littérature polémique \footnote{ Analyse commode sinon originale de J. RIVIÈRE, {\itshape Le problème de l’Église et de l’État au temps de Philippe le Bel}, Paris, 1926.}. La notion de l’histoire carolingienne et de la Gaule. Les idées de Pierre Dubois, avocat à Coutances. Il oscilla entre deux conceptions : cession à la royauté par l’Empire de lointaines étendues (dans le  \phantomsection
\label{p33} royaume d’Arles, la rive gauche du Rhin peut-être, en tout cas la Lombardie), cela moyennant un accord avec la maison de Habsbourg à laquelle serait reconnu l’Empire héréditaire, ou bien l’Empire même passait aux Capétiens. Sur un point, il est constant : une sorte de monarchie universelle (bien que Dubois proteste contre ce nom) au roi de France, aidé de princes capétiens auxquels sont distribuées diverses couronnes en Europe, — symbolisée par la direction de la croisade. Rêves fumeux d’un homme que les gouvernements ne prirent jamais au sérieux. Mais symptomatique jusque dans ses diverses orientations.\par
— {\itshape Les visées sur lEmpire.}\par
Rappel de l’essai de candidature de Philippe III, en 1272 ; en 1308, candidature ouverte du frère du roi, Charles de Valois, un Charles d’Anjou \emph{à la manque}, déjà candidat de par son mariage à l’Empire latin de Constantinople (ce fut l’élection de Henri de Luxembourg). En 1313, esquisse de candidature d’un prince français (on a hésité sur le nom). Vient en 1324, l’excommunication de Louis de Bavière, que Jean XXII dépose. Le pape, les Habsbourg même songent à Charles le Bel, qui entre dans le jeu. Mais il est impossible de gagner les électeurs.\par
{\itshape Les interventions en Italie.} Expédition de Charles de Valois en Italie, en 1301, pour pacifier la Toscane, d’accord avec le Pape et les Guelfes noirs bannis (Musciatto accompagnait Charles) ; puis conquérir pour les Angevins la Sicile. La brouille survenue, en 1302, entre le roi et Boniface, et la bataille de Courtrai firent rappeler Charles. La politique française continua à mettre sa main dans les intrigues d’Italie. Origine d’une longue tradition.\par
Mais, surtout, {\itshape les acquisitions territoriales sur la frontière de} l’Est.\par

\begin{enumerate}[itemsep=\baselineskip,]
\item[]\listhead{Ici, il faut de toute nécessité sérier géographiquement :}
\item Zone de rayonnement : l’affaire des décimes. En 1284, pour la guerre d’Aragon, considérée comme une croisade, le pape accorda une décime s’étendant aux diocèses de Liège, Metz, Toul, Verdun, aux provinces de Besançon, Lyon, Vienne, Tarentaise et Embrun. Les provinces d’Aix et Arles payaient à Charles d’Anjou. Sous Philippe le Bel, des décimes dans quelques-unes de ces provinces ou diocèses furent plusieurs fois accordées. La guerre d’Aragon ne finit officiellement qu’en 1295. Rodolphe de Habsbourg protesta sans rien faire.
\item Les Pays-Bas. Là, comme l’on sait, la grande pénétration s’est faite en Flandre, c’est-à-dire en terre du royaume (pour l’essentiel). Mais en Hainaut aussi, d’abord contre le centre (en occupant Valenciennes), puis en attirant dans son orbite cet ennemi né des Dampierre. Le comte fut forcé de prêter hommage pour la partie du comté sise sur la rive gauche (Ostrevant).
\item La France en Lorraine : 
\begin{itemize}[itemsep=0pt,]
\item Acquisition de Montfaucon ou, plus exactement, pariage de la seigneurie avec le chapitre de cette ville : 1273.
\item L’affaire de Beaulieu. L’abbaye se donne au roi par peur des comtes de Bar ; en 1287, le Parlement la reconnaît pour française ; Philippe le Bel y jette garnison. Alors c’est la lutte avec Thibaut de Bar. Rodolphe de Habsbourg s’émeut, fait enquêter sur la frontière. Mais en 1291, le comte dut se soumettre. Plus tard, quand éclata la guerre anglaise (1294), le successeur de Thibaut,  \phantomsection
\label{p34} son fils Henri, gendre d’Edouard 1\textsuperscript{er}, fut le grand allié de l’Angleterre sur le continent, où sterlings et livres tournois cherchaient des partisans. Mais en 1301 (4 juin), après l’entrevue de Quatrevaux, Henri de Bar dut reconnaître qu’il tenait du roi en hommage lige toute sa terre sur la rive gauche de la Meuse (c’est l’origine du Barrois mouvant). En outre, quelques cessions de terres et un pèlerinage.
\item Peu de temps auparavant, le 21 décembre 1300, Toul s’était mis sous la protection de la France. En 1315, Verdun fit de même.
\end{itemize}

 
\item  La Comté.\par
 (Noter, sous saint Louis, l’achat — en 1239 — du comté de Mâcon, où fut installé un bailli.)\par
 Le comte de Bourgogne, Otton IV, descendant du Barberousse par les femmes, était un prince brouillon et besogneux. Il conclut avec Philippe le Bel la convention de Vincennes en 1295. Sa fille, Jeanne, son héritière, devait épouser un fils du roi (ce fut Philippe de Poitiers — Philippe le Long) ; dès ce moment, Philippe le Bel recevait l’administration de la comté.\par
 Depuis ce moment, la comté (réunie au royaume sous Philippe V) ne devait plus cesser d’appartenir à des princes français.\par
 Philippe le Bel triompha de la résistance de la noblesse comtoise soutenue — en paroles — par Adolphe de Nassau.
 
\item Le royaume d’Arles. Les préliminaires. Croisade de Louis VIII. L’établissement de Charles d’Anjou. Hommage du Forez sous Louis VII, du Valentinois, rive gauche, sous Philippe Auguste.
\item  Lyon : situation de Lyon : ville pratiquement indépendante, où le fond de la lutte politique est la lutte des bourgeois contre l’archevêque ou le très puissant chapitre.\par
 Premières interventions : en 1269, saint Louis intervient comme arbitre entre les bourgeois et le chapitre (le siège était vacant, d’une vacance qui devait durer cinq ans).\par
 En 1271, les bourgeois se mettent sous la garde de Philippe III et, en échange de sa protection, lui promettent un impôt annuel. Mais, en 1273, le Parlement cassera le sceau des bourgeois. L’archevêque Adhémar de Valentinois sera enquêteur royal (en Languedoc : 1280).\par
 Sous Philippe le Bel, lutte monotone entre le bailli de Mâcon et le chapitre de Lyon. En 1292, le duc de Bourgogne, au nom de Philippe le Bel, installe dans la ville un {\itshape gardiator} (les bourgeois étaient en lutte contre leur archevêque alors favorable à la Savoie).\par
 Les deux Philippines : 1307 (conclues à Pontoise en septembre avec l’archidiacre Thibaut de Vassalieu), après des négociations où les arguments historiques avaient beaucoup servi. Un mémoire royal de 1307 : \emph{« Fondée dans une cité qui fut naguère la dame et la tête des Gaules, l’Église de Lyon est de même la tête et la dame de toutes les églises établies dans les Gaules. Or ces Gaules sont l’antique fondement et la {\itshape Pars} principale du royaume de France. Il est donc évident que l’église de Lyon est soumise au roi de France, comme à son prince temporel. »} Le roi déclare que l’Église a mérité confiscation ; il lui rend le pouvoir de gérer son patrimoine. L’Église de Lyon reconnaît être sous la protection, le ressort et la souveraineté du roi ; le roi s’appelle \emph{princeps} de Lyon. De plus,  \phantomsection
\label{p35} Philippe le Bel s’institua arbitre entre l’Église et les bourgeois, et (sur un point au moins) donna satisfaction à ces derniers.\par
 Il faut comprendre non seulement Lyon, mais aussi le Lyonnais (adresse au roi par les campagnes du Lyonnais).
 
\item  Vers le Sud, l’Uzegois avait été annexé avec le don des comtes de Toulouse. La grosse question fut Viviers. L’évêque et ses sujets étaient sans cesse harcelés de procédures par le sénéchal de Beaucaire. Finalement, le sénéchal, en 1283, ayant pénétré dans le territoire épiscopal, fut excommunié par l’évêque. Le roi fit saisir le temporel et construire une bastide. En 1286, l’évêque se soumit. Un accord de 1306 régla définitivement la question. L’Église reconnut le gouvernement du roi.\par
 
\begin{itemize}[itemsep=0pt,]
\item[]\listhead{Problèmes :}
\item Attitude de l’Empire. Rodolphe de Habsbourg a protesté vaguement, à grands coups d’arguments juridiques et historiques. Adolphe de Nassau aussi, qui fut l’allié d’Edouard 1\textsuperscript{er}, mais ne fit rien. Avec Albert d’Autriche, Philippe le Bel eut une entrevue à Quatrevaux, en 1299. Un mariage fut décidé entre les deux maisons. On raconta que Philippe avait promis l’Empire héréditaire ; Albert, la rive gauche du Rhin. On-dits dont il est difficile de voir s’ils contiennent une part de vérité : une part seulement, en tout cas. Il est certain qu’Albert a reconnu la frontière de la Meuse. Ce n’était pas l’Allemagne.
\item 2. Y a-t-il eu politique consciente ? Non pas de frontières naturelles. Mais désir méthodique de prendre pied dans une sorte de {\itshape no man’s land}, le long d’une ligne de moindre résistance ? Qui. Et cet instinct de grignotage des fonctionnaires royaux.
\end{itemize}

 
\end{enumerate}

\chapterclose


\chapteropen
\chapter[{4. L’administration}]{\textsc{4. }L’administration}\phantomsection
\label{c04}\renewcommand{\leftmark}{\textsc{4. }L’administration}


\chaptercont
\section[{A. Au centre de la monarchie}]{A. Au centre de la monarchie}\phantomsection
\label{c04a}
\noindent  \phantomsection
\label{p37} Autour du roi capétien se groupaient traditionnellement les grands officiers, dont les {\itshape signa} se voyaient et se voient encore au bas des diplômes solennels (de plus en plus rares). C’étaient traditionnellement le sénéchal, le bouteiller, le chambrier, le connétable. Mais le principal d’entre eux, le sénéchal, n’existe plus, sa dignité n’ayant plus été remplie depuis 1191. Les autres charges existent toujours. Le roi les confie d’ordinaire à de très hauts personnages et de beaux revenus leur sont attachés. Le rôle de l’un d’eux, le connétable, a beaucoup grandi à la suite de la suppression du dapiférat. On ne saurait dire qu’il ait le commandement en chef de l’armée. Mais il est certainement l’un des principaux chefs militaires, revendiquant comme un privilège, quand le roi est à l’ost, de se placer à la tête de l’avant-garde. Les fonctions des autres sont beaucoup moins bien définies. Certains ont eu un grand rôle, le chambrier Barthélémy de Roye sous Louis VIII et la régence de Blanche de Castille, le chambrier Jean de Beaumont sous saint Louis, les bouteillers Jean de Brienne et Henri de Sully sous Louis IX et Philippe V. Ils entretiennent à la cour l’influence de la haute noblesse. Mais à titre personnel.\par
Ce n’est pas sous l’autorité de ces hauts personnages, c’est sous celle de leurs anciens subordonnés que se sont développés les services de l’hôtel. C’est-à-dire de la Maison du Roi. Pour faire vivre le roi, la famille royale et tout le personnel, tantôt stable, tantôt passager, mais toujours nombreux, il faut toute une administration. Distributions de vivres, d’argent, de vêtements. Ses chefs, vivant dans l’intimité du souverain, sont susceptibles d’exercer une grande influence. Notamment, le personnel de la Chambre, chargé, en même temps que de la comptabilité, de l’Hôtel en général et auquel est confié le sceau secret. Certains de ses membres portent, dans le langage courant au moins, le titre de \emph{secrétaire du roi} (dès saint Louis). L’importance grandit à mesure que l’administration proprement dite, en se développant, s’écarte de la personne du roi. Le chambellan Pierre de Villebéon passa pour l’homme le plus puissant à la cour de saint Louis. Pierre de la Broce qui avait commencé comme chirurgien et valet de chambre de saint Louis, puis s’était élevé à la dignité de chambellan, fut, au début du règne de Philippe III, un véritable favori, jusqu’au jour où, s’étant heurté  \phantomsection
\label{p38} à la nouvelle reine, Marie de Brabant, il tomba en disgrâce et fut pendu au gibet de Montfaucon. Enguerrand de Marigny fut le chambellan de Philippe le Bel et l’un de ses principaux ministres, le véritable administrateur des finances du royaume. Lui aussi, mais après la mort de son maître, finit, en 1315, à Montfaucon. Naturellement, les titres que ces personnages portent à l’Hôtel ne représentent qu’une part de leurs fonctions, et la moins considérable. Leur véritable pouvoir, ils l’exercent grâce aux diverses missions que leur confie le roi. C’est une des caractéristiques de ces mœurs administratives que le caractère flottant des fonctions. Chaque année, la chancellerie distribue, sur l’ordre du roi, parmi ce personnel, de nombreuses lettres de commission — si nombreuses que souvent le même homme ayant reçu plusieurs missions presque simultanées, ne peut les exercer toutes à la fois.\par
Arrivons à la chancellerie qui est la vieille institution, mais dont le rôle, naturellement, s’est beaucoup développé. Elle comporte tout un personnel, dont les attributions et les droits sont réglés de façon précise au début du XIV\textsuperscript{e}  siècle par des ordonnances \footnote{ L. PERRICHET, {\itshape La grande chancellerie de France des origines à 1328}, Paris, 1912.}. C’est le moment, en effet — nous le verrons — où l’on tend à tout organiser. Expédition des lettres sur l’ordre de \emph{mentiones extra sigillum}. À sa tête, s’était trouvé placé traditionnellement un \emph{Chancelier}. Mais obéissant à son hostilité générale envers les grands officiers, Philippe Auguste, à la mort du chancelier Hugues du Puiset, n’avait pas accordé ce titre au personnage qui, en fait, le remplaçait, le frère Hospitalier Guérin. À l’avènement de Louis VIII, ce très puissant personnage, qui avait entre temps été nommé évêque de Senlis, reprit le vieux titre. Mais à sa mort, en 1227, on revint à la tradition créée par Philippe Auguste et la chancellerie ne fut plus dirigée que par des gardes du sceau. Ainsi en fut-il jusqu’à la mort de Philippe le Bel. Mais lorsque, en 1315, l’avènement de Louis X eut amené une révolution dans le personnel politique, Charles de Valois, alors fort influent, fit donner le sceau à son propre chancelier, Etienne de Mornay, qui ne renonça pas au titre qu’il avait porté au service d’un prince apanagé. Depuis lors, le nom de chancelier revint en honneur. La tradition était que ces fonctions étaient confiées à des ecclésiastiques, quitte d’ailleurs pour eux, en règle générale, à les abandonner s’ils parvenaient à l’épiscopat (mais il y a des exceptions : de frère Guérin et d’autres). Mais deux \emph{gens du secret} de Philippe le Bel, Pierre Flote et Nogaret (bien qu’il eût reçu peut-être les ordres), ont été des laïques. L’absence du titre de chancelier ne doit pas nous tromper : l’importance de la charge reste grande. Témoin les deux noms que je viens de citer. Mais ici encore — quelle que soit la fonction en elle-même — lorsque l’homme est puissant, elle s’accompagne de toutes sortes de missions, notamment diplomatiques. À côté de cela, des fonctionnaires sans titre défini, fonctionnaires de carrière, réservés aux missions : tel Philippe le Convers.\par
Les plus influents parmi les membres de l’Hôtel formaient, avec des administrateurs rappelés provisoirement des provinces, des prélats et de grands personnages laïcs, ce qu’on appelait traditionnellement la \emph{cour} du roi : entendez que, conformément aux conceptions générales de l’époque dite féodale, ils lui apportaient leurs conseils, quand ils en étaient requis, dans toutes ses grandes affaires : personnelles et du royaume, politiques, religieuses, judiciaires. En vérité, la conception féodale stricte eût fait des grands vassaux le conseil naturel du roi. Mais point n’est besoin d’expliquer longuement pourquoi ils n’apparaissaient que rarement. Telle était du moins la conception première, inorganique, de la {\itshape curia}. Elle se présente encore ainsi sous le règne de Louis VIII. Mais à partir de saint Louis, une spécialisation de plus en plus nette commence à se marquer. Deux commissions de la cour prennent peu à peu un caractère permanent : une commission judiciaire qui sera le Parlement, une commission financière qui sera la Chambre des Comptes. Enfin, la fonction de représentant du royaume auprès du roi avait été anciennement tenue par de grandes assemblées \footnote{Sous saint Louis, assemblée précédant la croisade d’Égypte ; sous Philippe III, assemblée précédant la croisade d’Aragon.} dans lesquelles on peut voir une sorte de sublimation de la cour du roi : elles inspireront à Philippe le Bel la convocation d’assemblées de délégués (vraiment élus pour une part) qui seront le début des États généraux. Mais ces institutions nouvelles n’épuisaient pas les fonctions anciennes de la Curia. En outre, Parlement et Chambre des Comptes, par le fait même que leur siège tendait à se fixer en permanence à Paris, se séparaient du roi, voué aux déplacements continuels, non par ses plaisirs, mais par ses guerres, et aussi pour rester en contact : c’est le vieux nomadisme. Il lui fallait un conseil plus proche, qui l’accompagnât. Mais cette institution, qui existait dans les faits, resta longtemps inorganique et sans une terminologie claire : un usage, en somme, plutôt qu’une institution. Le conseil étroit constitué par Philippe le Long, régent avant les couches de la reine veuve, n’était qu’une satisfaction donnée à la haute noblesse, un conseil aristocratique comme en avait vu l’Angleterre et auquel étaient remises certaines prérogatives royales. Peut-être survécut-il à la Régence, démuni peu à peu de ses éléments aristocratiques. Mais et peut-être à côté de lui, conformément au besoin général de régularité que nous avons déjà signalé, une ordonnance de 1318 établit un \emph{Conseil du Mois} plus restreint, pourvu d’attributions à peu près déterminées et dont on devait tenir un journal. Mais, sous cette forme, il n’est pas sûr que l’institution ait été durable.
\section[{B. Dans les provinces}]{B. Dans les provinces}\phantomsection
\label{c04b}
\noindent Là, l’œuvre a été une simple continuation de l’œuvre de Philippe Auguste. À l’avènement de saint Louis, la division du royaume en bailliages et sénéchaussées peut être tenue pour un fait accompli. Seule, la région parisienne y échappait encore, parce que baillis et sénéchaux avaient été, à l’origine, des délégués lointains de la {\itshape curia}. En 1261, ou peu après, saint Louis, sans changer le titre du prévôt de Paris, en fit, au lieu d’un fermier bourgeois, un fonctionnaire de carrière, révocable {\itshape ad nutum}, et lui soumit les prévôts voisins. Le découpage administratif était achevé.\par
Donc, à la base, les prévôts (viguiers, vicomtes ou bayles) qui sont des fermiers ; ils administrent les domaines ; rendent, au nom du roi, la justice seigneuriale ; font exécuter les actes venus d’en  \phantomsection
\label{p40} haut. Au-dessus d’eux, beaucoup plus puissants, les baillis ou sénéchaux (autour de Paris, les prévôts de Paris) qui sont de vrais fonctionnaires toujours révocables et que l’administration royale se fait un principe de déplacer assez souvent. Leurs pouvoirs sont multiples, parce qu’ils représentent le roi. Ils sont officiers de justice et de finances, mais aussi hommes de guerre. Nomades au surplus, par nécessité. Leurs circonscriptions sont étendues et elles ne se bornent pas au domaine ni au sens étroit ni au sens large. Car chacun d’eux a placé sous sa surveillance, et fréquemment sous sa justice, les grands fiefs voisins. Autour d’eux, peu à peu, quelques fonctionnaires spécialisés dans la justice ou les finances. Surtout autour d’eux, comme des prévôts, une foule de sergents, redoutés des populations.\par
Mais la difficulté est toujours la liaison. Nous verrons comment, pour les finances et la justice, des commissions permanentes sont envoyées dans certaines provinces. Il y a aussi des missions nombreuses. Mais surtout les enquêteurs ou enquêteurs-réformateurs créés par saint Louis en 1247 et devenus rapidement un rouage permanent : chargés d’écouter les plaintes des populations, de redresser à la fois les abus aux dépens des sujets et les usurpations au droit du roi, parfois de rétablir l’ordre par des procédures expéditives (Cf. en Angleterre les commissions de \emph{trailbaston}).
\section[{C. Le personnel}]{C. Le personnel}\phantomsection
\label{c04c}
\noindent Il est varié. D’ailleurs de degrés très variés, avec des cloisons assez étanches : les sergents ; les petits fonctionnaires ; les baillis, membres souvent de la cour du roi ; le monde de la cour. Néanmoins, il peut se décomposer par classes.\par
La haute noblesse joue un rôle encore important à la cour comme aidant le roi de ses conseils et à l’armée. Mais elle ne fait pas partie du corps des fonctionnaires.\par
Le haut clergé, les grands évêques et abbés royaux n’ont pas cessé d’entourer le roi. L’abbé de Saint-Denis, Mathieu de Vendôme, sous saint Louis et Philippe III. Sous Philippe IV, vers la fin du règne, l’archevêque de Narbonne, Gilles Aiscelin, sera un conseiller très écouté à Rome et en Angleterre, chargé de percevoir des décimes. Mais le plus souvent, surtout à partir de Philippe le Bel, ce sont des clercs d’ordre inférieur qui suivent le roi et reçoivent ensuite de lui l’épiscopat. Ils commencent par être \emph{clercs du roi}, reçoivent des bénéfices — en général des prébendes dans les chapitres — et finissent, s’ils sont heureux, par coiffer la mitre. Telle, par exemple, la carrière de Pierre de Latilly, une des victimes de la réaction qui suivit la mort de Philippe le Bel (il ne fut d’ailleurs qu’emprisonné) : membre du Parlement, chargé d’enquêter dans le Midi, et d’une mission en Angleterre, désigné pour arrêter les Lombards de Senlis, il sera, déjà évêque de Châlons, le dernier garde des sceaux du règne.\par
Les petits chevaliers du domaine. Ils continuent à tenir une place très importante. Notamment, c’est parmi eux que se recrutent les baillis et sénéchaux, {\itshape curriculum} qui tend nettement à se distinguer des carrières centrales. Un exemple : Beaumanoir. Mais c’est à ce milieu aussi qu’appartenait Enguerrand de Marigny, gentilhomme normand.\par
 \phantomsection
\label{p41} Philippe de Remy (Oise, canton d’Estrées-Saint-Denis), sire du château voisin de Beaumanoir qu’il tenait en fief de l’abbaye de Saint-Denis, d’une famille de chevalerie picarde. Son père, chevalier comme lui, avait été au service du comte Robert d’Artois, frère de saint Louis et de la comtesse, sa veuve. C’est également au service d’un prince apanagé que, né vers 1250, Beaumanoir commença sa carrière, à titre de bailli du comté de Clermont pour le comte Robert, fils de saint Louis (1279-1282). Puis il passa dans l’administration royale et fut successivement, jusqu’à sa mort en 1296, sénéchal du Poitou, chargé de mission à Rome, bailli de Vermandois, de Touraine et de Senlis.\par
Mais déjà l’élément bourgeois apparaît : de haute bourgeoisie. Jean le Saunier, bailli de Caen en 1263, était un bourgeois de Pontoise ; Vincent de Valricher, bailli de Caux en 1272, était un bourgeois de Rouen qui avait été maire de sa ville natale. Sous Philippe le Bel, Guillaume de Nogaret était un bourgeois du Midi. Mais avec lui apparaît un élément nouveau : celui des professeurs de droit. Pierre de Belleperche, qui lui aussi tint les sceaux sous Philippe IV à titre intérimaire et joua un grand rôle dans l’affaire de Lyon, avait professé le droit à Orléans. Pour Nogaret, né de parents toulousains, vraisemblablement des bourgeois qui possédaient un petit fief rural, Nogaret, sans doute près de Saint-Félix-de-Caraman. Ses ennemis racontaient que ses parents avaient été brûlés comme hérétiques. Date de naissance inconnue : probablement entre 1260 et 1270.\par
En 1291, on le trouve professeur de lois à Montpellier ; sans doute avait-il reçu les ordres mineurs. La ville appartenait en fief au roi de Majorque. En 1293, l’évêque de Maguelonne céda au roi de France sa part de la seigneurie de Montpellier. Dès 1294 Nogaret est au service du sénéchal de Beaucaire, Alfonse de Rouvray, comme juge mage. Peut-être dès 1295, en tout cas dès 1296, il paraît dans l’entourage royal ; en 1296, il est attesté comme un des fonctionnaires chargés de récupérer les droits usurpés sur le roi en Bigorre. À ce moment, il est encore qualifié de {\itshape clericus} : à partir de 1299, {\itshape miles} (ou {\itshape miles et legum professor}). En 1300, il est enquêteur en Champagne.\par
Pierre Flote ayant été tué le 11 juillet 1302 à Courtrai, Nogaret, sans lui succéder comme garde des sceaux (ce fut Étienne de Suisy) et en portant seulement le titre de conseiller du roi, revêt le rôle de premier plan que l’on sait. Il n’obtint les sceaux que le 22 septembre 1307, lorsque fut décidée l’arrestation des Templiers (Étienne de Suisy, devenu cardinal en 1305, était resté en fonctions jusqu’au début de 1307 ; il n’avait pas été remplacé, selon Perrichet, Pierre de Belleperche n’ayant été qu’un intérimaire). Il resta garde des sceaux jusqu’à sa mort (avril 1313).\par
Des noms, des {\itshape cursus honorum}. Derrière, quoi ? Ce sont des gens, en règle générale, instruits, de plus en plus à mesure que le siècle s’avance. Voyez Beaumanoir, littérateur et juriste. Le {\itshape Conseil à un ami} de Pierre de Fontaine, écrit entre 1254 et 1259, est l’œuvre d’un bailli de Vermandois, que Joinville nous montre parmi les deux conseillers, par lesquels le saint roi, assis sous le fameux chêne de Vincennes, faisait juger en sa présence les procès qu’on lui avait directement portés. Ce sont en même temps, pour la plupart, des hommes de guerre. Pierre Flote est mort à Courtrai. Ils sont bons juristes : voyez les exemples cités. On n’a pas attendu Philippe le Bel pour connaître le droit romain : le {\itshape Conseil à un ami} en est pénétré ;  \phantomsection
\label{p42} et l’on ne saurait dire que dans les détails, les gens s’en sont inspirés. Ils y ont pris surtout une méthode de raisonnement, et une idée très haute du pouvoir.\par
Sur les faits et gestes de ce personnel, nous sommes renseignés surtout par les témoignages rassemblés aux Enquêtes de saint Louis \footnote{ Ch. V. LANGLOIS, {\itshape Doléances recueillies par les Enquêteurs de saint Louis et des derniers Capétiens directs}, dans \href{http://gallica.bnf.fr/document?O=N018186}{\dotuline{{\itshape Revue historique} [http://gallica.bnf.fr/document?O=N018186]}}, t. 92 (1906), p. 1-41.} : ils nous renseignent surtout sur l’administration locale et sont évidemment poussés au noir. Certains faits cependant sont certains. La vénalité est très répandue. Elle est presque de nature chez les prévôts qui sont des fermiers. \emph{Mon ami} disait à un de ses administrés, le viguier de Beaucaire, sous saint Louis, \emph{« j’ai acheté bien cher ma viguerie, je veux avoir quelque chose de vous »} : d’où pots de vins et peut-être plus encore extorsions. Ils sont durs, comme tous les hommes de ce temps ; et ils font d’ailleurs un métier qui endurcit : parce qu’il est rude ; parce qu’ils se heurtent constamment à des résistances violentes. L’aventure de ce collecteur des mains-mortes, qui en 1318, s’étant rendu dans un petit village du Vermandois, pour y percevoir une amende de formariage, se fit rosser par les habitants, lui et ses sergents, dont deux furent en péril de mort, n’était point exceptionnelle et ne prédisposait point à la tendresse. Ils sont très puissants, soutenus par un grand esprit de corps, et à peu près assurés de l’impunité. Les carrières, même à la cour, sont en général possibles. Autour des chefs s’agitent d’innombrables intrigues, qui se terminent souvent par d’éclatantes et sanglantes disgrâces et des procès sans valeur, où les accusations de mauvaises mœurs et d’hérésie se mêlent à des histoires de philtres et d’envoûtement. Au moins à partir de Philippe III. Sous saint Louis, le personnel local n’a pas été bien différent de celui de ses successeurs. Le personnel central a eu incontestablement plus de tenue, parce que le roi choisissait au lieu d’être mené. Avec cela — je l’ai dit — un vif esprit de corps et un sens profond de l’autorité royale qui se confond avec la leur. Le XIII\textsuperscript{e} siècle finissant a certainement été un des moments où le fonctionnarisme naissant a été à la fois le plus puissant et le plus dans la main du roi, les habitudes et l’hérédité des charges et un recrutement par cooptation n’étant pas encore nés.
\section[{D. La justice }]{D. La justice \protect\footnotemark }\phantomsection
\label{c04d}
\footnotetext{ Cf. renseignements utiles dans CHÉNON, {\itshape Histoire générale du droit français public et privé}, 2 vol., Paris, 1926-1929.}
\noindent Je passerai brièvement, le sujet, à la différence des finances, étant aisé à connaître :\par
\subsection[{1° Les règles générales de l’organisation de la justice dans la société française}]{1° Les règles générales de l’organisation de la justice dans la société française}

\begin{listalpha}[itemsep=0pt,]
\item Les justices seigneuriales territoriales sur les classes inférieures : justice foncière ; basse justice ; haute justice (point d’aboutissement extrême : le Beauvaisis où, au XIII\textsuperscript{e} siècle, tous les seigneurs sont hauts justiciers).
\item Le cas des serfs et d’une façon générale des dépendants attachés à la personne. Rôle de l’arbitrage.
\item La justice vassalique. \phantomsection
\label{p43}
\item La justice ecclésiastique ; luttes du XIII\textsuperscript{e}  siècle. \emph{« Toute la juridiction du royaume est tenue du roi en fief ou arrière-fief »} (Beaumanoir).
\item Le système des appels.
\item La juridiction royale (idée générale).
\item Les procédés de juridiction : la lutte de la collégialité et du juge unique. En 1259, le châtelain de Blois est relaxé de faux jugement par le Parlement, parce qu’il n’a fait que lire le jugement prononcé par les membres de sa Cour.
\end{listalpha}

\subsection[{2° Le développement de la justice royale au cours du XIIIe siècle}]{2° Le développement de la justice royale au cours du XIII\textsuperscript{e} siècle}

\begin{listalpha}[itemsep=0pt,]
\item L’extension du domaine.
\item L’annexion des grands États féodaux, déjà pourvus d’une organisation judiciaire centralisée, parfois plus que dans le Domaine (Normandie).
\item L’extension des appels facilitée par la suppression du duel judiciaire, ordonnée par saint Louis. Il fut rétabli par ses successeurs seulement dans les causes criminelles, et tomba hors d’usage.
\item L’intervention entre les tenanciers des seigneuries et les seigneurs.
\item La notion des cas royaux. Cas qui touchent le roi, ses officiers, les institutions ou les personnes placées sous sa sauvegarde. La notion de la paix (biens d’asseurement, peut-être dessaisine). Obscurité de la définition. Réponse de Louis X aux nobles de Champagne : \emph{« Nous eussent requis que les cas nous leur voulsissions eclaircir ; nous les avons eclaircis en ceste manière, c’est assavoir que la royale majesté est entendue es cas qui de droit ou de ancienne coustume puent et doivent appartenir à souverain prince et à nul autre ».} 
\item La prévention.
\end{listalpha}

\subsection[{3° L’organisation de la justice royale}]{3° L’organisation de la justice royale}

\begin{listalpha}[itemsep=0pt,]
\item La justice locale. Les prévôts. Les baillis, à l’origine délégation de la cour du Roi, avec leurs \emph{assises} et parfois leurs juges spécialisés. Organisation de la hiérarchie d’appels : (1) prévôt, (2) bailli, (3) cour du Roi ; définitivement par une ordonnance de janvier 1278. Parallèlement, mais avec retard, les baillis seront peu à peu éliminés de la Cour (jamais dans leur cause en 1291 ; définitivement en 1303).
\item Le Parlement.
\end{listalpha}

\noindent Le mot : exemples à partir de saint Louis (Tristan et Yseut).\par
Attributions spécialisées par le développement de la juridiction baillivale.\par
Élimination progressive des grands seigneurs. Elle n’a pas été totale (le cas des pairs). Ils jugent encore sous saint Louis. Par exemple dans un arrêt de 1261 figurent entre autres, l’archevêque de Rouen, le comte de Soissons, le connétable. En 1296, le duc de Bourgogne est cité dans une ordonnance parmi les présidents de la Chambre des Plaids. Prenons la composition du Parlement fixée par une ordonnance de 1307 : on voit encore apparaître un archevêque, plusieurs évêques, deux comtes (Dreux et Boulogne) et le connétable. Philippe le Long, en 1319, exclura les prélats, parce qu’ils sont ou  \phantomsection
\label{p44} doivent être trop occupés par le \emph{gouvernement de leurs experitüautez}. Mais observé ?\par
Personnel fixé. Sous saint Louis, ce sont à peu près toujours les mêmes personnes qui jugent. À partir de Philippe le Bel au moins, le roi désigna pour chaque service. En fait, personnel à peu près permanent. Et gages réguliers.\par
Stabilité. De 1255 à 1314, nous trouvons hors de Paris la session : Nativité de la Vierge, 1257, Melun, et, jusqu’au XIV\textsuperscript{e} siècle, quelques séances exceptionnelles \footnote{ Séances du Parlement, tenues en dehors de Paris au XIV\textsuperscript{e} siècle : Vincennes, 1304 ; Cachan, 1308-1309 ; Pontoise, 1310-1311 ; Poissy, 1312 ; Orléans, 1325.}. En même temps, archives.\par
Organisation intérieure (fixée par les grandes ordonnances, surtout sous Philippe le Long). Apparition dès 1278 et nettement, dès 1296, des 3 Chambres : des Plaids, des Enquêtes et Requêtes (cette dernière trie les requêtes). En outre l’Auditoire de droit écrit (dès 1278 au plus tard ; mentionné pour la dernière fois en 1318).\par
Les commissions extérieures : Échiquier de Rouen (2 fois par an) dont on appelle à la Chambre des Plaids (depuis 1280 ; pratique interdite par la charte aux Normands de 1315), Grands Jours de Troyes, et, avec intermittence sous Philippe III et Philippe IV, Assises de Toulouse.\par
Le Parlement s’écarte du roi. Depuis 1273 \emph{per arrestum curie}. Les mentions de présence du roi deviennent exceptionnelles dès Philippe III. Sous ce règne, le Roi déjà plaide devant sa cour. Mais justice retenue : les requêtes de l’Hôtel.
\chapterclose


\chapteropen
\chapter[{5. Le roi et la nation}]{\textsc{5. }Le roi et la nation}\phantomsection
\label{c05}\renewcommand{\leftmark}{\textsc{5. }Le roi et la nation}


\chaptercont
\noindent \labelchar{1)}  \phantomsection
\label{p45} Ne pas croire qu’un roi du moyen âge pût se passer de ses sujets. Moralement, il devait prendre conseil. Pratiquement, le mécanisme administratif n’était pas assez parfait pour se passer de leur collaboration. Enfin, surtout lorsqu’on touchait à des problèmes religieux (tel Philippe le Bel), on avait besoin de l’opinion publique. Par ailleurs, il était fatal que pour obtenir quelque chose du gouvernement, les sujets fussent amenés à se confédérer. Par là, à prendre l’habitude de se réunir, et à imposer ou chercher à imposer au roi les décisions prises dans ces \emph{ligues}.\par
\bigbreak
\noindent \labelchar{2)} De bonne heure — dès la première moitié du XIII\textsuperscript{e} siècle en fait — nous voyons apparaître des réunions de sujets sous trois formes qui restent classiques :\par

\begin{listalpha}[itemsep=0pt,]
\item Assemblées générales convoquées par le roi, en grandes circonstances : nobles, prélats. Par exemple en 1270, celle où fut annoncée la croisade de Tunis ; en 1284, celle d’Aragon (nobles et prélats). Mais aussi les conseils de bourgeois, que saint Louis réunit plusieurs fois autour de lui, pour l’éclairer sur le fait des monnaies.
\item Des assemblées locales pratiquement beaucoup plus importantes. Par exemple, en 1235, le bailli de Vexin réunit des chevaliers pour prendre une ordonnance sur le relief ; pour le même objet, en 1246, des seigneurs de l’Anjou et du Maine autour du roi, lui-même ; les commissions de prud’hommes que l’ordonnance de 1254 recommande aux baillis et sénéchaux de réunir autour d’eux pour donner leur avis sur l’expédition des vivres.
\item Enfin, des ligues : les plus célèbres sont, sous saint Louis, celles des barons de France contre les empiètements du clergé ; probablement des ligues contraires des clercs, habitués d’ailleurs à la vie parlementaire.
\end{listalpha}

\bigbreak
\noindent \labelchar{3)} Sous le règne de Philippe le Bel, un développement notable se marqua de deux façons :\par
D’abord la multiplication des consultations locales pour l’impôt. Nous l’avons vu en 1303, lors de l’appel au concile : campagne d’opinion par des réunions locales. C’est probablement le plus important.\par
Enfin des assemblées générales beaucoup plus importantes que dans le passé. Il y en a eu essentiellement trois connues de nous :  \phantomsection
\label{p46} l’assemblée réunie le 10 avril 1302 à Notre-Dame pour délibérer \emph{« sur certaines affaires qui intéressent le roi et le royaume »} ; en fait, pour les faire protester contre l’attitude du pape et servir d’antidote à la convocation du concile.\par
— En mars 1308, une assemblée pareille à Tours pour approuver les mesures contre le Temple.\par
Ces deux-là n’étaient que d’opinion publique.\par
— Celle du 1\textsuperscript{er} août 1314, au Palais, à Paris, a un tout autre caractère et annonciateur de l’avenir. Il s’agissait d’obtenir l’approbation pour un subside général pour l’ost.\par

\begin{listalpha}[itemsep=0pt,]
\item[]\listhead{Quelques remarques :}
\item Le rôle effectif de ces assemblées est médiocre. Les établissements religieux ou les villes qui envoient des procureurs, les nomment souvent \emph{« ad audiendum ea quae per dominum regeni ordinabuntur »}, \emph{« pour oir et raporter la volenté le roy »}. En 1314, Enguerrand de Marigny, après avoir requis les bourgeois des villes de payer l’aide et expliquer les raisons, fit lever le roi pour que celui-ci pût voir \emph{« ceux qui luy voudroient faire aide »}.
\item Mais ce sont des réunions, et qui devaient être nombreuses. Des trois ordres (en 1302 au moins, ils ont délibéré à part). La présence des villes — parmi lesquelles on doit comprendre même certains bourgs de campagne — n’avait rien de révolutionnaire. Mais elle supposait déjà un rudiment de système électif. Les prélats et les seigneurs étaient en principe convoqués nommément, ce qui aboutissait à laisser de côté, à la différence de l’Angleterre, la masse des petits seigneurs ; ils pouvaient d’ailleurs se faire représenter par procureurs et le firent souvent. Mais pour les communautés religieuses et les villes, il fallait des représentants (dont le roi en général fixait le nombre). Pour les villes, ils étaient nommés de façon fort variable : par les magisirats ; par les communes ; par le prévôt royal ou le seigneur.
\end{listalpha}

\bigbreak
\noindent \labelchar{4)} Or, il arriva que le mécontentement même provoqué par le gouvernement royal amenât un développement des habitudes représentatives \footnote{ André ARTONNE, {\itshape Le mouvement de 1314 et les Chartes provinciales de 1315}, dans {\itshape Bibliothèque de la Faculté des Lettres de Paris}, t. XXIX, Paris, 1912.}. Sur le mécontentement, Joinville écrit en 1305 à la suite de propos de saint Louis sur les avertissements envoyés par Dieu aux hommes : \emph{« Si y preingne garde le roys qui ore est, car il est eschapé de aussi grant peril ou de plus que nous ne feimes ; si s’amende de ses mesfais en tel manière que Diex ne fière en li ne en ses choses cruelment »}.\par
Un subside avait été levé, en 1314, pour l’ost de Flandre, et l’ost se termina peu glorieusement. On ne se battit guère et Enguerrand de Marigny fit conclure la paix de Marquette, le 3 septembre. Elle avait été conclue par Louis de Nevers, fils du comte de Flandre ; mais Louis, aussitôt l’ost retiré, prétendit avoir été trompé et refusa de l’exécuter. Pourtant, on continua à lever le subside. D’où une double raison de mécontentement, et l’apparition des premières ligues qui unissent nobles et villes. Ce sont les \emph{alliés} ; les ligues se confédérèrent entre elles. La Normandie, l’Auvergne, le Languedoc restèrent à l’écart.\par
 \phantomsection
\label{p47} Philippe le Bel prescrivit de cesser la levée du subside, et mourut le 30 novembre 1314. Il y eut un changement de personnel : Marigny, arrestation de Michel de Bourdenai, Pierre d’Orgemont, Raoul de Presles ; les sceaux enlevés à Pierre de Latilly pour être donnés à Etienne de Mornay ; mais, semble-t-il, pas de changement réel de politique. Louis X cependant dut accorder des concessions. Ce furent une série de chartes provinciales dont la plus célèbre est la charte aux Normands (et il y en eut bien d’autres pour la Champagne, l’Auvergne, le Languedoc, etc.). Elles contiennent des concessions prudentes qui sont simplement le rappel de principes coutumiers et des promesses banales d’amendement : comme la grande ordonnance de réformation de 1303, à laquelle d’ailleurs les ligues s’étaient référées et que Louis X avait confirmée à nouveau. La plupart des requêtes d’ailleurs témoignaient d’un grand esprit de classe : respect des droits seigneuriaux, interdiction de convoquer directement les hommes.\par
Fait caractéristique : si les ligues s’étaient levées en Artois contre la comtesse, en Maine contre Charles de Valois, en Bourgogne au contraire, la ligue, malgré l’existence du gouvernement ducal, s’en prit au roi. Preuve éclatante du pouvoir royal.\par
\bigbreak
\noindent \labelchar{5)} Sous Philippe V, l’habitude des assemblées s’est prise\par

\begin{enumerate}[itemsep=0pt,]
\item Assemblées à peu près générales (Poitiers, juin 1321 : on y a délibéré à part non seulement par ordre, mais à l’intérieur du Tiers et du Clergé au moins, par région).
\item Assemblées du Nord, à Paris, et du Midi, à Bourges (mars 1317).
\item Assemblées locales d’amplitude extrêmement diverse et l’on commence d’y parler impôt. Habitude si ancrée qu’en 1318-1319, la chancellerie demanda aux autorités locales de lui fournir les listes des personnes nouvelles à convoquer.
\end{enumerate}

\chapterclose


\chapteropen
\chapter[{6. Le roi et l’église}]{\textsc{6. }Le roi et l’église}\phantomsection
\label{c06}\renewcommand{\leftmark}{\textsc{6. }Le roi et l’église}


\chaptercont
\section[{A. La part du roi dans le gouvernement de l’Église}]{A. La part du roi dans le gouvernement de l’Église}\phantomsection
\label{c06a}

\begin{enumerate}[itemsep=\baselineskip,]
\item Les fondements : le caractère sacré du roi ; la protection du roi sur la Sainte Église ; et la notion (dès l’époque carolingienne) de la fortune de l’Église, réservoir pour les nécessités du roi. Dans un mémoire d’un envoyé de saint Louis à Innocent IV : \emph{« istud etiam juris [rex] habet quod omnes ecclesiarum thesauros et omnia temporalia ipsarum pro sua et pro sui regni necessitate potest accipere, sicut sua »}. \\
Et tout cela se résume assez bien par ce mot que l’on prêtait à Philippe Auguste sur son lit de mort (Conon de Lausanne) : \emph{« Rogo te quod honores Deum et sanctam ecclesiam, sicut ego feci. Ego magnam utilitatem inde consecutus sum, et tu magnam inde consequeris \footnote{ Conon de Lausanne, {\itshape Notae}, éd. Waitz, {\itshape Monumenta Germaniae, Scriptores}, t. XXIV, p. 783.}. »}
\item  
\begin{listalpha}[itemsep=0pt,]
\item[]\listhead{Mais ce droit avait été restreint de diverses façons :}
\item Par la distinction entre les églises royales (ou les abbayes royales) et les autres. Les progrès du domaine ont, en ce qui regarde les évêques, peu à peu supprimé cette distinction. N’échappent plus au roi, à la fin du XIII\textsuperscript{e}  siècle, que les évêchés de la Guyenne et ceux de la Bretagne. Ni le duc de Bourgogne, ni le comte de Flandre n’avaient d’évêchés. Beaucoup d’abbayes de même sont tombées dans le domaine. D’autres sont devenues royales par lettres de sauvegarde. Et la théorie de la garde générale du roi (par dessus les gardes spéciales) est une arme puissante.
\item Par la réforme grégorienne. Celle-ci avait surtout supprimé les formes choquantes (pas toutes, nous le verrons). Elle a aussi dressé devant le roi le pouvoir de la papauté.
\end{listalpha}

 
\item  Comment s’exerce le droit du roi. D’abord pour la nomination : les procédés normaux. Prenons un évêché (pour les abbayes, il en va à peu près de même) :\par
 
\begin{listalpha}[itemsep=\baselineskip,]
\item Le corps électoral est (définitivement, depuis le concile de Latran de 1215) le chapitre ; les chanoines sont nommés par l’évêque, quelquefois par le chapitre lui-même ; pour certains d’entre eux,  \phantomsection
\label{p50} par une communauté monastique, voire un seigneur laïque. Provisions apostoliques. Intervention du roi (par l’intermédiaire de ces provisions ou par ordre direct, ou enfin à l’occasion de la régale). Ce sont des corps aristocratiques.
\item La première formalité : le congé d’élire. Certaines églises, assez rares, en sont dispensées. Le roi profita souvent de ce congé pour imposer son candidat. Par exemple : c’est ainsi que Philippe le Hardi nomme Pierre de Benais à l’évêché de Bayeux. \emph{Le chapitre}, dit Guillaume de Nangis, \emph{« ne l’osa contredire par la doubtance le roi »}.
\item Après l’élection, il faut l’approbation royale qui se traduit par la mainlevée de la régale. En général, le roi exige le serment de fidélité. Parfois celui-ci conserve la forme d’un hommage vassalique. Autant que je puis voir, la consécration n’a lieu qu’après ces formalités. \\
En somme, le roi entend rester maître des électeurs. Le mémoire de 1247 dit : \emph{« non est multum temporis quod reges Franciae conferebant omnes episcopatus in camera sua quibus voluerint... »}
\item  
\begin{itemize}[itemsep=0pt,]
\item[]\listhead{Le roi et la fortune de l’Église :}
\item  La régale. Régale temporelle. Régale spirituelle. Elles ne sont pas universelles, et l’étendue du droit, là même où il s’exerce, est souvent indécise : d’où d’assez nombreux procès. Mais le privilège est répandu, et lucratif.
\item Les impôts sur l’Église. Simple rappel.
\item L’utilisation des terres des Églises au moins royales : levées d’hommes et de tailles (rappel).
\end{itemize}

 
\end{listalpha}

 
\end{enumerate}

\section[{B. La concurrence de la Papauté}]{B. La concurrence de la Papauté}\phantomsection
\label{c06b}
\noindent Elle s’est produite sur deux domaines : l’imposition, les nominations. Nous savons déjà par quels procédés.\par

\begin{enumerate}[itemsep=0pt,]
\item[]\listhead{En fait, il y a eu visiblement :}
\item Jusque vers l’avènement du pape français Clément IV, une politique en somme assez raide.
\item Ensuite, une espèce d’alliance : le roi et le pape se partagent les décimes ou en lèvent tour à tour. Les nominations soulèvent des difficultés ; mais le roi souvent utilise le droit de nomination du pape. C’est cet accord que devait rompre provisoirement le passage sur le trône de saint Pierre d’un vieillard violent, héritier des théories d’Innocent III et d’Innocent IV : Boniface VIII.
\end{enumerate}

\section[{C. Le conflit de Philippe le Bel avec la Papauté }]{C. Le conflit de Philippe le Bel avec la Papauté \protect\footnotemark }\phantomsection
\label{c06c}
\footnotetext{ Voir J. RIVIÈRE, {\itshape Le problème de l’Église et de l’État sous Philippe le Bel}, Paris, 1926.}
\noindent \labelchar{1)} Pour comprendre les événements : les conditions de l’avènement. Après la mort du franciscain italien Nicolas IV (avril 1292), il y eut interrègne de plus de deux ans (la législation si raisonnable de Grégoire X sur le conclave — constitution {\itshape Ubi periculum}, du 7 juillet 1274 — avait été supprimée en 1276, par Jean XXI).  \phantomsection
\label{p51} Finalement, le 5 juillet 1294, les cardinaux, qui avaient quitté Rome pour Pérouse, nommèrent Pierre de Morrone. Celui-ci abdiqua le 13 décembre. Dante, contemporain de ces événements, l’a placé (anonymement) dans le vestibule de l’Enfer, parmi ces âmes qui n’ont pratiqué ni le bien ni le mal, disciples des anges restés neutres entre Dieu et Lucifer : \emph{« Celui qui par lâcheté a fait le grand refus »}. Mais il avait eu le temps de rétablir le {\itshape preceptum} de Grégoire X. Le conclave se réunit le 23 et, dès le lendemain, élut le cardinal Benoît Gaetani (Boniface VIII), un personnage d’assez haute naissance, neveu du feu pape Alexandre IV, qui avait derrière lui une longue carrière à la curie et comme légat : au surplus, un vieillard de près de quatre-vingts ans.\par
Seulement, l’abdication d’un pape était contestée. On pensait en outre que celle de Célestin V n’avait été qu’à demi-volontaire. L’impression fut confirmée par le fait que Boniface jugea utile de s’emparer de la personne de son prédécesseur qui, après une fuite dramatique, fut rattrapé et jeté dans un château de Campanie, où il mourut. Le bruit d’empoisonnement, naturellement, courut. Ajouter le caractère de Boniface VIII, ses violences contre ses ennemis italiens, notamment les Colonna, ses partis pris et ses rigueurs contre les Spirituels protégés par Célestin V. Dante qui n’était pas un ami des Capétiens, et qui a blâmé l’attentat, a dit (E. XXVII, 88) que \emph{« tel il n’avait d’ennemis que parmi les chrétiens »} et a fait dire à saint Pierre (Par. XXVII, 22-24) :\par


\begin{verse}
« Quegli ch’usurpa in terra il luogo mio\\
il luogo mio, il luogo mio, che vaca\\
nella presenza del Figliuol di Dio.»\\
\end{verse}

\noindent Nous verrons Nogaret s’exprimer en termes analogues.\par
\bigbreak
\noindent \labelchar{2)} On a parfois qualifié de « premier différend » les incidents soulevés par les décimes exigées depuis 1294, comme nous savons, et que condamna — comme d’ailleurs les décimes anglaises — la bulle {\itshape Clericis laicos} du 24 février 1296. Et il est bien vrai que les polémiques, soulevées par la querelle, attestent immédiatement une vive opposition doctrinale dont, d’ailleurs, le germe est visible dès le temps de saint Louis. Mais le pape, nous l’avons vu, céda très vite (bulle {\itshape Etsi statu}, du 31 juillet 1297). Puis il canonisa saint Louis.\par
\par
\bigbreak
\noindent \labelchar{3)} Le drame devait commencer après le grand Jubilé de 1300, qui apporta à Boniface une si vive exaltation. L’origine en fut l’arrestation par les gens du roi de l’évêque de Pamiers \footnote{Bernard Saisset, abbé de Saint-Antonin de Pamiers, avait été premier évêque du nouveau diocèse créé par Boniface VIII (qui le connaissait) en 1295.}, Bernard Saisset, accusé d’avoir conspiré contre le roi et les Français (12 juillet 1301) et victime de son ennemi, le comte de Foix. Le pape réclama la liberté de l’évêque qui devait se rendre à Rome pour se justifier : affaire déjà grave de conflit de juridiction. Elle fut aggravée par les affirmations doctrinales de la bulle {\itshape Ausculta fili} du 5 décembre 1301. Elle revendiquait, en termes violents, la supériorité du pape sur le roi, {\itshape ratione peccati ;} elle annonçait, nouvelle redoutable, la convocation à Rome d’un concile où viendraient les évêques français et où il serait traité \emph{« de la réformation du royaume et de la correction du roi »}. Le gouvernement royal répliqua par un appel à l’opinion. On fit circuler une fausse bulle qui résumait la vraie en termes, comme dit Langlois, \emph{clairs et  \phantomsection
\label{p52} durs} (c’est le coup de la dépêche d’Ems). On convoqua à Notre-Dame, le 10 avril 1302, une assemblée des trois ordres. Ceux-ci, après un exposé de Pierre Flote, écrivirent au pape pour affirmer l’indépendance du royaume, le bien-fondé de l’attitude du roi. Le pape se mit fort en colère ; après Courtrai (11 juillet 1302), il réunit à Rome le synode, où en effet parurent un bon nombre de prélats français, et là promulgua, en novembre 1302, la bulle {\itshape Unam Sanctam} qui affirmait la subordination du glaive temporel au glaive spirituel, le droit du pape de juger les rois et se terminait par la déclaration fameuse : \emph{« Porro subesse romano pontifici omni humanae creaturae declaramus, dicimus, diffinimus, et pronunciamus omnino esse de necessitate salutis »}.\par
Alors, le gouvernement royal, après une période d’hésitations, tomba sous la domination de Guillaume de Nogaret et des bannis italiens (notamment le florentin Mouche). Nogaret lut au Louvre, le 12 mars 1303, devant une assemblée de barons et de prélats, un violent réquisitoire contre Boniface où il l’accusait d’avarice, de haine pour la paix, de simonie et enfin de n’être point pape : \emph{« il a usurpé la place, car l’Église romaine était légitimement unie à Célestin quand il a commis le péché d’adultère avec elle »}. Il demandait la convocation d’un concile général pour juger \emph{ce faux prophète} et proposait qu’on l’enfermât avant le jugement. Après quoi, il partit pour l’Italie, afin d’exécuter cette tâche, pendant que, devant une assemblée beaucoup plus vaste, réunie en juin au Louvre encore, l’acolyte de Nogaret, Plaisians, renouvelait plus violemment le réquisitoire. Des commissaires, envoyés dans les provinces, recueillirent ou forcèrent des adhésions ; à l’étranger, on en obtint quelques-unes. Le 7 septembre, Nogaret avec un des Colonna, Sciarra, apparut à Anagni. Cependant que ses routiers commettaient divers excès, Nogaret s’empara, sans violences corporelles, de la personne du pape. Celui-ci, le 9, fut délivré par une émeute des gens de la ville. Boniface fut emmené à Rome, brisé — il avait 86 ans — et à demi-fou. Il mourut le 11 octobre.\par
Désormais, tout va tourner autour de deux problèmes : pour arme, le roi a le procès (posthume) de Boniface ; le pape, celui des exécuteurs de l’attentat d’Anagni qui avaient été excommuniés. Le jeu se dessina sous le pontificat de Benoît XI, un frère prêcheur, élu peu après la mort de Boniface. Puis, Benoît mort (7 juillet 1304), la vacance dura longtemps. Le 5 juin 1305, les cardinaux élirent enfin l’archevêque de Bordeaux, Bertrand de Got (Clément V).\par
On a dit \emph{un Français}. Oui, si l’on entend originaire du royaume. Non, si l’on veut bien se souvenir que la Guyenne était, en fait, anglaise. Par sa naissance (il était d’une famille seigneuriale des environs de Bazas), par son poste d’archevêque de Bordeaux, Clément était un sujet d’Edouard I\textsuperscript{er}. Mais il avait entretenu de bons rapports avec la cour de France et semble bien avoir été en fin de compte le candidat français. Il était, en tout cas, tout à fait étranger aux factions italiennes. Il demeura au nord des Alpes (à Avignon, depuis 1308). C’était un bon juriste, un homme hésitant et un malade.\par
Il commença par annuler les deux bulles {\itshape Clericis laicos} et {\itshape Unam Sanctam}. Puis, ce fut un long et double chantage dont nous connaissons les enjeux, et où se vint mêler l’affaire du Temple. En 1308, Philippe le Bel demanda l’exhumation des ossements du pape. Puis, en 1310, Clément dut consentir à ouvrir le procès de  \phantomsection
\label{p53} Boniface qu’il traîna en longueur ; l’année suivante, sur la promesse de régler au prochain concile le sort des Templiers, il obtint que l’accusation fût retirée — provisoirement — ce qui permit à Philippe de brandir encore cette arme. Les actes contre le roi et le royaume, émanés, depuis le 1\textsuperscript{er} novembre 1300, de Boniface ou Benoît, furent cassés. Le zèle bon et juste de Philippe fut loué. Nogaret fut absous, moyennant la promesse, non tenue, de se croiser. De même, ses principaux complices (Sciarra devait poursuivre une longue carrière, mêlée à toutes les affaires de Rome). Enfin, en 1313, Clément canonisa Célestin V, très prudemment, sous le nom de Pierre de Morrone. À la mort de Clément (20 avril 1314) et après un conclave qui dura jusqu’au 7 août 1316, le candidat du gouvernement royal, Jacques Duèze, (Jean XXII) fut élu. Entre lui et la cour, les rapports furent bons.
\section[{D. L’Affaire du Temple }]{D. L’Affaire du Temple \protect\footnotemark }\phantomsection
\label{c06d}
\footnotetext{ LIZERAND, {\itshape Le dossier de l’affaire des Templiers} (Les Classiques de l’Histoire de France au Moyen âge), Paris, 1923.}
\noindent L’Ordre du Temple, fondé au début du XII\textsuperscript{e} siècle à Jérusalem, avait reçu sa règle en 1128. Sa mission avait été de protéger les pèlerins de Palestine. En 1291, la prise de Saint-Jean-d’Acre le rendait inutile. Il en était de même des autres ordres : Teutoniques, Hospitaliers. Mais après un essai de colonisation en Hongrie, les Teutoniques s’étaient déjà établis en Prusse dès 1230. Les Hospitaliers, fortement établis dans le royaume de Chypre, avaient organisé une flotte importante et préludaient à la conquête de Rhodes (décidée en 1308 ; accomplie en 1310) qui devait rendre à l’Ordre sa justification. Rien de tel pour le Temple, malgré quelques tentatives. Il y avait longtemps que cette situation préoccupait. L’idée naturelle était de fondre Templiers et Hospitaliers. Elle avait été présentée par saint Louis, débattue au concile de Lyon en 1274, plusieurs fois étudiée depuis par la Curie. Elle était si bien dans l’air qu’elle figure naturellement dans le {\itshape De recuperatione Terrae Sanctae} de P. Dubois (1306-début de 1307). Le grand-maître, Jacques de Molay, écrivit un mémoire pour s’y opposer. Mais déjà à ce moment, des accusations sur la moralité de l’Ordre circulaient. On sait aujourd’hui — ce que Langlois ou Lavisse ne savaient pas — qu’un homme de Béziers, Esquieu de Floyran, les porta vers le début de 1305 au roi Jayme II d’Aragon. Le roi fut incrédule, mais promit à Esquieu une récompense, si le fait était prouvé. Esquieu se rendit alors à la cour de France, où il fut écouté.\par
Je ne reviendrai pas sur ces accusations \footnote{ Voir Ch. V. LANGLOIS, {\itshape Le procès des Templiers}, dans \href{http://gallica.bnf.fr/document?O=N087160}{\dotuline{{\itshape Revue des Deux Mondes} [http://gallica.bnf.fr/document?O=N087160]}}, t. 103, 1891, p. 382-421. — Id., {\itshape L’Affaire des Templiers}, dans \href{http://gallica.bnf.fr/document?O=N054717}{\dotuline{{\itshape Journal des Savants} [http://gallica.bnf.fr/document?O=N054717]}}, 1908, p. 417-435. — S. REINACH, {\itshape Cultes, mythes et religions}, t. IV, Paris, 1912, p. 252-3.}. À part certaines pratiques de sodomie, elles sont évidemment calomnieuses. La vérité semble être : l’idée de la suppression de l’Ordre ou de sa fusion comportait une confiscation de ses biens pour la Croisade (c’est la thèse de P. Dubois qui veut ne laisser au Temple que les domaines d’Orient ; ceux d’Europe formeront une sorte de fonds de croisade). Or, l’entourage royal était persuadé que les trésors de l’Église étaient siens (cf. plus haut). Et on pouvait toujours se servir du prétexte de croisade, comme pour les décimes...\par
 \phantomsection
\label{p54} Aussi, de quelque source que soient venues les premières accusations, la cour les accueillit. À Lyon, lors de son couronnement, en novembre 1305, les gens du roi en entretinrent Clément V. En avril 1307, lors de l’entrevue de Poitiers, Philippe en parla brièvement au pape. En août, le pape décida une enquête. Mais le 22 septembre, au monastère de Maubuisson, les sceaux furent remis à Nogaret et ce jour-là, nous dit une note du registre du Trésor des Chartes, il fut traité de l’arrestation des Templiers. On gagna l’Inquisition générale de France et, le 13 octobre, ce fut le coup de filet, dont la réussite fait honneur à la perfection du mécanisme administratif.\par
Pour le détail, voir Langlois ; c’est une belle expérience de critique du témoignage.\par

\begin{enumerate}[itemsep=\baselineskip,]
\item[]\listhead{Les phases sont :}
\item  
\begin{listalpha}[itemsep=0pt,]
\item[]\listhead{Premier interrogatoire à deux stades :}
\item par les commissaires royaux,
\item par l’Inquisition assistée de fonctionnaires royaux.
\end{listalpha}

 L’instruction rédigée en septembre 1307 prescrivait d’examiner \emph{« la vérité par gehene, se mestier est »}. Aux Templiers, les commissaires \emph{« leur prometeront pardon se il confessent verité en retornant à la foi de Saincte Église ou autrement que ils soient à mort condempné »}.
 
\item  Mais Clément V s’émut. Tout en ordonnant dans toute la chrétienté l’arrestation des Templiers et le séquestre de leurs biens, il envoya deux cardinaux devant lesquels les Templiers et Jacques de Molay (arrêté en France) rétractèrent leurs aveux. Il prétendit alors se réserver le procès (février 1308). D’où violente campagne d’intimidation. Philippe se fit approuver par l’Assemblée de Tours (mai 1308).\par
 
\begin{listalpha}[itemsep=0pt,]
\item[]\listhead{À l’entrevue de Poitiers (juin 1308), il fut décidé :}
\item Administration des biens par des commissaires du roi et des évêques.
\item Le sort de l’Ordre serait réglé en un concile, l’affaire préparée par des commissaires pontificaux enquêtants.
\item Dans chaque diocèse, le procès contre les personnes serait mené par l’évêque qui s’adjoindrait des inquisiteurs. Le jugement définitif devait être prononcé par un Conseil provincial (donc, des deux parts, enquête, puis concile) ; les hauts dignitaires réservés au Saint-Siège.
\end{listalpha}

 
\item  Alors s’ouvre la double enquête. Son déroulement en France inquiéta les gens du roi, en raison de la liberté relative des dépositions devant la Commission pontificale de Paris, cependant bien craintive. Les évêques français y dominaient et les gens du roi y assistaient \footnote{Cette présentation de l’affaire des Templiers est inachevée dans le manuscrit de Marc Bloch.}.
 
\end{enumerate}

\chapterclose


\chapteropen
\chapter[{7. Les cadres de la vie sociale}]{\textsc{7. }Les cadres de la vie sociale}\phantomsection
\label{c07}\renewcommand{\leftmark}{\textsc{7. }Les cadres de la vie sociale}


\chaptercont
\section[{A. La famille}]{A. La famille}\phantomsection
\label{c07a}
 \phantomsection
\label{p55}
\begin{enumerate}[itemsep=0pt,]
\item[]\listhead{Il faut être bref :}
\item parce que le sujet est très mal connu,
\item parce que, à tort, il est un peu en dehors des cadres de l’histoire, telle qu’on l’enseigne, et que ce n’est pas ici le lieu de critiquer — sauf pour indiquer discrètement que la conception aurait bien besoin d’en être renouvelée.
\end{enumerate}

\noindent Les liens du sang sont des liens très forts. Rien de plus caractéristique que l’observation de Joinville, à propos de la belle conduite, au combat de Mansourah, de la bataille de Gui de Monvoisin : \emph{« Et ce ne fu pas de merveille se il et sa gent se prouverent bien celle journée, car l’en me dist... que toute sa bataille, n’en fallait guères, estoit toute de chevaliers de son linnage et de chevaliers qui estoient ses hommes liges »}.\par
Cette solidarité n’exclut pas, bien entendu, les luttes à l’intérieur de la famille, nées des habitudes d’un âge de violence. Mais, très forte à l’égard des groupes extérieurs, elle se traduit de deux façons surtout : dans le droit criminel, dans la vie économique et le droit des biens.\par
La \emph{faide}, \emph{« vindicta parentum quod faidam dicimus »}. Solidarité active et passive. La classe chevaleresque tend à se la réserver. En fait, elle est générale, très vive notamment dans les villes. L’œuvre de paix des pouvoirs publics. Elle se borne essen­tiellement à imposer des trêves ou des asseurements, c’est-à-dire des paix. Il est, en Flandre et en Normandie, de principe que le prince ne peut gracier qu’avec l’assentiment des parents. Ou bien en exigeant, avant le début de la guerre, un délai (quarantaine le roi). Vers 1260, un chevalier, appelé Louis Defeux, avait été attaqué et blessé par un certain Thomas d’Ozouer. Il poursuivit l’agresseur devant la cour du roi. L’accusé ne vint point. Mais que lui reproche­-t-on ? Il avait été blessé lui-même par un neveu de Louis ; bien plus, il avait attendu quarante jours pour se venger. Sans doute, réplique Louis, mais ce que fait mon neveu ne me regarde point. Erreur, décidèrent les juges du saint roi. Le plaignant \emph{« n’a qu’à s’en prendre à lui-même ; s’il a reçu une blessure, il s’est mal gardé »}.\par
Mais on ne se vengeait pas toujours. Souvent, on composait. Là encore jouaient les solidarités active et passive. À Lille, au XIII\textsuperscript{e} siècle, la part dans le prix du sang (solidarité active) s’étendait  \phantomsection
\label{p56} jusqu’au groupe pourvu d’un trisaïeul commun ; à Saint-Omer, la solidarité passive jusqu’à celui qui avait pour ancêtre commun le père du trisaïeul.\par
Du point de vue économique, l’habitude de la copropriété familiale est très répandue dans la campagne, même dans la petite noblesse. À vrai dire, la vente, la donation sont plus généralement, au XIII\textsuperscript{e} siècle, soumises au consentement des proches. Mais partout fonctionne le retrait lignager.\par
La famille, groupe d’entraide. Les remariages. Joinville parlant de ses chevaliers tués en Égypte \emph{« par quoy il couvint leur femmes remarier toutes six »}.\par
Cependant, cette organisation familiale, qui avait toujours souffert de l’incertitude des deux lignées, est certainement en décroissance. Beaumanoir, qui écrit entre 1280 et 1290, a le sentiment très net que, de son temps, le groupe des proches liés par le devoir de vengeance est allé s’étrécissant : jusqu’aux cousins issus de germains, aux cousins germains même. Le retrait lignager, se substituant aux consentements, est un symptôme. L’effritement des liens de proche à proche a été parallèle à celui des liens vassaliques. Il fut le résultat des transformations économiques, mobilisant la propriété ; de la mobilité des hommes ; surtout de la consolidation de l’État.
\section[{B. Classes et groupes personnels}]{B. Classes et groupes personnels}\phantomsection
\label{c07b}
\noindent La société française, à l’époque des derniers Capétiens, se trouve placée à un moment décisif de l’évolution de la structure sociale. De l’époque précédente, elle a hérité un système de liens personnels, entre un seigneur qui commande et protège, un dépendant qui obéit et qui sert : vasselage, servage. Ces liens sont encore très forts. Voir le texte, déjà cité, de Joinville. Mais leur permanence tend cependant à s’effacer. Et parallèlement à cet affaiblissement des relations d’homme à homme, se produisent deux grands faits : consolidation des États (monarchies, grandes principautés) ; constitution de classes juridiques, beaucoup mieux hiérarchisées et plus strictement définies.
\section[{C. La seigneurie rurale et les classes dans la société paysanne }]{C. La seigneurie rurale et les classes dans la société paysanne \protect\footnotemark }\phantomsection
\label{c07c}
\footnotetext{ Voir H. SÉE, {\itshape Les classes rurales et le régime domanial en France au Moyen âge}, Paris, 1901. Marc BLOCH, {\itshape Caractères originaux de l’histoire rurale française}, Paris, 1931.}

\begin{enumerate}[itemsep=0pt,]
\item[]\listhead{Définitions :}
\item Territorialement, petites exploitations autour d’une grande.
\item Groupes d’hommes soumis à un même maître.
\end{enumerate}

\noindent L’étendue est variable. Mais il est rare qu’elle atteigne l’étendue d’un village avec son terroir. En pays d’habitat dispersé, elle s’étend à plusieurs hameaux.\par
Les relations sont réglées à l’intérieur de la seigneurie par la coutume. Un des grands faits, qui commence au XII\textsuperscript{e} siècle, est la mise par écrit de ces coutumes, par enquête et accord, qui aboutit parfois à certaines modifications et surtout à remplacer l’arbitraire  \phantomsection
\label{p57} par le fixe : soit \emph{chartes de coutumes} proprement dites, soit insertion dans d’autres actes (tels qu’un acte d’affranchissement) de clauses détaillées qui en font presque une charte de coutumes. Quelques exemples : Lorris, sous Louis VII ; Beaumont, 1182.\par
Examinons de plus près les divers aspects :\par
\subsection[{1° Les rapports du domaine avec les tenures}]{1° Les rapports du domaine avec les tenures}
\noindent Le grand fait est l’amenuisement ou disparition du domaine. Le plus souvent, il subsiste, mais ce n’est plus qu’une grosse ferme parmi d’autres. Parallèlement, diminution des corvées, réduites surtout à fournir un appoint de main-d’œuvre. Le salariat rural. La pratique, de plus en plus répandue, de l’affermage des domaines. Quelques exceptions : l’exploitation des abbayes cisterciennes.
\subsection[{2° Les droits sur les tenures}]{2° Les droits sur les tenures}
\noindent Ils sont véritablement héréditaires à l’époque où nous sommes (à très peu d’exceptions près). Mais le passage d’un héritier à l’autre est soumis en général à réinvestiture par le seigneur ou son représentant et à paiement d’une redevance (relief). De même, et cela régulièrement, pour l’aliénation (lods et ventes). Et ici le seigneur conserve, dans une certaine mesure, le droit de refuser le nouvel acquéreur ou tout au moins de le forcer à mettre hors la main la terre acquise. Il s’efforce d’en user souvent, pour éliminer les acquéreurs incommodes : institutions ecclésiastiques qui ne paient pas le relief (quand on les accepte, on leur demande un droit spécial : amortissement) ; autour de Paris, bourgeois de Paris.\par
Ne nous demandons pas qui est \emph{propriétaire} (encore qu’au XIII\textsuperscript{e} siècle les textes emploient parfois ce terme pour désigner le tenancier héréditaire). L’opinion juridique médiévale ne conçoit que des droits réels, divers, superposés et liés. Celui du tenancier est lié envers le seigneur (qui peut par exemple lui imposer l’obligation de cultiver), envers sa famille, envers la collectivité rurale. Naturellement, la tenure peut être à son tour amodiée à perpétuité, à vie ou à temps. Ce qui est rendu facile, nous le verrons, par la faiblesse du cens.
\subsection[{3° Les charges de la tenure}]{3° Les charges de la tenure}

\begin{listalpha}[itemsep=0pt,]
\item[]\listhead{Quelques tendances :}
\item Diminution des services par rapport aux redevances ;
\item Remplacement des redevances en nature par des redevances en argent ;
\item Fusion des redevances d’origines différentes, dont on ne sent plus la différence, parce qu’elles tendent toutes à se fixer sur la terre ;
\item Fixité et ses conséquences (pour l’argent).
\end{listalpha}


\begin{enumerate}[itemsep=0pt,]
\item[]\listhead{Essai de classement des principales redevances :}
\item Le cens, ses formes en nature (champart, terrier, agrier), taux variable : 1/30 à 1/6. Fréquemment, autour de 1/10. Sa faiblesse relative (chef cens ; menu cens).
\item Les redevances sur les maisons : œufs, poules, etc.
\item La taille ; étude réservée pour plus tard.
\item La dîme : dans quelle mesure elle fut et est redevenue seigneuriale ; grosses et menues dîmes ; importance de la dîme ; les problèmes de perception (vers 1250 rupture entre le chapitre de Paris et ses serfs, parce que le chapitre voulait percevoir la dîme sur-le-champ, comme le champart). Y ajouter les autres redevances afférentes à l’Église.  \phantomsection
\label{p58}
\item Les redevances (et éventuellement les services) aux avoués, le tensement, sa confusion typique (sur les terres de Notre-Dame) avec d’autres redevances.
\end{enumerate}

\subsection[{4° Les formes du droit de commandement seigneurial}]{4° Les formes du droit de commandement seigneurial}
\noindent Son caractère général : tendance à la fixité.\par

\begin{enumerate}[itemsep=0pt,]
\item La justice. Simple indication à reprendre ailleurs. Justice foncière, toujours. Basse justice, presque toujours. Haut justicier, très souvent. Le seigneur juge et partie. Prélude d’un système où cette notion sera abattue.
\item Le droit de ban. Pouvoir réglementaire. Le problème de la réglementation agraire mis en parallèle avec la réglementation des métiers.
\item Les monopoles : application du droit de ban. Le moulin ; le four ; les chevaux, le dépiquage ; le banvin (monopole de vente).
\end{enumerate}

\subsection[{5° Les formes de l’aide au seigneur}]{5° Les formes de l’aide au seigneur}

\begin{enumerate}[itemsep=\baselineskip,]
\item Le droit de gîte, fournitures aux chiens.
\item Le droit au crédit.
\item Le service d’ost. Tendance à sa limitation. Exemples.
\item  La taille, ou aide. Histoire sommaire. L’idée de prière. À qui s’étend l’obligation à la taille ? Que toute taille à l’ordinaire est arbitraire.\par
 
\begin{itemize}[itemsep=0pt,]
\item[]\listhead{Exemples de Notre-Dame de Paris :}
\item Entre 1198 et 1216 : taille pour le paiement de la décime papale ;
\item Vers 1212 : pour la construction des bâtiments capitulaires ;
\item En 1219 : pour la construction des maisons destinées aux chanoines ;
\item En 1221 : pour l’achat d’une dîme ;
\item Vers 1232 : motif inconnu ;
\item En 1247, sur un village seulement (Corbreuse) : pour le paiement d’une dette ;
\item En 1249 et 1251 : taille, motif inconnu, qui provoque une révolte.
\end{itemize}

 
\item Sans compter les \emph{sous du roi}, dont le chapitre — église royale — gardait parfois une part, et que nous retrouverons plus loin.
\end{enumerate}

\noindent Caractère toujours un peu contesté de la redevance, en raison de son caractère variable, dans le temps et dans le montant. Mais tendance à devenir fréquente (progrès de l’économie argent) et fixe.\par

\begin{enumerate}[itemsep=0pt,]
\item[]\listhead{Deux formes :}
\item Au cas : mais se borne en général à la taille du roi sur ses sujets ou du seigneur sur ses vassaux militaires.
\item Annuelle et de montant stable : c’est l’abonnement, forme type de la fixation de la taille rurale (parfois avec réserve de certains \emph{cas} — exemples). La taille se fixe sur le sol.
\end{enumerate}

\noindent Les droits que nous venons d’énumérer pèsent sur tous les hommes qui habitent la seigneurie ou y possèdent des biens. Mais ces hommes ne sont pas tous au même niveau juridique — ou social.  \phantomsection
\label{p59}
\subsection[{6° Liberté et servage}]{6° Liberté et servage}

\begin{listalpha}[itemsep=0pt,]
\item[]\listhead{Distinction de deux classes, par catégorie de liens :}
\item Le vilain libre \footnote{ Voir Marc BLOCH, {\itshape Liberté et servitude personnelles au Moyen âge, particulièrenient en France. Contribution à une étude des classes}, dans {\itshape Anuario de historia del Derecho español}, 1933, p. 19-115.}. Il n’est attaché au seigneur que par l’habitat et la possession. Sa condition n’est donc pas héréditaire (les efforts faits par les seigneurs pour retenir leurs vilains ont en général échoué, parce que la chose n’était pas possible ; ils ont dû se borner à enlever parfois aux partants leurs tenures : c’est la sanction que prévoient pour les \emph{taillables} les coutumes de Simon de Montfort, le 1\textsuperscript{er} décembre 1212). Il choisit donc son seigneur. En cela, consiste sa liberté. Les autres mots : \emph{hôtes}, \emph{manants}, \emph{couchants} et \emph{levants}.
\item Le serf, appelé aussi \emph{homme de corps}, parfois \emph{homme lige}. Le serf n’est pas un esclave. Il est, lui aussi, un \emph{hôte}, etc. Mais en outre, il est attaché à un seigneur (qui est ordinairement ou forcément le seigneur de sa terre) par les liens d’une dépendance personnelle et héréditaire. Par là même, il est astreint à certaines charges propres. Voyons-les, en remarquant bien qu’un tableau général est forcément amené à négliger les nuances des coutumes locales.
\end{listalpha}


\begin{enumerate}[itemsep=0pt,]
\item[]\listhead{Mais d’abord équivoques à prévenir :}
\item Le serf n’est pas \emph{attaché à la glèbe} (encore que les expressions \emph{servi glebe} ou \emph{glebe affixi} commencent à apparaître à notre époque dans quelques documents ; mais ce sont des expressions savantes maladroitement empruntées au vocabulaire du colonat). Il y a eu une tentative comme pour les vilains, un peu plus que pour eux. Les coutumes de Simon de Montfort soumettent les serfs émigrés à la confiscation des meubles, alors que le vilain émigré ne perd que ses immeubles. Mais elles sont restées là aussi fragmentaires et inefficaces. La particularité du serf est que, où il va, il reste attaché à son seigneur de corps. C’est la thèse (justifiée par les faits) de Beaumanoir qui, expressément, reconnaît la faculté de déplacement du serf. Les serfs forains. Leur sort.
\item De la même manière, il n’y a pas de tenure servile. Quelques efforts isolés au XII\textsuperscript{e} siècle ont été faits par divers seigneurs pour refuser d’admettre sur des terres — possédées originellement par leurs serfs — des tenanciers qui ne fussent pas de condition servile. Ils n’ont pas abouti à créer un droit. Raisons : l’abondance de la terre pendant longtemps. Le changement qui ne se fera qu’aux derniers siècles du moyen âge, sera le résultat de l’accroissement des disponibilités terriennes. Aussi de la constitution plus nette des traits de classe, de la disparition de la notion d’attache. Une réserve toutefois : le Midi, ou peut-être plus exactement le Sud-Ouest. Les terres de casalages sont spécifiquement serviles ; et l’attache apparaît au moins dès le début du XIV\textsuperscript{e} siècle.
\item La taille, même arbitraire, n’est pas spécifiquement servile. Le serf risque simplement de devoir deux fois la taille — tout comme d’ailleurs le vilain qui a des terres en deux seigneuries. Mais comme la plupart des affranchissements comportaient abonnements, la confusion se fait (en Champagne, dès le début du XIV\textsuperscript{e} siècle).
\end{enumerate}

 \phantomsection
\label{p60}
\begin{listalpha}[itemsep=0pt,]
\item[]\listhead{Ceci dit, voyons les charges typiques :}
\item Le chevage. Sa désuétude.
\item Le formariage. Sa raison d’être. Caractère contagieux en principe du servage.
\item La mainmorte. Les deux types. La pratique (rachat).
\item La justice suit le serf. Variabilité de la règle. En général, justice criminelle.
\end{listalpha}

\noindent Seulement, si le serf est attaché à son seigneur, il n’est pas que cela. Il est également le membre d’une classe inférieure, méprisé parce qu’il n’est pas libre. Il ne peut pas entrer dans les ordres, à moins d’affranchissement. Il ne peut témoigner contre les hommes libres (exception pour les serfs royaux et de certaines églises royales). Les personnes libres souvent refusent de se marier avec les personnes serves. Dualité caractéristique. Mais le trait \emph{classe} va grandissant, à mesure que le sens de l’attache personnelle se perdra.
\subsection[{7° La décadence du servage et ses transformations}]{7° La décadence du servage et ses transformations}
\noindent Quelle était la proportion des libres et des serfs sous Philippe Auguste ? Extrêmement variable selon les régions, parce que, si la notion des rapports personnels a été générale, elle a pris selon les circonstances locales des formes particulières. Il y a du moins un pays sans servage : la Normandie. En raison des invasions scandinaves. Il y a des régions où le servage est au moins presque totalement absent : le Forez. D’une façon générale, il semble bien que dans l’ensemble de la France les serfs aient été les plus nombreux, sans toutefois qu’il ait jamais eu, nulle part, absence totale de vilains \emph{libres}.\par
Or, les choses changent beaucoup au XIII\textsuperscript{e} siècle. Il est probable que certaines familles serves ont échappé au servage par simple désuétude. Là surtout où le chevage avait cessé d’être payé. Mais le grand instrument de sortie hors du servage a été l’affranchissement régulier.\par
Une vieille théorie, héritée du temps où il y avait des esclaves, voulait que l’affranchissement — la \emph{manumissio} — fût une œuvre pie. C’est pourquoi la plupart des affranchissements se donnent expressément pour des actes de charité ou de justice, cela même lorsque nous savons, soit par l’acte lui-même, soit par recoupements, que la liberté a été payée. Par exemple : la formule grégorienne \footnote{ M. BLOCH, {\itshape Rois et serfs}, p. 154. } ; le fameux préambule de 1315 \footnote{{\itshape Ibid.}, p. 132.}. Ne parlons pas d’hypocrisie : ce serait trop brutal.\par

\begin{enumerate}[itemsep=0pt,]
\item[]\listhead{Il y a :}
\item obéissance à la routine notariale,
\item obéissance aux convenances,
\item idée qu’une bonne action ne cesse pas d’être bonne si elle est rémunératrice,
\item influence des donations et aumônes aux églises (testament d’Alfonse de Poitiers).
\end{enumerate}

\noindent En fait, je le répète, la plupart des affranchissements étaient payés. Et cela est parfaitement compréhensible.\par

\begin{enumerate}[itemsep=0pt,]
\item[]\listhead{Il fallait donc deux conditions :}
\item Que le seigneur trouvât avantageux, et il le trouva souvent au XIII\textsuperscript{e} siècle, pour diverses raisons : parce que les charges serviles étaient de rendement irrégulier et de perception difficile (machinerie  \phantomsection
\label{p61} des collecteurs de mainmorte) ; parce que les villages de serfs risquaient de se dépeupler ; parce que, surtout, un besoin d’argent dans la société du temps se faisait fréquemment sentir : en 1255, à Saint-Germain-des-Prés, la chapelle de la Vierge s’élève avec l’argent des manumissions ; les chanoines de Sainte-Geneviève, un peu plus tard, emploient les deniers des libertés à des achats, au paiement d’une dette aux usuriers lombards, à l’acquittement d’une décime (impôt royal sur les églises).
\item Du côté des serfs, le désir de s’affranchir, très général et de plus en plus à mesure qu’il y avait moins de serfs (mariages) ; mais aussi la possibilité d’acheter l’affranchissement. C’est ce qui limite le mouvement. Contraste caractéristique entre les serfs des mêmes églises en Ile-de-France et Champagne ; en Beauce et en Sologne.
\end{enumerate}

\noindent L’état de nos connaissances ne permet pas de dresser une carte exacte, ni une courbe bien précise des affranchissements.\par

\begin{listalpha}[itemsep=0pt,]
\item[]\listhead{Du moins, quelques faits ressortent clairement :}
\item Ils ont été nombreux surtout depuis le milieu du XIII\textsuperscript{e} siècle, s’appliquant non seulement à des individus ou ménages isolés, mais à des seigneuries entières (plusieurs centaines d’individus parfois).
\item Ils ont, dans certaines régions, comme l’Ile-de-France, fait disparaître à peu près complètement la classe servile.
\item Dans le reste du royaume, ils n’ont fait, jusqu’en 1328, que l’entamer ; la pratique se poursuivra durant les siècles suivants. Ainsi, une part notable de l’{\itshape unearned increment} paysan s’est trouvée remise en vigueur ; et le nombre des serfs a considérablement diminué.
\end{listalpha}

\noindent Je n’ai parlé jusqu’ici que d’affranchissement complet. Mais il s’est produit dans certaines régions que certaines charges seules ont été, par actes partiels, supprimées : la mainmorte par exemple, sans le formariage ou le chevage. En Champagne notamment. Le résultat a été un effritement de la notion de servage. Surtout, là où il subsiste, le servage, dès la fin du XIII\textsuperscript{e}  siècle, tend à se modifier selon les lignes que nous connaissons déjà : attache au sol, tenure servile, adoption pour critère de la taille, et même de la corvée à volonté.
\subsection[{8° Les classes de fait dans la société paysanne}]{8° Les classes de fait dans la société paysanne}
\noindent \labelchar{a)} Les {\itshape sergents} ou {\itshape ministeriales.} Fonctionnaires seigneuriaux, souvent serfs (le seigneur juge ces derniers plus sûrs). En règle générale, dans le Nord, par village, {\itshape maire} et {\itshape doyen.} Surveillance du domaine, perception des redevances, justice souvent. Dans le centre, {\itshape juges ;} dans le Midi, {\itshape bayles.} Auxquels il faut ajouter, comme répondant à la même catégorie, dans les grandes seigneuries, les fonctionnaires centraux, sénéchaux, maréchaux, celleriers et — confondus à peu près avec eux ou séparés d’eux par une série de nuances — les artisans employés au service du maître. Laissons cependant ces derniers. Si ce sont des sergents, il leur manque les caractéristiques qui font des maires et doyens, comme des principaux fonctionnaires centraux, une classe sociale supérieure à la classe rurale : cela en dépit même, là où elle existe, de la terre servile.\par

\begin{enumerate}[itemsep=0pt,]
\item[]\listhead{Ce sont :}
\item la possession d’une terre assez importante destinée à rémunérer leurs services et qu’on appelle leur fief ;
\item les exemptions de droits, notamment de taille (préfiguration de l’exemption pareille des officiers royaux) ;
\item les gains de leurs charges légitimes (part des  \phantomsection
\label{p62} redevances), illégitimes (corvées, cadeaux exigés des tenanciers)
\item les pouvoirs de commandement ; notamment de commandement militaire : à Bonneval, d’après un acte de 1265, lorsque les hommes du lieu partent à l’ost, le maire prend leur tête et lève la bannière des moines ;
\item le genre de vie qui en est la conséquence ; le maire a dans le village sa \emph{maison forte}, il a parfois son sceau.
\end{enumerate}

\noindent En fait, au XIII\textsuperscript{e} siècle, gens puissants. Ils ne sont que rarement arrivés à expulser, plus ou moins partiellement, le seigneur. Mais ils sont arrivés à peu près partout à rendre leur charge en fait héréditaire (si en droit, les seigneurs affirment la non-hérédité). Ils sont si encombrants que parfois le seigneur leur rachète leurs charges. Grâce à l’hérédité, il y a de véritables familles de sergents où d’un tronc à l’autre, on se marie à son rang. Ou bien, ils cherchent femme — ou leurs filles cherchent mari — dans les familles chevaleresques. Eux-mêmes, souvent, se poussent à la chevalerie. Même s’ils sont serfs. Il y a plus d’un chevalier serf dans la France de la première partie du XIII\textsuperscript{e} siècle. La plupart sont des sergents.\par
Mais cette catégorie humaine n’a pas été extrêmement durable (en France). Dès le début du XIV\textsuperscript{e} siècle, les principaux d’entre eux s’échappent vers le haut. Ils se fondent dans la petite gentilhommerie. Et le nouveau clivage des classes met obstacle. Le Parlement sous saint Louis décide que le seigneur, qui fait son serf chevalier, par là même, bon gré mal gré, l’affranchit.\par
\bigbreak
\noindent \labelchar{b)} Plus durables sont les distinctions de simple fortune, notamment entre laboureurs et brassiers. Le témoignage des corvées. Un acte orléanais de 1210, parmi les manants d’une terre, distingue \emph{« ceux qui cultivent avec les bœufs, ceux qui œuvreront avec la houe »}.
\subsection[{9° Les paysans et le seigneur}]{9° Les paysans et le seigneur}
\noindent Comment les paysans sont-ils organisés ? Le village, communauté agraire (sauf dans les pays d’habitat dispersé). Les règlements agraires. Le communal.\par
Comment se fait le partage des attributions avec le seigneur. Il est d’autant plus délicat qu’il y a parfois plusieurs seigneurs, ce qui est gênant. À Hermonville, en Champagne, le village est divisé entre huit ou neuf justices. À partir de 1320, les habitants obtiennent des jurés communs, qui règlent la police agraire. Cette difficulté même mise à part, le partage des droits est délicat. Il est variable. Souvent, les habitants élisent eux-mêmes certains petits fonctionnaires — gardes des vignes, vachers — avec l’assentiment du seigneur ; ou bien, le représentant du seigneur les installe, avec l’avis des habitants. Même partage pour les bans de vendange et de moisson.\par
Mais ces habitants ont-ils une organisation ? Ils ont cherché à s’en donner une souvent. Il y a eu, au XII\textsuperscript{e} siècle et au début du XIII\textsuperscript{e}, des tentatives communales rurales qui ont réussi dans certains lieux de la Picardie, qui ont échoué dans l’Ile-de-France (où nous ne les connaissons plus que par les interdictions de communes, enregistrées dans certains actes). Dans le Midi, des \emph{consulats}. La charte de Beaumont promet l’élection du maire et des jurés. Il y eut utilisation des nécessités ecclésiastiques : la fabrique, les confréries, comme celle de Louvres, sous saint Louis : bâtir une église, entretenir les puits et les chemins, conserver \emph{les droits du village} (contre les maires) : caisse commune, boycottage. Il y  \phantomsection
\label{p63} avait des tendances très nettes à la confédération : dans les communes fédérales du Nord où plusieurs villages se groupent souvent en commune collective ; les villages de Notre-Dame de Paris avaient, sous saint Louis, cherché à traiter en commun de leur affranchissement. Un peu plus tard, quand les hommes d’un de ces villages, Orly, sont en lutte avec le seigneur au sujet de la taille, les hommes des autres villages sont \emph{legiers} avec eux et contribuent aux frais. Un gros problème juridique — la personnalité morale. Le Parlement répond non quand il n’y a pas corps ni commune. Mais pratiquement il faut bien que les habitants se réunissent : pour établir les impôts de répartition au profit du seigneur ; pour s’accorder avec lui ; nous le verrons, pour plaider contre lui.\par
Et il y a souvent désaccord. Beaucoup de petites révoltes sourdes devinées dans les textes. Mouvements généraux moins fréquents et graves qu’au siècle suivant. Il y en a néanmoins. Les Pastoureaux (1251) sont surtout un mouvement mystique et anticlérical à la fois, résultat des mauvaises nouvelles de la croisade ; mêlés de primitives revendications sociales et finalement écrasés, après avoir un moment été bien accueillis par les bourgeoisies urbaines et la reine. Le mouvement du Sénonais, en 1315, est mal connu ; là aussi mysticisme et révolution ; les insurgés avaient mis à leur tête \emph{un roi} et \emph{un pape}. Naturellement, il fut aussi écrasé.\par
Mais le grand fait, c’est l’intrusion de la justice royale. À part des hésitations, elle se décide à intervenir entre le seigneur et ses sujets. Toute une jurisprudence qui n’a pas été étudiée, mais qui marque notamment l’existence sous saint Louis de procès de servage en cours royales. Cette jurisprudence est parfaitement respectueuse des droits établis. Elle n’incline pas du tout vers les paysans. Mais elle est. Et par elle, en effet, se consolidera notamment l’hérédité de la tenure.
\section[{D. La classe chevaleresque. Le fief et l’hommage}]{D. La classe chevaleresque. Le fief et l’hommage}\phantomsection
\label{c07d}
\subsection[{1° La chevalerie }]{1° La chevalerie \protect\footnotemark }
\footnotetext{ Voir LANGLOIS, {\itshape Les origines de la noblesse en France}, dans \href{http://gallica.bnf.fr/document?O=N017467}{\dotuline{{\itshape Revue de Paris} [http://gallica.bnf.fr/document?O=N017467]}}, 1902, t. 5. p. 818-851.}
\noindent La constitution, en tant que caste juridique, de la classe nobiliaire, a résulté du caractère héréditaire peu à peu revêtu par l’aptitude à recevoir la chevalerie. C’est sur celle-ci qu’il convient d’abord de tourner nos regards. Plus précisément, sur les vicissitudes de la cérémonie qui crée un chevalier : l’adoubement.\par

\begin{enumerate}[itemsep=\baselineskip,]
\item  
\begin{listalpha}[itemsep=0pt,]
\item[]\listhead{Description de l’adoubement :}
\item remise des armes,
\item la \emph{collée} ({\itshape alapa militaris}),
\item les jeux à cheval, quintaine.
\end{listalpha}

 
\item Première étape (très mal connue). Remise des armes au jeune homme {\itshape libre} parvenu à l’âge adulte.
\item  Spécialisation d’une classe de guerriers professionnels.\par
 
\begin{listalpha}[itemsep=0pt,]
\item[]\listhead{Le plus grand nombre appartiennent :}
\item à la catégorie des vassaux,
\item à celle des possesseurs de la seigneurie.
\end{listalpha}

 À ceux, distinctement, que d’un terme désignant alors une situation de fait, on appelle {\itshape nobiles.} C’est l’état des choses du XI\textsuperscript{e} siècle et encore d’une grande partie du XII\textsuperscript{e}. Il y a, dans la plupart des cas, hérédité de fait.\par
  \phantomsection
\label{p64} 
\begin{listalpha}[itemsep=0pt,]
\item[]\listhead{Mais :}
\item des soldats de fortune, des serfs peuvent accéder à la chevalerie ;
\item seuls font partie de l’ordre des chevaliers, ceux qui ont reçu l’adoubement (n’y participent point ceux qui n’ont d’autres titres que d’avoir un père chevalier).
\end{listalpha}

 
\item  Il faut insister sur cette notion d’{\itshape ordre}, qui appartient au XII\textsuperscript{e} siècle. Elle se rattache au caractère religieux pris par l’adoubement. Au cours du XII\textsuperscript{e} siècle : prière, épée sur l’autel ; chants religieux durant la cérémonie ; création d’une véritable liturgie de la chevalerie. L’évolution, à la fin du règne de Philippe Auguste, peut être considérée comme accomplie — tout juste. Il s’y joint une notion morale.\par
 
\begin{listalpha}[itemsep=0pt,]
\item[]\listhead{Donc :}
\item classe définie par une hérédité de fait,
\item par un genre de vie,
\item par un rang social, entraînant des habitudes et obligations morales particulières.
\end{listalpha}

 
\item  
\begin{listalpha}[itemsep=0pt,]
\item[]\listhead{La cristallisation s’opère lorsque :}
\item on reconnaît certains privilèges aux fils de chevaliers,
\item lorsqu’on interdit l’adoubement à certaines personnes, qui ne sont pas fils de chevaliers,
\item lorsque naissent des privilèges et un droit spécial.
\end{listalpha}

 
\end{enumerate}


\astermono


\begin{enumerate}[itemsep=0pt,]
\item[]\listhead{Ici, quelques textes peuvent nous servir utilement de points de repère :}
\item {\itshape Summa de legibus in curia laicali}, composé sous saint Louis. Il s’agit du droit de monnayage. En sont exempts les moines, les clercs ordonnés, les chevaliers \emph{« et omnes ex milite de uxore propria procreati »}.
\item Enquête de 1247 sur l’administration de Josse de Bones, bailli de Tours \footnote{{\itshape Recueil des Historiens de la France}, t. XXIV, Paris, 1904, 1\textsuperscript{re} partie, p. 100.}. Un certain Pierre de Lercé, qui se qualifie de chevalier, se plaint d’avoir été frappé d’une amende par Josse \emph{« quod non erat recte miles, quoniam pater ejus non fuerat miles »} (on ne sait ce que Pierre avait fait, qui pût être réservé aux seuls chevaliers ; peut-être seulement pris le titre). Amende injuste, dit Pierre, pour deux raisons : son père tenait sa terre en fief (nous verrons plus loin la force de cette raison) ; sa mère était fille de chevalier et dame (l’argument témoigne d’un certain flottement ; en Champagne seulement, le droit admit que le ventre anoblissait).
\item Règle du Temple. Dès l’origine, on distinguait chevaliers et sergents, manteaux blancs et manteaux bruns. Mais la règle la plus ancienne — 1230 ou environ — ne parle pas des conditions d’admission, sans doute dépourvues de précision juridique. Par contre, la règle en français, du milieu du XIII\textsuperscript{e} siècle, introduit plus de rigueur. Au postulant, on demande s’il est chevalier et fils de chevalier, ou bien descendant de chevalier en ligne paternelle. Seulement si la réponse est affirmative, il sera accepté comme chevalier du Temple. Il y a plus : quiconque est chevalier (c’est-à-dire a reçu l’adoubement) et tel qui le doit être (c’est-à-dire dans les conditions d’hérédité plus haut mentionnées) s’il s’est fait sergent, sera mis aux fers.
\end{enumerate}

\noindent À ces textes, répondent les témoignages des chartes qui, volontiers, d’un fils de chevalier qui n’est pas chevalier, disent \emph{nondum miles}, Peu à peu un mot s’invente pour cette condition d’expectative (qui peut durer) : écuyer (armiger).\par
Il n’y a pas, en France, de loi qui interdise la chevalerie à quiconque n’est pas fils de chevalier.\par

\begin{enumerate}[itemsep=0pt,]
\item[]\listhead{Mais :}
\item  \phantomsection
\label{p65} Le texte de la Règle du Temple montre que l’on considérait cela comme illégitime (il ne suffit pas, pour être accepté parmi les chevaliers de l’Ordre, d’avoir reçu l’adoubement ; il faut encore être fils ou descendant de chevalier).
\item Les textes sur l’anoblissement, que nous verrons plus tard, montrent que, dès le milieu du XIII\textsuperscript{e}  siècle, le roi — et avec lui encore quelques hauts feudataires — revendique le monopole de permettre l’accès à la chevalerie des non-fils de chevaliers : preuve que la chose (si elle avait probablement encore lieu quelquefois en pratique) semblait anormale.
\item Un arrêt du Parlement, déjà cité, montre la barrière mise au moins devant une classe, les serfs.
\end{enumerate}

\noindent En ce qui regarde le droit spécial des nobles, impossible de l’étudier avant d’avoir vu les liens de droit qui, peu à peu, sont devenus propres à cette classe : l’hommage et le fief.\par
La synonymie noblesse = droit à la chevalerie est bien marquée par la lettre de commission de Philippe le Bel en 1302, déléguant des envoyés chargés \emph{« eos nobilitandi, ad hoc quod militiae cingulo valeant decorari »}.
\subsection[{2° L’hommage et le fief}]{2° L’hommage et le fief}
\noindent \labelchar{a) Description de l’hommage :}\par
l’hommage de bouche et de mains. La foi. Son caractère théoriquement viager. Forme ancienne et générale de dépendance, l’hommage s’est restreint pratiquement : 1) aux vassaux militaires, 2) à certains sergents seigneuriaux.\par
\bigbreak
\noindent \labelchar{b) L’aspect économique du contrat de vassalité : le fief.}\par
 Le fief tenure service. Notamment tenure militaire (franc fief). Liaison du fief et de l’hommage. L’investiture. Existe-t-il encore des vassaux non chasés ? Attestés sous Philippe Auguste (1188). Leur diminution s’explique par la généralisation du fief de chambre (cf. saint Louis et Joinville).\par
La hiérarchie des liens de fief et d’hommage. Les ruptures : l’alleu.\par
\bigbreak
\noindent \labelchar{c) Le problème de la pluralité des seigneurs.}\par
Comment il a été amené à se poser, par le fief surtout et aussi par le changement de tonalité de classe de la vassalité. Bien entendu, le problème suppose un conflit entre les divers seigneurs. L’essai de solution ; l’hommage lige (fin XI\textsuperscript{e} siècle) opposé à l’hommage {\itshape plain.} Mais l’hommage lige, à son tour, s’est attaché au fief. Il y a pluralité d’hommages liges. Déjà, il fallait, avant cette décadence, établir une espèce de hiérarchie, entre les hommages {\itshape plains.} Désormais, il faudra faire de même entre les divers hommages liges. Quelques types de solutions envisagées par la pratique ou par le droit.\par

\begin{enumerate}[itemsep=0pt,]
\item Il y a, en principe, un ordre reconnu entre deux hommages d’un même degré. Selon la solution de Guillaume Durand, évêque de Mende, dans son {\itshape Speculum judiciale}, qui est de 1271, c’est le premier auquel le vassal — ou son prédécesseur — a porté l’hommage. Dans les chartes, ceci s’explique sous forme de réserves.
\item Il ne s’ensuit pas de là que le vassal n’ait aucune obligation envers les autres seigneurs. Car, comme Durand le fait observer, le seigneur qu’il n’avait point servi aurait eu le droit de lui prendre son fief. D’où diverses solutions. Pour plus de clarté, appelons le vassal Jean, les deux seigneurs Pierre et Paul (Pierre jouissant de  \phantomsection
\label{p66} l’hommage supérieur) : Jean \emph{aide} Pierre, mais remet son fief à Paul, pour que celui-ci en tire profit pendant la guerre. Il le récompense à la fin de la guerre. Jean aide Pierre de sa personne, mais envoie aussi des troupes à Paul, de préférence prises à la terre qu’il tient de Paul.
\end{enumerate}

\noindent Le tout encore compliqué par des distinctions entre les motifs des guerres, qui peuvent modifier l’attitude à prendre par le vassal vis-à-vis des deux seigneurs.\par
On se doute bien que tout cela aboutissait, dans la pratique, à beaucoup d’arbitraires et d’infidélités.\par
\bigbreak
\noindent \labelchar{d) Quelle force avait donc le lien vassalique ?}\par
En quoi consistaient, en principe, les obligations du vassal ? D’une façon générale, à la protection que lui doit le seigneur, il répond par l’aide.\par

\begin{enumerate}[itemsep=0pt,]
\item[]\listhead{Cette aide prend la forme :}
\item militaire,
\item de service de cour,
\item de l’aide pécuniaire ou taille.
\end{enumerate}

\noindent Là, comme quand il s’agit de tenanciers paysans, le travail des générations a consisté à préciser, par suite à limiter.\par

\begin{enumerate}[itemsep=0pt,]
\item[]\listhead{Notamment :}
\item le service militaire dit d’ost et chevauchée, a une durée fixe, ordinairement de quarante jours ;
\item la taille, à certains cas.
\end{enumerate}

\noindent Dans le royaume ou les grandes principautés, le service de cour est demeuré par l’intermédiaire des fonctionnaires. Parfois, on est allé plus loin, l’obligation service a tout à fait disparu. Nous y reviendrons.\par
Il faut ajouter les rapports quasi familiaux, dont le seigneur tire le droit, souvent profitable, de marier les héritiers de ses vassaux.\par
Que le lien ait encore quelque force, c’est ce que prouve le texte de Joinville, déjà cité. De même, les efforts de la politique de saint Louis pour éviter qu’aucun de ses sujets ne prêtât hommage à un prince étranger (le double hommage de Guillaume le Maréchal avait empêché que ce loyal serviteur des rois anglais ne suivit en France Jean sans Terre contre Philippe Auguste). Mais le lien n’est vraiment puissant que lorsqu’il s’agit d’un seigneur \emph{naturel} que l’on connaît personnellement et de petits vassaux. Le paradoxe de la vassalité.\par
\bigbreak
\noindent \labelchar{e) Le droit des fiefs.}\par
La qualité de tenure de service n’a pas disparu en ce sens que, soit les torts du vassal, soit même son abstention, justifient en droit la commise. Un exemple éclatant en a été la confiscation des États angevins.\par
Le caractère primitivement viager se marque par le renouvellement de l’investiture à chaque changement dans le couple originel. Mais le fief — sauf exceptions expressément stipulées — est héréditaire, sans contestation possible. Au profit des filles également. Au profit des mineurs. Mais ici intervient l’usage du bail. Il était originellement familial. Dans certaines régions, il est devenu seigneurial ; ce qui prouve bien combien le fief a passé dans le patrimoine du vassal.\par
Que se passait-il, lorsqu’il y avait plusieurs héritiers de même rang ? L’intérêt du seigneur avait été longtemps de prévenir l’indivision. En fait, le droit d’aînesse absolu n’a triomphé que rarement (pays de Caux). Ordinairement, l’aîné n’a qu’une part supérieure en étendue, et le manoir principal. En fait, ce système amenait rapidement un grand morcellement. La propriété noble n’était protégée un peu efficacement dans le morcellement que dans le  \phantomsection
\label{p67} Midi, par le régime des substitutions, qui ne se transportera dans le Nord qu’après notre période.\par
Mais en cas de partage, de qui tiennent les cadets ? Tant que la notion de service eut de sa force, il fut de l’intérêt du seigneur de ne connaître qu’un répondant. Il fut donc admis que les cadets tenaient de l’aîné {\itshape (parage).} Mais vint un moment — et c’est l’intérêt de cette histoire — où les services comptent moins que les droits casuels. Expliquer en effet ce qu’était le relief (dans l’Ile-de-France du XIII\textsuperscript{e}  siècle, le \emph{roncin de service}, forme de relief, dispense de toute autre obligation, autre que négative). Alors, suppression du parage qui privait de reliefs des cadets. Ordonnance de 1209 de Philippe Auguste.\par
\bigbreak
\noindent \labelchar{f) Restriction à la classe noble}\par
C’est-à-dire la classe des chevaliers à titre héréditaire et descendants de chevaliers — du fief militaire et de l’hommage.\par
Il y aurait eu, aux yeux du roi et des grands barons, un système concevable : celui qui, forçant tout acquéreur de fief à en faire les services, l’eût par là contraint à la chevalerie et, par suite, à la noblesse. Il y a des traces de cette idée. Mais le principe de classe l’emporta. Législativement, semble-t-il. Beaumanoir fait plusieurs fois allusion à une ordonnance d’un roi, qu’il ne nomme pas, par laquelle il avait été défendu aux \emph{hommes de poesté} de tenir fief (homme de poesté = sujet d’une seigneurie, non noble). Seulement le roi a toujours été conçu comme pouvant dispenser des privilèges, notamment dans l’ordre hiérarchique. D’où la pratique des autorisations d’acquérir, moyennant paiement de \emph{francs fiefs}, codifié à partir de Philippe le Hardi et qui prête à de fructueuses tournées d’agents fiscaux, car le plus souvent, on ne fait que régulariser après coup l’acquisition.\par
Mais un cas assez grave se présentait : lorsque le non-noble avait acquis un fief d’où dépendaient à leur tour des fiefs. Un noble avait-il obligation de prêter hommage à un non-noble ? Le Parlement de Paris répondit non (décision, octave de la Chandeleur 1261).\par
Le droit des fiefs devint le droit des nobles. C’est un élément fondamental de leur droit propre, qui se développa par ailleurs et par suite de leur place privilégiée dans la société.
\subsection[{3° La place de la noblesse dans la société ; le genre de vie noble}]{3° La place de la noblesse dans la société ; le genre de vie noble}
\noindent \labelchar{a) Comment la classe reste ouverte ?}\par
Nous avons vu que la chevalerie est, en principe, réservée aux descendants des chevaliers. Si le principe avait été absolu, il y aurait eu fermeture radicale de la classe. En fait, on se disait toujours que les pouvoirs publics avaient le droit d’accorder des dispenses. Quels pouvoirs ? Le roi, évidemment, tout d’abord. En 1237, un bourgeois normand, nommé Robert de Beaumont, est frappé d’une lourde amende par la cour royale, parce qu’il s’est fait armer chevalier \emph{« sine licencia regis »}.\par
[…] \footnote{[Une page du manuscrit manque, soit une partie des paragraphes a) et b)]}\par
 \phantomsection
\label{p68} L’idée qu’il y avait des pairs de France et qu’ils étaient douze (comme les apôtres) était déjà répandue, à la fin du XI\textsuperscript{e} siècle, dans les milieux fidèles aux Capétiens, ainsi qu’en témoigne le Roland. Mais nous savons mal comment la liste s’établit. En 1216, un arrêt de la cour du roi, énumérant les juges qui ont participé aux plaids, cite avant les autres \emph{évêques et barons}, avec le titre de pairs du royaume, l’archevêque de Reims, les évêques de Langres, Chalon, Beauvais et Noyon et le duc de Bourgogne. Un peu plus tard, en 1224, nous apprenons que la comtesse de Flandre en fait partie. Vers le milieu du XIII\textsuperscript{e} siècle, la liste nous apparaît complète avec le sixième pair ecclésiastique, l’évêque de Laon, et les quatre autres pairs laïques : ducs d’Aquitaine et de Normandie, comtes de Toulouse et de Champagne (liste déjà, ou rapidement, fictive d’ailleurs, le duc de Normandie étant depuis Philippe Auguste le roi de France ; de même depuis 1270, le comte de Toulouse ; depuis Philippe le Bel, le comte de Champagne). Les raisons du choix sont peu claires. En 1297, Philippe IV créa trois pairies nouvelles (Anjou, Artois, Bretagne), arguant que le chiffre douze n’était plus atteint : ce qui était vrai. Mais plus tard, de nouvelles créations amenèrent à dépasser le nombre fatidique.\par
Il y a quelques tentatives du haut baronnat, pour mettre quelques avantages pratiques à la base du titre honorifique de pair. En 1224, les pairs présents à la cour prétendent juger seuls dans la cause de l’un d’eux (la comtesse de Flandre), excluant les officiers royaux. Ils sont déboutés. En 1306, la comtesse d’Artois se réclame, pour son domaine, de prétendues coutumes propres aux pairs de France. Elle est déboutée. Les pairs étant trop peu nombreux pour faire pression, le comte de Bretagne, les comtes de Nevers et d’Auvergne, la foule des moyens barons et des prélats n’était pas disposée à se donner du mal pour quelques grands princes territoriaux ou quelques évêques parmi beaucoup, les uns comme les autres assez arbitrairement choisis.\par
\bigbreak
\noindent \labelchar{c) La vie noble.}\par
Selon la tradition, l’éducation du noble continue à se faire en bonne part hors de chez lui, chez le seigneur de son père, ou à la rigueur un égal. Cette éducation est de plus en plus rarement uniquement de sport et de guerre et de bons usages. Il y a, à vrai dire, encore de hauts personnages qui ne savent pas lire : tel Jean de Nanteuil, chambellan de France sous saint Louis. Ces cas sont de plus en plus rares. Beaucoup, parmi les hauts et moyens barons, savent le latin. Mais comme le latin n’est plus seul à s’écrire, on peut être désormais \emph{litteratus} en ne connaissant que le français. Des nobles nombreux écrivent : tel, parmi les grands princes, Thibaut le Chansonnier, comte de Champagne et roi de Navarre, mort en 1253, qui fut un poète lyrique délicat ; parmi les \emph{povres chevaliers}, Robert de Clary, qui conta en prose picarde l’histoire de la 4e croisade ; parmi les chevaliers un peu plus aisés, qui font désormais carrière au niveau du roi, sous Philippe III et Philippe IV, Beaumanoir, poète de romans courtois, puis rédacteur lucide et intelligent des {\itshape Coutumes de Beauvaisis} ; enfin, dans ces bonnes places, immédiatement au-dessous des grands princes, Joinville. La \emph{courtoisie} comprend à la fois les bonnes manières, où Joinville était passé maître (nous savons qu’on le consultait sur les questions délicates d’usage de la table) et un certain idéal de culture.\par
 \phantomsection
\label{p69} La vertu laïque du chevalier est la prud’homie, que saint Louis prisait si haut. Elle comprend la largesse.\par
Malgré tout, le noble reste essentiellement un homme d’épée.\par
Le portrait du chevalier dans la littérature est avant tout un homme de sport (texte de Philippe de Novare, mort entre 1261 et 1264). Et la devise reste celle de Cligès :\par


\begin{verse}
« Ne s’acordent pas bien ansamble\\
Repos et los, si con moi samble. »\\
\end{verse}

\noindent La guerre et les pauvres chevaliers. Quelles guerres ? Celles du roi comme chevalier soldé. Les croisades. Les guerres privées aussi. En principe, elles sont vues avec beaucoup de défaveur par la royauté. Saint Louis les interdit en 1258 ; exactement, les guerres, les incendies et \emph{carrucorum perturbacionem}. Mais nous savons que la pratique de la faide empêchait de pareilles mesures d’être efficaces. Philippe le Bel se borna à des interdictions temporaires nécessitées par les guerres extérieures. Et les guerres privées — en fait — sont moins nombreuses (progrès de la justice ; de l’autorité de paix ; habitudes d’ordre). Mais elles se perpétuent jusqu’à la fin de l’époque capétienne... et plus tard. La noblesse y tenait. L’interdiction de saint Louis fut une des raisons pour lesquelles on le chansonna ; et en 1315, parmi les requêtes des nobles de Bourgogne, figure celle-ci : \emph{« Que li dis nobles puissent e doient user desormés quant il leur plaira et que il puissent guerroier et contregagier »}.\par
À côté de la guerre vraie, l’imitation de la guerre : les tournois. Imitation souvent sanglante et aussi lucrative. Il y avait des \emph{tournoyeurs} professionnels. L’Église hostile. Interdiction pontificale depuis 1148. Les rois, au moins depuis saint Louis, sont partagés entre des sentiments divers. Ils sont pieux. Ils n’aiment guère ces réunions de gens en armes (comme dit une ordonnance de Philippe le Bel), d’où peut naître une menace pour l’ordre. Ils redoutent d’y voir se décimer leur classe militaire. Mais les chevaliers eux-mêmes aimaient le spectacle des tournois : Philippe le Hardi, qui les interdit parfois, en fit faire cependant en sa présence et y laissa paraître ses frères dont un, Robert de Clermont, y reçut tant de coups de masse d’armes qu’il devint fou. Enfin ils se heurtaient à l’opinion : des nobles ; des marchands même. Aucun, en fait, même saint Louis, n’osa aller plus loin que des interdictions temporaires.\par
Enfin la chasse.\par
Une carrière nouvelle : le service royal (ou celui des grandes principautés) : Beaumanoir, Eustache de Beaumarchais passé du service d’Alfonse de Poitiers à celui de Philippe III. Les soudoyers.\par
Le château. Progrès relatif de l’habitat. La jonchée.\par
\bigbreak
\noindent \labelchar{d) Les fortunes nobiliaires.}\par
Elles sont variables. Il y a de pauvres gentilshommes honteux, envers lesquels Philippe III inscrit un legs dans son testament. Dans l’ensemble, la noblesse était la classe sociale tenant les rentes de la terre. Mais ces rentes vont, on le sait, en diminuant. En réalité, la plupart des nobles ont besoin, pour vivre noblement, d’un appoint : d’où les deux {\itshape rushes} en quelque sorte continus : vers le service des rois, vers la guerre génératrice de butin.
\chapterclose


\chapteropen
\chapter[{8. Les villes }]{\textsc{8. }Les villes \protect\footnotemark }\phantomsection
\label{c08}\renewcommand{\leftmark}{\textsc{8. }Les villes }

\footnotetext{ 
\begin{itemize}[itemsep=0pt,]
\item[]\listhead{Bibliographie}
\item H. PIRENNE, {\itshape Les villes du Moyen âge. Essai d’histoire économique et sociale}, Bruxelles, 1927.
\item A. LUCHAIRE, {\itshape Les communes françaises à l’époque des Capétiens directs}, 2\textsuperscript{e} édition, Paris, 1911 (dans l’Introduction, par L. Halphen, utiles indications bibliographiques).
\item P. VIOLLET, {\itshape Les communes françaises au Moyen âge}, dans {\itshape Mémoires de l’Académie des Inscriptions}, t. XXXVI, reproduit à peu près dans {\itshape Histoire des Institutions politiques et administratives de la France}, t. III.
\item Comme études locales : a) PIRENNE, {\itshape Histoire de Belgique}, Bruxelles, 1900-1932 ; b) sur Paris, HUISMAN, {\itshape La juridiction de la municipalité parisienne de saint Louis à Charles VII}, Paris, 1912 (renseignements généraux sur les origines de la municipalité) ; VIDIER, {\itshape Les origines de la municipalité parisienne}, dans {\itshape Mémoires Soc}. {\itshape Hist.} Paris, 1927, (important) ; Marcel POÈTE, {\itshape Une vie de cité. Paris de sa naissance à nos jours}, t. I, Paris, 1924.
\end{itemize}

 \noindent Pour le reste, je renvoie aux indications déjà citées d’Halphen, auxquelles rien d’absolument capital n’est venu depuis lors s’ajouter. Voir aussi les problèmes d’ensemble de la vie urbaine, régulièrement étudiés par ESPINAS, dans les \href{http://gallica.bnf.fr/document?O=N010020}{\dotuline{{\itshape Annales d’histoire économique et sociale} [http://gallica.bnf.fr/document?O=N010020]}}.
}

\chaptercont
\section[{A. Situation au début du XIIIe siècle}]{A. Situation au début du XIII\textsuperscript{e} siècle}\phantomsection
\label{c08a}
\noindent  \phantomsection
\label{p71} Lorsque s’achève le règne de Philippe Auguste, le mouvement des libertés urbaines, qui a profondément modifié l’aspect de la société française, a déjà atteint ses principaux résultats. Voyons quel est l’état de fait et de droit qu’il a créé.\par
Les Progrès des échanges et, à leur suite, de la production industrielle, ont abouti à la formation, dans toute la France, de groupes humains qui, de bien des façons, forment un contraste très vif avec les populations ambiantes. Les hommes qui les composent ne tirent point leur subsistance, sauf à titre accessoire, du travail agricole ; ils ne font point non plus figure de rentiers du sol ; leur genre de vie, pas plus qu’il n’est celui du paysan, n’est celui des guerriers. Ce sont des marchands ou des artisans qui, directement ou non, vivent de vente et d’achat. Ajoutez, ce qui est très important, que ces hommes vivent serrés au coude à coude en agglomérations relativement importantes ; enfin que la plupart de ces agglomérations sont désormais fortifiées. Ces gens des villes sont très loin d’être tous pareils entre eux. Comme nous le savons et le reverrons, de violents antagonismes non seulement de factions ou de lignages, mais aussi de classes, les jettent les uns contre les autres. Sans parler des rivalités de ville à ville. Enfin, il y a des villes de type bien différent : entre Bruges et un petit bourg de campagne à demi-rural. Mais chaque groupe urbain, par rapport aux autorités qui  \phantomsection
\label{p72} prétendent dominer la ville, a néanmoins des intérêts communs ; et les groupes urbains, dans leur ensemble, ont assez de ressemblance entre eux et au total d’originalité collective pour que les institutions publiques et le droit communal ou privé que chacun d’eux a cherché à dresser présente, d’un bout à l’autre de la France et de l’Europe, à peu près les mêmes traits. Politiquement, économiquement et du point de vue de la structure sociale, il y a une société urbaine.\par
Or les villes, comme les campagnes, avaient été soumises dans des conditions obscures au régime seigneurial. Le seigneur de la ville, souvent, percevait les cens de terres ; toujours il possédait des droits de justice et de ban et des redevances sur l’activité commerciale elle-même. {\itshape Le} seigneur. Disons plutôt {\itshape les} seigneurs. Car les villes, en général, étaient morcelées entre plusieurs seigneuries. De sorte que le mouvement de libertés eut deux aspects concomitants :\par

\begin{enumerate}[itemsep=0pt,]
\item effort vers une unification administrative, qui répondit à l’unité géographique et sociale du groupe,
\item lutte contre le ou les seigneurs pour obtenir décharge des obligations les plus gênantes et surtout pour substituer à l’autorité seigneuriale, souvent oppressive et maladroite, celle de magistrats ou de corps pris parmi les bourgeois eux-mêmes et élus par eux.
\end{enumerate}

\noindent Les villes qui étaient parvenues au plus haut point d’autonomie portaient, au XIII\textsuperscript{e} siècle, le nom de {\itshape commune} qui avait désigné originellement l’association formée par des hommes se prêtant serment mutuel d’entraide : serment qui avait paru singulièrement révolutionnaire à une société qui ne connaissait que le serment de fidélité d’inférieurs à supérieurs. Le nom désignait aussi l’amitié des communiers. Sa valeur sentimentale (Richer). Mais valeur juridique : la jurisprudence a élaboré, au XIII\textsuperscript{e} siècle, tout un système juridique de la Commune, dont le fondement était que la personnalité morale n’est reconnue qu’aux groupes qui sont expressément — par le seigneur ou par le roi — reconnus comme des communes. En 1273, le Parlement refusa de reconnaître aux Lyonnais le droit de donner procuration à quelques-uns d’entre eux sans un sceau, parce qu’ils ne formaient pas une commune. Jugement analogue relativement à Orléans, en 1312. Les insignes habituels : sceau, beffroi. Lorsque la commune de Boulogne sera temporairement supprimée, en 1268, le roi fait détruire le beffroi de Boulogne ; les bourgeois soumis, la délimitation est arrêtée. En 1296, l’arrêt qui prive Laon de commune, la prive de sceau et de cloches.\par

\begin{enumerate}[itemsep=0pt,]
\item[]\listhead{Mais deux erreurs sont à prévenir :}
\item que le mot de commune n’ait été connu que dans le Nord (fausse antithèse : consulat, commune),
\item que toutes les communes eussent atteint le même degré d’autonomie. Tant s’en faut. Les variétés de ville à ville, commune ou non, sont extrêmes.
\end{enumerate}

\noindent Essayons néanmoins de prendre une idée, région par région, du degré atteint par le mouvement de liberté.\par
En Flandre, le mouvement s’était déroulé en somme pacifiquement, en accord, sous la dynastie d’Alsace, avec le pouvoir comtal. Les villes avaient reçu surtout des privilèges économiques et juridiques. Le pouvoir politique, la justice, en dehors de la justice proprement communale, restaient aux mains du comte. Les \emph{échevins}, qui formaient le tribunal de la ville et sa plus haute autorité administrative, s’ils étaient pris parmi les bourgeois, étaient nommés à vie par le comte. Mais, dès la fin du XIIe siècle, les bourgeoisies avaient obtenu à peu près partout qu’ils fussent annuels, ce qui,  \phantomsection
\label{p73} pratiquement, avait pour résultat de donner une part de plus en plus grande au choix par les bourgeois : élection ou, plus souvent, cooptation. Cette situation de fait fut, en règle générale, reconnue en droit lors de la crise du comté qui suivit Bouvines.\par
La Picardie a été le domaine d’élection des communes, souvent créées par la violence, en opposition avec le pouvoir seigneurial, soit laïque (Saint-Quentin), soit épiscopat (Noyon, Beauvais).\par
Dans les deux grandes principautés territoriales d’entre Somme et Loire, la Champagne et la Normandie, le mouvement urbain s’était opéré comme en Flandre, d’accord avec le pouvoir comtal ou ducal. Par conséquent, comme en Flandre, avait eu pour résultat l’organisation de communes sans pouvoirs politiques. En Normandie, notamment, le modèle avait été donné par les Établissements de Rouen, qui réservent au duc la nomination du maire sur une liste de trois personnes présentée par les pairs héréditaires, qui sont les principaux bourgeois (certainement viagers en fait ou en droit) et leur remet également la plus grande part du pouvoir judiciaire. Lorsque les Plantagenets, ducs de Normandie, se furent rendus maîtres de toutes les provinces de l’Ouest, ils introduisirent les Établissements de Rouen, de Bayonne à Poitiers.\par
Dans le Languedoc et le Centre, l’autonomie urbaine, au moins dans les grandes villes, comme Toulouse, Montpellier ou Nîmes, a atteint un degré élevé. La commune de Toulouse, notamment, profita des troubles de la croisade, où elle joua un rôle considérable, pour consolider fortement son indépendance. Les consuls, annuels, se recrutent par cooptation. En 1248, le comte Raimond VII a dû promettre de ne point se mêler de l’élection.\par
Reste enfin le \emph{domaine royal}, mot difficile à définir... Là, la politique royale a été très claire, témoignant de beaucoup plus d’esprit de suite qu’on n’a parfois voulu le reconnaître. Les rois, Philippe Auguste notamment, ont admis ou même favorisé la formation de communes véritables sur les zones frontières, où les bourgeoisies de ces villes, d’ailleurs petites, devaient être avantagées, afin que l’on pût compter sur elles pour garder leurs enceintes : telles Meulan ou Mantes sur la frontière normande ; les communes du Valois ou celles du Nord-Est. Par ailleurs, Philippe Auguste n’a pas, en principe, touché aux institutions urbaines des pays qu’il avait soumis ou dont, comme pour le Vermandois, il avait hérité. Par contre, dans la partie ancienne, sa partie vitale, les grandes villes avaient été soigneusement maintenues sous l’autorité du roi. Sous Louis VII, une tentative de commune à Orléans avait été sévèrement réprimée. Sans doute, comme nous le verrons pour Paris, la bourgeoisie recevait en pratique, dans l’administration des villes, une part souvent assez large. Mais au roi restaient le contrôle, les principaux pouvoirs judiciaires, la possession de l’enceinte, le pouvoir d’ordonner.
\section[{B. Les Autorités urbaines}]{B. Les Autorités urbaines}\phantomsection
\label{c08b}
\noindent Mais lorsque la ville, dans une certaine mesure, se gouvernait elle-même — et il était assez rare qu’elle n’eût pas au moins une part à son administration — qui gouvernait et comment ? Le fait fondamental est que, du moins dans les centres importants, le mouvement d’autonomie avait été presque partout dirigé par les hauts bourgeois, plus ou moins suivis — pas toujours très aisément —  \phantomsection
\label{p74} par le menu peuple ; et c’est cette classe qui en avait recueilli les bénéfices.\par
De qui se composaient ces \emph{premiers de la ville} ? Il faut ici distinguer soigneusement deux grandes régions. La plus grande partie de la France d’une part ; le Toulousain et le Bas-Languedoc de l’autre. Dans la première, certains groupes appartenant à la classe chevaleresque avaient, au début, participé, dans quelques villes (comme Reims), au mouvement de liberté. Mais ils ont été rapidement éliminés. Le noble, dans ces contrées, est un rural. Salimbene : \emph{« In Francia solummodo burgenses in civitatibus habitant, milites vero et nobiles domine morantur in villis et possessionibus suis \footnote{ SALIMBENE, {\itshape Cronica}, éd. Holder-Egger, {\itshape Monum. Germ., Scriptores}, t. XXXII, p. 222.} »}. À la tête de la ville, sont des bourgeois non adoubés, non dignes de l’être. Il en fut autrement dans le Toulousain et le Languedoc, comme dans la Provence voisine. Là, les chevaliers jouaient encore, au XIII\textsuperscript{e} siècle, un rôle important. Ce sont, par leur origine, des groupes de vassaux militaires chargés de garder les châteaux ou les enceintes. Ils habitaient souvent groupés : à Nîmes, dans les arènes (\emph{chevaliers des Arènes}), d’où Louis VIII les fit déménager ; à Beaucaire, au pied du château ; ailleurs, dans des rues des Nobles. Souvent, ils avaient une part au Conseil urbain, au consulat. À Nîmes, durant tout le XIII\textsuperscript{e} siècle, il y eut quatre consuls bourgeois et quatre chevaliers. De même, dans un grand nombre de petites villes du Toulousain. Cependant, dans les grandes villes de commerce — comme Toulouse et Marseille — l’élément bourgeois n’avait pas accordé aux chevaliers le droit au moins à une représentation spéciale, sans les éliminer toutefois de la vie urbaine (notamment à Toulouse : {\itshape milites Tolosae).}\par
Ces cas, exceptionnels dans l’ensemble de la France, une fois mis à part, les hautes classes sont essentiellement un patriciat bourgeois (le mot de patriciat n’est pas du temps, mais il est commode). Naturellement, cette classe se présente avec des contours très nets, surtout dans les plus grandes villes. Nous connaissons déjà les sources de leur fortune : commerce lointain, organisation capitaliste qui, par l’achat des matières premières et la vente des produits, se superpose à la petite entreprise des métiers. Ce qu’il importe pourtant de se représenter, c’est que, à peu près partout au cours du XIII\textsuperscript{e} siècle, le mouvement naturel à cette classe a été — comme de toute classe capitaliste — de tendre à une sorte de condition de rentier : {\itshape otiosi}, disent les textes pour parler d’eux. Mouvement favorisé par l’avènement du commerce stable. Les hauts bourgeois n’abandonnaient pas tout le commerce. Mais beaucoup s’attachent de plus en plus à vivre de revenus tirés soit des prêts, notamment aux grands seigneurs ou aux États, soit des loyers des maisons urbaines, soit des seigneuries acquises à la campagne. Leur genre de vie se rapproche de celui de la classe chevaleresque. Ne forment-ils pas du reste, dans les contingents militaires des villes, la cavalerie ? Beaucoup vont de tournois en tournois. Des incidents survenus à une joute, donnée, le 1\textsuperscript{er} mai 1284, par les bourgeois de Douai, amenèrent une véritable guerre urbaine entre Douai et Lille. La puissance de l’oligarchie fut surtout dans les grandes villes riches où les différences de classe sont fortes. Les tours : Louis VIII, en 1226, à Avignon, en détruit, disait-on, trois cents. Le sentiment  \phantomsection
\label{p75} de classe est très développé malgré la littérature chevaleresque qui, volontiers, identifie bourgeois et vilain. Voir le {\itshape Jeu de Robin et Marion.} Voir le mot de saint Thomas (et les {\itshape Ongles bleus}). Culture bourgeoise : Arras (confrérie de la Sainte Chandelle remise à deux jongleurs par Notre-Dame. Adam de la Halle : vers 1135 - vers 1185).\par
Comment, en fait, ces grands bourgeois dominaient-ils la ville, et quels obstacles rencontrait leur domination ? Pour le comprendre, il faut prendre une image concrète des libertés urbaines.\par
En principe, la communauté se compose de tous les bourgeois. Par quoi, d’ailleurs, il ne faut pas entendre tous les habitants, mais ceux qui sont membres de l’association jurée. Sont exclus, en règle générale, les nobles, plus ou moins explicitement ; les clercs dont l’exclusion est érigée en principe par le Parlement de Paris ; enfin toute une catégorie de \emph{manants} qui n’ont pas accès à la bourgeoisie. Les conditions d’accès à celle-ci sont variables suivant les cas : mais, à peu près partout, il faut posséder ou une maison ou une certaine somme d’argent, ou tous les deux. En outre, certaines catégories infamantes : lépreux, et de plus en plus, au XIII\textsuperscript{e} siècle — conformément à l’accentuation générale des classes — les serfs.\par
Le nombre des bourgeois, ainsi conçu, est déjà relativement considérable. Leur réunion forme l’assemblée générale de la commune, qu’on appelle, dans le Midi, \emph{Parlement}. Mais cette assemblée se réunit en général rarement et seulement, à l’ordinaire. pour entériner des décisions prises d’avance. Là même où elle se réunit (le plus fréquemment dans le Midi), on s’arrange pour la limiter. Par exemple à Toulouse, après 1247, on ne la réunit plus que dans la Maison commune, trop petite certainement pour contenir tout le monde.\par
Le gouvernement de la ville appartient véritablement aux magistrats. Ceux-ci sont de types variables. Et il est impossible d’entrer ici dans le détail institutionnel. Quelques exemples suffiront.\par

\begin{enumerate}[itemsep=0pt,]
\item[]\listhead{Dans le Nord, la règle générale est qu’il y a :}
\item un conseil,
\item des magistrats pourvus d’une sorte de pouvoir exécutif.
\end{enumerate}

\noindent Dans certaines régions, le conseil a pris le nom d’échevins. Ce sont d’anciens magistrats carolingiens.\par

\begin{itemize}[itemsep=0pt,]
\item[]\listhead{Leur destin dans les villes a été double :}
\item ou bien, ils ont subsisté comme fonctionnaires royaux, indépendamment des magistratures urbaines (type picard : exemple Saint-Quentin) ;
\item ou bien, recrutés parmi les bourgeois, ils sont devenus des magistrats urbains (un autre échevinage pour le plat pays subsistant à part) : type flamand.
\end{itemize}

\noindent Là où le conseil ne s’est pas fondu avec l’échevinage, il porte un autre nom : jurés par exemple, ailleurs pairs comme à Beauvais ou à Rouen. Parfois il y a plusieurs conseils.\par
Le magistrat exécutif est, dans le Nord, le Maire ; parfois au nombre de deux (comme à Beauvais). Dans la Flandre thioise, le bourgmestre ; parfois aussi collégiaux (deux à Bruges).\par
Dans le Languedoc, le plus souvent, le gouvernement est purement collégial. À la tête de la ville, se trouve un collège de consuls. Par exemple à Toulouse, depuis 1180, ils sont vingt-quatre, douze pour la cité et douze pour le bourg abbatial de Saint-Sernin. On les appelle aussi, à Toulouse, capitouls (parce qu’ils formaient un chapitre, {\itshape capitulum}) et on tirera de là rapidement une étymologie romaine. Il arrive d’ailleurs, comme à Périgueux, que les  \phantomsection
\label{p76} consuls aient, à leur tête un maire. Le \emph{consulat} nordique est presque purement verbal. L’originalité véritable du Midi sera les consulats ruraux : un peu plus tard.\par
En outre, au cours du XIII\textsuperscript{e} siècle, les villes recourent de plus en plus aux fonctionnaires salariés.\par
Comment sont désignés ces magistrats ? Constitutions à l’origine mal précisées et, lorsqu’elles le sont, des plus compliquées. Mais presque tous, à des degrés variables, font intervenir la cooptation plutôt que l’élection. À Rouen, les pairs sont héréditaires ; ils proposent le maire (sur une liste de trois) au choix du duc ou roi. À Douai, système mitigé, mais encore très aristocrate. Les échevins sortants choisissent quatre bourgeois qui choisiront quatre échevins ; ceux-ci en choisiront quatre ; les huit, quatre autres encore ; et les douze, enfin, les quatre échevins réservés à la partie de la ville sise sur la rive gauche de la Scarpe (Douaxeul). À Toulouse, depuis 1222, les consuls de l’année sortante désignent leurs successeurs ; ce système ne sera supprimé à quelque moment que par la nomination par le comte. À Montpellier, au début du siècle, cooptation mitigée : les douze consuls sortants se réunissaient à sept représentants des corps de métier pour nommer leurs successeurs. Il s’ajoutait parfois des prescriptions proprement censitaires. À Bruges (1241), il faut, pour faire partie de l’échevinage : a) ne pas être manouvrier, b) faire partie de la Hanse de Londres. À Saint-Omer (1306) : a) avoir cinq livres tournois vaillant, b) ne pas ouvrer de son corps ni exercer de métier durant la charge. Ce système aboutissait parfois à de véritables roulements entre les mêmes individus. Déjà à Toulouse, les consuls de l’année précédente continuent à former un corps et participent à certains jugements. Ce régime atteignait à Gand sa forme la plus parfaite. Depuis 1222, la ville est gouvernée par le Conseil des Trente-neuf, qui se décompose en trois groupes de treize pourvus chacun d’attributions propres : 13 échevins de l’année en cours, 13 de l’année à venir, 13 de l’année précédente. Alternance perpétuelle, les vides étant comblés par cooptation.\par
Quels sont les pouvoirs de ces autorités ainsi recrutées ? Très variables, on le sait. La justice civile presque toujours. La justice criminelle plus ou moins partagée avec le seigneur. La paix. Le ban, qui se heurte d’ailleurs au droit suprême du seigneur. Police économique, notamment ravitaillement. Le Parlement, qui les conserve, interdisait au maire de faire proclamer dans les rues de la ville l’ordre à chacun de mettre, par le temps sec, un seau d’eau devant sa porte. Les institutions d’entraide. La milice urbaine ; parfois les remparts ; les sceaux militaires. La ville s’impose elle-même. Enfin, la banlieue (corvée de murs, fortifications, métiers ; biens ruraux des bourgeois ; allant, dans le Midi, jusqu’à élever des bastides).\par
Or, rappelons-nous que ces pouvoirs étaient entre les mains surtout des hauts bourgeois. Nous savons, ou nous devinons, quels instruments pouvaient être notamment, entre leurs mains, le pouvoir de juger et le pouvoir d’imposer. Beaumanoir \footnote{{\itshape Coutumes du Beauvaisis}, éd. A. Salmon, Coll. Picard, 2 vol., Paris, 1899-1900 : t. II, par. 1525, p. 270-271.} : \emph{« Mout de contens muevent es bonnes viles de commune pour leur tailles, car il avient souvent que li riche qui sont gouverneur des besoignes de la vile metent a meins qu’il ne doivent aus et leur parens, et  \phantomsection
\label{p77} deportent les autres riches hommes pour qu’il soient déportés, et ainsi queurent tuit li fres seur la communeté des povres. Et par ce ont esté maint mal fet, pour ce que li povre ne se vouloient soufrir ne il ne savoient pas bien la droite voie de pourchacier leur droit fors que par aus courre sus. Si en ont li aucun esté ocis... »} Ajoutez que l’administration était médiocre : endettement (normal alors pour les États, faute de facilité de trésorerie ; charges royales ; et mauvaise gestion : en 1251, Rouen dont le revenu annuel est d’environ 2 300 livres tournois, en doit près de 7 000. À Noyon, par exemple, recours à l’emprunt plutôt qu’à l’impôt. La jeunesse patricienne se permettait beaucoup d’excès de tout genre, du tapage nocturne à des violences sur les hommes et les femmes. Ajoutez les conflits économiques. Pensez enfin qu’à ces luttes de classe s’ajoutaient les querelles de factions ou de familles. À Bordeaux, en 1243, Henri III prescrit : \emph{« Puisqu’il y a dans cette ville deux factions rivales, les jurats seront pris chaque année dans chacune en nombre égal. »} Les deux factions avaient à leur tête des familles de gros marchands. Et bien que la ville fût, selon la vieille définition de Galbert de Bruges, \emph{locus pacificus}, on comprend que la paix n’y régnât point. Un poète arrageois :\par


\begin{verse}
« Arras ! Arras ! vile de plait\\
Et de haine et de destroit. »\\
\end{verse}

\noindent Voir l’opinion un peu suspecte de Beaumanoir : \emph{« Nous avons veu moult de debas es bonnes viles des uns contre les autres, si comme des povres contre les riches ou des riches meismes les uns contre les autres : si comme quant il ne se pueent acorder a fere maieurs ou procureeurs ou avocas, ou si comme quant li un metent sus as autres qu’il n’ont pas fet des reçoites de la vile ce qu’il doivent, ou qu’il ont contre de trop grans mises, ou si comme quant les besoignes de la vile vint mauvesement pour content ou mautolens qui muevent l’un lignage contre l’autre... \footnote{ BEAUMANOIR, {\itshape id}., paragr. 1520, p. 267-268.} »}.
\section[{C. Les luttes des classes urbaines}]{C. Les luttes des classes urbaines}\phantomsection
\label{c08c}
\noindent Durant tout le XIII\textsuperscript{e} siècle, l’effort contre l’oligarchie des classes moyennes de la population urbaine, groupées en général dans les métiers, s’est donc poursuivi. Sous des formes et avec des résultats différents selon les villes. Parfois, elle paraît avoir obtenu des résultats, pacifiquement, mais nous ne savons pas tout. Ailleurs, les violences sont entamées.\par
À Toulouse, depuis le rétablissement de l’autorité comtale, après la croisade sous Raimond VII, le comte paraît avoir joué des pauvres contre les riches. Il fut vaincu. Mais l’acte de 1248, par lequel il s’engage à ne pas intervenir dans la nomination des consuls, prévoit que ceux-ci seront pris désormais, mi-partie parmi les {\itshape majores}, mi-partie parmi les {\itshape medii.} À Nîmes, le règlement de 1272 adjoint aux quatre consuls chevaliers et aux quatre consuls bourgeois un conseil où, selon le système montpelliérain, figurent neuf représentants des \emph{Echelles}, c’est-à-dire des corps de métiers. Comme ce conseil prend part, avec les consuls sortants, à l’élection des consuls nouveaux, des gens de métier pénètrent dans le consulat. À Rouen, en 1321, les pairs deviennent des magistrats élus ( ? ) pour trois ans.\par
 \phantomsection
\label{p78} Ailleurs — surtout dans le Nord — l’évolution est beaucoup plus violente. À Beauvais, en 1233, une émeute sanglante éclata entre les deux parties : d’une part, les {\itshape populares}, de l’autre les {\itshape majores} que, du nom de leur principal groupe, on appelle aussi les {\itshape campsores}, les changeurs. Rien ne paraît alors changé à la constitution. Mais en 1281, le changement vient. Il y avait avant cette date — nous ne savons depuis quand — à Beauvais, vingt-deux métiers ; l’un, celui des changeurs, fournissait six pairs et un maire ; les autres vingt et un, les six autres et l’autre maire. Le \emph{commun} s’est plaint à la cour du roi. Celle-ci, si hésitante cependant, à l’ordinaire, à modifier une coutume, casse celle-là. Les deux maires et les dix pairs seront pris désormais dans tous les métiers.\par
En Flandre, enfin, la lutte prit une acuité particulière. Lors de l’avènement du faux Baudouin, en 1225, soulèvement à Valenciennes des gens de métier contre les patriciens. À partir du milieu du XIII\textsuperscript{e} siècle, les bans communaux sont pleins de défenses, sous lourdes peines, aux tisserands et foulons de porter des armes, voire de sortir munis des lourds outils de leur profession, de se rassembler à plus de sept. Il y a des grèves et un essai de résistance organisée du patriciat. En 1274, une émeute des tisserands et foulons de Gand contre l’échevinage avait échoué ; la plèbe vaincue s’enfuit vers le Brabant ; les échevins écrivirent aux villes brabançonnes pour les prier de ne les point accueillir. En 1280, enfin, une émeute éclata simultanément à Bruges, Ypres, Douai, et, en dehors du comté, à Tournai. Le comte Gui de Dampierre en profita pour prendre des mesures rigoureuses contre l’indépendance des villes. Le patriciat se tourna alors vers le roi de France. Les patriciens seront, en Flandre, les meilleurs soutiens de la politique conquérante de Philippe le Bel ({\itshape Leliaerts}). L’artisanat forma l’armée des {\itshape Claveswaerts}. C’est avec l’aide des premiers que, de 1297 à 1300, la Flandre fut soumise. La révolte de 1302, à Bruges, qui aboutit à la bataille de Courtrai, fut l’œuvre de l’artisanat.\par
Ainsi déjà élément politique. Nous allons voir comment — Beaumanoir nous l’a déjà fait prévoir — les mêmes mouvements favorisèrent en France l’autorité royale. Remarquons bien cependant que, dans l’ensemble de la France où les artisans étaient prolétarisés, il y eut en réalité trois classes :\par

\begin{enumerate}[itemsep=0pt,]
\item hauts bourgeois quasi rentiers,
\item artisans,
\item menu peuple.
\end{enumerate}

\noindent Ce dernier n’a figuré que comme armée d’émeute. Lorsque les hauts bourgeois ont été vaincus, ce fut au profit des {\itshape medii.}
\section[{D. La politique royale}]{D. La politique royale}\phantomsection
\label{c08d}
\noindent En un sens, elle est favorable aux villes. Saint Louis, dans ses \emph{Enseignements} au futur Philippe le Hardi : \emph{« Meismement les bones villes et les coustumes de ton royaume garde-en l’estat et en la franchise ou tes devanciers les ont gardées ; et se il y a aucune chose à amender, si l’amende et adresce et les tien en faveur et en amour ; car par la force et par les richesces des grosses villes, douteront les privez et les estranges de mespendre vers toy, especialment tes pers et tes barons. \footnote{JOINVILLE, éd. de Wailly, p. 494.} »}\par
 \phantomsection
\label{p79} Pour comprendre, il faut se rendre compte de la théorie de la commune développée depuis la fin du XII\textsuperscript{e} siècle par l’entourage du roi. Elle a été exprimée dans un passage des Gestes des évêques d’Auxerre, rédigé à une date incertaine, dans la première moitié du XIII\textsuperscript{e} siècle. L’évêque Guillaume de Toucy, rapporte-t-on, s’était opposé à la constitution d’une commune à Auxerre, favorisée par le comte : \emph{« malevolentiain illius piissimi Ludovici regis incurrit, qui ei improperebat quod Autissiodorensem civitatem ipsi et haeredibus suis auferre conabatur, reputans civitates omnes suas esse, in quibus communie essent \footnote{{\itshape Rec. des hist. de la France}, t. XII, p. 304.} »}.\par
Ce n’était peut-être pas là l’idée de Louis VII. C’était certainement une idée répandue autour des rois au temps de l’auteur. On la rattachera au privilège du roi de modifier le droit et de créer des corps nouveaux pourvus de personnalité collective. Beaumanoir : \emph{« De nouvel, nus ne puet fere vile de commune ou royaume de France sans l’assentement du roi, fors que li rois, parce que toutes nouveletés sunt defendues »}. En 1317, aux gens d’Ouveillan, dans le Languedoc, qui affirment que leur seigneur leur a naguère donné le consulat, le procureur royal répond que cela ne peut être : au roi seul, appartient de faire de telles concessions.\par
Donc, toute ville de commune forme, où qu’elle soit, comme un corps étranger, où le roi peut intervenir. C’est ce qui explique par exemple les interventions du roi, sous Philippe le Bel, relatives à la constitution de Gand. Ajoutez que beaucoup des communes non royales avaient pour seigneur des églises royales. Lors des troubles de Beauvais, en 1233, le roi commença par imposer à la ville un maire étranger à la communauté urbaine. Il y eut contre ce maire un soulèvement. Le roi entra à Beauvais avec une armée et refusa de céder aux protestations de l’évêque.\par
Mais plus précisément, le roi pensait pouvoir tirer des communes deux choses — souvenons-nous des mots de saint Louis \emph{force et richesse} — aide militaire et impôts. Sur le premier point, nous verrons plus tard. Indiquons simplement : 1) erreur de croire les milices incapables d’effort militaire ; 2) tendance générale au remplacement. Mais il reste la grosse question de la défense du rempart.\par
L’impôt eut donc de plus en plus grande importance. Là aussi, renvoyons. Mais surtout, avant l’établissement d’un impôt plus général sous Philippe le Bel, sous saint Louis, vraies vaches à lait de la fiscalité.\par

\begin{enumerate}[itemsep=0pt,]
\item[]\listhead{Pour tout cela, cependant, il fallait des communes :}
\item bien administrées ;
\item où l’ordre règne ;
\item obéissantes.
\end{enumerate}

\noindent Or, nous savons que le premier et le second points n’étaient pas. Et quant à la soumission, elle ne pouvait être obtenue qu’en bridant l’autonomie. Celle-ci répugnait notamment au fonctionnarisme de plus en plus installé et conscient.\par
Pratiquement, les interventions ont pris des formes diverses. Efforts législatifs pour introduire plus de régularité. Une ordonnance de saint Louis de 1256, relative aux communes de Normandie : —  \phantomsection
\label{p80} fixe un jour commun pour la présentation au roi des candidats à la mairie (principe des Établissements de Rouen) ; — organise la présentation des comptes au roi ; — interdit tout prêt ou don sans l’assentiment du roi ; — prend diverses mesures de bonne administration financière.\par
Un peu plus tard, en 1262, les mêmes prescriptions sont étendues à toutes les autres communes du royaume (mais ici, quoi qu’on en ait dit, le droit d’élection, du moins, reste intact \emph{« que tuit li mayeur de France soient fait lendemain de la feste saint Simon et saint Jude »}).\par
En fait, les gens du roi exigèrent ou s’efforcèrent d’exiger la présentation des comptes. Pour le reste, les ordonnances furent mal observées. L’intervention se traduisit bien plus tôt par des mesures locales.\par
Intervention dans la lutte des classes. Non pas en faveur des petits précisément. Les émeutes ont été en général durement répri­mées. À Beauvais, c’est l’évêque qui, selon un jeu classique, s’appuyait en 1233 sur les {\itshape minores}. Le roi et son maire favorisent l’autre parti. Cf. la Flandre. Mais pour maintenir l’ordre et suppri­mer les abus les plus flagrants. Nous avons vu le cas de Beauvais, en 1281. À Périgueux, en 1309, le Parlement dut aussi intervenir \footnote{{\itshape OLIM}, t. III, 1, p. 366.}. Il y avait eu des troubles, dont l’origine était dans un faussement du mode ancien d’élection. Le système régulier était le suivant : le jour de leur sortie de charge, le maire et les consuls, en présence de l’assemblée des bourgeois, choisissaient quatre électeurs ; ces quatre en élisaient huit ; ceux-ci à leur tour choisissaient maire et consuls. Or, depuis quelques années, il y avait collusion, le maire et les consuls faisaient auparavant choix des quatre, s’entendaient alors avec ceux-ci pour le choix des huit ; avec ceux-ci, enfin, pour le choix des consuls. Dans les troubles, il y avait eu violences exercées contre les agents du roi. Le Parlement porta des amendes et indemnités, décida que seraient arrachées et brûlées les portes de la maison du consulat, que les consuls du moment avaient refusé d’ouvrir au bayle royal ; enfin, rétablit le mode ancien d’élection. En outre, la juridiction partiellement exercée par la municipalité était mise dans la main du roi.\par
Interventions financières. Le gouvernement royal organisa, en 1291, la faillite depuis longtemps menaçante de la commune de Noyon ; bien entendu, au prix de lourdes confiscations : des biens des magistrats, des biens hors la ville de tous les bourgeois.\par
Enfin, suppression des communes. Temporaires : Rouen en 1292 ; Laon en 1296. Mesures fiscales : Rouen, 12 000 livres parisis. À Senlis, en 1320, à la suite d’une catastrophe financière, née en partie d’amendes royales et des luttes habituelles entre riches et menu peuple.\par
Plus fréquemment, restriction des pouvoirs : par exemple à Toulouse (où déjà efforts d’Alfonse de Poitiers). En 1283, les douze consuls sortants devaient désormais ne plus coopter les suivants,  \phantomsection
\label{p81} mais présenter au roi trente-six candidats sur lesquels il choisira les douze. Pressions électorales. L’institution, au début du XIV\textsuperscript{e} siècle, des capitaines de villes.\par
Apparition de la notion de tutelle administrative : Beaumanoir \emph{« Grand mestier est aucune fois que l’on sequeure les viles de comune en aucun cas come l’en feroit l’enfant sousaagié »}.\par
Un type de ville royale : Paris \footnote{ Compléter la bibliographie par : J. VIARD, {\itshape Paris sous Philippe le Bel}, dans {\itshape Bulletin de la Soc. Hist. de Paris}, 1934, p. 56-71. Ch. V. LANGLOIS, {\itshape Pierre Gencien}, dans {\itshape Histoire littéraire de la France}, t. XXXV, 1921, p. 284-301.}. La bourgeoisie. Passage des villes au pouvoir royal (partiel). Encore vif esprit de classe, en antagonisme avec la noblesse.\par

\begin{center}
\noindent \centerline{}
\end{center}

\chapterclose


\chapteropen
\chapter[{9. L’économie française }]{\textsc{9. }L’économie française \protect\footnotemark }\phantomsection
\label{c09}\renewcommand{\leftmark}{\textsc{9. }L’économie française }

\footnotetext{ 
\begin{itemize}[itemsep=0pt,]
\item[]\listhead{Bibliographie}
\item Fondamental : H. PIRENNE, \emph{Le mouvement économique et social} dans G. GLOTZ, \emph{Histoire générale, Histoire du Moyen âge}, t. VIII, Paris, 1933. J’en recommande d’autant plus la lecture que je ne le reproduirai pas. Comme manuel précédent, je n’ai guère à citer que le très précieux recueil de renseignements de W. KULISHER, \emph{Allegemeine Wirtschaftsgeschichte des Mittelalters und der Neuzeit}, t. I, Munich, 1928.
\item Histoire économique de la France : Henri SÉE, \emph{Esquisse d’une histoire économique et sociale de la France depuis les origines jusqu’à la guerre mondiale}, Paris, 1929, et du même \emph{Französische Wirtschaftsgeschichte}, t. I, Iena, 1930 (plus développé).
\item Histoire rurale de la France : Marc BLOCH, \emph{Les caractères originaux de l’histoire rurale française}, Oslo et Paris, 1931, qui ne dispense pas tout à fait de recourir à l’ouvrage de H. SÉE, \emph{Les classes rurales et le régime domanial en France au Moyen âge}, Paris, 1901.
\item Il n’y a pas de bons ouvrages récents sur le commerce ou l’industrie en France. On peut encore se reporter avec profit à H. PIGEONNEAU, {\itshape Histoire du commerce de la France}, Paris, 1885-1889, 2 vol. (sur lequel LEVASSEUR, \emph{Histoire du commerce de la France}, Paris, 1911, ne représente qu’un faible progrès). — E. LEVASSEUR, \emph{Histoire des classes ouvrières et de l’industrie en France avant 1789}, t. I, Paris, 1901.
\end{itemize}

 }

\chaptercont
\section[{A. Caractères généraux de l’économie européenne et principalement de l’économie française}]{A. Caractères généraux de l’économie européenne et principalement de l’économie française}\phantomsection
\label{c09a}
 \phantomsection
\label{p83}
\begin{enumerate}[itemsep=0pt,]
\item Immense prépondérance de la population agricole. Les caractères économiques de l’agriculture : peu de monoculture ; l’usine à blé.
\item Mais rôle des échanges. La grande révolution date du XII\textsuperscript{e} siècle (ou de la fin du XI\textsuperscript{e}). C’est le renversement de la balance commerciale vis-à-vis de l’Orient, en même temps que l’annexion du Nord au rayon d’action économique de l’Occident. Le rôle des draps.
\item Il en résulte que le grand commerce jouit d’une sorte de prépondérance économique. Uni à la finance : le grand commerçant étant en même temps financier. Ce commerçant, en outre, cesse de plus en plus d’être un colporteur. Naissance du commerce stable (avec ses techniques d’affaire : commission, lettre de change sur laquelle nous reviendrons). La production industrielle est subordonnée. Il n’a pas été indifférent pour la civilisation européenne que la première grande bourgeoisie ait été faite de marchands, non d’industriels : notamment retard de la technique.
\item Cette transformation a naturellement entraîné des modifications dans la table des valeurs morales. Hostilité ancienne du Christianisme au gain. Jean Chrysostome, cité par le décret de Gratien.  \phantomsection
\label{p84} Ces problèmes continuent à préoccuper la conscience du XIII\textsuperscript{e}-XIV\textsuperscript{e} siècle (et bien plus tard) : notamment sous la forme du problème de l’usure, que nous retrouverons. Mais un effort d’adaptation se poursuit, de modération, très sensible notamment chez Thomas d’Aquin (avec influence de l’aristotélisme) \footnote{Thomas d’Aquin, né en 1225 ou 1227, mort en 1274. Plusieurs fois à Paris, comme étudiant, puis comme professeur. La Somme a été rédigée entre 1265 et 1273.}.
\end{enumerate}

\noindent Voici les textes. \emph{« En chassant du Temple vendeurs et acheteurs, le Seigneur a signifié que le marchand, presque jamais ou même jamais — {\itshape vix aut nunquam —} ne put plaire à Dieu. C’est pourquoi aucun chrétien ne doit être marchand ; s’il veut l’être, qu’il soit chassé de l’Église de Dieu »}.\par
Saint Thomas copiant dans la première phrase saint Augustin. Les vices du négociant avide \emph{« sont de l’homme, non du métier, qui peut fort bien s’exercer sans eux »}. \emph{« Une cité parfaite avec des marchands modérément »}. Notion de modération, se traduisant par des idées comme celles-ci :\par

\begin{listalpha}[itemsep=0pt,]
\item légitimité en soi du bénéfice commercial ;
\item en stricte justice, le vendeur de blé, s’il sait que des arrivages vont faire baisser les cours, n’est pas tenu d’en avertir les acheteurs ;
\item interdiction de la vente à crédit.
\end{listalpha}

\section[{B. La place de la France, dans les grands courants d’échanges européens}]{B. La place de la France \\
dans les grands courants d’échanges européens}\phantomsection
\label{c09b}

\labelblock{1° Il y a deux grands foyers d’échanges. Le royaume participe, mais inégalement, à l’un et à l’autre :}

\noindent \labelchar{a)} D’abord {\itshape les ports méditerranéens}, tête de lignes du commerce avec l’Orient, musulman ou byzantin, et l’Afrique. Là s’embarquent les draps et étoffes de lin, des armes, des bois et des blés pour le Maghreb. Le débarquement des produits exotiques qui sont surtout : des épices, des produits divers, par exemple l’aloès. Parfois si lointains que Joinville croyait le gingembre, la rubarbe, l’aloès et la cannelle, abondants en Égypte, venus par le Nil du Paradis terrestre, par chute des arbres abattus par le vent. Ou des produits alimentaires de luxe : le sucre, qui provient pour partie des plantations génoises ou vénitiennes de Chypre ou de Syrie, ainsi que de l’Égypte. Des produits tinctoriaux ou servant à la teinture : indigo, bois de brésil, alun. Des matières premières de l’industrie textile, parfois élaborées en étoffe : coton, soie. Des esclaves, surtout concentrés dans les pays méditerranéens eux-mêmes.\par
Les principaux de ces ports, Gênes, Pise, Venise, sont hors de France. Marseille, hors du royaume également, tombe sous l’influence d’un prince français, Charles d’Anjou, comte de Provence depuis 1246 et qui établit définitivement sa domination à Marseille en 1257. Le Bas-Languedoc était du royaume. Depuis 1229, Narbonne est au roi. Montpellier forme une seigneurie sous la maison d’Aragon et l’évêque de Maguelonne dont, en 1293, le roi achète les droits. Plus près du delta, saint Louis crée Aigues-Mortes, au profit duquel il s’efforce en vain d’établir un véritable monopole commercial.\par
\bigbreak
\noindent \labelchar{b)} {\itshape Les ports de la mer du Nord}, notamment les ports flamands, qui sont du royaume (voir Pirenne, t. I). Le plus important de  \phantomsection
\label{p85} beaucoup est Bruges, sur le Zwin, avec ses avant-ports Damme et l’Ecluse. Une grande transformation s’opéra là au cours du XIII\textsuperscript{e} siècle. À la différence des ports méditerranéens, le commerce de Bruges devint purement passif ; la batellerie flamande ne joua plus de rôle important. Ce sont des vaisseaux étrangers qui apportent à Bruges les marchandises du Nord-Ouest et du Nord et en emportent les produits, soit aux Pays-Bas, soit dans la Méditerranée, pour lesquels Bruges fait figure dans l’Europe septentrionale de centre de diffusion.\par
Ces produits d’exportation sont, naturellement, essentiellement les draps et les produits méditerranéens.\par
Les produits d’importation sont : les laines anglaises, matière première pour l’industrie des Pays-Bas et même plus loin jusqu’en Italie (surtout vers 1300) ; du Nord, les blés, les bois, les harengs fumés, les fourrures, des métaux.\par

\labelblock{2° Où se faisait la jonction des deux grands foyers ?}

\noindent Problème très grave, dont nous allons voir l’importance fondamentale pour la société française.\par
\bigbreak
\noindent \labelchar{a)} Durant une première période, qui s’étend en gros jusqu’à la fin du XIII\textsuperscript{e} siècle, la jonction s’est faite à peu près uniquement par voie de terre. C’est dire que la voie de liaison passait par la France (je ne veux pas dire par le domaine royal). Les lourds chariots des caravanes ou les mules remontaient la vallée du Rhône — s’ils venaient du littoral languedocien ou du Bas-Languedoc ; ou bien les marchands rejoignaient la même route Rhône-Saône, sur des chariots ou sur des mules en franchissant un des cols des Alpes ; les plus fréquentés étaient le Genèvre, le Cenis et le Grand Saint-Bernard. Puis, par les plaines de Champagne, ils gagnaient celles du Nord. Une des routes les plus fréquentées passait par Bapaume, dont le péage qui, depuis Philippe Auguste, appartenait, avec l’Artois, au roi de France, rapportait au trésor royal de beaux revenus. Parfois, on passait de la voie de terre à la rivière.\par
Mais qui accompagnait les marchandises ? On pouvait concevoir que le marchand du lieu de production ou d’arrivée fît lui-même le voyage de bout en bout, l’échange n’ayant lieu qu’au point d’arrivée. En fait, surtout au XII\textsuperscript{e} siècle, cela se passait assez souvent ainsi ; des Flamands se rendaient à Gênes par exemple ; des marchands lombards sont signalés aux foires d’Ypres en 1127. De plus en plus, cependant, l’habitude se prit de se rencontrer en un lieu d’échange intermédiaire, qui fut les foires de Champagne.\par
On désigne sous ce nom un groupe de six foires, siégeant dans quatre villes différentes et qui se succédaient de façon à couvrir l’année entière. Certaines d’entre elles étaient fort anciennes (une de celles de Provins est attestée en 995). Mais elles ne prirent une grande importance qu’au cours du XII\textsuperscript{e} siècle.\par

\begin{itemize}[itemsep=0pt,]
\item[]\listhead{Tel qu’il est constitué définitivement, le cycle est le suivant :}
\item Foire de Lagny, du début de janvier au lundi qui précède la mi-carême ;
\item Bar-sur-Aube, du mardi avant la mi-carême jusqu’à la fin de février ou le début de mars ;
\item Foire centrale de Provins (dans la ville haute) : du mardi avant l’Ascension jusqu’à une durée de quarante-six jours ;  \phantomsection
\label{p86}
\item \term{Foire chaude} de Troyes, du mardi qui suit le 8 juillet jusqu’au 14 septembre ;
\item \term{Foire de Saint Ayoul} de Provins (dans la ville basse) : du 14 septembre jusqu’à la Toussaint ;
\item \term{Foire froide} de Troyes, du 2 novembre jusqu’à la semaine qui précède Noël.
\end{itemize}

\noindent Ces foires n’attiraient pas seulement les Flamands et les Italiens. Elles devinrent rapidement un centre d’échanges pour une grande partie de l’Europe. Des groupes de marchands, ville par ville, ou région par région, y avaient des maisons, destinées à leur logement et à l’étalage de leurs marchandises. Parmi eux, on trouve au XIII\textsuperscript{e} siècle, à côté des Italiens et des Flamands, des Allemands, des Espagnols, des Montpelliérains, des gens de diverses provinces françaises. Elles durent cette importance primordiale à leur situation géographique. Pas à cela seulement. Leur régularité même les rendait éminemment propres à l’établissement de cours réguliers — de marchandises et de change de monnaies — et aux règlements de paiements prévus à date fixe. La technique du crédit commercial, sur laquelle nous aurons à revenir, s’est développée essentiellement aux foires de Champagne. Une bonne organisation judiciaire s’ajouta à ces avantages. Les \term{gardes des foires de Champagne} — communs à toutes les foires et institués par le comte — veillaient au paiement des créances contractées là-bas et poursuivaient devant les juridictions étrangères les débiteurs défaillants. Aux diverses foires même, les opérations se suivaient selon un ordre strictement réglementé, les derniers jours étant consacrés à l’établissement des lettres de créance. Non seulement lieu d’échanges — très importants — de marchandises de tout ordre et de toute provenance (par exemple foires aux chevaux), mais bourse surtout monétaire, et {\itshape clearing house} européen. Les financiers italiens se faisaient envoyer les cours des monnaies aux foires de Champagne, et c’est, par exemple, par un rapport de cette espèce, heureusement conservé, que nous connaissons ce fait significatif : en 1265, avant d’entreprendre son expédition de Sicile, Charles d’Anjou fit acheter à la foire de Saint-Ayoul de Provins une grande quantité de monnaies d’or de Florence — le florin — si bien que les cours en haussèrent considérablement.\par
Or, ces foires si brillantes entrèrent en décadence à la fin du XIII\textsuperscript{e} siècle. Nous pouvons comparer les revenus que ces foires rapportaient au trésor comtal : en 1296 ils sont sensiblement plus faibles qu’en 1275. En 1339, un document rapporte ceci : les loyers des maisons appartenant au chapitre Saint Quiriace de Provins valaient \emph{« en icelluy temps que les foires de Champagne estoient en bon estat »} environ 1000 livres. Ils sont tombés à 300 livres.\par
Pourquoi ? On a souvent invoqué la fiscalité royale. À l’exception de la foire de Lagny, qui appartenait à l’abbaye de cette ville, et des sept premiers jours de la foire de Saint Ayoul de Provins, dont les revenus allaient au prieuré de ce nom, les foires étaient propriété comtale et le trésor des comtes percevait des droits importants. Or, en 1284, l’héritier du roi de France, le futur Philippe IV avait épousé l’héritière de Champagne. Désormais, le comte de Champagne fut le roi de France (à l’exception toutefois de la courte période : 1305-1314, pendant laquelle à la mort de sa mère, la reine Jeanne, le comté passa au fils aîné de Philippe IV, le futur Louis X). Et il est bien probable, en effet, que la fiscalité de Philippe le Bel et de ses fils eut ses excès en Champagne comme ailleurs. De même,  \phantomsection
\label{p87} et peut-être plus encore, les persécutions contre les Lombards. Mais là n’est pas la cause véritable de la décadence. Elle a pour origine la substitution de la voie de mer à la voie de terre, et, dans la voie de terre elle-même, un détournement.\par
\bigbreak
\noindent \labelchar{b)} La voie de terre avait toujours eu de graves inconvénients. Je ne sais si elle était moins sûre : piraterie et brigandages devaient s’équivaloir. Mais elle était certainement plus lente et plus onéreuse (à cause des péages). Pourquoi tarda-t-on à lui substituer la voie de mer ? À la fin du XIV\textsuperscript{e} siècle, on expliquait la révolution des communications par une curieuse légende : pour se rendre par mer en Flandre, les Génois auraient fait rompre \emph{« une roche en mer qui les empechoit »}. Cette roche n’a jamais existé que dans l’imagination des bons Champenois. En fait, progrès de toute part de la navigation lointaine par emploi de la boussole (qui se répand au XIII\textsuperscript{e} siècle) et de la construction des bateaux (koggen ou caraques). Parallèle à l’Umfahrt du Jutland, la confection de cartes (le plus ancien portulan connu, qui est pisan et du XIII\textsuperscript{e} siècle, donne la côte de l’Atlantique, d’ailleurs fort inexactement).\par
Nous sommes mal renseignés sur les dates. Dès 1232, nous voyons un vaisseau génois à La Rochelle. Un autre aborde en Angleterre en 1304. Au début du XIV\textsuperscript{e} siècle, d’autres apparaissent à l’Ecluse. En 1315 probablement, les Vénitiens commencent à organiser un service régulier de galères convoyées jusqu’aux ports des Pays-Bas.\par
En même temps, une modification se produisait dans les voies de terre. À une date peu antérieure à 1236, le passage du Gothard le long de la gorge de la Reuss avait été aménagé. Une route nouvelle s’ouvrit par là à travers les Alpes centrales, jusque là desservies par des cols médiocrement accessibles. En même temps, les villes de l’Allemagne du Sud et du Rhin, jusque là plutôt en retard, développaient leur commerce. Une route nouvelle s’établit, qui rejoignait par la Souabe et le Rhin la Flandre, Anvers ou la Hollande. Elle ne fit pas disparaître la route précédente. Mais elle la concurrença.\par
Importance de ces faits pour la France du XII\textsuperscript{e} siècle et de celle du XIV\textsuperscript{e}.\par

\labelblock{3° Il ne faudrait d’ailleurs pas borner les courants commerciaux qui traversaient la France ou en partaient, à la route Méditerranée-Champagne-Flandre.}

\noindent Sur la façade de la Manche ou de l’Océan, la France possède des ports actifs : Rouen, La Rochelle, Bordeaux, Bayonne (ces deux derniers en terre angevine-anglaise). Et elle prend place parmi les pays exportateurs.\par

\begin{itemize}[itemsep=0pt,]
\item[]\listhead{Pour cinq denrées surtout :}
\item L’argent du Massif Central, exporté en Angleterre et dans les pays musulmans ;
\item Les laines du Languedoc, exportées en Italie ;
\item Les draps d’un peu partout, notamment du Languedoc encore ;
\item Le sel (des marais salants de l’Atlantique) ;
\item Surtout peut-être le vin \footnote{ Voir H. PIRENNE, {\itshape Un grand commerce d’exportation au Mogen âge : les vins de France}, dans \href{http://gallica.bnf.fr/document?O=N010020}{\dotuline{{\itshape Annales d’histoire économique et sociale} [http://gallica.bnf.fr/document?O=N010020]}}, 1933., p. 225 sqq.}. Ici, servi par la navigation : en 1798 \term{vin de La Rochelle} à Liège. Vers l’Angleterre. Vers la Flandre.  \phantomsection
\label{p88} Vins de Gascogne ; mais aussi, dès la fin du siècle, de Bourgogne. Il y a eu là un des grands accrocs donnés à la production domaniale.
\end{itemize}


\labelblock{4° Ces rapports avec les pays étrangers ont contribué à donner ou à accroître en France l’importance économique d’éléments humains étrangers.}

\noindent Parmi ceux-ci, on peut ranger les Juifs. Non qu’ils ne fussent établis depuis bien des siècles sur le sol français ; leurs communautés remontent à l’époque romaine. Et ils parlaient français. Mais, outre que la différence de religion et de droit équivalaient au Moyen Age à une différence de nationalité, ils avaient dû une bonne part de leur rôle économique, au haut Moyen Age, à leurs liaisons internationales. C’est ce qui explique d’ailleurs que les progrès mêmes des relations internationales aient contribué à leur affaiblissement économique. De même et surtout, les progrès de l’intolérance, contemporains d’abord des croisades, puis de cette espèce de mise en bataille du catholicisme contre l’hérésie, qui caractérise le XIII\textsuperscript{e} siècle, d’autant qu’il y avait des conversions. Symptômes : la destruction des Talmuds en 1240 ; des bûchers isolés, comme celui où montèrent, en 1288, les célèbres martyrs de Troyes. Les Juifs avaient toujours exercé le commerce de l’argent, et accessoirement la ferme des domaines seigneuriaux et royaux ; mais ils étaient en même temps des commerçants très actifs et souvent des propriétaires terriens. En 1180, encore, un Juif était consul à Toulouse ; sous saint Louis même, un autre prévôt royal à Châtellerault. L’intolérance en même temps que la mainmise sur le commerce par les marchands indigènes les réduisirent de plus en plus, au XII\textsuperscript{e} siècle, au rôle de prêteurs (d’autant que, nous le verrons, l’usure était, en principe sinon en fait, interdite aux chrétiens). Ils étaient considérés — avec les étrangers d’ailleurs — comme attachés aux seigneurs hauts justiciers par des liens de dépendance stricts, voisins du servage ; et le roi tenait à étendre ses droits aux dépens des hauts seigneurs justiciers. Avec Philippe Auguste commence une série de mesures de persécution, qui sont en même temps d’exploitation, le roi en règle générale ne déclarant point éteintes les dettes contractées envers les Juifs, mais se substituant à eux comme créancier, ou s’efforçant de le faire. Expulsés du domaine royal en 1183, rappelés en 1198, ils furent de nouveau chassés en 1306 et leurs biens et titres de créance confisqués. L’opération cette fois menée par les officiers royaux et étendue à tout le royaume (sauf quelques principautés féodales). Les seigneurs eurent beaucoup de mal à obtenir la compensation, à laquelle ils avaient droit. Les Juifs furent d’ailleurs rappelés en 1315. Mais ils demeurèrent soumis à de fréquentes persécutions. Ils ne jouèrent de nouveau un rôle important dans la vie économique de la nation qu’au XVIII\textsuperscript{e} siècle.\par
En revanche les Italiens, qu’on appelait des Lombards. En réalité, ils ne venaient pas seulement des villes d’Italie du Nord comme Asti, mais aussi de la Toscane : Lucques, Florence, Sienne surtout. Ils étaient affiliés en général aux grandes compagnies commerçantes de là-bas. On en rencontrait même dans d’assez nombreuses villes : Montluçon en 1244, Varenne-en-Argonne au début du XIV\textsuperscript{e} siècle. Ne les imaginons pas seulement financiers, c’est-à-dire prêteurs. Le commerce de l’argent et celui des marchandises ordinaires n’étaient pas séparables au Moyen Age. Les Lucquois, par exemple, étaient les grands fournisseurs de soie brochée et brodée des cours royales ou seigneuriales. Mais il est exact qu’ils étaient  \phantomsection
\label{p89} grands prêteurs. Ils n’étaient pas les seuls à le faire, mais ils le faisaient largement.\par

\begin{enumerate}[itemsep=0pt,]
\item[]\listhead{Ils devaient l’importance de leur rôle :}
\item à l’accumulation des capitaux qu’avait provoquée en Italie le très ancien commerce méditerranéen ;
\item à la grande perfection de la technique des affaires italiennes. Ne croyons pas d’ailleurs qu’ils réussissaient toujours.
\end{enumerate}

\noindent La liste des faillites des grandes maisons italiennes, de 1298 à 1341, est impressionnante. C’est que leurs ambitions étaient souvent excessives, vues les conditions techniques et politiques de la finance du temps. C’est aussi qu’une bonne partie de leur actif consistait en créances sur des princes ou rois, aisément défaillants. Les besoins d’argent de la royauté l’ont mise, notamment depuis saint Louis et surtout depuis Philippe le Bel, en rapports fréquents avec les \term{Lombards} selon des procédés sur lesquels nous aurons à revenir. Un rôle considérable fut joué dans l’administration financière de Philippe le Bel et même parfois dans sa politique par trois frères, banquiers florentins établis à Paris et associés : Biccio, Musciato et Nicholuccio Guidi dei Francezi (Biche et Mouche). De même, dans l’administration de la Champagne, Renier Acorre, de Provins. Comme les Juifs, les Lombards furent à plusieurs reprises chassés et spoliés : en 1277, en 1291, en 1311, en 1320, le prétexte étant naturellement l’usure. Ils revinrent toujours.\par
À côté d’eux, commerçants et financiers indigènes sur lesquels nous reviendrons. La France n’est pas uniquement une dépendance de la finance italienne. Mais le rôle, que joue celle-ci, caractérise bien ses liens, désormais étroits, avec le commerce méditerranéen et l’antériorité de celui-ci.
\section[{C. Les échanges intérieurs}]{C. Les échanges intérieurs}\phantomsection
\label{c09c}
\noindent Lorsqu’il s’agit d’une époque antérieure au XII\textsuperscript{e} siècle, on a coutume de poser le problème : économie fermée et économie naturelle ? Je crois que le problème est mal posé. Mais laissons-le. Pour le XIII\textsuperscript{e} siècle, personne ne le soulève plus. Il est évident que les échanges d’un bout à l’autre de la société sont fréquents et que le numéraire y joue un rôle considérable. Reste non à apprécier leur volume (ce qui serait impossible), mais à analyser leur nature.\par

\labelblock{Quels sont les principaux objets des échanges ?}

\subsection[{1° Les denrées alimentaires de consommation courante}]{1° Les denrées alimentaires de consommation courante}

\begin{enumerate}[itemsep=0pt,]
\item Existence d’agglomérations urbaines, où vit une population dont l’activité n’est pas consacrée à produire ce qui la nourrit. À dire vrai, bien des villes sont encore à demi-rurales. Le bourgeois par ailleurs est souvent propriétaire terrien et mange volontiers le pain fait avec son propre blé ou les produits de basse-cour fournis par ses tenanciers ou fermiers. À côté du boulanger, le fournier. Il en sera ainsi jusqu’au XIX\textsuperscript{e} siècle. Mais il y a les artisans. Et dans une région urbaine comme la Flandre, malgré la puissance de l’agriculture environnante, il fallait même faire venir en grande partie les blés de l’étranger. Le souci que les villes ont de réglementer l’approvisionnement en atteste l’importance.
\item Les cours seigneuriales laïques ou ecclésiastiques. Les modifications de l’exploitation de la seigneurie ; moins de réserve, plus de rentes en argent. Il y a encore entre les mains de certains gros seigneurs, surtout ecclésiastiques, de gros produits : par les dîmes. Mais les conditions du transport rendent souvent utile de les vendre.
\item  \phantomsection
\label{p90} Les paysans eux-mêmes. Importance des disettes.
\end{enumerate}

\noindent La circulation des blés fonde des fortunes et est le grand commerce spéculatif du temps. Un sermon de la première moitié du XIII\textsuperscript{e} siècle vitupère contre les curés, qui mettent leurs blés en réserve pour les vendre plus cher, quand le marché sera moins fourni.
\subsection[{2° Les matières premières de provenance limitée : par exemple le sel}]{2° Les matières premières de provenance limitée : par exemple le sel}
\subsection[{3° Les produits fabriqués}]{3° Les produits fabriqués}
\noindent Dans les campagnes sans doute, le paysan fabrique encore lui-même souvent son drap ou tisse sa toile ; il fait ses outils de bois. Mais partout où un goût plus raffiné s’est répandu, ces produits grossiers ne suffisent plus. Il est très important de noter qu’à la différence de l’Allemagne, les seigneurs ont renoncé, dès les XI\textsuperscript{e} et XII\textsuperscript{e} siècles, aux redevances en produits fabriqués.\par

\begin{enumerate}[itemsep=0pt,]
\item[]\listhead{Ce commerce intérieur se fait de trois diverses formes :}
\item Boutiques permanentes urbaines. Parfois développées en halles ;
\item Foires : par exemple le Lendit ;
\item Marchés hebdomadaires très importants.
\end{enumerate}

\noindent Voici par exemple ce qu’on vendait à Brie-Comte-Robert, en 1209. Il y avait un marché, et en dehors de lui, parfois, des merciers et des regratiers. On vendait du bétail (au marché), du blé (au marché et ailleurs), du pain, du vin, de la laine non filée, du bois de construction ouvré ou non ; des coupes et écuelles en bois ; des pieux ; des fourches ; enfin du sel. Importance, dans la vie de relations, des marchés.\par
\bigbreak
\noindent En somme, une vie d’échanges assez active et qui pénètre assez à fond, et dans laquelle sont pris même d’obscurs producteurs. C’est ce qui explique les impôts et par conséquent les États.
\section[{D. Les moyens d’échange et le crédit}]{D. Les moyens d’échange et le crédit}\phantomsection
\label{c09d}

\labelblock{a) L’histoire de la monnaie en France au XIII\textsuperscript{e} et au XIV\textsuperscript{e} siècles est dominée par deux grands faits, qui la distinguent de l’époque précédente, et sont d’ailleurs intimement liés.}

\noindent \labelchar{1°} Concentration des droits de frappe.\par
Elle était dans l’intérêt général.\par

\begin{itemize}[itemsep=0pt,]
\item[]\listhead{Le pas décisif a été fait par saint Louis dans la fameuse ordonnance de 1262, qui pose les deux principes fondamentaux :}
\item La monnaie du roi court par tout le royaume ;
\item Celle des seigneurs, qui possèdent le droit de monnaie, ne peut courir que dans leurs propres terres.
\end{itemize}


\begin{itemize}[itemsep=0pt,]
\item[]\listhead{En outre, sous saint Louis ou sous ses successeurs, on veille à l’observation des autres règles suivantes :}
\item Il n’est plus accordé ni toléré d’autres frappes seigneuriales que celles qui ont été consacrées par l’usage ;
\item Les seigneurs ne peuvent frapper que les espèces consacrées par l’usage (principe clairement énoncé en 1315) ;
\item Il est interdit d’imiter la monnaie du roi ;
\item Il est interdit de frapper à un moins bon aloi que la monnaie royale.
\end{itemize}

 \phantomsection
\label{p91}\noindent Pratiquement, un nombre de plus en plus grand de seigneurs abandonnent leurs frappes, parfois en vendant le droit au roi. Il ne subsiste plus guère que les monnaies des grandes principautés féodales (Flandre, Guyenne). Les mutations amènent la suspension du monnayage seigneurial. La circulation n’est pas pour cela unifiée : monnaies étrangères.\par
\bigbreak
\noindent \labelchar{2°} Deux grands faits d’ordre monétaire, dans toute l’Europe :\par

\begin{itemize}[itemsep=0pt,]
\item La reprise de la frappe de l’or sur type indigène (1252, Gènes et Florence) ; débute en Italie, tentée par saint Louis, ne prend une réelle importance que sous Philippe le Bel.
\item La frappe de grosses monnaies d’argent, qui commence sous saint Louis avec le \term{gros tournois} de 1266 (l’initiative venait de l’Italie : Venise dès 1203).
\end{itemize}

\noindent Ce sont quelques-uns des changements les plus caractéristiques de l’économie du temps. Mais la création du gros devait avoir son revers : celui de favoriser les mutations.\par

\labelblock{b) Le problème des mutations est un des plus épineux qui soit. Il faut distinguer soigneusement deux périodes : avant et après la création des espèces d’or et du gros.}


\begin{itemize}[itemsep=0pt,]
\item {\itshape règne du denier} \footnote{ A. BLANCHET {\itshape et} A. DIEUDONNÉ, {\itshape Manuel de numismatique française}, t. II, Paris, 1916.}. La mutation monétaire consiste alors simplement à modifier la quantité d’argent contenue dans le denier, soit en modifiant son poids, soit en modifiant son titre (monnaie noire). Il y a mouvement général vers l’affaiblissement, avec quelques reprises, la mutation pouvant s’accompagner de diminution. Quelques chiffres : le denier de Charlemagne : en moyenne de 1 gr 80, avec un titre presque pur ; la monnaie melgorienne : en 1125, 0 gr 12, en 1273, 0 gr 07 ; le denier parisis de Philippe Auguste et de saint Louis : 1 gr 16 ou 1 gr 15 avec moins d’1/2 d’argent. Mais sous saint Louis le denier est réduit au rôle de monnaie d’appoint.
\item {\itshape Régime de \term{gros}.} Ici va intervenir le mécanisme de la monnaie de compte. Le compte par sous et livres. Comment il a été longtemps fictif (et ses deux origines). À partir du XIII\textsuperscript{e}  siècle, il tend à se matérialiser : depuis saint Louis et son gros, pour le denier tournois ; depuis Philippe le Bel avec l’agnel d’or, pour la livre tournois.
\item Seulement, la valeur n’était pas indiquée sur les pièces. Un denier était toujours un denier. Un gros tournois ne valait douze sous que par décision du gouvernement, sans que rien ne précisât sa valeur.
\end{itemize}


\begin{itemize}[itemsep=0pt,]
\item[]\listhead{Désormais, deux catégories de mutations sont possibles :}
\item Comme les anciennes, par modification du titre ou poids. Le dernier ne compte guère. Philippe pratiquera ces abaissements en teneur métallique en 1303 et 1311.
\item Mutation nominale, dont peut nous donner une idée approchée la dévaluation du franc ou celle du dollar.
\end{itemize}

\noindent {\itshape Affaiblissement} quand, en 1295, sans modification aucune de son poids (4 gr 22) ni de son titre (95\%), le gros tournois est porté de 12 deniers à 15 deniers.\par
 \phantomsection
\label{p92}{\itshape Renforcement} quand en 1305, le gros tournois, toujours du même poids et du même taux, est ramené à 13 d. 1/8.\par
Ces opérations ont été largement pratiquées sous Philippe le Bel et ses fils.\par

\begin{itemize}[itemsep=0pt,]
\item[]\listhead{En bref, cette histoire complexe peut être résumée ainsi :}
\item Affaiblissement nominal en 1295, accru et compliqué d’un affaiblissement matériel en 1303.
\item Renforcement nominal et matériel en 1306.
\item En 1309, affaiblissement nominal de l’or seul ; suivi en 1311 de la cessation de toutes autres monnaies d’argent que des monnaies noires.
\item En 1318, reprise de la frappe du gros, au titre de 0,95 et à la même valeur nominale qu’en 1295.
\item Sous Charles IV, affaiblissement nominal de l’or et retour aux monnaies noires.
\end{itemize}

\noindent En valeur or, la livre tournois : en 1258, 118 f. 22, en 1793, 4,82. Le point d’arrivée de l’histoire devait être encore l’affaiblissement.\par
Notons que mutation matérielle et mutation nominale revenaient au même procédé. Étant donné le rapport d’une quantité donnée de métal précieux et une valeur monétaire — par exemple 1 gramme d’argent et 1 denier — faire varier ce rapport. Plus généralement, la masse de métaux précieux dans un pays étant supposée immuable, augmentation des unités de paiement : inflation.\par
Mais le grand problème : pourquoi ?\par
\labelchar{c)} Avant de chercher à répondre à cette question, il faut éclaircir le terrain de quelques difficultés préliminaires.\par
Les mutations sont, au XIII\textsuperscript{e} siècle, un fait ancien. Le système de la mutation nominale leur donne des facilités nouvelles : elle ne crée rien.\par
Elles sont un fait général dans toute l’Europe, à ce point qu’en Allemagne, on trouve, dès la fin du XII\textsuperscript{e} siècle, le système de la mutation périodique : la monnaie s’affaiblissant dans l’année selon un rythme fixe.\par
Ceci ne veut pas dire que les causes aient été de tout temps et partout les mêmes. Mais qu’il s’agit du moins d’une phase à expliquer par des causes assez générales et profondes. Ceci dit, il convient, pour rechercher les causes des mutations, de se demander quelles étaient leurs conséquences, surtout celles auxquelles pensaient les contemporains.\par

\begin{enumerate}[itemsep=\baselineskip,]
\item[]\listhead{Il faut distinguer :}
\item  \textbf{Conséquences communes aux affaiblissements et renforcements :}\par
 Par le décri et grâce au seigneuriage, profit, pour le seigneur des ateliers. On notera que le profit était plus fort en cas d’affaiblissement, celui-ci ayant pour effet d’attirer le métal vers les ateliers comme nous le verrons.\par
 En période bi-métallique, possibilité d’un ajustement de valeur, dans un sens ou dans un autre, entre les deux métaux précieux.\par
 Introduit un trouble gênant dans la circulation.  \phantomsection
\label{p93}
 
\item  \textbf{Conséquences des affaiblissements :}\par
 Évite une perte pour le pouvoir émetteur en mettant sa monnaie au pair des espèces dépréciées circulantes. Dépréciées pourquoi ? Ceci exige un mot d’explication : — par les cisailleurs, tolérance ; usure — par l’afflux de monnaies étrangères de mauvais aloi. La loi de Gresham (signalée dans un mémoire du maître des monnaies, en 1312, pour la monnaie de Bretagne en Normandie).\par
 Augmente la masse des moyens de paiement en circulation (jusqu’à la hausse des prix, dont on n’avait pas une idée très nette, qui ne venait pas tout de suite, et n’était pas forcément proportionnelle).\par
 Augmentation des moyens de paiement immédiatement à la disposition du roi ; expressément, P. Dubois met en rapport avec l’introduction du service soldé.\par
 Problème des dettes \footnote{ Voir Marc BLOCH, {\itshape Les mutations monétaires et les dettes}, dans \href{http://gallica.bnf.fr/document?O=N010027}{\dotuline{{\itshape Annales d’Hist. écon. et sociale} [http://gallica.bnf.fr/document?O=N010027]}}, t. VI, 1934, p. 383.} : nominalisme ou réalisme. La question était bien connue depuis longtemps comme le prouvent les stipulations de contrat — dès le XII\textsuperscript{e} siècle — qui exigent le paiement de rentes en monnaie forte ou monnaie pesée, même si la monnaie était entre temps affaiblie. Ces accords privés continuent toujours. Mais le principe de ces ordonnances était le nominalisme. L’avantage pour le commun des débiteurs n’existant qu’au cas où, par ailleurs, la hausse des prix entraînait une hausse de leur gain. Il était sensible immédiatement pour le pouvoir émetteur, dans la mesure où celui-ci était plus débiteur que créancier et où il disposait d’une masse métallique antérieure à la mutation.\par
 Attirer, au moins au début, les métaux vers l’atelier (avant la hausse des prix). En 1295, l’affaiblissement coïncide avec la saisie de vaisselles d’or et d’argent chez les particuliers.
 
\item  \textbf{Conséquences des renforcements :}\par
 Diminuer la masse des moyens de paiement en circulation ; entraîne une hausse des prix.\par
 Rétablir la bonne réputation des monnaies locales et par suite prémunir contre la fuite vers les monnaies étrangères (inverse de la loi de Gresham).\par
 Problème des dettes. Appliquer purement et simplement le nominalisme eût acculé tous les débiteurs à la faillite ; Pratiquement, les ordonnances appliquent le réalisme aux dettes à courte durée — de même aux contrats de travail (sauf résiliation) ; appliquent le nominalisme aux dettes périodiques (loyers et cens seigneuriaux). L’exception n’était pas faite pour les contrats de travail.
 
\end{enumerate}

\noindent De tout cela que résulte-t-il ? Des raisons très diverses, quelques-unes, économiquement valables, pouvaient entraîner le roi aux mutations en général. La plupart inclinaient aux affaiblissements. Mais ceux-ci étaient véritablement avantageux pour un temps très court : ils fournissaient les moyens de parer à une crise en donnant des moyens de paiement accrus, en drainant les monnaies vers les ateliers. Ils n’étaient pas seulement un moyen d’ »  inflation » ; ils paraient aux difficultés nées de l’absence de pratiques de trésorerie :  \phantomsection
\label{p94} rappelons les dates : 1295, conflit avec l’Angleterre, saisie du duché de Guyenne, entreprise d’une coalition contre le Plantagenêt ; 11 juillet 1302, Courtrai ; 1324, reprise de la guerre anglaise, préparatifs d’une nouvelle guerre de Flandre. Puis la hausse des prix venant, le métal se haussant, l’opération perdait de son intérêt.\par
Mais il faut tenir compte aussi de la pression du sentiment public. Traditionnellement, les mutations passaient pour un grand mal ; comme en témoigne dès le XII\textsuperscript{e} siècle l’importance du monnayage (en Normandie, à Étampes et Orléans, 1137 ; Saint-Quentin 1195) surtout, l’affaiblissement ; par représentation collective très forte mythe de la \emph{« bonne monnaie de Monseigneur saint Louis »}. Il gênait d’ailleurs avant tout les marchands. Le renforcement, davantage les petites gens : dans l’hiver 1306-1307, émeutes de Paris, Châlons-sur-Marne, Laon. La pression des classes riches arrêtait donc sur la pente de l’affaiblissement. Atmosphère de trouble social.\par
Et à la longue, l’affaiblissement l’emportant, ruine des rentiers déjà pressentie sous Philippe le Bel par Pierre Dubois.\par
\bigbreak
\noindent \labelchar{d)} Comment se faisaient en réalité les paiements ? De la part des paiements en nature, et de la ferme du paiement \term{apprécié}. Les deux stades — diminution sensible au début du siècle — quelques reprises sous l’influence des grandes mutations.\par
Parmi les monnaies, le rôle des lingots (en voyage). Diminue devant la reprise du rôle de l’or. Les monnaies étrangères. Le rôle du change et l’importance sociale des changeurs.\par
L’absence de papier. Mais le rôle des instruments de crédit lettre de change (voir plus loin).\par
\bigbreak
\noindent \labelchar{e)} Ceci nous ramène à la question de crédit. L’interdiction de l’usure. Elle se tire, bien entendu, d’un texte de l’Ancien Testament proscrivant le prêt à intérêt à l’intérieur du peuple hébreu ; texte repris par le Nouveau Testament, notamment l’Évangile de Luc : \emph{« mutuum date, nihil inde sperantes »}. Elle n’a été originellement appliquée impérativement qu’aux clercs. La législation canonique et civile de l’époque carolingienne l’étend aux laïques. Elle est codifiée par la philosophie scolastique.\par
Les justifications — d’ordre théorique : idée du juste prix ; on ne vend pas le temps — pratiquement, les moralistes n’avaient guère sous les yeux que le prêt à la consommation, sans valeur économique et souvent oppressif.\par
L’application pratique. Les usuriers ont été poursuivis par les tribunaux d’Église ; frappés de peines surtout canoniques. Pratiquement, le prêt à intérêt n’a jamais cessé d’être appliqué et cela sous deux formes — prêt pur et simple — prêt détourné sous des formes diverses, dont la plus frappante est le change (avec majoration du change normal) — le gage. Distinction du mort-gage et du vif-gage. Le mort-gage, bien que déclaré usuraire par l’Église au XII\textsuperscript{e} siècle, n’a jamais cessé d’être pratiqué par les établissements ecclésiastiques.\par
Il est cependant possible que le caractère illicite et oblique de l’usure ait empêché le Moyen Age de connaître le poids social des dettes.\par
Problèmes particuliers et à peu près nouveaux dans la France du XIII\textsuperscript{e} siècle : le crédit commercial.\par
 \phantomsection
\label{p95} La première forme du prêt commercial a été un prêt surtout maritime. La commende : un bailleur de fonds et un commerçant s’associent ; le premier fournit au second, soit des marchandises, soit de l’argent pour aller vendre en pays lointain. Les bénéfices sont partagés. Il va de soi que si tout va bien, le prêteur recevra plus que son capital. Vieux dicton déjà chez Pierre le Chantre : le marchand a le choix entre l’Enfer et la pauvreté.\par
Mais cette forme ne pouvait suffire à tout. Et, sous l’impulsion de besoins économiques nouveaux, nous voyons se développer une attitude nouvelle tendant à limiter l’intérêt à certaines conditions et à un certain taux.\par

\begin{enumerate}[itemsep=0pt,]
\item[]\listhead{Double :}
\item Dans la doctrine : effort pour réintégrer l’intérêt comme compensation pour des désavantages subis par le prêteur. De là est venu le mot intérêt ({\itshape interesse}) légitime, opposé à l’usure (qui gardera un sens péjoratif). {\itshape Quod interest} : différence entre la situation présente du prêteur et sa situation soit au point de départ du prêt, soit au moment, lorsqu’il y a eu délai, où l’obligation eût dû être exécutée. Deux formes surtout — {\itshape damnum emergens} (justification d’un vieux truc de contrat) — {\itshape lucrum cessans} (beaucoup moins bien accepté).
\item Les actes des pouvoirs publics limitent le taux sur l’exemple antique : d’abord pour les Juifs ; en France, Philippe le Bel, en 1311 (à 20\% par an ; un taux plus faible est prescrit pour les foires de Champagne).
\end{enumerate}

\noindent Donc, le crédit s’est fait sa place. Qui l’exerce ? Les marchands financiers, les grands établissements ecclésiastiques. Cas des Templiers. Ce qu’il faut entendre par la \emph{banque} des Templiers : dépôt, mais il s’agit d’une simple garde d’espèces ; transport de place à place ; prêts dont l’intérêt semble s’être en général dissimulé sous la forme du délai ; gestion du Trésor royal.\par
Les procédés du crédit commercial ; la lettre de change de place à place, avec paiement, par exemple : aux foires de Champagne ; en Orient. La clause de mandataire. Premiers exemples de cessions.
\section[{E. La production}]{E. La production}\phantomsection
\label{c09e}
\noindent Il y a lieu d’insister d’abord sur le dualisme de la production, présent (à des degrés différents) dans presque tous les centres (idée d’Espinas) : pour la consommation intérieure ; pour le commerce extérieur. Opposition qu’on peut synthétiser dans celle des deux métiers : boulangers, drapiers.\par
La forme économique pourtant commune (presque universellement) est l’artisanat : le petit atelier.\par
Mais il faut distinguer la \emph{forme} et le \emph{régime économique}. Dans beaucoup de centres, et notamment les plus gros, comme les villes flamandes, on voit s’établir dès le XII\textsuperscript{e} siècle, et cela se confirme durant le siècle suivant, un véritable système capitaliste, fondé sur le commerce. Bien entendu, seulement pour la production de la catégorie commerce extérieur. Là se posent les gros problèmes de marchés ; achat de la matière première ; débouchés. Le cas est particulièrement net dans la draperie, qui peut servir de type. Le marchand achète la laine à l’état brut (en Flandre, essentiellement la laine anglaise). Il la vend successivement aux divers métiers  \phantomsection
\label{p96} chargés de la transformer en draps (j’énumère non exhaustivement fileurs ou fileuses, parfois ruraux ; cardeurs ; tisserands, peigneurs et tendeurs de draps ; foulons ; teinturiers). Il exige en général de ces artisans de ne vendre qu’à lui. L’opération se résout donc juridiquement en une série de ventes et d’achats. Pratiquement, la simplification des mouvements de fonds aboutissait en réalité au versement d’un salaire. D’autres moyens de dépendance : louage de la maison ; des outils ; crédit. Donc un système de capitalisme sans grande entreprise. Il n’a pas disparu de notre monde moderne (voyez les rapports de la boulangerie avec la minoterie ; des débits de boisson avec les brasseurs). Mais il a disparu de la grande production. C’est sur ce système que se fonde la fortune du grand patriciat urbain. Imaginons bien un système très dur.\par
Un type : Jean Boinebroke \footnote{ Voir G. ESPINAS, {\itshape Les origines du capitalisme.} 1. {\itshape Sire Jehan Boinebroke, patricien et drapier douaisien ( ? , 1286 env.)}, Lille, 1933, qui reprend et corrige un article paru antérieurement dans le {\itshape Vierteljahrschrift für Sozial-und Wirtschaftsgeschichte.}}. Nous le connaissons surtout par un acte émané de ses exécuteurs testamentaires, comportant règlement d’une saisie. Jean Boinebroke mourut avant février 1286. Il appartient à une très haute famille bourgeoise ; il fut neuf fois échevin. Très riche, il a, comme les hommes riches de son temps, placé une partie de sa fortune en biens fonds (maisons de ville, biens ruraux). Mais son métier essentiel est d’être drapier, entendez marchand drapier. Il achète non seulement la laine, mais aussi guède, alun et vend le drap. Ses relations avec l’artisanat sont celles qui ont précédemment été décrites. Il faut noter cependant un grand atelier : une teinturerie (mais tout le travail de teinturerie ne s’y fait probablement pas). Le centre est la maison de Boinebroke, où il a de véritables bureaux (avec trois clercs) et des magasins d’entrepôt. Il est créancier de certains des artisans qu’il emploie. Un créancier très dur. En revanche, un débiteur fort inexact. Il spécule dans ses achats de guède sur le cours, inconnu aux petits producteurs ruraux. Un maître impérieux, rapace, et volontiers goguenard — les textes parlent de son \emph{courroux}. À une pauvre femme, sa débitrice, sur laquelle il a opéré une saisie qui monte plus haut que son dû, il propose tout simplement d’aller gagner sa vie en \emph{ébourrant les draps}. Et comme la femme, tout en acceptant, proteste un peu, voici tout simplement sa réponse : \emph{« Commère, je ne sai mujike vos dei, mais je vos metrai en mon testament... »}\par
Cette production, comme le commerce d’une façon générale, n’est pas libre. Elle est réglementée ; et dans cette réglementation, s’exerce en partie par des associations.
\section[{F. Réglementation et associations}]{F. Réglementation et associations}\phantomsection
\label{c09f}
\noindent Le principe de la réglementation. Il est général. L’idée de la liberté économique répugne à l’esprit du temps. La boutade de von Below \footnote{ Voir W. GALLIEN, {\itshape Der Ursprung der Zünfte in Paris}, 1910. — G. FAGNIEZ, {\itshape Études sur l’industrie et la classe industrielle à Paris au XIIIe et au XIVe siècle}, Paris, 1877.} : jadis la libre concurrence était interdite et pratiquée (par les gros) ; aujourd’hui (avant guerre), elle est de principe et elle est tournée.\par
 \phantomsection
\label{p97}
\labelblock{Quels sont, outre une méfiance générale de la nature humaine, les raisons de cette réglementation ?}

\noindent Première raison : chaque maître (remarquons qu’il s’agit de maître) doit avoir son gain, conforme à son statut social. Le petit maître n’a jamais besoin que de protection. Donc, il faut éviter, à l’intérieur d’un même métier, les accaparements : de marchandises (en 1265, le règlement des éperonniers de Poitiers les astreint à l’achat en commun) ; de main-d’œuvre.\par
De même, de réprimer la réclame.\par
De même, chacun doit rester dans son métier. En d’autres termes, au même titre que la concentration horizontale, la concentration verticale est frappée d’interdiction. Par exemple, un ban urbain de Saint-Omer défend, en 1280, aux tanneurs de se faire cordonniers, aux cordonniers de se faire tanneurs. Extrême division du travail. Ici, naturellement, intervenait l’intérêt. En 1307, sur la requête des foulons et teinturiers, le Parlement interdit aux tisserands de Château-Landon l’exercice de ces deux métiers.\par
Enfin, la même notion de la stabilité sociale conduit non seulement à pratiquer en fait, mais à élever parfois à la hauteur d’un principe, la notion d’hérédité ; de même que d’innombrables textes mentionnent l’obéissance au seigneur \emph{naturel}, un diplôme de Louis VII mentionne à Paris des bouchers \emph{naturels} par droit héréditaire.\par
\bigbreak
\noindent Deuxième raison : maintenir la valeur de la production par une technique bonne, ce qui, aux yeux du Moyen Age, veut dire aussi le plus souvent \emph{conforme à la coutume}. Mais pourquoi tenir à maintenir la valeur de la production ? Ici, le motif profond diffère selon qu’il s’agit de satisfaire un marché intérieur, plus ou moins fermé, et les débouchés, plus ou moins lointains. Dans le premier cas, c’est bien l’intérêt du consommateur que la réglementation s’attache à protéger. Dans le second, c’est celui de la marque locale.\par
De tout cela, il résulte évidemment que, dans la pratique de la réglementation, s’opposent des intérêts souvent contraires : les productions tendent à échapper aux règles qui défendent le consommateur ; les grands marchands visent à l’accaparement... Nous allons voir cela plus clairement en étudiant les autorités qui réglementent.\par
La réglementation peut appartenir ou appartient pour portions variables : au seigneur ; à la communauté urbaine ; à des associations de producteurs ou de marchands. Par exemple, à Beauvais, un accord de la commune avec l’évêque, en 1276, remet la désignation de prud’hommes pris dans le métier de la draperie pour vérifier les draps : parce que les maires et les pairs \emph{« connaissent mieux que l’évêque les hommes sages et idoines du métier de la draperie »}. À Châlons, en 1250, les drapiers se plaignent que l’évêque a reçu de l’argent de la ville pour retirer au métier et remettre aux bourgeois la surveillance du métier.\par
\bigbreak
\noindent Ajoutez enfin l’ingérence du pouvoir royal (remarquez d’ailleurs que, dans certaines villes, il intervient en qualité moins de roi que de seigneur : ne fût-ce qu’à Paris). Il faut d’abord voir les associations.\par

\labelblock{Le plus ancien type d’association connu, ce sont les sociétés de marchands faisant le commerce lointain}

\noindent Elles sont désignées dans la France  \phantomsection
\label{p98} au Nord de la Loire, des mots germaniques de {\itshape guildes} ou de {\itshape hanses}, ou encore par ceux de {\itshape confréries —} ou de {\itshape charités.} Elles sont en plein développement au XII\textsuperscript{e} siècle, organisées avec chefs élus et trésor commun. Objets : religieux (offrandes en commun ; les confrères doivent assister aux funérailles les uns des autres) ; mondaines (beuveries en commun). Mais surtout entraide, notamment pour les voyages à l’extérieur. Par exemple, le règlement de la charité de Valenciennes prévoit l’aide en cas de chars brisés ou de chevaux fatigués. Mais ce caractère diminue à mesure que le commerce devient stable. Alors, aide sociale et économique.\par

\begin{listalpha}[itemsep=0pt,]
\item[]\listhead{D’où les tendances :}
\item exiger que tout marchand fît partie de la guilde. Les statuts de la guilde de Saint-Omer (1083-1127) interdisent de porter secours au marchand de la ville qui ne sera pas de la guilde ;
\item réserver le commerce aux affiliés. La Hanse des Marchands de l’eau, à Paris, a le privilège du trafic par eau à Paris et sur la Seine, entre des limites déterminées ; sauf association ;
\item écarter les pauvres par des droits d’entrée exorbitants, et favoriser les descendants ; souvent interdites aux gens des métiers.
\end{listalpha}

\noindent Complété par des associations inter-guilde comme la Hanse de Londres, et la Hanse des Dix-sept villes (pour les foires de Champagne). Résistances : interdiction des confréries au synode de Rouen, en 1190.\par
À côté de ces guildes marchandes, apparaissent, dès le XII\textsuperscript{e} siècle, les associations de métier que nous avons pris l’habitude d’appeler \emph{corporations}. Nous n’avons donc pas à rechercher leurs origines : mais il faut prévenir une équivoque. On les a parfois cru issues de groupes de travailleurs ouvrant pour le seigneur seul. La seule raison en était des impôts spécialisés, sous forme de fourniture et, par voie de conséquence, la juridiction des fonctionnaires spécialisés (à Paris, par exemple, les boulangers, dits \emph{talemeliers}, dépendent du panetier, les fripiers du chambrier). La conclusion dépasse de beaucoup les prémisses. Certes, le pouvoir seigneurial a eu sa part dans la formation de ces corporations, mais celles-ci ne sont pas de simples ateliers seigneuriaux libérés.\par

\begin{enumerate}[itemsep=0pt,]
\item[]\listhead{En fait, deux causes expliquent la formation des corporations :}
\item Le besoin spontané des hommes de même métier de s’unir dans des buts de solidarité : mêmes rues, confréries, entraide. Les saints patrons : saint Éloi, des orfèvres, saint Barthélémy l’écorché, des tanneurs ; sainte Claire des verriers. Les scieurs de long fêtent la Visitation (les deux scieurs s’inclinent l’un vers l’autre). Les épiciers, saint Michel, à cause de la balance. Dès la fin du XIII\textsuperscript{e} siècle, les foulons de Bruxelles ont leur hôpital.
\item Celui des autorités, seigneuries ou villes, de grouper les métiers pour la facilité de la réglementation. Là où, même dans les villes flamandes, domine le grand patriciat, c’est la ville ou la guilde qui réglementent directement la draperie.
\end{enumerate}

\noindent Il y a donc une lutte obligée entre le désir des groupes de s’administrer eux-mêmes et celui des pouvoirs (eux-mêmes en rivalité) à les dominer et réglementer directement. Le type de la corporation pure, qui devait aux siècles suivants faire tache d’huile sur toute la France, s’est en réalité constitué à Paris. Là, la ville n’existant pas en tant que communauté autonome, les métiers se sont trouvés face à face avec le pouvoir royal. Plus exactement (depuis 1261) avec le prévôt de carrière. Le premier de ces prévôts (Etienne Boileau) fit, entre 1261 et 1270, mettre par écrit les règlements des métiers, avec la préoccupation très nette d’amoindrir,  \phantomsection
\label{p99} dans la mesure du possible, les juridictions des officiers royaux. C’est un système de corporations s’administrant elles-mêmes, sous la surveillance très stricte du pouvoir royal et avec exploitation fiscale rigoureuse de la part de celui-ci.\par
Enfin, il convient de ne pas oublier que la corporation est essentiellement une association de maîtres, qui dominent les valets et les apprentis.\par

\begin{enumerate}[itemsep=\baselineskip,]
\item[]\listhead{Voyons plus précisément les problèmes du droit corporatif :}
\item Gouvernement de la corporation : par des \emph{jurés} et des \emph{gardes}, généralement élus (je n’insiste pas sur la question de terminologie), mais souvent soumis à l’approbation du pouvoir seigneurial et urbain.
\item Réglementation interne : mêmes difficultés ; même solution. Ici, très nettement, les intérêts s’opposent, lorsqu’il s’agit de métier de consommation intérieure.
\item  Accès à la maîtrise. Normalement, il faut pour cela l’acceptation par les gens du métier : un temps d’apprentissage ; achat du métier à la corporation ou au seigneur ; pas toujours ; souvent, une cérémonie symbolique, telle que celle-ci, décrite par Etienne Boileau, pour les talemeliers : \emph{« le nouveau talemelier doit d’abord acheter le métier. Puis ouvrer quatre ans. Ensuite, il prend un pot de terre, tout neuf, et le remplit de noix et de « meules »}, c’est-à-dire de petites pâtisseries légères. Alors, accompagné de tous les talemeliers, des maîtres-valets de chaque boutique, il se rend devant la maison du chef de la corporation, le \emph{maître} des talemeliers. Il remet à celui-ci son pot et lui dit : \emph{« Mestre, je ai fait et acompli mes quatre années ». — « Est-ce vrai ? »} demande le maître au coustumier. Si celui-ci répond \emph{oui}, le maître rend son pot au postulant. Celui-ci le jette contre le mur de la maison. Alors, tout le monde entre dans la maison du maître, qui leur offre \emph{feu et vin}. Mais non pas gratis. Chacun lui donne, pour son écot, un denier ».\par
 Le chef-d’œuvre n’est exigé, à Paris, que dans une corporation, celle des charpentiers pour selles ; encore est-ce pour la fin de l’apprentissage. Par conséquent, libre accès au rôle de valet, comme de maître.
 
\item  Mais tout le monde peut-il avoir accès à la maîtrise ? C’est le gros problème. La réponse varie extrêmement selon les métiers.\par
 Métiers strictement héréditaires : à Paris, les tisserands. Mais c’est exceptionnel et prouve que Paris n’était pas une ville de grande exportation drapière. De très bonne heure, c’est au contraire la règle pour les métiers d’alimentation, qui avaient à servir un marché restreint ou supposé tel. Notamment, à peu près partout, les étaux de bouchers étaient héréditaires, et les maîtres bouchers tendaient à se transformer en véritables rentiers. Ici d’ailleurs, les intérêts des villes risquaient d’entrer en conflit avec ceux des maîtres, notamment en ce qui regarde l’approvisionnement en pain. Par exemple, en 1307, la commune de Pontoise, plaidant contre les boulangers de la ville, obtint :\par
 
\begin{listalpha}[itemsep=0pt,]
\item que les droits d’entrée fussent considérablement rabaissés,
\item que les boulangers ne fussent plus seuls à surveiller la fabrication : à leurs délégués sont adjoints deux autres prud’hommes de la ville,
\item que, trois jours par semaine, les boulangers des alentours pourront venir vendre leur pain à Paris.
\end{listalpha}

  \phantomsection
\label{p100} Même dans beaucoup d’autres métiers, les fils de maîtres jouissent de dispenses diverses (droits d’achat, durée de l’apprentissage). L’acceptation par les maîtres pouvait équivaloir à un rejet des candidats étrangers par leur naissance à la corporation.
 
\item  
\begin{listalpha}[itemsep=0pt,]
\item[]\listhead{Deux raisons pouvaient provoquer les luttes de classes :}
\item les artisans contre le grand capitalisme. Sensible notamment dans les villes flamandes au cours du XIII\textsuperscript{e} siècle\footnote{Renvoyer au beau développement de Pirenne, t. I, livre II, chap. IV, p. 1}. Les \emph{ongles bleus}. Guildes contre métier ;
\item les compagnons contre les maîtres. Débuts du compagnonnage.
\end{listalpha}

 
\end{enumerate}

\chapterclose


\chapteropen
\chapter[{10. L’église et la vie religieuse}]{\textsc{10. }L’église et la vie religieuse}\phantomsection
\label{c10}\renewcommand{\leftmark}{\textsc{10. }L’église et la vie religieuse}


\chaptercont
\section[{A. La foi. Généralités}]{A. La foi. Généralités}\phantomsection
\label{c10a}
\noindent  \phantomsection
\label{p101} Qu’il convient de commencer par elle et pourquoi. Mais difficultés extrêmes de son étude. Manque de documents ; et manque d’études.\par
On peut dire du XII\textsuperscript{e} siècle, comme de tout le Moyen Age, qu’il est un siècle croyant. Mais en disant cela, on n’a pas dit beaucoup.\par

\begin{enumerate}[itemsep=0pt,]
\item[]\listhead{Essayons de préciser :}
\item Il est gênant de parler de croyance au surnaturel, la notion de \emph{naturel} elle-même étant trop imprécise. Parlons, si l’on veut, de croyance à l’intervention plus ou moins constante dans le monde visible de puissances invisibles. Sentiment, on peut le dire, universellement répandu. Je ne crois pas qu’il y ait à proprement parler d’athées.
\item La croyance à une vérité révélée, supérieure à toute démonstration rationnelle ou expérimentale ; sentiment presque universellement répandu : seule exception : les averroïstes, sur lesquels nous reviendrons.
\item Croyance à cette vérité révélée sous la forme de la doctrine chrétienne, élaborée par l’Église d’Occident : sentiment très général, mais qui a comporté quelques exceptions : les hérésies, et notamment le catharisme qui, sous ses formes extrêmes, est une religion à part, non chrétienne ; certaines explosions de scepticisme individuel, nées souvent d’une réaction anticléricale ; la christianisation certainement encore insuffisante, en profondeur, des campagnes.
\end{enumerate}

\noindent Notons d’ailleurs qu’il y a eu évolution. Le XIII\textsuperscript{e} siècle s’ouvre pour l’Église par une crise grave. Il est rempli, jusqu’à son troisième tiers environ, par une vigoureuse résistance de l’Église, couronnée de succès. La France et l’Europe sont, vers 1300, plus profondément catholicisées qu’en 1200. De cette crise, l’aspect intellectuel doit être réservé pour plus tard. Mais il faut voir ici d’abord l’hérésie et la lutte contre l’hérésie. Naturellement, il y a quelque inconvénient à commencer ainsi par l’hétérodoxie, qui n’est pas le plan le plus général. Mais cela est nécessaire : parce que l’évolution de l’orthodoxie ne serait guère intelligible sans ces connaissances préliminaires.
\section[{B. Les hérésies }]{B. Les hérésies \protect\footnotemark }\phantomsection
\label{c10b}
\footnotetext{ \noindent Un mot de bibliographie (j’y joins l’Inquisition). Renseignements généraux, mais peu sûrs dans H. Ch. LEA, {\itshape Histoire de l’Inquisition au Moyen âge}, trad. franç., Paris, 3 vol., 1900-1902 ; plus récent : Jean GUIRAUD, {\itshape Histoire de l’Inquisition}, t. I, Paris, 1935, utile, sérieux, partial (si j’en juge par les travaux précédents de l’auteur) ; le premier volume, seul paru, traite de l’hérésie et des débuts de la répression avant l’Inquisition.\par
 Sur les Cathares — C. SCHMIDT, {\itshape Histoire et doctrines de la secte des Cathares ou Albigeois}, 2 vol., Paris, 1849, demeure fondamental. Peu de choses y ont été ajoutées par E. BROECHX, {\itshape Le catharisme, étude sur les doctrines, la vie religieuse et morale, l’activité littéraire et les vicissitudes de la secte cathare avant la Croisade}, Hoogstraten, 1916.\par
 Sur les Vaudois, travaux insuffisants. On peut citer E. COMBA, {\itshape Histoire des Vaudois}, 2\textsuperscript{e} éd., t. I, Paris, 1901.\par
 Sur l’Inquisition : E. VACANDARD, {\itshape L’Inquisition, ses origines, sa procédure}, 1906, et du même auteur : {\itshape Inquisition}, dans le {\itshape Dict. de théologie cathol}., VII. Observations intelligentes dans la brochure de E. JORDAN, {\itshape La responsabilité de l’Église dans la répression de l’hérésie au Moyen âge}, 1915. Renseignements précis et vivants dans Ch. H. HASKINS. {\itshape Robert le Bougre and the beginnings of the Inquisition in the Northern France}, dans {\itshape American Hist. Review}, t. VII, 1901-1902.
}
 \phantomsection
\label{p102}\noindent Deux hérésies fondamentales : Vaudois et Cathares.\par

\begin{enumerate}[itemsep=\baselineskip,]
\item[]\listhead{Observations préliminaires :}
\item  Nous les connaissons l’une et l’autre fort mal.\par
 Les Vaudois ont survécu à la persécution, jusqu’au moment où ils finirent, aux XVI\textsuperscript{e} et XVII\textsuperscript{e} siècles, par se fondre dans le protestantisme. Mais ç’a été obscurément et précairement dans quelques vallées alpestres, sans grande littérature ; par surcroît, en se transformant beaucoup ; de sorte qu’il y aurait imprudence à conclure de leur doctrine relativement aux temps proches de nous à leurs doctrines anciennes.\par
 Les Cathares, eux, ont disparu, vers la fin de l’époque étudiée ; et leur littérature a été de fond en comble détruite.\par
 Il en résulte que, sur les deux doctrines, nous ne sommes plus guère renseignés que par leurs ennemis : sommes catholiques, interrogatoires d’inquisiteurs ; manuels à l’usage de ces derniers (dont le plus célèbre est la {\itshape Pratica officii Inquisitionis heretice pravitatis}, rédigé en 1323 ou 1324 par Bernard Gui, frère prêcheur et inquisiteur en Toulousain : est publié avec traduction par G. Mollat dans les {\itshape Classiques de l’histoire de France au Moyen âge}, 1926-1927, avec une Introduction, utile à la fois sur l’Inquisition et les hérésies).\par
 Insuffisance de ces sources.
 
\item Les Vaudois et les Cathares ont des origines et une couleur première très différentes. Ils étaient conscients de leurs contrastes. En 1207, à Lansac, sur la grand-place, un prédicateur cathare soutient une controverse contre un Vaudois. Il semble bien, cependant, quelqu’étrange que cela puisse paraître quand on connaît les principes de leurs doctrines, qu’il y ait eu de l’un à l’autre des gradations. Cela s’explique par diverses raisons : l’ignorance dogmatique des masses chez les hérétiques comme chez les orthodoxes ; un caractère commun, qui est la répugnance envers les sacrements de l’Église, notamment la négation de l’efficacité des messes pour les morts (par suite du Purgatoire) et de l’intercession des saints. Nous touchons là à ce que l’on peut appeler l’atmosphère hérétique du Moyen âge. L’idée était répandue en dehors même des hétérodoxes avec dogme. Un paysan de Cerizy en Normandie dit \emph{« qu’il n’a jamais pris le corps du Christ et que le pain bénit vaut autant que le corps  \phantomsection
\label{p103} du Christ (l’hostie), pourvu qu’il soit pris à bonne intention »}. La démarche mentale, dans la foule, a presque toujours été la même : sentiment antisacerdotal, très répandu, né de l’incontestable insuffisance morale d’une partie du clergé, surtout paroissial, des abus de l’Église possédante ; l’hérésie, d’antisacerdotale, devint antisacramentelle, par la répugnance à admettre la valabilité des sacrements distribués par le prêtre indigne ; quelques-uns en restent là : voyez le paysan de Cerizy. D’autres vont plus loin et se rattachent à une des doctrines, qui justifient les sentiments décrits.
\item Mode de propagation et influence de classe : les éléments mobiles (clercs vagants, religieux, les marchands ; dans le Nord de la France, les tisserands). L’atmosphère de classe : les tisserands encore. Mais pas seulement. À Cambrai, en 1236, l’Inquisition brûla trois anciens échevins.
\end{enumerate}

\subsection[{1° Les Vaudois}]{1° Les Vaudois}
\noindent L’hérésie est née dans un milieu urbain, celui de Lyon, et dans des conditions, dont on a bien souvent observé, qu’elles rappellent de fort près celles de la naissance de l’ordre franciscain.\par
Vers 1170, un marchand lyonnais, Pierre Valdo, distribue ses bien aux indigents, prêche la pauvreté évangélique et réunit autour de lui un petit groupe de laïques prêcheurs, qui ne tardèrent pas à s’appeler {\itshape Pauperes Christi}. La prédication était interdite aux laïques, l’archevêque les excommunie. Ils ne se soumirent pas. Au concile de Latran de 1179, ils ne sont pas mal accueillis par Alexandre III, qui confirma le vœu de pauvreté de Pierre Valdo, mais le droit de prêcher lui fut accordé seulement avec l’autorisation de l’ordinaire. Celle-ci lui étant refusée, ils persévèrent. \emph{« Cette usurpation du rôle d’apôtre, comme dit Étienne de Bourbon, marque le commencement de leur désobéissance. »} Et ils sont finalement excommuniés par le pape au concile de Vérone de 1184. Alors se développèrent chez eux les doctrines proprement hérétiques. Ils conservent une messe de rite simplifié, rompant avec la communion des saints, le Purgatoire, etc., se constituant un clergé à eux. D’autant plus attachés à la Bible qu’ils étaient infidèles à la tradition patristique, ils la firent traduire en langue romane. Pierre Valdo mourut probablement avant 1218. La secte se répandit un peu partout, surtout dans les milieux urbains. En 1230 ou 1231, un boulanger de Reims, appelé Echard, fut brûlé pour hérésie à la suite d’une condamnation par un concile provincial ; c’était, comme dit un texte et comme le prouve ce que l’on sait de sa doctrine, un disciple des \emph{« pauvres de Lyon \footnote{ HASKINS, {\itshape Robert le Bougre....} p. 442.} »}. Ils avaient essaimé jusqu’à Metz aux environs de 1200. Aussi et surtout dans les vallées du Rhône et en Italie où ils rejoignaient un mouvement issu de la Patarie. Mais il y avait eu, dès 1205, scission entre le groupe français et le groupe italien. On est mal renseigné sur les circonstances de l’émigration qui, avant le XIV\textsuperscript{e} siècle, fit se réfugier dans les vallées des Alpes un grand nombre de Vaudois venus probablement de la vallée du Rhône.
\subsection[{2° Les Cathares }]{2° Les Cathares \protect\footnotemark }
\footnotetext{Utiliser ici le dossier spécial.}
\noindent La répression du catharisme : dans le Midi, la Croisade des Albigeois, assez médiocrement terminée par la mort de Simon de  \phantomsection
\label{p104} Montfort sous les murs de Toulouse, le 25 juin 1218. Nous verrons comment elle fut reprise par la royauté et terminée politiquement par une transaction. Nous intéressent ici ses conséquences religieuses. Elle a porté un coup très dur à l’hérésie, notamment en la privant de la noblesse. Mais l’extirpation a été beaucoup plutôt l’œuvre de l’inquisition.
\section[{C. L’inquisition}]{C. L’inquisition}\phantomsection
\label{c10c}

\begin{listalpha}[itemsep=0pt,]
\item[]\listhead{Il faut distinguer :}
\item La répression violente de l’hérésie.
\item L’organisation à cette fin de tribunaux spéciaux.
\end{listalpha}

\noindent \labelchar{a)} La répression de l’hérésie par la violence est au début du XIII\textsuperscript{e} siècle. Elle répond au sentiment commun. Le roman de Floire et Blancheflor, à la fin du XII\textsuperscript{e} siècle, se termine par la conversion de Floire, roi jusque là païen en Espagne. Il fait baptiser ses sujets. Mais :\par


\begin{verse}
« Qui bauptizier ne se voloit\\
Ne en Dieu croire ne voloit\\
Floires le fesoit detrenchier\\
Ardoir en feu ou escorchier \footnote{{\itshape Floire et Blancheflor}, édité par Margaret Pelan, Paris, 1937, v. 3020-3023.}. »\\
\end{verse}

\noindent Et Beaumanoir, à la fin du XIII\textsuperscript{e} siècle (§ 833), écrit :\par

\begin{quoteblock}
\noindent « Qui erre contre la foi comme en mescreance de laquele il ne veut venir a vire de vérité … il doit estre ars et forfet tout le sien. »\end{quoteblock}

\noindent Les lois impériales romaines en donnaient l’exemple (lois de Justinien contre les Manichéens). L’Église avait été quelque temps hésitante devant des violences, qui répondaient plutôt au sentiment populaire. Mais, au XII\textsuperscript{e} siècle, les synodes et conciles s’étaient prononcés en faveur de la répression par la force, pour laquelle l’appui du pouvoir laïque était matériellement nécessaire.\par
\bigbreak
\noindent \labelchar{b)} Seulement, restaient la recherche et le jugement. Seuls, naturellement, des tribunaux d’Église avaient la compétence nécessaire. Le tribunal régulier, en pareil cas, était celui de l’évêque. À lui avait toujours appartenu, en effet, le jugement des causes de foi ; et c’est par lui qu’Innocent III s’applique à organiser la poursuite de l’hérésie.\par
À partir de 1227, Grégoire IX, le grand organisateur, en tout domaine, de la résistance, commença à confier la charge de procéder contre l’hérésie à des commissaires spéciaux, désignés par lettres apostoliques et qui, eux-mêmes, pouvaient se désigner des adjoints. Il semble s’être agi d’abord de mesures de circonstances. Mais très rapidement, en gros de 1231 à 1233, la pratique se généralisa. Et des circonscriptions régulières se créèrent. Parmi les commissaires désignés par la première lettre en 1227, pour l’Allemagne, avait été un prêtre séculier, le fameux Conrad de Marbourg. Mais, dès 1231, une autre lettre fit figurer à côté de lui certains prieurs de couvents dominicains. L’habitude se prit très vite de confier exclusivement les commissions à des membres de l’un des deux ordres mendiants, prêcheurs ou mineurs. Grégoire IX, qui avait contribué à les organiser dans le sein de l’Église, les aimait et les sentait en mains. J’ai dit des deux ordres : rien de plus faux en effet, que de croire la recherche de l’hérésie uniquement entre les mains de ceux que,  \phantomsection
\label{p105} dès le XIII\textsuperscript{e} siècle, on appelait \emph{Domini canes}. Il y a toujours des circonscriptions franciscaines. Mais les Dominicains, fondés, comme nous le verrons, dans le Midi de la France, pour prêcher contre les Cathares, étaient malgré tout les missionnaires les mieux désignés ; les provinces les plus nombreuses leur furent confiées : de bonne heure, en particulier, toute la France. La hiérarchie des deux ordres fut mise au service de la répression. Les généraux des Dominicains, en 1246, reçurent le droit de révoquer les commissaires. Ces commissaires employaient une procédure, dont nous verrons tout à l’heure la nature, qu’Innocent III avait inaugurée et que l’on appelait \emph{inquisitio}. D’où le nom sous lequel on les désignait \emph{inquisitores hereticae pravitatis} : Inquisiteurs.\par
Donc l’inquisition (au sens habituel du mot) a été une manifestation de la centralisation de l’Église. Mais pour s’exercer, il lui fallait l’autorisation et l’appui du pouvoir séculier. Ce ne lui fut pas accordé dans tous les pays. L’Inquisition n’a jamais pénétré en Angleterre, ce qui eut les plus grandes conséquences sur les procédures pénales anglaises en général. L’hérésie était poursuivie par les tribunaux ecclésiastiques ordinaires, punie parfois, sur leur requête, par les tribunaux laïques : le tout mal organisé. Mais c’est que l’hérésie type — le Catharisme — y était peu répandue. En France, la situation était différente. En outre, le roi, qui fut le contemporain des premiers décrets sur l’Inquisition, était saint Louis, si préoccupé dans toute sa législation (sur le blasphème par exemple) de mettre son pouvoir au service de l’orthodoxie, le roi qui disait : \emph{« le laïque quand il entend médire de la foi chrétienne, ne la doit défendre que de l’épée, dont il doit donner dans le ventre, tout comme elle y peut entrer »}. Il n’est pas indifférent de savoir que, sur son Psautier, figure la fête du grand martyr de l’inquisition, le dominicain italien Pierre de Vérone (ou Pierre Martyr), qui fut canonisé, en 1253. Les rois, ses successeurs, ne furent pas moins pieux. L’Inquisition fonctionna donc dans le royaume avec le plein appui du pouvoir royal. Bien mieux, dans une large mesure, à ses frais. Saint Louis déjà paie les dépenses des Inquisiteurs. Plus tard Bernard Gui, inquisiteur à Toulouse sous Philippe le Bel, recevait du budget royal un traitement annuel. Nous verrons que le roi n’y perdait pas.\par
Comment fonctionnait l’Inquisition ? La procédure est celle qui est dite \emph{inquisitoriale}, organisée par les cours ecclésiastiques, en général par Innocent III depuis 1199, codifiée au concile de Latran de 1215. Elle fut peu à peu adoptée depuis saint Louis par les cours royales. Elle remplaçait la procédure antérieure dite \emph{accusatoire}. Dans la procédure accusatoire, le juge ne pouvait entamer le procès que sur plainte. La procédure inquisitoriale lui confère, au contraire, un rôle actif. Une dénonciation, même anonyme, la rumeur publique, ses simples soupçons lui permettent d’entamer \emph{ex officio} une enquête contradictoire avec le prévenu. Elle fut d’ailleurs aggravée en matière d’hérésie par l’interdiction des avocats, le secret gardé sur le nom des témoins. L’emploi de la torture fut admis, en fait, au moins depuis 1250 et plus tard sanctionné en droit. Bernard Gui recommandait aussi les jeûnes rigoureux — en prison — favorables aux aveux. L’accusé pouvait être arrêté à la suite d’une requête adressée au pouvoir séculier.\par
Supposons la condamnation. Quelles étaient les peines ? Il faut distinguer deux cas : l’hérétique a abjuré ; ou bien, il demeure  \phantomsection
\label{p106} obstiné dans sa croyance ou encore (ce qui revient au même), ayant abjuré, il est relaps. Pour le premier cas, il y avait quelques peines relativement douces, tels que des pèlerinages imposés, ou le port (très redouté) de signes d’infamie, sur le vêtement, cela surtout lorsque le prévenu avait spontanément avoué et dénoncé. Mais la peine la plus fréquente était la prison à perpétuité : \emph{mur large} et surtout \emph{mur étroit}, cachot, fers aux mains et aux pieds, \emph{pain de douleur et eau de tribulation}. Le deuxième, c’était la peine de mort. Mais les tribunaux d’Église ne pouvaient pas l’infliger eux-mêmes. Ils les abandonnaient au bras séculier. Même, afin de préserver la fiction, c’était en priant \emph{affectueusement} les autorités séculières de n’infliger ni mort ni mutilation. Mais Gui, qui cite cette formule, dit ailleurs que, si la cour séculière n’avait pas puni de mort l’impénitent ou le relaps, elle se serait exposée à l’excommunication, comme favorisant l’hérésie. La peine de mort normale était le feu. Non pas pour épargner le feu de l’Enfer, comme l’a dit Victor Hugo, Torquemada, I, 6 :\par


\begin{verse}
« L’Enfer d’une heure annule un bûcher éternel.\\
Le péché brûle avec le vil haillon charnel,\\
Et l’âme sort, splendide et pure, de la flamme,\\
Car l’eau lave le corps, mais le feu lave l’âme. \footnote{ Œuvres complètes de V. HUGO, {\itshape Théâtre}, t. IV, Paris, Albin Michel, 1933, p. 35.} »\\
\end{verse}

\noindent C’était, au contraire, pour le préfigurer. Dans un sermon relatif au boulanger de Reims, Echard, brûlé en 1230 ou 1231, le chancelier de l’Université de Paris, Philippe de Grève, dit : \emph{« Translatus est ad furnum temporalis poenae et deinceps ad furnum gehennae »}.\par
Le corps, d’ailleurs, n’était pas seul atteint. Lorsqu’un hérétique était condamné à la prison et livré au bras séculier, ses biens étaient confisqués. En théorie au profit du seigneur. Pratiquement, très souvent à tort, on a confisqué — au profit du roi — légalement dans les sénéchaussées du Midi, qui entretenaient un percepteur spécial, dit \emph{receveur des encours}. Habitude dangereuse qui, sous une royauté obéie, tendait à faire de la condamnation pour hérésie un procédé fiscal. C’est l’origine de mesures comme le procès des Templiers.\par
Pratiquement, l’Inquisition fonctionne avec beaucoup de rigueur. Dans le Nord de la France, le premier inquisiteur fut le dominicain Robert le Petit, dit — parce qu’il avait été lui-même manichéen — Robert le Bougre. Son activité commença vers 1233 ; mais surtout depuis 1235. Il était en très bons termes avec le pape et saint Louis. Des sergents royaux l’accompagnaient. Le grand bûcher, allumé en mai 1239 à Mont-Aimé près de Provins, est resté célèbre : les sources évaluent le nombre de ceux qui y furent brûlés d’un coup, de 183 à 187 : l’autodafé eut lieu en présence du comte de Champagne, roi de Navarre, Thibaut IV, le doux chansonnier. Mais Robert, comme il arriva en même temps à Conrad de Marbourg, semble avoir tourné à la manie homicide. Il fut révoqué peu après 1239, non bien entendu pour avoir condamné des hérétiques, mais pour avoir condamné des innocents, en leur extorquant des aveux, par magie disaient les uns, par intimidation disaient d’autres. Il fut même, semble-t-il, un moment mis en prison par les chefs de son ordre, puis obtint la permission de changer d’ordre et mena une vie  \phantomsection
\label{p107} errante et misérable. Après lui, d’ailleurs, les rigueurs, moins indistinctes, continuèrent. Il ne semble pas qu’elles aient détruit entièrement l’hérésie. Mais elles l’ont certainement désorganisée, remise sous sa forme d’anti-Église et réduite à quelques foyers mal reliés, sans doctrine fixe.\par
Mais c’est dans le Midi qu’il faut surtout l’étudier. Là l’hérésie était surtout forte. Elle se doublait de la haine de l’étranger : à la suite de la croisade ; de l’annexion du Bas-Languedoc à la Couronne (1229) ; de la prise de possession du Toulousain par un prince français, Alfonse de Poitiers (1249), auquel succède, en 1270, le roi de France. En 1273, un homme traduit devant le tribunal de l’Inquisition était accusé d’avoir dit : \emph{« Quod multum sibi displicebat dominium Gallorum et quia clerici et gallici unum erant. »}\par
La rigueur était extrême. Elle souleva des révoltes, mi-religieuses, mi-nationales. En 1240, Raimond Trencavel, fils du dernier vicomte de Béziers, dépossédé par la croisade, tenta un coup de main sur le pays de Carcassonne et eut l’appui des habitants. En mai 1242, plusieurs inquisiteurs furent massacrés par la foule à Avignonet. À plusieurs reprises, sous Philippe III et Philippe le Bel, les villes du Midi protestèrent auprès du roi ou du pape contre les abus de l’Inquisition. Le pape sévit, en 1286 ou 1287, contre certains d’entre eux. Puis, la tête du mouvement fut prise depuis 1296 par un franciscain, originaire de Montpellier et lecteur du couvent de Carcassonne, puis de Narbonne, Bernard Délicieux \footnote{ Michel DE DMITREWSKI, {\itshape Frère Bernard Délicieux}, dans {\itshape Archivum franciscanum historicum}, 1924 et 1925.}. Il s’adressa, en 1301, à deux enquêteurs royaux envoyés dans le Toulousain, fut conduit par eux auprès du roi à Senlis. L’évêque d’Albi, coupable d’avoir favorisé des procédés fâcheux, fut frappé d’amende ; le roi exigea de l’ordre dominicain la révocation de l’inquisiteur de Toulouse. Mais l’agitation ne fut pas apaisée, notamment à Carcassonne. Bernard y prononça des discours incendiaires, avec appel sous forme de paraboles à la violence. Une milice urbaine fut organisée, pour se protéger contre les Inquisiteurs. L’enquêteur royal, revenu dans le pays, transféra les prisonniers de l’Inquisition de Carcassonne dans les cachots du roi. De l’argent fut recueilli de ville en ville pour poursuivre la lutte. Puis le roi arriva à Toulouse, en 1303, et visita ensuite Carcassonne. Le bruit, la hardiesse des plaintes semblent avoir déplu à un prince naturellement très froid et très pieux. Aussi bien, après Agnani, 7 septembre 1303, il cherchait l’accord avec la curie. Un complot se forma alors pour offrir le Languedoc à l’infant Ferrand, fils du roi de Majorque, qui était le seigneur de Montpellier, ville natale de Bernard. Bernard l’alla trouver. Mais le roi Jacques, qui redoutait Philippe, eut vent de la chose, brutalisa son fils et chassa les envoyés. Le pape Benoît XI lança un mandat d’arrêt contre Bernard. Celui-ci se rendit à Paris et fut arrêté. Les conspirateurs carcassonnais furent pendus. Mis en liberté par Clément V, Bernard se rallia aux Spirituels. Son procès fut repris sous Jean XXII, en 1317. Il fut condamné au mur en 1319, à Carcassonne, et mourut peu après. Une à une les villes méridionales avaient demandé et obtenu la \emph{réconciliation} avec l’Inquisition.\par
 \phantomsection
\label{p108}
\begin{enumerate}[itemsep=0pt,]
\item[]\listhead{Quels résultats ? Il faut distinguer les étapes :}
\item La croisade a eu pour résultat de priver l’hérésie du soutien, au moins ouvert, des grandes maisons baronales.
\item Depuis que commence, en 1233, l’œuvre de l’Inquisition et qu’elle s’intensifie après le massacre de 1242, elle frappe à coups redoublés sur toutes les classes sociales.
\end{enumerate}

\noindent Elle a un double résultat : désorganiser l’église cathare, les parfaits étant les principales victimes. \emph{« Il y a deux églises, dira vers la fin du siècle le parfait Pierre Autier : l’une qui fait et pardonne, l’autre qui tient et écorche ; la première est celle qui demeure fidèle à la voie des Apôtres ; la seconde est l’Église romaine »}. En 1273 ou 1274, un grand exode des parfaits et de quelques croyants a lieu vers la Lombardie, notamment l’évêque cathare de Toulouse fuit à Sermione, un autre évêque à Pavie. Les croyants vont là-bas recevoir le {\itshape consolamentum}, envoient des subsides à leurs chefs exilés. De là-bas s’organise une tournée de parfaits en Languedoc. Mais les croyants languedociens gémissent de ne pas avoir de parfaits. C’est la vie d’une église du Désert. Un texte nous décrit les réunions des {\itshape Bonshommes} dans quelques cabanes d’un bois, près de Palajac. Et voici les pérégrinations de Pierre Autier \footnote{ J.-M. VIDAL, {\itshape Les derniers ministres de l’Albigéisme en Languedoc. Leurs doctrines}, dans \href{http://gallica.bnf.fr/document?O=N016999}{\dotuline{{\itshape Revue des Questions historiques} [http://gallica.bnf.fr/document?O=N016999]}}, 1906, p. 61-107.} : il était d’Ax (les Thermes) et de famille cathare, assez riche. Lui-même notaire. Vers 1296, il alla en Lombardie où, à Côme, il se fit ordonner parfait, ainsi que son frère. Ses biens furent confisqués par le comte de Foix. En 1298 ou 1299, il reparut au pays et y erra, distribuant les bénédictions et le {\itshape consolamentum.} On disait qu’il l’avait donné, sur son lit de mort, au comte de Foix lui-même. Un épisode donne une idée de cette vie. Un béguin l’avait dénoncé aux Prêcheurs de Pamiers, s’offrant à se faire passer pour son disciple et le livrer. Mais un des dominicains de Pamiers était le propre neveu de Pierre et, paraît-il, lui servait d’espion. Il dénonça le béguin. Deux croyants se saisirent une nuit de ce traître, sur un pont, et allèrent le jeter dans un gouffre de la montagne. Mais en 1305, Pierre pénétra jusqu’à Toulouse et même, dit-on, réunit quelques croyants dans une église. Une trahison le livra, en 1309, à l’Inquisition. Pierre mourut sur le bûcher en 1311, s’écriant que s’il lui était loisible de prêcher, tout le peuple se convertirait à sa foi. La plupart de ses compagnons, avant ou après lui, furent brûlés.\par
Or, conséquence : une église sans enseignement, où s’exagère la part des fictions légendaires. Et, parce qu’elle n’a pas d’enseignement, mal protégée.\par
L’Inquisition frappe tout le monde, mais elle rejette dans le prolétariat, par les confiscations, les riches. Et par là même aussi — et parce que l’hérétique a la vie difficile — elle compromet d’abord la fidélité des classes riches. La noblesse, à partir de 1250, donne (comme pour la Réforme au XVII\textsuperscript{e} siècle) le signal de la fuite. Puis, beaucoup plus lentement, la bourgeoisie ; le catharisme finit comme une religion de pauvres.\par
Finit quand ? On le sait mal, mais il semble bien qu’en France, au moment où s’éteint la lignée des Capétiens directs, il ait à peu près disparu. Lea a écrit : \emph{« S’il était vrai que le sang des martyrs est la semence de l’Église, le Manichéisme serait aujourd’hui la religion dominante de l’Europe. »} Mais cela n’est pas vrai.
 \phantomsection
\label{p109}\section[{D. La vie religieuse}]{D. La vie religieuse}\phantomsection
\label{c10d}
\subsection[{1° Comment le laïque connaît sa religion}]{1° Comment le laïque connaît sa religion}

\begin{listalpha}[itemsep=0pt,]
\item \textbf{Le clergé paroissial (chapelains, confesseurs)}. Il est médiocre. Voyez par exemple le fameux livre des visitations épiscopales d’Eudes Rigaud (1248-1269) \footnote{ Voir L. DELISLE, {\itshape Le clergé français au XIIIe siècle, d’après le Journal des visites pastorales d’Eudes Rigaud}, dans {\itshape Bibl. Éc. Chartes}, t. VIII, 1846, p. 479-499.}. Médiocre par les mœurs. La règle du célibat est mal observée. Au point que, dans un mémoire présenté au Concile de Vienne de 1312 par l’évêque de Mende, Guillaume Durant le jeune, celui-ci propose d’en venir au système de l’Église d’Orient : prêtres paroissiaux mariés. Il est ignorant. Son autorité, en outre, est souvent peu aimée du paysan : parce qu’on l’accuse de rapacité. Prêtre, sergent seigneurial, meunier, les trois ennemis : voir les fabliaux. D’où cela vient-il ? Mauvais système de nomination, le patronat ; les cures données en bénéfice, avec vicaire ; les cures monastiques. La mauvaise formation.
\item \textbf{La prédication} : c’est le grand fait nouveau du XIII\textsuperscript{e} siècle. Par les évêques, les séculiers, surtout les réguliers, notamment les nouveaux ordres mendiants. Le type du sermon : les {\itshape Exempla ;} saint Louis à son fils : \emph{« Écoute volontiers parler de Notre Seigneur en sermons »} (et en privé).
\item \textbf{L’iconographie}. Importance à cet égard du développement artistique du XIII\textsuperscript{e}  siècle.
\item Pour ceux qui lisent, \textbf{les livres de piété à l’usage des laïques}. Pas absolument séparable de la catégorie précédente. Joinville, à Acre (1250-1251), pour faire de la propagande dans son entourage, fait faire un petit livre, exposé de la foi catholique \footnote{{\itshape Le Credo de Joinville}, éd. N. de Wailly, p. 508-537.}. Littérature de piété très abondante \footnote{ Voir Ch. V. LANGLOIS, {\itshape La vie en France au Moyen âge du milieu du XIIIe au milieu du XIVe. Enseignements, méditations et controverses. La vie spirituelle d’après des écrits en français à l’usage des laïcs}, Paris, 1928.}. Prise de conscience religieuse, favorable d’ailleurs à l’hérésie.
\end{listalpha}

\subsection[{2° Ce qu’il y a de pieux et de vivant dans la religion}]{2° Ce qu’il y a de pieux et de vivant dans la religion}
\noindent Problème difficile. À noter : la préoccupation du diable ; l’Enfer et la vie de l’autre Monde. D’où les Indulgences. Saint Louis à son fils : \emph{« Recherche volontiers les pardons. »} La Vierge, culte des saints, d’où les fausses reliques et les statues miraculeuses truquées, dénoncées par Guillaume de Digulleville, dans le {\itshape Pèlerinage de la Vie humaine}, entre 1330 et 1332. Le Christ et le culte de la Croix.
\subsection[{3° Le rôle du prêtre}]{3° Le rôle du prêtre}
\noindent \emph{« Cher fils, dit saint Louis, je t’enseigne que tu t’accoutumes à te confesser souvent. »} La même règle se retrouve dans deux traités d’édification à l’usage des laïcs. La confession.
\section[{E. L’ascétisme orthodoxe}]{E. L’ascétisme orthodoxe}\phantomsection
\label{c10e}
\noindent \labelchar{1)} {\itshape Les ordres anciens} se rattachent tous soit à la règle de saint Benoît, ou telle quelle, ou sous la forme réformée des ordres érémitiques de la fin du XII\textsuperscript{e} siècle : Citeaux, Chartreux ; soit à la règle dite de saint Augustin (chanoines réguliers), sous sa forme primitive, ou sous la forme que lui ont donnée en 1121 les Prémontrés ;  \phantomsection
\label{p110} soit aux règles, apparentées entre elles, des ordres militaires Temple, Hôpital.\par
Ces institutions, malgré le {\itshape rush} vers les Ordres mendiants, sont encore extrêmement vivantes, religieusement et socialement. Si d’ailleurs on laisse de côté les ordres militaires, les autres groupements bénédictins et même augustins tendent à effacer leurs distinctions : la pratique de Cîteaux tend à se rapprocher des autres pratiques bénédictines ; les monastères bénédictins isolés tendent, à l’imitation de Cîteaux et de Cluny, à s’unir entre eux par l’institution de chapitres communs.\par

\begin{itemize}[itemsep=0pt,]
\item[]\listhead{Quelques grands faits :}
\item Monastères et prieurés ;
\item L’abbé, sa fortune distincte de celle des moines ;
\item L’élection de l’abbé : l’intrusion des pouvoirs laïques et de la papauté ;
\item La fortune des monastères : les terres et droits seigneuriaux ; les dîmes ; les églises paroissiales. L’administration de la fortune ; les \emph{offices} ; rédaction des cartulaires ; les placements en capitaux ; l’endettement ;
\item La fonction du monastère. Essentiellement, la prière et la liturgie. Le monastère, centre d’études (mais beaucoup moins d’écoles). Les moines au service de l’État : exemple, l’abbé de Saint-Denis, Mathieu de Vendôme, régent (avec un laïque, Simon de Nesle) durant la croisade de saint Louis, puis (seul ? ) durant celle d’Aragon. \emph{« Il régnait en France, dit un annaliste normand, tout se faisait à sa volonté. »}
\end{itemize}

\noindent Aux monastères, on peut joindre les chapitres séculiers : même type de fonction, même règle de maison de prières et de liturgie ; mais point de vœux monastiques pour les chanoines, et absentéisme fréquent de ces derniers, qui fait d’eux les serviteurs types de la royauté. Un juif converti, Philippe le Convers, qui, filleul de Philippe le Bel, fut un des bons administrateurs des derniers Capétiens, était archidiacre en Brie en l’église de Meaux, archidiacre d’Eu en celle de Rouen, trésorier de Reims et Troyes, chanoine de Paris, Tournai et Notre-Dame de Mantes : le tout peut-être pas à la fois, mais certainement plusieurs dignités ensemble.\par
Mais forme monastique nouvelle.\par
\bigbreak
\noindent \labelchar{2)} {\itshape Les ordres mendiants.} Laissons les petits ordres mendiants (Carmes, constitués en ordre monastique en 1247, Ermites augustins constitués en ordre monastique en 1256). Entrent en ligne de compte, avant tout, Mineurs et Prêcheurs. Rappel de leur histoire primitive.\par

\begin{itemize}[itemsep=0pt,]
\item \textbf{Les Mineurs} sont nés d’un groupe de pénitents surtout laïques réunis, à partir de 1209, autour du fils d’un marchand d’Assise. L’objet était de mener une vie apostolique, soit pauvreté et prêche. Intelligemment reconnue par Innocent III en 1210, adroitement incorporée aux institutions de l’Église grâce surtout à l’action du cardinal Hugolin d’Ostie, le futur Grégoire IX, l’association des {\itshape poenitentes in Assisia nati} s’est, au moment où mourut saint François (3 octobre 1226), véritablement transformée en ordre, avec une organisation administrative, une règle approuvée par la Papauté, un noviciat régulier.
\item \textbf{Les Prêcheurs} ont eu pour fondateur un chanoine castillan, Dominique. Mais ils ont été fondés dans le Midi de la France.  \phantomsection
\label{p111} L’origine est un petit groupe de prédicateurs, formé vers 1206 autour de Dominique pour prêcher l’orthodoxie, et fixé à Toulouse en 1215. La première règle fut celle de saint Augustin. En 1217, Dominique quitte le Languedoc et disperse ses \emph{Frères Prêcheurs} a travers le monde. Lorsqu’il meurt à Bologne, le 6 août 1221, la règle de saint Augustin a été remplacée par une règle nouvelle, qui en demeure inspirée, mais en y ajoutant, semble-t-il, la pauvreté, probablement à l’imitation des Franciscains (nous n’avons d’ailleurs pas cette règle ; les plus anciennes constitutions connues sont celles de 1228).
\end{itemize}


\begin{itemize}[itemsep=0pt,]
\item[]\listhead{Qu’apportaient de neuf ces deux ordres ?}
\item Le principe de \textbf{pauvreté} : sa signification, son abaissement nécessaire. Il en reste du moins une constitution de fortune différente, où la rente seigneuriale tient peu de place (sujet mal connu).
\item Une \textbf{concentration} beaucoup plus grande et par suite un instrument beaucoup plus en mains d’une curie pontificale, qui aspire à dominer une Église centralisée et unifiée.
\item La rupture avec l’idée du moine loin du siècle. Les \textbf{ordres urbains}. La \textbf{prédication}. L’administration des sacrements et notamment la confession. Les premiers établissements fondés sont extra urbains. Puis ils entrèrent en ville (voir ci-dessous, Paris). Texte relatif à Ecouges. Ils parent à l’insuffisance du clergé séculier.
\end{itemize}

\noindent La différence entre les deux ordres. Les Franciscains sont venus plus tard et avec plus d’hésitation au travail intellectuel \footnote{Voir texte de saint Bonaventure, général en 1257, mort en 1274.}. Courant mystique et pieusement anarchique.\par
Le premier établissement en France. Il remontait à la fin du règne de Philippe Auguste. Les Dominicains avaient leur patrie dans le Midi. La première colonie dominicaine se fixa à Paris en 1217 ; d’abord dans une maison voisine de Notre-Dame, puis dans un Hôtel-Dieu qui leur fut donné, près de la porte d’Orléans et qui était dédié à saint Jacques (parce qu’il était sur la route de Compostelle). On les appela \emph{Jacobins}. Peu à peu d’autres groupes s’établirent dans les principales villes. En 1219, les frères mineurs partent en même temps pour la Gascogne et pour Paris, ou plus exactement pour Saint-Denis. La première maison qu’ils édifièrent dans les environs de Paris fut détruite sur l’ordre de saint François. Mais peu après 1229, un nouveau couvent, Vauvert, fut édifié dans le Bourg Saint-Germain : Cordeliers (École de Médecine). Le succès fut grand, notamment dans les milieux d’études. Un prédicateur du XIII\textsuperscript{e} siècle écrit : \emph{« Les parents disent aux maîtres de leurs enfants, quand ils les envoient à Paris : ne permettez pas à nos enfants d’aller vers les Frères Prêcheurs ou Mineurs parce que ce sont des larrons et qu’ils les raviraient aussitôt »}. Beaucoup d’évêques sortirent des Mendiants : par exemple Eudes Rigaud. Texte sur saint Louis \footnote{ Voir M. POÈTE, {\itshape Paris, de sa naissance à nos jours}, t. I, Paris, 1924, p. 180.}.\par

\labelblock{Les Tiers Ordres.}

\noindent Néanmoins, pas sans troubles, ni habileté (extr. dans Rutebeuf) : querelle universitaire que nous verrons plus tard. Affaire des Spirituels. Le mouvement est double : facilité à la pauvreté collective, à l’ascétisme strict, y compris l’ascétisme intellectuel. Refus de se mêler aux pratiques de l’Église possédante : aliénation  \phantomsection
\label{p112} économique des reliques ; argent pris pour les sépultures etc., ni même aux constitutions ordinaires des autres ordres monastiques. La Papauté ayant en général pris le parti des conventuels, attitude d’opposition. L’autonomie. Jean Olevi (un Languedocien). Le Pape n’a pas le pouvoir d’accorder des dispenses ou absolutions (à la règle). Encore moins peut-il autoriser quelques pratiques contraires à la règle. Bulle de Jean XXII, 13 avril 1317 : \emph{« la pauvreté est une grande chose, mais plus grande encore est l’innocence, et l’obéissance parfaite est la plus grande de toutes »}.\par
Le courant Joachimite : Joachim, abbé du monastère cistercien de Corozzo en Calabre, puis fondateur du monastère de Saint-Jean-de-Flore, mort en 1302. Ses œuvres principales ont été réunies, avec une introduction, sous le titre de l’{\itshape Évangile éternel}, vers 1252 par un franciscain (probablement Gérard di Borgo S. Donnino), condamné en 1255 après l’attaque de Guillaume de Saint-Amour (voir plus loin). Messianisme : règne du Saint-Esprit.\par
C’est vers la fin du règne de saint Louis que ce mouvement complexe et assez trouble commença à agiter l’Église de France. À peu après contemporain de la publication à Paris de \emph{l’Évangile Éternel}, a lieu en Provence, dans le Midi languedocien la prédication de Hugues de Digne, ministre provincial des Mineurs, qui devait devenir un des saints du mouvement spirituel, uni dans cette sorte de canonisation non officielle à sa sœur Douceline, simple Tertiaire. Joinville qui l’avait connu, dit que (mort en 1284) \emph{« il gest en la citée de Marseilles, là où il fait moult belles miracles »}. Autour des frères spirituels se forment de petites communautés de Tiers Ordre, qu’on appelle Béguins. L’origine du mot est obscure. Il se rapproche de celui de béguines donné en Flandre à des femmes qui, sans entrer en religion, vivaient pieusement en communauté. Ils ont surtout été répandus dans le Midi. Leur chef fut un languedocien mineur du couvent de Béziers, Jean-Pierre Olive ou Olevi mort en 1298). Longtemps on hésita à les persécuter. Leurs œuvres étaient condamnées, non leur personne. Jean-Pierre Olive, comme H. de Digne, est mort paisiblement. Mais l’attitude changea lors de l’avènement à la Papauté de Jean XXII et, presque simultanément, au généralat de Michel de Césène (1314). Le bûcher des quatre franciscains \emph{martyrs}, livrés au bras séculier par l’Inquisition de Marseille (7 mai 1318), fut suivi dans le Languedoc de beaucoup d’autres. Bernard Gui les tient pour hérétiques et leur consacre un long développement. Mais l’agitation et le ferment persista. Remuement des masses par tout cela.
\section[{F. Les Institutions de l’Église}]{F. Les Institutions de l’Église}\phantomsection
\label{c10f}

\begin{enumerate}[itemsep=0pt,]
\item La paroisse. Rappel des procédés de nomination.
\item L’évêque. Son rôle. La juridiction de l’ordinaire (l’official). Nomination des évêques, l’élection par le chapitre ; le rôle du roi ; et du pape (indications).
\item La Papauté. Les faits nouveaux. Centralisation. Bureaucratie. La Papauté avignonnaise \footnote{ Voir G. MOLLAT, {\itshape Les papes d’Avignon (1305-1370)}, 6\textsuperscript{e} éd., 1930.}.
\end{enumerate}

\noindent Élu au conclave en 1305 comme successeur de Benoît XI, Clément V ne se rendit jamais en Italie, extrêmement troublée, et après  \phantomsection
\label{p113} avoir erré de ville en ville, en France et dans le royaume d’Arles, se fixa en 1308 à Avignon, ou plus exactement tantôt à Avignon, tantôt dans les châteaux du Comtat-Venaissin. Le Comtat, depuis 1271, était terre papale. La ville d’Avignon appartenait à la maison d’Anjou (elle ne fut achetée qu’en 1348 par Clément VI).\par
Il serait inexact de se représenter à ce moment et après ce moment une papauté toute française. Mais certainement influence française très forte. Dès le XIII\textsuperscript{e} siècle. Jusqu’à la mort d’Alexandre IV (1261), tous les papes sont italiens. D’Urbain IV à Benoît XI (1261-1304), sur 13 papes, on compte trois français. Depuis Clément V jusqu’à Grégoire XI (1305-1378), ils sont tous français, à l’exception du provençal Urbain V, sujet des Angevins, princes français.\par

\begin{enumerate}[itemsep=0pt,]
\item[]\listhead{Comment se marque la centralisation ?}
\item Centralisation de la justice.
\item Les impôts. Ce sont essentiellement les décimes, nées à l’origine des impôts de croisade, levées par Grégoire IX à l’occasion de la lutte contre Frédéric II (1225), parfois partagées entre le Pape et le roi. Il s’y ajoute au début du XIV\textsuperscript{e} siècle des subsides. En vérité toutes sortes de taxes de chancellerie ou afférentes aux nominations.
\item Les nominations. Réserve générale : sous Clément IV, une décrétale du 27 août 1265 donne au pape la pleine disposition des bénéfices des mourants à la Curie. Boniface VIII étend la mesure à la distance de deux journées de marche. Clément V et Jean XXII étendent aux bénéfices des cardinaux, à ceux qui seraient vacants par disparition ou résignation etc. Désir de pouvoir et de profit ; hostilité à l’élection ; accrue avec les rois (Philippe le Bel faisait fournir au pape des prébendes à ses serviteurs).
\end{enumerate}

\section[{G. Les attitudes vis-à-vis de l’Église}]{G. Les attitudes vis-à-vis de l’Église}\phantomsection
\label{c10g}
\noindent Le problème de l’Église possédante. Il est double : — religieux, nous avons vu — politique. Attitude familière des laïques. L’hostilité à la rapacité cléricale. S’ajoute dans le clergé français l’hostilité à la rapacité de la Curie. Il se pose sous le triple aspect — finance — justice — nomination. Des ligues de la noblesse contre les empiètements du clergé ont été formées à plusieurs reprises sous saint Louis, plus ou moins inspirées par les évêques de l’Empire, tolérées par le roi de France. Des membres du clergé de France et le roi ont protesté sous saint Louis contre les abus de Rome. Mais, vers la fin de son règne, saint Louis ne proteste plus, parce qu’il avait besoin du pape pour lever les décimes royales (décret de Latran de 1265 : l’autorisation pontificale est nécessaire aux rois pour imposer les églises) et aussi en raison de ses sentiments de plus en plus religieux. Mais l’hostilité des agents royaux et seigneuriaux contre les justices d’Église et du clergé contre les exactions se poursuit. C’est l’arrière-plan du règne de Philippe le Bel. Elle correspond à ce que nous avons vu — une société croyante de plus en plus consciente de sa foi, mais fort peu encline à la théocratie.
\chapterclose


\chapteropen
\chapter[{11. La vie intellectuelle}]{\textsc{11. }La vie intellectuelle}\phantomsection
\label{c11}\renewcommand{\leftmark}{\textsc{11. }La vie intellectuelle}


\chaptercont
\section[{A. La mentalité}]{A. La mentalité}\phantomsection
\label{c11a}
\noindent  \phantomsection
\label{p115} Comment la saisir ?\par

\begin{enumerate}[itemsep=\baselineskip,]
\item  L’art. Difficulté de l’étudier. Date des monuments. Quelques faits de civilisation pourtant se dégagent.\par
 L’héritage carolingien. La basilique couverte en charpentes. Les innovations : le type à double abside, parfois à double transept. Clochers accolés à l’édifice et souvent au chevet. Le déambulatoire à chapelles rayonnantes. Quelques plans à coupole (Aix-la-Chapelle, Germigny-les-Prés. Voir en Alsace, mais 1045 seulement, Ottmarsheim).\par
 La décoration intérieure : stucs, mosaïques, peintures. Pas de grande sculpture décorative.\par
 La continuation de l’héritage carolingien (notamment en Allemagne). Mais les courants nouveaux.\par
 Le groupe catalan (avec influences mozarabes). Ares outrepassés. Mais surtout couverture des nefs en voûtes. Remarquer l’influence sur l’iconographie mozarabe (illustration au X\textsuperscript{e} siècle du commentaire de Béatus dans le monastère espagnol de San Miguel d’Escolada). Enfin d’origine incertaine et en tout cas très générales, les arcatures aveugles et les \emph{bandes lombardes} (bandes verticales de peu de saillie). Apparition du narthex (celui de Saint Philibert de Tournus date probablement de la fin du X\textsuperscript{e} siècle) (le bas voûté en arête, le haut en berceaux longitudinaux). Fréquence de la construction en bois.
 
\item La culture. Son caractère uniquement latin, à la seule exception de l’Angleterre. Éclipse en Allemagne de la littérature en langue vulgaire. Son insignifiance en France, (Eulalie, IX\textsuperscript{e} siècle ; Passion du Christ et Saint Léger, X\textsuperscript{e}).
\end{enumerate}

\noindent Les grandes écoles destinées à l’enseignement des clercs (et des clercs seuls). On appelle \foreign{studia generalia} celles qui confèrent des grades reconnus par l’usage, puis par décision pontificale, dans toute la chrétienté. On dit aussi de plus en plus, dans ce sens, {\itshape Universitas}, mot qui désignait à l’origine l’association des maîtres et étudiants, douée de personnalité morale.
\section[{B. Les Universités}]{B. Les Universités}\phantomsection
\label{c11b}
\subsection[{1° La formation et la constitution des Universités françaises}]{1° La formation et la constitution des Universités françaises}
\noindent L’Université de Paris. Ses origines échappent à notre période. L’importance des écoles de Paris remonte à la première moitié du XII\textsuperscript{e} siècle, aux temps d’Abélard et de Guillaume de Champeaux. C’est à cette époque que les écoles des Arts, pour échapper au chapitre, se fixent sur la rive gauche, dans le bourg Sainte-Geneviève. Les premiers privilèges de juridiction ont été accordés par Philippe Auguste en 1200. C’est en 1215 que l’enseignement est organisé par le cardinal légat Robert de Courçon ; en 1219 que nous voyons la mention d’officiers élus par les maîtres. La fin du règne de Philippe Auguste, celui de Louis VIII, le début du règne de saint Louis sont remplis par la grande querelle des maîtres avec le chancelier du chapitre : elle est marquée par la sécession, vers 1227, des maîtres en droit et en théologie sur la rive gauche. Le cardinal légat Romain de Saint-Ange, très puissant sous la régence à la cour de France, était en mauvais termes avec l’Université. Il en fit briser le sceau, en 1225, sur requête du chapitre. Une rixe entre les écoliers et les habitants du bourg Saint-Marcel amena une plainte du chapitre au légat ; celui-ci exhorta la reine à l’action. Les sergents royaux maltraitèrent les écoliers, en tuèrent plusieurs. L’Université répondit d’abord par l’arme déjà classique de la \emph{grève}, puis par celle, plus grave encore de la sécession ; elle se dispersa. Il est intéressant de voir le pape Grégoire IX s’émouvoir — c’est l’indice du rôle d’école universelle — et notamment d’école théologique — pris par les écoles parisiennes \emph{« fleuve qui irrigua et féconda le paradis de l’Église universelle »} (dit-il lui-même dans une bulle de 1229). Il rappela son légat. Le résultat fut — après réconciliation avec le gouvernement royal — la grande bulle {\itshape Parens scientiarum} de 1231.\par

\begin{itemize}[itemsep=0pt,]
\item[]\listhead{Les clauses essentielles sont les suivantes :}
\item Le chancelier ne confèrera la licence en théologie et en droit qu’après enquête auprès des maîtres. Pour les médecins et les artistes, il n’accordera la licence qu’à des personnes dignes (pour les artistes, le problème n’était pas brûlant, parce qu’ils avaient le recours de s’adresser à l’abbé de Sainte-Geneviève ou à son représentant). Pratiquement, la licence sera désormais décernée par des commissions de maîtres.
\item Droit pour les maîtres de faire des règlements touchant les heures de l’enseignement, la discipline intérieure, le prix des chambres ; et d’expulser qui n’obéira pas à ces règlements.
\item Prescriptions relatives à la juridiction de l’évêque.
\item Prescriptions d’orthodoxie caractéristiques. Les maîtres ès arts ne se serviront pas des livres interdits au concile de 1210 (nous verrons que ce sont ceux d’Aristote). Les maîtres et écoliers en théologie ne feront pas les philosophes ({\itshape nec philosophos se ostendent}), ne disputeront que des questions qui peuvent être déterminées \emph{« per libros theologicos et sanctorum patrum instrumenta »} ; ils ne parleront pas dans la \emph{langue du peuple}.
\end{itemize}

\noindent Noter que, depuis 1219, l’enseignement du droit civil a été interdit à Paris (par Honorius III).\par

\begin{itemize}[itemsep=0pt,]
\item Orléans, dont la licence a été pratiquement reconnue comme universelle depuis la seconde moitié du XIII\textsuperscript{e} siècle : droit civil et  \phantomsection
\label{p117} arts surtout. Défense à Paris : bulle d’Honorius III, 1219 ; en 1235, Grégoire IX précise qu’elle ne s’applique pas à Orléans.
\item Angers vers l’époque de la sécession, en 1229.
\item Toulouse en 1230, fondée par le pape comme centre contre les Albigeois (Grégoire IX).
\item Montpellier : l’école de médecine jointe longtemps à celles des arts était anciennement reconnue comme {\itshape Studium generale.} Au début du XIII\textsuperscript{e} siècle, les arts et la médecine se séparent, le droit s’ajoute. Et les statuts furent confirmés par la Papauté.
\end{itemize}

\subsection[{2° Reprenons l’histoire de l’Université de Paris, principal Centre d’études}]{2° Reprenons l’histoire de l’Université de Paris, principal Centre d’études}
\noindent Les traits généraux de sa constitution se dégagent sous le règne de saint Louis. L’Université est une université de maîtres (à la différence de Bologne et même d’Orléans). Elle se divise en quatre Facultés : Arts (trivium : grammaire, rhétorique, dialectique ; quadrivium : arithmétique, géométrie, astronomie, musique) ; théologie, droit canon, médecine. Elles sont hiérarchisées en ce sens qu’en règle générale on a passé par les Arts (à Paris ou ailleurs) pour être admis dans l’une des autres Facultés. Mais les maîtres ès Arts sont les plus nombreux et ont mené la lutte contre les chanceliers. Nous allons en voir les conséquences. Les Arts se divisent en quatre nations . française, normande, picarde, anglaise (la française comprend les Italiens ; l’anglaise, les Allemands). Chacun a son procureur. L’ensemble de la Faculté des Arts a à sa tête un recteur, qui est en même temps le chef de toute l’Université. Chacune des trois autres Facultés, un doyen. Les maîtres se divisent en trois catégories : bacheliers ( jeunes gens), qui ont le droit d’agir comme des espèces d’assistants ; licenciés {\itshape (licentia docendi) ;} docteurs (dans les trois facultés supérieures). Il n’y a pas de chaires pourvues de traitement. Les locaux sont ou prêtés par des institutions religieuses, ou loués par les maîtres ; à la fin du XIII\textsuperscript{e} siècle, les Nations se chargent du loyer. Les leçons se paient et il y a les bénéfices ecclésiastiques. Mais là, grande cause de supériorité des ordres et des collèges. L’enseignement se fait en général sous forme de commentaires d’ouvrages, auxquels s’ajoutent les {\itshape disputationes.}\par

\begin{enumerate}[itemsep=0pt,]
\item[]\listhead{Les degrés :}
\item l’étudiant commence par être un {\itshape scholaris} tout court ;
\item après un examen sous forme de discussion publique ({\itshape determinatio}), il devient bachelier, et peut alors donner un enseignement relativement élémentaire ;
\item la licence. Les licenciés sont dits \emph{maîtres} dans toutes les Facultés à l’exception de celle de Droit Canon où, à l’imitation de Bologne, on parle de docteurs \footnote{ Voir Charles H. HASKINS, {\itshape The Life of medieval Students as illustrated by their letters}, Madison, Wisconsin, U.S.A, t. III, 1897-1898. — {\itshape The University of Paris in the sermons of the thirteenth Century}, {\itshape Ibid}., t. X, 1904-1905.}.
\end{enumerate}

\noindent Les collèges. À l’origine hôtels pour étudiants, ainsi dispensés de graves soucis matériels et soumis à une certaine discipline. Le premier fut celui des Dix-huit, annexe de l’Hôtel-Dieu et fondé en 1180 par le chapitre de Notre-Dame. Ces établissements se multiplient au cours du siècle. Les premiers étaient pour les artistes (plus jeunes). Vers 1257, Robert de Sorbon, chapelain de saint Louis, établit un collège pour seize pauvres maîtres ès Arts, étudiants en théologie, visiblement dans le dessein de perpétuer la race des théologiens laïques. Le célèbre collège de Navarre, fondé en 1314 par la reine Jeanne, femme de Philippe le Bel, comprit des artistes et des théologiens. Ainsi l’Université se bâtissait en pierres. Les ordres monastiques avaient leurs maisons d’études. Dès ce moment, on commençait d’enseigner dans les collèges.\par

\labelblock{Les ordres mendiants.}

\noindent  \phantomsection
\label{p118} En 1229, au moment de la sécession, le chancelier de Notre-Dame avait accordé la licence au maître en théologie du couvent des Dominicains ! Elle lui fut maintenue après 1231. Un peu plus tard, un maître en théologie séculier prit l’habit dominicain, un autre, le célèbre Alexandre de Hales, celui de franciscain, tout en gardant leurs enseignements. Se multiplient : parmi eux, de 1245 à 1248, Albert le Grand. En 1250, le pape ordonne au chancelier d’accorder la licence aux religieux, qui en sembleraient dignes. Heurts : de tour d’esprit, de positions intellectuelles, de concurrence. En 1253, une \emph{cessation} fut proclamée à propos du meurtre d’un écolier par les sergents du roi. Les Mendiants refusèrent d’y participer sans l’assentiment du pape. Ils furent expulsés de la société des maîtres. Alexandre IV, ancien dominicain lui-même, en 1255, ordonna de les admettre. À ce moment, Guillaume de Saint-Amour lança son {\itshape Tractatus brevis de periculis novissimorum temporum.} Il y eut deux résultats : condamnation de Jean de Flore ; interdiction faite à Guillaume, en 1257, de résider à Paris. La lutte se poursuivit. Parmi les Mendiants, qu’en 1256 les maîtres s’étaient refusé d’accepter, se trouvent Bonaventure et Thomas. L’Université finit par admettre les Mendiants. Mais l’antagonisme dura pendant tout le siècle, avec des controverses, tantôt vives, tantôt assoupies. Les Mendiants ont contribué à tenir l’Université dans le devoir papal. Mais l’hostilité contre eux a développé un esprit \emph{gallican} dans l’Université (qui se traduit par l’attitude sous Boniface VIII).\par
La vie universitaire : l’âge (dès 14 ans pour les artistes). La turbulence. Qui sort de l’Université ?
\section[{C. Les grands problèmes intellectuels }]{C. Les grands problèmes intellectuels \protect\footnotemark }\phantomsection
\label{c11c}
\footnotetext{ Comme guide, les livres de E. GILSON, {\itshape La philosophie au Mogen âge}, 2 vol., Paris, 1922. Autres ouvrages du même auteur, plus développés, notamment : {\itshape L’Esprit de la philosophie médiévale}, 2 vol., Paris, 1932. Références rapides UEBERVEG, {\itshape Grundriss der Geschichte der Philosophie}, t. II.}
\noindent Il ne saurait être question ici d’étudier le développement de la philosophie du XIII\textsuperscript{e} siècle, qui commence à nous être relativement bien connue, ni celui des sciences : histoire qui, étroitement liée à celle des techniques, l’est au contraire fort mal. Quelques indications en ce qui touche les grands courants devront suffire.\par
Observation : comme pour la religion, différentes couches. La fréquentation du monde d’un saint Thomas n’est certainement pas exactement celle de son contemporain saint Louis et celle-ci est encore fort éloignée de celle des rustres, qui peuplaient les villages royaux. Les grands problèmes, dont je vais parler ici, ont surtout touché la première couche.\par
Le grand danger pour la foi traditionnelle provenait, au début du siècle, de l’aristotélisme, surtout sous la forme de l’averroïsme.\par
 \phantomsection
\label{p119} Il faut nous représenter que, jusque vers la fin du XII\textsuperscript{e} siècle, les philosophes et les théologiens du Moyen âge n’avaient guère connu la philosophie antique qu’à travers l’utilisation, qu’en avaient faite les écrivains chrétiens des premiers siècles. Tel devait rester, jusqu’à la Renaissance, le cas de Platon, fort respecté, influent même à travers la philosophie chrétienne ancienne, mais en somme peu pratiqué directement. Il n’est pas indifférent que ce type du sage non chrétien que dépeignent, en Socrate, les grands dialogues, ait été sans action sur la pensée du XIII\textsuperscript{e} siècle. Quant à Aristote, seuls furent longtemps connus ses traités de logique, traduits par Boèce. Ils ont fourni à tout le Moyen âge l’armature de sa logique. Tout le reste de l’œuvre était ignoré.\par
Elle cessa de l’être à partir de la fin du XII\textsuperscript{e} siècle. Non dans les textes directs. Mais par des traductions. Les unes furent faites de l’arabe, surtout en Espagne. À Tolède fonctionna, durant la seconde moitié du XII\textsuperscript{e} siècle, un véritable atelier de traductions. D’autres du grec, notamment en Sicile. Vers 1200, la plus grande partie de l’œuvre métaphysique et physique d’Aristote était ainsi à la disposition de quiconque cherchait à philosopher. Le reste (la fin des œuvres sur la nature, les œuvres morales, politiques et de doctrine littéraire) suivirent dans la première moitié du siècle.\par
Or, c’était là un événement intellectuel gros de conséquences : pour la première fois, la pensée occidentale se trouvait en présence d’une philosophie parfaitement organisée, séduisante et d’où la doctrine chrétienne était relativement absente. Aristote n’était pas arrivé seul. En même temps que lui, on avait traduit ses commentateurs de la fin du monde antique ou arabes. Parfois même, il arrivait que le texte, que l’on tenait pour aristotélicien, fùt en réalité de ses continuateurs : soit par interpolations (dans les manuscrits traduits de l’arabe surtout), soit sous forme d’apocryphes. Or, ces continuateurs étaient pour la foi chrétienne plus dangereux encore que le maître. Je me bornerai à citer le plus célèbre et le plus influent en Occident, celui que les Latins appelaient Averroës et dont le nom arabe était Ibn-Rochd (né à Cordoue en 1126, mort en 1178), plusieurs fois persécuté par l’orthodoxie mahométane. C’est \emph{« Averroës qui fit le grand commentaire »} (d’Aristote) comme dit Dante, qui l’a placé dans les limbes, avec son maître Aristote et tous les justes de l’Antiquité, dont la seule peine est d’être éloignés de Dieu.\par

\begin{itemize}[itemsep=0pt,]
\item[]\listhead{Deux thèses essentielles :}
\item L’éternité du monde donc pas de création ;
\item unité de l’intellect humain : donc pas d’immortalité personnelle, ni de jugement après la mort.
\end{itemize}

\noindent Contre l’influence très redoutable d’Aristote et des Averroïstes, l’Église a réagi de diverses façons.\par
D’abord par l’interdiction et la persécution. Sans doute le \foreign{Mauritius} d’Espagne, dont les statuts de 1215 défendent d’enseigner la doctrine à Paris était averroïste.\par

\begin{listalpha}[itemsep=\baselineskip,]
\item Le plus illustre des averroïstes latins fut Siger de Brabant, un maître ès arts de l’Université de Paris, contemporain de saint Louis et de Philippe III. Il écrivit d’abord assez librement, puis après une double condamnation de ses thèses, en 1270 et 1277, par l’évêque de Paris, Etienne Tempier, fut traduit en 1277 devant le tribunal de l’Inquisition, conduit ensuite à la cour pontificale, où il fut tué par le glaive, dit un poème du temps, dans des circonstances, qui nous demeurent mystérieuses.
 \phantomsection
\label{p120}\item  Contre Aristote également, on procéda d’abord par l’interdiction. Le centre de la nouvelle philosophie était incontestablement Paris. Et ce fut là qu’on frappa.\par
 
\begin{enumerate}[itemsep=0pt,]
\item[]\listhead{Une série de mesures se suivent :}
\item  1210, interdiction par un concile de la province de Sens, réuni à Paris, \emph{« de lire à Paris en public ou secrètement les livres d’Aristote sur la philosophie naturelle ni les commentaires »}. Il semble qu’on ait brûlé des manuscrits ;
\item en 1215, les statuts donnés à l’Université par le légat Robert de Courçon, tout en permettant expressément l’usage des écrits aristotéliciens sur la dialectique, interdisaient la lecture de la {\itshape Métaphysique} et des livres sur la philosophie naturelle.
\end{enumerate}

 Une méfiance analogue transparaît dans la bulle {\itshape Parens Scientiarum} de 1231, qui d’une part interdit aux maîtres ès arts l’usage des {\itshape Libri naturales} d’Aristote, de l’autre, recommande aux maîtres et écoliers en théologie de ne pas faire les philosophes ({\itshape nec philosophos se ostendent}) et de ne disputer que des questions qui pourront être déterminées \emph{« per libros theologicos et sanctorum patrum tractatus »}.
 
\end{listalpha}

\noindent Bien que ces interdictions aient été encore renouvelées par la suite, il est certain qu’elles avaient échoué. Elles étaient d’ailleurs proprement parisiennes, et les maîtres de Toulouse pouvaient, en 1229, allécher leurs auditeurs en leur promettant de leur expliquer les {\itshape Libri naturales} proscrits à Paris. En 1255, nous voyons toute l’œuvre, peu à peu expurgée de ses commentaires, enseignée officiellement à la Faculté des Arts. C’est par une autre voie qu’il devait être remédié au danger : l’absorption de la philosophie aristotélicienne dans la philosophie chrétienne.\par
La première tentative gauchement tentée en ce sens le fut par le pape Grégoire IX. En avril 1231, Grégoire IX décida de faire expurger les {\itshape Libri Naturales} de leurs erreurs et constitua des commissions à cet effet (qui devaient, semble-t-il, siéger à Paris). Après quoi, il serait loisible de s’en servir. Il est tout à l’honneur de cette commission qu’elle semble ne jamais avoir exécuté son travail. L’incorporation fut l’œuvre des philosophes eux-mêmes.\par
Au premier chef deux étrangers, de race noble, dominicains, et qui ont tous deux enseigné à Paris : l’Allemand Albert le Grand, qui enseigne à Paris de 1245 à 1248, mort à Cologne en 1280 ; son élève, l’Italien Thomas d’Aquin qui, après un premier séjour à Paris (1245-1248) et quatre années passées à Cologne auprès d’Albert, revint à Paris en 1257 et — réserve faite d’un séjour à la cour pontificale — y enseigne jusqu’en 1272. Il retourne à Naples et meurt, sur sa route vers le concile de Lyon, en 1274.\par
On n’attend pas de moi que je résume en trois mots le thomisme.\par

\begin{itemize}[itemsep=0pt,]
\item[]\listhead{Ce qu’il faut retenir je crois, surtout, du point de vue qui nous occupe ici, ce sont deux traits :}
\item Une préoccupation très nette de marquer les différences entre la foi révélée et la raison, de faire à chacune sa part, en laissant à la raison, non seulement le droit, mais le devoir de démontrer tout ce qui est démontrable dans la foi : d’où utilisation de la philosophie.
\item Par suite, une sympathie médiocre pour tout ce qui est intuition mystique ; il y a a) la révélation ; b) la raison. Il n’est guère d’autre manière que celle-là d’arriver à Dieu.
\end{itemize}


\begin{enumerate}[itemsep=0pt,]
\item[]\listhead{Il faut bien observer que cette philosophie parut très vivement révolutionnaire :}
\item parce qu’il y avait quelque chose de choquant à voir un théologien passer son temps à étudier un philosophe païen ;
 \phantomsection
\label{p121}\item par la part très faible, qu’elle faisait à la connaissance mystique.
\end{enumerate}

\noindent Elle fut très vivement attaquée, notamment en Angleterre. Elle eut néanmoins une très forte action. Mais elle n’est devenue philosophie officielle que de nos jours. Il y eut d’autres courants. Voici les principaux.\par

\begin{listalpha}[itemsep=\baselineskip,]
\item D’une part, un courant qui fond avec l’augustinisme et, jusqu’à un certain point, la tradition platonicienne avec la source vive de mysticisme, qui coulait dans les milieux franciscains Bonaventure.
\item  Un courant, où l’étude à la fois mathématique et expérimentale du monde sensible s’unit à un violent mysticisme : l’unité étant faite de l’hostilité à la fois à la raison raisonnante et, en matière de foi, à l’argument d’autorité. Le plus illustre représentant fut un Franciscain anglais, Roger Bacon, mort peu après 1292, après avoir été emprisonné de longues années, à qui l’on doit le mot destiné à un si bel avenir : \foreign{scientia experimentalis}.\par
 Mais Bacon avait passé à Paris et, là notamment, connu un homme qu’il a indiqué comme un de ses maîtres les plus chers : le Picard Pierre de Maricourt qui, contemporain de saint Louis (nous ne connaissons pas la date de sa naissance ni de sa mort), est l’auteur d’un traité sur l’aimant, longtemps célèbre, et que Bacon a salué du beau titre de \foreign{dominus experimentorum}.
 
\item Enfin, le mouvement averroïste n’était point mort. Il est significatif que Dante, malgré son orthodoxie, n’ait pas seulement colloqué Averroës, fort honorablement, dans les limbes ; dans la couronne d’âmes lumineuses, qu’il rencontre au IV\textsuperscript{e} ciel, figure avec saint Thomas et Albert son maître, nul autre que Siger. Au temps de Philippe le Bel et de ses fils, enseigne à Paris le maître ès Arts Jean de Jandun, qui, il est vrai, fut excommunié en 1327 par le pape Jean XXII, en même temps que son aîné Marsile de Padoue, mais moins pour ses opinions philosophiques que pour ses théories politiques ; il mourut en 1328. Dans son {\itshape Defensor pacis}, composé avec Marsile, Jean attaque violemment la papauté et veut soumettre l’Église à l’État (en l’espèce l’Empire), dont le chef est le représentant du peuple souverain. Mais en même temps, Jean est un averroïste déclaré, et probablement un véritable incrédule, sa théorie de la double vérité ne lui servant guère que de masque assez transparent pour ses lecteurs. Il n’est pas sans intérêt, pour l’historien des doctrines politiques, de voir un homme de cette trempe passer du Paris de Nogaret à la cour de Louis de Bavière ; ni pour l’historien des idées comme de la politique ecclésiastique en général, de voir quels contacts pouvait avoir dans la capitale même, autour du très pieux Philippe le Bel, un entourage peut-être moins pieux.
\end{listalpha}

\noindent Que savait-on, que disait-on de tout cela dans les couches moins intellectuelles ? Mystère. Quelques traits néanmoins à retenir.\par

\begin{enumerate}[itemsep=0pt,]
\item Le XIII\textsuperscript{e} siècle est le siècle des Sommes, logiquement et clairement ordonnées. Celles de saint Thomas sont des chefs-d’œuvre. Le {\itshape Speculum majus} du dominicain Vincent de Beauvais, qui fut {\itshape lector} de saint Louis, témoigne d’un esprit d’un degré bien inférieur ; mais elle n’en est pas moins caractéristique. C’est une vaste encyclopédie, divisée en trois parties : s{\itshape peculum naturale, doctrinale, historiale.} Dans la plupart des œuvres du temps, se manifestent ce goût et, jusqu’à un certain point, ce talent d’organisation qui est un des signes distinctifs du temps.
\item  \phantomsection
\label{p122} Toutes ces philosophies du XIII\textsuperscript{e} siècle témoignent d’une puissance de construction très supérieure à quoi que ce soit, qu’on ait vu en Europe depuis la fin de l’Empire romain. Il serait néanmoins injuste de croire que cette systématisation a arrêté une fermentation intellectuelle qui, nous l’avons vu, dura jusqu’au bout du siècle. Il n’en est pas moins vrai que la marque du siècle, dans son développement, est plutôt un assagissement de la pensée dans des formes traditionnelles. La grande crise de pensée comme de foi est plutôt des environs de 1200 que de 1300.
\end{enumerate}

\section[{D. L’expression de la mentalité dans la littérature et dans l’art}]{D. L’expression de la mentalité dans la littérature et dans l’art}\phantomsection
\label{c11d}
\noindent La littérature française est seule en ligne de compte. La provençale se meurt. La littérature latine est de plus en plus rejetée vers les œuvres de caractère didactique. Un des effets curieux de l’embrigadement des clercs dans les Universités est d’avoir tari la source de poésie latine, qu’entretenaient les {\itshape clerici vagantes.}\par
Il n’y aurait aucun intérêt à chercher à énumérer ici les œuvres ou les genres. Cherchons simplement à situer l’époque des successeurs de Philippe Auguste dans le courant général de la littérature française.\par
Une première observation s’impose. À beaucoup d’égards, le siècle nous apparaît comme une simple continuation. On remanie au goût du jour les vieilles gestes épiques, ou on brode sur les anciens héros de nouveaux poèmes. On écrit des romans d’aventures selon la formule créée au XII\textsuperscript{e} siècle, tels ceux de Beaumanoir : la {\itshape Manekine} et {\itshape Jean de Dammartin et Blonde d’Oxford.} On compose comme par le passé des poésies amoureuses ou satiriques. On ajoute de nouvelles branches au cycle du Renart. Il y a dans tout cela beaucoup de choses agréables, servies par une langue, qui n’avait jamais été et ne sera de longtemps plus simple et plus riche. Rien de très nouveau, en somme. Rien qui sente l’originalité du génie individuel ou collectif.\par
Quelques œuvres, cependant, rendent un son nouveau. C’est d’abord l’extraordinaire développement du goût des contes, et des contes de caractère populaire sous la forme de fabliaux. Il y a là une source de verve drue, et de verve comique : à retenir, pour ne pas se figurer tout le siècle sous l’aspect d’une mystique cathédrale.\par
C’est ensuite le premier développement du drame religieux, définitivement sorti des répons en latin liturgique. Il supposait un public, des foules urbaines. Un des plus anciens exemples connus (le Jeu Saint Nicolas), fin du XII\textsuperscript{e} siècle, est arrageois. Plusieurs autres ont été composés pour les \emph{Puys} ou confréries urbaines du Nord de la France. Parallèlement se développe le théâtre comique : Jeu de la Feuillée.\par
C’est enfin et surtout, à mesure que le siècle s’avance, le progrès d’une verve à la fois satirique et didactique. Elle se marque dans plusieurs continuateurs de Renart (notamment le Couronnement de Renart, peu après 1250) et s’épanouira au début du XIV\textsuperscript{e} siècle, dans Renart le Contrefait. Elle marque l’étrange transformation du Roman de la Rose, qui avait été une des œuvres de ce temps destinée au plus long retentissement. La première partie du poème fut écrite, entre 1225 et 1240, par Guillaume de Lorris dans une veine d’allégories et d’amour courtois. Le poème s’arrête au moment où l’Amour se plaint devant le château, où est enfermée  \phantomsection
\label{p123} la Rose. On ne sait s’il était demeuré inachevé, ou bien si, par discrétion amoureuse, Guillaume ne souhaitait pas le pousser plus loin. L’œuvre fut continuée, entre 1275 et 1280, par un clerc des écoles de Paris, Jean Clopinel, dit Jean de Meung. Ce très long récit, sous le voile de l’allégorie, prend souvent l’allure d’une somme, très érudite, mais très réaliste, avec une sorte d’apologie de la nature, dont le son est à retenir pour qui veut se faire une idée des courants d’idées dans les milieux intellectuels parisiens, non pas révolutionnaire certes, mais sans mysticisme politique et d’un christianisme fort rebelle à l’ascétisme. Toute cette littérature et jusqu’aux poèmes du pauvre jongleur Rutebœuf, qui vécut surtout à Paris sous saint Louis et Philippe III, témoigne d’une opinion très active.\par
Mais peut-être l’originalité du XIII\textsuperscript{e} siècle est-elle d’avoir véritablement créé la littérature française en prose. Cela surtout sous la forme de la prose historique, juridique et parfois didactique (par exemple avec le {\itshape Trésor} composé, en 1265, par l’Italien Brunetto Latini). Toute cette prose n’est pas prose d’art, et l’on en a parfois exagéré les mérites. Mais Joinville est un pur chef-d’œuvre, peut-être le premier chef-d’œuvre de prose de toute la littérature européenne.\par
Par là, on voit définitivement la littérature sortir des cadres des professionnels.\par
Quel public ? C’est ce qu’en vérité nous savons très mal. Je ne sais s’il faut parler d’un public plus large qu’au temps des Chansons de geste. Mais certainement d’un public plus varié — bourgeoisie ! — et plus sensible aux qualités d’art ; un public qui ne se contente plus d’entendre lire — qui lit — bien que Joinville écrive encore \emph{« cel qui orront ce livre »}.\par
\bigbreak
\noindent Ne cherchons pas ici à faire plus qu’une histoire littéraire, une histoire de l’art. Seulement à voir, ce que l’on peut nous apprendre sur les aspects du temps.\par
Inutile de rappeler que l’époque étudiée a vu une admirable floraison artistique et qu’elle est la grande époque de l’art que, traditionnellement, nous appelons gothique ou encore (d’une forme de voûte) ogivale. Quelques dates d’édifices : Chartres moins le clocher roman, pour l’essentiel autour de 1220 ; Reims de 1211 à 1300 ; Amiens de 1220 à 1288 ; Bourges de 1200 environ à 1270 ; la Sainte Chapelle entre 1243 et 1248. Le Palais Royal dans la Cité, date pour l’essentiel de Philippe le Bel.\par
La première idée, qui puisse venir au matérialisme de l’historien devant ces constructions, c’est qu’elles ont coûté beaucoup d’argent. Nous savons mal comment elles ont été subventionnées : fortunes des évêques, des chapitres, des monastères ; dons des grands et quêtes auprès de la masse ; contributions peut-être des villes. Mais il faut bien entendre que l’effort a rarement pu être soutenu jusqu’au bout. Sans aller jusqu’à l’exemple de Beauvais, dont la cathédrale, commencée en 1247 sur un plan gigantesque, n’a jamais compris qu’un chœur et un transept, peu d’édifices religieux ont été achevés jusqu’au bout.\par
De pareils édifices, très grands, plus hauts et plus percés de fenêtres que ceux de l’âge antérieur, attestent certainement une grande habileté technique (dans le fameux album de Villard de Honnecourt, carnet de croquis techniques). Mais ici gardons-nous aussi d’exagération. La cathédrale est plus qu’une épure et comme  \phantomsection
\label{p124} épure, elle n’est pas impeccable. L’histoire est pleine de voûtes écroulées. La perfection même de l’ogive a été récemment mise en question par des ingénieurs et des archéologues. Et les lignes n’ont pas été tracées seulement pour la commodité, mais pour la beauté.\par
Peut-être, en effet, ce qu’il y a de plus instructif dans cette floraison de l’art gothique, c’est le besoin que les hommes ont éprouvé de modifier un décor, dont ils ne voulaient plus, par un décor, qu’ils jugeaient plus beau. On nous parle habituellement d’incendies pour expliquer les reconstructions d’églises. D’accord. Mais pourquoi tant d’incendies en un si court intervalle de temps ? On nous dit aussi qu’on a voulu faire des églises plus grandes. Cela a été assez souvent le cas et en un sens, les reconstructions sont un effet du progrès démographique. Celui-ci, cependant, dans les villes, semble avoir entraîné surtout la multiplication des paroisses. Il faut bien penser avant tout à un souci esthétique.\par
On a vu que, jusqu’ici, nous avons parlé surtout d’édifices religieux. Ce ne sont pas les seuls. On a construit beaucoup de maisons épiscopales, ou bourgeoises, de châteaux, de halles. Mais il serait naturellement puéril de nier que l’on fût surtout au service de la religion. On a dit et redit que, par son iconographie — sculpture, vitraux, plus rarement peinture — une église gothique est un véritable livre d’enseignement. Cela est vrai, tout en faisant la part de techniques d’ateliers et parfois de simples fantaisies esthétiques.\par
Vitraux, miniatures, sculptures sont instructifs à un autre point de vue encore. Ils nous montrent le progrès fait dans le rendu de la figure humaine et une sorte de transition entre le schématisme et le franc réalisme, qui sera la loi de la fin du Moyen Age. Il y a de grandes différences entre les ateliers et sans doute entre les hommes qui nous sont inconnus. Mais dans l’ensemble, la marche est sensible : simplification et grandeur dans les grands portails de Chartres et le Beau Dieu d’Amiens, grâce plus souriante des œuvres rémoises avec peut-être dans certaines d’entre elles une influence de l’Antique, naturalisme plus accentué dans les œuvres du début du XIV\textsuperscript{e} siècle. Le premier des tombeaux des rois à Saint-Denis, où se marque un certain souci d’individualité — sans que d’ailleurs nous puissions être sûrs de l’exactitude du portrait — est celui de Philippe III.\par
Un caractère est net : c’est la disparition du fantastique et de l’apocalyptique. L’art du XIII\textsuperscript{e} siècle tout entier est, comme la prose de Joinville, un art simple et humain et les beaux feuillages qui, au lieu des monstres de naguère, ornent tant de chapiteaux, se développent sur les espaces vides des portails, voire — comme à Amiens — courent en guirlandes tout le long de la nef au-dessus des travées ouvrant sur les collatéraux, ce sont des feuillages de chez nous.\par
Enfin, autre chose encore et de plus profond peut-être à retenir de cet art. Beaucoup mieux que la littérature, encombrée d’œuvres médiocres, l’architecture, la sculpture monumentale, les savants ivoires, les vitraux, les miniatures (comme celles du psautier de saint Louis) nous rappellent que ces hommes, qui savaient peu de chose, qui se plaisaient encore à des contes bien naïfs, dont les mœurs étaient rudes et violentes, étaient cependant des esprits raffinés, avides de délicates et saines jouissances des yeux. Et ce n’est pas là, pour comprendre une époque, un fait indifférent.
\section[{E. Le bagage intellectuel}]{E. Le bagage intellectuel}\phantomsection
\label{c11e}
\subsection[{1° Les langues}]{1° Les langues}

\begin{listalpha}[itemsep=\baselineskip,]
\item  Les langues vulgaires. La division de la France en deux groupes linguistiques. Leur non-pénétration : Jacques Duèse devenu, en 1316, le pape Jean XXII, Cahorsin, ne sait pas le français. Mais décadence de la littérature provençale. Le dernier des grands troubadours, Guiraut Riquier, mort à la fin du XIII\textsuperscript{e} siècle, doit passer une partie de sa vie à la cour de Castille. La poésie subsiste dans les milieux bourgeois.\par
 Le français : tendance à se former une langue littéraire de l’Ile-de-France.
 
\item Le latin, langue de culture et de conversation, langue véhiculaire de l’enseignement. Mais en plus, depuis saint Louis et surtout Philippe le Bel, langue unique de l’administration. L’administration royale qui, depuis saint Louis et Philippe le Bel, use du français dans le Nord, use du latin dans le Midi.
\end{listalpha}

\noindent La littérature en français est : poétique ou romanesque, d’édification, historique, didactique seulement sous la forme de traductions, assez rares (par exemple les {\itshape Météorologiques} d’Aristote ont été traduites vers 1250), et surtout de compilations, de sommes en général très médiocres et en retard sur la pensée des gens les plus instruits. La vraie littérature philosophique ou scientifique de ce temps — c’est tout un — est en latin.
\subsection[{2° L’enseignement}]{2° L’enseignement}

\begin{listalpha}[itemsep=0pt,]
\item[]\listhead{Il faut distinguer soigneusement différents types :}
\item L’instruction des grands personnages laïques, assez poussée. Comme l’on sait déjà, elle se fait par précepteurs.
\item Les écoles des villages et des villes. Elles sont ecclésiastiques en général, annexes d’un monastère ou d’un chapitre. Dans les villes, elles se sont beaucoup développées au cours des XII\textsuperscript{e} et XIII\textsuperscript{e} siècles, sous l’impulsion des bourgeoisies, qui tantôt empiètent sur les droits reconnus aux autorités ecclésiastiques, tantôt cherchent à s’accorder avec celles-ci ; elles servent à la population marchande, sauf dans les grandes familles, où on a recours, semble-t-il, à des précepteurs privés. Il y en a également, dans certains bourgs de campagne, mais là elles sont, semble-t-il, surtout fréquentées par les enfants que l’on destine au clergé (et qui d’ailleurs n’y entrent pas toujours). Aller aux écoles et se faire prêtre, c’est tout un : le prouvent les textes relatifs aux serfs (Midi, Arras). J’ai pu suivre, de 1298 à 1306, des nominations de maîtres des écoles de grammaire de Montfort (l’Amaury) par l’abbé de Saint-Magloire. Du bienheureux Thomas de Biville, mort en 1257, son biographe nous dit qu’avant de recevoir les ordres, il exerça la métier de maître d’école \emph{« en beaucoup de lieux \footnote{ Voir D’IRSAY, {\itshape Histoire des Universités françaises et étrangères des origines à nos jours}, Paris, t. I, 1933. — ROSHDALL, {\itshape The Universities of Europe in the Middle Ages}, 3 vol. 1895, Abrégé, Cambridge.} »}.
\end{listalpha}

\chapterclose

 


% at least one empty page at end (for booklet couv)
\ifbooklet
  \pagestyle{empty}
  \clearpage
  % 2 empty pages maybe needed for 4e cover
  \ifnum\modulo{\value{page}}{4}=0 \hbox{}\newpage\hbox{}\newpage\fi
  \ifnum\modulo{\value{page}}{4}=1 \hbox{}\newpage\hbox{}\newpage\fi


  \hbox{}\newpage
  \ifodd\value{page}\hbox{}\newpage\fi
  {\centering\color{rubric}\bfseries\noindent\large
    Hurlus ? Qu’est-ce.\par
    \bigskip
  }
  \noindent Des bouquinistes électroniques, pour du texte libre à participation libre,
  téléchargeable gratuitement sur \href{https://hurlus.fr}{\dotuline{hurlus.fr}}.\par
  \bigskip
  \noindent Cette brochure a été produite par des éditeurs bénévoles.
  Elle n’est pas faîte pour être possédée, mais pour être lue, et puis donnée.
  Que circule le texte !
  En page de garde, on peut ajouter une date, un lieu, un nom ; pour suivre le voyage des idées.
  \par

  Ce texte a été choisi parce qu’une personne l’a aimé,
  ou haï, elle a en tous cas pensé qu’il partipait à la formation de notre présent ;
  sans le souci de plaire, vendre, ou militer pour une cause.
  \par

  L’édition électronique est soigneuse, tant sur la technique
  que sur l’établissement du texte ; mais sans aucune prétention scolaire, au contraire.
  Le but est de s’adresser à tous, sans distinction de science ou de diplôme.
  Au plus direct ! (possible)
  \par

  Cet exemplaire en papier a été tiré sur une imprimante personnelle
   ou une photocopieuse. Tout le monde peut le faire.
  Il suffit de
  télécharger un fichier sur \href{https://hurlus.fr}{\dotuline{hurlus.fr}},
  d’imprimer, et agrafer ; puis de lire et donner.\par

  \bigskip

  \noindent PS : Les hurlus furent aussi des rebelles protestants qui cassaient les statues dans les églises catholiques. En 1566 démarra la révolte des gueux dans le pays de Lille. L’insurrection enflamma la région jusqu’à Anvers où les gueux de mer bloquèrent les bateaux espagnols.
  Ce fut une rare guerre de libération dont naquit un pays toujours libre : les Pays-Bas.
  En plat pays francophone, par contre, restèrent des bandes de huguenots, les hurlus, progressivement réprimés par la très catholique Espagne.
  Cette mémoire d’une défaite est éteinte, rallumons-la. Sortons les livres du culte universitaire, cherchons les idoles de l’époque, pour les briser.
\fi

\ifdev % autotext in dev mode
\fontname\font — \textsc{Les règles du jeu}\par
(\hyperref[utopie]{\underline{Lien}})\par
\noindent \initialiv{A}{lors là}\blindtext\par
\noindent \initialiv{À}{ la bonheur des dames}\blindtext\par
\noindent \initialiv{É}{tonnez-le}\blindtext\par
\noindent \initialiv{Q}{ualitativement}\blindtext\par
\noindent \initialiv{V}{aloriser}\blindtext\par
\Blindtext
\phantomsection
\label{utopie}
\Blinddocument
\fi
\end{document}
