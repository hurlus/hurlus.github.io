%%%%%%%%%%%%%%%%%%%%%%%%%%%%%%%%
% LaTeX model for hurlus print %
%%%%%%%%%%%%%%%%%%%%%%%%%%%%%%%%

% Needed before document class
\RequirePackage{pdftexcmds} % needed for tests expressions
\RequirePackage{fix-cm} % correct units

% Define mode
\def\mode{a4}

\newif\ifaiv % a4
\newif\ifav % a5
\newif\ifbooklet % booklet

\ifnum \strcmp{\mode}{booklet}=0
  \booklettrue
\else\ifnum \strcmp{\mode}{a5}=0
  \avtrue
\else
  \aivtrue
\fi\fi

\ifbooklet % do not enclose with {}
  \documentclass[french,twoside]{book} % ,notitlepage
  \usepackage[%
    papersize={105mm, 297mm},
    inner=10mm,
    outer=10mm,
    top=18mm,
    bottom=18mm,
    marginparsep=0pt,
  ]{geometry}
  \usepackage[fontsize=9.5pt]{scrextend} % for Roboto
\else\ifav
  \documentclass[french,twoside]{book} % ,notitlepage
  \usepackage[%
    a5paper,
    inner=25mm,
    outer=15mm,
    top=15mm,
    bottom=15mm,
    marginparsep=0pt,
  ]{geometry}
  \usepackage[fontsize=12pt]{scrextend} % for Roboto
\else% A4 2 cols
  \documentclass[twocolumn]{book}
  \usepackage[%
    a4paper,
    inner=15mm,
    outer=10mm,
    top=25mm,
    bottom=18mm,
    marginparsep=0pt,
  ]{geometry}
  \setlength{\columnsep}{20mm}
  \usepackage[fontsize=9.5pt]{scrextend} % for Roboto
\fi\fi

%%%%%%%%%%%%%%
% Alignments %
%%%%%%%%%%%%%%
% before teinte macros

\setlength{\arrayrulewidth}{0.2pt}
\setlength{\columnseprule}{\arrayrulewidth} % twocol
\setlength{\parskip}{0pt} % classical para with no margin
\setlength{\parindent}{1.5em}

%%%%%%%%%%
% Colors %
%%%%%%%%%%
% before Teinte macros

\usepackage[dvipsnames]{xcolor}
\definecolor{rubric}{HTML}{902c20} % the tonic
\def\columnseprulecolor{\color{rubric}}
\colorlet{borderline}{rubric!30!} % definecolor need exact code
\definecolor{shadecolor}{gray}{0.95}
\definecolor{bghi}{gray}{0.5}

%%%%%%%%%%%%%%%%%
% Teinte macros %
%%%%%%%%%%%%%%%%%
%%%%%%%%%%%%%%%%%%%%%%%%%%%%%%%%%%%%%%%%%%%%%%%%%%%
% <TEI> generic (LaTeX names generated by Teinte) %
%%%%%%%%%%%%%%%%%%%%%%%%%%%%%%%%%%%%%%%%%%%%%%%%%%%
% This template is inserted in a specific design

\makeatletter % <@@@


\usepackage{blindtext} % generate text for testing
\usepackage{contour} % rounding words
\usepackage[nodayofweek]{datetime}
\usepackage{DejaVuSans} % font for symbols
\usepackage{enumitem} % <list>
\usepackage{etoolbox} % patch commands
\usepackage{fancyvrb}
\usepackage{fancyhdr}
\usepackage{fontspec} % XeLaTeX mandatory for fonts
\usepackage{footnote} % used to capture notes in minipage (ex: quote)
\usepackage{framed} % bordering correct with footnote hack
\usepackage{graphicx}
\usepackage{lettrine} % drop caps
\usepackage{lipsum} % generate text for testing
\usepackage[framemethod=tikz,]{mdframed} % maybe used for frame with footnotes inside
\usepackage{pdftexcmds} % needed for tests expressions
\usepackage{polyglossia} % non-break space french punct, bug Warning: "Failed to patch part"
\usepackage[%
  indentfirst=false,
  vskip=1em,
  noorphanfirst=true,
  noorphanafter=true,
  leftmargin=\parindent,
  rightmargin=0pt,
]{quoting}
\usepackage{ragged2e}
\usepackage{setspace}
\usepackage{tabularx} % <table>
\usepackage[explicit]{titlesec} % wear titles, !NO implicit
\usepackage{tikz} % ornaments
\usepackage{tocloft} % styling tocs
\usepackage[fit]{truncate} % used im runing titles
\usepackage{unicode-math}
\usepackage[normalem]{ulem} % breakable \uline, normalem is absolutely necessary to keep \emph
\usepackage{verse} % <l>
\usepackage{xcolor} % named colors
\usepackage{xparse} % @ifundefined
\XeTeXdefaultencoding "iso-8859-1" % bad encoding of xstring
\usepackage{xstring} % string tests
\XeTeXdefaultencoding "utf-8"

\PassOptionsToPackage{hyphens}{url} % before hyperref, which load url package
\usepackage{hyperref} % supposed to be the last one, :o) except for the ones to follow
% TOTEST
% \usepackage{hypcap} % links in caption ?
% \usepackage{marginnote}
% TESTED
% \usepackage{background} % doesn’t work with xetek
% \usepackage{bookmark} % prefers the hyperref hack \phantomsection
% \usepackage[color, leftbars]{changebar} % 2 cols doc, impossible to keep bar left
% \usepackage[utf8x]{inputenc} % inputenc package ignored with utf8 based engines
% \usepackage{firamath} % unavailable in Ubuntu 21-04
% \usepackage{flushend} % bad for last notes, supposed flush end of columns
% \usepackage[stable]{footmisc} % BAD for complex notes https://texfaq.org/FAQ-ftnsect
% \usepackage{multicol} % not compatible with too much packages (longtable, framed, memoir…)
% \usepackage{sectsty} % \chapterfont OBSOLETE
% \usepackage{soul} % \ul for underline, OBSOLETE with XeTeX
% \usepackage[breakable]{tcolorbox} % text styling gone, footnote hack not kept with breakable


% Metadata inserted by a program, from the TEI source, for title page and runing heads
\title{{\bf Mémoire sur le paupérisme}\\ \medskip
Du développement progressif du paupérisme chez les modernes et des moyens employés pour le combattre\\ \medskip
Mémoire présenté à la Société académique cherbourgeoise}
\date{1835}
\author{Alexis de Tocqueville}
\def\elbibl{Alexis de Tocqueville. 1835. \emph{Mémoire sur le paupérisme}}
\def\elabstract{%

\labelblock{Préface hurlue}


\begin{quoteblock}
\noindent Il n’y a rien qui, en général, élève et soutient plus haut l’esprit humain que l’idée des droits. On trouve dans l’idée du droit quelque chose de grand et de viril qui ôte à la demande son caractère suppliant, et place celui qui réclame sur le même niveau que celui qui accorde.\end{quoteblock}

\noindent Tocqueville est un grand esprit, malheureusement.\par
\bigbreak
\noindent Il sait rendre toutes les idées, même celles qu’il conteste, dans leur plus grande force, si bien que ses réfutations en sont d’autant plus brillantes et convaincantes. Qui n’est pas d’accord se sent effacé par la force de ses évidences. Il est l’esprit de la classe dominante, d’hier et de demain : intelligent, mesuré, humain, et donc écrasant de son universalité. N’a-t-on jamais défendu en si peu de mots que ci-dessus, la dignité que donne le droit à la sécurité sociale ?\par
Qui, à part un socialiste, a su lire dès 1835, le mouvement de l’histoire, l’aliénation de la classe ouvrière, entièrement soumise aux aléas du marché, et pourtant base de la production ?\par

\begin{quoteblock}
\noindent Les hommes quittaient la charrue pour prendre la navette et le marteau ; de la chaumière ils passaient dans la manufacture […] La classe industrielle a reçu la mission spéciale et dangereuse de pourvoir à ses risques et périls au bonheur matériel de toutes les autres.\end{quoteblock}

\noindent Cependant Tocqueville n’est pas socialiste, mais libéral. Il comprend bien que l’industrie produit la misère et nécessite une redistribution, mais il trouve aussi les arguments les plus vicieux contre l’\emph{assistanat}. Ses témoignages sur l’assistance publique anglaise de l’époque restent piquants, quoique vus depuis un \emph{Lord}. Il en conclut que le droit à l’assistance publique déprave les âmes et la société, et qu’il vaut mieux lui préférer la charité privée. Plutôt des milliardaires qui donne de bon cœur, qu’une sécurité sociale contraignante et administrative… \par

\begin{quoteblock}
\noindent L’aumône individuelle établit des liens précieux entre le riche et le pauvre. Le premier s’intéresse par le bienfait même au sort de celui dont il a entrepris de soulager la misère ; le second, soutenu par des secours qu’il n’avait pas droit d’exiger et que peut-être il n’espérait pas obtenir, se sent attiré par la reconnaissance.\end{quoteblock}

\noindent Où est donc le vice du raisonnement ? Dans une évidence implicite au détour d’une phrase : \emph{« deux nations rivales existent depuis le commencement du monde qu’on appelle les riches et les pauvres »}.\par
Tocqueville avait pourtant commencé son mémoire par une première partie presque copiée de Rousseau \emph{sur l'origine et les fondements de l'inégalité parmi les hommes} (un autre discours en réponse à une académie) ; expliquant que les riches sont apparus avec la religion de la propriété. Les inégalités ne sont pas de tous les temps mais produites par l’histoire.\par
Il aurait dû en conclure que la pauvreté, et les riches, peuvent disparaître ; mais il croit au contraire que cet ordre est éternel et voulu par Dieu. Le pauvre est nécessaire au riche pour le salut de son âme.\par
Selon Tocqueville, c’est même l’assistance publique qui est la cause de la dépravation des riches qui ne donnent plus. La charité individuelle était \emph{« suffisante au Moyen Age, parce que l’ardeur religieuse lui donnait une immense énergie »}, tandis qu’avec l’assistance publique, \emph{« la loi dépouille le riche d’une partie de son superflu sans le consulter, et il ne voit dans le pauvre qu’un avide étranger appelé par le législateur, au partage de ses biens »}. Et voilà, c’est de la faute au pauvre si le riche n’a plus envie de donner. Le pauvre doit savoir se vendre.\par
Mais Tocqueville ne se soucie de l’intérêt général que par accident, et avoue en conclusion sa véritable peur.\par

\begin{quoteblock}
\noindent [L’assistance publique] réduira avec le temps les riches à n’être que les fermiers des pauvres, tarira les sources de l’épargne, arrêtera l’accumulation des capitaux, comprimera l’essor du commerce, engourdira l’activité et l’industrie humaines et finira par amener une révolution violente dans l’État, lorsque le nombre de ceux qui reçoivent l’aumône sera devenu presque aussi grand que le nombre de ceux qui la donnent, et que l’indigent ne pouvant plus tirer des riches appauvris de quoi pourvoir à ses besoins trouvera plus facile de les dépouiller tout à coup de leurs biens que de demander leurs secours.\end{quoteblock}

\noindent Donc, si l’assistance publique fonctionne trop bien, elle va appauvrir les riches, qui ne pourront plus alimenter les caisses, et donc les pauvres en colère se mettraient en \emph{Révolution}, contre les riches.\par
Mais si les riches sont pauvres, pourquoi les pauvres en voudraient à leur propriété ? Et si l’assistance fonctionne, pourquoi les riches craindraient la ruine ? La dénonciation de l’\emph{assistanat} ne se soucie pas vraiment du moral des pauvres ; Tocqueville et ses éternels lecteurs s’inquiètent surtout de leurs privilèges de ne plus se sentir riches s’ils ne peuvent plus donner aux pauvres.\par
\bigbreak
\noindent Cette préface devrait vous avoir définitivement mal tourné l’esprit pour ne pas être séduit par un texte, qui n’en reste pas moins instructif parce qu’il est fort et clair.

}

% Default metas
\newcommand{\colorprovide}[2]{\@ifundefinedcolor{#1}{\colorlet{#1}{#2}}{}}
\colorprovide{rubric}{red}
\colorprovide{silver}{Gray}
\@ifundefined{syms}{\newfontfamily\syms{DejaVu Sans}}{}
\newif\ifdev
\@ifundefined{elbibl}{% No meta defined, maybe dev mode
  \newcommand{\elbibl}{Titre court ?}
  \newcommand{\elbook}{Titre du livre source ?}
  \newcommand{\elabstract}{Résumé\par}
  \newcommand{\elurl}{http://oeuvres.github.io/elbook/2}
  \author{Éric Lœchien}
  \title{Un titre de test assez long pour vérifier le comportement d’une maquette}
  \date{1566}
  \devtrue
}{}
\let\eltitle\@title
\let\elauthor\@author
\let\eldate\@date


\defaultfontfeatures{
  % Mapping=tex-text, % no effect seen
  Scale=MatchLowercase,
  Ligatures={TeX,Common},
}

\@ifundefined{\columnseprulecolor}
  {%
    \patchcmd\@outputdblcol{% find
      \normalcolor\vrule
    }{% and replace by
      \columnseprulecolor\vrule
    }{% success
    }{% failure
      \@latex@warning{Patching \string\@outputdblcol\space failed}%
    }
  }
  {}

\hypersetup{
  % pdftex, % no effect
  pdftitle={\elbibl},
  % pdfauthor={Your name here},
  % pdfsubject={Your subject here},
  % pdfkeywords={keyword1, keyword2},
  bookmarksnumbered=true,
  bookmarksopen=true,
  bookmarksopenlevel=1,
  pdfstartview=Fit,
  breaklinks=true, % avoid long links
  pdfpagemode=UseOutlines,    % pdf toc
  hyperfootnotes=true,
  colorlinks=false,
  pdfborder=0 0 0,
  % pdfpagelayout=TwoPageRight,
  % linktocpage=true, % NO, toc, link only on page no
}





% generic typo commands
\newcommand{\astermono}{\medskip\centerline{\color{rubric}\large\selectfont{\syms ✻}}\medskip\par}%
\newcommand{\astertri}{\medskip\par\centerline{\color{rubric}\large\selectfont{\syms ✻\,✻\,✻}}\medskip\par}%
\newcommand{\asterism}{\bigskip\par\noindent\parbox{\linewidth}{\centering\color{rubric}\large{\syms ✻}\\{\syms ✻}\hskip 0.75em{\syms ✻}}\bigskip\par}%

% lists
\newlength{\listmod}
\setlength{\listmod}{\parindent}
\setlist{
  itemindent=!,
  listparindent=\listmod,
  labelsep=0.2\listmod,
  parsep=0pt,
  % topsep=0.2em, % default topsep is best
}
\setlist[itemize]{
  label=—,
  leftmargin=0pt,
  labelindent=1.2em,
  labelwidth=0pt,
}
\setlist[enumerate]{
  label={\bf\color{rubric}\arabic*.},
  labelindent=0.8\listmod,
  leftmargin=\listmod,
  labelwidth=0pt,
}
\newlist{listalpha}{enumerate}{1}
\setlist[listalpha]{
  label={\bf\color{rubric}\alph*.},
  leftmargin=0pt,
  labelindent=0.8\listmod,
  labelwidth=0pt,
}
\newcommand{\listhead}[1]{\hspace{-1\listmod}\emph{#1}}

\renewcommand{\hrulefill}{%
  \leavevmode\leaders\hrule height 0.2pt\hfill\kern\z@}

% General typo
\DeclareTextFontCommand{\textlarge}{\large}
\DeclareTextFontCommand{\textsmall}{\small}


% commands, inlines
\newcommand{\anchor}[1]{\Hy@raisedlink{\hypertarget{#1}{}}} % link to top of an anchor (not baseline)
\newcommand\abbr{}
\newcommand{\autour}[1]{\tikz[baseline=(X.base)]\node [draw=rubric,thin,rectangle,inner sep=1.5pt, rounded corners=3pt] (X) {#1};}
\newcommand\corr{}
\newcommand{\ed}[1]{ {\color{silver}\sffamily\footnotesize (#1)} } % <milestone ed="1688"/>
\newcommand\expan{}
\newcommand\gap{}
\renewcommand{\LettrineFontHook}{\color{rubric}}
\newcommand{\initial}[2]{\lettrine[lines=2, loversize=0.3, lhang=0.3]{#1}{#2}}
\newcommand{\initialiv}[2]{%
  \let\oldLFH\LettrineFontHook
  % \renewcommand{\LettrineFontHook}{\color{rubric}\ttfamily}
  \IfSubStr{Q}{#1}{
    \lettrine[lines=4, lhang=0.2, loversize=-0.1, lraise=0.2]{\smash{#1}}{#2}
  }{\IfSubStr{É}{#1}{
    \lettrine[lines=4, lhang=0.2, loversize=-0, lraise=0]{\smash{#1}}{#2}
  }{\IfSubStr{ÀÂ}{#1}{
    \lettrine[lines=4, lhang=0.2, loversize=-0, lraise=0, slope=0.6em]{\smash{#1}}{#2}
  }{\IfSubStr{A}{#1}{
    \lettrine[lines=4, lhang=0.2, loversize=0.2, slope=0.6em]{\smash{#1}}{#2}
  }{\IfSubStr{V}{#1}{
    \lettrine[lines=4, lhang=0.2, loversize=0.2, slope=-0.5em]{\smash{#1}}{#2}
  }{
    \lettrine[lines=4, lhang=0.2, loversize=0.2]{\smash{#1}}{#2}
  }}}}}
  \let\LettrineFontHook\oldLFH
}
\newcommand{\labelchar}[1]{{\color{rubric}\bf #1}}
\newcommand{\milestone}[1]{\autour{\footnotesize\color{rubric} #1}} % <milestone n="4"/>
\newcommand\name{}
\newcommand\orig{}
\newcommand\orgName{}
\newcommand\persName{}
\newcommand\placeName{}
\newcommand{\pn}[1]{{\sffamily\textbf{#1.}} } % <p n="3"/>
\newcommand\reg{}
% \newcommand\ref{} % already defined
\newcommand\sic{}
\def\mednobreak{\ifdim\lastskip<\medskipamount
  \removelastskip\nopagebreak\medskip\fi}
\def\bignobreak{\ifdim\lastskip<\bigskipamount
  \removelastskip\nopagebreak\bigskip\fi}
% commands, blocks
\newcommand{\byline}[1]{\bigskip{\RaggedLeft{#1}\par}\bigskip}
\newcommand{\bibl}[1]{{\footnotesize\RaggedLeft{#1}\par}}
\newcommand{\biblitem}[1]{{\noindent\hangindent=\parindent   #1\par}}
\newcommand{\dateline}[1]{\medskip{\RaggedLeft{#1}\par}\bigskip}
\newcommand{\labelblock}[1]{\bigskip{\color{rubric}\bfseries\centering\RaggedLeft{#1}\par}\bignobreak}
\newcommand{\salute}[1]{\bigskip{#1}\par\medskip}

% environments for blocks (some may become commands)
\newenvironment{borderbox}{}{} % framing content
\newenvironment{citbibl}{\ifvmode\hfill\fi}{\ifvmode\par\fi }
\newenvironment{docAuthor}{\ifvmode\vskip4pt\fontsize{16pt}{18pt}\selectfont\fi\itshape}{\ifvmode\par\fi }
\newenvironment{docDate}{}{\ifvmode\par\fi }
\newenvironment{docImprint}{\vskip6pt}{\ifvmode\par\fi }
\newenvironment{docTitle}{\vskip6pt\bfseries\fontsize{18pt}{22pt}\selectfont}{\par }
\newenvironment{msHead}{\vskip6pt}{\par}
\newenvironment{msItem}{\vskip6pt}{\par}
\newenvironment{titlePart}{}{\par }


% environments for block containers
\newenvironment{argument}{\fontlight\parindent0pt}{\vskip1.5em}
\newenvironment{biblfree}{}{\ifvmode\par\fi }
\newenvironment{bibitemlist}[1]{%
  \list{\@biblabel{\@arabic\c@enumiv}}%
  {%
    \settowidth\labelwidth{\@biblabel{#1}}%
    \leftmargin\labelwidth
    \advance\leftmargin\labelsep
    \@openbib@code
    \usecounter{enumiv}%
    \let\p@enumiv\@empty
    \renewcommand\theenumiv{\@arabic\c@enumiv}%
  }
  \sloppy
  \clubpenalty4000
  \@clubpenalty \clubpenalty
  \widowpenalty4000%
  \sfcode`\.\@m
}%
{\def\@noitemerr
  {\@latex@warning{Empty `bibitemlist' environment}}%
\endlist}
\newenvironment{quoteblock}% may be used for ornaments
  {\begin{quoting}}
  {\end{quoting}}

% table () is preceded and finished by custom command
\newcommand{\tableopen}[1]{%
  \ifnum\strcmp{#1}{wide}=0{%
    \begin{center}
  }
  \else\ifnum\strcmp{#1}{long}=0{%
    \begin{center}
  }
  \else{%
    \begin{center}
  }
  \fi\fi
}
\newcommand{\tableclose}[1]{%
  \ifnum\strcmp{#1}{wide}=0{%
    \end{center}
  }
  \else\ifnum\strcmp{#1}{long}=0{%
    \end{center}
  }
  \else{%
    \end{center}
  }
  \fi\fi
}


% text structure
\newcommand\chapteropen{} % before chapter title
\newcommand\chaptercont{} % after title, argument, epigraph…
\newcommand\chapterclose{} % maybe useful for multicol settings
\setcounter{secnumdepth}{-2} % no counters for hierarchy titles
\setcounter{tocdepth}{5} % deep toc
\markright{\@title} % ???
\markboth{\@title}{\@author} % ???
\renewcommand\tableofcontents{\@starttoc{toc}}
% toclof format
% \renewcommand{\@tocrmarg}{0.1em} % Useless command?
% \renewcommand{\@pnumwidth}{0.5em} % {1.75em}
\renewcommand{\@cftmaketoctitle}{}
\setlength{\cftbeforesecskip}{\z@ \@plus.2\p@}
\renewcommand{\cftchapfont}{\bfseries}
\renewcommand{\cftchapdotsep}{\cftdotsep}
\renewcommand{\cftchapleader}{\normalfont\cftdotfill{\cftchapdotsep}}
\renewcommand{\cftchappagefont}{\normalfont}
\setlength{\cftbeforechapskip}{0em \@plus\p@}
% \renewcommand{\cftsecfont}{\small\relax}
\renewcommand{\cftsecpagefont}{\normalfont}
% \renewcommand{\cftsubsecfont}{\small\relax}
\renewcommand{\cftsecdotsep}{\cftdotsep}
\renewcommand{\cftsecpagefont}{\normalfont}
\renewcommand{\cftsecleader}{\normalfont\cftdotfill{\cftsecdotsep}}
\setlength{\cftsecindent}{1em}
\setlength{\cftsubsecindent}{2em}
\setlength{\cftsubsubsecindent}{3em}
\setlength{\cftchapnumwidth}{1em}
\setlength{\cftsecnumwidth}{1em}
\setlength{\cftsubsecnumwidth}{1em}
\setlength{\cftsubsubsecnumwidth}{1em}

% footnotes
\newif\ifheading
\newcommand*{\fnmarkscale}{\ifheading 0.70 \else 1 \fi}
\renewcommand\footnoterule{\vspace*{0.3cm}\hrule height \arrayrulewidth width 3cm \vspace*{0.3cm}}
\setlength\footnotesep{1.5\footnotesep} % footnote separator
\renewcommand\@makefntext[1]{\parindent 1.5em \noindent \hb@xt@1.8em{\hss{\normalfont\@thefnmark . }}#1} % no superscipt in foot


% orphans and widows
\clubpenalty=9996
\widowpenalty=9999
\brokenpenalty=4991
\predisplaypenalty=10000
\postdisplaypenalty=1549
\displaywidowpenalty=1602
\hyphenpenalty=400
% Copied from Rahtz but not understood
\def\@pnumwidth{1.55em}
\def\@tocrmarg {2.55em}
\def\@dotsep{4.5}
\emergencystretch 3em
\hbadness=4000
\pretolerance=750
\tolerance=2000
\vbadness=4000
\def\Gin@extensions{.pdf,.png,.jpg,.mps,.tif}
% \renewcommand{\@cite}[1]{#1} % biblio


\makeatother % /@@@>
%%%%%%%%%%%%%%
% </TEI> end %
%%%%%%%%%%%%%%


%%%%%%%%%%%%%
% footnotes %
%%%%%%%%%%%%%
\renewcommand{\thefootnote}{\bfseries\textcolor{rubric}{\arabic{footnote}}} % color for footnote marks

%%%%%%%%%
% Fonts %
%%%%%%%%%
% \usepackage[]{noto} % works
\usepackage[defaultmono]{droidsansmono}
\usepackage[]{roboto} % SmallCaps, Regular is a bit bold
% \linespread{0.90} % too compact, keep font natural
\newfontfamily\fontrun[]{Roboto Condensed Light} % condensed runing heads
\ifav
  \setmainfont{Roboto}
\else
  \setmainfont[
    BoldFont="Roboto",
  ]{Roboto Light}
  \renewcommand{\LettrineFontHook}{\bfseries\color{rubric}}
\fi
% \renewenvironment{labelblock}{\begin{center}\bfseries\color{rubric}}{\end{center}}

%%%%%%%%
% MISC %
%%%%%%%%

\setdefaultlanguage[frenchpart=false]{french} % bug on part


\newenvironment{quotebar}{%
    \def\FrameCommand{{\color{rubric!10!}\vrule width 0.5em} \hspace{0.9em}}%
    \def\OuterFrameSep{\itemsep} % séparateur vertical
    \MakeFramed {\advance\hsize-\width \FrameRestore}
  }%
  {%
    \endMakeFramed
  }
\renewenvironment{quoteblock}% may be used for ornaments
  {%
    \savenotes
    \setstretch{0.9}
    \begin{quotebar}
  }
  {%
    \end{quotebar}
    \spewnotes
  }


\renewcommand{\pn}[1]{{\footnotesize\color{rubric}\autour{#1}}} % <p n="3"/>
\renewcommand{\headrulewidth}{\arrayrulewidth}
\renewcommand{\headrule}{\color{rubric}\hrule}


\ifaiv
  \fancypagestyle{fancy}{%
    \fancyhf{}
    \setlength{\headheight}{1.5em}
    \fancyhead{} % reset head
    \fancyfoot{} % reset foot
    \fancyhead[L]{\truncate{0.45\headwidth}{\fontrun\elbibl}} % book ref
    \fancyhead[R]{\truncate{0.45\headwidth}{\fontrun\nouppercase\leftmark}} % Chapter title
    \fancyhead[C]{\thepage}
  }
  \fancypagestyle{plain}{% apply to chapter
    \fancyhf{}% clear all header and footer fields
    \setlength{\headheight}{1.5em}
    \fancyhead[L]{\truncate{0.9\headwidth}{\fontrun\elbibl}}
    \fancyhead[R]{\thepage}
  }
  \renewcommand\chapteropen{} % before chapter title NO more \savenotes needed
  \renewcommand\chaptercont{} % after title, argument, epigraph… \spewnotes
  \renewcommand\chapterclose{} % maybe useful for multicol settings
\else
  \fancypagestyle{fancy}{%
    \fancyhf{}
    \setlength{\headheight}{1.5em}
    \fancyhead{} % reset head
    \fancyfoot{} % reset foot
    \fancyhead[LO]{\truncate{0.9\headwidth}{\fontrun\elbibl}} % book ref
    \fancyhead[RE]{\truncate{0.9\headwidth}{\fontrun\nouppercase\leftmark}} % Chapter title, \nouppercase needed
    \fancyhead[RO,LE]{\thepage}
  }
  \fancypagestyle{plain}{% apply to chapter
    \fancyhf{}% clear all header and footer fields
    \setlength{\headheight}{1.5em}
    \fancyhead[L]{\truncate{0.9\headwidth}{\fontrun\elbibl}}
    \fancyhead[R]{\thepage}
  }
\fi

% delicate tuning, image has produce line-height problems in title on 2 lines
\titleformat{name=\chapter} % command
  [display] % shape
  {\vspace{16pt}\centering} % format
  {} % label
  {0pt} % separator between n
  {% before code
    % \vspace{16pt}
  }
[{\color{rubric}\huge\bfseries #1}\vspace{16pt}] % after code
% \titlespacing{command}{left spacing}{before spacing}{after spacing}[right]
\titlespacing*{\chapter}{0pt}{-30pt}{0pt}[0pt]

\titleformat{name=\section}
  [block]
  {}
  {} % \thesection
  {} % separator \arrayrulewidth
  {}
[\vbox{\color{rubric}\hrule\bigskip \large\bfseries\raggedleft #1}]

\titlespacing{\section}{0pt}{0pt plus 4pt minus 2pt}{\baselineskip}

\titleformat{name=\subsection}
  [block]
  {\bfseries\raggedright}
  {} % \thesection
  {} % separator \arrayrulewidth
  {}
[\uline{#1}]
% \titlespacing{\subsection}{0pt}{0pt plus 4pt minus 2pt}{\baselineskip}


\newcommand\chapo{{%
  \vspace*{-3em}
  \centering % no vskip ()
  {\Large\addfontfeature{LetterSpace=25}\bfseries{\elauthor}}\par
  \smallskip
  {\large\eldate}\par
  \bigskip
  {\Large\selectfont{\eltitle}}\par
  \bigskip
  {\color{rubric}\hline\par}
  \bigskip
  {\Large LIVRE LIBRE À PRIX LIBRE, DEMANDEZ AU COMPTOIR\par}
  \centerline{\small\color{rubric} {hurlus.fr, tiré le \today}}\par
  \bigskip
}}


\begin{document}
\pagestyle{fancy}
\thispagestyle{empty}


\ifaiv
  \twocolumn[\chapo]
\else
  \chapo
\fi


\elabstract
\bigskip

\makeatletter\@starttoc{toc}\makeatother % toc without new page
\pagebreak % request a page break after toc

\ifdev % autotext in dev mode

\fontname\font — \textsc{Les règles du jeu}\par
(\hyperref[utopie]{\underline{Lien}})\par
\noindent \initialiv{A}{lors là}\blindtext\par
\noindent \initialiv{À}{ la bonheur des dames}\blindtext\par
\noindent \initialiv{É}{tonnez-le}\blindtext\par
\noindent \initialiv{Q}{ualitativement}\blindtext\par
\noindent \initialiv{V}{aloriser}\blindtext\par
\Blindtext
\phantomsection
\label{utopie}
\Blinddocument
\fi

\section[Première partie.]{Première partie.}
\noindent \initial{L}{ORSQU'ON} parcourt les diverses contrées de l’Europe, on est frappé d’un spectacle très extraordinaire et en apparence inexplicable.\par

\labelblock{Les pays qui paraissent les plus misérables sont ceux qui, en réalité, comptent le moins d’indigents, et chez les peuples dont vous admirez l’opulence, une partie de la population est obligée pour vivre d’avoir recours aux dons de l’autre.}

\noindent Traversez les campagnes de l’Angleterre, vous vous croirez transporté dans l’Eden de la civilisation moderne. Des routes magnifiquement entretenues, de fraîches et propres demeures, de gras troupeaux errant dans de riches prairies, des cultivateurs pleins de force et de santé, la richesse plus éblouissante qu’en aucun pays du monde, la simple aisance plus ornée et plus recherchée qu’ailleurs ; partout l’aspect du soin, du bien-être et des loisirs ; un air de prospérité universelle qu’on croit respirer dans l’atmosphère elle-même et qui fait tressaillir le cœur à chaque pas : telle apparaît l’Angleterre aux premiers regards du voyageur.\par
Pénétrez maintenant dans l’intérieur des communes ; examinez les registres des paroisses, et vous découvrirez avec un inexprimable étonnement que le sixième des habitants de ce florissant royaume vit aux dépens de la charité publique.\par
Que si vous transportez en Espagne, et surtout en Portugal, la scène de vos observations, un spectacle tout contraire frappera vos regards. Vous rencontrez sur vos pas une population mal nourrie, mal vêtue, ignorante et grossière, vivant au milieu de campagnes à moitié incultes et dans des demeures misérables ; en Portugal cependant, le nombre des indigents est peu considérable. M. de Villeneuve estime qu’il se trouve dans ce royaume un pauvre sur vingt-cinq habitants. Le célèbre géographe Balbi avait précédemment indiqué le chiffre d’un indigent sur quatre-vingt-dix-huit habitants.\par
\bigbreak
\noindent Au lieu de comparer entre elles des contrées étrangères, opposez les unes aux autres diverses parties du même empire, et vous arriverez à un résultat analogue : vous verrez croître proportionnellement, d’une part, le nombre de ceux qui vivent dans l’aisance, et, de l’autre, le nombre de ceux qui ont recours pour vivre aux dons du public.\par
La moyenne des indigents en France, suivant les calculs d’un écrivain consciencieux \footnote{\emph{(Note de Tocqueville)} M. de Villeneuve.} dont je suis loin, du reste, d’approuver toutes les théories, est d’un pauvre sur vingt habitants. Mais on remarque, entre les différentes parties du royaume, d’immenses différences. Le département du Nord, qui est à coup sûr le plus riche, le plus peuplé et le plus avancé en toute chose, compte près du sixième de sa population auquel les secours de la charité sont nécessaires. Dans la Creuse, le plus pauvre et le moins industrialisé de tous nos départements, il ne se rencontre qu’un indigent sur cinquante-huit habitants. Dans cette statistique, la Manche est indiquée comme ayant un pauvre sur vingt-six habitants.\par
Je pense qu’il n’est pas impossible de donner une explication raisonnable de ce phénomène. L'effet que je viens de signaler tient à plusieurs causes générales qu’il serait trop long d’approfondir, mais qu’on peut au moins indiquer.\par
\bigbreak
\noindent Ici, pour bien comprendre ma pensée, je sens le besoin de remonter pour un moment jusqu’à la source des sociétés humaines. Je descendrai ensuite rapidement le fleuve de l’humanité jusqu’à nos jours.\par
\bigbreak
\noindent Voici les hommes qui se rassemblent pour la première fois. Ils sortent des bois, ils sont encore sauvages ; ils s’associent non pour jouir de la vie, mais pour trouver les moyens e vivre. Un abri contre l’intempérie des saisons, une nourriture suffisante, tel est l’objet de leurs efforts. Leur esprit ne va pas au-delà e ces biens, et, s’ils les obtiennent sans peine, ils s’estiment satisfaits de leur sort et endorment dans leur oisive aisance. J'ai vécu au milieu des peuplades barbares de l’Amérique du Nord ; j’ai plaint leur destinée, mais eux ne la trouvaient point cruelle. Coulé au milieu de la fumée de sa hutte, couvert de grossiers vêtements, ouvrage de ses mains ou produit de sa chasse, l’Indien regarde en pitié nos arts, considérant comme un assujettissement fatigant et honteux les recherches de notre civilisation ; il ne nous envie que nos armes.\par
Parvenus à ce premier âge des sociétés, les hommes ont donc encore très peu de désirs, ne ressentent guère que des besoins analogues à ceux qu’éprouvent les animaux ; ils ont seulement découvert dans l’organisation sociale le moyen de les satisfaire avec moins de peine. Avant que l’agriculture leur soit connue, ils vivent de la chasse ; du moment qu’ils ont appris l’art de faire produire à la terre des moissons, ils deviennent cultivateurs. Chacun tire alors du champ qui lui est échu en partage de quoi pourvoir à sa nourriture et à celle de ses enfants. La propriété foncière est créée et avec elle on voit naître l’élément le plus actif du progrès.\par
Du moment où les hommes possèdent la terre, ils se fixent. Ils trouvent dans la culture du sol des ressources abondantes contre la faim. Assurés de vivre, ils commencent à entrevoir qu’il se rencontre dans l’existence humaine d’autres sources de jouissances que la satisfaction des premiers et des plus impérieux besoins de la vie.\par

\labelblock{Tant que les hommes avaient été errants et chasseurs, l’inégalité n’avait pu s’introduire parmi eux d’une manière permanente.}

\noindent Il n’existait point de signe extérieur qui pût établir d’une façon durable la supériorité d’un homme et surtout d’une famille sur une autre famille ou sur un autre homme ; et ce signe eût-il existé, on n’aurait pu le transmettre à ses enfants. Mais dès l’instant où la propriété foncière fut connue et où les hommes eurent converti les vastes forêts en riches guérets et grasses prairies, de ce moment on vit des individus réunir dans leurs mains beaucoup plus de terre qu’il n’en fallait pour se nourrir et en perpétuer la propriété dans les mains de leur postérité. De là l’existence du superflu ; avec le superflu naît le goût des jouissances autres que la satisfaction des besoins les plus grossiers de la nature physique.\par
C'est à cet âge des sociétés qu’il faut placer l’origine de presque toutes les aristocraties.\par
Tandis que quelques hommes connaissent déjà l’art de concentrer dans les mains d’un petit nombre, avec la richesse et le pouvoir, presque toutes les jouissances intellectuelles matérielles que peut présenter l’existence, la foule à demi sauvage ignore encore le secret de répandre l’aisance et la liberté sur tous. A cette époque de l’histoire du genre humain, les hommes ont déjà abandonné les grossières et orgueilleuses vertus qui avaient pris naissance dans les bois ; ils ont perdu ces avantages de la barbarie, sans acquérir ce que la civilisation peut donner. Attachés à la culture du sol comme à leur seule ressource, ils ignorent l’art de défendre les fruits de leurs travaux. Placés entre l’indépendance sauvage qu’ils ne peuvent plus goûter, et la liberté civile et politique qu’ils ne comprennent point encore, ils sont livrés sans recours à la violence et à la ruse, et se montrent prêts à subir toutes les tyrannies, pourvu qu’on les laisse vivre ou plutôt végéter près de leurs sillons.\par
C'est alors que la propriété foncière s’agglomère outre mesure ; que le gouvernement se concentre dans quelques mains. C'est alors que la guerre, au lieu de mettre en péril l’état politique des peuples ainsi qu’il arrive de nos jours, menace la propriété individuelle de chaque citoyen ; que l’inégalité atteint dans le monde ses extrêmes limites et qu’on voit s’étendre l’esprit de conquête qui a été comme le père et la mère de toutes les aristocraties durables.\par
Les Barbares qui ont envahi l’Empire romain à la fin du IV\textsuperscript{e} siècle étaient des sauvages qui avaient entrevu ce que la propriété foncière présente d’utile et qui voulurent s’attribuer exclusivement les avantages qu’elle peut offrir. La plupart des provinces romaines qu’ils attaquèrent étaient peuplées par des hommes attachés depuis longtemps déjà à la culture de la terre, dont les mœurs s’étaient amollies parmi les occupations paisibles des champs et chez lesquels cependant la civilisation n’avait point encore fait d’assez grands progrès pour les mettre en état de lutter contre l’impétuosité primitive de leurs ennemis. La victoire mit dans les mains des Barbares non seulement le gouvernement, mais la propriété des tiers. Le cultivateur, de possesseur devint fermier. L'inégalité passa dans les lois ; elle devint un droit après avoir été un fait. La société féodale s’organisa et l’on vit naître le Moyen Age. Si l’on fait attention à ce qui ce passe dans le monde depuis l’origine des sociétés, on découvrira sans peine que l’égalité ne se rencontre qu’aux deux bouts de la civilisation. Les sauvages sont égaux entre eux parce qu’ils sont tous également faibles et ignorants. Les hommes très civilisés peuvent tous devenir égaux parce qu’ils ont tous à leur disposition des moyens analogues d’atteindre l’aisance et le bonheur. Entre ces deux extrêmes se trouvent l’inégalité des conditions, la richesse, les lumières, le pouvoir des uns, la pauvreté, l’ignorance et la faiblesse de tous les autres.\par
D'habiles et savants écrivains ont déjà travaillé à faire connaître le Moyen Age ; d’autres y travaillent encore et parmi eux il nous est permis de compter le secrétaire de la Société académique de Cherbourg. Je laisse donc cette grande tâche à des hommes plus capables que moi de la remplir ; je ne veux ici qu’examiner un coin de l’immense tableau que les siècles féodaux déroulent à nos yeux.\par

\astermono

\noindent Au XII\textsuperscript{e} siècle, ce qui a été appelé depuis le tiers état n’existait pour ainsi dire point encore. La population n’était divisée qu’en deux catégories : d’un côté ceux qui cultivaient le sol sans le posséder ; de l’autre ceux qui possédaient le sol sans le cultiver.\par
Quant à cette première classe de la population, j’imagine que, sous certains rapports, son sort était moins à plaindre que celui des hommes du peuple de nos jours. Ces hommes, qui en faisaient partie avec plus de liberté, d’élévation et de moralité que les esclaves de nos colonies, se trouvaient cependant dans une position analogue. Leurs moyens d’existence étaient presque toujours assurés ; l’intérêt du maître se rencontrait sur ce point d’accord avec le leur. Bornés dans leurs désirs aussi bien que dans leur pouvoir, sans souffrance pour le présent, tranquilles sur un avenir qui ne leur appartenait pas, ils jouissaient de ce genre de bonheur végétatif dont il est aussi difficile à l’homme très civilisé de comprendre le charme que de nier l’existence.\par
L'autre classe présentait un spectacle opposé. Là se rencontrait avec un loisir héréditaire l’usage habituel et assuré d’un grand superflu. Je suis loin de croire cependant qu’au sein même de cette classe privilégiée, la recherche des jouissances de la vie fût poussée aussi loin qu’on le suppose généralement. Le luxe peut facilement exister au sein d’une nation encore à moitié barbare, mais non l’aisance. L'aisance suppose une classe nombreuse dont tous les membres s’occupent simultanément à rendre la vie plus douce et plus aisée. Or, dans les temps dont je parle, le nombre de ceux que le soin de vivre ne préoccupait pas uniquement était très petit. L’existence de ces derniers était brillante, fastueuse, mais non commode. On mangeait avec ses doigts dans des plats d’argent ou acier ciselé ; les habits étaient couverts d’hermine et d’or et le linge était inconnu ; on logeait dans des palais dont l’humidité couvrait les murs, et l’on s’asseyait sur des sièges de bois richement sculptés près d’immenses foyers où se consumaient des arbres entiers sans répandre la chaleur autour d’eux. Je suis convaincu qu’il n’est pas aujourd’hui de ville de province dont les habitants aisés ne réunissent dans leur demeure plus de véritables commodités de la vie et ne trouvent plus de facilité à satisfaire les mille besoins que la civilisation fait naître, que le plus orgueilleux baron du Moyen Age.\par
Si nous attachons nos regards sur les siècles féodaux, nous découvrons donc que la grande majorité de la population vivait presque sans besoins et que le reste n’en éprouvait qu’un petit nombre. La terre suffisait pour ainsi dire à tous, l’aisance n’était nulle part ; partout le vivre.\par
Il était nécessaire de fixer ce point de départ pour faire bien comprendre ce que je vais dire.\par
\bigbreak
\noindent A mesure que le temps suit son cours, la population qui cultive la terre conçoit des goûts nouveaux. La satisfaction des plus grossiers besoins ne saurait plus la contenter. Le paysan, sans quitter ses champs, veut s’y trouver mieux logé, mieux couvert ; il a entrevu les douceurs de l’aisance et il désire se les procurer. D'un autre côté, la classe qui vit de la terre sans cultiver le sol étend le cercle de ses jouissances ; ses plaisirs sont moins fastueux, mais plus compliqués, plus variés. Mille besoins inconnus aux nobles du Moyen Age viennent aiguillonner leurs descendants. Un grand nombre d’hommes qui vivaient sur la terre et de la terre quittent alors les champs et trouvent moyen de pourvoir à leur existence en travaillant à satisfaire ces besoins nouveaux qui se manifestent. La culture, qui était l’occupation de tous, n’est plus que celle du plus grand nombre. A côté de ceux qui subsistent des produits du sol sans travailler, se place une classe nombreuse qui vit en travaillant de son industrie mais sans cultiver le sol.\par
Chaque siècle, en s’échappant des mains du Créateur, vient développer l’esprit humain, étendre le cercle de la pensée, augmenter les désirs, accroître la puissance de l’homme ; le pauvre et le riche, chacun dans sa sphère, conçoit l’idée de jouissances nouvelles qu’ignoraient leurs devanciers. Pour satisfaire ces nouveaux besoins auxquels la culture de la terre ne peut suffire, une portion de la population quitte chaque année les travaux des champs pour s’adonner à l’industrie.\par
Si l’on considère attentivement ce qui se passe en Europe depuis plusieurs siècles, on demeure convaincu qu’à mesure que la civilisation faisait des progrès, il s’opérait un grand déplacement dans la population. \labelchar{Les hommes quittaient la charrue pour prendre la navette et le marteau ; de la chaumière ils passaient dans la manufacture} ; en agissant ainsi, ils obéissaient aux lois immuables qui président à la croissance des sociétés organisées. On ne peut donc pas plus assigner un terme à ce mouvement qu’imposer des bornes à la perfectibilité humaine. La limite de l’un comme des autres n’est connue que de Dieu.\par
Quelle a été, quelle est la conséquence du mouvement graduel et irrésistible que nous venons de décrire ?\par
Une somme immense de biens nouveaux a été introduite dans le monde ; la classe qui était restée à la culture de la terre a trouvé à sa disposition une foule de jouissances que le siècle précédent n’avait pas connues ; la vie du cultivateur est devenue plus douce et plus commode ; la vie du grand propriétaire plus variée et plus ornée ; l’aisance s’est trouvée à la portée du plus grand nombre, mais ces heureux résultats n’ont point été obtenus sans qu’il fallût les payer.\par
J'ai dit qu’au Moyen Age l’aisance n’était nulle part, le vivre partout. Ce mot résume d’avance ce qui va suivre. Lorsque la presque totalité de la population vivait de la culture du sol, on rencontrait de grandes misères et des mœurs grossières, mais les besoins les plus pressants de l’homme étaient satisfaits. Il est très rare que la terre ne puisse au moins fournir à celui qui l’arrose de ses sueurs de quoi apaiser le cri de la faim. La population était donc misérable, mais elle vivait. Aujourd’hui la population est plus heureuse, mais il se rencontre toujours une minorité prête à mourir de besoin si l’appui du public vient à lui manquer.\par
Un pareil résultat est facile à comprendre. Le cultivateur a pour produit des denrées de première nécessité. Le débit peut en être plus ou moins avantageux, mais il est à peu près sûr ; et si une cause accidentelle empêche l’écoulement des produits du sol, ces fruits fournissent au moins de quoi vivre à celui qui les a recueillis et lui permettent d’attendre des temps meilleurs.\par
L'ouvrier, au contraire, spécule sur des besoins factices et secondaires que mille causes peuvent restreindre, que de grands évènements peuvent entièrement suspendre. quels que soient le malheur des temps, la cherté ou le bon marché des denrées, il faut à chaque homme une certaine somme de nourriture sans laquelle il languit et meurt, et l’on est toujours assuré de lui voir faire des sacrifices extraordinaires pour se les procurer ; mais des circonstances malheureuses peuvent porter la population à se refuser certaines jouissances, auxquelles elle se livrait sans peine en d’autres temps. Or c’est le goût et l’usage de ces jouissances sur lesquels l’ouvrier compte pour vivre. S'ils viennent à lui manquer, il ne lui reste aucune ressource. Sa moisson, à lui, est brûlée ; ses champs sont frappés de stérilité, et pour peu qu’un pareil état se prolonge, il n’aperçoit qu’une horrible misère et la mort.\par
Je n’ai parlé que du cas où la population restreindrait ses besoins. Beaucoup d’autres causes peuvent amener le même effet : une production exagérée chez les régnicoles, la concurrence des étrangers…\par

\labelblock{La classe industrielle a reçu de Dieu la mission spéciale et dangereuse de pourvoir à ses risques et périls au bonheur matériel de toutes les autres}

\noindent La classe industrielle qui sert si puissamment au bien-être des autres est donc bien plus exposée qu’elles aux maux subits et irrémédiables. Dans la grande fabrique des sociétés humaines, je considère la classe industrielle comme ayant reçu de Dieu la mission spéciale et dangereuse de pourvoir à ses risques et périls au bonheur matériel de toutes les autres. Or le mouvement naturel et irrémédiable de la civilisation tend sans cesse a augmenter a quantité comparative de ceux qui la composent. Chaque année, les besoins se multiplient et se diversifient, et avec eux croît le nombre des individus qui espèrent se créer une plus grande aisance en travaillant à satisfaire ces besoins nouveaux qu’en restant occupes de l’agriculture : grand sujet de méditation pour les hommes d’État de nos jours !\par
C'est à cette cause qu’il faut principalement attribuer ce qui se passe au sein des sociétés riches où l’aisance et l’indigence se rencontrent dans de plus grandes proportions qu’ailleurs. La classe industrielle, qui fournit aux jouissances du plus grand nombre, est exposée elle-même à des misères qui seraient presque inconnues, si cette classe n existait pas.\par
\bigbreak
\noindent Cependant d’autres causes encore contribuent au développement graduel du paupérisme.\par
\bigbreak
\noindent L'homme naît avec des besoins, et il se fait des besoins. Il tient les premiers de sa constitution physique, les seconds de l’usage et de l’éducation. J'ai montré qu’à l’origine des sociétés les hommes n’avaient guère que des besoins naturels, ne cherchant qu’à vivre ; mais à mesure que les jouissances de la vie sont devenues plus étendues, ils ont contracté l’habitude de se livrer à quelques-unes d’entre elles, et celles-là ont fini par leur devenir presque aussi nécessaires que la vie elle-même. Je citerai l’usage du tabac, parce le tabac est un objet de luxe qui a pénétré jusque dans les déserts et qui a créé parmi les sauvages une jouissance factice, qu’il faut à tout prix se procurer. Le tabac est presque plus indispensable aux Indiens que la nourriture ; ils sont aussi tentés de recourir à la charité de leurs semblables, quand ils sont privés de l’un, que quand l’autre leur manque. Ils ont donc une cause de mendicité inconnue à leurs pères. Ce que j’ai dit pour le tabac s’applique à une multitude d’objets dont on ne saurait se passer dans la vie civilisée. Plus une société est riche, industrieuse, prospère, plus les jouissances du plus grand nombre deviennent variées et permanentes ; plus elles sont variées et permanentes, plus elles s’assimilent par l’usage et l’exemple, à de véritables besoins. L'homme civilisé est donc infiniment plus exposé aux vicissitudes de la destinée que l’homme sauvage. Ce qui n’arrive au second que de loin en loin et dans quelques circonstances, peut arriver sans cesse et dans des circonstances très ordinaires au premier. Avec le cercle de ses jouissances, il a agrandi le cercle de ses besoins et il offre une plus large place aux coups de la fortune. De là vient que le pauvre d’Angleterre paraît presque riche au pauvre de France ; celui-ci à l’indigent espagnol. Ce qui manque à l’Anglais n’a jamais été en la possession du Français. Et il en est ainsi à mesure qu’on descend l’échelle sociale. Chez les peuples très civilisés, le manque d’une multitude de choses cause la misère ; dans l’ état sauvage, la pauvreté ne consiste qu’à ne pas trouver de quoi manger.\par
Les progrès de la civilisation n’exposent pas seulement les hommes à beaucoup de misères nouvelles ; ils portent encore la société à soulager des misères auxquelles, dans un État à demi-policé, on ne songerait pas. Dans un pays où la majorité est mal vêtue, mal logée, mal nourrie, qui pense à donner au pauvre un vêtement propre, une nourriture saine, une commode demeure ? Chez les Anglais, où le plus grand nombre, possesseur de tous ces biens, regarde comme un affreux malheur de ne pas en jouir, la société croit devoir venir au secours de ceux qui en sont privés, et guérit les maux qu’elle n’apercevrait même pas ailleurs.\par

\labelblock{En Angleterre, la moyenne des jouissances que doit espérer un homme dans la vie est placée plus haut que dans un autre pays du monde. Ceci facilite singulièrement l’extension du paupérisme dans ce royaume.}

\noindent Si toutes ces réflexions sont justes, on concevra sans peine que plus les nations sont riches, plus le nombre de ceux qui ont recours à la charité publique doit se multiplier, puisque deux causes très puissantes tendent à ce résultat : chez ces nations, la classe la plus naturellement exposée aux besoins augmente sans cesse, et d’un autre côté, les besoins s’augmentent et se diversifient eux-mêmes à l’infini ; l’occasion de se trouver exposé à quelques-uns devient plus fréquente chaque jour.\par
Ne nous livrons donc point à de dangereuses illusions, fixons sur l’avenir des sociétés modernes un regard calme et tranquille. Ne nous laissons pas enivrer par le spectacle de sa grandeur ; ne nous décourageons pas à la vue de ses misères. A mesure que le mouvement actuel de la civilisation se continuera, on verra croître les jouissances du plus grand nombre ; la société deviendra plus perfectionnée, plus savante ; l’existence sera plus aisée, plus douce, plus ornée, plus longue ; mais en même temps, sachons le prévoir, le nombre de ceux qui auront besoin de recourir à l’appui de leurs semblables pour recueillir une faible part de tous ces biens, le nombre de ceux-là s’accroîtra sans cesse. On pourra ralentir ce double mouvement ; les circonstances particulières dans lesquelles les différents peuples sont placés précipiteront ou suspendront son cours ; mais il n’est donné à personne de l’arrêter. Hâtons-nous donc de chercher les moyens d’atténuer les maux inévitables qu’il est déjà facile de prévoir.
\section[Seconde partie]{Seconde partie}
\noindent \initialiv{I}{l y a deux} espèces de bienfaisances : l’une, qui porte chaque individu à soulager, suivant ses moyens, les maux qui se trouvent à sa portée. Celle-là est aussi vieille que le monde ; elle a commencé avec les misères humaines ; le christianisme en a fait une vertu divine, et l’a appelée la charité.\par
L'autre, moins instinctive, plus raisonnée, moins enthousiaste, et souvent plus puissante, porte la société elle-même à s’occuper des malheurs de ses membres et à veiller systématiquement au soulagement de leurs douleurs. Celle-ci est née du protestantisme et ne s’est développée que dans les sociétés modernes.\par
La première est une vertu privée, elle échappe à l’action sociale ; la seconde est au contraire produite et régularisée par la société. C'est donc de celle-là qu’il faut spécialement nous occuper.\par

\labelblock{Il n’y a pas, au premier abord, d’idée qui paraisse plus belle et plus grande que celle de la charité publique.}

\noindent La société, jetant un regard continu sur elle-même, sondant chaque jour ses blessures et s’occupant à les guérir ; la société, en même temps qu’elle assure aux riches la jouissance de leurs biens, garantissant les pauvres de l’excès de leur misère, demande aux uns une portion de leur superflu pour accorder aux autres le nécessaire. Il y a certes là un grand spectacle en présence duquel l’esprit s’élève et l’âme ne saurait manquer d’être émue.\par
Pourquoi faut-il que l’expérience vienne détruire une partie de ces belles illusions ?\par
Le seul pays de l’Europe qui ait systématisé et appliqué en grand les théories de la charité publique est l’Angleterre.\par
A l’époque de la révolution religieuse qui changea la face de l’Angleterre, sous Henri VIII, presque toutes les communautés charitables du royaume furent supprimées, et comme les biens de ces communautés passèrent aux nobles et ne furent point partagés entre les mains du peuple, il s’ensuivit que le nombre de pauvres alors existants resta le même, tandis que les moyens de pourvoir à leurs besoins étaient en partie détruits. Le nombre des pauvres s’accrut donc outre mesure, et Élisabeth, la fille de Henri VIII, frappée de l’aspect repoussant des misères du peuple, songea à substituer aux aumônes que la suppression des couvents avait fort réduites, une subvention annuelle, fournie par les communes.\par
Une loi \footnote{\noindent \emph{(Note de Tocqueville)} Voyez\par

\begin{enumerate}[itemsep=0pt,]
\item Blackstone, livre I, ch. IV
\item Les principaux résultats de l’enquête faite en 1833 sur l’état des pauvres, contenus dans le livre intitulé : \emph{Extracts from the infotmation received by Ris Majesty's commissioners as to the administration and operation of the Poor-laws} ;
\item \emph{The report of the Poorlaws commissioners ;}
\item Et enfin la loi de 1834 qui a été le résultat de tous ces travaux.

\end{enumerate}} promulguée dans la quarante-troisième année du règne de cette princesse dispose que dans chaque paroisse des inspecteurs des pauvres seront nommés ; que ces inspecteurs auront le droit de taxer les habitants à l’effet de nourrir les indigents infirmes, et de fournir du travail aux autres. A mesure que le temps avançait dans sa marche, l’Angleterre était de plus en plus entraînée à adopter le principe de la charité légale. Le paupérisme croissait plus rapidement dans la Grande-Bretagne que partout ailleurs. Des causes générales et d’autres spéciales à ce pays produisaient ce triste résultat. Les Anglais ont devancé les autres nations de l’Europe dans la vie de la civilisation ; toutes les réflexions que j’ai faites précédemment leur sont donc particulièrement applicables, mais il en est d’autres qui ne se rapportent qu’à eux seuls.\par
\bigbreak
\noindent La classe industrielle d’Angleterre ne pourvoit pas seulement aux besoins et aux jouissances du peuple anglais, mais d’une grande partie de l’humanité. Son bien-être ou ses misères dépendent donc non seulement de ce qui arrive dans la Grande-Bretagne, mais en quelque façon de tout ce qui se passe sous le soleil. Lorsqu’un habitant des Indes réduit sa dépense et resserre sa consommation, il y a un fabricant anglais qui souffre. L’Angleterre est donc le pays du monde où l’agriculteur est tout à la fois le plus puissamment attiré vers les travaux de l’industrie et s’y trouve le plus exposé aux vicissitudes de la fortune.\par
Il arrive depuis un siècle, chez les Anglais, un événement qu’on peut considérer comme un phénomène, si l’on fait attention au spectacle offert par le reste du monde. Depuis cent ans, la propriété foncière se divise sans cesse dans les pays connus ; en Angleterre, elle s’agglomère sans cesse. Les terres de moyenne grandeur disparaissent dans les vastes domaines, la grande culture succède à la petite. Il y aurait sur ce sujet à donner des explications qui peut-être ne manqueraient pas de quelque intérêt, mais elles m’écarteraient de mon sujet : le fait me suffit, il est constant. Il en résulte que, tandis que l’agriculteur est sollicité par son intérêt de quitter la charrue et d’entrer dans les manufactures, il est, en quelque façon, poussé malgré lui à le faire par l’agglomération de la propriété foncière. Car, proportion gardée, il faut infiniment moins de travailleurs pour cultiver un grand domaine qu’un petit champ. La terre lui manque et l’industrie l’appelle. Ce double mouvement l’entraîne. Sur vingt-cinq millions d’habitants qui peuplent la Grande-Bretagne, il n’y en a plus que neuf millions qui s’occupent à cultiver le sol ; quatorze ou près des deux tiers suivent les chances périlleuses du commerce et de l’industrie \footnote{\emph{(Note de Tocqueville)} En France, la classe industrielle ne forme encore qu’un quart de la population. }. Le paupérisme a donc dû croître plus vite en Angleterre que dans des pays dont la civilisation eût été égale à celle des Anglais. L'Angleterre, ayant une fois admis le principe de la charité légale, n’a pu s’en départir. Ainsi la législation anglaise des pauvres ne présente-t-elle, depuis deux cents ans, qu’un long développement des lois d’Elizabeth. Près de deux siècles et demi se sont écoulés depuis que le principe de la charité égale a été pleinement admis chez nos voisins, et l’on peut juger maintenant les conséquences fatales qui ont découlé de l’adoption de ce principe. Examinons-les successivement.\par
Le pauvre, ayant un droit absolu aux secours de la société, et trouvant en tous lieux une administration publique organisée pour les lui fournir, on vit bientôt renaître et se généraliser dans une contrée protestante les abus que la Réforme avait reprochés avec raison à quelques-uns des pays catholiques. L'homme, comme tous les êtres organisés, a une passion naturelle pour l’oisiveté. Il y a pourtant deux motifs qui le portent au travail : le besoin de vivre, le désir d’améliorer les conditions de l’existence. L’expérience a prouvé que la plupart des hommes ne pouvaient être suffisamment excités au travail que par le premier de ces motifs, et que le second n’était puissant que sur un petit nombre. Or un établissement charitable, ouvert indistinctement à tous ceux qui sont dans le besoin, ou une loi qui donne à tous les pauvres, quelle que soit l’origine de la pauvreté, un droit au secours du public, affaiblit ou détruit le premier stimulant et ne laisse intact que le second. Le paysan anglais comme le paysan espagnol, s’il ne se sent pas le vif désir de rendre meilleure la position dans laquelle il est né et de sortir de sa sphère, désir timide et qui avorte aisément chez la plupart des hommes, - le paysan de ces deux contrées, dis-je, n’a point d’intérêt au travail, ou, s’il travaille, il n’a pas d’intérêt à l’épargne ; il reste donc oisif, ou dépense inconsidérément le fruit précieux de ses labeurs. Dans l’un ou l’autre de ces pays, on arrive par des causes différentes à ce même résultat, que c’est la partie la plus généreuse, la plus active, la plus industrieuse de la nation, qui consacre ses secours à fournir de quoi vivre à ceux qui ne font rien ou font un mauvais usage de leur travail.\par
\bigbreak
\noindent Nous voilà certes bien loin de la belle et séduisante théorie que j’exposais plus haut. Est-il possible d’échapper à ces conséquences funestes d’un bon principe ? Pour moi, j’avoue que je les considère comme inévitables.\par
\bigbreak
\noindent Ici l’on m’arrête en disant : vous supposez que, quelle que soit la cause de la misère, la misère sera secourue ; vous ajoutez que les secours du public soustrairont les pauvres à l’obligation de travail ; c’est poser en fait ce qui reste douteux. Qui empêche la société, avant d’accorder le secours, de s’enquérir des causes du besoin ? Pourquoi la condition du travail ne serait pas imposée à l’indigent valide qui s’adresse à la pitié du public ? Je réponds que les lois anglaises ont conçu l’idée de ces palliatifs ; mais elles ont échoué, et cela se comprend sans peine.\par
Il n’y a rien de si difficile à distinguer que les nuances qui sépare un malheur immérité d’une infortune que le vice a produite. Combien de misères sont à la fois le résultat de ces deux causes ! Quelle connaissance approfondie du caractère de chaque homme et des circonstances dans lesquelles il a vécu suppose le jugement d’un pareil point ; que de lumières, quel discernement sûr, quelle raison froide et inexorable ! Ou trouver le magistrat qui aura la conscience, le temps, le talent, les moyens de se livrer à un pareil examen ? Qui osera laisser mourir de faim le pauvre parce que celui-ci meurt par sa faute ? Qui entendra ses cris et raisonnera sur ses vices ? A l’aspect des misères de nos semblables, l’intérêt personnel lui-même se tait ; l’intérêt du trésor public en serait-il plus puissant ? Et si l’âme du surveillant des pauvres demeurait inaccessible à ces émotions, toujours belles, lors même qu’elles égarent, restera-t-elle fermée à la crainte ? Tenant entre ses mains les douleurs ou les joies, la vie ou la mort d’une portion considérable de ses semblables, de la portion la plus désordonnée, la plus turbulente, la plus grossière, ne reculera-t-il pas devant l’exercice de ce terrible pouvoir ? Et si l’on rencontre l’un de ces hommes intrépides, en trouvera-t-on plusieurs ? Cependant de pareilles fonctions ne peuvent être exercées que sur un petit territoire ; il faut donc en revêtir un grand nombre de citoyens. Les Anglais ont été obligés de placer des surveillants des pauvres dans chaque commune. Qu'arrive-t-il donc infailliblement de tout ceci ? La misère étant constatée, les causes de la misère restent incertaines : l’une résulte d’un fait patent, l’autre prouvée par un raisonnement toujours contestable ; le secours ne pouvant faire qu’un tort éloigné à la société, le refus du secours un mal instantané aux pauvres et au surveillant lui-même, le choix de ce dernier ne sera pas douteux. Les lois auront déclaré que la misère innocente sera seule secourue, la pratique viendra au secours de toutes les misères. Je ferai des raisonnements analogues et également appuyés sur l’expérience quant au second point.\par
\bigbreak
\noindent On veut que l’aumône soit le prix du travail. Mais d’abord existe-t-il toujours des travaux publics à faire ? Sont-ils également répartis sur toute la surface du pays, de manière qu’on ne voie jamais dans un district beaucoup de travaux à exécuter et peu de personnes à pourvoir ; dans un autre, beaucoup d’indigents à secourir et peu de travaux à exécuter ? Si cette difficulté se présente à toutes les époques, ne devient-elle pas insurmontable lorsque, par suite du développement progressif de la civilisation, des progrès de la population, de l’effet de la loi des pauvres elle-même, le nombre des indigents atteint comme en Angleterre le sixième, d’autres disent le quart de la population totale ?\par
Mais en supposant même qu’il se rencontrât toujours des travaux à exécuter, qui se chargera d’en constater l’urgence, d’en suivre l’exécution, d’en fixer le prix ? Le surveillant, cet homme, indépendamment des qualités d’un grand magistrat, aura donc les talents, l’activité, les connaissances spéciales d’un bon entrepreneur d’industrie ; il trouvera dans le sentiment du devoir ce que l’intérêt personnel lui-même serait peut-être impuissant à créer : le courage de contraindre à des efforts productifs et continus la portion la plus inactive et la plus vicieuse de la population. Serait-il sage de s’en flatter ? Est-il raisonnable de le croire ? Sollicité par les besoins du pauvre, le surveillant imposera un travail fictif, ou même, comme cela se pratique presque toujours en Angleterre, donnera le salaire sans exiger le travail. Il faut que les lois soient faites pour les hommes et non en vue d’une perfection idéale que la nature humaine ne comporte pas, ou dont elle ne présente que de loin en loin des modèles.\par
Toute mesure qui fonde la charité légale sur une base permanente et qui lui donne une forme administrative crée donc une classe oisive et paresseuse, vivant aux dépens de la classe industrielle et travaillante. C'est là, sinon son résultat immédiat, du moins sa conséquence inévitable. Elle reproduit tous les vices du système monacal, moins les hautes idées de moralité et de religion qui souvent venaient s’y joindre. Une pareille loi est un germe empoisonné, déposé au sein de la législation ; les circonstances, comme en Amérique, peuvent empêcher le germe de prendre des développements rapides, mais non le détruire, et si la génération actuelle échappe à son influence, il dévorera le bien-être des générations à venir.\par

\astermono

\noindent Si vous étudiez de près l’état des populations chez lesquelles une pareille législation est depuis longtemps en vigueur, vous découvrirez sans peine que les effets n’agissent pas d’une manière moins fâcheuse sur la moralité que sur la prospérité publique, et qu’elle déprave les hommes plus encore qu’elle ne les appauvrit.\par

\labelblock{Il n’y a rien qui, en général, élève et soutient plus haut l’esprit humain que l’idée des droits. On trouve dans l’idée du droit quelque chose de grand et de viril qui ôte à la demande son caractère suppliant, et place celui qui réclame sur le même niveau que celui qui accorde.}

\noindent Mais le droit qu’a le pauvre d’obtenir les secours de la société a cela de particulier, qu’au lieu d’élever le cœur de l’homme qui l’exerce, il l’abaisse. Dans les pays où la législation n’ouvre pas un pareil recours, le pauvre, en s’adressant à la charité individuelle, reconnaît, il est vrai, son état d’infériorité par rapport au reste de ses semblables ; mais il le reconnaît en secret et pour un temps ; du moment où un indigent est inscrit sur la liste des pauvres de sa paroisse, il peut, sans doute, réclamer avec assurance des secours ; mais qu’est-ce que l’obtention de ce droit, sinon la manifestation authentique de la misère, de la faiblesse, de l’inconduite de celui qui en est revêtu ? Les droits ordinaires sont conférés aux hommes en raison de quelque avantage personnel acquis par eux sur leurs semblables. Celui-ci est accordé en raison d’une infériorité reconnue. Les premiers mettent cet avantage en relief et le constatent ; le second place en lumière cette infériorité et la légalise.\par
Plus les uns sont grands et assurés, plus ils honorent ; plus l’autre est permanent et \emph{étendu}, plus il dégrade.\par
Le pauvre qui réclame l’aumône au nom de la loi est donc dans une position plus humiliante encore que l’indigent qui la demande à la pitié de ses semblables au nom de celui qui voit d’un même oeil et qui soumet à d’égales lois [Dieu] le pauvre et le riche.\par
Mais ce n’est pas tout encore\par

\labelblock{L’aumône individuelle établit des liens précieux entre le riche et le pauvre. Le premier s’intéresse par le bienfait même au sort de celui dont il a entrepris de soulager la misère ; le second, soutenu par des secours qu’il n’avait pas droit d’exiger et que peut-être il n’espérait pas obtenir, se sent attiré par la reconnaissance.}

\noindent Un lien moral s’établit entre ces deux classes que tant d’intérêts et de passions concourent à séparer, et, divisées par la fortune leur volonté les rapproche ; il n’en est point ainsi dans la charité légale.\par
Celle-ci laisse subsister l’aumône, mais elle lui ôte sa moralité. Le riche, que la loi dépouille d’une partie de son superflu sans le consulter, ne voit dans le pauvre qu’un avide étranger appelé par le législateur, au partage de ses biens. Le pauvre, de son côté, ne sent aucune gratitude pour un bienfait qu’on ne peut lui refuser et qui ne saurait d’ailleurs le satisfaire ; car l’aumône publique, qui assure la vie, ne la rend pas plus heureuse et plus aisée que ne le ferait l’aumône individuelle ; la charité légale n’empêche donc point qu’il n’y ait dans la société des pauvres et des riches, que les uns ne jettent autour d’eux des regards pleins de haine et de crainte, que les autres ne songent à leurs maux avec désespoir et avec envie. Loin de tendre à unir dans un même peuple ces deux nations rivales qui existent depuis le commencement du monde et qu’on appelle les riches et les pauvres, elle brise le seul lien qui pouvait s’établir entre elles, elle les range chacune sous sa bannière ; elle les compte et, les mettant en présence, elle les dispose au combat.\par
\bigbreak
\noindent J'ai dit que le résultat inévitable de la charité légale était de maintenir dans l’oisiveté le plus grand nombre des pauvres et d’entretenir leurs loisirs aux dépens de ceux qui travaillent.\par
Si l’oisiveté dans la richesse, l’oisiveté héréditaire, achetée par des services ou des travaux, l’oisiveté entourée de la considération publique, accompagnée du contentement d’esprit, intéressée par les plaisirs de l’intelligence, moralisée par l’exercice de la pensée : si cette oisiveté, dis-je, a été la mère de tant de vices, que sera-ce d’une oisiveté dégradée acquise par la lâcheté, méritée par l’inconduite, dont on jouit au milieu de l’ignominie et qui ne devient supportable qu’à mesure que l’âme de celui qui la souffre achève de se corrompre et de se dégrader ?\par
Qu'espérer d’un homme dont la position ne peut s’améliorer, car il a perdu la considération de ses semblables, qui est la condition première de tous les progrès ; dont la fortune ne saura devenir pire, car s’étant réduit à la satisfaction des plus pressants besoins, il est assuré qu’ils seront toujours satisfaits ? Quelle action reste-t-il à la conscience et à l’activité humaines dans un être ainsi borné de toutes parts, qui vit sans espoir et sans crainte parce qu’il connaît l’avenir, comme fait l’animal, parce qu’il ignore les circonstances de la destinée ; concentré ainsi que lui dans le présent et dans ce que le présent peut offrir de jouissances ignobles et passagères à une nature abrutie ?\par
Lisez tous les livres écrits en Angleterre sur le paupérisme ; étudiez les enquêtes ordonnées par le Parlement britannique ; parcourez les discussions qui ont eu lieu à la Chambre les Lords et à celle des communes sur cette difficile question ; une seule plainte retentira à vos oreilles : on déplore l’état de dégradation où sont tombées les classes inférieures de ce grand peuple ! le nombre des enfants naturels augmente sans cesse, celui des criminels s’accroît rapidement ; la population indigente se développe outre mesure ; l’esprit de prévoyance et d’épargne se montre de plus en plus étranger au pauvre ; tandis que dans le reste de la nation les lumières se répandent, les mœurs s’adoucissent, les goûts deviennent plus délicats, les habitudes plus polies, - lui, reste immobile, ou plutôt il rétrograde ; on dirait qu’il recule vers la barbarie, et, placé au milieu des merveilles de la civilisation, il semble se rapprocher par ses idées et par ses penchants de l’homme sauvage.\par

\labelblock{La charité légale n’exerce pas une moins funeste influence sur la liberté du pauvre que sur sa moralité.}

\noindent Ceci se démontre aisément : du moment où l’on fait aux communes un devoir strict de secourir les indigents, il s’ensuit immédiatement et forcément cette conséquence que les communes ne doivent des secours qu’aux pauvres qui sont domiciliés sur leur territoire ; c’est le seul moyen équitable d’égaliser la charge publique qui résulte de la loi, et de la proportionner aux moyens de ceux qui doivent la supporter. Or, comme dans un pays où la charité publique est organisée, la charité individuelle est à peu près inconnue, il en résulte que celui que des malheurs ou des vices rendent incapable de gagner sa vie est condamné, sous peine de mort, à ne pas quitter le lieu où il est né. S'il s’en éloigne, il ne marche qu’en pays ennemi ; l’intérêt individuel des communes, bien autrement puissant et bien plus actif que ne saurait l’être la police nationale la mieux organisée, dénonce son arrivée, épie ses démarches, et s’il veut se fixer dans un nouveau séjour, le désigne à la force publique qui le ramène au lieu du départ. Par leur législation sur les pauvres, les Anglais ont \emph{immobilisé} un sixième de leur population. Ils l’ont attaché à la terre comme l’étaient les paysans du Moyen Age. La glèbe \emph{forçait} l’homme à rester \emph{malgré sa volonté} dans le lieu de sa naissance ; la charité légale \emph{l’empêche de vouloir} s’en éloigner. Je ne vois que cette différence entre les deux systèmes. Les Anglais ont été plus loin, et ils ont tiré du principe de la bienfaisance publique des conséquences plus funestes encore et auxquelles je pense qu’il est permis d’échapper. Les communes anglaises sont tellement préoccupées de la crainte qu’un indigent ne vienne tomber à leur charge et n’obtienne un domicile dans leur sein, que quand un étranger dont l’extérieur n’annonce pas l’opulence s’établit momentanément au milieu d’elles, ou lorsqu’un malheur inattendu vient à le frapper, l’autorité municipale se hâte de lui faire demander caution contre la misère à venir, et si l’étranger ne peut fournir cette caution, il faut qu’il s’éloigne.\par
Ainsi la charité légale n’a pas seulement ravi la liberté locomotrice aux pauvres d’Angleterre, mais à tous ceux que la pauvreté menace.\par
\bigbreak
\noindent Je ne saurais, je pense, mieux compléter ce triste tableau qu’en transcrivant ici le morceau suivant et que je trouve dans mes notes sur l’Angleterre.\par
Je parcourais en 1833 la Grande-Bretagne. D'autres étaient frappés de la prospérité intérieure du pays : moi, je songeais à l’inquiétude secrète qui travaillait visiblement l’esprit de tous ses habitants. Je pensais que de grandes misères devaient se cacher sous ce manteau brillant que l’Europe admire. Cette idée me porta à examiner avec une attention toute particulière le paupérisme, cette plaie hideuse et immense qui est attachée à un corps plein de vigueur et de santé.\par
J'habitais alors la maison d’un grand propriétaire du sud de l’Angleterre ; c’était le temps où les juges de paix se réunissent pour prononcer sur les réclamations que font entendre les pauvres contre leurs communes, ou les communes contre les pauvres. Mon hôte était juge de paix, et je le suivais régulièrement au tribunal. Je trouve dans mes notes de voyage cette peinture de la première audience à laquelle j’assistai ; elle résume en quelques mots et met en relief tout ce qui précède. Je transcris avec une extrême exactitude afin de laisser au tableau le simple cachet de la vérité.\par

\begin{quoteblock}
\noindent « Le premier individu qui se présente devant les juges de paix est un vieillard ; sa figure est fraîche et vermeille, il est coiffé d’une perruque et couvert d’un excellent habit noir, il a tout l’air d’un rentier, il s’approche pourtant de la barre et réclame avec emportement contre l’injustice des administrateurs de sa commune. Cet homme est un pauvre, et l’on vient de diminuer injustement la part qu’il recevait dans la charité publique. On remet la cause pour entendre les administrateurs de la commune.\par
« Après ce frais et pétulant vieillard paraît ne jeune femme enceinte, dont les vêtements annoncent une pauvreté récente et qui porte sur ses traits flétris l’empreinte des douleurs. Elle expose que son mari est parti, il y a quelques jours pour un voyage de mer, que depuis lors elle n’a reçu de lui ni nouvelles ni secours, elle réclame l’aumône publique, mais l’administrateur des pauvres hésite à la lui accorder. Le beau-père de cette femme est un marchand aisé, il habite la ville même où le tribunal tient ses séances, et on espère aussi qu’il voudra bien, dans l’absence de son fils, se charger de l’entretien de sa belle-fille ; les juges de paix font venir cet homme ; mais il refuse de remplir les devoirs que la nature lui impose et que la loi ne lui commande pas. Les magistrats insistent ; ils cherchent à faire naître le remords ou la compassion dans l’âme égoïste de cet homme, leurs efforts échouent, et la commune est condamnée à payer le secours qu’on réclame.\par
« Après cette pauvre femme abandonnée, viennent cinq ou six hommes grands et vigoureux. Ils sont dans la force de la jeunesse, leur démarche est ferme et presque insultante. Ils se plaignent des administrateurs de leurs villages qui refusent de leur donner du travail, ou, à défaut de travail, un secours.\par
« Les administrateurs répliquent que la commune n’a en ce moment aucuns travaux à exécuter ; et quant au secours gratuit, il n’est pas dû, disent-ils, parce que les demandeurs trouveraient facilement un emploi de leur industrie chez les particuliers s’ils le voulaient. »
\end{quoteblock}

\noindent Lord X, avec lequel j’étais venu, me dit :\par

\begin{quoteblock}
\noindent « Vous venez de voir dans un cadre étroit une partie des nombreux abus que produit la loi des pauvres. Ce vieillard, qui s’est présenté le premier, a très probablement de quoi vivre, mais il pense qu’il a le droit d’exiger qu’on l’entretienne dans l’aisance, et il ne rougit pas de réclamer la charité publique, qui a perdu aux yeux du peuple son caractère pénible et humiliant. Cette jeune femme, qui paraît honnête et malheureuse, serait certainement secourue par son beau-père si la loi des pauvres n’existait pas, mais l’intérêt fait taire chez ce dernier le cri de la honte, et il se décharge sur le public d’une dette qu’il devrait seul acquitter. Quant à ces jeunes gens qui se sont présentés les derniers, je les connais, ils habitent mon village : ce sont de très dangereux citoyens, et de fait, mauvais sujets ; ils dissipent en peu d’instants dans les cabarets l’argent qu’il gagnent parce qu’ils savent que l’État viendra à leur secours ; ainsi, vous voyez qu’à la première gêne, causée par leur faute, ils s’adressent à nous. »
\end{quoteblock}


\begin{quoteblock}
\noindent « L'audience continua. Une jeune femme se présente à la barre, le surveillant des pauvres de sa commune la suit, un enfant l’accompagne ; elle s’approche sans donner le moindre signe d’hésitation, la pudeur ne fait pas même incliner son regard. Le surveillant l’accuse d’avoir eu en commerce illégitime l’enfant qu’elle porte dans ses bras.\par
« Elle en convient sans peine. Comme elle est indigente, et que l’enfant naturel, si le père restait inconnu, tomberait, avec sa mère, à la charge de la commune, le surveillant la somme de nommer le père ; le tribunal lui fait prêter serment. Elle désigne un paysan du voisinage. Celui-ci, qui est présent à l’audience, reconnaît très complaisamment l’exactitude du fait, et les juges de paix le condamnent à entretenir l’enfant. Le père, la mère se retirent sans que cet incident soulève la moindre émotion dans l’assemblée accoutumée à de semblables spectacles.\par
« Après cette jeune femme s’en présente une autre. Celle-ci vient volontairement ; elle aborde les magistrats avec la même insouciance effrontée qu’a montrée la première.\par
« Elle se déclare enceinte et nomme le père de l’enfant qui doit naître ; cet homme est absent. Le tribunal remet à un autre jour pour le faire citer. »
\end{quoteblock}

\noindent Lord X me dit :\par

\begin{quoteblock}
\noindent « Voici encore de funestes effets produits par les mêmes lois. La conséquence la plus directe de la législation sur les pauvres est de mettre à la charge du public l’entretien des enfants abandonnés qui sont les plus nécessiteux de tous les indigents. De là est né le désir de décharger les communes de l’entretien des enfants naturels que leurs parents seraient en état de nourrir. De là aussi cette recherche de la paternité provoquée par les communes et dont la preuve est délaissée à la femme. Car quel autre genre de preuve peut-on se flatter d’obtenir en pareille matière ? En obligeant les communes à se charger des enfants naturels et en leur permettant de rechercher la paternité, afin d’alléger ce poids accablant, nous avons facilité autant qu’il était en nous l’inconduite des femmes dans les basses classes. La grossesse illégitime doit presque toujours améliorer leur situation matérielle. Si le père de l’enfant est riche, elles peuvent se décharger sur lui du soin d’élever le fruit de leurs communes erreurs ; s’il est pauvre, elles confient ce soin à la société : les secours qu’on leur accorde de part ou d’autre dépassent presque toujours les dépenses du nouveau-né. Elles s’enrichissent donc par leurs vices mêmes, et il arrive souvent que la fille qui a été plusieurs fois mère fait un mariage plus avantageux que la jeune vierge qui n’a que ses vertus à offrir. La première a trouvé une sorte de dot dans son infamie. »
\end{quoteblock}

\noindent Je répète que je n’ai rien voulu changer à ce passage de mon journal ; je l’ai reproduit dans les mêmes termes, parce qu’il m’a semblé qu’il rendait avec simplicité et vérité les impressions que je voudrais faire partager au lecteur.\par
Depuis mon voyage en Angleterre, la loi des pauvres a été modifiée. Beaucoup d’Anglais se flattent que ces changements exerceront une grande influence sur le sort des indigents, sur leur moralité, sur leur nombre. Je voudrais pouvoir partager ces espérances, mais je ne saurais le faire. Les Anglais de nos jours ont consacré de nouveau dans la nouvelle loi le principe admis il y a deux cent cinquante ans par Élisabeth. Comme cette princesse, ils ont imposé à la société l’obligation de nourrir le pauvre. C'en est assez ; tous les abus que j’ai essayé de décrire sont renfermés dans le premier principe comme le plus grand chêne dans le gland qu’un enfant peut cacher dans sa main. Il ne lui faut que du temps pour se développer et pour croître. Vouloir établir une loi qui vienne d’une manière régulière, permanente, uniforme au secours des indigents, sans que le nombre des indigents augmente, sans que leur paresse croisse avec leurs besoins, leur oisiveté avec leurs vices, c’est planter le gland et s’étonner qu’il en paraisse une tige, puis des feuilles, plus tard des fleurs, enfin des fruits qui, se répandant au loin, feront sortir un jour une verte forêt des entrailles de la terre.\par
\bigbreak
\noindent Je suis certes bien loin de vouloir faire ici le procès à la bienfaisance qui est tout à la fois la plus naturelle, la plus belle et la plus sainte des vertus. Mais je pense qu’il n’est pas de principe si bon dont on puisse admettre comme bonnes toutes les conséquences. Je crois que la bienfaisance doit être une vertu mâle et raisonnée, non un goût faible et irréfléchi ; qu’il ne faut pas faire le bien qui plaît le plus à celui qui donne, mais le plus véritablement utile à celui qui reçoit ; non pas celui qui soulage le plus complètement les misères de quelques-uns, mais celui qui sert au bien-être du plus grand nombre. Je ne saurais calculer la bienfaisance que de cette manière ; comprise dans un autre sens, elle est encore un instinct sublime, mais elle ne mérite plus à mes yeux le nom de vertu.\par
Je reconnais que la charité individuelle produit presque toujours des effets utiles. Elle s’attache aux misères les plus grandes, elle marche sans bruit derrière la mauvaise fortune, et répare à l’improviste et en silence les maux que celle-ci a faits. Elle se montre partout où il y a les malheureux à secourir ; elle croît avec leurs souffrances, et cependant on ne peut sans imprudence compter sur elle, car mille accidents pourront retarder ou arrêter sa marche ; on ne sait où la rencontrer, et elle n’est point avertie par le cri de toutes les douleurs.\par
J'admets que l’association des personnes charitables, en régularisant les secours, pourrait donner à la bienfaisance individuelle plus d’activité et plus de puissance ; je reconnais non seulement l’utilité, mais la nécessité d’une charité publique appliquée à des maux inévitables, tels que la faiblesse de l’enfance, la caducité de la vieillesse, la maladie, la folie ; j’admets encore son utilité momentanée dans des temps de calamités publiques qui de loin en loin échappent des mains de Dieu, et viennent annoncer aux nations sa colère. L'aumône de l’État est alors aussi instantanée, aussi imprévue, aussi passagère que le mal lui-même.\par
J'entends encore la charité publique ouvrant les écoles aux enfants des pauvres et fournissant gratuitement à l’intelligence les moyens d’acquérir par le travail les biens du corps.\par
\bigbreak
\noindent Mais je suis profondément convaincu que tout système régulier, permanent, administratif, dont le but sera de pourvoir aux besoins du pauvre, fera naître plus de misères qu’il n’en peut guérir, dépravera la population qu’il veut secourir et consoler, réduira avec le temps les riches à n’être que les fermiers des pauvres, tarira les sources de l’épargne, arrêtera l’accumulation des capitaux, comprimera l’essor du commerce, engourdira l’activité et l’industrie humaines et finira par amener une révolution violente dans l’État, lorsque le nombre de ceux qui reçoivent l’aumône sera devenu presque aussi grand que le nombre de ceux qui la donnent, et que l’indigent ne pouvant plus tirer des riches appauvris de quoi pourvoir à ses besoins trouvera plus facile de les dépouiller tout à coup de leurs biens que de demander leurs secours.\par

\astermono

\noindent Résumons en peu de mots tout ce qui précède.\par
La marche progressive de la civilisation moderne augmente graduellement, et dans une proportion plus ou moins rapide, le nombre de ceux qui sont portés à recourir à la charité.\par
Quel remède apporter à de pareils maux ?\par
L'aumône légale se présente d’abord à l’esprit, l’aumône légale sous toutes ses formes, tantôt gratuite, tantôt cachée sous la forme d’un salaire, tantôt accidentelle et passagère dans certains temps, tantôt régulière et permanente dans d’autres. Mais un examen approfondi ne tarde pas à démontrer que ce remède, qui semble tout à la fois si naturel et si efficace, est d’un emploi très dangereux ; qu’il n’apporte qu’un soulagement trompeur et momentané aux douleurs individuelles, et qu’il envenime les plaies de la société, quelle que soit la manière dont on l’emploie.\par
Reste donc la charité particulière ; celle-là ne saurait produire que des effets utiles. Sa faiblesse même garantit contre ses dangers ; elle soulage beaucoup de misères et n’en fait point naître. Mais en présence du développement progressif des classes industrielles et de tous les maux que la civilisation mélange aux biens inestimables qu’elle produit, la charité individuelle paraît bien faible. Suffisante au Moyen Age, quand l’ardeur religieuse lui donnait une immense énergie, et lorsque sa tâche était moins difficile à remplir, le deviendrait-elle de nos jours où le fardeau qu’elle doit supporter est lourd, et où ses forces sont affaiblies ? La charité individuelle est un agent puissant que la société ne doit point mépriser, mais auquel il serait imprudent de se confier : elle est un des moyens et ne saurait être le seul.\par
Que reste-t-il donc à faire ? De quel côté tourner ses regards ? Comment adoucir les maux qu’on a la faculté de prévoir, mais non de guérir ?\par
Jusqu’ici j’ai examiné les moyens lucratifs de la misère. Mais n’existe-t-il que cet ordre de moyens ? Après avoir songé à soulager les maux, ne serait-il pas utile de chercher à les prévenir ? Ne saurait-on empêcher le déplacement rapide de la population, de telle sorte que les hommes ne quittent la terre et ne passent à l’industrie qu’autant que cette dernière peut facilement répondre à leurs besoins ? La somme des richesses nationales ne peut-elle continuer à augmenter sans qu’une partie de ceux qui produisent ces richesses aient à maudire la prospérité qu’ils font naître ? Est-il impossible d’établir un rapport plus fixe et plus régulier entre la production et la consommation des matières manufacturées : Ne peut-on pas faciliter aux classes ouvrières l’accumulation de l’épargne qui, dans des temps de calamité industrielle, leur permette d’attendre sans mourir le retour de la fortune ?\par
Ici l’horizon s’étend de toutes parts devant moi. Mon sujet s’agrandit ; je vois une carrière qui s’ouvre, mais je ne puis dans ce moment la parcourir. Le présent mémoire, trop court pour ce que j’avais à traiter excède déjà cependant les bornes que j’avais cru devoir me prescrire. Les mesures à l’aide desquelles on peut espérer de combattre d’une manière préventive le paupérisme feront l’objet d’un second ouvrage dont je compte faire hommage l’année prochaine à la société académique de Cherbourg.



\end{document}
