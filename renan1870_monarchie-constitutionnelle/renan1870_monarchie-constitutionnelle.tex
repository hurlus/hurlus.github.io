%%%%%%%%%%%%%%%%%%%%%%%%%%%%%%%%%
% LaTeX model https://hurlus.fr %
%%%%%%%%%%%%%%%%%%%%%%%%%%%%%%%%%

% Needed before document class
\RequirePackage{pdftexcmds} % needed for tests expressions
\RequirePackage{fix-cm} % correct units

% Define mode
\def\mode{a4}

\newif\ifaiv % a4
\newif\ifav % a5
\newif\ifbooklet % booklet
\newif\ifcover % cover for booklet

\ifnum \strcmp{\mode}{cover}=0
  \covertrue
\else\ifnum \strcmp{\mode}{booklet}=0
  \booklettrue
\else\ifnum \strcmp{\mode}{a5}=0
  \avtrue
\else
  \aivtrue
\fi\fi\fi

\ifbooklet % do not enclose with {}
  \documentclass[french,twoside]{book} % ,notitlepage
  \usepackage[%
    papersize={105mm, 297mm},
    inner=12mm,
    outer=12mm,
    top=20mm,
    bottom=15mm,
    marginparsep=0pt,
  ]{geometry}
  \usepackage[fontsize=9.5pt]{scrextend} % for Roboto
\else\ifav
  \documentclass[french,twoside]{book} % ,notitlepage
  \usepackage[%
    a5paper,
    inner=25mm,
    outer=15mm,
    top=15mm,
    bottom=15mm,
    marginparsep=0pt,
  ]{geometry}
  \usepackage[fontsize=12pt]{scrextend}
\else% A4 2 cols
  \documentclass[twocolumn]{report}
  \usepackage[%
    a4paper,
    inner=15mm,
    outer=10mm,
    top=25mm,
    bottom=18mm,
    marginparsep=0pt,
  ]{geometry}
  \setlength{\columnsep}{20mm}
  \usepackage[fontsize=9.5pt]{scrextend}
\fi\fi

%%%%%%%%%%%%%%
% Alignments %
%%%%%%%%%%%%%%
% before teinte macros

\setlength{\arrayrulewidth}{0.2pt}
\setlength{\columnseprule}{\arrayrulewidth} % twocol
\setlength{\parskip}{0pt} % classical para with no margin
\setlength{\parindent}{1.5em}

%%%%%%%%%%
% Colors %
%%%%%%%%%%
% before Teinte macros

\usepackage[dvipsnames]{xcolor}
\definecolor{rubric}{HTML}{800000} % the tonic 0c71c3
\def\columnseprulecolor{\color{rubric}}
\colorlet{borderline}{rubric!30!} % definecolor need exact code
\definecolor{shadecolor}{gray}{0.95}
\definecolor{bghi}{gray}{0.5}

%%%%%%%%%%%%%%%%%
% Teinte macros %
%%%%%%%%%%%%%%%%%
%%%%%%%%%%%%%%%%%%%%%%%%%%%%%%%%%%%%%%%%%%%%%%%%%%%
% <TEI> generic (LaTeX names generated by Teinte) %
%%%%%%%%%%%%%%%%%%%%%%%%%%%%%%%%%%%%%%%%%%%%%%%%%%%
% This template is inserted in a specific design
% It is XeLaTeX and otf fonts

\makeatletter % <@@@


\usepackage{blindtext} % generate text for testing
\usepackage[strict]{changepage} % for modulo 4
\usepackage{contour} % rounding words
\usepackage[nodayofweek]{datetime}
% \usepackage{DejaVuSans} % seems buggy for sffont font for symbols
\usepackage{enumitem} % <list>
\usepackage{etoolbox} % patch commands
\usepackage{fancyvrb}
\usepackage{fancyhdr}
\usepackage{float}
\usepackage{fontspec} % XeLaTeX mandatory for fonts
\usepackage{footnote} % used to capture notes in minipage (ex: quote)
\usepackage{framed} % bordering correct with footnote hack
\usepackage{graphicx}
\usepackage{lettrine} % drop caps
\usepackage{lipsum} % generate text for testing
\usepackage[framemethod=tikz,]{mdframed} % maybe used for frame with footnotes inside
\usepackage{pdftexcmds} % needed for tests expressions
\usepackage{polyglossia} % non-break space french punct, bug Warning: "Failed to patch part"
\usepackage[%
  indentfirst=false,
  vskip=1em,
  noorphanfirst=true,
  noorphanafter=true,
  leftmargin=\parindent,
  rightmargin=0pt,
]{quoting}
\usepackage{ragged2e}
\usepackage{setspace} % \setstretch for <quote>
\usepackage{tabularx} % <table>
\usepackage[explicit]{titlesec} % wear titles, !NO implicit
\usepackage{tikz} % ornaments
\usepackage{tocloft} % styling tocs
\usepackage[fit]{truncate} % used im runing titles
\usepackage{unicode-math}
\usepackage[normalem]{ulem} % breakable \uline, normalem is absolutely necessary to keep \emph
\usepackage{verse} % <l>
\usepackage{xcolor} % named colors
\usepackage{xparse} % @ifundefined
\XeTeXdefaultencoding "iso-8859-1" % bad encoding of xstring
\usepackage{xstring} % string tests
\XeTeXdefaultencoding "utf-8"
\PassOptionsToPackage{hyphens}{url} % before hyperref, which load url package

% TOTEST
% \usepackage{hypcap} % links in caption ?
% \usepackage{marginnote}
% TESTED
% \usepackage{background} % doesn’t work with xetek
% \usepackage{bookmark} % prefers the hyperref hack \phantomsection
% \usepackage[color, leftbars]{changebar} % 2 cols doc, impossible to keep bar left
% \usepackage[utf8x]{inputenc} % inputenc package ignored with utf8 based engines
% \usepackage[sfdefault,medium]{inter} % no small caps
% \usepackage{firamath} % choose firasans instead, firamath unavailable in Ubuntu 21-04
% \usepackage{flushend} % bad for last notes, supposed flush end of columns
% \usepackage[stable]{footmisc} % BAD for complex notes https://texfaq.org/FAQ-ftnsect
% \usepackage{helvet} % not for XeLaTeX
% \usepackage{multicol} % not compatible with too much packages (longtable, framed, memoir…)
% \usepackage[default,oldstyle,scale=0.95]{opensans} % no small caps
% \usepackage{sectsty} % \chapterfont OBSOLETE
% \usepackage{soul} % \ul for underline, OBSOLETE with XeTeX
% \usepackage[breakable]{tcolorbox} % text styling gone, footnote hack not kept with breakable


% Metadata inserted by a program, from the TEI source, for title page and runing heads
\title{\textbf{ Monarchie constitutionnelle }}
\date{1870}
\author{Renan, Ernest (1823-1892)}
\def\elbibl{Renan, Ernest (1823-1892). 1870. \emph{Monarchie constitutionnelle}}

% Default metas
\newcommand{\colorprovide}[2]{\@ifundefinedcolor{#1}{\colorlet{#1}{#2}}{}}
\colorprovide{rubric}{red}
\colorprovide{silver}{lightgray}
\@ifundefined{syms}{\newfontfamily\syms{DejaVu Sans}}{}
\newif\ifdev
\@ifundefined{elbibl}{% No meta defined, maybe dev mode
  \newcommand{\elbibl}{Titre court ?}
  \newcommand{\elbook}{Titre du livre source ?}
  \newcommand{\elabstract}{Résumé\par}
  \newcommand{\elurl}{http://oeuvres.github.io/elbook/2}
  \author{Éric Lœchien}
  \title{Un titre de test assez long pour vérifier le comportement d’une maquette}
  \date{1566}
  \devtrue
}{}
\let\eltitle\@title
\let\elauthor\@author
\let\eldate\@date


\defaultfontfeatures{
  % Mapping=tex-text, % no effect seen
  Scale=MatchLowercase,
  Ligatures={TeX,Common},
}


% generic typo commands
\newcommand{\astermono}{\medskip\centerline{\color{rubric}\large\selectfont{\syms ✻}}\medskip\par}%
\newcommand{\astertri}{\medskip\par\centerline{\color{rubric}\large\selectfont{\syms ✻\,✻\,✻}}\medskip\par}%
\newcommand{\asterism}{\bigskip\par\noindent\parbox{\linewidth}{\centering\color{rubric}\large{\syms ✻}\\{\syms ✻}\hskip 0.75em{\syms ✻}}\bigskip\par}%

% lists
\newlength{\listmod}
\setlength{\listmod}{\parindent}
\setlist{
  itemindent=!,
  listparindent=\listmod,
  labelsep=0.2\listmod,
  parsep=0pt,
  % topsep=0.2em, % default topsep is best
}
\setlist[itemize]{
  label=—,
  leftmargin=0pt,
  labelindent=1.2em,
  labelwidth=0pt,
}
\setlist[enumerate]{
  label={\bf\color{rubric}\arabic*.},
  labelindent=0.8\listmod,
  leftmargin=\listmod,
  labelwidth=0pt,
}
\newlist{listalpha}{enumerate}{1}
\setlist[listalpha]{
  label={\bf\color{rubric}\alph*.},
  leftmargin=0pt,
  labelindent=0.8\listmod,
  labelwidth=0pt,
}
\newcommand{\listhead}[1]{\hspace{-1\listmod}\emph{#1}}

\renewcommand{\hrulefill}{%
  \leavevmode\leaders\hrule height 0.2pt\hfill\kern\z@}

% General typo
\DeclareTextFontCommand{\textlarge}{\large}
\DeclareTextFontCommand{\textsmall}{\small}

% commands, inlines
\newcommand{\anchor}[1]{\Hy@raisedlink{\hypertarget{#1}{}}} % link to top of an anchor (not baseline)
\newcommand\abbr[1]{#1}
\newcommand{\autour}[1]{\tikz[baseline=(X.base)]\node [draw=rubric,thin,rectangle,inner sep=1.5pt, rounded corners=3pt] (X) {\color{rubric}#1};}
\newcommand\corr[1]{#1}
\newcommand{\ed}[1]{ {\color{silver}\sffamily\footnotesize (#1)} } % <milestone ed="1688"/>
\newcommand\expan[1]{#1}
\newcommand\foreign[1]{\emph{#1}}
\newcommand\gap[1]{#1}
\renewcommand{\LettrineFontHook}{\color{rubric}}
\newcommand{\initial}[2]{\lettrine[lines=2, loversize=0.3, lhang=0.3]{#1}{#2}}
\newcommand{\initialiv}[2]{%
  \let\oldLFH\LettrineFontHook
  % \renewcommand{\LettrineFontHook}{\color{rubric}\ttfamily}
  \IfSubStr{QJ’}{#1}{
    \lettrine[lines=4, lhang=0.2, loversize=-0.1, lraise=0.2]{\smash{#1}}{#2}
  }{\IfSubStr{É}{#1}{
    \lettrine[lines=4, lhang=0.2, loversize=-0, lraise=0]{\smash{#1}}{#2}
  }{\IfSubStr{ÀÂ}{#1}{
    \lettrine[lines=4, lhang=0.2, loversize=-0, lraise=0, slope=0.6em]{\smash{#1}}{#2}
  }{\IfSubStr{A}{#1}{
    \lettrine[lines=4, lhang=0.2, loversize=0.2, slope=0.6em]{\smash{#1}}{#2}
  }{\IfSubStr{V}{#1}{
    \lettrine[lines=4, lhang=0.2, loversize=0.2, slope=-0.5em]{\smash{#1}}{#2}
  }{
    \lettrine[lines=4, lhang=0.2, loversize=0.2]{\smash{#1}}{#2}
  }}}}}
  \let\LettrineFontHook\oldLFH
}
\newcommand{\labelchar}[1]{\textbf{\color{rubric} #1}}
\newcommand{\milestone}[1]{\autour{\footnotesize\color{rubric} #1}} % <milestone n="4"/>
\newcommand\name[1]{#1}
\newcommand\orig[1]{#1}
\newcommand\orgName[1]{#1}
\newcommand\persName[1]{#1}
\newcommand\placeName[1]{#1}
\newcommand{\pn}[1]{\IfSubStr{-—–¶}{#1}% <p n="3"/>
  {\noindent{\bfseries\color{rubric}   ¶  }}
  {{\footnotesize\autour{ #1}  }}}
\newcommand\reg{}
% \newcommand\ref{} % already defined
\newcommand\sic[1]{#1}
\newcommand\surname[1]{\textsc{#1}}
\newcommand\term[1]{\textbf{#1}}

\def\mednobreak{\ifdim\lastskip<\medskipamount
  \removelastskip\nopagebreak\medskip\fi}
\def\bignobreak{\ifdim\lastskip<\bigskipamount
  \removelastskip\nopagebreak\bigskip\fi}

% commands, blocks
\newcommand{\byline}[1]{\bigskip{\RaggedLeft{#1}\par}\bigskip}
\newcommand{\bibl}[1]{{\RaggedLeft{#1}\par\bigskip}}
\newcommand{\biblitem}[1]{{\noindent\hangindent=\parindent   #1\par}}
\newcommand{\dateline}[1]{\medskip{\RaggedLeft{#1}\par}\bigskip}
\newcommand{\labelblock}[1]{\medbreak{\noindent\color{rubric}\bfseries #1}\par\mednobreak}
\newcommand{\salute}[1]{\bigbreak{#1}\par\medbreak}
\newcommand{\signed}[1]{\bigbreak\filbreak{\raggedleft #1\par}\medskip}

% environments for blocks (some may become commands)
\newenvironment{borderbox}{}{} % framing content
\newenvironment{citbibl}{\ifvmode\hfill\fi}{\ifvmode\par\fi }
\newenvironment{docAuthor}{\ifvmode\vskip4pt\fontsize{16pt}{18pt}\selectfont\fi\itshape}{\ifvmode\par\fi }
\newenvironment{docDate}{}{\ifvmode\par\fi }
\newenvironment{docImprint}{\vskip6pt}{\ifvmode\par\fi }
\newenvironment{docTitle}{\vskip6pt\bfseries\fontsize{18pt}{22pt}\selectfont}{\par }
\newenvironment{msHead}{\vskip6pt}{\par}
\newenvironment{msItem}{\vskip6pt}{\par}
\newenvironment{titlePart}{}{\par }


% environments for block containers
\newenvironment{argument}{\itshape\parindent0pt}{\vskip1.5em}
\newenvironment{biblfree}{}{\ifvmode\par\fi }
\newenvironment{bibitemlist}[1]{%
  \list{\@biblabel{\@arabic\c@enumiv}}%
  {%
    \settowidth\labelwidth{\@biblabel{#1}}%
    \leftmargin\labelwidth
    \advance\leftmargin\labelsep
    \@openbib@code
    \usecounter{enumiv}%
    \let\p@enumiv\@empty
    \renewcommand\theenumiv{\@arabic\c@enumiv}%
  }
  \sloppy
  \clubpenalty4000
  \@clubpenalty \clubpenalty
  \widowpenalty4000%
  \sfcode`\.\@m
}%
{\def\@noitemerr
  {\@latex@warning{Empty `bibitemlist' environment}}%
\endlist}
\newenvironment{quoteblock}% may be used for ornaments
  {\begin{quoting}}
  {\end{quoting}}

% table () is preceded and finished by custom command
\newcommand{\tableopen}[1]{%
  \ifnum\strcmp{#1}{wide}=0{%
    \begin{center}
  }
  \else\ifnum\strcmp{#1}{long}=0{%
    \begin{center}
  }
  \else{%
    \begin{center}
  }
  \fi\fi
}
\newcommand{\tableclose}[1]{%
  \ifnum\strcmp{#1}{wide}=0{%
    \end{center}
  }
  \else\ifnum\strcmp{#1}{long}=0{%
    \end{center}
  }
  \else{%
    \end{center}
  }
  \fi\fi
}


% text structure
\newcommand\chapteropen{} % before chapter title
\newcommand\chaptercont{} % after title, argument, epigraph…
\newcommand\chapterclose{} % maybe useful for multicol settings
\setcounter{secnumdepth}{-2} % no counters for hierarchy titles
\setcounter{tocdepth}{5} % deep toc
\markright{\@title} % ???
\markboth{\@title}{\@author} % ???
\renewcommand\tableofcontents{\@starttoc{toc}}
% toclof format
% \renewcommand{\@tocrmarg}{0.1em} % Useless command?
% \renewcommand{\@pnumwidth}{0.5em} % {1.75em}
\renewcommand{\@cftmaketoctitle}{}
\setlength{\cftbeforesecskip}{\z@ \@plus.2\p@}
\renewcommand{\cftchapfont}{}
\renewcommand{\cftchapdotsep}{\cftdotsep}
\renewcommand{\cftchapleader}{\normalfont\cftdotfill{\cftchapdotsep}}
\renewcommand{\cftchappagefont}{\bfseries}
\setlength{\cftbeforechapskip}{0em \@plus\p@}
% \renewcommand{\cftsecfont}{\small\relax}
\renewcommand{\cftsecpagefont}{\normalfont}
% \renewcommand{\cftsubsecfont}{\small\relax}
\renewcommand{\cftsecdotsep}{\cftdotsep}
\renewcommand{\cftsecpagefont}{\normalfont}
\renewcommand{\cftsecleader}{\normalfont\cftdotfill{\cftsecdotsep}}
\setlength{\cftsecindent}{1em}
\setlength{\cftsubsecindent}{2em}
\setlength{\cftsubsubsecindent}{3em}
\setlength{\cftchapnumwidth}{1em}
\setlength{\cftsecnumwidth}{1em}
\setlength{\cftsubsecnumwidth}{1em}
\setlength{\cftsubsubsecnumwidth}{1em}

% footnotes
\newif\ifheading
\newcommand*{\fnmarkscale}{\ifheading 0.70 \else 1 \fi}
\renewcommand\footnoterule{\vspace*{0.3cm}\hrule height \arrayrulewidth width 3cm \vspace*{0.3cm}}
\setlength\footnotesep{1.5\footnotesep} % footnote separator
\renewcommand\@makefntext[1]{\parindent 1.5em \noindent \hb@xt@1.8em{\hss{\normalfont\@thefnmark . }}#1} % no superscipt in foot
\patchcmd{\@footnotetext}{\footnotesize}{\footnotesize\sffamily}{}{} % before scrextend, hyperref


%   see https://tex.stackexchange.com/a/34449/5049
\def\truncdiv#1#2{((#1-(#2-1)/2)/#2)}
\def\moduloop#1#2{(#1-\truncdiv{#1}{#2}*#2)}
\def\modulo#1#2{\number\numexpr\moduloop{#1}{#2}\relax}

% orphans and widows
\clubpenalty=9996
\widowpenalty=9999
\brokenpenalty=4991
\predisplaypenalty=10000
\postdisplaypenalty=1549
\displaywidowpenalty=1602
\hyphenpenalty=400
% Copied from Rahtz but not understood
\def\@pnumwidth{1.55em}
\def\@tocrmarg {2.55em}
\def\@dotsep{4.5}
\emergencystretch 3em
\hbadness=4000
\pretolerance=750
\tolerance=2000
\vbadness=4000
\def\Gin@extensions{.pdf,.png,.jpg,.mps,.tif}
% \renewcommand{\@cite}[1]{#1} % biblio

\usepackage{hyperref} % supposed to be the last one, :o) except for the ones to follow
\urlstyle{same} % after hyperref
\hypersetup{
  % pdftex, % no effect
  pdftitle={\elbibl},
  % pdfauthor={Your name here},
  % pdfsubject={Your subject here},
  % pdfkeywords={keyword1, keyword2},
  bookmarksnumbered=true,
  bookmarksopen=true,
  bookmarksopenlevel=1,
  pdfstartview=Fit,
  breaklinks=true, % avoid long links
  pdfpagemode=UseOutlines,    % pdf toc
  hyperfootnotes=true,
  colorlinks=false,
  pdfborder=0 0 0,
  % pdfpagelayout=TwoPageRight,
  % linktocpage=true, % NO, toc, link only on page no
}

\makeatother % /@@@>
%%%%%%%%%%%%%%
% </TEI> end %
%%%%%%%%%%%%%%


%%%%%%%%%%%%%
% footnotes %
%%%%%%%%%%%%%
\renewcommand{\thefootnote}{\bfseries\textcolor{rubric}{\arabic{footnote}}} % color for footnote marks

%%%%%%%%%
% Fonts %
%%%%%%%%%
\usepackage[]{roboto} % SmallCaps, Regular is a bit bold
% \linespread{0.90} % too compact, keep font natural
\newfontfamily\fontrun[]{Roboto Condensed Light} % condensed runing heads
\ifav
  \setmainfont[
    ItalicFont={Roboto Light Italic},
  ]{Roboto}
\else\ifbooklet
  \setmainfont[
    ItalicFont={Roboto Light Italic},
  ]{Roboto}
\else
\setmainfont[
  ItalicFont={Roboto Italic},
]{Roboto Light}
\fi\fi
\renewcommand{\LettrineFontHook}{\bfseries\color{rubric}}
% \renewenvironment{labelblock}{\begin{center}\bfseries\color{rubric}}{\end{center}}

%%%%%%%%
% MISC %
%%%%%%%%

\setdefaultlanguage[frenchpart=false]{french} % bug on part


\newenvironment{quotebar}{%
    \def\FrameCommand{{\color{rubric!10!}\vrule width 0.5em} \hspace{0.9em}}%
    \def\OuterFrameSep{\itemsep} % séparateur vertical
    \MakeFramed {\advance\hsize-\width \FrameRestore}
  }%
  {%
    \endMakeFramed
  }
\renewenvironment{quoteblock}% may be used for ornaments
  {%
    \savenotes
    \setstretch{0.9}
    \normalfont
    \begin{quotebar}
  }
  {%
    \end{quotebar}
    \spewnotes
  }


\renewcommand{\headrulewidth}{\arrayrulewidth}
\renewcommand{\headrule}{{\color{rubric}\hrule}}

% delicate tuning, image has produce line-height problems in title on 2 lines
\titleformat{name=\chapter} % command
  [display] % shape
  {\vspace{1.5em}\centering} % format
  {} % label
  {0pt} % separator between n
  {}
[{\color{rubric}\huge\textbf{#1}}\bigskip] % after code
% \titlespacing{command}{left spacing}{before spacing}{after spacing}[right]
\titlespacing*{\chapter}{0pt}{-2em}{0pt}[0pt]

\titleformat{name=\section}
  [block]{}{}{}{}
  [\vbox{\color{rubric}\large\raggedleft\textbf{#1}}]
\titlespacing{\section}{0pt}{0pt plus 4pt minus 2pt}{\baselineskip}

\titleformat{name=\subsection}
  [block]
  {}
  {} % \thesection
  {} % separator \arrayrulewidth
  {}
[\vbox{\large\textbf{#1}}]
% \titlespacing{\subsection}{0pt}{0pt plus 4pt minus 2pt}{\baselineskip}

\ifaiv
  \fancypagestyle{main}{%
    \fancyhf{}
    \setlength{\headheight}{1.5em}
    \fancyhead{} % reset head
    \fancyfoot{} % reset foot
    \fancyhead[L]{\truncate{0.45\headwidth}{\fontrun\elbibl}} % book ref
    \fancyhead[R]{\truncate{0.45\headwidth}{ \fontrun\nouppercase\leftmark}} % Chapter title
    \fancyhead[C]{\thepage}
  }
  \fancypagestyle{plain}{% apply to chapter
    \fancyhf{}% clear all header and footer fields
    \setlength{\headheight}{1.5em}
    \fancyhead[L]{\truncate{0.9\headwidth}{\fontrun\elbibl}}
    \fancyhead[R]{\thepage}
  }
\else
  \fancypagestyle{main}{%
    \fancyhf{}
    \setlength{\headheight}{1.5em}
    \fancyhead{} % reset head
    \fancyfoot{} % reset foot
    \fancyhead[RE]{\truncate{0.9\headwidth}{\fontrun\elbibl}} % book ref
    \fancyhead[LO]{\truncate{0.9\headwidth}{\fontrun\nouppercase\leftmark}} % Chapter title, \nouppercase needed
    \fancyhead[RO,LE]{\thepage}
  }
  \fancypagestyle{plain}{% apply to chapter
    \fancyhf{}% clear all header and footer fields
    \setlength{\headheight}{1.5em}
    \fancyhead[L]{\truncate{0.9\headwidth}{\fontrun\elbibl}}
    \fancyhead[R]{\thepage}
  }
\fi

\ifav % a5 only
  \titleclass{\section}{top}
\fi

\newcommand\chapo{{%
  \vspace*{-3em}
  \centering % no vskip ()
  {\Large\addfontfeature{LetterSpace=25}\bfseries{\elauthor}}\par
  \smallskip
  {\large\eldate}\par
  \bigskip
  {\Large\selectfont{\eltitle}}\par
  \bigskip
  {\color{rubric}\hline\par}
  \bigskip
  {\Large TEXTE LIBRE À PARTICPATION LIBRE\par}
  \centerline{\small\color{rubric} {hurlus.fr, tiré le \today}}\par
  \bigskip
}}

\newcommand\cover{{%
  \thispagestyle{empty}
  \centering
  {\LARGE\bfseries{\elauthor}}\par
  \bigskip
  {\Large\eldate}\par
  \bigskip
  \bigskip
  {\LARGE\selectfont{\eltitle}}\par
  \vfill\null
  {\color{rubric}\setlength{\arrayrulewidth}{2pt}\hline\par}
  \vfill\null
  {\Large TEXTE LIBRE À PARTICPATION LIBRE\par}
  \centerline{{\href{https://hurlus.fr}{\dotuline{hurlus.fr}}, tiré le \today}}\par
}}

\begin{document}
\pagestyle{empty}
\ifbooklet{
  \cover\newpage
  \thispagestyle{empty}\hbox{}\newpage
  \cover\newpage\noindent Les voyages de la brochure\par
  \bigskip
  \begin{tabularx}{\textwidth}{l|X|X}
    \textbf{Date} & \textbf{Lieu}& \textbf{Nom/pseudo} \\ \hline
    \rule{0pt}{25cm} &  &   \\
  \end{tabularx}
  \newpage
  \addtocounter{page}{-4}
}\fi

\thispagestyle{empty}
\ifaiv
  \twocolumn[\chapo]
\else
  \chapo
\fi
{\it\elabstract}
\bigskip
\makeatletter\@starttoc{toc}\makeatother % toc without new page
\bigskip

\pagestyle{main} % after style

  \frontmatter 
\begin{titlepage}
MONARCHIE CONSTITUTIONNELLE EN France PAR ERNEST RENANPARIS \\
MICHEL LÉVY FRÈRES, ÉDITEURS \\
RUE VIVIENNE, 2 BIS, ET BOULEVARD DES ITALIENS, 15 A LA LIBRAIRIE NOUVELLE 1870
\end{titlepage}
\mainmatter \noindent L’histoire n’est ni une géométrie inflexible ni une simple succession d’incidents fortuits. Si l’histoire était dominée d’une manière absolue par la nécessité, on pourrait tout prévoir ; si elle était un simple jeu de la passion et de la fortune, on ne pourrait rien prévoir. Or, la vérité est que les choses humaines, bien qu’elles déjouent souvent les conjectures des esprits les plus sagaces, prêtent néanmoins au calcul. Les faits accomplis contiennent, si on sait distinguer l’essentiel de l’accessoire, les lignes générales de l’avenir. « Le petit grain de sable qui se mit dans l’urètre de Cromwell » fut, au xviie siècle, un événement capital ; la philosophie de l’histoire d’Angleterre est indépendante d’un pareil détail. Santé ou maladie, bonne ou mauvaise humeur des princes, brouilles ou raccommodements des personnages considérables, intrigues diplomatiques, chances diverses de la guerre, le plus grand génie ne sert de rien pour deviner tout cela. Ces sortes de choses se passent dans un monde où le raisonnement n’a aucune application. Le valet de chambre d’un souverain pourrait, en fait de nouvelles importantes, redresser les idées du meilleur esprit ; mais ces accidents, impossibles à prévoir et à déterminer {\itshape a priori}, s’effacent dans l’ensemble. Le passé nous apparaît comme un dessein suivi, où tout se tient et s’explique. L’avenir jugera notre temps comme nous jugeons le passé, et verra des conséquences rigoureuses où nous sommes souvent tentés de ne voir que des volontés individuelles et des rencontres du hasard.\par
C’est dans cet esprit que nous voudrions proposer quelques observations sur les graves événements accomplis en cette année 1869. La philosophie que nous porterons dans cet examen n’est pas celle de l’indifférence. Nous ne nous exagérons pas la part de la réflexion dans la conduite des choses humaines ; nous ne croyons pas cependant que le temps soit déjà venu de déserter la vie publique et d’abandonner les affaires de ce monde à l’intrigue et à la violence. Un reproche peut toujours être adressé à celui qui critique les affaires de son siècle sans avoir consenti à s’en mêler ; mais celui qui a fait ce qu’un honnête homme peut faire, celui qui a dit ce qu’il pense sans souci de plaire ou de déplaire à personne, celui-là peut avoir la conscience merveilleusement à l’aise. Nous ne devons pas à notre patrie de trahir pour elle la vérité, de manquer pour elle de goût et de tact ; nous ne lui devons pas de suivre ses caprices ni de nous convertir à la thèse qui réussit ; nous lui devons de dire bien exactement, et sans le sacrifice d’une nuance, ce que nous croyons être la vérité.
\section[{I}]{I}\renewcommand{\leftmark}{I}

\noindent La Révolution française est un événement si extraordinaire, que c’est par elle qu’il faut ouvrir toute série de considérations sur les affaires de notre temps. Rien d’important n’arrive en France qui ne soit la conséquence directe de ce fait capital, lequel a changé profondément les conditions de la vie dans notre pays. Comme tout ce qui est grand, héroïque, téméraire, comme tout ce qui dépasse la commune mesure des forces humaines, la Révolution française sera durant des siècles le sujet dont le monde s’entretiendra, sur lequel on se divisera, qui servira de prétexte pour s’aimer et se haïr, qui fournira des sujets de drames et de romans. En un sens, la Révolution française (l’Empire, dans ma pensée, fait corps avec elle) est la gloire de la France, l’épopée française par excellence ; mais presque toujours les nations qui ont dans leur histoire un fait exceptionnel expient ce fait par de longues souffrances et souvent le payent de leur existence nationale. Il en fut ainsi de la Judée, de la Grèce et de l’Italie. Pour avoir créé des choses uniques dont le monde vit et profite, ces pays ont traversé des siècles d’humiliation et de mort nationale. La vie nationale est quelque chose de limité, de médiocre, de borné. Pour faire de l’extraordinaire, de l’universel, il faut déchirer ce réseau étroit ; du même coup, on déchire sa patrie, une patrie étant un ensemble de préjugés et d’idées arrêtées que l’humanité entière ne saurait accepter. Les nations qui ont créé la religion, l’art, la science, l’empire, l’église, là papauté (toutes choses universelles, non nationales), ont été plus que des nations ; elles ont été par là même moins que des nations en ce sens qu’elles ont été victimes de leur œuvre. Je pense que la Révolution aura pour la France des conséquences analogues, mais moins durables, parce que l’œuvre de la France a été moins grande et moins universelle que les œuvres de la Judée, de la Grèce, de l’Italie. Le parallèle exact de la situation actuelle de notre pays me parait être l’Allemagne au xviie siècle. L’Allemagne au \textsc{xvi}\textsuperscript{e} siècle avait fait pour l’humanité une œuvre de premier ordre, la Réforme. Elle l’expia au xviie par un extrême abaissement politique. Il est probable que le xixe siècle sera de même considéré, dans l’histoire de France, comme l’expiation de la Révolution. Les nations, pas plus que les individus, ne sortent impunément de la ligne moyenne, qui est celle du bon sens pratique et de la possibilité.\par
Si la Révolution en effet a créé pour la France dans le monde une situation poétique et romanesque de premier ordre, il est sûr d’un autre côté qu’à considérer seulement les exigences de la politique ordinaire, elle a engagé la France dans une voie pleine de singularités. Le but que la France a voulu atteindre par la Révolution est celui que toutes les nations modernes poursuivent : une société juste, honnête, humaine, garantissant les droits et la liberté de tous avec le moins de sacrifices possible des droits et de la liberté de chacun. Ce but, la France, à la date où nous sommes, après avoir versé des flots de sang, en est fort loin, tandis que l’Angleterre, qui n’a pas procédé par révolutions, l’a presque ‘atteint. La France, en d’autres termes, offre cet étrange spectacle d’un pays qui essaye tardivement de regagner son arriéré sur les nations qu’elle avait traitées d’arriérées, qui se remet à l’école des peuples auxquels elle avait prétendu donner des leçons, et s’efforce de faire par imitation l’œuvre où elle avait cru déployer une haute originalité.\par
La cause de cette bizarrerie historique est fort simple. Malgré le feu étrange qui l’animait, la France, à la fin du \textsc{xviii}\textsuperscript{e} siècle, était assez ignorante des conditions d’existence d’une nation et de l’humanité. Sa prodigieuse tentative impliqua beaucoup d’erreurs ; elle méconnut tout à fait les règles de la liberté moderne. Qu’on le regrette ou qu’on s’en réjouisse, la liberté moderne n’est nullement la liberté antique ni celle des républiques du moyen âge. Elle est bien plus réelle,mais beaucoup moins brillante. Thucydide et Machiavel n’y comprendraient rien, et cependant un sujet de la reine Victoria est mille fois plus libre que ne l’a été aucun citoyen de Sparte, d’Athènes, de Venise ou de Florence. Plus de ces fiévreuses agitations républicaines, pleines de noblesse et de danger ; plus de ces villes composées d’un peuple fin, vivant et aristocratique ; au lieu de cela, de grandes masses pesantes, chez lesquelles l’intelligence est le fait d’un petit nombre, mais qui contribuent puissamment à la civilisation en mettant au service de l’État, par la conscription et l’impôt, un merveilleux trésor d’abnégation, de docilité, de bon esprit. Cette manière d’exister, qui est assurément celle qui use le moins une nation, et conserve le mieux ses forces, l’Angleterre en a donné le modèle. L’Angleterre est arrivée à l’état le plus libéral que le monde ait connu jusqu’ici en développant ses institutions du moyen âge, et nullement par la révolution. La liberté en Angleterre ne vient pas de Cromwell ni des républicains de 1649 ; elle vient de son histoire entière, de son égal respect pour le droit du roi, pour le droit des seigneurs, pour le droit des communes et des corporations de toute espèce. La France suivit la marche opposée. Le roi avait depuis longtemps fait table rase du droit des seigneurs et des communes ; la nation fit table rase des droits du roi. Elle procéda philosophiquement en une matière où il faut procéder historiquement : elle crut qu’on fonde la liberté par la souveraineté du peuple et au nom d’une autorité centrale, tandis que la liberté s’obtient par de petites conquêtes locales successives, par des réformes lentes. L’Angleterre, qui ne se pique de nulle philosophie, l’Angleterre, qui n’a rompu avec sa tradition qu’à un seul moment d’égarement passager suivi d’un prompt repentir, l’Angleterre, qui, au lieu du dogme absolu de la souveraineté du peuple, admet seulement le principe plus modéré qu’il n’y a pas de gouvernement sans le peuple, ni contre le peuple, s’est trouvée mille fois plus libre que la France, qui avait si fièrement planté le drapeau philosophique des droits de l’homme. C’est que la souveraineté du peuple ne fonde pas le gouvernement constitutionnel. L’État ainsi établi à la française est trop fort ; loin de garantir toutes les libertés, il absorbe toutes les libertés ; sa forme est la convention ou le despotisme. Ce qui devait sortir de la Révolution ne pouvait, après tout, beaucoup différer du Consulat et de l’Empire ; ce qui devait sortir d’une telle conception de la société ne pouvait être autre chose qu’une administration, un réseau de préfets, un code civil étroit, une machine servant à étreindre la nation, un maillot où il lui serait impossible de vivre et de croître. Rien de plus injuste que la haine avec laquelle l’école radicale française traite l’œuvre de Napoléon. L’œuvre de Napoléon, si l’on excepte quelques erreurs qui furent personnelles à cet homme extraordinaire, n’est en somme que le programme révolutionnaire réalisé en ses parties possibles. Napoléon n’eût pas existé que la constitution définitive de la République n’eût pas différé essentiellement de la constitution de l’an \textsc{viii}.\par
Une idée à plusieurs égards très fausse de la société humaine est en effet au fond de toutes les tentatives révolutionnaires françaises. L’erreur originelle fut d’abord masquée par le magnifique élan d’enthousiasme pour la liberté et le droit qui remplit les premières années de la Révolution ; mais, ce beau feu une fois tombé, il resta une théorie sociale qui fut dominante sous le Directoire, le Consulat et l’Empire, et marqua d’un sceau profond toutes les créations du temps.\par
D’après cette théorie, qu’on peut bien qualifier de matérialisme en politique, la société n’est pas quelque chose de religieux ni de sacré. Elle n’a qu’un seul but, c’est que les individus qui la composent jouissent de la plus grande somme possible de bien-être, sans souci de la destinée idéale de l’humanité. Que parle-t-on d’élever, d’ennoblir la conscience humaine ? Il s’agit seulement de contenter le grand nombre, d’assurer à tous une sorte de bonheur vulgaire et bien relatif assurément, car l’âme noble aurait en aversion un pareil bonheur, et se mettrait en révolte contre la société qui prétendrait le procurer. Aux yeux d’une philosophie éclairée, la société est un grand fait providentiel ; elle est établie non par l’homme, mais par la nature elle-même, afin qu’à la surface de notre planète se produise la vie intellectuelle et morale. L’homme isolé n’existe pas pour la philosophie politique. La société humaine, mère de tout idéal, est le produit direct de la volonté suprême qui veut que le bien, le vrai, le beau, aient dans l’univers des contemplateurs. Cette fonction transcendante de l’humanité ne s’accomplit pas au moyen de la simple coexistence des individus. La société est une hiérarchie. Tous les individus sont nobles et sacrés, tous les êtres (même les animaux) ont des droits ; mais tous les êtres ne sont pas égaux, tous sont des membres d’un vaste corps, des parties d’un immense organisme qui accomplit un travail divin. La négation de ce travail divin est l’erreur où verse facilement la démocratie française. Considérant les jouissances de l’individu comme l’objet unique de la société, elle est amenée à méconnaître les droits de l’idée, la primauté de l’esprit. Ne comprenant pas d’ailleurs l’inégalité des races, parce qu’en effet les différences ethnographiques ont disparu de son sein depuis un temps immémorial, la France est amenée à concevoir comme la perfection sociale une sorte de médiocrité universelle. Dieu nous garde de rêver la résurrection de ce qui est mort ; mais, sans demander la reconstitution de la noblesse, il est bien permis de trouver que l’importance accordée à la naissance vaut mieux à beaucoup d’égards que l’importance accordée à la fortune : l’une n’est pas plus juste que l’autre, et la seule distinction juste, qui est celle du mérite et de la vertu, se trouve mieux d’une société où les rangs sont réglés par la naissance que d’une société où la richesse seule fait l’inégalité.\par
La vie humaine deviendrait impossible, si l’homme ne se donnait le droit de subordonner l’animal à ses besoins ; elle ne serait guère plus possible, si l’on s’en tenait à cette conception abstraite qui fait envisager tous les hommes comme apportant en naissant un même droit à la fortune et aux rangs sociaux.\par
Un tel état de choses, juste en apparence, serait la fin de toute vertu ; ce serait fatalement la haine et la guerre entre les deux sexes, puisque la nature a créé là, au sein même de l’espèce humaine, une différence de rôle indéniable. La bourgeoisie trouve juste qu’après avoir supprimé la royauté et la noblesse héréditaires, on s’arrête devant la richesse héréditaire. L’ouvrier trouve juste qu’après avoir supprimé la richesse héréditaire, on s’arrête devant l’inégalité de sexe, et même, s’il est un peu sensé, devant l’inégalité de force et de capacité. L’utopiste le plus exalté trouve juste qu’après avoir supprimé en imagination toute inégalité entre les hommes, on admette le droit qu’a l’homme d’employer l’animal selon ses besoins. Et, pourtant, il n’est pas plus juste que tel individu naisse riche qu’il n’est juste que tel individu naisse avec une distinction sociale ; il n’a pas plus gagné l’un que l’autre par son travail personnel. On part toujours de l’idée que la noblesse a pour origine le mérite, et, comme il est clair que le mérite n’est pas héréditaire, on démontre facilement que la noblesse héréditaire est chose absurde ; mais c’est là l’éternelle erreur française d’une justice distributive dont l’État tiendrait la balance. La raison sociale de la noblesse, envisagée comme institution d’utilité publique, était non pas de récompenser le mérite, mais de le provoquer, de rendre possibles, faciles même certains genres de mérite. N’aurait-elle eu pour effet que de montrer que la justice ne doit pas être cherchée dans la constitution officielle de la société, c’eût été déjà quelque chose. La devise « au plus digne » n’a en politique que bien peu d’applications.\par
La bourgeoisie française s’est donc fait quelque illusion en croyant, par son système de concours, d’écoles spéciales et d’avancement régulier, fonder une société juste. Le peuple lui démontrera facilement que l’enfant pauvre est exclu de ces concours, et lui soutiendra que la justice ne sera complète que quand tous les Français seront placés, en naissant, dans des conditions identiques. En d’autres termes, aucune société n’est possible, si l’on pousse à la rigueur les idées de justice distributive à l’égard des individus. Une nation qui poursuivrait un tel programme se condamnerait à une incurable faiblesse. Supprimant l’hérédité, et par là détruisant la famille ou la laissant facultative, elle serait bientôt vaincue soit par les parties d’elle-même où se conserveraient les anciens principes, soit par les nations étrangères qui conserveraient ces principes. La race qui triomphe est toujours celle où la famille et la propriété sont le plus fortement organisées. L’humanité est une échelle mystérieuse, une série de résultantes procédant les unes des autres. Des générations laborieuses d’hommes du peuple et de paysans rendent possible l’existence, du bourgeois honnête et économe, lequel rend possible à son tour l’homme dispensé du travail matériel, voué tout entier aux choses désintéressées. Chacun à son rang est le gardien d’une tradition qui importe aux progrès de la civilisation. Il n’y a pas deux morales, il n’y a pas deux sciences, il n’y a pas deux éducations. Il y a un seul ensemble intellectuel et moral, ouvrage splendide de l’esprit humain, que chacun, excepté l’égoïste, crée pour une petite part et auquel chacun participe à des degrés divers.\par
On supprime l’humanité, si l’on n’admet pas que des classes entières doivent vivre de la gloire et de la jouissance des autres. Le démocrate traite de dupe le paysan d’ancien régime qui travaille pour ses nobles, les aime et jouit de la haute existence que d’autres mènent avec ses sueurs. Certainement, c’est là un non-sens avec une vie étroite, renfermée, où tout se passe à huis clos comme de notre temps. Dans l’état actuel de la société, les avantages qu’un homme a sur un autre sont devenus choses exclusives et personnelles : jouir du plaisir ou de la noblesse d’autrui parait une extravagance ; mais il n’en a pas toujours été ainsi. Quand Gubbio ou Assise voyait défiler en cavalcade la noce de son jeune seigneur, nul n’était jaloux. Tous alors participaient de la vie de tous : le pauvre jouissait de la richesse du riche, le moine des joies du mondain, le mondain des prières du moine ; pour tous, il y avait l’art, la poésie, la religion.\par
Les froides considérations de l’économiste sauront-elles remplacer tout cela ? suffiront-elles pour refréner l’arrogance d’une démocratie sûre de sa force, et qui, après ne s’être pas arrêtée devant le fait de la souveraineté, sera bien tentée de ne pas s’arrêter devant le fait de la propriété ? Y aura-t-il des voix assez éloquentes pour faire accepter à des jeunes gens de dix-huit ans des raisonnements de vieillards, pour persuader à des classes sociales jeunes, ardentes, croyant au plaisir, et que la jouissance n’a pas encore désabusées, qu’il n’est pas possible que tous jouissent, que tous soient bien élevés, délicats, vertueux même dans le sens raffiné, mais qu’il faut qu’il y ait des gens de loisir, savants, bien élevés, délicats, vertueux, en lesquels et par lesquels les autres jouissent et goûtent l’idéal ? Les événements le diront. La supériorité de l’Église et la force qui lui assure encore un avenir consiste en ce que seule elle comprend cela et le fait comprendre. L’Église sait bien que les meilleurs sont souvent victimes de la supériorité des classes prétendues élevées ; mais elle sait aussi que la nature a voulu que la vie de l’humanité fût à plusieurs degrés. Elle sait et elle avoue que c’est la grossièreté de plusieurs qui fait l’éducation d’un seul, que c’est la sueur de plusieurs qui permet la vie noble d’un petit nombre ; cependant, elle n’appelle pas ceux-ci privilégiés, ni ceux-là déshérités, car l’œuvre humaine est pour elle indivisible. Supprimez cette grande loi, mettez tous les individus sur le même rang, avec des droits égaux, sans lien de subordination à une œuvre commune ; vous avez égoïsme, médiocrité, isolement, sécheresse, impossibilité de vivre, quelque chose comme la vie de notre temps, la plus triste, même pour l’homme du peuple, qui ait jamais été menée. À n’envisager que le droit des individus, il est injuste qu’un homme soit sacrifié à un autre homme ; mais il n’est pas injuste que tous soient assujettis à l’œuvre supérieure qu’accomplit l’humanité. C’est à la religion qu’il appartient d’expliquer ces mystères et d’offrir dans le monde idéal de surabondantes consolations à tous les sacrifiés d’ici-bas.\par
Voilà ce que la Révolution, dès qu’elle eut perdu sa grande ivresse sacrée des premiers jours, ne comprit pas assez. La Révolution en définitive fut irréligieuse et athée. La société qu’elle rêva dans les tristes jours qui suivirent l’accès de fièvre, quand elle chercha à se recueillir, est une sorte de régiment composé de matérialistes, et où la discipline tient lieu de vertu. La base toute négative que les hommes secs et durs de ce temps donnèrent à la société française ne peut produire qu’un peuple rogue et mal élevé ; leur code, œuvre de défiance, admet pour premier principe que tout s’apprécie en argent, c’est-à-dire en plaisir. La jalousie résume toute la théorie morale de ces prétendus fondateurs de nos lois. Or, la jalousie fonde l’égalité, non la liberté ; mettant l’homme toujours en garde contre les empiétements de son semblable, elle empêche l’affabilité entre les classes. Pas de société sans amour, sans tradition, sans respect, sans mutuelle aménité. Dans sa fausse notion de la vertu, qu’elle confond avec l’âpre revendication de ce que chacun regarde comme son droit, l’école démocratique ne voit pas que la grande vertu d’une nation est de supporter l’inégalité traditionnelle. La race la plus vertueuse est pour cette école, non la race qui pratique le sacrifice, le dévouement, l’idéalisme sous toutes ses formes, mais la plus turbulente, celle qui fait le plus de révolutions. On étonne beaucoup les plus intelligents démocrates quand on leur dit qu’il y a encore en effet dans le monde des races vertueuses, les Lithuaniens, par exemple, les Dithmarses, les Poméraniens, races encore féodales, pleines de forces vives en réserve, comprenant le devoir comme Kant, et pour lesquelles le mot de révolution n’a aucun sens.\par
La première conséquence de cette philosophie revêche et superficielle, trop tôt substituée à celle des Montesquieu et des Turgot, fut la suppression de la royauté. À des esprits imbus d’une philosophie matérialiste, la royauté devait paraître une anomalie. Bien peu de personnes comprenaient, en 1792, que la continuité des bonnes choses doit être gardée par des institutions qui sont, si l’on veut, un privilège pour quelques-uns, mais qui constituent des organes de la vie nationale sans lesquels certains besoins restent en souffrance. Ces petites forteresses où se conservent des dépôts appartenant à la société paraissaient des tours féodales. On niait toutes les subordinations traditionnelles, tous les pactes historiques, tous les symboles. La royauté était le premier de ces pactes, un pacte remontant à mille ans, un symbole que la puérile philosophie de l’histoire alors en vogue ne pouvait comprendre. Aucune nation n’a jamais créé une légende plus complète que celle de cette grande royauté capétienne, sorte de religion, née à Saint-Denis, consacrée à Reims par le concert des évêques, ayant ses rites, sa liturgie, son ampoule sacrée, son oriflamme. À toute nationalité correspond une dynastie en laquelle s’incarnent le génie et les intérêts de la nation ; une conscience nationale n’est fixe et ferme que quand elle a contracté un mariage indissoluble avec une famille, qui s’engage par le contrat à n’avoir aucun intérêt distinct de celui de la nation. Jamais cette identification ne fut aussi parfaite qu’entre la maison capétienne et la France. Ce fut plus qu’une royauté, ce fut en sacerdoce ; prêtre-roi comme David, le roi de France porte la chape et tient l’épée. Dieu l’éclaire en ses jugements. Le roi d’Angleterre se soucie peu de justice, il défend son droit contre ses barons ; l’empereur d’Allemagne s’en soucie moins encore, il chasse éternellement sur ses montagnes du Tyrol pendant que la boule du monde roule à sa guise ; le roi de France, lui, est juste : entouré de ses prud’hommes et de ses clercs solennels, avec sa main de justice, il ressemble à un Salomon. Son sacre, imité des rois d’Israël, était quelque chose d’étrange et d’unique. La France avait créé un huitième sacrement qui ne s’administrait qu’à Reims, le sacrement de la royauté. Le roi sacré fait des miracles ; il est revêtu d’un « ordre » ; c’est un personnage ecclésiastique de premier rang. Au pape, qui l’interpelle au nom de Dieu, il répond en montrant son onction : « Moi aussi, je suis de Dieu ! » Il se permet avec le successeur de Pierre des libertés sans égales. Une fois, il le fait souffleter et déclarer hérétique ; une autre fois, il le menace de le faire brûler ; appuyé sur ses docteurs de Sorbonne, il lé semonce, le dépose. Nonobstant cela, son type le plus parfait est un saint canonisé, saint Louis, si pur, si humble, si simple et si fort. Il a ses adorateurs mystiques ; la bonne Jeanne Darc ne le sépare pas de saint Michel et de sainte Catherine ; cette pauvre fille vécut à la lettre de la religion de Reims. Légende incomparable ! fable sainte ! C’est le vulgaire couteau destiné à faire tomber la tête des criminels qu’on lève contre elle ! Le meurtre du 21 janvier est, au point de vue de l’idéaliste, l’acte de matérialisme le plus hideux, la plus honteuse profession qu’on ait jamais faite d’ingratitude et de bassesse, de roturière vilenie et d’oubli du passé.\par
Est-ce à dire que cet ancien régime, dont la société nouvelle cherchait à faire disparaître le souvenir avec le genre particulier d’acharnement qu’on ne trouve que chez le parvenu contre le grand seigneur auquel il doit tout, est-ce à dire que cet ancien régime ne fût pas gravement coupable ? Certes, il l’était ; si je faisais en ce moment la philosophie générale de notre histoire, je montrerais que la royauté, la noblesse, le clergé, les parlements, les villes, les universités de la vieille France, avaient tous manqué à leurs devoirs, et que les révolutionnaires de 1793 ne firent que mettre le sceau à une série de fautes dont les conséquences pèsent lourdement sur nous. On expie toujours sa grandeur. La France avait conçu sa royauté comme quelque chose d’illimité. Le roi à la façon anglaise, sorte de stathouder payé et armé pour défendre la nation et détenir certains droits, était pour elle un non-sens. Dès le xiiie siècle, le roi d’Angleterre, sans cesse en lutte avec ses sujets et lié par des chartes, est pour nos poëtes français un objet de dérision ; il n’est pas assez puissant. La royauté française était quelque chose de trop sacré ; on ne contrôle pas l’oint du Seigneur ; Bossuet était conséquent en dressant la théorie du roi de France avec l’Écriture sainte. Si le roi d’Angleterre avait eu cette teinte de mysticité, les barons et les communes n’auraient pas réussi à le mater. La royauté française, pour produire ce brillant météore du règne de Louis XVI, avait absorbé tous les pouvoirs de la nation. Le lendemain du jour où l’État se trouva constitué sous la main d’un seul en cette puissante unité, il était inévitable que la France se prit telle que l’avait faite le grand roi avec son pouvoir central tout-puissant, ses libertés détruites, et, jugeant le roi une superfétation, le traitat comme un moule devenu inutile dès que la statue est coulée. Richelieu et Louis XIV ont été de la sorte les grands révolutionnaires, les vrais fondateurs de la République. Le pendant exact de la colossale royauté de Louis XIV est la république de 1793, avec sa concentration effrayante des pouvoirs, monstre inouï, tel qu’on n’en avait jamais vu de semblable. Les exemples de républiques ne sont pas rares dans l’histoire ; mais ces républiques sont des villes ou de petits États confédérés. Ce qui est absolument sans exemple, c’est une république centralisée de trente millions d’âmes. Livrée pendant quatre on cinq ans aux vacillations de l’homme ivre, comme un {\itshape Great-Eastern} en perdition, l’énorme machine tomba dans son lit naturel, entre les mains d’un puissant despote, qui sut d’abord avec une habileté prodigieuse organiser le mouvement nouveau, mais qui finit comme tous les despotes. Devenu fou d’orgueil, il attira sur le pays qui s’était mis à sa discrétion la plus cruelle avanie que puisse éprouver une nation, et amena le retour de la dynastie que la France avait expulsée avec les derniers affronts.
\section[{II}]{II}\renewcommand{\leftmark}{II}

\noindent L’analogie d’une telle marche des événements avec ce qui se passa en Angleterre au xviie siècle se remarque sans peine. Elle frappa tout le monde en 1830, quand on vit un mouvement national substituer à la branche légitime des Bourbons une branche collatérale plus disposée à tenir compte des besoins nouveaux. Louis-Philippe dut paraître un Guillaume III, et l’on put espérer que la conséquence dernière de tant de convulsions serait le paisible établissement du régime constitutionnel en France. Une sorte de paix, un peu de quiétude et d’oubli entra avec cette consolante pensée dans notre pauvre conscience française si troublée ; on amnistia tout, même les folies et les crimes, on s’envisagea comme la génération privilégiée destinée à goûter les fruits des fautes des générations passées. C’était là une grande illusion ; la surprise la plus inconcevable de l’histoire réussit ; une bande d’étourdis, contre lesquels aurait dû suffire le bâton du constable, renversa une dynastie sur laquelle la partie sensée de la nation avait fait reposer toute sa foi politique, toutes ses espérances. Pour emporter une théorie conçue par les meilleurs esprits d’après les plus séduisantes apparences, une heure d’irréflexion chez les uns, de défaillance chez les autres, suffit.\par
Pourquoi cette singulière déconvenue ? Pourquoi ce qui s’était passé en Angleterre ne se passa-t-il pas en France ? Pourquoi Louis-Philippe ne fut-il pas un Guillaume III, fondateur glorieux d’une ère nouvelle dans l’histoire de notre pays ? Dira-t-on que ce fut la faute de Louis-Philippe ? Cela serait injuste. Louis-Philippe fit des fautes ; mais il faut qu’il soit loisible à tous les gouvernements d’en commettre. Qui prendrait la conduite des choses humaines à la condition d’être infaillible et impeccable ne régnerait pas un jour. En tout cas, si Louis-Philippe mérita d’être détrôné, Guillaume III le mérita beaucoup plus. Ce qu’on a le plus reproché à Louis-Philippe, impopularité, inhabileté à se faire aimer, goût du pouvoir personnel, insouciance de la gloire extérieure, retours vers le parti légitimiste au détriment du parti qui l’avait fait roi, efforts pour reconstituer la prérogative royale, on put le reprocher bien plus encore à Guillaume III. Pourquoi donc les résultats furent-ils si divers ? Sans doute cela tint à la différence des temps et des pays. Des opérations historiques possibles chez un peuple sérieux et lourd, plein de confiance dans l’hérédité, ayant une répugnance invincible à forcer la dernière résistance du souverain, peuvent être impossibles à une époque de légèreté spirituelle et d’étourderie raisonneuse. Le mouvement républicain de 1649, d’ailleurs, avait été infiniment moins profond que ne fut celui de 1792. Le mouvement anglais de 1649 n’arriva pas à constituer un pouvoir impérial ; Cromwell ne fut pas un Napoléon. Enfin le parti républicain anglais n’eut pas de seconde génération. Écrasé sous la restauration des Stuarts, décimé par la persécution ou réfugié en Amérique, il cessa d’avoir sur les affaires d’Angleterre une influence considérable. Au xviiie siècle, l’Angleterre semble prendre à tâche d’expier par une sorte d’exagération de loyalisme et d’orthodoxie ses écarts momentanés du milieu du \textsc{xvi}\textsuperscript{e}. Il fallut plus de cent cinquante ans pour que la mort de Charles I\textsuperscript{er} cessât de peser sur la politique, pour qu’on osât penser librement et ne pas se croire obligé d’afficher un légitimisme effréné. Les choses se seraient passées à peu près de la même manière en France, si la réaction royaliste de 1796 et 1797 l’eût emporté. La Restauration se fût faite alors avec de bien plus franches allures, et la République n’eût été dans l’histoire de France que ce qu’elle est dans l’histoire d’Angleterre, un incident sans conséquence. Napoléon, par son génie, aidé des merveilleuses ressources de la France, sauva la Révolution, lui donna une forme, une organisation, un prestige militaire inouï.\par
La faible et inintelligente restauration de 1814 ne put en aucune manière déraciner une idée qui avait vécu si profondément dans la nation et entraîné après elle une génération énergique. La France sous la Restauration et sous Louis-Philippe continua de vivre des souvenirs de l’Empire et de là République. La Révolution reprit faveur. Tandis qu’en Angleterre, à partir de la restauration de Charles II et après 1688, la république ne cesse d’être maudite, qu’un homme était mal posé dans la société s’il nommait Charles I\textsuperscript{er} sans l’appeler le roi martyr, ou Cromwell sans le qualifier d’usurpateur, en France il devint de règle de faire des histoires de la Révolution sur le ton apologétique et admiratif. Ce fat un fait grave que le père du nouveau roi eût pris à la Révolution une part considérable ; on s’habitua à considérer la dynastie nouvelle comme un compromis avec, la Révolution, non comme l’héritière par substitution d’une légitimité. Un nouveau parti républicain, se rattachant à quelques vieux patriarches survivants de 1793, parvint [à se reformer. Ce parti, qui avait joué un rôle considérable en juillet 1830, mais qui dès lors n’avait pu faire prévaloir ses idées théoriques absolues, ne cessa de battre en brèche le gouvernement nouveau. Le changement de 1688 en Angleterre n’avait rien eu de révolutionnaire, dans le sens où nous entendons ce mot ; ce changement ne se fit point par le peuple ; il ne viola aucun droit, si ce n’est celui du roi détrôné. Chez nous, au contraire, 1830 déchaîna des forces anarchiques et humilia profondément le parti légitimiste. Ce parti, renfermant à quelques égards les portions les plus solides et les plus morales du pays, fit une cruelle guerre à la dynastie nouvelle, soit par son abstention, en l’empêchant de s’asseoir sur la seule base qui fonde une dynastie, l’élément lourdement conservateur, — soit par sa connivence avec le parti républicain. De la sorte, le gouvernement de la maison d’Orléans ne put se fonder sérieusement ; un souffle le renversa. On avait tout pardonné à Guillaume III ; on ne pardonna rien à Louis-Philippe. Le principe royaliste fut assez fort en Angleterre pour subir une transformation ; il ne le fut pas en France. Certainement, si le parti républicain avait eu en Angleterre sous Guillaume III l’importance qu’il eut en France sous Louis-Philippe, si ce parti avait eu l’appui de la faction des Stuarts, l’établissement constitutionnel dé l’Angleterre n’eût pas duré. En cela, l’Angleterre bénéficia d’un avantage énorme qu’elle possède, son aptitude colonisatrice. L’Amérique fut le déversoir du parti républicain ; sans cela, ce parti fût resté comme un virus dans la mère patrie, et eût empêché l’établissement constitutionnel. Rien ne se perd dans le monde de ce qui est fort et sincère. Ces exilés républicains furent les pères de ceux qui firent la guerre de l’indépendance à la fin du \textsc{xviii}\textsuperscript{e} siècle. L’élément révolutionnaire en Angleterre, au lieu d’être un dissolvant, fut de la sorte créateur ; le radicalisme anglais, au lieu de déchirer la mère patrie, fit l’Amérique. Si la France eût été colonisatrice au lieu d’être militaire, si l’élément hardi et entreprenant qui ailleurs colonise était capable chez nous d’autre chose que de conspirer et de se battre pour des principes abstraits, nous n’aurions pas eu Napoléon ; le parti républicain, chassé par la réaction, eût émigré vers 1798 et eût fondé au loin une Nouvelle-France qui, selon la loi des colonies, serait maintenant sans doute une république séparée. Malheureusement, nos discordes civiles n’aboutirent qu’à des déportations. Au lieu des États-Unis, nous avons eu Sinnamary et Lambèse ! Pendant que, dans ces tristes séjours, des colons déplorables mouraient, s’échappaient comme des forçats, attendaient quelque nouvelle révolution ou quelque amnistie, la mère patrie continuait à broyer les redoutables problèmes qui avaient amené leur exil sans une ombre de progrès.\par
Une grosse erreur de philosophie historique contribuait au moins autant que le goût particulier de la France pour les théories à fausser le jugement national sur cette grave question des formes du gouvernement, c’était justement l’exemple de l’Amérique. L’école républicaine citait toujours cet exemple comme bon et facile à suivre. Rien de plus superficiel. Que des colonies habituées à se gouverner d’une façon indépendante rompent le lien qui les unit à la mère patrie, que, ce lien rompu, elles se passent de royauté et pourvoient à leur sûreté par un pacte fédératif, il n’y a rien en cela que de naturel. Cette façon de se séparer du trône comme une bouture portant en elle son germe de vie est te principe éternel de la colonisation, principe qui est une des conditions du progrès de l’humanité, de la race aryenne en particulier. La Virginie, la Caroline, étaient des républiques avant la guerre de l’indépendance. Cette guerre ne changea rien à la constitution intérieure des États ; elle coupa seulement la corde, devenue gênante, qui les liait à l’Europe, et y substitua un lien fédéral. Ce ne fut pas là une œuvre révolutionnaire ; une conception du droit éminemment conservatrice, un esprit aristocratique et juridique de liberté provinciale était au fond de ce grand mouvement. De même, quand le Canada et l’Australie verront se rompre le lien léger qui les rattache à l’Angleterre, ces pays, habitués à se gouverner eux-mêmes, continueront leur vie propre, sans presque s’apercevoir du changement. Si la France avait entrepris sérieusement la colonisation de l’Algérie, l’Algérie aurait chance d’être une république avant la France. Les colonies, formées de personnes qui ne se trouvent pas à l’aise dans leur pays natal et qui cherchent plus de liberté qu’elles n’en ont chez elles, sont toujours plus près de la république que la mère patrie, liée par ses vieilles habitudes et ses vieux préjugés.\par
Ainsi continua de vivre en France un parti qui ne permet pas à la royauté constitutionnelle de se développer, le parti républicain radical. La situation de la France ne fut nullement celle de l’Angleterre ; à côté de la droite, de la gauche et du centre, il y eut un parti irréconciliable, négation totale du gouvernement existant, un parti qui ne dit pas au gouvernement : « Faites telle chose, et nous sommes à vous ; » mais qui lui laisse entendre : « Quoi que vous fassiez, nous serons contre vous. » La république est en un sens le terme de toute société humaine, mais on conçoit deux manières bien différentes d’y venir. Établir la république de haute lutte, en détruisant tous les obstacles, est le rêve des esprits ardents. Il est une autre voie plus douce et plus sûre : conserver les anciennes familles royales comme de précieux monuments et d’antiques souvenirs n’est pas seulement une fantaisie d’antiquaire ; les dynasties ainsi conservées deviennent des rouages infiniment commodes du gouvernement constitutionnel à certains jours de crise. Les pays qui ont suivi cette marche, comme l’Angleterre, arriveront-ils un jour à la république parfaite, sans dynastie héréditaire et avec suffrage universel ? C’est demander si l’hyperbole atteint ses asymptotes. Qu’importe, puisqu’en réalité elle en approche si près, que la distance est insaisissable à l’œil ! Voilà ce que le parti républicain français ne comprend pas. Pour la forme de la république, il en sacrifie la réalité. Pour ne pas suivre une grande roule, tracée, faisant quelques détours, il préfère se jeter dans les précipices et les fondrières. On vit rarement avec autant d’honnêteté aussi peu d’esprit politique et de pénétration.\par
L’année 1848 mit la plaie à nu, et posa pour tout esprit exercé le principe fondamental de la philosophie de notre histoire. La révolution de 1848 ne fut pas un effet sans cause (une telle assertion serait dénuée de sens), ce fut un effet complètement disproportionné avec sa cause apparente. Le choc ne fut rien, la ruine fut immense. Il arriva en 1848 ce qui serait arrivé en Angleterre, si Guillaume III eût été emporté par un des accès de vif mécontentement que provoqua son gouvernement. L’histoire d’Angleterre eût été bouleversée dans une telle hypothèse. En Angleterre, le goût du peuple pour la légitimité et la crainte de la république furent assez forts pour faire traverser à la nouvelle dynastie les moments difficiles. En France, l’affaiblissement moral de la nation, son manque de foi en la royauté, l’énergie du parti républicain, suffirent pour jeter par terre un trône qui n’avait que des assises ruineuses. On vit ce jour-là la funeste situation où la France est restée depuis la Révolution. Si en France la Révolution et la République avaient jeté des racines moins profondes, la maison d’Orléans et avec elle le régime parlementaire se fussent sûrement consolidés ; si l’idée républicaine avait été dominante, elle aurait, après diverses actions et réactions, entraîné le pays, et la République se fût fondée : ni l’une ni l’autre de ces deux suppositions ne se réalisa. L’esprit républicain s’était trouvé assez fort pour empêcher la royauté constitutionnelle de durer ; il ne fut pas assez fort pour établir la République. De là une position fausse, bizarre et faite pour amener un triste abaissement. Ce qui s’est passé en 1848 pourrait se passer plusieurs fois encore ; tâchons d’en bien démêler la loi secrète et l’intime raison.\par
Quand nous voyons un homme mourir d’un rhume, nous en concluons, non pas que le rhume est une maladie mortelle, mais que cet homme était poitrinaire. La maladie dont mourut le gouvernement de juillet fut de même si légère, qu’il faut admettre que sa constitution était des plus chétives. La petite agitation des banquets était de celles qu’un gouvernement doit pouvoir supporter sous peine de n’être pas capable de vivre. Comment, avec toutes les apparences de la santé, le gouvernement de juillet se trouva-t-il si faible ? C’est qu’il n’avait pas ce qui donne à un gouvernement de bons poumons, un cœur vigoureux, de solides viscères ; je veux dire la sérieuse adhésion des parties résistantes du pays. Le sentiment de profonde humanité qui empêcha Louis-Philippe de livrer la bataille, outre qu’il impliquait une défiance de son droit, ne suffit pas pour expliquer sa chute. Le parti républicain qui fit la révolution était une imperceptible minorité. Dans un pays où le gouvernement eût été moins centralisé, et où l’opinion se fût trouvée moins divisée, la majorité eut fait volte-face ; mais la province n’avait pas encore l’idée de résister à un mouvement venant de Paris ; de plus, si la faction qui prit part au mouvement le 24 février 1848 fut insignifiante, le nombre de ceux qui eussent pu défendre la dynastie vaincue était peu considérable. Le parti légitimiste triompha, et, sans faire de barricades, eut ce jour-là sa revanche. La dynastie d’Orléans n’avait pas su, malgré sa profonde droiture et sa rare honnêteté, parler au cœur du pays ni se faire aimer.\par
Ainsi mise en présence du fait accompli par une minorité turbulente, que va faire la France ? Un pays qui n’a pas de dynastie unanimement acceptée est toujours dans ses actions un peu gauche et embarrassé. La France plia ; elle accepta la République sans y croire, sournoisement, et bien décidée à lui être infidèle. L’occasion ne manqua point. Le vote du 10 décembre fut une évidente répudiation de la République. Le parti qui avait fait la révolution de février subit la loi du talion. Qu’on nous permette une expression vulgaire : il avait joué un mauvais tour à la France, la France lui joua un mauvais tour. Elle fit comme un bourgeois honnête dont les gamins s’empareraient en un jour d’émeute et qu’ils affubleraient du bonnet rouge ; ce digne homme pourrait se laisser faire par amour de la paix, mais en garderait probablement quelque rancune. La surprise du scrutin répondit à la surprise de l’émeute. Sûrement la conduite de la France eût été plus digne et plus loyale, si, à l’annonce de la révolution, elle avait résisté en face, arrêté poliment les commissaires du gouvernement provisoire à leur descente de diligence, et convoqué des espèces de conseils généraux qui eussent rétabli la monarchie ; mais plusieurs raisons qui s’entrevoient trop facilement pour qu’il soit bien besoin de les développer rendaient alors cette conduite impossible ; en outre, la nation à qui l’on donne le suffrage universel devient toujours un peu dissimulée. Elle a entre les mains une arme toute-puissante, qui dispense des guerres civiles. Quand on est sûr que l’ennemi sera obligé de passer par un défilé dont on est maître et où il sera forcé de subir le feu sans répondre, on ne va pas l’attaquer. La France attendit, et en décembre 1848 infligea au parti républicain un affront sanglant. Si février avait prouvé que la France ne tenait pas beaucoup à la monarchie constitutionnelle de la maison d’Orléans, le scrutin du 10 décembre prouva qu’elle ne tenait pas davantage à la République. L’impuissance politique de ce grand pays parut dans tout son jour.\par
Que dire de ce qui se passa ensuite ? Nous n’aimons pas plus les coups d’État que les révolutions ; nous n’aimons pas les révolutions, justement parce qu’elles amènent les coups d’État. On ne peut cependant accorder au parti de 1848 sa prétention fondamentale.\par
Ce parti, au nom de je ne sais quel droit divin, s’arroge le pouvoir qu’il n’accorde à aucun autre parti d’avoir pu enchaîner la France, si bien que les illégalités qu’on a faites pour briser les liens dont il avait entouré le pays sont des crimes, tandis que sa révolution de février, à lui, n’a été qu’un acte glorieux. Voilà qui est inacceptable. {\itshape Quis tulerit Gracchos de seditione querentes} ? Qui frappe avec l’épée finira par l’épée. Si les fusils qui couchèrent en joue M. Sauzet et la duchesse d’Orléans le 24 février 1848 furent innocents, les baïonnettes qui envahirent la chambre le 2 décembre 1851 ne furent pas coupables. Pour nous, chacune de ces violences est un coup de poignard à la patrie, une blessure qui atteint les parties les plus essentielles de sa constitution, un pas de plus dans un labyrinthe sans issue, et nous avons le droit de dire de toutes ces néfastes journées :\par
Excidat illa dies ævo, nec postera credant Secula ; nos etiam taceamus, et oblita multa Nocte tegi nostræ patiamur crimina gentis.
\section[{III}]{III}\renewcommand{\leftmark}{III}

\noindent L’empereur Napoléon III et le petit groupe d’hommes qui partagent sa pensée intime apportèrent au gouvernement de la France un programme qui, pour n’être pas fondé sur l’histoire, ne manquait pas d’originalité : relever la tradition de l’Empire, profiter de sa légende grandiose, si vivante encore dans le peuple, faire parler le sentiment populaire à cet égard par le suffrage universel, amener par ce suffrage une délégation engageant l’avenir et fondant l’hérédité, provoquer, suivant l’idée chère à la France, une élection dynastique\footnote{L’idée que l’élection a joué un rôle à l’origine des dynasties de la France, quoique historiquement fausse, se retrouve dès la fin du \textsc{xiii}\textsuperscript{e} siècle. Voir les romans de {\itshape Hugues Capet} et de {\itshape Baudouin de Sebourg}.} ; au-dedans, gouvernement personnel de l’empereur, avec des apparences de gouvernement parlementaire habilement réduites à la nullité ; au-dehors, rôle brillant et actif, rendant peu à peu à la France, par la guerre et la diplomatie, la place de premier ordre qu’elle possédait, il y a soixante ans, parmi les nations de l’Europe, et que depuis 1814 elle a perdue.\par
La France, pendant dix-sept ans, a laissé faire cette expérience avec une patience qu’on pourrait appeler exemplaire, si jamais il était bon pour une nation de trop pratiquer l’abnégation quand il s’agit de ses destinées. Où en est l’expérience ? Quels résultats a-t-elle amenés ?\par
Peut-on dire d’abord que la nouvelle maison napoléonienne se soit fondée, c’est-à-dire ait créé autour d’elle ces sentiments d’affection et de dévouement personnel qui font la force d’une dynastie ? Il ne faut pas à cet égard se faire d’illusion. L’égoïsme, le scepticisme, l’indifférence envers les gouvernants, la persuasion qu’on ne leur doit aucune reconnaissance, ont totalement desséché le cœur du pays. La question est devenue une question d’intérêt. La fortune publique ayant pris un grand accroissement, si la question se posait en ces termes : {\itshape révolution}, —{\itshape  pas de révolution}, le second terme obtiendrait une immense majorité ; mais souvent un pays qui ne veut pas de la révolution fait ce qu’il faut pour l’amener. En tout cas, ces sentiments d’effusion tendre et de fidélité que le pays avait autrefois pour ses rois, il n’y faut plus penser. Les personnes ayant pour la dynastie napoléonienne les sentiments que le royaliste de la Restauration avait pour la famille royale pourraient se compter. Il\phantomsection
\label{\_GoBack} n’y a presque pas de légitimistes napoléoniens ; voilà un fait dont le gouvernement ne peut assez se pénétrer.\par
La partie du programme de l’empereur Napoléon III relative à la gloire militaire et au rôle prépondérant de la France avait sa grandeur, et ceux qui, du point de vue des intérêts généraux de la civilisation, sont reconnaissants à l’empereur de la guerre de Crimée et de celle d’Italie, ne peuvent juger avec sévérité tous les points de la politique étrangère du second Empire ; mais il est clair que la France n’est nullement à l’unisson de pareilles idées. Mis au suffrage universel, le plébiscite {\itshape pas de guerre} réunirait une majorité bien plus forte encore que {\itshape pas de révolution.} La France actuelle n’est pas plus héroïque que sentimentale. La prépondérance d’une nation européenne sur les autres est d’ailleurs devenue impossible dans l’état actuel des sociétés. Les intentions menaçantes imprudemment exprimées de ce côté du Rhin (et ce n’est pas le gouvernement qui à cet égard a été le plus coupable ou le plus maladroit) ont provoqué chez les nations germaniques une émotion qui tombera le jour où elles seront rassurées sur l’ambition qu’elles ont pu nous supposer. Ce jour-là cessera la force de la Prusse dans le corps germanique, force qui n’a pas d’autre raison d’être que la crainte de la France. Ce jour-là même cessera probablement le désir d’unité politique, désir si peu conforme à l’esprit germanique et qui n’a jamais été chez les Allemands qu’une mesure défensive, impatiemment tolérée, contre un voisin fortement organisé.\par
Ce seul point changé dans le programme primitif de l’empereur Napoléon III suffirait pour modifier tout ce qui a trait au gouverment intérieur. L’empereur Napoléon III n’a jamais cru pouvoir gouverner sans une chambre élective ; seulement, il a espéré rester longtemps, sinon toujours, maître des élections. C’était là un calcul qui n’aurait pu se réaliser qu’avec de perpétuelles guerres, de perpétuelles victoires. Le gouvernement personnel ne se maintient qu’à la condition d’avoir toujours et partout gloire et succès. Comment pouvait-on espérer qu’à moins d’un éblouissement de prospérité le pays déposerait éternellement dans l’urne le bulletin que l’administration lui mettait dans la main ? Il était inévitable qu’un jour la France voulût se servir de l’arme puissante qu’on lui avait laissée, et prit une part de responsabilité dans ses affaires. En politique, on ne joue pas longtemps avec les apparences. On devait s’attendre à ce que le simulacre de gouvernement parlementaire que l’empereur Napoléon III avait toujours conservé devînt une réalité sérieuse. Les dernières élections ont fait passer cette supposition dans le domaine des faits accomplis. Les élections de mai et juin 1869 ont montré que la loi de notre société ne pouvait être celle du césarisme romain. Le césarisme romain fut également à son origine un despotisme entouré de fictions républicaines ; le despotisme tua les fictions ; chez nous, au contraire, les fictions représentatives ont tué le despotisme. Gela n’arriva pas sous le premier Empire, car le mode d’élection du Corps législatif était alors tout à fait illusoire. Rien ne prouve mieux que les événements de ces derniers mois combien l’idéal de gouvernement créé par l’Angleterre s’impose forcément à tous les États. On dit souvent que la France n’est pas faite pour un tel gouvernement. La France vient de prouver qu’elle pense le contraire ; en tout cas, si cela était vrai, je dirais qu’il faut désespérer de l’avenir de la France. Le régime libéral est une nécessité absolue pour toutes les nations modernes. Qui ne pourra s’y accommoder périra. D’abord le régime libéral donnera aux nations qui l’ont adopté une immense supériorité sur celles qui ne pourront s’y plier. Une nation qui ne sera capable ni de la liberté de la presse, ni de la liberté de réunion, ni de la liberté politique, sera certainement dépassée et vaincue par les nations qui peuvent supporter de telles libertés. Ces dernières seront toujours mieux informées, plus instruites, plus sérieuses, mieux gouvernées.\par
Une autre raison encore établit que, si la France est condamnée à une fatale alternative d’anarchie et de despotisme, sa perte est inévitable. On ne sort de l’anarchie que par un grand état militaire, lequel, outre qu’il ruine et épuise la nation, ne peut conserver son ascendant sur la nation qu’à la condition d’être toujours victorieux à l’étranger. Le régime de compression militaire à l’intérieur amène nécessairement la guerre étrangère ; une armée vaincue et humiliée ne peut comprimer énergiquement. Or, dans l’état actuel de l’Europe, une nation condamnée à faire par système la guerre à l’extérieur est une nation perdue. Cette nation provoquera sans cesse contre elle des coalitions et des invasions. Voilà comment l’état instable du gouvernement intérieur de la France constitue pour elle un danger au dehors, et fait d’elle une nation guerrière, bien que l’opinion générale y soit très-pacifique. L’équilibre de l’Europe exige que toutes les nations qui la composent aient à peu près la même constitution politique. Un {\itshape ebrius inter sobrios} ne saurait être toléré dans ce concert.\par
De toutes parts, on arrive donc à cette conséquence, que la France doit entrer sans retard dans la voie du gouvernement représentatif. Une question préalable se poserait ici : l’empereur Napoléon III se résignera-t-il à ce changement de rôle ? Modifiera-t-il à ce point un programme qui est pour lui non un simple calcul d’ambition, mais une foi, un enthousiasme, la croyance qui explique toute sa vie ? Après avoir aimé jusqu’au fanatisme un idéal qu’il tient pour le seul noble et grand, mais dont la France n’a pas voulu, n’éprouvera-t-il pas un invincible dégoût pour ce régime de paix, d’économie, de petites batailles ministérielles qui s’est toujours présenté à lui comme une image de décadence, et qu’il associe au souvenir d’une dynastie tenue de lui en peu d’estime ? Sortira-t-il de ce cercle de conseillers et de ministres médiocres où il paraît se complaire ? Le souverain investi par plébiscite de la plénitude des droits populaires peut-il être parlementaire ? Le plébiscite n’est-il pas la négation de la monarchie constitutionnelle ? Un tel gouvernement est-il jamais sorti d’un coup d’État ? peut-il exister avec le suffrage universel ? Le respect dû à la personne du souverain nous interdit d’examiner ces questions. Le caractère de l’empereur Napoléon III est d’ailleurs un problème sur lequel, même quand on possédera des données que personne maintenant ne peut avoir, on fera bien de s’exprimer avec beaucoup de précautions. Il y aura peu de sujets historiques où il sera plus important d’user de retouches, et, si dans cinquante ans il n’y a pas un critique aussi profond que M. Sainte-Beuve, aussi consciencieux, aussi attentif à ne pas effacer les contradictions et à les expliquer, l’empereur Napoléon III ne sera jamais bien jugé. Nous ne ferons qu’une seule réflexion. Les considérations de race et de sang, qui étaient jadis décisives en histoire, ont beaucoup perdu de leur force. Des substitutions qui eussent été impossibles sous l’ancien régime peuvent être devenues possibles. Le caractère des familles, qui était autrefois inflexible, si bien qu’un Bourbon par exemple ne pouvait convenir qu’à un rôle déterminé, est maintenant susceptible de bien des modifications. Le rôle historique et la race ne sont plus deux choses inséparables. Qu’un héritier de Napoléon I\textsuperscript{er} accomplisse une œuvre en contradiction avec l’œuvre de Napoléon I\textsuperscript{er}, il n’y a en cela rien d’absolument inadmissible. L’opinion publique est tellement devenue le souverain maître, que chaque nom, chaque homme n’est que ce qu’elle le fait. Les objections {\itshape a priori} que certaines personnes élèvent contre la possibilité d’un avenir constitutionnel avec la famille Bonaparte ne sont donc pas décisives. La famille capétienne, qui devint bien réellement la représentation de la nationalité française et du tiers état, fut, à l’origine, ultra-germanique, ultra-féodale.\par
De même que l’architecture fait un style avec des fautes et des inexpériences, de même un pays tire tel parti qu’il veut des actes où la fatalité l’a poussé. Nous jouissons des bienfaits de la royauté, quoique la royauté ait été fondée par une série de crimes ; nous profilons des conséquences de la Révolution, quoique la Révolution ait été un tissu d’atrocités. Une triste loi des choses humaines veut qu’on devienne sage quand on est usé. On a été trop difficile, on a repoussé l’excellent ; on reste dans le médiocre par crainte de pire. La coquette qui a refusé les plus beaux mariages finit souvent par un mariage de raison. Ceux qui ont rêvé la République sans républicains se laissent aller de même à concevoir un règne de la famille Bonaparte sans bonapartistes, un état de choses où cette famille, débarrassée de l’entourage compromettant de ceux qui ont fondé son second avènement, trouverait ses meilleurs appuis, ses conseillers les plus sûrs dans ceux qui ne l’ont pas faite, mais l’ont acceptée comme voulue par la France et susceptible d’ouvrir quelque issue à l’étrange impasse où nous a engagés la destinée. Il est très-vrai qu’il n’y a pas un exemple de dynastie constitutionnelle sortie d’un coup d’État. Des Visconti, des Sforza, tyrans issus de discordes républicaines, ne sont pas l’étoffe dont on fait des royautés légitimes. De telles royautés ne se sont fondées que par la particulière dureté et hauteur de là race germanique aux époques barbares et inconscientes, où l’oubli est possible et où l’humanité vit dans ces ténèbres mystérieuses qui fondent le respect. {\itshape Fata viam invenient…} Le défi étrange que la France a jeté à toutes les lois de l’histoire impose en de telles inductions une extrême réserve. Montons plus haut, et, négligeant ce qui peut être déjoué par l’accident de demain, recherchons quelles sont dans le pays les raisons d’être de la monarchie constitutionnelle, quels motifs peuvent en faire espérer le triomphe, quelles craintes peuvent rester sur son établissement.
\section[{IV}]{IV}\renewcommand{\leftmark}{IV}

\noindent Nous avons vu que le trait particulier de la France, trait qui la sépare profondément de l’Angleterre et des autres États européens (l’Italie et l’Espagne jusqu’à un certain point exceptées), est que le parti républicain constitue dans son sein un élément considérable. Ce parti, qui fut assez fort pour renverser Louis-Philippe et pour imposer quelques mois sa théorie à la France, fut, après le 2 décembre, l’objet d’une sorte de proscription. A-t-il disparu pour cela ? Non, certes.\par
Les progrès qu’il a faits en ces dix-sept dernières années ont été très-sensibles. Non seulement il s’est maintenu en possession de la majorité dans Paris et les grandes villes, mais encore il a conquis des pays entiers ; toute la zone des environs de Paris lui appartient. L’esprit démocratique, tel que nous le connaissons à Paris, avec sa raideur, son ton absolu, sa simplicité décevante d’idées, ses soupçons méticuleux, son ingratitude, a conquis certains cantons ruraux d’une façon qui étonne. Dans tel village, la situation des fermiers et des valets de ferme est exactement celle des ouvriers et des patrons dans une ville de manufactures ; des paysans vous y feront de la politique rogue, radicale et jalouse avec autant d’assurance que des ouvriers de Belleville ou du faubourg Saint-Antoine. L’idée des droits égaux de tous, la façon de concevoir le gouvernement comme un simple service public qu’on paye et auquel on ne doit ni respect ni reconnaissance, une sorte d’impertinence américaine, la prétention d’être aussi sage que les meilleurs hommes d’État et de réduire la politique à une simple consultation de la volonté de la majorité, voilà l’esprit qui envahit de plus en plus, même les campagnes. Je ne doute pas que cet esprit ne fasse tous les jours des progrès, et qu’aux prochaines élections, il ne se montre, partout où il sera le maître, plus exigeant, plus intraitable encore qu’il ne l’a été cette année.\par
Le parti républicain pourra-t-il cependant devenir un jour la majorité et faire prévaloir en France les institutions américaines ? Je ne le crois pas. L’essence de ce parti est d’être une minorité. S’il aboutissait à une révolution sociale, il pourrait créer de nouvelles classes, mais ces classes deviendraient monarchiques le lendemain de leur enrichissement. Les intérêts les plus pressants de la France, son esprit, ses qualités et ses défauts lui font de la royauté un besoin. Le lendemain du jour où le parti radical aura jeté bas une monarchie, les journalistes, les littérateurs, les artistes, les gens d’esprit, les gens du monde, les femmes, conspireront pour en établir une autre, car la monarchie répond à des besoins profonds de la France. Notre amabilité seule suffit pour faire de nous de mauvais républicains. Les charmantes exagérations de la vieille politesse française, la courtoisie qui nous met aux pieds de ceux avec qui nous sommes en rapport, sont le contraire de cette raideur, de cette âpreté, de cette sécheresse que donne au démocrate le sentiment perpétuel de son droit. La France n’excelle que dans l’exquis, elle n’aime que le distingué, elle ne sait faire que de l’aristocratique. Nous sommes une race de gentilshommes ; notre idéal a été créé par des gentilshommes, non, comme celui de l’Amérique, par d’honnêtes bourgeois, de sérieux hommes d’affaires. De telles habitudes ne sont satisfaites qu’avec une haute société, une cour et des princes du sang. Espérer que les grandes et fines œuvres françaises continueraient de se produire dans un monde bourgeois, n’admettant d’autre inégalité que celle de la fortune, c’est une illusion. Les gens d’esprit et de cœur qui dépensent le plus de chaleur pour l’utopie républicaine seraient justement ceux qui pourraient le moins s’accommoder d’une pareille société. Les personnes qui poursuivent si avidement l’idéal américain oublient que cette race n’a pas notre passé brillant, qu’elle n’a pas fait une découverte de science pure ni créé un chef-d’œuvre, qu’elle n’a jamais eu de noblesse, que le négoce et la fortune l’occupent tout entière. Notre idéal à nous ne peut se réaliser qu’avec un gouvernement donnant de l’éclat à ce qui approche de lui, et créant des distinctions en dehors de la richesse. Une société où le mérite d’un homme et sa supériorité sur un autre ne peuvent se révéler que sous forme d’industrie et de commerce nous est antipathique ; non que le commerce et l’industrie ne nous paraissent honnêtes, mais parce que nous voyons bien que les meilleures choses (par exemple, les fonctions du prêtre, du magistrat, du savant, de l’artiste et de l’homme de lettres sérieux) sont l’inverse de l’esprit industriel et commercial, le premier devoir de ceux qui s’y adonnent étant de ne pas chercher à s’enrichir, et de ne jamais considérer la valeur vénale de ce qu’ils font.\par
Le parti républicain pourra donc empêcher tout gouvernement libéral de s’établir, car, en provoquant des séditions, il lui sera toujours loisible de forcer les gouvernements à s’armer de lois répressives, à restreindre les libertés, à fortifier l’élément militaire ; il est douteux qu’il soit capable de s’établir lui-même. La haine entre lui et la partie paisible du pays ira toujours s’envenimant, car il paraîtra de plus en plus au pays un éternel trouble-fête. Il ne réussira, je le crains, qu’à provoquer des espèces de crises périodiques, suivies d’expulsions violentes, que le parti conservateur montrera comme des assainissements, mais qui seront en réalité des affaiblissements, et qui en tout cas useront d’une manière déplorable le tempérament de la France. Dans ces vomissements convulsifs en effet, des éléments excellents, nécessaires à la vie d’une nation, seront rejetés avec les éléments impurs. Comme il est arrivé après 1848, les idées libérales souffriront de leur inévitable solidarité avec un parti qui, plein d’illusions généreuses, exerce un grand attrait sur les imaginations jeunes, et qui, d’ailleurs, a toute une partie de son programme en commun avec l’école libérale. Il est à craindre que de longues habitudes d’esprit, une certaine raideur, beaucoup de routine et d’habitude de tout juger d’après Paris (habitude facile à comprendre chez un parti qui fut à l’origine essentiellement parisien) n’induisent ce parti à croire que des révolutions dans le genre de 1830 et de 1848 pourraient se renouveler.\par
Rien ne serait plus funeste. Le temps des révolutions parisiennes est fini. Je fonde cette opinion beaucoup moins sur les changements matériels accomplis dans Paris que sur deux raisons qui pèseront, selon moi, d’un poids énorme sur les destinées de l’avenir.\par
L’une est l’établissement du suffrage universel. Un peuple en possession de ce suffrage ne laissera pas faire de révolution par sa capitale. Si une telle révolution s’opérait dans Paris (chose heureusement impossible), je suis persuadé que les départements ne l’accepteraient pas, que des barricades s’élèveraient sur les chemins de fer pour arrêter la propagation de l’incendie et empêcher l’approvisionnement de la capitale, que l’émeute parisienne, vite affamée, n’aurait que quelques jours de vie, l’émancipation de la province a fait depuis 1848 de grands progrès. Un autre événement, d’ailleurs, doit être pris en grande considération. Toute la philosophie de l’histoire est dominée par la question de l’armement. Rien n’a autant contribué au triomphe de l’esprit moderne que l’invention de la poudre à canon. L’artillerie a tué la chevalerie et la féodalité, créé la force des royautés et des États, maté définitivement la barbarie, rendu impossibles ces cyclones étranges du monde tartare qui, se formant au centre de l’Asie, venaient ébranler l’Europe et terrifier le monde chrétien. L’application délicate de la science à l’art de la guerre amènera de nos jours des révolutions presque aussi graves. La guerre deviendra de plus en plus un problème scientifique et industriel ; l’avantage sera pour la nation la plus riche, la plus scientifique, la plus industrieuse. Que si nous examinons les effets de ce changement à l’intérieur des États, il est clair que l’application en grand de la science à l’armement profitera uniquement aux gouvernements. L’effet de l’artillerie fut de démolir les uns après les autres tous les châteaux féodaux ; une décharge de tel engin perfectionné arrêtera une révolution. Aux époques où l’armement est peu perfectionné, un citoyen égale presque un soldat ; mais, dès que le procédé agressif devient une chose savante, exigeant des instruments de précision et demandant une éducation spéciale, le soldat a une immense supériorité sur la masse désarmée. Tout porte donc à croire que des révolutions commencées par les citoyens seraient désormais écrasées dans leur germe. C’est ce que comprennent avec leur habileté ordinaire les jésuites quand ils s’emparent des avenues de l’école de Saint-Cyr et de l’École polytechnique. Ils voient l’avenir de ceux qui savent manier les armes savantes et les forces disciplinées, et ils reconnaissent très bien que l’avantage, sous ce rapport, est aux anciennes classes nobles, moins préoccupées que la bourgeoisie d’industrie ou de positions civiles lucratives et par là même plus capables d’abnégation.\par
La France paraît donc devoir longtemps encore échapper â la République, même quand le parti républicain formerait la majorité numérique. La France voit grandir chaque jour dans son sein une masse populaire dénuée d’idéal religieux, et repoussant tout principe social supérieur à la volonté des individus. L’autre masse, non encore pénétrée de cette idée égoïste, est chaque jour diminuée par l’instruction primaire et par l’usage du suffrage universel ; mais, contre ce flot montant d’idées envahissantes, lesquelles, étant jeunes et inexpérimentées, ne tiennent compte d’aucune difficulté, se dressent des intérêts et des besoins supérieurs, qui veulent une organisation et une direction de la société par un principe de raison et de science distinct de la volonté des individus. Le démocrate s’imagine toujours que la conscience de la nation est parfaitement claire, il n’admet rien d’obscur, d’hésitant, de contradictoire dans l’opinion : compter les voix et faire ce que veut la majorité lui paraissent choses fort simples ; mais ce sont là des illusions. Longtemps encore l’opinion devra être devinée, pressentie, supposée et jusqu’à un certain point dirigée. De là des intérêts monarchiques qui, le lendemain de l’établissement de la République, se montreront formidables, même dans l’esprit de ceux qui auront fait ou laissé faire la République.\par
Le mouvement qui s’opère dans les classes populaires et qui tend à donner aux individus une conscience de plus en plus nette de leurs droits est un fait si évident, que vouloir s’y opposer serait de la pure folie. Le devoir de la politique est, non pas de combattre un tel mouvement, mais de le prévoir et de s’en accommoder. Les savants n’ont jamais cherché des moyens pour arrêter la marée ; ils ont mieux fait : ils ont si bien déterminé les lois du phénomène, que le navigateur sait minute par minute l’état de la mer et en tire grand profit. L’essentiel est que le flot ascendant n’emporte pas les digues nécessaires et ne produise pas, en se retirant, de funestes réactions. Or, c’est là, suivant les apparences, ce qui arrivera toutes les fois que la démocratie française sera conduite par le jacobinisme âpre, hargneux, pédantesque, qui remue le pays, parfois même lui donne de l’essor, mais ne le conduira jamais à une constitution assurée. Ce parti peut faire une révolution, il ne régnera pas plus de deux mois après l’avoir faite. Même le jour où (chose peu probable) il arriverait à une majorité de scrutin, il ne fonderait rien encore, car les éléments dont il dispose, excellents pour agiter, sont instables, faciles à diviser, et tout à fait incapables de fournir les éléments solides d’une construction. Sa force, quoique grande, est en partie une force de circonstance. Dix fois il m’a été donné, pendant une campagne électorale, d’entendre le dialogue que voici, « Nous ne sommes pas contents du gouvernement ; il coûte trop cher ; il gouverne au profit.d’idées qui ne sont pas les nôtres ; nous voterons pour le candidat de l’opposition la plus avancée. — Vous êtes donc révolutionnaires ? — Nullement ; une révolution serait le dernier malheur. Il s’agit seulement de faire impression sur le gouvernement, de le forcer à changer, de le contenir vigoureusement. — Mais, si la Chambre est composée de révolutionnaires, c’est le renversement du gouvernement — Non ; il n’y en aura que vingt ou trente, et puis le gouvernement est si fort ! Il a les chassepots ! » Ce naïf raisonnement donne la mesure de l’illusion que se fait la gauche radicale, quand elle s’imagine que le pays la veut pour elle-même. Une grande partie du pays la prend comme un bâton pour châtier le pouvoir, non comme un appui pour s’étayer. « On nous nomme, donc on nous aime, » serait de la part des honorables membres de l’opposition dite avancée la plus dangereuse des conclusions. On les nomme pour donner une leçon au gouvernement, et avec la persuasion que le gouvernement est assez fort pour supporter la leçon. Le jour où il n’en serait plus ainsi et où l’on s’apercevrait qu’on a mis en danger l’existence du gouvernement, il se ferait une volte-face, si bien que le parti radical est soumis à cette loi étrange, que l’heure de sa victoire est le commencement de sa défaite. Son triomphe est sa fin ; souvent ceux qui l’ont nommé et mis en avant applaudissent eux-mêmes à sa proscription.\par
L’ordre en effet est devenu dans nos sociétés modernes ! d’Europe une condition si impérieuse, que de longues guerres civiles sont impossibles. On cite quelquefois l’exemple de ces illustres républiques grecques et italiennes, qui créèrent une admirable civilisation au milieu d’un État politique assez analogue à notre Terreur ; mais on ne saurait rien conclure de là pour des sociétés comme les nôtres, où les ressorts sont bien plus compliqués. L’Espagne, les républiques espagnoles de l’Amérique, l’Italie même, peuvent supporter plus d’anarchie que la France, parce que ce sont des pays où la vie matérielle est plus facile, où il y a moins de sources de richesse, où les intérêts et le crédit ont pris moins de développement. La Terreur, à la fin du dernier siècle, fut la suspension de la vie. Ce serait de nos jours bien pis encore. De même qu’un être d’une structure simple résiste à des milieux très-différents, et que les animaux fins, tels que l’homme, ont des limites de vie très restreintes, si bien que de légers changements dans leurs habitudes amènent pour eux la mort, de même nos civilisations montées comme de savants appareils ne supportent pas de crises. Elles ont, si j’ose le dire, le tempérament délicat ; un degré de plus ou de moins les tue. Huit jours d’anarchie amèneraient des pertes incalculables ; au bout d’un mois peut-être, les chemins de fer s’arrêteraient. Nous avons créé des mécanismes d’une précision infinie, des outillages qui marchent par la confiance et qui tous supposent une profonde tranquillité publique, un gouvernement à la fois fortement établi et sérieusement contrôlé. Je sais qu’aux États-Unis les choses ne se passent point de la sorte ; on y supporte des désordres qui chez nous feraient pousser des cris d’alarme. Cela vient de ce que l’assise constitutionnelle des États-Unis n’est jamais réellement compromise. Ces pays américains, peu gouvernés, ressemblent aux pays européens où la dynastie est hors de question. Ils ont le respect de la loi et de la constitution, qui représentent chez eux ce qu’est en Europe le dogme de la légitimité. Comparer les pays à tendances socialistes, comme le nôtre, où tant de personnes attendent d’une révolution l’amélioration de leur sort, à de pareils États, complètement exempts de socialisme, où l’homme, tout occupé de ces affaires privées, demande au gouvernement très peu de garanties, est la plus profonde erreur qu’on puisse commettre en fait d’histoire philosophique.\par
Le besoin d’ordre qu’éprouvent nos vieilles sociétés européennes, coïncidant avec le perfectionnement des armes, donnera en somme aux gouvernements autant de force que leur en enlève chaque jour le progrès des idées révolutionnaires. Comme la religion, l’ordre aura ses fanatiques. Les sociétés modernes offrent cette particularité, qu’elles sont d’une grande douceur quand leur principe n’est pas en danger, mais qu’elles deviennent impitoyables si on leur inspire des doutes sur les conditions de leur durée. La société qui a eu peur est comme l’homme qui a eu peur : elle n’a plus toute sa valeur morale. Les moyens qu’employa la société catholique au xiiie et au \textsc{xvi}\textsuperscript{e} siècle pour défendre son existence menacée, la société moderne les emploiera, sous des formes plus expéditives et moins cruelles, mais non moins terribles. Si les vieilles dynasties y sont impuissantes, ou si, comme il est probable, elles refusent le pouvoir dans des conditions indignes d’elles, on recourra aux {\itshape paciers} et aux podestats de l’Italie du moyen âge, que l’on chargera à forfait, et sur un sanglant programme réglé d’avance, de rétablir les conditions de la vie. Des dictateurs d’aventure analogues aux généraux de l’Amérique espagnole se chargeront seuls d’une telle besogne. Comme nos races cependant ont un fonds de fidélité dont elles ne se déparlent pas, comme d’ailleurs il restera longtemps des survivants des anciennes dynasties, il y aura probablement des retours de légitimité après chaque cruelle dictature. Plus d’une fois encore, on suppliera les vieux détenteurs traditionnels de rôles nationaux de reprendre leur tâche et de rendre, à tout prix aux pays qui contractèrent jadis avec leurs ancêtres, un peu de paix, de bonne foi et d’honneur. Peut-être se feront-ils prier et mettront-ils à leur acceptation des clauses qu’on ne marchandera pas. En présence de certains faits comme ceux qui se sont passés récemment en Grèce, au Mexique, en Espagne, le parti démocratique dit parfois avec un sourire : « On ne trouve plus de rois. » En effet, nous verrons un temps où la royauté dépréciée n’aura plus assez d’attraits pour tenter les princes capables et se respectant eux-mêmes. Dieu veuille qu’un jour, pour avoir trop fait fi des libertés octroyées, on ne soit pas amené à prier les souverains de les réserver toutes, ou de n’en délier le faisceau que lentement, par des concessions et des chartes personnelles, locales, momentanées !\par
Un retour des barbares, c’est-à-dire un nouveau triomphe des parties moins conscientes et moins civilisées de l’humanité sur les parties plus conscientes et plus civilisées, parait, au premier coup d’œil, impossible. Entendons-nous bien à cet égard. Il existe encore dans le monde un réservoir de forces barbares, placées presque toutes sous la main de la Russie. Tant que les nations civilisées conserveront leur forte organisation, le rôle de cette barbarie est à peu près réduit à néant ; mais certainement, si (ce qu’à Dieu ne plaise !) la lèpre de l’égoïsme et de l’anarchie faisait périr nos États occidentaux, la barbarie retrouverait sa fonction, qui est de relever la virilité dans les civilisations corrompues, d’opérer un retour vivifiant d’instinct quand la réflexion a supprimé la subordination, de montrer que se faire tuer volontiers par fidélité pour un chef (chose que le démocrate tient pour basse et insensée) est ce qui rend fort et fait posséder la terre. Il ne faut pas se dissimuler en effet que le dernier terme des théories démocratiques socialistes serait un complet affaiblissement. Une nation qui se livrerait à ce programme, répudiant toute idée de gloire, d’éclat social, de supériorité individuelle, réduisant tout à contenter les volontés matérialistes des foules, c’est-à-dire à procurer la jouissance du plus grand nombre, deviendrait tout à fait ouverte à la conquête, et son existence courrait les plus grands dangers.\par
Comment prévenir ces tristes éventualités, que nous avons voulu montrer comme des possibilités et non comme des craintes déterminées ? Par le programme réactionnaire ? En comprimant, éteignant, serrant, gouvernant de plus en plus ? Non, mille fois non ; cette politique a été l’origine de tout le mal ; elle serait le moyen de tout perdre. Le programme libéral est en même temps le programme vraiment conservateur. Monarchie constitutionnelle, limitée et contrôlée ; décentralisation, diminution du gouvernement, forte organisation de la commune, du canton, du département ; large essor donné à l’activité individuelle dans le domaine de l’art, de l’esprit, de la science, de l’industrie, de la colonisation ; politique décidément pacifique, abandon de toute prétention à des agrandissements territoriaux en Europe ; développement d’une bonne instruction primaire et d’une instruction supérieure capable de donner aux mœurs de la classe instruite la base d’une solide philosophie ; formation d’une chambre haute provenant de modes d’élection très-variés et réalisant à côté de la simple représentation numérique des citoyens la représentation des intérêts, des fonctions, des spécialités, des aptitudes diverses ; dans les questions sociales, neutralité du gouvernement ; liberté entière d’association ; séparation graduelle de l’Église et de l’État, condition de tout sérieux dans les opinions religieuses : voilà ce qu’on rêve quand on cherche, avec la réflexion froide et dégagée des aveuglements d’un patriotisme intempérant, la voie du possible. À quelques égards, c’est là une politique de pénitence, impliquant l’aveu que, pour le moment, il s’agit moins de continuer la Révolution que de la corriger. Je me figure souvent en effet que l’esprit français traverse une période de jeûne, une sorte de diète politique, durant laquelle l’attitude qui nous convient est celle de l’homme d’esprit qui expie les fautes de sa jeunesse, ou bien du voyageur déçu qui contourne par le plus long chemin la hauteur qu’il avait prétendu escalader à pic. Les révolutions, comme les guerres civiles, fortifient, si l’on en sort ; elles tuent, si elles durent. Les brillantes et hardies entreprises nous ont mal réussi ; essayons des voies plus humbles. Les initiatives de Paris ont été funestes ; voyons ce que peut le terre-à-terre provincial. Craignons ces revendications impérieuses et hautaines, si rarement suivies d’effet. Qu’on me montre un exemple, au moins en France, d’une liberté prise de haute lutte et gardée.\par
Nul plus que moi n’admire et n’aime ce centre extraordinaire de vie et de pensée qui s’appelle Paris. Maladie si l’on veut, mais maladie à la façon de la perle, précieuse et exquise hypertrophie, Paris est la raison d’être de la France. Foyer de lumière et de chaleur, je veux bien qu’on l’appelle aussi foyer de décomposition morale, pourvu qu’on m’accorde que sur ce fumier naissent des fleurs charmantes, dont quelques-unes de première rareté. La gloire de la France est de savoir entretenir cette prodigieuse exhibition permanente de ses produits les plus excellents ; mais il ne faut pas se dissimuler à quel prix ce merveilleux résultat est obtenu. Les capitales consomment, elles ne produisent pas. Il ne faut pas, en portant le mal aux extrêmes, risquer de faire de la France alternativement une tête sans corps et un corps sans tête. L’action politique de Paris doit cesser d’être prépondérante. Les deux choses que la province a jusqu’ici reçues de Paris, les révolutions et le gouvernement, la province commence à les accueillir avec une égale antipathie. Seule, la démocratie parisienne ne fondera rien de solide ; si l’on n’y prend garde, elle amènera des exterminations périodiques, funestes pour la France, puisque la démocratie parisienne est d’un autre côté un ferment nécessaire, un excitant sans lequel la vie de la France languirait. Les réunions publiques de la dernière période électorale à Paris ont révélé un manque complet d’esprit politique. Maîtresse du terrain, la démocratie a mis à l’ordre du jour une sorte de surenchère en fait de paradoxes ; les candidats se sont laissé conduire par les exigences de la foule, et n’ont guère été appréciés qu’en proportion de leur vigueur déclamatoire ; l’opinion modérée n’a pu se faire entendre, ou bien a été obligée de forcer sa voix. Paris ignore les deux premières vertus de la vie politique, la patience et l’oubli. La politique du patriarche Jacob, qui voulait que la marche de toute sa tribu se réglât sur le pas des agneaux nouveau-nés, n’est pas du tout son fait.\par
En général, l’erreur du parti libéral français est de ne pas comprendre que toute construction politique doit avoir une base conservatrice. En Angleterre, le gouvernement parlementaire n’a été possible qu’après l’exclusion du parti radical, exclusion qui s’est faite avec une sorte de frénésie de légitimité. Rien n’est assuré en politique jusqu’à ce qu’on ait amené les parties lourdes et solides, qui sont le lest de la nation, à servir le progrès. Le parti libéral de 1830 s’imagina trop facilement emporter son programme de vive force, en contrariant en face le parti légitimiste. L’abstention ou l’hostilité de ce parti est encore le grand malheur de la France. Retirée de la vie commune, l’aristocratie légitimiste refuse à la société ce qu’elle lui doit, un patronage, des modèles et des leçons de noble vie, de belles images de sérieux. La vulgarité, le défaut d’éducation de la France, l’ignorance de l’art de vivre, l’ennui, le manque de respect, la parcimonie puérile de la vie provinciale, viennent de ce que les personnes qui devraient au pays les types de gentilshommes remplissant les devoirs publics avec une autorité reconnue de tous désertent la société générale, se renferment de plus en plus dans une vie solitaire et fermée. Le parti légitimiste est en un sens l’assise indispensable de toute fondation politique parmi nous ; même lès États-Unis possèdent à leur manière cette base essentielle de toute société dans leurs souvenirs religieux, héroïques à leur manière, et dans cette classe de citoyens moraux, fiers, graves, pesants, qui sont les pierres avec lesquelles on bâtit l’édifice de l’État. Le reste n’est que sable ; on n’en fait rien de durable, quelque esprit et même quelque chaleur de cœur qu’on y mette d’ailleurs.\par
Ce parti provincial, qui prend de jour en jour conscience de sa force, que pense-t-il ? que veut-il ? Jamais état d’opinion ne fut plus évident. Ce parti est libéral, non révolutionnaire, constitutionnel, non républicain ; il veut le contrôle du pouvoir, non sa destruction, la fin du gouvernement personnel, non le renversement de la dynastie. Je ne doute pas que, si, il y a huit mois, le gouvernement eût nettement pris son parti, renoncé aux candidatures officielles, au morcellement artificiel des circonscriptions, et laissé les élections se faire spontanément par le pays, le scrutin n’eût envoyé une Chambre décidément imbue de ces principes, et qui, étant considérée par le pays comme une représentation de sa volonté, aurait eu assez de force pour traverser les circonstances les plus difficiles. On aura un jour autant de peine à comprendre que l’empereur Napoléon III n’ait pas saisi ce moyen pour obtenir une seconde signature du pays à son contrat de mariage et pour partager avec lui la responsabilité d’un obscur avenir, qu’on en éprouve à comprendre que Louis-Philippe n’ait pas vu dans l’adjonction des capacités une manière d’élargir les bases de sa dynastie. La province en effet prend les élections beaucoup plus au sérieux que Paris. N’ayant de vie politique qu’une fois tous les six ans, elle prête aux élections une importance que Paris, avec sa perpétuelle légèreté, ne leur accorde pas. Paris, préoccupé de sa protestation radicale, voit dans les élections non un choix de graves délégués, mais une occasion de manifestations ironiques. La province ne comprend pas ces finesses ; son député est vraiment son mandataire, et elle y tient. Une Chambre élue librement et sans l’intervention de l’administration eût-elle été dangereuse pour la dynastie ? L’opposition radicale y eût-elle été représentée par un nombre plus considérable de députés ? Je crois juste tout le contraire. Dans un grand nombre de cas, l’élection des candidats hostiles ou même injurieux a été une façon de protester contre le candidat officiel ou complaisant. La candidature officielle trouble complètement l’opération électorale et en altère la sincérité, non seulement par la pression directe que l’administration exerce en sa faveur, mais surtout par la fausse situation où elle met l’électeur indépendant. Pour celui-ci, en effet, il ne s’agit plus de choisir le candidat qui représente le mieux son opinion, ou qu’il croit le plus capable de rendre des services au pays ; il s’agit d’écarter à tout prix le candidat officiel. Dès lors, plus de nuances, plus de préférences. Les opinions extrêmes trouvant une faveur assurée dans la foule, sur laquelle les assertions tranchées, les déclamations brayantes, ont plus de force que les opinions moyennes, le parti démocratique d’ailleurs ayant une organisation que n’a aucun autre parti et disposant d’un vrai fanatisme, les libéraux suivent le torrent, et adoptent malgré leurs répugnances le candidat radical. C’est une erreur fort répandue en France de croire qu’en demandant plus on obtient moins, et que l’opposition radicale est l’instrument du progrès, la force d’impulsion du gouvernement ; cela est vrai de l’opposition modérée, mais non de l’opinion radicale, laquelle est un obstacle au progrès, un empêchement aux concessions, par la terreur qu’elle inspire et les mesures de répression qu’elle amène.\par
Plus que jamais l’effort de la politique doit être non pas de résoudre les questions, mais d’attendre qu’elles s’usent. La vie des nations, comme celle des individus, est un compromis entre des contradictions. De combien de choses il faut dire qu’on ne peut vivre ni avec elles ni sans elles, et pourtant l’on vit toujours ! Le prince Napoléon disait, il y a quelques jours, avec esprit, à ceux qui veulent ajourner la liberté jusqu’à ce qu’il n’y ait plus en France ni dynasties rivales ni parti révolutionnaire : « Vous attendrez longtemps. » L’histoire ne blâmera pas la politique de ceux qui, dans un tel état de choses, se seront résignés à vivre d’expédients. Supposez qu’un membre de la branche aînée ou de la branche cadette de Bourbon règne un jour sur la France, ce ne sera point parce que la majorité de la France se sera faite légitimiste ou orléaniste, c’est parce que la {\itshape roue de fortune} aura ramené des circonstances où tel membre de la maison de Bourbon se sera trouvé l’utilité du moment. La France a si complètement laissé mourir en elle l’attachement dynastique, que même la légitimité n’y rentrerait que par aventure, à titre transitoire. Le positivisme contemporain a tellement supprimé toute métaphysique, qu’une idée des plus étroites tend à se répandre : c’est qu’un suffrage populaire a d’autant plus de force qu’il est plus récent, si bien qu’au bout d’une quinzaine d’années on fait cet étrange raisonnement : « La génération qui avait voté tel plébiscite est morte en partie, le suffrage a perdu sa valeur et a besoin d’être renouvelé. » C’est le contraire de l’idée du moyen âge, selon laquelle un pacte valait d’autant plus qu’il était plus ancien. C’est en un sens la négation du principe national, car le principe national, comme la religion, suppose des pactes indépendants de la volonté des individus, des pactes transmis et reçus de père en fils comme un héritage. En refusant à la nation le pouvoir d’engager l’avenir, on réduit tout à des contrats viagers, que dis-je ? passagers ; les exaltés, je crois, les voudraient même annuels,en attendant ce qu’ils appellent le gouvernement direct, état où la volonté nationale ne serait plus que le caprice de chaque heure. Que devient avec de pareilles conceptions politiques l’intégrité de la nation ? Comment nier le droit à la succession quand on réduit tout au fait matériel de la volonté actuelle des citoyens ? La vérité est qu’une nation est autre chose que la collection des unités qui la composent, qu’elle ne saurait dépendre d’un vote, qu’elle est à sa manière une idée, une chose abstraite, supérieure aux volontés particulières. Le principe du gouvernement ne saurait non plus être réduit à une simple consultation du suffrage universel, c’est-à-dire à constater et exécuter ce que le plus grand nombre regarde comme son intérêt. Cette conception matérialiste renferme au fond un appel à la lutte ; en se proclamant {\itshape ultima ratio}, le suffrage universel part de cette idée que le plus grand nombre est un indice de force, en sorte que, si la minorité ne pliait pas devant l’opinion de la majorité, elle aurait toute chance d’être vaincue. Mais ce raisonnement n’est pas exact, car la minorité peut être plus énergique et plus versée dans le maniement des armes que la majorité. « Nous sommes vingt, vous êtes un, dit le suffrage universel ; cédez, ou nous vous forçons ! — Vous êtes vingt, mais j’ai raison, et à moi seul je peux vous forcer, cédez, » dira l’homme armé.\par
{\itshape Fata viam invenient} ! Heureux qui peut, comme Boèce, sur les ruines d’un monde, écrire sa {\itshape Consolation de la philosophie.} L’avenir de la France est un mystère qui déjoue toute sagacité. Certes, d’autres pays agitent de graves problèmes : l’Angleterre, avec un calme qu’on ne peut assez admirer, résout des questions hardies qui chez nous passent pour le domaine des seuls utopistes ; mais partout le débat est circonscrit, partout il y a une arène limitée, des lois du combat, des hérauts et des juges. Chez nous, c’est la constitution même, la forme et jusqu’a un certain point l’existence de la société qui sont perpétuellement en question. Un pays peut-il résister à un tel régime ? Voilà ce qu’on se demande avec inquiétude. On se rassure en songeant qu’une grande nation est, comme le corps humain, une machine admirablement pondérée et équilibrée, qu’elle se crée les organes dont elle a besoin, et que, si elle les a perdus, elle se les redonne. Il se peut que, dans notre ardeur révolutionnaire, nous ayons poussé trop loin les amputations, qu’en croyant ne retrancher que des superfluités maladives, nous ayons touché à quelque organe essentiel de la vie, si bien que l’obstination du malade à ne pas se bien porter tienne à quelque grosse lésion faite par nous dans ses entrailles. C’est une raison pour y mettre désormais beaucoup de précautions et pour laisser ce corps, robuste après tout, quoique profondément atteint, réparer ses blessures intérieures et revenir aux conditions normales de la vie.\par
Hâtons-nous de le dire, d’ailleurs : des défauts aussi brillants que ceux de la France sont à leur manière des qualités. La France n’a pas perdu le sceptre de l’esprit, du goût, de l’art délicat, de l’atticisme ; longtemps encore, elle fixera l’attention de l’humanité civilisée, et posera l’enjeu sur lequel le public européen engagera ses paris. Les affaires de la France sont de telle nature, qu’elles divisent et passionnent les étrangers autant et souvent plus que les affaires de leur propre pays. Le grand inconvénient de son état politique, c’est l’imprévu ; mais l’imprévu est à double face : à côté des mauvaises chances, il y a les bonnes, et nous ne serions nullement surpris qu’après de déplorables aventures, la France ne traversât des années d’un singulier éclat. Si, lasse enfin d’étonner le monde, elle voulait prendre son parti d’une sorte d’apaisement politique, quelle ample et glorieuse revanche elle pourrait prendre dans les voies de l’activité privée ! Comme elle saurait rivaliser avec l’Angleterre dans la conquête pacifique du globe et dans l’assujettissement de toutes les races inférieures à notre civilisation ! La France peut tout, excepté être médiocre. Ce qu’elle souffre, en somme, elle le souffre pour avoir trop osé contre les dieux. Quels que soient les malheurs que l’avenir lui réserve et dût son sort exciter un jour la pitié du monde, le monde n’oubliera point qu’elle fit d’audacieuses expériences dont tous profitent, qu’elle aima la justice jusqu’à la folie, et que son crime) si elle en commit, fut d’avoir admis avec une généreuse imprudence la possibilité d’un idéal que les misères de l’humanité ne comportent pas.\par


\begin{raggedleft}FIN\end{raggedleft}
 


% at least one empty page at end (for booklet couv)
\ifbooklet
  \pagestyle{empty}
  \clearpage
  % 2 empty pages maybe needed for 4e cover
  \ifnum\modulo{\value{page}}{4}=0 \hbox{}\newpage\hbox{}\newpage\fi
  \ifnum\modulo{\value{page}}{4}=1 \hbox{}\newpage\hbox{}\newpage\fi


  \hbox{}\newpage
  \ifodd\value{page}\hbox{}\newpage\fi
  {\centering\color{rubric}\bfseries\noindent\large
    Hurlus ? Qu’est-ce.\par
    \bigskip
  }
  \noindent Des bouquinistes électroniques, pour du texte libre à participation libre,
  téléchargeable gratuitement sur \href{https://hurlus.fr}{\dotuline{hurlus.fr}}.\par
  \bigskip
  \noindent Cette brochure a été produite par des éditeurs bénévoles.
  Elle n’est pas faîte pour être possédée, mais pour être lue, et puis donnée.
  Que circule le texte !
  En page de garde, on peut ajouter une date, un lieu, un nom ; pour suivre le voyage des idées.
  \par

  Ce texte a été choisi parce qu’une personne l’a aimé,
  ou haï, elle a en tous cas pensé qu’il partipait à la formation de notre présent ;
  sans le souci de plaire, vendre, ou militer pour une cause.
  \par

  L’édition électronique est soigneuse, tant sur la technique
  que sur l’établissement du texte ; mais sans aucune prétention scolaire, au contraire.
  Le but est de s’adresser à tous, sans distinction de science ou de diplôme.
  Au plus direct ! (possible)
  \par

  Cet exemplaire en papier a été tiré sur une imprimante personnelle
   ou une photocopieuse. Tout le monde peut le faire.
  Il suffit de
  télécharger un fichier sur \href{https://hurlus.fr}{\dotuline{hurlus.fr}},
  d’imprimer, et agrafer ; puis de lire et donner.\par

  \bigskip

  \noindent PS : Les hurlus furent aussi des rebelles protestants qui cassaient les statues dans les églises catholiques. En 1566 démarra la révolte des gueux dans le pays de Lille. L’insurrection enflamma la région jusqu’à Anvers où les gueux de mer bloquèrent les bateaux espagnols.
  Ce fut une rare guerre de libération dont naquit un pays toujours libre : les Pays-Bas.
  En plat pays francophone, par contre, restèrent des bandes de huguenots, les hurlus, progressivement réprimés par la très catholique Espagne.
  Cette mémoire d’une défaite est éteinte, rallumons-la. Sortons les livres du culte universitaire, cherchons les idoles de l’époque, pour les briser.
\fi

\ifdev % autotext in dev mode
\fontname\font — \textsc{Les règles du jeu}\par
(\hyperref[utopie]{\underline{Lien}})\par
\noindent \initialiv{A}{lors là}\blindtext\par
\noindent \initialiv{À}{ la bonheur des dames}\blindtext\par
\noindent \initialiv{É}{tonnez-le}\blindtext\par
\noindent \initialiv{Q}{ualitativement}\blindtext\par
\noindent \initialiv{V}{aloriser}\blindtext\par
\Blindtext
\phantomsection
\label{utopie}
\Blinddocument
\fi
\end{document}
