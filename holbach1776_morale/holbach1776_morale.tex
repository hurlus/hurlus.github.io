%%%%%%%%%%%%%%%%%%%%%%%%%%%%%%%%%
% LaTeX model https://hurlus.fr %
%%%%%%%%%%%%%%%%%%%%%%%%%%%%%%%%%

% Needed before document class
\RequirePackage{pdftexcmds} % needed for tests expressions
\RequirePackage{fix-cm} % correct units

% Define mode
\def\mode{a4}

\newif\ifaiv % a4
\newif\ifav % a5
\newif\ifbooklet % booklet
\newif\ifcover % cover for booklet

\ifnum \strcmp{\mode}{cover}=0
  \covertrue
\else\ifnum \strcmp{\mode}{booklet}=0
  \booklettrue
\else\ifnum \strcmp{\mode}{a5}=0
  \avtrue
\else
  \aivtrue
\fi\fi\fi

\ifbooklet % do not enclose with {}
  \documentclass[french,twoside]{book} % ,notitlepage
  \usepackage[%
    papersize={105mm, 297mm},
    inner=12mm,
    outer=12mm,
    top=20mm,
    bottom=15mm,
    marginparsep=0pt,
  ]{geometry}
  \usepackage[fontsize=9.5pt]{scrextend} % for Roboto
\else\ifav
  \documentclass[french,twoside]{book} % ,notitlepage
  \usepackage[%
    a5paper,
    inner=25mm,
    outer=15mm,
    top=15mm,
    bottom=15mm,
    marginparsep=0pt,
  ]{geometry}
  \usepackage[fontsize=12pt]{scrextend}
\else% A4 2 cols
  \documentclass[twocolumn]{report}
  \usepackage[%
    a4paper,
    inner=15mm,
    outer=10mm,
    top=25mm,
    bottom=18mm,
    marginparsep=0pt,
  ]{geometry}
  \setlength{\columnsep}{20mm}
  \usepackage[fontsize=9.5pt]{scrextend}
\fi\fi

%%%%%%%%%%%%%%
% Alignments %
%%%%%%%%%%%%%%
% before teinte macros

\setlength{\arrayrulewidth}{0.2pt}
\setlength{\columnseprule}{\arrayrulewidth} % twocol
\setlength{\parskip}{0pt} % classical para with no margin
\setlength{\parindent}{1.5em}

%%%%%%%%%%
% Colors %
%%%%%%%%%%
% before Teinte macros

\usepackage[dvipsnames]{xcolor}
\definecolor{rubric}{HTML}{800000} % the tonic 0c71c3
\def\columnseprulecolor{\color{rubric}}
\colorlet{borderline}{rubric!30!} % definecolor need exact code
\definecolor{shadecolor}{gray}{0.95}
\definecolor{bghi}{gray}{0.5}

%%%%%%%%%%%%%%%%%
% Teinte macros %
%%%%%%%%%%%%%%%%%
%%%%%%%%%%%%%%%%%%%%%%%%%%%%%%%%%%%%%%%%%%%%%%%%%%%
% <TEI> generic (LaTeX names generated by Teinte) %
%%%%%%%%%%%%%%%%%%%%%%%%%%%%%%%%%%%%%%%%%%%%%%%%%%%
% This template is inserted in a specific design
% It is XeLaTeX and otf fonts

\makeatletter % <@@@


\usepackage{blindtext} % generate text for testing
\usepackage[strict]{changepage} % for modulo 4
\usepackage{contour} % rounding words
\usepackage[nodayofweek]{datetime}
% \usepackage{DejaVuSans} % seems buggy for sffont font for symbols
\usepackage{enumitem} % <list>
\usepackage{etoolbox} % patch commands
\usepackage{fancyvrb}
\usepackage{fancyhdr}
\usepackage{float}
\usepackage{fontspec} % XeLaTeX mandatory for fonts
\usepackage{footnote} % used to capture notes in minipage (ex: quote)
\usepackage{framed} % bordering correct with footnote hack
\usepackage{graphicx}
\usepackage{lettrine} % drop caps
\usepackage{lipsum} % generate text for testing
\usepackage[framemethod=tikz,]{mdframed} % maybe used for frame with footnotes inside
\usepackage{pdftexcmds} % needed for tests expressions
\usepackage{polyglossia} % non-break space french punct, bug Warning: "Failed to patch part"
\usepackage[%
  indentfirst=false,
  vskip=1em,
  noorphanfirst=true,
  noorphanafter=true,
  leftmargin=\parindent,
  rightmargin=0pt,
]{quoting}
\usepackage{ragged2e}
\usepackage{setspace} % \setstretch for <quote>
\usepackage{tabularx} % <table>
\usepackage[explicit]{titlesec} % wear titles, !NO implicit
\usepackage{tikz} % ornaments
\usepackage{tocloft} % styling tocs
\usepackage[fit]{truncate} % used im runing titles
\usepackage{unicode-math}
\usepackage[normalem]{ulem} % breakable \uline, normalem is absolutely necessary to keep \emph
\usepackage{verse} % <l>
\usepackage{xcolor} % named colors
\usepackage{xparse} % @ifundefined
\XeTeXdefaultencoding "iso-8859-1" % bad encoding of xstring
\usepackage{xstring} % string tests
\XeTeXdefaultencoding "utf-8"
\PassOptionsToPackage{hyphens}{url} % before hyperref, which load url package

% TOTEST
% \usepackage{hypcap} % links in caption ?
% \usepackage{marginnote}
% TESTED
% \usepackage{background} % doesn’t work with xetek
% \usepackage{bookmark} % prefers the hyperref hack \phantomsection
% \usepackage[color, leftbars]{changebar} % 2 cols doc, impossible to keep bar left
% \usepackage[utf8x]{inputenc} % inputenc package ignored with utf8 based engines
% \usepackage[sfdefault,medium]{inter} % no small caps
% \usepackage{firamath} % choose firasans instead, firamath unavailable in Ubuntu 21-04
% \usepackage{flushend} % bad for last notes, supposed flush end of columns
% \usepackage[stable]{footmisc} % BAD for complex notes https://texfaq.org/FAQ-ftnsect
% \usepackage{helvet} % not for XeLaTeX
% \usepackage{multicol} % not compatible with too much packages (longtable, framed, memoir…)
% \usepackage[default,oldstyle,scale=0.95]{opensans} % no small caps
% \usepackage{sectsty} % \chapterfont OBSOLETE
% \usepackage{soul} % \ul for underline, OBSOLETE with XeTeX
% \usepackage[breakable]{tcolorbox} % text styling gone, footnote hack not kept with breakable


% Metadata inserted by a program, from the TEI source, for title page and runing heads
\title{\textbf{ La Morale universelle ou Les Devoirs de l’homme fondés sur sa nature }}
\date{1776}
\author{Baron d’Holbach}
\def\elbibl{Baron d’Holbach. 1776. \emph{La Morale universelle ou Les Devoirs de l’homme fondés sur sa nature}}

% Default metas
\newcommand{\colorprovide}[2]{\@ifundefinedcolor{#1}{\colorlet{#1}{#2}}{}}
\colorprovide{rubric}{red}
\colorprovide{silver}{lightgray}
\@ifundefined{syms}{\newfontfamily\syms{DejaVu Sans}}{}
\newif\ifdev
\@ifundefined{elbibl}{% No meta defined, maybe dev mode
  \newcommand{\elbibl}{Titre court ?}
  \newcommand{\elbook}{Titre du livre source ?}
  \newcommand{\elabstract}{Résumé\par}
  \newcommand{\elurl}{http://oeuvres.github.io/elbook/2}
  \author{Éric Lœchien}
  \title{Un titre de test assez long pour vérifier le comportement d’une maquette}
  \date{1566}
  \devtrue
}{}
\let\eltitle\@title
\let\elauthor\@author
\let\eldate\@date


\defaultfontfeatures{
  % Mapping=tex-text, % no effect seen
  Scale=MatchLowercase,
  Ligatures={TeX,Common},
}


% generic typo commands
\newcommand{\astermono}{\medskip\centerline{\color{rubric}\large\selectfont{\syms ✻}}\medskip\par}%
\newcommand{\astertri}{\medskip\par\centerline{\color{rubric}\large\selectfont{\syms ✻\,✻\,✻}}\medskip\par}%
\newcommand{\asterism}{\bigskip\par\noindent\parbox{\linewidth}{\centering\color{rubric}\large{\syms ✻}\\{\syms ✻}\hskip 0.75em{\syms ✻}}\bigskip\par}%

% lists
\newlength{\listmod}
\setlength{\listmod}{\parindent}
\setlist{
  itemindent=!,
  listparindent=\listmod,
  labelsep=0.2\listmod,
  parsep=0pt,
  % topsep=0.2em, % default topsep is best
}
\setlist[itemize]{
  label=—,
  leftmargin=0pt,
  labelindent=1.2em,
  labelwidth=0pt,
}
\setlist[enumerate]{
  label={\bf\color{rubric}\arabic*.},
  labelindent=0.8\listmod,
  leftmargin=\listmod,
  labelwidth=0pt,
}
\newlist{listalpha}{enumerate}{1}
\setlist[listalpha]{
  label={\bf\color{rubric}\alph*.},
  leftmargin=0pt,
  labelindent=0.8\listmod,
  labelwidth=0pt,
}
\newcommand{\listhead}[1]{\hspace{-1\listmod}\emph{#1}}

\renewcommand{\hrulefill}{%
  \leavevmode\leaders\hrule height 0.2pt\hfill\kern\z@}

% General typo
\DeclareTextFontCommand{\textlarge}{\large}
\DeclareTextFontCommand{\textsmall}{\small}

% commands, inlines
\newcommand{\anchor}[1]{\Hy@raisedlink{\hypertarget{#1}{}}} % link to top of an anchor (not baseline)
\newcommand\abbr[1]{#1}
\newcommand{\autour}[1]{\tikz[baseline=(X.base)]\node [draw=rubric,thin,rectangle,inner sep=1.5pt, rounded corners=3pt] (X) {\color{rubric}#1};}
\newcommand\corr[1]{#1}
\newcommand{\ed}[1]{ {\color{silver}\sffamily\footnotesize (#1)} } % <milestone ed="1688"/>
\newcommand\expan[1]{#1}
\newcommand\foreign[1]{\emph{#1}}
\newcommand\gap[1]{#1}
\renewcommand{\LettrineFontHook}{\color{rubric}}
\newcommand{\initial}[2]{\lettrine[lines=2, loversize=0.3, lhang=0.3]{#1}{#2}}
\newcommand{\initialiv}[2]{%
  \let\oldLFH\LettrineFontHook
  % \renewcommand{\LettrineFontHook}{\color{rubric}\ttfamily}
  \IfSubStr{QJ’}{#1}{
    \lettrine[lines=4, lhang=0.2, loversize=-0.1, lraise=0.2]{\smash{#1}}{#2}
  }{\IfSubStr{É}{#1}{
    \lettrine[lines=4, lhang=0.2, loversize=-0, lraise=0]{\smash{#1}}{#2}
  }{\IfSubStr{ÀÂ}{#1}{
    \lettrine[lines=4, lhang=0.2, loversize=-0, lraise=0, slope=0.6em]{\smash{#1}}{#2}
  }{\IfSubStr{A}{#1}{
    \lettrine[lines=4, lhang=0.2, loversize=0.2, slope=0.6em]{\smash{#1}}{#2}
  }{\IfSubStr{V}{#1}{
    \lettrine[lines=4, lhang=0.2, loversize=0.2, slope=-0.5em]{\smash{#1}}{#2}
  }{
    \lettrine[lines=4, lhang=0.2, loversize=0.2]{\smash{#1}}{#2}
  }}}}}
  \let\LettrineFontHook\oldLFH
}
\newcommand{\labelchar}[1]{\textbf{\color{rubric} #1}}
\newcommand{\milestone}[1]{\autour{\footnotesize\color{rubric} #1}} % <milestone n="4"/>
\newcommand\name[1]{#1}
\newcommand\orig[1]{#1}
\newcommand\orgName[1]{#1}
\newcommand\persName[1]{#1}
\newcommand\placeName[1]{#1}
\newcommand{\pn}[1]{\IfSubStr{-—–¶}{#1}% <p n="3"/>
  {\noindent{\bfseries\color{rubric}   ¶  }}
  {{\footnotesize\autour{ #1}  }}}
\newcommand\reg{}
% \newcommand\ref{} % already defined
\newcommand\sic[1]{#1}
\newcommand\surname[1]{\textsc{#1}}
\newcommand\term[1]{\textbf{#1}}

\def\mednobreak{\ifdim\lastskip<\medskipamount
  \removelastskip\nopagebreak\medskip\fi}
\def\bignobreak{\ifdim\lastskip<\bigskipamount
  \removelastskip\nopagebreak\bigskip\fi}

% commands, blocks
\newcommand{\byline}[1]{\bigskip{\RaggedLeft{#1}\par}\bigskip}
\newcommand{\bibl}[1]{{\RaggedLeft{#1}\par\bigskip}}
\newcommand{\biblitem}[1]{{\noindent\hangindent=\parindent   #1\par}}
\newcommand{\dateline}[1]{\medskip{\RaggedLeft{#1}\par}\bigskip}
\newcommand{\labelblock}[1]{\medbreak{\noindent\color{rubric}\bfseries #1}\par\mednobreak}
\newcommand{\salute}[1]{\bigbreak{#1}\par\medbreak}
\newcommand{\signed}[1]{\bigbreak\filbreak{\raggedleft #1\par}\medskip}

% environments for blocks (some may become commands)
\newenvironment{borderbox}{}{} % framing content
\newenvironment{citbibl}{\ifvmode\hfill\fi}{\ifvmode\par\fi }
\newenvironment{docAuthor}{\ifvmode\vskip4pt\fontsize{16pt}{18pt}\selectfont\fi\itshape}{\ifvmode\par\fi }
\newenvironment{docDate}{}{\ifvmode\par\fi }
\newenvironment{docImprint}{\vskip6pt}{\ifvmode\par\fi }
\newenvironment{docTitle}{\vskip6pt\bfseries\fontsize{18pt}{22pt}\selectfont}{\par }
\newenvironment{msHead}{\vskip6pt}{\par}
\newenvironment{msItem}{\vskip6pt}{\par}
\newenvironment{titlePart}{}{\par }


% environments for block containers
\newenvironment{argument}{\itshape\parindent0pt}{\vskip1.5em}
\newenvironment{biblfree}{}{\ifvmode\par\fi }
\newenvironment{bibitemlist}[1]{%
  \list{\@biblabel{\@arabic\c@enumiv}}%
  {%
    \settowidth\labelwidth{\@biblabel{#1}}%
    \leftmargin\labelwidth
    \advance\leftmargin\labelsep
    \@openbib@code
    \usecounter{enumiv}%
    \let\p@enumiv\@empty
    \renewcommand\theenumiv{\@arabic\c@enumiv}%
  }
  \sloppy
  \clubpenalty4000
  \@clubpenalty \clubpenalty
  \widowpenalty4000%
  \sfcode`\.\@m
}%
{\def\@noitemerr
  {\@latex@warning{Empty `bibitemlist' environment}}%
\endlist}
\newenvironment{quoteblock}% may be used for ornaments
  {\begin{quoting}}
  {\end{quoting}}

% table () is preceded and finished by custom command
\newcommand{\tableopen}[1]{%
  \ifnum\strcmp{#1}{wide}=0{%
    \begin{center}
  }
  \else\ifnum\strcmp{#1}{long}=0{%
    \begin{center}
  }
  \else{%
    \begin{center}
  }
  \fi\fi
}
\newcommand{\tableclose}[1]{%
  \ifnum\strcmp{#1}{wide}=0{%
    \end{center}
  }
  \else\ifnum\strcmp{#1}{long}=0{%
    \end{center}
  }
  \else{%
    \end{center}
  }
  \fi\fi
}


% text structure
\newcommand\chapteropen{} % before chapter title
\newcommand\chaptercont{} % after title, argument, epigraph…
\newcommand\chapterclose{} % maybe useful for multicol settings
\setcounter{secnumdepth}{-2} % no counters for hierarchy titles
\setcounter{tocdepth}{5} % deep toc
\markright{\@title} % ???
\markboth{\@title}{\@author} % ???
\renewcommand\tableofcontents{\@starttoc{toc}}
% toclof format
% \renewcommand{\@tocrmarg}{0.1em} % Useless command?
% \renewcommand{\@pnumwidth}{0.5em} % {1.75em}
\renewcommand{\@cftmaketoctitle}{}
\setlength{\cftbeforesecskip}{\z@ \@plus.2\p@}
\renewcommand{\cftchapfont}{}
\renewcommand{\cftchapdotsep}{\cftdotsep}
\renewcommand{\cftchapleader}{\normalfont\cftdotfill{\cftchapdotsep}}
\renewcommand{\cftchappagefont}{\bfseries}
\setlength{\cftbeforechapskip}{0em \@plus\p@}
% \renewcommand{\cftsecfont}{\small\relax}
\renewcommand{\cftsecpagefont}{\normalfont}
% \renewcommand{\cftsubsecfont}{\small\relax}
\renewcommand{\cftsecdotsep}{\cftdotsep}
\renewcommand{\cftsecpagefont}{\normalfont}
\renewcommand{\cftsecleader}{\normalfont\cftdotfill{\cftsecdotsep}}
\setlength{\cftsecindent}{1em}
\setlength{\cftsubsecindent}{2em}
\setlength{\cftsubsubsecindent}{3em}
\setlength{\cftchapnumwidth}{1em}
\setlength{\cftsecnumwidth}{1em}
\setlength{\cftsubsecnumwidth}{1em}
\setlength{\cftsubsubsecnumwidth}{1em}

% footnotes
\newif\ifheading
\newcommand*{\fnmarkscale}{\ifheading 0.70 \else 1 \fi}
\renewcommand\footnoterule{\vspace*{0.3cm}\hrule height \arrayrulewidth width 3cm \vspace*{0.3cm}}
\setlength\footnotesep{1.5\footnotesep} % footnote separator
\renewcommand\@makefntext[1]{\parindent 1.5em \noindent \hb@xt@1.8em{\hss{\normalfont\@thefnmark . }}#1} % no superscipt in foot
\patchcmd{\@footnotetext}{\footnotesize}{\footnotesize\sffamily}{}{} % before scrextend, hyperref


%   see https://tex.stackexchange.com/a/34449/5049
\def\truncdiv#1#2{((#1-(#2-1)/2)/#2)}
\def\moduloop#1#2{(#1-\truncdiv{#1}{#2}*#2)}
\def\modulo#1#2{\number\numexpr\moduloop{#1}{#2}\relax}

% orphans and widows
\clubpenalty=9996
\widowpenalty=9999
\brokenpenalty=4991
\predisplaypenalty=10000
\postdisplaypenalty=1549
\displaywidowpenalty=1602
\hyphenpenalty=400
% Copied from Rahtz but not understood
\def\@pnumwidth{1.55em}
\def\@tocrmarg {2.55em}
\def\@dotsep{4.5}
\emergencystretch 3em
\hbadness=4000
\pretolerance=750
\tolerance=2000
\vbadness=4000
\def\Gin@extensions{.pdf,.png,.jpg,.mps,.tif}
% \renewcommand{\@cite}[1]{#1} % biblio

\usepackage{hyperref} % supposed to be the last one, :o) except for the ones to follow
\urlstyle{same} % after hyperref
\hypersetup{
  % pdftex, % no effect
  pdftitle={\elbibl},
  % pdfauthor={Your name here},
  % pdfsubject={Your subject here},
  % pdfkeywords={keyword1, keyword2},
  bookmarksnumbered=true,
  bookmarksopen=true,
  bookmarksopenlevel=1,
  pdfstartview=Fit,
  breaklinks=true, % avoid long links
  pdfpagemode=UseOutlines,    % pdf toc
  hyperfootnotes=true,
  colorlinks=false,
  pdfborder=0 0 0,
  % pdfpagelayout=TwoPageRight,
  % linktocpage=true, % NO, toc, link only on page no
}

\makeatother % /@@@>
%%%%%%%%%%%%%%
% </TEI> end %
%%%%%%%%%%%%%%


%%%%%%%%%%%%%
% footnotes %
%%%%%%%%%%%%%
\renewcommand{\thefootnote}{\bfseries\textcolor{rubric}{\arabic{footnote}}} % color for footnote marks

%%%%%%%%%
% Fonts %
%%%%%%%%%
\usepackage[]{roboto} % SmallCaps, Regular is a bit bold
% \linespread{0.90} % too compact, keep font natural
\newfontfamily\fontrun[]{Roboto Condensed Light} % condensed runing heads
\ifav
  \setmainfont[
    ItalicFont={Roboto Light Italic},
  ]{Roboto}
\else\ifbooklet
  \setmainfont[
    ItalicFont={Roboto Light Italic},
  ]{Roboto}
\else
\setmainfont[
  ItalicFont={Roboto Italic},
]{Roboto Light}
\fi\fi
\renewcommand{\LettrineFontHook}{\bfseries\color{rubric}}
% \renewenvironment{labelblock}{\begin{center}\bfseries\color{rubric}}{\end{center}}

%%%%%%%%
% MISC %
%%%%%%%%

\setdefaultlanguage[frenchpart=false]{french} % bug on part


\newenvironment{quotebar}{%
    \def\FrameCommand{{\color{rubric!10!}\vrule width 0.5em} \hspace{0.9em}}%
    \def\OuterFrameSep{\itemsep} % séparateur vertical
    \MakeFramed {\advance\hsize-\width \FrameRestore}
  }%
  {%
    \endMakeFramed
  }
\renewenvironment{quoteblock}% may be used for ornaments
  {%
    \savenotes
    \setstretch{0.9}
    \normalfont
    \begin{quotebar}
  }
  {%
    \end{quotebar}
    \spewnotes
  }


\renewcommand{\headrulewidth}{\arrayrulewidth}
\renewcommand{\headrule}{{\color{rubric}\hrule}}

% delicate tuning, image has produce line-height problems in title on 2 lines
\titleformat{name=\chapter} % command
  [display] % shape
  {\vspace{1.5em}\centering} % format
  {} % label
  {0pt} % separator between n
  {}
[{\color{rubric}\huge\textbf{#1}}\bigskip] % after code
% \titlespacing{command}{left spacing}{before spacing}{after spacing}[right]
\titlespacing*{\chapter}{0pt}{-2em}{0pt}[0pt]

\titleformat{name=\section}
  [block]{}{}{}{}
  [\vbox{\color{rubric}\large\raggedleft\textbf{#1}}]
\titlespacing{\section}{0pt}{0pt plus 4pt minus 2pt}{\baselineskip}

\titleformat{name=\subsection}
  [block]
  {}
  {} % \thesection
  {} % separator \arrayrulewidth
  {}
[\vbox{\large\textbf{#1}}]
% \titlespacing{\subsection}{0pt}{0pt plus 4pt minus 2pt}{\baselineskip}

\ifaiv
  \fancypagestyle{main}{%
    \fancyhf{}
    \setlength{\headheight}{1.5em}
    \fancyhead{} % reset head
    \fancyfoot{} % reset foot
    \fancyhead[L]{\truncate{0.45\headwidth}{\fontrun\elbibl}} % book ref
    \fancyhead[R]{\truncate{0.45\headwidth}{ \fontrun\nouppercase\leftmark}} % Chapter title
    \fancyhead[C]{\thepage}
  }
  \fancypagestyle{plain}{% apply to chapter
    \fancyhf{}% clear all header and footer fields
    \setlength{\headheight}{1.5em}
    \fancyhead[L]{\truncate{0.9\headwidth}{\fontrun\elbibl}}
    \fancyhead[R]{\thepage}
  }
\else
  \fancypagestyle{main}{%
    \fancyhf{}
    \setlength{\headheight}{1.5em}
    \fancyhead{} % reset head
    \fancyfoot{} % reset foot
    \fancyhead[RE]{\truncate{0.9\headwidth}{\fontrun\elbibl}} % book ref
    \fancyhead[LO]{\truncate{0.9\headwidth}{\fontrun\nouppercase\leftmark}} % Chapter title, \nouppercase needed
    \fancyhead[RO,LE]{\thepage}
  }
  \fancypagestyle{plain}{% apply to chapter
    \fancyhf{}% clear all header and footer fields
    \setlength{\headheight}{1.5em}
    \fancyhead[L]{\truncate{0.9\headwidth}{\fontrun\elbibl}}
    \fancyhead[R]{\thepage}
  }
\fi

\ifav % a5 only
  \titleclass{\section}{top}
\fi

\newcommand\chapo{{%
  \vspace*{-3em}
  \centering % no vskip ()
  {\Large\addfontfeature{LetterSpace=25}\bfseries{\elauthor}}\par
  \smallskip
  {\large\eldate}\par
  \bigskip
  {\Large\selectfont{\eltitle}}\par
  \bigskip
  {\color{rubric}\hline\par}
  \bigskip
  {\Large TEXTE LIBRE À PARTICPATION LIBRE\par}
  \centerline{\small\color{rubric} {hurlus.fr, tiré le \today}}\par
  \bigskip
}}

\newcommand\cover{{%
  \thispagestyle{empty}
  \centering
  {\LARGE\bfseries{\elauthor}}\par
  \bigskip
  {\Large\eldate}\par
  \bigskip
  \bigskip
  {\LARGE\selectfont{\eltitle}}\par
  \vfill\null
  {\color{rubric}\setlength{\arrayrulewidth}{2pt}\hline\par}
  \vfill\null
  {\Large TEXTE LIBRE À PARTICPATION LIBRE\par}
  \centerline{{\href{https://hurlus.fr}{\dotuline{hurlus.fr}}, tiré le \today}}\par
}}

\begin{document}
\pagestyle{empty}
\ifbooklet{
  \cover\newpage
  \thispagestyle{empty}\hbox{}\newpage
  \cover\newpage\noindent Les voyages de la brochure\par
  \bigskip
  \begin{tabularx}{\textwidth}{l|X|X}
    \textbf{Date} & \textbf{Lieu}& \textbf{Nom/pseudo} \\ \hline
    \rule{0pt}{25cm} &  &   \\
  \end{tabularx}
  \newpage
  \addtocounter{page}{-4}
}\fi

\thispagestyle{empty}
\ifaiv
  \twocolumn[\chapo]
\else
  \chapo
\fi
{\it\elabstract}
\bigskip
\makeatletter\@starttoc{toc}\makeatother % toc without new page
\bigskip

\pagestyle{main} % after style

  \frontmatter \noindent « Natura duce utendum est : hanc ratio observat, hanc consultit : idem est ergo beate vivere, \& secundum naturam. »\par
Sénèque, {\itshape De Vita beata}, chap. VIII.
\mainmatter \section[{Préface}]{Préface}\renewcommand{\leftmark}{Préface}

\noindent Quoique depuis un grand nombre de siècles l’esprit humain se soit occupé de la morale, cette science, la plus digne d’intéresser les hommes, ne semble pas avoir fait tous les progrès que l’on avait lieu d’attendre. Ses principes sont encore sujets à des disputes, et les philosophes ont été de tout temps peu d’accord sur les fondements que l’on devait leur donner. Entre les mains de la plupart des sages de l’Antiquité, la philosophie morale, faite pour éclairer également la conduite de tous les hommes, est devenue communément abstraite et mystérieuse. Par une fatalité qui lui est commune avec toutes les connaissances humaines, elle négligea l’expérience et se laissa d’abord guider par l’enthousiasme et l’amour du merveilleux. De là toutes les hypothèses si variées de tant de philosophes anciens et modernes qui, bien loin d’éclaircir la morale et de la rendre populaire, n’ont fait que l’envelopper de ténèbres épaisses, au point que l’étude la plus importante pour l’homme lui devint presque inutile par le soin qu’on prit de la rendre impénétrable. Par une faiblesse commune presque à tous les premiers savants, ils donnèrent à leurs leçons un ton d’inspiration et de mystère, dans la vue de les rendre plus respectables au vulgaire étonné.\par
L’Antiquité ne nous montre aucun système de morale bien lié, elle ne nous offre dans les écrits de la plupart des philosophes que des mots vagues dépourvus de définitions exactes, de principes détachés et souvent contradictoires. Nous n’y trouvons qu’un petit nombre de maximes très belles et très vraies quelquefois, mais isolées et qui ne concourent point à former un ensemble, un corps de doctrine capable de servir de règle constante dans la conduite de la vie.\par
Pythagore, qui le premier prit le nom de {\itshape philosophe} ou d’{\itshape ami de la sagesse}, puisa ses connaissances mystérieuses chez les prêtres de l’Égypte, de l’Assyrie, de l’Hindoustan. Nous n’avons de lui que quelques préceptes obscurs, ou plutôt des énigmes, recueillis par ses disciples, dont il serait bien difficile de former un ensemble. Socrate, que l’on regarde comme le père de la morale, la fit, dit-on, descendre du ciel pour éclairer les hommes ; mais ses principes, tels qu’ils nous sont présentés par Xénophon et Platon ses disciples, quoique ornés des charmes d’une éloquence poétique, n’offrent à l’esprit que des notions embrouillées, des idées peu arrêtées, accompagnées des élans d’une imagination brillante peu capable de nous fournir une instruction réelle.\par
Le stoïcisme, par ses vertus fanatiques et farouches, ne rendit la vertu nullement attrayante pour les hommes {\itshape ;} les perfections impossibles qu’il exigea ne purent faire du sage qu’un être de raison. Toute morale qui prétendra tirer l’homme de la sphère, l’élever au-dessus de sa nature, qui lui dira de ne point sentir, d’être indifférent sur le plaisir et la douleur, de se rendre impassible à force de raisonnement, de cesser d’être un homme, pourra bien être admirée par ses enthousiastes mais ne conviendra jamais à des êtres que la Nature a fait sensibles et remplis de désirs. Les hommes admirent toujours une morale austère, ils révèrent ceux qui la prêchent, ils les regardent comme des hommes rares et divins, mais ils ne la pratiquent jamais.\par
Si la morale d’Épicure fut telle qu’elle nous fut présentée par ses adversaires, qui l’accusent d’avoir lâché la brise à toutes les passions, elle ne fut nullement propre à régler la conduite de l’homme. Mais si, comme ses partisans le soutiennent, cette morale invitait l’homme à la vertu, présentée sous les noms de {\itshape plaisir}, de {\itshape bien-être}, de {\itshape volupté}, elle est vraie, elle n’a rien à redouter des imputations de ses ennemis ; elle ne pêche que pour ne s’être pas suffisamment expliquée.\par
Quelle morale pouvait-on fonder sur les principes outrés et bizarres des cyniques, qui semblaient ne s’être proposé que de s’attirer les regards du vulgaire par leur impudence choquante et leur singularité ? La science des mœurs ne devait pas faire de grands progrès dans l’école d’un Pyrrhon et de ses sectateurs, dont le principe était de douter des vérités les mieux démontrées. Elle ne pouvait que s’obscurcir, devenir très incertaine et très vague dans Aristote, dont les disciples, à force de distinctions et de subtilités, paraissaient avoir formé le projet d’embrouiller les vérités les plus simples et les plus claires. Cependant, la doctrine de ces derniers philosophes, servant longtemps de guide à l’Europe, empêcha de découvrir les vrais principes de toute philosophie et tint l’esprit humain enchaîné sous le joug d’une autorité tyrannique que l’on fut obligé de révérer comme infaillible. Chez les scolastiques la morale ne fut qu’un jeu d’esprit, un amas de sophismes et de pièges dans lequel il fut presque impossible de démêler la vérité.\par
Ces réflexions, que tout confirme, peuvent nous faire voir ce que l’on doit penser du préjugé qui voudrait sans cesse nous mettre en adoration devant la sagesse ancienne, ainsi que de celui qui se persuade qu’en morale {\itshape tout est dit}. On trouvera que les anciens philosophes n’ont point eu des idées bien nettes sur les vrais principes de cette science ; s’ils les ont quelque fois aperçus, ils les ont souvent perdus de vue et n’en ont presque jamais tiré les conséquences les plus immédiates.\par
Quant à ceux qui pensent qu’il ne reste plus rien dire sur la morale, nous croyons pouvoir leur montrer que jusqu’ici l’on a fait que rassembler les matériaux propres à construire un édifice que les méditations rassemblées de hommes pourront un jour conduire à sa perfection. Les Anciens nous ont fourni une grande partie de ces matériaux ; quelques modernes y ont depuis amplement contribué. La postérité, profitant des lumières et des fautes de ses prédécesseurs, pourra mettre avec le temps la dernière main à ce grand ouvrage. Le fameux temple d’Éphèse fut construit aux dépens de tous les rois et peuples de l’Asie ; le temple de la sagesse doit s’élever par les travaux communs de tous les êtres pensants.\par
En général, on peut dire que les premiers efforts de la philosophie, faute de principes sûrs, ne produisirent que des erreurs entremêlées de quelques vérités. L’esprit subtil des Grecs les éloigna de la simplicité, leur imagination porta les choses à l’extrême ; la philosophie ne devint souvent pour eux que de la charlatanerie pure, que chacun fit valoir de son mieux. L’amour-propre de tout chef de secte lui fit croire qu’il avait seul rencontré la vérité, tandis que toutes les sectes s’en écartaient également par des routes différentes. Ces prétendus sages ne semblaient se proposer pour l’ordinaire que de se contredire, de se décrier, de se combattre, de s’embarrasser réciproquement par des sophismes et des chicanes interminables. La saine philosophie, sincèrement occupée de la recherche de ce qui est utile et vrai, ne doit point être outrée ni proposer des choses impraticables ou inintelligibles : elle doit se mettre en garde et contre l’enthousiasme, et contre une vanité puérile, et contre l’esprit de contradiction. Toujours de bonne foi avec elle-même, toujours calme, elle ne doit suivre que la raison éclairée par l’expérience, qui seule nous montre les objets tels qu’ils sont ; elle doit accepter la vérité de toutes les mains qui la présentent et rejeter l’erreur et le préjugé, de quelque autorité que l’on veuille les appuyer.\par
Les philosophes de l’Antiquité semblent encore avoir souvent à dessein enveloppé leur doctrine de nuages. La plupart d’entre eux, pour la rendre plus inaccessible au vulgaire, ont eu une {\itshape double doctrine}, l’une publique et l’autre particulière, qu’il est difficile de distinguer dans leurs écrits, surtout après qu’un grand nombre de siècles en a fait perdre la clef. La philosophie, pour être utile dans tous les âges et à tous les hommes, doit être franche et sincère ; celle qui n’est intelligible que pour un temps ou à quelques initiés devient une énigme inexplicable pour la postérité.\par
Ainsi, ne suivons pas en aveugle les idées des Anciens, n’adoptons leurs principes ou leurs opinions qu’autant que l’examen les montrera évidents, lumineux, conformes à la Nature, à l’expérience, à l’utilité constante des hommes de tous les siècles. Profitons avec reconnaissance d’une foule de maximes sages et vraies que les philosophes les plus célèbres de l’Antiquité nous ont souvent transmises avec une foule d’erreurs ; distinguons-les, s’il se peut, de celles que l’enthousiasme a produites. Suivons Socrate quand il nous recommande de {\itshape nous connaître nous-mêmes}, écoutons Pythagore et Platon quand ils nous donnent des préceptes intelligibles, recevons les conseils de Zénon quand nous les trouvons conformes à la nature de l’homme, doutons avec Pyrrhon des choses dont jusqu’ici les principes n’ont pas été suffisamment développés, employons la subtilité d’Aristote pour démêler le vrai, si souvent confondu avec le faux. Dès que l’erreur est manifeste, que l’autorité de ces noms respectés ne nous en impose plus.\par
En traitant de la morale, ne nous enfonçons point dans les abîmes d’une métaphysique subtile ou d’une dialectique tortueuse ; les règles des mœurs étant faites pour tous, doivent être simples, claires, démonstratives, à la portée de tous. Les principes sur lesquels nos devoirs se fondent doivent être si frappants et si généraux que chacun puisse s’en convaincre et en tirer les conséquences relatives à ses besoins et au rang qu’il occupe dans la société.\par
Des notions obscures, abstraites et compliquées, des autorités souvent suspectes, un fanatisme exalté ne peuvent éclairer ni guider sûrement. Pour que la morale soit efficace, il faut rendre raison à l’homme des préceptes qu’on lui donne, il faut lui faire sentir les motifs pressants qui doivent le porter à les suivre, il faut lui faire connaître en quoi la vertu consiste, il faut la lui faire aimer en la montrant comme la source du bonheur. L’enthousiasme et l’autorité, s’ils ont quelque utilité, ne sont bons qu’à gouverner quelques temps des peuples ignorants et sans expérience, dont l’esprit n’est point encore suffisamment exercé.\par
Étonner les hommes pour les persuader, dérouter l’esprit humain par des énigmes, l’éblouir par des merveilles, telle fut communément la méthode des premiers sages qui s’occupèrent de l’instruction et du gouvernement des nations grossières. Mais si ces premiers législateurs eurent recours au surnaturel pour les soumettre aux règles qu’ils voulurent leur prescrire, s’ils se servirent, pour les conduire, de l’enthousiasme, qui ne raisonne guère, et du merveilleux, qui fait plus d’impression sur le vulgaire que les meilleurs raisonnements, ces moyens ne sont plus de saison quand il s’agit de parler à des peuples moins sauvages et sortis de l’enfance. L’homme devenu plus raisonnable doit être conduit par la raison. Les philosophes doivent le rappeler à sa propre nature, la fonction des législateurs est de l’inviter et de l’obliger à la suivre.\par
Les moralistes modernes, très souvent entraînés par l’autorité des Anciens, ont trop fidèlement suivi leurs traces sans se mettre fort en peine de se frayer des routes nouvelles pour découvrir la vérité. La plupart d’entre eux, faute d’examiner l’homme avec assez d’attention, ne l’ont point vu tel qu’il est. Ils ont cru, comme quelques Anciens, qu’il recevait de la Nature des idées qu’ils ont appelées {\itshape innées}, à l’aide desquelles il jugeait sainement et du bien et du mal. Ils ont regardé la raison, la vertu, la justice, la bienveillance, la pitié, comme des qualités essentiellement inhérentes à la nature humaine. Selon eux, la Nature a gravé dans tous les cœurs les vérités primitives, l’amour du bien, la haine du mal moral, dont l’homme jugeait sainement à l’aide d’un {\itshape sens moral}, c’est-à-dire d’une qualité occulte, d’un certain {\itshape critérium} qu’il apporte en naissant et qui le met à portée de prononcer avec certitude sur le mérite ou le démérite des actions. En vain le profond Locke a-t-il prouvé que les idées {\itshape innées} n’étaient que des chimères : ces moralistes persistent dans leur préjugé. Ils veulent croire, ou persuader, que l’homme, même sans avoir senti le bien ou le mal qui résulte des actions, est capable de décider si elles sont bonnes ou mauvaises. Nous ferons voir, d’après ces philosophes plus éclairés, que l’homme ne possède en venant au monde que la faculté de sentir, et que sa façon de sentir est le vrai {\itshape critérium}, ou la seule règle de ses jugements, ou de ses sentiments moraux sur les actions ou sur les causes qui se font sentir à lui, vérité si palpable qu’il est bien surprenant qu’il y ait des hommes à qui l’on soit encore réduit à la prouver !\par
Enfin, nous ferons voir que les lois ou les règles que l’on suppose {\itshape écrites par la Nature dans tous les cœurs}, ne sont que des suites nécessaires de la façon dont les hommes sont conformés par la Nature et de la manière dont leurs dispositions ont été cultivées. Le vrai système de nos devoirs doit être celui qui résulte de notre propre nature convenablement modifiée.\par
D’autres, d’après Cudworth, ont fondé la morale sur des {\itshape règles, des convenances éternelles et immuables}, qu’ils ont supposé antérieures à l’homme et totalement indépendantes de lui. D’où l’on voit qu’ils n’ont fait que réaliser des abstractions pures, qu’ils ont supposé des modifications ou qualités antérieures aux êtres ou sujets susceptibles de les recevoir, et des rapports indépendants des êtres entre lesquels ils pussent subsister. Cependant, si la morale est la règle des hommes vivant en société, elle ne peut que coexister avec les hommes et se fonder sur les rapports qui s’établissent entre eux. Une morale antérieure à l’existence des hommes et de leurs rapports est une morale aérienne, une chimère véritable. Il ne peut y avoir ni règles, ni devoirs, ni rapports entre des êtres qui n’existent que dans les régions imaginaires.\par
Nous ne parlerons point ici de la morale religieuse, dont l’objet étant de conduire les hommes par des voies surnaturelles, ne reconnaît point dans sa marche les droits de la raison. Nous ne prétendons proposer dans cet ouvrage que les principes d’une morale humaine et sociale, convenable au monde où nous vivons, dans lequel la raison et l’expérience suffisent pour guider vers la félicité présente que se proposent des êtres vivants en société. Les motifs que cette morale expose sont purement humains, c’est-à-dire uniquement fondés sur la nature de l’homme telle qu’elle se montre à nos yeux, abstraction faite des opinions qui divisent le genre humain, auxquelles une morale faite également pour tous les habitants de la terre ne doit point s’arrêter. On est homme avant que d’avoir une religion, et quelque religion qu’on adopte, sa morale doit être la même que celle que la Nature prescrit à tous les hommes, sans quoi elle serait destructive pour la société.\par
Les philosophes, en effet, ont été et sont encore partagés sur la nature de l’homme, sur le principe de ses opérations et facultés, tant visibles que cachées. Les uns, et c’est le plus grand nombre, prétendent que ses pensées, ses volontés, ses actions ne doivent point être attribuées à son corps, qui n’est qu’un assemblement d’organes matériels, incapables de penser et d’agir s’ils n’étaient remués par une âme ou par un agent spirituel distingué de ce corps qui leur sert d’enveloppe ou d’instrument. D’autres, mais en plus petit nombre, rejettent l’existence de ce moteur invisible et croient que l’organisation humaine suffit pour opérer les actes, pour produire les pensées, les facultés, les mouvements dont l’homme est susceptible.\par
Nous ne nous arrêterons point à discuter ces sentiments divers : pour savoir ce que l’homme doit faire dans la société, il n’est pas besoin de remonter si haut. Ainsi, nous n’examinerons ni la cause secrète qui peut remuer le corps, ni les efforts invisibles dont ce corps est composé : nous laissons ces recherches à la métaphysique et à l’anatomie. Pour découvrir les principes de la morale, contentons-nous de savoir que l’homme agit, que sa façon d’agir est en général la même dans tous les individus de son espèce, nonobstant les nuances qui les différencient. La façon d’être et d’agir commune à tous les hommes est assez connue pour pouvoir en déduire avec certitude la manière dont ils doivent se conduire dans la route de la vie. L’homme est un être sensible ; à quelque cause que sa sensibilité soit due, cette qualité réside essentiellement en lui et suffit pour lui faire connaître et ce qu’il se doit à lui-même, et ce qu’il doit aux êtres avec lesquels son destin est de vivre sur la terre.\par
Les variétés presque infinies que l’on remarque entre les individus dont l’espèce humaine est composée, n’empêchent pas qu’une même morale ne leur convienne à tous ; ils s’accordent tous au fond et ce n’est que dans la forme qu’ils varient : tous désirent être heureux, mais ils ne peuvent l’être de la même façon. S’il se trouvait des hommes tellement conformés que les principes de la morale ne pussent leur convenir, cette morale n’en serait pas moins certaine {\itshape ;} il faudrait en conclure simplement qu’elle n’est pas faite pour des êtres constitués différemment de tous les autres. Il n’existe point de morale pour les monstres ou pour les insensés : la morale universelle n’est faite que pour des êtres susceptibles de raison et bien organisés. Dans ceux-ci la Nature ne varie point ; il ne s’agit que de la bien observer pour en déduire les règles invariables qu’ils doivent observer.\par
Ce n’est pas non plus ici le lieu d’examiner si l’homme est destiné à une autre vie, c’est-à-dire si son âme est faite pour survivre à la ruine de son corps ou si la mort anéantit l’homme tout entier : c’est à la métaphysique et à la théologie qu’il appartient de discuter ces questions, auxquelles nous ne prétendons toucher ici en aucune manière. La morale que nous présentons est la connaissance naturelle des devoirs de l’homme dans la vie de ce monde. Quelque sentiment que l’on adopte sur son âme et sur son sort à venir, soit que cette âme soit immortelle ou non, les devoirs de la vie sociale seront toujours les mêmes, et pour les démêler il suffira de savoir que l’homme est susceptible d’éprouver du plaisir et de la douleur, et qu’il vit avec des êtres qui sentent comme lui, dont il est obligé de mériter la bienveillance pour obtenir ce qui lui plaît et pour écarter ce qui peut lui déplaire.\par
Quelque spéculation qu’on adopte, à quelque degré que l’on porte le scepticisme et l’incrédulité, jamais, si l’on est de bonne foi, l’on ne pourra se faire illusion au point de douter de sa propre existence et de celle d’êtres qui nous ressemblent, dont nous sommes entourés, sur lesquels nos actions influent, et qui réagissent sur nous selon la manière dont ils sont affectés par nos propres actions. En un mot, on ne doutera jamais qu’il ne subsiste des rapports nécessaires entre les hommes vivants en société, et qu’ils ne contribuent à leur bien-être ou à leur malheur réciproque.\par
Si quelqu’un même adoptait le système de Berkeley — ce sceptique extravagant qui prétendait qu’il n’existait rien ne réel entre nous et que tous les objets que la Nature présente à l’homme ne sont que dans son imagination, dans son propre cerveau — cette hypothèse subtile et bizarre n’exclurait pas la morale si, comme ce philosophe le suppose, tout ce que nous voyons dans le monde n’est qu’une illusion, un rêve continuel. En suivant les préceptes de la morale, les hommes se procureront au moins des rêves suivis, agréables, utiles à leur repos, conformes à leur bien-être durant le temps de leur sommeil en ce monde, et les individus qui rêveront ne se troubleront point les uns les autres par des songes funestes.\par
{\itshape Je croirai}, dit un illustre moderne, {\itshape qu’il y a du vice et de la vertu comme il y a de la santé et des maladies}\footnote{M. de Voltaire, dans son {\itshape Homélie sur l’Athéisme}.}. Les notions primitives de la morale ne peuvent être aucunement contestées ; elles suffisent pour en déduire tous les devoirs de l’homme social et pour fixer la route qui doit le conduire au bonheur dans la vie présente, dans les différents états où son destin se place, dans les rapports divers qui s’établissent entre lui et les êtres de son espèce.\par
Cela posé, le système que nous tentons de présenter n’attaque aucunement ni les cultes ni les opinions religieuses établies chez les différents peuples de la terre ; il se propose uniquement de montrer aux hommes, de quelque pays ou de quelque religion qu’ils soient, les moyens que la Nature leur fournit pour obtenir le bien-être qu’elle les oblige de désirer, et de leur indiquer les motifs naturels faits pour les exciter soit à faire le bien, soit à fuir le mal. En un mot, je le répète, une morale humaine n’a pour objet que la conduite des hommes en ce monde : elle laisse à la théologie le soin de les conduire à l’autre vie. Les religions des peuples varient dans les différentes contrées de notre globe, mais les intérêts, les devoirs, les vertus, le bien-être sont les mêmes pour tous ceux qui l’habitent.\par
Quelques sages de l’Antiquité ont prétendu que la philosophie n’était que {\itshape la méditation de la mort}\footnote{« Tota philosophorum vita commentatio mortis est. » Cicéron, {\itshape Tusculanes}, I, C. 30. 31.}. Mais des idées plus conformes à nos intérêts et moins lugubres nous feront définir la philosophie {\itshape la méditation de la vie}. L’art de mourir n’a pas besoin d’être appris ; l’art de bien vivre intéresse bien plus des êtres intelligents et devrait occuper toutes leurs pensées en ce monde. Quiconque aura bien médité ses devoirs et les aura fidèlement pratiqués, jouira d’un bonheur véritable durant sa vie et la quittera sans crainte et sans remords. « La vie, dit Montaigne, n’est de soi ni un bien ni un mal ; c’est la place du bien et du mal selon que vous la leur faites. À mon avis, c’est le vivre heureusement et non le mourir heureusement qui fait l’humaine félicité. » Une vie ornée de vertus est nécessairement heureuse et nous conduit tranquillement vers un terme où nul homme ne pourra se repentir d’avoir suivi la route que sa nature lui a tracée. Une morale conforme à la Nature ne peut jamais déplaire à l’être que l’on réserve comme l’auteur de cette Nature.\par
L’homme est partout un être sensible, c’est-à-dire susceptible d’aimer le plaisir et de craindre la douleur. Dans toute société il est entouré d’êtres sensibles qui, comme lui, cherchent le plaisir et craignent la douleur. Ceux-ci ne contribuent au bien-être de leurs semblables que lorsqu’on les y détermine par le plaisir qu’on leur procure ; ils refusent d’y contribuer dès qu’on leur fait du mal. Voilà les principes sur lesquels on peut fonder une morale universelle ou commune à tous les individus de l’espèce humaine. C’est pour méconnaître ces principes incontestables que les hommes se rendent souvent si malheureux, que bien des sages ont cru que la félicité était pour toujours bannie de leur séjour.\par
N’adoptons point ces idées affligeantes, croyons fermement que l’homme est fait pour être heureux ; ne lui conseillons point de renoncer à la vie sociale sous prétexte de se soustraire aux inconvénients dont elle est souvent accompagnée : montrons-lui qu’ils sont balancés par des avantages inestimables. Les vices, les crimes, les défauts dont la société est tourmentée sont des suites de l’ignorance, de l’inexpérience et des préjugés dont les peuples sont encore les victimes, parce que bien des causes se sont continuellement opposées au développement de leur raison. La morale, ainsi que la plupart des connaissances humaines, n’a été jusqu’à présent si imparfaite et si ténébreuse que parce qu’elle n’a pas suffisamment consulté l’expérience, et que souvent elle a follement contrarié la Nature, qu’elle aurait dû prendre incessamment pour guide. Les mœurs des hommes sont corrompues parce que ceux qui auraient dû les conduire au bonheur en leur faisant observer les devoirs de la morale, faute de connaître leurs propres intérêts, ont cru qu’il fallait que les hommes fussent aveugles et déraisonnables afin de les mieux dompter et de les tenir dans les fers. Si la morale fut incapable de contenir les peuples, c’est que les puissances de la terre ne lui ont jamais prêté le secours des récompenses et des peines dont elles étaient dépositaires. Des gouvernements injustes ont redouté la vraie morale, des gouvernements négligents l’ont regardée comme une science de pure spéculation dont la pratique était totalement indifférente à prospérité des empires. Ils n’ont pas senti qu’elle seule pouvait être la base de la félicité publique et particulière, et que sans elle les États les plus puissants en apparence marchaient à leur ruine.\par
Ainsi, n’admettons pas les principes insensés d’un philosophe célèbre par ses paradoxes qui s’est mis à la torture pour nous prouver que {\itshape les vices particuliers tournaient au profit de la société}\footnote{Mandeville, dans {\itshape La Fable des Abeilles.} Il est très probable que cet auteur ingénieux s’est proposé dans son ouvrage de faire voir qu’il fallait totalement renoncer aux bonnes mœurs dans un pays tel que le sien, où toutes les vues du gouvernement et des particuliers sont tournées vers les richesses. Voyez ce qu’il sera dit dans le chap. 1 de la IV\textsuperscript{e} section.}, à moins que cet auteur n’ait voulu par une satire ingénieuse prouver à ses concitoyens l’impossibilité de concilier les vertus sociales avec la passion désordonnée pour les richesses et le luxe dont le propre est de les anéantir totalement. Nous dirons au contraire que les vices des particuliers influent toujours d’une façon plus ou moins fâcheuse sur le bien-être des nations. Les vices épidémiques leur causent souvent des transports et des délires dont elles sont tôt ou tard les victimes. Les vices des individus détruisent le bonheur des familles, et c’est l’assemblage des familles qui forme les nations. L’activité prétendue que les vices donnent aux hommes, est la même que celle que la fièvre produit en eux : les pays où le luxe domine ressemblent à des malades inconsidérés chez qui les aliments dont ils se surchargent se convertissent en poison. Les richesses, en s’accumulant de plus en plus chez un peuple, ne servent qu’à le rendre de jour en jour plus vicieux et plus misérable.\par
On nous dira peut-être qu’il est indifférent au gouvernement, pourvu qu’il soit riche et puissant, de s’occuper des mœurs des hommes ; mais nous répondrons que ces mœurs intéressent tous les citoyens, auxquels ils n’est point indifférent que leurs concitoyens soient honnêtes gens ou fripons, puisqu’ils ont à vivre avec eux. Nous dirons, de plus, qu’un État, pour être florissant et puissant, a plus besoin de vertus que de richesses. Enfin, nous dirons qu’il est bien plus important pour une nation d’être heureuse que d’avoir de grands trésors et de grandes forces dont à tout moment elle serait tentée d’abuser. L’opulence et la puissance d’une nation, que l’on a mal à-propos confondues avec sa félicité, sont souvent pour elle des occasions prochaines de destruction.\par
Ainsi, les vices et les passions des particuliers ne sont jamais utiles à l’État ; ils peuvent bien l’être pour les despotes, les tyrans et leurs suppôts, qui se servent des vices de leurs sujets pour les diviser d’intérêts et les subjuguer les uns les autres. Si c’est l’utilité de ces personnages que l’auteur dont nous parlons avait en vue, il a confondu l’intérêt d’une nation avec celui de ses plus cruels ennemis. Au reste, tout notre ouvrage présentera dans chaque ligne une réfutation de ce système téméraire et fera voir les conséquences funestes de la tyrannie ou de la négligence de ceux qui devraient régler les mœurs des hommes.\par
Par une suite de la même perversité ou de la même indifférence, l’éducation fut partout négligée, ou celle que l’on donna ne fut nullement capable de former des êtres sociables ou vertueux. Enfin, au sein de la dissipation et des plaisirs frivoles, la morale, trop sérieuse et trop incommode pour des êtres vicieux et légers, ne fut point étudiée ; chacun se contenta de quelques notions superficielles, chacun crut en savoir assez pour se conduire dans le monde. Très peu de gens se sont donné la peine de saisir la chaîne des principes et des motifs faite pour régler leurs actions à chaque pas. Tout le monde prétend être bon juge en morale, tandis qu’il n’est rien de plus rare que des hommes qui en aient les idées les plus simples ; tout le monde dans la théorie reconnaît son utilité mais peu de gens s’embarrassent de la mettre en pratique. Chacun du bout des lèvres rend hommage à la vertu, et presque personne ne se l’est bien définie. Chacun nous parle de la raison, et rien de moins ordinaire que des êtres qui la cultivent. Enfin, dans cette foule immense de traités de morale dont l’univers est inondé, on trouve rarement des vues capables d’éclairer l’homme sur ses devoirs.\par
D’un autre côté, un préjugé très universel persuade non seulement que les Anciens ont tout dit, mais encore que les mœurs antiques valaient bien mieux que celles qu’ils voient régner de leur temps. Bien des gens semblent admettre la fable de {\itshape l’âge d’or}, ou du moins s’imaginent que les peuples dans leur origine étaient et plus vertueux et plus heureux que leur postérité.\par
La moindre réflexion sur les annales du monde suffit pour détruire une pareille opinion. Les nations n’ont été d’abord que des hordes sauvages, et des sauvages ne sont ni heureux, ni sages, ni vraiment sociables. S’ils ont été exempts de mille besoins enfantés depuis par le luxe et par les vices qu’il engendre, ils ont été féroces, cruels, injustes, turbulents, totalement étrangers aux sentiments de l’équité et de l’humanité. Si les premiers temps de Rome nous montrent dans les Curius, les Cincinnatus, des exemples de frugalité, ils nous font voir dans tous les Romains une ambition injuste, perfide, inhumaine, qui ne doit pas prévenir en faveur de leur morale. Dans la république de Sparte, dont on nous vante si souvent les vertus, tout homme de bien ne peut voir qu’une troupe de brigands très austères et très méchants. L’Antiquité nous montre des peuples guerriers, des peuples très puissants, mais elle ne nous montre pas des peuples sages et vertueux. N’en soyons point étonnés : les mœurs des nations sont toujours le fruit des idées que leur inspirent ceux qui les gouvernent. La vraie morale eut à combattre en tout temps les préjugés enracinés dans l’esprit des peuples, des opinions et des usages que le temps avait rendu sacrés, et surtout les faux intérêts de ceux qui faisaient mouvoir la machine politique. Quelle morale et quelles vertus réelles pouvaient avoir les Romains à qui tout inspirait dès la plus tendre enfance un amour exclusif pour la patrie propre à les rendre injustes envers tous les peuples de la terre ? Un philosophe qui dans Rome eût recommandé les vertus sociales, eût-il été favorablement écouté par un Sénat pervers dont l’intérêt voulait que le peuple fût toujours en guerre afin de le dompter plus facilement et de le rendre plus soumis à ses décrets ? On l’eût admiré, peut-être, comme un discoureur éloquent, mais on eût regardé ses maximes comme contraires aux intérêts de l’État. Un homme vraiment sensible, équitable et vertueux eût passé dans Rome pour un très mauvais citoyen.\par
Les vrais principes de la morale paraissent en effet heurter de front des notions, des coutumes, des institutions visiblement opposées à la sociabilité que l’on voit établies chez presque tous les peuples. En développant à leurs yeux les règles de l’équité, les fondements de l’autorité, les droits des citoyens, quel est le gouvernement qui ne soupçonne aussitôt que l’on fait la critique de sa conduite et que l’on veut attaquer son pouvoir ? La politique n’ayant été autrefois, et n’étant encore pour l’ordinaire, que l’art fatal d’aveugler les peuples et de les mettre en servitude, se crut presque toujours intéressée à supprimer les lumières et à réduire la raison au silence. Enfin, la vraie morale trouva des contradicteurs opiniâtres dans l’ignorance, la pusillanimité, l’inertie des citoyens mêmes qui auraient le plus besoin qu’elle modérât les passions de ceux dont à tout moment ils éprouvent les rigueurs.\par
Ces obstacles ne sont pas faits pour rebuter les âmes qui brûlent d’un désir sincère d’être utiles au genre humain et qui sont échauffées de l’amour de la vertu. La morale est la vraie science de l’homme, la plus importante pour lui, la plus digne d’occuper un être vraiment sociable. C’est à la morale qu’il appartient de mûrir l’esprit humain, de rendre l’homme raisonnable, de le dégager des bandelettes de l’enfance, de lui apprendre à marcher d’un pas ferme vers les objets vraiment désirables pour des êtres intelligents. Les talents réunis des hommes qui pensent devraient enfin conspirer à faire connaître, et aux peuples et à leurs chefs, leurs véritables intérêts, afin de les détromper de tant de frivolités, de tant de vains jouets, de tant de passions aveugles qui causent leurs misères. Assez et trop longtemps les talents n’ont servi qu’à flatter bassement la grandeur, à propager les erreurs, à fomenter des vices, à charmer l’ennui des hommes : l’esprit et le génie devraient enfin s’occuper de leur instruction et de leur félicité. Est-il un objet plus digne de notre curiosité que la science de bien vivre et de se rendre heureux ?\par
La morale est la science du bonheur. Elle est utile et nécessaire à tous les habitants de la terre, elle est utile aux nations, aux souverains, aux citoyens, aux grands et aux petits, aux riches et aux pauvres, aux parents et aux enfants, aux maîtres et aux esclaves, qu’elle invite également à chercher leur bien-être. Sans elle, comme on le prouvera, la politique n’est qu’un brigandage fait pour anéantir les mœurs des peuples, sans elle le genre humain est perpétuellement troublé par l’ambition des rois, sans elle une société ne rassemble que des ennemis toujours prêts à se nuire, sans elle les familles en discorde ne font que rapprocher des malheureux qui se tourmentent journellement par leurs caprices et leurs humeurs incommodes, sans elle, enfin, chaque homme est à tout moment le jouet et la victime des vices et des excès auxquels son imprudence le livre.\par
En un mot, la morale est faite pour régler le destin de l’univers. Elle embrasse les intérêts de toute la race humaine, elle a droit de commander à tous les peuples, à tous les rois, à tous les citoyens, et ses décrets ne sont jamais impunément violés. La {\itshape politique}, comme on verra bientôt, n’est que la morale appliquée à la conservation des États. La {\itshape législation} n’est que la morale rendue sacrée par les lois. Le {\itshape droit des gens} n’est que la morale appliquée à la conduite des nations entre elles. Le {\itshape droit de la Nature} n’est que l’assemblage des règles de la morale puisées dans la nature de l’homme. C’est donc à juste titre que l’on peut appeler cette science {\itshape universelle}, puisque son vaste empire comprend toutes les actions de l’homme dans toutes les positions de la vie.\par
Que les hommes qui méditent cherchent donc à dégager cette science importante des nuages dont depuis tant de siècles on n’a fait que l’entourer, que ses principes soigneusement discutés prennent enfin ce degré de certitude propre à convaincre les esprits. Qu’aucunement guidée par l’expérience, elle n’affecte plus le langage de l’allégorie, qu’elle ne rende plus du haut de l’Empyrée des oracles ambigus, qu’elle renonce aux rêveries du platonisme, qu’elle quitte le ton rebutant du stoïcisme, qu’elle abjure les singularités du cynisme, qu’elle se dégage des labyrinthes de l’aristotélisme. Enfin, toujours guidée par la bonne foi et la droiture, qu’elle parle avec franchise et simplicité, qu’elle n’étonne plus par des paradoxes et qu’elle rougisse de la charlatanerie dont des hommes vains et trompeurs l’ont tant de fois revêtue.\par
Pour être utile, je le répète, la morale doit être simple et vraie. Il faut qu’elle s’explique clairement. Elle ne cherchera point à éblouir par de vains ornements qui trop souvent défigurent la vérité, elle ne promettra pas un {\itshape souverain bien} idéal attaché à une apathie insociable, à une misanthropie dangereuse, à une sombre mélancolie, elle ne conseillera pas aux hommes de s’éloigner les uns des autres ou de se haïr eux-mêmes, elle ne les rebutera point par des préceptes austères, par des conseils impraticables, par des perfections inaccessibles, elle ne leur prescrira jamais des vertus contraires à leur nature, elle les consolera de leurs peines et leur dira d’en espérer la fin et d’en chercher les remèdes, elle leur commandera d’être hommes, de réfléchir, de consulter leur raison, qui toujours les rendra justes, humains, bienfaisants, sociables, qui leur apprendra en quoi consiste leur bien-être réel, qui leur permettra les plaisirs honnêtes, qui leur indiquera les moyens légitimes de s’assurer un bonheur solide durant une vie exempte de honte et de remords.\par
Tel est le but auquel on s’est efforcé de contribuer dans cet ouvrage où l’on essaie de développer la nature de l’homme moral, sa tendance invariable, les désirs ou les passions qui le remuent, les principes de la vie sociale, les vertus qui maintiennent et les vices qui troublent son harmonie. Dans la première partie l’on a tâché de donner des définitions simples et d’exposer clairement les principes de la science des mœurs. Dans la seconde partie on appliquera les principes établis dans la première à tous les états de la vie. Au risque de paraître diffus, on s’est permis de rappeler et d’appliquer plus d’une fois les mêmes principes, afin de les rendre toujours présents à ceux des lecteurs qui n’en auraient pas saisi l’ensemble. Une morale élémentaire demande qu’on sacrifie la brièveté au désir de la mettre à la portée de tout le monde. Les ouvrages serrés et précis, plus agréables sans doute aux personnes éclairées, ne sont pas toujours utiles à celles qui cherchent à s’instruire ; souvent on se rend obscur en voulant être trop court. Enfin, pour joindre l’autorité au raisonnement, l’on a cru devoir enrichir cet ouvrage de pensées remarquables et de maximes utiles tirées des Anciens et des modernes, dans la vue de former une espèce de concordance capable de {\itshape fortifier chacun des chaînons du système moral que l’on a tenté d’établir}.
\section[{Section I. Principes généraux et Définitions}]{Section I. Principes généraux et Définitions}\renewcommand{\leftmark}{Section I. Principes généraux et Définitions}

\subsection[{Chapitre I. De la Morale, des Devoirs, de l’Obligation morale}]{Chapitre I. De la Morale, des Devoirs, de l’Obligation morale}
\noindent La morale est la science des rapports qui subsistent entre les hommes ou des devoirs qui découlent de ces rapports. Ou, si l’on veut, la morale est la connaissance de ce que doivent nécessairement faire ou éviter des êtres intelligents et raisonnables qui veulent se conserver et vivre heureux en société.\par
Pour être universelle, la morale doit être conforme à la nature de l’homme en général, c’est-à-dire fondée sur son essence, sur les propriétés et qualités que l’on trouve constamment dans tous les êtres de son espèce et par lesquelles on le distingue des autres animaux. D’où l’on voit que la morale suppose la science de la nature humaine.\par
Toute science ne peut être que le fruit de l’expérience. Savoir une chose, c’est avoir éprouvé les effets qu’elle produit, la manière dont elle agit, les différents points de vue sous lesquels on peut l’envisager. La science des mœurs, pour être sûre, ne doit être qu’une suite d’expériences constantes, réitérées, invariables, qui seules peuvent fournir une connaissance vraie des rapports subsistant entre les êtres de l’espèce humaine.\par
Les {\itshape rapports} subsistant entre les hommes sont les différentes manières dont ils agissent les uns sur les autres ou dont ils influent sur leur bien-être réciproque.\par
Les {\itshape devoirs} de la morale sont les moyens qu’un être intelligent et susceptible d’expérience doit prendre pour obtenir le bonheur vers lequel sa nature le force de tendre sans cesse. Marcher est un devoir pour qui veut se transporter d’un endroit à un autre, être utile est un devoir pour qui veut mériter l’affection et l’estime de ses semblables, s’abstenir de faire du mal est un devoir pour qui craint de s’attirer la haine et le ressentiment de ceux qu’il sait pouvoir contribuer à son propre bonheur. En un mot, le devoir est la convenance des moyens avec la fin qu’on se propose. La sagesse consiste à proportionner ces moyens à cette fin, c’est-à-dire à les employer utilement pour obtenir la félicité que l’homme est fait pour désirer.\par
L’{\itshape obligation} morale est la nécessité de faire ou d’éviter de [faire] certaines actions en vue du bien-être que nous cherchons dans la vie sociale. Celui qui veut la fin doit vouloir les moyens ; tout être qui désire se rendre heureux est obligé de suivre la route la plus propre à le conduire au bonheur et de s’éloigner de celle qui l’écarte de son but, sous peine d’être malheureux. La connaissance de cette route ou de ces moyens est le fruit de l’expérience, qui seule peut nous faire connaître et le but que nous devons nous proposer, et les voies les plus sûres pour y parvenir.\par
Les liens qui unissent les hommes les uns aux autres ne sont que les obligations et les devoirs auxquels ils sont soumis d’après les rapports qui subsistent entre eux. Ces obligations ou devoirs sont les conditions sans lesquelles ils ne peuvent se rendre réciproquement heureux. Tels sont les liens qui unissent les pères et les enfants, les souverains et les sujets, la société avec ses membres, etc.\par
Ces principes suffisent pour nous convaincre que l’homme n’apporte point en naissant la connaissance des devoirs de la morale, et que rien n’est plus chimérique que l’opinion de ceux qui attribuent à l’homme des sentiments moraux {\itshape innés}. Les idées qu’il a du bien et du mal, du plaisir et de la douleur, de l’ordre et du désordre, des objets qu’il doit chercher ou fuir, désirer ou craindre, ne peuvent être que des suites de ses expériences, et il ne peut compter sur ses expériences que lorsqu’elles sont constantes, réitérées, accompagnées de jugement, de réflexion et de raison.\par
L’homme n’apporte en venant au monde que la faculté de sentir, et de sa sensibilité découlent toutes ses facultés que l’on nomme {\itshape intellectuelles}. Dire que nous avons des idées morales antérieures à l’expérience du bien et du mal que les objets font éprouver, c’est dire que nous connaissons les causes sans avoir senti leurs effets.
\subsection[{Chapitre II. De l’Homme et de sa Nature}]{Chapitre II. De l’Homme et de sa Nature}
\noindent L’homme est un être sensible, intelligent, raisonnable, sociable, qui dans tous les instants de sa durée cherche sans interruption à se conserver et à rendre son existence agréable.\par
Quelle que soit la variété prodigieuse que l’on trouve dans les individus de l’espèce humaine, ils ont une nature commune qui ne se dément jamais. Il n’est point d’homme qui ne se propose quelque bien dans tous les moments de sa vie, il n’en est point qui, par les moyens qu’il suppose les plus propres, ne cherche à se procurer le bonheur et à se garantir de la peine. Nous nous trompons souvent et sur le but et sur les moyens, soit parce que nous manquons d’expériences, soit parce que nous ne sommes pas en état de faire usage de celles que nous avons pu recueillir. L’ignorance et l’erreur sont les vraies causes des égarements des hommes et des malheurs qu’ils s’attirent.\par
Pour ne s’être pas formé des idées vraies de la nature de l’homme, beaucoup de moralistes se sont trompés dans leurs systèmes sur la morale et nous ont donné des romans et des fables au lieu de l’histoire de l’homme ; le mot {\itshape nature} fut communément pour eux un terme vague auquel, le plus souvent, ils ne surent pas attacher de sens bien déterminé. Cependant, la morale étant la science de l’homme, il est important de commencer par s’en faire des idées véritables, sans quoi l’on serait en danger de s’égarer à tout moment. Mais pour connaître l’homme, il ne faut pas, comme il est arrivé trop souvent, à l’aide d’une métaphysique incertaine et trompeuse, rechercher les ressorts cachés qui le remuent : il suffit de considérer l’homme tel qu’il se présente à notre vue, tel qu’il agit constamment sous nos yeux, et d’examiner les qualités et les propriétés qui se trouvent visiblement et constamment en lui.\par
Cela posé, nous appellerons {\itshape nature} dans l’homme, l’assemblage des propriétés et qualités qui le constituent ce qu’il est, qui sont inhérentes à son espèce, qui la distinguent des autres espèces d’animaux ou qui lui sont communes avec elles. Sans remonter péniblement par la pensée jusqu’aux principes invisibles auxquels sont dus le sentiment et la pensée, il suffit en morale de savoir que tout homme sent, pense, agit, et cherche le bien-être dans tous les instants de sa durée : voilà les qualités et les propriétés qui constituent la nature humaine et que l’on trouve constamment dans tous les individus de notre espèce. Il n’est pas besoin d’en savoir davantage pour découvrir la conduite que tout homme doit tenir pour atteindre le but qu’il se propose.
\subsection[{Chapitre III. De la Sensibilité. Des Facultés intellectuelles}]{Chapitre III. De la Sensibilité. Des Facultés intellectuelles}
\noindent Dans l’homme, ainsi que dans tous les animaux, la sensibilité est une disposition naturelle qui fait qu’il est agréablement ou désagréablement remué par les objets qui agissent sur lui ou avec lesquels il a quelques rapports. Cette faculté dépend de la structure du corps humain, de son organisation particulière, des sens dont il est pourvu. Cette organisation rend l’homme susceptible de recevoir des impressions durables ou passagères de la part des objets dont ses sens sont frappés. Ces sens sont la vue, le toucher, le goût, l’odorat et l’ouïe. Les impressions que l’homme reçoit par ces différentes voies sont des impulsions, des mouvements des changements opérés en lui-même, et dont il a la {\itshape conscience}. Celle-ci n’est que la connaissance intime des changements ou des effets que les objets qui le remuent produisent dans sa machine. Ces effets se nomment {\itshape sensations} ou {\itshape perceptions}, parce que, éprouvés par les sens, ils lui font percevoir que les objets agissent sur lui.\par
Les sensations font naître des idées, c’est-à-dire des images, des traces, des impressions que nos sens ont reçues. Le sentiment continué ou renouvelé des impressions ou des idées qui se sont tracées en nous se nomme {\itshape pensée}. La faculté de contempler ces idées imprimées ou tracées au dedans de nous-mêmes par les objets qui ont agi sur nos sens, se nomme {\itshape réflexion}. La faculté de nous représenter de nouveau les idées ou les images que nos sens nous ont apportées, lors même que les objets qui les ont produites sont absents, se nomme {\itshape mémoire}. L’on appelle {\itshape jugement} la comparaison des objets qui nous remuent ou qui nous ont remués, des idées qu’ils produisent ou qu’ils ont produites en nous, des effets que nous sentons ou que nous avons sentis. {\itshape L’esprit} est la facilité de comparer avec promptitude les rapports des causes et des effets. {\itshape L’imagination} est la faculté de nous représenter avec force les images, les idées ou les effets que les objets ont fait naître en nous. L’intelligence, la raison, la prévoyance, la prudence, l’adresse et l’industrie ne sont que des suites de nos façons de sentir.\par
Tous les animaux donnent évidemment des signes plus ou moins marqués de sensibilité. De même que l’homme, nous les voyons affectés par les objets qui agissent sur eux, nous les voyons chercher avec ardeur ce qui est utile à leur conservation, ce qui est propre à satisfaire leurs besoins, ce qui est capable de leur procurer du bien-être ; nous voyons qu’ils fuient les objets dont ils ont éprouvé des sensations douloureuses, nous trouvons en eux des réflexions, de la mémoire, de la prévoyance, de la sagacité.\par
Enfin, tout nous prouve qu’ils ont quelquefois dans leurs organes une finesse supérieure à celle de l’homme. Ce que nous appelons {\itshape instinct} dans les animaux, est la faculté de se procurer les moyens de satisfaire des besoins ; il ressemble beaucoup à ce que l’on nomme {\itshape intelligence, raison, sagacité} dans l’homme. Beaucoup d’hommes, par leur conduite, donnent si peu de signes d’intelligence et de raison que leurs facultés intellectuelles semblent fort au-dessous de ce qu’on nomme l’instinct des bêtes. Il est des hommes qui diffèrent bien plus d’autres hommes, que l’homme en général ne diffère de la brute. L’enfant qui vient de naître a moins d’industrie et de ressources que les animaux les plus dépourvus de raison. Tout homme qui se livre inconsidérément à la débauche, à l’intempérance, à l’ivrognerie, à la colère, à la vengeance, se montre-t-il réellement supérieur aux bêtes ?\par
L’homme diffère des autres animaux et se montre supérieur à eux par son activité, par l’énergie de ses facultés, par la force de sa mémoire, par la multiplicité de ses expériences, par son industrie, qui le mettent à portée de satisfaire avec plus de facilité ses besoins. En un mot, l’homme, à force d’expériences et de réflexions, non seulement éprouve les sensations présentes mais encore se rappelle les sensations passées et prévoit les sensations futures ; une sagacité supérieure le met en état de faire contribuer la Nature entière à son propre bonheur. Mais ces facultés demandent à être développées, sans cela il demeurerait dans un abrutissement peu différent de celui des bêtes. En naissant il apporte des dispositions naturelles qui, bien ou mal cultivées, le rendent raisonnable ou insensé, bon ou méchant, prudent ou inconsidéré, capable ou incapable de réflexion et de jugement, expérimenté ou ignorant.\par
D’un autre côté, quoique tous les hommes en général paraissent conformés de la même manière et sujets aux mêmes passions, cependant la sensibilité n’est pas la même dans tous les individus dont le genre humain est composé. Cette sensibilité est plus ou moins vive suivant le plus ou le moins de finesse et de mobilité dont la Nature a doué leurs organes, suivant la qualité des fluides et des solides dont leur machine est composée, d’où découle la variété de leurs tempéraments et de leurs facultés.\par
Le {\itshape tempérament} n’est que la façon d’être particulière à chaque individu de l’espèce humaine ; elle résulte de l’organisation ou de la conformation qui lui est propre. Par une suite de ce tempérament, parmi les hommes les uns sont plus sensibles que les autres, c’est-à-dire plus susceptibles d’être promptement remués par les objets qui frappent leurs sens. Les uns ont de la vigueur, de l’esprit, de l’imagination, des passions vives, de l’enthousiasme, de l’impétuosité, tandis que d’autres sont faibles, lâches, stupides, paresseux, languissants ; les uns ont une mémoire heureuse, un jugement sain, sont capables d’expérience, de réflexion, de raison, de prudence, de prévoyance, tandis que d’autres sont totalement privés de ces facultés.\par
Les uns sont disposés à la gaieté, remuants, inquiets, dissipés, les autres sont posés, mélancoliques, furieux, recueillis en eux-mêmes, etc.\par
En un mot, les différents degrés de sensibilité produisent cette diversité merveilleuse que nous voyons entre les caractères, les penchants et les goûts des hommes ; cette qualité les distingue autant que les traits de leurs visages. Les hommes ne diffèrent entre eux que parce qu’ils ne sentent pas précisément de la même manière. Dès lors ils ne peuvent avoir précisément les mêmes sensations, les mêmes idées, les mêmes inclinations, les mêmes opinions des choses, ni par conséquent tenir la même conduite dans la vie.
\subsection[{Chapitre IV. Du Plaisir et de la Douleur. Du Bonheur}]{Chapitre IV. Du Plaisir et de la Douleur. Du Bonheur}
\noindent Nonobstant les nuances infinies qui distinguent les hommes, de façon qu’il n’en est pas deux qui soient exactement semblables, ils ont un point général sur lequel tous sont d’accord : c’est l’amour du plaisir et la crainte de la douleur. Dans la même famille de plantes, il n’en est pas qui soient rigoureusement les mêmes, il n’est pas deux feuilles sur un arbre qui ne montrent des différences à l’œil observateur, et cependant, ces plantes, ces arbres, ces feuilles sont de la même espèce et tirent également leurs sucs nourriciers de la terre et des eaux. Placées dans un sol convenablement préparé, échauffées par les rayons d’un soleil favorable, soigneusement arrosées, ces plantes s’animent, végètent, s’élèvent et présentent à nos yeux les marques d’une sorte de gaieté. Au contraire, si elles se trouvent sur un terrain aride, elles languissent, elles paraissent souffrir, se fanent et se détruisent, quelque soin qu’on se doit donner pour les cultiver\footnote{L’ingénieux auteur du livre {\itshape De l’Esprit} croit que l’éducation, ou la modification, suffit pour faire des hommes ce que l’on veut ; ce philosophe célèbre ne semble pas avoir fait attention que si la Nature ne fournit pas un sujet idoine, il est impossible de le bien modifier. C’est en vain qu’on sèmerait sur un roc, ainsi que dans un terrain trop aquatique. Nous aurons occasion de revenir sur cette question lorsque nous parlerons que l’éducation. Voyez section V, chap. 3 de la II\textsuperscript{e} partie. Plutarque dit, suivant la traduction d’Amyot : « La Nature sans doctrine et nourriture est une chose aveugle, la doctrine sans Nature est défectueuse, et l’usage, dans les deux premières, est chose imparfaite. Ni plus ni moins qu’au labourage, il faut premièrement que la terre soit bonne, secondement que le laboureur soit homme entendu, et tiercement que la semence soit choisie et élue. Aussi, la Nature représente la terre, le maître qui enseigne ressemble au laboureur et les enseignements et exemples reviennent à la semence. » Voyez Plutarque, {\itshape Comme il faut nourrir les enfants}, tome 2, page 2. B., opp., édition de Paris, 1624.}.\par
Parmi les impressions ou sensations que l’homme reçoit des objets qui le frappent, les unes, par leur conformité avec la nature de sa machine, lui plaisent, et d’autres, par le trouble et le dérangement qu’elles y portent, lui déplaisent. En conséquence, il approuve les unes, il souhaite qu’elles continuent ou se renouvellent en lui, tandis qu’il désapprouve les autres et désire qu’elles disparaissent. D’après la façon agréable ou fâcheuse dont nos sens sont remués, nous aimons et nous haïssons les objets, nous les désirons et nous les craignons, nous les cherchons ou nous tâchons d’en écarter les influences.\par
Aimer un objet, c’est souhaiter sa présence, c’est désirer qu’il continue à produire sur nos sens des impressions convenables à notre être, c’est vouloir le posséder afin d’être souvent à portée d’éprouver ses effets agréables. Haïr un objet, c’est désirer son absence afin de voir terminer l’impression pénible qu’il produit sur nos sens. Nous aimons un ami parce que sa présence, sa conversation, ses qualités estimables nous causent du plaisir ; nous désirons de ne point rencontrer un ennemi parce que sa présence nous gêne.\par
Toute sensation ou tout mouvement agréable qui s’excite en nous-mêmes et dont nous désirons la durée se nomme {\itshape bien, plaisir}, et l’objet qui produit cette impression en nous se nomme {\itshape bon, utile, agréable.} Toute sensation dont nous désirons la fin, parce qu’elle nous trouble et dérange l’ordre de notre machine, s’appelle {\itshape mal} ou {\itshape douleur}, et l’objet qui l’excite se nomme {\itshape mauvais, nuisible, méchant, désagréable.} Le plaisir durable et continué se nomme {\itshape bonheur, bien-être, félicité} ; la douleur continuée se nomme {\itshape malheur, infortune}. Le bonheur est un état d’acquiescement continué aux façons de sentir et d’exister que nous trouvons agréables ou conformes à notre être.\par
L’homme par sa nature doit aimer nécessairement le plaisir et haïr la douleur, parce que l’un est convenable à son être, c’est-à-dire à son organisation, à son tempérament, à l’ordre nécessaire à sa conservation ; la douleur, au contraire, dérange l’ordre de la machine humaine, empêche ses organes de remplir leurs fonctions, nuit à sa conservation.\par
L’ordre est en général la façon d’être par laquelle toutes les parties d’un tout conspirent sans obstacles à procurer la fin que sa nature lui propose. L’ordre, dans la machine humaine, est cette façon d’être qui fait que toutes les parties de son corps concourent à sa conservation et au bien-être de l’ensemble. {\itshape L’ordre moral} ou social est cet heureux concours des actions et des volontés humaines d’où résulte la conservation et le bonheur de la société. Le désordre est tout dérangement de l’ordre, ou tout ce qui nuit au bien-être de l’homme ou de la société.\par
Le plaisir n’est un bien qu’autant qu’il est conforme à l’ordre. Dès qu’il produit du désordre, soit immédiatement, soit par ses conséquences, ce plaisir est un mal réel, vu que la conservation de l’homme et son bonheur durable sont des biens plus désirables que des plaisirs passagers qui seraient suivis de peines. Au moment où, trempé de sueur, un homme boit avec ardeur une eau glacée, il éprouve sans doute un plaisir très vif, mais il peut être suivi d’une maladie terminée par la mort.\par
Le plaisir cesse d’être un bien pour devenir un mal dès qu’il produit en nous, soit sur le champ, soit par la suite, des effets nuisibles à notre conservation et contraires à notre bien-être permanent.\par
D’un autre côté, la douleur peut devenir un bien préférable au plaisir même, lorsqu’elle tend à nous conserver et à nous procurer des avantages constants. Un convalescent souffre patiemment les aiguillons de la faim et s’abstient des aliments qui flatteraient passagèrement son palais, en vue de recouvrer sa santé, qu’il envisage comme un bonheur plus désirable que le plaisir fugitif de contenter son appétit.\par
L’expérience seule peut nous apprendre à distinguer les plaisirs auxquels on peut se livrer sans crainte ou qu’on doit préférer, de ceux qui peuvent avoir pour nous des conséquences dangereuses. Quoique l’amour du plaisir soit essentiellement inhérent à l’homme, il doit être subordonné à l’amour de sa propre conservation et au désir d’un bien-être durable qu’il se propose à chaque instant. S’il veut être heureux, tout concourt à lui prouver que pour parvenir à cette fin, il doit mettre du choix dans ses plaisirs, en user avec modération, rejeter comme des maux ceux qui seront suivis de peines, et préférer les douleurs momentanées lorsqu’elles peuvent lui procurer un bonheur plus solide et plus grand.\par
Cela posé, les plaisirs doivent être distingués d’après leur influence sur le bonheur des hommes. Les {\itshape plaisirs vrais} sont ceux que l’expérience nous montre conformes à la conservation de l’homme et incapables de lui causer de la douleur. Les {\itshape plaisirs trompeurs} sont ceux qui, le flattant quelques instants, finissent par lui causer des maux durables. Les plaisirs raisonnables sont ceux qui conviennent à un être susceptible de distinguer l’utile du nuisible, le réel de l’apparent ; les plaisirs honnêtes sont ceux qui ne sont pas suivis de regrets, de honte et de remords. Les plaisirs déshonnêtes sont ceux dont nous sommes forcés de rougir, parce qu’ils nous rendent méprisables à nous-mêmes et aux autres ; le plaisir finit toujours par tourmenter quand il n’est pas conforme à nos devoirs. Les plaisirs légitimes sont ceux qui sont approuvés par les êtres avec qui nous sommes en société. Les plaisirs illicites sont ceux qui nous sont défendus par la loi, etc.\par
Les plaisirs ou sensations agréables qui se font immédiatement sentir à nos organes s’appellent plaisirs {\itshape physiques}. Quoiqu’ils procurent à l’homme une façon d’être qu’il approuve, ils ne peuvent longtemps durer sans causer l’affaiblissement de ces mêmes organes, dont la force est naturellement limitée. Ainsi, les mêmes plaisirs finissent par nous fatiguer si nous ne mettons entre eux des intervalles qui permettent aux sens de se reposer ou de reprendre des forces. La vue d’un objet éclatant nous plaît d’abord, mais finit par blesser nos yeux quand ils s’y arrêtent trop longtemps. Les plaisirs les plus vifs sont communément les moins durables, parce qu’ils produisent les secousses les plus violentes à la machine humaine, d’où il suit qu’un homme sage doit en être économe en vue de sa propre conservation. On voit par là que la tempérance, la modération, l’abstinence de quelques plaisirs, sont des vertus fondées sur la nature humaine.\par
L’homme, jouissant de plusieurs sens, a besoin que ses sens soient alternativement exercés, sans cela il tombe bientôt dans la langueur et l’ennui. D’où il suit que la nature de l’homme exige qu’il varie ses plaisirs. L’ennui est la fatigue de nos sens remués par des sensations uniformes.\par
Les plaisirs que l’on nomme {\itshape intellectuels} sont ceux que nous éprouvons au dedans de nous-mêmes ou qui sont produits par la pensée ou la contemplation des idées que nos sens nous ont fournies, par la mémoire, par le jugement, par l’esprit, par l’imagination. Telles sont les jouissances variées que procurent l’étude, la méditation, les sciences. Ces sortes de plaisirs sont préférables aux plaisirs physiques, parce que nous possédons en nous-mêmes les causes capables de les exciter ou de les renouveler en nous à volonté. Lorsque la lecture de l’Histoire a gravé dans la mémoire des faits curieux, agréables, intéressants, en parcourant ces faits, en les contemplant au-dedans de lui-même, l’homme de lettres éprouve un sentiment analogue mais supérieur à celui d’un curieux dont les yeux considèrent les tableaux rassemblés dans une vaste galerie. Lorsque la philosophie a fait connaître l’homme, ses rapports, ses variétés, ses passions, ses désirs, le philosophe, en méditant, jouit de la contemplation des matériaux dont sa tête s’est ornée. Enfin, l’homme vertueux jouit au-dedans de lui-même du bien qu’il fait aux autres et se nourrit agréablement de l’idée d’en être aimé.\par
D’ailleurs, les plaisirs intellectuels et les jouissances qu’ils nous procurent, sont plus à nous que celles que nous donnent les avantages extérieurs tels que les richesses, les grandes possessions, les dignités, le crédit, la faveur, que la fortune accorde et ravit à son gré. Nous sommes toujours en état de jouir des plaisirs dont nous portons la source au-dedans de nous-mêmes et dont les autres hommes ne peuvent point nous priver. Il n’y a que des maladies capables de causer un renversement total dans notre machine qui puissent nous empêcher de jouir de nos facultés intellectuelles et de nos vertus. Ces qualités inhérentes à l’homme peuvent seules lui mériter un attachement sincère, une amitié vraiment désintéressée. Aimer quelqu’un pour lui-même, c’est l’aimer non en vue de son pouvoir ou de son opulence, mais en vue des qualités agréables, des dispositions louables dont on jouit dans sa société, qui résident habituellement en lui, sur lesquelles on peut compter, parce qu’elles ne peuvent lui être enlevées que par des accidents peu communs dans la vie.
\subsection[{Chapitre V. Des Passions, des Désirs, des Besoins}]{Chapitre V. Des Passions, des Désirs, des Besoins}
\noindent Les passions, dans l’homme, sont des mouvements plus ou moins vifs d’amour pour les objets qu’il croit propres à lui fournir des impressions, des sensations, des idées agréables, ou bien ce sont des mouvements de haine pour les objets qu’il trouve ou qu’il suppose capables de l’affecter d’une façon douloureuse. Toutes les passions se réduisent à désirer quelque bien, quelque plaisir, quelque bonheur réel ou faux, et à craindre et fuir quelque mal, soit véritable, soit imaginaire. Les désirs sont des mouvements d’amour pour un bien véritable ou supposé que l’on ne possède pas. L’espérance est l’amour d’un bien que l’on attend mais dont on n’a pas encore la jouissance. La colère est une haine subite pour un objet que l’on croit nuisible, etc.\par
Rien n’est donc plus naturel à l’homme que d’avoir des passions et des désirs ; ces mouvements d’attraction qu’il éprouve pour certains objets et de répulsion pour d’autres, sont dus à l’analogie ou à la discordance qui se trouve entre ses organes et les choses qu’il aime ou qu’il hait. La plupart des enfants aiment avec passion le lait, les fruits doux, les aliments sucrés, et détestent les choses amères, parce que les premières substances produisent sur les houppes nerveuses de leur palais des sensations qui leur plaisent, tandis que l’amertume y excite des mouvements désagréables.\par
Les stoïciens, et beaucoup d’autres moralistes comme eux, ont pris les passions pour des {\itshape maladies de l’âme} qu’il fallait totalement déraciner. Mais les passions des hommes ne sont pas plus des maladies que la faim, qui leur est naturelle, qui les sollicite à se nourrir, qui leur fait désirer les aliments les plus conformes à leurs goûts, qui les avertit d’un besoin de leur machine qu’ils doivent satisfaire s’ils veulent se conserver. De ce que bien des hommes se surchargent l’estomac d’aliments nuisibles à la santé, l’on ne peut pas en conclure que la faim soit une maladie, ni que le désir de la satisfaire soit blâmable et ne doive point être écouté. Une philosophie fanatique est cause qu’en morale les hommes n’ont presque jamais pu convenir de rien.\par
Pour peu que l’on veuille réfléchir, on reconnaîtra que les passions, en elles-mêmes, ne sont ni bonnes ni mauvaises : elles ne deviennent telles que par l’usage qu’on en fait. Tout homme étant né avec des besoins, rien de plus naturel en lui que le désir de le satisfaire ; susceptible de sentir le plaisir et la douleur, rien de plus naturel que d’aimer l’un et de haïr l’autre. D’où il suit que les passions et les désirs sont essentiels à l’homme, inhérents à sa nature, inséparables de son être, nécessaires à sa conservation. Un être sensible qui haïrait le plaisir, qui fuirait le bien-être, qui désirerait le mal, enfin qui n’aurait aucun besoin, ne serait plus un homme ; incapable de se conserver lui-même, il serait totalement inutile aux autres.\par
L’on appelle {\itshape besoins} tout ce qui est utile ou nécessaire, soit à la conservation, soit à la félicité de l’homme. Les besoins que l’on nomme {\itshape naturels} sont de se nourrir, de vêtir, de se garantir des injures de l’air, et de se propager. Les besoins de tous les hommes sont les mêmes, ils ne varient que par les moyens de les satisfaire. Du pain sec suffit à l’homme pauvre pour apaiser le besoin de la faim ; il faut à l’homme opulent une table somptueuse, couverte des mets les plus rares pour contenter son appétit, et surtout sa vanité qui, pour lui, est devenue un besoin bien plus pressant que la faim parce que son imagination lui représente habituellement le faste comme un bien nécessaire à sa félicité. La peau des bêtes suffit pour vêtir un sauvage, au lieu que l’habitant d’un pays où règne le luxe se trouve malheureux et rougit lorsqu’il n’a pas des habits magnifiques dans lesquels son imagination lui montre le moyen de donner aux autres une grande idée de soi.\par
C’est ainsi que l’imagination, l’habitude, les conventions, les préjugés, nous font une multitude de besoins qui nous éloignent de notre nature ; nous nous trouvons fort à plaindre lorsque nous sommes hors d’état de les satisfaire. Rien de plus important que de borner ses besoins, afin de pouvoir les contenter sans peine. Nos besoins naturels sont en petit nombre et bornés, au lieu que les besoins crées par l’imagination sont insatiables et sans nombre. Plus les hommes ont de besoins, et plus il leur est difficile de se rendre heureux. La félicité consiste dans l’accord de nos besoins avec le pouvoir de les satisfaire.\par
Nous avons dit plus haut que les différents degrés de sensibilité dans les hommes étaient les causes de la diversité prodigieuse que l’on remarquait entre eux ; c’est de la même source que part la diversité de leurs passions, de leurs appétits, de leurs besoins, de leurs goûts, des volontés qui les font agir. Suivant l’organisation particulière à chaque homme, qui constitue en lui le tempérament, son imagination, ses besoins mêmes sont variés. Quoique tous les hommes aient besoin de nourriture, les mêmes aliments ne leurs plaisent point à tous : l’estomac de l’un en demande une plus grande quantité que celui d’un l’autre, ceux qui réussissent aux uns ne conviennent point aux autres et leur causent souvent des maladies fâcheuses.\par
C’est de là que résulte cette grande variété que l’on peut remarquer dans les passions ; elles diffèrent non seulement pour les objets vers lesquels elles se portent, mais encore pour la force et la durée. Toutes les passions sont excitées par les besoins des hommes ; ces besoins sont dus, soit au tempérament, soit à l’imagination, soit à l’habitude, soit à l’exemple, soit à l’éducation. D’où il suit qu’ils ne sont pas les mêmes dans tous les êtres de notre espèce ; bien plus, ils sont sujets à varier dans le même individu. Tous les hommes éprouvent la soif ou le besoin de boire ; aux uns de l’eau suffit pour l’apaiser, d’autres demandent du vin, devenu nécessaire pour ranimer leur estomac, d’autres, accoutumés à la délicatesse, ont besoin de vins délicieux ; enfin, les meilleurs vins répugnent à quelques personnes malades ou dégoûtées. Le besoin et le désir de boire sont bien plus forts dans un homme que l’exercice a violemment échauffé que dans le même homme qui s’est tenu tranquille. Un homme dont l’imagination vive lui peint fortement les plaisirs de l’amour attachés à un objet, se sent tourmenté par des désirs plus violents ou des passions plus fortes que celui dont l’imagination est plus paisible. Un amant bien épris des charmes de sa maîtresse, que son imagination lui exagère, éprouve une passion naturelle excitée par un besoin que cette imagination redouble à tout moment.\par
Ainsi, les besoins dans les hommes sont des choses qu’ils trouvent véritablement, ou qu’ils supposent faussement nécessaires à leur conservation, à leurs plaisirs, à leur bien-être. Les besoins {\itshape naturels} sont, comme on vient de le dire, les choses que notre nature a rendu nécessaires au maintien de notre être dans une existence heureuse. Les besoins {\itshape imaginaires} sont ceux qu’une imagination souvent déréglée nous peint très faussement comme indispensables à notre félicité. Une imagination perpétuellement enflammée par les exemples, les opinions, les habitudes que nous trouvons établies dans la société, nous rendent esclaves d’une foule de besoins dont nous sommes tourmentés sans cesse, et nous mettent dans la dépendance de ceux qui peuvent les satisfaire.\par
Pour être heureux et libre, il faudrait n’éprouver que les besoins que l’on peut satisfaire par soi-même et sans trop de peines. Des besoins immenses demandent des travaux et des secours multipliés, souvent très inutiles, et dès lors ces besoins nous rendent si malheureux que bien des gens ont cru que, pour les empêcher de s’accroître, l’homme devait combattre de toute sa force ses besoins, même les plus naturels, vivre en sauvage ou en anachorète, se priver de toute nourriture agréable, et faire du mal, se vouer au célibat, etc.\par
Cette morale outrée n’est point faite pour les hommes. Une morale plus sage leur dit de contenter leurs besoins naturels d’une façon qui ne soit nuisible ni pour eux-mêmes ni pour les autres, de circonscrire ces besoins afin de n’être point malheureux, faute de pouvoir les satisfaire, de prendre garde de les multiplier, parce qu’ils les entraîneraient dans le vice ou le crime. Nos besoins font naître nos désirs ; en diminuant les premiers, les désirs diminuent ou disparaissent.\par
Tant d’hommes ne sont malheureux et méchants que parce qu’ils se font des besoins qui rendent leurs désirs indomptables. Le bonheur consiste à ne désirer que ce qu’on peut obtenir.
\subsection[{Chapitre VI. De l’Intérêt, ou de l’amour de soi}]{Chapitre VI. De l’Intérêt, ou de l’amour de soi}
\noindent Nos désirs, excités par des besoins réels ou imaginaires, constituent l’{\itshape intérêt}, par où l’on désigne en général ce que chaque homme souhaite parce qu’il le croit utile ou nécessaire à son propre bien-être ; en un mot, l’objet dans la jouissance duquel chacun fait consister son plaisir ou son bonheur. L’intérêt du voluptueux est dans la jouissance des plaisirs des sens, l’avare a placé le sien dans la possession de ses trésors, le fastueux attache le plus grand intérêt à faire un vain étalage de ses richesses, l’ambitieux, dont l’imagination s’allume par l’idée d’exercer son empire sur d’autres hommes, place son intérêt dans la jouissance d’un grand pouvoir ; l’intérêt de l’homme de lettres consiste à mériter la gloire ; enfin, l’intérêt de l’homme de bien consiste à se faire estimer et chérir de ses semblables. Quand on dit que les intérêts des hommes sont variés, on indique simplement que leurs besoins, leurs désirs, leurs passions et leurs goûts ne sont pas les mêmes ou qu’ils attachent l’idée de bien-être à des objets divers.\par
Il est donc indubitable que tous les individus de l’espèce humaine n’agissent et ne peuvent agir que par intérêt. Le mot {\itshape intérêt}, ainsi que le mot {\itshape passion}, ne présente à l’esprit que l’amour d’un bien, le désir du bonheur. On ne peut donc blâmer les hommes d’être intéressés (ce qui signifie avoir des besoins et des passions), que lorsqu’ils ont des intérêts, des passions, des besoins nuisibles, soit pour eux-mêmes, soit pour les êtres avec les intérêts desquels les leurs ne s’accordent pas.\par
C’est d’après leurs intérêts que les hommes sont bons et méchants. En faisant le bien, comme en faisant le mal, nous agissons toujours en vue d’un avantage que nous croyons devoir résulter de notre conduite. L’idée de bien-être, ou l’intérêt attaché à des plaisirs ou à des objets contraires à notre propre bonheur, constitue ce que l’on appelle l’intérêt {\itshape mal entendu}. Il est la source des erreurs et des égarements des hommes qui, faute d’expérience, de réflexion et de raison, méconnaissent trop souvent leurs intérêts véritables et n’écoutent que des besoins imaginaires et des passions aveugles enfantées par leur ignorance, leurs préjugés, par les saillies d’une imagination déréglée.\par
L’intérêt personnel et les passions qu’il met en jeu, ne sont des dispositions blâmables que quand elles sont contraires au bien-être de ceux avec qui nous vivons, c’est-à-dire quand elles nous font tenir une conduite qui leur est incommode ou nuisible. Les hommes n’approuvent que ce qui leur est utile ; ainsi, leur intérêt les force à blâmer, haïr et mépriser tout ce qui contrarie leur tendance au bonheur. L’intérêt est louable et légitime lorsqu’il a pour objet des choses vraiment utiles et à nous-mêmes, et aux autres. L’amour de la vertu n’est que notre intérêt attaché à des actions avantageuses au genre humain. Si un intérêt sordide est le mobile de l’avare, un intérêt plus noble anime l’être bienfaisant. Il veut gagner l’affection, l’estime, la tendresse de ceux qui sont à portée de sentir les effets de sa générosité.\par
{\itshape Sacrifier son intérêt} signifie sacrifier un objet qui plaît ou qu’on aime à un objet que l’on aime plus fortement ou qui plaît davantage. Un ami consent à sacrifier une partie de sa fortune pour son ami, parce que cet ami lui est bien plus cher que la portion des biens qu’il lui sacrifie. L’enthousiasme est la passion pour un objet que l’on envisage uniquement, portée jusqu’à une sorte d’ivresse qui fait que l’homme lui sacrifie tout, jusqu’à lui-même. Nous allons voir dans un moment que dans ce cas, c’est toujours à son propre intérêt, c’est à lui-même que l’homme se sacrifie.\par
Agir sans intérêt, ce serait agir sans motif. Un être intelligent, c’est-à-dire qui se propose le bien-être à chaque instant de sa durée et qui sait employer les moyens propres à le conduire à ce but, ne peut pas un instant perdre de vue son intérêt. Pour que cet intérêt soit louable, il doit sentir que, la Nature l’ayant placé dans la société, son intérêt véritable exige qu’il s’y rende utile et agréable, parce que les êtres dont il est entouré, sensibles, amoureux du bien-être, intéressés comme lui, ne contribueront à son bonheur qu’en vue du bonheur qui qu’ils attendent de lui. D’où l’on voit que c’est sur l’intérêt que la morale doit fonder solidement tous ses préceptes pour les rendre efficaces. Elle doit prouver aux hommes que leur véritable intérêt exige qu’ils s’attachent à la vertu, sans laquelle il ne peut y avoir pour eux de bien-être sur la terre.\par
Quelques philosophes ont fondé la morale sur une {\itshape bienveillance} innée qu’ils ont cru inhérente à la nature humaine. Mais cette bienveillance ne peut être que l’effet de l’expérience et de la réflexion, qui nous montrent que les autres hommes sont utiles à nous-mêmes, sont en état de contribuer à notre propre bonheur. Une bienveillance désintéressée, c’est-à-dire de laquelle il ne résulterait pour nous de la part de ceux qui nous l’inspirent ni tendresse ni retour, serait un sentiment dépourvu de motifs ou un effet sans cause. C’est relativement à lui-même que l’homme montre de la bienveillance aux autres. Il veut s’en faire des amis, c’est-à-dire des êtres qui s’intéressent à lui, ou bien il éprouve ce sentiment pour ceux dont il a lui-même expérimenté les dispositions favorables, ou enfin il veut s’attirer l’estime de lui-même et de la société.\par
On nous dira peut-être que des personnes vertueuses poussent le désintéressement jusqu’à montrer de la bienveillance à des ingrats, et que d’autres la montrent à des hommes qu’ils n’ont jamais connus et ne verront jamais. Mais cette bienveillance même n’est point désintéressée ; si elle vient de la pitié, nous verrons bientôt que l’homme compatissant se soulage lui-même en faisant du bien aux autres. Enfin, nous prouverons que tout homme qui fait du bien trouve toujours en lui-même la récompense que les ingrats lui refusent ou que les inconnus ne peuvent lui témoigner.\par
Toutes les passions, les intérêts, les volontés et les actions de l’homme, n’ont pour objet constant que de satisfaire l’amour qu’il a pour lui-même. Cet {\itshape amour de soi}, tant blâmé par quelques moralistes et confondu mal à propos par eux avec un {\itshape égoïsme} insociable, n’est dans le fait que le désir permanent de se conserver et de se procurer une existence heureuse.\par
Condamner l’homme parce qu’il s’aime lui-même, c’est le blâmer d’être homme. Prétendre que cette affection vient de sa nature corrompue, c’est dire qu’une nature plus parfaite lui eût fait négliger sa conservation et son propre bien-être ; soutenir que ce principe des actions humaines est ignoble et bas, c’est dire qu’il est bas et ignoble d’être un homme.\par
Mettant à l’écart les préjugés dont les ouvrages d’un grand nombre de moralistes abondent, si nous voulons examiner l’homme tel que la Nature l’a fait, nous reconnaîtrons qu’il ne pourrait subsister s’il perdait de vue l’amour qu’il a pour lui-même. Tant qu’il jouit d’organes sains et bien constitués, il ne peut se haïr, il ne peut être indifférent au bien ou au mal qui lui arrive, il ne peut s’empêcher de désirer le bien-être qu’il n’a pas, ni de craindre le mal dont il se voit menacé. Il ne peut aimer les êtres de son espèce qu’autant qu’il les trouve favorables à ses désirs et disposés à contribuer à sa conservation et à sa propre félicité. C’est toujours en vue de lui-même qu’il a de l’affection pour les autres et qu’il s’unit avec eux.\par
C’est en vue du plaisir que font à notre cœur la présence, les conseils, les consolations d’un ami, que nous aimons cet ami ; c’est nous qui éprouvons les effets agréables de son commerce qui nous attachent à lui. C’est en vue du plaisir qu’une maîtresse procure à son imagination ou à ses sens, qu’un amant aime cette maîtresse au point même quelquefois de se sacrifier pour elle. C’est en vue du plaisir qu’une tendre mère éprouve en voyant un enfant chéri, qu’elle aime, qu’elle lui prodigue ses soins aux dépens même de sa santé et de sa propre vie. C’est nous-mêmes que nous aimons dans les autres, ainsi que dans tous les objets auxquels nous attachons notre amour. C’est lui-même que l’ami chérit dans son ami, l’amant dans sa maîtresse, la mère dans son enfant, l’ambitieux dans les honneurs, l’avare dans les richesses, l’homme de bien dans l’affection de ses semblables, et, à son défaut, dans le contentement intérieur que procure la vertu.\par
Si quelquefois l’amour de soi semble n’avoir aucune part à nos actions, c’est qu’alors le cœur se trouble, l’enthousiasme l’enivre, il ne raisonne, il ne calcule plus ; et dans le désordre où l’homme se trouve, il est capable se sacrifier lui-même à l’objet dont il n’était épris que parce qu’il y trouvait sa félicité. Voilà comme l’amitié sincère a quelquefois été portée jusqu’à vouloir périr pour un ami.\par
Nous nous attendrissons sur nous-mêmes lorsque nous mêlons nos larmes à celles d’un malheureux ; nous nous pleurons nous-mêmes lorsque nous pleurons sur les cendres d’un objet dans lequel nous n’avions placé notre affection que parce qu’il nous procurait de grands plaisirs.\par
Enfin, c’est à l’amour de la gloire qui rejaillira sur lui ou à la crainte de la honte qui retombera sur lui, que le héros s’immole et se dévoue dans les combats. Il ne fait alors que sacrifier sa vie au désir de mériter la considération et la gloire, dont l’idée allume son imagination et l’étourdit sur le danger, ou bien il se sacrifie à la crainte de vivre déshonoré, ce qui lui paraît le comble de l’infortune. C’est pour lui-même que le guerrier veut de l’estime et craint la honte ; c’est donc par amour pour lui-même qu’il expose ses jours et qu’il brave la mort. Dans la chaleur de son imagination, il ne songe pas que s’il vient à périr il ne recueillera pas les fruits de cet honneur dans lequel il s’est habitué à faire consister son bien-être.\par
Ainsi, ne blâmons point l’amour que tout homme a pour lui-même ; ce sentiment est naturel et nécessaire à sa conservation propre, à son utilité, à celle de la société. Un homme qui se haïrait lui-même ou qui serait indifférent sur son propre bonheur, serait un insensé peu disposé à faire du bien à ses associés. Un homme qui cesserait de s’aimer serait un malade à qui sa propre vie deviendrait incommode et qui ne s’intéresserait aucunement aux autres. Les mélancoliques qui se tuent sont des êtres de cette trempe, ainsi que les fanatiques qui, devenus les ennemis d’eux-mêmes, se séparent de la société et se rendent inutiles au monde.\par
Néanmoins, le solitaire et l’anachorète ne sont pas exempts d’intérêt ou d’amour pour eux-mêmes : leur haine pour le monde, pour ses plaisirs et pour les choses que les autres hommes désirent, est fondée sur l’espoir d’être un jour plus heureux en se privant durant leur vie des objets qui excitent les passions des autres. D’où l’on voit que c’est eux-mêmes qu’ils aiment en se rendant malheureux pour un temps.\par
Dans l’homme qui réfléchit, l’amour de soi est toujours accompagné d’affection pour les autres. En aimant les êtres avec lesquels il a des rapports, il ne fait que s’aimer plus efficacement lui-même puisqu’il aime les instruments de sa propre félicité. « Celui, dit Sénèque, qui s’aime bien lui-même, est l’ami de tous les autres\footnote{« Qui sibi amicus est, scito hune amicum omnibus esse. » Sénèque, {\itshape Épître VI}, in fine.}. » Il dit encore ailleurs « qu’il faut apprendre à l’homme comment il doit s’aimer, car il serait fou de douter qu’il ne s’aimât lui-même\footnote{« Modus ergo diligcndi præcipiendus est homini, id est quomodo se diligat aut prosit sibi : quin autem se diligat aut prosit sibi dubitare dementis est. — omne animal, simul ut ortum est, seipsum et omnes partes suas diligit. » Cicéron, {\itshape De Finibus}, livre II, chap. XI. Arrien dit que tous les actes des êtres animés, et même ceux de la divinité, partent de l’amour de soi. Voyez Arrien, livre I, chap. XIX. Cicéron reconnaît encore « que tous nos désirs et nos aversions, nos projets de toute espèce, ont pour mobile unique le plaisir ou la douleur ; d’où il suit que toutes les actions bonnes et louables n’ont pour objet qu’une vie commode et heureuse. » Voyez Cicéron, {\itshape De Finibus}, livre I, chap. 12. Avant tous ces auteurs, Aristote avait très bien combattu l’opinion de ceux qui de son temps, comme quelques personnes aujourd’hui, regardaient l’amour de soi, ou l’intérêt, comme un principe ignoble et vicieux. Voyez Aristote, {\itshape Éthique}, livre IX, chap. 8. D’où l’on voit que plusieurs philosophes de l’Antiquité ont très bien connu le vrai mobile des actions humaines ou le vrai principe de toute morale, dont ils se sont écartés faute de lui avoir donné toute l’étendue convenable.} ». Un être sociable ne peut en effet s’aimer véritablement qu’en intéressant ses semblables à son bonheur, ce qu’il ne peut effectuer qu’en leur faisant éprouver les bonnes dispositions de son cœur. C’est toujours pécher contre soi que de violer les devoirs qui nous lient aux autres.\par
Ainsi, loin de former le projet insensé d’éteindre dans le cœur de l’homme l’amour essentiel et naturel qu’il a pour lui-même, la morale doit s’en servir pour lui montrer l’intérêt qu’il a d’être bon, humain, sociable, fidèle à ses engagements. Loin de vouloir anéantir les passions inhérentes à sa nature, elle les dirigera vers la vertu, sans laquelle nul homme sur la terre ne peut jamais jouir d’un bonheur véritable. Cette morale dira donc à tout homme de s’aimer et lui indiquera les vrais moyens de contenter ce besoin qui le ramène à tout moment sur lui-même, en le faisant partager à ceux qui l’environnent. Les passions ainsi dirigées contribueront à son bien-être, soit quand il est isolé, soit quand il vit en société ; elles le rendront intéressant comme époux, comme père, comme ami, comme citoyen, comme souverain, comme sujet. Enfin, ses passions et ses intérêts, d’accord avec ceux de la société, le rendront lui-même heureux du bonheur des autres.\par
Celui chez qui l’amour de soi étouffe toute affection pour les autres est un être insociable, un insensé qui ne voit pas que tout homme, vivant avec d’autres hommes, est dans une impossibilité complète de travailler à son bonheur sans l’assistance des autres. Toutes nos passions aveugles, nos intérêts mal entendus, nos vices et nos défauts, nous séparent de la société. En indisposant nos associés contre nous, ils en font des ennemis peu favorables à nos désirs. Tous les méchants que l’on déteste vivent comme s’ils étaient seuls dans la société, le tyran qui l’opprime vit en tremblant au milieu de son peuple qui le hait. Le riche avare vit méprisé comme un être inutile, l’homme dont le cœur glacé ne s’échauffe pour personne n’a pas lieu de s’attendre qu’on s’intéresse à lui.\par
En un mot, il n’est point en morale de vérité plus claire que celle qui prouve que l’homme en société ne peut se rendre heureux sans le secours des autres.
\subsection[{Chapitre VII. De l’Utilité des Passions}]{Chapitre VII. De l’Utilité des Passions}
\noindent Plutarque compare les passions aux vents, sans lesquels un vaisseau ne peut point avancer. Rien n’est donc plus inutile que de déclamer contre les passions, rien de plus impraticable que le projet de les détruire. Le moraliste doit exposer les avantages de la vertu et les inconvénients du vice. La tâche du législateur est d’inviter, d’intéresser, de forcer même chacun, pour son propre intérêt, à contribuer à l’intérêt général.\par
Instruire les hommes, c’est leur indiquer ce qu’ils doivent aimer ou craindre, c’est exciter leurs passions pour les objets utiles, c’est leur apprendre à réprimer et à ne point irriter les désirs qui pourraient avoir des effets funestes, soit pour eux-mêmes, soit pour les autres. En opposant des passions à d’autres passions, la crainte à l’impétuosité des désirs déréglés, la haine et la colère aux actions nuisibles, des intérêts réels à des intérêts fictifs et imaginaires, un bien-être constant à des fantaisies du moment, on pourra se promettre de faire des passions un usage avantageux, on les dirigera vers l’utilité publique à laquelle dans la vie sociale l’utilité particulière de chaque homme se trouve nécessairement liée. Voilà comme les intérêts divers peuvent être combinés avec l’intérêt général.\par
Un homme dépourvu de passions ou de désirs, loin d’être un homme parfait, comme quelques penseurs l’ont prétendu, serait un être inutile à lui-même et aux autres, et dès lors peu fait pour la vie sociale. Un homme qui ne serait susceptible ni d’amour ni de haine, ni d’espérance ni de crainte, ni de plaisir ni de douleur, en un mot, le sage du stoïcisme, serait une masse inerte que l’on ne pourrait nullement mettre en action\phantomsection
\label{footnote1}\footnote{Quelqu’un, entendant parler des maximes d’Épictète, dit que c’était {\itshape un homme de bois}.}.\par
Comment modifier, façonner, élever un enfant qui, privé de passions, n’aurait aucun ressort et serait indifférent au plaisir et à la douleur, aux récompenses et aux châtiments qu’on lui proposerait ? Comment exciter au bien des êtres dépouillés de passions et d’intérêts, et pour lesquels il n’existe point de motifs propres à les faire agir ? Que pourrait faire un législateur d’une société d’hommes également insensibles à ses menaces et à ses récompenses, aux richesses et à l’indigence, à la gloire et à l’ignominie, à la louange et au blâme ?\par
La science du politique et du moraliste, dont les vues doivent être les mêmes, consiste à exciter, diriger et régler les passions des hommes de manière à les faire conspirer à leur bonheur mutuel. Il n’est aucune passion qui ne puisse être tournée vers le bien de la société et qui ne soit nécessaire à son maintien, à son bonheur.\par
La passion de l’amour, si justement décriée par ses ravages, est l’effet d’un besoin naturel ; elle est nécessaire à la conservation de notre espèce. Il ne s’agit donc que de régler l’amour de manière à ne point nuire, ni à celui qui l’éprouve, ni à l’être qui en est l’objet, ni à la société.\par
La colère et la haine, si funestes quelquefois par leurs effets terribles, étant contenues dans de justes bornes, sont des passions utiles et nécessaires pour écarter de nous et de la société les choses capables de nuire. La colère, l’indignation, la haine, sont des mouvements légitimes que la morale, la vertu, l’amour du bien public doivent exciter dans les cœurs honnêtes contre l’injustice et la méchanceté.\par
La passion du pouvoir, que l’on nomme {\itshape ambition} et que l’on est si souvent forcé de détester, est un sentiment naturel à l’homme qui veut être à portée de faire contribuer les autres à sa félicité propre. Ce sentiment est utile à la société lorsqu’il porte le citoyen à se rendre digne de commander et d’exercer le pouvoir par les talents qu’il acquiert.\par
La passion de la gloire, que l’on regarde souvent comme une vaine fumée, n’est que le désir d’être estimé des autres hommes. Ce désir est nécessaire à la société, dans le sein de laquelle il fait naître le courage, le sentiment de l’honneur, la bienfaisance, la générosité, et tous les talents qui contribuent soit au bien-être, soit aux plaisirs du genre humain.\par
Le désir des richesses n’est que le désir des moyens de subsister commodément et d’engager les autres à concourir à notre félicité particulière. Cette passion, bien dirigée, est la source de l’industrie, du travail, de l’activité nécessaire à la vie sociale.\par
La crainte, ce sentiment qui souvent fait des lâches, des âmes basses et serviles, est utile et nécessaire pour contenir toutes les passions dont les effets pourraient être fatals à nous-mêmes et aux autres. La crainte de nuire à notre conservation propre, à notre bonheur durable, est le frein naturel de tout être qui s’aime véritablement ; la crainte de déplaire aux autres est le lien de toute société, de principe de toute vertu ; enfin, la crainte des châtiments en impose souvent aux hommes les plus déraisonnables.\par
L’amour de nous-mêmes, que l’on nomme {\itshape orgueil} ou {\itshape amour-propre} et qui déplaît lorsqu’il déprime les autres, est un sentiment très louable quand il nous fait craindre de nous avilir par des actions basses et dignes de mépris.\par
L’envie, cette passion si commune et si vile, s’ennoblit quand, au lieu de nous faire lâchement haïr les grands hommes et les grands talents, elle nous porte à les imiter et à mériter, comme eux, l’estime de nos concitoyens ; elle se change pour lors en émulation louable.\par
Ainsi, n’écoutons plus les vaines déclamations d’une philosophie qui fait consister le bonheur et la vertu dans la privation totale des passions et des désirs. Que l’éducation sème dans les cœurs des passions utiles et à nous et aux autres, qu’elle empêche d’éclore ou qu’elle étouffe avec soin celles dont il résulterait du mal pour nous ou pour nos associés. Qu’elle excite l’activité nécessaire à la société, qu’elle comprime ou brise les ressorts dangereux. Qu’elle dirige les volontés particulières vers le bien général du tout, auquel le bien des membres est toujours attaché.\par
Enfin, que le gouvernement, toujours d’accord avec la morale, se serve des passions des hommes pour les faire vouloir et agir de la manière la plus conforme à leur véritable intérêt. L’homme de bien n’est pas celui qui n’a point de passion, c’est celui qui n’a que des passions conformes à son bonheur constant, qu’il ne peut séparer de celui des êtres faits pour concourir avec lui à sa propre félicité. La sagesse ne nous dit pas de n’aimer rien mais de n’aimer que ce qui est vraiment digne d’amour, de ne désirer que ce que nous sommes à portée d’obtenir, de ne vouloir que ce qui est capable de nous rendre solidement heureux. « Chaque homme, dit Cicéron, devrait se proposer uniquement de faire que ce qui est utile à lui-même devienne utile à tous\footnote{« Unum debet esse omnibus propositum, ut eadem sit utilitas uniuscuiusque et universorum. » Voyez Cicéron, {\itshape De Officiis}, livre I.}. »
\subsection[{Chapitre VIII. De la Volonté et des Actions}]{Chapitre VIII. De la Volonté et des Actions}
\noindent La volonté est dans l’homme une direction, une tendance, une disposition interne donnée par le désir d’obtenir les objets dans lesquels il voit de l’agrément ou de l’utilité, ou par la crainte de ceux qu’il juge contraires à son bien-être. Cette direction n’est déterminée que par l’idée d’un bien ou d’un mal attachée à l’objet qui excite le désir ou la crainte, l’affection ou l’aversion. Notre volonté est flottante, vague, indéterminée, tant que nous ne sommes pas assurés du bien ou du mal qui peut résulter de l’objet que nous désirons. Alors nous hésitons, nous nous trouvons, pour ainsi dire, placés dans une balance qui s’élève et s’abaisse alternativement, jusqu’à ce qu’un nouveau poids la fasse pencher d’un côté. Ces poids qui déterminent la volonté de l’homme sont les idées d’un intérêt, d’un bien-être ou d’un plaisir plus grand qui, comparées aux idées du mal ou d’un intérêt moins grand, finissent nécessairement par nous entraîner, par décider notre volonté, par nous diriger vers le but ou l’objet que nous jugeons le plus utile pour nous-mêmes.\par
Tant que nous ne connaissons pas suffisamment les qualités d’un objet, c’est-à-dire ses effets utiles ou nuisibles, nous sommes dans l’incertitude : nous nous sentons tantôt attirés et tantôt repoussés, nous délibérons. {\itshape Délibérer} sur un objet, c’est alternativement l’aimer pour les qualités utiles qu’on croit trouver en lui, et le haïr pour les qualités nuisibles qu’on lui suppose. Délibérer sur ses actions, c’est peser les avantages et les désavantages qui peuvent en résulter pour soi. Lorsque nous croyons être sûrs des effets de nos actions, nous ne balançons plus, notre volonté cesse d’être chancelante. Nous sommes dirigés ou déterminés dans notre choix par l’idée du bien-être attaché à l’objet sur lequel nous étions incertains ; nous agissons alors pour l’obtenir ou l’éviter.\par
Les {\itshape actions} sont les mouvements organiques produits par la volonté, déterminée par l’idée du bien ou du mal qui réside dans un objet. Toutes les actions d’un être qui cherche le plaisir et qui craint la douleur tendent à lui procurer la possession des objets qu’il croit utiles, ou à lui faire éviter ceux qu’il juge nuisibles.\par
Un exemple peut servir à expliquer cette théorie. Au moment où la faim me presse, ma vue est frappée par un fruit que l’expérience me fait connaître comme agréable. Cette vue fait naître mon désir, ma volonté est dirigée ou déterminée vers cet objet. Je ne balance point, parce que je suis assuré de la bonté de ce fruit. En conséquence, j’agis ou je produis les mouvements nécessaires pour me le procurer. Mes pieds s’avancent, je m’approche de l’arbre, j’étends les bras pour cueillir l’objet de mes désirs, et sans hésiter je le porte à ma bouche. Mais si j’ignore la nature du fruit qui s’est offert à ma vue, j’hésite, je balance, je le considère, je le flaire, je cherche à démêler ses qualités, je le goûte avec précaution. Quand le résultat de mon examen me fait connaître que le fruit est mauvais ou peut me nuire, la volonté excitée par la faim est anéantie par la crainte du danger. La volonté de me conserver contrebalance alors la volonté de me procurer une satisfaction passagère : je m’abstiens de manger ce fruit, je le rejette avec dédain.\par
On loue et l’on blâme les hommes pour les actions qui partent de leur volonté parce que leur volonté est susceptible d’être dirigée ou modifiée d’une manière conforme au bien de la société. Tout homme qui vit avec d’autres hommes est censé devoir être habitué, façonné, modifié, de manière à ne vouloir que ce qui peut plaire à ses associés et à ne point vouloir ce qui peut lui attirer leur ressentiment ou leur haine. D’un autre côté, l’homme qui cherche incessamment le bonheur ne doit vouloir que ce qui peut l’y conduire sûrement, et doit suspendre ses actions jusqu’à ce que l’expérience et l’examen lui aient fait connaître clairement ce qu’il est utile pour lui de vouloir et de faire. Tant que nous ignorons la nature des objets, notre intérêt nous ordonne de les considérer avec attention, afin de bien connaître s’ils sont vraiment utiles ou nuisibles et si les actions propres à nous les procurer ne sont point sujettes à des inconvénients. Un être raisonnable est celui qui dans toutes ses actions se sert des moyens les plus sûrs pour obtenir la fin qu’il se propose, et dont les volontés sont continuellement dirigées par la prudence et la réflexion.
\subsection[{Chapitre IX. De l’Expérience}]{Chapitre IX. De l’Expérience}
\noindent La morale, ainsi que toute autre science, ne peut être solidement établie que sur l’expérience. Toute sensation, tout mouvement agréable ou fâcheux qui s’excite dans nos organes est un fait. Par le plaisir ou la douleur qui se produisent en nous à l’occasion d’un objet qui nous remue, nous nous formons l’idée de cet objet, nous nous instruisons de sa nature par ses effets sur nous-mêmes, nous acquérons l’expérience, que l’on peut définir {\itshape la connaissance des causes par leurs effets sur les hommes}.\par
L’homme est susceptible d’expérience, c’est-à-dire [qu’]il est par sa nature capable de sentir, de se retracer ses sensations à l’aide de sa mémoire, de réfléchir ou de revenir sur les sensations et les idées qu’il a reçues, de les comparer entre elles et de connaître par là ce qu’il doit aimer ou craindre. L’expérience est la faculté de connaître les rapports ou la manière dont les êtres de la Nature agissent les uns sur les autres. En portant un charbon ardent sur de la poudre à canon, j’apprends que cette poudre s’enflamme avec explosion et qu’elle imprime un sentiment de douleur en moi si j’en approche de trop près ; par là j’acquiers une expérience, et l’idée de la poudre se présentera toujours à ma mémoire accompagnée d’inflammation, d’explosion et de douleur.\par
La morale, pour être sûre, ne doit être qu’une suite d’expériences faites sur les dispositions essentielles, les passions, les volontés, les actions des hommes et leurs effets. Avoir de l’expérience en morale, c’est connaître avec certitude les effets résultant de la conduite des hommes. Faute d’expérience, un enfant commet une action qui déplaît à son père : celui-ci le châtie. Par là l’enfant apprend à ne plus réitérer la même action, parce que la mémoire la lui représente comme devant être suivie d’un châtiment, c’est-à-dire d’une douleur.\par
Ce n’est qu’à force d’expériences que les hommes peuvent apprendre ce qu’ils doivent faire ou éviter. L’expérience seule peut nous montrer la vraie nature des objets, ceux que nous devons désirer ou craindre, les actions utiles ou nuisibles à nous-mêmes et aux autres. Sans expérience et sans réflexion l’on demeure dans une enfance perpétuelle. « Celui, dit un Arabe, qui fait des expériences, augmente sa science, mais celui qui est crédule augmente son ignorance\footnote{Voyez {\itshape Sentent. Arab., In Erpenii Grammatic. Arab.}}. »\par
Les hommes sont sujets à se tromper dans leurs expériences : la trop grande sensibilité, ainsi que la raideur de leurs organes, font que souvent ils sont incapables de se former des idées vraies, de se rappeler exactement les impressions qu’ils ont reçues, de prévoir les effets éloignés que leurs actions produiront sur eux. Un tempérament trop ardent, une imagination trop exaltée, des passions impétueuses, des désirs inconsidérés empêchent de juger sainement, troublent la mémoire et rendent l’expérience inutile et fautive. Un homme stupide est celui dont les sens sont engourdis, qui ne sent que faiblement, qui lie difficilement ses idées, qui saisit avec peine les rapports, qui manque de mémoire. Avec de telles dispositions, il est presque impossible d’acquérir de l’expérience ou de juger sainement des choses. D’un autre côté, l’homme d’esprit est souvent trop sensible, trop précipité, d’une imagination trop emportée. De là les erreurs et les fréquents écarts de l’imagination et du génie, dont l’effervescence nuit à la réflexion et par conséquent à l’exactitude des expériences. Enfin, le tumulte des passions, la dissipation, l’amour désordonné du plaisir, aussi bien que l’insensibilité, l’apathie, la stupidité, mettent des obstacles continuels au développement de la raison humaine, qui ne peut être que le fruit de l’expérience.\par
Il faut un tempérament justement balancé, il faut des organes sains, du jugement, de la réflexion, pour faire des expériences sûres. Être bien né, c’est avoir reçu de la Nature ou de l’art les dispositions propres à juger sainement des choses. Une main ébranlée par une action violente n’est capable de tracer qu’imparfaitement les caractères de l’écriture, qu’elle forme avec facilité et précision dès qu’elle est reposée.\par
Nos sens nous trompent ou nous font des rapports infidèles lorsque nous ne les appelons pas successivement à notre secours. Une tour carrée nous paraît ronde dans un certain éloignement, mais en s’approchant de plus près de cette tour, en la touchant, l’erreur de nos yeux se trouve rectifiée.\par
La première impression d’un objet me le fait envisager comme un bien désirable, mais l’expérience aidée par la réflexion m’apprend bientôt qu’il peut me nuire et que le plaisir momentané qu’il paraît me promettre sera tôt ou tard suivi de regrets et de peines.\par
La prévoyance est fondée sur l’expérience, qui m’enseigne que les mêmes causes doivent produire les mêmes effets. Celui qui a senti l’amertume d’un fruit s’en abstient par la suite, attendu qu’il prévoit qu’il produirait encore sur lui la même sensation. Voilà comme l’expérience, le jugement et la mémoire mettent l’homme à portée de pressentir l’avenir, c’est-à-dire de voir d’avance les effets que les objets dont il connaît la nature opéreront sur lui.
\subsection[{Chapitre X. De la Vérité}]{Chapitre X. De la Vérité}
\noindent L’expérience, accompagnée des circonstances qui la rendent sûre, nous découvre la {\itshape vérité}, qui n’est que la conformité des jugements que nous portons avec la nature des choses, c’est-à-dire avec les propriétés, les qualités, les effets immédiats ou éloignés des êtres qui agissent ou qui peuvent agir sur nous, que l’expérience nous fait ou connaître ou prévoir.\par
Quand je dis que le feu excite de la douleur, je dis une vérité ; c’est-à-dire je prononce un jugement conforme à la nature du feu, fondé sur l’expérience constante de tous les êtres sensibles. Quand je dis que l’intempérance et la débauche détruisent la santé, je dis une vérité, je porte un jugement confirmé par l’expérience journalière, qui prouve qu’une suite naturelle de ces vices est d’énerver le corps et de causer tôt ou tard une existence misérable. Quand je dis que la vertu est aimable, je juge d’une façon conforme à l’expérience constante de tous les habitants de la terre.\par
La vérité consiste à voir les choses telles qu’elle sont, à leur attribuer les qualités qu’elles possèdent réellement, à prévoir avec certitude leurs effets bons ou mauvais, à distinguer ce qui est utile, louable et désirable, de ce qui n’est que chimérique et apparent.\par
L’erreur est le fruit d’expériences mal faites, de jugements précipités, de l’inexpérience totale, que l’on appelle ignorance, du délire de l’imagination, du trouble de nos sens. En un mot, l’erreur est l’opposition de nos jugements avec la nature des choses. Je suis dans l’erreur lorsque je pense que des plaisirs déshonnêtes peuvent procurer le bonheur, parce que l’expérience, la réflexion, la prévoyance, auraient du me convaincre que ces plaisirs, suivis de longues peines, me rendront méprisables aux yeux de mes concitoyens.\par
Les préjugés sont des jugements destitués d’expériences suffisantes. Les individus, ainsi que les nations, sont les dupes d’une foule de préjugés dangereux qui les écartent sans cesse du bien-être vers lequel ils croient s’acheminer. Les opinions des peuples, leurs institutions, leurs usages et leurs lois souvent si contraires à la raison, sont dus à leur inexpérience, sont consacrés par l’habitude, se transmettent sans examen des pères aux enfants. Voilà comme les erreurs les plus nuisibles, les idées les plus fausses, les coutumes les plus dépravées et les plus opposées au bien des sociétés, les abus les plus criants se perpétuent parmi les hommes.\par
Faute de voir les choses sous leur vrai point de vue, les principes de la morale sont ignorés de la plupart des hommes. Nous les voyons guidés par des préjugés destructeurs, par des usages barbares, par des opinions fausses, par la routine aveugle, dont l’effet est de les tromper, de les empêcher de connaître leurs intérêts et les objets qu’ils doivent estimer ou mépriser. La vraie gloire, le véritable honneur, les devoirs les plus évidents, les vérités les plus frappantes sont totalement obscurcis par une foule d’erreurs qui forment un labyrinthe d’où l’esprit a peine à se tirer.\par
Quelle morale en effet que celle que l’on fonderait sur les préjugés, les opinions, les coutumes souvent abominables que l’on voit établies chez la plupart des peuples de la terre ! Presque partout la violence et la force constituent des droits.\par
Des intérêts frivoles rendent des peuples ennemis des autres peuples. L’homicide, les guerres, les duels, les cruautés, les adultères, la rapine, la mauvaise foi ne sont point des crimes aux yeux de bien des nations qui se disent civilisées. En un mot, à la vue de la conduite que la plupart des hommes tiennent entre eux, des spéculateurs ont cru que la morale n’avait aucuns principes sûrs, n’était qu’une pure chimère, et que ses devoirs dépendaient uniquement des caprices des législateurs et des conventions des hommes.\par
C’est à la vérité fondée sur l’expérience qu’il appartient de juger les hommes, leurs institutions, leur conduite et leurs mœurs. L’ignorance et l’erreur sont les sources du mal moral ; la vérité seule, en éclairant les mortels sur la nature des choses, peut un jour parvenir à les rendre meilleurs ou plus raisonnables.
\subsection[{Chapitre XI. De la Raison}]{Chapitre XI. De la Raison}
\noindent En morale, la raison est la connaissance de la vérité appliquée à la conduite de la vie, c’est la faculté de distinguer le bien du mal, l’utile du nuisible, les intérêts réels des intérêts apparents, et de se conduire en conséquence.\par
Quand on dit que l’{\itshape homme est un être raisonnable}, on ne veut point faire entendre par là qu’il apporte en naissant la connaissance de ce qui lui est avantageux ou nuisible. On veut seulement indiquer qu’il jouit de la faculté de sentir et de distinguer ce qui lui est favorable de ce qui lui est contraire, ce qu’il doit aimer et chercher de ce qu’il doit haïr et craindre, ce qui procure un bien durable de ce qui ne procure qu’un bien passager.\par
D’où l’on est forcé de conclure que la raison dans l’homme ne peut être que le fruit tardif de l’expérience, de la connaissance du vrai, de la réflexion, ce qui suppose, comme on a vu, une organisation bien constituée, un tempérament modéré, une imagination réglée, un cœur exempt de passions turbulentes. C’est de cette heureuse et rare combinaison de circonstances que résulte la raison éclairée, faite pour guider les hommes dans la conduite de la vie. « Il n’y a, dit Sénèque, que la science du bien et du mal qui porte l’esprit à sa perfection\footnote{« Una re consummatur animus, scientiâ bonorum ac malerum immutabilis. » Sénèque, {\itshape Épîtres}, 88, p. 389, tome 2, édition Varior.}. »\par
L’homme dans son enfance montre aussi peu de raison que les brutes. Que dis-je ! Bien moins capable de s’aider lui-même que la plupart des bêtes, sans le secours de ses parents il serait exposé à périr dès le moment de sa naissance. Ce n’est qu’à force d’expériences, qui se tracent plus ou moins facilement et durablement dans sa mémoire, qu’il apprend à se conserver, à connaître les objets, à distinguer ceux qui lui plaisent de ceux qui lui déplaisent, ceux qui peuvent lui faire du bien de ceux qui lui sont nuisibles. L’enfant poussé par le besoin de la faim porte naturellement à sa bouche tout ce qui lui tombe sous la main ; s’il éprouve alors par le sens du goût une impression agréable, cette expérience suffit pour qu’il attache l’idée du plaisir à l’objet qui a une fois fait naître en lui des sensations favorables. Dès lors il aime cet objet, il le désire, il s’y habitue, il tend les bras pour l’obtenir, il s’irrite et pleure lorsqu’on le lui refuse. Au contraire, si un objet a une fois excité dans sa bouche une sensation douloureuse ou désagréable, il apprend à le haïr, sa vue lui cause une répugnance, parce qu’il se rappelle l’impression fâcheuse qu’il a éprouvée : on ne peut le déterminer à le prendre sans l’affliger.\par
En naissant l’homme n’est qu’une masse inerte mais capable de sentir. Ce n’est que peu à peu qu’il apprend à connaître ce qu’il doit aimer ou craindre, ce qu’il doit vouloir ou ne point vouloir, les moyens qu’il faut employer pour obtenir les choses qu’il désire et pour éviter celles qui peuvent lui nuire ; ce n’est qu’avec le temps qu’il apprend à se mouvoir, à faire usage de ses membres, à marcher, à parler, à exprimer ses passions et ses volontés. En un mot, c’est avec beaucoup de lenteur que l’homme apprend à agir. Ce n’est qu’en réitérant des expériences que ses parents, sa nourrice, ses instituteurs lui aident à faire, qu’il acquiert l’habitude ou la facilité de se remuer, de s’énoncer, de parler, d’écrire, de penser comme les autres hommes\footnote{Les auteurs anciens, ainsi que les relations modernes, nous parlent de peuples tellement grossiers qu’il ignoraient encore l’usage de la parole. Diodore de Sicile attribue cette ignorance aux {\itshape Ichtyophages} qui, selon lui, n’avaient que quelques gestes pour se communiquer leurs idées. Garcilasso de la Véga dit la même chose de quelques peuplades voisines de l’Empire des Incas du Pérou.}.
\subsection[{Chapitre XII. De l’Habitude, de l’Instruction, de l’Éducation}]{Chapitre XII. De l’Habitude, de l’Instruction, de l’Éducation}
\noindent Élever, instruire un enfant, développer sa raison, c’est l’aider à faire des expériences, c’est lui communiquer celles que l’on a recueillies soi-même, c’est lui transmettre les idées, les notions, les opinions que l’on s’est formées. L’expérience supérieure ou la raison plus exercée des parents et des maîtres, est le fondement naturel de l’empire ou de l’autorité qu’ils exercent sur les enfants et les jeunes gens. Le respect que l’on montre dans la société aux vieillards, aux magistrats, aux souverains, suppose en eux plus d’expérience, de raison et de lumières que dans les autres hommes. La considération que l’on a pour les savants, les prêtres, les médecins, etc., n’est fondée que sur l’idée de l’expérience qu’ils ont acquise relativement aux objets dont ils se sont occupés. Le sage n’est estimable que parce qu’il jouit d’une raison plus éclairée que le vulgaire.\par
L’homme ne devient ce qu’il est qu’à l’aide de son expérience propre ou de celle que d’autres lui fournissent ; l’éducation parvient à le modifier. D’une masse qui ne fait que sentir, d’une machine presque inanimée, à l’aide de la culture il devient peu à peu un être expérimenté qui connaît la vérité et qui, suivant la façon dont sa matière première a été modifiée, montre par la suite plus ou moins de raison.\par
Dans l’enfance l’homme apprend non seulement à agir, mais encore à penser. Nos idées, nos opinions, nos affections, nos passions, nos intérêts, les notions que nous avons du bien et du mal, de l’honneur ou de la honte, du vice et de la vertu, nous sont infuses par l’éducation d’abord, et ensuite par la société. Si ces idées sont vraies, conformes à l’expérience et à la raison, nous devenons des êtres raisonnables, honnêtes, vertueux ; si ces idées sont fausses, notre esprit se remplit d’erreurs et de préjugés, nous devenons des animaux déraisonnables, incapables de procurer le bonheur soit à nous-mêmes, soit aux autres.\par
C’est encore dans l’enfance que nous contractons nos habitudes bonnes ou mauvaises, c’est-à-dire les façons d’agir utiles ou nuisibles à nous-mêmes et aux autres. L’{\itshape habitude}, en général, est une disposition dans nos organes causée par la fréquence des mêmes mouvements, d’où résulte la facilité de les produire. L’enfant apprend avec assez de peine à marcher, mais à force d’exercer ses jambes il en acquiert l’habitude, il marche avec facilité, il souffre quand on l’empêche de courir. Dans la tendre enfance l’homme ne produit que des cris ou des sons inarticulés, mais peu à peu sa langue exercée prononce des paroles et finit par les rendre avec rapidité.\par
Nos idées en morale ne sont donc que des effets de l’habitude\footnote{Le caractère, dit Hobbes, naît du tempérament, de l’expérience, de l’habitude, de la prospérité, de l’adversité, des réflexions, des discours, de l’exemple, des circonstances. Changez ces choses et le caractère changera. Les mœurs sont formées dès que l’habitude a passé dans le caractère.}. Les nourrices, les instituteurs, les parents ne communiquent à leurs élèves que les notions vraies ou fausses dont ils sont eux-mêmes imbus. Si leurs notions sont conformes à l’expérience, ces élèves auront des idées vraies et contracteront des habitudes convenables ; si leurs notions sont fausses, les sujets qu’ils auront abreuvés dès l’enfance dans la coupe de l’erreur, seront déraisonnables et méchants.\par
Les opinions des hommes ne sont que des associations vraies ou fausses des idées qui leur deviennent habituelles à force de se réitérer dans leurs cerveaux. Si dès l’enfance on ne montrait jamais l’idée de vertu que jointe à l’idée de plaisir, de bonheur, d’estime, de vénération, si des exemples funestes ne démentaient pas ensuite cette association des idées, il y a tout lieu de croire qu’un enfant instruit de cette manière deviendrait un homme de bien ou un citoyen estimable. Lorsque dès sa plus tendre jeunesse, l’homme, d’après les idées de ses parents ou de ses maîtres, s’habitue à joindre l’idée de bonheur à la parure, à l’argent, à la naissance, au pouvoir, est-il bien étonnant que l’on en fasse un homme vain, un avare, un orgueilleux, un ambitieux ?\par
La raison n’est que l’habitude contractée de juger sainement des choses et de démêler promptement ce qui est conforme ou contraire à notre félicité. Ce qu’on appelle {\itshape l’instinct moral} est la faculté de juger avec promptitude, sans hésiter, sans que la réflexion semble avoir part à notre jugement. Cet instinct ou cette promptitude à juger n’est due qu’à l’habitude acquise par l’exercice fréquent. Dans le physique, nous nous portons par instinct vers les objets propres à causer du plaisir à nos sens ; dans le moral, nous éprouvons un sentiment prompt d’estime, d’admiration, d’amour pour les actions vertueuses, et d’horreur pour les actions criminelles, dont nous connaissons au premier coup d’œil la tendance et la fin.\par
La promptitude avec laquelle cet {\itshape instinct} ou ce {\itshape tact} moral s’exerce par les personnes éclairées et vertueuses, a fait croire à plusieurs moralistes que cette faculté était inhérente à l’homme, qui l’apportait en naissant.\par
Cependant, il est le fruit de la réflexion, de l’habitude, de la culture, qui met à profit nos dispositions naturelles ou qui nous inspire les sentiments que nous devons avoir. Dans la morale comme dans les arts, le goût ou l’aptitude à bien juger des actions des hommes est une faculté acquise par l’exercice ; elle est nulle dans la plupart des hommes.\par
L’homme sans culture, le sauvage, l’homme du peuple, n’ont ni {\itshape l’instinct} ni le goût moral dont nous parlons ; au contraire, ils jugent communément très mal\footnote{« Interdum volgus rectum videt, est ubi peccat. » Horace, {\itshape Épîtres}, livre II, 1, vers 63.}. La multitude admire quelquefois les plus grands crimes, les injustices et les violences les plus criantes dans les héros et les conquérants, qu’elle proclame de grands hommes. Il n’y a que la réflexion et l’habitude qui nous apprennent à juger sainement et promptement en morale, ou à saisir d’un coup d’œil rapide les beautés et les difformités des actions humaines.\par
Ces réflexions nous font sentir l’importance d’une bonne éducation. Elle seule peut former des êtres raisonnables, vertueux par habitude, capables de se rendre heureux par eux-mêmes et de contribuer au bonheur des autres. L’homme ne doit être regardé comme intelligent et raisonnable que lorsqu’il prend les vrais moyens de se procurer son bonheur ; il est déraisonnable, imprudent, ignorant, dès qu’il suit une route opposée.\par
Les plaisirs de l’homme sont raisonnables lorsqu’ils contribuent à lui procurer un bien-être solide, qu’il doit toujours préférer à ses jouissances passagères. Les passions et les volontés de l’homme sont raisonnables lorsqu’elles se proposent des objets vraiment avantageux pour lui. Les actions de l’homme sont raisonnables toutes les fois qu’elles contribuent à lui faire obtenir des biens réels sans nuire aux autres. L’homme guidé par la raison ne veut, ne désire, ne fait que ce qui lui est vraiment utile ; il ne perd point de vue ce qu’il se doit à lui-même et ce qu’il doit aux êtres avec lesquels il vit en société. Toute la vie d’un être sociable doit être accompagnée d’une attention continuelle sur lui-même et sur les autres.
\subsection[{Chapitre XIII. De la Conscience}]{Chapitre XIII. De la Conscience}
\noindent Les expériences que nous faisons, les opinions vraies ou fausses que l’on nous donne ou que nous prenons, notre raison plus ou moins soigneusement cultivée, les habitudes que nous contractons, l’éducation que nous recevons, développent en nous un sentiment intérieur de plaisir ou de douleur que l’on nomme {\itshape conscience.} On peut la définir [comme] la connaissance des effets que nos actions produisent sur nos semblables et par contrecoup sur nous-mêmes.\par
Pour peu que l’on y réfléchisse, on reconnaîtra que, de même que l’{\itshape instinct} ou le sentiment moral dont on vient de parler, la conscience est une disposition acquise, et que c’est avec très peu de fondement que tant de moralistes l’ont regardée comme un sentiment {\itshape inné}, c’est-à-dire comme une qualité inhérente à notre nature. Quand on voudra s’entendre en morale, on sera forcé de convenir que le cœur de l’homme n’est qu’une table rase, plus ou moins disposée à recevoir les impressions que l’on peut y faire. « Les lois de la conscience, dit Montaigne, que nous croyons naître de la Nature, naissent de la coutume : chacun ayant en vénération interne les opinions et les mœurs approuvées et reçues autour de lui, ne peut s’en déprendre sans remords, ni s’y appliquer sans applaudissement. » Plutarque avait dit avant lui « que les mœurs et conditions sont qualités qui s’impriment par long trait de temps ; et qui dira que les vertus morales s’acquièrent aussi par accoutumance, à mon avis, il ne fourvoiera point\phantomsection
\label{footnote2}\footnote{Voyez {\itshape Essais} de Montaigne, livre I, chap. 22 ; et Plutarque, traité {\itshape Comment il faut nourrir les enfants}, traduction d’Amyot, voyez Plutarque, opp. tom. 2, page 2. F. et page 3. A., édition citée ub. sup.} ».\par
Comment un homme qui n’aurait point des idées nettes de la justice pourrait-il avoir la conscience d’avoir fait une action injuste ? Il faut avoir appris, soit par notre propre expérience, soit par celle qui nous est communiquée, les effets que les causes peuvent produire sur nous pour juger de ces causes, c’est-à-dire pour savoir si elles nous sont favorables ou nuisibles. Il faut des expériences et des réflexions encore plus multipliées pour découvrir et prévoir les influences de notre conduite sur d’autres ou pour pressentir ses conséquences souvent très éloignées.\par
Une conscience éclairée est le guide de l’homme moral ; elle ne peut être le fruit que d’une grande expérience, d’une connaissance parfaite de la vérité, d’une raison cultivée, d’une éducation qui ait convenablement modifié un tempérament propre à recevoir la culture qu’on a pu lui donner. Une conscience de cette trempe, loin d’être dans l’homme l’effet d’un {\itshape sens moral} inhérent à sa nature, loin d’être commune à tous les êtres de notre espèce, est infiniment rare et ne se trouve que dans un petit nombre d’hommes choisis, bien nés, pourvus d’une imagination vive ou d’un cœur très sensible et convenablement modifié.\par
Pour peu que l’on regarde autour de soi, l’on reconnaîtra ces vérités, on trouvera que très peu de gens sont à portée de faire des expériences et des réflexions nécessaires à la conduite de la vie. Très peu de gens ont le calme et le sang-froid qui rendent capable de penser et de prévoir les conséquences de leurs actions. Enfin, la conscience de la plupart des hommes est dépravée par les préjugés, les exemples, les idées fausses, les institutions déraisonnables qu’ils rencontrent dans la société.\par
Dans le plus grand nombre des hommes, on ne trouve qu’une conscience {\itshape erronée}, c’est-à-dire qui juge d’une façon peu conforme à la nature des choses ou à la vérité. Cela vient des opinions fausses que l’on s’est formées ou que l’on a reçues des autres, qui font attacher l’idée de bien à des actions que l’on trouverait très nuisibles si on les avait plus mûrement examinées. Beaucoup de gens font le mal, et commettent même des crimes en sûreté de conscience, parce que leur conscience est faussée par des préjugés.\par
Il n’est point de vice qui ne perde la difformité de ses traits quand il est approuvé par la société où nous vivons {\itshape ;} le crime lui-même s’ennoblit par le nombre et l’autorité des coupables. Personne ne rougit de l’adultère ou de la dissolution des mœurs chez un peuple corrompu. Personne ne rougit d’être bas à le cour. Le soldat n’est pas honteux de ses rapines et de ses forfaits {\itshape ;} il en fera même trophée devant ses camarades, qu’il connaît disposés à faire comme lui. Pour peu qu’on ouvre les yeux, l’on trouve des hommes très injustes, très méchants et très inhumains, et qui pourtant ne se reprochent ni leurs injustices fréquentes, qu’ils prennent souvent pour des actions légitimes ou des droits, ni leurs cruautés, qu’ils regardent comme les effets d’un courage louable, comme des devoirs. Nous voyons des riches à qui leur conscience ne dit rien pour avoir acquis une fortune immense au dépens de leurs concitoyens. Les voyageurs nous montrent des sauvages qui se croient obligés de faire mourir leurs pères lorsque la décrépitude les a rendus inutiles. Nous trouvons des zélés que leur conscience, aveuglée par des idées fausses de vertu, sollicite à exterminer sans remords et sans pitié ceux qui n’ont pas les mêmes opinions qu’eux. En un mot, il est des nations tellement viciées que la conscience ne reproche rien à des hommes qui se permettent des rapines, des homicides, des duels, des séductions, etc., parce que ces crimes et ces vices sont approuvés ou tolérés par l’opinion générale ou ne sont pas réprimés par des lois. Dès lors chacun s’y livre sans honte et sans remords. Ces excès ne sont évités que par quelques hommes plus modérés, plus timides, plus prudents que les autres.\par
La honte est un sentiment douloureux excité en nous par l’idée du mépris que nous savons avoir encouru.\par
Le remords est la crainte que produit en nous l’idée que nos actions sont capables de nous attirer la haine ou le ressentiment des autres.\par
Le repentir est une douleur interne d’avoir fait quelque chose dont nous envisageons les conséquences désagréables ou dangereuses pour nous-mêmes.\par
Les hommes n’ont communément ni honte, ni remords, ni repentir des actions qu’ils voient autorisées par l’exemple, tolérées ou permises par les lois, pratiquées par le grand nombre. Ces sentiments ne s’élèvent en eux que lorsqu’ils s’aperçoivent que ces actions sont universellement blâmées ou peuvent leur attirer des châtiments. Un Spartiate ne rougissait pas d’un larcin ou d’un vol adroit, qu’il voyait autorisé par les lois de son pays. Un despote, continuellement applaudi par ses flatteurs, n’a point de honte du mal qu’il fait à ses sujets. Un traitant ne rougit guère d’une fortune acquise injustement sous l’autorité du prince. Un duelliste ne se repent pas d’un assassinat qui l’honore souvent aux yeux de ses concitoyens. Un fanatique s’applaudit des ravages et des troubles que son zèle produit dans la société.\par
Il n’y a que des réflexions profondes et suivies sur les rapports immuables et les devoirs de la morale qui puissent éclairer la conscience et nous montrer ce que nous devons éviter ou faire, indépendamment des notions fausses que nous trouvons établies. La conscience est nulle, ou du moins elle se fait très faiblement, très passagèrement entendre dans les sociétés trop nombreuses, où les hommes ne sont point assez remarqués, où les êtres les plus méchants se perdent dans la foule. Voilà pourquoi les grandes villes deviennent communément les rendez-vous des fripons, qui s’y rendent des campagnes ou des provinces. Les remords sont bientôt évaporés, et la honte disparaît dans le tumulte des passions, dans le tourbillon des plaisirs, dans la dissipation continuelle. L’étourderie, la légèreté, la frivolité rendent souvent les hommes aussi dangereux que la méchanceté la plus noire. La conscience de l’homme léger ne lui reproche rien, ou du moins sa voix est bientôt étouffée chez celui qui voltige sans cesse, qui ne pèse rien et qui jamais n’a l’attention nécessaire pour prévoir les suites de ses actions. Tout homme qui ne réfléchit point n’a pas le temps de se juger. Dans les méchants confirmés, les coups réitérés de la conscience produisent à la longue un endurcissement que la morale est dans l’impossibilité de détruire.\par
La conscience ne parle qu’à ceux qui rentrent en eux-mêmes, qui raisonnent leurs actions et dans lesquels une éducation convenable a fait naître le désir, l’intérêt de plaire et la crainte habituelle de se faire mépriser ou haïr. Un être ainsi modifié devient capable de se juger ; il se condamne quand il a commis quelque action qu’il sait pouvoir altérer les sentiments qu’il voudrait constamment exciter dans ceux dont l’estime et la tendresse sont nécessaire à son bien-être. Il éprouve de la honte, des remords, du repentir, toutes les fois qu’il a mal fait ; il s’observe, il se corrige par la crainte d’éprouver encore par la suite ces sentiments douloureux qui le forcent souvent à se détester lui-même, parce qu’il se voit alors des mêmes yeux qu’il est vu par les autres.\par
D’où l’on voit que la conscience suppose une imagination qui nous peigne d’une façon vive et marquée les sentiments que nous excitons dans les autres. Un homme sans imagination ne se représente que peu ou point ces impressions ou sentiments ; il ne se met point en leur place. Il est très difficile de faire un homme de bien d’un stupide à qui l’imagination ne dit rien, ainsi que d’un insensé que cette imagination tient dans une ivresse continuelle.\par
Tout nous prouve donc que la conscience, loin d’être une qualité innée ou inhérente à la nature humaine, ne peut être le fruit que de l’expérience, de l’imagination guidée par la raison, de l’habitude de se replier sur soi, de l’attention sur ses actions, de la prévoyance de leurs influences sur les autres et de leur réaction sur nous-mêmes.\par
La bonne conscience est la récompense de la vertu ; elle consiste dans l’assurance que nos actions doivent nous procurer les applaudissements, l’estime, l’attachement des êtres avec qui nous vivons. Nous avons droit d’être contents de nous lorsque nous avons la certitude que les autres en sont ou doivent en être contents. Voilà ce qui constitue la vraie béatitude, le repos de la bonne conscience, la tranquillité de l’âme, la félicité durable, que l’homme désire sans cesse et vers laquelle la morale doit le guider. Ce n’est que dans une bonne conscience que consiste le {\itshape souverain bien} ; {\itshape et la vertu seule est capable de le procurer}.
\subsection[{Chapitre XIV. Des Effets de la Conscience en Morale}]{Chapitre XIV. Des Effets de la Conscience en Morale}
\noindent Par une loi constante de la Nature, le méchant ne peut jamais jouir d’un bonheur pur dans le monde\footnote{« Nemo malus felix. » Juvénal, {\itshape Satires}, IV, vers 8.}. Ses richesses, son pouvoir, ne le garantissent pas contre lui-même. Dans les moments lucides que ses passions lui laissent, s’il rentre dans son intérieur, c’est pour essuyer les reproches d’une conscience troublée par les peintures affreuses que l’imagination lui présente. C’est ainsi que l’assassin, durant la nuit, même éveillé, croit voir l’ombre plaintive de ceux qu’il a cruellement égorgés, il voit les regards plein d’horreur du public irrité qui crie à la vengeance, il voit des juges sévères qui prononcent son arrêt ; enfin, il voit les apprêts du supplice qu’il reconnaît avoir très justement mérité. Ce spectacle imaginaire est quelquefois si cruel pour des esprits doués d’une imagination très forte, que l’on a vu des coupables s’offrir d’eux-mêmes aux coups de la justice et chercher dans les tourments et dans la mort un asile contre les remords dont ils se sentaient incessamment agités. Tels sont les terribles effets du désespoir dans quelques êtres que l’horreur de leurs forfaits met dans l’impuissance de se réconcilier avec eux-mêmes.\par
On se tromperait néanmoins si l’on croyait que la conscience agisse d’une façon si puissante sur tous les coupables. Elle ne dit presque rien aux esprits engourdis, elle ne parle qu’à la dérobée à des êtres frivoles et dissipés, elle se tait entièrement dans l’orage des passions, elle s’oppose vainement aux penchants de l’habitude : celle-ci devient un besoin impérieux qui rend sourd à ses cris.\par
Ne soyons pas étonnés si tant de gens dans le monde commettent le mal sans y songer, persistent jusqu’au tombeau dans des vices et des désordres qu’ils se reprochent rarement, et ne s’embarrassent guère du soin de réparer les injustices qu’ils ont fait éprouver au autres. On ne répare le mal que lorsque la conscience tourmente assidûment. La continuité des blessures qu’elle nous fait nous force non seulement au repentir, mais encore à détruire, autant qu’il est en nous, le mal dont l’idée nous assiège et qui nous a dû rendre odieux pour les êtres avec lesquels nous vivons. En réparant le mal, tout homme se propose de se remettre bien avec lui-même et avec les autres. Il tâche alors de bannir de son esprit les images hideuses dont il est infesté, il s’efforce d’effacer de l’esprit des autres les impressions défavorables que sa conduite a dû nécessairement y produire.\par
Il est des vices, des fautes, des crimes mêmes qui se réparent. Une injustice faite à quelqu’un se répare en lui rendant justice, en le dédommageant d’une façon généreuse du tort qu’on a pu lui causer. La restitution répare le crime du vol. Une déclaration solennelle peut réparer les injures faites à la réputation d’un autre. Des marques de soumission et de repentir peuvent désarmer le ressentiment produit par une offense. Le cœur de l’homme semble s’épanouir toutes les fois qu’il a réparé le mal dont l’idée le comprime et le flétrit.\par
Rien de plus rare pourtant qu’une réparation complète, c’est-à-dire capable d’anéantir en nous-mêmes les cicatrices de la conscience, et dans les autres le souvenir du mal que nous leur avons fait endurer.\par
L’homme est toujours forcé d’éprouver de la douleur, un sentiment secret de mépris pour lui-même, lorsqu’il se rappelle qu’il s’est rendu haïssable ou méprisable aux yeux des êtres de son espèce. Ceux-ci, de leur côté, ont de la peine à mettre totalement en oubli des actions qui les ont cruellement affligées.\par
D’un autre côté, la réparation des torts paraît toujours infiniment coûter, soit à la vanité, soit à la cupidité des hommes. Elle suppose une grandeur d’âme, un courage, dont les méchants, sans un changement total, ne sont guère capables. Voilà pourquoi tant de coupables se repentent de leur conduite, paraissent y renoncer, mais consentent rarement à réparer le mal dont ils sont les auteurs. Ces regrets infructueux, ces sentiments de justice avortés, sont dus soit à l’ignorance, soit au manque de force, soit à la faiblesse des aiguillons de la conscience, qui ne tourmentent pas assez pour qu’on cherche à s’en débarrasser tout à fait.\par
La plupart des hommes, quand ils ne sont pas confirmés dans le vice et le crime, passent leur vie à lutter contre eux-mêmes, à se faire des reproches, puis à chercher des sophismes propres à rendormir leur conscience toutes les fois qu’elle s’éveille pour les importuner.\par
Les hommes devraient trembler s’ils songeaient aux suites inévitables de leurs passions pour eux-mêmes. Par un juste châtiment de la Nature, il est des crimes qui ne peuvent aucunement se réparer. Comment rendre à la vie un ami fidèle que le délire de la colère a fait périr dans un duel ? Comment un tyran dont les excès ont rendu tout un peuple malheureux pour des siècles, pourra-t-il se réconcilier avec lui-même ? Comment calmer les remords d’un conquérant lorsque son imagination vient à lui faire entendre les cris des nations désolées ? Comment apaiser la conscience d’un ministre dont les conseils perfides ont anéanti le bonheur de ses concitoyens ? Est-il quelque moyen de faire rentrer la paix dans le cœur du juge dont l’ignorance ou l’iniquité ont fait périr l’innocent ? Enfin, comment rassurer l’esprit de celui qui s’est engraissé de la substance du pauvre, de la veuve et de l’orphelin ?\par
Des hommes de cette trempe n’entendent guère le cri de la conscience. Chez eux elle est perpétuellement étouffée par le tumulte des affaires, des plaisirs bruyants, du vice effronté, des applaudissements serviles et par les consolations perfides des imposteurs dont ils sont entourés. Quand par hasard la conscience élève en eux sa voix, quand leur imagination alarmée leur peint les effets vastes et souvent irréparables de leurs passions, on la tranquillise communément par des remèdes imaginaires ; la superstition se charge d’expier tous leurs crimes. À l’aide de quelques pratiques, elle leur fournit les moyens d’apaiser les mânes de ceux que leur ambition, leur cupidité, leurs vengeances ont immolés. Dès lors les plus grands criminels se croient lavés de leurs souillures ; mais bientôt ils retomberont dans les crimes dont il est si facile d’écarter les remords. Voilà comment tout contribue à soulager la conscience de ceux dont la conduite influe de la façon la plus cruelle sur le bien-être et les mœurs des nations !\par
La morale fondée sur la Nature ne possède aucune recette pour guérir les plaies invétérées de la conscience de ceux que l’habitude affermit dans le crime. À ses yeux le repentir stérile ne peut rien réparer ; elle ne croit pas que de vains regrets suffisent pour tranquilliser le méchant qui persiste dans ses iniquités. Elle le condamne à gémir jusqu’à la mort sous le fouet des furies, elle veut que ses blessures ne cessent point de saigner, elle veut qu’au défaut des châtiments que la tyrannie ne craint point de la part des hommes, elle se punisse elle-même. C’est une cruauté, une trahison de calmer les remords de ceux qui font le malheur de la terre. Qu’ils éprouvent, s’il se peut, tous les tourments de la honte, de la terreur et du mépris d’eux-mêmes, jusqu’à ce qu’ils fassent cesser les infortunes qu’ils font éclore. La seule expiation que la morale puisse fournir aux criminels, c’est de rompre avec le crime. C’est en faisant de très grands biens aux hommes qu’on peut leur faire oublier les peines qu’on leur a fait éprouver, c’est en reconnaissant ses égarements qu’on apprend à s’en corriger, c’est en s’occupant du bonheur de ses semblables que l’on peut soulager la conscience toutes les fois qu’elle reproche les ravages qu’une conduite criminelle a pu causer.\par
Une conscience toujours sereine et sans nuages est une récompense qui n’appartient qu’à l’innocence. La conscience du méchant ne peut lui montrer que des plaies effrayantes, la conscience du vicieux désabusé lui montre des cicatrices, la conscience de l’homme de bien ne lui annonce qu’une santé constante. Porter les hommes à établir l’ordre et la paix en eux-mêmes par le contentement qu’ils procurent aux autres, voilà le grand objet que la morale doit se proposer.
\section[{Section II. Devoirs de l’Homme dans l’état de nature et dans l’état de société. Des Vertus sociales}]{Section II. Devoirs de l’Homme dans l’état de nature et dans l’état de société. Des Vertus sociales}\renewcommand{\leftmark}{Section II. Devoirs de l’Homme dans l’état de nature et dans l’état de société. Des Vertus sociales}

\subsection[{Chapitre I. Devoirs de l’Homme isolé ou dans l’état de nature}]{Chapitre I. Devoirs de l’Homme isolé ou dans l’état de nature}
\noindent L’homme peut être considéré sous deux points de vue généraux : comme seul, ou comme vivant avec d’autres hommes avec lesquels il a des rapports. Les moralistes et philosophes ont appelé état de nature la position de l’homme isolé, c’est-à-dire sans avoir égard à ses rapports avec les êtres de son espèce. Quoique l’homme ne se trouve point, ou du moins rarement, dans cet état abstrait, lorsqu’il se trouve seul, dégagé de toute liaison avec les autres, incapable d’influer sur eux par ses actions et d’éprouver les effets des leurs, il ne laisse pas d’être soumis à des devoirs envers lui-même.\par
Les devoirs, comme on l’a dit plus haut, sont les moyens nécessaires pour obtenir la fin qu’on se propose. L’homme isolé ou dans l’état de nature a sans doute une fin, qui est de se conserver et de rendre son existence heureuse ; l’homme isolé étant un être sensible, c’est-à-dire capable d’éprouver des plaisirs et des peines, sa nature le force d’aimer les uns et de craindre les autres. Il a des désirs, des craintes, des passions, des volontés ; il peut agir, faire des expériences, et quelques faibles que soient les connaissances qu’il acquiert dans cet état d’abandon, il est à portée de recueillir assez d’expériences pour régler sa conduite dans sa vie solitaire.\par
Un sauvage, s’il vivait tout seul, ou un homme que le naufrage aurait jeté dans une île déserte, voulant se conserver, sont obligés d’en prendre les moyens. Conséquemment, ils s’occuperont du soin de se nourrir, ils mettront de la différence entre les fruits doux et les fruits amers que leur séjour produit, ils auront soin de s’abstenir des aliments qui leur auront causé des douleurs, des maladies, ils s’en tiendront à ceux que l’expérience leur aura montré comme incapables de nuire à leur santé. Sous peine d’être punis de leur imprudence, ils résisteront à la tentation de manger les choses qui, après leur avoir fourni des sensations délectables, auront produit quelque dérangement fâcheux dans leur machine.\par
On voit donc que l’homme, dans quelque position qu’il se trouve, est soumis à des devoirs, c’est-à-dire se voit obligé de prendre les voies nécessaires pour obtenir le bien-être qu’il désire et pour écarter le mal que sa nature lui fait craindre.\par
Lorsqu’un homme vit tout seul, ses actions ne peuvent influer sur les autres, mais elles influent sur lui-même. Un être sensible, intelligent, raisonnable, ne peut se perdre de vue ; lors même qu’il n’a pas de témoins de sa conduite, il est son propre témoin, il a la conscience de se faire du bien ou du mal, il éprouve des regrets et des remords lorsqu’il sait qu’il s’est attiré par son imprudence des maux qu’il aurait pu éviter s’il eût consulté l’expérience et la raison.\par
La conscience, dans l’homme isolé, est la connaissance acquise par l’expérience des effets que ses actions peuvent produire sur lui-même. La conscience dans l’homme en société est, comme on l’a dit ailleurs, la connaissance des effets que ses actions doivent produire sur les autres et, par contrecoup, sur lui.\par
La honte dans l’homme isolé est le mépris de lui-même, excité par l’idée de sa déraison et de sa propre faiblesse. Le remords est en lui l’idée du châtiment que la Nature réserve à sa conduite insensée.\par
En réfléchissant sur ce qui se passe en nous lorsque nous sommes tous seuls, chacun peut se convaincre que l’homme isolé est forcé de se juger lui-même, de se repentir de ses passions et de ses actions inconsidérées lorsqu’elles ont pour lui des conséquences fâcheuses, de rougir de ses vices et de ses faiblesses ; en un mot, de se condamner d’avoir manqué à ce qu’il se devait à lui-même. Quoique tout seul, un être intelligent doit aimer l’ordre et haïr le désordre dont le théâtre se trouve au-dedans de lui. Il doit être inquiet toutes les fois que ses fonctions organiques sont troublées. Il doit éprouver des sentiments de crainte, il se dépite contre lui-même quand il soupçonne que ses forces et ses facultés ne sont pas capables de lui fournir les biens qu’il se propose ou d’écarter les maux dont il est menacé.\par
D’un autre côté, l’homme seul s’applaudit quand tout chez lui se passe dans l’ordre, quand ses facultés le servent à son gré, quand ses forces, son adresse, son industrie répondent à ses vues ou le mettent en état d’obtenir le bien-être et de repousser les dangers qui pourraient se présenter.\par
Ces réflexions nous prouvent clairement que l’homme considéré comme isolé ou, si l’on veut, dans l’état de nature, doit être raisonnable, consulter l’expérience, suspendre les actions dont les effets lui paraissent incertains, se refuser aux plaisirs suivis de peines, réprimer ses passions désordonnées. Quand bien même il serait tout seul au monde, cette solitude absolue ne le dispenserait aucunement de vivre d’une façon conforme à sa nature. Les qualités que l’on nomme force, prudence, modération, tempérance, sont aussi nécessaires à l’homme seul qu’à l’homme en société. En refusant de se soumettre à ces devoirs, l’homme isolé s’en trouvera puni, il se verra languissant et malade, il sera dans l’incapacité de jouir des plaisirs qu’il désire, il se dégoûtera de son être, il n’aura qu’une existence incommode dont il sera forcé d’accuser sa propre folie. Vivant dans des inquiétudes continuelles, la vie ne sera pour lui qu’un fardeau difficile à supporter.\par
Quoique l’état de nature, ou de l’homme totalement privé de rapports avec ses semblables, soit purement idéal, cependant chacun de nous se trouve souvent pour quelque temps dans une solitude complète durant laquelle il n’a d’autre témoin que lui-même. C’est alors qu’il peut appliquer à sa conduite les principes qui viennent d’être posés. Ils lui apprendront à se respecter et se craindre, à contenir ses passions, à ne point se permettre des actions dont il aurait lieu de se repentir, à ne pas même s’abandonner à des pensées déshonnêtes qui pourraient enflammer son imagination, en un mot, à s’abstenir de ce qui pourrait l’obliger de rougir à ses propres yeux de son imprudence ou de sa faiblesse.
\subsection[{Chapitre II. De la Société, des Devoirs de l’Homme social}]{Chapitre II. De la Société, des Devoirs de l’Homme social}
\noindent Ce n’est que par abstraction que l’homme peut être envisagé dans un état de solitude ou privé de tous rapports avec les êtres de son espèce. Ce qu’on appelle l’{\itshape état de nature} serait un état contraire à sa nature, c’est-à-dire opposé à la tendance des facultés de l’homme, nuisible à sa conservation, opposé au bien-être qu’il est de sa nature de désirer constamment. Tout homme est le fruit d’une association formée par l’union de son père et de sa mère, sans le secours desquels il n’eût jamais pu se conserver. Né dans la société, entouré d’êtres utiles et nécessaires à sa conservation, à ses plaisirs, à son bonheur, il serait contre sa nature de vouloir renoncer à un état dont il éprouve à chaque instant le besoin et dont il ne pourrait se passer sans se rendre malheureux.\par
Quand on dit que l’homme est un être {\itshape sociable}, on indique par là que sa nature, ses besoins, ses désirs, ses habitudes, l’obligent de vivre en société avec des êtres semblables à lui, afin de se garantir par leurs secours des maux qu’il craint et de se procurer les biens nécessaires à sa propre félicité.\par
Une société est l’assemblage de plusieurs êtres de l’espèce humaine réunis dans la vue de travailler de concert à leur bonheur mutuel. Toute société suppose invariablement ce but ; il serait contraire à la Nature que des êtres animés sans cesse du désir de se conserver et de se rendre heureux se rapprochassent ou s’unissent les uns aux autres pour travailler à leur destruction ou à leur malheur réciproque. Dès que deux êtres s’associent, on doit conclure qu’ils ont besoin l’un de l’autre pour obtenir quelque bien qu’ils désirent en commun. Ainsi, le bonheur commun des associés est le but nécessaire de toute société composée d’êtres intelligents et raisonnables.\par
Le genre humain dans son ensemble n’est qu’une vaste société composée de tous les êtres de l’espèce humaine. Les différentes nations ne doivent être envisagées que comme des individus de cette société générale. Les peuples divers que nous voyons sur notre globe sont des sociétés particulières, distinguées des autres par les noms des pays qu’elles habitent ; si elles avaient plus de raison, au lieu de se combattre et de se détruire, elles devraient conspirer à se rendre réciproquement heureuses. Dans chaque nation, une cité ou une ville forme une société particulière composée d’un certain nombre de familles et de citoyens intéressés également et au bien-être de cette association particulière, et à la conservation de la nation dont ils font partie. Une famille est une société plus particulière encore, composée d’un nombre plus ou moins considérable d’individus descendus de la même souche et distingués par le nom de ceux qui ont une origine différente. Le mariage est une société formée par l’homme et la femme pour travailler à leurs besoins et à leur bonheur mutuel. L’amitié est une association de plusieurs hommes qui se jugent capables de contribuer à leur félicité réciproque. Les réunions durables ou passagères de ceux qui s’associent pour quelques entreprises, pour le commerce, etc., n’ont et ne peuvent avoir pour but que de mettre leurs forces en commun afin de se procurer des avantages communs.\par
En un mot, aussitôt que plusieurs individus se rassemblent dans la vue d’obtenir une fin commune, ils forment une société. Les associations des différents peuples et de leurs chefs se nomment {\itshape alliances} ; elles ont pour objet leur défense, leur conservation, leurs intérêts réciproques ; enfin, des avantages que seuls ils ne pourraient se procurer.\par
La connaissance des devoirs de l’homme envers lui-même le conduit directement à la découverte de ce qu’il doit à ses semblables, ses associés. Quelle que soit la variété qui se trouve entre les individus dont le genre humain est composé, tous s’accordent, comme on a vu, à chercher le plaisir, à fuir la douleur. La moindre réflexion devrait donc apprendre à chacun d’eux ce qu’il doit à des êtres organisés, conformés, sensibles comme lui, dont l’assistance, l’affection, l’estime, la bienveillance sont nécessaires à sa propre félicité dans tous les moments de sa vie. Ainsi, chaque homme en société devrait se dire à lui-même : « Je suis homme, et les êtres qui m’entourent sont des êtres comme moi. Je suis sensible, et tout me prouve que les autres sont, comme moi, susceptibles de sentir le plaisir et la douleur. Je cherche l’un et je crains l’autre, donc des êtres semblables à moi éprouvent les mêmes désirs et les mêmes craintes. Je hais ceux qui me font du mal ou qui mettent des obstacles à mon bonheur, donc je deviens un objet désagréable pour tous ceux dont mes volontés ou mes actions contrarient les souhaits. J’aime ceux qui contribuent à ma propre félicité, j’estime ceux qui me procurent une existence agréable ; je suis prêt à tout faire pour eux. Donc, pour être chéri, estimé, considéré par des êtres qui me ressemblent, je dois contribuer à leur bien-être, à leur utilité. »\par
C’est sur des réflexions si simples, si naturelles, que toute morale doit se fonder. Que l’homme considère ce qu’il est, ce qu’il désire, et il trouvera que la Nature lui indique ce qu’il doit faire pour mériter l’affection des autres, et que cette Nature le porte à la vertu.
\subsection[{Chapitre III. De la Vertu en général}]{Chapitre III. De la Vertu en général}
\noindent La vertu en général est une disposition ou volonté habituelle et permanente de contribuer à la félicité constante des êtres avec lesquels nous vivons en société. Cette disposition ne peut être solidement fondée que sur l’expérience, la réflexion, la vérité, à l’aide desquelles nous connaissons et nos vrais intérêts, et les intérêts de ceux avec qui nous avons des rapports. Sans expériences vraies, nous agissons au hasard et sans règle, nous confondons le bien et le mal, nous pouvons nuire à nous-mêmes et aux autres, même en croyant faire du bien. La vertu ne consiste pas dans des mouvements passagers qui nous portent au bien mais dans des dispositions solides et permanentes\phantomsection
\label{footnote3}\footnote{« Je trouve, dit Montaigne, qu’il y a bien à dire entre les beautés et saillies de l’âme, ou une résolue et constante habitude. » Voyez {\itshape Essais}, livre II, chap. 29.}. Procurer aux hommes des plaisirs frivoles et passagers mais bientôt suivis de regrets ou de peines durables, ce n’est point être vertueux. Il n’y a point de vertu à favoriser les hommes dans leurs vices, leurs préjugés, leurs opinions fausses, leurs penchants déréglés. La vertu doit être éclairée et se proposer le bien durable des êtres de l’espèce humaine. La vertu doit être aimée parce qu’elle est utile à la société et à chacun de ses membres. Ce qui est vraiment utile est ce qui procure en tout temps la plus grande forme de bonheur. Cette disposition que l’on nomme {\itshape vertu} doit être habituelle ou permanente dans l’homme. Un homme n’est point vertueux pour avoir fait quelques actions utiles aux autres hommes ; il ne mérite ce nom que lorsque l’habitude excite constamment en lui l’amour des actions conformes au bien-être des autres hommes ou la haine de celles qui peuvent lui nuire. Cette habitude, contractée de bonne heure, s’identifie avec l’homme de bien et le dispose en tout temps à faire ce qui est avantageux, à s’abstenir de ce qui peut être contraire à la félicité des autres.\par
D’un autre côté, l’homme vertueux peut être quelquefois trompé ou séduit par le premier aspect des choses mais, accoutumé à réfléchir sur les conséquences de ses actions, il est bientôt retenu par la crainte des effets qui, devenue habituelle en lui, l’arrête et l’empêche de se prêter à la séduction des passions et de l’imagination, dont il sait qu’il doit se défier. Sans cesser d’être vertueux, un homme peut désirer le plaisir, mais bientôt la raison le rappelle à son devoir en lui montrant les suites des actions qu’il commettrait pour l’obtenir. La vertu suppose de la réflexion, de l’expérience, de la crainte, de la modération. L’homme de bien est un homme qui calcule, qui combine avec justesse, qui s’observe, qui craint de déplaire ; le méchant est un homme qui se laisse entraîner et qui ne raisonne point sa conduite. « L’incertitude et le vertige, dit Juvénal, furent toujours le caractère du méchant\footnote{« Mobilis et varia est ferme natura malorum. » {\itshape Satire XIII}, vers 236.}. » C’est donc avec raison que Sénèque nous dit que « la vertu est un art qu’il faut apprendre\footnote{« Discenda est virtus, ars est bonum fieri. »} ». Elle est évidemment le fruit, malheureusement trop rare, de l’expérience et de la réflexion. C’est en se repliant sur soi que l’on parvient à apprendre, à se familiariser avec elle, à se l’identifier. C’est à force d’exercice que l’on en contracte l’habitude, c’est en pesant les avantages qu’elle procure, en savourant ses douceurs, en contemplant les sentiments désirables qu’elle excite dans ceux qui en sentent les influences, que l’on apprend à l’aimer. Après en avoir connu le mérite et le prix, l’on se trouve assez fort pour résister à des intérêts futiles, à des plaisirs méprisables, quand on les compare aux avantages constants de la vertu. Lorsqu’on dit que la {\itshape vertu est sa propre récompense}, on indique que tout homme qui la pratique est fait pour jouir de la tendresse, de l’estime, de la considération, de la gloire, en un mot, d’un bien-être nécessairement attaché à une conduite conforme au bien de la société. Celui qui fait le bonheur de ceux avec lesquels il a des rapports acquiert des droits sur leur affection et se met en droit de s’estimer, de s’applaudir, de jouir des douceurs d’une bonne conscience, qui souvent le dédommage de l’ingratitude des hommes. Quelques moralistes nous représentent la vertu comme pénible, comme un sacrifice continuel de nos intérêts les plus chers, comme une haine implacable des plaisirs que la Nature nous porte à désirer, comme un combat fatigant contre nos passions et nos penchants les plus doux. Mais ce n’est point en devenant des ennemis de nous-mêmes que nous pourrons devenir des amis de la vertu. Elle ne nous ordonne pas de renoncer au plaisir, elle nous dit de les choisir et d’en user avec sagesse ; elle ne nous défend pas de jouir des bienfaits de la Nature mais elle nous dit de ne pas nous y livrer en aveugles et de ne point fonder sur eux notre bonheur permanent ; elle ne nous commande pas le sacrifice impossible de toutes nos passions, elle nous prescrit de bien connaître les objets que nous devons aimer et de leur sacrifier les passions inconsidérées pour des objets qui ne nous donneraient que des jouissances momentanées suivies de longs regrets. En un mot, la vertu n’est point contraire aux penchants de notre nature : elle est, comme dit Cicéron, {\itshape la Nature perfectionnée}\footnote{« Est autem nihil aliud quam in se perfecta et ad summum perducta Natura. » Cicéron, {\itshape De Legibus}, livre I, chap. 8.}. Elle n’est point austère et farouche, elle n’est point un enthousiasme fanatique ; elle est une douce habitude de trouver un plaisir constant et pur dans l’usage de notre raison, qui nous apprend à goûter le bien-être que nous répandons sur les autres.\par
Non, la vraie vertu ne consiste pas dans un renoncement total à l’amour de soi, dans un dégagement idéal de tout intérêt, dans un mépris affecté de ce que les hommes désirent. Elle consiste à s’aimer véritablement, à placer son intérêt dans des objets louables, à ne faire que des actions desquelles peut résulter l’estime, l’affection, la considération, la gloire réelles, à se procurer par des voies sûres ce que les hommes veulent obtenir par des routes incertaines et fausses. Est-ce l’affection de vos concitoyens que vous cherchez ? C’est en leur faisant du bien que vous pourrez la mériter. Est-ce la gloire qui fait l’objet de vos vœux ? Elle ne peut être que le salaire de vos actions universellement utiles. Est-ce le pouvoir que votre ambition demande ? En est-il un et plus doux et plus sûr que celui que vos bienfaits vous feront exercer sur vos semblables ? Est-ce le contentement intérieur que votre cœur désire ? Vous êtes certain d’en jouir par la vertu ; elle seule vous donnera le juste droit de vous applaudir quand même l’injustice des hommes vous priverait des hommages que vous aurez mérités.\par
Ainsi, ne croyons pas que la vertu soit un sacrifice cruel de ses intérêts : personne ne connaît mieux comment il faut s’aimer que l’homme qui la pratique. Qu’est-ce en effet que l’on désire le plus dans ce monde, sinon de se faire aimer, chérir, honorer, respecter des autres, de leur donner une bonne opinion de soi, de jouir constamment d’une satisfaction que rien ne peut ravir ? La vertu fournit tous ces avantages ; elle est le plus sûr moyen de conquérir les cœurs, de parvenir à la considération, d’acquérir de la supériorité, d’exercer sur les autres hommes un pouvoir qu’ils approuvent.\par
L’honneur véritable est, comme on verra, le droit que la vertu nous donne à l’estime de nos semblables. Le mérite est en général l’assemblage des qualités utiles ou louables, ou auxquelles on attache du prix dans la société. La supériorité d’un homme sur un autre ne peut être fondée que sur les avantages plus marqués dont il fait jouir. L’autorité légitime, c’est-à-dire reconnue par ceux sur qui elle est exercée, ne peut avoir pour base que le bien qu’on leur fait éprouver. La vraie gloire ne peut être, aux yeux d’un être raisonnable, que la reconnaissance publique, l’admiration générale, excitées par des actions, des talents, des dispositions universellement utiles au genre humain.\par
Telles sont les récompenses que la société, pour son propre intérêt, doit décerner à la vertu. Lorsque aveuglée par l’ignorance, elle lui refuse son salaire, quand ses idées fausses la rendent insensible au mérite, quand le gouvernement, au lieu d’exciter les citoyens à s’occuper du bien public ou du bonheur dont ils sont faits pour jouir en commun, ne montre à la vertu que de la haine ou du dédain, la société ne tarde pas à être punie de son injustice et de sa folie. Les vertus nécessaires à l’ordre, à l’harmonie sociale, à la concorde, à la paix, disparaissent, les liens de la société se relâchent ou se brisent, les intérêts particuliers font oublier l’intérêt général, les citoyens se divisent et le monde devient l’arène des combats continuels que se livrent les vices et les passions des hommes.\par
La vertu n’est si rare que parce que la folie des hommes la prive très souvent des récompenses qu’elle a droit de prétendre. Les sociétés, ainsi que les individus, livrées à des erreurs funestes, méconnaissent leurs intérêts, ont des idées fausses de l’honneur, de la gloire, du bien-être, et rendent leurs hommages à des objets futiles et souvent aux crimes les plus nuisibles. C’est ainsi que chez la plupart des peuples de la terre, l’équité est totalement méconnue, la force se confond avec le droit, l’autorité est le partage non des bienfaits, mais de la violence, la gloire est attachée à des attentats contre le genre humain, l’idée d’honneur à des actions féroces et cruelles, l’idée de supériorité se trouve liée dans tous les esprits à des vanités, à des distinctions puériles dont il ne résulte aucun bien pour la société.\par
Faute de raison et de lumières, les hommes, pour la plupart, ignorent ce que c’est que la vertu et prostituent son nom respectable aux dispositions les plus contraires au bonheur du genre humain. Des nations entières n’ont-elles pas regardé comme la vertu par excellence la valeur guerrière, cette qualité barbare qui met si souvent les nations en larmes ?\par
Pour aimer la vertu il faut s’en former des idées véritables, il faut avoir médité ses effets, il faut en connaître les avantages constants, il faut avoir senti son influence nécessaire sur le bonheur général des sociétés et sur le bonheur particulier des individus. L’amour de la vertu n’est que l’amour de l’ordre, de la concorde, de la félicité publique et privée. Il n’est point de société qui n’ait besoin de vertus pour se conserver et pour jouir des bienfaits de la Nature, il n’est point de famille qui ne trouve dans la vertu de la douceur, de la consolation, de la force, il n’est point d’individu qui n’ait le plus grand intérêt à éprouver les effets de la vertu et à montrer les vertus aux autres. Sous quelque point de vue qu’on l’envisage, l’idée de la vertu est nécessairement liée à celle d’utilité, de bien-être, de contentement, de paix.\par
Au milieu de la société la plus déraisonnable, l’homme de bien, souvent forcé de gémir de la dépravation publique dont il est la victime, se console en rentrant en lui-même, s’applaudit de trouver dans son cœur une joie pure, un contentement solide, le droit de prétendre à la tendresse et à l’estime de ceux sur qui son sort lui permet d’influer. Voilà ce qui constitue le repos de la {\itshape bonne conscience}, qui n’est que l’assurance de mériter l’affection et l’estime des êtres avec qui l’on vit, et l’idée de sa propre supériorité sur les méchants que l’on voit tourmentés par leurs vices et les jouets continuels de leurs tristes folies.\par
Ce qui vient d’être dit nous prouve que l’homme vertueux peut seul passer pour l’homme vraiment {\itshape sociable}, c’est-à-dire pour un membre qui contribue de bonne foi au but que toute société se propose. Examinons maintenant en détail les vertus sociales, ou les dispositions que l’expérience nous montre comme les plus capables de faire obtenir aux nations, ainsi qu’aux citoyens, une félicité permanente.
\subsection[{Chapitre IV. De la Justice}]{Chapitre IV. De la Justice}
\noindent La morale, à proprement parler, n’a qu’une seule vertu à proposer aux hommes\footnote{Suivant Plutarque, le philosophe Ménédémus prétendait qu’il n’y avait point de différence réelle entre les vertus, et qu’il n’en existait qu’une seule, qu’on ne faisait que dénigrer sous des noms divers. Il disait que c’était toujours la même vertu que l’on appelait tantôt justice, tantôt prudence, tantôt tempérance, etc. Voyez Plutarque, {\itshape De la Vertu morale}.}. L’unique devoir de l’être sociable, c’est d’être juste. La justice est la vertu par excellence, elle sert de base à toutes les autres. On peut la définir [comme] une volonté ou une disposition habituelle et permanente de maintenir les hommes dans la jouissance de leurs droits et de faire pour eux tout ce que nous voudrions qu’ils fissent pour nous-mêmes.\par
Les {\itshape droits} des hommes consistent dans le libre usage de leurs volontés et de leurs facultés pour se procurer les objets nécessaires à leur propre bonheur. Dans l’état de nature l’homme isolé a droit de prendre tous les moyens qu’il juge convenables pour se conserver et se procurer le bien-être : il ne fait tort à personne. Cependant, on a vu que même dans cet état, les droits de l’homme sont limités par la raison, qui lui prescrit de ne faire de ses facultés qu’un usage conforme à sa conservation et à son bonheur véritable. Nul homme, sans folie ou sans un dérangement total de sa machine, ne peut exercer le droit de se nuire ou de se détruire ; tout homme intelligent et raisonnable se doit donc la justice à lui-même. Ses droits, à cet égard, sont fixés par la Nature ; ce ne serait pas user de ses droits ou de sa liberté, ce serait en abuser que de se nuire de plein gré.\par
Dans l’état de société les droits des hommes, ou la liberté d’agir, sont limités par la justice, qui leur montre qu’ils ne doivent agir que d’une façon conforme au bien-être de la société, faite pour les intéresser parce qu’ils en sont les membres. Tout homme vivant en société serait injuste si l’exercice de ses droits propres ou de sa liberté nuisait aux droits, à la liberté, au bien-être de ceux avec lesquels il se trouve associé. Ainsi, les droits de l’homme en société consistent dans un usage de sa liberté conforme à la justice qu’il doit à ses associés.\par
La justice n’ôte point à l’homme la liberté ou la faculté de travailler à son propre bonheur : elle l’empêche seulement d’exercer ce pouvoir d’une façon nuisible aux droits de tous que la société doit maintenir. Cela posé, la liberté de l’homme, dans la vie sociale, est le droit que chaque citoyen peut exercer sans porter préjudice à ses associés. Tout usage du pouvoir qui nuit aux autres est injuste et se nomme {\itshape licence.} Chaque homme, ne consultant souvent que son intérêt propre, ses passions, ses désirs déréglés, peut être injuste et méconnaître les droits des autres et leur faire du mal. Aussi, pour le bien de tous la société l’oblige d’observer la justice envers ses associés, elle règle sa conduite pour la rendre conforme à l’intérêt général.\par
C’est par les {\itshape lois} que la société peut régler les actions de ses membres et les empêcher de se nuire réciproquement. Les lois sont les volontés de la société ou les règles de conduite qu’elle prescrit à chacun de ses membres pour les obliger d’observer entre eux les devoirs que la justice leur impose, ou pour les empêcher de se troubler les uns les autres dans l’usage de leurs droits.\par
Les lois sont justes quand elles maintiennent chaque membre de la société dans ses droits, quand elles le garantissent de toute violence, quand elles procurent à chacun la jouissance de sa personne et des biens nécessaires à sa conservation propre et à sa félicité. Ce sont là les objets que la société doit assurer également à tous ses membres ; son autorité sur eux n’a pour base que les avantages qu’elle leur procure. Cette autorité est juste quand elle est conforme au but de la société, c’est-à-dire quand elle contribue au bonheur qu’elle doit à ses membres.
\subsection[{Chapitre V. De l’Autorité}]{Chapitre V. De l’Autorité}
\noindent L’autorité est le pouvoir de régler les actions des hommes. Toute société, pour le bien de ses membres, doit exercer son pouvoir sur eux ; sans cela, leurs passions discordantes, leurs intérêts divers troubleraient à tout moment et la tranquillité, et la félicité particulière des familles et des citoyens. Les hommes vivent en société dans la vue de leur bien-être ; chacun d’entre eux trouve dans la vie sociale une sécurité, des avantages, des secours, des plaisirs dont il serait privé s’il vivait séparé. Conséquemment, chaque membre d’une famille, d’un corps, d’une association quelconque, est forcé de dépendre de la société générale.\par
Dépendre de quelqu’un, c’est avoir besoin de lui pour se conserver et se rendre heureux. Le besoin est le principe et le motif de la vie sociale. Nous dépendons de ceux qui nous procurent des biens que nous serions incapables d’obtenir par nous-mêmes. L’autorité des parents et la dépendance des enfants ont pour principe le besoin continuel qu’ont ces derniers de l’expérience, des conseils, des secours, des bienfaits, de la protection de leurs parents pour obtenir des avantages qu’ils sont incapables de se procurer. C’est sur les mêmes motifs que se fonde l’autorité de la société et de ses lois qui, pour le bien de tous, doivent commander à tous.\par
La diversité et l’inégalité que la Nature a mises entre les hommes donne une supériorité naturelle à ceux qui surpassent les autres par les forces du corps, par les talents de l’esprit, par une grande expérience, par une raison plus éclairée, par des vertus et des qualités utiles à la société. Il est juste que celui qui se trouve capable de faire jouir les autres de grands biens soit préféré à celui qui ne leur est bon à rien. La Nature ne soumet les hommes à d’autres hommes que par les besoins qu’elle leur donne et qu’ils ne peuvent satisfaire sans leurs secours. Toute supériorité, pour être juste, doit être fondée sur les avantages réels dont on fait jouir les autres hommes. Voilà les titres légitimes de la souveraineté, de la grandeur, des richesses, de la noblesse, de toute espèce de puissance, voilà la source raisonnable des distinctions et des rangs divers qui s’établissent dans une société. L’obéissance et la subordination consistent à soumettre ses actions à la volonté de ceux que l’on juge capables de procurer les biens que l’on désire, ou d’en priver. L’espérance de quelque bien ou la crainte de quelque mal sont les motifs de l’obéissance du sujet envers son prince, du respect du citoyen pour ses magistrats, de la déférence du peuple pour les grands, de la dépendance où les pauvres sont des riches et des puissants, etc.\par
Mais si la justice approuve la préférence ou la supériorité que les hommes accordent à ceux qui sont les plus utiles à leur bien-être, la justice cesse d’approuver cette préférence aussitôt que ces hommes supérieurs abusent de leur autorité pour nuire. La justice se nomme {\itshape équité} parce que, nonobstant l’inégalité naturelle des hommes, elle veut qu’on respecte également les droits de tous, et défend aux plus forts de se prévaloir de leurs forces contre les plus faibles.\par
On voit d’après ces principes que la société, ou ceux qu’elle a choisis pour annoncer ses lois, exercent une autorité qui doit être reconnue par tous ceux qui jouissent des avantages de la société. Si les lois sont justes, c’est-à-dire conformes à l’utilité générale et au bien des êtres associés, elles les obligent tous également et punissent très justement ceux qui les violent. Punir quelqu’un, c’est lui causer du mal, c’est le priver des avantages dont il jouissait et dont il aurait continué de jouir s’il eût suivi les règles de la justice indiquées par la prudence de la société.\par
Destinée à conserver les droits des hommes et à les garantir de leurs passions mutuelles, la loi doit punir ceux qui se montrent rebelles aux volontés générales. Elle peut priver du bien-être et réprimer ceux qui troublent la félicité publique, afin de contenir par la crainte ceux que leurs passions empêchent d’entendre la voix publique et qui refusent de remplir les engagements du {\itshape pacte social}.
\subsection[{Chapitre VI. Du Pacte social}]{Chapitre VI. Du Pacte social}
\noindent Ce pacte est la somme des conditions tacites ou exprimées sous lesquelles chaque membre d’une société s’engage envers les autres de contribuer à leur bien-être et d’observer à leur égard les devoirs de la justice. En un mot, le pacte social est la somme des devoirs que la vie sociale impose à ceux qui vivent ensemble pour leur avantage commun.\par
En se réunissant pour leur bonheur mutuel, les hommes, par le but même qu’ils se proposent, se trouvent évidemment engagés et nécessités de prendre la route capable de les y conduire. Soit que ces engagements aient été écrits, exprimés, publiés ou non, ils sont toujours les mêmes. Il est facile de les connaître : ils sont indispensables et sacrés, ils sont fondés sur la nécessité d’employer les moyens propres à obtenir la fin qu’on se propose en vivant avec des hommes.\par
Il suffit de vivre en société pour être obligé de concourir au but de la société ou pour se trouver engagé, même sans déclaration formelle, à servir suivant ses talents et ses forces, à secourir, à défendre ses associés, à respecter leurs droits, à se conformer à la justice, à se soumettre aux lois propres à maintenir l’ordre nécessaire à la conservation de l’ensemble.\par
En échange, la société toute entière, ou les dépositaires de son autorité, se trouvent naturellement et nécessairement engagés à secourir, défendre, protéger, maintenir dans ses justes droits celui qui, sous cette garantie, s’oblige à remplir fidèlement les devoirs de la vie sociale.\par
En conséquence de ces engagements naturels et réciproques, chaque membre acquiert des droits sur la société, c’est-à-dire peut espérer que l’obéissance qu’il lui montre, que l’affection qu’il a pour elle, que les servies qu’il lui rend, seront payés par des avantages tels que la protection, la sûreté de sa personne et de ses biens, la portion de félicité dont la vie sociale met à portée de jouir. Chaque membre de la société est en droit d’exiger un bien-être plus grand que celui dont il jouirait s’il vivait isolé. La société ne peut sans injustice le priver de ce droit ; sans cela elle contrarierait son but, elle nuirait à sa propre conservation, elle ne ferait que rassembler des êtres injustes, animés d’intérêts personnels, dont les passions seraient continuellement en guerre avec le bien public.\par
L’amour sincère de la patrie ne peut être dans les citoyens que l’effet des avantages que la patrie leur procure. Une société sans justice ou gouvernée par des lois iniques et partiales invite tous ses membres à l’injustice, à la méchanceté, ou les rend indifférents sur les intérêts des autres.\par
Par l’imprudence et la déraison des peuples et de ceux qui les gouvernent, les hommes sont très souvent guidés par des lois injustes, des usages pervers, des opinions erronées, des préjugés capables d’anéantir la félicité publique. Enchaînées par des coutumes ou des habitudes peu raisonnées, les nations se trouvent malheureuses et se remplissent de mauvais citoyens perpétuellement occupés à se nuire ouvertement ou sourdement pour des intérêts particuliers toujours opposés à l’intérêt général.\par
La réunion des intérêts particuliers avec l’intérêt général ne peut être que l’effet d’une société fidèle à remplir les engagements du pacte social. Des lois impartiales obligeraient tous les citoyens d’observer les lois de la justice, et tout homme raisonnable se trouverait dans la nécessité d’être vertueux, c’est-à-dire serait dans la disposition habituelle de respecter les droits de ses semblables.\par
C’est dans la balance de l’équité que l’on doit peser les lois, les coutumes, les institutions humaines. Pour distinguer le bien du mal, l’utile du nuisible, le juste de l’injuste, il faut de l’expérience et de la raison. Faute de réfléchir, les hommes, pour la plupart, regardent comme juste tout ce que les lois ou les usages ordonnent ou permettent et regardent comme injuste ce qu’ils défendent. De pareils principes sont faits pour confondre, obscurcir, anéantir toutes les idées de la justice naturelle.\par
Ce que les lois ou les usages d’un peuple permettent se nomme {\itshape licite}, ce qu’ils défendent se nomme {\itshape illicite}. Ce qui est licite ou permis par la loi ou par l’usage peut-être quelquefois très injuste. Chez les Lacédémoniens le larcin ou le vol, fait avec adresse, était permis ou licite, sans être une action juste pour cela. La moindre réflexion nous prouve que c’est nuire aux droits des hommes que de leur ravir des biens dont la société doit être garante. Dans une association de brigands telle que celle des Romains, ces conquérants du monde, ces fléaux du genre humain, le vol, le meurtre, la violence exercés contre les autres peuples étaient des actions non seulement permises, mais encore approuvées et louées comme des vertus.\par
Ce n’est donc pas la volonté souvent déraisonnable d’un peuple, ce ne sont pas ses intérêts particuliers, ce ne sont pas ses lois et ses usages qui rendent juste ce qui ne l’est point par sa nature : il n’y a de vraiment juste que ce qui est conforme aux droits du genre humain.\par
La violence et la conquête peuvent être conformes aux intérêts d’un peuple ambitieux ; ceux qui contentent ses passions peuvent être à ses yeux des personnages estimables et vertueux, mais un tel peuple n’est qu’un amas de malfaiteurs et d’assassins pour quiconque a des idées saines du droit des gens, insolemment violé par une nation ennemie de toutes les autres. L’intérêt permanent de l’homme en général, du genre humain, de la grande société du monde, veut qu’un peuple respecte les droits d’un autre peuple, de même que l’intérêt général de toute société particulière veut que chacun des membres respecte les droits de ses associés.\par
Rien ne peut dispenser les hommes d’être justes : la justice est nécessaire à tous les habitants de la terre, elle est la pierre angulaire de toute association. Sans elle, il ne peut y avoir de société ; son but n’est que de mettre les hommes à l’abri de leurs injustices mutuelles. Le gouvernement et les lois ne peuvent avoir pour objet légitime que d’inviter et de forcer les citoyens à vivre ensemble selon les règles de la justice. La politique ne peut être que les règles immuables de la justice, fortifiées par les récompenses et les châtiments de la société. Obliger les hommes à être justes, c’est les obliger à être humains, bienfaisants, paisibles, sociables, c’est les forcer à travailler au bien-être de leurs semblables afin d’acquérir de justes droits à l’affection, à la bienveillance, à l’assistance, à l’estime, à la protection des autres.\par
Être juste, c’est remplir fidèlement des devoirs que prescrit la vie sociale, c’est sentir l’intérêt que l’on a de mériter de la part de ses associés les sentiments et les dispositions que l’on reconnaît utiles à son propre bonheur dans toutes les positions où l’on peut se trouver. La justice apprend à l’homme à réprimer ses passions, parce qu’elle lui montre qu’en leur donnant un libre cours, il déchaînerait contre lui la haine et les passions des autres. La justice fait que l’homme observe la bonne foi dans les traités, modère son amour-propre, se juge impartialement lui-même, ne s’arroge que ce qui lui est dû, rend aux autres ce qu’ils peuvent exiger. L’homme qui se juge ainsi retient les saillies de l’orgueil, de la vanité, de l’envie, de la jalousie, qui produisent à tout moment tant de divisions sur la terre. S’apprécier soi-même, se mettre à sa place dans la société, montrer des égards, de la politesse, de l’indulgence à tous les hommes, témoigner de la déférence, de la considération, du respect à ceux qui jouissent de la supériorité sur nous par les avantages qu’ils procurent à la société, montrer de la reconnaissance à ceux dont nous recevons des bienfaits, faire du bien aux autres hommes pour mériter leur amour, ne sont évidemment que des actes de justice.\par
On ne peut trop insister sur les avantages que la justice procure aux hommes, ni leur trop répéter que cette vertu suffit pour les rendre heureux\phantomsection
\label{footnote4}\footnote{« Le juste, dit Épicure, est le seul de tous les hommes qui puisse vivre sans trouble et sans désordre ; l’injuste, au contraire, est toujours dans la crainte et dans l’agitation. » « Justus a perturbationibus maxime liber est : injustus autem a plurimis perturbationibus obsidetur. » Voyez Diogène Laërce, {\itshape Vies et Doctrines des Philosophes illustres}, livre X, section 120.}, et que son absence est la cause immédiate de tout le mal moral. Faute de connaître les avantages de l’équité, les gouvernements, destinés à maintenir la justice, dégénèrent en despotisme et en tyrannie. Pour avoir méconnu les droits de l’équité, les peuples de tous temps se sont détruits les uns les autres par des guerres fatales dont l’objet fut communément l’ambition, les prétentions injustes, l’avidité de quelques souverains. Faute de sentir les devoirs de l’équité, dans la plupart des nations les puissants oppriment les faibles et veulent jouir, à l’exclusion des autres citoyens, des droits que la justice assigne à tous également. C’est l’injustice qui transforme tant de fois les pères de familles, les époux, les maîtres, les riches et les grands, en tyrans détestables qui cependant ont le courage de prétendre à l’affection, à la soumission, aux hommages sincères de ceux qu’ils rendent continuellement malheureux.\par
La justice est donc évidemment la base de toutes les vertus, la source commune d’où elles sont émanées, le centre où elles viennent se terminer. Cette vertu renferme toutes les vertus morales ou sociales. La probité, l’intégrité, la bonne foi, la fidélité, l’humanité, la bienfaisance, la reconnaissance, etc, ne sont, comme nous le verrons bientôt, que des dispositions fondées sur la justice ; ou plutôt elles ne sont que la justice même, envisagée sous différents points de vue. Ainsi, ne demandons aux hommes que d’être justes et bientôt ils auront toutes les qualités nécessaires pour rendre la société constamment agréable et fortunée. L’homme juste peut seul être appelé l’être sociable par excellence.
\subsection[{Chapitre VII. De l’Humanité}]{Chapitre VII. De l’Humanité}
\noindent L’humanité est l’affection que nous devons aux êtres de notre espèce comme membres de la société universelle, à qui, par conséquent, la justice veut que nous montrions de la bienveillance et que nous donnions les secours que nous exigeons pour nous-mêmes.\par
Avoir de l’humanité, comme le nom même de cette vertu l’indique, c’est connaître ce que tout homme en cette qualité doit à tous les êtres de son espèce, c’est la vertu de l’homme par essence\phantomsection
\label{footnote5}\footnote{Sénèque dit que la vertu constitue l’homme ({\itshape Virtus virum facit}). En effet, le mot latin {\itshape virtus}, duquel on a dérivé celui de {\itshape vertu}, vient de {\itshape vir}, et indique une qualité essentiellement propre à l’homme, et pourrait se traduire par {\itshape humanité}. D’où l’on voit que le mot {\itshape virtus}, si indignement appliqué par les Romains à la valeur guerrière, était directement opposé à son sens véritable.}.\par
Un être sensible qui aime le plaisir et fuit la douleur, qui désire d’être secouru dans ses besoins, qui s’aime lui-même et veut être aimé des autres, pour peu qu’il réfléchisse, reconnaîtra que les autres sont des hommes comme lui, forment les mêmes vœux, ont les mêmes besoins ; cette analogie ou conformité lui montre l’intérêt qu’il doit prendre à tout être son semblable, ses devoirs envers lui, ce qu’il doit faire pour son bonheur et les choses dont l’équité lui ordonne de s’abstenir à son égard.\par
La justice m’ordonne de montrer de la bienveillance à tout homme qui se présente à mes regards, parce que j’exige des sentiments de bonté des êtres les plus inconnus parmi lesquels le sort peut me jeter.\par
Le Chinois, le mahométan, le Tartare, ont droit à ma justice, à mon assistance, à mon humanité, parce que comme homme j’exigerais leurs secours si je me trouvais moi-même transplanté dans leurs pays.\par
Ainsi, l’humanité, fondée sur l’équité, condamne ces antipathies nationales, ces haines religieuses, ces préjugés odieux qui ferment le cœur de l’homme à ses semblables.\par
Elle condamne cette affection resserrée qui ne se porte que sur les hommes connus, elle proscrit cette affection exclusive pour les membres d’une même société, pour les citoyens d’une même nation, pour les membre d’un même corps, pour les adhérents d’une même secte.\par
L’homme vraiment humain et juste est fait pour s’intéresser au bonheur et au malheur de tout être de son espèce. Une âme vraiment grande embrasse dans son affection le genre humain tout entier et désirerait de voir tous les hommes heureux\footnote{Homère a bien exprimé le sentiment de l’humanité dans {\itshape L’Odyssée} ; il fait dire par Eumée à Ulysse, son maître, déguisé en pauvre mendiant : « Il ne m’est point permis de mépriser un étranger ni un indigent, quand même il serait dans un état plus abject que celui où vous me paraissez réduit ; car c’est Jupiter qui nous envoie l’inconnu et le pauvre. » — « Honore, dit Phocylide, également l’étranger et le concitoyen, car nous sommes tous des voyageurs répandus sur la terre. » Voyez Phocylide, {\itshape Carm.} — Cicéron et Arrien nous proposent l’exemple de Socrate : quelqu’un lui ayant demandé de quel pays il était, il répondit {\itshape du monde}. Voyez Cicéron, {\itshape Tusculanes}, livre I ; Arrien, livre I, chap. IX. — Antonin dit : « Étant par ma nature un être raisonnable et sociable, quelles que soient ma ville ou mon pays, je dirai, comme Antonin, que je suis de Rome, et je dirai, comme homme, que je suis du monde. » Voyez Antonin, livre VI, §. 44.}.\par
Ainsi, n’écoutons point les vains propos de ceux qui prétendent qu’aimer tous les hommes soit une chose impossible et que l’amour du genre humain, si vanté par quelques sages, est un prétexte pour n’aimer personne. Aimer les hommes, c’est désirer leur bien-être, c’est avoir la volonté d’y contribuer autant qu’il est en nous. Avoir de l’humanité, c’est être habituellement disposé à montrer de la bienveillance et de l’équité à quiconque se trouve à portée d’avoir besoin de nous. Il est sans doute dans nos affections, des degrés fixés par la justice : nous devons plus d’amour à nos parents, à nos amis, à nos concitoyens, à la société dont nous sommes les membres, à ceux, en un mot, dont nous éprouvons les secours et les bienfaits, dont nous avons un besoin continuel, qu’à des étrangers qui ne nous tiennent par d’autres liens que ceux de l’humanité.\par
Les besoins plus ou moins pressants rendent les devoirs des hommes plus ou moins indispensables ou sacrés. Pourquoi devons-nous plus d’amour à notre patrie qu’à un autre pays ? C’est parce que notre patrie renferme les personnes et les choses les plus utiles à notre propre bonheur. Pourquoi un fils doit-il à son père son affection et ses soins préférablement à tout autre ? C’est parce que son père est de tous les êtres le plus nécessaire à sa propre félicité, celui auquel il se trouve attaché par les liens de la plus grande reconnaissance.\par
Le besoin est donc le principe des liens qui unissent les hommes et les retiennent en société. C’est en raison du besoin qu’ils ont les uns des autres qu’ils s’attachent réciproquement. Un homme qui n’aurait aucun besoin de personne serait un être isolé, amoral, insociable, dépourvu de justice et d’humanité. Celui qui s’imagine pouvoir se passer des autres se croit communément dispensé de leur montrer des sentiments.\par
Les princes et les grands, sujets à se persuader qu’ils sont des êtres d’une espèce différente des autres, sont peu tentés de leur montrer de l’humanité. Il faut communément avoir éprouvé le malheur ou le craindre pour prendre part aux peines des misérables.\par
Si l’humanité est une disposition distinctive des hommes, combien en trouve-t-on peu qui méritent de porter le nom de leur espèce !\par
La morale doit se proposer de réunir d’intérêts tous les individus de l’espèce humaine, et surtout les membres d’une même société. La politique devrait sans cesse concourir à resserrer les liens de l’humanité, soit en récompensant ceux qui montrent cette vertu, soit en flétrissant ceux qui refusent de l’exercer. En un mot, tout devrait faire sentir aux mortels qu’ils ont besoin les uns des autres et leur prouver que le pouvoir suprême, que la rang, la naissance, les dignités, les richesses, bien loin d’être des titres pour mépriser ceux qui n’ont pas ces avantages, imposent à ceux qui les possèdent le devoir d’être humains, de secourir, de protéger leurs semblables. Le mépris pour la misère, la pauvreté, la faiblesse, est un outrage pour l’espèce humaine : au lieu d’exalter celui qui s’en rend coupable, il doit le ravaler, lui faire perdre sa dignité et les droits à l’affection et aux respects de ses concitoyens.
\subsection[{Chapitre VIII. De la Compassion ou de la Pitié}]{Chapitre VIII. De la Compassion ou de la Pitié}
\noindent Compatir aux maux des hommes, c’est sentir ce qu’ils sentent, c’est souffrir avec eux, c’est partager leurs peines, c’est en quelque façon se mettre dans leur place pour éprouver la situation pénible qui les tourmente.\par
Ainsi, la compassion dans l’homme est une disposition habituelle à sentir plus ou moins vivement les maux dont les autres sont affligés.\par
Pour expliquer les causes de cette sensibilité qui intéresse les hommes aux peines de leurs semblables, quelques moralistes ont eu recours à une certaine {\itshape sympathie}, c’est-à-dire à une cause occulte et chimérique qui ne peut rien expliquer. C’est dans l’organisation de l’homme, dans sa sensibilité, dans une mémoire fidèle, dans une imagination active qu’il faut chercher la vraie cause de la compassion\phantomsection
\label{footnote6}\footnote{On sait le trait d’un sybarite qui, en voyant des ouvriers travailler dans son jardin, se sentit tellement troublé qu’il défendit de jamais rien y faire en sa présence.}.\par
Celui qui a des organes sensibles sent vivement la douleur, s’en rappelle exactement l’idée ; son imagination la lui peint avec force à la vue de l’homme qui souffre.\par
Dès lors il est troublé lui-même, il frémit, son cœur se serre, il éprouve une vraie douleur, qui dans les personnes très sensibles se manifeste quelquefois par des évanouissements ou des convulsions. L’effet naturel de la douleur qu’éprouve alors la personne vivement affectée, est de chercher les moyens de faire cesser dans les autres la situation pénible qui s’est communiquée à elle-même.\par
Du soulagement donné à celui qui souffre, il en résulte un soulagement réel pour la personne qui lui donne du secours, plaisir très doux que la réflexion augmente encore par l’idée d’avoir fait du bien à quelqu’un, d’avoir acquis des droits sur son affection, d’avoir mérité sa reconnaissance, d’avoir agi d’une façon qui prouve que l’on possède un cœur tendre et sensible, disposition que tous les hommes désirent trouver dans leurs semblables et dont l’absence ferait croire que l’on est mal conformé.\par
Les hommes étant très variés pour l’organisation et la force de l’imagination, ne peuvent être susceptibles de sentir avec une égale vivacité les maux de leurs semblables. Il est des êtres pour qui la compassion est nulle, ou du moins n’est pas assez forte pour les déterminer à faire cesser les peines qu’ils voient souffrir aux autres. On ne rencontre que trop souvent des hommes que l’habitude du bien-être, la jouissance des commodités\phantomsection
\label{footnote7}\footnote{« Plus on est favorisé des biens de la fortune, dit un moraliste moderne, moins on est disposé à soulager ceux qui en sont dénués. Les pauvres tirent plus de secours de gens presque aussi pauvres qu’eux que des gens riches. Il semble qu’on ne soit compatissant que pour les maux qu’on éprouve en partie : car un homme accablé de peine épuise sur lui-même toute sa sensibilité, et l’excès du malheur rend aussi incapable de commisération que le comble de la prospérité. » Voyez un livre intitulé {\itshape Les Mœurs}, partie II, chap. IV, article II.}, l’inexpérience du mal endurcissent sur les maux d’autrui et empêchent même de s’en faire une idée. Le malheureux est communément bien plus compatissant que celui qui n’a jamais éprouvé les coups du sort. Celui qui a ressenti les douleurs de la goutte ou de la pierre est bien plus disposé qu’un autre à plaindre ceux qu’il voit affligés des mêmes maladies. L’indigent qui a souvent éprouvé les horreurs de la faim connaît toute sa force et plaint celui qui l’éprouve, tandis que le riche, perpétuellement rassasié, semble ignorer qu’il existe au monde des millions de malheureux privés du nécessaire.\par
Quelques moralistes ont cru que la compassion — ou cette disposition à prendre part aux infortunes des autres qui se trouve dans les personnes sensibles, bien nées, convenablement élevées — devrait être regardée comme la base de toutes les vertus morales et sociales\phantomsection
\label{footnote8}\footnote{Les stoïciens ont eu une opinion totalement opposée ; ils regardaient la pitié comme une faiblesse au-dessus de laquelle le sage devait s’élever.}. Mais la pitié, comme tout le prouve, est très rare sur la terre ; le monde est rempli d’une foule d’êtres insensibles dont les cœurs ne sont que peu ou point remués par les infortunes de leurs semblables. Dans les uns ce sentiment n’existe pas, dans d’autres il est si faible que le moindre intérêt, la moindre passion, la plus légère fantaisie, sont capables de l’étouffer.\par
Quoique tous les hommes désirent de passer pour sensibles, il en est très peu qui donnent les signes d’une sensibilité véritable. Si une première impulsion les montre vivement touchés, ces sentiments sont sans suite et vont bientôt avorter. Des princes contemplent d’un œil sec les malheurs de tout un peuple, auxquels un mot de leur bouche pourrait souvent remédier. Des pères de famille voient souvent de sang-froid couler les larmes d’une femme, des enfants, des serviteurs, dont les mauvaises humeurs ou leurs folies causent les infortunes. Des hommes avides voient sans pitié la misère des peuples que leurs extorsions réduisent à la mendicité. Enfin, il est très peu de gens assez touchés de malheurs de leurs semblables pour daigner leur donner des consolations ou pour leur tendre une main secourable\phantomsection
\label{footnote9}\footnote{« La vue de l’infortuné, dit un philosophe célèbre, fait sur la plupart des hommes l’effet de la tête de Méduse : à son aspect, les cœurs se changent en rochers. » Voyez le livre {\itshape De l’Esprit}, discours III, chap. XIV, p. 358, édition in-4°.}. On fuit communément le spectacle du malheur, que l’on trouve fâcheux, et l’on cherche mille prétextes pour se dispenser de secourir le malheureux, que l’on regarde pour l’ordinaire comme un être incommode et totalement inutile.\par
Que dis-je ! Les hommes, pour la plupart, se croient autorisés par la faiblesse ou l’infortune des autres à les outrager impunément et prennent un barbare plaisir à les affliger, à leur faire sentir leur supériorité, à les traiter cruellement, à les tourner en ridicule. Ainsi, des êtres exposés eux-mêmes aux caprices de la fortune, loin de s’attendrir sur le sort des malheureux, aggravent encore leurs peines par des airs hautains, des railleries piquantes, des mépris insultants\footnote{« Nil habet infelix paupertas durius in se, Quam quod ridiculos homines facit. » Juvénal, {\itshape Satires}, III, vers 152.}. Rien de plus barbare, de plus inhumain, de plus lâche, que d’insulter le faible et le malheureux que l’on voit dénué de secours, rien de plus révoltant pour le cœur de l’homme que de se voir exposé au mépris, à la dureté de ses semblables.\par
Pour un être habituellement disposé à plaindre et soulager les malheureux, il ne suffit pas d’avoir un cœur sensible qui, comme on a vu, est un don de la Nature\footnote{« Mollissima corda humano generi dare se Natura fatetur quæ lacrimas dedit. » Juvénal, {\itshape Satires}, XV, vers 131.} : il faut encore que cette sensibilité naturelle ait été soigneusement cultivée. L’éducation devrait sans cesse exercer la sensibilité des princes, des grands et de ceux qui sont destinés à jouir de l’opulence. On devrait de bonne heure étouffer cet orgueil qui leur persuade qu’ils n’ont besoin de personne, qu’ils sont des êtres d’un ordre plus relevé que le peuple indigent. On devrait leur répéter qu’ils sont des hommes faibles, sujets à mille accidents et que mille circonstances inopinées peuvent à chaque instant plonger dans l’infortune ; on devrait attendrir leurs âmes endurcies par le spectacle si touchant, et souvent si déchirant, de la misère ; on devrait échauffer leur imagination en leur peignant sous les traits les plus forts la situation déplorable à laquelle, pour contenter le luxe et la vanité de quelques favoris du sort, les autres sont condamnés pour la vie à manger du pain arrosé de sueurs et de larmes. À la vue de ces tableaux si frappants, quel est l’homme dont le cœur ne fût au moins fortement ébranlé ! Élevé dans ces idées, quel est le monarque, le grand ou le riche qui ne se reprocherait pas de jouir d’un inutile superflu tandis que tant de leurs semblables languissent dans l’infortune et maudissent leur existence.\par
C’est ainsi que le sentiment de la pitié pourrait être développé dans les cœurs que la Nature a doués de sensibilité, mais comme cette disposition est malheureusement très rare, l’équité doit y suppléer pour ceux que la Nature en a privés. On leur représentera donc qu’ils sont eux-mêmes exposés comme les autres à des revers, et que pour acquérir des droits sur la pitié des autres, ils doivent se montrer sensibles, prendre part aux misères humaines, ou du moins les soulager. Le riche dédaigneux doit apprendre qu’un accident imprévu peut, au moment qu’il s’y attend le moins, le réduire au même état que le malheureux dont il détourne les yeux. Enfin, tout homme qui se dit sociable devrait savoir qu’étant homme, il est obligé de prendre part aux infortunes de ses semblables et de les soulager autant qu’il est en son pouvoir.\par
Néanmoins, très peu de gens remplissent ce devoir si sacré : chacun trouve des prétextes pour se dispenser de montrer de la pitié à ceux mêmes qui devraient en exciter la plus forte. C’est ainsi que l’on trouve souvent dans un saint zèle un prétexte pour haïr ceux qui sont dans l’erreur, lors même que l’on croit que leurs égarements peuvent les conduire à des malheurs infinis. Conséquemment, on tourmente, on persécute, on extermine quelquefois des hommes que l’on pourrait peut-être ramener par la douceur, et pour qui l’on devrait sentir la plus tendre commisération. Pareillement, on n’a guère de pitié pour ceux qui par leur faute sont tombés dans l’infortune, tandis qu’on devrait les plaindre d’être ainsi constitués. Les égarements des hommes viennent de leurs tempéraments, de leur ignorance, de leur éducation, de leurs passions indomptées, de leur inadvertance, de leur étourderie. Aux yeux de l’homme de bien, le méchant qu’il est forcé d’éviter est bien plus digne de pitié que de haine, vu qu’il travaille incessamment à se rendre malheureux.
\subsection[{Chapitre IX. De la Bienfaisance}]{Chapitre IX. De la Bienfaisance}
\noindent C’est violer le pacte social, c’est être injuste, que de négliger ou de refuser de faire du bien, quand on le peut, aux êtres avec lesquels on vit en société. Tout est échange parmi les hommes ; la bienfaisance est le moyen le plus sûr d’enchaîner les cœurs, elle est payée par la tendresse, l’estime, l’admiration de ceux qui en éprouvent les effets.\par
La bienveillance est une disposition habituelle à contribuer au bien-être de ceux avec qui notre destin nous lie, en vue de mériter leur bienveillance et leur reconnaissance. Ainsi, la bienfaisance ne peut pas être désintéressée ou dépourvue de motifs\footnote{« Qu’est-ce qu’un bienfait ? dit Sénèque ; c’est un acte de bienveillance fait pour donner de la joie et pour en recevoir. » « Quid est ergo beneficium ? benevola actio, tribuens gaudium, capteasque tribuendo. » Voyez Sénèque, {\itshape De Beneficiis}, livre I, chap. 5 et 6.}. Si tout homme, par sa nature, désire l’affection de ses semblables, rien de plus naturel et de plus légitime que d’en prendre les moyens. Il est vrai que les bienfaits ne sont pas toujours payés des sentiments qu’ils devraient naturellement exciter, mais en dépit des ingrats, un être bienfaisant est toujours estimable aux yeux de la société, ses heureuses dispositions sont applaudies par tous les cœurs sensibles, dont le jugement équitable le venge de l’injustice des autres.\par
« Celui qui vous donne vous ôte toujours quelque chose », dit un ancien Arabe\footnote{Voyez {\itshape Sentent. Arab. in Erpenii grammatica}.}. Tout bienfait donne à celui qui en est l’auteur une supériorité nécessaire sur celui qui le reçoit : « Celui, dit Aristote, qui fait du bien à quelqu’un, l’aime mieux qu’il n’en est aimé\footnote{Montaigne ajoute que « celui à qui il est dû aime mieux que celui qui doit ; et tout ouvrier aime mieux son ouvrage qu’il n’en serait aimé, si l’ouvrage avait du sentiment. » Voyez {\itshape Essais} de Montaigne, livre II, chap. 8. Nous reviendrons sur ce principe en parlant de l’ingratitude et de l’affection paternelle, qui est plus commune que la piété filiale.}. » Chacun craint de trouver dans un bienfaiteur un maître orgueilleux qui mette un prix trop grand au bien qu’il a pu faire. Voilà sans doute pourquoi les âmes nobles et fières refusent souvent les bienfaits et sont en garde contre les secours qui peuvent devenir onéreux. La bienfaisance est un art souvent très difficile ; il consiste à ménager la délicatesse de ceux qui en sont les objets. On rougit très souvent des bienfaits qu’on reçoit parce qu’on les regarde comme des chaînes, comme des engagements à la servitude\footnote{« Beneficium accipere, libertatem vendere est », disaient les Anciens.}. Les bienfaits accompagnés de hauteur révoltent ceux qui les reçoivent et ne font que des ingrats. C’est très souvent la faute du bienfaiteur s’il ne trouve pas dans les cœurs les sentiments qu’il prétend y faire éclore. On ne reçoit un bienfait avec reconnaissance que lorsqu’on a la confiance que le bienfaiteur ne s’en prévaudra pas pour faire sentir sa supériorité d’une façon incommode à l’amour-propre. Les bienfaits dont l’objet est d’asservir sont des insultes et des outrages, et dès lors sont de nature à déplaire à tout homme qui veut conserver sa liberté. Les âmes basses et vénales sont prêtes à recevoir à toutes mains, mais l’homme de bien, qui a la conscience de sa propre valeur, ne peut consentir à perdre le droit de s’estimer ; il ne reçoit des bienfaits que lorsqu’il est assuré de pouvoir les payer par sa reconnaissance. Il n’y a que l’homme sensible et vertueux qui sache vraiment obliger, il n’y a que l’homme sensible qui soit vraiment reconnaissant. « Il faut, disait Chilon, oublier le bien qu’on fait aux autres et ne se ressouvenir que de celui que l’on reçoit. »\par
La bienfaisance exercée sans choix est souvent moins une vertu qu’une faiblesse ; pour être estimable, elle doit être réglée par la justice et la prudence. Faire du bien aux méchants, c’est être dupe, c’est les confirmer dans leur méchanceté. Faire du bien à des insensés, c’est leur faire un mal réel, c’est les entretenir dans leurs dispositions nuisibles. La bienfaisance de l’homme faible ne fait que des ingrats ; on se croit dispensé de lui savoir gré de ce qu’il n’a pas la force de refuser. L’homme bienfaisant par faiblesse mérite plus la pitié que l’estime des honnêtes gens, et devient la proie des fripons\footnote{Plutarque reproche à Nicias « d’avoir été toujours prêt à donner aux méchants qui ne songeaient qu’à mal faire, et aux bons qui étaient dignes de ses libéralités. En un mot, sa faiblesse était un fond sûr pour les méchants, et son humanité pour les gens de bien ». Voyez Plutarque, {\itshape Vie de Nicias}. Celui à qui un homme faible a fait du bien se félicite communément d’avoir {\itshape attrapé} son bienfaiteur.}.\par
Pour être juste, la bienfaisance doit se proposer le bien public, et récompenser la vertu. Le vice et la méchanceté méritent-ils un salaire ? « Ne répands pas les bienfaits sur les méchants, dit Phocylide, car ce serait de la semence sur la mer. »\par
Des bienfaits versés sans choix, des faveurs accordées à des hommes indignes sont des injustices réelles dont l’effet est de décourager le mérite et les talents nécessaires au bonheur de la vie sociale. En comblant de faveur des hommes vils et rampants, en répandant les trésors de l’État sur des citoyens inutiles ou pervers, un prince n’est nullement bienfaisant : il est injuste envers son peuple, dont il récompense les ennemis à ses dépens.\par
La bienfaisance doit-elle s’étendre jusqu’à ceux qui nous ont fait du mal à nous-mêmes ? La plus noble des vengeances est sans doute celle qui nous porte à faire du bien à ceux dont nous avons lieu de nous plaindre ; elle est propre à changer le cœur d’un ennemi. Est-il rien de plus satisfaisant que d’exercer son empire sur celui même qui nous a marqué du mépris ? Est-il rien qui marque plus de grandeur et de vraie force dans l’âme que de montrer à son ennemi qu’il n’a pas le pouvoir de la troubler ? « Ne point se venger d’un ennemi, dit Plutarque, quand on en trouve l’occasion, est une preuve d’humanité ; mais avoir pitié de lui quand il est tombé dans l’adversité, lui donner les secours qu’il demande, est la marque la plus grande de bienveillance et de générosité\footnote{Voyez Plutarque, {\itshape De l’Utilité des Ennemis}. « Relève, dit Phocylide, la bête de somme de ton ennemi si elle est tombée dans le chemin. » Voyez Phocylide, {\itshape Carm.}, vers 133.}. »\par
La bienfaisance n’est point l’apanage exclusif de la puissance, du crédit, de la grandeur, de l’opulence : tout citoyen vertueux peut être bienfaisant dans la sphère où le sort l’a placé. On sert utilement sa patrie par ses vertus, par ses talents, par ses lumières, par son travail ; le sage qui éclaire ses concitoyens, le savant et l’artiste habile, le cultivateur laborieux méritent de l’estime et de l’amour ; ils peuvent avec justice se flatter d’être des bienfaiteurs de leur pays. Ce que l’on nomme {\itshape esprit public} est la bienfaisance appliquée à la société en général. Une sage politique devrait l’exciter, surtout dans les cœurs des riches et des grands, qui trouveraient dans la gloire et dans des distinctions honorables la récompense d’un emploi de leur fortune préférable sans doute aux folles dépenses qui n’ont pour objet que le luxe et la vanité. L’esprit public, ou la bienfaisance étendue sur toute une nation, annonce un bon gouvernement et des citoyens empressés de mériter l’estime de leurs concitoyens ; ces dispositions font voir que chacun prend à cœur le bien-être de son pays.\par
Mais nous verrons bientôt que la bienfaisance doit être accompagnée de modestie ; il vaut mieux, dit-on, donner que recevoir. Donner est en effet une marque de pouvoir ou de supériorité, au lieu que recevoir est un signe de faiblesse ou d’infériorité.\par
La reconnaissance, suivant la force du mot, est l’aveu de sa dépendance et de la puissance du bienfaiteur. Il faut donc que le bienfaiteur ménage la délicatesse des hommes s’il veut mériter leur affection et leur reconnaissance. Quiconque par sa conduite annonce du mépris à ceux qu’il oblige, se paie de ses propres mains. L’homme arrogant révolte, et dès lors il n’est pas un être bienfaisant. S’applaudir intérieurement du bien que l’on fait aux hommes est un sentiment naturel et légitime, mais leur faire sentir sa supériorité, c’est les affliger sensiblement.\par
La libéralité est une suite de la bienfaisance ; elle consiste à faire part des biens de la fortune à ceux qui en ont besoin. Elle doit être réglée par l’équité, la prudence et la raison. Une libéralité sans choix se nomme prodigalité ; elle est, comme on verra bientôt, un vice, et non pas une vertu.\par
La générosité est encore un effet de la bienfaisance. Elle consiste à faire le sacrifice d’une partie de nos droits en vue du bien-être de la société ou de ceux à qui nous voulons donner des marques de notre bienveillance. Cette disposition si noble qui semble nous détacher de nous-mêmes, de nos intérêts les plus chers, quelquefois même de la vie, a pour motif un grand amour des hommes, un désir ardent de leur plaire, un grand enthousiasme pour la gloire, sans même pouvoir se flatter d’en jouir. Les Codrus, les Curtius, les Decius étaient des hommes généreux, enivrés de l’amour de leur pays au point de courir à une mort assurée dans l’espérance d’être admirés et chéris par leurs concitoyens.\par
On demandera peut-être quelle est la mesure de la bienfaisance, de la libéralité, de la générosité. Elle est fixée par l’équité, qui nous dit que nous devons faire pour les autres ce que nous voudrions qu’ils fissent pour nous. Mais d’un autre côté, cette même équité nous montre que nous ne pouvons justement exiger de la bienfaisance ou de la générosité des autres, que les sacrifices que nous ferions pour eux.\par
La bienfaisance, la libéralité, la générosité, pour être bien réglées, doivent avoir pour objet primitif les personnes qui ont les rapports les plus intimes avec nous ; ces dispositions sont des dettes quand il s’agit de la patrie, de nos parents, de nos proches, de nos amis sincères. Elles sont des actes de bienveillance, d’humanité, de pitié quand elles nous portent à secourir des indifférents, des inconnus, des personnes avec lesquelles nous ne sommes liés que faiblement ; elles sont des marques de grandeur d’âme quand elles s’étendent à ceux dont nous avons à nous plaindre. « La méchanceté de l’homme, disait Dion, suivant Plutarque, quoique difficile à déraciner, n’est pourtant d’ordinaire ni si farouche ni si rebelle qu’elle ne se corrige et ne s’adoucisse enfin lorsqu’elle est vaincue par des bienfaits réitérés\phantomsection
\label{footnote10}\footnote{Voyez Plutarque, dans la {\itshape Vie de Dion}.}. »\par
En un mot, la bienfaisance est de toutes les vertus la plus propre à rendre l’homme cher à ses semblables et content de lui-même. Ainsi, nous finirons cet article par l’avis de Polybe à Scipion, qu’il exhortait à ne point rentrer chez lui sans s’être fait un ami par ses bienfaits. « Partout où l’on rencontre un homme, dit Sénèque, on peut exercer la bienfaisance\phantomsection
\label{footnote11}\footnote{« Ubicumque Homo est, ibi beneficio locus est. » Sénèque, {\itshape De Vita Beata}, chap. 24.}. »
\subsection[{Chapitre X. De la Modestie. De l’Honneur. De la Gloire}]{Chapitre X. De la Modestie. De l’Honneur. De la Gloire}
\noindent La modestie est une vertu qui consiste à ne point se prévaloir de ses talents et de ses vertus d’une façon désagréable pour ceux avec qui nous vivons. Un jugement trop favorable de nous-mêmes offense nos semblables qui, voulant juger librement de nos actions, ne souffrent qu’avec peine que l’on s’assigne à soi-même dans leur opinion un rang ou des récompenses qu’ils n’ont point décernés. Pour sentir que la modestie est fondée sur la justice, il suffit que chacun ait éprouvé à quel point la société se trouve fatiguée par des hommes superbes et vains qui ne semblent y vivre que pour faire essuyer aux autres leurs mépris insultants, ou par ces personnages ridicules qui, sans cesse occupés de leur mérite réel ou prétendu, font essuyer aux autres l’ennui de leur {\itshape égoïsme} impertinent. D’ailleurs, un être sociable doit se connaître, sentir qu’il a des imperfections et des défauts, se juger avec équité et réprimer par cette considération les mouvements d’orgueil qui s’élèvent en lui lorsqu’il se compare aux autres. La conscience de nos propres défauts est un remède assuré contre la trop haute opinion que nous avons de nous-mêmes.\par
Nul homme qui a la juste confiance d’avoir de la vertu, de la probité ou des talents, ne peut se mépriser lui-même ; ce sentiment, s’il était possible, serait injuste. Toutes les fois que l’homme a la conscience d’avoir bien fait, de posséder des qualités estimables ou des talents utiles, il acquiert le droit de s’applaudir et de sentir les droits qu’il a sur l’estime des autres. Mais il perdrait ces droits s’il se croyait autorisé à leur nuire, il déplairait et blesserait véritablement s’il montrait de la hauteur et du mépris à des êtres essentiellement épris d’eux-mêmes, jaloux de leur égalité, et qui jamais ne reconnaissent qu’à regret la supériorité des autres.\par
La modestie seule est capable de désarmer l’envie, qui souvent rend les hommes très injustes. Tout homme vraiment grand ou qui montre des talents extraordinaires, s’annonce dans la société comme un maître dont chacun redoute la supériorité. Voilà sans doute la cause de l’aversion et de la jalousie trop communes que font éclore les grands talents, dont l’éclat offusque les esprits médiocres\footnote{« Vrit enim fulgore suo qui prægravat artes, infra se positas. » Horace, {\itshape Épîtres}, livre II, 1, vers 13-15.}. C’est par la modestie que l’on peut ramener les hommes à l’équité et leur faire oublier la disproportion que les vertus ou le génie mettent entre eux et les êtres les plus distingués de leur espèce.\par
L’on craint naturellement les princes, les grands et les puissants de la terre. Pour les aimer on exige qu’ils descendent de leur rang et se mettent à niveau des autres ; il est de la nature de l’homme de redouter ceux qui lui semblent plus grands et plus forts que lui, parce qu’ils lui rappellent à tout moment sa bassesse ou sa médiocrité.\par
Tout être vraiment sociable doit se prêter à la faiblesse des hommes. S’il veut mériter leur amour et leur estime, il doit être modeste et résister aux mouvements d’un amour-propre qui lui attirerait de la haine ou du mépris, au lieu de l’affection et de l’estime qu’il est fait pour attendre. L’homme vertueux doit désirer la bonne opinion de ses semblables, mais la réflexion lui prouve que ses vues seraient frustrées si, par son arrogance, son orgueil et sa présomption, il affligeait les êtres dont il veut mériter l’amour.\par
On voit donc que le désir de l’estime et l’amour de la gloire guidés par la raison, sont compatibles avec la modestie qui, loin d’ôter leur prix au mérite et à la vertu, les rendent bien plus propres à toucher les cœurs des hommes. Celui qui a la conscience de sa propre valeur attend en paix qu’on lui rende justice ; celui qui n’est point sûr de son propre mérite se croit obligé d’en avertir les autres, et par une sotte vanité ne s’attire le plus souvent que des mépris.\par
Un amour-propre inquiet, un orgueil insensé, une hauteur peu raisonnée annoncent de la faiblesse et de la défiance de son propre mérite. La vertu réelle, les vrais talents, la grandeur d’âme, l’honneur véritable, sont tranquilles sur leurs droits.\par
L’{\itshape honneur} est le droit légitime que nous avons acquis par notre conduite et sur l’estime des autres, et sur notre propre estime. L’homme n’a le droit de prétendre à l’estime de la société que lorsqu’il en est un membre utile. Il n’a le droit de s’estimer ou de s’applaudir lui-même que lorsqu’il est assuré d’avoir mérité l’estime de ses semblables. Ainsi, l’homme d’honneur (qui jamais ne peut être distingué de l’honnête homme) ne peut être déshonoré que lorsque, changeant de conduite, il se prive du droit à l’estime des autres et à sa propre estime. Sans cela il peut bien être noirci par la calomnie et déchiré par l’envie, des circonstances malheureuses pourront pour un temps ternir sa réputation, mais il ne perdra jamais le droit de s’estimer lui-même, que nul pouvoir sur la terre ne pourra lui ravir.\par
Ce que le préjugé décore du nom d’honneur n’est le plus souvent qu’un orgueil inquiet, une vanité chatouilleuse, une présomption de ses droits incertains sur l’estime publique. Des gens d’honneur de cette espèce sont toujours {\itshape sur le qui-vive}, ils craignent qu’un mot, qu’un geste ne leur ravisse un honneur chimérique, et pour montrer leur droit à l’estime publique, vous les verrez souvent commettre des crimes et des meurtres pour mettre leur honneur à couvert. C’est sur de pareilles notions que se fonde l’usage barbare des combats singuliers qui, bien loin de déshonorer aux yeux des nations qui se disent raisonnables et civilisées, font estimer comme gens d’honneur ceux qui commettent de pareils attentats. Le véritable honneur ne se détruit point par un affront et ne se rétablit point par un assassinat. Un homme ne peut être blessé dans son honneur que par lui-même. Le courage est une faiblesse quand il ne peut rien supporter. L’honneur réel ne peut consister que dans la vertu ; la vertu ne peut être ni cruelle ni sanguinaire : elle est paisible, elle est douce, elle est juste, patiente et modeste. Elle n’est point arrogante et superbe, parce qu’elle se rendrait odieuse et méprisable.\par
Cicéron nous apprend que Socrate maudissait ceux qui avaient séparé l’utile de l’honnête et regardait cette distinction comme la source de tous les maux\footnote{Voyez Cicéron, {\itshape De Legibus}, livre I, chap 12. Idem, {\itshape De Officiis}, livre III, chap. 3.}.\par
Les anciens philosophes appelaient {\itshape honnête} ce que nous appelons bon, juste, louable, utile à la société. En effet, ce qui porte ces caractères est honnête ou, suivant la force du mot, mérite d’être honoré. Cela posé, la vertu seule est honorable et l’honnête homme ne doit jamais être distingué de l’homme d’honneur. D’un autre côté, ces mêmes philosophes appelaient {\itshape honteux} ce que nous nommons mauvais ou nuisible à la société. D’après ce principe, une vengeance féroce, un homicide, bien loin d’être des actions honorables, devraient couvrir de honte et d’infamie celui qui s’en rend coupable.\par
Tacite remarque que « le mépris de la gloire conduit au mépris de la vertu\footnote{« Contemptu famæ, contemni virtutes. » {\itshape Annales}, livre IV, chap. 38, in fine.} ». Le désir de l’estime et de la réputation est un sentiment naturel que l’on ne peut blâmer sans folie, c’est un motif puissant pour exciter les grandes âmes à s’occuper d’objets utiles au genre humain. Cette passion n’est blâmable que lorsqu’elle est excitée par des objets trompeurs ou lorsqu’elle emploie des moyens destructeurs de l’ordre social\footnote{« L’honneur, dit Platon, est une jouissance divine. » Voyez Platon, {\itshape De Legibus}, livre V. « La gloire, dit Cicéron, est la vraie récompense de la vertu ; il n’y a rien de plus propre à exciter aux actions honnêtes les hommes d’un génie supérieur. » Voyez Cicéron, {\itshape In Consol.}}.\par
« Nous ne devons pas, dit Antonin, désirer les louanges de la multitude ; nous ne devons ambitionner que celles des personnes qui vivent conformément à la Nature. » La gloire a été bien définie [comme étant] {\itshape la louange des bons}, c’est-à-dire de ceux qui jugent bien et qui méritent eux-mêmes d’être loués. Il n’y a que la vertu qui mérite l’estime des gens de bien, et la vertu ne consiste que dans des dispositions utiles au bonheur de notre espèce. La gloire n’est donc faite que pour ceux qui font de très grands biens aux hommes ; elle n’est aucunement destinée à ceux qui les détruisent. Combien de prétendus grands hommes sont dégradés aux yeux de ceux qui se sont faits des idées vraies de la gloire ! Mais les grands crimes en imposent tellement à l’imagination du vulgaire qu’il honore très souvent des forfaits détestables ; il met au rang des dieux des monstres qui ne méritent pas d’être regardés comme des hommes ! Le préjugé enivre tellement les peuples qu’ils admirent ceux mêmes dont ils éprouvent les fureurs. L’admiration que l’on montre à des héros de cette espèce annonce de la noirceur, de la bassesse ou de la stupidité.\par
Un conquérant s’imagine que ses exploits le conduiront à la gloire. Il commence par voler des provinces et des royaumes, et pour parvenir à un but si honnête, il ruine ses propres États, il immole ses propres sujets pour avoir l’avantage d’exterminer ceux des autres ! Dans un héros de cette trempe la raison ne peut voir qu’un furieux, un brigand, un malheureux sans honneur et sans gloire. Le sage Plutarque a remarqué très bien que le surnom de {\itshape juste}, qu’il appelle {\itshape très royal et très divin}, donné au bon Aristide, n’a été nullement ambitionné par les grands rois. « Ils ont, dit-il, bien mieux aimé être appelés {\itshape poliorcètes}, preneurs de villes, {\itshape cerauni}, foudres de guerre, {\itshape nicanors}, ou vainqueurs ; quelques-uns même ont pris plaisir à se voir donner les noms d’{\itshape aigle} et de {\itshape vautours}, préférant ainsi le vain honneur de ces titres, qui ne marquent que la force et la puissance, à la solide gloire de ceux qui marquent la vertu\phantomsection
\label{footnote12}\footnote{Voyez Plutarque, {\itshape Vie d’Aristide}. À ces fléaux de l’Antiquité, l’Histoire moderne peut opposer des Richard {\itshape Cœur de Lion}, des Robert {\itshape le Diable}, et la troupe des princes qui ont mérité le surnom de grands par les grands maux qu’ils ont fait à leurs propres nations et à celles qui ont eu le malheur d’exercer leurs grandes âmes.}. »\par
Un conquérant estimable est celui qui se dompte lui-même et sait mettre un frein à ses passions. On prétend que la morale n’est point faite pour les héros. Dans ce cas, un héros n’est qu’une bête féroce qui n’est faite ni pour vivre avec des hommes ni pour les gouverner. Ceux qui ont la bassesse de louer ces prétendus grands hommes, dont la gloire consiste à écraser les nations sous le char de la victoire, les encouragent au crime et méritent d’être, comme eux, dévoués à l’infamie.
\subsection[{Chapitre XI. De la Tempérance. De la Chasteté. De la Pudeur}]{Chapitre XI. De la Tempérance. De la Chasteté. De la Pudeur}
\noindent Les passions sont des effets naturels de l’organisation des hommes et des idées qu’ils se font ou qu’on leur donne du bonheur ; mais si l’homme est un être raisonnable et sociable, il doit avoir des idées vraies de son bien-être et tâcher de l’obtenir par des voies compatibles avec les intérêts de ceux auxquels la société l’unit. Un inconsidéré qui suit les impulsions aveugles de ses passions n’est ni un être intelligent, ni un être sociable et doué de raison. L’être intelligent est celui qui prend de justes mesures pour obtenir son bonheur ; l’être sociable est celui qui concilie son bien-être avec celui de ses semblables. L’être raisonnable est celui qui distingue le vrai du faux, l’utile du nuisible, et qui sait qu’il doit mettre un frein à ses désirs. L’homme n’est jamais ce qu’il doit être s’il ne montre de la retenue dans sa conduite.\par
La tempérance est dans l’homme l’habitude de contenir les désirs, les appétits, les passions nuisibles soit à lui-même, soit aux autres. Cette vertu, de même que toutes les autres, est fondée sur l’équité. Que deviendrait une société dans laquelle chacun se permettrait de suivre ses fantaisies les plus déréglées ? Si chacun pour son intérêt souhaite que ses associés résistent à leurs caprices, il doit reconnaître que les autres ont droit d’exiger qu’il contienne les siens dans les bornes prescrites par l’intérêt général.\par
D’un autre côté, si, comme on l’a dit plus haut, l’homme isolé lui-même doit, en vue de sa conservation et de son bonheur durable, refuser de satisfaire ses appétits désordonnés, il y est encore plus obligé dans la vie sociale, où ses actions influent sur un grand nombre d’êtres qui réagissent sur lui-même. Si les excès du vin sont capables de nuire à tout homme qui s’y livre, ils lui nuiront encore bien plus dans la société, où ces excès l’exposent au mépris et peuvent, en troublant sa raison, le porter à des actions punissables par les lois.\par
Quelques moralistes sévères, pour rendre l’homme tempérant lui ont prescrit un divorce total avec tous les plaisirs, et même lui ont ordonné de les haïr, de les fuir. Des maxime si dures mettraient l’homme dans une guerre continuelle contre sa propre nature et sembleraient se proposer d’en faire un misanthrope ennemi de lui-même et désagréable à la société.\par
Les appétits de l’homme doivent être sans doute réglés par la raison ; tout lui prouve qu’il est des plaisirs dont il doit se priver pour son propre avantage, et cela par la crainte des conséquences souvent terribles qu’ils pourraient avoir pour lui-même et pour ses associés. C’est contre les séductions des plaisirs de cette espèce que l’être sociable doit se mettre en garde. C’est contre des passions injustes et criminelles qu’il doit apprendre à combattre sans cesse, afin de contracter l’habitude d’y résister.\par
L’habitude, en effet, nous rend faciles des choses qui d’abord nous paraissaient impossibles\phantomsection
\label{footnote13}\footnote{« Gravissimum est imperium consuetudinis. » Publius Syrus.}. Un des principaux objets de l’éducation devrait être d’accoutumer de bonne heure les hommes à résister aux impulsions inconsidérées de leurs désirs, par la crainte des effets qui peuvent en résulter.\par
La tempérance a pour principe la crainte de déplaire aux autres et de se nuire à soi-même ; cette crainte, rendue habituelle, suffit pour contrebalancer les efforts des passions qui peuvent nous solliciter au mal. Tout homme qui ne serait point susceptible de crainte ne pourrait guère réprimer les mouvements de son cœur. Nous voyons que les hommes exempts de crainte par le privilège de leur état, sont communément les plus nuisibles à la société. Une crainte juste et bien fondée des êtres qui nous environnent et dont nous sentons le besoin pour notre propre félicité, constitue l’homme vraiment sociable et lui fait un devoir de la tempérance. C’est par elle qu’il s’habitue à réprimer les effervescences subites de la colère ou de la haine pour les objets qui mettent quelques obstacles à ses désirs. C’est par elle qu’il apprend à se refuser aux plaisirs déshonnêtes, c’est-à-dire qui le rendraient odieux ou méprisable à la société. C’est par elle qu’il résiste aux séductions de l’amour, cette passion qui produit tant de ravages parmi les hommes.\par
La chasteté, qui résiste aux désirs déréglés de l’amour, est une suite de la tempérance ou de la crainte des effets de la volupté. La passion naturelle qui porte un sexe vers l’autre est une des plus violentes dans un très grand nombre d’hommes, mais l’expérience et la raison font connaître les dangers de s’y livrer. Les lois de presque toutes les nations, les opinions de la plupart des peuples policés, conformes en ce point à la Nature et à la droite raison, ont mis des entraves à l’amour déréglé, pour prévenir les désordres qu’ils causeraient dans la société. C’est d’après les mêmes idées que la continence absolue, le célibat, le renoncement total aux plaisirs même légitimes de l’amour, ont été admirés comme des perfections, comme les efforts d’une vertu surnaturelle.\par
Les pensées enflamment les désirs, échauffent l’imagination, donnent de l’activité à nos passions. D’où il suit que la tempérance nous prescrit de mettre un frein même à nos pensées, de bannir de notre esprit celles qui peuvent nous rappeler des idées déshonnêtes capables d’irriter nos passions pour les objets dont l’usage nous est interdit.\par
Il est certain qu’en méditant sans cesse le plaisir qu’un objet peut nous causer ou que l’imagination nous exagère, nous ne faisons qu’attiser nos désirs, leur donner de nouvelles forces, les rendre habituels, les changer en des besoins impérieux que l’on ne peut dompter. « La tempérance, dit Démophile, est le vigueur de l’âme. » Elle suppose la force, qui mérita toujours la considération des hommes.\par
Ces réflexions confirmées par l’expérience nous doivent découvrir l’utilité de la {\itshape pudeur.} On peut la définir la crainte d’allumer en soi-même ou dans les autres des passions dangereuses, par la vue des objets capables de les exciter.\par
Quelques penseurs ont cru que le sentiment de la pudeur n’avait pour base que le préjugé, les conventions des hommes, les usages des peuples policés. Mais en regardant la chose de près, on sera forcé de reconnaître que la pudeur est fondée sur la raison naturelle, qui nous montre que si la volupté et la débauche sont capables de produire des ravages dans la société, il est évidemment démontré que l’intérêt de la société demande que l’on voile avec soin les objets faits pour éveiller des désirs criminels. Si l’on nous cite l’exemple des sauvages qui vont tous nus et qui n’ont aucune idée de la pudeur, nous dirons que les sauvages sont des hommes que leur raison peu cultivée ne doit aucunement faire prendre pour modèles. L’impudent Diogène lui-même disait que {\itshape la pudeur est la couleur de la vertu}.\par
Par la même raison, la tempérance, qui met un frein à nos pensées et à nos actions, nous prescrit d’en mettre en nos paroles, nous interdit les discours déshonnêtes, condamne ces écrits obscènes dont l’effet nécessaire est d’allumer la pudeur, de présenter des images lascives capables d’allumer les passions des hommes.\par
Ce fut évidemment pour habituer les hommes à la tempérance que le cynisme et le stoïcisme ont engagé leurs sectateurs à se priver des plaisirs et des commodités de la vie. Sur le même principe, Pythagore prescrivit un silence rigoureux à ses disciples. Enfin, c’est pour affaiblir les passions des hommes que quelques religions ont prescrit des abstinences, des jeûnes, des mortifications, dont le but était visiblement d’habituer à la tempérance, d’accoutumer à se priver des choses capables d’enflammer les passions. Si ces préceptes ont été quelquefois outrés par quelques législateurs bizarres, ils partaient au moins d’un principe raisonnable. La médecine ne nous montre-t-elle pas dans la diète ou le jeûne, le remède le plus sûr contre un grand nombre de maladies ? L’abstinence totale du vin ordonnée par l’Alcoran, si elle était plus fidèlement observée, exempteraient les musulmans d’un grand nombre d’accidents auxquels l’ivrognerie si commune expose les habitants de nos contrées.\par
Les vertus portées à l’excès cessent d’être des vertus et deviennent des folies ; les idées de perfection poussées trop loin sont fausses dès qu’elles nous invitent à nous détruire. Elles sont alors des effets de l’orgueil, qui prétend s’élever au-dessus de la nature humaine, ou d’une imagination en délire. La vraie tempérance est accompagnée de la {\itshape modération}, qui nous fait éviter les excès en tout genre. La vraie morale, toujours guidée par la raison et la prudence, prescrit à l’homme de vivre suivant sa nature et de ne point prétendre s’élever au-dessus d’elle ; elle sait que des préceptes trop rigoureux sont inutiles pour le plus grand nombre des mortels et ne tendent qu’à faire des enthousiastes orgueilleux ou des fourbes hypocrites. Les joghis, ou pénitents de l’Inde, sont des fourbes et non des hommes tempérants. Le fanatique qui fait consister la perfection à s’affaiblir ou à se détruire peu à peu, devient un membre inutile de la société.
\subsection[{Chapitre XII. De la Prudence}]{Chapitre XII. De la Prudence}
\noindent L’homme en société est obligé de concerter ses mouvements avec ceux des êtres qui l’entourent {\itshape ;} il a besoin de leur assistance, de leur affection, de leur estime, et il doit prendre les moyens de les concilier. Voilà ce qui constitue la {\itshape prudence}, que l’on met communément au nombre des vertus. La prudence n’est que l’expérience et la raison appliquées à la conduite de la vie. On peut la définir [comme] l’habitude de choisir les moyens les plus propres à nous concilier la bienveillance et les secours des autres, et de nous abstenir de ce qui peut les indisposer. L’expérience, fondée sur la connaissance des hommes, nous rend prudents, c’est-à-dire nous indique comment il faut agir pour leur plaire et ce qu’il faut éviter pour ne pas perdre leur attachement ou leur estime, dont nous avons un besoin continuel.\par
La justice est la base de la prudence, comme de toutes les autres vertus. Perpétuellement exposés à souffrir impatiemment des imprudences, des étourderies, des défauts et des caprices des autres, nous sommes forcés d’en conclure qu’une conduite qui nous déplaît en eux doit nécessairement leur déplaire en nous et nuire aux sentiments que nous voulons éprouver de leur part.\par
La circonspection qui, suivant la force du mot, consiste à {\itshape regarder autour de soi}, à faire attention aux êtres qui nous environnent, est une qualité nécessaire à quiconque veut vivre en société. L’étourdi semble oublier qu’il est avec d’autres hommes dont il doit respecter les droits, ménager l’amour-propre, mériter la bienveillance ; il agit comme un insensé qui, les yeux fermés, se précipiterait dans une foule où il heurterait tous ceux qu’il trouverait sur son chemin sans songer qu’il est lui-même exposé aux coups de ceux dont il provoque la colère.\par
Telle est communément la position du méchant ; armé contre tous, il s’expose aux coups de tous. L’imprudence, l’inadvertance, l’étourderie, fruits ordinaires de la légèreté, de la dissipation, de la frivolité, sont des sources de désagréments.\par
L’homme sociable est fait pour réfléchir, pour s’observer lui-même et pour songer aux autres. Si le bonheur est un objet qui mérite notre attention, il suit que chacun de nous a le plus grand intérêt d’être à ce qu’il fait, de peser ses démarches, d’examiner si la route qu’il tient peut le conduire au but qu’il se propose. Le tumulte des plaisirs, la dissipation continuelle, une vie trop agitée, sont des obstacles au développement de la raison humaine. La frivolité, la légèreté, l’incurie, sont des dispositions fâcheuses en ce qu’elles nous empêchent d’accorder aux objets les plus intéressants pour nous des moments que nous ne croyons dus qu’au plaisir. Voilà la source véritable de la plupart des maux qui troublent notre vie. Beaucoup d’hommes demeurent dans une enfance perpétuelle et meurent sans être jamais parvenus à l’âge de maturité ; la gravité des mœurs y paraît ridicule et déplacée. Personne n’est sérieusement occupé de ce qu’il fait, personne ne s’embarrasse des objets les plus nécessaires à la félicité durable, chacun ne songe qu’à se procurer des amusements passagers, sans travailler à fonder un bien-être solide.\par
« La gravité, dit un illustre philosophe, est le rempart de l’honnêteté publique ; aussi le vice commence par déconcerter celle-là, afin de renverser plus sûrement celle-ci\footnote{M. Diderot, {\itshape Encyclopédie}, article « Gravité ».}. » La gravité dans les mœurs est une attention sur soi fondée sur la crainte de faire par inadvertance des actions capables d’indisposer les êtres avec qui nous vivons. Cette sorte de gravité est le fruit de l’expérience ou d’une raison exercée {\itshape ;} elle convient à tout être vraiment sociable qui, pour mériter la bienveillance des autres, doit mesurer sa conduite, ses discours, et montrer par son maintien même qu’il prête l’attention nécessaire aux objets qui le méritent. La gravité devient ridicule et se change en pédanterie quand, fondée sur une vanité puérile, elle n’a pour objet que des minuties qu’elle traite avec importance ; alors elle est méprisable, parce qu’elle exige du respect pour des choses peu dignes d’occuper des êtres raisonnables. La gravité décente et convenable est celle qui fait respecter des objets vraiment importants pour la société et qui montre que nous nous respectons nous-mêmes, ainsi que nos associés. Elle est alors fondée sur la prudence ou sur la juste crainte de perdre la bonne opinion de ceux avec qui nous avons des rapports.\par
Dans le langage ordinaire, rien de plus commun que de confondre la prudence avec la finesse, la ruse, avec l’art souvent blâmable de parvenir à ses fins. La vraie prudence est le choix des moyens nécessaires pour nous rendre heureux dans le monde. Ulysse était un fourbe sans être un homme prudent.
\subsection[{Chapitre XIII. De la Force, de la Grandeur d’âme, de la Patience}]{Chapitre XIII. De la Force, de la Grandeur d’âme, de la Patience}
\noindent Les moralistes, tant anciens que modernes, ont fait une vertu de la {\itshape force}. Les uns ont désigné sous ce nom la valeur guerrière, le courage, qui fait braver les dangers et la mort quand il s’agit des intérêts de la patrie. Cette disposition est sans doute utile et nécessaire ; par conséquent, elle est une vertu quand elle a véritablement pour but la justice, la conservation des droits de la société, la défense de la félicité publique.\par
Mais la force n’est plus une vertu quand elle cesse d’avoir la justice pour base, quand elle nous fait violer les droits des hommes, quand elle se prête à l’injustice. Le courage ou la force d’un Romain, que nous trouvons qualifié de vertu par excellence, n’était qu’un attentat contre les droits des plus saints de tous les peuples de la terre. C’est sous ce point de vue qu’un écrivain célèbre a dit avec raison, que « le courage n’est point une vertu, mais une qualité heureuse commune aux scélérats et aux grands hommes\phantomsection
\label{footnote14}\footnote{M. de Voltaire.} ». Caton a dit dans le même esprit « qu’il y a bien de la différence entre estimer la vertu et mépriser la vie\footnote{Voyez Plutarque dans la {\itshape Vie de Pélopidas} — « Ne tire point l’épée, dit Phocylide, pour tuer, mais pour défendre. » Voyez Phocylide, {\itshape Carm.}, vers 29. Plutarque rapporte dans la vie du même Pélopidas une belle épitaphe faite en l’honneur de quelques Lacédémoniens qui avaient péri dans un combat : « Ceux-ci sont morts persuadés que le bonheur ne consiste ni à vivre ni à mourir, mais à faire l’un et l’autre avec gloire. »} ».\par
La force est, suivant les stoïciens, la vertu qui combat pour la justice. D’où l’on voit qu’elle n’est aucunement la vertu des conquérants et de tant de héros célébrés dans l’Histoire. La force de l’homme de bien est la vigueur de l’âme affermie dans l’amour de ses devoirs et inviolablement attachée à la vertu. C’est une disposition habituelle et raisonnée à défendre les droits de la société et à lui sacrifier ses intérêts les plus chers. Les âmes bien pénétrées de l’amour du bien public sont susceptibles d’un enthousiasme heureux, d’une passion si forte qu’elle les transporte au point de s’oublier elle-même ; des cœurs bien épris du désir de la gloire ne voient rien que cet objet et s’immolent pour l’obtenir. La crainte de l’ignominie a souvent plus de pouvoir que la crainte de la mort.\par
Ces dispositions sont rendues habituelles par l’exemple, par l’opinion publique, qui, prêtant des forces continuelles aux imaginations ardentes, les déterminent à des actions qui souvent paraissent surnaturelles.\par
Dans une société tous ses membres ne sont point susceptibles de cette ardeur louable et de cette grandeur d’âme qui raisonne : la valeur militaire n’est dans le plus grand nombre des soldats que l’effet de l’imprudence, de la légèreté, de la témérité, de la routine. Les idées de bien public, de justice, de patrie sont nulles pour la plupart des guerriers ; ils sont peu accoutumés à réfléchir sur ces objets trop vastes pour leurs esprits frivoles ; ils combattent soit par la crainte du châtiment, soit par la crainte de se déshonorer aux yeux de leurs camarades, dont l’exemple les entraîne.\par
Si la valeur guerrière n’est pas également nécessaire à tous les membres d’une société, la fermeté, le courage sont des qualités très utiles dans tous les états de la vie : la force morale est une disposition avantageuse et pour nous-mêmes et pour les autres. Elle produit la {\itshape constance, la fermeté, la grandeur d’âme, la patience}. La tempérance, comme on a vu, suppose la force de résister à nos passions, de réprimer les impulsions de nos désirs déréglés. Il faut de la force pour persévérer dans la vertu, qui dans mille circonstances semble contraire à nos intérêts du moment.\par
La force, la constance, la fermeté, seront toujours regardées comme des dispositions louables dans les êtres de notre espèce. Les femmes elles-mêmes haïssent les lâches, parce qu’elles ont besoin de protecteurs. Nous admirons la force de l’âme quand elle porte à des grands sacrifices ; nous n’aimons que les hommes sur la constance et la fermeté desquels nous croyons pouvoir compter. Par la même raison, la faiblesse, l’inconstance nous déplaisent ; nous n’aimons à traiter qu’avec des hommes en qui nous supposons un caractère solide, capable de résister aux séductions momentanées qui détournent les autres du but qu’ils se proposent.\par
Les hommes ont une telle estime pour la force qu’ils l’admirent même dans le crime. C’est là, comme on a vu plus haut, la source de l’admiration que les peuples ont souvent pour les destructeurs du genre humain. En général, tout ce qui annonce une grande vigueur, une grande fermeté, une grande opiniâtreté, paraît surnaturel au vulgaire qui s’en trouve incapable. Voilà sans doute le principe de la vénération qu’excitent en lui les grandes austérités, les genres de vie extraordinaires, les singularités par lesquelles des fanatiques ou des imposteurs s’attirent quelquefois les regards. En un mot, tout ce qui marque la force, tant au physique qu’au moral, en impose toujours. « Le monde, dit Montaigne, ne pense rien utile qui ne soit pénible ; la facilité lui est suspecte. » Voilà pourquoi souvent il admire des tours de force qui ne prouvent aucunement la vertu : tels sont peut-être les fondements de la vénération que les Anciens et les modernes ont eu pour la morale austère et souvent insociable des stoïciens.\par
La force n’est une vertu que lorsqu’elle est utile ou lorsqu’elle donne de la consistance aux autres vertus. La force et la fermeté dans les choses qui ne sont d’aucune utilité ne prouvent qu’une vanité puérile. La fermeté dans des choses nuisibles ou désagréables aux autres, vient d’un orgueil coupable et doit attirer le mépris. La vraie force est la fermeté dans le bien ; l’opiniâtreté est la fermeté dans le mal. L’obstination, la raideur dans le caractère, la dureté, une humeur implacable, le défaut d’indulgence, l’impolitesse sont des vices réels par lesquels des hommes bornés s’imaginent quelquefois se rendre très estimables. Ces dispositions, qui causent des ravages et des désagréments dans le monde, partent pour l’ordinaire de présomption et de petitesse. Se rendre à la raison, ne jamais résister à l’équité ou à la sensibilité de son cœur, avoir égard aux conventions et aux usages raisonnables, faire céder son amour-propre à celui des autres sont des qualités qui nous rendent aimables et qui montrent bien plus de noblesse et de force qu’une inflexibilité farouche ou qu’une sotte vanité. La vraie force est celle qui rend inflexible toutes les fois qu’il s’agit de la vertu ; pour être louable, elle doit toujours être accompagnée d’une timidité qui fait craindre de déplaire aux autres, de les blesser, de perdre ses droits sur leur estime et leur amour. Cette sorte de timidité est très compatible avec le courage, la grandeur d’âme et la force ; elle est, comme celle-ci, la gardienne des vertus\footnote{Plutarque dit que « ceux qui sont les plus craintifs et les plus timides pour les lois, sont ordinairement les plus vaillants et les plus intrépides contre les ennemis ; et ceux qui craignent le plus la mauvaise réputation craignent le moins les douleurs, les peines et les blessures. C’est pourquoi celui-là a eu une grande raison qui a dit là où est la peur, là est aussi la honte. » Il avait dit auparavant que les Lacédémoniens avaient des chapelles consacrées à la peur, persuadés que la peur est le lien de toute bonne police. Voyez Plutarque, dans la {\itshape Vie d’Agis et de Cléomène}.}.\par
La grandeur d’âme véritable suppose de la vertu. Sans cela elle ne serait qu’une vaine présomption. Ce n’est que la juste confiance dans ses facultés qui permet d’entreprendre de grandes choses sans s’étonner des obstacles si effrayants pour le commun des hommes.\par
La grandeur d’âme, fondée sur la conscience de sa propre dignité, met l’homme vertueux au-dessus des injures, des affronts et des discours qui troublent et flétrissent tant de cœurs pusillanimes. Suivant Plutarque, les Spartiates, si fameux pour leur courage, demandaient aux dieux dans leurs prières {\itshape la force de supporter les injures}. La grandeur d’âme les fait pardonner ; supérieure à l’envie, à la médisance, à la calomnie, elle méprise leurs traits impuissants, qu’elle sait incapables de la blesser ou de troubler sa sérénité. La grandeur d’âme est franche et vraie parce que fortifiée par la conscience de son mérite ; elle ne sent pas le besoin de tromper et de séduire par des ruses : ce sont de vils moyens qu’elle abandonne à la faiblesse. La grandeur d’âme est bienfaisante et généreuse parce qu’il faut de l’énergie pour sacrifier ses intérêts à ceux des autres. La grandeur d’âme donne aux actions de l’homme inviolablement attaché à la vertu, cette vigueur que l’on regarde comme un désintéressement héroïque. Par elle, comme dit Sénèque, « la mauvaise opinion qu’on donne de soi cause souvent du plaisir quand c’est par une bonne action ». La conscience assurée de l’homme de bien le met alors au-dessus des jugements du public et le dédommage de ses iniquités. Il n’est personne à qui l’homme vertueux ne paraisse plus grand [que] lorsqu’il supporte avec courage les injustices du sort : il semble alors mesurer ses forces contre celles du destin et lutter avec lui corps à corps. Sénèque dit « qu’il n’est pas de spectacle plus grand pour les dieux et les hommes, que de voir l’homme de bien aux prises avec la fortune ». Mais ce spectacle (indigne, sans doute, des dieux, maîtres de la fortune) est fait pour intéresser et toucher vivement les mortels qui sont eux-mêmes en butte aux coups du sort. C’est à la grandeur d’âme ou à la force qu’est due la patience, cette qualité que tant de braves prétendus regardent comme une marque de petitesse et de lâcheté. Il est important pour les hommes de fortifier leurs âmes et de se préparer d’avance à supporter tant de maux dont la vie est à tout moment assiégée. Que deviendrait la société si ceux qui la composent ne pouvaient consentir à se tolérer les uns les autres ? La patience est donc une vertu sociale ; elle nous met en état de soutenir les disgrâces de la fortune, les défauts et les infirmités des hommes, les malheurs de la vie.\par
Rien de plus nécessaire dans les vicissitudes continuelles auxquelles les choses humaines sont sujettes, que d’être prêt à les soutenir avec fermeté. « C’est un grand mal, dit Anacharsis, que de ne pouvoir souffrir aucun mal ; il faut souffrir, afin de moins souffrir. » Se livrer en effet à des mouvements continuels d’impatience, s’irriter de tout ce qui nous contrarie, ce n’est pas soulager sa peine, c’est la redoubler sans cesse, c’est envenimer à tout moment les plaies que le temps pourrait guérir. L’homme impatient est très malheureux dans la société, qui lui fournit incessamment des causes de trouble et de mauvaise humeur. Celui qui est privé de patience est un homme faible dont le bien-être dépend de quiconque veut le tourmenter. La patience est la mère de l’indulgence, si nécessaire, comme nous le verrons bientôt, dans toutes les positions de la vie. Une sotte vanité persuade à quelques gens qu’il y va de leur gloire de ne rien endurer, mais l’expérience journalière nous montre que l’homme doux et patient intéresse tout le monde, et qu’on l’estime bien plus que celui qui se laisse emporter par la colère. Il serait essentiel d’accoutumer la jeunesse bouillante à calmer l’impatience, à se soumettre à la nécessité, contre laquelle il est toujours inutile de se révolter, et de la prémunir ainsi contre les adversités dont personne ne peut se flatter d’être toujours exempt. En un mot, la force est une vertu qui sert d’appui à toutes les autres. Il faut de la fermeté dans un monde corrompu ; des hommes lâches et pusillanimes ne font que chanceler dans le chemin de la vie. Sans une audace généreuse, il ne se trouverait personne qui eût le courage d’annoncer la vérité. Elle ne trouve communément que des ennemis implacables dans ceux qui devraient l’aimer et la prendre pour guide.
\subsection[{Chapitre XIV. De la Véracité}]{Chapitre XIV. De la Véracité}
\noindent Socrate disait que la vertu et la vérité étaient la même chose\footnote{Wollaston réduit toutes les notions du bien et du mal moral à celles de la vérité et du mensonge. Mais cette idée paraît être plus subtile que vraie. Sénèque disait pareillement que « le bien est toujours joint au vrai ; car s’il n’était pas vrai, il ne serait pas bien, il n’en aurait que l’apparence ». « La vérité, dit Pindare, est le fondement de la vertu la plus sublime. »}. En effet, si la vérité, comme tout le prouve, est un besoin pressant pour l’homme, si elle est de la plus grande utilité à tout le genre humain, si elle est l’objet des recherches de l’être raisonnable, il semble que les moralistes auraient dû placer la véracité au nombre des vertus sociales. Nous la définirons [comme] une disposition habituelle à manifester aux hommes des choses utiles et nécessaires à leur félicité.\par
Cette vertu, comme toutes les autres, est visiblement dérivée de la justice puisqu’elle est fondée sur le pacte social qui nous oblige de contribuer au bien-être de nos semblables, objet que nous ne pouvons remplir qu’en les assistant de nos conseils, de nos expériences, de nos lumières. Tout homme sociable doit la vérité à ses associés, par la même raison qu’il leur doit ses secours afin d’acquérir le droit de compter sur les leurs. Celui qui trompe ressemble à ceux qui répandent de la fausse monnaie dans le public. Celui qui refuse de communiquer à ses semblables des vérités utiles à leur bonheur, peut être comparé à l’avare qui ne fait part de son trésor à personne. Les hommes n’aiment la vérité que parce qu’elle leur est utile ; ils cessent de l’aimer lorsqu’ils la croient contraire à leurs intérêts. Mais nos égarements viennent pour l’ordinaire de ce que nous attachons l’idée d’utilité à des choses nuisibles, et ensuite l’idée de vérité à ce que nous avons jugé faussement être utile. Dire la vérité aux hommes, c’est leur apprendre ce qui est réellement et constamment utile à leur bien-être, et non ce qui n’est utile que d’après leurs préjugés. Les vérités que l’on nomme {\itshape dangereuses} sont celles qui contrarient les préjugés publics ; mais ces vérités n’en sont pas moins utiles pour cela, puisque les plus grandes calamités des nations sont dues à des opinions fausses, à des préjugés dangereux dont elles sont les victimes. Quiconque eût dit à Rome qu’un peuple conquérant n’est qu’une troupe de brigands détestables eût passé pour un insensé, et le Sénat ambitieux n’eût pas manqué de le punir comme un perturbateur du repos public, comme un ennemi de la patrie. Cependant, aux yeux de tout homme vertueux, ce citoyen courageux aurait paru très sage, très ami de la paix, très ami du genre humain, très ami des Romains mêmes, qu’il eût cherché à détromper de leurs préjugés injustes et barbares auxquels ils se sacrifiaient tous les jours. Les magistrats des Amycléens, fatigués des fausses nouvelles qui plusieurs fois avaient menacé leur ville d’un siège, défendirent sous peine de mort qu’on en parlât davantage. En conséquence du silence imposé par cette loi, les ennemis vinrent tout de bon, la ville fut prise et ses habitants furent égorgés. Il ne se trouva pas de citoyen assez généreux pour avertir sa patrie du péril auquel elle se trouvait exposée. Un Amycléen eût-il donc été coupable si, méprisant une loi extravagante, il eût annoncé hardiment une vérité dangereuse mais nécessaire au salut de tous ses concitoyens ?\par
La véracité n’est une vertu que lorsqu’elle découvre aux hommes des objets nécessaires à leur bonheur, à leur conservation, à leur félicité permanente. Elle cesse d’être utile et devient même un mal quand elle les afflige sans profit ou lorsqu’elle nuit à leurs intérêts réels. Si j’annonce brusquement à une mère tendre, sensible, accablée par la maladie, que son enfant chéri est en danger de mourir tandis qu’elle est dans l’impossibilité de sauver ses jours, je lui dis une vérité inutile et nuisible, je lui cause un mal réel, je lui porte le coup de la mort. Si un tyran envoie des assassins pour égorger mon ami vertueux, suis-je obligé de leur découvrir que cet ami s’est réfugié chez moi ? Non, sans doute : je me rendrais criminel en découvrant la vérité à des hommes assez pervers pour se rendre les ministres de l’ennemi de la société. Je ne dois la vérité que lorsqu’elle est utile ; elle est toujours inutile aux méchants. C’est donc à la prudence, à la raison, à la justice qu’il appartient de distinguer les vérités qu’il faut dire, de celles qu’il faut taire ou dissimuler, les vérités vraiment utiles de celles qui sont inutiles ou dangereuses. Toute vérité qui tend évidemment au bien de la société ne peut être cachée sans crime ; toute vérité qui, sans profit pour la société, peut nuire à quelques-uns de ses membres, est une vérité nuisible. La vérité dans la conduite se nomme {\itshape droiture, bonne foi, franchise, naïveté, candeur, fidélité}. Toutes ces dispositions sont désirables dans la vie sociale ; l’homme droit peut prétendre à l’estime et à la confiance de tous ceux qui ont des rapports avec lui. Les fourbes les plus décidés désirent de trouver dans les autres les qualités dont ils sont eux-mêmes dépourvus. Vouloir connaître les hommes, c’est désirer de savoir leurs dispositions véritables ; ceux qui montrent de la candeur, de la simplicité, ou qui ont, comme on dit, {\itshape le cœur sur les lèvres}, sont des êtres précieux dans le commerce de la vie. Nous craignons tout homme sombre et caché, parce que nous ignorons les moyens de traiter avec lui. Nous aimons un caractère ouvert et souvent, en faveur de la franchise, nous fermons les yeux sur ses défauts. La bonne foi et la véracité sont si rares parce que dès la plus tendre enfance, on s’accoutume au mensonge, à la dissimulation, à la fausseté. Ensuite, les vices et les mauvaises dispositions du cœur semblent forcer les hommes à ne se montrer que masqués ; il n’y a que l’homme de bien qui n’ait pas à craindre de se montrer à visage découvert. « Celui qui marche avec simplicité, dit le sage, marche avec confiance. »
\subsection[{Chapitre XV. De l’Activité}]{Chapitre XV. De l’Activité}
\noindent La vertu doit être agissante ; les vertus contemplatives sont inutiles à la société lorsqu’elle n’en peut pas ressentir les effets. De l’aveu de tous les moralistes, l’oisiveté et la paresse sont des dispositions méprisables et qui conduisent infailliblement au vice. L’intérêt de la société demande que chacun de ses membres contribue selon son pouvoir à la prospérité du corps. Il semblerait donc qu’on aurait dû faire une vertu de {\itshape l’activité}, de l’occupation, de l’amour du travail, dans lequel on peut trouver le moyen le plus juste et le plus honnête de subsister, ou du moins de se soustraire à l’ennui, cet impitoyable tyran de tous les désœuvrés.\par
Cela posé, nous définirons l’activité une disposition habituelle à contribuer par notre travail au bien de la société. Sénèque compare très justement la société à une voûte soutenue par la pression réciproque des pierres qui la composent\footnote{« Societas nostra lapidum fornicationi simillima est, quae, casura nisi in vicem obstarent ; hoc ipso sustinetur. » Sénèque, {\itshape Épître} 95, p. 471, tome 2, édition Varior. Je cite la page parce que cette épître est fort longue.}. Chaque corps, chaque ordre de citoyens, chaque famille, chaque individu doit à sa manière contribuer au soutien de l’ensemble ou, pour suivre la comparaison de Sénèque, il ne doit point y avoir de pierres détachées ; le législateur est la clef destinée à les contenir chacune dans leur place. Le souverain doit veiller à tout, ses ministres sont faits pour seconder ses vues, les magistrats doivent s’occuper de faire observer les lois, les grands et les puissants doivent soutenir les faibles, les riches doivent assister les pauvres, le cultivateur doit nourrir la société, le savant et l’artiste doivent l’éclairer et rendre ses travaux plus faciles, le soldat doit défendre ceux qui le font subsister.\par
L’homme désœuvré qui ne fait rien pour la société en est un membre inutile et ne peut sans injustice prétendre aux avantages de la vie sociale, à l’estime, aux honneurs, aux distinctions ; ces récompenses ne sont dues qu’à ceux dont la patrie peut tirer des secours. Voilà comment les intérêts particuliers se trouvent nécessairement unis à l’intérêt public et ne peuvent en être aucunement séparés.\par
Ces réflexions naturelles peuvent nous faire voir ce que nous devons penser de ces moralistes inconsidérés qui conseillent à des êtres sociables de se rendre sauvages, de se détacher de la société, de s’occuper uniquement d’eux-mêmes sans prendre aucune part à l’intérêt général. Une morale plus sensée fait un devoir à tout citoyen de contribuer suivant ses forces à l’utilité publique. Une sage politique doit appeler tous les citoyens au service de l’État et, guidée par la justice, elle devrait ne préférer à tous les autres que ceux qui se distinguent par leur activité, leurs talents et leur mérite personnel.\par
Dans une société juste et bien constituée, il ne doit être permis à personne de s’isoler ou de vivre inutile : ce n’est que dans une société corrompue que l’homme de bien, écarté par l’injustice, est forcé de se concentrer en lui-même. Toute nation soumise à la tyrannie peut être comparée à une voûte écrasée par le poids de sa clef, dont toutes les pierres sont disjointes. Dans cet édifice ruineux l’on ne trouve ni liaison, nul ensemble ; les corps sont ennemis des corps, chacun ne vit que pour soi, les citoyens se dispersent, il n’est plus d’esprit public, une profonde indifférence s’empare de tous les cœurs. Le sage, obligé de s’envelopper tristement du manteau philosophique, est réduit à jouir dans le cercle étroit de ses pareils du bien-être qu’il chercherait vainement au dehors.\par
L’ambition est une passion louable, noble et juste quand elle est excitée par l’idée de la considération attachée aux services que l’on peut rendre à son pays ; cette passion est légitime quand elle est accompagnée de la volonté et de la capacité de faire un grand nombre d’heureux. Mais elle est très condamnable quand elle ne se propose que l’exercice d’un pouvoir injuste ; elle est basse quand elle ne veut exercer son empire que sur des malheureux ou profiter des débris du naufrage de la patrie. Le désœuvrement, l’inaction, la retraite sont des devoirs pour l’homme honnête toutes les fois qu’il se voit dans l’impossibilité de faire le bien ; l’activité n’est une vertu que lorsqu’elle contribue à l’utilité générale.\par
En réfléchissant à ces principes, on pourra facilement découvrir les causes de la plupart des désordres que l’on voit régner dans les sociétés. Par une suite nécessaire de l’injustice des politiques qui ne se proposent que leurs vils intérêts, l’activité de tous ceux qui participent au pouvoir n’a pour objet que leur intérêt personnel ; la vertu et les talents, exclus des places, sont forcés de languir dans l’inaction. La société se remplit de méchants qui ne sont actifs que pour lui faire du mal, ou de désœuvrés perpétuellement occupés à tromper leurs ennemis, soit par des amusements frivoles, soit par des vices honteux. C’est ainsi que le miel est continuellement dévoré par des frelons malfaisants très peu disposés à contribuer au bien d’une société pour laquelle ils n’ont aucun attachement.\par
Exciter au travail les citoyens, les employer suivant leurs talents, les empêcher d’être oisifs ou de profiter sans rien faire des travaux de la société, devrait être l’objet des soins continuels d’une sage politique. Tout homme qui travaille est un citoyen estimable ; tout homme qui vit dans l’inaction est un membre inutile que ses vices ne tarderont point à rendre incommode pour ses associés. Il faut avoir travaillé pour être en droit de goûter les douceurs du repos ; le repos continuel est de tous les états le plus fatigant pour l’homme\footnote{Un grand seigneur disait un jour en présence de son fermier « qu’il s’ennuyait à la mort » ; le fermier lui répondit : « C’est qu’il est toujours dimanche pour vous. »}. L’inaction rend l’esprit malade, de même que le défaut d’exercice remplit le corps d’infirmités\footnote{« L’inaction, dit l’auteur du livre sur {\itshape Les Mœurs} déjà cité, est une sorte de léthargie également pernicieuse à l’âme et au corps. » Partie II, chap. II, article II, §. I.}.
\subsection[{Chapitre XVI. De la Douceur. De l’Indulgence. De la Tolérance. De la Complaisance. De la Politesse, ou des qualités agréables dans la vie sociale}]{Chapitre XVI. De la Douceur. De l’Indulgence. De la Tolérance. De la Complaisance. De la Politesse, ou des qualités agréables dans la vie sociale}
\noindent Des vertus sociales qui viennent d’être examinées, il découle des qualités propres à nous rendre chers ceux qui les possèdent et dont l’absence devient souvent très fatale à l’harmonie sociale et à la douceur de la vie. Ces qualités sont vraiment utiles à la société, puisqu’elles tendent à rapprocher ses membres. Sans être des vertus, elles en dérivent ; toutes, comme elles, se fondent sur la justice, qui nous apprend que nous devons nous rendre aimables si nous voulons acquérir le droit d’être aimés. Un être vraiment sociable doit, pour son intérêt, posséder ou acquérir des dispositions propres à lui concilier l’attachement de ceux dont les sentiments favorables contribuent à sa félicité. Tout homme qui s’aime véritablement doit désirer de voir ce sentiment si naturel partagé par les autres. L’homme le plus vain, le plus présomptueux, est affligé lorsqu’il se voit privé des suffrages de ceux mêmes qu’il paraît mépriser.\par
L’indulgence et la douceur sont des dispositions très nécessaires dans la vie sociale, qui nous font supporter les défauts et les faiblesses des autres. Elles se fondent sur l’équité, qui nous fait sentir que pour obtenir grâce pour les défauts ou faiblesses auxquelles nous sommes sujets, nous devons pardonner et souffrir les infirmités que nous voyons dans ceux avec qui nous vivons. L’indulgence est le fruit d’une patience raisonnée, d’une grande habitude de nous vaincre nous-mêmes, de résister à la colère qui trop souvent nous soulève contre les personnes ou les choses dont nous sommes choqués.\par
Cette disposition est visiblement émanée de l’humanité, cette vertu qui, comme on a vu, nous fait aimer les hommes tels qu’ils sont. La compassion nous fait plaindre les méchants même, parce que tout nous montre qu’ils sont les premières victimes de leurs folies criminelles.\par
La douceur et l’indulgence véritables sont les fruits rares de la réflexion, de l’expérience et de la raison ; on peut les regarder dans les hommes vifs et sensibles comme le plus grand effort de la raison humaine. Ces dispositions ne se trouvent naturelles que dans un petit nombre d’âmes fortes et tendres à la fois, dont la Nature a pris soin de tempérer les passions. Les imaginations vives, les esprits impétueux, trouvent dans leur tempérament des obstacles invincibles à l’indulgence. La douceur a des droits sur tous les cœurs ; les hommes les plus emportés lui rendent hommage et se laissent désarmer par elle.\par
Plus l’homme est éclairé, et plus il sent le besoin d’indulgence\phantomsection
\label{footnote15}\footnote{« L’indulgence, dit un philosophe célèbre, est une justice que la faible humanité est en droit d’exiger de la sagesse. Or, rien de plus propre à nous porter à l’indulgence, à fermer nos yeux à la haine, à les ouvrir aux principes d’une morale humaine et douce, que la connaissance profonde du cœur humain ; aussi les hommes les plus éclairés ont-ils presque toujours été les plus indulgents. » Voyez le livre {\itshape De l’Esprit}, Discours I, chap. IV, p. 35, édition in-4°.}. Rien de moins indulgent que les ignorants et les sots. Le grand homme devrait être trop fort pour être blessé par des minuties indignes de l’occuper ; il ne s’aperçoit presque point des ridicules ou des défauts si frappants pour la malignité vulgaire. Les ignorants sont privés d’indulgence parce qu’ils n’ont jamais réfléchi à la fragilité humaine ; les sots manquent d’indulgence parce que les sottises des autres, et surtout des gens d’esprit, semblent dégrader ceux-ci et les rapprocher des sots. Il faut être né sensible et doux, il faut avoir de l’humanité, il faut s’être habitué à la modération, à la tempérance, à l’équité, pour avoir ou pour acquérir cette indulgence si nécessaire et si rare dans la vie sociale.\par
L’indulgence que nous avons pour les opinions et les erreurs des hommes est appelée {\itshape tolérance.} Pour peu que l’on consultât l’expérience, la raison, l’équité, l’humanité, on reconnaîtrait que rien n’est plus nécessaire que cette disposition, que rien n’est à la fois plus tyrannique et plus insensé que de haïr ou tourmenter nos semblables parce qu’ils ne pensent pas comme nous. Les hommes sont-ils donc les maîtres d’avoir ou de ne point avoir les opinions qui leur ont été inculquées dès l’enfance et qu’on leur a fait regarder comme essentielles à leur bonheur ? Est-il moins déraisonnable de détester un homme pour ses erreurs que pour n’être pas né des mêmes parents, pour n’avoir pas reçu les mêmes idées, pour n’avoir point appris la même langue que nous ? Les opinions vraies ou fausses sont des habitudes contractées dès l’âge le plus tendre et tellement identifiées avec celui qui les a reçues, qu’il est communément impossible de les déraciner\footnote{Montaigne dit avec grande raison « et ne fut jamais au monde deux opinions pareilles, non plus que deux poils, ou deux grains. Leur plus universelle qualité, c’est la diversité. » Voyez {\itshape Essais}, livre II, chap. 37, à la fin du chapitre. — Le docteur Swift a très bien remarqué que « les hommes ont communément assez de religion pour se haïr, mais qu’ils en ont rarement assez pour s’aimer les uns les autres ».}. Il est aussi peu juste d’haïr quelqu’un parce qu’il se trompe, que de haïr pour n’avoir pas d’aussi bons yeux, autant de dextérité, autant d’esprit que nous. Les erreurs des hommes sur des objets qu’ils jugent très importants pour eux sont toujours involontaires. Ils ne sont opiniâtres dans leurs idées que parce qu’ils croient très dangereux d’en changer. Vouloir les leur arracher, c’est vouloir qu’ils renoncent à leur bonheur par complaisance pour nous. Tout homme qui, se trouvant le plus fort, fait violence à un autre pour lui faire adopter ses propres opinions, met évidemment cet autre en droit de le violenter à son tour lorsqu’il sera le plus fort. Le mahométan qui, ayant la force de son côté, se croit en droit de tourmenter le bramine, le parsis ou le chrétien, donne évidemment à ceux-ci le droit de le tourmenter quand ils en auront le pouvoir.\par
En un mot, rien de plus injuste, de plus inhumain, de plus extravagant, de plus contraire au repos de la société, que de haïr et de persécuter ses semblables pour des opinions. Mais, dira-t-on, si ces opinions sont dangereuses, ne faut-il pas les étouffer ? Les opinions ne sont dangereuses que lorsqu’on veut les faire adopter par force à d’autres. Le crime est toujours du côté de celui qui le premier emploie la violence. Quiconque veut tyranniser mérite qu’on lui oppose la force et n’a pas droit de se plaindre quand on se sert des mêmes armes contre lui. Des agresseurs injustes peuvent être justement punis ou repoussés. On nous dira peut-être que celui qui a des opinions vraies est en droit d’user de la force pour ramener à la vérité ceux qui s’en sont écartés. Mais en matière d’opinions, chacun se tient assuré d’avoir la vérité pour lui ; et si d’après cette présomption l’on est autorisé à contraindre ou persécuter les autres, il est évident que tous les peuples de la terre, dont chacun croit jouir exclusivement de la vérité, seront autorisés à s’exterminer les uns les autres pour leurs systèmes divers.\par
D’où l’on voit que rien n’est plus propre à rendre les hommes insociables que le défaut d’indulgence en matière d’opinions. Si quelqu’un méritait d’être privé des droits de l’humanité, ce serait évidemment celui qui voudrait que l’on égorgeât sans pitié tous ceux qui ne penseraient pas comme lui.\par
L’homme en société devant pour son propre bien chercher à se rendre agréable, la {\itshape complaisance} honnête doit être regardée comme une qualité louable. On peut la définir [comme] une disposition habituelle de se conformer aux volontés justes et aux goûts raisonnables des êtres avec qui nous vivons. Quiconque refuse de se prêter aux désirs et aux plaisirs légitimes des autres montre de la présomption, annonce une humeur peu sociable et perd le droit d’exiger la complaisance de ses associés. La complaisance est un des liens les plus doux de la vie ; elle suppose la douceur du caractère, une facilité, une flexibilité propres à nous faire aimer. On ne doit pas la confondre avec une lâche condescendance pour les vices, ni avec une basse flatterie dont l’effet est de nourrir les dispositions les plus criminelles. Les bornes de la complaisance, ainsi que celles de toutes les autres qualités sociales, sont évidemment fixées par l’équité, qui défend de se conformer à des goûts vicieux et pervers. La complaisance devient coupable quand elle nuit soit à ceux à qui nous la montrons, soit à la société ; elle n’est pour lors qu’une bassesse indigne de tous nos mépris.\par
La complaisance juste, humaine, sociable, est l’âme de la vie ; elle resserre les liens de l’union conjugale, elle entretient l’amitié, elle nous habitue à contenter tous les êtres avec lesquels nous avons des rapports. La complaisance retenue dans ses justes limites nous rend chers à tout le inonde ; mais lorsqu’elle est excessive, elle nous fait mépriser de ceux mêmes à qui nous la témoignons. Elle doit être fondée sur la bonté, sur la philanthropie, sur un désir de plaire par des moyens équitables ; elle dégoûte et nous avilit dès qu’elle ne se propose qu’un intérêt sordide. La complaisance du courtisan, du parasite, du flatteur, n’indique que la bassesse de leurs âmes et les rend méprisables à ceux mêmes qui se repaissent de leur encens. Le véritable ami estime celui qu’il aime et ne lui demande que des choses incapables de le dégrader ; en exigeant une complaisance lâche, l’ami serait un vrai tyran.\par
Toutes les qualités sociales dont on vient de parler ne peuvent être sincères ou solidement établies que sur la bonté, la douceur de caractère, don précieux de la Nature que l’on ne rencontre guère dans les âmes impétueuses, dans les esprits hautains, dans les personnes privées d’éducation et de l’usage du monde : l’homme du peuple n’a point appris à se vaincre. Cependant, la morale fournit à ceux qui voudront la consulter des motifs pour combattre les impulsions de l’orgueil et d’un tempérament trop irascible : elle nous rappelle à l’équité. Elle nous montre que des êtres dépourvus de douceur, d’indulgence, de complaisance, révoltent tout le monde, et surtout les personnes les plus emportées. Elle nous prouve que la douceur, au contraire, vient à bout de la violence et réussit bien plus sûrement que la force ou la ruse. En rentrant souvent en lui-même, tout homme raisonnable peut parvenir à dompter son caractère et à donner à sa conduite le ton nécessaire pour plaire à la société. L’exemple des courtisans ne nous prouve-t-il pas à quel point le caractère peut être modifié ? L’on voit à la cour les hommes les plus fiers, les plus colères, les plus vains, supporter avec patience les affronts les plus cruels et n’opposer qu’un silence respectueux aux discours les plus offensants de leurs maîtres.\par
L’homme sociable est fait pour s’observer, pour se réprimer, pour travailler sur lui-même lorsque la Nature ne lui a point accordé les dispositions nécessaires pour se rendre agréable. Sous peine d’être puni par l’aversion de tous ceux qui l’entourent, un être susceptible de raison et de réflexion est obligé de se replier sur lui-même, de juger ses actions, de se condamner quand il a tort, de se corriger de ses défauts. Quiconque refuse de réprimer ses passions et son humeur fait nécessairement souffrir les autres et ne peut guère se flatter de s’attirer leur affection.\par
Il est encore d’autres qualités qui contribuent à rendre l’homme agréable dans le commerce de la vie ; telle est surtout la {\itshape politesse}, que l’on peut définir [comme] l’habitude de montrer aux personnes avec qui nous vivons les sentiments et les égards que se doivent réciproquement des êtres réunis en société. Telle est encore le soin de se conformer aux règles de la décence. Enfin, on doit mettre au nombre des dispositions faites pour contribuer à l’agrément de la vie, l’esprit, l’enjouement, la gaieté, les connaissances, soit utiles, soit agréables, les sciences, les goûts, les talents, etc. ; mais nous nous promettons d’entrer dans quelques détails sur ces qualités dans la suite de cet ouvrage\footnote{Voyez la seconde partie, section II, chap. VII.}.\par
En général, la vie sociale exige une attention sur nous-mêmes, un désir de plaire aux autres, une timidité raisonnable qui doit nous faire écarter de nos discours et de nos manières tout ce qui peut nous indisposer ; sans cette timidité louable, la société serait incommode et fâcheuse. Si la justice prescrit à tout homme de respecter son semblable, l’humanité lui fait un devoir de ménager ses faiblesses. Quiconque est trop altier pour plier son caractère et pour dompter son humeur, doit vivre seul et se montre peu fait pour le commerce des hommes.\par
Tout homme qui veut vivre agréablement ne doit jamais perdre de vue ses associés. Suivant un moraliste moderne très sensé, toute la vie de l’homme ne doit être « qu’un enchaînement d’attention sur le présent, de prévoyance pour l’avenir, et de retour sur le passé\footnote{Voyez les {\itshape Leçons de la Sagesse}.} ». Ainsi, comme nous allons le prouver, le méchant n’est jamais qu’un imprudent, un insensé, un étourdi qui, dans son ivresse ou sa folie, travaille continuellement à détruire le bonheur qu’il croit trouver en commettant le mal. Nul homme ne se suffit à lui-même, nul homme en société ne peut se rendre heureux aux dépens de tous les autres ; d’où il suit que, par la nature même des choses, nul homme ne peut nuire à ses semblables sans se nuire à lui-même.
\section[{Section III. Du Mal moral, ou des crimes, des vices et des défauts des hommes}]{Section III. Du Mal moral, ou des crimes, des vices et des défauts des hommes}\renewcommand{\leftmark}{Section III. Du Mal moral, ou des crimes, des vices et des défauts des hommes}

\subsection[{Chapitre I. Des Crimes, de l’Injustice, de l’Homicide, du Vol, de la Cruauté}]{Chapitre I. Des Crimes, de l’Injustice, de l’Homicide, du Vol, de la Cruauté}
\noindent L’examen qui vient d’être fait des vertus sociales, ainsi que des qualités désirables qui en sont dérivées ou qui les accompagnent, nous prouve que ce n’est qu’en les pratiquant que l’homme en société peut obtenir l’affection, l’estime, le bien-être vers lequel il ne cesse de soupirer. Des intérêts si évidents devraient être des motifs assez puissants pour déterminer tout être raisonnable, soit à cultiver les dispositions heureuses qu’il a reçues de la Nature, soit à tâcher de les acquérir et de se les rendre habituelles et familières en vue des récompenses qu’il y voit attachées, soit enfin à combattre, réprimer, anéantir, s’il est possible, les penchants déréglés, les passions dangereuses, les vices et les défauts dont l’effet infaillible serait de le rendre odieux, méprisable, punissable, malheureux. Montrons donc à tout homme, de la façon la plus claire, qu’il n’est point de vice qui ne soit sévèrement châtié et par la nature même des choses et par la société, et que toute conduite nuisible aux autres finit toujours par retomber sur celui qui la tient. « La peine, dit Platon, suit toujours le vice. » Hésiode dit {\itshape qu’elle naît avec lui}. L’homme cesse d’être heureux dès qu’il devient coupable.\par
Si, comme on l’a tant prouvé, la vertu est l’habitude de contribuer au bien-être de la vie sociale, le {\itshape vice} doit être défini [comme] l’habitude de nuire au bonheur de la société, dont étant nous-mêmes les membres, nous éprouvons la réaction nécessaire. Si la vertu seule mérite l’affection, l’estime, la vénération des hommes, le vice mérite leur haine, leur mépris, leurs châtiments. Si c’est dans la vertu seule que consiste la vraie gloire et l’honneur véritable, le vice ne peut attirer que la honte et l’ignominie. Si la bonne conscience ou l’estime méritée de soi est un bien réservé à l’innocence et à la vertu, la crainte, l’opprobre, le remords, le mépris de soi doivent être le partage du crime. Si l’homme vertueux peut seul passer pour véritablement sage, raisonnable, éclairé, le vicieux n’est qu’un aveugle, un insensé, un enfant dépourvu d’expérience et de raison qui méconnaît ses intérêts les plus chers. Si l’homme qui pratique la vertu est l’être vraiment sociable, tout nous montre que le méchant est un furieux qui s’occupe à briser les liens de la société, qui démolit la maison faite pour lui servir d’asile. Enfin, si toutes les vertus sont dérivées de la justice, tous les crimes, les vices et les défauts des hommes sont des violations plus ou moins marquées de l’équité, des droits de l’homme, de ce que l’être sociable se doit à lui-même et aux autres.\par
C’est être injuste que de nuire à ses associés, parce que nul homme n’a le droit de faire du mal à ses semblables ; c’est se nuire à soi-même que de s’attirer par sa conduite le mépris ou le ressentiment de la société, qui pour sa propre conservation est obligé de punir ceux qui l’outragent. L’on nomme {\itshape crimes, forfaits, attentats} les actions qui troublent évidemment la société. Le meurtre, l’oppression, la violence, l’adultère, le vol sont des crimes ou des violations graves de la justice, faites pour inspirer la terreur à tous les citoyens. Il n’est pas de membre de la société qui ne soit intéressé au châtiment de pareils excès dont chacun peut craindre de devenir la victime. Tout homme qui s’y livre se déclare l’ennemi de tous ; par là même il les avertit qu’il renonce à l’association, et par conséquent à la protection et au bien-être que la société ne s’est engagée de lui procurer que sous la condition expresse d’être juste, de contribuer à sa félicité, ou du moins de n’y mettre aucun obstacle. Le méchant déchaîne tous les hommes contre lui, il anéantit ses propres droits, il s’expose au ressentiment de ceux dont il a besoin pour sa félicité.\par
Si chez les hommes la vie est réputée le plus grand des biens, il n’en est pas que la société soit plus intéressée à défendre. L’homicide est donc très justement regardé comme l’attentat le plus noir que l’on puisse commettre. Celui qui arrache la vie à son semblable paraît dépourvu de justice, d’humanité, de pitié, et par conséquent il est un monstre contre lequel la société doit s’armer. Celui qui tue son bienfaiteur, à ces dispositions si criminelles il joint encore l’ingratitude la plus atroce. Celui qui tue son père doit inspirer une horreur particulière ; il paraît avoir foulé aux pieds des sentiments que l’habitude devrait avoir identifiés en lui. On suppose qu’après avoir franchi les obstacles et brisé les liens qui auraient dû l’empêcher de commettre un tel forfait, le parricide doit s’être familiarisé avec le crime au point de ne faire plus qu’un jeu de la vie des autres hommes.\par
Les crimes, en effet, de même que les vertus, sont souvent des effets de l’habitude. C’est peu à peu que les hommes deviennent méchants\phantomsection
\label{footnote16}\footnote{« Nemo repente fuit turpissimus. » Juvénal, {\itshape Satires}, II, vers 83.}. Le crime réfléchi paraît bien plus odieux que celui qui n’est que l’effet de l’effervescence passagère de quelque passion subite qui a pu produire dans l’homme une folie momentanée. Celui qui a commis un crime de cette manière devient un objet de pitié ; un crime unique n’annonce pas toujours un cœur totalement dépravé, mais le crime prémédité ou réitéré indique un naturel endurci dans le mal, pour qui la méchanceté est un besoin et qui dès lors est indigne de toute compassion. Les grands crimes annoncent un tempérament indompté, une sorte de délire, ou bien des dispositions funestes enracinées par l’habitude, qui rendent souvent l’homme capable de commettre les actions les plus atroces de sang-froid. Les Caligula, les Néron, les Commodore paraissent avoir été des fous très dangereux, sans doute, mais beaucoup moins odieux qu’un Tibère, dont la cruauté fut toujours tranquille et réfléchie.\par
Penser avec plaisir aux avantages qui peuvent résulter d’un crime, s’occuper sans relâche de l’intérêt qu’on peut trouver à le commettre, échauffer incessamment son imagination par la peinture du profit qui peut en revenir, voilà des degrés qui conduisent les hommes au crime ; ils s’enivrent au point de n’en plus voir les conséquences. Tout homme sujet à la colère souhaiterait dans le moment la destruction de celui qui l’irrite, mais accoutumé à réfléchir aux suites de ses actions, il frissonne à la vue du danger où pouvait l’exposer l’impulsion d’une passion téméraire. S’il a de la grandeur d’âme, il oublie l’offense qu’il a reçue et ne songe plus à s’en venger.\par
Les grands crimes annoncent communément le défaut d’une éducation propre à modifier les hommes, c’est-à-dire à les habituer à résister à leurs penchants aveugles. Les personnes bien élevées sont accoutumées à ne penser au crime qu’avec horreur ; l’idée seule d’un assassinat les fait trembler. Le vol ne se montre à leurs yeux qu’accompagné d’infamie, mais ces mêmes personnes cesseront de regarder l’homicide sous le même point de vue quand le préjugé leur aura persuadé qu’un dieu est une chose nécessaire à leur honneur. D’autres se permettront le vol et la rapine parce qu’ils s’y croiront autorisés par la loi, par l’usage et l’opinion ; combien de gens qui s’imaginent que la permission du prince les autorise à dépouiller les citoyens !\par
Pour fixer nos idées sur les actions des hommes, il est utile de les définir avec précision. Cela posé, le {\itshape vol} est toute action qui prive un homme injustement et contre son gré de ce qu’il a droit de posséder ; c’est une violation de la propriété que toute société s’engage de conserver à chacun de ses membres. Nulle loi ne peut autoriser des actions contraires au but de la société. Ainsi, tout homme juste ne se prêtera jamais à des opinions introduites par la tyrannie et contredites hautement par l’équité naturelle. Celle-ci défend à tous les hommes de s’emparer du bien des autres et fait un crime du vol, sous quelque nom que l’on cherche à le couvrir. Elle montre que les conquêtes sont des vols de royaumes et de provinces, et que les guerres injustes sont des assassinats. Elle montre que les impôts qui n’ont pas pour objet l’utilité publique sont des vols avérés, que les profits illicites, les émoluments injustes, le refus de payer ses dettes, les extorsions, les rapines et les concussions du despote sont des vols aussi criminels que ceux qui se font sur les grands chemins\phantomsection
\label{footnote17}\footnote{Les fripons se soucient fort peu d’appeler les choses par leur vrai nom. Quand les Arabes bédouins ont pillé une caravane ou détroussé des voyageurs, ils disent {\itshape qu’ils ont gagné} ce qu’ils ont pris. Les traitants appellent leur métier {\itshape travail}, et donnent le nom de {\itshape profits} aux fruits de leurs extorsions, qu’ils désignent sous le nom d’une {\itshape bonne affaire}. En bonne morale, tout homme qui s’empare du bien des autres ou qui jouit du salaire et des récompenses de la société sans aucun profit pour elle, est un voleur.}. Les voleurs ordinaires peuvent du moins rejeter leurs crimes sur la misère, sur le besoin, sur la nécessité qui ne connaît point de lois, au lieu que les tyrans et leurs suppôts ne volent souvent que pour acquérir du superflu, dont ils ne font qu’un usage évidemment contraire au bonheur et de la société particulière et de tout le genre humain.\par
Lorsque les nations sont corrompues, elles s’apprivoisent aisément avec les actions les plus criminelles. D’ailleurs, le nombre et le rang des coupables semble anoblir la conduite la plus déshonorante, et la négligence des législateurs paraît en quelque façon l’absoudre. Un grand qui emprunte de tous côtés, un prodigue qui, après avoir follement dissipé sa fortune, ruine ses créanciers, un commerçant qui, abusant de la confiance qu’on lui montre, dérange par son inconduite ou ses entreprises hasardeuses ses affaires propres et fait banqueroute aux autres, ne sont le plus souvent ni punis ni déshonorés ; ils se montrent effrontément dans le monde et quelquefois même y font trophée de leurs escroqueries. Mais aux yeux de l’homme juste, ces différents personnages ne sont que d’infâmes voleurs que les lois devraient punir, ou du moins qu’à leur défaut la bonne compagnie devrait exclure sans pitié. Si ceux qui vivent aux dépens des autres sont des voleurs, les adhérents et les parasites du prodigue ou du fripon endetté sont de vrais receleurs.\par
La morale nous fait porter un même jugement de tous ces vendeurs de mauvaise foi qui sans pudeur et sans remords profitent de la simplicité, du peu de connaissance ou du besoin des autres, pour les tromper indignement.\par
Bien des marchands se persuadent que leur profession les met en droit de faire saisir toutes les occasions de gagner, que tout gain est légitime, et ceux même qui en toute autre chose craindraient de violer les règles de la probité la plus sévère et de blesser leur conscience, n’ont plus ni probité ni conscience dès qu’il s’agit de leur métier. Bien plus, il est des hommes assez pervers pour se vanter ouvertement de l’abus honteux qu’ils ont fait de la crédulité des autres. L’ignorance trop commune où vit le peuple des vrais principes de la justice, fait que surtout dans les grandes villes, presque tous les petits marchands sont voleurs et fripons. Ce n’est que chez les commerçants d’un ordre plus relevé que l’on trouve de l’honneur et de la bonne foi, sentiments que la bonne éducation peut seule inspirer.\par
L’indigence, la paresse, le vice, poussent communément au crime. Les hommes qui jouissent du nécessaire ou qui l’obtiennent par leur travail, qui n’ont point de vices à satisfaire, ne sont guère tentés de voler ni de troubler la société. Les vices font commettre des crimes pour contenter des vices dont on a contracté la malheureuse habitude. L’homme du peuple, dès qu’il est sans rien faire, devient nécessairement vicieux et se livre à toutes sortes de crimes pour assouvir ses nouveaux besoins. L’homme opulent et puissant est communément rempli de vices et de besoins parce qu’il est désœuvré. La fortune la plus ample suffisant à peine pour rassasier sa cupidité, il se croit forcé de recourir au crime dans l’espoir frivole de se rendre plus heureux.\par
L’{\itshape injustice} peut se définir en général [comme] une disposition à violer les droits des autres en faveur de notre intérêt personnel. La {\itshape tyrannie} est l’injustice exercée contre toute la société par ceux qui la gouvernent. Toute autorité légitime n’étant fondée que sur les avantages que l’on procure à ceux sur qui elle est exercée, cette autorité devient une tyrannie dès qu’on en abuse contre eux ; elle n’est alors qu’une usurpation odieuse. Comme ce n’est qu’en vue de jouir des avantages de la justice que les hommes vivent en société, on voit très clairement que l’injustice anéantit le pacte social, et que pour lors la société ne rassemble plus que des ennemis toujours prêts à se nuire, des oppresseurs et des opprimés.\par
L’injustice relâche et dissout les liens de la société conjugale : un mari devenu tyran n’est pas en droit d’attendre de sa femme des sentiments d’amour, un père injuste ne trouve que des ennemis dans ses propres enfants, un maître injuste ne doit pas compter sur l’attachement de ses serviteurs ; tout homme injuste semble par sa conduite annoncer à tous ceux qui ont des rapports avec lui qu’il renonce à leur affection, qu’il consent à leur haine, qu’il n’a besoin de personne, qu’il ne songe qu’à lui. En un mot, la justice est le soutien du monde, et l’injustice est la source de toutes les calamités dont il est affligé.\par
Si l’humanité, la compassion, la sensibilité, sont des vertus nécessaires à la société, l’absence de ces dispositions doit être regardée comme odieuse et criminelle. Un homme qui n’aime personne, qui refuse ses secours à ses semblables, qui se montre insensible à leurs peines, qui se plaît à les voir souffrir au lieu d’être touché de leurs misères, est un monstre indigne de vivre en société et que son affreux caractère condamne à rester dans un désert avec les bêtes qui lui ressemblent. Être inhumain, c’est cesser d’être un homme ; être insensible, c’est avoir reçu de la Nature une organisation incompatible avec la vie sociale, ou bien c’est avoir contracté l’habitude de s’endurcir sur les maux que l’on devrait soulager. Être cruel, c’est trouver du plaisir dans les souffrances des autres, disposition qui ravale l’homme au-dessous de la brute : le loup déchire sa proie, mais c’est pour la dévorer, c’est-à-dire pour satisfaire le besoin pressant de la faim, au lieu que l’homme cruel se repaît agréablement l’imagination par l’idée des tourments de ses semblables, se plaît à les faire durer, cherche des manières ingénieuses de rendre plus piquants les aiguillons de la douleur, et se fait un spectacle, une jouissance des maux qu’il voit souffrir aux autres. Pour peu qu’on réfléchisse, on a lieu d’être consterné en voyant le penchant que les hommes, pour la plupart, ont à la cruauté. Tout un peuple accourt en foule pour jouir du supplice des victimes que les lois condamnent à la mort ; nous le voyons contempler d’un œil avide les convulsions et les angoisses du malheureux qu’on abandonne à la fureur des bourreaux. Plus ses tourments sont cruels, plus ils excitent les désirs d’une populace inhumaine sur le visage de laquelle on voit pourtant bientôt l’horreur se peindre. Une conduite si bizarre et si contradictoire est due à la curiosité, c’est-à-dire au besoin d’être fortement remué, effet que rien ne produit aussi vivement sur l’homme que la vue de son semblable en proie à la douleur et luttant contre sa destruction. Cette curiosité contentée fait place à la pitié, c’est-à-dire à la réflexion, au retour que chacun fait sur lui-même, à l’imagination qui le met en quelque façon à la place du malheureux qu’il voit souffrir. Au commencement de cette affreuse tragédie, attiré par sa curiosité, le spectateur est quelque temps soutenu par l’idée de sa propre sûreté, par la comparaison avantageuse de sa situation avec celle du criminel, par l’indignation et la haine que causent les crimes dont ce malheureux va subir le châtiment, par l’esprit de vengeance que la sentence du juge lui inspire ; mais à la fin, ces motifs cessant lui permettent de s’intéresser au sort d’un être de son espèce, que la réflexion lui montre sensible et déchiré par la douleur.\par
C’est ainsi que l’on peut expliquer ces alternatives de cruauté et de pitié si communes parmi les gens du peuple. Les personnes bien élevées sont pour l’ordinaire exemptes de cette curiosité barbare, mais plus accoutumées à penser, elles en deviennent plus sensibles et leurs organes moins forts auraient peine à résister au spectacle d’un homme cruellement tourmenté. D’où l’on peut conclure, comme on l’a dit ailleurs, que la pitié est le fruit d’un esprit exercé dans lequel l’éducation, l’expérience, la raison, ont amorti cette curiosité cruelle qui pousse le commun des hommes aux pieds des échafauds.\par
Les enfants sont communément cruels, comme on peut en juger par la manière dont ils traitent les oiseaux et les animaux qu’ils tiennent en leur puissance. On les voit pleurer ensuite lorsqu’ils les ont fait périr, parce qu’ils en sont privés ; leur cruauté a pour motif la curiosité, à laquelle vient se joindre le désir d’essayer leur force ou d’exercer leur pouvoir.\par
Un enfant n’écoute que les impulsions subites de ses désirs et de ses craintes ; s’il en avait la force, il exterminerait tous ceux qui s’opposent à ses fantaisies. C’est dans l’âge le plus tendre que l’on doit réprimer les passions de l’homme ; c’est alors qu’il faudrait soigneusement étouffer toutes les dispositions cruelles, l’accoutumer à s’attendrir sur les peines des autres, l’exercer à la pitié, si nécessaire et si rare dans la vie sociale\footnote{On dit qu’une nation sage refusa une charge de magistrature à un homme considérable, parce qu’on avait remarqué que dans sa jeunesse il prenait plaisir à déchirer des oiseaux. Dans un autre pays, un homme fut chassé du Sénat pour avoir écrasé un oiseau qui était venu se réfugier dans son sein. Voyez Addisson, {\itshape Mentor Moderne}, n° 61.}.\par
L’Histoire nous montre les trônes souvent remplis par des tyrans farouches et cruels ; rien de plus rare que des princes à qui l’on ait appris dans l’enfance à réprimer leurs mouvements déréglés. On leur donne au contraire une si haute idée d’eux-mêmes, une idée si basse du reste des humains, qu’ils regardent les peuples comme destinés par la Nature à leur servir de jouets. C’est ainsi que l’on forma tant de monstres qui se firent un amusement de sacrifier des millions d’hommes à leurs passions indomptées et même à leurs fantaisies passagères. En mettant Rome en feu, Néron ne chercha qu’à satisfaire sa curiosité ; il voulut voir un incendie immense et repaître son orgueil de l’idée de son pouvoir sans bornes, qui lui permettait de tout oser contre un peuple asservi. L’orgueil fut toujours un des principaux mobiles de la cruauté et de l’oubli de ce qu’on doit aux hommes.\par
Loin de donner aux puissants de la terre un cœur sensible et tendre, tout concourt à leur inspirer des sentiments féroces. En excitant leur ardeur guerrière, on les familiarise avec le sang, on les habitue à contempler d’un œil sec une multitude égorgée, des villes réduites en cendres, des campagnes ravagées, des nations entières baignées de larmes, le tout pour contenter leur propre avidité ou pour amuser leurs passions. Les plaisirs mêmes dont on amuse leur oisiveté sont gothiques et sauvages, ils semblent n’avoir pour objet que de les rendre insensibles et barbares ; on leur fait de bonne heure une occupation importante de poursuivre des bêtes, de les tourmenter sans relâche, de les réduire aux abois\footnote{Rien de plus cruel que la chasse au cerf, plaisir qui est communément réservé pour les rois et les princes ; cet animal gémit et répand des larmes quand il se voir forcé. « Questuque cruentus, atque imploranti similis », dit Ovide, il semble implorer la pitié de l’homme son ennemi : cependant, c’est à des femmes que l’on réserve communément l’honneur de lui plonger le couteau ! Rien de plus propre à rendre les hommes cruels que de souffrir que les enfants s’amusent à tourmenter les bêtes. Locke parle d’une mère sensée qui permettait aux siens d’avoir des oiseaux, mais qui les récompensait ou punissait, suivant qu’ils en usaient bien ou mal avec eux. Voyez {\itshape Traité de l’Éducation}. Plutarque, chez les Anciens, et M. Rousseau dans son {\itshape Émile} ont très éloquemment plaidé la cause des bêtes, qu’ils ont vengées de la cruauté des hommes. Les {\itshape Papiers anglais} de 1770 rapportent qu’un chauffeur, voyant un pauvre homme qui portait dans sa main une tête de mouton pour son dîner et celui de sa femme et de ses enfants, s’écria : « Ce sont ces coquins-là qui font qu’il nous en coûte si cher pour nourrir nos chiens. »}, de les voir se débattre et lutter contre la mort.\par
Est-ce donc là le moyen de former des âmes pitoyables ? Le prince qui s’est accoutumé à voir les angoisses d’une bête palpitante sous le couteau, daignera-t-il prendre part aux souffrances d’un homme qu’on lui montre toujours comme un être d’une espèce inférieure à la sienne ? La guerre, ce crime affreux et si fréquent des rois, est évidemment très propre à perpétuer l’injustice et l’inhumanité sur la terre. La valeur guerrière est-elle donc autre chose qu’une cruauté véritable exercée de sang-froid ? Un homme nourri dans l’horreur des combats, accoutumé à ces assassinats collectifs que l’on nomme des batailles, qui par état doit mépriser la douleur et la mort, sera-t-il bien disposé à s’attendrir sur les maux de ses semblables ? Un être sensible et compatissant serait à coup sûr un très mauvais soldat.\par
Ainsi, la cruauté des rois contribue nécessairement à fomenter cette disposition fatale dans les cœurs d’un grand nombre de citoyens. Si les guerres sont devenues moins cruelles qu’autrefois, c’est que les peuples, à mesure qu’ils s’éloignent de l’état sauvage et barbare, font des retours plus fréquents sur eux-mêmes ; ils sentent les dangers qui résulteraient pour eux s’ils ne mettaient des bornes à leur inhumanité. En conséquence, on s’efforce de concilier autant qu’on peut la guerre avec la pitié. Espérons donc qu’à l’aide des progrès de la raison, les souverains, devenus plus humains et plus doux, renonceront aux plaisirs féroces de sacrifier tant d’hommes à leurs injustes fantaisies. Espérons que les lois, devenues plus humaines, diminueront le nombre des victimes de la justice et modéreront la rigueur des supplices dont l’effet est d’exciter la curiosité du peuple, d’alimenter sa cruauté, sans jamais diminuer le nombre des criminels. Pour être inhumain et cruel, il n’est pas nécessaire d’exterminer des hommes ou de leur faire éprouver des supplices rigoureux. Tout homme qui pour satisfaire sa passion, sa fureur, sa vengeance, son orgueil, sa vanité, fait le malheur durable des autres, possède une âme dure et doit être taxé de cruauté : un cœur sensible et tendre doit donc abhorrer tous ces tyrans domestiques qui s’abreuvent journellement des larmes de leurs femmes, de leurs enfants, de leurs proches, de leurs serviteurs et de tous ceux sur lesquels ils exercent leur autorité despotique. Combien de gens, par leur humeur indomptée, font éprouver de longs supplices à ceux qui les entourent ! Combien d’hommes, qui rougiraient de passer pour cruels, et qui font savourer journellement le poison du chagrin aux malheureux que le sort a mis en leur puissance ? L’avare n’est-il pas endurci contre la pitié ? Le débauché, le prodigue, le fastueux, ne refusent-ils pas souvent le nécessaire aux personnes qui devraient leur être les plus chères, tandis qu’ils sacrifient tout à leur vanité, à leur luxe, à leurs plaisirs criminels ? La négligence, l’incurie, le défaut de réflexion, deviennent très souvent des cruautés avérées. Celui qui, lorsqu’il le peut, néglige ou refuse de faire cesser le malheur de son semblable, est un barbare que la société devrait punir d’infamie et que les lois devraient rappeler aux devoirs de tout être sociable.
\subsection[{Chapitre II. De l’Orgueil, de la Vanité, du Luxe}]{Chapitre II. De l’Orgueil, de la Vanité, du Luxe}
\noindent L’orgueil est une idée haute de soi-même accompagnée de mépris pour les autres. L’orgueilleux est injuste en ce qu’il ne s’apprécie jamais lui-même avec équité. Il s’exagère son propre mérite et ne rend pas justice à celui des autres. L’orgueilleux annonce de l’imprudence et de la sottise. Il prétend s’attirer l’estime, la considération, les égards des autres, tandis qu’il les révolte par sa conduite et ne s’attire pour l’ordinaire que leur haine et leur mépris. L’orgueilleux est un être insociable ; il se fait le centre unique de la société, dont il veut exclusivement obtenir l’attention sans avoir aucun égard aux droits de ses associés. L’homme orgueilleux ne voit partout que lui seul. Il semble croire que ses semblables ne sont faits que pour l’adorer et lui rendre des hommages, sans être obligé de leur montrer du retour. L’orgueilleux est coléreux, inquiet, très prompt à s’alarmer, ce qui toujours dénote l’absence d’un mérite réel. La bonne conscience, c’est-à-dire l’estime méritée de soi-même et des autres, donne de la force, de la confiance, de la sécurité ; elle ne craint pas d’être privée de ses droits.\par
N’est-ce pas méconnaître ses intérêts que de montrer de l’orgueil ? Affligeant pour les autres, il les porte naturellement à examiner les titres de celui qui prétend s’élever au-dessus d’eux ; de cet examen il résulte rarement que l’orgueilleux soit digne de la haute opinion qu’il a ou qu’il veut donner de lui. Le mérite réel n’est jamais orgueilleux ; il est communément accompagné de modestie\phantomsection
\label{footnote18}\footnote{« Qui s’examine profondément, dit le philosophe déjà cité, se surprend trop souvent en erreur pour n’être pas modeste. Il ne s’enorgueillit point de ses lumières ; il ignore sa supériorité. L’esprit est comme la santé : quand on en a, l’on ne s’en aperçoit point. » Voyez le livre {\itshape De l’Esprit}, Discours II, chap. VII, p. 900, édition in-4°.}, vertu si nécessaire pour amener les hommes à reconnaître la supériorité que l’on a sur eux, dont ils ont toujours tant de peine à convenir.\par
Tout homme s’aime, sans doute, et se préfère aux autres ; mais tout homme désire de voir ces sentiments confirmés par les autres. Pour avoir le droit de d’estimer et de voir son amour-propre étayé des suffrages publics, il faut montrer des talents, des vertus, des dispositions vraiment utiles, des qualités que l’on puisse sincèrement considérer. L’amour légitime de soi, l’estime fondée sur la juste confiance que l’on mérite la tendresse et la bienveillance des autres, n’est point un vice, c’est un acte de justice qui doit être ratifié par la société, et auquel sans être injuste elle ne peut refuser de souscrire.\par
Défendre à l’homme de bien de s’aimer, de s’estimer, de se rendre justice, de sentir son mérite et son prix, c’est lui défendre de jouir des avantages et des douceurs d’une bonne conscience qui, comme on l’a fait voir, n’est que la connaissance des sentiments favorables qu’une conduite louable doit exciter. Le sentiment de sa propre dignité est fait pour soutenir l’homme de bien contre l’ingratitude, qui souvent lui refuse les récompenses auxquelles il a droit de prétendre. La confiance que donne le vrai mérite permet en effet au sage cette ambition légitime qui suppose la volonté et le pouvoir de faire du bien à ses semblables. Où en serait la société s’il n’était jamais permis aux âmes honnêtes d’aspirer aux honneurs, aux dignités, aux places dans lesquelles un grand cœur peut exercer sa bienfaisance ? Enfin, c’est le sentiment de l’honneur, c’est le respect pour lui-même, c’est une noble fierté qui empêche l’homme vertueux de s’avilir, de se prêter à des bassesses et aux moyens honteux par lesquels tant de gens s’efforcent de parvenir en sacrifiant leur honneur à la fortune. Les âmes basses et rampantes n’ont rien à perdre ; elles sont accoutumées aux mépris des autres et à s’estimer très faiblement elles-mêmes.\par
Ainsi, ne défendons pas à l’homme vertueux, bienfaisant, éclairé, de s’estimer lui-même, puisqu’il en a le droit. Mais défendons à tout homme qui veut plaire à la société de s’exagérer son propre mérite ou de l’étaler avec faste d’une façon humiliante pour les autres : il perdrait dès lors l’estime de ses concitoyens. Disons-lui que la présomption ou la confiance peu fondée sur des talents et des vertus qu’on n’a pas, est un orgueil très ridicule et ne peut être le partage que d’un sot dont la folie est de croire un mérite qu’il n’a point. Craignons de nous rendre méprisables par une fatuité qui fait que l’on ne se montre occupé que de soi-même et des qualités que l’on croit posséder. Si ces qualités sont réellement en nous, nous fatiguons les autres à force de les leur représenter. Sont-elles fausses ? Nous leur paraissons impertinents et ridicules dès qu’ils ont une fois démêlé l’imposture ou l’erreur. Évitons l’arrogance et la hauteur dont l’effet est de repousser et de blesser, rejetons comme une folie toute insolence qui consiste à faire sentir son orgueil à ceux mêmes à qui l’on doit de la soumission et du respect. La grossièreté, la brutalité, l’impolitesse sont des effets ordinaires d’un orgueil qui se met au-dessus des égards, qui refuse de se conformer aux usages et de montrer les déférences et les attentions que des êtres sociables se doivent les uns aux autres. Tout orgueilleux semble croire qu’il existe tout seul dans la société.\par
L’impudence peut être définie [comme] l’orgueil du vice ; l’effronterie est le courage de la honte : il n’y a que la corruption la plus complète qui puisse rendre fier de ce qui devrait faire rougir aux yeux de ses concitoyens. Tout esclave, tout homme bas ou corrompu qui se glorifie, doit être regardé comme un impudent, un effronté.\par
La vanité est un orgueil fondé sur des avantages qui ne sont d’aucune utilité pour les autres. {\itshape La vanité est}, dit-on, {\itshape la gloire des petites âmes}. Un grand homme ne peut être flatté de la possession des choses qu’il reconnaît inutiles à la société. L’orgueil de la naissance est une pure vanité puisqu’il se fonde sur une circonstance du hasard qui ne dépend aucunement de notre propre mérite, dont il ne résulte aucun bien pour le reste des hommes. L’ostentation, le faste, la parure, sont des marques de vanité ; elles annoncent qu’un homme s’estime et veut être estimé des autres par des endroits qui ne sont aucunement intéressants pour le public. Quel avantage résulte-t-il qu’un homme étale aux yeux des passants des équipages dorés, des livres magnifiques, des coursiers d’un grand prix ? Les repas somptueux du prodigue ne sont utiles qu’à quelques parasites qui paient en flatterie le sot qui les régale.\par
Le luxe est une émulation de vanité qu’on voit éclore parmi les citoyens des nations opulentes. Cette vanité, alimentée par l’exemple, devient pour les riches le plus pressant des besoins, auquel par conséquent tout est sacrifié. À la vue des forfaits et des crimes que cette vanité épidémique fait commettre chaque jour, il est impossible de souscrire au jugement que des écrivains, d’ailleurs bien intentionnés, ont porté au luxe. Il est vrai qu’il attire des richesses dans un État ; mais ces richesses tendent-elles à soulager la misère du plus grand nombre ? Non, sans doute ; l’argent attiré par le luxe se concentre bientôt dans un petit nombre de mains et n’en sort que pour alimenter le luxe des richesses, sans porter le moindre secours aux cultivateurs, aux citoyens laborieux, aux arts vraiment utiles que le luxe dédaigne. Les trésors de l’homme vain sont réservés pour entretenir son faste, sa mollesse, ses voluptés. Il les répand à pleines mains sur des flatteurs, des proxénètes, des courtisanes, des fripons de toute espèce. Le plaisir de la bienfaisance étant ignoré de lui, il n’a jamais de quoi encourager ni consoler les talents affligés ; les dépenses nécessaires à son luxe ne lui laissent jamais les moyens de faire du bien. La vanité endurcit l’âme et ferme le cœur à la bienveillance. Enfin, comme de petites causes multipliées produisent les plus grands effets, c’est la vanité puérile du luxe qui produisit toujours la ruine des plus grands États. Une vanité nationale est toujours l’effet d’un gouvernement injuste et vain : chacun, mécontent de sa place, veut se mettre au-dessus de son niveau.\par
Il est donc également de l’intérêt de la politique et de la saine morale de réprimer, de décrier le luxe et de guérir les hommes de la fatale vanité qui le fait naître. Pour cet effet, il est utile de se faire des idées précises de ce mal contagieux si funeste aux sociétés et aux individus. Il semble que l’on doit appeler {\itshape luxe} toute dépense qui n’a pour objet que la vanité, que le désir d’égaler ou de surpasser les autres, que le dessein de faire de ses richesses une parade inutile. De plus, on doit appeler {\itshape dépenses de luxe} toutes celles qui excèdent nos facultés ou qui devraient être employées à des usages plus nécessaires et plus conformes aux principes de la morale. Le souverain d’une nation opulente ne peut être accusé de luxe quand, sans opprimer ses sujets, il fait élever un palais somptueux dont la magnificence annonce aux citoyens le résidence d’un chef occupé de leur bien-être et qu’ils doivent respecter. Ce souverain peut sans blâme donner à sa demeure tous les ornements que son goût lui suggère, tant qu’ils ne sont point achetés aux dépens de la félicité publique. Mais un monarque qui, pour contenter son orgueil, écrase son peuple d’impôts, le plonge dans l’indigence et l’insulte ensuite par des monuments superbes, un tel monarque est un tyran coupable du luxe le plus criminel et dont les travaux coûteux doivent être détestés par toutes les âmes honnêtes.\par
Qu’un prince animé par sa reconnaissance bâtisse un asile ample et commode pour les guerriers qui l’ont servi, on ne pourra l’accuser de luxe ou de vanité ; mais si, ne consultant que son goût pour le faste, il fait de cette retraite de l’indigence un superbe palais onéreux pour son peuple, il n’est plus bienfaisant : il repaît son orgueil en étalant un luxe très inutile. Il aurait bien mieux employé son argent s’il avait épargné de vains ornements pour nourrir un plus grand nombre d’infortunés. Un grand, un particulier opulent, peuvent sans luxe se construire une habitation agréable ornée de meubles commodes mais ils sont des insensés s’ils se proposent de copier la magnificence d’un roi ; ils deviennent criminels s’ils bâtissent aux dépens de leurs concitoyens, ils se rendent coupables de la folie la plus condamnable s’ils contentent leur vanité en ruinant leur postérité.\par
Tout homme qui jouit de l’aisance peut s’habiller d’une façon qui le distingue de l’indigent ; il peut sans luxe se procurer des voitures et des serviteurs. Mais s’il lui faut chaque jour des vêtements riches et nouveaux, des équipages brillants, des bijoux précieux, s’il peuple sa maison de valets inutiles, il fait tort à tous ceux qu’il voulait soulager. Il enrichit des tailleurs, des bijoutiers, des selliers, mais il prive les campagnes de cultivateurs, il multiplie les fainéants et les vices ; il nuit à la société, et s’il dérange ses affaires, il se nuit à lui-même et vole ses créanciers. Enfin, il fait tort à l’homme moins aisé dont son exemple anime la vanité, mais pour qui les commodités et la parure du riche sont un luxe destructeur.\par
Des riches et des grands peuvent se procurer les plaisirs de la table, rassembler des amis, leur faire très bonne chère, mettre du choix dans les mets qu’ils leur présentent. Mais n’y a-t-il pas une vanité extravagante à ne pouvoir se contenter des denrées et des mets que fournit le climat qu’on habite ? Il y a de la folie à vouloir, aux dépens de sa fortune, jouter contre les banquets des souverains ; il y a de la dureté à sacrifier à sa vanité chimérique ce qui ferait subsister bien des familles honnêtes qui souvent n’ont pas du pain.\par
Le nécessaire du riche devient luxe pour le pauvre. L’homme opulent contracte mille besoins que l’indigent devrait toujours ignorer. L’usage du tabac est un luxe ruineux pour le manœuvre qui gagne à peine de quoi vivre. Le riche sans se déranger peut aller au spectacle, l’artisan est perdu dès qu’il en a pris le goût.\par
Le luxe, enfin, pousse tous les hommes hors de leur sphère ; il les enivre de mille besoins imaginaires auxquels ils ont souvent la folie de sacrifier les besoins les plus réels, les devoirs les plus sacrés. Dans un pays de luxe, l’agréable l’emporte toujours sur l’utile ; la vanité de paraître fait que personne ne se sent à son aise. Depuis le souverain jusqu’aux moindres des sujets, chacun exerce ses forces et personne n’est content de son sort. Chacun est tourmenté d’une vanité inquiète et jalouse qui le fait rougir de se voir surpassé par les autres. Il se croit méprisable dès qu’il ne peut les égaler. Cette vanité dégénère en une telle manie, que le suicide n’est point rare dans les villes dont le luxe s’est emparé : la honte d’être déchu réduit l’homme au désespoir.\par
L’ambition qu’on nomme, par les ravages qu’elle produit sur la terre, la passion des grandes âmes, n’est communément l’effet que d’une vanité remuante ou mécontente de son sort. Cette faim excessive de la domination et de la gloire est une folie qui, au lieu de conduire à la vraie gloire, devrait conduire à l’exécution publique. Un conquérant est communément un génie rétréci qui, très peu capable de bien gouverner les anciens sujets que le destin lui avait soumis, a la présomption de croire qu’il gouvernera bien mieux les nouveaux qu’il va subjuguer. Si par la sagesse de sa conduite et de ses lois, Alexandre eût fait le bonheur des États qu’il avait hérité de ses pères, on lui pardonnerait peut-être ses conquêtes en Asie ; mais ce héros, gonflé de ses victoires, a la sotte vanité de se faire passer pour fils de Jupiter ; il meurt sans avoir donné à l’univers la moindre marque de sagesse, de lumières, de vertu, sans lesquelles pourtant il n’existe ni honneur ni gloire.\par
Ce que vulgairement on nomme {\itshape honneur} dans la plupart des nations, n’est, comme on l’a fait remarquer, qu’une vanité chatouilleuse qui, toujours inquiétée par la conscience de son peu de mérite et craignant d’être abaissée dans l’opinion des autres, est capable de porter les hommes aux plus affreux excès. En vertu des préjugés sur lesquels cet honneur se fonde, l’homme coupable d’un assassinat, d’un crime, lève sa tête altière au milieu de la société ; sa vanité féroce lui persuade qu’il a droit à l’estime publique pour avoir eu le courage de tuer de sang-froid un citoyen et de braver les lois.\par
Enfin, de tous les vices des hommes il n’en est peut-être pas qui fasse commettre un plus grand nombre de crimes que la vanité, sans compter les folies et les travers dans lesquels elle les précipite à chaque pas. Cette vanité persuade aux puissants de la terre que c’est par un faste ruineux pour les peuples qu’il faut s’attirer les regards des imbéciles mortels ; d’après ces vaines idées, les nations sont forcées d’arroser la terre de sang et de sueur pour mettre leurs vains tyrans en état de paraître avec éclat, d’élever des édifices pompeux, de soutenir la splendeur de leur trône. Princes ! laissez-là votre faste ; gouvernez vos sujets avec justice, occupez-vous du soin de leur procurer le bonheur, et vous n’aurez pas besoin de les éblouir par un vain appareil qui décèle toujours une âme rétrécie qui s’efforce de se cacher sous le masque d’une grandeur empruntée.\par
Les grands, les nobles, les citoyens les plus distingués des nations, par un effet de leurs préjugés sacrifient continuellement leur bonheur permanent et durable aux besoins imaginaires que leur crée la vanité. On les voit échanger leur temps, leur liberté, leur honneur, leur fortune et leur vie contre des titres, des sons, des ornements, des rubans, marques futiles dont, au défaut de mérite et de vertus, tant de gens ont besoin pour s’illustrer aux yeux de leurs concitoyens ! Des privilèges injustes, des préséances vaines, des prérogatives idéales sont communément les causes des querelles, des divisions, des cabales qui déchirent les cours, qui mettent les nations en guerre, qui finissent quelquefois par embraser l’univers !\par
La morale ne peut donc, au risque même de ne parler qu’à des sourds, assez répéter aux hommes de cultiver leur raison, de peser les conséquences de leurs folles vanités, de sentir que c’est dans la vertu seule que consiste la gloire, l’honneur, la noblesse, la grandeur véritable. Que les hommes les plus grands sont petits aux yeux de ceux qui réfléchissent et qui voient la faiblesse des ressorts dont souvent la machine du monde est ébranlée ! Des disputes minutieuses, des opinions frivoles, des hypothèses puériles soutenues obstinément par des hommes bouffis de la plus sotte vanité, suffisent pour allumer des haines immortelles et pour troubler le repos des nations !\par
L’opiniâtreté, que l’on confond si souvent avec la fermeté, avec l’amour de la vérité, avec le zèle pour la justice, n’est le plus communément que l’effet d’une vanité méprisable qui se fait un point d’honneur de ne jamais se rendre. L’homme opiniâtre a la folie de croire que sa raison supérieure ne peut nullement l’égarer ; son amour-propre lui permet rarement d’être juste : il persiste dans l’injustice, il s’imagine qu’il y va de sa gloire de ne jamais se rétracter. Est-il un égarement plus commun et plus funeste ? Tout ne concourt-il pas à prouver que rien n’est plus honorable et plus noble qu’un aveu franc de son erreur, qu’un hommage sincère rendu à la vérité ? Nous trouvons toujours de la grandeur d’âme et de la force dans celui qui sait dompter sa vanité, et nous méprisons les obstinés dont l’orgueil inflexible ne veut jamais plier. De combien de flots de sang la terre fut mille fois inondée par l’opiniâtreté de quelques spéculateurs qui voulurent faire adopter aux nations leurs opinions comme des oracles infaillibles ! Quels ravages n’a pas causé la maxime hautaine et pernicieuse de tant de souverains à qui l’on persuada que {\itshape l’autorité ne doit jamais reculer} ! Un prince n’est jamais plus grand et plus cher à son peuple que lorsque, reconnaissant qu’il s’est trompé, il remédie aux maux que ses erreurs ont pu causer.\par
L’on aime les personnes timides et qui ne résistent point, parce qu’on se promet d’en disposer à son gré. Cependant, la timidité que d’ordinaire on aime et que l’on prend souvent pour de la modestie, n’est quelquefois l’effet que d’une vanité secrète qui craint de n’être point autant considérée qu’elle croit le mériter : cet amour-propre délicat ne veut pas s’exposer à des assauts qu’il se sent incapable de soutenir.\par
En un mot, il n’est point de forme que l’amour-propre n’emprunte pour se masquer. Cette passion, hypocrite quand elle n’a pas le courage de se montrer à découvert, prend des détours que les observateurs les plus attentifs peuvent à peine démêler. Mais on ne se trompera guère quand on dira qu’une vanité couverte ou visible est le mobile universel de la conduite du plus grand nombre des hommes ; souvent sa marche est si secrète qu’elle se dérobe à nous-mêmes. Elle nous donne le change à tout moment. Elle nous trompe et, quelquefois à notre insu, elle nous conduit peu à peu à des actions très brusques et très criminelles suivies de longs regrets.\par
Des intérêts mal entendus, un amour-propre inconsidéré, une vanité puérile, voilà les vrais fléaux et des nations et des sociétés particulières ; elles deviennent des arènes où chacun vient, pour ainsi dire, faire assaut de vanité. Chacun y veut {\itshape primer, dominer} les autres, jouer un rôle distingué. Parmi des êtres qui se sentent sociables, il faut une circonspection incommode, une crainte continuelle de blesser les prétentions impertinentes de tous ceux que l’on rencontre. Les amis les plus intimes et les plus familiers sont prêts à se brouiller, à se séparer pour toujours, à s’égorger pour une parole indiscrète que ne peut endurer une vanité soupçonneuse. Rien de plus difficile et de plus périlleux que de vivre avec des hommes qui ne placent leur honneur et leur gloire que dans des puérilités. Elles rendent souvent les citoyens d’une nation civilisée aussi coléreux, aussi vindicatifs, aussi cruels que les sauvages les plus inconsidérés. En voyant les objets dans lesquels la plupart des hommes font consister leur vanité ou leurs prétentions, on serait tenté de les regarder comme des enfants incapables de jamais parvenir à maturité\footnote{Le chevalier Digby remarque « que les hommes ont un tel désir de paraître supérieurs aux autres, qu’ils vont jusqu’à se vanter d’avoir vu ce qu’ils n’ont jamais vu ». De là les menteries des voyageurs, les exagérations des conteurs, etc., etc.}. On ne voit dans le monde que des gens dont l’amour-propre est continuellement blessé de celui des autres, on n’y rencontre que des insensés qui ont la folie d’exiger ce qu’ils ne rendent à personne.\par
C’est en effet à l’orgueil, à la présomption, à une folle vanité que l’on doit attribuer le défaut de ces tyrans de la société que l’on nomme {\itshape exigeants}. Une arrogance très injuste leur persuade qu’on leur manque sans cesse, que l’on n’a pas pour eux les attentions qu’ils méritent, tandis qu’ils manquent souvent eux-mêmes à leurs amis, à tout le monde. Rien de plus incommode dans le commerce de la vie que des hommes de ce caractère. Rien de plus injuste que des orgueilleux qui veulent être aimés sans montrer aucune affection pour les autres. Rien de plus commun que des êtres qui veulent être considérés de ceux même qu’ils méprisent et à qui souvent ils témoignent sans détour le peu de cas qu’ils en font. Rien de plus insociable qu’un amour-propre qui rapporte tout à lui-même, sans jamais avoir égard à l’amour-propre des autres. Ce sont communément les hommes les plus exigeants qui ont les droits les moins fondés sur l’estime de ceux dont ils exigent le dévouement le plus complet.\par
En considérant la conduite de la plupart des hommes que l’on voit sans cesse occupés de leurs vanités puériles, on serait tenté de croire qu’ils ne sont que des enfants que la raison ne pourra jamais guérir de leurs folies. Une sotte vanité, un orgueil méprisable percent dans toutes les actions et semblent être les leviers qui font mouvoir le monde.\par
D’un autre côté, celui qui se mépriserait totalement lui-même serait peu curieux de mériter l’estime de se semblables, dont tout homme doit être jaloux. Toux ceux qui ont la conscience d’être peu dignes de considération s’abandonnent, pour ainsi dire, eux-mêmes et finissent par des bassesses dont leur amour-propre flétri ne sait plus rougir. S’il leur reste encore quelque énergie, ils deviennent imprudents et bravent insolemment le {\itshape qu’en dira-t-on}. Rien de plus dangereux que les hommes avilis qui ont totalement renoncé à l’estime publique\footnote{« Dire moins de soi qu’il y en a, c’est sottise, non modestie : se payer de moins qu’on ne vaut, c’est lâcheté, et pusillanimité selon Aristote. » Voyez {\itshape Essais} de Montaigne, livre II, chap. 6.}.\par
En se rendant justice, en rentrant quelquefois dans le fond de son propre cœur, on pourra modérer peu à peu les saillies d’une vanité qui semble tenir si fortement à la nature humaine. L’équité nous apprend à ne point nous surfaire des qualités que nous pouvons posséder. Si chaque homme, de bonne foi avec lui-même, se demandait en quoi consiste donc cette prééminence qu’il s’arroge sur les autres, s’il examinait de sang-froid les titres d’après lesquels il exige les égards des autres et qu’à leur défaut il s’adjuge de sa propre autorité, il y a tout lieu de croire que cet examen habituel le rendrait plus réservé, et dès lors plus agréable à la société, qui lui saurait gré des sacrifices qu’il consentirait à lui faire. Rendons-nous vraiment estimables et nous n’aurons pas besoin de manège pour nous faire estimer. Combien les hommes s’épargnerait de soucis et de peines s’ils consentaient à être ce qu’ils sont !\par
Faute de faire des réflexions si simples, une vanité désagréable empoisonne toutes les actions, elle peuple la société d’une foule de gens assez insensés pour préférer le sot plaisir de paraître heureux à celui de l’être réellement, elle remplit les compagnies de {\itshape petits-maîtres}, de fats, d’impertinents, {\itshape d’avantageux, d’importants}, d’étourdis qui font des dépenses et des efforts incroyables pour se rendre ridicules et même insupportables. Une portion du genre humain est continuellement occupée à se moquer de l’autre pour se venger des blessures que se font leurs vanités réciproques. Chacun s’efforce de briller au dehors, de s’attirer tous les regards, d’en imposer par les qualités fictives qu’il croit propres à lui faire obtenir la préférence qu’il ambitionne. Mais personne ne descend en lui-même\footnote{« At nemo in sese tentat descendre, nemo. »Pers. {\itshape Satire 4}, vers 23.}, personne ne s’embarrasse d’acquérir des qualités auxquelles le public ne pourrait refuser son hommage. Enfin, personne ne songe à montrer dans sa conduite cette modestie qui lui plaît toutes les fois qu’il la rencontre dans les autres. Pour tâcher d’obtenir une place distinguée dans l’opinion publique, la plupart des hommes se donnent des tourments continuels qui se terminent d’ordinaire par les rendre incommodes et méprisables aux yeux de ceux dont ils prétendent se faire considérer. Le chemin le plus sûr à l’estime, c’est de la mériter par des vertus réelles. Tout homme qui se surfait, finit communément par être mis au-dessous même de sa juste valeur.
\subsection[{Chapitre III. De la Colère, de la Vengeance, de l’Humeur, de la Misanthropie}]{Chapitre III. De la Colère, de la Vengeance, de l’Humeur, de la Misanthropie}
\noindent La colère est une haine subite, plus ou moins permanente, contre les objets que nous jugeons contraires à notre bien-être. Rien de plus naturel que cette passion dans un être perpétuellement occupé de sa propre conservation et sa félicité. Mais rien de plus nécessaire à un être raisonnable et sociable que de réprimer des mouvements impétueux aussi dangereux pour lui que pour ceux avec lesquels son destin est de vivre. En général, la raison prouve que, pour son propre intérêt, tout homme vivant en société doit être en garde contre toutes les impulsions qui le troublent et l’empêchent de faire usage de son jugement, de sa réflexion, de l’expérience destinée à lui servir de guide. « Le sage, dit Épicure, peut être outragé par la haine, par l’envie et par le mépris des hommes ; mais il croit qu’il dépend de lui de se mettre au-dessus de tout préjudice par la force de la raison. La sagesse est un bien si solide, qu’elle ôte à celui qui l’a en partage toute disposition à sortir de son état naturel, et l’empêche de changer de caractère par la colère, quand même il en aurait la volonté\phantomsection
\label{footnote19}\footnote{« Detrimenta quæ ex hominibus, sive odii, sive individiæ, sive contemptûs causa fiunt, sapientem autumat ratione superare. Eum vera qui semel fuerit sapiens, in contrarium habitum transire non posse nec sponte variare. » Voyez Diogène Laërce, {\itshape De Vitis et Dogmatibus Philosophorum}, X, Seg. 117.}. »\par
De même que toutes les passions, la colère peut être retenue, balancée, comprimée par la crainte des suites fâcheuses qu’elle peut avoir et pour nous-mêmes et pour les autres. Tout homme sociable doit être raisonnable, c’est-à-dire doit distinguer les mouvements naturels qu’il peut suivre sans danger de ceux auxquels il doit prudemment résister. Il doit être modifié de manière à régler ces mouvements de la façon qui convient à la vie sociale, il doit avoir de bonne heure contracté l’habitude de se vaincre et l’exercice doit lui procurer la force nécessaire pour y parvenir. On ne peut trop le répéter, tout homme qui n’a point appris à résister aux penchants de sa nature ne peut être qu’un membre nuisible dans la société. Les princes, les grands, les riches, ainsi que les gens du peuple, sont les plus sujets à la colère parce que leurs passions dans l’enfance ont été ou flattées ou négligées. Il serait inutile de parler ici des effets redoutables de la colère de rois ; tout l’univers a retenti dans tous les temps des affreux rugissements de ces lions déchaînés ou des cris des nations désolées par leurs fureurs.\par
Quoiqu’au premier coup d’œil les emportements de la colère semblent annoncer un grand ressort, une force, une énergie dans l’âme, les moralistes pour la plupart ont attribué cette passion à la faiblesse : elle suppose en effet une mobilité dans les organes qui les rend susceptibles d’être aisément affectés. Cette décomposition si facile de la machine, ou cette {\itshape irritabilité}, se remarque surtout dans les femmes, que la Nature a rendues communément plus sensibles, plus faibles, et dès lors plus sujettes à la colère que les hommes. Pareillement, les enfants, dès l’âge le plus tendre, donnent par leurs cris, leurs larmes, leurs trépignements et leurs convulsions, des signes peu équivoques de la colère dont ils sont agités toutes les fois qu’on ne se rend pas à leurs caprices. Si les forces répondaient à ses fureurs, un enfant serait capable d’exterminer sa nourrice ou sa mère sur le refus d’un {\itshape bonbon}. Peu à peu ses organes se fortifient, il devient plus tranquille, on le châtie de ses emportements qui mettraient quelquefois sa santé ou sa vie en danger. La crainte lui apprend à se contenir ; de cette manière il acquiert de la raison par degrés et se trouve insensiblement modifié de façon à pouvoir vivre en société.\par
Tout homme vivant avec des hommes doit savoir qu’il est entouré d’êtres qui, comme lui, sont remplis de défauts, de passions, de vanités, de faiblesses : il doit donc en conclure que son propre intérêt lui fait un devoir de les supporter, et qu’une colère continuelle le mettrait dans un état de guerre continuelle avec tous ceux qu’il fréquente. Celui qui est sujet à la colère est habituellement malheureux ; tout le blesse, la haine est perpétuellement dans son cœur, et il excite ce sentiment fâcheux dans tous les êtres que ses emportements effraient et rendent très misérables. L’homme coléreux ne peut jamais jouir d’un bonheur durable, vu que la moindre chose est capable de le troubler. Mécontent de tout le monde, il ne rend personne heureux ; il est comme un tyran au milieu des esclaves dont il soupçonne l’aversion. Il est forcé de lire la terreur qu’il inspire sur le visage de sa femme, de ses enfants, de ses valets, qui ne respirent qu’en son absence.\par
La douceur est un moyen assez sûr de désarmer la colère ; néanmoins il est des hommes tellement dominés par cette passion que la douceur même les irrite encore plus et les jette dans une sorte de désespoir et de rage ; alors la honte d’avoir tort ou la vanité se joignant à la colère, semble lui rendre de nouvelles forces et la porte jusqu’au délire. Ce phénomène, en morale, nous prouve évidemment que l’homme doux jouit d’une supériorité que, même dans la folie, l’homme coléreux est contraint de sentir. En effet, la colère est dans quelques personnes une frénésie, une courte rage, une véritable folie. Sans cela comment expliquer la conduite de quelques emportés, de ceux qui dans les accès de leur aveugle furie s’en prennent aux objets inanimés, frappent avec violence une table, une muraille, se blessent souvent grièvement et vont jusqu’à braver la mort ?\par
On voit donc que l’homme livré à la colère, redoutable à tout le monde, doit se craindre lui-même et ne peut jamais prévoir jusqu’où ses emportements le pousseront. Mais, étant tout seul, s’il est capable de se nuire, que sera-ce lorsqu’il se trouvera dans la compagnie des autres ? Il n’est jamais assuré de revoir sa maison ; incapable de rien endurer, il peut à chaque instant rencontrer des hommes aussi dangereux que lui qui le puniront de son humeur insociable. « La colère, dit un sage d’Orient, commence par la folie et finit par le regret. »\par
Aristote a prétendu que la colère pouvait quelquefois servir d’âme à la vertu, mais nous dirons avec Sénèque et Montaigne qu’en tout cas « c’est une arme de nouvel usage car, dit-il, nous remuons les autres armes, celle-ci nous remue ; notre main ne la guide pas, c’est elle qui guide notre main, nous ne la tenons pas\footnote{Voyez {\itshape Essais}, livre 2, chap. 31, vers la fin.} ».\par
Quoique la colère soit une passion dangereuse, il en est cependant une que nous devons approuver. C’est cette colère sociale que doivent nécessairement exciter dans toutes les âmes honnêtes le crime, l’injustice, la tyrannie, sur lesquels il n’est point permis d’être indifférent et qui doivent irriter tout bon citoyen ou faire naître dans son cœur une indignation durable. Cette colère légitime, appelée par Cicéron une {\itshape haine civile}, est un sentiment fait pour animer tous ceux qui s’intéressent fortement au bien-être du genre humain. Tout homme qui n’est pas troublé à la vue des injustices et des oppressions que l’on fait éprouver à ses semblables, est un lâche, un mauvais citoyen. C’est, disent les Arabes, {\itshape dans la colère qu’on reconnaît le sage}\footnote{Voyez {\itshape Sentent. Arab. in Erpenii grammat.}}.\par
La colère cachée, nourrie au fond du cœur et longtemps retenue, n’est pas moins cruelle dans ses effets ; c’est elle qui produit la {\itshape vengeance}. Cette passion redoutable, couvée par la pensée, attisée par l’imagination, fortifiée par réflexion, devient encore plus dangereuse que la colère la plus vive, qui bientôt s’exhale. La violence ouverte mérite plus d’indulgence : elle est bien moins à craindre que la fureur cachée de ces hommes assez maîtres d’eux-mêmes pour dissimuler leurs sentiments jusqu’au moment qui leur procure l’occasion de se venger à leur aise. On peut souvent compter sur la bonté du cœur et sur la générosité de celui qui est prompt à s’irriter ; plus ses emportements sont vifs, moins ils ont de durée ; au lieu que l’on ne peut jamais compter sur la réconciliation sincère d’un homme assez dissimulé pour cacher et comprimer longtemps dans son cœur la colère excitée par un outrage. Le sentiment de la colère est d’autant plus incommode qu’on a plus de peine à l’empêcher d’éclater ; ainsi, le vindicatif est le bourreau de lui-même, en même temps qu’il épie les occasions de faire éprouver sa cruauté aux autres.\par
La vengeance a toujours l’orgueil ou la vanité pour mobile. Se venger, c’est punir celui qui a excité notre colère, c’est trouver du plaisir à lui faire sentir que l’on a le pouvoir de le rendre malheureux. La vengeance est communément cruelle parce que l’imagination et la pensée exagèrent l’outrage qu’on a reçu. Le vindicatif croit que sa vengeance est incomplète si celui dont il se venge ignore de quelle main partent les coups qu’il reçoit. Voilà sans doute pourquoi Caligula prenait un grand plaisir à faire venir en sa présence les victimes qu’il destinait à périr dans les tourments. Voilà pourquoi il disait à ses satellites de les {\itshape frapper de manière à leur faire sentir les horreurs de la mort}\footnote{L’Italie nous fournit l’exemple d’une vengeance bien atroce, et si étrange qu’on a cru pouvoir la rapporter. Une femme de mauvaise vie, irritée de l’infidélité de son amant, dissimule le désir de se venger pendant deux ans que dura la nouvelle passion de son perfide. Au bout de ce temps, celui-ci revient à sa première maîtresse, qui le reçoit avec ardeur, ne lui fait aucun reproche mais lui plonge un poignard dans le cœur immédiatement après lui avoir permis un péché pour lequel elle présumait qu’il devait être éternellement damné.}.\par
Comme les hommes sont toujours des juges suspects et récusables dans leur propre cause, les lois, dans tous les pays policés, se sont réservés le droit de venger les citoyens ; elles ont ôté à ceux-ci le droit de punir les outrages qu’on leur a faits : ces lois sont en cela très conformes à l’intérêt de la société et des individus. Elles sont justes en ce qu’elles empêchent les hommes d’être injustes et cruels. Elles sont sociables puisque par là elles indiquent que des êtres perpétuellement exposés à s’irriter réciproquement doivent réfléchir aux conséquences de leurs actions et mettre en oubli des offenses qui ne sont le plus souvent que des minuties et des effets de la faiblesse humaine. La nature, la justice, l’humanité, la grandeur d’âme, la philosophie s’accordent à proscrire la vengeance et à nous faire un devoir du pardon des injures\footnote{La philosophie avait enseigné de bonne heure aux hommes la doctrine du pardon des injures. Plutarque nous apprend que les pythagoriciens se faisaient toujours un devoir de se donner la main en signe de réconciliation avant le coucher du soleil, lorsqu’ils s’étaient réciproquement offensés. — « Celui-là, dit Ménandre, est le plus vertueux entre les mortels, qui sait le mieux supporter les injures avec patience. » — Juvénal a dit depuis que la vengeance n’est un plaisir que pour les âmes rétrécies. « Quippe, minuti semper et infirmi est animi, exiguique voluptas ultio. » Voyez Juvénal, {\itshape Satires}, XIII, vers 189.}.\par
On a dit que la vengeance était {\itshape le mets des dieux}, c’est-à-dire un plaisir si grand qu’ils l’enviaient aux mortels. Mais quels dieux que ces êtres vindicatifs de la mythologie qui, sensibles aux mépris des hommes, ne diffèrent de les punir que pour en tirer une vengeance plus éclatante et plus capable d’effrayer ! Ces dieux coléreux, cachés dans leurs vengeances, implacables, insociables, ne sont pas faits pour servir de modèles à des êtres qui vivent en société : tout prouve que la vanité est une vraie petitesse, que l’indulgence et l’humanité sont des vertus aimables et nécessaires, que la vraie force suppose de la patience. N’est-ce pas se rendre soi-même très malheureux que de porter sans cesse la haine et la rage au fond du cœur ? La vengeance n’est propre qu’à éterniser les inimitiés dans le monde ; le plaisir futile qu’elle donne est toujours suivi de repentirs durables. Elle nous montre à la société comme des membres dangereux. « Celui, dit Philémon, qui pardonne une injure, force son ennemi à s’injurier lui-même. » Tout doit nous convaincre que l’homme qui sait pardonner, paraît aux yeux de tous les êtres sociables et raisonnables beaucoup plus estimable, plus fort et plus grand que l’insensé qui l’a blessé, ou que le lâche qui ne peut rien supporter. « Un lâche, dit un moderne, peut combattre ; un lâche peut vaincre ; mais un lâche ne peut jamais pardonner\phantomsection
\label{footnote20}\footnote{Voyez Addison dans {\itshape Le Mentor moderne}, n° 20.}. »\par
La générosité qui fait pardonner les injures est un sentiment inconnu des petites âmes, des gens du peuple, des hommes du commun. Les sauvages, suivant les relations des voyageurs, sont implacables dans leurs vengeances, qui chez eux se perpétuent de races en races et finissent par amener la destruction totale de leurs diverses hordes. L’esprit vindicatif qui subsiste encore dans un grand nombre de peuples que l’on croie policés, et l’idée qui fait croire qu’un homme de cœur ne doit jamais endurer un affront, sont visiblement des restes de la barbarie répandue en Europe par des nations féroces et guerrières qui jadis ont subjugué le vaste empire des Romains. Mais des hommes de cette trempe, des soldats farouches et déraisonnables ne sont pas des modèles à suivre par des hommes devenus sages, c’est-à-dire plus instruits des intérêts de la société et de ce qui constitue la grandeur d’âme, la gloire véritable. L’homme inculte et sauvage ne réfléchit point, il suit en aveugle les impulsions momentanées de sa fureur ; l’homme policé est vraiment sociable et s’accoutume à contenir ses passions parce qu’il en connaît les suites dangereuses. Ce n’est que par l’expérience que l’homme déraisonnable diffère de l’enfant, du sauvage, de l’insensé\phantomsection
\label{footnote21}\footnote{Dans tous les pays où la justice ne se rend point fidèlement, on voit communément régner les vengeances les plus cruelles. Lorsque la loi ne venge pas l’homme, il se venge lui-même, souvent outre mesure. Voilà la cause à laquelle on peut attribuer les fréquents assassinats qui se commettent dans les pays despotiques où la justice est toujours très mal administrée. Rien n’est plus capable de pousser les hommes au désespoir que le déni de justice.}.\par
Il est encore une disposition qui, sans avoir les effets impétueux de la colère ou les cruautés lentes et réfléchies de la vengeance, ne laisse pas de rendre bien des gens incommodes à la société : je veux parler de l’{\itshape humeur}. C’est une disposition habituelle à s’irriter ; elle dérive communément d’un tempérament vicié. Elle influe d’une façon très fâcheuse sur le caractère, à moins que le vice de l’organisation n’ait été soigneusement prévenu ou rectifié par l’éducation, par l’habitude, par l’usage du monde, par la réflexion. Il est des personnes tellement dominées par l’humeur ou dont la bile est si facile à émouvoir, que les moindres choses les irritent. Elles ne semblent jamais jouir d’aucune sérénité ; on dirait qu’elles se nourrissent d’amertume et de fiel, et que, ne trouvant de plaisir qu’à se tourmenter elles-mêmes, elles ne peuvent souffrir la paix et le contentement des autres. Tout homme sujet à cette colère habituelle est aussi malheureux qu’insociable. Il est bien difficile que celui qui est mécontent de tout le monde soit capable de se concilier l’amitié de personne.\par
Faute de vouloir faire des réflexions si naturelles, bien des atrabilaires se rendent les fléaux de leurs familles et de la société. Combien d’époux, sans motifs valables, vivent en vrais ennemis et semblent ne pouvoir s’envisager de sang-froid ou se parler sans colère ? Combien de pères chagrins qui ne peuvent sans s’irriter considérer les jeux les plus innocents de leurs enfants ? Combien de maîtres qui croiraient se dégrader s’ils ne parlaient avec aigreur à leurs domestiques tremblants ? Il est des hommes qui ne paraissent avoir des amis que pour leur faire à tout moment essuyer les effets de leur mauvaise humeur. Enfin, il est des gens tellement remplis de bile qu’ils ne se montrent dans le monde que pour avoir occasion de la répandre. Tout révolte ces misanthropes aux yeux desquels la Nature entière paraît défigurée.\par
Les personnes qu’une humeur noire domine ignorent-elles donc que dans toutes les positions de la vie l’homme doit aimer pour être aimé ? Est-il un état plus cruel que celui d’une femme condamnée pour la vie à souffrir les caprices d’un mari dont les caresses ne peuvent adoucir l’humeur invétérée ? Des enfants repoussés par le front austère d’un père pourront-ils avoir une tendresse véritable pour ce tyran qui ne leur sourit jamais ? Un maître grondeur et que tout mécontente sera-t-il servi avec zèle par des serviteurs perpétuellement intimidés ? Quels amis peut mériter un homme insociable et brutal dont le commerce les afflige et les humilie ? N’y a-t-il pas une présomption bien ridicule à croire que tout le monde, et ceux mêmes qui ne dépendent aucunement de lui, sont faits pour supporter l’humeur d’un homme qui ne veut rien supporter ?\par
Communément, un sot orgueil joint à la bile constitue le caractère de ces hommes farouches et chagrins qui trop souvent empoisonnent le commerce de la vie. Qu’ils ne nous disent pas qu’{\itshape on ne peut se résoudre}, que leur humeur est l’effet de leur tempérament. C’est en travaillant sur nous-mêmes, en nous observant avec soin, en combattant les défauts de notre organisation que nous pouvons devenir des êtres vraiment sociables : la conscience de nos propres défauts devrait sans cesse nous ramener à l’indulgence pour ceux des autres.\par
D’ailleurs, souvent la mauvaise humeur nous les exagère et quelquefois même leurs torts n’existent que dans leur imagination malade. Que dans les accès de son mal l’homme bilieux se sépare, s’il le faut, pour quelque temps, de la société qui le fatigue et qu’il afflige, que dans des intervalles plus calmes il se demande raison de sa mauvaise humeur, le plus souvent il trouvera que son chagrin n’a point de motifs et qu’il a tort de s’irriter contre les autres ou de se tourmenter lui-même.\par
L’indulgence, la patience, la douceur, le désir de plaire sont les seuls liens qui puissent unir entre eux deux êtres imparfaits. La colère et la mauvaise humeur, loin de remédier à quelque chose, ne peuvent que troubler et dissoudre la société.\par
La misanthropie ou l’aversion pour les hommes est une humeur habituelle et continue qui nous fait haïr les êtres avec lesquels nous devons vivre en société. Cette disposition vraiment inhumaine et sauvage paraît venir de plusieurs causes que tout homme raisonnable devrait soigneusement combattre : elle est due à un orgueil très irascible qui, nous fermant les yeux sur nos propres défauts, nous exagère ceux des autres et nous les fait juger avec trop de rigueur. Le misanthrope ne connaît ni l’indulgence ni la pitié. L’envie et la jalousie, passions toujours mécontentes, ont communément beaucoup de part à l’humeur que l’on éprouve contre le genre humain. La bile est surtout remuée à la vue de la prospérité de ceux que l’on en suppose moins dignes que soi. L’envie fait la philosophie de bien des courtisans ; leurs mauvais succès les rendent souvent caustiques et misanthropes.\par
Cependant, il peut se faire que l’éloignement pour les hommes parte quelquefois d’une source moins impure. Un homme honnête et sensible peut à la fin s’indigner d’avoir été longtemps le spectateur ou le jouet, soit de la méchanceté, soit de la folie de ses semblables, et concevoir dès lors beaucoup d’aversion ou de mépris pour eux. Quoique cette misanthropie fondée sur une expérience fâcheuse paraisse moins blâmable que celle qui naît de l’envie, elle décèle néanmoins un défaut de justice, en ce qu’elle enveloppe tous les hommes dans la même condamnation.\par
La vraie sagesse, toujours exempte de préjugés, ne peut approuver la haine des hommes dans un être fait pour vivre avec eux. Elle approuve la prudence, qui nous fait éviter la société des insensés et des méchants, mais elle blâme une humeur sombre qui ne s’accommode avec personne ; elle condamne une haine opiniâtre qui dispose très peu à se rendre utile aux autres ou qui bannit la bienveillance universelle. Le misanthrope est très souvent un méchant qui, ne sachant se faire aimer de personne, prend le parti de haïr tout le monde.\par
La morale doit travailler à rendre l’homme sociable, elle doit lui montrer ses intérêts toujours liés à ceux de ses pareils. La raison guidée par l’expérience lui fera voir que son destin est de vivre dans une foule où il sera nécessairement poussé, tantôt par des méchants et tantôt par des étourdis, bien plus communs encore. Il s’armera donc de patience, de courage et d’indulgence, afin de fournir tranquillement sa carrière ; il tâchera de contenir son indignation et sa colère qui, l’agitant lui-même d’une façon très incommode, le rendraient sans cesse mécontent de son sort et le mettraient dans un état de guerre continuelle avec eux qui l’entourent.\par
L’humeur, l’insociabilité, la misanthropie sont des vices réels. Les moralistes qui en font des perfections, des vertus, qui persuadent à l’homme qu’il y a du mérite à se séparer de ses semblables, à s’isoler, à vivre inutiles à la société, ont visiblement ignoré que la vertu doit toujours être utile et bienfaisante.
\subsection[{Chapitre IV. De l’Avarice et de la Prodigalité}]{Chapitre IV. De l’Avarice et de la Prodigalité}
\noindent Pour peu que l’on se soit fait une idée des intérêts de la société et du mérite attaché à l’humanité, à la bienfaisance, la compassion, la libéralité, on reconnaîtra que l’avarice est une disposition inhumaine et méprisable puisqu’elle est incompatible avec toutes ces vertus. Cette passion consiste dans une soif inextinguible des richesses pour elles-mêmes, sans jamais en faire usage ni pour son propre bien-être, ni pour celui des autres. Les richesses ne sont point le bonheur entre les mains de l’homme sensé ; elles ne sont que des moyens de l’obtenir parce qu’elles le mettent à portée de faire concourir un grand nombre d’hommes à sa propre félicité. L’avare est un homme isolé, concentré en lui-même, dont le cœur ne s’ouvre point à ses semblables. Accoutumé à se priver de tout, comment serait-il tenté d’entrer dans les besoins des autres ou de leur tendre une main secourable ? Il ne vit qu’avec son or ; cette idole inanimée est l’unique objet de son culte et de ses soins ; il l’adore en secret et lui sacrifie à chaque instant toutes ses autres passions, ainsi que toutes les vertus sociales. Il se refuse tout et s’applaudit de ses privations mêmes, qui deviennent pour lui des jouissances continuelles puisqu’elles le mènent au but qu’il se propose, qui est uniquement d’amasser.\par
Les moralistes ont avec raison condamné l’avarice. Les poètes ont jeté à pleines mains les traits de la satire sur elle ; il ne paraît cependant pas qu’ils aient suffisamment analysé les motifs cachés et puissants qui servent à nourrir dans quelques hommes cette passion insociable et qui les y attachent par des liens impossibles à briser. On nous dépeint l’avare comme un être malheureux parce qu’il se refuse des plaisirs que nous jugeons dignes d’envie ; mais l’avare est peu sensible à ces plaisirs ; il s’est fait un contentement à part qui, dans son imagination, l’emporte sur tout, ou plutôt qui lui présente tous les plaisirs réunis. Pourquoi va-t-il tout seul contempler son or ? C’est que son trésor peint à son esprit toutes les jouissances du monde ; ce trésor lui représente le pouvoir d’acquérir des honneurs, des palais, des terres, des possessions, des bijoux rares, des femmes, s’il a quelque sentiment de volupté. En un mot, dans son coffre l’avare voit tout, c’est-à-dire la facilité de se procurer, s’il voulait, tout ce qui fait l’objet des désirs des autres ; cette possibilité lui suffit, il ne va point au-delà. En employant son argent à l’acquisition de quelque objet particulier, son illusion cesserait, il ne lui resterait que la chose acquise ou le souvenir de quelque plaisir passé ; il ne verrait plus en imagination la faculté d’avoir tout ce que l’on peut se procurer avec l’argent.\par
L’avare se refuse tout, il est vrai, mais chaque privation devient un bienfait pour lui ; il lui fait des sacrifices souvent coûteux, peut-être, mais c’est le propre de toute passion dominante d’immoler toutes les autres à l’objet qu’elle chérit. Il sait bien qu’on le méprise\phantomsection
\label{footnote22}\footnote{« Populus me sibilat, at mihi plaudo ipse domi, simul ac nummos contemplor in arca. » Horace, {\itshape Satires}, livre I, 1, vers 66-67.}, mais il s’estime assez à la vue de son coffre, qu’il regarde comme sa force, comme son ami le plus sûr, comme renfermant ce qui peut lui procurer des avantages qu’il ne pourrait attendre du reste de la société. Il est sans compassion parce qu’il est sans besoins, ou du moins parce qu’il a le pouvoir de leur imposer silence ; il n’aime personne parce que son argent absorbe toutes ses affections ; il refuse le nécessaire à sa femme, à ses enfants, à son domestique, parce que le nécessaire lui paraît du superflu. Il est tourmenté par des inquiétudes, mais toute passion n’est-elle pas agitée par la crainte de perdre l’objet qu’elle chérit le plus ? Il n’est ni plus heureux ni plus malheureux que l’ambitieux qui se tourmente et qui craint de perdre son pouvoir, que l’amant jaloux qui soupçonne la fidélité de sa maîtresse, que l’enthousiaste de la gloire qui craint qu’elle ne lui échappe. Il n’est point de passion forte qui ne soit agitée et qui n’excite par intervalles de la honte et des remords ; mais ces sentiments pénibles sont bientôt effacés par les illusions que présente à l’imagination l’objet dont on est bien fortement enflammé.\par
Ainsi, l’avare est malheureux, sans doute, et par les tourments de sa propre passion, et par l’idée qu’elle produit sur les autres. Non seulement il les prive de tout mais encore il est capable des actions les plus basses pour assouvir la soif qui le brûle sans relâche ; enfin, dans l’excès de sa folie il est capable de se pendre après avoir perdu son or, parce que cette perte le prive du seul objet qui l’attache à la vie.\par
L’avarice est, comme beaucoup d’autres, une passion exclusive qui sépare l’homme de la société. Ce serait une erreur de croire que l’on est avare pour d’autres. Un père de famille prudent et sage est économe sans être avare ; il résiste à ses goûts, à ses fantaisies, il se prive des choses inutiles, il diminue ses dépenses pour faire un sort agréable à ses enfants. Mais l’avare est {\itshape personnel} ; ce n’est jamais par l’affection pour d’autres que l’on se charge d’une passion insupportable pour ceux qui n’en sont pas pleinement infectés. Nous voyons tous les jours des hommes qui, sans avoir d’héritiers, sans aimer leurs parents, sans dessein de faire jamais le moindre bien à personne, ne se permettant pas d’user de leur fortune immense, vivent dans une véritable indigence et, jusqu’au bord du tombeau, ne cessent d’accumuler des trésors dont ils ne feront aucun usage\footnote{« Non propter vitam faciunt patrimonia quidam, sed vitio cæci propter patrimonia vivunt. » Juvénal, {\itshape Satires}, XII, vers 50-51.}. Les vrais avares aiment l’argent et pour lui-même, et pour eux seuls. Ils le regardent comme un bien réel et non comme la représentation du bonheur ou comme un moyen de l’obtenir. L’homme sociable et raisonnable regarde l’argent uniquement comme le moyen d’obtenir des jouissances honnêtes, et l’homme vertueux ne connaît pas de jouissance plus vraie que de faire des heureux. Il est bienfaisant et libéral parce qu’il sait que c’est dans l’exercice de la bienfaisance que consiste tout l’avantage de la richesse sur l’indigence ou la médiocrité.\par
Le fils d’un avare est communément prodigue. Il a beaucoup souffert du vice de son père, et dès lors il se jette dans l’extrémité contraire : d’ailleurs ce père, en lui refusant tout, ne lui a pas permis d’apprendre le bon usage qu’on peut faire de son bien. Le prodigue se croit estimable en se livrant à une autre folie.\par
La prodigalité est le vice opposé à l’avarice. Cette passion fondée sur la vanité consiste à répandre sans mesure et sans choix les biens de la fortune, ou à faire de ses richesses un usage peu utile et pour soi-même et pour la société. Un prodigue n’est point un être bienfaisant, c’est un insensé qui ne connaît pas le véritable usage de l’argent, qui ne refuse rien à ses désirs les plus déréglés, qui veut s’illustrer par des dépenses dépourvues d’utilité ou par une sorte de mépris affecté pour les richesses, dont l’emploi devrait faire tout le prix\footnote{« Nescit quo valeat nummus, quem præbeat usum ? » Horace, {\itshape Satires}, livre I, 1, vers 73.}. César donnait au peuple romain des fêtes qui lui coûtaient des millions de sesterces ; ces prodigalités faites pour servir son ambition, n’avaient pour but que de corrompre de plus en plus un peuple déjà vicieux et corrompu. Les prodigalités de Marc Antoine et de Cléopâtre, qui faisaient dissoudre des perles d’un prix immense pour les avaler dans un repas, étaient de vraies folies produites par l’ivresse de l’opulence.\par
La prodigalité dans les princes, que l’on décore souvent du nom de bienfaisance, n’est qu’une faiblesse très criminelle : les peuples sont forcés de gémir pour les mettre en état de les satisfaire. Un souverain prodigue est bientôt obligé de devenir un tyran. Il est cruel pour son peuple parce qu’il veut contenter les courtisans qui l’entourent et qu’il voit, tandis qu’il ne voit pas son peuple et ne s’en soucie guère : on a soin de l’empêcher d’entendre les murmures du vulgaire méprisé.\par
Est-ce donc être bienfaisant que de piller la société toute entière pour enrichir les plus inutiles ou les plus nuisibles de ses membres ? Les prodigalités de Néron et d’Héliogabale étaient des outrages impudents faits à la misère publique.\par
Le prodigue se fait tort à lui-même ; parvenu à ruiner sa fortune, il ne lui reste guère de ressources chez ses amis ; inconsidéré dans son choix, il n’a communément répandu ses largesses que sur des flatteurs, des parasites, des hommes dépourvus de mœurs et de sentiments, sur des ingrats qui croient l’avoir suffisamment payé par leur basse complaisance et leurs lâches flatteries. Il n’y a que l’homme sage qui sache user de la fortune. L’homme vicieux, vain et frivole, ne fait qu’en abuser.\par
L’avare et le prodigue ont cela de commun que ni l’un ni l’autre ne connaissent l’usage des richesses qu’ils désirent également. L’un est avide pour amasser, l’autre est avide pour dépenser. Tous deux, quand ils le peuvent, montrent une égale rapacité qui les rend injustes et criminels, tous deux ne sont ni aimés ni estimés parce que l’avare ne fait du bien à personne. L’avare pille pour s’enrichir lui-même ; le prodigue vole et fraude ses créanciers, il se ruine et n’enrichit que des fripons et des gens méprisables qui seuls savent mettre son extravagance à profit.
\subsection[{Chapitre V. De l’Ingratitude}]{Chapitre V. De l’Ingratitude}
\noindent « Rien, dit un Ancien, ne vieillit plus promptement qu’un bienfait\footnote{Un Espagnol a dit aussi : « Celui à qui vous donnez, l’écrit sur le sable ; et celui à qui vous l’ôtez, l’écrit sur l’acier. »}. » Il n’est pas de vice plus détestable, et pourtant plus commun, que l’ingratitude. Platon le regarde comme renfermant tous les autres : il consiste dans l’oubli des bienfaits et quelquefois il va jusqu’à faire haïr le bienfaiteur. Rien de plus odieux, de plus injuste, de plus insociable que cette disposition criminelle. Elle rend celui qui s’en trouve coupable en quelque façon l’ennemi de lui-même. D’ailleurs, elle ne peut manquer de lui attirer la haine de toute la société : chacun sent en effet que l’ingratitude tend à décourager les âmes bienfaisantes, à bannir du commerce de la vie la compassion, la bonté, la libéralité, le désir d’obliger, qui sont ses plus doux liens. Il n’est donc point d’homme qui ne soit personnellement intéressé à partager l’inimitié que l’on doit aux ingrats. Méconnaître les bienfaits qu’on a reçus annonce une insensibilité, une injustice, une folie, une lâcheté surprenantes ; haïr celui qui nous a fait du bien indique une étrange férocité. Si les hommes réunis doivent se prêter des secours mutuels, quels motifs leur restera-t-il pour exercer leur bienveillance lorsqu’ils auront tout lieu de craindre qu’elle ne soit payée que par l’ingratitude et la haine ?\par
Quelque désintéressées que l’on suppose la bienveillance, la générosité, la libéralité, ces vertus ont nécessairement pour but d’acquérir des droits sur les cœurs de ceux que l’on oblige. Nul homme ne fait du bien à son semblable en vue d’en faire un ennemi : le citoyen généreux, en servant sa patrie, ne peut avoir le dessein de se rendre haïssable ou méprisable à ses yeux. Quiconque fait du bien s’attend avec raison à la reconnaissance, à la tendresse ou du moins à l’équité de ceux qu’il distingue. Lors même que la bienfaisance s’étend jusqu’aux ennemis, celui qui l’exerce a lieu de se flatter qu’il désarmera leur haine et qu’il en fera des amis. Les prétentions à l’affection et à la gratitude sont donc justes et fondées. Elles sont les motifs naturels de la bienfaisance, et ces mêmes prétentions ne peuvent être frustrées sans injustice et sans folie. L’ingratitude est si révoltante qu’elle est capable n’anéantir l’humanité au fond des cœurs les plus honnêtes.\par
Obliger des ingrats, faire du bien à des êtres injustes serait, dit-on, la preuve de la vertu la plus robuste, de la magnanimité la plus merveilleuse, de la générosité la plus rare, et peut-être souvent de la plus grande faiblesse. Mais peu d’hommes sont capables d’un désintéressement si parfait ; il supposerait un enthousiasme peu commun, une imagination assez féconde pour se dédommager par elle-même de l’injustice des autres. Tout homme qui nous oblige annonce qu’il veut acquérir sur notre affection et notre estime des droits que nous ne pouvons lui refuser sans crime ; il nous montre évidemment qu’il nous veut du bien, qu’il s’intéresse à nous, qu’il est à notre égard dans les dispositions que nous désirons naturellement de rencontrer. Ainsi, quels que soient ses motifs, nous ne pouvons nous dispenser d’accorder du retour à quiconque nous témoigne de l’intérêt, de la bonne volonté.\par
D’après des vérités si faciles à sentir, n’est-il pas surprenant de rencontrer tant d’ingrats sur la terre ? Néanmoins plusieurs causes semblent concourir pour les multiplier. L’orgueil et la vanité paraissent être en général les vraies sources de l’ingratitude. On surfait son propre mérite, et chacun alors regarde les bienfaits qu’il reçoit comme des dettes. Chacun croit trouver en soi la raison suffisante des services qu’on lui rend et n’en veut avoir obligation qu’à lui-même. D’ailleurs, on craint les avantages que l’on peut donner à ceux de qui l’on reçoit des bienfaits ; on appréhende qu’ils ne soient tentés d’abuser de la supériorité ou des droits qu’ils acquièrent. On a honte d’avouer que l’on dépend d’eux et que l’on a besoin de leurs secours pour sa propre félicité. Enfin, on craint qu’ils ne mettent à leurs bienfaits un si haut prix qu’on ne puisse les payer. On a très bien comparé les ingrats aux mauvais débiteurs qui redoutent la rencontre de leurs créanciers. Enfin, l’envie, cette passion fatale qui s’irrite même des bienfaits qu’elle reçoit et qui rend injuste et cruel envers ceux que l’on devrait chérir et considérer, devient souvent la cause de la plus noire ingratitude.\par
D’un autre côté, l’art de faire du bien, comme on l’a fait remarquer en parlant de la bienfaisance, est inconnu du plus grand nombre des hommes ; il exige une modestie, une délicatesse, un tact fin, qui puisse rassurer l’amour-propre de ceux que l’on oblige et dont on veut mériter la gratitude. Cet amour-propre est si prompt à s’allumer que le bienfaiteur a besoin de toutes les ressources de l’esprit pour ne point offenser les personnes qu’il a dessein d’obliger. Les orgueilleux, les hommes vains, impérieux, fastueux et prodigues, ne connaissent aucunement l’art de faire du bien. Aussi sont-ils communément des ingrats : il n’y a que les personnes sensibles qui sachent obliger. En faisant du bien, l’orgueilleux ne veut qu’étendre son empire, augmenter le nombre de ses esclaves, leur montrer à chaque instant son pouvoir et sa supériorité. L’homme fastueux ne veut que faire parade de ses richesses ou de son crédit et répand indistinctement ses faveurs pour augmenter sa cour. Tous ceux qui en faisant du bien ne cherchent qu’à multiplier autour d’eux des flatteurs, des esclaves, des jouets de leurs fantaisies, ne doivent guère s’attendre à beaucoup de reconnaissance. Ces hommes abjects croiront toujours s’être pleinement acquittés par leurs bassesses et leurs viles complaisances. Il n’y a que la vertu modeste qui puisse s’attirer la confiance des âmes honnêtes et vertueuses ; il n’y a que les âmes de cette trempe qui soient véritablement reconnaissantes.\par
Il est rare que les grands sachent véritablement obliger ou faire du bien : peu habitués à se contraindre, ils obligent avec hauteur et demandent souvent des sacrifices trop coûteux en échange de leurs faveurs. Rien de plus cruel pour une âme honnête que de ne pouvoir aimer ni estimer ceux qui lui font du bien, et que d’être intérieurement forcé de les haïr ou de les mépriser. Comment s’attacher sincèrement à des hommes qui, par leur conduite altière et leurs procédés humiliants, prennent soin de dispenser d’avance tous ceux qu’ils obligent de la reconnaissance que ceux-ci voudraient sentir pour eux ? Est-il une position plus affreuse que celle d’un fils bien né que la tyrannie d’un père force à ne point aimer l’auteur de ses jours, celui à qui son cœur voudrait pouvoir montrer la gratitude la plus tendre, l’attachement le plus vrai ? Les tyrans de toute espèce ne peuvent faire que des ingrats.\par
D’un autre côté, les princes, les riches et les grands de la terre se rendent ordinairement coupables de la plus noire ingratitude. Élevés au-dessus des autres, ils s’imaginent que personne ne peut les obliger, que nul homme n’est en droit de penser qu’il a pu leur rendre des services assez grands pour mériter de leur part de la reconnaissance. Entourés de sycophantes et de flatteurs, vous les voyez disposés à croire que tout leur est dû, qu’ils ne sont jamais en reste avec ceux qui les servent, qu’ils ne doivent rien à personne, que l’avantage de les servir est un honneur assez grand pour les dispenser des sentiments qu’ils exigent des autres. Les tyrans, toujours inquiets et lâches, sont prêts, sur les moindres soupçons, à payer les services par la disgrâce et souvent par la mort\phantomsection
\label{footnote23}\footnote{Le sultan Bajazet II fit mourir Acomat son vizir, qui avait assuré son trône et considérablement étendu son empire, parce que, comme ce prince en convenait lui-même, « il se trouvait dans l’impossibilité de reconnaître dignement les services qu’il en avait reçus. » — Par un motif semblable, Caligula fit périr Macron, à qui il était redevable de l’Empire. — Tibère, ayant appris que l’auguste Lentulus l’avait par son testament institué son héritier, lui envoya des satellites pour le tuer afin de jouir de sa succession. — Louis XI, qui s’y connaissait, avait coutume de dire que « les grands bienfaits faisaient les grands ingrats ».}. D’ailleurs, les services éclatants donnent à ceux qui les rendent un lustre capable d’allumer les âmes rétrécies de ces orgueilleux potentats. Ils sont communément assez petits pour être jaloux de la gloire acquise par des citoyens que leurs grandes actions semblent mettre au niveau de leurs superbes maîtres : l’envie ne permet jamais aux tyrans d’aimer sincèrement les hommes qui les effacent.\par
C’est, comme nous le verrons bientôt, à la crainte de la supériorité et à l’envie qu’excitent les grands talents, que sont dues ces marques révoltantes de la plus noire ingratitude dont des peuples entiers se sont rendus coupables envers les magistrats et les chefs qui les avaient servis le plus utilement. Les républiques d’Athènes et de Rome nous fournissent des exemples mémorables de l’injustice des nations envers leurs plus grands bienfaiteurs. Les hommes en corps ne semblent jamais rougir de leur ingratitude. Celui qui fait du bien au public n’est souvent récompensé par personne.\par
C’est à l’envie toujours subsistante que l’on doit attribuer les injustices si fréquentes du public pour ceux qui lui ont autrefois procuré les plaisirs les plus grands, les découvertes les plus intéressantes. Voilà pourquoi les hommes de génie furent en tout temps persécutés, punis des services qu’ils avaient rendus à leurs contemporains, forcés d’attendre de la postérité, plus équitable, la récompense et la gloire que méritent leurs talents. Le public est composé d’un petit nombre de personnes justes et d’une foule immense d’être injustes, lâches, envieux, que les grands hommes offusquent et qui font tous leurs efforts pour les déprimer.\par
Faut-il obliger des ingrats ? Oui. Il est grand de mépriser l’envie. Il faut faire du bien aux hommes en dépit d’eux, il faut se contenter des suffrages des gens de bien, il faut en appeler de ses contemporains ingrats à la postérité, toujours favorable aux bienfaiteurs du genre humain. Enfin, au défaut des applaudissements et des récompenses qu’il mérite, tout homme vraiment utile à ses semblables, tout homme généreux trouvera dans les applaudissements de sa propre conscience le salaire le plus doux des services qu’il rend à la société. L’injustice et l’ingratitude des hommes réduit souvent la vertu à se payer de ses propres mains.
\subsection[{Chapitre VI. De l’Envie. De la Jalousie. De la Médisance}]{Chapitre VI. De l’Envie. De la Jalousie. De la Médisance}
\noindent L’envie, ce tyran acharné du mérite, des talents, de la vertu, est une disposition insociable qui fait haïr tous ceux qui possèdent des avantages et des qualités estimables.\par
La jalousie, qui tient beaucoup à l’envie, est l’inquiétude produite en nous par l’idée d’un bonheur dont nous supposons que les autres jouissent tandis que nous en sommes privés nous-mêmes.\par
L’orgueil est la source de l’envie ; l’amour de préférence que chaque homme a pour soi, lui fait haïr dans les autres les avantages capables de leur donner dans la société une supériorité que chacun désirerait pour lui-même. « Ceux, dit Sophocle, qui insultent les grands hommes, semblent ne point faire du mal ; il sont sûrs de s’entendre applaudir. » Tout mortel qui se fait remarquer par des talents, du mérite, du bonheur, du crédit, des richesses, devient l’objet de l’envie publique. Chacun voudrait jouir préférablement à lui de tous ces avantages. On porte envie aux princes, aux grands, aux riches, parce qu’on sait que leur pouvoir et leur fortune les mettent à portée d’exercer un empire que l’on voudrait exercer en leur place et dont on se flatte que l’on ferait un bien meilleur usage.\par
La jalousie, au contraire, suppose une idée basse de soi-même, une absence des avantages ou qualités que l’on voit ou que l’on suppose exister dans ceux dont on est jaloux. Un amant est jaloux de son rival parce qu’il craint de n’avoir pas aux yeux de sa maîtresse autant d’agréments que celui qui cause ses inquiétudes. Les pauvres sont jaloux des riches parce qu’ils se sentent dépourvus des moyens que ceux-ci peuvent employer pour obtenir tous les plaisirs dont les premiers sont privés.\par
L’envie et la jalousie sont des sentiments naturels à tous les hommes, mais que pour son propre repos et le bien de la société, un être sociable doit soigneusement réprimer. L’envieux est celui qui n’a point appris à combattre et à vaincre une passion aveugle aussi funeste à lui-même qu’aux autres. La vie sociale devient un tourment continuel pour un être affligé de cette passion malheureuse, tout devient à ses yeux un spectacle déchirant ; il n’est point d’avantages obtenus par quelqu’un qui ne portent un coup mortel à l’envieux. L’opulence de ses concitoyens le désole, leur élévation l’irrite, leur réputation le blesse, les éloges qu’on leur donne sont des coups de poignard, la gloire qu’ils acquièrent le met au désespoir ; en un mot, il n’est point de paix pour un homme assez mal conformé pour s’irriter de tous les biens qu’il voit arriver aux autres.\par
S’il veut se soustraire au spectacle désolant de la félicité publique, il n’a rien de mieux à faire que fuir pour dévorer son propre cœur dans une affreuse solitude.\par
L’envie est un sentiment honteux qui n’ose se montrer parce qu’il blesserait tous ceux qu’on en rendrait témoins {\itshape ;} aussi sait-il se cacher sous une infinité de formes diverses. Nul homme n’ose convenir qu’il porte envie aux autres : sa passion se masque sous le nom d’amour du bien public quand elle veut déprimer ceux qui lui déplaisent. Alors elle s’indigne à la vue des places éminentes accordées à des hommes dépourvus de mérite, elle gémit de l’opulence qu’elle voit entre les mains de gens peu faits pour la posséder. Prétextant un amour pur de la vérité, elle va fouiller dans les secrets des cœurs pour donner des motifs odieux et bas aux actions les plus belles. Elle cherche dans la conduite des hommes tout ce qui peut les rabaisser. Elle chérit la médisance parce qu’elle dégrade ses rivaux.\par
L’envie tient lieu de morale à bien des gens ; peu sensible aux intérêts de la vertu ou au bien de la société, l’envieux devient un lynx quand il s’agit de dévoiler les vices et les défauts de ceux dont le bien-être l’offusque. L’envie devient audacieuse, emportée, quand elle peut se déguiser sous le nom de zèle pour la vertu.\par
Sous prétexte de bon goût, elle critique sans cesse et ne trouve rien de bon. Elle écoute avidement les sarcasmes et les épigrammes ; la raillerie, la satire la plus cruelle sont pour elle des aliments délicieux. Ils suspendent quelques instants la douleur que lui causent le mérite et les talents. Elle adopte sans examen la calomnie, parce qu’elle sait qu’elle laisse toujours après elle des cicatrices qu’il sera difficile de faire disparaître. En un mot, la malignité, la méchanceté, la noirceur sont les dignes compagnes de l’envie, à l’aide desquelles elle réussit au moins à tourmenter le mérite, à le décourager lorsqu’elle ne parvient pas à l’étouffer.\par
La médisance est une vérité nuisible à ceux qui en sont les objets. Le médisant n’est pas un homme véridique : il n’est qu’un envieux, un malin, un méchant dont les discours ne peuvent plaire qu’à des êtres qui lui ressemblent. S’il n’existait point d’envieux, la médisance serait bannie de la société ; on n’écoute la médisance avec tant d’empressement que parce qu’elle déprime les autres dans l’opinion publique. Chacun voit un ennemi de moins dans le grand homme que l’on attaque ou que la méchanceté veut détruire. « Le médisant, dit Quintilien, ne diffère du méchant que par l’occasion. Il ne fait du mal par ses discours que parce qu’il est trop lâche pour en faire par ses actions\footnote{« Maledicus a malefico non distat nisi occasione. » Quintillien, {\itshape Institutions oratoires}, livre 12, chap. 9, n° 9, édition Gesner, Göttingue, 1738, in-4°.}. »\par
Le médisant est un homme vain qui, en révélant les infirmités des autres, ne veut souvent que persuader qu’il est sain. D’ailleurs, il se pique d’être véridique, tandis qu’il n’est qu’un hypocrite qui fait un étalage de sentiments honnêtes, mais toujours faux dès qu’ils ne sont pas accompagnés de bonté, d’indulgence, d’humanité. Le médisant devrait être regardé comme un ennemi public ; cependant, on l’écoute, et l’on dirait que les hommes ne le fréquentent que pour avoir le plaisir de se dire du mal les uns des autres.\par
Pour guérir les hommes de l’envie et de la jalousie qui les tourmentent, ainsi que de la médisance et de la détraction, il serait à propos de leur faire voir que leurs efforts sont inutiles contre le mérite et les vrais talents. En vain la médisance s’exerce sur l’homme de bien. Eh ! Ne sait-on pas que nul mortel sur la terre n’est exempt de défauts ? Une injuste critique veut-elle déprécier les productions du génie ? Ne sait-on pas que le génie est inégal et ne peut être régulier dans sa marche ? Des fautes minutieuses ont-elles jamais fait tomber dans l’oubli les ouvrages immortels de l’esprit humain ? La calomnie veut-elle noircir la probité ? Tôt ou tard l’iniquité se découvre, elle tourne à la confusion de l’envieux qui la fait éclore et rend l’innocence, qu’elle voulait opprimer, plus aimable et plus intéressante.\par
Qu’il y aurait peu d’envieux si l’on réfléchissait combien il y a peu d’hommes vraiment heureux ou dignes d’être enviés ! Les grands sont enviés parce qu’on les suppose les plus contents des mortels ; mais comment un homme qui pense pourrait-il envier des courtisans perpétuellement tourmentés par une envie mutuelle, par des alarmes continuelles, par des chagrins cuisants, par des intrigues aussi longues que la vie ? Le riche est l’objet de la jalousie et de l’envie du pauvre ; pour détromper celui-ci, qu’on lui apprenne qu’avec tous les moyens capables de se procurer le bien-être et le repos, le riche n’en met souvent aucun en usage. Dévoré de la soif des richesses, il n’en a jamais assez. Rongé par l’ambition, il n’est jamais satisfait de son sort. Rassasié de plaisirs, il ne connaît plus aucun moyen de s’amuser. Fatigué de son désœuvrement, il est tombé dans l’ennui, le plus cruel de tous les tourments dont la Nature puisse punir l’homme qui ne veut point travailler. Enfin, tout prouve à l’indigent laborieux que son destin, qui lui paraît si lamentable, l’exempte d’une infinité de besoins imaginaires, d’intrigues, de peines d’esprit, dont la grandeur et l’opulence sont sans cesse agitées.\par
Pour détromper les auditeurs envieux ou malins du plaisir que leur cause la médisance, nous les avertirons qu’ils doivent s’attendre que le même personnage dont ils écoutent avidement les discours malins, dont ils savourent les satires impitoyables, en quittant la compagnie, va divertir à ses dépens un autre cercle de gens aussi bien disposés.\par
Enfin, pour détromper le médisant lui-même du plaisir qu’il trouve à nuire, nous lui représentons la bassesse du rôle qu’il joue, qui ne peut que le faire craindre sans jamais le faire aimer ni estimer. La réputation de méchant est-elle donc bien digne de l’ambition d’un être sociable ? Est-il un métier plus vil et plus bas que celui de délateur public ? N’est-ce pas se rendre complice de son infamie que de l’écouter avec plaisir ? N’est-ce pas se déshonorer que de l’admettre dans sa familiarité ? « Le délateur, dit un moderne, étant le plus vil des hommes, déshonore les personnes qui le fréquentent, bien plus que ne le ferait le bourreau ; la conduite du premier est l’effet de son mauvais caractère, au lieu que le bourreau fait son métier\footnote{Voyez l’ouvrage anglais nommé {\itshape Adventurer}, n° 46.}. » Celui-ci fait du mal par devoir, l’autre en fait pour son plaisir. Est-il donc un plaisir plus détestable que de courir de maisons en maisons pour dénigrer ses concitoyens, pour divulguer les traits qui peuvent leur nuire, pour leur ravir la réputation et le repos sans profit réel pour la société ?\par
Le médisant nous dira peut-être qu’il faut être vrai et qu’il importe au public de connaître les hommes ; il ajoutera qu’il ne médit que des personnes indifférentes auxquelles il ne doit rien. Mais nous lui répondrons que la vérité n’est utile au public que lorsqu’il s’agit de crimes, et non de défauts et d’infirmités cachés. L’homme véridique n’est qu’un lâche assassin lorsqu’il apprend des vérités capables d’anéantir la bonne opinion, de refroidir la bienveillance, de nuire à la fortune de ses concitoyens ; on n’est guère porté à faire du bien à ceux dont on a mauvaise idée. Enfin, nous lui dirons qu’un être sociable doit, même aux inconnus, aux indifférents, aux étrangers, des égards et des ménagements, et qu’en y manquant il donne au premier venu le droit de le dénigrer lui-même et de divulguer ses secrets. Est-il un homme assez vain pour se flatter d’être sans défauts ? S’il n’est personne qui consente que ses faiblesses soient exposées, il s’ensuit que nous devons couvrir celles des autres.\par
Sous quelque point de vue que l’on envisage la médisance, elle est très condamnable par les ravages, les inimitiés, les querelles qu’elle produit à tout moment. Elle cause beaucoup de mal et ne fait aucun bien ; on hait le médisant, quoique la médisance plaise. La médisance est fille de la haine, de l’humeur, de l’envie et de l’oisiveté. Elle n’a point à se glorifier d’une origine si méprisable. Le vide de l’esprit, l’incapacité de s’occuper, le désœuvrement alimentent ce vice odieux. Faute de pouvoir parler de choses, on parle de personnes. Rien de plus utile que de savoir se taire. Le besoin de parler est un des grands fléaux de toutes les sociétés.
\subsection[{Chapitre VII. Du Mensonge. De la Flatterie. De l’Hypocrisie. De la Calomnie}]{Chapitre VII. Du Mensonge. De la Flatterie. De l’Hypocrisie. De la Calomnie}
\noindent La parole doit servir aux hommes pour se communiquer leurs pensées, pour se prêter des secours mutuels, pour se transmettre les vérités qui peuvent leur être utiles, et non pour se détruire réciproquement et se tromper. Le menteur pèche contre tous ces devoirs, et par conséquent se rend nuisible à ses associés. Mentir, c’est parler contre sa pensée, c’est induire les autres en erreur, c’est violer les conventions sur lesquelles est fondé le commerce du langage, qui deviendrait très funeste si les hommes ne s’en servaient que pour s’abuser les uns les autres. Disons donc avec la franchise de Montaigne : « En vérité, le mentir est un maudit vice. Nous ne sommes hommes et ne nous tenons les uns aux autres que par la parole : si nous en connaissions l’horreur et le poids, nous le poursuivrions à feu plus justement que d’autres crimes\footnote{Voyez {\itshape Essais} de Montaigne, livre I, chap. 9.}. » Aristote dit que « la récompense du menteur est de n’être point cru, quand même il parle vrai ».\par
Tous les moralistes sont d’accord sur l’horreur que doit inspirer le mensonge. Ceux qui en ont contracté la malheureuse habitude perdent toute confiance de la part des autres. La parole leur devient pour ainsi dire inutile. En effet, le vice est bas et servile, il annonce toujours la crainte ou la vanité. L’homme de bien est sincère, il n’a rien à craindre en montrant la vérité, qui ne peut que lui être avantageuse. Les enfants et les valets sont les plus sujets à mentir, parce que leur conduite inconsidérée les expose sans cesse à des corrections désagréables. Apollonius disait qu’il n’appartenait qu’aux esclaves de mentir. Les Perses, selon Hérodote, notaient les menteurs d’{\itshape infamie}. Les lois des Indiens, suivant Philostrate, voulaient que tout homme convaincu de mensonge fût déclaré incapable de remplir aucune magistrature. Cette infamie attachée au mensonge subsiste encore parmi les nations modernes, chez lesquelles un démenti est réputé une insulte si grave que l’on se croit obligé de la laver dans le sang.\par
Suivant Plutarque, Epænetus avait coutume de dire que « les menteurs sont la cause de tous les crimes qui se commettent dans le monde\footnote{Voyez Plutarque, dans {\itshape Les Dits notables des Lacédémoniens}.} ». Il a raison, sans doute : l’erreur et l’imposture sont les sources fécondes de toutes les calamités dont le genre humain est affligé. Indépendamment des erreurs qui sont dues à l’ignorance des hommes, il en est un grand nombre qui leur viennent des menteurs qui ont pris soin de tromper leur crédulité pour les soumettre plus sûrement à leur empire.\par
Un imposteur s’élève en Arabie, et débite au nom du Ciel des mensonges qu’il parvient à faire respecter d’une partie de ses concitoyens ; bientôt ces mensonges, devenus sacrés, se propagent par le fer dans l’Asie, l’Afrique et l’Europe. Ils autorisent des fanatiques ambitieux à conquérir toute la terre et à l’arroser de sang. La loi de Mahomet s’établit sur la violence, elle renverse les trônes, et sur les ruines du monde établit la tyrannie musulmane. C’est ainsi que des menteurs forment des frénétiques qui se font un devoir de troubler l’univers, des hypocrites qui cherchent à profiter des malheurs des hommes, des tyrans qui enchaînent les peuples et les obligent à contribuer, aux dépens de leur vie, à leurs injustes projets.\par
Parmi les moyens de tromper les hommes, il n’en est point qui ait produit dans tous les temps de plus grands malheurs que la flatterie. Diogène disait que « le plus dangereux des animaux sauvages c’est le médisant, et des animaux privés, c’est le flatteur ».\par
On a bien défini la flatterie en disant qu’elle est un commerce de mensonge, fondé d’un côté sur l’intérêt le plus vil et de l’autre sur la vanité. Le flatteur est un menteur qui trompe pour se rendre agréable à celui dont il a le projet de séduire la vanité. C’est un perfide qui lui plonge un glaive enduit de miel\footnote{« Adulatio mellitus gladius. » Hieron.}. « Qui vous flatte, vous hait », a dit un sage arabe\footnote{Voyez {\itshape Sentences arabes}, in {\itshape Gramm. Erpenii}.}.\par
En effet, tout flatteur est forcé de s’abaisser devant le sot qu’il encense. C’est une humiliation qui doit coûter à sa vanité ; il doit haïr et mépriser celui qui le réduit à s’avilir. Les princes et les grands se trompent lourdement quand ils se croient aimés par des hommes vils qui les entourent. Personne ne peut aimer celui qui le dégrade. Nonobstant la bassesse de convention à la cour, nul flatteur n’est assez intrépide pour ne jamais rougir.\par
« La flatterie, dit Charron, est pire que le faux témoignage. Il ne corrompt pas le juge, il ne fait que le tromper. Au lieu que la flatterie corrompt le jugement et le rend inaccessible à la vérité\footnote{Voyez Charron, {\itshape De la Sagesse}, livre III, chap. 10.}. » Tant de princes ne font le mal avec tant de constance que parce qu’ils sont entourés de flatteurs qui leur disent qu’ils font bien, que leurs sujets sont heureux, que l’on bénit leur règne, qu’ils peuvent continuer sans crainte à donner un libre cours à toutes leurs passions.\par
Ainsi, des empoisonneurs publics parviennent à rendre inutiles les dispositions les plus heureuses. Ils infectent les meilleurs princes dès l’enfance, ils en font des tyrans stupides qui deviennent par degrés les fléaux de leurs sujets. S’il n’y avait point de flatteurs, il n’y aurait pas de tyrans sur la terre. La flatterie est donc évidemment la trahison la plus noire, c’est un crime détestable qui, après avoir livré la société à la tyrannie, expose le tyran à des révolutions terribles et souvent à sa propre destruction. Le flatteur est l’ennemi le plus dangereux et des peuples et des rois. Tous les hommes aiment la flatterie parce que tous ont plus ou moins d’orgueil, de vanité, de bonne opinion d’eux-mêmes. Rien de plus rare que ceux qui ont la prudence ou la force de résister aux pièges des flatteurs ; chacun adopte la flatterie, même en reconnaissant qu’elle est un pur mensonge. Chacun dit, avec Térence : « Je sais bien que tu mens, mais continue de mentir, car tu me fais grand plaisir\phantomsection
\label{footnote24}\footnote{« Mentiris, Dave ; perge tamen, places. » Térence.}. » Un poète célèbre assure avec raison que « personne n’est entièrement inaccessible à la flatterie, et que l’on flatte un homme qui montre de la haine aux flatteurs, en le louant de haïr la flatterie\phantomsection
\label{footnote25}\footnote{Shakespeare, dans la tragédie d’{\itshape Othello}.} ».\par
La flatterie commence toujours par aveugler les hommes. En examinant avec soin le faible de celui qu’ils ont envie de tromper, les flatteurs finissent par le trouver. On les a très bien comparés aux voleurs de nuit, dont le premier soin est d’éteindre les lumières dans les maisons qu’ils veulent piller. Antisthène disait avec autant de justesse, que « les courtisanes souhaitent à leurs amants tous les biens, hors le bon sens et la sagesse ». Les flatteurs font les mêmes vœux pour tous ceux qu’ils veulent attirer dans leurs pièges. « Si tu ne reconnais pas en toi des choses estimables, dit Démophile, soit assuré que les autres te flattent. » On a très justement remarqué que les tyrans les plus détestés ont été les plus flattés {\itshape ;} n’en soyons point surpris : les princes les plus méchants sont communément les plus vains, les plus ombrageux, les plus à redouter ; ainsi, la crainte, venant se joindre à la bassesse, la pousse au-delà de toutes les bornes.\par
Elle ne peut aller trop loin quand il s’agit de plaire à un tyran, qui est pour l’ordinaire et méchant et stupide. La flatterie ne fait qu’enorgueillir la sottise et donner de l’audace à la perversité ; « c’est faire un grand mal aux sots, dit le même poète, que de les applaudir\phantomsection
\label{footnote26}\footnote{Voyez {\itshape Poetæ græci minores. Demophili sentencia}. — Dion Cassius, parlant de Séjan, remarque que plus les hommes sont sots ou dépourvus de mérite, et plus ils sont affamés de flatterie et de soumissions. Vid. Dion. Cass. {\itshape Histor. in} Tiber., livre 58, chap. 5, p. 879.}. »\par
La flatterie la plus basse, la plus servile, la plus fade ne révolte pas un esprit rétréci ; mais il faut à l’homme vain, quand il a quelque pudeur, une flatterie plus délicate, il lui faut un poison préparé par des mains plus habiles : une flatterie grossière effaroucherait sa vanité. Tibère haussait les épaules à la vue des bassesses que des sénateurs maladroits\phantomsection
\label{footnote27}\footnote{« Memoriæ proditur Tiberium, quoties curia egrederetur, Græcis verbis in hunc modum eloqui solitum : {\itshape O homines ad servitutem paratos} ! Scilicet etiam illum qui libertatem publicam nollet tant proiectæ servientium patientiæ tædebat. » Tacite, {\itshape Annales}, livre 5, chap. 65, in fine.} employaient pour le flatter. Le même Alexandre qui poussa la folie jusqu’à vouloir se faire passer pour un dieu, réprima quelquefois les flatteurs qui lui offraient un encens peu délicat. L’adulation est désagréable quand elle annonce trop de bassesse dans celui qui la prodigue. Les personnes les plus sensibles à la flatterie n’en sont que peu ou point touchées quand elle part d’un homme qu’elles sont forcées de mépriser. Il faut, pour leur plaire, que le flatteur annonce quelque mérite, et surtout qu’il affecte de la sincérité. Nul homme ne peut aimer des flatteries dépourvues de vraisemblance : on veut qu’elles aient au moins quelque lueur de vérité. Quoiqu’il en soit, la flatterie annonce toujours bassesse dans celui qui la prodigue et sotte vanité dans celui qui s’y laisse surprendre. L’adulateur semble faire à celui qu’il flatte un sacrifice entier de son orgueil et de son amour-propre ; ce n’est pas qu’il soit exempt de ces vices, mais il sait en suspendre l’effet. Rien de plus commun que de voir les esclaves les plus rampants en présence du maître, montrer la hauteur la plus insolente à leurs inférieurs. Quoique l’ambition soit le fruit de l’orgueil, elle s’abaisse à flatter pour obtenir la faculté de faire sentir aux autres le poids de sa puissance subalterne. Rien de plus arrogant et de plus fier qu’un esclave : il se dédommage sur les autres des outrages qu’il essuie de la part de ceux qu’il est obligé de flatter. En s’abaissant jusqu’à terre, le flatteur ambitieux ne fait que prendre son élan.\par
Quelques moralistes outrés ont prétendu qu’il n’était jamais permis de mentir, quand même il s’agirait du salut de l’univers\footnote{St. Augustin.}. Mais une morale plus sage ne peut adopter cette maxime insociable. Un mensonge qui sauverait le genre humain serait l’action la plus noble dont un homme fût capable, un mensonge qui sauverait la patrie serait une action très vertueuse et digne d’un bon citoyen. Une vérité qui la ferait périr serait un crime détestable. Un mensonge qui sauverait la vie d’un père, d’un ami, d’un homme innocent injustement opprimé, ne peut paraître criminel qu’aux yeux d’un insensé. La vertu est toujours l’utilité des êtres de notre espèce. Une vérité qui nuit à quelqu’un, sans profit pour la société, est un mal réel {\itshape \textbackslash} un mensonge utile à ceux que nous devons aimer et qui ne fait tort à personne, ne mérite aucunement d’être blâmé.\par
Le mensonge peut se trouver dans la conduite, ainsi que dans le discours. Il est des hommes dont la conduite est un mensonge continuel. L’hypocrisie est un mensonge dans le maintien ainsi que dans les paroles, dont l’objet est de tromper en montrant au dehors des vertus dont on est totalement dépourvu. Le méchant le plus décidé est beaucoup moins dangereux que le perfide qui nous trompe sous le masque de la vertu. On peut se mettre en garde contre le premier, au lieu qu’il est presque impossible de se garantir des coups imprévus de l’homme qui nous séduit par des dehors imposteurs.\par
L’hypocrite a été très justement comparé au crocodile, qui semble, dit-on, déplorer le sort de ceux qu’il est prêt à dévorer.\par
L’hypocrisie demande un art infini pour tromper longtemps sans se démasquer elle-même ; il en coûterait cent fois moins pour acquérir les vertus qu’elle affecte, que pour les montrer. Que de tourments et d’avanies les hommes s’épargneraient s’ils étaient plus vrais ou s’ils se faisaient un principe de ne paraître que ce qu’ils sont ! Tromper longtemps suppose une attention, un travail assidu dont peu de gens sont capables. La meilleure des politiques consiste évidemment à être bon et sincère.\par
La trahison est un mensonge dans la conduite ou le discours ; elle consiste à faire du mal à ceux à qui nous devons faire du bien, ou que nous avons trompés par des marques de bienveillance. Trahir sa patrie, c’est livrer à ses ennemis la société que nous sommes obligés de défendre ; trahir son ami, c’est nuire à l’homme que nous avions mis en droit de compter sur notre affection. La trahison suppose une lâcheté et une dépravation détestables ; ceux mêmes qui en profitent le plus ne peuvent estimer ou aimer les infâmes qui s’en rendent coupables. On aime quelquefois la trahison mais on déteste les traîtres, parce que jamais il n’est possible de s’y fier. Tout tyran est un traître qui nuit à la société, au bonheur de laquelle il s’est engagé de veiller. Tout citoyen qui favorise et soutient la tyrannie est un traître que ses concitoyens devraient regarder avec horreur.\par
La vanité, dont tant d’hommes frivoles et légers sont infectés, fait éclore une infinité de mensonges dans la conduite, que l’on nomme des prétentions. Elles font le tourment et de ceux qui les ont, et de ceux qu’elles importunent dans le commerce de la vie. Si l’hypocrisie et l’imposture sont des mensonges, il est évident que ceux qui montrent des prétentions en tout genre sont des menteurs. Les personnes sensées méprisent une foule d’hommes qui, par leur jactance, leur fatuité, leur affectation, leur vanité, portent continuellement la discorde et le trouble dans la société. Les compagnies destinées à l’amusement de ceux qui les composent, deviennent souvent des rendez-vous où des menteurs viennent se fatiguer réciproquement par leurs prétentions, leurs impertinences et leurs sottises. L’un prétend à l’esprit, l’autre à la science, d’autres mêmes à la vertu, tandis que personne ne se met en peine d’acquérir les qualités qui le rendraient vraiment estimable. Sois ce que tu veux paraître, voilà la maxime que doit suivre tout homme prudent et sage.\par
Si les vaines prétentions des hommes sont des mensonges incommodes pour la société et qu’elle punit le ridicule, il en est d’autres pour lesquelles elle montre une juste horreur, relativement aux désordres affreux qu’ils y causent ; de ce nombre est la calomnie. Elle consiste à mentir contre l’innocence, à lui imputer faussement des fautes ou des actions capables de lui ravir l’estime publique, et même de lui attirer d’injustes châtiments. D’où l’on voit que ce crime viole insolemment la justice, l’humanité, la pitié, en un mot, les vertus les plus saintes ; par conséquent, il intéresse également tous les citoyens, dont chacun est exposé aux traits publics ou cachés de la calomnie.\par
Quelque affreux que soit le crime, il est pourtant très commun sur la terre ; rien de plus surprenant que la promptitude avec laquelle la calomnie se répand parmi les hommes. Par un phénomène très étrange, au premier coup d’œil ils détestent la calomnie et en sont perpétuellement les complices et les dupes. Pour cesser d’en être étonné, il suffit de voir les sources d’où part ce crime destructeur : il est dû principalement à l’envie, à la vengeance, à la colère, à la malignité, qui prend un secret plaisir à démolir ou troubler la félicité des autres. D’un autre côté, l’imprudence, la légèreté, l’étourderie empêchent de voir les choses telles qu’elles sont et de pressentir les conséquences des discours que l’on tient. Les mêmes causes qui font naître la calomnie, les propagent avec la plus grande facilité. On l’adopte sans examen parce qu’on se plaît à voir déprimer ses semblables. La malignité est toujours intimement liée à l’envie. Le zèle pour la vertu anime souvent l’homme de bien trop crédule contre celui qu’on calomnie, et le trouble au point de n’en pas peser suffisamment les preuves. Enfin, l’imprudence, si commune parmi les hommes, fait qu’ils n’accordent pas l’attention convenable à l’examen des faits qu’on leur débite ; on les reçoit légèrement et on les répand de même, sans prévoir à quel point cette légèreté peut devenir funeste à celui dont on immole la réputation, et peut-être la vie.\par
La discrétion, la réflexion, la suspension de jugement, voilà les moyens de se garantir d’un crime si détestable par ses effets et dans lequel la crédulité devient elle-même coupable. Les princes, perpétuellement entourés d’hommes envieux et légers, devraient surtout ne point prêter l’oreille à des discours qui les exposent souvent à sacrifier les hommes les plus vertueux à la haine ou à l’envie de quelques scélérats qui ne possèdent que l’art affreux de nuire.\par
Pour se mettre en garde contre les impressions de la calomnie, il suffit de réfléchir aux passions des hommes. D’ailleurs, l’expérience nous prouve que très peu de gens ont la capacité de bien voir les faits mêmes dont ils sont les témoins. Très peu de gens rapportent fidèlement ce qu’ils ont vu, ce qu’ils ont entendu. Souvent il est difficile de vérifier les faits que nous devrions être à portée de connaître le mieux ; des circonstances qui semblent indifférentes ou minutieuses peuvent aggraver ou diminuer les imputations. Enfin, tout nous invite à nous défier les uns des autres et de nous-mêmes ; souvent nous sommes sujets à nous tromper de la meilleure foi au monde.\par
Tout doit donc nous faire sentir à quel point le mensonge peut devenir funeste, sous quelque forme qu’il se présente. C’est à lui que sont dues la mauvaise foi, la perfidie, la fraude, la duplicité, les charlataneries et fourberies de toutes espèces, les fables dont tant de nations sont abreuvées. Si la véracité, comme nous l’avons prouvé, est une vertu nécessaire, tout ce qui tend à tromper les mortels doit être blâmé. D’ailleurs, tout imposteur alarme l’amour-propre des autres : personne ne veut être dupe, et chacun se venge de l’homme qui a prétendu lui en imposer. L’affection que l’on avait pour lui se change souvent en haine, on croit ne pouvoir trop le rabaisser ; la vengeance de l’amour-propre blessé, souvent injuste, va jusqu’à lui refuser tout mérite et toute vertu.\par
Gardons-nous non seulement de tromper les hommes, mais encore de les entretenir dans leurs erreurs ; il n’est point de préjugé, de mensonge, d’imposture, qui ne soit pour la race humaine de la plus grande conséquence. Si nous ne devons pas toute vérité aux individus parce que souvent elle leur deviendrait inutile ou nuisible, nous la devons constamment à la société, dont elle est le guide et le flambeau : le mensonge n’a jamais pour elle qu’une utilité passagère. On peut cacher à l’homme la vérité, on peut la lui dissimuler, et même le tromper pour son bien, mais jamais on ne trompe pour son bien la société toute entière, pour laquelle les erreurs générales sont toujours des suites qui se font sentir jusque dans les siècles les plus éloignés\phantomsection
\label{footnote28}\footnote{Voyez la section IV de cet ouvrage, chap. X.}.
\subsection[{Chapitre VIII. De la Paresse. De l’Oisiveté. De l’Ennui et de ses effets. De la Passion du jeu, etc.}]{Chapitre VIII. De la Paresse. De l’Oisiveté. De l’Ennui et de ses effets. De la Passion du jeu, etc.}
\noindent Le travail paraît à tous les hommes une peine dont ils voudraient s’exempter. L’homme laborieux, forcé de gagner son pain à la sueur de son front, porte envie à l’homme riche qu’il voit plongé dans l’oisiveté, tandis que celui-ci est souvent plus à plaindre que lui. Le pauvre travaille pour amasser dans l’espoir de se reposer un jour. Les préjugés de quelques peuples leur font regarder le travail comme abject, comme le partage méprisable des malheureux\footnote{Dans tous les pays chauds, les hommes sont indolents et paresseux, et conséquemment esclaves, indigents, ennuyés, misérables. La maxime des habitants de l’Hindoustan est « qu’il vaut mieux s’asseoir que de marcher, se coucher que de s’asseoir, dormir que de veiller, et mourir que de vivre ». Le gouvernement, encore plus que le climat, rend les hommes indolents et paresseux. Le despotisme ne fait que des esclaves découragés ou des bandits audacieux qui infestent les pays. Telle est la véritable source de la paresse, de la misère et des désordres que l’on voit régner en Espagne, en Italie, en Sicile, c’est-à-dire dans les plus belles contrées de l’Europe.}. En un mot, on remarque dans les hommes, en général, un penchant naturel à la paresse qui, envisagée sous son vrai point de vue, est un vice réel, une disposition nuisible à nous-mêmes et aux autres, que la morale condamne et que notre intérêt propre, ainsi que celui de la société, nous excite à combattre sans relâche. L’apathie, l’indolence, la mollesse, l’incurie, l’indifférence, la lâcheté, la haine du travail, l’ignorance sont des qualités qui nous rendent inutiles et incommodes au corps dont nous sommes membres et qui nous mettent hors d’état de nous procurer le bien-être que nous sommes faits pour désirer. Enfin, si comme on l’a fait voir, l’activité ou l’amour du travail est une vertu réelle, il est évident que l’inaction et la fainéantise sont des vices ou des violations de nos devoirs. Ce n’est que pour travailler à leur bonheur mutuel que les hommes vivent en société.\par
La paresse, la négligence, l’inertie sont des crimes véritables dans les souverains, destinés à veiller sans cesse aux besoins, aux intérêts, au bonheur des nations. L’oisiveté et l’apathie sont des vices honteux dans un père de famille, chargé par la Nature de s’occuper du bien-être de ceux qui lui sont subordonnés. La paresse est un défaut punissable dans les serviteurs, qui se sont engagés à travailler pour leurs maîtres. Tout homme qui reçoit les récompenses et les bienfaits de la société s’est engagé à contribuer, selon ses forces, à l’utilité publique, et n’est plus qu’un voleur dès qu’il manque à ses engagements. L’artisan, l’ouvrier, l’homme du peuple travaillent sous peine de mourir de faim ou de périr pour des crimes que la paresse leur fera commettre tôt ou tard. « Jamais, dit Xénophon, un esprit livré à la paresse ne produit rien de bon » ; un adage très connu nous dit que l’oisiveté est la mère de tous les vices. C’est d’elle, en effet, que l’on voit sortir les fantaisies les plus bizarres, les goûts les plus pervers, les plaisirs les plus insensés, les amusements les plus futiles, les dépenses les plus extravagantes. Ces choses n’ont véritablement pour objet que de suppléer à des occupations honnêtes qui empêcheraient les princes, les riches et les grands de sentir le fardeau de l’oisiveté dont ils sont incessamment accablés. « Il n’y a pas, dit Démocrite, de fardeau plus pesant que celui de la paresse. » En effet, elle est toujours accompagnée de l’ennui, supplice rigoureux dont la Nature se sert pour punir tous ceux qui refusent de s’occuper.\par
L’ennui est cette langueur, cette stagnation mortelle que produit dans l’homme l’absence des sensations capables de l’avertir de son existence d’une façon agréable. Pour échapper à l’ennui, il faut que les organes, soit extérieurs, soit intérieurs de la machine humaine soient mis en action d’une façon qui les exerce sans douleur. Le fer se rouille lorsqu’il n’est pas continuellement frotté ; il en est de même des organes de l’homme : trop de travail les use, et l’absence du travail leur fait perdre la facilité ou l’habitude de leurs fonctions.\par
L’indigent travaille au corps pour subsister ; dès qu’il cesse de travailler de ses membres, il travaille de l’esprit ou de la pensée, et comme pour l’ordinaire cet esprit n’est point cultivé, son désœuvrement le conduit au mal : il ne voit que le crime qui puisse suppléer au travail du corps que sa paresse lui a fait abandonner. « Tout paresseux, dit Phocylide, a des mains prêtes à voler\footnote{Phocylide, {\itshape Catin.}, vers 144. « Le travail, dit-il plus loin, augmente la vertu. Que celui qui n’a point appris à cultiver les arts travaille avec la bêche. » Vers 147.}. »\par
L’homme opulent, que son état dispense du travail du corps, a communément l’esprit ou la pensée dans un mouvement perpétuel. Continuellement tourmenté du besoin de sentir, il cherche dans ses richesses les moyens de varier ses sensations, il a recours à des exercices quelquefois très pénibles. La chasse, la promenade, les spectacles, la bonne chère, les plaisirs des sens, la débauche contribuent à donner à sa machine des secousses diversifiées qui suffisent quelques temps pour le maintenir dans l’activité nécessaire à son bien-être ; mais bientôt les objets qui le remuaient agréablement ont produit sur ses sens tout l’effet dont ils étaient capables, ses organes se fatiguent par la répétition des mêmes sensations, il leur faut de nouvelles façons de sentir, et la Nature, épuisée par l’abus qu’on a fait des plaisirs qu’elle présente, laisse le riche imprudent dans une langueur mortelle. « Personne, disait Bion, n’a plus de peines que celui qui veut n’en prendre aucune. »\par
Le bœuf qui laboure est évidemment un animal plus estimable ou plus utile, que le riche ou le grand livré à l’oisiveté. Ainsi que la vie du corps, la vie sociale consiste dans l’action. Les hommes qui ne font rien pour la société ne sont que des cadavres faits pour infecter les vivants. Vivre, c’est faire du bien à ses semblables, c’est être utile, c’est agir conformément au but de la société. « Amis, j’ai perdu la journée », s’écriait le bon Titus lorsqu’il n’avait eu l’occasion de faire aucun bien à ses sujets. Mais par une étrange fatalité, les princes, les riches et les puissants de la terre, qui devraient animer et vivifier les nations, se plongent communément dans l’indolence, ne sont que des corps morts, incommodes pour ceux qui les entourent ; ou, s’ils agissent et donnent quelques signes de vie, ce n’est que pour troubler la société. Le désœuvrement habituel dans lequel vivent les riches et les grands est visiblement la vraie source des vices dont ils sont infectés et qu’ils communiquent aux autres. Exciter tous les citoyens au travail, les occuper utilement, flétrir l’oisiveté, devrait être un des premiers soins de tout bon gouvernement.\par
La curiosité si mobile et toujours insatiable que l’on voit régner dans les sociétés opulentes, n’est qu’un besoin continuel d’éprouver des sensations nouvelles capables de rendre quelques instants de vie à des machines engourdies. Ce besoin devient si impérieux que l’on brave des dangers réels, des incommodités sans nombre, pour le satisfaire ; c’est lui qui pousse en foule aux spectacles et aux nouveautés de toute espèce : chacun espère d’y trouver quelque soulagement momentané à sa langueur habituelle. Mais des âmes vides et des esprits incapables de se suffire rencontrent en tout lieu cet ennui dont ils sont obstinément poursuivis. On le retrouve dans les amusements mêmes, dans des visites périodiques, dans les cercles brillants, dans les parties, dans ces repas, ces soupers et ces fêtes où l’on comptait goûter les plaisirs les plus piquants.\par
Ce n’est qu’en lui-même que l’homme peut trouver un asile assuré contre l’ennui. Pour prévenir les effets de cette stagnation fatale, l’éducation devrait inspirer dès l’enfance aux personnes destinées à jouir sans travail de l’aisance ou de l’opulence, le goût de l’étude, du travail d’esprit, de la science, de la réflexion. En exerçant leurs facultés intellectuelles, on leur fournirait un moyen de s’occuper agréablement, de varier leurs jouissances, de s’ouvrir une source inépuisable de plaisirs utiles pour eux-mêmes et pour la société, qui les rendraient heureux et qui pourraient leur attirer de la considération. Enfin, on leur ferait contracter l’habitude du travail de la tête, à l’aide duquel ils sauraient un jour se soustraire à la langueur qui désole l’opulence épaisse, la grandeur ignorante et la mollesse incapable d’agir.\par
En habituant de bonne heure la jeunesse à la réflexion, à la lecture, à la recherche de la vérité, on lui procure une façon d’employer le temps agréable pour elle-même et profitable pour la société. L’homme, ainsi, s’accoutume à vivre sans peine avec lui-même et se rend utile aux autres. Ses occupations mentales, quand il a le bonheur de s’y attacher, remplissent ses moments, détournent son esprit des futilités, des vanités puériles, des dépenses ruineuses et surtout des plaisirs déshonnêtes ou des amusements criminels, ressources malheureuses que les hommes désœuvrés trouvent contre l’ennui qui les persécute. Tout le monde se plaint de la brièveté du temps et de la courte durée de la vie, tandis que presque tout le monde prodigue ce temps que l’on dit si précieux ; les hommes pour la plupart meurent sans avoir su jouir véritablement de rien. Le repos ne peut être doux que pour celui qui travaille. Le plaisir n’est senti que par ceux qui n’en ont point abusé\phantomsection
\label{footnote29}\footnote{« Voluptates commendat rarior usus. » Juvénal, {\itshape Satires}, XI, vers 208.}. Les amusements les plus vifs deviennent insipides pour l’imprudent qui s’y est inconsidérément livré. On sort à regret d’un monde où l’on a perdu son temps à courir vers un bien-être que l’on a jamais pu fixer. L’art d’employer le temps est ignoré du plus grand nombre de ceux qui se plaignent de sa rapidité. Une mort toujours redoutée termine une vie dont ils n’ont su tirer aucun parti pour leur propre bonheur.\par
L’ignorance est un mal, parce qu’elle laisse l’homme dans une sorte d’enfance, dans une inexpérience honteuse, dans une stupidité qui le rend inutile à lui-même et de peu de ressources pour les autres. Un homme dont l’esprit est sans culture n’a d’autre moyen de se distinguer dans le monde que par son faste, sa parure, son luxe, sa fatuité. Il ne sait jamais comment employer son temps ; il porte de cercle en cercle ses ennuis, son ineptie, sa présence incommode. Toujours à charge à lui-même, il le devient aux autres. Sa conversation stérile ne roule que sur des minuties indignes d’occuper un être raisonnable. Caton disait avec raison que « les fainéants sont les ennemis jurés des personnes occupées » ; ce sont les vrais fléaux de la société : toujours malheureux eux-mêmes, ils tourmentent sans relâche les autres. Le temps, si précieux et toujours si court pour les personnes qui savent l’employer utilement, devient d’une longueur insupportable pour l’ignorant désœuvré. Il le prodigue indignement à des riens, à des extravagances, à des discours frivoles, à des occupations souvent plus funestes que l’oisiveté\phantomsection
\label{footnote30}\footnote{Le célèbre Locke, étant un jour chez le comte de Shaftesbury, trouva ce Lord et ses amis fortement occupés à jouer. Notre philosophe, ennuyé d’avoir été longtemps le spectateur muet de ce stérile amusement, tira brusquement ses tablettes et se mit à écrire d’un air très attentif. Un des joueurs s’en étant aperçu, le pria de communiquer à la compagnie les bonnes idées qu’il venait de consigner sur ses tablettes ; sur quoi Locke, s’adressant à tous, répondit : « Messieurs, voulant profiter des lumières que j’ai droit d’attendre de personnes de votre mérite, je me suis mis à écrire votre conversation depuis deux heures. » Cette réponse fit rougir les joueurs, qui laissèrent là les cartes pour s’amuser d’une manière plus convenable à des gens d’esprit. — « Nous devons, dit Sénèque, accorder quelquefois du relâche à notre esprit et lui rendre des forces par des amusements ; mais ces amusements mêmes doivent être des occupations utiles. »}.\par
Le jeu, fait pour délasser par intervalles l’esprit, devient pour le fainéant une occupation si sérieuse que souvent il s’expose à la perte totale de sa fortune. Son âme engourdie a besoin de secousses vigoureuses et réitérées ; elle ne les trouve que dans un amusement terrible, durant lequel elle est continuellement ballottée entre l’espérance de s’enrichir et la crainte de la misère.\par
C’est évidemment l’ignorance et l’incapacité de s’occuper convenablement qui font naître et perpétuent la passion du jeu, de laquelle on voit si souvent résulter les effets les plus déplorables. Un père de famille, pour donner quelque activité à son esprit, risque sur une carte ou sur un coup de dé son aisance, sa fortune, celle de sa femme et de ses enfants ; esclave une fois de cette passion détestable, accoutumé aux mouvements vifs et fréquents que produisent l’intérêt, l’incertitude, les alternatives continuelles de la terreur et de la joie, le joueur est ordinairement un furieux que rien ne peut convertir, que la perte de tout son bien.\par
D’après les conventions des joueurs entre eux, l’on appelle dans le monde dettes d’honneur celles que le jeu fait contracter. Suivant les principes d’une morale inventée par la corruption, les dettes de cette nature doivent être acquittées préférablement à toutes les autres. Un homme est déshonoré s’il manque à payer ce qu’il a perdu au jeu sur sa parole, tandis qu’il n’est aucunement puni ou méprisé lorsqu’il néglige ou refuse de payer des marchands, des artisans, des ouvriers indigents, dont sa mauvaise foi ou sa négligence plongent souvent les familles dans la misère la plus profonde !\par
Ce n’est pas encore assez des périls inhérents au jeu lui-même : cette passion cruelle expose à beaucoup d’autres. Ceux que le jeu favorise montrent de la sérénité, ceux contre lesquels la fortune se déclare sont en proie au plus sombre chagrin et quelquefois éprouvent les fureurs convulsives des frénétiques les plus dangereux. De là ces querelles fréquentes que l’on voit s’élever entre des hommes qui, voulant d’abord tuer le temps ou s’amuser, finissent quelquefois par s’égorger.\par
Sans produire toujours des effets si cruels, le jeu doit être blâmé dès qu’il intéresse l’avarice et la cupidité. Est-il rien de moins sociable que des concitoyens, des hommes qui se donnent pour amis, qui se réunissent pour s’amuser et qui font tous leurs efforts pour s’arracher une partie de leur fortune ? Jamais le jeu ne devrait aller jusqu’à chagriner celui que le sort n’a point favorisé. Le gros jeu suppose toujours des âmes bassement intéressées qui désirent de se ruiner et de s’affliger réciproquement.\par
C’est encore au désœuvrement que l’on doit attribuer tant d’extravagances et de crimes qui finissent par troubler le repos et le bonheur des familles. C’est lui qui multiplie la débauche, les galanteries, les dérèglements, les adultères : tant de femmes ne s’écartent du chemin de la vertu que parce qu’elles ne savent aucunement s’occuper des objets les plus intéressants pour elles.\par
Tels sont les effets terribles que produisent à tout moment l’oisiveté et l’ennui, qui toujours marche à sa suite. C’est à cet ennui que l’on doit attribuer presque tous les vices, les folles dépenses, les travers des grands, des riches, des princes mêmes, qui ne connaissent d’autre occupation que les plaisirs, et qui après les avoir épuisés de bonne heure, passent toute la vie dans une langueur continue en attendant que des plaisirs nouveaux viennent rendre quelque activité à leurs âmes endormies.\par
Tout fainéant est un membre inutile de la société ; il ne tarde pas communément à devenir aussi dangereux pour elle qu’incommode pour lui-même\phantomsection
\label{footnote31}\footnote{Par les lois de Solon, il était permis de dénoncer tout citoyen qui n’avait aucune occupation. — Chez les gymnosophistes, on ne donnait point à manger aux jeunes gens sans qu’ils n’eussent rendu compte de ce qu’ils avaient fait pendant la journée.}. C’est en occupant l’homme du peuple sans l’accabler d’un travail trop pénible, qu’on lui rendra son état agréable et qu’on le détournera du vice et du crime. Les malfaiteurs et les scélérats ne sont si communs sous de mauvais gouvernements, que parce que les hommes découragés par la tyrannie préfèrent l’oisiveté à une vie laborieuse. Alors, le crime devient pour eux l’unique moyen de subsister.\par
L’oisiveté d’un souverain est un crime aussi grand que la tyrannie la plus avérée. Les sujets d’un monarque fainéant ne peuvent, par les travaux les plus rudes, fournir aux besoins infinis, aux fantaisies immenses, aux vices qui lui sont nécessaires pour remplir son temps.\par
En accoutumant de bonne heure les princes, les grands et les riches à s’occuper, on les garantira des folies et des excès auxquels trop souvent le désœuvrement et l’ignorance les livrent. La paresse et les vices des grands sont imités par le peuple. Celui-ci, pour satisfaire les passions que l’exemple a fait germer en lui, se livre en aveugle au mal et brave insolemment les lois et les supplices.\par
Indépendamment de l’oisiveté, dont nous venons encore de décrire les funestes effets, il existe encore une paresse de tempérament qui, par l’engourdissement et l’inertie qu’elle produit dans les cœurs, devient aussi dangereuse que l’inaction et l’incapacité de s’occuper : on pourrait la comparer à une véritable léthargie. Tandis que les autres passions ont souvent les emportements du délire, celle-ci semble endormir les facultés ; celui qui s’en trouve atteint devient indifférent, même sur les objets qui devraient intéresser tout être raisonnable. Les paresseux de cette espèce, loin de rougir d’une disposition si peu sociable, s’en applaudissent, y trouvent un charme secret, et quelquefois s’en vantent comme de la possession d’un très grand bien, comme d’une vraie philosophie.\par
« C’est se tromper, dit un moraliste célèbre, de croire qu’il n’y a que les violentes passions comme l’ambition et l’amour qui puissent triompher des autres. La paresse, toute languissante qu’elle est, ne laisse pas d’en être souvent la maîtresse. Elle usurpe sur tous les desseins et sur toutes les actions de la vie : elle y consume insensiblement les passions et les vertus\footnote{Voyez les {\itshape Réflexions morales} du duc de la Rochefoucault.}. »\par
Il dit ailleurs que « de toutes les passions, celle qui nous est la plus inconnue à nous-mêmes, c’est la paresse ; elle est la plus ardente et la plus maligne de toutes, quoique sa force soit insensible et que les dommages qu’elle cause soient très cachés. Si nous considérons attentivement son pouvoir, nous verrons qu’elle se rend en toute rencontre maîtresse de nos sentiments, de nos intérêts et de nos plaisirs. C’est la Rémore qui a la force d’arrêter les vaisseaux… Pour donner enfin la véritable idée de cette passion, il faut dire que la paresse est comme la béatitude de l’âme, qui la console de toutes ses pertes et lui tient lieu de tous les biens… De tous les défauts, celui dont nous demeurons le plus aisément d’accord, c’est de la paresse. Nous nous persuadons qu’elle tient toutes les vertus paisibles et que, sans détruire entièrement les autres, elle en suspend seulement les fonctions. » Bien plus, ceux qui sont enchantés par cette sorte de paresse s’en font un mérite, une vertu. Mais cette apathie du cœur, cette indifférence pour tout, cette privation de toute sensibilité, ce détachement de l’estime et de la gloire ne peuvent être aucunement regardés comme des vertus morales ou sociales. Un être vraiment sociable doit s’intéresser au bonheur et aux malheurs des hommes. Il doit partager leurs plaisirs et leurs peines. Il doit s’attacher fortement à la justice, il doit être toujours prêt à rendre à ses semblables les services et les soins dont il est capable. Le paresseux est un poids inutile à la terre, il est mort pour la société. Il ne peut être ni bon prince, ni bon père de famille, ni bon ami, ni bon citoyen. Un homme de ce caractère, concentré en lui-même, n’existe que pour lui seul. Une vie purement contemplative, la paresse philosophique des épicuriens, l’apathie des stoïciens, exaltées par tant de moralistes, sont des vices réels : tout homme qui vit avec des hommes est fait pour être utile.\par
Solon voulait que tout citoyen qui refusait de prendre part aux factions de la république, en fût retranché comme un membre incommode. Si cette loi paraît trop rigoureuse, il serait au moins à désirer que tout citoyen indifférent aux maux de son pays ou qui ne contribue en rien à sa félicité, fût puni par le mépris\footnote{« La paresse et l’indolence, dit Démosthène, dans la vie domestique comme dans la vie civile, ne se rendent pas d’abord sensibles par chacune des choses que l’on a négligées, mais elles se font enfin sentir par leur somme totale. » Voyez Démosthène, {\itshape Philippiques}, IV.}.
\subsection[{Chapitre IX. De la Dissolution des Mœurs. De la Débauche. De l’Amour. Des Plaisirs déshonnêtes.}]{Chapitre IX. De la Dissolution des Mœurs. De la Débauche. De l’Amour. Des Plaisirs déshonnêtes.}
\noindent L’homme social, comme on l’a souvent répété, doit, pour son propre intérêt et celui de ses associés, mettre un frein à ses passions naturelles et résister aux impulsions déréglées de son tempérament. Rien de plus naturel à l’homme que d’aimer le plaisir, mais un être guidé par la raison fuit les plaisirs, qu’il sait pouvoir se changer en peines, craint de se nuire et s’abstient de ce qui peut lui faire perdre l’estime de ses semblables.\par
Cela posé, l’on doit mettre au nombre des vices toutes les dispositions qui, soit immédiatement, soit par leurs conséquences nécessaires, peuvent causer du dommage à celui qui s’y livre ou produire quelque trouble dans la société. Tant d’hommes sont entraînés par leurs penchants les plus pervers parce qu’ils ne raisonnent point leurs actions. Le vice est brusque, inconsidéré, au lieu que la raison, ainsi que l’équité, tient toujours la balance. Les hommes ne sont vicieux que parce qu’ils ne pensent qu’au présent.\par
L’amour, cette passion si follement exaltée par les poètes et si décriée par les sages, est un sentiment inhérent à la nature de l’homme ; il est l’effet de ses plus pressants besoins, mais s’il n’est contenu dans de justes bornes, tout nous prouve qu’il est la source des plus affreux ravages. C’est aux plaisirs de l’amour que la Nature attache la conservation de notre espèce, et par conséquent de la société. Ainsi que l’homme, les animaux sont sensibles à l’amour et cherchent ses plaisirs avec ardeur, mais la tempérance et la prudence nous apprennent et nous habituent à résister aux sollicitations d’un tempérament impétueux ou d’une nature toujours aveugle quand elle n’est pas guidée par la raison.\par
En parlant de la tempérance, nous avons suffisamment prouvé l’importance de cette vertu dans la conduite de la vie : sans elle, l’homme, continuellement emporté par l’attrait du plaisir, deviendrait à tout moment l’ennemi de lui-même et porterait le désordre dans la société. Nous avons fait voir pareillement les avantages de la pudeur, cette gardienne respectable des mœurs, et nous avons prouvé qu’en voilant aux regards les objets capables d’exciter les passions destructives, elle opposait d’heureux obstacles à la fougue d’une imagination souvent indomptable quand elle est bien allumée.\par
L’amour est pour l’ordinaire un enfant nourri dans la mollesse et l’oisiveté. Nous avons déjà fait entrevoir que c’est elle surtout qui conduit les hommes à la débauche et qui leur en fait une habitude, un besoin : elle remplit le vide immense que le désœuvrement laisse communément dans la tête des princes, des riches, des grands, et particulièrement des femmes du grand monde, que leur état semble condamner à la mollesse, à l’inertie. Voilà, comme on a vu, la vraie source de la galanterie, fruit d’ailleurs nécessaire de la communication trop fréquente des deux sexes. C’est, dans des hommes désœuvrés, la volonté de plaire à toutes les femmes sans s’attacher sincèrement à aucunes. Quelque innocent que paraisse ce commerce frauduleux qui ne semble fondé que sur la politesse, la déférence et les égards que l’on doit au beau sexe, il ne laisse pas de devenir très dangereux par ses effets : il amollit les âmes des hommes\footnote{César nous apprend que les anciens Germains faisaient le plus grand cas de la chasteté, comme propre à fortifier les hommes, et notaient d’infamie ceux qui avant l’âge de vingt ans avaient fréquenté les femmes. — Suivant le père Laffiteau, les jeunes sauvages n’ont la liberté d’user des droits du mariage qu’un an après s’y être engagés. Voyez {\itshape Les Moeurs des Sauvages} par le père Laffiteau, et César, {\itshape La Guerre des Gaules}, livre  chap. 21, non procul ab init.} et dispose les femmes à se familiariser avec des idées qui peuvent avoir pour elles les conséquences les plus funestes. La faiblesse n’est en sûreté qu’en évitant le danger ; il est bien difficile qu’une femme perpétuellement exposée aux séductions d’un grand nombre d’adorateurs ait toujours la force d’y résister. Rien de plus important que de prévoir et prévenir les périls dont la vertu, dans un monde pervers, se trouve continuellement environnée.\par
Si, comme on l’a démontré ci-devant, l’homme isolé, c’est-à-dire considéré relativement à lui-même, est obligé de résister aux impulsions d’une nature aveugle et brute et de lui opposer les lois d’une nature plus expérimentée, il suit que l’homme, dans quelque position qu’il se trouve, doit, pour la conservation de son être, combattre et réprimer des pensées et des désirs qui le porteraient souvent à faire de ses forces un abus toujours funeste à lui-même. D’où l’on voit que les plaisirs qui ont rapport à l’amour sont interdits à l’homme ou à la femme isolés {\itshape ;} l’intérêt de leur conservation et de leur santé exige qu’ils respectent leur propre corps et qu’ils craignent de contracter des habitudes et des besoins qu’ils ne pourraient contenter sans se causer par la suite un dommage irréparable. L’expérience nous montre en effet que l’habitude d’écouter un tempérament trop ardent est, de toutes les habitudes, la plus contraire à la conservation de l’homme et la plus difficile à déraciner. D’où il suit que la retenue, la tempérance, la pureté devraient accompagner l’homme, même au fond d’un désert inaccessible au reste des humains.\par
Cette obligation devient encore plus forte dans la vie sociale, où les actions de l’homme non seulement influent sur lui-même mais encore sont capables d’influer sur les autres. La chasteté, la retenue, la pudeur sont des qualités respectées dans toutes les nations civilisées ; l’impudicité, la dissolution, l’impudence, au contraire, y sont généralement regardées comme honteuses et méprisables. Ces opinions ne seraient-elles fondées que sur des préjugés ou sur des conventions arbitraires ? Non ; elles ont pour base l’expérience, qui prouve très constamment que tout homme livré par habitude à la débauche est communément un insensé qui se perd et qui n’est nullement disposé à s’occuper utilement pour les autres. Le débauché, tourmenté d’une passion exclusive, irrite continuellement son imagination lascive et ne songe qu’aux moyens de satisfaire les besoins qu’elle lui crée. Une fille qui a violé les règles de la pudeur, dominée par son tempérament, hait le travail, est ennemie de toute réflexion, se moque de la prudence, n’est nullement propre à devenir une mère de famille attentive et laborieuse, ne songe qu’au plaisir ou, quand par ses dérèglements il est devenu moins attrayant pour elle, elle ne pense qu’au profit qu’elle peut tirer du trafic de ses charmes.\par
Pour connaître les sentiments que la débauche, le goût habituel des plaisirs déshonnêtes et de la crapule doivent exciter dans les âmes vertueuses, que l’on examine les suites de ces dispositions abrutissantes dans ceux que le sort destine à gouverner les empires : elles éteignent visiblement en eux toute activité, elles les endorment dans une mollesse continue qui, souvent plus que la cruauté, conduit les États à la ruine. Quels soins les peuples d’Asie peuvent-ils attendre de leurs sultans voluptueux perpétuellement occupés des pâles plaisirs de leurs sérails, où ils sont eux-mêmes gouvernés par les caprices et les menées de quelques favorites ou de quelques eunuques ? Sous un Néron, un Héliogabale, Rome ne fut qu’un lieu de perdition où d’infâmes courtisanes du sein de la débauche décidaient du sort des citoyens, dissipaient les trésors de l’État, distribuaient les honneurs et les grâces à des hommes à qui la corruption tenait lieu de mérite, de talents et de vertu. Une nation est perdue\phantomsection
\label{footnote32}\footnote{« Desinit esse remedio locus ubi quæ fuerant vitia mores sunt. » Sénèque, {\itshape Épître XXXIX}, in fine.} lorsque la dissolution des mœurs, autorisée par l’exemple des chefs et récompensée par eux, devient universelle ; alors, le vice effronté ne cherche plus à se couvrir des ombres du mystère et la débauche infecte toutes les classes de la société. Peu à peu, la décence, devenue ridicule, est forcée de rougir à son tour.\par
L’horreur et le mépris que l’on doit avoir pour la débauche sont donc très justement fondés sur ses effets naturels : les idées que l’on a de ses malheureuses victimes ne sont donc pas l’effet d’un préjugé. Dans les sociétés où la vertu et l’honneur des femmes sont principalement attachés au soin qu’elles prennent de conserver la chasteté, où l’éducation a pour objet de les prémunir soit contre la faiblesse de leurs cœurs, soit contre la force de leur tempérament, on peut naturellement supposer qu’une fille qui a franchi les barrières de la pudeur est perdue sans ressource, n’est plus propre à rien, ne peut être désormais regardée que comme l’instrument vénal de la brutalité publique. Conséquemment, une prostituée est exclue des compagnies décentes, elle est un objet d’horreur pour les femmes honnêtes, elle s’attire peu d’égards même de ceux que le goût de la débauche amène auprès d’elle ; bannie, pour ainsi dire, de la société, elle est forcée de s’étourdir par la dissipation, l’intempérance, les folles dépenses, la vanité. Incapable de réfléchir, dépourvue de toute prévoyance, elle vit à la journée, ne songe aucunement au lendemain, périt promptement de ses débauches ou traîne douloureusement jusqu’au tombeau une vieillesse indigente, languissante et méprisée.\par
C’est pourtant en faveur de ces objets méprisables que l’on voit tous les jours tant de riches et de grands abandonner des femmes aimables et vertueuses, se ruiner de gaieté de cœur, ne laisser que des dettes à la postérité. Mais la vertu n’a plus de droits sur les âmes corrompues par la débauche ; les hommes dépravés par elle méconnaissent les charmes de la pudeur, de la décence : il leur faut désormais de l’impudence. Le vice effronté, les propos obscènes et grossiers les ont dégoûtés pour toujours de toute conversation honnête et d’une conduite réservée. Voilà pourquoi des maris libertins préféreront souvent une courtisane sans agréments et du plus mauvais ton à des épouses pleines de charmes et de vertus qui ne leur procureraient pas les mêmes plaisirs, qu’un goût pervers leur fait trouver dans le commerce des prostituées qu’ils ne peuvent au fond s’empêcher de mépriser et qu’ils abandonnent à leur malheureux sort quand ils en sont ennuyés.\par
Telles sont les suites ordinaires de l’amour déréglé ; c’est à cet avilissement déplorable que des filles trop faibles sont conduites par d’infâmes séducteurs que les lois devraient punir. Mais dans la plupart des nations, la séduction n’est point regardée comme un crime. Ceux qui s’en rendent coupables s’en applaudissent comme d’une conquête et font trophée des victoires qu’ils remportent sur un sexe fragile et crédule que sa faiblesse semble autoriser à tromper de la façon la plus cruelle. Quelle doit être la dépravation des idées dans des nations où des actions pareilles n’attirent ni châtiments ni déshonneur ! Quelles âmes doivent avoir ces monstres de luxure dont les attentats portent la désolation et la honte durable dans des familles honnêtes ? Est-il une plus grande cruauté que celle de ces débauchés qui, pour satisfaire un désir passager, vouent pour la vie les victimes qu’ils ont séduites à l’opprobre, aux larmes, à la misère ? Mais la débauche devenue habituelle anéantit tout sentiment dans le cœur, toute réflexion dans l’esprit ; c’est par de nouveaux excès que le libertin étouffe les remords que les premiers crimes pourraient faire naître en lui. D’ailleurs, assez aveugle pour ne pas voir le mal qu’il se fait à lui-même, comment se reprocherait-il le tort qu’il fait aux autres ?\par
Ceux qui regardent la débauche et la dissolution des mœurs comme des objets sur lesquels le gouvernement doit fermer les yeux en ont-ils sérieusement envisagé les conséquences ? Ne voit-on pas à tout moment des familles ruinées par des pères libertins qui ne transmettent à leurs enfants que leurs goûts dépravés avec l’impossibilité de les satisfaire ? Des exemples trop fréquents ne prouvent-ils pas à quels excès d’aveuglement et de délire des attachements honteux peuvent souvent porter ? Il n’est guère de fortune capable de résister aux séductions de ces systèmes, à la voracité de ces harpies affamées qui se sont une fois emparé de l’esprit d’un débauché. Rien ne peut contenter les désirs effrénés, les caprices bizarres, la vanité impertinente de ces femmes qui ne connaissent aucunes mesures. La ruine complète de leurs amants met seule un terme à leurs demandes ; alors, une dupe ruinée est obligée de faire place à une dupe nouvelle qui à son tour sera dépouillée, car telle est la tendresse et la constance que des amants insensés peuvent attendre de ces êtres abjects et mercenaires auxquels ils ont eu la folie de s’attacher.\par
Si le libertinage produit journellement tant d’effets déplorables même sur les riches et les personnes les plus aisées, quels ravages ne produit-il pas quand il gagne les gens d’une fortune bornée ? Il abrutit l’homme de lettres, dont il endort le génie. Il détourne le marchand de son commerce et le force bientôt à devenir fripon. Il fait sortir l’artiste de son atelier, il dégoûte l’artisan du travail nécessaire à sa subsistance journalière. Enfin, après avoir dérangé l’homme opulent, la débauche conduit l’homme du peuple à l’hôpital ou au gibet. On ne voit guère de malfaiteurs à la perte desquels des femmes de mauvaise vie n’aient grandement contribué. Un misérable, le plus souvent, ne vole, n’assassine, ne commet des forfaits que pour contenter la vanité ou les besoins d’une maîtresse qui le trahira peut-être et le livrera tôt ou tard au supplice.\par
C’est encore au dérèglement des mœurs que l’on doit le plus souvent imputer ces disputes fréquentes et ces combats sanglants qui mettent tant de jeunes étourdis au tombeau. Combien d’imprudents fougueux, par une sotte jalousie, ont la cruelle extravagance de hasarder leur propre vie pour disputer les faveurs banales et méprisables d’une vile prostituée ? Ne faut-il pas avoir des idées bien étranges de l’honneur pour le faire consister dans la possession de ces femmes dissolues qui sont au premier occupant ? Mais c’est le propre de l’amour, ou plutôt de la débauche crapuleuse, d’éteindre toute réflexion sensée, toute pensée raisonnable.\par
Indépendamment du juste mépris que le libertinage attire à ceux qui s’y livrent, indépendamment de l’épuisement qu’il cause, la Nature a pris soin de châtier de la façon la plus directe les inconsidérés que les idées de décence ou de raison ne peuvent arrêter dans leurs penchants déréglés. La jeunesse devrait frémir à la vue des contagions affreuses dont la volupté la menace. De quelle horreur les débauchés ne devraient-ils pas être saisis en songeant que les fruits de leurs désordres peuvent encore infecter la postérité la plus reculée. Mais ces considérations n’ont point de force sur l’esprit de ces êtres abrutis qui, même aux dépens de leur vie, cherchent à satisfaire leurs honteuses passions. Le vice est un tyran qui donne à ses esclaves un fatal courage, capable de leur faire affronter les maladies et la mort.\par
Tout dans la société semble exciter et fomenter, dans les âmes des riches et des grands surtout, le goût funeste du vice et de la volupté. L’éducation publique, des discours obscènes, des spectacles peu chastes\footnote{Les gouvernements, dans quelques nations, semblent autoriser la corruption publique par des spectacles très licencieux. Le théâtre anglais est évidemment une école de prostitution. Beaucoup de pièces du théâtre français, telles que {\itshape La Fille Capitaine, La Femme Juge et Partie, George Dandin, L’École des Femmes}, etc., ne donnent assurément pas à la jeunesse des leçons utiles aux mœurs. {\itshape L’opéra}, dans quelques pays, paraît n’être imaginé que pour allumer dans les cœurs le goût de la débauche par des chants, des maximes et des danses lubriques. Les {\itshape parades} font perdre le temps du peuple et corrompent ses mœurs. Les pièces les moins licencieuses offrent toujours aux yeux et à l’imagination des jeunes gens, des objets propres à irriter les passions.}, des romans séducteurs, des exemples pervers contribuent chaque jour à semer dans tous les cœurs les germes de la débauche ; une corruption contagieuse s’y insinue pour ainsi dire par tous les pores, et souvent les esprits sont gâtés avant même que la Nature ait donné aux organes du corps une consistance suffisante. De là cette vieillesse précoce que l’on remarque surtout dans les grands et les habitants corrompus des cours, dont la race chétive et faible annonce évidemment les dérèglements des parents. Le débauché non seulement se nuit à lui-même mais encore il substitue sa faiblesse et ses vices à ses malheureux descendants.\par
Nous ne parlerons point ici de ces goûts bizarres et pervers, contraires aux vues de la Nature, dont on voit quelquefois des nations entières infectées. Nous dirons seulement que ces goûts inconcevables paraissent être les effets d’une imagination dépravée qui, pour ranimer des sens usés par les plaisirs ordinaires, en inventent de nouveaux propres à réveiller pour un temps des malheureux que leur anéantissement ou leur faiblesse réduit au désespoir. C’est ainsi que la Nature se venge de ceux qui abusent de la volupté : elle les réduit à chercher le plaisir par des voies qui mettent l’homme au-dessous de la brute. Les débauches ingénieuses des Grecs, des Romains, des orientaux\footnote{Les relations de l’Orient nous disent que, par un effet de la polygamie, les mahométans riches, les Persans, les Mongols, les Chinois sont communément épuisés à l’âge de trente ans ou totalement insensibles aux plaisirs naturels ; voilà sans doute la cause des goûts honteux et dépravés qui règnent en Asie.} annoncent dans ces peuples une imagination troublée qui ne fait plus qu’inventer pour satisfaire des malades dont l’appétit est déréglé.\par
On nous demandera peut-être quels remèdes on peut apporter à la dissolution des mœurs, qui semble tellement enracinée dans quelques contrées que l’on serait tenté de croire qu’il est impossible de la faire disparaître. Nous dirons qu’une éducation plus vigilante empêcherait la jeunesse de contracter des habitudes capables d’influer sur le bien-être de toute sa vie ; nous dirons que des parents plus réglés dans leur conduite formeraient infailliblement des enfants moins corrompus ; nous dirons que des souverains vertueux influeraient par leurs exemples sur leurs sujets. En fermant aux vices le chemin de la faveur, des honneurs, des dignités, des récompenses, un prince parviendrait bientôt à diminuer au moins la corruption publique et scandaleuse dont la cour est le vrai foyer. L’exemple des grands, toujours fidèlement copié par les petits, ramènerait en peu de temps la décence et la pudeur depuis longtemps bannies du sein des nations opulentes. Celles-ci n’ont communément sur les pauvres que le funeste avantage d’avoir plus de mollesse et de vices, et beaucoup moins de force et de vertus.\par
En parlant du devoir des époux, nous ferons voir les inconvénients, non moins terribles que fréquents, qui résultent pour les familles et pour la société de l’infidélité conjugale, de la coquetterie et de ces galanteries que, dans quelques nations apprivoisées avec la corruption, l’on a la témérité de regarder comme des bagatelles, des amusements, des jeux d’esprit.\par
Si la raison condamne la débauche, elle proscrit nécessairement tout ce qui peut y provoquer. Ainsi, elle interdit les discours licencieux, les lectures dangereuses, les habillements lascifs, les regards déshonnêtes ; elle ordonne de détourner l’imagination de ces pensées lubriques qui pourraient peu à peu porter à des actions criminelles : celles-ci, réitérées, forment des habitudes permanentes capables de résister à tous les conseils de la raison. « Il ne faut pas qu’un homme sage contienne ses mains, disait Isocrate, mais encore il faut qu’il contienne ses yeux. »\par
Les plaisirs de l’amour étant les plus vifs de ceux que la machine humaine puisse éprouver, sont de nature à pouvoir être difficilement remplacés. Par la même raison, l’expérience nous montre qu’ils sont les plus destructeurs pour l’homme ; ses organes ne peuvent essuyer sans un détriment notable les mouvements convulsifs que ces plaisirs y causent. Voilà pourquoi, emporté par ses habitudes dangereuses, le débauché en est communément l’esclave jusqu’au tombeau ; au défaut même de la faculté de satisfaire ses besoins invétérés, son imagination perpétuellement au travail ne lui laisse aucun repos. Rien de plus digne de pitié que la vieillesse infirme et méprisable de ces hommes dont la vie fut consacrée à la volupté.
\subsection[{Chapitre X. De l’Intempérance}]{Chapitre X. De l’Intempérance}
\noindent Tout ce qui nuit à la santé du corps, tout ce qui trouble les facultés intellectuelles ou la raison de l’homme, tout ce qui le rend nuisible soit à lui-même, soit aux autres, doit être réputé vicieux et criminel et ne peut être approuvé par la saine morale. Si la tempérance est une vertu, l’intempérance est un vice que l’on peut définir par l’habitude de se livrer aux appétits déréglés du sens du goût. Tous les excès de la bouche, la gourmandise, l’ivrognerie doivent être regardés comme des dispositions dangereuses pour nous-mêmes et ceux avec qui nous vivons. C’est à la médecine qu’il appartient de faire sentir les dangers auxquels l’intempérance expose le corps. D’accord avec la morale, elle prouve que le gourmand, esclave d’une passion avilissante, est sujet à des maladies cruelles et fréquentes, végète dans un état de langueur et trouve communément une mort prématurée dans des plaisirs auxquels son estomac ne peut suffire.\par
La morale, de son côté, ne voit dans l’homme intempérant qu’un malheureux dont l’esprit, absorbé par une passion brutale, ne s’occupe que des moyens de la contenter. Dans les pays où le luxe a fixé sa demeure, les riches et les grands, dont tous les organes se trouvent communément émoussés par l’abus qu’ils en ont fait, sont réduits à chercher dans des aliments précoces, rares, dispendieux, des moyens de ranimer un appétit languissant. Leur pays ne leur fournissant plus rien d’assez piquant, vous les voyez se faire une occupation sérieuse d’imaginer de nouvelles combinaisons capables d’irriter leurs palais endormis. Ils mettent à contribution les mers et les contrées éloignées pour réveiller leurs sens usés. À cet affaiblissement physique de la machine se joint encore une sotte vanité qui se fait un mérite de présenter à des convives étonnés des productions coûteuses destinées à leur donner une haute idée de l’opulence de celui qui les régale. Celui-ci a la noble ambition de passer pour faire la chère la plus délicate ; il ne rougit pas de partager une gloire qui devrait n’être faite que pour son maître d’hôtel ou son cuisinier.\par
C’est surtout dans les plaisirs de la table et dans la gloire d’offrir à ses convives des mets bien préparés, bien choisis et bien chers, que beaucoup de gens font consister la représentation et la grandeur. Des repas somptueux leur paraissent annoncer du goût, de la générosité, de la noblesse, de la sociabilité. L’homme opulent et l’homme en place jouissent intérieurement des applaudissements qu’ils croient obtenir d’une foule de flatteurs, de gourmands et souvent d’inconnus qu’ils rassemblent au hasard et sans choix pour les rendre témoins de leur prétendue grandeur et de leur félicité.\par
C’est ainsi que les maisons des riches et des grands se changent en hôtelleries ouvertes à tout venant, dont les propriétaires ont la sottise de déranger et leur fortune et leur santé pour des gens qu’ils connaissent à peine et qu’ils ont pourtant la folie de prendre pour des amis. Rien de plus méprisable que ces amis de table que la bonne chère attire uniquement, et que l’on pourrait qualifier avec plus de raison d’amis du cuisinier que d’amis de son maître\footnote{Plutarque qualifie les amis de cette espèce d’{\itshape amis de la marmite}.}. Celui-ci, après avoir dérangé sa fortune, ce qui arrive très fréquemment, est tout surpris de se voir abandonné de ses prétendus amis ; il s’aperçoit trop tard qu’il ne rassemblait chez lui que des gourmands dont la sensibilité n’était que dans l’estomac et qui ne lui savent aucun gré des folles dépenses qu’il a faites pour eux, ou plutôt en faveur de sa sotte vanité.\par
En effet, le prodigue, comme on a vu, n’est point un être bienfaisant : c’est un extravagant souvent dépourvu de sensibilité qui sacrifie sa fortune à l’envie de paraître. Comment un être vraiment sensible ne se reprocherait-il pas les dépenses souvent énormes de ses festins s’il venait à réfléchir qu’elle suffiraient pour procurer le nécessaire à des familles indigentes qui mangent à peine du pain ? Mais des bienfaits de ce genre n’ont pas pour l’homme riche tout l’éclat que demande sa vanité ; il aime mieux représenter et se ruiner sottement que de donner les secours les plus légers aux misérables. Il trouve dans son rang, dans sa place, une obligation de dépenser qui lui fournit des prétextes pour ne jamais soulager les besoins les plus pressants du pauvre.\par
Les dépenses extravagantes des grands et des riches, les déprédations de leurs tables contribuent encore à rendre le sort de l’indigent encore plus fâcheux ; c’est en effet à ces causes que l’on peut attribuer la cherté des provisions, des denrées comestibles, que l’on voit régner dans les contrées où le luxe ne fait que rendre la pauvreté plus malheureuse. Des festins continuels, des ragoûts recherchés, les dégâts des valets consomment et détruisent souvent en un jour, dans une grande ville, autant de vivres qu’il en faudrait pour nourrir pendant un mois les cultivateurs de toute une province.\par
Tels sont pourtant les effets de ce luxe dont bien des gens entreprennent l’apologie ! La réflexion nous le montre comme le destructeur impitoyable du riche, qu’il ruine, et du pauvre, qu’il prive à tout moment du nécessaire. Tout nous prouve que la saine politique, d’accord avec la morale, devrait le proscrire, et ramener les citoyens à la frugalité — non moins utile à la santé, à la fortune des riches et des grands qu’à l’aisance et au bien-être du peuple, auquel les gouvernements, pour l’ordinaire, semblent très peu s’intéresser.\par
C’est sans doute à leur négligence ou à des intérêts futiles et mal entendus que l’on doit attribuer l’ivrognerie dont on voit si communément le bas peuple infecté. Tout prouve les ravages que les excès du vin et une crapule habituelle causent parmi les classes les plus subalternes de la société. Cependant, on ne cherche aucuns moyens d’y remédier ; bien loin de là, dans quelques nations la politique se rend complice de ces désordres, en vue d’un profit sordide ou des droits que le gouvernement lève sur les boissons. L’intempérance du peuple est regardée comme un bien pour l’État, et l’on craindrait une diminution dans les finances si le peuple devenait plus sobre et plus raisonnable\footnote{Dans l’Empire de Russie, le souverain se réserve exclusivement le monopole de l’eau-de-vie et tient un registre exact de ce qu’il en faut tous les ans à chaque famille. Dans toutes les nations européennes, les gouvernements mettent des impôts très forts sur les boissons ; ils ont par conséquent le plus grand intérêt que le peuple s’enivre. Les liqueurs distillées sont la ressource des pauvres, surtout dans les pays où le vin est trop cher.}.\par
La paresse, l’oisiveté, la difficulté de se procurer des aliments convenables déterminent le peuple à l’ivrognerie, et surtout lui font contracter l’habitude des liqueurs fortes, qui le détruisent en peu de temps. Celles-ci lui deviennent nécessaires pour ranimer sa machine d’ailleurs peu nourrie ; elles procurent de plus à son palais des sensations vives mais, le privant habituellement de sa raison, elles finissent tôt ou tard par l’abrutir tout à fait et par le rendre incapable de subsister par son travail.\par
Dans quelques nations les institutions religieuses, obligeant le peuple à demeurer dans l’inaction, semblent trop souvent l’inviter à l’intempérance. Des solennités et des fêtes multipliées, qui condamnent l’artisan et l’homme du peuple à ne point faire usage de ses bras, ne lui laissent dans son désœuvrement d’autre ressource que de s’enivrer. Par là il se prive du profit qu’il a pu faire par son travail et se met souvent hors d’état de donner du pain à ses enfants. D’ailleurs, son ivrognerie l’expose à des rixes fortuites, à des dangers sans nombre ; souvent même elle le conduit à des crimes. En prévenant l’oisiveté, la politique préviendrait bien des désordres qu’elle est continuellement obligée de punir sans pouvoir les faire cesser.\par
Quoique chez quelques nations l’ivrognerie semble bannie de la bonne compagnie, ce vice subsiste dans les provinces et paraît la ressource commune de tous les désœuvrés. Combien d’hommes qui se disent raisonnables ne trouvent d’autre moyen d’employer un temps qui leur pèse, qu’en noyant dans le vin le peu de bon sens dont ils jouissent ? Si les habitants des pays méridionaux montrent plus de sobriété, ceux des pays du nord croient trouver dans la rigueur de leur climat des motifs pressants de s’enivrer habituellement et font trophée de leur honteuse intempérance. Belle gloire, sans doute, que celle qui résulte, pour un être intelligent, de se priver périodiquement de sa raison et de se ravaler souvent au-dessous de la condition des bêtes.\par
L’ivrognerie est évidemment un plaisir de sauvages. Nous voyons ces hordes d’hommes, ou plutôt d’enfants inconsidérés, dont le nouveau monde est peuplé, subjuguées par les liqueurs fortes dont les Européens leur ont procuré la fatale connaissance. C’est à l’usage immodéré de ces funestes breuvages que bien des voyageurs attribuent la destruction presque entière de ces peuples dépourvus de prudence et de raison.\par
Anacharsis prétendait que la vigne produisait trois raisins : le premier de plaisir, le second d’ivrognerie, le troisième de repentir. L’expérience journalière suffit pour nous convaincre que rien n’est plus contraire à l’homme physique, ainsi qu’à l’homme moral, que l’intempérance. En affaiblissant le corps, elle amène à grands pas la vieillesse, les infirmités et la mort. « L’intempérance, dit Démocrite, donne de courtes joies et de longs déplaisirs. » Une vie sensuelle et délicate nous fait contracter une mollesse qui nous rend intuiles et méprisables ; l’excès du vin, en troublant perpétuellement le cerveau, abrutit l’homme qui s’y livre, le rend incapable de travail, l’empêche de penser ou de remplir aucuns de ses devoirs, et souvent le conduit à des crimes propres à lui attirer des châtiments.\par
L’être vraiment raisonnable doit veiller à sa conservation ; l’être vraiment sociable doit conserver son sang-froid et ne jamais troubler ses qualités intellectuelles, de peur d’être entraîné à son insu et malgré lui à des actions qui le dégraderaient et dont, rendu à lui-même, il serait forcé de rougir\footnote{« Hic murus æneus esto nil conscire sibi, nulla pallescere culpa. » Horace, {\itshape Épîtres}, livre I, 1, vers 60-61.}.
\subsection[{Chapitre XI. Des Plaisirs honnêtes et déshonnêtes}]{Chapitre XI. Des Plaisirs honnêtes et déshonnêtes}
\noindent Une morale farouche et répugnante à la nature de l’homme lui fait un crime de tous les plaisirs, mais une morale plus humaine l’invite à la vertu en lui prouvant qu’elle seule peut lui procurer des plaisirs exempts d’amertume et de regrets. La raison nous permet et nous ordonne de jouir des bienfaits de la Nature, de suivre des penchants réglés, de chercher des plaisirs et des amusements qui ne nuisent ni à nous-mêmes ni aux autres. Elle nous conseille d’en user dans la mesure fixée par l’intérêt de chaque homme, ainsi que par le bon ordre ou l’intérêt général de la société.\par
Dans toutes leurs actions les hommes cherchent le plaisir, c’est lui que nos passions ou nos désirs ont pour but. Nous le rencontrons si rarement soit parce que nous le cherchons où il n’est pas, soit parce que nous avons l’imprudence d’en abuser.\par
Nous avons déjà ci-devant\footnote{Section I, chap. IV.} défini le plaisir ; nous en avons distingué deux espèces : nous avons dit que les plaisirs qui agissent immédiatement sur nos organes visibles se nomment {\itshape plaisirs des sens} ou {\itshape plaisirs corporels}, et que ceux qui se font sentir au-dedans de nous-mêmes s’appellent {\itshape plaisirs intellectuels} ou plaisirs de l’esprit et du cœur.\par
C’est surtout contre les plaisirs des sens qu’une foule de moralistes s’est de tous temps élevée ; quelques-uns les ont totalement proscrits. Cependant, ces plaisirs en eux-mêmes n’ont rien de criminel lorsque, vraiment utiles à nous, ils ne peuvent causer aucun dommage à personne. Les plaisirs de la table, dont nous venons d’examiner les abus, n’ont en eux-mêmes rien de blâmable. Il est très naturel, très conforme à la raison d’aimer les aliments flatteurs pour le palais et de les préférer à ceux qui lui seraient insipides ou désagréables. Mais il serait contraire à la Nature de prendre ces aliments sans mesure et, pour satisfaire un plaisir passager, de s’exposer à de longues infirmités. Il serait odieux et criminel de dévorer dans des festins la substance du pauvre. Il serait insensé de déranger la fortune pour contenter un appétit trop écouté : la passion pour des mets recherchés ou pour des vins délicieux est faite pour nous rendre méprisables. Un gourmand ne parut jamais un être bien estimable ; un homme trop difficile est souvent malheureux.\par
Les yeux peuvent sans crime se porter sur les charmes divers que la Nature répand sur ses ouvrages. Une belle femme est un objet digne d’attirer les regards, il est très naturel d’éprouver du plaisir à sa vue. Mais ce plaisir deviendrait fatal pour nous s’il allumait dans nos cœurs une ardeur importune, il se changerait en crime s’il excitait en nous une passion capable de nous faire entreprendre des actions déshonorantes pour l’objet que nous avons d’abord innocemment admiré. Il ne peut y avoir aucun mal à entendre avec plaisir des sons harmonieux qui flattent notre oreille ; mais ce plaisir peut avoir des conséquences blâmables s’il nous amollit le cœur en le disposant à la volupté, à la débauche, ou s’il nous fait oublier nos devoirs essentiels.\par
Il est très naturel d’aimer et de chercher les agréments et les commodités de la vie, de préférer des vêtements moelleux à ceux qui font une impression désagréable sur les doigts ; mais il est puéril de n’avoir l’esprit occupé que de vaines parures, il serait insensé de déranger sa fortune pour contenter une sotte vanité. La morale ne condamne le luxe et les plaisirs qu’il procure, que parce qu’ils servent d’aliment à des passions extravagantes qui nous font méconnaître ce que nous devons à la société. L’amour du faste ferme nos cœurs aux besoins de nos semblables, il amène notre propre ruine et celle de la patrie. Les spectacles et les amusements divers que la société nous présente sont des délassements que la raison approuve tant qu’ils n’ont pas des conséquences dangereuses, mais elle condamne des spectacles licencieux qui ne rempliraient l’esprit d’une jeunesse emportée que d’images lubriques, et son cœur de maximes empoisonnées. La saine morale pourrait-elle ne pas s’élever contre tout ce qui fait éclore ou ce qui fomente des passions capables de ravager la société ? Comment des femmes faibles et d’une imagination vive résisteraient-elles à des passions que le théâtre leur montre chaque jour sous les traits les plus propres à séduire ?\par
Bien des moralistes, que l’on accuse communément d’une sévérité ridicule, ont blâmé les spectacles et les ont regardés comme une source de corruption. Quelque rigoureux que paraisse ce jugement, la saine morale se trouve à bien des égards obligée d’y souscrire. Si l’amour et une passion funeste par les ravages qu’elle produit, si la débauche est un mal, si la volupté est dangereuse, quels effets ces passions présentées sous les traits les plus séduisants ne doivent-elles pas produire sur une jeunesse imprudente qui ne court au théâtre que pour attiser des désirs qu’elle porte déjà dans son cœur ? Sans parler de ces pièces licencieuses admises ou tolérées dans quelques pays, la jeunesse, si elle parlait de bonne foi, conviendrait que c’est bien plutôt les charmes d’une actrice et des images lascives qu’elle va chercher au théâtre, que les sentiments vertueux qu’un drame peut renfermer. C’est le doux poison du vice que vont boire à longs traits tant de voluptueux désœuvrés dont les spectacles sont devenus la principale affaire. Les plus opulents d’entre eux nous prouvent par leur conduite que ce n’est nullement la vertu qu’ils y vont applaudir ou chercher. Le théâtre est un écueil où la fidélité conjugale, la raison, les fortunes et les mœurs vont à tout moment échouer. On peut, sans risque de se tromper, porter le même jugement de ces assemblées publiques et nocturnes connues sous le nom de {\itshape bals}, où le libertinage curieux, les intrigues criminelles, les aventures inopinées ou concertées rapprochent les personnes des deux sexes. Il est difficile de croire que ce soit le désir de prendre un exercice utile à sa santé qui excite une si vive ardeur pour la danse dans un grand nombre de femmes délicates ou d’hommes efféminés. Des exemples multipliés nous prouvent que pour bien des gens le bal n’est rien moins qu’un plaisir innocent. Mais par une cruelle nécessité, dans les sociétés corrompues les plaisirs originairement les plus simples, par l’abus que le vice en fait faire, se convertissent en poison et ne servent qu’à étendre et multiplier la corruption : celle-ci est un besoin indispensable pour une foule d’opulents vicieux et désœuvrés qui cherchent partout le vice, devenu l’unique aliment convenable à leurs âmes flétries. La morale la plus simple doit paraître révoltante et farouche à des hommes sans mœurs ou à des étourdis incapables d’envisager les conséquences souvent terribles de leurs vains amusements. Ce n’est point à des êtres de cette trempe que la raison peut adresser des leçons. Entre les mains de l’homme imprudent et dépravé, tout change, tout se dénature et devient dangereux. La lecture ne lui plaît qu’autant qu’elle contribue à nourrir ses penchants déréglés. De là tant de romans amoureux, tant de vers et de productions dont la frivolité n’est que le moindre défaut, [qui] font l’unique étude des gens du monde et dont ils ne servent qu’à fortifier les inclinaisons très funestes au repos des familles et de la société.\par
Au risque donc de déplaire à bien du monde, la morale n’approuvera nullement des plaisirs ou des amusements d’où résultent visiblement les maux les plus réels. L’homme de bien résiste à l’opinion publique toutes les fois qu’elle est contraire à la félicité publique, toujours invinciblement liée à la bonté des mœurs. Tous les plaisirs capables de favoriser des passions naturelles que l’on doit contenir, ne peuvent être innocentes aux yeux de la raison. Les hommes ne peuvent-ils donc s’amuser sans se salir l’imagination, sans s’exciter au vice, sans se nuire à eux-mêmes et aux autres ? Le grand mal des riches vient de ce qu’ils veulent se délasser sans jamais s’être vraiment occupés. Les jeux divers inventés pour donner du relâchement aux esprits fatigués de leurs occupations habituelles, ne sont blâmables que lorsqu’ils prennent eux-mêmes la place de ces occupations plus importantes. Le jeu n’est qu’une fureur insensée quand il nous expose à la ruine {\itshape ;} il prouve le vide de ceux qui ne sauraient sans lui ni s’occuper ni converser les uns avec les autres. Un joueur de profession n’est bon à rien et s’ennuie dès qu’il cesse de tenir des cartes ou des dés\footnote{Il est bon de remarquer que les cartes à jouer furent inventées pour amuser Charles VI, roi de France, lorsqu’il fut tombé en démence. On dirait que depuis, le mal de ce prince a gagné toute l’Europe, où les cartes font le bonheur ou la ressource de la bonne compagnie, et même de la plus mauvaise.}.\par
En un mot, ce ne sont point les plaisirs des sens que la raison condamne, c’est l’abus qu’on en fait communément, c’est l’usage trop fréquent qui les rend insipides ou qui nous en fait des besoins pressants que nous ne pouvons plus satisfaire qu’au détriment de nous-mêmes ou des autres.\par
Les plaisirs {\itshape intellectuels} ou de l’esprit sont, comme on l’a dit ailleurs, les plaisirs que les sens nous ont offerts, renouvelés par la mémoire, contemplés par la réflexion, comparés par le jugement, animés, exaltés, embellis, multipliés par notre imagination. Lorsque retirés, pour ainsi dire, en nous-mêmes, nous nous rappelons les objets ou les sensations qui nous ont plu, nous les considérons sous plusieurs faces, nous les comparons entre eux, nous nous les peignons sous des traits souvent plus séduisants que la réalité. Mais de même que les plaisirs des sens, les plaisirs intellectuels peuvent devenir louables ou blâmables, honnêtes ou criminels, avantageux ou nuisibles soit pour nous, soit pour la société. C’est à la raison qu’il appartient de régler notre esprit et de mettre des limites à notre imagination, trop souvent sujette à nous enivrer, nous égarer, nous entraîner au mal. Un esprit vif, une imagination ardente sont des guides bien dangereux lorsqu’ils perdent de vue le flambeau de la raison. La morale doit diriger nos pensées et bannir de notre esprit les idées qui peuvent avoir pour nous des conséquences fâcheuses. Les égarements de la pensée sont bientôt suivis des égarements de la conduite.\par
Les plaisirs de l’esprit peuvent être ou très honnêtes ou très criminels. La science, l’étude, des lectures utiles laissent dans notre cerveau des traces ou des idées qui, embellies par une imagination brillante, deviennent une source intarissable de jouissances pour nous-mêmes et pour ceux à qui nous communiquons nos découvertes. Mais le cerveau de l’homme ignorant, désœuvré, vicieux, ne se remplit que d’images futiles, lubriques, déshonnêtes, capables de mettre ses passions et celles des autres dans une fermentation dangereuse. L’imagination réglée d’un homme de bien lui peint avec vérité les avantages de la vertu, la gloire qui en résulte, la tendresse qu’elle lui attire, les douceurs de la paix d’une bonne conscience. L’imagination égarée d’un ambitieux lui représente les futiles avantages d’une puissance incertaine dont il ne sait point user. Celle d’un fat lui montre tous les yeux étonnés de son faste, de ses équipages, de ses livrées, de sa parure ; celle de l’avare lui représente des biens sans nombre dont il ne jouira jamais.\par
L’imagination est donc la source commune du vice et de la vertu, des plaisirs honnêtes et déshonnêtes ; c’est elle qui, réglée par l’expérience, exalte aux yeux de l’homme de bien les plaisirs moraux, les charmes de la science, les attraits de la vertu. Ces plaisirs sont totalement inconnus d’un tas d’esprits bornés, de ces âmes rétrécies pour qui la vertu n’est qu’un vain nom, ou pour tant d’hommes dépourvus de réflexion qui ne croient voir en elle qu’un objet triste et lugubre. Qu’est-ce que la bienfaisance, l’humanité, la générosité, pour la plupart des riches, sinon la privation d’une portion de leur bien qu’ils destinent à se procurer des plaisirs peu solides ? Ces vertus présentent une toute autre idée à celui qui médite leurs effets sur les cœurs des mortels, qui connaît la réaction de la reconnaissance, qui se voit dans sa propre imagination un objet digne de l’amour de ses concitoyens.\par
La conscience est presque nulle pour l’étourdi qui ne réfléchit point, pour celui que la passion aveugle, pour le stupide qui n’a point d’imagination : il en faut pour se peindre avec force les sentiments divers que nos actions, bonnes ou mauvaises, produiront sur les autres, il faut avoir médité l’homme pour savoir la manière dont il peut être affecté, soit en bien, soit en mal. Cette imagination prompte et cette réflexion constituent la sensibilité, sans laquelle les plaisirs moraux ne touchent guère et la conscience ne parle que faiblement. Quel plaisir peut trouver à soulager un autre, celui qui ne se sent pas assez vivement affecté de la peinture de ses maux pour avoir un grand besoin de se soulager lui-même ? Il faut avoir entendu retentir dans son cœur le cri de l’infortune pour trouver du plaisir à la faire cesser.\par
L’homme qui ne sent point ou qui ne pense point, ne sait jouir de rien ; la Nature entière est comme morte pour lui, les art qui la réprésentent n’affectent point ses yeux appesantis. La réflexion et l’imagination nous font goûter les charmes et les plaisirs qui résultent de la contemplation de l’univers ; c’est par elles que le monde physique et le monde moral deviennent un spectacle enchanteur dont toutes les scènes nous remuent vivement. Tandis qu’une foule imprudente court après des plaisirs trompeurs qu’elle ne peut jamais fixer, l’homme de bien sensible, éclairé, rencontre partout des jouissances. Après avoir trouvé du plaisir dans le travail, il en retrouve dans des délassements honnêtes, dans des conversations utiles, dans l’examen d’une Nature diversifiée à l’infini. La société, si fatigante pour des êtres qui réciproquement s’incommodent et s’ennuient, fournit à l’homme qui pense des observations multipliées dont son esprit se remplit ; il amasse des faits, il accumule des provisions propres à l’amuser dans sa solitude. Les champs, si uniformes pour les habitants agités de nos villes, lui offrent à chaque pas mille plaisirs nouveaux. Le fracas bruyant des villes et les extravagances du vulgaire sont pour lui des spectacles intéressants. En un mot, tout nous prouve qu’il n’est de vrais plaisirs que pour l’être qui sent et qui médite ; tout lui démontre les avantages de la vertu et les inconvénients qui résultent des folies et des défauts des hommes.
\subsection[{Chapitre XII. Des Défauts, des Imperfections, des Ridicules ou des qualités désagréables dans la vie sociale}]{Chapitre XII. Des Défauts, des Imperfections, des Ridicules ou des qualités désagréables dans la vie sociale}
\noindent Après l’examen qui vient d’être fait des vices ou des dispositions nuisibles à la vie sociale, il nous reste encore à parler des défauts ou des imperfections dont l’effet est de nous rendre incommodes ou désagréables à ceux avec qui nous vivons. Ainsi que les vices, les défauts des hommes sont des suites de leur tempérament diversement modifié par l’habitude ; on peut les définir [comme] des privations des qualités nécessaires pour se rendre agréable dans la société.\par
Comme un être sociable se sent toujours intéressé à plaire aux personnes avec lesquelles il doit vivre, non seulement il se croit obligé de résister à ses passions et de combattre ses penchants déréglés, mais encore il cherche à corriger les défauts qui pourraient affaiblir la bienveillance qu’il désire exciter. Chacun est aveugle sur ses propres défauts, mais l’homme sociable doit s’étudier lui-même, tâcher de se voir des mêmes yeux dont il est vu par les autres, juger ses imperfections comme il juge celles qu’il aperçoit dans ses semblables. Ce qu’il trouve désagréable ou choquant en eux suffit pour lui faire connaître ce qui doit les choquer ou leur déplaire en lui. C’est ainsi que le sage peut tirer un profit réel des imperfections, des faiblesses des hommes ; il apprend de cette manière à éviter dans ses actions ce qui lui déplaît dans leur conduite. Il sait qu’il ne doit rien négliger pour mériter l’estime et l’affection, et que les moindres défauts, quoiqu’ils ne causent pas des effets si sensibles et si prompts que le crime, ne laissent pas à la longue de blesser profondément les personnes qui en sentent les effets continués : « La moindre surcharge, dit Montaigne, brise les barrières de la patience\footnote{Montaigne, {\itshape Essais}, livre I.}. »\par
Tous les hommes ont des défauts plus ou moins incommodes à ceux qui en ressentent les effets ; nous souffrons quelquefois de ceux auxquels nous sommes sujets nous-mêmes sans nous en apercevoir : ils nous déplaisent dans les autres, tandis que nous ne songeons nullement à nous en corriger. Nous sommes très pénétrants lorsqu’il s’agit de voir les imperfections et les faiblesses des autres, et nous sommes des aveugles dès qu’il s’agit des nôtres. Comment expliquer ce phénomène ? Il est très facile à résoudre. Nous sommes, par l’habitude, accoutumés à notre façon d’être ; bonne ou mauvaise, nous la croyons nécessaire à notre bonheur. Il n’en est pas de même des défauts des autres, auxquels nous ne nous accoutumons presque jamais. Nous désirons qu’ils se corrigent parce que leurs défauts nous blessent, et nous ne nous corrigeons pas parce que nos défauts nous font plaisir ou nous paraissent des biens.\par
On est tout surpris de voir dans le monde des personnes accoutumées à vivre ensemble se séparer quelquefois brusquement et se séparer pour toujours ; mais on cessera d’être étonné de cette conduite si l’on considère que ces défauts qui paraissent d’abord faciles à supporter, en se faisant sentir journellement, deviennent insupportables. Ce sont des piqûres légères qui, continuellement réitérées, forment enfin des plaies douloureuses que rien ne peut guérir. Voilà sans doute pourquoi rien n’est plus rare que de voir persévérer jusqu’à la fin des personnes dont l’humeur ou le caractère se conviennent assez pour vivre ensemble dans une grande familiarité. Cette familiarité même, semblant les autoriser à bannir d’entre elles la gêne, contribue à leur faire mieux sentir leurs défauts réciproques. Telle est la vraie cause de la fréquente désunion que l’on voit entre les époux, les parents et les amis les plus intimes.\par
Que l’homme social se juge donc impartialement lui-même, qu’il se corrige des défauts capables d’altérer ou d’anéantir la bienveillance qu’il veut rencontrer. Mais, d’un autre côté, l’humanité lui recommande d’avoir de l’indulgence pour l’imperfection de ses semblables et, d’accord avec la justice, elle lui prouve que ce n’est qu’à ce prix qu’il peut s’attendre lui-même à faire tolérer ses propres faiblesses. Celui qui n’a pas d’indulgence est, comme on l’a prouvé, un être insociable qui se condamne à subir un jugement rigoureux. Nul homme sur la terre n’est exempt de défauts\footnote{Horace, {\itshape Satires}, livre I, 3, vers 68.} ; s’irriter sans cesse contre les faiblesses des autres, c’est se déclarer peu fait pour vivre en société. Il n’y a qu’une grande indulgence, une douceur continue dans le caractère, une attention suivie, une aménité dans l’humeur, une facilité dans les mœurs qui puissent cimenter les unions entre les hommes : souvent, dès qu’ils se sont vus de près, ils cessent de s’aimer.\par
Trop de crainte d’être blessé par les défauts de nos semblables nous conduit à la défiance et à la misanthropie, dispositions très contraires à la vie sociale et qui donnent lieu de croire que celui où elles se trouvent est lui-même d’un caractère suspect. Ceux qui ne croient pas à la vertu des autres doivent faire présumer qu’ils n’en ont guère eux-mêmes. {\itshape Tous les hommes sont des scélérats}, disait un misanthrope à un très honnête homme qu’il voyait assez souvent. {\itshape Où donc voyez-vous cela} ? lui répondit celui-ci. {\itshape En moi}, répliqua sur le champ le premier.\par
L’homme défiant, soupçonneux, à qui tout fait ombrage, est nécessairement très misérable. Perpétuellement entouré de pièges et de dangers imaginaires, il ne connaît ni les charmes de l’amitié, ni les douceurs du repos, ni les agréments de la société. Il se voit seul dans le monde exposé aux embûches d’une foule d’ennemis. La défiance continuelle est un supplice long et cruel dont la Nature se sert pour punir les tyrans et tous ceux qui ont la conscience d’avoir attiré sur eux l’inimitié des hommes. Le méchant est toujours armé de craintes et de soupçons.\par
D’un autre côté, la confiance excessive n’est rien moins qu’une vertu ; elle est une marque de faiblesse et d’inexpérience. C’est après avoir éprouvé les hommes que l’on peut leur accorder sa confiance. Malheur à celui qui n’a trouvé personne digne de la mériter ! La prudence est la vertu qui tient un juste milieu entre la défiance misanthropique et la confiance excessive. On ne peut sans danger se fier à tout le monde mais c’est être bien malheureux que de ne se fier à personne. « Se fier à tout le monde, ne se fier à personne sont deux vices, dit Sénèque {\itshape ;} mais il y a plus d’honnêteté dans l’un, plus de sûreté dans l’autre. »\par
La fermeté, le courage, la constance, la force étant des qualités sociales ou des vertus, nous devons regarder la faiblesse, la mollesse, l’inconstance comme des défauts réels et souvent même comme des vices impardonnables. L’homme faible est toujours chancelant dans sa conduite ; peu maître de lui, il est sans cesse au premier occupant et prêt à se laisser aller où l’on veut le conduire. Il est impossible de compter sur l’homme sans caractère : il n’a point de but arrêté, il n’oppose aucune résistance aux impulsions qu’on lui donne, il devient le jouet continuel de ceux qui prennent facilement de l’ascendant sur son esprit. Sans système et sans principes dans sa conduite, il est irrésolu, inconstant, toujours flottant entre le vice et la vertu. Celui qui n’est pas fortement attaché à des principes est aussi peu capable de résister à ses propres passions qu’à celles des autres. La faiblesse est communément l’effet d’une paresse habituelle et d’une indolence qui va jusqu’à se prêter quelquefois au crime même. Un souverain sans fermeté devient un vrai fléau pour son peuple. L’homme faible peut être aimé et plaint, mais jamais il ne peut être sincèrement estimé ; il fait sans le savoir quelquefois plus de mal que le méchant décidé, dont la marche connue fait au moins qu’on l’évite. Un caractère trop facile inspire une confiance qui finit presque toujours par être trompée.\par
Rien de plus désagréable et de moins sûr dans le commerce de la vie que ces caractères lâches et pusillanimes qui, pour ainsi dire, tournent à tout vent. Comment compter un instant sur des hommes qui n’ont presque jamais d’avis que celui des personnes qu’ils rencontrent, prêts à en changer dès qu’ils changeront de cercle, disposés à livrer leurs amis mêmes à quiconque voudra les déchirer ? Jamais un homme lâche et sans caractère ni fermeté ne peut être considéré comme un ami solide.\par
Il est très peu de gens dans le monde qui soient bien fermement ce qu’ils sont, qui montrent un caractère bien marqué, qui aient un but vers lequel ils marchent d’un pas sûr. Rien de plus rare que l’homme solide qui suive un plan sans le perdre de vue\footnote{Horace, {\itshape Épîtres}, livre I, 1, vers 82.} : de là toutes les variations, les contradictions, les inconséquences que nous observons dans la conduite de la plupart des êtres avec qui nous vivons. On les voit, pour ainsi dire, continuellement égarés, sans objet déterminé, prêts à se laisser détourner de leur route par le moindre intérêt qu’on vient leur présenter. La morale doit se proposer de fixer invariablement les yeux des hommes sur leurs intérêts véritables et leur offrir les motifs les plus capables de les affermir dans la route qui conduit au bonheur.\par
C’est le défaut de fixité dans les principes et de stabilité dans le caractère qui rend les vices et les défauts des hommes si contagieux. L’usage du monde, la fréquentation de la cour et des grands, le commerce des femmes, en même temps qu’ils servent à polir, contribuent trop souvent à effacer le caractère et gâter le cœur. On veut plaire, on prend le ton de ceux que l’on fréquente, et l’on devient quelquefois vicieux ou méchant par pure complaisance. L’habitude de sacrifier ses volontés et ses propres idées à celles des autres fait que l’on n’ose plus être soi, on n’a plus de physionomie, on change à tout moment de principes et de conduite : on craindrait sans cela d’être accusé de raideur, de singularité, d’impolitesse ou de pédanterie. {\itshape Il faut être comme tout le monde} est la maxime banale de tant de gens sans courage, sans principes, sans caractère, dont le monde est rempli. Voilà comment les vices se répandent, les travers se perpétuent, et presque tous les hommes finissent par se ressembler\footnote{Un homme d’esprit disait que les gens du monde étaient comme les monnaies, dont les empreintes se sont presque entièrement effacées à force d’avoir passé de mains en mains.}. Voilà comment ils sont continuellement entraînés par l’exemple, par la crainte de déplaire à des êtres dépravés. Enfin, voilà comment l’ignorance ou l’incertitude du but que l’on doit se proposer et la faiblesse, sont les vraies sources du mal moral, des vices, des extravagances et même souvent de la perversité qu’on voit régner parmi les hommes.\par
Il faut de la vigueur pour être vertueux au milieu d’un monde insensé ou pervers. {\itshape Osez être sage}, a dit un Ancien. Mais faute de lumières, peu de gens ont ce courage, que tout d’ailleurs s’efforce d’amortir. En effet, on ne peut douter que le gouvernement, fait pour agir si puissamment sur les hommes, n’influe de la façon la plus marquée sur leurs caractères et leurs mœurs. Le despotisme ne fait de ses esclaves que des automates prêts à recevoir toutes les impulsions qu’il leur donne ; et ces impulsions les portent toujours au mal. Un gouvernement militaire donne à toute une nation le ton de l’étourderie, de la vanité, de l’arrogance, de la présomption, de la licence. Il faut être bien ferme et bien nerveux pour résister constamment à des forces qui agissent incessamment sur nous.\par
La légèreté, l’étourderie, la dissipation, la frivolité forment, encore plus que la malice du cœur humain, des obstacles à la félicité sociale. Il est des pays où la légèreté paraît un agrément, mais il est bien difficile de faire d’un homme léger un ami solide sur les sentiments et la discrétion duquel il soit permis de compter. Comment compter sur un être qui n’est jamais sûr de lui-même ! La morale, pour être mise en pratique, exige de la réflexion, de l’attention, de fréquents retours sur soi, un recueillement intérieur dont peu de gens sont capables. Voilà pourquoi la morale paraît si rebutante à des esprits frivoles, qui lui préfèrent des bagatelles. L’habitude de penser peut seule donner à tout être raisonnable la faculté de combiner promptement ses rapports et ses devoirs. La félicité de l’homme est un objet si grave qu’il semblerait mériter quelques soins de sa part et devoir fixer les regards sur les moyens de l’obtenir. « Consulte-toi deux ou trois fois, dit le poète Théognis, car l’homme précipité est toujours un homme nuisible\phantomsection
\label{footnote33}\footnote{Voyez {\itshape Poetæ Græci minores, Theognidis Carmina}.}. »\par
Tout nous prouve l’importance de mettre un frein à notre langue dans un monde désœuvré, curieux, rempli de malignité ; cependant, rien de plus commun que {\itshape l’indiscrétion}, qui est un besoin de parler dont tant de gens paraissent tourmentés. Ce défaut terrible quelquefois par ses conséquences, n’annonce pas toujours un mauvais cœur, quoiqu’il produise souvent des effets aussi cruels que la méchanceté. Il est dû à l’étourderie, à la légèreté, et souvent à une forte vanité qui se fait un mérite de repaître la curiosité des autres. L’indiscret est si dépourvu de réflexion qu’il divulgue son propre secret et se compromet lui-même aussi facilement que les autres. Il est communément faible et sans caractère {\itshape ;} il n’a pas la force de garder le dépôt qu’on a eu la sottise de lui confier. Quoique l’indiscrétion soit quelquefois aussi dangereuse qu’une trahison, elle passe néanmoins pour une faute légère dans un monde frivole, oisif et curieux.\par
La {\itshape curiosité}, ou le désir de pénétrer les secrets des autres, est un défaut qui annonce communément le vide la tête. Le curieux est pour l’ordinaire un fainéant qui n’a que très peu d’idées ; d’ailleurs, on ne peut guère compter sur sa discrétion. « Fuyez le curieux, dit Horace, car il est toujours indiscret ou bavard\phantomsection
\label{footnote34}\footnote{Horace, {\itshape Épîtres}, livre I, 18, vers 69.}. » Enfin, l’on est curieux par vanité. L’on attache de la gloire à pouvoir dire que {\itshape l’on sait} ou qu’on {\itshape a vu} : c’est un mérite pour les sots auprès des désœuvrés.\par
Il est difficile de bien parler et de beaucoup parler. Quoi de plus fatiguant que ces discoureurs impitoyables, que ces dissertateurs éternels qui semblent toujours se croire dans la tribune aux harangues, sans jamais vouloir en descendre ? C’est avoir peu d’égards à l’amour-propre des autres que de ne point leur permettre de parler à leur tour. Mais bien des gens sont dans l’idée que ce n’est qu’en parlant beaucoup qu’on montre beaucoup d’esprit. Un proverbe trivial mais très sensé nous dit {\itshape qu’un vaisseau plein fait moins de bruit qu’un vaisseau vide.}\par
D’un autre côté, rien de plus rare que des personnes qui sachent écouter, et rien de plus commun que des gens qui veulent qu’on les écoute : cette injustice, cet amour-propre exclusif se montre fréquemment dans la société. La conversation étant faite pour instruire ou pour amuser, chacun se croit en droit d’y contribuer ; c’est faire un affront aux autres que de les en exclure. Par une suite de cette vanité, l’on voit quelquefois des gens d’esprit ne se plaire que dans la compagnie des sots. {\itshape C’est un sot}, disait un homme d’esprit, {\itshape mais il m’écoute}. « Il y a, dit un auteur moderne, des gens qui aiment mieux être rois dans la mauvaise compagnie que citoyens dans la bonne\footnote{Voyez Moncrif, {\itshape Art de plaire}.}. »\par
Si la conversation doit avoir pour objet d’éclairer et de plaire, on peut parler quand on se croit en état d’y réussir ; mais il ne faut point oublier que les autres sont capables de contribuer à notre instruction et à notre amusement. Il faut écouter et se taire quand on n’a rien d’agréable ou d’utile à communiquer. C’est, comme on l’a dit ailleurs, le vide de la conversation qui rend la médisance et la calomnie si communes : quand on ne sait point parler des choses, on se jette sur les personnes.\par
Le grand art de la conversation consiste à ne blesser, à n’humilier personne, à ne parler que des choses qu’on sait, à n’entretenir les autres que de ce qui peut les intéresser. Cet art que tout le monde croit posséder n’est rien moins que commun. Les sociétés sont remplies ou d’importants qui préviennent contre eux par leur sotte vanité, qui veulent parler de tout, ou d’ennuyeux qui nous fatiguent en nous parlant d’objets peu faits pour nous intéresser. Un sot s’imagine que ce qui frappe sa tête rétrécie a droit d’intéresser l’univers.\par
L’expérience, la réflexion, l’étude, et surtout la bienveillance et la bonté du cœur peuvent seules nous rendre utiles et agréables dans le commerce de la vie. Les entretiens des gens du monde ne sont communément si stériles, leurs visites si fastidieuses, leurs assemblées les plus brillantes et leurs banquets somptueux ne sont remplis d’ennui que parce que la société rapproche des gens qui s’aiment et s’estiment fort peu, qui se connaissent à peine, qui n’ont rien de bon à se dire, qui ne se disent que des riens. Ce qu’on appelle {\itshape le grand monde} n’est le plus souvent composé que de personnes très vaines qui ne croient réciproquement se rien devoir, qui, privées d’instruction, ne portent dans la société que de la raideur, de la sécheresse, du dégoût : la conversation doit être nécessairement stérile et languissante quand le cœur et l’esprit n’y peuvent entrer pour rien. Il n’y a que l’amitié franche et sincère, la science, la vertu qui puissent donner de la vie au commerce des hommes.\par
La vanité rend insociable. L’ignorance, l’oisiveté, l’inhabitude de penser et l’aridité du cœur sont les causes qui font pulluler les {\itshape ennuyeux}, les diseurs de riens, les importuns et les fats, dont les cours, les villes et les campagnes sont perpétuellement infestées. Tout homme dont l’esprit est vide devient très incommode aux autres par le besoin qu’il a de remuer son âme engourdie et de suspendre son ennui ; tourmenté sans relâche par cet ennemi domestique, il ne s’aperçoit nullement qu’il est un vrai fléau pour les autres. Un des grands inconvénients du commerce du monde est d’exposer les personnes occupées à devenir les victimes d’une foule d’importuns, de fainéants, d’ennuyeux, qui viennent périodiquement leur apprendre qu’ils n’ont rien à leur dire. Un peu de bon sens ne devrait-il pas suffire pour apprendre à respecter les moments de l’homme occupé ? Il est des instants où l’ami même doit craindre d’incommoder son ami. Mais des réflexions si naturelles n’entrent pas dans la tête de ces stupides que la politesse fait tolérer, tandis qu’ils en violent eux-mêmes toutes les règles.\par
En regardant les choses de près, on trouvera que même parmi ceux qui se piquent le plus de politesse, de savoir-vivre, d’usage du monde, il est très peu de gens que l’on puisse appeler vraiment {\itshape polis}. Si la vraie politesse consiste à ne choquer personne, tout homme vain est impoli. Le fat, le {\itshape petit maître}, la coquette évaporée pèchent aussi grossièrement contre la bienséance et la politesse que le rustre le plus mal élevé. Peut-on regarder comme vraiment polis ces personnages dont le maintien arrogant, les regards effrontés, les manières dédaigneuses ou négligées semblent insulter tout le monde ? Un élégant enivré de ses perfections, uniquement occupé de sa futile parure, qui, se présentant dans un cercle, ne fait attention à personne, qui joue la distraction et n’écoute jamais ce qu’on lui dit ni la réponse qu’on lui fait, qui se glorifie de ses travers, est évidemment un impudent qui se met au-dessus des égards que l’on doit à la société. Les gens les plus épris d’eux-mêmes font communément de leur mieux pour en dégoûter les autres. L’impudence consiste dans un mépris insolent de l’estime et de l’opinion publique, que tout homme, quel qu’il soit, doit toujours respecter.\par
Bien des gens se montrent arrogants et fiers dans la crainte d’être méprisés, ou du moins de ne pas s’attirer la dose de considération qu’ils croient mériter. {\itshape Il faut se faire valoir}, nous disent-ils. Oui, sans doute ; mais c’est par des qualités aimables et respectables. L’arrogant se fait haïr de peur de n’être pas suffisamment estimé !\par
Si le mérite le plus réel déplaît quand il se montre avec ostentation, quels sentiments peut exciter celui dont le mérite ne consiste que dans ses habits, ses équipages et dans des manières qui sont des affronts continuels pour les autres ? Mais les impertinents de cette trempe se suffisent à eux-mêmes ; ils dédaignent les jugements du public dont ils se flattent, à force d’insolence, d’arracher l’admiration. Une bonne opinion de soi constitue l’orgueil ; il déplaît, même avec du mérite, parce qu’il usurpe les droits de la société, qui veut demeurer en possession d’apprécier ses membres. La vanité est la haute opinion de soi fondée sur des futilités. D’où l’on voit que la suffisance, le faste, les grands airs annoncent des avantages qui n’en imposent qu’à des sots. La simplicité, la modestie, la défiance de soi-même sont des moyens bien plus sûrs de réussir que les prétentions, les hauteurs, les airs importants et le jargon de tant d’impertinents qui semblent méconnaître ce que l’on doit aux hommes. La suffisance et la fatuité sont des maladies presque incurables. Comment guérir un homme content de lui-même et qui se croit au-dessus du jugement des autres ? L’esprit de contradiction, l’opiniâtreté, la trop grande chaleur dans la dispute, l’amour de la singularité sont encore des défauts qu’engendre la vanité. Bien des gens s’imaginent qu’il est glorieux de n’être de l’avis de personne : ils croient par là faire preuve d’une sagacité supérieure mais ils ne prouvent souvent que leur mauvaise humeur et leur impolitesse. Ils nous diront sans doute qu’ils se sentent animés d’un grand amour pour la vérité ; mais nous leur répondrons que c’est ne la point aimer que de la présenter d’une façon propre à rebuter. La raison ne peut plaire lorsqu’elle prend le ton de l’impolitesse et de la dureté. Il est bien difficile de convaincre celui dont l’amour-propre est blessé.\par
L’opiniâtreté est l’effet d’une sotte présomption et d’un préjugé puéril qui nous suggèrent qu’il est honteux de se tromper, qu’il y a de la bassesse à l’avouer, qu’il est beau {\itshape d’avoir toujours le dernier [mot]}. Mais n’est-il pas plus honteux et plus insensé de résister à la vérité ? N’est-il pas plus noble et plus grand de céder avec douceur même lorsqu’on est sûr d’avoir la raison pour soi, que de disputer sans fin avec des personnes déraisonnables ? Le peuple et les sots donnent raison à ceux qui crient le plus longtemps et le plus fort, mais les personnes sensées la donnent à celui qui a le courage de se rétracter quand il a tort ou de ne point abuser de sa victoire\footnote{Racine et Boileau se trouvant ensemble à l’Académie des Inscriptions, ce dernier avança par mégarde une proposition qui n’était pas juste. Racine, auprès duquel ses amis même ne trouvaient point de grâce quand il leur échappait quelque chose qui pût lui donner prise, ne s’en tint pas à une simple plaisanterie mais tomba rudement sur son ami et alla même jusqu’à l’insulte. Boileau se contenta de lui dire : « Je conviens que j’ai tort, mais j’aime encore mieux l’avoir, que d’avoir aussi orgueilleusement raison que vous l’avez. »} quand il a combattu pour la vérité. La singularité ne prouve aucun mérite réel : s’écarter des opinions ou des usages admis par la société montre communément plus d’orgueil que de sagesse ou de lumières. Il faut résister au torrent de la coutume quand elle est évidemment contraire à la vertu, il faut s’y laisser entraîner dans les choses indifférentes. Une conduite opposée à celle de tout le monde étonne quelquefois un moment mais ne peut point attirer une considération durable.\par
En général, toute affectation déplaît : elle décèle de la vanité. Le vrai, le simple, le naturel nous rendent chers à ceux avec qui nous vivons parce qu’ils veulent toujours nous voir tels que nous sommes. Il faut être soi pour bien jouer son rôle sur la scène du monde ; on ne risque point alors de se voir démarquer. Une gravité affectée n’annonce qu’un sot orgueil qui voudrait usurper des respects, une pédanterie minutieuse est le propre des petits esprits ; ces défauts ne doivent pas se confondre avec la gravité des mœurs et l’exactitude sévère à remplir ses devoirs, qui partent d’une attention suivie sur nous-mêmes et d’une crainte louable d’offenser les autres par des inadvertances et des légèretés.\par
Rien de plus gênant dans la vie que ces hommes {\itshape pointilleux} dont la vanité sensible et délicate est toujours prête à s’offenser. Celui qui se sent si faible ne devrait point s’exposer au choc de la société, dans laquelle il ne peut jeter que de la contrainte et de l’ennui. Une vanité trop prompte à s’alarmer annonce une faiblesse, une petitesse d’esprit, une inexpérience puérile ; tout homme trop facile à piquer devient nécessairement malheureux dans un monde rempli de plus d’étourderie que de méchanceté. Est-il rien de plus fâcheux que d’avoir une âme assez faible pour être à tout moment troublé par les inadvertances ou par le moindre oubli des personnes que l’on fréquente ? Cependant, ces petitesses, dont un homme raisonnable ne devrait point s’apercevoir, ont souvent dans un monde vain et frivole les conséquences les plus graves.\par
En général, la vanité, comme on l’a dit ailleurs, est le vice qui produit le plus de ravages dans le monde. Des personnes de tout âge et de tout rang, par le prix qu’elles attachent à des minuties, semblent n’être que de grands enfants. Bien des hommes, en grandissant, ne font que changer de jouets. Des vêtements plus riches, des équipages plus brillants, des bijoux plus coûteux, des parures plus variées, des inutilités plus recherchées remplacent chaque jour les objets dont s’amusait leur enfance. Combien petite et rétrécie doit être l’âme de tant de gens dont le soin de leur parure absorbe et la fortune et le temps ! Quelle idée peut-on se former de ces femmes et de ces hommes dégradés dont la toilette et les pompons occupent toutes les journées ? Le vrai châtiment de ces enfants est de ne point les remarquer.\par
Les nations où le luxe domine, sont remplies d’êtres frivoles sérieusement occupés de babioles devenues à leurs yeux des objets très importants. C’est pour elles qu’ils perdent et leur temps et leur argent ; c’est à des petitesses qu’ils sacrifient leur bonheur et leur repos ; c’est pour une vanité puérile qu’ils courent, qu’ils se portent envie, qu’ils se disputent et se blessent. La raison mûre ou la sagesse consiste à n’estimer les choses que selon leur juste valeur. Celui qui s’est mis au-dessus des bagatelles est plus heureux et plus grand que tous ceux qui s’en rendent les esclaves. La vanité choque tout le monde ; la modération et la modestie ne peuvent choquer personne.\par
La route de la vie est un chemin étroit où se trouve une foule innombrable de passagers qui, chacun à sa manière, s’efforcent d’arriver au bonheur. Vous les voyez se mouvoir avec plus ou moins d’activité suivant des directions très variées qui se croisent, et qui souvent sont totalement opposées. Au milieu de cette troupe confuse, les méchants sont des aveugles qui, au risque de s’attirer le ressentiment général, frappent et blessent tous ceux qui se rencontrent sur leur chemin. Des voyageurs imprudents, légers, distraits, inconsidérés, n’ayant point de but fixe, s’agitent en tous sens, pressent et sont pressés, heurtent et sont heurtés, sont incommodes à tout le monde. Le sage marche avec précaution, il regarde autour de lui, il prévoit et prévient les obstacles et les dangers, il évite la foule et, favorisé du secours de ses associés, il s’avance d’un pas sûr vers le terme du voyage que les plus empressés ne peuvent point atteindre. L’estime, la considération, la bienveillance, la tranquillité sont le prix de l’attention que l’homme de bien apporte dans sa conduite.\par
Faute de réfléchir au but de toute société, les hommes ne semblent réunis que pour se blesser réciproquement par des défauts dont chacun reconnaît les inconvénients dans les autres sans daigner s’apercevoir que les siens doivent nécessairement produire des effets tout semblables. La {\itshape légèreté} n’est que l’incapacité de s’attacher fortement aux objets intéressants pour nous. {\itshape L’inconstance} consiste à changer perpétuellement d’intérêts ou d’objets. {\itshape L’étourderie} consiste à ne pas se donner le temps de bien envisager les objets ou de réfléchir mûrement aux suites de nos actions. La {\itshape frivolité} consiste à n’accorder son attention qu’à des objets incapables de nous procurer un bonheur véritable.\par
Tels sont les ennemis que la raison a souvent à combattre dans la société. L’imprudence, les distractions continuelles, la dissipation, la vanité, l’ivresse des plaisirs, des passions sérieuses pour des futilités sont des barrières qui s’opposent à la réflexion et qui tiennent la plupart des hommes dans une enfance perpétuelle.\par
La distraction est une application de nos pensées à d’autres objets que ceux dont nous devons nous occuper ; elle est un manque d’égards pour ceux avec qui nous vivons. Ce défaut que nous trouvons si ridicule dans de certaines occasions, est pourtant très commun et presque universel. Combien peu de personnes s’occupent des affaires les plus intéressantes pour elles ! Chacun les met de côté pour ne penser qu’aux intérêts souvent futiles qui se sont emparés de son imagination et qui absorbent ses facultés. Chacun, dans sa rêverie, semble oublier qu’il vit en compagnie avec des êtres auxquels il doit son attention et ses soins. Il est aisé de sentir à combien d’inconvénients cette distraction morale nous expose. Un homme sensé doit toujours être attentif et sur lui-même et sur les autres : {\itshape je n’y avais pas songé} est une mauvaise excuse pour un être qui vit en société. Envisager son but et {\itshape faire ce que l’on fait}, voilà la base de toute morale.\par
La vie sociale est un acte religieux dans lequel tout homme doit se dire : {\itshape sois à ce que tu fais}\phantomsection
\label{footnote35}\footnote{Plutarque nous apprend que dans les sacrifices des Anciens un crieur avertissait le prêtre de recueillir son attention en lui disant : « Hoc age ! Soyez à ce que vous faites ! »}.\par
Bien des gens se croient disculpés de leurs fautes en les rejetant sur {\itshape l’oubli}. Mais la conduite de la vie suppose une mémoire assez fidèle pour ne pas oublier des devoirs essentiels qui doivent incessamment se représenter à notre esprit. Des oublis sont très criminels quand ils nous font perdre de vue des devoirs importants de la justice, de l’humanité, de la pitié. Un ministre ou un juge qui oublieraient un innocent dans les prisons au détriment de sa fortune, de sa santé ou de sa vie, sont-ils donc moins coupables qu’un assassin ? Sans nous rendre si criminels, l’habitude d’oublier nous rend désagréables dans la vie sociale ; elle produit l’inaptitude dans nos propres affaires et dans celles des autres. La vie de l’homme, on ne peut assez le redire, demande de l’attention, de la mémoire, de la présence d’esprit.\par
L’ignorance, que l’on allègue très souvent comme une excuse valable, qu’on pardonne quelquefois trop aisément, que l’on punit seulement par le ridicule, peut quelquefois devenir un crime très grave. Quels reproches n’a pas à se faire un juge sans lumières qui décide imprudemment du sort de ses concitoyens ? Quels remords doit éprouver un médecin ignorant qui, aux dépens de la vie des hommes, exerce une profession dans laquelle il ne s’est pas suffisamment instruit ? Il n’est pas permis d’ignorer les principes d’un art important au bien-être de nos semblables ; la suffisance est un crime dès qu’elle se joue du salut des hommes. Tout homme qui a le front d’exercer un office, un emploi public dont il se connaît incapable, est évidemment étranger aux vrais principes de la probité. L’ignorance est la source intarissable des maux sans nombre sous lesquels les peuples sont forcés de gémir. Dans tous les états de la vie, l’homme, pour son propre intérêt et pour celui des autres, doit tâcher de s’instruire. Les lumières contribuent à développer la raison, dont l’effet est de nous rendre meilleurs, plus utiles, plus chers à nos associés.\par
Le défaut d’expérience et de réflexion constitue l’ignorance, qui ne peut être que désavantageuse soit pour nous-mêmes, soit pour les autres. L’ignorant est méprisé parce qu’il n’est d’aucune ressource pour la société ; l’ignorant est à plaindre parce qu’il est communément incapable de s’aider lui-même. La science qui, comme on l’a dit ci-devant, n’est que le fruit de l’expérience et de l’habitude de réfléchir, est estimée parce qu’elle met celui qui la possède à portée de porter des secours, des conseils, des agréments que l’on ne peut attendre de l’ignorant.\par
Dans tous les états de la vie, depuis le monarque jusqu’à l’artisan, l’homme le plus expérimenté ou le plus instruit est nécessairement plus estimé, plus recherché que celui que l’on voit privé de lumières ou d’habileté.\par
Si la raison, comme on l’a fait voir, n’est que l’expérience et la réflexion appliquées à la conduite de la vie, il est très difficile que l’ignorant devienne un être raisonnable, un homme solidement vertueux. Il faut connaître les usages du monde pour y vivre avec agrément et pour éviter le ridicule attaché à l’ignorance de ces mêmes usages. L’ignorant est un aveugle, un étourdi qui marche au hasard dans la route de ce monde, au risque de heurter les autres ou de faire des chutes à tout moment. En un mot, sans expérience ou sans lumières il est impossible d’être bon.\par
On nous dira peut-être que l’on rencontre parfois des personnes simples, grossières, dépourvues d’instruction ou de science, et qui pourtant, comme {\itshape par instinct}, sont vertueuses et fidèles à leurs devoirs, tandis que des hommes doués de l’esprit le plus sublime et des connaissances les plus vastes se conduisent très mal et ne se font remarquer que par des écarts ou des méchancetés. Nous répondrons que des hommes très simples peuvent aisément sentir les avantages attachés à la vertu, ainsi que les inconvénients et les embarras sans nombre dont le vice est accompagné. Sans montrer au dehors des lumières bien éclatantes, ils ont fait intérieurement, pour régler leurs actions, des expériences et des réflexions faciles qui très souvent échappent à la pétulance de l’homme d’esprit, ou que sa vanité dédaigne. D’où il résulte que, malgré sa simplicité, l’homme de bien est quelquefois plus chéri et plus aimable que l’homme de beaucoup d’esprit : celui-ci se fait craindre, le {\itshape bon homme} se fait aimer. On n’est jamais ni sot ni méprisable quand on a le talent de mériter l’estime et l’affection de ses semblables. L’homme simple, vertueux et modeste, peut compter sur une bienveillance plus durable que celui qui ne plaît que par des saillies passagères et qui, plus souvent encore, se rend désagréable par son orgueil ou sa malignité. L’homme véritablement éclairé est celui qui connaît et qui suit les moyens nécessaires pour être constamment aimé. Tout homme qui croit se faire estimer par des moyens faits pour déplaire, est un ignorant, un étourdi, un sot.\par
Le ridicule consiste dans le peu de proportion entre les moyens et le but qu’on se propose. Tourner le dos à l’objet que l’on veut obtenir constitue évidemment l’ignorance, le ridicule et la sottise. N’est-ce pas être bien ignorant que de ne point savoir que la crainte n’attire pas la tendresse, que l’arrogance indispose, que la jactance et la fatuité se punissent par le ridicule ? Combien de gens dans le monde, dont l’objet est de se faire admirer et considérer, et qui par leur conduite insensée ne parviennent qu’à se faire haïr et mépriser ? Voilà ce que produisent leurs airs de hauteur, leurs manières impertinentes, leurs prétentions mal fondées, leur faste et leurs dépenses qu’ils ne peuvent soutenir, leur ton décisif sur des matières qu’ils n’entendent pas.\par
En regardant la chose de près, on trouvera toujours que l’orgueil et la vanité sont des preuves indubitables de sottise ; ils montrent une parfaite ignorance de la route qu’il faut tenir pour gagner la bienveillance et l’estime des hommes. Un esprit stupide et borné qui se tient humblement dans sa sphère est beaucoup moins ridicule ou méprisable que l’homme à prétentions qui se réjouit à ses dépens. En morale il n’est point de maladie plus incurable que celle d’un ignorant présomptueux ou d’un sot qui a le malheur d’être content de lui-même. Le premier pas vers la sociabilité est de connaître ce qui nous manque et de nous corriger de nos défauts.\par
Un être vraiment sociable ne doit jamais perdre ses associés de vue. Les distractions, l’étourderie, les folies et les fautes sont toujours punies, soit par l’indignation ou la haine, soit par le mépris et le ridicule. On craint le {\itshape ridicule} parce qu’il suppose le mépris. Or le mépris est révoltant pour un être amoureux de lui-même. L’homme raisonnable écarte de sa conduite tout ce qui peut le faire mépriser avec justice parce qu’alors il serait forcé de ratifier le jugement des autres, mais il brave le ridicule qui, dans un monde vicié, tombe souvent sur le mérite et la vertu.\par
En effet, si le ridicule consiste à choquer l’opinion et la mode, qui très communément tiennent lieu de la décence et de la raison, il est clair qu’une conduite sage et réglée doit souvent paraître singulière et bizarre dans une société frivole ou corrompue. Voilà pourquoi l’on voit parfois la vertu, la probité, la pudeur, l’équité même, exposées aux sarcasmes du vice : il croit se disculper en se moquant des qualités qui le forceraient à rougir. Dans le monde, la vertu ressemble souvent à la dame d’Horace qui danse en rougissant au milieu des satyres impudents\phantomsection
\label{footnote36}\footnote{Horace, {\itshape Art poétique}, vers 233.}.\par
Les vertus les plus respectables peuvent être quelquefois exposées aux impertinences de la raillerie et aux traits du ridicule mais, assuré de sa propre dignité, l’homme de bien méprise ces fléaux si redoutables pour les gens du monde, ces idoles imaginaires auxquelles on les voit sacrifier leur fortune, leur conscience et leur vie. Une crainte puérile de l’opinion met très souvent des obstacles insurmontables à la vertu ; cette vaine terreur fait que contre sa conscience, contre ses propres lumières, on suit le torrent du monde, on fait {\itshape comme les autres} et l’on se livre au mal sans pouvoir s’arrêter. Les hommes les plus éclairés se rendent quelquefois les esclaves de l’usage et vivent dans une lutte perpétuelle avec leur propre raison. {\itshape Le déshonorant}, dit un moraliste célèbre, {\itshape offense moins que le ridicule}.\par
La {\itshape raillerie}, presque toujours armée par l’envie et la malignité, déconcerte souvent la sagesse et la probité, mais elle n’a de prise réelle que sur le vice ; elle finit par se déshonorer lorsqu’elle attaque la vertu. Il faut de la force pour oser être vertueux dans les nations où le vice,\par
tout fier du nombre et du rang de ses adhérents, pousse l’impudence jusqu’à vouloir se moquer des qualités devant lesquelles il devrait baisser les yeux.\par
Tout railleur est un homme vain et méchant. La raillerie suppose toujours le dessein de blesser plus ou moins celui sur qui on l’exerce ; elle renferme le reproche de quelques défauts que l’on expose à la risée. Une dame célèbre a dit avec raison « que les personnes qui ont le besoin de médire et qui aiment à railler ont une malignité secrète dans le cœur. De la plus douce raillerie à l’offense, il n’y a qu’un pas à faire. Souvent le faux ami, abusant du droit de plaisanter, vous blesse ; mais la personne que vous attaquez a seule droit de juger si vous plaisantez. Dès qu’on la blesse, elle n’est plus raillée : elle est offensée\footnote{Madame de Lambert.} ». « La raillerie, disait un Ancien, est comme le sel, qu’il ne faut employer qu’avec précaution. »\par
La raillerie est presque toujours une arme dangereuse et ses traits sont quelquefois plus cruels et plus insupportables qu’une injure. Railler celui que l’on appelle son ami, c’est se déshonorer par une véritable trahison, c’est l’immoler à des indifférents, c’est montrer qu’on l’aime beaucoup moins qu’un bon mot. Railler les indifférents, c’est s’exposer follement à leur ressentiment, c’est provoquer gratuitement leur mauvaise humeur. Railler ses supérieurs serait une folie dont on craindrait d’être châtié. La raillerie ne peut donc impunément s’exercer que sur les amis, et pour lors elle est une perfidie, ou sur les inférieurs et les malheureux, ce qui est une lâcheté détestable.\par
Cependant, rien de plus commun que cette cruauté. Les hommes ne se plaisent pour l’ordinaire à railler que ceux qu’ils devraient et plaindre et consoler. Ils versent à pleines mains le ridicule et les sarcasmes sur des gens dont les infortunes ou les défauts devraient exciter la pitié. Un homme est-il contrefait ? A-t-il l’esprit borné ? A-t-il commis quelque bévue ? Est-il nécessiteux et forcé de tout endurer ? Aussitôt il est en butte à des railleries continuelles ; il devient le jouet de la société, il essuie les piqûres d’une foule de lâches qui cherchent à briller à ses dépens et qui lui font sentir le poids de leur supériorité. Il n’est personne qui ne se croit en droit d’insulter les misérables.\par
Ces dispositions se trouvent surtout dans les enfants, toujours très prompts à saisir les défauts, les infirmités, les faiblesses, les ridicules des personnes qui s’offrent à leur vue. On les rencontre encore dans ceux en qui l’éducation et la réflexion n’ont pas fait disparaître ce penchant inhumain. Les gens du peuple exercent communément les saillies de leur esprit inculte contre ceux qui découvrent quelque disgrâce naturelle. Les enfants et les gens du peuple, comme on l’a fait voir ailleurs, sont communément cruels.\par
Rien de plus commun que de voir les hommes rire des accidents et des malheurs qu’ils voient arriver aux autres. Ce sentiment odieux paraît venir de la comparaison avantageuse pour soi que l’on fait de sa propre sécurité, de ses propres perfections, avec la situation fâcheuse ou les défauts des autres. L’homme, d’après sa nature toute brute et sans culture, est si peu un être doué de compassion et de pitié que, si son cœur n’a pas été convenablement modifié, il est tenté de se réjouir du mal de ses semblables parce que ce mal l’avertit qu’il est bien lui-même. Quand il ne réfléchit pas, il ne songe nullement qu’il est exposé aux accidents dont il voit les autres affligés, et qu’il est très odieux de rire de leurs malheurs, de leurs défauts, de leurs faiblesses. C’est ainsi que l’homme borné devient communément le jouet de l’homme plus favorisé du côté de l’esprit ; celui-ci, gonflé de l’idée des avantages qu’il possède, ne voit pas qu’il est injuste et cruel pour un être qui devrait exciter la pitié.\par
Les hommes ne devraient jamais oublier qu’ils se doivent des égards. Les gens d’esprit surtout devraient s’observer encore plus que les autres et craindre de blesser. La vivacité de l’esprit, la chaleur de l’imagination, la gaieté produisent souvent une ivresse, une pétulance, contre lesquelles il est bon de se mettre en garde. Les gens d’esprit, en vertu de la supériorité qu’ils se sentent sur les autres, sont ordinairement tentés de s’en prévaloir contre ceux qu’ils trouvent moins heureux du côté des facultés intellectuelles. Voilà, sans doute, ce qui fait souvent regarder les gens de lettres comme des êtres dangereux à fréquenter.\par
L’ironie sanglante, des plaisanteries offensantes ne peuvent plaire qu’à des envieux, à des méchants, dont tout homme d’un vrai mérite ne peut point ambitionner les suffrages : ce sont des lâchetés puisqu’elles attaquent communément des personnes incapables de se défendre. Rien de plus barbare et de plus lâche que la plaisanterie ou l’ironie dans la bouche d’un prince ; elle imprime quelquefois des taches ineffaçables et suffit pour anéantir le bonheur de toute la vie.\par
Tout homme assez vain, assez inconsidéré pour offenser par ses bons mots ou par ses plaisanteries non seulement un ami mais des indifférents, n’est pas fait pour être admis dans des sociétés honnêtes dont les membres doivent se respecter les uns les autres. Les railleurs, les plaisants de profession, les diseurs de bons mots, les bouffons sont quelquefois des gens d’esprit dont la malignité amuse ; mais on les trouve rarement estimables par les qualités du cœur, bien plus importantes dans le commerce de la vie que ces saillies dont souvent on fait tant de cas dans le monde. « Défiez-vous, dit Horace, de celui qui médit de son ami absent, de celui qui ne le défend pas quand on l’accuse, de celui qui cherche à faire rire par ses bons mots : il possède à coup sûr une âme dépravée\phantomsection
\label{footnote37}\footnote{Horace, {\itshape Satires}, livre I, 4, vers 81 et suivants.}. »\par
Cependant, l’inattention, la légèreté, le défaut de réflexion contribuent autant que le mauvais cœur à la raillerie, qui ne peut être approuvée ou tolérée que lorsque sans blesser celui même qui s’en trouve l’objet, elle ne sert qu’à l’animer et répandre une vivacité agréable dans la conversation. Une vie vraiment sociable exige que personne ne quitte ses associés mécontent de lui-même ou des autres.\par
La raillerie, le ridicule, la plaisanterie ne sont utiles et louables que lorsqu’ils s’exercent en général sur les vices régnant dans la société, dont ils peuvent quelquefois réprimer l’impudence ou la folie. Quoi de plus ridicule, de plus digne d’exercer la satire que la vanité de tant d’hommes et de femmes gravement occupés de riens pompeux, de parures, de bijoux, de modes bizarres, d’ajustements ? Sont-ce donc des hommes ou des enfants que ces êtres frivoles dont la tête n’est remplie que de jouets dont ils se dégoûtent à tout moment ? Est-il au monde un être plus risible qu’un fat qui ne se présente dans la société que pour lui montrer sa sottise, son impertinence, son carrosse, son habit ? Peut-on considérer sans rire les prétentions d’une coquette surannée qui jusqu’au tombeau affecte les airs évaporés, la parure et l’étourderie de la jeunesse ? Verra-t-on sans pitié la vanité bourgeoise et maladroite de tant de gens du commun qui ont la folie de croire qu’ils copient la grandeur par leur impertinence ? Quoi de plus fatigant qu’un discoureur insipide qui s’empare de la conversation pour étourdir par son caquet importun ? Est-il rien de plus méprisable que l’arrogance de tant d’importants qui jugent et raisonnent de tout sans se connaître à rien ? L’homme sensé peut-il voir sans dégoût ces oisifs insupportables pour eux-mêmes qui vont périodiquement promener de cercles en cercles leur ennui et leur inutilité ? De quel œil peut-on voir ces fâcheux, ces misanthropes pétris de fiel et d’envie qui ne sortent de leur tanière que pour répandre au dehors leur humeur incommode ? Est-il rien de plus propre à bannir la gaieté, l’harmonie sociable, que ces esprits contredisants qui se font un principe de n’être jamais de l’avis de personne ? Est-il un objet plus digne de la satire que ce jeu continuel fait pour suppléer à la stérilité des conversations de tant d’êtres qui s’ennuient parce qu’ils n’ont rien à dire ?\par
Mais le sage dont le cœur est sensible est bien plus porté à jouer le rôle d’Héraclite que celui de Démocrite dans la société. Ces travers et ces folies cessent d’être ridicules à ses yeux et lui paraissent déplorables quand il voit que des puérilités deviennent, chez les êtres frivoles qu’elles occupent uniquement, la source des crimes les plus destructeurs, des injustices les plus criantes, des querelles les plus tragiques. On gémit et l’on cesse de rire en voyant que de vains titres, des préséances, des places, des rubans, des jouets excitent l’ambition et font éclore les intrigues, les menées sourdes, les perfidies et les crimes de tant de grands enfants qui d’abord ne paraissent que ridicules. Il faut verser des larmes quand on voit qu’un sot orgueil déguisé sous le nom d’honneur fait chaque jour répandre le sang de ces méchants enfants, qui cessent alors d’être divertissants. On doit éprouver une indignation profonde en voyant que ce faste impertinent par lequel tant de gens se distinguent, est cause de la ruine d’une foule de malheureux dont l’industrie et le travail ne sont point payés. On gémit quand on réfléchit que ce jeu fait pour délasser les fainéants absorbe quelquefois les plus amples fortunes. Enfin on ne rit plus de ces galanteries indécentes qui troublent pour toujours l’harmonie, la confiance et l’estime si nécessaires au maintien de la paix domestique.\par
Les faiblesses, les défauts, les extravagances des hommes les conduisent souvent au crime et à l’infortune. Il n’est point de vice qui ne se punisse lui-même\phantomsection
\label{footnote38}\footnote{Sénèque.} et qui, tôt ou tard, ne produise dans la société des ravages qu’une âme sensible est forcée de déplorer.\par
Plaignons donc les mortels de leurs égarements, suites nécessaires de leur étourderie, de leur inexpérience, des fausses idées qu’il se font du bonheur, des routes trompeuses qu’ils prennent pour y parvenir. Vivre avec des hommes, c’est vivre avec des êtres dont la plupart sont faibles, aveugles, imprudents ; les haïr, ce serait joindre l’injustice à l’inhumanité, ce serait se tourmenter sans profit pour les autres. Fuir les hommes, ce serait se priver des avantages de la vie sociale qui, malgré ses défauts, nous offre encore des charmes. Nul homme n’est gratuitement méchant. Il ne commet le mal que parce qu’il en attend quelque bien, il est méchant parce qu’il est ignorant, dépourvu de réflexion, peu prévoyant sur les effets nécessaires de ses actions. Détester les hommes pour leurs faiblesses et leurs vices, ce serait les détester parce qu’ils sont dignes de la pitié la plus tendre.\par
Aimons donc nos semblables, afin d’attirer leur amour. Ne les fuyons pas si nous pouvons leur prêter des secours. Ne les révoltons point par une humeur atrabilaire ; invitons-les à la vertu en leur montrant ses charmes, détournons-les du vice en dévoilant sa difformité. N’insultons pas à leurs misères, invinciblement liées aux préjugés de toute espèce qu’ils ont puisé dès l’enfance dans la coupe de l’erreur. Ne les désespérons pas en déclarant que leurs maux sont sans remède et qu’ils sont condamnés à languir toujours. Consolons-les plutôt par l’espoir de voir cesser leurs peines ; montrons-leur dans les progrès de la raison et dans la vérité l’antidote du poison dont les esprits sont infectés. Qu’ils entrevoient des temps plus propices où les nations mûries par l’expérience renonceront enfin à leurs cruelles folies et placeront la vertu dans un temple qui lui appartient : c’est alors qu’elle établira l’harmonie sociable en inspirant un esprit de paix à tous les peuples du monde, en réunissant d’intérêts les nations et leurs chefs, en confondant le bonheur du citoyen avec celui de la patrie, en faisant sentir à chaque membre de la société que son bien-être est lié à celui de ses semblables et que jamais il ne doit s’en séparer.\par
S’il n’était point permis de se livrer à des espérances si vastes et si flatteuses, qu’il le soit au moins de croire que des principes puisés dans la nature de l’homme seront adoptés par quelques êtres pensants à qui tout prouvera que la vertu est la seule base de la félicité publique et particulière, tandis que le vice anéantit chaque jour le bien-être des nations, des familles, des individus. Telles sont les vérités que nous tenterons de développer de plus en plus dans la suite de cet ouvrage où l’on trouvera l’application de nos principes aux hommes considérés dans leurs états divers.
\section[{Section IV. Pratique de la Morale. Morale des Peuples, des Souverains, des Grands, des Riches, etc., ou devoirs de la vie publique et des différents états}]{Section IV. Pratique de la Morale. Morale des Peuples, des Souverains, des Grands, des Riches, etc., ou devoirs de la vie publique et des différents états}\renewcommand{\leftmark}{Section IV. Pratique de la Morale. Morale des Peuples, des Souverains, des Grands, des Riches, etc., ou devoirs de la vie publique et des différents états}

\subsection[{Chapitre I. Du Droit des Gens ou de la morale des nations et de leurs devoirs réciproques}]{Chapitre I. Du Droit des Gens ou de la morale des nations et de leurs devoirs réciproques}
\noindent Nous avons tâché jusqu’ici d’établir les principes de la morale sur la nature de l’homme ; en donnant l’analyse et la définition des vertus et des vices, nous avons fait sentir les avantages inestimables des unes et les conséquences déplorables des autres. Cet examen nous a mis à portée de découvrir les motifs naturels les plus capables d’exciter les hommes au bien et de les détourner du mal, et ces motifs se sont trouvés fondés sur leurs propres intérêts. Enfin, nous avons fait connaître la nature et le but de la vie sociale et les devoirs qu’elle impose. Appliquons maintenant les faits ou les expériences morales que nous avons recueillies aux différentes sociétés dont la terre est peuplée. Considérons les devoirs de l’homme dans ses états divers ou sous les rapports variés qu’il peut avoir avec les êtres de son espèce. Commençons par examiner les devoirs réciproques des nations qui se sont partagé les différentes contrées de notre globe.\par
Le genre humain entier forme une vaste société dont les nations diverses sont les membres répandus sur la face de la terre, éclairés, échauffés par le même soleil, entourés par les eaux des mêmes océans, conformés de la même manière, sujets aux mêmes besoins, formant les mêmes désirs, occupés du soin de se conserver, de se procurer le bien-être et d’écarter la douleur. La Nature ayant rendu semblables à ces égards tous les citoyens du monde, il s’ensuit que la conformité de leur essence les rapproche, met des rapports entre eux, fait qu’ils agissent de même et que leurs actions ont une influence nécessaire sur leur existence, sur leur bonheur ou leur malheur réciproques.\par
De ces principes incontestables il faudra nécessairement conclure que les peuples sont liés à d’autres peuples par les mêmes liens, par les mêmes intérêts que chaque homme dans une nation ou une société particulière est lié à chacun de ses concitoyens ; conséquemment, chaque nation doit observer envers les autres nations les mêmes devoirs, les mêmes règles que la vie sociale prescrit à chaque individu envers les membres d’une société particulière. Une nation est obligée, pour son propre intérêt, de pratiquer les mêmes vertus que tout home doit montrer à son semblable, fût-il étranger ou inconnu. Un peuple doit la justice à un autre peuple, c’est-à-dire est obligé de respecter ses droits, ses possessions, sa liberté, son bien-être, par la même raison que tout peuple veut qu’on respecte ces choses dont il jouit lui-même. Si, comme on l’a suffisamment prouvé, la justice est la source commune de toutes les vertus sociales, il s’ensuit nécessairement qu’elle prescrit à chaque peuple de prêter aux autres peuples les secours de l’humanité, de leur montrer de la bienveillance, de la compassion dans leurs calamités, de la protection dans leur faiblesse, de la reconnaissance pour leurs services, de la sincérité et de la fidélité dans les conventions réciproques ou traités. Il s’ensuit encore des mêmes principes que, pour entretenir l’union et la paix si utiles à la félicité mutuelle des nations, un peuple, en vue de ces avantages, doit montrer de la générosité aux autres peuples, sacrifier à la concorde et à la gloire une portion même de ses droits, ne point faire sentir aux autres le poids de son orgueil et de sa supériorité ; enfin, il ne doit pas manquer aux égards que des citoyens du monde sont en droit d’exiger les uns des autres.\par
Des peuples limitrophes se doivent évidemment les bons offices et l’assistance que se doivent réciproquement des voisins dans une même cité. Les peuples alliés, c’est-à-dire que des intérêts communs unissent plus intimement, sont des amis et doivent dès lors observer les devoirs toujours sacrés de l’amitié. Les nations éloignées les unes des autres se doivent au moins réciproquement l’équité et l’humanité, que nul habitant sur terre n’a le droit de méconnaître. Les nations en guerre doivent, pour leur intérêt propre, mettre à leur haine, à leur colère et à leurs vengeances les bornes fixées par l’équité, par la juste défense de soi, par l’humanité, par la pitié, toujours faites pour reprendre leurs droits sur les hommes raisonnables et pour les attendrir sur le sort des malheureux.\par
Tels sont évidemment les devoirs que la Nature impose aux nations comme à tous les autres hommes. Tels sont les principes du {\itshape droit des gens}, qui n’est au fond que la morale des peuples. Faute de faire attention à des vérités si claires, on a cru que la morale, destinée à régler les actions des particuliers, n’était point faite pour les peuples ou pour les chefs qui les représentent. On a prétendu que les souverains et les États étaient toujours dans un {\itshape état de nature} que l’on a constamment opposé à l’{\itshape état social}.\par
Mais cet état de nature est visiblement une chimère, une abstraction toute pure. Il exista toujours une famille qui en se multipliant fit éclore plusieurs familles ou sociétés, d’où naquirent des nations qui se choisirent des souverains. Jamais, comme on l’a prouvé, l’homme ne fut isolé sur la terre. Dès qu’il y eut plusieurs familles, sociétés ou nations, il s’établit entre elles des rapports plus ou moins intimes, en raison de leurs positions et de leurs besoins réciproques. Ces rapports et ces besoins produisirent des devoirs, dont l’assemblage est l’objet de la morale.\par
D’ailleurs, si la morale doit se fonder sur la nature de l’homme, elle doit convenir à l’homme dans son état de nature, et par conséquent elle est faite pour régler la conduite des nations, même dans l’état de nature où l’on suppose qu’elles sont restées. Ainsi, sous quelque point de vue que l’on envisage les hommes, soit qu’on les voit partagés en grandes ou en petites masses, ils sont toujours sous l’empire de la morale. Les mêmes règles sont faites pour les obliger tous ; ils seront soumis aux mêmes devoirs, ils seront forcés de s’y conformer sous peine d’encourir tôt ou tard les châtiments attachés par la nature même des choses à la violation de ses lois.\par
Les hommes, soit séparés, soit en masse, dans tous les temps et dans tous les lieux, sont les mêmes. Les nations sont susceptibles des mêmes passions et tourmentées des mêmes vices que les individus : elles ne sont, en effet, que des amas d’individus. Les mœurs nationales, les usages, bons ou mauvais, les opinions vraies ou fausses des peuples ne sont jamais que des résultats soit de l’ignorance, soit de la raison plus ou moins exercée du plus grand nombre de ceux dont un corps politique est composé. Un peuple n’est guerrier que parce que les passions du plus grand nombre sont tournées vers la guerre. Un peuple n’est commerçant que parce que les désirs du grand nombre sont tournés vers les richesses que le commerce procure. Un peuple est fier parce que tous les citoyens s’enorgueillissent de leurs succès, de leur bonne fortune, de leurs richesses, etc. Un peuple est injuste, inhumain, sanguinaire, parce que les hommes qui le composent sont élevés et nourris dans des principes insociables.\par
Ce sont communément les législateurs et les chefs des peuples qui fomentent entre eux les passions, les goûts, les vices, les préjugés et les folies dont on les voit tourmentés. Le brigand Romulus rassembla de tous côtés des brigands ; ceux-ci formèrent, pour le malheur de la terre, une race de brigands ou de guerriers qui ne connurent d’autre vertu, d’autre honneur, d’autre gloire que d’opprimer ou de vaincre tous les peuples du monde. L’ambitieux Mahomet fait d’une troupe d’Arabes des forcenés qui se font un principe religieux de conquérir et de répandre les rêveries du Coran.\par
La gloire attachée dans presque tous les pays à la conquête, à la guerre, à la bravoure, est visiblement un reste des mœurs sauvages qui subsistaient chez toutes les nations avant qu’elles fussent civilisées. Il n’est guère de peuples qui soient encore détrompés de ce préjugé si fatal au repos de l’univers. Les sociétés même qui devraient sentir le mieux les avantages de la paix admirent les grands exploits, attachent une idée noble au métier de la guerre et n’ont pas pour les injustices et les forfaits qu’elle entraîne toute l’horreur qu’ils mériteraient.\par
Qu’est-ce en effet que faire la guerre, (excepté dans le cas d’une juste défense), sinon la violation la plus crainte des droits les plus sains de la justice et de l’humanité ? Si un assassin, un voleur, un brigand, paraissent des hommes détestables, quelle indignation ne devrait pas exciter dans tous les cœurs un peuple conquérant qui, pour satisfaire son ambition, pour augmenter ses domaines, pour assouvir son avarice, sa vengeance et sa rage, et quelquefois pour contenter les caprices de sa vanité, fait périr des millions d’hommes, inonde les campagnes de sang, réduit les villes en cendres, ravage en un instant les espérances du laboureur et, placé insolemment sur les débris des nations et des trônes, s’applaudit de ses crimes, se glorifie des maux sans nombre qu’il a fait souffrir au genre humain. « Pendant la guerre, dit Thucydide, l’avarice se réveille, la justice est terrassée, la violence et la force règnent, la débauche se donne un libre essor, le pouvoir est entre les mains des plus méchants des hommes, les bons sont opprimés, l’innocence est écrasée, les filles et les femmes sont déshonorées, les contrées sont ravagées, les maisons sont brûlées, les temples sont détruits, les tombeaux sont violés… Enfin, la famine et la peste suivent constamment les pas de la guerre. »\par
Tels sont les jeux qui servent d’amusement à des peuples forcenés, guidés par des chefs dépourvus de justice et d’entrailles. Si quelque chose semble devoir rabaisser l’homme au-dessous de la bête, c’est sans doute la guerre. Les lions et les tigres ne combattent que pour satisfaire leur faim. L’homme est le seul animal qui de gaieté de cœur et sans cause, vole à la destruction de ses semblables et se félicite d’en avoir beaucoup exterminé. Pendant la longue durée de la République romaine, il serait très difficile, peut-être, de trouver une seule guerre légitime : si le Romain féroce fut attaqué par d’autres peuples, ce fut communément pour le punir de quelque entreprise injuste dont il s’était lui-même rendu coupable le premier.\par
Mais la Nature prend soin de châtier tôt ou tard ces peuples odieux qui se déclarent les ennemis du genre humain : forcés d’acheter leurs conquêtes et leurs victoires par leur propre sang, ils s’affaiblissent nécessairement, les richesses amassées par la guerre les corrompent et les divisent\footnote{« La luxure s’est ruée sur nous et venge l’univers asservi. » Juvénal, {\itshape Satires}, VI, vers 292.}. Des guerres civiles vengent les nations opprimées ; le peuple ennemi de tous les peuples est assailli de toutes parts, son empire devient la proie de cent nations barbares dont ses violences avaient provoqué la colère. Telle fut la destinée de Rome qui, après avoir dépouillé, ravagé, désolé le monde connu, devint enfin la proie des Goths, des Visigoths, des Hérules, des Lombards, etc.\par
D’ailleurs, un peuple continuellement en armes ne peut jouir longtemps ni d’un bon gouvernement, ni d’un bonheur véritable et permanent. La guerre amène toujours la licence : les lois se taisent au bruit des armes. Des soldats insolents croient qu’elles ne sont pas faites pour eux\footnote{« Votre ville, disait Numa aux Romains, est si accoutumée aux armes et tellement enflée de ses succès, qu’on voit bien qu’elle ne veut que s’agrandir et commander aux autres. Il serait donc ridicule de vouloir enseigner à servir les dieux, à aimer la justice, à haïr la violence et la guerre, à un peuple qui demande bien plus à suivre un général qu’à obéir à un roi. » Voyez Plutarque, {\itshape Vie de Numa Pompilius}.} : les chefs se divisent, se combattent, se rendent maîtres de l’État affaibli par d’affreuse convulsions, le vainqueur, croyant assurer sa conquête, devient tyran ; ainsi, le despotisme achève de ruiner jusque dans ses fondements la félicité publique. Il anéantit tout d’un coup la justice, la liberté, les lois. Tel est communément l’écueil où vont échouer les États qui se sont enivrés de la vanité des conquêtes ! C’est ainsi que par leurs guerres injustes tous les grands peuples de la terre n’ont eu que la gloire fatale de se détruire successivement !\par
Un peuple toujours en guerre ne peut être ni libre ni bien gouverné ? « Mars, dit le poète Timothée, est le tyran, mais le droit est le souverain du monde. » Un peuple sans cesse armé est un furieux qui tôt ou tard tourne sa rage contre lui-même. Il n’est point de nation qui n’ait le plus grand intérêt au maintien de l’ordre, de la justice, de la paix\footnote{Plutarque appelle {\itshape divin} l’amour que Nicias avait pour la paix. Voyez la {\itshape Vie de Nicias}. Voyez idem dans la {\itshape Vie de Démétrius}. hommes est la première des vertus que l’on devrait enseigner aux souverains ou les forcer de pratiquer.}. Les guerres fréquentes sont incompatibles avec la population, l’agriculture, le commerce, l’industrie, les arts utiles, qui seuls peuvent rendre les États fortunés. La guerre, par les dépenses qu’elle exige, accable et décourage le citoyen laborieux, s’oppose à son activité, met des entraves au négoce, dépeuple les campagnes et ruine communément un royaume pour conquérir une forteresse ou une province qu’elle commence ordinairement par ravager avant d’en prendre possession. « J’aime mieux, disait Marc Aurèle, conserver un seul citoyen que de détruire mille ennemis. » L’économie du sang des hommes est la première des vertus que l’on devrait enseigner aux souverains ou les forcer de pratiquer.\par
Si nous consultons les annales du monde, nous verrons que la guerre fut de tout temps le principe de la ruine des empires les plus formidables et qui paraissaient pouvoir se flatter de la plus longue durée. Les États les plus vastes ne procurent à ceux qui se sont injustement agrandis que le funeste avantage d’avoir perpétuellement à combattre de nouveaux ennemis, des voisins alarmés par les projets des conquérants ambitieux. Aucun pays n’améliorera son sort par les plus vastes conquêtes ; le plus grand État est communément le plus mal gouverné. En étendant leurs limites, jamais les rois n’ont augmenté ni leur puissance réelle, ni le bonheur des peuples. « Les longues guerres, dit Xénophon, ne se terminent jamais que par le malheur des deux partis. » Agésilas, à la vue de la guerre du Péloponnèse, si fatale à tous les Grecs, s’écria : « Ô malheureuse Grèce ! Qui a fait périr elle-même autant de ses citoyens qu’il en eût fallu pour vaincre tous les barbares\footnote{Voyez Plutarque, {\itshape Dits notables des Princes}.} ! »\par
Les nations belliqueuses ont la folie de sacrifier ce qu’elles possèdent à l’espoir incertain de dominer, de jouer un grand rôle, de s’agrandir. Les plus vastes monarchies formées par des guerres et des victoires se sont affaissées sous le poids de leur propre grandeur. En un mot, sous quelque point de vue que l’on envisage la guerre, elle est une calamité pour ceux mêmes qui la font avec le plus de succès. « Le vaincu se désole, et déjà son vainqueur n’est plus\footnote{« Flet victus, et victor interiti. » Érasme, {\itshape Apophtegmes}. Plutarque attribue la décadence de Sparte à la passion de s’agrandir et de dominer sur la Grèce. Il ajoute que Lycurgue était bien persuadé qu’une ville qui veut être heureuse n’a pas besoin de conquêtes. Voyez Plutarque, {\itshape Vie d’Agésilas}.}. » Un empire peut-il jouir d’une vraie prospérité quand son ambition est cause que tous les citoyens gémissent dans la misère ou se font égorger pour étendre ses bornes ?\par
Quoique les princes et les peuples ne semblent pas être jusqu’ici revenus de la fureur qui les pousse à la guerre, l’humanité pourtant a depuis quelques siècles fait des progrès relativement à la façon de la faire. Autrefois les peuples féroces exterminaient sans pitié les vaincus qui tombaient entre leurs mains, ou du moins leur faisait subir le joug d’un esclavage souvent plus cruel que la mort. Aujourd’hui la voix sainte de l’humanité se fait entendre même au milieu des combats. Des mœurs plus douces ont fait abolir l’esclavage, l’on est parvenu à sentir qu’un ennemi était un homme et que pour acquérir le droit d’être humainement traité dans les revers de la fortune, il fallait épargner les vaincus. « C’est être, dit Tite-Live, une bête féroce et non pas un homme, que de croire que la guerre n’a pas des droits comme la paix\footnote{Tite-Live, {\itshape Histoire romaine}.}. »\par
Les injustices de la guerre et les malheurs qui l’accompagnent ne sont-ils donc pas assez terribles pour que les hommes reconnaissent la nécessité de mettre quelques bornes à leurs fureurs ? Ils écoutent à quelques égards la Nature qui leur crie qu’il y a de l’infamie à exercer sa cruauté contre un ennemi qui ne peut plus nuire et qui rend les armes.\par
Lassés enfin de leurs cruautés, de leurs crimes et de leurs folies, les peuples terminent leurs guerres par des traités que l’on doit regarder comme des contrats ou des engagements réciproques. L’équité, la bonne foi, la raison devraient concourir à faire respecter ces conventions solennelles dans lesquelles communément les parties contractantes prennent le Ciel à témoin de leurs promesses ; mais le Ciel n’est pas capable d’en imposer à des hommes dépourvus d’équité. Ces traités communément imposés par la force à la faiblesse abattue ou surpris par la ruse, sont presque à tout moment éludés ou rompus. N’en soyons point surpris ; la violence, la fraude, la mauvaise foi président pour l’ordinaire à tous les engagements faits par des êtres dépourvus de droiture, et souvent la justice est forcée d’approuver la rupture des liens formés par l’iniquité. Il n’y a que des hommes équitables et traitant de bonne foi qui puissent acquérir des droits que la justice rende inviolables et sacrés\footnote{Plutarque, dans la {\itshape Vie de Pyrrhus}, en parlant des politiques injustes, dit : « La guerre et la paix, ces noms si respectables, sont pour eux deux sortes de monnaies dont ils se servent pour leurs intérêts et jamais pour la justice. Encore sont-ils plus louables quand ils font une guerre ouverte que quand il déguisent sous les saints noms de justice, d’amitié, de paix, ce qui n’est qu’une trêve d’injustices et de crimes. »}.\par
Cette ambition si vaine et si fière ne rougit donc souvent pas de recourir en lâche au mensonge et à la fraude pour parvenir à ses fins ! Le parjure, la perfidie, la trahison paraissent des moyens honorables aux grandes âmes de ces héros qui marchent à la gloire ! Ne le croyons pas ; les peuples et les rois se déshonorent lorsqu’ils manquent à la bonne foi. Les fourbes découverts finissent par ne plus tromper ; ils laissent à leurs noms des tâches ineffaçables aux yeux de la postérité. La meilleure politique pour les princes et les peuples, ainsi que pour les particuliers, sera toujours d’être vrais. Mais pour être sincère et vrai, il faut être équitable. L’iniquité fut et sera toujours obligée de suivre des routes obliques et ténébreuses, incompatibles avec la droiture et la sincérité. Quiconque a des projets déshonnêtes est forcé d’employer la ruse, de se cacher avec soin et de recourir bassement à la fraude, au mensonge, à la supercherie.\par
Parmi les passions dont les peuples se trouvent agités comme les particuliers, l’on doit compter l’avarice, la cupidité, qui souvent les mettent aux prises. Nous voyons des nations éprises de cette passion abjecte former le projet ridicule, impraticable, injuste, d’attirer dans leurs mains le commerce exclusif du monde. Polybe observe avec grande raison que {\itshape dans les États maritimes et livrés au commerce, rien ne paraît honteux quand il donne du profit}, principe capable d’anéantir les mœurs et la probité, principe qui doit rendre chaque citoyen ou injuste ou avare, et qui dispose les âmes à la vénalité. D’ailleurs, la cupidité des peuples semble perpétuellement se punir elle-même et frustrer ses propres vues. Des guerres entreprises à tout moment pour augmenter la masse des richesses nationales font réellement disparaître celles qui étaient acquises pour en obtenir d’imaginaires. Un peuple avare sacrifie continuellement son bien-être, son repos, son aisance à l’espoir de s’enrichir. Il se met dans l’indigence pour parvenir à l’opulence\footnote{Voici comment un orateur moderne fait le tableau allégorique de la politique actuelle : « Un colosse sans proportions dans son énorme stature ; sa tête excessive, s’élève fièrement sur un corps desséché… Ses pieds s’appuient sur les deux mondes : sa main droite est armée d’une épée, et dans sa gauche elle tient la plume de la finance et la balance du commerce : impétueuse et sensible, un souffle l’agite et la met en convulsion ; toutes les parties de la terre tremblent sous ses moindres mouvements. Cependant froide dans sa fureur et méthodique dans ses violences, elle calcule en combattant. Elle évalue les hommes avec des monnaies et pèse le sang avec des marchandises. » Voyez {\itshape Discours sur les Mœurs}, par M. Servan.}.\par
D’ailleurs, cette opulence ne tarde pas à conduire la nation à sa ruine. Elle amène le luxe, qui traîne toujours à sa suite la mollesse, la débauche, les vices de toute espèce. L’avidité fut et sera toujours le principe de la destruction des empires. {\itshape Un État est malheureux quand il renferme des citoyens trop riches ou trop avides de richesses}\footnote{Cette pensée est d’Avidius Cassius. Elle est rapportée par Vulcatius Gallicanus in {\itshape Vita Cassii}, chap. 13. Voyez {\itshape Hist. aug. script.}, tome I, édition Lugd. Batav. 1671.}. Platon refusa de donner des lois aux Cyrénéens parce qu’ils étaient trop riches. Les Arcadiens et les Thébains ayant demandé une législation à ce même philosophe, il voulut établir chez eux une plus grande égalité ; mais comme les riches refusèrent d’y consentir, il les abandonna à leur mauvais sort, à leurs discussions intestines, à leurs vices. Un gouvernement montre des signes indubitables d’imprudence et de folie lorsqu’il inspire à ses sujets une passion forte pour les richesses, dont la nature est d’absorber bientôt toutes les autres et de faire disparaître toutes les vertus nécessaires à la société.\par
Ainsi, les nations, de même que les individus, portent la peine des passions dont elles se laissent aveugler ! Concluons donc que la modération, la tempérance sont aussi nécessaires à la conservation et à la félicité permanente des peuples et des empires qu’à celles des particuliers. Concluons que la morale est faite pour guider les souverains et les nations. Concluons enfin que jamais la politique ne peut impunément séparer ses intérêts de ceux de la vertu, toujours utile aux hommes, sous quelque face qu’on les considère.\par
Ainsi, je le répète, la morale est la même pour tous les habitants du monde ; les peuples sont obligés d’observer ses devoirs les uns envers les autres, ils ne peuvent les violer sans se nuire à eux-mêmes. La politique extérieure, pour être saine, ne doit être que la morale appliquée à la conduite des nations. « La politique, dit très bien le savant traducteur de Plutarque, n’est digne de louange que lorsqu’elle est employée par la justice pour obtenir un but louable\phantomsection
\label{footnote39}\footnote{Voyez Dacier, {\itshape Comparaison d’Alexandre et de César}, pag. 316. Il dit ailleurs : « La saine politique enseigne qu’il vaut mieux gagner les hommes par la bonne foi que de s’en rendre maître par les armes. » Voyez {\itshape idem, Comparaison de Phocion et de Caton}, page 551, tome 6.}. »\par
Si la raison pouvait se faire entendre des peuples ou de ceux qui dirigent leurs mouvements, elle leur dirait d’être justes, de jouir eux-mêmes et de laisser jouir en paix les autres du sol et des avantages que le destin leur accorde ; de renoncer pour toujours à ces conquêtes criminelles qui attirent aux conquérants la haine du genre humain, de maudire ces guerriers qui rassemblent à la fois tous les fléaux dont les hommes puissent être accablés ; de ne recourir du moins à ces moyens terribles que lorsqu’ils sont indispensablement nécessaires à leur conservation, à leur sûreté, à leur bonheur réel ; de gémir de ces victoires sanglantes qui s’achètent aux dépens du sang, des richesses et du bien-être de la patrie ; de réunir leurs forces pour réprimer les projets de ces peuples remuants ou de ces rois ambitieux qui ne trouvent la gloire qu’à troubler la tranquillité des autres ; de chérir la paix, sans laquelle nul État ne peut être florissant et fortuné, de sacrifier de bon cœur à ce bien si désirable des intérêts frivoles, toujours indignes de lui être comparés ; d’agir avec franchise, de respecter le bonne foi qui seule peut faire naître et maintenir la confiance, de renoncer aux détours d’une politique tortueuse, également pénible et déshonorante pour les souverains et les peuples et qui ne sert le plus souvent qu’à éterniser leurs sanglants démêlés ; d’étouffer pour toujours ces haines nationales si contraires aux droits saints de l’humanité, à cette bienveillance universelle que doivent se montrer les êtres de la même espèce ; de contenir dans de justes bornes l’amour de la patrie, qui devient un attentat contre le genre humain dès qu’il rend injuste et cruel ; de cultiver chez eux les mœurs, l’agriculture, les arts utiles et agréables à la vie, d’y faire fleurir un commerce raisonnable, de se défendre d’une avidité inquiète et toujours insatiable ; et surtout de se garantir des effets destructeurs du luxe, qui anéantit constamment l’amour du bien public et de la vertu pour élever sur ses ruines les vices, la vénalité, l’injustice, la rapine, la dissolution, l’indifférence pour la félicité générale, en un mot : les dispositions les plus contraires au bonheur de la société.\par
Telles sont en peu de mots les vérités et les leçons que la morale enseigne à toutes les nations de la terre. Tels sont les principes de la vraie politique, qui n’est que l’art de rendre les hommes heureux. Ils sont connus et sentis par tous les princes éclairés ; tout leur prouve que leurs intérêts réels, leur gloire véritable, leur vraie grandeur, leur conservation propre et leur sûreté, sont inséparablement attachés au bien-être et aux vertus des peuples.\par
On nous parle sans cesse de la {\itshape gloire des nations}, de {\itshape l’honneur des couronnes}. Cette gloire ne peut consister que dans un gouvernement qui rende les peuples fortunés, dans la félicité publique. Cet honneur consiste à mériter l’estime des autres nations.\par
Les peuples se déshonorent et se rendent coupables aux yeux des autres peuples par les mêmes crimes et les mêmes actions qui rendent les individus odieux ou méprisables. Les attentats, les perfidies, les iniquités des souverains retombent presque toujours sur les nations que l’on regarde comme complices des excès auxquels on ne les voit pas refuser de se prêter. Voilà comme des peuples entiers acquièrent souvent la réputation d’être turbulents, inhumains, fourbes et sans foi ; ils perdent la confiance et s’attirent l’indignation, la haine, la fureur des autres sociétés. Un gouvernement qui manque à ses engagements, qui viole ses promesses, soit envers ses sujets, soit envers les étrangers, ne diffère en rien d’un banqueroutier frauduleux ou d’un prodigue insensé et fripon qui ruine ses créanciers : il anéantit son crédit, il se prive de ressources, il autorise la fraude et la mauvaise foi de ses sujets, il les rend suspects les uns aux autres et méprisables aux yeux de tous les peuples du monde. C’est des souverains que dépend la bonne ou mauvaise renommée des nations, qui devraient être infiniment jalouses de leur honneur et de leur vraie gloire, auxquels tous les citoyens sont fortement intéressés. Les peuples, ainsi que les particuliers, font consister leur grandeur et leur gloire dans le pouvoir de nuire, de faire la loi aux autres, de rassembler une grande masse de richesses, d’être injustes impunément ; en un mot, l’orgueil national consiste dans une sotte vanité, tandis qu’il devrait consister dans l’équité, dans la probité, dans un gouvernement sage qui procurerait le bonheur et la liberté sans lesquels un peuple n’a aucune raison pour s’enorgueillir ou se préférer à d’autres\footnote{Agésilas ayant entendu nommer le roi de Perse, {\itshape le grand roi} : « Eh ! comment, s’écria-t-il, serait-il plus grand que moi s’il n’est pas plus juste et plus vertueux ? » Voyez Plutarque, {\itshape Dits notables des Lacédémoniens}.}.\par
Les hommes approuvent sans examen et par habitude, ou cherchent à imiter ce qu’ils ont dès leur enfance entendu louer et célébrer. Telle est la source ordinaire des préjugés nationaux dont le vulgaire est imbu et dont les personnes les plus sages ont souvent de la peine à se défaire totalement. Rien de plus propre à corrompre l’esprit et le cœur des princes et des peuples que la vénération peu raisonnée que l’on inspire communément à la jeunesse pour les grands hommes, les guerriers, les conquérants de l’Antiquité, qui trop souvent méconnurent tous les principes de la morale. Des instituteurs imprudents ne parlent qu’avec emphase des Grecs et des Romains, qu’on vous fait regarder comme des modèles de sagesse, de vertu, de politique. L’on apprend dès l’âge le plus tendre à révérer comme des vertus le courage bouillant, la férocité barbare, les attentats heureux soit des héros fabuleux chantés par les poètes, soit des grands capitaines qui ont subjugué des nations et rendu leurs nations fameuses. On représente comme des hommes divins et rares des Lacédémoniens farouches, injustes, sanguinaires, des Athéniens souvent souillés de crimes, et surtout des Romains toujours prêts à violer les droits les plus saints de l’humanité et à sacrifier tous les habitants de la terre à l’insatiable patrie qui leur commandait des forfaits. Grâce à ces instructions fatales, les hommes s’accoutument à respecter la violence, l’injustice et la fraude dès qu’elles sont utiles à leur pays. Les souverains se croient assez grands quand ils sont assez forts pour commettre de grands crimes à la face de l’univers. Les peuples s’imaginent être couverts de gloire quand ils ont été les instruments abjects des iniquités de leurs chefs, qui bientôt deviennent leurs tyrans. D’après ces idées, il n’est presque personne qui n’admire ou ne justifie le Macédonien furieux dont la témérité criminelle renversa le trône des Perses, on révère les Émiles, on est saisi de vénération au seul nom du destructeur de Carthage, on applaudit dans un César le génie et les travaux qui, après avoir arrosé les Gaules de sang, le mirent en état d’enchaîner ses concitoyens.\par
C’est ainsi que dans les souverains et les sujets l’on voit se perpétuer l’ambition, la passion de jouer un grand rôle, la fureur de faire trembler ses voisins, la folie des conquêtes. Les exemples de tant de prétendus héros font éclore de siècle en siècle des insensés et des pervers qui communiquent leur frénésie à leurs peuples imprudents et qui, sûrs d’être applaudis, s’illustrent par des forfaits que l’on appelle {\itshape exploits}. Encouragés par les éloges des poètes et d’un vulgaire imbécile, les princes ne se croient puissants que pour avoir fait beaucoup de mal au genre humain, et les peuples se croient estimables quand ils ont eu l’honneur de seconder avec courage leurs infâmes projets. La grandeur, dans l’opinion de la plupart des hommes, consiste dans le funeste avantage de faire bien des malheureux.\par
Loin de nous faire admirer des peuples destructeurs qui ont ravagé la terre, l’Histoire devrait montrer que les nations injustes n’ont jamais travaillé qu’à se forger des fers : les conquêtes font des tyrans, jamais elles n’ont fait des peuples fortunés. Des lois sages appuyées par la volonté constante des nations devraient pour toujours lier les mains de ces potentats fougueux qui, peu capables de s’occuper du bien-être de leurs propres sujets, ne songent qu’à faire sentir leurs coups à leurs voisins. Pour être grand et respectable, un peuple doit être heureux. Ni ses armées, ni ses richesses, ni l’étendue de ses provinces ne lui procureront une vraie félicité, qui ne peut être que l’effet de ses vertus. Une nation sera toujours puissante et respectée lorsqu’elle sera composée de citoyens réunis sous des chefs vertueux. Une nation guerrière, turbulente, avide du bien des autres, devient l’objet de la haine universelle et finit tôt ou tard par succomber sous les efforts des ennemis qu’elle s’est fait.
\subsection[{Chapitre II. Devoirs des Souverains}]{Chapitre II. Devoirs des Souverains}
\noindent Gouverner les hommes, c’est avoir le droit d’employer les forces remises par la société dans les mains d’une ou de plusieurs personnes pour obliger tous ses membres à se conformer aux devoirs de la morale. Ces devoirs, comme nous l’avons prouvé ci-devant, sont contenus dans le pacte social par lequel chacun des associés s’engage à être juste, à respecter les droits des autres, à leur prêter les secours dont il est capable, à concourir de toutes ses forces à la conservation du corps, sous la condition qu’en échange de son obéissance et de sa fidélité à remplir ses devoirs, la société lui accordera protection pour sa personne et pour les biens que son industrie et son travail ont pu légitimement lui procurer. D’après les principes répandus dans cet ouvrage, il est évident que ce pacte renferme tous les devoirs de la morale puisqu’il engage chaque citoyen à se conformer aux règles de l’équité qui est la base de toutes les vertus sociales et à s’abstenir de tous les crimes ou vices qui sont, comme on a vu, des violations plus ou moins marquées de ce contrat fait pour lier tous les membres de la société. Mais comme les passions des hommes leur font souvent perdre de vue leurs engagements, ou comme leur légèreté leur fait souvent oublier que leur bien-être propre est lié à celui de leurs associés, il fallut dans chaque société une force toujours subsistante qui veillât sur tous les membres du corps politique et qui fût capable de les ramener sans cesse à l’observation des devoirs qu’ils semblent méconnaître. Cette force se nomme {\itshape gouvernement} ; l’on peut le définir la force de la société destinée à obliger ses membres à remplir les engagements du pacte social. C’est par le moyen des lois que le gouvernement exprime la volonté générale et prescrit aux citoyens les règles qu’ils doivent suivre pour la conservation, la tranquillité, l’harmonie de la société. L’autorité du gouvernement est juste parce qu’elle a pour objet de procurer à tous les membres de la société des avantages que leurs désirs inconsidérés, leurs intérêts mal entendus et discordants, leur inexpérience et leur faiblesse les empêcheraient d’obtenir par eux-mêmes. Si tous les hommes étaient éclairés ou raisonnables, ils n’auraient aucun besoin d’être gouvernés, mais comme ils ignorent ou semblent méconnaître et le but qu’ils doivent se proposer et les moyens d’y parvenir, il faut que le gouvernement, en leur présentant la raison publique exprimée par la loi, les remette dans la voie dont ils pourraient s’écarter. « Le magistrat, dit Cicéron, est une loi parlante\footnote{« L’on peut dire avec vérité que le magistrat est la loi parlante, et la loi, le magistrat muet. » Cicéron, {\itshape De Legibus}, Livre III, chap. 1.}. »\par
D’après leurs circonstances variées et leurs besoins divers, les nations ont donné des formes différentes à leurs gouvernements. Les unes ont remis l’autorité publique entre les mains d’un seul homme, et ce gouvernement s’est appelé {\itshape monarchique}. Les autres ont déposé le pouvoir de la société entre les mains d’un nombre plus ou moins grand de citoyens distingués par leurs vertus, leurs talents, leurs richesses, leur naissance, et ce gouvernement se nomme {\itshape aristocratique}. D’autres ont conservé l’autorité toute entière ; alors le peuple se gouverna lui-même ou du moins par des magistrats de son choix : ce gouvernement fut nommé {\itshape démocratique}. D’autres nations ont fait un mélange de ces différentes manières de gouverner. Elles ont cru trouver des avantages à combiner ensemble les trois formes de gouvernement dont on vient de parler : ce mélange produisit ce qu’on appelle un gouvernement {\itshape mixte.} L’on nomme gouvernement {\itshape absolu} celui dont la nation n’a point limité les droits par des conventions expresses. L’on appelle {\itshape limité} celui dont l’autorité est resserrée par des règles expresses imposées par la nation à ceux qui la gouvernent. Les dépositaires de l’autorité sociale se nomment {\itshape souverains}, quelle que soit la forme du gouvernement adoptée par une société.\par
Des spéculateurs ont longtemps et vainement disputé pour savoir quelle était la meilleure forme de gouvernement, c’est-à-dire la plus conforme au bien des sociétés, la plus capable de procurer le bonheur aux nations. Mais le but de tout gouvernement est toujours le même : il ne peut être que la conservation et la félicité de la société gouvernée. Ses droits sont toujours les mêmes, quelque forme qu’on lui donne puisqu’il n’y a que l’équité qui puisse conférer des droits réels et valables. Son autorité, soit qu’elle ait des limites prescrites, soit qu’on ait oublié de lui fixer des bornes, est toujours également tempérée ou limitée par l’avantage qu’elle doit procurer à la société sur laquelle on l’exerce. Une autorité exercée sans profit pour la société ou qui serait contraire à ses intérêts et à sa volonté, changerait de nature et ne serait plus qu’une usurpation manifeste, une tyrannie à laquelle la société ne pourrait être soumise que par la violence, qui jamais ne peut donner des droits. Toutes les formes de gouvernement sont bonnes quand elles sont conformes à l’équité. Tout souverain exerce une autorité légitime quand, se conformant au but invariable de la société, il observe religieusement lui-même et fait observer à tous les citoyens, sans distinction, les engagements du pacte social dont il est le gardien et le dépositaire. Le souverain absolu peut faire tout ce qu’il veut, mais il ne doit rien vouloir que de conforme au bien de la société, dont le salut est la loi primitive et fondamentale que la Nature impose à tous ceux qui gouvernent les hommes. « La bonne cité, dit Plutarque, est celle où les bons commandent et où les méchants n’ont aucune autorité. »\par
« Jupiter même, dit ailleurs ce philosophe, ne peut bien gouverner sans justice. » Cependant, l’on a souvent disputé et l’on dispute encore pour savoir si le souverain absolu doit être soumis aux lois, s’il est lié par les engagements du contrat social qui servent à lier tous les membres du corps politique. Mais comment des êtres raisonnables ont-ils pu sérieusement disputer pour savoir si le souverain, uniquement destiné à servir la justice, à conserver les droits de chacun et de tous, à veiller incessamment au bien public, était tenu d’être juste et de remplir les conditions qui, quand même elles n’auraient jamais été exprimées, sont évidemment renfermées dans le pouvoir qu’il exerce dans la société ? A-t-on pu de bonne foi douter qu’un souverain, le chef d’une nation, fût lié au corps politique dont il est la tête ? puisse se passer du tronc ou des membres et ne ressente pas les coups dont ils sont affectés ? Peut-on mettre en problème si des hommes rassemblés par leurs besoins naturels pour jouir en sûreté des avantages de la vie sociale, pour être garantis des passions de leurs semblables, ont jamais pu accorder à leurs chefs le droit d’anéantir pour eux tous les biens en vue desquels ils vivent en société ? Enfin, les nations ont-elles pu sans folie conférer à celui ou à ceux qu’elles ont rendu dépositaires de leurs droits, le droit de les rendre constamment malheureux ? « La juridiction, dit Montaigne, ne se donne point en faveur du judiciant, c’est en faveur du juridicié\footnote{\noindent Voyez les {\itshape Essais} de Montaigne, livre III, chap. 6.\par
« Que ceux-là donc qui élèvent l’autorité des souverains jusque-là, qu’ils osent dire qu’ils n’ont d’autre juge que Dieu, quelque chose qu’ils fassent, me montrent qu’il y ait jamais eu de nation qui sciemment et sans crainte ou force se soit oubliée jusqu’à se soumettre à la volonté de quelque souverain sans cette condition expresse et tacitement entendue d’être justement et équitablement gouvernée… Quand même un peuple sciemment et de son plein gré a consenti à une chose qui de soi-même est manifestement irréligieuse et contre le droit naturel, une telle obligation ne peut valoir… Certainement ce serait une chose trop inique de n’accorder à toute une nation ce que l’équité octroie aux personnes particulières comme aux mineurs, aux femmes, à ceux qui ont le sens blessé, à ceux qui sont trompés de plus de la moitié du juste prix, principalement s’il appert de la mauvaise foi de celui auquel de telles personnes se seraient obligées… Les peuples sont-ils esclaves ? Par le droit romain l’esclave auquel étant malade n’aura été pourvu par son seigneur, est tenu pour affranchi… Certainement ce qu’ils allèguent qu’un roi n’est astreint aux lois, ne doit ni ne peut être généralement entendu, ainsi que chantent les flatteurs des rois et ruineurs de royaumes… Il s’ensuit nécessairement : ou que les rois ne sont pas hommes, ou qu’ils sont obligés aux lois divines et humaines ou naturelles. » Voyez le livre du {\itshape Droit des Magistrats sur les Sujets}, publié en 1550.
}. »\par
Ainsi, sous quelque point de vue que l’on envisage l’autorité souveraine, elle est toujours soumise aux lois immuables de l’équité ; destinée à les maintenir, elle ne peut les enfreindre sans dégénérer en tyrannie. Les lois qu’elle prescrit doivent être justes, conformes à la nature de l’homme en société ; les lois positives ne peuvent jamais être opposées aux lois de la Nature. Elles ne doivent être que ces lois appliquées aux besoins, aux circonstances, aux intérêts particuliers des peuples à qui elles sont destinées. Elles ne peuvent en aucun cas heurter de front la félicité publique qu’elles sont faites pour assurer. De là découlent évidemment tous les devoirs des souverains.\par
On a vu dans le chapitre qui précède les devoirs des peuples et de leurs chefs envers les autres peuples. Nous allons maintenant jeter un coup d’œil rapide sur les devoirs de ces chefs envers les nations qu’ils gouvernent, et tout nous prouvera que la morale prescrit aux princes les mêmes règles, les mêmes devoirs qu’aux membres les plus obscurs de la société, que l’autorité suprême ne fait qu’étendre ces devoirs indispensables à un plus grand nombre d’objets. Si chaque citoyen, dans la sphère étroite qui l’entoure, est obligé pour son propre intérêt de montrer des vertus, le souverain est obligé, dans la vaste sphère où il agit, de déployer avec plus d’énergie les vertus de son état. Ses actions influent non seulement sur sa nation mais encore sur les autres peuples de la terre. Les crimes et les vices du particulier ont des effets bornés, au lieu que les vices et les défauts des princes produisent l’infortune et des hommes qui vivent et des races futures. De mauvaise lois, des résolutions imprudentes, des démarches précipitées sont très souvent suivies de malheurs qui se transmettent à la postérité la plus reculée.\par
« La vertu, dit Confucius, doit être commune au laboureur et au monarque. » La vertu primitive et fondamentale du souverain comme du citoyen doit être la justice. Elle suffit pour lui montrer tous ses devoirs et lui tracer la route qu’il doit suivre. La justice des rois ne diffère de celle du citoyen que parce qu’elle s’étend plus loin. Le souverain a des rapports non seulement avec son propre peuple, mais encore avec les autres peuples de la terre. Son ambition, réglée par la justice, se trouve satisfaite dès qu’il commande à des sujets heureux ; il ne cherche point à s’emparer des provinces des autres parce qu’il trouve qu’un prince est assez grand quand il règne sur une nation qui lui est bien attachée. Le monarque humain et juste frémit au seul nom de la guerre parce que, même accompagnée des plus brillants succès, elle n’est propre qu’à ruiner et dépeupler un État. Il est fidèle à ses traités parce que l’équité, la bonne foi lui donneront de l’ascendant sur des politiques fourbes dont l’univers entier devient bientôt l’ennemi. Le bon prince est pacifique parce que c’est dans la paix qu’il peut travailler librement au bonheur des citoyens.\par
C’est au sein de la tranquillité que le souverain vraiment grand peut montrer sa sagesse, ses talents, son génie : semblable à l’astre du jour dont les rayons éclairent et fécondent tout le globe, le prince juste vivifie tous les corps, les familles, les individus de la société ; d’une main ferme il tient la balance entre tous ses sujets. La prévention, la faveur, l’amitié, la pitié même ne l’empêchent nullement de maintenir invariablement les règles de l’équité, qui place sur une même ligne et le fort et le faible, le grand et le petit, le riche et l’indigent. La bienfaisance et la sensibilité du prince ne s’arrêtent point à des individus : elles embrassent l’ensemble de l’État, le peuple tout entier.\par
Sa pitié l’attendrit non sur les plaintes de la cupidité qui le trompe, mais sur la misère plus réelle d’une foule qu’il ne voit pas et sur les larmes des malheureux que souvent on s’efforce de cacher à ses regards.\par
Une justice inébranlable constitue seule la bienfaisance et la pitié d’un monarque aux yeux duquel tout son peuple doit être toujours présent. Il est sûr que les riches et les grands se feront jour pour parvenir aux pieds du trône, mais il craint de ne point entendre les cris de l’innocent et du pauvre.\par
Les droits, la liberté, les biens, les intérêts de tous lui paraissent plus respectables que les prétentions et les demandes des courtisans qui l’entourent. Il n’accorde à personne le droit funeste d’opprimer parce qu’il sait qu’il ne pourrait sans crime se l’attribuer à lui-même : il sait qu’il est le défenseur, et non le propriétaire des biens de ses sujets ; il sait qu’un impôt est un vol quand il n’a pas pour objet la conservation de l’État. Il sait qu’une loi, qu’un édit ne rendront point légitimes une violation manifeste des droits du citoyen. Il reconnaît que les trésors de l’État sont à l’État et ne peuvent sans prévarication être consacrés à ses propres plaisirs. Il sait que son temps même n’est plus à lui mais appartient à son peuple auquel il doit tous ses soins. Il se reprocherait comme des crimes une vie molle, indolente, dissipée, et des amusements ruineux pour son pays. Il sait que la vie d’un souverain est pénible et laborieuse, et ne doit point être uniquement destinée aux plaisirs. Il s’abstient surtout de ceux qui tendraient évidemment à corrompre les mœurs de son peuple, parce qu’il sait qu’un peuple sans mœurs ne peut pas être bien gouverné. Il sait enfin qu’il est responsable de la conduite de ceux sur qui il se décharge des détails de l’administration, que leurs crimes deviendraient les siens et qu’il souffrirait lui-même de leurs négligences. Il met donc au néant ces privilèges injustes qui élèvent des favoris au-dessus des lois et qui leur permettent d’employer leur crédit et leur force pour écraser l’innocence. Il ne croit pas que tout son peuple a tort quand il se plaint des oppression d’un vizir. Sa faveur disparaît dès qu’il s’agit de la justice, ou plutôt sa faveur et ses bienfaits sont guidés par cette justice même, qui lui montre les citoyens les plus utiles, les plus vertueux, les plus distingués par leur mérite, comme seuls dignes des récompenses, des emplois et des grâces.\par
Quiconque ose troubler par ses crimes la félicité publique, quelque rang qu’il occupe, est abandonné à la sévérité des lois. Quiconque se déshonore par ses actions est puni par la disgrâce. Quiconque remplit négligemment les devoirs de son état est privé de sa place, que l’équité n’adjuge qu’à des sujets capables de la remplir dignement.\par
Enfin, un souverain inviolablement attaché à la justice corrige à tout moment le vice en lui montrant un front sévère, et fortifie la vertu en l’appelant aux honneurs.\par
La morale sera toujours inutile tant que les leçons ne seront point appuyées par l’exemple et la volonté des souverains\phantomsection
\label{footnote40}\footnote{« Rex velit honesta, nemo non eadem volet. » Sénèque, in {\itshape Thyeste}.}. Les peuples seront corrompus tant que les chefs qui règlent leurs destinées ne sentiront pas l’intérêt qu’ils ont d’être eux-mêmes vertueux. C’est en vain que la religion menacera les mortels de la colère du Ciel pour les détourner de leurs vices et de leurs méchancetés. C’est en vain qu’elle leur promettra les récompenses ineffables d’une autre vie pour les inviter à la vertu : la voix puissante des rois, les récompenses et les châtiments de la vie présente seront toujours les moyens les plus efficaces pour faire agir des êtres occupés de leurs intérêts actuels et qui ne songent que faiblement à leur sort futur. La morale la plus démontrée peut bien convaincre les esprits d’un petit nombre de penseurs, mais elle n’influera sur les actions de tout un peuple que lorsqu’elle aura reçu la sanction de l’autorité suprême.\par
Tout prince ami de la justice peut, même sans effort, rappeler ses sujets à leurs devoirs, les leur faire pratiquer avec joie, encourager le mérite et les talents, réformer les mœurs. Les hommes attachent un si haut prix à la faveur de leurs maîtres, ils sont si troublés de l’idée de leur déplaire, on les voit tellement empressés à mériter leur bienveillance, que la vertu du prince suffit pour faire régner en peu de temps la vertu dans son empire et pour établir avec elle la félicité publique qui en sera toujours la compagne inséparable.\par
Tel est le but que paraît se proposer un monarque, jeune encore, que le destin favorable vient, pour le bonheur de ses sujets, de placer sur le trône de ses pères. Plein de sagesse dans l’âge de la dissipation et des plaisirs, ce prince a déjà porté les regards sur les mœurs si longtemps méprisés. Pénétré des sentiments de l’équité, son cœur a déjà fait éclater le désintéressement, la fidélité dans les engagements, le désir de soulager un peuple malheureux. Ennemi de l’oppression, il a banni de sa présence les instruments détestés du despotisme, les auteurs des calamités publiques ; désabusé des subtilités du luxe, il a montré son aversion pour ce mal si dangereux dans un État. Enfin, l’aurore d’un nouveau règne semble promettre à tout un peuple engourdi dans de longues ténèbres le jour le plus serein.\par
Reçois, ô {\scshape Louis XVI} ! l’hommage pur et désintéressé d’un inconnu qui te révère. Continue, prince vraiment bon, de mériter la tendresse d’un peuple sociable, docile, soumis même sous l’autorité la plus dure. Que par tes mains généreuses les fers du despotisme soient brisées. Que les portes de ces prisons, tant de fois le séjour de l’innocence opprimée, soient à jamais fermées. Après avoir rétabli la justice dans son sanctuaire, anéantis ces lois barbares, cette jurisprudence obscure et tortueuse, ces formes arbitraires, ces coutumes souvent contraires à la Nature et désolantes pour les sujets. Deviens le législateur d’un grand peuple, sois le restaurateur d’une nation illustre, le réformateur de ses mœurs, le créateur de sa félicité. Réprime la tyrannie du crédit et de la puissance, la rapacité de l’exacteur, les cabales et les querelles du fanatisme, les excès de l’opulence, les folies d’un luxe destructeur, les impudences de la débauche. Fais succéder à la licence une liberté légitime aussi utile au souverain qu’aux sujets. Établis pour tous les citoyens la sûreté qui met le pauvre à couvert de toute violence. La pauvre est ton sujet ; c’est lui qui travaille et pour toi, et pour les grands qui t’environnent. Le pauvre a le plus de droit à ta justice, à ta protection, à ta bonté ; ainsi, juste toi-même, ô prince, ne permets pas qu’aucun des tiens soit opprimé. Que tes regards courroucés repoussent les courtisans pervers, l’homme injuste, le flatteur désintéressé, le délateur odieux, le débauché qui se dégrade, le dissipateur inconsidéré, le débiteur qui retient le salaire du citoyen, l’insensé qui se dérange par une vanité ruineuse. Punis le crime par la loi, dans quelque rang qu’il se trouve. Montre du mépris au vice, récompense le mérite, les talents, la vertu. Appelle-les à tes conseils auprès de ta personne ; ainsi tu seras vraiment grand et puissant ; ton peuple sera florissant et tu seras cher à tes sujets, respecté de tes voisins, admiré de la postérité.\par
Si cette conduite d’un sage monarque déplaît à quelques courtisans pervers, à quelques grands orgueilleux, à quelques hommes corrompus qui désirent profiter des vices et des faiblesses de leurs maîtres, elle excitera l’enthousiasme d’un peuple entier qui ne cessera de bénir un souverain dont les bienfaits se feront sentir à toute la société. Un tel prince deviendra l’idole des citoyens ; son nom ne sera prononcé qu’avec les transports de la tendresse, chacun de ses sujets le regardera comme son protecteur et son père. Il vivra sous leurs yeux comme au sein de sa famille. Ses jours précieux seront défendus par sa nation intéressée à conserver en lui le gage de son bonheur. Agasiclès, roi de Sparte, disait « qu’un roi n’avait pas besoin de gardes quand il gouvernait ses sujets comme un père gouverne ses enfants ». Pline dit à Trajan « qu’un prince n’est jamais plus fidèlement gardé que par son innocence et sa vertu ».\par
Un souverain bienfaisant ou bon n’est pas celui qui prodigue sans choix les trésors de l’État sur la troupe affamée dont il est entouré ; un prince clément n’est pas celui qui pardonne les attentats commis contre son peuple ; un monarque débonnaire n’est pas celui qui répand des grâces sur des courtisans et des favoris sans mérite : c’est celui qui récompense justement le mérite. Un prince, lorsqu’il est juste, n’accorde point de grâces ou de faveurs gratuites : tous ses bienfaits ne sont que des actes d’équité par lesquels il paie les avantages qu’on procure à sa nation, au nom et aux dépens de laquelle les dignités, les pensions, les honneurs, se distribuent. Un souverain digne d’amour n’est pas un homme facile, une dupe qui se laisse guider en aveugle par ses favoris ou ses ministres ; un potentat respectable n’est pas celui qui se distingue par une étiquette orgueilleuse, par des dépenses énormes, par un luxe effréné, par des édifices somptueux.\par
Le souverain vraiment bon est celui qui est bon pour tout son peuple, qui respecte ses droits, qui se sert de ses trésors avec économie pour exciter le mérite ou les talents nécessaires au bonheur de l’État. Un prince clément pour les coupables est cruel pour la société. Un Ancien disait {\itshape que c’est perdre les bons que de pardonner aux méchants}. Un souverain qui se laisse guider par des courtisans flatteurs ne connaît jamais la vérité et souffre que l’on rende ses sujets malheureux. Un monarque orgueilleux qui ne fait consister la gloire que dans un vain appareil, dans ses prodigalités ruineuses, dans une magnificence sans bornes, dans des plaisirs coûteux, dans des conquêtes, est un souverain dont l’âme rétrécie ne connaît pas la gloire que la vertu seule peut décerner. « Il est, dit Pline à Trajan, bien plus honorable pour la mémoire d’un prince de passer à la postérité pour avoir été bon que pour avoir été heureux. » Un prince peut-il se croire heureux lorsque ses sujets sont plongés dans la misère ? Un souverain ne peut être puissant et fortuné que lorsqu’il fondera sa grandeur et sa puissance sur la liberté et le bonheur de son peuple.\par
En voyant la conduite de la plupart des princes, on dirait que leur état ne les oblige à rien. On croirait qu’ils ne sont sur la terre que pour la ravager, l’asservir, dévorer les peuples ou pour s’amuser sans cesse, sans rien faire d’utile pour les nations. Est-ce donc régner que d’abandonner les rênes de l’empire à quelques favoris, tandis que celui qui devrait gouverner vit dans une honteuse oisiveté ou ne pense qu’à faire diversion à ses ennuis par des plaisirs souvent honteux, par des fêtes ruineuses, par des édifices inutiles qui coûtent des larmes à tout un peuple occupé à repaître les vices et la vanité d’un chef peu disposé à rien faire pour lui ?\par
Une sotte vanité serait-elle faite pour entrer dans le cœur d’un monarque ? Un sentiment si petit ne serait-il pas déplacé dans une âme vraiment noble ? La vraie grandeur des rois consiste dans la félicité des peuples, leur vraie puissance dans l’attachement de ces peuples, leur vraie richesse dans l’aisance et l’activité de leurs sujets, leur vraie magnificence dans l’abondance qu’ils font régner. C’est dans les cœurs des nations que les princes doivent s’ériger des monuments, bien plus flatteurs et plus dignes d’admiration que ces bâtiments superbes faits aux dépens de la félicité nationale : les pyramides de l’Égypte qui subsistent encore, les monuments de Babylone qui ne subsistent plus, les palais ruinés des tyrans de Rome ne retracent à l’esprit que la folie de ceux qui les ont élevés. Montaigne a dit avec très grande raison « que c’est une espèce de pusillanimité aux monarques, et un témoignage de ne point assez sentir ce qu’ils font, de travailler à se faire valoir par des dépenses excessives\footnote{Voyez {\itshape Essais}, livre 3, ch. 6.} ». « Le plus grand roi, dit Zoroastre, est celui qui rend la terre plus fertile\footnote{Voyez {\itshape Zend-avesta}, ou le livre sacré des Parsis.}. »\par
Ceux qui sont chargés de l’éducation des princes, au lieu de leur montrer la gloire dans la guerre, dans d’injustes conquêtes, dans un faste éblouissant, dans des dépenses frivoles, devraient les habituer dès l’enfance à combattre leurs passions et leurs caprices, et leur proposer la conquête de leurs sujets comme l’objet vers lequel tous leurs vœux doivent se porter.\par
Au lieu d’endurcir les princes, au lieu de leur apprendre à mépriser les hommes, leurs instituteurs devraient remuer leur imagination par la peinture touchante des misères auxquelles tant de millions de leurs semblables sont condamnés pour les faire vivre eux-mêmes dans le luxe et la splendeur. Les peuples et leurs maîtres seraient bien plus heureux si, au lieu de persuader à ceux-ci qu’ils sont des dieux ou des êtres d’un ordre supérieur au reste des mortels, on leur répétait sans cesse qu’ils sont des hommes et que sans ce peuple méprisé ils seraient eux-mêmes très malheureux.\par
Carnéade disait que « les enfants des princes n’apprennent rien avec plus de soin que l’art de monter à cheval, parce qu’en toute autre étude chacun leur cède, au lieu qu’un cheval n’est point courtisan : il renverse par terre le fils d’un roi comme celui d’un paysan ». L’empereur Sigismond disait que « tout le monde refusait d’exercer un métier qu’il n’avait point appris, et qu’il n’y avait que le métier de roi, le plus difficile de tous, que l’on exerçât sans y être formé ». Cependant, le grand Cyrus reconnaissait qu’il n’appartient à nul homme de commander s’il n’est meilleur que ceux à qui il commande\footnote{Voyez Plutarque dans les {\itshape Dits notables des Princes}. Il dit ailleurs que « gouverner un État et être philosophe est la même chose ». Pittacus disait « qu’il était difficile de commander et d’être homme de bien ».}.\par
« Ne fais pas le prince, dit Solon, si tu n’as pas appris à l’être. Apprends à te gouverner avant de gouverner les autres. »\par
L’éducation des enfants des rois, bien loin de les éclairer et de leur donner des entrailles, semble se proposer d’étouffer en eux les germes de la justice et de l’humanité. On ne leur parle que de combats, de conquêtes, on ne les entretient que de leur propre grandeur et du néant des autres ; on leur montre les peuples comme des vils troupeaux dont ils peuvent disposer à leur gré et qu’ils ont droit de dépouiller et dévorer. On leur dit qu’ils doivent fermer l’oreille à leurs plaintes importunes et toujours destituées de raison.\par
Voilà pourquoi les princes sont rarement équitables ou pourvus d’un cœur sensible. C’est ainsi qu’on en fait des idoles inaccessibles à leurs sujets, sur lesquels, à leur insu, l’on exerce les plus étranges cruautés. C’est ainsi qu’on en fait des ingrats qui sans cesse refusent au mérite ses justes récompenses pour les prodiguer à la bassesse et à la flatterie. Enfin, c’est ainsi qu’au sein des plaisirs, de la pompe et des fêtes, les souverains sont dans une ivresse continuelle ou s’endorment dans une sécurité fatale qui les conduit tôt ou tard à une perte certaine\phantomsection
\label{footnote41}\footnote{Lorsque Lucullus combattit contre Mithridate, les généraux de ce monarque lui laissèrent ignorer que l’armée où il se trouvait en personne souffrait la disette la plus cruelle. — Le premier qui annonça au roi Tigrane l’approche du même Lucullus eût la tête tranchée par ordre de ce prince. Voyez Plutarque dans la {\itshape Vie de Lucullus}.}.\par
La Nature, toujours juste dans ses châtiments, n’épargne aucun de ceux qui méconnaissent ses lois. Les mauvais rois rendent leurs sujets malheureux et les malheurs des sujets retombent nécessairement sur leurs injustes maîtres. Les provinces épuisées par des guerres inutiles n’offrent que des cultivateurs découragés par la rigueur des impôts. Le commerce disparaît par les entraves dont il est continuellement accablé. Un gouvernement négligent finit toujours par des violences et dégénère en tyrannie. Les fantaisies du souverain deviennent inépuisables parce que, faute de s’occuper de ses devoirs, il a besoin de plaisirs et d’amusements continuels ; les besoins et les demandes du prince augmentent dans la même progression que sa nation s’épuise et que ses moyens diminuent. Les impôts sont redoublés à mesure que les peuples deviennent plus pauvres ; enfin, l’on a recours à mille extorsions, à la perfidie, à la fraude pour achever de ruiner un État obéré par un gouvernement en délire.\par
Ainsi, le despote, devenu lui-même plus misérable et plus affamé, ne connaît plus de frein. Il écrase les lois sous le poids de ses volontés arbitraires, et bientôt il ne règne que sur des esclaves sans activité et sans industrie. La conscience tourmente alors le tyran sur son trône : il sait qu’il a mérité la haine universelle, il craint tous les regards, il voit des ennemis dans tous ceux qui l’approchent, il a peur de son peuple dont il a rebuté la tendresse. Inquiet et malheureux, il devient ombrageux et bientôt inhumain et cruel.\par
Enfin, la tyrannie parvenue à son comble produit des soulèvements, des révoltes, des révolutions, dont le tyran est la première victime. De l’esclavage au désespoir il n’y a souvent qu’un pas. Un despote est un souverain qui met sa volonté propre à la place de l’équité, son intérêt personnel à la place de l’intérêt de la société. Un souverain de cette trempe a la folie de croire que lui seul fait l’État, que sa nation n’est rien, que la société toute entière n’est destinée par le Ciel qu’à servir ses fantaisies. Le tyran est le souverain qui met en pratique les principes du despote et qui, croyant se rendre heureux lui seul, rend tout son peuple malheureux.\par
Mais se rend-il en effet heureux lui-même ? Non, il est rempli de trouble et d’inquiétudes. « Il faut, dit un Ancien, que celui qui se fait craindre de beaucoup de gens vive lui-même dans la crainte\footnote{« Necesse est multos timeat, quem multi riment. » Voyez {\itshape Publ. Syri. Sent.} Aratus détermina Lysiades, tyran de Mégalopolis, à renoncer au pouvoir qu’il avait usurpé en lui montrant les dangers et les inquiétudes dont il était accompagné. Voyez Plutarque, {\itshape Vie d’Aratus}. Le premier acte que fit Numa en prenant possession de la souveraineté fut de casser la compagnie de ses gardes « car, dit Plutarque, il ne voulait ni se défier de ceux qui se fiaient en lui, ni être le roi de ceux qui n’avaient aucune confiance en lui ». Voyez Plutarque, {\itshape Vie de Numa Pompilius}.}. » « Les tyrans, dit Plutarque, craignent leurs sujets ; les bons princes craignent pour leurs sujets. » Nulle puissance sur la terre ne peut longtemps commettre le mal en sûreté.\par
Désirer le despotisme, c’est désirer le pouvoir de faire du mal à tout un peuple et de se rendre soi-même très misérable. Le tyran est un malheureux qui gouverne des malheureux avec un glaive tranchant dont il se blesse lui-même. « Il n’est point de puissance assurée si elle ne se soumet aux lois de l’équité\footnote{« Ea demum tuta est potentia quæ viribus suis modum imponit. » Pline, {\itshape Panégyrique}.}. » Mais un penchant naturel à tous les hommes et que tout contribue à fortifier dans les princes, les porte à désirer un pouvoir sans bornes ; ils détestent tous les obstacles que leur autorité peut rencontrer. Les princes les plus faibles et les plus incapables en sont même les plus jaloux ; il n’en est pas que l’on ne réveille en leur parlant de l’extension de leur puissance. Tous se croient malheureux lorsqu’ils ne peuvent contenter toutes leurs fantaisies ; tous soupirent après le despotisme comme l’unique moyen d’obtenir la suprême félicité, tandis que ce despotisme ne leur met en main que les moyens d’écraser leurs sujets et de s’ensevelir eux-mêmes sous les ruines de l’État. Le pouvoir absolu fut et sera toujours la cause de la décadence et des malheurs des peuples que les rois sont tôt ou tard forcés de partager.\par
Cette vérité confirmée par l’expérience de tant de siècles semble être totalement ignorée de la plupart de ceux qui gouvernent le monde ; elle leur est soigneusement cachée par des ministres complaisants dont l’objet est de profiter de leurs désordres. Ce sont en effet ces âmes viles et désintéressées que l’on doit regarder comme les vraies causes de l’ignorance des princes et des malheurs des nations. Ce sont les flatteurs qui forment les tyrans et ce sont les tyrans qui, corrompant incessamment les mœurs des nations, rendent la vertu si pénible et si rare. Polybe a raison de dire que « la tyrannie est coupable de toutes les injustices et de tous les crimes des hommes ».\par
En effet, toujours injuste, elle ne peut être servie à son gré que par des hommes sans mœurs et sans probité, par des esclaves en proie à l’intérêt le plus sordide qui, sous des maîtres avides et corrompus, deviennent les seuls distributeurs des grâces, des dignités, des honneurs, des récompenses. Ceux-ci n’accordent leur bienveillance qu’à des hommes de leur trempe. Ils craignent le mérite et la vertu, qui les forceraient de rougir. Par la négligence ou l’injustice d’un mauvais gouvernement, une nation entière est forcée de se pervertir ; la vertu étant exclue de la faveur et des places, il faut y renoncer pour parvenir à la fortune, il faut suivre le torrent qui toujours entraîne vers le mal. La morale est inutile et déplacée sous un gouvernement despotique où tout citoyen vertueux doit nécessairement déplaire et au prince et à ceux qui gouvernent sous lui. Le tyran pour régner n’a besoin ni de talents ni de vertus. Il ne lui faut que des soldats, des fers et des prisons. Un tyran n’est souvent qu’un automate, une idole immobile qui ne se meut que par les impulsions que lui donnent les esclaves assez habiles pour s’emparer de son pouvoir. Un despote qui a jeté son pays dans la servitude finit presque toujours par n’être lui-même qu’un sot esclave ; ce n’est jamais lui qui recueille les fruits de la tyrannie.\par
La science la plus essentielle à celui qui veut gouverner sagement est, suivant Plutarque, de {\itshape rendre les hommes capables d’être bien gouvernés}. Les mœurs des souverains décident nécessairement des mœurs de leurs sujets. Distributeurs de tous les biens, des honneurs, des dignités que les hommes désirent, ils peuvent à leur gré tourner les cœurs vers le vice ou la vertu. Les cours donnent le ton aux villes, les villes corrompent les campagnes : voilà comment, de proche en proche, les peuples se trouvent imbus des préjugés, des vanités, du luxe, des frivolités, des folies et des vices que l’on voit infecter les cours. Les souverains donnent partout l’impulsion première aux volontés des grands et ceux-ci communiquent à leurs inférieurs l’impulsion qu’ils ont reçue : si la première impulsion portait au bien, les mœurs seraient bientôt réformées.\par
Tout le monde convient que le luxe, cette émulation fatale de vanité, est principalement dû au faste des souverains et des grands, que chacun s’efforce plus ou moins d’imiter ou de copier. Ce mal si dangereux paraît être inhérent à la monarchie et surtout au despotisme, où le prince, transformé en une espèce de divinité, veut en imposer à ses esclaves par un faste éblouissant. Pour arrêter les effets de cette épidémie dangereuse, on a quelquefois imaginé des lois que l’on a cru capables de la réprimer ; mais elles furent communément très inutiles. La meilleure des lois somptuaires pour un État, ce serait un prince frugal, économe, ennemi du luxe et de la frivolité. En permettant le luxe aux grands, en l’interdisant aux petits, on ne fait qu’irriter de plus en plus la vanité de ceux-ci, qui peu à peu vient à bout des lois les plus sévères.\par
Rien ne serait donc plus important pour la félicité des peuples que d’inspirer de bonne heure à ceux qui doivent régner sur eux l’amour de la vertu, sans laquelle il n’est point de prospérité sur la terre. Mais les maximes d’une politique injuste, dont l’objet est d’exercer impunément la licence, tiennent lieu trop souvent de science et de morale aux souverains ; par là les intérêts des chefs ne s’accordent jamais avec ceux du corps. Étrange politique, sans doute, par laquelle ceux qui ne sont destinés qu’à faire observer les devoirs de la morale sont continuellement occupés à les violer et à briser les liens qui devraient les unir avec leurs citoyens !\par
« Priver la vertu des honneurs qui lui sont dus, c’est, disait Caton, ôter la vertu à la jeunesse. » Mais éloigner la vertu des grandes places, corrompre les hommes pour les subjuguer, les diviser afin de les asservir les uns par les autres, c’est à quoi se réduisent tous les principes d’une politique odieuse visiblement imaginée non pour la conservation mais pour la dissolution d’un État. D’après de telles maximes, les souverains deviennent nécessairement les ennemis de leurs sujets et doivent déclarer une guerre sanglante à la raison qui pourrait les éclairer et à la vertu qui pourrait les réunir. Il vaut donc bien mieux les aveugler et les corrompre, les tenir dans une enfance éternelle, leur inspirer des vices capables de les mettre en discorde, afin de les empêcher de s’unir contre ceux qui les oppriment. La vertu doit être nécessairement détestée par tous ceux qui gouvernent injustement. La morale, d’ailleurs, ne peut convenir à des esclaves. Un esclave ne doit connaître de vertu qu’une soumission aveugle à la volonté de son maître\footnote{« Si les princes ne visent qu’à leur propre sûreté au lieu de l’honnêteté, ils ne devraient chercher à commander qu’à plusieurs moutons, plusieurs bœufs et plusieurs chevaux, non pas à plusieurs hommes… Un tyran qui aime mieux commander à des esclaves qu’à des hommes entiers me semble proprement faire comme le laboureur qui aimerait mieux recueillir des sauterelles, des oiseaux, que non pas du bon grain de froment et d’orge. » Voyez Plutarque, {\itshape Banquet des sept Sages}.}.\par
Les courtisans, toujours extrêmes dans leurs bassesses, ont voulu faire de leurs rois des divinités sur la terre. Mais il est aisé de voir qu’en exaltant ainsi leurs maîtres, ils ont fait de vains efforts pour justifier leur propre servitude et pour ennoblir leur lâcheté. D’ailleurs, ils étaient les prêtres des dieux qu’ils avaient ainsi crées.\par
Une politique plus saine et plus utile veut que les souverains se regardent comme des hommes, des citoyens, et qu’ils ne séparent jamais leurs intérêts de ceux de leurs sujets : de la réunion de ces intérêts résulte la concorde sociale, la félicité commune et du chef et des membres. Le prince n’est jamais vraiment grand et puissant s’il n’est soutenu par l’affection de son peuple. Le peuple est toujours malheureux si le souverain refuse de s’occuper de son bonheur. Éléas roi de Scythie disait « que quand il était oisif, il ne différait en rien de son valet d’écurie ». Une vie fainéante et dissipée est toujours honteuse et criminelle dans un roi, dont tout le temps appartient à ses sujets.\par
Pour gouverner de manière à rendre les nations heureuses, il ne faut ni un travail excessif, ni des lumières surnaturelles, ni un génie merveilleux : il ne faut que de la droiture, de la vigilance, de la fermeté, de la bonne volonté. Une âme trop exaltée peut quelquefois manquer de prudence : un bon esprit est souvent plus propre à gouverner les hommes qu’un génie transcendant. Que les nations ne demandent point à leurs chefs des talents sublimes et rares, des qualités difficiles à rencontrer. Tout homme de bien a ce qu’il faut pour gouverner un État, tout prince qui voudra sincèrement le bien de ses sujets trouvera sans peine des coopérateurs ; il fera naître dans sa cour une émulation de talents et de mérite non moins utile à ses intérêts qu’à ceux de ses sujets. Tout monarque qui voudra connaître la vérité aura bientôt les lumières nécessaires pour administrer sagement. Enfin, tout souverain qui s’attachera fortement à la justice la fera régner dans ses États et la rendra respectable à ses sujets. La justice et la force, voilà les vertus des rois.\par
La vaine pompe dont les rois sont environnés, la facilité et la promptitude avec laquelle leurs ordres sont exécutés, les amusements continuels dont on les voit jouir, les plaisirs dans lesquels on les voit nager, font que le vulgaire les regarde comme les plus heureux des mortels. En un mot, une erreur très commune fait supposer que le pouvoir suprême doit être accompagné de la suprême félicité. Mais la vie d’un souverain qui remplit ses devoirs est active, laborieuse, vigilante, incessamment occupée ; celle d’un prince désœuvré, dissipé, ennemi du travail, est un ennui perpétué. Tout monarque juste et sensible doit éprouver à chaque instant les sollicitudes les plus vives. Le souverain qui ne daigne pas s’occuper de ses propres affaires s’expose à tous les maux résultants de l’inconduite ou de la perversité de ses ministres, qu’il n’est guère en état de bien choisir. Les rois ont autant et plus à craindre de leurs amis que de leurs ennemis, ou plutôt ils n’ont jamais d’amis : ils n’ont que des flatteurs, des hommes vicieux attachés à leurs personnes, soit par un intérêt sordide, soit par la vanité. D’ailleurs, n’ayant point d’égaux, n’ayant aucun besoin, ils ne jouissent ni des douceurs de l’amitié, ni des charmes de la confiance, ni des plus grands agréments de la vie sociale : ils en sont privés par la distance énorme que le trône met entre eux et leurs sujets les plus distingués. Ceux-ci sont toujours gênés en présence d’un maître devant lequel on ne peut rien hasarder. D’où l’on voit que la gaieté, qui suppose toujours liberté, sécurité, égalité, ne peut jamais se montrer à la cour des rois. Ce fut au milieu d’un festin que le grand Alexandre assassina Clitus, qu’il regardait lui-même comme son ami le plus vrai\phantomsection
\label{footnote42}\footnote{Ce prince disait qu’{\itshape Héphestion aimait le roi}, mais que Clitus aimait Alexandre.}.\par
Enfin, le plus grand malheur attaché à la condition des rois, c’est de ne pouvoir presque jamais savoir la vérité. On la leur cache, surtout quand elle est affligeante, c’est-à-dire lorsqu’elle serait plus importante à connaître. « Quelques princes, dit Gordon, ont appris qu’ils étaient détrônés avant d’avoir appris qu’ils n’étaient point aimés\phantomsection
\label{footnote43}\footnote{Voyez {\itshape Discours préliminaire} de sa traduction de Tacite.}. » C’est ce qui arrive surtout aux souverains absolus, aux despotes, aux tyrans à qui leurs passions indomptées ne permettent jamais que l’on parle avec sincérité ; peu accoutumés à la contradiction, tout ce qui s’oppose à leurs fantaisies suffit pour provoquer la colère de ces enfants imprudents qui veulent pouvoir tout oser impunément. Ce sont pourtant les princes dont le pouvoir est illimité qui auraient le plus grand intérêt à connaître les vraies dispositions de leurs sujets ; ceux-ci, ne pouvant faire parvenir leurs plaintes jusqu’au trône, ne s’expliquent que par des révoltes, des révolutions et des massacres dont le tyran est la première victime.\par
Voilà donc la félicité suprême à laquelle conduit la puissance sans bornes que les princes désirent avec tant d’ardeur et qu’ils se croient malheureux de ne point posséder ! Cette puissance les prive de la confiance, des conseils, des secours, des consolations que l’amitié peut procurer. Bien plus, le monarque qui veut être juste doit se mettre en garde contre les séductions de ceux que son choix favorise, et craindre que son affection pour eux ne le fasse pécher contre la justice universelle qu’il doit à tout son peuple. C’est de ce peuple qu’il doit ambitionner l’amitié, c’est ce peuple qu’il doit entendre pour savoir la vérité, c’est sur ce peuple qu’il doit fonder sa propre sûreté, c’est sur le bien-être de ce peuple qu’il doit établir sa propre grandeur, sa gloire, sa félicité : ce sont ceux qui lui feront obtenir ces avantages que le prince doit regarder comme ses amis. Théopompe disait qu’un grand roi est celui qui permet à ses amis de lui dire la vérité, qui rend justice à ses sujets et qui obéit aux lois.\par
Quelle que soit la forme du gouvernement adoptée par une nation, les devoirs, les intérêts de ses chefs seront toujours les mêmes. La politique et la morale veulent que dans un gouvernement aristocratique un sot orgueil, un vain esprit de corps, un attachement opiniâtre à des prérogatives injustes ne l’emportent jamais sur les droits de la patrie. Rien de plus fâcheux dans les aristocraties et de plus insupportable aux peuples, que la vanité puérile des nobles, des magistrats ou des souverains collectifs. Ceux-ci devraient se distinguer par la décence et la gravité de leurs mœurs, leur équité, leur probité, leur affabilité, leur modestie, qualités bien plus propres à les faire chérir et révérer qu’une morgue insociable qui ne peut que les faire détester de leurs concitoyens et qui se trouve déplacée dans les gouvernements républicains.\par
Que les chefs d’une aristocratie laissent aux esclaves favorisés du despotisme la vaine gloire de se distinguer par leur hauteur et leur insolence ; qu’ils se distinguent par leur bonté, leur modération, leur intégrité. L’arrogance et l’orgueil doivent être bannis des États où l’on jouit de quelque liberté. L’aristocratie doit compter le peuple pour quelque chose ; elle ne le regarde pas des mêmes yeux que la monarchie, qui ne distingue que ses nobles, ou que le despotisme, qui méprise également le vil troupeau qu’il écrase.\par
En un mot, tout gouvernement républicain suppose une sorte d’égalité entre des citoyens également soumis aux lois. Les magistrats y sont des chefs sans cesser d’être citoyens. D’où il suit que leurs manières hautaines sont plus choquantes et plus importunes au peuple que sous la monarchie, qui l’a de longue main accoutumé à endurer l’insolence et les mépris des grands et de tous ceux qui jouissent de quelque pouvoir. Dans tout État bien constitué, nul citoyen n’a le droit d’être insolent. Ces aristocrates communément si jaloux de leur pouvoir, si défiants, s’épargneraient bien des dépenses, des embarras et des gênes s’ils daignaient se souvenir qu’ils sont des citoyens et non des tyrans ou des despotes, que la vanité n’est propre qu’à les faire abhorrer, qu’elle fait journellement des ennemis et des mécontents dont l’humeur éclate quelquefois par des révolutions terribles\phantomsection
\label{footnote44}\footnote{« La trop grande jalousie du pouvoir, dit Tite-Live, et l’obstination à ne jamais descendre de sa grandeur dans un des ordres d’une république, produit souvent de grands démêlés très inutiles et qui souvent deviennent funestes à cet ordre lui-même. » — « Le peuple, dit Plutarque, regarde toujours comme un très grand honneur de n’être pas méprisé des grands. » Voyez {\itshape Vie de Nicias}.}.\par
Nous trouvons des preuves de cette vérité dans l’histoire de la plupart des aristocraties anciennes, qui communément dégénérèrent en tyrannies véritables. L’histoire romaine nous montre un Sénat orgueilleux, avare, jaloux de ses prérogatives usurpées, perpétuellement en querelle avec le peuple, qu’il s’arrogeait le droit de mépriser, de vexer par ses usures, d’opprimer de toutes manières et d’envoyer à la boucherie au dehors quand il l’incommodait. Bientôt la division entre les chefs de cette république toujours armée produit des factions cruelles, d’affreuses guerres civiles s’allument, les citoyens s’arment contre les citoyens ; enfin, après les sanglants démêlés de Marius et de Sylla, l’ambitieux César, appuyé de la faction du peuple, s’élève sur les ruines de l’État. Il établit le despotisme d’un seul à la place du despotisme des magistrats, il laisse le gouvernement en proie à une longue suite de monstres qui semblèrent se disputer à qui commettrait le plus de crimes et d’infamies. La noblesse romaine devient surtout l’objet de la cruauté des Tibère, des Caligula, des Néron. Tandis que ces monstres caressaient le peuple ou l’amusaient par des spectacles, ils faisaient couler le noble sang des sénateurs et des patriciens dont la race faisait ombrage à leur ambition tyrannique. En un mot, l’orgueil d’un Sénat divisé mit fin à la République la plus puissante qui fut jamais au monde. « C’est par les grands, dit Solon, que les cités périssent ; c’est par l’imprudence du peuple qu’elles tombent dans les fers. »\par
Les démocraties, ou gouvernements populaires, ne périssent communément si tôt que par l’injustice, la licence, la jalousie et l’envie du peuple, que son pouvoir enivre et rend insolent. Une populace arrogante, flattée par ses démagogues, devient souvent le plus cruel des tyrans : elle immole la vertu même à son envie, à son caprice, au plaisir de faire sentir sa puissance aux citoyens qu’elle devrait chérir et respecter. Elle commet le crime sans remords parce qu’elle est inconsidérée et parce que d’ailleurs la honte en est supportée par un plus grand nombre de coupables. L’ingratitude des Athéniens pour\par
Aristide, Cimon et Phocion, fait que personne n’est tenté de plaindre un peuple frivole et méchant d’avoir enfin totalement perdu sa liberté dont il faisait un si terrible usage\footnote{L’ingratitude des Athéniens pour Périclès, à qui ils voulurent faire rendre compte de son administration, détermina cet homme célèbre à exciter la guerre du Péloponnèse, qui fut la cause de la destruction de toutes les républiques de la Grèce. Thémistocle disait aux Athéniens : « Ô pauvres gens ! Pourquoi vous lassez-vous de recevoir souvent des bienfaits des mêmes gens ? » Plutarque observe très justement que dans les révolutions de la démocratie, c’est ordinairement le plus méchant qui prospère et qui s’élève au plus haut degré. Voyez Plutarque, {\itshape Vie de Nicias}.}. Platon fait dire à Socrate que la démocratie « est l’empire des méchants sur les bons. Et que la multitude, lorsqu’elle jouit de l’autorité, est le plus cruel des tyrans. » Un despote peut être quelquefois retenu par la crainte, la honte, le remords, au lieu qu’un peuple tyran emporté par ses passions a perdu toute crainte et toute pudeur.
\subsection[{Chapitre III. Devoirs des Sujets}]{Chapitre III. Devoirs des Sujets}
\noindent Tout gouvernement équitable exerce, comme on a vu, une autorité légitime à laquelle tout citoyen vertueux est obligé d’obéir, mais un gouvernement injuste n’exerce qu’un pouvoir usurpé. Sous le despotisme et la tyrannie il n’y a plus d’autorité : il n’y a qu’un brigandage ; la société, contre son gré, est forcée de subir le joug qui lui est imposé par le crime et la violence. Opprimée elle-même, elle ne peut plus procurer aux citoyens aucun des avantages qu’elle s’est engagée de leur assurer par le pacte social. Un mauvais gouvernement anéantit ce pacte ; en empêchant la société de remplir ses engagements avec ses membres, il semble annoncer à ceux-ci qu’ils ne doivent rien à la société.\par
Pour que la société soit en droit d’exiger l’attachement de ses membres, elle doit leur montrer un tendre intérêt à tous. Elle ne s’est point engagée à rendre tous les citoyens également aisés, heureux et puissants mais elle s’est engagée à les protéger également, à les garantir de l’injustice, à leur procurer la sûreté nécessaire à leurs entreprises et à leurs travaux, à les récompenser en raison des services qu’ils lui rendront. C’est à ces conditions que les citoyens peuvent aimer leur patrie, s’intéresser à son bonheur, contribuer fidèlement à sa conservation et à sa félicité. Qu’est-ce que l’amour de la patrie sous un gouvernement tyrannique ? L’exiger d’un esclave, ce serait évidemment vouloir qu’un prisonnier chérît sa prison et fût amoureux de ses chaînes. L’amour de la patrie dans un pays soumis à la tyrannie ne consiste que dans un attachement servile pour ses tyrans, de qui l’on espère obtenir les dépouilles de ses concitoyens. Dans une pareille constitution l’homme vraiment attaché à son pays passe pour un rebelle, pour un mauvais citoyen, pour un ennemi de l’autorité\footnote{« La Cité, dit Plutarque, est très bien gouvernée… en laquelle ceux qui ne sont point outragés haïssent autant et poursuivent aussi âprement celui qui a fait une oppression et outrage, que celui qui est outragé. » Voyez {\itshape Banquet des sept Sages}.}.\par
Les hommes, presque toujours gouvernés par des mots, s’imaginent que tout ce qui porte l’empreinte du pouvoir est fait pour être aveuglément obéi. Ils ne voient pas que l’autorité légitime (c’est-à-dire celle qui contribue au bien de la société et qui est reconnue par elle) est la seule qui ait le droit de se faire obéir ; ils ne voient pas que l’autorité, dès qu’elle devient injuste, n’a plus le droit d’obliger des hommes rassemblés pour jouir des avantages de l’équité et de la protection des lois : « Personne, dit Cicéron, ne doit obéir à ceux qui n’ont pas le droit de commander. » La tyrannie est faite pour être détestée par tout bon citoyen. Ses ordres ne peuvent être suivis que par des esclaves corrompus qui cherchent à profiter des malheurs de leur patrie. Un intérêt sordide et la crainte, et non l’affection, peuvent être les motifs de l’obéissance forcée du citoyen obligé de haïr intérieurement l’autorité malfaisante sous laquelle son destin le force de gémir. Les Grecs, suivant Plutarque, regardaient le gouvernement despotique des Perses comme indigne de commander à des hommes.\par
Ces réflexions si naturelles doivent nous empêcher d’être surpris de trouver la plupart des nations remplies de citoyens indifférents sur le sort de la patrie, dépourvus de toute idée du bien public, uniquement occupés de leurs intérêts personnels, sans jamais faire le moindre retour sur la société. Les intérêts de celle-ci n’ont en effet rien de commun avec ceux de la plupart des membres qui la composent. On ne trouve nulle part des lois qui établissent une justice exacte parmi les citoyens. Les nations se divisent en oppresseurs et en opprimés. Des préjugés injustes, des vanités méprisables, des privilèges iniques mettent perpétuellement la discorde entre les différents ordres de l’État. Un fatal esprit de corps prend la place de l’esprit public et du patriotisme. Les riches et les grands s’arrogent le droit de vexer les pauvres et les petits, le noble méprise le roturier, le guerrier ne connaît que la force et n’obéit qu’à la voix du despote qui le paie. Le magistrat ne songe qu’aux prérogatives de sa charge et s’embarrasse fort peu des droits de ses concitoyens. Le prêtre ne s’occupe que de ses immunités. Ainsi, des intérêts discordants s’opposent sans cesse à l’intérêt général et détruisent efficacement l’harmonie sociale. Le despotisme habile se prévaut de ces divisions continuelles pour abattre la justice et les lois. Il fomente ses dissensions, il met ses créatures à portée de profiter des ruines de la patrie ; aveuglés par les faveurs trompeuses, ceux qui devraient se montrer les meilleurs citoyens ne cherchent qu’à se procurer le crédit ou le pouvoir d’opprimer, ils travaillent à fortifier de plus en plus la puissance fatale sous laquelle la nation entière sera tôt ou tard accablée. Les pauvres et les faibles perpétuellement écrasés par l’injustice des puissants et des grands, qu’ils voient seuls prospérer, deviennent leurs ennemis, et par des crimes se vengent de la partialité du gouvernement qui ne répand ses bienfaits que sur les heureux de la terre et qui oublie totalement les malheureux.\par
On ne peut trop le répéter : tous les citoyens d’un État sont également intéressés à y voir régner l’équité. Il n’est point un seul homme qui, s’il était raisonnable, ne dût trembler dès qu’il voit la violence opprimer le dernier des citoyens. L’oppression, après avoir fait sentir ses coups aux dernières classes du peuple, finit par les faire éprouver aux classes les plus élevées. Les corps les plus puissants, dès qu’ils sont divisés, n’offrent qu’une faible barrière à la tyrannie qui marche incessamment vers son but. Tous les corps, toutes les familles, tous les citoyens, n’ont qu’un seul intérêt : c’est d’être gouvernés par des lois équitables. Les lois ne sont telles que lorsqu’elles protègent également le grand et le petit, le riche et l’indigent. Le bon citoyen est celui qui dans sa sphère contribue de bonne foi à l’intérêt général parce qu’il reconnaît que son intérêt personnel ne peut en être détaché sans péril pour lui-même, vérité que nous ferons sentir en parcourant les devoirs de toutes les classes suivant lesquelles les citoyens d’un État sont partagés.\par
Un bon gouvernement ne mérite ce nom que lorsqu’il est juste pour tout le monde. Il a seul le pouvoir de former de bons citoyens, il a seul le droit d’attendre de la part de ses sujets l’attachement, la fidélité, les sacrifices généreux, en un mot : l’accomplissement des devoirs de la vie sociale. L’autorité légitime est la seule qui puisse être sincèrement aimée, obéie, respectée ; elle seule peut inspirer aux hommes l’amour de la patrie, qui n’est évidemment que l’amour de leur sûreté et de leur prospérité.\par
Tout le monde a dans la bouche cet adage : {\itshape la patrie est là où l’on se trouve bien}. D’où il résulte qu’il n’y a plus de patrie où l’on se trouve sous l’oppression sans espérance de voir finir ses peines. Le citoyen est fait pour supporter avec patience les inconvénients nécessaires de la vie sociale et pour partager avec ses concitoyens les calamités passagères qu’ils éprouvent. Mais il a droit de renoncer à l’association dès qu’il voit qu’elle lui refuse constamment les avantages qu’il a droit d’en attendre. Il n’y a plus de patrie où il n’y a ni justice, ni bonne foi, ni concorde, ni vertu. Sacrifier ses biens et sa vie pour des tyrans, c’est s’immoler non à sa patrie, mais à ses plus cruels ennemis. « Le bon citoyen, dit Cicéron, est celui qui ne peut souffrir dans sa patrie une puissance qui prétende s’élever au-dessus des lois. »\par
Le citoyen ne doit obéir qu’aux lois, et ces lois, comme on a vu, ne peuvent avoir pour objet que la conservation, la sûreté, le bien-être, l’union, le repos de la société. Celui qui obéit en aveugle aux caprices d’un despote n’est point un citoyen, c’est un esclave. Il n’y a point de citoyens sous le despotisme : « il n’y a point de cité pour des esclaves\phantomsection
\label{footnote45}\footnote{« Servorum nulla est civitas. » {\itshape Publ. Syri. Sentent.}}. » La patrie n’est pour eux qu’une vaste prison gardée par des satellites sous les ordres d’un geôlier impitoyable. Ces satellites sont des mercenaires dont l’obéissance est une vraie trahison. « Rien, dit Cicéron, n’est plus contraire à l’équité que des hommes armés et rassemblés ; rien de plus opposé au droit que la violence\phantomsection
\label{footnote46}\footnote{Cicéron, {\itshape Plaidoyer pour Cécina}.}. » La vraie cité, la vraie patrie, la vraie société est celle où chacun jouit de ses droits maintenus par la loi.\par
Partout où l’homme est plus fort que la loi, la justice est obligée de se taire et la société ne tarde point à se dissoudre. Pausanias, roi de Sparte, disait « qu’il faut que les lois soient maîtresses des hommes, et non pas que les hommes soient les maîtres des lois ». Solon disait que « pour faire durer un empire il faut que le magistrat obéisse aux lois, et le peuple aux magistrats ». Enfin, Platon dit que « les meilleurs princes sont ceux qui obéissent le plus fidèlement aux lois. Partout, ajoute-t-il, où la loi est la maîtresse et où les magistrats sont ses esclaves, l’on voit prospérer les villes et abonder tous les biens qu’on peut attendre des dieux, au lieu que partout où le magistrat est le maître et la loi la servante, l’on ne doit attendre que ruine et désolation ».\par
Mais pour être en droit de régler la conduite des souverains et des sujets, les lois doivent être justes, conformes au bien public, au but de la société, à ses besoins, à ses circonstances particulières. Des lois qui n’auraient pour objet que les intérêts personnels du souverain ou de ceux que sa faveur distingue, seraient injustes et contraires au bien-être de tous. Des lois tyranniques ne peuvent être respectées : elles sont faites par des hommes qui n’ont pas droit de commander. Le bien public et l’équité naturelle sont la mesure invariable de l’obéissance que le citoyen doit même aux lois. Quiconque a des idées vraies de la justice peut aisément distinguer les lois qu’il doit suivre de celles auxquelles il ne pourrait se soumettre sans blesser sa conscience et sans se rendre coupable envers la société. Nul homme qui a quelque idée de justice ou quelque sentiment d’honneur ne se prévaudra d’une loi forgée par la tyrannie pour autoriser quelques citoyens à dépouiller les autres. Nul homme qui n’est pas totalement aveuglé par un intérêt sordide ne croira que le souverain puisse lui conférer le droit de s’enrichir injustement aux dépens de sa patrie. Tout homme de bien renoncera plutôt à la fortune, à la grandeur, au crédit, que de conserver un emploi qu’il ne peut exercer au gré du prince sans faire le malheur de ses concitoyens.\par
La justice serait vraiment bannie de la terre si les ordres des princes étaient des lois auxquelles il ne fût jamais permis de résister. Le courtisan moderne qui disait « qu’il ne concevait pas comment on pouvait résister à la volonté de son maître\footnote{{\itshape Journal historique de la révolution opérée par le chancelier de Maupéou}, tome II.} », parlait comme un esclave nourri dans les maximes du despotisme d’Orient, suivant lesquelles le sultan est un dieu aux caprices de qui c’est un crime de s’opposer, lors même qu’ils répugnent au bon sens.\par
Cependant, à la honte des personnes qui occupent le rang le plus distingué dans plusieurs nations éclairées, ces principes odieux et destructeurs sont la règle de la conduite de bien des grands et de la plupart des nobles et des gens de guerre. Bien plus, cette doctrine fut très souvent prêchée par les ministres d’un dieu que l’on suppose la source de toute justice et de toute morale !\par
Ou en seraient des nations si, malheureusement infectées de ces idées funestes, des magistrats n’avaient jamais le courage de s’exposer à la colère du souverain en refusant de souscrire à ses volontés arbitraires ? Que deviendraient les peuples si la justice dépendait des caprices variables d’un sultan, d’un vizir, d’une favorite, que le pouvoir absolu ferait passer pour des lois ? Sur quoi serait fondée l’autorité du monarque lui-même s’il se faisait un jeu d’anéantir l’équité qui sert de base à son trône, qui fait également la sûreté des rois et des sujets ?\par
Ainsi, les vils flatteurs qui prétendent que le prince ne doit jamais ni reculer ni trouver de résistance à ses volontés suprêmes, sont non seulement de mauvais citoyens mais encore des ennemis du prince. N’est-ce pas servir fidèlement le souverain que de lui désobéir quand ses ordres sont contraires à ses propres intérêts ? Il n’y a que des insensés qui puissent se prêter aux fantaisies d’un inconsidéré résolu de ravager son héritage ; lui résister, c’est l’empêcher de se nuire, lui obéir, c’est se rendre complice de sa folie et de sa ruine.\par
Tout prince qui se révolte contre des lois équitables invite ses sujets à se révolter contre lui. Tous ceux qui l’excitent ou le soutiennent dans ses entreprises insensées sont de mauvais citoyens, des adulateurs infâmes qui trahissent à la fois et la patrie et son chef. Ceux qui adoptent les maximes d’une obéissance aveugle et passive aux lois imposées par le despotisme en délire sont ou des stupides qui méconnaissent leurs propres intérêts, ou des esclaves qui méritent d’éprouver pendant toute leur vie la dureté de leurs fers.\par
Si l’on s’en rapportait aux notions vagues de quelques spéculateurs, on serait tenté de croire que tous les sujets d’un État, changés en automates, devraient une obéissance aveugle et implicite à tout ce qui ferait loi ou porterait la sanction de l’autorité souveraine. Mais cette autorité est-elle donc toujours juste, infaillible, exempte de passions, incapable de s’égarer ? La tyrannie, qui n’est que le gouvernement de l’injustice unie avec la force, a-t-elle le droit de fabriquer des lois contraires à l’équité, et chacun est-il tenu de s’y soumettre sans murmurer ? Si ces principes étaient vrais, la société ne serait plus qu’un amas de victimes obligées de se laisser dépouiller et de tendre le col au glaive des citoyens obéissants que le tyran aurait choisis pour être ses bourreaux.\par
Distinguons donc les lois faites pour être obéies et respectées par des citoyens honnêtes, de ces lois injustes et destructives que la tyrannie, la violence, la déraison, la routine, qui ne raisonne point, ont souvent introduites. « La justice, dit un docteur célèbre\phantomsection
\label{footnote47}\footnote{St. Augustin.}, a le droit de briser d’injustes liens. » Ce n’est pas le citoyen qui a droit de juger la loi de son pays, c’est la justice dont tout homme sensé est en droit de se faire des idées sûres. Les lois ne sont respectables que lorsqu’elles sont équitables. Elles doivent être abrogées dès qu’elles sont contraires au bien public. « Les lois, dit Locke, sont faites pour les hommes, et non les hommes pour les lois. » Les plus grands maux des nations sont dus à des lois visiblement injustes sous lesquelles la violence les force de plier. « Les lois, dit Montaigne, se maintiennent en crédit non parce qu’elles sont justes, mais parce qu’elles sont lois\footnote{Voyez {\itshape Essais}, livre III, ch. 13.}. »\par
Le respect dû aux lois ne peut être fondé que sur l’équité de ces lois, que pour son propre intérêt tout citoyen doit observer et maintenir. « Les lois, disait Démonax, sont inutiles aux bons parce que les gens de bien n’en ont aucun besoin, et aux méchants parce qu’ils n’en deviennent pas meilleurs. » Socrate, qui poussa jusqu’au fanatisme la soumission aux lois d’un peuple ingrat et frivole, et qui voulut en être le martyr, fut injuste envers lui-même ; s’il fût sorti de prison, il eût épargné aux Athéniens un crime qui les a couverts d’une éternelle infamie.\par
La morale n’aurait aucuns principes constants et sûrs si des lois quelconques, souvent insensées et criminelles, devaient être plus respectées que la voix de la Nature éclairée par la raison. En promenant ses regards sur toutes les contrées de la terre, on est surpris de trouver que les plus grands forfaits ont été non seulement approuvés mais encore commandés par les lois. Dans tous les États despotiques on ne voit pour l’ordinaire que les caprices des tyrans les plus extravagants consacrés sous le nom de lois. Des peuples se sont permis le parricide\footnote{Élien (livre IV, ch. I) nous dit qu’en Sardaigne les enfants étaient obligés de tuer leurs pères lorsqu’ils étaient tombés dans la décrépitude. Les derviches tuaient pareillement tous ceux qui vivaient au-delà de soixante ans.} ! Les Carthaginois étaient forcés de sacrifier leurs enfants à leurs dieux sanguinaires. Les Égyptiens, qui passent pour avoir été si policés, si sages, ont approuvé le vol. Chez les Scythes on égorgeait des milliers d’hommes et de femmes pour honorer les funérailles des princes. Pourquoi n’aurait-on pas désobéi à de pareilles lois ou réclamé contre elles ? « Les hommes, demande Cicéron, ont-ils donc le pouvoir de rendre bon ce qui est mauvais, et mauvais ce qui est bon ? »\par
On nous dira peut-être que ces lois n’ont lieu que chez des peuples barbares qui n’avaient aucune idée de la morale. Mais les peuples modernes nous offrent-ils des lois plus justes et plus sensées ? L’équité, le bon sens, l’humanité ne sont-ils pas indignement violés par des lois de sang établies dans un grand nombre de pays contre tous ceux qui ne professent pas la religion du prince ? Trouvera-t-on quelque ombre de justice dans la plupart de ces lois fiscales dont l’objet est de fournir aux extravagances des souverains en dépouillant les peuples du nécessaire ? dans ces lois féodales imposées par des nobles armés à des nations tremblantes ?\par
Mais il faut s’arrêter, car l’on ne finirait pas si l’on voulait faire l’énumération des lois iniques dont les peuples sont les victimes forcées ou volontaires.\par
Quelles idées claires et vraies de l’équité naturelle les peuples pourraient-ils puiser dans cet amas informe de coutumes et de lois injustes, déraisonnables, bizarres, ténébreuses, inconciliables, qui presque en tout pays forment la jurisprudence et la règle des hommes ? Quelles notions peut-on se former de la justice quand on la voit perpétuellement anéantie par des formalités insidieuses ? Quelles ressources les citoyens peuvent-ils trouver dans une jurisprudence captieuse qui semble favoriser la mauvaise foi, les emprunts et les contrats frauduleux, les friponneries les plus indignes, les ruses les plus capables de bannir la probité des engagements réciproques des citoyens ?\par
Quelle confiance peut-on prendre ou quelle protection peut-on trouver dans des lois qui donnent lieu à des chicanes interminables destinées à ruiner les plaideurs, à engraisser des praticiens imposteurs, à mettre des gouvernements avides à portée de lever des impôts sur les dissensions éternelles des sujets ? Dans la plupart des nations l’étude des lois, qui devraient être simples et à la portée de tous les citoyens, est une étude pénible de laquelle résulte une science très incertaine uniquement réservée à quelques hommes qui profitent de son obscurité pour tromper et dépouiller les malheureux qui tombent dans leurs mains.\par
En un mot, les lois faites pour guider les nations ne sont propres qu’à les égarer, à leur faire méconnaître les principes les plus évidents de l’équité\phantomsection
\label{footnote48}\footnote{Pour se convaincre de l’absurdité et même de la perversité de la jurisprudence romaine, et surtout des lois de Justinien qui servent encore de base à la législation européenne, on a qu’à lire le {\itshape Traité des lois civiles par M. P. de T.}, publié depuis peu à La Haye, 1774, et l’on verra qu’à proprement parler les nations n’ont pas encore de législation véritable, c’est-à-dire vraiment conforme au bien de la société. Par une négligence ou une impéritie bien funeste, les législateurs modernes ont trouvé plus court d’adopter des lois anciennes maladroitement corrigées ou modifiées, que d’en faire de neuves plus justes, plus morales, plus analogues à la position actuelle des peuples. Des Francs, des Goths, des Lombards, des Saxons, des brigands ignorants nourris dans le carnage étaient-ils des législateurs en état de donner des lois sensées aux peuples vaincus ou de rectifier celles que ces peuples avaient déjà ?}.\par
Les lois ne devant être que les règles de la morale promulguées par l’autorité, devraient être claires, précises, intelligibles pour tout le monde. Mais elles ne sont d’ordinaire que des pièges tendus à la simplicité, des chaînes incommodes dont la puissance a de tout temps surchargé la faiblesse. Des lois ainsi formées corrompent évidemment les mœurs, elles autorisent le fripon habile à se montrer sans pudeur dans la société ; enfin, souvent elles ne font que des transgresseurs.\par
Les hommes sont communément ennemis des lois parce qu’ils ne trouvent en elles que des obstacles continuels à l’exercice de leur liberté et de leurs droits naturels, qui les empêchent de satisfaire leurs besoins, de contenter leurs désirs les plus légitimes. De l’aveu même des jurisconsultes, rien de plus injuste et conséquemment de plus contraire à la morale que le droit, s’il était rigoureusement observé\phantomsection
\label{footnote49}\footnote{« Summum jus, summa injuria. »}.\par
L’homme qui n’est juste que conformément aux lois peut être dépourvu de toute vertu sociale : à l’aide de ces lois un fils attaquera très indécemment son père, des époux se diffameront réciproquement, des proches se dépouilleront sans pitié ; les débiteurs ruineront leurs créanciers, les traitants s’approprieront la subsistance du pauvre, des juges immoleront sans remords l’innocent, et des hommes si pervers marcheront la tête levée au milieu de leurs concitoyens !\par
Nul climat, nul gouvernement, nul pouvoir n’a le droit de porter atteinte à l’empire universel que la justice doit exercer sur les hommes ; cependant, aucune législation ne semble avoir consulté les intérêts des peuples. On dirait que le genre humain entier n’existe et ne vit sur la terre que pour un petit nombre d’individus privilégiés qui s’embarrasse fort peu de lui procurer le bonheur qu’il aurait droit d’attendre en échange de sa soumission\footnote{« humanum paucis uiuit genus. » Lucain, La Pharsale, livre V.}.\par
Une législation vraiment sacrée serait celle qui consulterait les intérêts de tous et non les intérêts de quelques chefs ou de ceux qu’ils favorisent. Des lois utiles et justes sont celles qui maintiennent chaque citoyen dans ses droits et se garantissent de la méchanceté des autres. Les nations n’auront une législation respectable et fidèlement obéie que lorsqu’elle sera conforme à la nature de l’homme vivant en société, c’est-à-dire guidée par la morale, dont elle doit rendre les préceptes inviolables.\par
C’est alors que la loi doit être religieusement observée, c’est alors que ses infracteurs pourront être justement châtiés comme des ennemis de la patrie et des enfants rebelles.\par
On regarde communément la réforme des lois comme une entreprise si difficile qu’elle surpasse les forces de l’esprit humain. Mais disons avec Quintilien\footnote{Quintilien, livre XII, chap. I.} : pourquoi n’oserait-on pas avancer que la durée des siècles fera découvrir quelque chose de plus parfait que ce qui a ci-devant existé ?\par
Cette difficulté ou cette impossibilité prétendue ne vient point de la chose elle-même : elle est due aux préjugés des hommes, à la négligence ou à la mauvaise volonté de ceux qui les gouvernent. Des souverains équitables acquièrent le droit de commander à l’opinion des peuples.\par
Ceux-ci ne sont en garde contre les nouveautés et les changements que parce qu’une expérience fatale leur apprend qu’ils ne font communément que redoubler leurs misères. Partout les peuples sont mal, mais ils craignent toujours d’être plus mal encore. Le prince qui par sa vertu s’attirera la confiance de ses sujets, dissipera ses craintes, substituera quand il voudra des lois justes et claires à ces lois obscures et si souvent déraisonnables pour lesquelles les nations ont un attachement machinal.\par
Le souverain éclairé développe la raison de son peuple ; rien de plus aisé que de gouverner des sujets raisonnables, rien de plus difficile que de contenir des hommes ignorants et privés de raison. Une bonne législation se trouvera formée lorsqu’elle armera la morale de l’autorité suprême. Elle sera fidèlement suivie quand tous les citoyens reconnaîtront que leur intérêt les oblige de s’y conformer. La morale ne peut rien sans le secours des lois, et les lois ne peuvent rien sans les mœurs\phantomsection
\label{footnote50}\footnote{« Quid leges sine moribus, vanae proficiunt ? » Horace, {\itshape Odes}, livre III, 24, vers 35. Aristote, avant lui, avait dit : « La loi n’a d’autre force pour se faire obéir que celle qu’elle tire de l’accoutumance ; et c’est l’accoutumance qui forme les mœurs. » Voyez Aristote, {\itshape Politique}, livre II, chap. 8.}.\par
Ainsi, ne désespérons point que l’on ne puisse voir un jour les hommes soumis à des lois plus sages, plus conformes à leur nature, plus propres à les rendre vertueux et fortunés.\par
Un bon roi, comme un Hercule, peut bannir de ses États les monstres, les vices, les préjugés qui s’opposent également au bien-être des souverains et des sujets. Les peuples seront heureux quand les rois seront des sages\phantomsection
\label{footnote51}\footnote{Voyez Plutarque, {\itshape Vie de Numa}, Cicéron, {\itshape A. Q. Fratrem}.}. « Les villes et les hommes, dit Platon, ne seront délivrés de leurs maux que lorsque, par une fortune divine, la souveraine puissance et la philosophie se rencontrant dans le même homme rendront la vertu triomphante du vice. »
\subsection[{Chapitre IV. Devoirs des Grands}]{Chapitre IV. Devoirs des Grands}
\noindent L’on nomme {\itshape grands} ceux qui sont élevés au-dessus de leurs concitoyens par leur pouvoir, leurs places, leur naissance et leurs richesses. Dans un État bien constitué, c’est-à-dire où la justice serait fidèlement observée, les citoyens les plus vertueux, les plus utiles, les plus éclairés, seraient les plus grands ou les plus distingués. Le pouvoir ne serait remis que dans les mains les plus capables de l’exercer pour le bien de la société. Les dignités, les places, les honneurs, les marques de la considération publique ne serait accordés qu’à ceux qui les auraient mérités par leurs talents et leur conduite. Les richesses et les récompenses ne seraient le partage que de ceux qui sauraient en faire un usage vraiment avantageux à leurs concitoyens. D’où l’on voit que la vertu seule donne des droits légitimes à la grandeur.\par
Si, comme on l’a fait voir, toute autorité que l’on exerce sur les hommes ne peut être fondée que sur les avantages qu’on leur procure, si toute supériorité, toute distinction ou prééminence sur nos semblables, pour être reconnue par eux, suppose des qualités supérieures, des talents estimables, un mérite peu commun, on sera forcé de convenir que l’absence de ces qualités fait rentrer dans la foule, que le pouvoir exercé par des hommes indignes, que l’autorité dont ils sont revêtus, que leur supériorité ne sont que des usurpations auxquelles leurs citoyens ne peuvent se soumettre que par la violence.\par
L’amour de préférence que chaque homme a pour lui-même fait qu’il désire de s’élever au-dessus de ses égaux et le rend envieux et jaloux de tout ce qui lui fait sentir sa propre infériorité. Mais s’il a des sentiments équitables, ces jalousies disparaissent dès qu’il voit que ceux qu’on lui préfère ou qu’on distingue de lui possèdent des talents et des qualités estimables dont il est à portée de profiter lui-même. Ainsi, le mérite et la vertu calment l’envie des hommes, les forcent de reconnaître la supériorité de ceux qu’on élève au-dessus de leurs têtes par des honneurs légitimes, par un rang mérité ; alors ils consentent à leur donner des signes plus marqués de soumission et de respect qu’à leurs autres citoyens.\par
En respectant et conservant les droits de tous les citoyens forts ou faibles, riches ou pauvres, grands ou petits, l’équité naturelle veut pourtant, pour l’utilité générale, que ceux qui procurent de plus grands avantages soient récompensés par les marques de considération et d’estime, par les déférences qui leur sont dues en vertu des services qu’ils rendent à la société. Voilà l’origine naturelle et légitime des rang divers dans lesquels les citoyens d’un même État se trouvent partagés : cette inégalité est juste puisqu’elle tend au bien-être de tous, elle est louable parce qu’elle est fondée sur la reconnaissance sociale, qui doit payer les services qu’on reçoit. Elle est utile parce qu’elle se sert de l’intérêt personnel pour exciter les hommes à faire le bien, comme un moyen d’obtenir la supériorité que chacun désire avec ardeur.\par
Ce n’est donc qu’en donnant des preuves de son mérite que l’on obtient à juste titre le droit de s’élever au-dessus des autres. Toute autre voie serait inique, démentie par la société, contraire à ses vrais intérêts et regardée par elle comme une usurpation manifeste. Même dans les gouvernements les plus despotiques, les places, le pouvoir, les dignités conférés à des citoyens incapables ou pervers, révoltent leurs concitoyens. La crainte peut bien les empêcher de faire éclater leur indignation et leur arracher des signes d’une soumission que leur cœur désavoue, mais la vertu seule obtient des hommages sincères et les reçoit avec un plaisir pur, tandis que le vice, toujours inquiet et soupçonneux, sait à quoi s’en tenir sur les respects qu’on lui montre.\par
La vraie grandeur de l’homme et sa vraie dignité consistent donc à faire du bien aux hommes, à leur montrer des sentiments d’affection, à leur rendre les services, à répandre sur eux les bienfaits en faveur desquels ils consentent à reconnaître des supérieurs. D’où il suit que les grands, s’ils veulent se rendre dignes de l’attachement vrai et des respects volontaires de leurs concitoyens, doivent surtout écarter de leur conduite l’orgueil, des manières hautaines, un ton impérieux, en un mot tout ce qui peut humilier les hommes en leur faisant sentir leur faiblesse et leur infériorité. L’affabilité, la douceur, une compassion tendre, un profond respect pour les infortunés, un désir sincère d’obliger sont les qualités par lesquelles les grands devraient toujours se distinguer. La grandeur qui ne s’annonce que par sa dureté, sa fierté, son mépris, repousse tous les cœurs. Les bienfaits que lui arrache l’importunité sont regardés comme des insultes et ne font que des ingrats. Est-il rien de plus puéril et de plus bas que la vanité tyrannique de quelques grands qui ne paraissent désirer le pouvoir que pour se faire des ennemis ? Ils semblent dire à tout le monde : {\itshape respectez-moi, j’ai le pouvoir de vous exterminer}. Le pouvoir a-t-il quelque chose de flatteur s’il ne sert qu’à faire trembler et à s’attirer des malédictions ? La grandeur inaccessible n’est d’aucune utilité, la grandeur dépourvue de pitié est une férocité véritable. Un ministre impitoyable fait retomber sur son maître une partie de la haine dont il est lui-même accablé. Combien de révoltes ont été produites par les manières insupportables de quelques favoris incapables de contenir leurs humeurs ? Combien de guerres sanglantes n’ont eu pour cause première que l’insolence de quelque ministre altier dont la témérité a fait couler le sang des nations\phantomsection
\label{footnote52}\footnote{La hauteur insolente du marquis de Louvois à l’égard d’un Hollandais distingué fut, dit-on, la principale cause de la haine des Hollandais pour Louis XIV et des avanies qu’ils firent éprouver à ce prince durant la guerre pour la succession d’Espagne.} ! De quel frémissement tout ministre des rois devrait-il être agité quand il se voit forcé de leur conseiller la guerre la plus juste, surtout s’il réfléchit à toutes ses horreurs ! Ne doit-on pas trembler lorsqu’il se propose un impôt désolant, un édit dont la rigueur se fera sentir pour des siècles jusqu’aux extrémités d’un empire !\par
Mais le pouvoir et la grandeur, pour l’ordinaire, enorgueillissent le cœur de l’homme, l’enivrent et produisent dans sa tête une sorte de délire\footnote{« Fortuna nimium quem fovet, stultum facit. » Publius Syrus.}. On dirait que les grands ne cherchent qu’à se rendre terribles et s’embarrassent fort peu de mériter l’amour. Dans la classe élevée où la fortune les place, ils croient ne point tenir à leurs concitoyens, à la patrie, à la nation. Ce sont ces idées fausses qui rendent si souvent la grandeur odieuse et qui font tant d’ennemis au pouvoir.\par
L’éducation que l’on donne communément à ceux que leur naissance destine aux grandes places est presque aussi négligée que celle des princes qu’ils doivent un jour représenter. Indépendamment des lumières que ces emplois demandent, les personnes appelées à partager les soins de l’administration devraient surtout apprendre à connaître les hommes, à découvrir ce qu’ils sont, afin de savoir ce qu’ils leur doivent et la manière de les remuer d’une façon avantageuse à leurs propres intérêts.\par
L’éducation des grands devrait donc surtout leur enseigner la morale, qui n’est que l’art de se faire aimer des hommes, de les connaître, d’unir leurs intérêts aux nôtres. Mais dans presque tous les pays, ce n’est point le mérite ou la vertu qui appellent aux dignités : c’est la faveur, la cabale et l’intrigue.\par
On dirait que la volonté du prince ou la protection de ses favoris suffisent pour faire descendre sur un homme tous les dons nécessaires à l’administration d’un État.\par
Est-ce donc au milieu des affaires multipliées et compliquées, au milieu des intrigues et des pièges qu’un ministre peut apprendre son métier ? Pour se maintenir en place il négligera les affaires, il se reposera sur le travail des autres ; dépourvu de lumières, sa confiance sera perpétuellement trompée : il ne l’accordera qu’à des hommes pris sans choix, à des protégés qui, n’ayant acquis le droit de lui plaire que par leurs bassesses et leurs flatteries, contribuent par leur impéritie, leurs sottises, leurs vices et leurs trahisons même, à la chute de leurs protecteurs. Ainsi que les richesses, tout le monde désire le pouvoir et la grandeur, sans savoir en tirer parti pour sa propre félicité.\par
À quoi sert la puissance si elle ne fait obtenir l’attachement, la bienveillance, la considération sincère des hommes sur lesquels cette puissance nous fournit les moyens d’agir ? Pourquoi la disgrâce jette-t-elle communément un favori, un ministre, dans un abandon universel ? C’est qu’il ne s’est servi de son pouvoir pour obliger personne, ou qu’il n’a jamais obligé que des ingrats en ne répandant ses bienfaits et ses grâces que sur des êtres sans mérite et sans vertu.\par
Le mérite doit être cherché ; il se présente rarement à la cour des rois : la vertu, communément timide, n’oserait s’y produire. D’ailleurs, elle s’y trouverait presque toujours déplacée. Le mérite s’estime lui-même et ne consent point à se déshonorer par des bassesses et des intrigues. Au contraire, le vice effronté se montre avec audace dans un pays où il connaît les moyens de réussir. Il faut à des ministres intrigants et pervers des instruments qui se prêtent à toutes leurs fantaisies. La probité déconcerte les méchants, le mérite fait peur à la médiocrité, les grands talents alarment l’incapacité : ils n’ont pas la souplesse requise pour plaire à des hommes dont les intérêts ne s’accordent nullement avec ceux de l’équité.\par
Esclaves de la flatterie, les gens en place sont presque toujours entourés d’une foule de fripons ligués contre la vertu, de traîtres prêts à sacrifier leurs protecteurs à quiconque leur fait envisager quelque avantage à trahir leur confiance ou à les abandonner. Le serpent, à force de ramper, s’élève à des hauteurs inaccessibles aux animaux les plus légers, mais son venin n’en est que plus subtil par les efforts qu’il a faits pour monter.\par
La morale, qui seule apprend à connaître les hommes, à démêler les ressorts qui les font agir, à les juger, n’est donc pas une science inutile aux ministres, aux gens en place, aux puissants de la terre.\par
La vertu, que la grandeur dédaigne, qu’elle repousse, à laquelle souvent elle ne croit pas, est pourtant quelque chose de réel ? Oui, sans doute ; ce n’est que dans le cœur de l’homme de bien que l’on doit trouver l’attachement sincère, l’amitié véritable et la reconnaissance. On les chercherait vainement dans les âmes abjectes de ces sycophantes dont les ministres et les grands sont perpétuellement accompagnés ; ils sèment presque toujours dans une terre ingrate qui jamais ne produira que des épines et des ronces. Un ministre est presque toujours expulsé par les intrigues de ceux que ses faveurs n’ont fait que mettre à portée de lui nuire plus sûrement à lui-même.\par
Mais la puissance aveugle l’homme ; le ministre, le favori, le courtisan, trompés par leur amour-propre, se flattent que leur pouvoir ne doit jamais finir. Les exemples des fréquentes disgrâces dont ils ont été les témoins ne peuvent désabuser des personnages assez vains pour présumer que la fortune fera des exceptions pour eux ou que leur génie supérieur et leur adresse les tireront des écueils où tant d’autres ont échoué.\par
C’est, sans doute, cette illusion qui fait que tant de ministres en place travaillent sans relâche à seconder les efforts d’un despote destructeur, à démolir la puissance des lois, à renverser la liberté publique, à forger des fers à la patrie. Les imprudents ne voient pas que ces lois, cette liberté qu’ils accablent, ces barrières qu’ils renversent ne seront plus capables de les protéger eux-mêmes au jour de l’affliction\footnote{L’Histoire tant ancienne que moderne nous fournit des exemples aussi terribles que fréquents des revers que la fortune fit de tout temps éprouver à des ministres et à des favoris. Quoi de plus effrayant que la chute des Séjan, des Rufin, des Marigny, des connétables de Luynes, des Strafford, etc., etc., etc. En ce moment même, une nation longtemps opprimée jouit avec transport de la disgrâce méritée de deux ministres tyrans (le chancelier de Maupéou et l’abbé Terray). L’un, après avoir insolemment anéanti les lois et les tribunaux de son pays et cruellement dispersé les magistrats, s’est vu relégué à son tour dans une retraite isolée d’où il entend les cris de joie de tout un peuple s’applaudissant de sa chute. L’autre, après avoir sans pitié pressé les dernières gouttes du sang de ses concitoyens, malgré la dureté de son cœur insensible est forcé de rougir de la bassesse avec laquelle il s’est rendu le bourreau de sa nation. Que l’on compare le sort de ces vils instruments de la tyrannie avec celui dont, au milieu de sa disgrâce, jouissait peu auparavant un ministre noble, généreux, bienfaisant (le duc de Choiseul), que les cabales de ces monstres avait fait éloigner. Celui-ci dans la retraite trouva la sérénité, le contentement, l’amitié constante et fidèle, tandis que les autres n’y trouvent que la honte, la fureur impuissante, un abandon général, la haine des honnêtes gens.} !\par
Les ministres devraient apprendre à se défier des faveurs toujours trompeuses d’un despote qui, communément privé d’équité, de lumières et de reconnaissance, ne suit que ses caprices et n’est guidé dans ses affections et sa haine que par les impulsions de ceux qui pour quelques instants s’emparent de son faible esprit. Les services les plus fidèles et les plus signalés sont bientôt oubliés par des tyrans stupides incapables de les apprécier, et qui ne sont eux-mêmes que les esclaves et les instruments de ceux qui sont utiles à leurs passions momentanées. Il n’est point de ministre dont la faveur puisse contrebalancer auprès de son maître vicieux celle d’une maîtresse, d’un proxénète, d’un nouveau favori ; ceux qui contribuent aux plaisirs du prince l’intéressent bien plus que ceux qui n’ont que le mérite de bien servir l’État. Le bon ministre n’est assuré de la faveur que sous un maître éclairé et vertueux.\par
Les ministres sont donc eux-mêmes intéressés à la vertu d’un prince. Ainsi, loin de flatter ces despotes auxquels ils veulent sans cesse asservir la patrie, loin d’agacer contre les peuples ces lions déchaînés, ils devraient opposer la raison, la vérité, la justice, la terreur même à leurs emportements ; ils devraient se souvenir qu’il n’est point sans les lois de grandeur, de rangs, de privilèges assurés, qu’un gouvernement injuste, toujours guidé par le caprice, détruit en un moment tout ce qui déplaît à ses fantaisies, qu’à ses yeux les hommes les plus élevés, les plus capables, ne sont que des esclaves qu’un souffle fait rentrer dans la poussière. Chez les tyrans de l’Asie, le vizir qui a le plus contribué à soutenir ou étendre la tyrannie de son maître se voit souvent obligé de tendre humblement le col au cordon que l’ingrat lui envoie par ses muets. Tout favori d’un souverain devrait toujours se souvenir qu’il est un citoyen choisi pour assister de ses lumières un autre citoyen chargé par sa nation de l’administration générale. Tout ministre devrait sentir que servir un despote dans ses vues, c’est se rendre esclave avec sa postérité, c’est se dégrader soi-même, c’est s’exposer sans défense aux coups de la tyrannie, c’est renoncer au titre de citoyen pour prendre celui d’un traître. Tout ministre vertueux doit renoncer à sa place quand la perversité ou la tyrannie le mettent dans l’impossibilité d’être utile à sa patrie. Le ministre complaisant pour les caprices et les vices d’une cour dissolue sert aussi mal son maître que son pays. Un dépositaire de l’autorité, s’il n’a pas étouffé dans son âme tout sentiment d’honneur ou de pudeur, ne doit pas balancer à fuir et à remettre un pouvoir qui ne servirait qu’à lui attirer le mépris et la haine de ses contemporains et l’exécration de la postérité : le crédit d’un ministre de la tyrannie, communément de peu de durée, est suivi d’un opprobre éternel. La fonction de concussionnaire, d’exacteur, de bourreau de ses concitoyens, peut-elle paraître glorieuse et digne d’exciter l’ambition d’un homme d’honneur !\par
C’est par les ministres que les sujets jugent de leurs souverains, les aiment ou les haïssent, les estiment ou les méprisent. Les princes ont donc le plus grand intérêt de ne confier la puissance qu’à des hommes justes, modérés, vertueux, les seuls qui puissent faire sincèrement chérir et respecter l’autorité. Le souverain peut se tromper sur les talents de l’esprit mais il se trompera difficilement sur les mœurs dans la vie privée. Il doit savoir qu’un avare, un voluptueux, un homme livré aux femmes, un prodigue, un homme dur et dépourvu d’entrailles, un être frivole et léger ne peuvent être propres à faire aimer la puissance. La probité, l’amour du travail, l’affabilité, les bonnes mœurs sont des qualités plus importantes dans un ministre que le génie, toujours très rare, ou que l’esprit, qui très souvent s’égare et qui devient nuisible quand il n’est pas tempéré par le sang-froid de la raison. Un préjugé très commun persuade aux souverains comme au vulgaire que l’esprit seul suffit pour remplir les grandes places ; mais cet esprit est sujet à des fâcheux écarts quand il n’est pas uni à la bonté du cœur. L’esprit et le génie joints à la justice, à la droiture, à l’expérience, aux bonnes mœurs, constituent le grand homme d’État, le ministre qu’on révère. Elles en font un Sully, un Maurepas, un Turgot, un ministre citoyen qui jamais ne séparera les intérêts du pays de ceux de ses sujets.\par
Ce n’est pas seulement en servant l’injustice et la tyrannie que le ministre se rend coupable envers sa patrie, c’est encore en négligeant ses devoirs, en donnant à la dissipation, à l’intrigue, aux plaisirs, des moments qu’il doit aux affaires de l’État. L’homme en place appartient au public, à ses concitoyens ; s’il est léger, inappliqué, indolent, il peut se rendre aussi criminel que s’il était décidément méchant. Que de reproches, s’il rentrait quelquefois en lui-même, n’aurait-il point à se faire en réfléchissant que ses amusements, son inadvertance, son incurie font gémir une foule de citoyens indigents qui, après avoir bien mérité de l’État, achèvent de se ruiner en sollicitations inutiles et sont réduits à mendier dans une antichambre ? N’est-ce donc pas une cruauté véritable que de tenir suspendus entre l’espérance et la crainte des malheureux qu’une décision prompte aurait pu sauver du naufrage ? Mais au sein de l’abondance et des plaisirs, les grands n’ont aucune idée des angoisses des pauvres. Ils écrasent en passant, et même sans y songer, des milliers d’infortunés. Le sentiment des peines les plus communes aux hommes sera-t-il toujours ignoré de ceux qui peuvent et qui doivent les soulager ? Dans quelles transes ne devrait pas vivre un dépositaire du pouvoir s’il pensait que ses légèretés, ses inadvertances peuvent causer le malheur d’un grand nombre de familles honnêtes et les forcer à vivre dans les larmes et le désespoir ?\par
« Ne conseille pas aux princes ce qui leur plaît, dit Solon, mais ce qui leur est utile. » Un ministre complaisant et flatteur ne fait qu’alimenter dans l’esprit de son maître les vices dont et ce maître, et l’État, et lui-même seront un jour les victimes. La véracité devrait être la première vertu d’un ministre fidèle ; fait pour voir de plus près que le prince les besoins, les désirs, les malheurs de son peuple, il ne peut, sans trahir et son pays et son maître, le tromper ou lui dissimuler la vérité. Le prince doit être touché quand ses sujets sont dans la peine ; il doit trembler quand ils sont mécontents. C’est lui qui par état doit connaître les maux et les dispositions de son peuple. C’est à lui de faire cesser ses murmures et ses plaintes. Tout ministre fidèle doit être et l’œil du maître, et l’organe du peuple. Ces courtisans flatteurs qui craignent d’inquiéter les rois ou de les affliger sont des prévaricateurs et des traîtres ; un roi doit-il être tranquille lorsque sa nation est misérable ?\par
Mais sous des gouvernements imprudents, frivoles et corrompus, la vraie grandeur est méconnue. Ainsi que le despote, ses favoris sont des enfants qui, contents de jouir de quelques avantages frivoles et passagers, ne portent guère leurs vues sur l’avenir. Chacun cherche à tirer parti de sa puissance éphémère et s’embarrasse fort peu de ce que deviendront après lui et le prince et l’État. S’il est impossible que le pouvoir absolu forme de bons souverains, il n’est pas moins difficile qu’il forme des ministres vraiment attachés à leurs maîtres et fidèles à leurs devoirs.\par
Les citoyens les plus puissants, ainsi que les plus faibles, sont évidemment intéressés au maintient de l’équité. Ils peuvent trouver dans les lois des secours contre la noirceur et l’intrigue qui voudraient les accabler. La grandeur, pour être stable, doit se fonder sur la justice ; dès que cette vertu règne dans la société, elle soutient tous ses membres, elle empêche que personne ne soit puni sans cause ou injustement opprimé. Cette justice universelle et sociale est un rempart bien plus sûr contre la violence que de vains privilèges, des titres inutiles, des distinctions frivoles que le caprice peut donner et reprendre. Peut-on se regarder comme quelque chose quand la puissance et la grandeur dont on jouit dépendent uniquement de la fantaisie d’un despote, d’une maîtresse ou d’un vizir ? Le citoyen obscur, sous un gouvernement libre, n’est-il pas plus assuré de ses droits que le ministre le plus accrédité sous l’empire du despotisme, qui n’est qu’une mer orageuse perpétuellement soulevée par des vents opposés ? Tout despote est un enfant volontaire et méchant qui se plaît à briser les jouets dont il s’est amusé.\par
Si les ministres ou les personnes revêtues du pouvoir sont destinées à représenter un souverain équitable dans les différentes parties de l’administration, ils doivent le faire chérir des peuples, être justes comme lui, rendre aimable son autorité. Un des principaux devoirs du ministre et de l’homme en place est donc d’être accessible, de recevoir avec bonté les demandes ou les représentations des sujets, de leur rendre une justice impartiale et prompte. Un ministre dur, sec, inaccessible, nuit à la réputation de son maître. Celui qui n’est qu’homme de plaisir fait tort à ses affaires ou devient inutile. Le ministère doit être exact et sérieux ; il demande non de la hauteur mais de l’attention, de la gravité dans les mœurs, la décence convenable à un état fait pour être respecté. Le ministre qui n’a des oreilles que pour ceux qui l’entourent, sera perpétuellement trompé et risquera de passer pour ignorant, pour faible, et souvent pour injuste ou corrompu.\par
Un des plus grands malheurs attaché à la grandeur et au pouvoir, c’est que celui qui les possède est obligé de craindre sa famille, ses amis les plus chers et de se mettre en garde contre les sentiments de son propre cœur. Son attachement pour l’État doit l’emporter toujours sur ses liaisons particulières : l’homme public n’est plus le maître des mouvements de sa tendresse ; il ne doit recevoir l’impulsion que de la justice et de l’intérêt de l’État, desquels il doit faire dépendre son honneur et sa gloire. Un ministre qui n’est bon que pour les siens est un homme dont l’âme est faible et rétrécie. « Je ne ferai point ce que vous demandez, vous êtes trop de mes amis », disait un homme digne de sa place à l’un de ses favoris qui lui faisait une demande peu équitable.\par
Un ministre prodigue ou qui ne peut rien refuser n’est pas un homme bienfaisant. C’est un homme faible, un administrateur infidèle, un prévaricateur. On se rend très coupable en répandant les trésors de l’État pour se faire des créatures ; tout ministre qui fait le bien n’a besoin ni d’adhérents ni de cabales : l’innocence de sa conduite doit lui suffire pendant qu’il est en place, et sa conscience doit être sa force et son appui lorsqu’il en est sorti. Jeter les richesses de l’État à la tête des courtisans faméliques ou des grands toujours avides, c’est arracher le nécessaire au malheureux, dont les besoins réels doivent être préférés aux besoins imaginaires de la vanité.\par
Quoi ! Les hommes les plus riches sont-ils faits pour absorber tous seuls les richesses et les récompenses des nations ? Non, sans doute ; elles sont principalement destinées à payer, à ranimer, à consoler le mérite laborieux, l’indigence timide, le talent dans la détresse, les services rendus à l’État. C’est à la probité réduite à la misère que l’homme en place doit tendre une main secourable. Le riche et le grand n’ont que trop de ressources et de manège pour obtenir les objets de leurs désirs souvent injustes et criminels. Ce n’est le plus souvent que pour opprimer l’innocent, étouffer le cri de l’infortuné, dépouiller le citoyen, jeter le faible dans les fers, que des courtisans odieux importunent le ministre qu’ils veulent rendre complice de leurs iniquités. Sous un gouvernement injuste les grands se croient dégradés s’ils n’ont pas le privilège affreux de faire du mal aux autres ; c’est en cela qu’ils font communément consister leur prééminence.\par
Par une fatalité trop commune, les hommes qui devraient se distinguer par l’élévation de leurs âmes montrent souvent une petitesse inconcevable ; ils ne semblent occupés que de vanités, de minuties, de jouets auxquels ils ont la folie de sacrifier leur repos, leur fortune, leur sûreté propre, la liberté de leurs descendants et de leurs concitoyens. On dirait que la grandeur d’âme et la raison ne sont point faites pour les grands et que les personnages élevés au-dessus des autres ne s’en distinguent réellement que par leur imprudence et leur folie !\par
Un étrange renversement des idées fait que les grands, pour la plupart, s’imaginent ne point jouir du pouvoir s’ils ne peuvent en abuser. Crédit, pouvoir, privilège, grandeur deviennent des synonymes de licence, de corruption, d’impunité. Les souverains et leurs suppôts ne veulent que se faire craindre et s’embarrassent fort peu de se faire estimer ; ils ne désirent la puissance que pour écraser tous ceux qui leur déplaisent, sans s’occuper du soin de mériter l’affection de personne. Dans l’esprit de la plupart des grands, être puissant c’est être redoutable et par conséquent haïssable ; être grand c’est jouir du droit d’être injuste, de faire du mal impunément, de se mettre au-dessus des lois, d’opprimer le faible et l’innocent, de mépriser et d’insulter le citoyen obscur et malheureux, de fouler aux pieds ce que les hommes ont de plus respectable. Être grand, aux yeux du vulgaire imbécile, c’est annoncer son rang par des palais somptueux, par des possessions amples et souvent injustement acquises, par des équipages élégants, par des chevaux, par un cortège de valets insolents, par des habits magnifiques, par des rubans et des colliers faits pour indiquer la faveur du prince ou de ses ministres ; c’est souvent, sans richesses réelles, représenter aux dépens d’une foule de créanciers qu’on immole indignement à sa vanité.\par
Enfin, être grand, c’est avoir par sa naissance le droit d’aller grossir la troupe des esclaves titrés qui vont lâchement faire la cour à un despote ou recevoir les dédains d’une idole qui laisse à peine tomber ses regards sur la foule avilie dont elle est environnée. C’est dans ces bassesses ou dans ces crimes que les peuples eux-mêmes font consister la grandeur des citoyens qui les accablent ! Plus un gouvernement est injuste, plus les grands sont insolents ou fastueux. Ils se vengent sur le pauvre des avanies qu’ils essuient souvent eux-mêmes ; ils masquent leur esclavage et leur petitesse réelle sous le vain appareil de la magnificence. Une cour bien brillante annonce toujours une nation misérable et des grands qui se ruinent pour ne le point paraître.\par
Aux yeux de la raison, le pouvoir et la grandeur ne sont des biens désirables que parce qu’ils peuvent fournir les moyens de se faire estimer et chérir. Être véritablement grand, c’est montrer de la grandeur d’âme, avoir du pouvoir et du crédit, c’est être en état de se garantir de toute injustice et de protéger les autres. Jouir de privilèges stables et de prérogatives assurées, c’est les posséder en commun avec tous ses concitoyens. Être libre, c’est ne craindre personne et ne dépendre que de lois solidement fondées sur l’équité. Avoir de la puissance, c’est posséder les moyens de faire du bien aux hommes et non le fatal pouvoir de leur nuire, c’est jouir de la faculté de faire des heureux et non de l’affreuse licence d’insulter aux misérables. C’est être maître de soi et refuser de se rendre esclave, c’est être à portée de répandre ses bienfaits sur les autres et non pas pratiquer l’art infâme de les ruiner par des escroqueries punissables. Être noble, c’est penser noblement, c’est avoir des sentiments plus élevés que le vulgaire ; être {\itshape titré}, c’est avoir acquis des droits incontestables à l’estime de ses concitoyens. Être {\itshape homme de qualité}, c’est avoir les qualités faites pour se distinguer du commun des mortels. Qu’est-ce que des grands qui ne se distinguent des autres que par des mots, des habits, des rubans ?
\subsection[{Chapitre V. Devoirs des Nobles et des Guerriers}]{Chapitre V. Devoirs des Nobles et des Guerriers}
\noindent L’on appelle {\itshape noblesse}, parmi nous, la considération attachée dans l’opinion publique aux descendants de ceux qui ont bien servi la patrie. En reconnaissance des services de leurs ancêtres, la société les {\itshape distingue}, c’est-à-dire leur marque plus d’estime qu’aux autres. Cette considération, ces distinctions accordées même au souvenir d’une utilité passée, furent sans doute imaginées pour encourager ces descendants à marcher sur les traces de leurs pères et à se distinguer comme eux par leurs talents et leur zèle. Tout citoyen qui contribue à la félicité publique doit être réputé {\itshape noble}, c’est-à-dire mérite d’être préféré à ceux qui ne procurent aucuns avantages à leurs associés.\par
Sur ce principe, toute société pour son propre intérêt doit témoigner une considération particulière à des guerriers généreux qui, aux dépens de leur fortune et de leur vie, s’occupent du soin de la défendre contre ses ennemis. Elle doit pareillement une considération distinguée aux magistrats chargés de maintenir l’équité entre ses membres et de contenir les passions qui troubleraient son repos. Le droit de rendre justice à ses concitoyens est la fonction la plus utile et la plus noble à laquelle un citoyen puisse se livrer ; si l’homme de guerre défend son pays contre les ennemis du dehors, le magistrat le défend contre les ennemis renfermés dans son sein, non moins dangereux que les premiers. Si l’homme de guerre consacre sa vie à la défense de la patrie, le magistrat dévoue la sienne et sacrifie son temps au maintien de la justice, sans laquelle nulle société ne pourrait subsister. « Il faut, dit Cicéron, anéantir l’opinion de ceux qui s’imaginent que les vertus guerrières sont plus estimables que celles qui ont pour objet l’intérieur de l’État\footnote{Voyez Cicéron, {\itshape De Officiis}, I.}. »\par
Par la même raison, les nations doivent accorder une place distinguée dans leur estime à tous les citoyens que leurs talents et leurs mérites divers mettent à portée de leur rendre des services éminents. La société, sous peine d’être injuste et de décourager les membres qui pourraient contribuer à son bien-être, doit proportionner sagement sa considération et ses récompenses à l’étendue des avantages dont on la fait jouir. « Tous, dit Sénèque, peuvent aspirer à ce qui fait la vraie noblesse de l’homme : c’est la droite raison, l’esprit juste, la sagesse et la vertu. » Telles sont les qualités qu’une association équitable doit honorer et récompenser dans ses membres.\par
Dans toute nation il s’établit donc nécessairement une sorte de {\itshape hiérarchie} politique dont le souverain est le chef, parce qu’il dirige les volontés et les mouvements des différents corps de la nation. En conséquence, le prince devient le distributeur des grâces au nom de la société, le dispensateur de ses récompenses ; chargé de la reconnaissance publique, il juge et du mérite des citoyens et de l’étendue de l’estime que l’on doit leur montrer. S’il est juste, la société applaudit son jugement et la fidélité qu’il montre à payer les services qu’on lui rend ; s’il est injuste, la société contredit ses jugements comme capables de décourager le mérite et les talents nécessaires à son bonheur, et refuse sa considération à celui qu’elle trouve injustement récompensé.\par
Lorsqu’un prince ennoblit un citoyen ou lui donne quelque titre honorable, il déclare à sa nation qu’un tel homme ayant bien mérité d’elle paraît digne d’occuper un rang distingué parmi ses concitoyens et a des droits fondés à leur reconnaissance. Si la faveur, l’intrigue, la bassesse ont fait obtenir cette nouvelle distinction, la société, loin de souscrire aux honneurs accordés en pareil cas, loin d’accorder à l’homme ainsi décoré son estime ou sa gratitude, le punit par le ridicule, le rejette, en appelle de la décision du souverain surpris ou prévenu. Nul monarque, quelque absolu qu’il puisse être, ne peut subjuguer l’opinion publique au point de lui faire considérer ou respecter un citoyen qui n’est ni estimable ni respectable par lui-même.\par
Elle respecte encore bien moins une noblesse acquise à prix d’argent, qui ne suppose dans celui qu’elle décore que des richesses et non le mérite et les talents auxquels la reconnaissance publique est due ; ce moyen vil d’obtenir des distinctions fut un effet de l’avarice de quelques princes qui surent tirer parti de la vanité de leurs sujets opulents en leur vendant bien cher la fumée dont elle voulut se repaître. Mais les souverains furent privés par là d’un moyen facile de récompenser le vrai mérite : ils donnèrent à la richesse une distinction qui, sagement économisée, eût été très utile pour exciter le mérite. Par ce honteux trafic la noblesse fut prostituée à des hommes nouveaux qui, sans avoir bien mérité de la république, furent en droit de jouir de privilèges souvent très incommodes pour le reste des citoyens.\par
Mais l’opinion publique ne put jamais souscrire à ce commerce déshonorant et visiblement contraire au bien de la société ; d’ailleurs il se trouvait opposé à des préjugés antérieurs. Les nations, peu disposées à reconnaître la prééminence de tant de nobles nouveaux et sans mérite, réservèrent leur considération pour une noblesse plus antique, qu’elles voyaient perpétuée dans la postérité des anciens défenseurs de la patrie. Tout ce qui porte le caractère de l’Antiquité, que l’on crut toujours très sage, en impose aux nations. Ainsi, par un préjugé confirmé depuis des siècles, les peuples continuent de respecter les descendants de ces antiques guerriers sans examiner les mérites de leurs ancêtres et, bien plus, sans s’assurer si ces descendants ont eux-mêmes rendu quelques services réels à la patrie. Comment un homme peut-il se croire honoré par ce qui n’est point à lui ? Est-ce donc hors de soi que l’on peut chercher la véritable grandeur ?\par
Ainsi, des préjugés anciens s’opposèrent aux distinctions nouvelles introduites dans la société ; les peuples stupides admirèrent la noblesse antique uniquement parce que leurs pères l’avaient longtemps redoutée et respectée. Une routine aveugle décide de l’opinion des hommes, qui rarement se rendent raison des motifs de leurs façons de penser et d’agir ; par une espèce de contagion, ils héritent même de préjugés avilissants pour eux.\par
Pour peu que, la balance de la raison et de la justice en main, l’on pèse les idées que l’on se forme en Europe de la noblesse antique, qui va jusqu’à la révérer même dans ses rejetons les plus éloignés, on sera forcé de convenir que cette opinion n’a rien de solide. On trouvera que ces anciens guerriers desquels les nobles d’aujourd’hui ont tiré leur origine, ont bien plus souvent troublé la patrie qu’ils ne l’ont servie, ils ont plutôt contribué à lui forger des chaînes qu’à lui procurer des avantages réels ; s’ils l’ont fidèlement défendue contre les ennemis du dehors, ils l’ont communément livrée aux ennemis du dedans en la soumettant au pouvoir des tyrans.\par
Même en supposant la grandeur et la réalité des services rendus à la patrie par les anciens héros des nations, la reconnaissance de celles-ci n’aurait au moins pas dû s’étendre jusqu’à leur postérité la plus reculée. Si l’équité défend de punir les descendants des crimes de leurs ancêtres, elle ne peut exiger que l’on récompense sans fin ces descendants des vertus ou des talents de leurs aïeux. La vertu ne se transmet point avec le sang ; le mérite est une qualité personnelle. Ainsi, la raison et l’intérêt public sembleraient exiger que les honneurs, les distinctions, la noblesse, au lieu d’être héréditaires, demeurassent entre les mains d’un gouvernement équitable comme des moyens sûrs d’exciter à servir utilement l’état et de récompenser ceux qui auraient vraiment contribué à sa félicité présente. Est-il juste en effet qu’un homme, dont souvent la race ignorée a croupi pendant des siècles dans le fond de ses terres sans rendre à l’État aucun service marqué, jouisse d’une considération et de privilèges destinés à récompenser la valeur guerrière ? Est-il juste que l’homme inutile soit honoré, distingué, respecté, récompensé par des prérogatives immenses au détriment du citoyen laborieux, parce qu’il y a sept ou huit siècles qu’un des ancêtres du noble a porté les armes pour son pays ? Que cet homme possède les terres jadis accordées à ses pères, mais l’équité semblerait exiger que, s’il prétend jouir des distinctions et privilèges de la noblesse, il les méritât lui-même et cessât de s’enorgueillir des prouesses de ses aïeux qu’il n’a point imitées. « L’estimation, dit Montaigne, et le prix d’un homme consistent au cœur et en la volonté : c’est là où gît son vrai honneur\footnote{Voyez {\itshape Essais}, livre I, chap. 30.}. »\par
La vanité est le vice de la noblesse : fondé sur des opinions dont nous venons de reconnaître la frivolité, le noble se croit réellement un être d’un ordre supérieur au reste des citoyens : on dirait que pétri d’un limon bien plus pur, il n’a rien de commun avec le reste de ses compatriotes. « L’illusion de la plupart des nobles, dit M. Nicole, est de croire que leur noblesse est en eux un caractère naturel. » Un autre moraliste avait dit avant lui : « À le bien prendre, la noblesse est un don du hasard, une qualité d’autrui. Qu’y a-t-il de plus inepte que de se glorifier de ce qui n’est pas sien… Ceux qui n’ont pour eux que cette noblesse la font valoir et en parlent toujours. Toute leur gloire est dans les tombeaux de leurs ancêtres… Que sert à un aveugle que ses pères aient eu la vue bonne… Être issu de gens qui ont bien mérité du public, c’est être obligé de les imiter\footnote{Voyez Charron, {\itshape De la Sagesse}, livre I, chap. 59.}. »\par
Il pouvait ajouter que le mérite réel ou prétendu de ses pères ne donnait point au noble le droit de marquer du mépris à ses concitoyens, et qu’une vanité rebutante n’était propre qu’à faire oublier ce mérite, quand même il eût été plus réel que l’Histoire ne semble l’indiquer.\par
Les annales de toutes les nations nous montrent en effet dans les anciens nobles un corps de guerriers turbulents perpétuellement divisés entre eux pour des querelles aussi injustes que futiles, uniquement occupés à se tourmenter les uns les autres ou à faire sentir cruellement le poids de leur autorité à leurs vassaux et à leurs serfs. Nous voyons ces furieux continuellement en guerre, déchirant les nations par leurs sanglants démêlés. Nous les voyons imposer à leurs sujets des devoirs souvent aussi bizarres que tyranniques, et s’en faire des droits. Nous voyons, dans ces temps de troubles et d’infortunes, les rois beaucoup trop faibles pour réprimer les violences de ces frénétiques sans cesse occupés à s’entre-détruire, méprisant l’autorité souveraine, se révoltant contre elle toutes les fois qu’elle entreprit de les contenir. Des meurtres, des vols, des rapines, des infamies sont les titres respectables que la noblesse nous présente dans l’Histoire. Enfin, cette noblesse toujours en délire et en discorde, toujours séparée d’intérêts du reste de la nation, succomba sous la force agissante et réunie des princes ambitieux qui domptèrent ces guerriers si fiers, au point de les réduire à solliciter l’avantage de jouer le rôle d’esclaves à la cour ou de devenir les satellites et les soutiens des plus injustes tyrans contre leur patrie et leurs concitoyens. Une servitude volontaire peut-elle être compatible avec la vraie noblesse ? « Tout homme, dit Sophocle, qui est entré libre dans le palais des rois, y devient bientôt esclave. »\par
Telle fut et telle dut être nécessairement la fin des excès continuels d’une noblesse ignorante, agitée, imprudente, qui jamais ne connut ses véritables intérêts. Une sotte vanité, des privilèges souvent injustes obtenus ou arrachés des souverains rendirent en tout temps les nobles et les grands insociables ; ils crurent qu’il ne leur convenait pas de faire cause commune avec des {\itshape roturiers}, des {\itshape vilains}, des {\itshape bourgeois} ; ils les dédaignèrent, les écrasèrent, et la nation n’eut plus de forces qu’elle pût opposer au despotisme ; celui-ci vint à bout d’accabler successivement tous les ordres de l’État\footnote{Les grands et les nobles polonais arrachèrent de Louis, roi de Pologne et de Hongrie, le privilège de n’être jugés que par eux-mêmes afin de se soustraire aux tribunaux ordinaires, ce qui leur procura l’impunité de tous les crimes et fit régner une anarchie qui a fini de nos jours par amener la destruction et le démembrement de ce royaume. Frédéric I, roi de Danemark, pour obtenir les secours des nobles de son royaume, fut obligé de leur livrer les peuples pieds et poings liés. Il leur donna le droit de vie et de mort sur leurs paysans et celui de les condamner à la perte de leurs biens immeubles, sans appel aux tribunaux ordinaires. Voyez Mallet, {\itshape Histoire de Danemark}, tome 4, p. 10.}. Un esprit de corps, toujours contraire à l’esprit patriotique, causa la perte des États et l’avilissement de la noblesse elle-même.\par
Par un préjugé contraire à toute justice, les hommes se croient faibles et malheureux quand ils n’ont pas le droit de faire du mal à ceux qu’ils voient au-dessous d’eux. Le crédit, le pouvoir, les prérogatives ne sont pour l’ordinaire que la faculté d’opprimer les plus faibles et de leur faire sentir le poids de son autorité. « Ceux même, dit Juvénal, qui ne veulent tuer personne, désirent d’en avoir la puissance\footnote{Juvénal, {\itshape Satires}, X, vers 96.}. » Les insensés ne voient pas que le pouvoir le plus désirable est celui qui se fait aimer ! Ils ne sentent pas que la force injuste peut être domptée par une force plus grande ! Enfin, ces nobles qui mettaient au nombre de leurs privilèges le droit infâme de tourmenter et de piller, de faire périr leurs malheureux sujets, ne s’apercevaient pas que cette anarchie et ces désordres frayaient une route facile au despotisme. Les peuples opprimés aiment toujours mieux avoir un seul tyran que d’obéir à cinquante dont les discordes sont un malheur continuel\footnote{La tyrannie des nobles détermina les Danois en 1660 à déférer le pouvoir absolu au roi. La mauvaise administration du Sénat de Suède fut en 1772 la cause de la dernière révolution arrivée dans ce royaume.}.\par
Tant d’exemples mémorables qui prouvent ces tristes vérités ne devraient-ils pas ouvrir les yeux de la noblesse et lui prouver que rien n’est plus contraire au bien de la société, à la prospérité nationale, à la saine politique, à la saine morale, que cet orgueil imbécile qui la sépare du corps des nations ? Tous les citoyens d’un même État, grands ou petits, nobles ou roturiers, riches ou pauvres, étant membres du même corps, ne sont-ils pas destinés à s’aimer, à se soutenir, à travailler de concert à la félicité publique ? De quel droit le noble mépriserait-il le laboureur qui le nourrit et l’enrichit, l’artisan qui le vêtit, le commerçant qui lui procure les agréments de la vie, l’homme de lettres qui l’amuse et l’instruit, le savant qui travaille pour lui ?\par
Mais par une suite de ses préjugés, la noblesse trop souvent dédaigne de s’instruire et semble même se glorifier de son ignorance\phantomsection
\label{footnote53}\footnote{Le tyran Licinius disait que la science était la peste pour un État. Un roi de Castille ayant dit que {\itshape l’étude des sciences ne convenait pas à un noble}, Alphonse, roi d’Aragon à qui on rapporta ce propos, s’écria {\itshape que ce mot était d’un bœuf et non pas d’un homme}.}. Presque toujours destiné au métier de la guerre, que de sottes préventions lui font regarder comme seul digne de lui, le noble méprise la science et cherche rarement à s’éclairer. S’il est d’une race illustre ou favorisée du prince, il se tient assuré de parvenir aux grades les plus élevés sans se donner le soin pénible d’acquérir des talents. Si le noble est ignoré de la cour, il ne se livre point au métier de la guerre : il vit totalement inutile et désœuvré dans les possessions de ses pères, où souvent il exerce une tyrannie fatale à ses vassaux.\par
Les héros et les grands capitaines de l’Antiquité, qui ne le cédaient en rien à nos guerriers modernes pour le courage et les talents militaires, ne dédaignaient pas de s’instruire dans les écoles de la philosophie. Les Épaminondas, les Périclès, les Alexandre ne regardaient pas la culture de l’esprit comme un ornement superflu dans un homme de guerre. Scipion, le vainqueur de Carthage, vivait dans la plus grande intimité avec Térence l’affranchi ; ce grand homme cultivait les lettres et la philosophie : « Il n’était, suivant Cicéron, jamais plus occupé que lorsqu’il paraissait vivre dans le plus profond repos. »\par
Il n’est point de citoyens qui eussent un plus grand besoin de la ressource des lettres et des sciences que ces nobles et ces guerriers qui parmi nous se font gloire de tout ignorer. C’est à l’ignorance et à l’oisiveté fastidieuse, auxquelles trop souvent la noblesse moderne se condamne, que l’on doit attribuer les vices, les excès et les bassesses par lesquels on la voit souvent se déshonorer. Le guerrier n’est en action que pendant un temps très court relativement à la durée de sa vie. Ses fonctions une fois remplies, il n’a plus rien à faire ; la paix le plonge dans une indolence, une paresse complète. Alors vous le voyez, aux dépens de sa fortune, se livrer immodérément au jeu, à la débauche, à la galanterie, aux désordres de toute espèce, à des dépenses ruineuses. Enfin, sa fortune délabrée l’oblige à contracter des dettes, à devenir escroc et fripon, à {\itshape vivre d’industrie}, et souvent à se permettre des choses qui feraient rougir les derniers des citoyens.\par
C’est au désœuvrement des nobles et des guerriers, à leur passion pour le jeu, à leur libertinage, et surtout à leur vanité turbulente, que l’on doit attribuer leurs querelles fréquentes qui se terminent si souvent par des combats sanglants. L’honneur, chez nos guerriers modernes, n’est pas la juste estime de soi confirmée par les autres : celle-ci ne peut être fondée que sur le sentiment de sa propre dignité que donne la vertu seule ; cet honneur futile est bien plutôt la crainte d’être méprisé parce que l’on se reconnaît réellement méprisable. Se battre ne prouvera jamais que l’on a de l’honneur ; un duel ne prouve rien sinon beaucoup d’impatience, de vanité, d’étourderie, qualités très opposées à la force, à la vraie grandeur d’âme, à l’humanité. L’homme d’honneur est celui qui mérite d’être honoré. Qu’y a-t-il d’honorable dans une petitesse accompagnée de cruauté ? Les fameux capitaines de la Grèce et de Rome, avec autant de bravoure et d’honneur que nos guerriers modernes, supportaient une insulte et ne cherchaient point à la laver dans le sang de leurs concitoyens\footnote{Dans les siècles barbares de l’Europe, la religion et la politique approuvaient également les combats singuliers, et l’on en regardait le succès comme un jugement du Ciel qui toujours était censé se déclarer contre le coupable. Depuis ce temps les lois religieuses et civiles ont vainement tenté d’abolir ces usages inhumains. Aujourd’hui dans toute l’Europe l’homme qui se bat en duel s’expose à périr sur un échafaud et celui qui refuse de se battre se trouve déshonoré. Si l’on eût voulu supprimer les duels, il eût fallu commencer par rectifier l’opinion nationale en attachant l’infamie à quiconque s’en serait rendu coupable. Si l’on eût déclaré infâme et dégradé tout noble qui se serait battu, l’on eût fait plus d’impression que par la crainte de la mort, que l’homme de guerre est fait pour mépriser. Fabius disait que « celui qui ne peut endurer une injure est plus poltron que celui qui fuit devant l’ennemi. » Tout le monde connaît le trait de Thémistocle sur lequel Eurybiade dans un conseil de guerre leva la canne comme pour le frapper. Thémistocle, peu sensible à cet outrage, se contenta de lui dire froidement : « Frappe, mais écoute. » Ceux qui prétendent que l’esprit militaire a besoin de duels pour être maintenu n’ont qu’à lire l’histoire grecque et romaine. Ils y verront que des guerriers redoutables pour leurs ennemis n’avaient pas la folie de s’égorger les uns les autres pour des gestes ou pour des mots.}.\par
Si les distinctions attachées à la noblesse ont le mérite et la vertu pour fondement réel ou supposé, si cette noblesse veut avoir véritablement de l’honneur, les nobles paraissent avoir pris des engagements plus forts que les autres de montrer à la société des talents et des vertus. « La vraie noblesse, c’est la vertu », dit Juvénal\footnote{Juvénal, {\itshape Satires}, VIII, vers 20.}. Ainsi, un noble ignorant, un noble sans mérite et sans talents, un noble bas et rampant, un noble avili par ses débauches, ses vices, ses dettes, ses friponneries, en un mot : un noble sans vertu, sont des contradictions dans les termes. Il n’est pas douteux que le plébéien le plus obscur, dès qu’il est honnête et laborieux, ne soit un citoyen plus estimable que le noble inutile ou pervers qui souvent se croit en droit de l’accabler de mépris : celui qui sert bien la patrie n’est jamais ignoble ou roturier. {\itshape Il y a}, dit un Arabe, {\itshape bien peu de nobles sur la terre}.\par
Que la noblesse cesse donc de s’enorgueillir des mérites et des services de ses pères. Qu’elle gémisse plutôt de leur aveuglement et de leurs crimes qui ont tant de fois anéanti le bonheur de la patrie. Qu’elle expie par ses bienfaits leurs folies si nuisibles et pour eux-mêmes et pour leurs concitoyens, qu’elle rougisse de ce qu’ils ont si souvent contribué à livrer leur patrie au joug du despotisme dont ils n’ont fait que se rendre les défenseurs et les premiers esclaves. Que cette noblesse renonce à son ignorance et à ses préjugés qui ne lui laissent d’autre profession dans la société que de s’immoler aux injustes caprices des conquérants : ceux-ci ne regardent leur noblesse que comme une pépinière de victimes destinée à servir leur propre ambition. Toujours dupe de l’opinion transmise par ses sauvages ancêtres et maintenue par une politique trompeuse, cette noblesse se dévoue et se ruine pour une vaine fumée ; enfin, séduite par la vanité, un luxe ruineux multipliant ses besoins, la force de renoncer à sa liberté et de ramper lâchement aux pieds des maîtres qui peuvent les satisfaire. Sous un gouvernement arbitraire, le luxe est un moyen puissant pour humilier les nobles et les forcer à recevoir le joug. L’honneur et le despotisme seront toujours incompatibles.\par
Il n’est point de citoyens à qui l’instruction, la vertu, les talents soient plus nécessaires qu’aux nobles et aux grands : destinés par état à régler le sort des nations, appelés aux conseils des rois, faits pour commander les armées et pour soutenir les empires, combien ne devraient-ils pas amasser de connaissances !\par
Mais par une fatalité trop commune, les hommes nés pour diriger les autres se rient de la vertu, méprisent la science et dédaignent l’instruction. Le militaire s’imagine que sa profession ne lui impose que le devoir de montrer du courage et de braver la mort. Ne voit-il donc pas que la guerre est un art qui suppose de l’expérience, des réflexions et quelquefois le génie le plus étendu ? La rareté des grands généraux ne prouve-t-elle pas suffisamment la difficulté de leur métier ? Ce n’est pas au sein des villes occupées de frivolités, ce n’est pas aux genoux des belles, ce n’est pas au milieu des intrigues d’une cour, ce n’est pas dans les antichambres des ministres qu’un capitaine peut apprendre à défendre sa patrie, à tracer des campements, à discipliner des soldats, à déployer des bataillons. Est-il rien de plus funeste pour l’État et de plus criminel que la présomption de ces généraux qui, dépourvus de lumières, ont l’audace de se présenter pour commander des armées dont les opérations décideront peut-être à jamais de la destinée d’un Empire ? Comment un général ose-t-il lever les yeux devant son maître et ses concitoyens lorsqu’il sait que son incapacité est la vraie cause des revers de son pays ? Son cœur ne devrait-il pas être déchiré de remords lorsqu’il y entend les cris plaintifs de tant de familles que son impéritie téméraire a plongées dans le deuil ? Quels reproches ne doit-il pas se faire en songeant aux légions que son imprudente vanité a fait inutilement égorger !\par
Que l’on ne dise donc plus que la science est inutile aux guerriers et que le courage leur suffit. Sans lumières, le courage n’est qu’une étourderie ou une férocité. L’étude, la réflexion, le savoir sont de la plus grande importance et pour les gens de guerre, et pour l’État dont ils sont les défenseurs. La morale, ainsi que la politique, se réunissent évidemment pour couvrir d’ignominie cette honteuse ignorance qui trop communément est l’apanage du militaire. L’officier, pour l’ordinaire, n’est guère plus instruit que le simple soldat. Suivre sans réflexion la routine du service, se battre en aveugle quand les chefs l’ordonnent, végéter dans l’oisiveté d’une garnison, languir dans un ennui qui n’est diversifié que par le désordre et la débauche : telle est la vie machinale et fastidieuse dans laquelle le militaire croupit jusqu’à sa vieillesse qui, bien loin de le faire considérer, le rend très méprisable ; voilà pour l’ordinaire ce qu’on appelle {\itshape servir}\footnote{« Avec la seule pratique sans théorie, dit M. de Puységur, on aura beau montrer des tranchées, on ne saura pas pour cela conduire une attaque devant une place, non plus que se précautionner contre des sorties. On se sera trouvé dans beaucoup de circonvallations, et l’on ne saura point en faire. On aura de même été dans des armées d’observations et vu faire tous les mouvements pour couvrir un siège, et l’on ne saura pas pour cela les diriger. » Voyez le {\itshape Traité de l’Art de la guerre}, par M. de Puységur.}. Pour avoir négligé d’amasser dans sa jeunesse les connaissances que l’étude et la méditation peuvent seules fournir, l’officier blanchi sous le harnais n’est souvent qu’un objet fatiguant pour lui-même et pour ses concitoyens. Un militaire sans culture, quelque vaillant qu’il puisse être, sera toujours inutile et méprisé durant la paix. Nonobstant les préjugés de la plupart des peuples qui font regarder la profession des armes comme la plus relevée, il n’est point de position plus déplorable que celle d’un vieux militaire sans fortune et sans lumières ; trompé souvent par un gouvernement ingrat au service duquel il s’est follement ruiné, il est forcé de solliciter en pure perte une pension modique pour subsister. Les princes et leurs ministres ne songent guère à répandre des bienfaits sur des sujets inutiles ; aigri par l’infortune, notre héros rebuté porte ses plaintes continuelles dans des cercles qu’il ennuie. Incommode à tout le monde, ses infirmités l’accablent et terminent dans la misère une vie qu’il eût été plus avantageux pour lui de perdre dans les combats. Les qualités du cœur et de l’esprit peuvent seules mériter une considération qui dure jusqu’au tombeau.\par
D’un autre côté, le militaire communément dépourvu d’instruction et de mœurs ne porte très souvent dans la société civile que la morale qu’il a puisée dans les garnisons, les camps et les armées. Cette morale, d’ordinaire peu délicate sur tout le reste, fait consister le mérite dans une férocité facile à ranimer, dans une rudesse habituelle ou dans une fatuité qui ne préviennent pas en faveur des guerriers et qui rendent leur commerce suspect et dangereux. Les devoirs et les règles que la morale, la raison, la saine politique imposent aux nobles et aux militaires, les obligent à s’attirer la considération publique et à mériter les honneurs, les grades, les récompenses (qui sont toujours accordées au nom et aux dépens de la nation) par leurs services réels, par leurs talents utiles, par leur attachement à leur pays. Bien loin de les mettre en droit d’opprimer ou de mépriser leurs concitoyens, leur rang au contraire les engage à leur donner l’exemple de l’équité, de la modération, de la vraie force, de la magnanimité, de la générosité, de l’amour du bien public. Les guerriers et les nobles sont communément des citoyens que tout devrait attacher le plus intimement à la patrie. Le mérite militaire consiste à défendre avec courage les personnes et les possessions de tous contre ceux qui voudraient les envahir. D’où l’on voit que l’homme de guerre deviendrait un traître et même un lâche s’il vendait sa vie au despotisme et à la tyrannie, qui furent toujours les plus implacables ennemis de toute société\phantomsection
\label{footnote54}\footnote{« Ce ne sont pas, dit Firmicus, des hommes courageux que ceux qui trafiquent de leur sang et qui s’exposent à la mort pour les caprices d’un autre. » Voyez Jul. Firmicus, livre VIII, chap. 13. « N’est-ce pas, dit Antiphane, être aux gages de la mort que de gagner de quoi vivre aux dépens de sa vie ? »}. Un guerrier assez fou pour s’immoler aux caprices d’un tyran n’est qu’un gladiateur mercenaire. Un citoyen qui donne des fers à son pays est un furieux qui met le feu à sa propre maison au risque de se ruiner lui-même avec sa postérité. Quel affreux héritage que de laisser à sa famille l’opprobre de la servitude\phantomsection
\label{footnote55}\footnote{Un Lacédémonien répondit à Indarnes, officier persan qui le sollicitait de demeurer en Perse : « Tu ne connais pas le prix de la liberté, car celui qui le connaît, s’il a du jugement, ne s’échangerait pas avec le royaume de Perse. » Voyez Plutarque, {\itshape Dits notables des Lacédémoniens}.} !\par
Obéir en aveugle, c’est à quoi se réduit toute la morale de l’homme de guerre. Mais si cette morale convient dans des camps et des armées, on ne doit pas l’enseigner dans les villes ou dans la société : elle ne ferait évidemment des guerriers que de pures machines, des instruments abjects qui dans les mains des tyrans anéantiraient les lois et la liberté. L’obéissance machinale à des chefs injustes est une trahison contre la patrie, que le guerrier doit défendre contre tous ses ennemis. Si cette obéissance est louable dans le simple soldat, toujours incapable de raisonner et de se former des idées de justice, elle est coupable et déshonorante dans ceux qui le commandent ; l’éducation devrait leur avoir inspiré des sentiments plus nobles et plus généreux qu’aux automates dont ils dirigent les mouvements. Mais la politique des tyrans prit soin d’élever en tout temps un mur d’airain entre les nobles, les soldats et ses autres sujets. La noblesse militaire, en formant une classe distinguée, se dévoua servilement aux volontés des plus mauvais princes, et leurrée par de vains privilèges, par des pensions et de vains titres, elle n’eut rien de commun avec les différents ordres de l’État. Tout guerrier fut l’homme du prince et se crut dégagé de tout lien envers sa nation ; il cessa d’être citoyen pour devenir un satellite, un mercenaire, un esclave. Les lois, la liberté, la justice, et avec elles la félicité, sont bientôt bannies des États dont les chefs ont à leurs ordres des troupes stipendiées.\par
Parler de patrie, de morale, de devoirs à ceux qui composent aujourd’hui les armées, c’est évidemment s’exposer à la risée. La vanité, l’étourderie, le libertinage, la paresse, le désir de jouir d’une licence impunie, voilà les motifs ordinaires qui portent une jeunesse inconsidérée à la profession des armes. Des guerriers de cette trempe sont tentés de croire que la raison, la réflexion, l’équité, la vertu, ne sont point faites pour eux. La morale semble devoir en imposer encore bien moins à des soldats grossiers choisis pour l’ordinaire parmi les fainéants, les vagabonds, des gens {\itshape sans feu ni lieu}, et même souvent les malfaiteurs, trop heureux de trouver dans une légion le moyen de se soustraire soit à l’indigence, soit aux châtiments qu’ils ont mérités\footnote{Xénophon attribue la décadence des Perses après Cyrus à la façon dont alors on formait leurs armées, qui n’étaient plus composées que d’une vile canaille ramassée à peu près comme on fait pour former les armées d’aujourd’hui.}.\par
Un gouvernement militaire influe de la façon la plus marquée sur les mœurs des nations ; chacun veut ressembler à ceux qui composent le corps le plus distingué : conséquemment, chacun affecte des manières militaires, chacun se montre vain, léger, sans soucis et sans mœurs.\par
Ce n’est pas ainsi qu’étaient composées ces armées courageuses des Grecs et des Romains dont l’Histoire nous a transmis les exploits : leurs généraux étaient des hommes désintéressés, instruits, guidés par la passion de la gloire ; les simples soldats n’étaient pas de vils mercenaires : c’étaient des citoyens, des cultivateurs, des propriétaires ; ils avaient une patrie qui leur était chère parce qu’elle renfermait et protégeait leurs femmes, leurs enfants et leurs biens. Ils combattaient avec force pour la liberté et non pour le despotisme ; la guerre terminée les rendait à leurs foyers où ils jouissaient des louanges de leurs concitoyens pour les avoir vaillamment défendus. La milice romaine, devenue mercenaire par la suite, cessa d’être animée du même esprit. Les soldats ne furent plus alors que les instruments détestables des ambitieux qui surent les gagner ; ils asservirent l’État à des tyrans qu’ils détruisirent à volonté. À force de massacres, de rapines, d’indiscipline, ils amenèrent la ruine de l’empire, qu’ils auraient dû défendre bien plutôt contre ses indignes maîtres que contre les Germains, les Parthes ou les Daces.\par
Tel est le sort que des troupes mercenaires préparent aux nations ! Telles sont les destinées de ces tyrans qui se confient à une soldatesque inconstante et perverse ! Celle-ci, après avoir démoli l’équité, la liberté, les lois, fière de ses succès et remplie d’avidité, finit par s’élancer en bête féroce sur le maître qui a déchaîné sa fureur. Les empereurs les plus justes, les plus sages, les Probus, les Alexandre Sévère furent les victimes de ces soldats forcenés à qui la vertu des princes était devenue odieuse. Enfin, tel est encore de nos jours le sort que des janissaires rebelles font éprouver à leurs sultans. Les despotes eux-mêmes ne peuvent pas toujours compter sur les esclaves qui gardent leur personne. Des bêtes féroces exterminent très souvent leurs gardiens. La licence et la corruption des soldats, que les princes semblent favoriser, devient aussi funeste aux maîtres qu’aux nations que ceux-ci se proposent d’asservir. Les instruments qu’emploie la tyrannie contribuent tôt ou tard à la destruction des tyrans.\par
Sous les gouvernements introduits par les peuples barbares qui partagèrent les provinces de l’empire romain, les généraux, les grands, les nobles, les guerriers, uniquement obligés de suivre les rois à la guerre, se rendirent peu à peu indépendants de leur autorité durant la paix ; ils furent de plus les représentants, les magistrats et les juges des nations réduites en servitude par la force de leurs bras. Mais quelle put être la justice que des serfs malheureux obtinrent de ces hommes brutaux, ignorants, nourris de carnage et de rapines ? Quelle protection les citoyens dédaignés trouvèrent-ils dans des nobles qui jamais ne songèrent qu’à stipuler leurs propres intérêts ? Les rois, trop faibles pour mettre à la raison des vassaux indomptés, les divisèrent comme on a vu, profitèrent de leurs dissensions et de leur impéritie pour leur associer dans les tribunaux des {\itshape clercs}\phantomsection
\label{footnote56}\footnote{On appelait {\itshape clerc}, dans les siècles d’ignorance, tous ceux qui avaient quelque teinture des lettres qui étaient alors réservées au clergé.} ou des juges plus instruits qui peu à peu remplacèrent ces guerriers incapables et formèrent la magistrature que l’on voit subsister en Europe.\par
Des représentants armés deviennent bientôt des tyrans redoutables pour le peuple et des sujets rebelles au souverain. Une noblesse militaire orgueilleuse de sa force méprise la justice et n’est pas faite pour juger les citoyens. Il faut aux nations pour les représenter des hommes justes, intègres, éclairés, soumis aux lois, inaccessibles aux séductions des cours, qui obligent le prince lui-même à respecter les droits de la société et qui surtout les respectent eux-mêmes. Des représentants vénaux ou faciles à séduire sont des traîtres qui bientôt tomberont dans les fers du despotisme après avoir sottement donné dans ses pièges.\par
Ainsi, faute d’équité, de raison, de science, la haute noblesse, qui jadis marchait presque l’égale des monarques, fut non seulement terrassée, dépouillée de son pouvoir, mais encore privée de la prérogative si noble de représenter et de juger les peuples. Sa chute ne devrait-elle pas apprendre à tous les grands que nulle puissance, quelque forte qu’elle paraisse, ne peut se soutenir sans justice et sans lumières ? Nul ordre dans l’État, nul corps ne peut sans péril séparer ses intérêts de ceux de la nation ; en un mot : la morale et les talents sont utiles et nécessaires à la noblesse et n’ont rien qui leur doive attirer ses mépris. « Un esclave, dit un poète, n’a pas droit de marcher la tête levée\phantomsection
\label{footnote57}\footnote{Voyez {\itshape Poetæ græci min., Theognidis Carmina}.}. »\par
La noblesse impose évidemment à ceux qui la possèdent le devoir de s’attacher plus fortement à la patrie que les autres. Plus on reçoit de la société et plus on doit lui montrer de gratitude et de zèle. Personne plus que le noble n’est intéressé à la prospérité de l’État qui renferme ses biens, où il jouit de la considération et des honneurs qu’il est fait pour désirer. Rien de plus légitime et de mieux fondé que le choix des souverains lorsque dans la distribution des emplois importants ils préfèrent les sujets les plus distingués par la naissance.\par
On doit supposer sans doute que des personnes bien nées ont été bien élevées, c’est-à-dire ont reçu de leurs parents des principes d’honneur, des sentiments généreux, une ambition noble, des qualités estimables, un esprit et un cœur soigneusement cultivés. Lorsque ces dispositions manquent au noble, il n’est plus qu’un homme du commun capable de nuire et au maître qu’il sert, et à ceux sur lesquels il a de l’autorité.\par
Mais pour être justement considéré, il n’est pas toujours nécessaire que le noble prodigue son sang dans les batailles ou remplisse des emplois distingués : lorsque dénué d’ambition il vit retiré dans les possessions de ses ancêtres, son opulence ou son aisance le mettent à portée de faire beaucoup de bien aux malheureux dont il se voit entouré. Un seigneur bienfaisant et puissant n’est-il pas et plus grand et plus heureux dans son domaine que ces grands qui s’exposent aux orages des cours ? Quand le noble ne jouit que d’une fortune médiocre, sa retraite le met à couvert des aiguillons de l’ambition ; elle lui dérobe le spectacle affligeant des indignes personnages que l’injustice élève si souvent aux honneurs. Ses besoins sont bornés parce qu’il n’est point infecté de la contagion du luxe ; il fait valoir en paix son champ, il cultive son esprit dans ses moments de loisir, il élève des enfants que leurs talents pourront un jour tirer de l’obscurité et faire paraître avec éclat dans le monde.\par
Mais le malheur cesse d’intéresser quand il est accompagné de vanité. Le rejeton vertueux d’une famille antique et déchue est un objet attendrissant qui nous rappelle les jeux cruels de la fortune ; un noble modeste est fait pour gagner plus sûrement les cœurs qu’un gentilhomme indigent et superbe. Trop souvent la hauteur ne quitte point la noblesse au sein même de la misère. Dans quelque position que le noble se trouve, il est fait pour {\itshape se sentir}, c’est-à-dire il doit se respecter lui-même, ne jamais s’avilir, être jaloux de l’estime des autres. Ces sentiments louables devraient-ils se confondre avec une vanité pusillanime, inquiète, avec une indolence honteuse, une crainte futile de se dégrader par un travail honnête ou par des talents estimables ? Les préjugés barbares qui subsistent encore font que dans bien des nations, tout noble se croit par l’unique droit de sa naissance fondé à dédaigner des emplois honorables, les ressources du commerce, et à mépriser ceux que le destin n’a pas fait naître comme lui ; nul talent, nulle vertu ne lui paraissent comparables à l’avantage d’être né de parents nobles. Ce préjugé pitoyable le rend souvent injuste, insociable, désagréable à tous ceux que le hasard n’a pas si bien servis. Il faut être singulièrement dépourvu de mérite personnel pour attacher tant de valeur à un pur accident !\par
Les hommes ne sont point égaux par la Nature ; ils ne sont point égaux par les conventions sociales qui, pour être équitables, ne doivent jamais mettre sur la même ligne l’homme inutile ou méchant et le citoyen vertueux. Le noble n’est respectable que lorsqu’il agit noblement : il ne mérite nullement d’être distingué de la foule quand ses sentiments et ses vertus ne tiennent point ce que semblait promettre son origine. Ses concitoyens sont en droit de lui dire : « Si vous êtes vraiment du sang de ces guerriers généreux qui se sont autrefois dévoués pour la patrie, prouvez-nous votre origine par des actions nobles, par une façon de penser digne de tels ancêtres. Si vous descendez des bienfaiteurs de nos pères, ne traitez point leurs descendants avec une hauteur insultante. Si vous voulez être honoré, méritez notre estime par vos vertus, par un attachement inviolable aux lois sacrées de l’honneur. Si vous êtes membre du corps le plus distingué de l’État, ne vous rendez pas complice des méchants qui, après avoir tout renversé par vos mains, anéantiront vos privilèges et vous mettront un jour au rang de ces plébéiens que vous avez la cruauté ou la folie de mépriser\phantomsection
\label{footnote58}\footnote{Un noble allemand ne fait aucune société avec un négociant. Les habitants de l’Hindoustan sont partagés en {\itshape castes} ou tribus dont les supérieurs non seulement méprisent les tribus inférieures, mais encore les maltraitent cruellement. Un {\itshape naïre}, ou noble du Malabar, a droit de tuer un {\itshape pouliet}, ou pauvre, qui l’aurait touché par mégarde. Les nobles chingulais traitent les plébéiens de la même manière tandis qu’ils ne s’approchent du roi qu’à quatre pattes et se qualifient de {\itshape chiens} quand ils lui parlent d’eux-mêmes. Un gentilhomme polonais peut tuer un paysan sans conséquence. En Europe un grand seigneur n’est tout au plus puni que par la prison pour les crimes et les assassinats, hormis en Angleterre où les lois ne font pas acception des personnes.}. »\par
Trop longtemps enivrés de distinctions frivoles, de prérogatives puériles et précaires, de vains titres, de prétendus droits quelquefois très injustes, les nobles se crurent des êtres d’une autre nature que le reste des hommes ; ils rougirent de confondre leurs intérêts avec ceux des bourgeois, qu’ils regardèrent comme des affranchis de leurs ancêtres. Autorisés par une jurisprudence féodale et barbare, ils exercèrent sur les peuples mille vexations juridiques. Le droit si noble de la chasse rendit les terres stériles, les campagnes furent dévastées et les cultivateurs ruinés pour l’amusement des seigneurs ; la vie des bêtes fauves devint plus précieuse que celle des hommes\phantomsection
\label{footnote59}\footnote{Les lois imaginées pour conserver la chasse sont atroces chez quelques peuples. On assure qu’en Allemagne des princes ont fait lier des braconniers sur des cerfs que l’on mettait ensuite en liberté dans les bois où ces malheureux étaient déchirés.}. Sous prétexte de maintenir leurs droits, les grands firent éprouver à leurs sujets les injustices les plus criantes. C’est un bel amusement, sans doute, un plaisir bien noble et bien grand que celui qui change de vastes contrées en forêts, en déserts, qui quelquefois anéantit les récoltes et qui coûte des larmes à cent familles désolées !\par
La morale et la politique s’élèvent également contre ces abus révoltants. Les nobles et les grands ne peuvent-ils donc s’amuser sans ravager leurs propres terres ou sans affliger les malheureux dont ils devraient être les protecteurs et les pères ? De quel œil le laboureur indigné doit-il voir son seigneur qui ne se montre dans les campagnes que pour y porter la disette et le désordre ? Mais l’humanité ne dit rien à des orgueilleux à l’abri de la misère ; ils rient des pleurs des misérables, ils s’applaudissent du pouvoir de tout oser contre la faiblesse impuissante. Que dis-je ! Ils châtieraient celui qui aurait la témérité de se plaindre humblement du mal qu’on lui fait éprouver\footnote{J’ai vu un grand seigneur menacer de la bastonnade et du cachot un paysan qui, lui servant le guide à la poursuite d’un cerf, lui avait fait faire un détour pour épargner un champ non encore moissonné.} !\par
Si les princes, les nobles et les grands, dans l’emportement de leurs plaisirs, sont incapables d’écouter la voix de la pitié, qu’ils écoutent du moins celle de leur propre intérêt. Qu’ils renoncent à des droits qui laissent en friche et dépeuplent leurs domaines, qui découragent et mettent en fuite les cultivateurs dont ils ont besoin pour contenter leur luxe et leur vanité, qui rendent la grandeur et la noblesse odieuses à des citoyens dont elles devraient mériter la tendresse et encourager les travaux. N’est-ce qu’en faisant du mal aux faibles que les grands croient montrer leur puissance et leur supériorité ? L’équité naturelle, dont les lois sont plus saintes que les folles conventions des hommes, met au néant des privilèges accordés par l’injustice, soutenus par la violence et confirmés par les siècles. Le pacte social exige que nulle classe de citoyens ne s’arroge le droit de tourmenter les autres ; il met le faible sous la sauvegarde du puissant, le cultivateur sous la protection de son seigneur : le château du noble est fait, ainsi que son cœur, pour être l’asile de ses villageois opprimés. Une noblesse vertueuse, citoyenne, éclairée, serait la protectrice et le modèle des peuples, ses membres bien unis seraient de droit les représentants des nations ; ils formeraient un rempart que jamais la tyrannie ne pourrait renverser. Des nobles oppresseurs, divisés, sans lumières et sans mœurs, après avoir accablé les peuples, finissent par être accablés à leur tour.\par
La vraie morale, toujours d’accord avec l’équité et la saine politique, ne doit pas se proposer de déprimer la noblesse mais de lui mettre sous les yeux ses engagements envers la société, de la rappeler à sa véritable origine, à son institution naturelle. La justice, toujours unie aux intérêts de l’État, ne peut pas se proposer d’introduire dans les nations une égalité démocratique qui bientôt dégénérerait en confusion. Tous les empires ont besoin de défenseurs animés par l’honneur ou à qui l’éducation ait inspiré des sentiments élevés. Ils doivent être récompensés par des distinctions honorables, par la considération publique, par des récompenses méritées. Mais la justice ne peut pas approuver que la noblesse, même lorsqu’elle vit dans l’oisiveté, jouisse de privilèges onéreux pour le reste des citoyens et qu’elle ne supporte point des fardeaux qui sont cruellement rejetés sur la partie la plus pauvre et la plus laborieuse des nations. Le noble qui par état est le défenseur de son pays, le grand qui donne ses conseils aux souverains, le magistrat qui consacre ses veilles au maintien de la justice et du bon ordre, sont des citoyens justement distingués des autres et qui ne doivent être aucunement confondus avec le citoyen obscur qui ne rend pas les mêmes services à la patrie.\par
Que l’on n’écoute donc pas les maximes d’une philosophie mécontente et jalouse qui, sous prétexte de ramener la justice ou le règne d’Astrée sur la terre, voudrait anéantir tous les rangs pour introduire dans les sociétés civilisées une égalité chimérique qui ne subsista pas même dans les hordes les plus sauvages. Dans ces peuplades errantes dont la guerre est la passion habituelle (ainsi qu’elle l’est malheureusement encore dans la plupart des nations policées), les hommes les plus braves ne sont-ils pas les plus distingués et les mieux récompensés ? La raison ne veut donc pas que dans la nécessité cruelle qui met si fréquemment les nations en armes, l’on anéantisse l’esprit militaire et l’on arrache à la valeur la considération qui lui est due. La vraie morale prescrit uniquement aux nobles, aux guerriers, aux grands, aux hommes élevés en dignité, de se distinguer par les vertus et les connaissances qui conviennent à leur état : elle leur défend de se dégrader par une conduite servile ou par des vices capables de les confondre avec des esclaves ou avec la plus vile populace. Le mot {\itshape noblesse} est fait pour annoncer courage, grandeur d’âme, volonté ferme et constante de maintenir les droits de la société. Le rang annonce une supériorité de vertus, de talents, d’expériences, à laquelle le respect et la considération sont dus.\par
Les grandes places annoncent la puissance, la capacité, la volonté de faire du bien, une autorité légitime à laquelle pour leur propre intérêt les hommes sont obligés de se soumettre. La noblesse, le rang et la grandeur sont des mots vides de sens dès qu’ils ne procurent aucun avantage au public ; ils méritent d’être méprisés et détestés quand ils ne font que du mal. Ce serait être injuste que d’exiger pour les dignités, la naissance ou les places, des sentiments qui ne sont dus qu’aux qualités personnelles que ces mots représentent.
\subsection[{Suite du chapitre V. Des Devoirs des Nobles et des Guerriers}]{Suite du chapitre V. Des Devoirs des Nobles et des Guerriers}
\noindent Jusqu’ici nous n’avons parlé que des devoirs des nobles et des gens de guerre relativement à leurs concitoyens et à la patrie où ils sont nés, au bien-être de laquelle tout leur prouve qu’ils sont pour le moins autant intéressés que les autres ordres de l’État. Il nous reste encore à exposer en peu de mots les devoirs qui les lient envers ceux contre qui leur profession les oblige de porter les armes. Ce serait en effet méconnaître les principes les plus évidents de la raison ou de la morale que de croire que l’homme ne dût rien à son ennemi. Ce serait dégrader le guerrier et le supposer une bête féroce que de penser que né dans des nations policées, il pût ignorer les maximes humaines et justes qu’elles ont établies entre elles et qui demeurent en vigueur même au milieu du tumulte des combats. Enfin, ce serait regarder le militaire comme un vil automate, comme un bourreau sans pitié, comme un sauvage furieux, que d’imaginer qu’il pût ne pas savoir jusqu’où son courage doit le pousser contre les ennemis que sa patrie lui désigne.\par
Il n’y a que des sauvages stupides dépourvus de raison, de prévoyance et de vertu, qui se persuadent que tout est permis contre des vaincus et que l’on ne doit mettre aucun terme à sa fureur et à sa vengeance. Les insensés n’ont donc pas vu que {\itshape les armes sont journalières}, que celui qui use cruellement de sa victoire peut bientôt tomber à son tour entre les mains d’un ennemi dont il n’a fait que redoubler la rage ? Les aveugles ne s’aperçoivent pas que leurs guerres continuelles et toujours impitoyables ont presque réduit leurs nations, jadis nombreuses, à de chétives hordes incapables de se défendre contre une poignée d’Européens.\par
Déjà depuis longtemps la voix sainte de l’humanité, la raison, l’intérêt éclairé ont détrompé les nations de nos contrées de leur férocité primitive. Plus les peuples se sont instruits et plus ils ont montré de modération dans la guerre. Si des faits récents fournissent des exemples d’atrocité, ils sont dus à des nations qui n’ont point encore été suffisamment guéries de l’ignorance et de la frénésie de leurs ancêtres sauvages\footnote{Les Croates et les Pandoures, peuples stupides et barbares, ont commis, durant la guerre qui a suivi la mort de l’empereur Charles VI, des cruautés inouïes. Les Kalmouks et les Tartares au service de la Russie ne se sont pas mieux comportés dans la dernière guerre. La destruction du Palatinat, ordonnée dans le siècle passé par Louis XIV, nous prouve que ce prince si vanté par les poètes était un sauvage aussi cruel qu’un Attila. Au reste, cet acte de barbarie n’eut d’autre effet que de le rendre exécrable à toute l’Europe.}.\par
Grâce aux préceptes de la raison qui ont adouci peu à peu les souverains et les guerriers, les hommes ne sont plus si cruellement acharnés à leur destruction réciproque. Le soldat entend le cri de l’humanité au sein même du carnage, au milieu du bruit des armes. Il accorde la vie à l’ennemi désarmé qui la demande : il serait déshonoré s’il frappait son adversaire abattu à ses genoux. Il fait des prisonniers et non pas des esclaves tels que ceux à qui les barbares romains ne laissaient la vie que pour la leur rendre plus insupportable que la mort. Aujourd’hui dans les armées les prisonniers faits à la guerre sont traités avec douceur, garantis de toute insulte et rendus par échange ou par rançon à leur pays. Enfin, les armes mêmes si bruyantes de nos guerriers modernes sont bien moins destructives que celles des anciens.\par
Tels sont les effets que la morale a peu à peu produit sur les cœurs des princes et de leurs soldats. Il faut donc espérer que les maîtres du monde, détrompés de plus en plus de leur ambition meurtrière, s’apercevront du mal que les guerres les plus heureuses font toujours à leurs États. Ramenés à l’humanité, à la justice, à la raison par leur intérêt mieux connu, ils deviendront moins prodigues du sang de leurs sujets, ils ne décideront plus si légèrement la destruction des peuples. Rendus plus pacifiques, ils réduiront ces armées innombrables qui absorbent inutilement tous les revenus de leurs empires, ils s’occuperont de l’administration intérieure, de la législation et des mœurs, ils réuniront d’intérêts les sujets à leurs souverains ; et sous leurs sages lois le guerrier et le noble deviendront des citoyens.\par
Indépendamment des devoirs généraux que le droit des gens, adopté par les nations policées, impose à l’homme de guerre, il en est d’autres que la morale lui prescrit et qu’il ne peut négliger sans crime et sans déshonneur. Si sa patrie lui ordonne de combattre et de détruire ses ennemis qu’il trouve armés, elle ne doit pas lui ordonner d’exercer une vengeance aussi injuste qu’inutile sur le citoyen désarmé, sur le laboureur paisible, sur l’habitant des villes. N’est-ce donc pas assez des ravages, des massacres, des violences de toute espèce que la guerre traîne à sa suite, sans étendre encore ses effets sur des hommes tranquilles dont le malheur est d’être nés dans les États d’un autre maître ?\par
S’il existe donc quelque idée de justice et quelque sentiment de pitié dans les chefs des armées ou dans les officiers soumis à leurs ordres, ils épargneront des citoyens infortunés dont la ruine totale ne peut aucunement contribuer au succès de leurs armes et qui n’ont rien de commun avec les querelles des rois. Ainsi, qu’une discipline sévère mette un frein puissant à la licence, à la cupidité, à la débauche d’une soldatesque toujours ignorante et barbare. Que ces chefs vraiment nobles et désintéressés dont l’honneur doit être le mobile unique, n’aillent pas s’avilir par une avarice sordide. Est-il rien de plus honteux que la conduite abjecte de ces généraux d’armées entre les mains de qui la guerre est un trafic et qui, se rabaissant au métier cruel et bas des traitants et des usuriers, cherchent à exprimer des veines des peuples le peu de sang que la guerre y a laissé ! Tels sont les devoirs que la morale et l’honneur prescrivent aux gens de guerre. Ils furent généreusement observés par les Scipion, les Turenne, les Catinat ; ils le seront par tous ceux qui préféreront une gloire solide à l’amour de l’argent, passion qui décèle communément des âmes lâches et rétrécies. L’avarice est un vice peu fait pour les grands cœurs. La valeur militaire s’anéantit bientôt chez les nations énervées par le luxe où le guerrier souvent préfère sa fortune à sa gloire. Les Romains pauvres et enivrés de l’amour de leur patrie ont subjugué le monde ; enrichis des dépouilles des nations, leur avarice les mit aux prises les uns avec les autres. Amollis par le luxe, ces guerriers si redoutables ne furent qu’un vil troupeau d’esclaves tremblants sous les plus lâches, les plus méprisables des tyrans.\par
Le sentiment de l’honneur doit entièrement disparaître et faire place à l’intérêt le plus sordide dans une nation asservie ; l’honneur n’est point fait pour des esclaves : ils ne peuvent ni s’estimer eux-mêmes, ni prétendre à l’estime de leurs concitoyens. La grandeur d’âme, la fierté noble, le courage seraient des qualités inutiles, déplacées, nuisibles même dans des êtres destinés à ramper. Comment un homme avili par la crainte aurait-il une haute idée de lui-même tandis que tout lui prouve sa dépendance et sa faiblesse ? Un courtisan, dont le rang, la fortune, la liberté, la vie sont à la merci d’un despote méchant ou faible, d’un ministre pervers, d’une maîtresse étourdie, peut-il avoir la force et l’élévation que donne la sécurité ? Quel intérêt cet esclave uniquement occupé du soin de plaire à son maître trouverait-il à mériter l’estime d’un public qui, s’il montrait des vertus, ne lui accorderait qu’une approbation tacite et stérile ou peut être le blâmerait d’avoir eu des qualités peu compatibles avec son état ?\par
Le vrai courage suppose une vigueur, une énergie produite par l’amour de la patrie ; mais où est la patrie dans une contrée que le despotisme a subjuguée ? Le guerrier n’y a d’autre fonction que celle de défendre le geôlier qui la tient en captivité.\par
Il ne peut y avoir ni vraie noblesse ni distinctions réelles, ni rangs ni privilèges durables parmi des hommes également asservis aux caprices d’un maître. Quelques-uns des esclaves que sa faveur inconstante distinguera pour un moment s’enorgueilliront peut être de leur crédit passager et se croiront quelque chose ; mais la moindre réflexion doit bientôt les ramener à l’idée de leur propre néant et leur fera sentir que la main qui les élève et les soutient, peut en se retirant les faire tomber dans la poussière. Une noblesse qui n’est illustrée que par de vains titres, des prérogatives imaginaires, des privilèges injustes, des signes futiles, n’a rien de solide et de réel. La noblesse véritable ne peut se trouver que sous un gouvernement capable d’inspirer des sentiments généreux dans une patrie qui procure la justice, la liberté, la sûreté. Nul citoyen n’est donc plus que le noble intéressé au bien-être de son pays, au maintien des lois qui mettent tous les ordres de l’État à couvert contre les coups de la tyrannie. L’homme véritablement généreux\footnote{Le mot {\itshape généreux} vient du mot latin {\itshape genus} qui signifie {\itshape race illustre}. On a toujours supposé qu’un homme bien né devait avoir des sentiments plus nobles que les autres et se montrer capable de plus grands sacrifices pour la patrie.}, suivant la force du mot, est celui qui a reçu de ses aïeux une âme assez grande, assez noble, assez courageuse pour sacrifier des intérêts puérils et méprisables, des avantages incertains et précaires à des intérêts solides et permanents qui l’attachent à sa patrie, au désir d’être estimé de ses concitoyens, à la gloire, qui n’est jamais que l’estime des honnêtes gens. « C’est par le temple de la vertu, dit Cicéron, que l’on arrive au temple de la gloire. »\par
Quels droits à l’estime publique pourraient donc avoir des nobles et des guerriers totalement dépourvus de grandeur d’âme, de vrai courage, de sentiments généreux ? Une nation peut-elle avoir une considération sincère pour des courtisans occupés à flatter à ses dépens le despote qui la dépouille ou pour des guerriers dont la fonction est de tenir leurs concitoyens sous le joug de l’oppression ? Non ; des hommes de ce caractère ne peuvent aucunement prétendre à l’estime qui constitue le véritable honneur. Ils peuvent bien en imposer par leur faste et leur arrogance, ils peuvent inspirer de la crainte, ils peuvent arracher des signes extérieurs de complaisance et de respect, mais ils n’obtiendront jamais ni des hommages sincères, ni la gloire, qui ne sont dus qu’à la générosité, au patriotisme, à la vertu.\par
Comment le pouvoir de nuire donnerait-il quelques droits à l’estime des hommes ? Ce serait se former des idées bien fausses de l’honneur que de le croire compatible avec le vice, la licence, la perversité. C’est néanmoins dans ces désordres que tant de prétendus nobles et de guerriers ne rougissent pas de le faire consister. On voit souvent les hommes les plus coupables, les plus notés, les plus dignes du mépris des honnêtes gens, s’annoncer comme des {\itshape gens d’honneur}, se présenter impudemment dans toutes les compagnies à l’ombre d’un grand nom ou d’un grade militaire, braver insolemment les regards et recevoir même très souvent un accueil favorable. Les friponneries les plus basses, les dettes les plus frauduleuses ne les font point exclure de la {\itshape bonne compagnie}. Sous des gouvernements injustes ou faibles les grands sont assurés de l’impunité ; les crimes les plus avérés ne les exposent pas à la rigueur des lois : on craindrait que leur châtiment ne déshonorât leurs familles. Comme si les crimes n’étaient point personnels ! Comme si ces crimes ne déshonoraient pas bien plus que l’échafaud\footnote{En 1763, le Lord Ferres, d’une maison alliée à la maison royale, fut pendu publiquement à Londres pour avoir assassiné son domestique. Ce qui n’empêcha pas son frère de prendre séance en sa place dans la Chambre des Pairs d’Angleterre. Dans les autres royaumes de l’Europe, les grands seigneurs ne sont punis exemplairement que pour cause de rébellion contre le souverain ou ses ministres. Les crimes contre la nation sont aisément pardonnés.} ! En un mot, la naissance est un manteau qui couvre toutes les iniquités.\par
En tenant ainsi une balance inégale entre des sujets qui devraient jouir d’un droit égal à la justice, des princes injustes ou faibles ne semblent-ils pas livrer le citoyen obscur à la discrétion des grands ? Voilà comment un mauvais gouvernement peu content d’opprimer les peuples, les abandonne indignement aux outrages et aux attentats d’une foule de tyrans subalternes qui, assurés de n’être point punis, font éprouver leur licence à leurs inférieurs. Ce n’est souvent que par le vice, plus audacieux, que les nobles et les grands se distinguent du vulgaire et s’élèvent au-dessus de leurs concitoyens. Ils les méprisent parce qu’ils sont trop faibles pour pouvoir leur résister.\par
Si des souverains accordent l’impunité à ceux qu’ils daignent favoriser, l’homme de guerre se la procure à lui-même au moyen de son épée, toujours prête à percer quiconque oserait lui témoigner le mépris que ses vices devraient lui attirer\footnote{L’usage de porter l’épée dans les villes en temps de paix, au milieu de ses concitoyens, est un reste de barbarie gothique qui, vu les accidents et les crimes qu’il produit, devrait être aboli dans toute nation policée. Cet usage était inconnu des Grecs et des Romains, qui pourtant par la valeur guerrière ne le cédaient nullement aux descendants des Francs, des Vandales ou des Visigoths. En France, par un abus très dangereux, des valets, des cuisiniers, des artisans portent l’épée et souvent se croient en droit d’insulter des citoyens paisibles qu’ils devraient à tous égards respecter. Le valet d’un grand seigneur a l’impertinence de se croire fort au-dessus d’un bon bourgeois.}. Il résulte un très grand mal dans le commerce du monde d’un préjugé sauvage qui fait passer pour honorable un courage aveugle et forcené, et qui souvent empêche un fripon, un escroc, un homme très méprisable d’être justement réprimandé ou banni de la société. Des personnages de cette trempe peuvent avoir la témérité de se battre : rien de plus ordinaire que de voir l’étourderie et la folie s’unir avec le vice et la perversité. D’un autre côté, l’homme le plus honnête et le plus brave peut succomber sous l’adresse d’un impudent, d’un {\itshape ferrailleur}, d’un spadassin exercé. Pour éviter des querelles et des combats, on est souvent forcé de tolérer dans la bonne compagnie des impertinents, de fort malhonnêtes gens que, parce qu’ils savent se {\itshape battre}, on ne peut en exclure et qui se croient eux-mêmes des gens d’honneur. Ces funestes préjugés rendent la société militaire aussi désagréable que dangereuse.\par
Cependant, les lumières de la raison en se répandant peu à peu ont fait disparaître en partie ces notions si contraires à l’agrément et au repos de la société. Des corps militaires devenus plus sensés savent se débarrasser de ces querelleurs, de ces gladiateurs effrontés qu’on regardait autrefois avec une sorte d’admiration. Un intérêt mieux entendu a fait enfin reconnaître que l’on pouvait montrer du courage contre les ennemis de l’État sans être prêt à tout moment d’insulter, de combattre et d’égorger ses concitoyens. Plus les hommes s’éclaireront et plus leurs mœurs deviendront humaines ou sociales.\par
Il est pourtant des militaires qui semblent regretter encore l’antique barbarie de ces temps où les guerriers s’assassinaient les uns les autres avec la plus grande facilité ; ils prétendent que ces fréquents combats servaient à entretenir l’esprit militaire. Ainsi, ces aveugles spéculateurs s’imaginent qu’un homme de guerre, pour conserver l’esprit de son métier, doit être une bête féroce, un sauvage, un brutal incapable de tout sentiment humain ou raisonnable !\par
En effet, en voyant la conduite insensée du plus grand nombre de ceux qui suivent la profession des armes, l’étourderie et l’incurie qui président à leurs actions, le mépris qu’ils montrent pour les règles de l’équité et pour les bonnes mœurs, on serait tenté de croire que la morale est totalement incompatible avec le métier de la guerre et que le militaire est destiné par son état à ne jamais réfléchir ou faire usage de sa raison. Une politique aussi fausse qu’injuste a trop souvent adopté ces maximes pernicieuses ; croyant mieux s’attacher ses soldats, le despotisme les tint dans l’ignorance et leur permit la rapine, l’injustice et la licence dans les mœurs. Politique bien imprudente que celle qui lâche ainsi la bride à des inconsidérés aveuglément emportés par toutes leurs passions ! Les princes qui suivent de pareilles idées ne voient donc pas que ces satellites à qui l’on permet l’injustice et d’exercer leur férocité contre les citoyens désarmés finissent très souvent par les exercer ensuite contre le souverain lui-même. Comment contenir les fureurs d’une soldatesque abrutie que l’on a pris soin d’entretenir dans le désordre ? Ainsi, sans écouter les maximes d’une politique aveugle et barbare, tout prince raisonnable, pour sa propre sûreté et pour le bien de ses États, doit réprimer la licence du soldat, s’occuper des mœurs de ses chefs, les inviter par des récompenses à s’instruire, en y consacrant une portion du loisir immense et fastidieux que leur laissent en temps de paix leurs fonctions militaires. Par là le souverain se verra servi par des hommes plus habiles, plus expérimentés, moins turbulents, et les nations trouveront dans les nobles et les guerriers des concitoyens plus utiles, plus sociables, plus dignes d’être aimés et considérés.\par
En général, rien ne semble contribuer plus efficacement à la corruption des mœurs d’une nation que le gouvernement militaire : le désordre, la licence, la débauche qui l’accompagnent en tous lieux sont par lui communiqués à toutes les classes de la société et fixent surtout leur domicile dans les endroits où les gens de guerre font leur séjour. C’est là qu’on voit à chaque instant le guerrier travailler à la séduction de l’innocence, attaquer sans relâche la vertu des femmes, se venger de leurs refus par d’affreuses calomnies, en un mot : se jouer insolemment de leur réputation et du repos des familles les plus honnêtes\phantomsection
\label{footnote60}\footnote{Il est un grand nombre de villes de garnison où le militaire est exclu de toutes les maisons honnêtes. Cette exclusion est due à la conduite impertinente de la plupart des officiers surtout avec les femmes, dont, par une vanité bien lâche, ils flétrissent souvent la réputation lors même qu’elles l’ont le moins mérité. Est-il rien de plus bas, de plus indigne d’un homme d’honneur, que ces {\itshape listes} infamantes et souvent calomnieuses par lesquelles des officiers ont l’impudence de déshonorer un sexe que tout honnête homme doit respecter et dont il se ferait même un devoir de cacher les faiblesses ?}.\par
Ajoutez à ces désordres la vanité, la frivolité, l’étourderie, la fatuité, l’arrogance qui font pour ainsi dire le caractère distinctif de la plupart des gens de guerre et qui rendent leur société déplaisante pour les personnes sensées ? Enfin, le militaire, presque toujours désœuvré, rougirait de s’occuper ; il se glorifie de son ineptie et de sa fainéantise, qu’il croit honorables dans son état, il méprise comme des pédants ceux de ses camarades qui cherchent dans l’étude un moyen d’employer leur loisir utilement.\par
On ne peut trop le répéter : l’ignorance et l’oisiveté seront toujours pour les guerriers des sources intarissables de désordres, de malheurs et d’ennuis. Ils ne peuvent s’en garantir qu’en s’ornant plus soigneusement et le cœur et l’esprit. Qu’ils apprennent au moins en quoi consiste cet honneur dont ils se piquent, tandis qu’ils n’en ont pas souvent la plus légère idée, qu’ils ne le confondent plus avec la vanité, l’arrogance, l’impudence ou le vice effronté, qui ne peuvent que les rendre odieux et méprisables, qu’ils sachent que l’instruction et les mœurs ne leur sont pas moins utiles qu’au reste des citoyens.\par
Par une sotte vanité que trop souvent l’on substitue à la grandeur d’âme, à la noble fierté, à l’honneur véritable, un luxe ruineux fait des ravages affreux dans les armées et dérange la fortune de ceux qui se consacrent à la défense de l’État. C’est à ce luxe destructeur que des familles nobles sont redevables de l’indigence et de l’obscurité dans lesquelles on les voit souvent croupir. C’est à cette misère que l’on doit attribuer la dépendance servile dans laquelle le despotisme tient continuellement une noblesse que ses folles dépenses ont ruinée. En un mot, le luxe et la vanité des nobles et des guerriers servent à consolider les chaînes qui les retiennent eux-mêmes sous le pouvoir des tyrans.\par
C’est, pour tout homme qui pense, un spectacle étrange et digne de pitié que de voir à quel point l’opinion est parvenue à fasciner la noblesse et à la tromper sur ses intérêts les plus réels. Pour briller à la guerre par une dépense qui surpasse ses forces, un noble, un riche propriétaire, s’endette, engage ses terres, se dépouille de la fortune qu’il possède et dont il peut jouir, le tout dans la vue de plaire à une cour ingrate des caprices de laquelle il sera forcé de dépendre le reste de sa vie ! Pour remplacer les biens solides dont sa vanité l’a privé, il obtiendra quelquefois un grade, une pension précaire, quelque distinction puérile, s’il est favorisé ; mais s’il n’a point la faveur, il sera négligé et méprisé de ceux mêmes pour qui il a eu la simplicité de se ruiner. En un mot, c’est à des espérances chimériques, à des préjugés trompeurs, au hasard, que tant de guerriers et de nobles ont la folie de sacrifier leur fortune, leur repos, leur honneur, leur vie et très souvent la patrie, dont ils se disent les défenseurs.\par
Une politique moins perfide et plus éclairée devrait réprimer un luxe et une mollesse incompatibles avec le métier de la guerre. Comment des hommes vraiment pleins de courage n’ont-ils pas la force de les mépriser ? Des princes plus justes et plus sages banniront ces fléaux des armées pour introduire en leur place la simplicité, la tempérance, la frugalité, la discipline, plus propres à fortifier les corps et à soutenir le courage. Quels spectacles révoltants pour des malheureux que les repas somptueux des chefs qui, par leur luxe et leurs profusions, affament le camp, font nager dans l’abondance une foule de valets fainéants tandis que le soldat exténué de fatigues manque souvent du nécessaire ?\par
Que dirons-nous de ces plaisirs amenés à grands frais, de ces théâtres, des amusements frivoles, des jeux ruineux, d’une foule de prostituées, des débauches continuelles que le luxe et l’habitude du vice rendent nécessaires à des guerriers corrompus et totalement efféminés ? Il semblerait qu’une politique affreuse se fait un principe d’affaiblir, de détruire les corps, la fortune et les mœurs de ceux qu’elle destine à la défense de l’État. Telle est la récompense que le despotisme réserve communément aux insensés qui ont eu l’imprudence de soutenir son injuste pouvoir ! Il les corrompt, il les ruine et les abandonne ensuite au repentir, à la misère, aux infirmités, au mépris. Par une loi constante de la Nature dont le noble et le guerrier ne sont point exceptés, il n’est point de désordre qui ne trouve tôt ou tard son châtiment sur la terre. Les gens de guerre font souvent le malheur des nations sans se rendre eux-mêmes plus fortunés.\par
Rentrez donc enfin en vous-mêmes, grands, nobles et guerriers ! Ouvrez les yeux sur de vains préjugés qui depuis trop longtemps vous aveuglent. Apprenez à mieux connaître l’honneur auquel votre rang et votre profession semble devoir vous attacher plus particulièrement. Faites-le consister dans le droit incontestable à l’estime de vos concitoyens et non dans une naissance qui n’est due qu’au hasard, dans des prérogatives et des privilèges contraires à l’équité, dans un crédit et des faveurs qu’un moment peut enlever, dans une vanité fastueuse qui vous ruine, dans une ignorance qui vous dégrade, dans une licence qui vous déshonore. Devenez citoyens dans des nations que vos ancêtres ont trop souvent asservies et ravagées. Ne soyez plus les fauteurs du despotisme, les contempteurs des lois, les ennemis orgueilleux des magistrats qui les soutiennent. De concert avec eux, soyez les défenseurs de la patrie, qui ne peut exister sans justice, sans liberté, sans règles permanentes. Montrez-vous les vrais soutiens du trône en l’établissant sur la félicité publique, à laquelle tout vous prouve que vous êtes intéressés et que le souverain lui-même doit sa sûreté. Voilà la route qui conduit à l’honneur. C’est ainsi que vous serez véritablement estimés et distingués et que vous transmettrez à la postérité des noms chéris et respectables.
\subsection[{Chapitre VI. Devoirs des Magistrats et Gens de loi}]{Chapitre VI. Devoirs des Magistrats et Gens de loi}
\noindent Ce qui vient d’être dit des grands et des nobles peut donc encore s’appliquer aux magistrats, aux juges, aux organes des lois à qui les nations ont assigné de tout temps un rang honorable parmi les citoyens. Des hommes destinés à rendre justice aux autres, à leur faire observer les conventions sociales, à réprimer leurs passions, à punir les crimes au nom de la société doivent se montrer dignes des respects du public par une équité inébranlable, par une probité à toute épreuve, par une intégrité parfaite, par une connaissance profonde des lois si compliquées et si multipliées qui composent la jurisprudence de tant de nations. Destinée à censurer et contenir les vices, à punir les dérèglements des autres, la magistrature impose à ses membres une décence, une gravité particulière dans les mœurs, une conduite intacte et pure totalement exempte des excès qu’ils doivent corriger.\par
Un magistrat inique vendu à la faveur qui se laisse séduire par la sollicitation, par le crédit, la richesse, l’autorité, est un monstre dans l’ordre social : c’est un bourreau. Le juge sans étude et sans lumières est capable par son ignorance de renverser les fortunes des familles et de punir l’innocence à tout moment. « Il n’y a point, dit un magistrat célèbre, de différence entre un juge méchant et un juge ignorant\footnote{M. le Chancelier D’Aguesseau. — Un autre magistrat se plaint du peu de lumières des sénateurs de son temps : « Plérumque tamen, dit Cicéron, ad honores adipiscendos et ad Rempublicam gerendam nudi veniunt et inermes, nullâ cognitione rerum, nullâ scientiâ ornati. » Voyez Cicéron, {\itshape De Legibus}. Le même orateur dit ailleurs : « Senatorius ordo vitio careat ; ceteris specimen sit : nec veniat quidem in eum oraïnem quisquam vitii particeps. » Cicéron, {\itshape De Legibus}, Livre III, chap. 12 et 13.}. » Le magistrat livré à la débauche, à la dissipation, à la galanterie, aux plaisirs, est indigne de sa place ; il ne mérite que le mépris de ses concitoyens et devrait être honteusement chassé du rang que ses mœurs déshonorent. Une censure très sévère devrait, comme chez les Romains, veiller sur les magistrats, purger les tribunaux des membres qui les dégradent. La magistrature est un état qui doit se distinguer par sa décence, par l’innocence de sa conduite, par la sagesse de ses jugements, par sa pénétration et l’étendue de ses lumières : un magistrat frivole, dissipé, sans étude, est une contradiction à laquelle la dépravation générale peut seule accoutumer les yeux. Le ministre des lois est fait pour les connaître ; le protecteur des mœurs doit avoir lui-même des mœurs. Celui qui juge les autres doit craindre à son tour les jugements du public, qui n’accorde son estime qu’au mérite personnel.\par
Comment estimer un magistrat lorsqu’il ne regarde sa place que comme un vain titre qui ne l’oblige à rien ? Comment respecter un juge ignorant, inappliqué, esclave de ses plaisirs, qui s’avilit par ses vices et se méprise lui-même ? Comment considérer un juge dont les arrêts sont souvent dictés par la corruption et la débauche ? Quelle idée se former d’un sénateur assez petit pour imiter la vanité, le faste, les hauteurs, les désordres même que l’on ne trouve qu’avec indignation dans un militaire étourdi ?\par
Plusieurs causes semblent avoir concouru à l’avilissement de la magistrature : la multiplicité des lois, leurs contradictions continuelles, leur obscurité ont rendu l’étude de la jurisprudence fastidieuse, impossible même, au plus grand nombre de ceux qui devraient s’y livrer. Combien de travaux, de pénétration et d’assiduité ne faut-il pas pour parcourir le labyrinthe que les lois accumulées présentent à ceux qui voudraient s’en instruire ? Aussi, rien de plus rare qu’un juge qui sache ou qui puisse savoir son métier. La tourbe des magistrats est guidée par la forme, par la routine aveugle, depuis longtemps en possession de décider du sort des hommes. De l’obscurité des lois et de leur multiplicité résultent non seulement l’ignorance des juges mais encore l’imposture et la mauvaise foi d’une foule de praticiens qui enlacent adroitement les citoyens dans leurs filets pour dévorer leur substance et qui, surprenant habilement la religion du magistrat, font souvent triompher l’injustice et la fraude. Une jurisprudence ténébreuse et compliquée est une source de crimes et de maux dans les nations opulentes et policées, plus malheureuses à cet égard que les nations les plus pauvres et les plus barbares.\par
La vénalité des offices de la magistrature, introduite par l’avidité ou les prétendus besoins de quelques gouvernements, a rempli les tribunaux de sujets à qui l’opulence tenait lieu de science, de mérite et de vertu. Le droit de juger les peuples fut vendu à une foule d’hommes dépourvus des connaissances et des qualités nécessaires pour s’acquitter dignement d’une fonction si noble. Ceux-ci transmirent ce droit éminent à une postérité qui, sûre d’hériter des places de ses pères, se crut dès lors dispensée de la peine de les mériter.\par
Lorsque le choix des ministres de la justice dépendit d’une cour communément corrompue, les peuples n’eurent pas lieu de s’applaudir des magistrats qui leur furent donnés. L’étude et le concours devraient seuls faire adjuger les offices de la magistrature.\par
Des magistrats fiers de leur pouvoir en abusèrent souvent et firent sentir d’une façon incommode le poids de leur autorité au reste des citoyens ; ceux-ci n’eurent que de faibles ressources contre les injustices ou les violences de ceux qui étaient destinés à les protéger. Ainsi, la magistrature forma dans quelques États une classe à part qui, profitant du droit de juger, s’arrogea bientôt celui de dominer et d’opprimer : au lieu de faire aimer son pouvoir par son affabilité, sa modération, sa justice, au lieu de s’attacher les différents ordres de l’État par un zèle sincère pour le bien général, au lieu de se faire considérer par son mérite et ses lumières, le magistrat, enivré de sa puissance précaire, ne voulut que se rendre redoutable à ses concitoyens.\par
Gonflée de ses prérogatives, que la magistrature voulut toujours étendre, on la vit quelquefois s’efforcer de former sans l’aveu des nations une sorte d’aristocratie qui fit ombrage aux souverains ; sous prétexte de défendre les lois et les droits des peuples, les magistrats prétendirent représenter les nations. Mais ces prétentions qu’une conduite équitable, intègre et mesurée, aurait peut-être fait adopter, déplut à la noblesse jalouse qui, comme on a vu, regrette toujours pour elle-même un droit dont son imprudence l’a fait déchoir ; d’ailleurs, les vues ambitieuses des magistrats ne furent point appuyées par les différentes classes, perpétuellement divisées. Le despotisme combattit donc et subjugua sans peine un corps sans force réelle et qui par son arrogance, son peu de lumières, son indifférence pour le bien de l’État, avait anéanti l’attachement et la considération du public sans lesquels aucun corps ne peut longtemps se soutenir.\par
Pour acquérir de la consistance, qui n’est l’effet que de la considération publique, l’équité, les lumières, le mérite et la vertu sont nécessaires aux corps comme aux individus. Un corps dont les membres sont corrompus et divisés ne peut jouir que d’une puissance précaire. Tout corps qui se fait des intérêts séparés de ceux de sa nation ou des autres corps de l’État ne peut longtemps résister à la force, aux artifices, aux pièges du despotisme, qui cherche sans relâche à diviser et démolir tout ce qui peut mettre obstacle à ses fantaisies.\par
Le despotisme fut et sera toujours l’ennemi des formes et des lois, qui souvent le gênent ou le retardent dans sa marche insensée. Le despote hait et méprise le magistrat qui, défenseur des lois de son pays, lui rappelle toujours l’importune idée de l’équité. Ne soyons donc pas étonnés en voyant que l’étiquette des cours monarchiques et despotiques a mis une très grande différence entre la noblesse militaire et la magistrature même la plus élevée : l’homme de guerre présente au chef de la société un esclave par état dévoué à toutes ses volontés, tandis que l’homme de lois lui présente un défenseur des droits du peuple, un ministre de l’équité, avec lesquels un mauvais gouvernement est continuellement en guerre.\par
Les despotes, affamés d’une autorité sans bornes, éprouvent une antipathie naturelle pour la vérité, pour les formes, les règles, les lois et leurs interprètes ; l’intégrité des magistrats déplaît à des cours injustes, leur résistance la plus noble est une révolte aux yeux d’un prince entouré de courtisans toujours vils et soumis. Les remontrances les plus humbles fatiguent des souverains que la vérité ne peut qu’effaroucher ; les plaintes les plus légitimes alarment des ministres et des favoris communément les vrais auteurs des calamités nationales et qui ont le plus grand intérêt qu’aucun cri ne réveille le monarque endormi par leurs soins. En un mot, le prince et sa cour ne voient dans des magistrats fidèles à leurs devoirs que des censeurs incommodes qu’il faut réduire au silence ou rendre complices des désordres qu’ils voudraient arrêter.\par
Les lois sont inutiles quand il existe dans l’État une autorité plus forte que la leur. Sous un gouvernement injuste, la justice n’est qu’un fantôme fait pour effrayer les faibles et qui n’en impose aucunement aux puissants. La magistrature est un vain titre qui ne donne ni fixité, ni pouvoir, ni considération réelle. Les tribunaux destinés à se prêter aux volontés momentanées du prince ou de ses favoris, ne peuvent suivre aucuns principes constants et doivent faire plier les lois sous les caprices des grands. Le magistrat n’est plus alors qu’un vil esclave à tout moment forcé de renoncer à la fortune ou de perdre sa liberté, sa vie même, s’il refuse de sacrifier son honneur et sa conscience aux fantaisies variables du maître ou de ses agents.\par
Sous de tels chefs le juge doit s’armer d’un cœur d’airain, il doit trouver coupables et détruire les victimes les plus innocentes dès que le despotisme lui ordonne de frapper. Le despotisme n’a jamais tort : il s’arroge le pouvoir de créer le juste et l’injuste. Lui déplaire est un crime, lui obéir est l’unique devoir et l’unique vertu.\par
En un mot, le magistrat dégradé par la servitude ne devient qu’un automate qui reçoit les impulsions que le crédit, la sollicitation, la puissance lui donnent. Il se méprise lui-même et ne s’attire que la haine et le mépris des autres, et cherche en vain dans le faste, l’opulence, la dissipation, à s’étourdir sur les remords qui se renouvellent en lui. Les ministres de la justice deviennent les plus injustes, les plus cruels, les plus méprisables des hommes sous la tyrannie dont l’injustice est la base et la cruauté le soutien.\par
Pour un homme de cœur, est-il une position plus affreuse que celle d’un magistrat honnête qui, forcé de prêter ses secours à la tyrannie et à ses agents, se trouve continuellement obligé d’inquiéter les familles et de vivre dans un commerce perpétuel avec des espions, des sycophantes, des délateurs, en un mot avec des hommes infâmes, les seuls qui soient disposés à se prêter aux vues d’une administration violente et soupçonneuse ? Un gouvernement est bien lâche et bien petit quand il se sert de pareils instruments ! Un magistrat est bien grand lorsque sous le despotisme il conserve son intégrité et l’amour des citoyens !\par
La magistrature ne peut être honorable et considérée que lorsque, fidèle à ses devoirs, elle remplit noblement ses augustes fonctions ; elle ne peut être justement respectée et chérie que sous un gouvernement équitable qui lui laisse la liberté de se conformer à la raison, aux lois, à sa conscience, à son honneur.\par
En simplifiant la jurisprudence, en la rendant plus claire, en élaguant prudemment cette multitude de lois et de coutumes obscures, injustes, contradictoires, sous lesquelles tant de peuples sont accablés, les magistrats n’auront plus tant de peine à se procurer les lumières nécessaires à leur état. Des lois plus précises et plus claires n’auraient pas besoin d’être sans cesse commentées, expliquées, interprétées, les décisions des juges seraient plus stables et moins arbitraires, la raison et l’équité naturelle anéantiraient l’hydre de la chicane qui dévore les nations, qui ruine les familles, qui si fréquemment fait succomber le bon droit.\par
Enfin, une réforme sage soulagerait les peuples du fardeau insupportable de tant de juges, de tribunaux, de suppôts de la justice, dont ils sont écrasés. Un bon gouvernement ne devrait-il pas préférer le bonheur de commander à des sujets paisibles, honnêtes et justes, au méprisable avantage de profiter de leurs procès et de leurs querelles ? Un gouvernement équitable devrait-il tolérer des nuées de sauterelles affamées qui dévorent impunément les moissons du citoyen ? La cruelle administration de la justice et les iniquités sans nombre auxquels on est exposé dès qu’on poursuit ses droits, sont un des plus grands fléaux dont les nations soient partout accablées.\par
En attendant une réforme salutaire qui, comme on a fait voir, ne peut être opérée que par un gouvernement instruit de ses vrais intérêts, tout magistrat qui voudra mériter sa propre estime et les respects du public s’attachera fortement à la justice, défendra courageusement ses droits, sacrifiera généreusement sa fortune, son crédit, une faveur incertaine, à la satisfaction permanente qui suit toujours une conduite irréprochable. Il quittera son état lorsqu’il n’y trouvera plus la possibilité d’être juste, il portera dans la retraite un contentement intérieur que l’homme honnête doit préférer à tout. Il n’y sera même privé ni des applaudissements ni de la gloire qui, même au milieu de la plus grande corruption des mœurs, sous les gouvernements les plus pervers, dans les nations les plus frivoles, accompagnent la vertu.\par
C’est dans l’estime de ses concitoyens et non dans la faveur d’une cour souvent injuste et tyrannique que le magistrat doit faire consister sa gloire. La persécution rendit toujours le grand homme plus intéressant et plus cher aux honnêtes gens. À l’admiration que le courage est fait pour exciter se joint alors l’attendrissement de la compassion.\par
Tels sont les sentiments que tu fis naître dans tous les cœurs honnêtes et sensibles, illustre Malesherbes\footnote{Premier président de la cour des Aydes de Paris, qui fut dépouillé de sa charge et exilé par le chancelier de Maupéou en 1771. Ce grand magistrat fut surnommé {\itshape le dernier des Français}.}, lorsque le pouvoir odieux d’un ministre cruel te priva de ta dignité, de ta fortune, de ton état, et te força d’enfouir dans la solitude tes sublimes talents dont tu t’étais si noblement servi pour faire entendre jusqu’au trône les cris de la liberté expirante de ta patrie !\par
L’Europe entière n’a-t-elle pas pris part à tes peines, généreux La Chalotais, lorsque, sans respect pour ton âge, tes barbares ennemis machinaient ta ruine et déjà te préparaient des échafauds\phantomsection
\label{footnote61}\footnote{M. {\itshape Caradeuc de La Chalotais}, Procureur général du Parlement de Bretagne.} !\par
La tendresse publique n’a-t-elle pas accompagné ta prison et tes disgrâces, jeune Du Paty ! Toi qui fis voir la fermeté d’un sénateur consommé dans l’âge même des plaisirs et de la frivolité\phantomsection
\label{footnote62}\footnote{M. {\itshape Mercier du Paty}, Avocat général du Parlement de Bordeaux, qui à l’âge de 25 ans, quoique attaqué d’une maladie dangereuse, fut emprisonné cruellement par le chancelier de Maupéou en 1771 et ensuite envoyé en exil.} !\par
Il est donc des consolations, des récompenses, des honneurs et même des applaudissements publics pour les magistrats généreux ; ils sont chéris et vénérés au sein même des nations flétries par le despotisme. Les esclaves les plus lâches ou les plus frivoles ne peuvent s’empêcher d’admirer leurs défenseurs et de donner au moins quelques larmes passagères aux malheurs qu’ils s’attirent en prenant en main la cause de la patrie. Non, toutes les violences de la tyrannie ne pourront jamais ravir à la grandeur d’âme les hommages des cœurs sensibles et vertueux. Tous ceux qui auront le courage d’être utiles aux hommes en seront de leur vivant même fidèlement récompensés.\par
Des magistrats vraiment nobles et grands, des magistrats sincèrement échauffés de l’amour du bien public et détachés des petitesses de l’amour-propre, de l’intérêt particulier, de l’esprit de corps, de leurs vains privilèges, s’attireraient l’affection de tous leurs concitoyens réunis d’intérêts avec les défenseurs de leurs lois. Une magistrature animée de cet esprit patriotique, secondée par les vœux de tous les bons citoyens, deviendrait une barrière puissante contre le despotisme et la tyrannie. La justice et la vertu sont aussi nécessaires aux différents corps d’un État qu’à chacun des individus. Le vice, l’arrogance, l’orgueil, l’imprudence mettent la division entre les classes diverses de la société, détruisent l’harmonie sociale et rendent chaque ordre trop faible pour résister à l’oppression. Une sotte vanité, un attachement puéril à de vaines prérogatives, des prétentions souvent déraisonnables, des chimères suffisent pour mettre la division entre des citoyens qui devraient se soutenir mutuellement. Il en résulte que tous tombent successivement dans les pièges du despotisme, qui finit par être lui-même la victime de sa propre vanité.\par
Depuis le monarque jusqu’au dernier des citoyens, il n’est personne qui n’ait le plus grand intérêt au maintien de l’équité : chacun doit être juste et faire le bien dans sa sphère, chacun doit être chéri, considéré, quand il remplit exactement les devoirs de son état. Par le sien, le magistrat est le ministre de l’équité, l’organe de la loi et non son interprète, le défenseur du faible, le refuge du pauvre, le consolateur de la veuve et de l’orphelin, le protecteur de l’innocent, la terreur du coupable, quelque grand, quelque opulent qu’il puisse être. Tous les citoyens ont besoin de la justice, sans doute, tous ont droit d’y prétendre ; mais la loi doit surtout sa force au malheureux, à l’indigent, au citoyen dénué de secours. Le cœur du magistrat doit toujours par préférence s’ouvrir à l’infortuné : c’est lui qui a le plus grand besoin de justice, et pourtant c’est à lui qu’elle est pour l’ordinaire impitoyablement refusée ! Enfin, des magistrats attentifs, que leurs fonctions mettent tous les jours à portée de reconnaître les inconvénients des lois souvent injustes et des usages nuisibles introduits par la barbarie ou par la tyrannie, devraient en représenter les mauvais effets au législateur. Ces juges animés par l’humanité devraient surtout faire abroger ces tortures vraiment sauvages par lesquelles on multiplie, sans avantage pour la société, les tourments des malheureuses victimes de la justice. Ils devraient encore faire mitiger des lois de sang qui rendent la peine de mort trop fréquente en la décernant contre des délits qui ne méritent nullement un châtiment si terrible, par lequel les nations sont privées de beaucoup d’hommes dont elles pourraient éprouver les services. En un mot, le magistrat, même en punissant le crime, ne doit pas montrer de colère ni se dépouiller des sentiments d’humanité. Au milieu de l’obscurité, de la déraison, des contradictions perpétuelles et même de la perversité que l’on voit régner dans la jurisprudence qui sert de règle à bien des nations, il est très difficile que la saine morale, toujours conforme à la Nature, trouve des préceptes qu’elle puisse donner avec succès à la plupart de ces hommes dont la profession est de guider, de défendre, d’éclairer les citoyens dans leurs démêlés juridiques et de les conduire dans l’affreux dédale des formes qui trop souvent servent à rendre l’accès du temple de Thémis inaccessible aux citoyens. Cette morale parlerait en vain à des mercenaires toujours prêts à prendre en vain la cause du riche injuste, de l’oppresseur puissant, du plaideur de mauvaise foi contre le pauvre, l’innocent et le faible. Quelle conscience ou quel front doivent avoir ces guides trompeurs, ces appuis de l’injustice qui, par d’affreuses connivences avec leurs perfides confrères, par des menées criminelles, des trahisons, des détours, des chicanes et des formes insidieuses, se glorifient quelquefois des victoires infâmes qu’ils ont remportées sur le bon droit ? Est-il un attentat plus détestable et plus digne d’être châtié que celui de ces impudents qui font métier de tromper sciemment les juges et de leur faire dicter des arrêts favorables à l’iniquité ? Au défaut des lois, l’opprobre ne devrait-il pas s’imprimer sur le front de ces voleurs autorisés qui par mille moyens ingénieux trouvent le secret de ruiner en procédures les familles les plus opulentes et d’absorber en frais les prétentions des créanciers ? Est-il un citoyen sûr de sa propriété dès qu’il tombe entre les mains de ces vautours rongeurs dont rien ne peut assouvir la rapacité ? Enfin, quelle protection l’homme honnête peut-il attendre des lois, qui ne sont trop communément que des pièges tendus à l’innocence, à la simplicité, à la bonne foi ?\par
Dans bien des nations, se défendre dans la cause la plus juste, c’est s’exposer à la ruine. Les formes en tout pays semblent donner des avantages inestimables aux plaideurs de mauvaise foi\phantomsection
\label{footnote63}\footnote{Un avocat célèbre disait que « lorsqu’une cause évidemment juste, le plus sage est de s’accommoder ; mais lorsqu’elle est douteuse, il faut plaider ». On remarque en général que les habiles gens de loi sont ceux qui plaident le moins.}. La multiplicité des lois, souvent contradictoires, rend la jurisprudence incertaine, impénétrable, arbitraire pour ceux mêmes qui s’en occupent uniquement ; elle fait que les juges les plus intègres sont surpris à tout moment par des praticiens rusés qui se font une gloire de triompher dans les causes les plus désespérées. En général, les gens de loi sont, chez presque tous les peuples, l’un des plus grands fléaux dont ils soient tourmentés. Les ministres de la justice sont très souvent ceux qui lui montrent le mépris le plus outrageant.\par
Ce serait cependant être injuste que d’envelopper dans la même condamnation tous ceux qui professent la jurisprudence. Il se trouve dans leur nombre des hommes honnêtes, nobles, vertueux, qui gémissent hautement de l’iniquité des lois, de l’absurdité des formes, du brigandage de leurs indignes confrères. L’innocence délaissée rencontre souvent en eux des champions généreux qui osent la défendre contre la puissance altière. L’indigent opprimé fut souvent garanti des entreprises de la force par des protecteurs courageux et désintéressés. Des plaideurs acharnés ont plus d’une fois calmé leurs animosités par les conseils pacifiques de jurisconsultes bienfaisants qui les ont préservés de la ruine. En un mot, si parmi les suppôts de la justice on trouve communément des êtres méprisables par le trafic honteux qu’ils font de leurs talents, d’autres nous montrent des exemples éclatants de vertu, de justice et de générosité.\par
Bien plus, un ordre d’hommes que la grandeur orgueilleuse se croit en droit de mépriser a donné dans les plus grands dangers des marques d’un patriotisme, d’une noblesse, d’un courage, d’un véritable honneur inconnus aux fiers esclaves dont les cours sont remplies et que leurs lâches cœurs seraient incapables d’imiter\footnote{Les annales de la France conserveront à la postérité les noms illustres des {\itshape La Chalotais}, des {\itshape Lamoignon de Malesherbes}, magistrats autant distingués par des talents sublimes que par leur fermeté dans l’infortune et par le courage qu’ils ont opposé aux fureurs du despotisme. Ces même annales n’oublieront pas de transmettre aux races futures le nom respectable du généreux {\itshape Target} (avocat au Parlement de Paris) dont la grande âme a résisté constamment aux séductions et aux menaces de la tyrannie.}.\phantomsection
\label{footnote64} Ces lions indomptés à la guerre deviennent très souvent des moutons à la cour.\par
Gardons-nous donc de confondre des citoyens respectables tels que ceux dont on vient de parler, avec la troupe méprisable de ceux pour qui l’étude des lois n’est qu’un moyen d’exercer impunément le brigandage le plus affreux. Au milieu même des périls où des lois confuses et très souvent injustes mettent les nations, il est utile que des citoyens honnêtes en démêlent le chaos et nous avertissent des écueils contre lesquels nous pouvons à tout moment échouer. Quoi de plus estimable que des hommes modérés dont le sang-froid puisse apaiser les passions et l’humeur querelleuse d’une foule d’insensés toujours prêts à s’attaquer ! Est-il une fonction plus noble et plus honorable que celle d’un avocat à qui ses lumières et sa probité attirent la confiance du public, dont le cabinet devient un sanctuaire respecté qui se rend le conseil, l’arbitre, le juge de ses concitoyens ? Par des voies licites et très honnêtes un jurisconsulte estimé n’acquiert-il pas facilement et sans remords une fortune dont il n’a point à rougir ?\par
Telle est en général la conduite que la morale semble indiquer à ceux qui se destinent à l’étude des lois, que tant de causes concourent à rendre si pénible. C’est à des gouvernements plus sages, plus justes, plus vertueux, qu’il appartient de former une jurisprudence plus claire, plus conforme à la Nature et aux besoins des nations. Voilà le seul moyen de faire disparaître une engeance affamée qui dévore impunément la substance des citoyens et qui détruit souvent dans les esprits les idées les plus naturelles du juste et de l’injuste. Tacite regarde avec raison la multiplicité des lois comme le signe indubitable d’un mauvais gouvernement et d’un peuple corrompu\footnote{« In pessimâ autem republicâ plurimæ Leges. »}.
\subsection[{Chapitre VII. Devoirs des Ministres de la Religion}]{Chapitre VII. Devoirs des Ministres de la Religion}
\noindent Il n’entre pas dans le plan de cet ouvrage, uniquement destiné à développer les principes de la morale naturelle, d’examiner les fondements des religions variées que nous voyons établies dans les diverses contrées du monde. Quelles que soient les idées que les différents peuples se forment de la divinité ou du moteur invisible de la Nature, ce fut toujours à la bonté de cet être que les hommes rendirent leurs hommages. Ils ont dû supposer qu’il leur voulait du bien, qu’il écoutait leurs prières, qu’il avait la puissance et la volonté de les rendre heureux, d’où ils ont dû conclure que l’homme devait faire du bien à ses semblables pour se conformer aux vues de cet être bienfaisant. Envisagée sous cette face, la religion ne peut être que la morale naturelle, ou les devoirs de l’homme confirmés par l’autorité connue, ou présumée, du maître de la Nature et des hommes, qui ne peut contrarier les lois auxquelles leur conservation et leur bien-être sont visiblement attachés.\par
Suivant les principes de toutes les religions, les qualités morales et les volontés divines doivent servir de modèles et de règles aux hommes : tous les cultes qui supposent la divinité méchante, cruelle, injuste, vindicative, ennemie des hommes, en un mot immorale, ne peuvent être regardés que comme des superstitions et des mensonges inventés par des imposteurs intéressés à troubler le repos du genre humain. Toute morale serait inconciliable avec un système religieux qui supposerait un dieu despote ou sans règle, aux yeux duquel les malheurs des nations et les pleurs des mortels seraient un spectacle amusant. « Jupiter lui-même, dit Plutarque, n’a pas le droit d’être injuste. » « Un dieu, dit Cicéron, cesserait d’être dieu, s’il déplaisait à l’homme. » Ailleurs, cet orateur philosophe représente Dieu comme {\itshape le protecteur et l’ami de la vie sociale} ; il est parfaitement d’accord avec la sagesse éternelle qui déclare que {\itshape la société des enfants des hommes fait ses délices les plus chères}\footnote{Voyez Cicéron, {\itshape De Legibus}, III. Voyez {\itshape Proverbes}, chap. 8, vers. 31.}.\par
Cela posé, toute opinion, toute doctrine, tout culte qui contrarient la nature de l’homme raisonnable et vivant en société, doivent être rejetés comme contraires aux intentions de l’auteur de la nature humaine. Tout système religieux qui porterait à violer la justice, la bienfaisance, l’humanité, ou à fouler aux pieds les vertus sociales, doit être détesté comme un blasphème contre la divinité : enfin, toute hypothèse qui produirait en son nom des dissensions, des haines, des persécutions et des guerres, doit être regardée comme un mensonge abominable. Nous avons donc des moyens de juger si une religion est bonne ou mauvaise, c’est-à-dire conforme ou contraire aux idées que l’on se fait de la divinité. D’après ces principes qui paraissent incontestables, la religion la plus convenable à la morale, à la nature de l’être sociable, à la conservation, à l’harmonie, à la paix des nations, doit être préférée à des opinions opposées, qui devraient être proscrites avec indignation. Ce n’est que la conformité avec les préceptes de la morale naturelle qui peut constituer l’excellence d’une religion et fixer sa prééminence sur tant de superstitions dont les hommes sont infectés.\par
La morale est donc, relativement au monde où nous vivons, la pierre de touche de la religion et l’objet qui intéresse le plus la société politique. Si la théologie règle les pensées des hommes sur des objets célestes et surnaturels, la morale se contente de régler leurs actions et de les diriger vers leur plus grand bien sur la terre. Si la religion promet des récompenses ineffables à la vertu et menace le crime de châtiments rigoureux dans une autre vie, la morale promet dans la vie présente des récompenses sensibles à tout homme vertueux, elle menace le pervers de châtiments très marqués et ses arrêts confirmés par la société sont souvent fortifiés par l’autorité des lois. La société ne peut ni ne doit s’occuper des pensées secrètes de ses membres, sur lesquelles elle n’a point de prise ; elle ne peut les juger que sur leurs actions, dont elle éprouve l’influence. Pourvu que le citoyen soit juste, paisible, vertueux et remplisse fidèlement ses devoirs dans sa sphère, ni la société ni le gouvernement ne peuvent sans folie fouiller dans sa pensée ou s’arroger le droit de régler ses opinions vraies ou fausses relativement à des choses qui ne sont aucunement du ressort de l’expérience ou de la raison. Il doit être permis à l’homme d’errer à ses propres risques sur des matières inaccessibles aux sens, mais la société ou la loi peuvent justement l’empêcher d’errer dans sa conduite et le punir lorsque ses actions nuisent à ses concitoyens. En un mot, c’est une tyrannie aussi cruelle qu’insensée, de punir un homme pour n’avoir pu voir des objets invisibles avec les mêmes yeux que les tyrans qui le tourmentent pour sa façon particulière de penser. D’un autre côté, un dieu très juste, très puissant et très bon, qui permet que les mortels s’égarent dans leurs pensées, ne peut pas approuver qu’on les tourmente pour leurs pensées diverses, qui ne dépendent point de leurs volontés. D’où il suit que la religion, d’accord avec la morale et la raison, défend de maltraiter les hommes pour leurs opinions religieuses.\par
Cependant, rien n’a coûté plus de sang et de larmes aux nations, que l’imposture qui persuade que la société est fortement intéressée à régler les opinions particulières des citoyens sur des dogmes abstraits de la religion.\par
Cette idée, qui ne peut venir d’une divinité bienfaisante, a produit des persécutions, des supplices multipliés, des révoltes sans nombre, des massacres affreux, des régicides, en un mot : les crimes les plus destructeurs. Des prêtres ambitieux ont voulu régner sur l’univers, subjuguer les souverains, établir leur empire sur les pensées mêmes des hommes. Ils furent secondés par des fanatiques zélés et par des imposteurs qui osèrent prétendre que le dieu de la paix et des miséricordes voulait que sa cause fût défendue par le fer et par le feu ; ils poussèrent la démence et l’effronterie jusqu’à soutenir que ce dieu se plaisait à voir fumer le sang humain et demandait qu’on égorgeât tous ceux qui n’auraient pas des idées justes de son essence impénétrable ! Des opinions si cruelles, si contraires aux notions que l’on se forme de la divinité, ont souvent révolté des philosophes éclairés, des gens de bonnes mœurs, et en ont fait des ennemis du dieu qu’on leur peignait sous des traits si bizarres et si propres à effrayer. Frappés des excès qu’ils voyaient commettre en son nom, ils ont quelquefois rejeté toute religion comme incompatible avec les principes de la morale et n’ont regardé ses ministres que comme des tyrans, des imposteurs, des perturbateurs de la société, des brigands ligués pour asservir le genre humain. Mais à quelque degré que l’on porte le doute ou l’incrédulité, quelles que soient les opinions des hommes sur la divinité, sur la religion et ses ministres, ces opinions ne changent rien à celles qu’ils doivent se faire de la morale. Celle-ci a la raison et l’expérience pour base ; elle se fonde sur le témoignage de nos sens. Soit que cette morale ait reçu la sanction de la divinité, soit qu’elle ne soit point revêtue de cette autorité surnaturelle, elle oblige également tous les êtres sociables ou vivants avec des hommes.\par
Celui qui n’aurait point la foi, qui ne croirait point une religion révélée ou une morale expressément confirmée par la volonté divine, ne pourrait pas pour cela s’empêcher d’admettre une morale humaine dont la réalité est constatée par des expériences incontestables, confirmée par les suffrages constants de tous les siècles et de tous les êtres raisonnables. Celui qui nierait même l’existence d’un dieu rémunérateur de la vertu et vengeur des crimes, ne pourrait pas refuser de croire l’existence des hommes, et serait forcé de s’apercevoir à tout moment que ces hommes chérissent ce qui leur est utile ou considèrent la vertu, tandis qu’ils méprisent le vice et punissent le crime. Si, comme on a dit ailleurs\phantomsection
\label{footnote65}\footnote{Voyez la préface ou discours préliminaire.}, les vues d’un homme ne s’étendaient pas au-delà des bornes de sa vie présente, il serait au moins obligé de reconnaître que pour vivre heureux et tranquille en ce monde, il ne peut se dispenser d’obéir aux lois que la Nature impose à des êtres nécessaires à leur félicité mutuelle. En se conformant à ces lois évidentes, tout homme aura droit à l’affection, à l’estime, aux bienfaits de la société, quelles que soient d’ailleurs ses notions vraies ou fausses sur la religion. Bien plus, des hommes très pieux ont cru que tous ceux qui suivaient la sagesse ou la raison pouvaient être regardés comme très religieux, {\itshape même quand ils seraient athées}\phantomsection
\label{footnote66}\footnote{C’est le sentiment de saint Justin martyr. Voyez son {\itshape Apologie}.}.\par
Ces principes nous mettront à portée de juger la doctrine et la conduite des ministres de la religion. Nous les reconnaîtrons pour les organes de la divinité, les interprètes de l’auteur de la Nature, lorsqu’ils nous parleront le langage de la Nature, qui ne peut jamais être contraire au bien de la société\footnote{« Jamais la Nature n’eut un langage, et la philosophie un autre. » Juvénal, {\itshape Satires}, XIV, vers 321.}. Nous regarderons comme des organes de quelque génie malfaisant, comme des menteurs, ceux dont les préceptes nous inviteraient au mal ou tendraient visiblement à rendre les hommes malheureux ou méchants. Enfin, nous applaudirons la conduite et les mœurs de ceux qui seront vertueux, sociables, utiles à la société, et nous gémirons sur les égarements de ceux qui par leurs actions se rendront haïssables et méprisables aux yeux des êtres sensés.\par
Le sacerdoce forma chez tous les peuples du monde un ordre très distingué ; ses fonctions sublimes lui firent partager avec les dieux la vénération des mortels. Les prêtres furent, comme on verra bientôt\footnote{Voyez le chapitre IX de la présente section.}, les premiers savants, les premiers fondateurs des nations. Une longue prescription leur donna et leur conserve en tout pays le droit d’élever la jeunesse, d’enseigner la morale aux hommes, de diriger leurs consciences et leurs mœurs en cette vie, de façon à les y rendre heureux. Enfin, étendant leurs idées au-delà même du trépas, les ministres de la religion se proposent de guider l’homme à une félicité plus grande que celle dont il jouit sur la terre.\par
Bornés dans nos recherches à ne nous occuper que des mobiles humains et naturels qui doivent porter l’homme à faire le bien en ce monde, nous ne nous élancerons pas par la pensée dans un monde qui ne peut être connu que par la foi. Ainsi, nous examinerons seulement les devoirs qu’impose aux ministres des autels le rang qu’ils tiennent dans la société. Egalement respecté par les souverains et les peuples, le clergé occupe le premier rang ou constitue l’ordre le plus considéré dans toutes les nations. En vue des services qu’il rend ou qu’on attend de lui, il est pour l’ordinaire très amplement doté. Ses chefs, ses membres les plus illustres, jouissent de possessions qui les mettent à portée de paraître avec splendeur aux yeux de leurs concitoyens. Tant de marques d’honneur, des distinctions si frappantes, des richesses accumulées imposent évidemment, surtout aux membres les plus favorisés du clergé, le devoir indispensable d’une reconnaissance éternelle, d’un attachement inviolable pour une patrie qui les comble de bienfaits. Sans se rendre coupables de la plus noire ingratitude, des évêques, des prélats, dans les nations européennes, doivent se signaler par leur patriotisme, par leur zèle à contribuer au bien-être, à la conservation des sociétés qui ont généreusement contribué à leur félicité particulière. D’où l’on voit que le prêtre doit, encore plus que tout autre, se montrer citoyen, chérir son pays, défendre sa liberté, stipuler ses intérêts, s’occuper de la félicité publique, maintenir les droits de tous, enfin s’opposer avec noblesse aux progrès du despotisme qui, après avoir dévoré les autres ordres de l’État, pourrait engloutir le clergé à son tour. Nul ordre dans un État n’est plus respectable que le clergé aux yeux des princes mêmes ; c’est donc aux ministres de la religion qu’il appartient de faire connaître aux rois la vérité que des courtisans flatteurs ne leur montrent jamais. Au lieu de calmer les remords des tyrans par des expiations faciles, le prêtre devrait remplir de terreurs salutaires les âmes lâches et cruelles de ces monstres qui causent tous les malheurs des peuples. Placés au grand jour, les prêtres devraient, encore plus par leurs exemples que par leurs discours, exhorter les citoyens à l’union, à la concorde, à l’humanité, à l’indulgence, à la tolérance pour les égarements et les défauts des hommes. Un prêtre intolérant et cruel ne peut pas être l’organe d’un dieu plein de patience et de bonté. Un prêtre qui fait immoler des hommes est un prêtre de Moloch et non de Jésus-Christ. Un prêtre persécuteur, un fanatique qui prêche la discorde, ne sont que des fourbes qui parlent en leur propre nom et dont la langue est guidée par l’intérêt, par le délire et la fureur. Un inquisiteur qui livre un hérétique aux flammes est évidemment un scélérat que l’intérêt de son corps a changé en une bête féroce.\par
Disciples d’un dieu de paix dont le royaume n’était pas de ce monde, les prêtres de nos contrées ne peuvent sans outrager leur divin maître refuser le tribut à César ou se dispenser de contribuer aux charges de l’État sous prétexte d’immunités et de {\itshape droits divins}. Ils peuvent encore bien moins résister aux puissances, soulever les sujets contre les souverains, exercer un empire sur les princes, les priver de leurs couronnes, armer des mains parricides pour immoler des rois. Des prêtres coupables de pareils attentats prouveraient à l’univers qu’ils ne croient pas au dieu qu’ils annoncent aux autres.\par
Imitateurs d’un dieu qui naquit dans l’indigence, successeurs d’apôtres qui vécurent dans la pauvreté, les prêtres du christianisme ne possèdent rien en propre. Dépositaires des aumônes que les fidèles ont remis en leurs mains, ils ne doivent jamais les fermer quand il s’agit de soulager la misère. Un prêtre avare et sans pitié pour les pauvres serait un économe infidèle, un voleur, un assassin. Un prêtre intéressé, ainsi qu’un prêtre orgueilleux, ne pourraient sans démence se donner pour des disciples de Jésus.\par
Occupés d’études pénibles ou livrés à la vie contemplative, les prêtres ont des moyens d’amortir en eux-mêmes l’ambition, l’avarice, la vanité, le goût du luxe et de la volupté dont les autres hommes sont les jouets. La vie du prêtre doit être irréprochable, son état doit le garantir de la contagion du vice ; il est fait pour nous montrer en sa personne le sage, le philosophe, que l’Antiquité promettait vainement.\par
Échauffés, attendris par les exemples touchants de la primitive Église, les prêtres chrétiens sont destinés à faire renaître entre eux les temps fortunés où les fidèles n’avaient qu’un cœur et qu’un esprit. Des querelles interminables et continuelles seraient des scènes scandaleuses très capables de refroidir la confiance des citoyens ; ceux-ci, dans leurs guides, ne devraient trouver que des anges de paix, des modèles de charité, des exemples vivants de toutes les vertus sociales.\par
Si, comme on ne peut en douter, les sciences sont de la plus grande utilité pour les hommes, quels avantages inestimables ne pourraient pas lui procurer tant de cénobites et de moines richement dotés ? Qui oserait se plaindre de leur oisiveté et reprocher leur aisance ou leur opulence à des savants qui emploieraient le temps que leur fournit la retraite à faire des découvertes utiles, des expériences intéressantes, des recherches capables de faciliter en tout genre les progrès de l’esprit humain et les travaux de la société ?\par
Enfin, les ministres de la religion étant presque en tout pays exclusivement chargés de l’éducation de la jeunesse, quelles obligations les nations ne devraient-elles pas leur avoir s’ils s’acquittaient avec soin de la tâche importante et pénible de façonner le cœur et l’esprit de ceux qui deviendront un jour des citoyens !\par
Le clergé serait sans doute le corps le plus utile, le plus digne de la confiance et de l’attachement des peuples, s’il remplissait les fonctions auxquelles il semble destiné. Tels sont en peu de mots les devoirs que la vie sociale et la reconnaissance imposent aux ministres de la religion. En s’y conformant fidèlement, ils mériteraient vraiment le rang et les richesses dont ils jouissent au sein des sociétés, ils s’assureraient la vénération de leurs concitoyens, ils seraient des hommes utiles et respectables aux yeux même de ceux qui, écoutant la voix de la raison, refuseraient de souscrire à leurs dogmes. Il est à présumer que la conduite d’un grand nombre de prêtres et de pasteurs, souvent si peu conforme à leur doctrine, est une des principales causes du dégoût que tant de personnes éclairées conçoivent pour la religion : à la vue de l’esprit despotique, de l’ambition, de l’avidité, de l’intolérance, de l’inhumanité dont les docteurs et les guides des peuples se rendent souvent coupables, bien des gens rejettent cette religion comme incompatible avec les principes les plus évidents de la saine morale.\par
Tout homme ou tout corps qui s’éloigne du chemin de la vertu travaille à sa propre destruction. Un clergé sans lumières et sans mœurs prêche hautement l’irréligion et l’incrédulité. Un corps trop orgueilleux pour faire cause commune avec les autres citoyens ne peut avoir d’appui vraiment solide. Des prêtres ambitieux et turbulents déplaisent également aux souverains et au reste des sujets. Des guides avides et corrompus perdent la confiance et l’affection des peuples. Des docteurs dépourvus de science se rendront méprisables aux yeux des personnes éclairées.\par
Enfin, des prêtres fauteurs du despotisme et de la tyrannie ne peuvent manquer de devenir un jour la proie des despotes et des tyrans : comme Ulysse dans l’antre du Cyclope, ils auront l’unique avantage d’être dévorés les derniers\footnote{Les jésuites, qui pendant plus de deux siècles ont formé une société redoutable à tout l’univers par sa puissance, son crédit, ses intrigues et ses richesses, ont été constamment les trompettes de l’intolérance, les fauteurs de l’ignorance, les flatteurs du despotisme. Un jésuite confesseur de Louis XIV rassura sa conscience sur un impôt que ce prince trouvait lui-même aussi injuste qu’onéreux, en lui disant « qu’il était le maître des biens de tous ses sujets. » C’est sans doute en punition de cette maxime odieuse que nous avons vu depuis peu d’années la société des jésuites détruite sans aucune réclamation dans toute l’Europe et dépouillée par les princes de ses richesses immenses. « Necque enim lex aquior ulla est. Quam necis artifices arte perire sua. » Ovide. Cette doctrine jésuitique fut encore renouvelée en France à l’occasion de la destruction des Parlements en l’année 1771 par l’abbé Du Bault, curé d’Épiais, qui vint exprès à Paris du fond de sa province pour prêcher que les Français étaient esclaves et que leur roi était maître des biens, de la personne et de la vie de ses sujets. Voyez J{\itshape ournal historique de la révolution opérée dans la monarchie française}, etc. Tome II, page 47’. En général, les chefs du clergé de France ont montré la joie la plus indécente quand les actes réitérés du plus affreux despotisme eurent anéanti tous les tribunaux de leurs pays. Faut-il que les ministres de la religion soient presque toujours les ennemis de la liberté des nations, à laquelle ils sont eux-mêmes si fortement intéressés !}.
\subsection[{Chapitre VIII. Devoirs des Riches}]{Chapitre VIII. Devoirs des Riches}
\noindent Les richesses donnent et doivent donner à ceux qui les possèdent, un rang distingué parmi leurs concitoyens. L’homme riche est pour ainsi dire plus citoyen qu’un autre ; son opulence le met à portée de prêter à ses semblables des secours dont l’indigence est incapable. Il tient à la société par un plus grand nombre de liens qui l’obligent de s’intéresser beaucoup plus à son sort que le pauvre qui, n’ayant rien ou peu de chose à perdre, doit s’intéresser moins vivement aux révolutions qu’il voit arriver dans son pays. Celui qui n’a rien que ses bras n’a point à proprement parler de patrie : il est bien partout où il trouve les moyens de subsister, au lieu que l’homme opulent peut être utile à bien des gens, est en état d’assister sa patrie, au destin de laquelle il se trouve intimement uni par ses possessions, dont la conservation dépend de celle de la société. Tandis qu’au siège de Corinthe les habitants s’empressaient à repousser l’ennemi par toutes sortes de moyens, Diogène, pour se moquer de leurs embarras, s’amusait follement à remuer son tonneau.\par
Ne soyons donc pas étonnés de voir que presque en tout pays les lois, les usages, les institutions, souvent injustes et cruelles pour les pauvres, ont été plus favorables aux riches et montrent une partialité marquée pour les favoris de la fortune. Les grands, les puissants, les opulents durent communément être préférés à des indigents qui parurent moins utiles à la société.\par
Cependant, ces usages et ces lois furent évidemment injustes quand elles permirent aux heureux de la terre d’opprimer et d’écraser les faibles et les malheureux. L’équité, dont la fonction est de remédier à l’inégalité des hommes, dut apprendre aux riches qu’ils devaient respecter la misère du pauvre, et cela pour leur propre intérêt. En effet, sans le travail et les secours continuels du pauvre, le riche ne serait-il pas lui-même dans la misère, et ces secours venant à lui manquer, ne le rendraient-ils pas plus malheureux que le pauvre lui-même ?\par
Ainsi, la justice, d’accord avec l’humanité, avec la commisération et avec toutes les vertus sociales, apprend à l’homme riche à voir dans l’indigent l’un de ses associés, nécessaire à son propre bonheur, dont il doit mériter les secours en lui facilitant, en échange de ses peines, les moyens de subsister, de se conserver, de se rendre heureux à sa manière.\par
C’est ainsi que la vie sociale met les hommes dans une dépendance mutuelle. Voilà comme les grands ont besoin des petits, sans lesquels ils seraient eux-mêmes petits. L’opulent, pour jouir de l’aisance, des plaisirs, des commodités de la vie, a besoin des bras et de l’industrie de l’indigent, que sa misère rend laborieux, actif, industrieux. En un mot, la moindre réflexion nous prouve que dans la société les membres sont unis les uns aux autres par des nœuds indissolubles que nul d’entre eux ne peut briser sans se faire tort à lui-même. Elle nous fait sentir que nul citoyen n’a le droit de mépriser les autres, d’abuser de leur faiblesse ou de leur indigence, de les traiter avec hauteur ou dureté ; elle nous montre que le riche est continuellement intéressé à faire du bien, sous peine d’être haï ou méprisé pour n’avoir pas rempli sa tâche dans la vie sociale. Le citoyen que la société fait jouir d’une grande somme de bonheur doit plus à cette société que les malheureux qu’elle néglige.\par
Les riches peuvent être comparés aux sources, aux ruisseaux, aux rivières destinées à répandre leurs eaux pour féconder les terres arides afin de leur faire produire des plantes et des fruits. Le riche avare ressemble à ces fleuves dont les eaux pour quelque temps se perdent sous la terre. Le riche prodigue agit comme les rivières débordées qui se répandent dans les campagnes sans y produire la fécondité.\par
Enfin, pour suivre notre comparaison, les richesses mal acquises et follement prodiguées ressemblent à ces torrents qui détruisent les endroits par où ils passent et qui finissent le plus souvent par laisser à sec le lit qu’ils ont formé avec tant de violence.\par
Les réflexions qui viennent d’être présentées peuvent donc servir à fixer notre jugement sur ce que la plupart des moralistes ont dit des richesses. Le plus grand nombre des sages les a blâmées comme des obstacles à la vertu, comme des moyens de corruption, comme la source intarissable de mille besoins imaginaires qui nous plongent dans le luxe, la volupté, la mollesse, qui nous endurcissent le cœur et nous rendent injustes ; enfin, qui nous détournent de la recherche des vérités nécessaires au vrai bonheur de l’être intelligent. Tel est en général le jugement que les anciens philosophes ont porté sur l’opulence, qu’ils ont montrée comme le plus dangereux écueil de la vertu.\par
Écoutons un moment Sénèque qui, du sein des richesses, ose en faire la satire : « Depuis, dit-il\phantomsection
\label{footnote67}\footnote{Voyez Sénèque, {\itshape Épîtres}, 115.}, que les richesses ont été mises en honneur parmi les hommes et sont devenues en quelque façon la mesure de la considération publique, le goût des choses vraiment honnêtes et louables s’est entièrement perdu. Nous sommes tous devenus des marchands, tellement corrompus par l’argent que nous ne demandons plus de quelle utilité une chose peut être, mais de quel agrément ; l’amour des richesses nous rend tour à tour honnêtes gens ou fripons, selon que notre intérêt ou les circonstances l’exigent… Enfin, ajoute-t-il, les mœurs sont si dépravées que nous maudissons la pauvreté et que nous la regardons comme une chose déshonorante, comme une véritable infamie ; en un mot, elle est l’objet du mépris des riches et de la haine des pauvres. »\par
Platon décide formellement, qu’« il est impossible d’être à la fois bien riche et honnête homme, et que comme il n’existe pas de bonheur sans vertu, les riches ne peuvent pas être réellement heureux\footnote{Platon, {\itshape Des Lois}, livre V, page 742, E. et 743., A. B. tome 2, édit. Henri Stéphani, année 1578.} ». Les moralistes nous font encore une peinture des inquiétudes, compagnes assidues de l’opulence et qui empoisonnent sa possession, que tout le monde envie ; on nous la montre comme l’instrument de toutes les passions. Mais, comme dit Bacon : « Les richesses sont le gros bagage de la vertu ; le bagage est nécessaire à une armée mais il en retarde quelquefois la marche et fait perdre l’occasion de remporter la victoire. »\par
Pour réduire ces opinions à leur juste valeur, nous dirons qu’en elles-mêmes les richesses ne sont rien : elles ne sont que ce que les font valoir ceux qui les possèdent. Un lit doré ne soulage point un malade, une fortune brillante ne rend pas un sot plus sage. « L’aisance et l’indigence, dit Montaigne, dépendent de l’opinion d’un chacun, et non plus la richesse que la gloire, que la santé, n’ont qu’autant de beauté et de plaisir que leur en prête celui qui les possède\footnote{Voyez {\itshape Essais} de Montaigne, livre I, chap. 40, p. 198, tome 2, 1745.}. » Entre les mains d’un homme sage, humain, libéral, l’opulence est évidemment la source d’un bien-être et d’un contentement autant de fois renouvelé qu’il trouve d’occasions d’exercer ses dispositions estimables.\par
Nous dirons que l’homme sensible, dont le cœur sait goûter le plaisir de faire des heureux, d’être utile à son pays, de répandre ses bienfaits sur tout le genre humain, ne serait point embarrassé quand il aurait en son pouvoir toutes les richesses et du Potose et du Pérou. Nous dirons que ce qui rend souvent la pauvreté et la médiocrité fâcheuses pour l’homme honnête qui s’attendrit sur les maux de ses semblables, c’est l’impossibilité où elles le mettent de satisfaire les désirs de sa grande âme, qui voudrait pouvoir soulager tous les malheureux que le sort lui présente, exciter tous les talents utiles à ses concitoyens, essuyer les larmes de tous ceux que l’infortune accable ; avec un cœur bien placé, les trésors de Crésus ne seraient jamais des obstacles à sa félicité. « Quand tu auras, dit Plutarque, profité des leçons de la philosophie, tu vivras partout sans déplaisir et tu jouiras du bonheur en tout état : la richesse te réjouira parce que tu auras plus de moyens de faire du bien à plusieurs ; la pauvreté, d’autant que tu auras moins de soucis {\itshape ;} la gloire, d’autant que tu te verras honoré ; l’obscurité, d’autant que tu seras moins envié\footnote{Voyez Plutarque, {\itshape Du Vice et de la Vertu}.}. » « Avec la vertu, dit-il ailleurs, toute façon de vivre est agréable. Tu seras toujours content de la fortune quand tu auras bien appris en quoi consiste la probité et la bonté ».\par
Nous conviendrons qu’il est rare que les richesses se trouvent dans les mains de personnes de cette trempe : l’opulence ne se voit guère combinée soit avec de grandes lumières\footnote{« On ne trouve pas beaucoup de sens commun dans cette haute fortune. » Juvénal, {\itshape Satires}, VIII, vers 73.}, soit avec de grandes vertus. Le plus souvent, la fortune aveugle se plaît à combler de ses dons d’indignes favoris qui ne savent en faire usage ni pour leur propre bonheur, ni pour celui des autres ; enfin, il est très peu de gens qui aient des âmes assez fortes pour soutenir le poids d’une grande opulence\footnote{C’est faiblesse d’âme de ne pouvoir supporter les richesses. Sénèque, {\itshape Épîtres}, I, V, 6 (fin). — Plutarque observe très sagement que, « comme tous les tempéraments ne sont pas propres à porter beaucoup de vin, tous les esprits ne sont pas plus capables de supporter une grande fortune sans tomber dans l’ivresse et sans perdre la raison ». Voyez Plutarque, {\itshape Vie de Lucullus}.}. « L’or, disait Chilon, est la pierre de touche de l’homme. »\par
N’en soyons point surpris : les richesses dont la plupart des hommes jouissent sont ou le fruit de leurs propres travaux, de leurs intrigues, de leurs bassesses, ou bien elles sont transmises par leurs ancêtres ; dans ces deux cas il est assez difficile qu’elles tombent en des mains vraiment capables d’en faire un usage conforme à la raison\footnote{« Dives ut iniquus, aut iniqui hæres. » S. Hiéron. — L’homme riche est injuste ou l’héritier d’un homme injuste. « Beaucoup de méchants, dit le poète Théognis, deviennent riches et beaucoup de gens de bien demeurent pauvres. Mais nous ne voudrions pas changer notre vertu pour leurs richesses car la vertu reste toujours, tandis que les richesses changent de maîtres à tout moment. » Voyez {\itshape Poetæ Græci Minores}. — Quelqu’un disait à Sylla, qui se vantait de sa vertu : « Eh ! comment serais-tu vertueux, toi qui, n’ayant rien hérité de ton père, te trouves pourtant avoir de si grands biens ? Voyez Plutarque, dans {\itshape La Vie de Sylla}. — Un proverbe vulgaire dit qu’heureux sont les enfants dont les pères sont damnés.}. Ceux qui travaillent à leur fortune n’ont ni le temps ni la volonté de se former le cœur ou l’esprit ; uniquement occupés du soin de leurs affaires, ils n’ont aucune idée des avantages qui résulteraient pour eux de la culture de leurs facultés intellectuelles. D’un autre côté, les hommes, quand ils sont fortement animés du désir des richesses, se rendent pour l’ordinaire peu délicats sur les moyens d’en obtenir. « Le gain, dit Juvénal, a toujours bonne odeur, quelle que soit la marchandise\footnote{Juvénal, {\itshape Satires}, XIV, vers 204.}. » Il faut, pour parvenir à la fortune, une conduite si basse, si rampante, si oblique, que les honnêtes gens ont de la peine à se prêter à mille démarches qui ne coûtent rien à ceux qui veulent s’enrichir à tout prix. Enfin, rien de plus difficile que d’acquérir de grands biens sans faire quelques outrages à la probité. D’où l’on voit que l’occupation pénible de faire sa fortune par soi-même est assez incompatible avec une observation scrupuleuse des règles de la morale. La fortune ne paraît aveugle dans la distribution de ses faveurs que parce que les hommes qui en seraient les plus dignes ne veulent pas les acheter au prix qu’elle y met communément. « Il est, disait Thalès, aussi facile au sage de s’enrichir qu’il est difficile de lui en faire naître l’envie. »\par
« Il n’y a, dit Homère, que les âmes honnêtes qui puissent être guéries. » La morale, qui ne peut jamais s’écarter des règles immuables de l’équité, n’a point de préceptes pour des hommes avides, sans honneur, sans probité, qui ne trouvent rien de plus important que de faire leur fortune ; ses leçons paraîtraient ridicules et déplacées si elles osaient s’adresser à des courtisans sans âme, à des exacteurs impitoyables, à des publicains qui s’engraissent du sang des peuples et qui s’abreuvent des larmes des malheureux. L’équité naturelle ne serait point écoutée de tous ceux qui se persuadent que la volonté des princes rend juste la rapine et le vol, ni de ces hommes endurcis qui ne trouvent leur intérêt que dans l’infortune des autres.\par
La morale ne donnerait pareillement que des conseils inutiles, ou trop vagues à ceux des commerçants dont les profits les plus licites ou permis par l’usage et les lois ne sont pas toujours approuvés par une justice sévère : le marchand est trop souvent juge et partie dans sa propre cause pour n’être pas fréquemment tenté de faire pencher la balance du côté de son intérêt particulier. Cet intérêt se trouve communément prêt à lui suggérer des sophismes qu’il n’a ni le temps ni la volonté de bien se démêler. Enfin, il faut bien de la force et de la vertu pour qu’un homme dans le commerce ne succombe pas souvent à la tentation de mettre à profit soit les besoins, soit l’ignorance et la simplicité de ses concitoyens.\par
En général, la morale, au risque de n’être point écoutée, dira toujours aux hommes d’être justes, de résister à la cupidité, de respecter la bonne foi, de craindre d’avoir un jour à rougir d’une fortune acquise aux dépens de la conscience et de la probité, parce que sa possession serait troublée soit par des remords importuns, soit par l’indignation publique, soit par des avanies.\par
Quand l’opulence est le fruit du travail des ancêtres, il est encore assez difficile que celui qui en hérite ait appris l’art d’en bien user. Comment des pères dépourvus eux-mêmes de principes, de sentiments louables et de vertus, en pourraient-ils inspirer à leurs enfants ? L’éducation des personnes nées dans l’opulence ne se propose communément rien moins que de leur former un cœur juste, sensible, bienfaisant. Bien plus, elle réussit difficilement à leur donner le goût de l’étude et de la réflexion. Des parents ignorants et peu touchés des charmes de la vertu laisseront leur fortune à des enfants qui leur ressembleront. Des avares, des usuriers, des concussionnaires, des monopoleurs, des courtisans, des financiers seraient-ils capables d’inspirer à leurs descendants des sentiments nobles et généreux qui seraient incompatibles avec tous les moyens d’aller à la fortune ? Bien plus, ces parents si avides n’ont pas même le talent de leur apprendre à conserver les richesses qu’ils leur laisseront. On remarque assez constamment que l’opulence la plus énorme se transmet rarement jusqu’à la troisième génération. La folie des enfants parvient très promptement à dissiper les trésors accumulés par l’injustice des pères.\par
Le fils d’un courtisan, d’un homme sans cœur, d’un flatteur, est-il fait pour avoir quelque estime pour la vertu ? Un père fastueux et vain, plongé dans le luxe et la débauche, daignera-t-il s’occuper à façonner l’âme de son fils et à lui montrer la manière de faire un usage sensé des biens qu’il doit un jour posséder ? Enfin, le fils d’un homme qui nage dans l’abondance sera-t-il de lui-même tenté d’acquérir la modération, la douceur, les vertus, les talents et les connaissances qui peuvent un jour contribuer à son propre bien-être ? Les enfants nés au sein de l’opulence ne deviennent pour l’ordinaire que des furieux qui se croient tout permis. « La satiété, dit Théognis, fait naître la férocité\phantomsection
\label{footnote68}\footnote{Plutarque observe, au sujet de Sylla, que la fortune produisit en lui un changement total et le rendit farouche et cruel. Et par ce grand changement, dit ce philosophe, « il donna lieu d’accuser les grands honneurs et les grandes richesses et de leur reprocher qu’elles ne permettent pas aux hommes de conserver leurs premières mœurs mais qu’elles engendrent dans leurs cœurs l’emportement, la vanité, l’inhumanité, l’insolence. » Voyez Plutarque, {\itshape Vie de Sylla}. — La plupart des riches se font haïr du pauvre non seulement par l’envie qu’ils excitent en lui mais encore par le mal qu’ils lui font gratuitement et par les incommodités qu’ils lui causent. Dans les grandes villes surtout, le peuple est perpétuellement embarrassé dans ses travaux les plus nécessaires par les équipages toujours en mouvement des grands et des riches désœuvrés qui, dans la précipitation avec laquelle ils tâchent de fuir l’ennemi, écrasent, renversent impunément et sans remords les malheureux qui se trouvent sur leur chemin.}. »\par
Des fortunes énormes, des richesses immenses amassées dans peu de mains annoncent un gouvernement injuste qui s’embarrasse fort peu de l’aisance et de la subsistance du plus grand nombre de ses sujets. Cent familles aisées sont plus utiles à l’État que le riche engourdi dont les trésors enfouis exciteraient l’activité de toute une province. Des richesses réparties font le bien de l’État : elles augmentent l’industrie et conservent les mœurs que la grande opulence, ainsi que la profonde misère, corrompent et détruisent. La grande fortune enivre l’homme ou l’engourdit totalement. « Les beaux habits, dit Démophile, gênent le corps ; les grandes richesses gênent l’esprit. » D’un autre côté, une grande indigence, comme on verra bientôt, sollicite souvent au crime. Il n’est point de pays où l’on trouve des particuliers plus riches et autant de malfaiteurs que dans les nations opulentes.\par
Thalès disait que « la république la mieux ordonnée est celle où personne n’est ni trop riche ni trop pauvre. » L’état de médiocrité fut toujours l’asile de la probité. Un gouvernement est bien imprudent et bien coupable quand il inspire à ses sujets une passion effrénée pour les richesses : il anéantit par là tout sentiment d’honneur ou de vertu !\par
Le philosophe Cratès s’écriait : « Ô hommes ! Où vous précipitez-vous en prenant des peines pour amasser des richesses, tandis que vous négligez l’éducation de vos enfants à qui vous devez les laisser ? »\par
Rien ne modifie plus puissamment les hommes que l’éducation : l’exemple, l’instruction, les maximes des parents leur donnent les premières impulsions. Il ne faut donc pas s’étonner de trouver dans des nations infectées par le luxe, par la dissipation et la débauche, tant de riches dépourvus des qualités nécessaires pour se rendre heureux par leurs richesses et encore bien moins disposés à s’occuper du bien-être des autres.\par
Le faste, la représentation, le besoin illimité de vivre {\itshape suivant son état}, dont la vanité se fait toujours une haute idée, les dépenses énormes qu’exigent des plaisirs recherchés, font que l’homme le plus opulent n’a jamais de superflu : une fortune immense lui suffit à peine pour faire face à tous les besoins que sa vanité, jointe au dégoût des plaisirs ordinaires, fait naître dans sa tête.\par
Il n’est point de trésors capables de satisfaire les caprices et les fantaisies innombrables que le luxe, la dissipation et l’ennui enfantent à tout moment : à peine les revenus des rois pourraient-ils suffire pour apaiser la soif inextinguible d’une imagination déréglée.\par
L’ennui, comme on a déjà pu s’en convaincre, est un bourreau qui perpétuellement châtie au nom de la Nature ceux qui n’ont point appris à régler leurs désirs, à s’occuper utilement, à mettre l’économie dans leurs amusements. Pourquoi voit-on sans cesse les grands et les riches montrer si rarement un front serein ? C’est qu’au sein même des honneurs, de la fortune et des plaisirs, ils ne jouissent de rien ; tous les amusements sont épuisés pour eux : il faudrait que la Nature créât en leur faveur de nouvelles jouissances et de nouveaux organes.\par
La bonne chère, la volupté, les spectacles, les plaisirs les plus variés n’ont plus rien qui les touche\footnote{« Leurs plaisirs mêmes sont agités ; ils sont en proie à mille terreurs ; et au sein de leurs jouissances cette pensée importune se présente à leur esprit : “Combien ce bonheur doit-il durer ?” » Sénèque, {\itshape De Brevitate Vitæ}, chap. 17, 1.} ; rien ne les réveille. Au milieu des fêtes les plus brillantes l’ennui les assiège, l’imagination les tourmente et leur persuade toujours que le plaisir doit se trouver à l’endroit où ils ne sont pas.\par
De là cette agitation, cette inquiétude convulsive que l’on remarque communément dans les princes, les grands et les riches. Ils semblent passer leur vie à courir pour chercher le plaisir sans jamais en jouir lorsqu’ils l’ont sous les yeux.\par
« L’un, dit Lucrèce, quitte son riche palais pour se dérober à l’ennui mais il y rentre un moment après, ne se trouvant pas plus heureux ailleurs. Cet autre se sauve à toutes brides dans ses terres, comme pour éteindre un incendie ; mais à peine en a-t-il touché les limites, qu’il y trouve l’ennui ;...il regagne la ville avec la même promptitude… chacun se fuit sans cesse. Etc.\footnote{Voyez Lucrèce, livre III. : « Je croyais autrefois, ô Phanias ! (faisait dire Ménandre à un acteur) que ceux qui n’ont pas besoin de gagner leur vie jouissaient d’un sommeil tranquille et jamais ne s’écriaient {\itshape que je suis malheureux} ! Je pensais qu’il n’y avait que le pauvre qui s’agitait dans son lit, mais je vois maintenant que vous autres, qui passez pour être heureux, n’êtes pas mieux que nous. » Voyez {\itshape Poetæ Græci Minores}.} »\par
S’occuper d’une façon utile et faire du bien à ses semblables, voilà les seuls moyens d’échapper à l’ennui qui tourmente tant de riches pour lesquels il n’existe plus de plaisirs sur la terre. Les plaisirs des sens s’épuisent, le contentement puéril que peut donner la vanité disparaît quand il est habituel ; mais les plaisirs du cœur se renouvellent à tout moment et le contentement inexprimable qui résulte de l’idée du bonheur que l’on répand sur les autres est une jouissance qui jamais ne s’altère.\par
{\itshape Essayez de faire des heureux, pour être heureux vous-mêmes} ; voilà le meilleur conseil que la morale ait pour les riches.\par
Aristote, en parlant des richesses, dit que « les uns n’en usent point, et que les autres en abusent ». Que l’homme riche serait heureux s’il savait profiter des avantages que la fortune lui met entre les mains !\par
Comment l’ennui pourrait-il l’assaillir lorsqu’avec une âme sensible et tendre il posséderait un esprit cultivé ? Tout se changerait en plaisirs sous la main du riche bienfaisant.\par
Essuyer les larmes du malheureux, porter inopinément la consolation et la joie dans une famille affligée, réparer les injustices du sort quand il opprime le mérite infortuné, récompenser libéralement les services qu’on a reçus, déterrer et mettre au jour les talents flétris par l’indigence, exciter le génie aux découvertes utiles, savoir jouir en secret du bonheur de faire des heureux sans leur montrer la main de leur bienfaiteur, rendre à la gaieté le cœur d’un ami vertueux qui se trouve dans la détresse, par des travaux utiles à la patrie occuper et faire subsister la pauvreté laborieuse, ranimer le cultivateur découragé, mériter les bénédictions et la tendresse des êtres dont on est environné : voilà des moyens sûrs de se procurer des jouissances durables et variées, de calmer l’envie que cause presque toujours une grande fortune, et même de faire pardonner les voies par lesquelles cette fortune a pu s’acquérir par d’injustes pères.\par
Des descendants vertueux peuvent parvenir à faire oublier la source impure de leur opulence : l’indignation et l’envie se taisent à la vue du bon usage que l’homme de bien sait faire de ses richesses ; il se rend heureux lui-même en méritant les applaudissements de ses concitoyens\phantomsection
\label{footnote69}\footnote{L’Antiquité nous fournit dans Pline le Jeune un exemple bien touchant de ce que peut l’opulence bienfaisante. Cet homme excellent se montre dans ses lettres perpétuellement occupé du sort de ses amis et de tous ceux qui l’entourent. À l’un il remet des dettes considérables, il se charge de payer celles d’un autre ; il augmente la dot de la fille d’un ami qui n’est plus afin de lui faire trouver un meilleur parti. Il vend une terre au-dessous de sa valeur pour enrichir à son insu un homme qui lui est cher. Il fait à un autre ami un sort qui le met à portée de vivre dans l’indépendance et le repos jusqu’à la fin de ses jours. Il fonde une bibliothèque à Côme, sa patrie, ainsi qu’une maison d’asile pour les orphelins. Enfin, il nous apprend lui-même qu’une sage économie, encore plus que la richesse, le mettait en état de satisfaire son humeur bienfaisante. Voyez les {\itshape Lettres de Pline}. — Nous trouvons des dispositions semblables dans Gillias, citoyen d’Agrigente, qui, suivant Valère Maxime, ne parut s’occuper toute sa vie qu’à faire de ses immenses richesses un usage utile à ses concitoyens. Il dotait de pauvres filles, il venait au secours de tous les malheureux, il exerçait l’hospitalité indistinctement envers tous les étrangers, il approvisionnait sa patrie dans les temps de disette ; en un mot, le bien de Gillias semblait être un patrimoine commun à tous les hommes. Voyez Valère Maxime, livre IV, chap. 8. — Que l’on compare la conduite de ces riches avec celle d’une foule de millionnaires stupides qui n’imaginent que des folies pour dissiper leur fortune ou qui ne songent qu’aux moyens d’en augmenter la masse. Des traitants toujours avides, des monopoleurs engraissés par les calamités nationales, des riches débauchés, des hommes livrés à la vanité du luxe ne sont guère touchés du bien public, auquel ils ne se croient aucunement intéressés. Quelle idée la postérité prendra-t-elle de notre siècle lorsqu’elle saura qu’au milieu de Paris, de la capitale d’un royaume opulent et puissant, où le luxe élève chaque jour des monuments aussi coûteux qu’inutiles, parmi tant de gens qui ne savent que faire de leur argent, il ne se trouve pas des personnes assez généreuses pour contribuer à la reconstruction des écoles de médecine qui menacent depuis longtemps d’ensevelir sous leurs ruines les maîtres et les disciples de l’art le plus intéressant ! L’art de guérir n’est-il donc rien pour des insensés sujets à tant d’infirmités ? Des salles de spectacle, des Cotisées, sont-ils des monuments plus importants que le séjour de ceux qui veillent à la santé de tous les citoyens ? Quelle honte pour une ville qui fait vivre dans l’abondance et le luxe des légions de farceurs, de chanteuses, de baladins, et qui ne daigne rien pour favoriser les études longues et pénibles des savants les plus utiles à la société ! Tandis qu’un Opéra corrupteur lève chaque année une contribution de {\itshape cinq à six cent mille livres} sur un public désœuvré, la Faculté de Médecine ne possède que {\itshape dix-huit cents livres} de rentes ; ses professeurs ne reçoivent presque aucun salaire et le pauvre est dans l’impossibilité de se faire agréger à un corps dont, s’il était secouru, il pourrait devenir l’ornement. Ô Athéniens ! Vous êtes des enfants.}.\par
C’est surtout dans les campagnes où les riches, éloignés de l’atmosphère empestée des villes et de la contagion du luxe, trouveraient des occasions de faire un usage honorable de leur opulence et de se montrer citoyens. Mais trop souvent accoutumés à l’air infecté des grandes sociétés, au tourbillon des plaisirs frivoles, aux vices qui sont devenus des besoins pour eux, les riches regardent les capitales comme leur véritable patrie ; ils se croient en exil dans leurs terres à moins d’y transporter les désordres, le bruit, les funestes amusements auxquels ils se sont habitués. Sans cela, les plaisirs champêtres, les charmes de la Nature, leur paraissent insipides : ils ignorent totalement le plaisir de faire du bien.\par
Ces plaisirs sont pourtant plus solides et plus purs que ceux dont se repaît la vanité. Peut-on leur comparer le futile avantage de se faire remarquer du vulgaire par des habits, des équipages, des livrées, des ameublements recherchés et partout le méprisable étalage auquel le luxe attache un si haut prix ? Le riche injuste peut-il se flatter de mériter l’estime publique en déployant insolemment aux yeux de ses concitoyens appauvris une magnificence insultante ? Dans la crainte d’exciter l’indignation générale, ces hommes gorgés de la substance des peuples ne feraient-ils pas mieux de dérober à tous les regards une opulence achetée par des iniquités et des crimes ? L’amour-propre de ces favoris de Plutus peut-il les aveugler au point de croire qu’une nation opprimée pour les enrichir leur pardonnera l’impudence avec laquelle ils osent étaler les fruits de leurs rapines ? Non ; les applaudissements et les hommages des flatteurs, des parasites dont leur table est entourée, ne les persuaderont jamais de leur mérite. Ils ne feront point taire les reproches d’une conscience inquiète : tout leur faste imposant, leurs repas somptueux, ne feront que des envieux de ceux même qu’ils prennent pour leurs amis. Les convives du traitant enrichi, en l’aidant à consumer ses richesses, ne lui en ont aucune obligation : ils regardent sa dépense comme un devoir, comme une restitution faite à la société et qu’ils se chargent de recevoir en son nom. L’homme qui n’a que de la vanité n’est pas fait pour avoir des amis : il n’a que des adulateurs, de lâches complaisants prêts à lui tourner le dos aussitôt que les richesses, dont ils prennent assidûment leur part, se seront écoulées\phantomsection
\label{footnote70}\footnote{Des voyageurs nous apprennent qu’il se trouve des mahométans qui se font scrupule de manger avec ceux qu’ils soupçonnent d’avoir mal acquis leur fortune. Un calife de Bagdad s’était fait une loi de n’employer à se nourrir et se vêtir que l’argent provenu du travail de ses mains.}.\par
On est tout surpris de voir les grands et les riches abandonnés de tout le monde dès que la fortune les abandonne ; mais il y aurait bien plus lieu d’être surpris si leurs prétendus amis en usaient autrement. Le riche fastueux et prodigue ne considère que lui-même dans les dépenses qu’il fait ; c’est à sa propre vanité qu’il sacrifie sa fortune, c’est pour être applaudi qu’il répand l’or à pleines mains, c’est pour exercer une sorte d’empire sur des hommes avilis qu’il les invite à venir prendre part à ses festins. Ceux-ci comptent être quittes avec lui lorsqu’ils ont régalé sa sottise de la fumée de leur encens. En effet, le même homme qui consent à dépenser dans un repas des sommes suffisantes pour tirer toute une famille de la misère ne se déterminerait jamais à faire une dépense beaucoup moindre si elle était ignorée. Bien plus, cet homme qui veut paraître si généreux et si noble aux yeux des flatteurs dont il est environné ne voudrait peut-être pas leur donner en secret leur repas en argent.\par
Ce n’est ni la bienveillance ni le désir d’obliger qui sont les vrais mobiles du faste et qui causent le dérangement des prodigues : c’est une vanité concentrée, qui très souvent leur tient lieu de bonté, d’affection, d’amitié, et d’amour même. Rien de plus commun que de voir un homme riche se ruiner pour une maîtresse pour laquelle au fond du cœur il ne sent point d’amour ; il ne veut que la gloire de supplanter ses rivaux et de remporter à force d’argent la victoire sur eux. Comment d’ailleurs un tel homme pourrait-il se flatter de posséder le cœur d’une femme usée par le plaisir et toujours prête à préférer l’amant qui mettra le plus haut prix à ses faveurs ?\par
Les goûts souvent ruineux que des riches affectent sont rarement vrais et sincères : ils sont pour l’ordinaire uniquement fondés sur une sotte vanité qui leur persuade qu’ils seront admirés comme des gens d’un goût exquis et rare, comme des {\itshape connaisseurs}, et surtout comme des hommes très riches et très heureux. C’est ainsi qu’un financier privé de goût réel rassemble souvent à grands frais une collection immense de curiosités dont il n’a nulle idée, de livres qu’il ne lira jamais, de tableaux dont il ne sait aucunement juger\footnote{On peut aisément remarquer que les artistes qui servent au luxe — les brocanteurs, les bijoutiers, les tailleurs, etc. — sont communément peu délicats sur les profits. Accoutumés à traiter avec des dupes, ils deviennent ordinairement fripons. D’un autre côté, en fréquentant les grands ils contractent l’habitude de la fatuité. Voilà les gens que le luxe fait prospérer aux dépens des cultivateurs et des citoyens utiles ! Joignez aux gens de cette espèce des filles de joie, des actrices, des proxénètes, des danseurs, des fripons de toutes couleurs, et vous aurez la liste des personnages intéressants que la corruption des mœurs fait briller, qui absorbent plus ou moins promptement les facultés des hommes les plus opulents et qui s’attirent même souvent des distinctions et des récompenses de la part du gouvernement. « Les quêteurs, les actrices de mime, les parasites hâbleurs, toute cette engeance… » Horace, {\itshape Satires}, livre I, 2, vers 2.}. Cependant, il faut convenir que l’ennui a souvent autant de part que la vanité aux dépenses inutiles qui dérangent les plus grandes fortunes ; c’est lui qui détermine à payer chèrement des objets faits pour dégoûter, ou du moins pour paraître insipides aussitôt qu’on les a possédés. C’est à l’ennui des riches que sont dues les productions si variées, si changeantes et quelquefois si bizarres, de la mode, et qui semblent faire pardonner au luxe tout le mal que d’ailleurs il fait aux nations.\par
Mais les consolations passagères que le luxe fournit aux ennuis et à la vanité de quelques riches désœuvrés ne doivent pas le justifier des maux sans nombre qu’il cause aux pauvres, c’est-à-dire à la partie la plus nombreuse de toute société. Le luxe n’est avantageux qu’aux artisans du luxe ; il ne procure que des maux à la portion vraiment utile et laborieuse des citoyens. Le prix qu’il en coûte à un riche ennuyé pour un chef-d’œuvre de la peinture ou de la sculpture, pour une superbe tapisserie, pour les dorures dont il orne son palais, pour un habit brodé, pour un bijou stérile, suffirait quelquefois pour vivifier plusieurs familles de cultivateurs honnêtes, bien plus nécessaires à l’État que tant d’artistes qui ne font que repaître les yeux ou les oreilles. Que l’homme de goût admire les productions sublimes des arts, qu’il rende justice aux talents divers qui amusent ses yeux ; mais le vrai sage, toujours sensible aux afflictions et aux besoins du plus grand nombre, ne pourra jamais les préférer aux arts utiles et nécessaires à la société qui feraient subsister des millions de malheureux. Une province défrichée et rendue fertile pour ses habitants, des marais desséchés pour donner un air plus salubre, des canaux creusés pour faciliter les transports sont pour un bon citoyen des objets plus intéressants que des palais ornés des tableaux de {\itshape Raphaël}, des statues de {\itshape Michel-Ange}, accompagnés des jardins de {\itshape Le Nôtre}.\par
Mais les riches, pour l’ordinaire, ne sont pas accoutumés à s’occuper du bien qu’ils pourraient faire au peuple qu’ils méprisent : ils aiment mieux lui faire sentir leur puissance d’une façon propre à se faire haïr. Loin de diminuer l’envie des indigents, ils semblent la réveiller sans cesse par une conduite arrogante et tyrannique. On dirait que les hommes à qui la fortune a donné tous les moyens de se faire aimer ne savent s’en servir que pour se rendre odieux et méprisables. Au lieu de soulager la misère du pauvre, les riches ne semblent répandus sur la terre que pour la multiplier ; au lieu de féconder les terres arides et stériles, l’opulence et la puissance ne font que les ravager. Est-on heureux soi-même quand on ne voit autour de soi que des infortunés ? Les richesses peuvent-elles avoir quelque chose de flatteur quand elles ne font qu’attirer les malédictions et la haine de ceux dont elles pourraient concilier l’amour ?
\subsection[{Chapitre IX. Devoirs des Pauvres}]{Chapitre IX. Devoirs des Pauvres}
\noindent Avec quelle indignation un cœur sensible regardera-t-il le luxe quand il s’apercevra qu’il endurcit le cœur des princes, des grands et des riches, dès qu’il est parvenu à leur forger des besoins infinis et toujours insatiables qui les empêchent de soulager les misères des peuples en ne leur laissant jamais de superflu ! De quel œil une saine politique pourra-t-elle envisager l’aversion que ce luxe inspire aux riches pour les campagnes, que leurs richesses devraient ranimer ? Ne gémira-t-elle pas en voyant ces campagnes qui, loin d’être secourues, sont dépeuplées pour procurer un nombre inutile de valets à l’opulence indolente ? Enfin, tout homme de bien ne sera-t-il pas sensiblement touché en voyant ces serviteurs corrompus par l’exemple de leurs maîtres porter jusque dans les dernières classes de la société la corruption et les vices dont ils se sont abreuvés dans les villes ?\par
Dans un État corrompu, les influences du luxe, funestes aux riches qu’il met en délire, se font sentir d’une façon plus cruelle encore aux pauvres et à tous ceux qui n’ont qu’une fortune bornée. Ceux-ci veulent imiter de loin les manières, les dépenses, le faste des opulents et des grands ; chacun rougit de son indigence et veut au moins la masquer par sa parure. Le pauvre et l’homme peu aisé, entraînés par le torrent, sont nécessités à suivre le ton fastueux que les riches, les grands, les femmes, presque toujours frivoles et vaines, donnent à la société. Chacun se voit obligé de surpasser ses facultés sous peine de ne pouvoir pas approcher des êtres fastueux et peu humains qui seraient faits pour soulager et consoler l’indigent : celui-ci se voit donc forcé de sortir de son état, qui ne serait pas un titre pour être secouru. Ainsi, le malheureux que ses besoins obligent de solliciter les grands est contraint, pour n’être point repoussé par des valets insolents, de faire de la dépense lorsqu’il doit paraître devant ses protecteurs : il craindrait de les blesser s’il leur laissait apercevoir son infortune, il se ruine de peur d’être rebuté et finit très souvent par ne point obtenir les secours dans l’espérance desquels il a dérangé ses affaires.\par
Voilà comment les riches, incapables de se rendre eux-mêmes heureux, loin de procurer du soulagement ou du bien-être aux autres, leur font contracter leurs maladies ! L’épidémie de la cour se répand dans les cités ; bientôt elle la répand dans les campagnes où elle porte le germe de tous les vices, de tous les dérèglements et même de tous les crimes. C’est ainsi que la vanité se propage ; le goût de la parure, si fatale à l’innocence, s’empare de l’esprit du peuple, l’indolence et la paresse remplacent l’amour du travail. Les mœurs se perdent dans l’oisiveté, qui bientôt remplit la société de brigands, de voleurs, de fripons, d’assassins, de prostituées, que la terreur des lois ne peut aucunement réprimer. En décourageant le pauvre, en le dégradant par d’indignes préjugés, un mauvais gouvernement le force à se livrer au crime, qu’on ne peut arrêter sans détruire un grand nombre de victimes. Cette sévérité, néanmoins, ne corrige personne : en avilissant les hommes, on les excite à tout oser ; en les rendant malheureux, on ôte à la mort même ce qu’elle a de terrible. Rendez le pauvre heureux, délivrez-le de l’oppression ; bientôt il travaillera, il aimera la vie, il craindra de la perdre, il sera content de son état.\par
C’est toujours le despotisme qui multiplie les fainéants. C’est l’exemple et l’oppression des riches et des puissants qui corrompent l’innocence du pauvre ; celui-ci, dans sa misère, est forcé de se prêter aux vices de ceux dont il a besoin pour subsister. Avec l’argent, le débauché vient aisément à bout de séduire une fille que le désir de se parer rendra facile à ses vœux ; avec l’argent, il rendra ses parents même complices de son déshonneur. Enfin, l’argent, triomphant de tout, fait que l’homme du peuple devient à tout moment l’instrument des caprices et des crimes de ceux qui veulent l’employer.\par
D’ailleurs, le pauvre, accablé de l’idée de sa propre faiblesse, s’accoutume à regarder l’homme opulent comme un être d’une espèce différente de la sienne et faite pour être exclusivement heureuse. Il l’imite autant qu’il peut ; il devient avide et vain comme lui. Il désire de s’enrichir afin de jouir des avantages qu’il croit attachés aux richesses, et les voies les plus courtes lui paraissent les meilleures\footnote{« Aucun vice de l’âme humaine n’a préparé plus de breuvages empoisonnés, n’a plus souvent animé un fer homicide, que l’implacable désir d’un revenu sans mesure. » Juvénal, {\itshape Satires}, XIV, vers 175 et suivants.}.\par
Voilà comme le pauvre, dégoûté du travail, devient d’abord vicieux, puis criminel ; il ne voit de ressources que dans le vol pour suppléer au travail qui le ferait honnêtement subsister.\par
C’est l’avidité d’un gouvernement tyrannique, ce sont les extorsions de tant d’hommes qui veulent promptement s’enrichir, ce sont les exemples funestes des riches libertins qui peuplent les sociétés d’un si grand nombre de fainéants, de vagabonds, de malfaiteurs que la sévérité des lois ne peut plus les supprimer. La rigueur des impôts, des servitudes, des corvées, dégoûte le cultivateur d’un labeur pénible par lui-même ; il ne travaille plus dès qu’il s’est aperçu que toutes ses peines ne lui produisent rien et ne suffisent pas pour le faire subsister : il aime mieux mendier ou voler que de cultiver une terre ingrate que la tyrannie l’oblige de détester. Rien n’annonce d’une façon plus marquée la négligence et la dureté d’un gouvernement, que la mendicité. Dans un État bien constitué, tout homme qui jouit de l’usage de ses membres devrait être utilement employé et celui que son sort malheureux ou ses infirmités empêchent de travailler a des droits\footnote{« L’honnête pauvreté, dit M. Helvétius, n’a d’autre patrimoine que les trésors de la vertueuse opulence. » Voyez le livre {\itshape De l’Esprit}, discours II, VI, page 51, in-4°.} sur l’humanité de ses semblables, et devrait être soigné par ses concitoyens sans qu’il lui fût permis de chercher à subsister par une vie vagabonde trop souvent vicieuse et criminelle. Pour peu qu’on y réfléchisse, on reconnaîtra que ces hôpitaux somptueux que la pitié mal entendue fait élever au sein des villes ne font souvent à grands frais que redoubler les malheurs du pauvre et les soulagent très peu. Une humanité plus raisonnée fournirait aux malades des secours plus efficaces et plus grands dans leurs propres domiciles et ferait épargner les dépenses énormes d’une administration ruineuse.\par
Une compassion imprudente sert encore à multiplier au sein des nations une classe de malheureux connus sous le nom de {\itshape pauvres honteux.} Rien de plus abusif que la bienfaisance exercée sur des indigents de cette trempe, qui pour l’ordinaire ne sont que des fainéants orgueilleux. Le pauvre ne doit point être honteux de sa misère, faite pour attendrir les cœurs sensibles ou plutôt pour s’attirer les secours fixés par la société. L’homme tombé dans l’indigence doit renoncer à sa vanité primitive pour se conformer à son humble état ; le malheureux cesse d’intéresser dès qu’il est orgueilleux. Enfin, au lieu de se livrer aux chimères d’un orgueil paresseux, tout homme déchu doit chercher dans un travail honnête des ressources contre ses infortunes, de quelque rang qu’il soit tombé.\par
L’humanité, l’équité, l’intérêt général de la société se réunissent pour crier aux souverains de cesser de faire des mendiants, de montrer quelque pitié à ces peuples dont ils troublent cruellement les travaux et la félicité, et que souvent ils réduisent au désespoir. Loin de la saine politique [sont] ces maximes affreuses qui persuadent à tant de princes que les peuples doivent être retenus dans la misère pour être gouvernés avec plus de facilité. L’oppression et la violence ne feront jamais que des esclaves engourdis ou des méchants déterminés qui braveront les supplices pour se venger des injustices qu’on leur fait à tout moment éprouver. C’est aux princes qu’il appartient de consoler efficacement les malheureux et de les ramener à la vertu, que la morale leur prêchera vainement tant que des gouvernements iniques les forceront au crime.\par
Accoutumé dès l’enfance à des occupations très pénibles, l’homme du peuple n’est point malheureux de travailler : il ne l’est que lorsque son travail excessif ne lui fournit plus les moyens de subsister. La pauvreté est, dit-on, la mère de l’industrie ; mais elle est aussi la mère du crime quand cette industrie est découragée, quand elle est gênée, quand elle n’est récompensée que par des impôts accablants. C’est alors que se changeant en fureur, elle devient fatale à la société.\par
Une sage administration doit donc faire en sorte que le pauvre soit occupé ; elle doit, pour le bien de la société, l’encourager au travail nécessaire à la conservation de ses mœurs, à sa propre subsistance, à sa félicité. Il n’est point en politique de vue plus fausse que de favoriser l’oisiveté du peuple.\par
La vraie source de la corruption des Romains partait évidemment de la paresse qu’entretenaient dans le peuple les distributions fréquentes de grains, et les spectacles continuels que lui donnaient des ambitieux qui cherchaient à captiver sa faveur ou à l’endormir dans ses fers. Sous les tyrans qui ravagèrent cet État autrefois si puissant, le peuple dépravé s’embarrassait fort peu des cruautés que ces monstres exerçaient sur les citoyens les plus illustres ; il ne demandait que {\itshape du pain et des spectacles}\phantomsection
\label{footnote71}\footnote{« Panem et circenses ». Juvénal, {\itshape Satires}, X, vers 81. Plutarque dit que Xérès, voulant punir les Babyloniens d’une révolte, les obligera de quitter les armes, de danser, de chanter, de se livrer à la débauche. — « Numa partagea des terres aux pauvres citoyens afin que, tirés de la misère, ils ne fussent plus dans la nécessité de mal faire et pour que, livrés à la vie champêtre, ils s’adoucissent et se cultivassent eux-mêmes en cultivant leurs champs. » Voyez Plutarque dans la {\itshape Vie de Numa}. Les troubles d’Athènes, les folies qui anéantirent cette république frivole et corrompue doivent être attribuées aux extravagances et à la perversité des citoyens oisifs et pauvres nommés {\itshape thêtes}, dont l’esprit était gâté par la fainéantise, les flatteries des orateurs et des spectacles continuels. Les Athéniens en général avaient de l’esprit, de la finesse et du goût, mais très peu de vertu. Ils avaient soin de la punir toutes les fois qu’elle blessait leurs yeux malades et jaloux. Voyez Xénophon, {\itshape Econom.}}. À ce prix Néron lui-même fut un prince adoré de son vivant, regretté après sa mort.\par
Une politique éclairée devrait faire en sorte que le plus grand nombre des citoyens possédât quelque chose en propre : la propriété, attachant l’homme à sa terre, fait qu’il aime son pays, qu’il s’estime lui-même, qu’il craint de perdre les avantages dont il jouit. Il n’est point de patrie pour le malheureux qui n’a rien. Mais dans presque tous les pays les riches et les grands ont tout envahi ; ils se sont emparés de la terre pour ne la cultiver que faiblement ou point du tout.\par
Des parcs démesurés, des jardins sans bornes, des forêts immenses occupent des terrains qui suffiraient pour employer tous les bras des fainéants que l’on rencontre dans les cités et les campagnes. Si les riches renonçaient, en faveur des indigents, aux possessions superflues qu’ils ont entre les mains et dont ils ne savent tirer aucun profit réel, leurs propres revenus seraient considérablement augmentés, la terre serait mieux cultivée, les récoltes seraient plus abondantes et les pauvres, si souvent incommodes à la nation, deviendraient d’utiles citoyens, aussi heureux que leur état le comporte. Gélon menait souvent lui-même les Syracusains aux champs, afin de les exciter à l’agriculture.\par
Ne nous y trompons pas, l’indigence n’exclut point le bonheur\footnote{« Neque divitibus contingunt gaudia solis, nec vixit male, qui natus moriensque fefellit. » Horace, {\itshape Épîtres}, livre I, 17, vers 9-10.} ; elle est capable d’en jouir plus sûrement, par un travail modéré, que l’opulence perpétuellement engourdie ou sans cesse agitée par les besoins continuels de sa folle vanité. La pauvreté occupée a des mœurs ; la pauvreté craint de déplaire ; la pauvreté a des entrailles ; l’indigent est sensible aux maux de ses semblables, auxquels il est lui-même exposé. S’il est privé d’une foule de jouissances, il est, à l’ennui près, au même point que le riche, dont le cœur épuisé ne jouit de rien et ne connaît plus de plaisirs assez piquants. Les désirs du pauvre sont bornés, comme ses besoins ; content de subsister, il n’étend guère ses vues sur l’avenir ; possédant peu, il est exempt des alarmes qui troublent à chaque instant le repos de l’opulence et de la grandeur qu’il croit si dignes d’envie. Ne tenant rien de la fortune, il craint peu ses revers. « C’est, dit Épicure, une chose estimable que la pauvreté, pourvu qu’elle soit tranquille et contente de son sort : on est riche aussitôt que l’on est familiarisé avec la disette : ce n’est pas celui qui a peu qui est pauvre, c’est celui qui ayant beaucoup désire d’avoir encore davantage. » — « Veux-tu être riche, dit-il encore, ne songe point à augmenter ton bien, diminue seulement ton avidité\footnote{« Le chemin le plus court pour s’enrichir, suivant Sénèque, c’est le mépris des richesses. » Voyez Sénèque, {\itshape Épîtres}, 88. Il dit encore ailleurs : « En décourageant le luxe, un roi pourrait tout d’un coup enrichir toute sa cour et soulager tout son peuple. »}. »\par
C’est du sein de la pauvreté que l’on voit communément sortir la science, le génie et les talents. Homère, ce chantre immortel de la Grèce, donna l’immortalité à ces héros fameux dont sans lui les noms seraient ensevelis dans un éternel oubli. Virgile, Horace, Érasme naquirent dans l’obscurité. C’est aux talents divers des hommes dont l’indigence a développé le génie, que les rois, les conquérants, les généraux sont redevables de leur gloire. C’est aux lumières des savants, qui souvent ont vécu dans l’indigence et la détresse, que les sociétés sont redevables des plus grandes découvertes ; c’est à des hommes qu’ils ont l’ingratitude de mépriser, que ces grands si fiers et ces riches si vains doivent chaque jour leurs amusements et leurs plaisirs.\par
De quel droit les riches et les grands dédaigneraient-ils donc le pauvre ? Celui-ci devrait trouver en eux des bienfaiteurs et des appuis contre la violence et les rigueurs du sort ; au lieu de le flétrir par des mépris cruels, qu’ils le regardent comme un citoyen fait pour les intéresser par sa misère même, nécessaire à leur bien-être, souvent bien au-dessus d’eux par des talents qu’ils devraient respecter. Qu’ils se souviennent que dans sa cabane l’indigence ou la médiocrité jouissent quelquefois d’une félicité pure, inconnue de ces mortels qui habitent des palais élevés par le crime\footnote{« Licet sub paupere tecto reges et regum vita præcurrere amicos. » Horace, {\itshape Épîtres}, livre I, 10, vers 32 et 33.}.\par
Que l’indigent, trop souvent envieux, demeure convaincu que l’innocence occupée est infiniment plus heureuse que la grandeur et l’opulence, qui rarement savent mettre des bornes à leurs désirs.\par
Que le pauvre se console donc et se conforme à son humble fortune ; il a droit de prétendre aux secours et aux bienfaits de ses concitoyens plus fortunés dès qu’il travaille utilement pour eux. S’il a besoin des riches et des grands, qu’il leur montre la soumission, la déférence, les respects et les soins qu’ils ont droit d’en attendre en échange de leur assistance et de leur protection.\par
Qu’il s’efforce de gagner leur bienveillance par des voies honnêtes et légitimes, par la douceur et la patience convenables à son état, et non par des bassesses ou des infamies que le vice tyrannique peut exiger.\par
Lorsqu’il trouve dans les grands des protecteurs de sa faiblesse, dans les riches des consolateurs de sa misère, qu’il les paie fidèlement par sa reconnaissance, mais que jamais une lâche crainte ou une indigne complaisance ne lui fassent sacrifier son honneur et sa conscience. L’honneur du pauvre, ainsi que celui du citoyen le plus illustre, consiste à s’attacher fermement à la vertu. La probité, la bonne foi, la droiture, la fidélité à remplir ses devoirs sont des qualités plus honorables que l’opulence ou la grandeur lorsqu’elles en sont dépourvues.\par
Est-il rien de plus noble et de plus respectable que la vertu qui ne se dément pas au sein même de la misère, et qui refuse d’en sortir par des moyens déshonnêtes que les riches et les grands, sans aucuns besoins urgents, ne rougissent pas d’employer ? La pauvreté noble et courageuse d’un Aristide ou d’un Curius ne fut-elle pas plus honorable que l’opulence d’un Crassus ou d’un Trimalcion ?\par
Si la vertu est aimable dans quelque état qu’on la trouve, elle est plus vénérable et plus touchante encore dans l’indigent et le malheureux, que tout semble en dégoûter. La probité se rencontre plus communément dans la médiocrité satisfaite de son sort, que chez la grandeur ambitieuse et toujours inquiète, chez l’opulence toujours avide, chez l’indigence profonde que tout invite au mal.\par
Il serait presque impossible d’entrer dans le détail des devoirs que la morale impose à toutes les classes diverses dans lesquelles les nations sont partagées.\par
On se contentera donc de leur représenter que la probité, l’intégrité, la vertu, non seulement sont propres à faire considérer chacun dans sa sphère, mais encore peuvent être utiles à sa fortune. Le marchand de bonne foi et qui s’est acquis la réputation de ne jamais tromper, ne manquera pas d’être préféré à ses concurrents ; des profits modiques et souvent réitérés, accompagnés d’une conduite économe et réglée mènent plus sûrement à l’opulence que la fraude ; celui que l’on a trompé d’une façon marquée n’est point tenté de se faire tromper une autre fois. L’artisan raisonnable, attentif, consciencieux, sera plus recherché que celui que sa négligence, sa crapule et ses vices rendent inexact et fripon.\par
La morale est la même pour tous les hommes, grands ou petits, nobles ou roturiers, riches ou pauvres. Ses leçons peuvent être entendues par le monarque et le laboureur ; elles leur seront également utiles et nécessaires, et leur pratique procure des droits également fondés à l’estime publique.\par
Un prince dont les injustices produisent la disette dans ses États, est-il un homme plus estimable que le cultivateur qui les vivifie en faisant sortir des moissons de la terre ?\par
Un citoyen laborieux n’est-il pas préférable à tant de grands inutiles à la patrie qu’ils dévorent ? Un négociant honnête, un artisan industrieux sont-ils donc plus méprisables que le seigneur injuste qui refuse de payer ce qu’il leur doit ?\par
Enfin, l’homme de lettres indigent qui consacre ses veilles à l’instruction ou aux amusements de ses concitoyens, ne mérite-t-il pas d’être plus considéré que l’opulent imbécile qui affecte de mépriser les talents ?\par
Que l’homme pauvre qui vit de son labeur et de son industrie cesse d’être méprisé par ces hommes altiers qui le jugent d’une autre espèce que la leur.\par
Que le citoyen obscur ne gémisse plus de son sort, qu’il ne se croie plus malheureux, qu’il ne se méprise point lorsqu’il remplit honnêtement sa tâche dans la société. Content de son état, qu’il ne porte point envie aux courtisans inquiets, aux grands rongés de désirs et troublés par des alarmes continuelles, aux riches que rien ne peut satisfaire. La médiocrité fait que, placé à l’écart, on jouit du mouvement de ce monde sans en éprouver les embarras.\par
Que le cultivateur — si respectable et si peu respecté par les insensés qu’il nourrit, qu’il enrichit, qu’il vêtit — se félicite d’ignorer cette foule de besoins, de frivolités et de peines dont les favoris de la fortune sont journellement tourmentés.\par
Que l’habitant des champs, dans sa paisible chaumière, sente le bonheur d’être exempt des soucis qui voltigent dans les villes sous les lambris dorés.\par
Que sur l’humble grabat où profondément il repose, il ne rêve pas au duvet sur lequel le crime agité cherche en vain le sommeil.\par
Qu’il s’applaudisse de la santé, de la vigueur que lui procurent des repas frugaux et simples, en comparant ses forces avec la faiblesse et les infirmités de ces intempérants, dont les mets les plus piquants ne réveillent plus l’appétit\footnote{Virgile a bien décrit le bonheur du cultivateur dans ces vers : « Interea dulces pendent circum oscula nati : casta pudicitiam servat domus : ubera racca ; lactea demittunt, etc. » Voyez Virgile, {\itshape Géorgiques}, livre III, vers 523.}.\par
Lorsqu’en rentrant dans sa cabane, après le coucher du soleil, il trouve le souper préparé par sa laborieuse ménagère, accueilli, caressé par des enfants charmés de son retour, ne doit-il pas préférer son sort à celui de tant de riches obligés de fuir leur propre maison, où ils ne trouvent souvent que des femmes de mauvaise humeur et des enfants rebelles ?\par
Que le laboureur apprenne donc à se plaire dans son état ; qu’il sache que le nourricier de son pays est un homme plus libre, plus heureux, plus digne d’estime que le grand avili, que le guerrier féroce, que le courtisan servile, que le traitant affamé qui désolent la patrie sans pouvoir se rendre eux-mêmes heureux par tout le mal qu’ils font à leurs concitoyens.\par
Il existe donc une félicité pour ces êtres que l’opulence et la grandeur regardent comme les rebuts de la nature humaine et que pourtant ils s’empressent si peu de soulager. Il existe pour les indigents une morale capable d’être saisie par les esprits les plus simples, encore bien mieux que par les esprits exaltés que l’on ne peut convaincre ou que par ces cœurs endurcis que rien ne peut amollir. Il est bien plus facile de faire sentir les avantages de l’équité à celui que sa faiblesse expose à l’oppression, qu’à des princes, des nobles, des riches qui font consister leur bien-être et leur gloire dans le pouvoir d’opprimer. Il est plus aisé de faire naître les sentiments de la compassion, de l’humanité, dans celui qui souffre souvent lui-même, que dans ces hommes que leur état semble garantir des misères de la vie.\par
Enfin, l’on a moins de peine à contenir les passions timides de l’indigent, que ses malheurs n’ont pas encore conduit au crime, que les passions indomptables des tyrans qui croient n’avoir rien à craindre sur la terre. L’ignorance heureuse dans laquelle le pauvre vit de mille objets divers qui tourmentent l’esprit du riche, l’exempte d’une infinité de besoins et de désirs ; accoutumé aux privations, il s’abstient des choses nuisibles que tant de gens ne peuvent se refuser sans douleur.\par
Ainsi, les moralistes qui d’ordinaire se proposent uniquement l’instruction des classes les plus florissantes de la société, ne devraient pas dédaigner celle des êtres les moins favorisés par le sort ; en proportionnant les leçons de la morale à l’état et à la capacité du pauvre, le sage mériterait autant de gloire et pourrait recueillir plus de fruits, qu’en annonçant aux puissants de la terre des vérités stériles ou déplaisantes. Mais on regarde communément le peuple comme un vil troupeau peu fait pour raisonner ou pour s’instruire, et qui doit être trompé afin de pouvoir être impunément opprimé.
\subsection[{Chapitre X. Devoirs des Savants, des Gens de lettres, des Artistes}]{Chapitre X. Devoirs des Savants, des Gens de lettres, des Artistes}
\noindent De tout temps et dans tous les pays les talents de l’esprit ont mérité à ceux qui les possédaient l’estime et la considération de leurs concitoyens et leur ont fait assigner un rang honorable et distingué. Bien plus, dans l’origine des nations les hommes les plus éclairés, les plus expérimentés, les plus instruits, ont acquis tant de crédit ou d’ascendant sur les peuples que ceux-ci reçurent avec reconnaissance les lois qu’ils leur dictèrent : ils les regardèrent comme des oracles, comme des êtres surnaturels. Les {\itshape prêtres} en Égypte, les {\itshape Chaldéens} en Assyrie, les {\itshape mages} en Perse, les {\itshape brahmanes} dans l’Hindoustan, les {\itshape philosophes} chez les Grecs, furent des personnages que leurs lumières firent respecter également des souverains et des peuples, auxquels ils se rendirent utiles par leurs connaissances, leurs découvertes, leur science, fruits de leurs recherches et de leurs méditations. L’Histoire nous les montre comme les inventeurs des mythologies, des religions, des cultes et des législations qui s’établirent chez la plupart des nations de la terre. Les premiers savants sont souvent devenus les premiers souverains. « Ceux, dit le grand auteur de {\itshape L’Esprit des Lois}, qui avaient inventé des arts, fait la guerre pour le peuple, assemblé des hommes dispersés, ou qui leur avaient donné des terres, obtenaient le royaume pour eux et le transmettaient à leurs descendants. Ils étaient rois, prêtres et juges\footnote{Voyez {\itshape L’Esprit des Lois}, livre I.}. »\par
Ainsi, la considération publique pour ces hommes divins et rares ne fut point stérile. Les prêtres, jouissant de la confiance des peuples, furent richement dotés par la reconnaissance nationale ; ils eurent des immunités et des privilèges qui les mirent à portée de vaquer tranquillement à leurs méditations, à leurs fonctions respectées, aux recherches dont la société pouvait tirer quelque fruit. En conséquence, ces personnages révérés, livrés à la contemplation et à l’expérience, se trouvèrent à portée de faire des découvertes utiles ou curieuses, et les peuples les prirent pour des êtres d’un ordre supérieur qui commerçaient avec le Ciel. Les nations furent redevables à ces premiers savants de la théologie, de l’astronomie, de la géométrie, de la médecine, de la physique et d’un grand nombre d’arts capables de contribuer soit aux travaux, soit aux agréments de la vie.\par
Quelque informes que fussent les premières notions de ces spéculateurs, elles parurent sublimes à des sauvages dépourvus d’expérience, et pour les leur faire encore plus respecter, on les enveloppa d’allégories, d’énigmes et de mystères ; intelligibles pour les seuls prêtres, ils servirent à perpétuer leur ascendant sur les peuples.\par
C’est ainsi que la science, les talents de l’esprit, l’industrie et la ruse élevèrent les savants au-dessus des autres. C’est ainsi que les prêtres, qui possédaient exclusivement les connaissances intéressantes pour les nations, furent regardés comme leurs guides ; ils passèrent pour les interprètes des dieux, devant lesquels les princes et les peuples demeurèrent prosternés. D’où l’on voit que l’utilité sociale fut la source primitive de la vénération que les hommes ont marquée dans tous les siècles au sacerdoce, ainsi que des honneurs, des richesses, des privilèges par lesquels ils l’ont amplement récompensé.\par
Telle est la véritable origine des sciences et des arts qui, de siècle en siècle, se sont plus ou moins perfectionnés et que chaque jour peut enrichir de découvertes nouvelles. Des peuples ignorants furent curieux, inquiets, superstitieux ; frappés du spectacle des astres, leurs faibles yeux n’y découvrirent que des sujets d’étonnement. Des prêtres observateurs prétendirent avoir le secret d’y lire leurs destinées : cette curiosité fit naître l’astronomie ; celle-ci ne fut au commencement que l’{\itshape astrologie judiciaire}, science trompeuse que les lumières postérieures ont fait justement mépriser par les personnes sensées.\par
Pour l’homme dépourvu d’expérience, tout est miracle ; conséquemment, la médecine, la physique, la chimie, la botanique, etc., dans leur berceau furent des sciences {\itshape magiques} fondées sur le commerce supposé des prêtres avec les dieux. L’ignorance ayant fait naître le goût du merveilleux, celui-ci fit éclore à son tour la poésie, qui l’orna de ses charmes, qui contribua plus que toute autre chose à enflammer l’imagination des hommes pour les objets qu’on voulut leur faire admirer et respecter, enfin qui grava profondément dans les esprits les notions, les histoires, les fables dont on voulut les occuper.\par
La morale de ces premiers docteurs des peuples fut encore une science ténébreuse. Faute de connaître suffisamment la nature de l’homme et les motifs les plus capables de l’exciter à la vertu et de le détourner du mal, on ne lui présenta que des motifs surnaturels, des idées vagues de ses devoirs ; au lieu de les établir sur ses rapports avec les autres hommes, on les fonda sur ses rapports avec des puissances cachées, par qui l’on supposait le monde gouverné et dont on pouvait s’attirer la bienveillance ou la colère. On imagina de plus pour les peuples, des pratiques et des cérémonies par lesquelles on prétendit que l’on pouvait rendre ces puissances favorables ou désarmer leur fureur. Ce n’est pas dans un monde invisible et inconnu qu’il faut aller puiser les devoirs de l’homme sur la terre qu’il habite : c’est dans les besoins de sa nature, c’est dans son propre cœur que l’on doit les puiser. Ce n’est pas dans la faveur ou la colère des puissances invisibles qu’il faut chercher des motifs pour inviter l’homme au bien ou le détourner du mal, c’est dans l’affection et la haine de ses semblables qu’il a toujours devant les yeux. Des cérémonies et des rites ne purifient point le cœur de l’homme : ils ne font le plus souvent qu’endormir sa conscience.\par
Mais on se crut obligé de conduire des peuples grossiers et sauvages par l’enthousiasme, soit parce qu’on voulut les tromper, soit parce qu’on les regarda comme incapables d’être conduits par la raison. Conséquemment, la science des mœurs et la politique, chez les premiers savants ou prêtres, fut étayée par des fables. On a lieu de soupçonner en effet que les mythologies religieuses que l’on voit établies dans les contrées diverses de notre globe, ne sont que la science primitive et grossière de la Nature et de l’homme, ornée par la poésie, consacrée par la religion, enveloppée de mystères afin de la rendre vénérable aux yeux des peuples, toujours bien plus avides du merveilleux que de principes simples et raisonnés. On voulut en tout temps tromper, étonner, aveugler les hommes pour les engager à remplir leurs devoirs. Une doctrine simple et raisonnable n’était point encore trouvée ; d’ailleurs, elle n’eût pas été conforme aux vues politiques des premiers instituteurs des nations : ceux-ci traitèrent leurs disciples comme des enfants qu’il faut séduire par des contes, des récits étonnants, des prodiges.\par
La clarté et la simplicité sont les derniers efforts de la science et ne conviennent aux hommes que dans leur maturité. « Les hommes, dit Tacite, sont toujours plus portés à croire ce qu’ils n’entendent point ; ils trouvent plus de charmes dans les choses obscures que dans celles qui sont claires et faciles à comprendre. » Euripide avait dit avant lui « qu’il y a dans les ténèbres une sorte de majesté ». Lucrèce disait aussi que « la stupidité n’admire que les opinions cachées sous des termes mystérieux\footnote{« Omnia stolidi magis admirantur amantque, Inversis quæ sub verbis latitantia cernunt. » Voyez Lucrèce, livre I, vers 642.} ».\par
Ainsi, les premières connaissances qui furent données aux nations sortirent communément des nuages de l’imposture. Par une fatalité trop ordinaire, les hommes moins ignorants que les autres sont tentés d’en faire des dupes d’abord, et par la suite des esclaves. C’est sur cette politique peu sincère qu’est sans doute fondé l’esprit mystérieux qu’on voit régner dans l’Antiquité ; cet esprit, pendant un grand nombre de siècles, infecta les écrits des philosophes les plus célèbres, qui, par état, semblaient faits pour éclairer le genre humain en lui montrant la vérité si nécessaire à son bonheur.\par
En conséquence de ces principes, les docteurs des nations firent descendre leurs préceptes du ciel ; c’est ainsi que {\itshape Brahma} présenta aux habitants de l’Hindoustan une doctrine, des lois et des pratiques qu’il dit avoir reçu du maître invisible du monde. C’est ainsi qu’Osms, après avoir reçu du ciel l’art de l’agriculture, devint le législateur, le souverain et même le dieu tutélaire de l’Égypte ; c’est ainsi que {\itshape Zoroastre}, au nom d’Oromase, régla le culte, les mœurs et les devoirs des habitants de la Perse.\par
D’après les mêmes idées, {\itshape Orphée} instruisit les Grecs et fonda les mystères d’Éleusis ; {\itshape Numa} donna ses lois aux habitants de Rome ; {\itshape Mahomet} aux Arabes, etc.\par
Tous ces législateurs, trouvant dans les peuples grossiers une passion forte pour le merveilleux, un grand respect pour les énigmes et les mystères, en profitèrent habilement pour les soumettre à leur empire\phantomsection
\label{footnote72}\footnote{« Le vrai champ et sujet de l’imposture, dit Montaigne, sont les choses inconnues ; d’autant qu’en premier lieu l’étrangeté même donne crédit ; et puis n’étant point sujettes à nos discours ordinaires, elles nous ôtent les moyens de les combattre. » Voyez livre I, ch. 31. — César avait dit avant lui que « par un vice commun de la nature nous avons plus de confiance dans les choses invisibles cachées, inconnues, et nous en sommes plus troublés ». « Communi fit vitio naturæ, ut invisis, latirantibus atque incognitis rebus magis considamus, vehementiusque exerreamur. » {\itshape De Bello Civili}, livre II, section 4.}. Un langage obscur irrite la curiosité, des notions merveilleuses étonnent les esprits et mettent les cerveaux en travail. Semblable au tonnerre, une science entourée de nuages fait considérer ceux qui se vantent de la posséder ; mais si elle leur est avantageuse, elle est inutile ou nuisible aux progrès de l’esprit humain, qu’elle amuse sans profit et qu’elle retient dans une longue enfance.\par
C’est évidemment de l’Égypte et de la Phénicie que les Grecs reçurent leur culte, leurs premières notions sur la Nature et sur la morale ; en un mot : leur {\itshape philosophie.} Pythagore, comme on l’a dit ailleurs, alla chercher sa science mystique dans les écoles des prêtres égyptiens et des savants de Chaldée. Platon, après lui, puisa dans la même source la doctrine ténébreuse et sublime qu’il répandit dans sa patrie\phantomsection
\label{footnote73}\footnote{Platon même paraît avoir enchéri sur le ton mystérieux des prêtres égyptiens ; il semble reprocher à ceux-ci {\itshape d’avoir fait un tort irréparable aux sciences en inventant l’écriture}. Cependant l’écriture est l’unique moyen de répandre et de conserver les connaissances humaines. Les sauvages demeurent dans l’enfance parce que les découvertes, les expériences, les réflexions de leurs ancêtres, faute d’écriture, sont toujours perdues pour eux. Chaque race, dépourvue des secours de cet art est forcée de recommencer, sur nouveaux frais. Il faut parler clairement pour être utile aux hommes. Le savant mystérieux et caché n’est propre qu’à embrouiller les esprits et retarder leurs progrès ; un tel homme n’est pas un bienfaiteur du genre humain. La vérité donne tout leur lustre aux sciences : celui qui méprise la vérité et lui préfère une vaine éloquence n’est qu’un vain charlatan. Un Grec, parlant de Pythagore, a dit : « Pythagore l’enchanteur, qui n’aime que la vaine gloire et affecte un langage grave et mystérieux pour attirer les hommes dans ses filets… » Voyez Plutarque, {\itshape Vie de Numa}.}. La Grèce peu à peu se remplit de philosophes et de penseurs qui s’attirèrent de la considération par leurs systèmes et leurs découvertes, adoptées ensuite par les Romains ; ces conquérants les communiquèrent aux différents peuples soumis à leur empire. C’est de leurs mains que les modernes ont reçu les connaissances dont ils jouissent et qu’ils doivent chercher à perfectionner, à simplifier, à rendre plus claires et plus utiles.\par
Ainsi, les sciences et les talents de l’esprit furent de tout temps en honneur parmi les peuples. Cet ascendant de la science s’est montré dans toutes les contrées de la terre. Depuis un grand nombre de siècles {\itshape Confucius}, par les préceptes moraux qu’on lui attribue, gouverne encore la Chine ; sa mémoire y est toujours chère. Ses maximes y sont respectées comme des oracles par les féroces Tartares mêmes, qui plus d’une fois ont subjugué ce vaste empire ; pour parvenir aux places il faut avoir étudié les livres de ce sage, à qui l’on rend un culte et que l’on a surnommé {\itshape le roi des lettrés}. Ces hommages rendus par une nation à la mémoire de cet homme célèbre prouvent au moins que les Chinois, tout corrompus qu’ils sont, se croient obligés de montrer à l’extérieur de la vénération pour les talents et la vertu, lors même qu’ils en sont totalement dépourvus. Nonobstant leur respect pour les écrits attribués à Confucius, les chinois sont misérables et sans mœurs parce qu’ils vivent sous un gouvernement despotique et barbare, fait pour mettre des obstacles invincibles aux progrès de la vraie science et pour rendre inutiles les leçons de morale la plus sensée\footnote{Nous observons en passant que la morale de ce sage fameux, telle qu’elle nous a été transmise par quelques missionnaires européens, n’est pas faite pour nous donner une haute idée des lumières des Chinois. Les ouvrages attribués à {\itshape Confucius} et à son disciple {\itshape Mentzius} ne renferment que des maximes communes et triviales qui ne peuvent aucunement être comparées à celles des Grecs et des Romains. D’ailleurs ces écrits, si vantés par quelques modernes, sont favorables au despotisme, c’est-à-dire, au plus injuste des gouvernements, à la tyrannie paternelle qu’ils confondent avec une autorité raisonnable, à la polygamie et à la tyrannie exercée sur les femmes ; enfin ils n’ont pour objet que des esclaves. D’où l’on voit que ce sage d’Orient ou ceux qui ont adopté ses maximes n’ont point eu les première notions de la vraie morale et du droit naturel. On frémit quand on pense que la loi permet en Chine aux pères d’exposer leurs enfants qui souvent, dans les rues de Pékin, sont écrasés sous des voitures ou dévorés par les bêtes.}.\par
Si pendant plusieurs siècles la science fut méprisée en Europe et parut languir dans l’oubli, cet état d’abjection doit être attribué à la confusion et aux troubles produits par les révolutions et les guerres continuelles dont les nations furent agitées. Alors l’esprit humain retomba dans l’ignorance primitive ; des guerriers stupides et forcenés ne connurent d’autre mérite que de savoir se battre. Les peuples, totalement privés de lumières et de raison, végétèrent dans un abrutissement funeste, accompagné de tous les maux qu’entraînent l’erreur et les préjugés. Les hommes engourdis croupirent dans l’infortune parce qu’ils manquèrent des secours, des consolations, des plaisirs, des commodités que les sciences et les arts peuvent seuls procurer. Des soldats farouches ne connurent aucunement les avantages inestimables que les talents, le génie, l’industrie pouvaient fournir à la vie sociale. Les nations furent aveugles et sans mœurs parce qu’il n’y a que la raison, fruit de l’expérience ou de la science, qui puisse rendre les hommes plus humains ou plus sociables.\par
Enfin, les ténèbres de cette longue nuit commencèrent à se dissiper ; des souverains amis des lettres, des sciences et des arts, leur tendirent une main secourable ; l’esprit humain, sorti de sa longue léthargie, reprit son activité ; les talents furent considérés, honorés, récompensés ; dès lors, ils excitèrent dans les âmes une fermentation vive, une émulation favorable. Les mœurs s’adoucirent, la réflexion prit la place de l’impétuosité et de l’étourderie ; l’étude devint l’occupation de beaucoup de citoyens enflammés par le désir de la réputation, de la gloire et même de la fortune, à laquelle on vit que les talents pouvaient conduire. Les lettres devinrent au moins un amusement agréable pour un grand nombre de personnes, qui sans elles languiraient dans une oisiveté fatigante. Aristote disait « que les savants avaient sur les ignorants les mêmes avantages que les vivants sur les morts, que la science est un ornement dans la prospérité et un refuge dans l’adversité ». La science, suivant Diogène, sert de frein à la jeunesse, de soulagement aux vieillards, de richesse aux pauvres, et d’ornement aux riches. « Les sciences et les lettres, dit Cicéron\phantomsection
\label{footnote74}\footnote{Cicéron, {\itshape Orat. pro. Archia Poëta}, chap. 7, §. 16.}, sont l’aliment de la jeunesse et l’amusement de la vieillesse ; elles nous donnent de l’éclat dans la prospérité et sont une ressource, une consolation dans l’adversité ; elles font les délices du cabinet, sans causer ailleurs aucun embarras : la nuit elles nous tiennent compagnie ; aux champs et dans nos voyages elles nous suivent, etc. »\par
Tel est le jugement que portait de l’étude un homme d’État à qui fut confié le gouvernement du plus puissant empire du monde : il devrait faire rougir tant de grands et de nobles qui affectent de mépriser la science, la regardent comme inutile et dangereuse et semblent se glorifier d’une ignorance qui fut toujours la source de l’erreur et du vice. La science n’est en droit de déplaire qu’aux imposteurs et aux tyrans\phantomsection
\label{footnote75}\footnote{Caligula voulait détruire les ouvrages d’Homère. Un empereur de la Chine fit brûler tous les livres de ses États. Les mauvais princes et sont toujours déclarés les ennemis de la science. Valencien et Licinius la nommaient un poison, une peste dans l’État. L’imposteur Mahomet proscrivit prudemment toute science, dans la crainte qu’elle vînt détruire ses impostures. « Le grand Turc, dit La Boëtie, s’est bien avisé de cela, que les livres et la doctrine donnent plus que toute autre chose aux hommes le sens de reconnaître et de haïr la tyrannie. » Voyez {\itshape Discours sur la Servitude volontaire}, imprimé à la suite des {\itshape Essais} de Montaigne de l’édition donnée par Coste.}.\par
Serait-ce donc pour mériter les suffrages des hommes de cette trempe, que quelques gens de lettres ont employé leurs talents et leur esprit à déclamer contre l’utilité des sciences ? Mais examinons en peu de mots les raisons sur lesquelles un célèbre détracteur des lettres fonde ses imputations contre elles. « Les sciences, selon Mr. Rousseau de Genève, sont défectueuses dans leur origine, dans leur objet, dans leurs effets.\par
« Dans leur origine, l’astronomie est née de la superstition ; l’éloquence de l’ambition, de la haine, de la flatterie, du mensonge ; la géométrie de l’avarice ; la physique d’une vaine curiosité ; toutes, et la morale même, de l’orgueil humain.\par
« Dans leur objet : point d’Histoire sans tyrans, sans guerres, sans conspirateurs ; point d’arts sans luxe. Point de sciences sans l’oubli des devoirs les plus indispensables. Que de dangers, que de fausses routes rencontrent dans la carrière des sciences ceux qui cherchent sincèrement la vérité ! Son {\itshape critérium} même est incertain.\par
« Dans leurs effets : les sciences sont filles et mères de l’oisiveté ; elles sont inutiles au bonheur ; elles avancent mille paradoxes qui sapent les fondements de la foi et anéantissent la vertu. Elles étouffent le sentiment de notre liberté originelle et introduisent une fausse politesse qui, en éteignant la confiance et l’amitié, ouvre la porte à mille vices : elles produisent le luxe et la folle envie de se distinguer, d’où naissent la dépravation des mœurs, la corruption du goût et la mollesse\footnote{Voyez le discours de Mr. Rousseau couronné par l’Académie de Dijon, sur cette question : {\itshape si le rétablissement des sciences et des arts a contribué à épurer les mœurs}.}. »\par
Pour répondre pied à pied à des accusations si graves, nous dirons que l’astronomie est née d’un désir légitime et raisonnable de connaître les mouvements des corps célestes, que les hommes avaient besoin de les connaître pour régler les travaux les plus nécessaires à la vie, tels que l’agriculture et la navigation ; que l’astrologie, qui n’est point une science réelle, est née de la superstition. L’éloquence est née du besoin de mettre en action les passions, les intérêts des hommes afin de les déterminer à faire ce qui leur est utile ou pour leur persuader la vérité, si nécessaire à leur bien-être ; si des imposteurs en ont fait usage pour tromper, c’est que les choses les plus utiles deviennent très nuisibles par l’abus qu’on en fait. La physique est l’effet d’une curiosité louable qui porte l’homme à chercher dans la Nature ce qui peut contribuer à son propre bonheur, connaissance sans laquelle il ne pourrait ni se conserver ni vivre. La géométrie n’est point le fruit de l’avarice mais du besoin de distinguer les possessions des hommes, distinction sans laquelle tout tomberait dans la confusion. La morale n’est point due à l’orgueil mais au besoin indispensable de savoir comment doivent se comporter des êtres qui vivent en société.\par
L’Histoire nous apprend des faits utiles à notre instruction ; elle nous montre des tyrans, des révolutions, des guerres, des conspirations, pour nous en faire sentir l’horreur et nous engager à chercher les moyens de nous garantir des maux dont le genre humain fut si souvent affligé. Les arts, il est vrai, fleurissent au sein du luxe ; mais ces arts, qui n’ont pas pour objet l’utilité réelle, ne doivent pas être confondus avec ceux dont la société ne saurait se passer. La science ne produit pas l’oubli de nos devoirs ; au contraire, la vraie science est faite pour nous y ramener : elle nous fait remplir un devoir dès qu’elle nous rend utiles à nos semblables par les vérités ou les expériences qu’elle nous met à portée de leur communiquer. L’on ne peut faire un crime aux sciences des dangers auxquels s’exposent ceux qui cherchent la vérité ; ce crime doit être imputé à la méchanceté de ceux qui rendent la vérité dangereuse à ses apôtres ou qui s’efforcent d’en priver le genre humain. Les fausses routes que l’on rencontre dans la carrière des sciences ne prouvent aucunement que les sciences soient mauvaises ou fausses ; elles prouvent que les hommes sont sujets à s’égarer quelquefois très longtemps avant de rencontrer la vérité, et à se tromper toutes les fois qu’ils ne partent pas d’après des expériences sûres : ces fausses routes font voir que le savant doit se défier de lui-même et que c’est à force de chutes que l’on apprend à marcher. Le {\itshape critérium} de la vérité est certain quand on ne s’occupe que des objets que l’on peut soumettre à l’expérience et quand on rejette ceux qui n’ont que l’imagination pour base.\par
Les sciences vraiment utiles ne sont pas les filles et les mères de l’oisiveté : elles sont filles des vrais besoins de l’homme, et le poussent à chercher ce qui peut contribuer à sa conservation et rendre son existence heureuse ; elles ne sont inutiles au bonheur que lorsqu’elles s’occupent de spéculations vagues et d’objets inaccessibles à l’expérience. Les paradoxes qui anéantissent la vertu ne peuvent être que des effets d’un délire, que l’on ne peut pas plus appeler une science que l’ivresse ou le transport du cerveau. Les sciences n’étouffent pas le sentiment de notre liberté naturelle ; au contraire, toute science véritable nous y ramène : elle nous la fait chérir et désirer à la vue des malheurs dont l’esclavage est toujours accompagné. Les sciences supposent de la réflexion, et la réflexion nous rend polis parce qu’elle nous rend sociables en nous apprenant les égards que se doivent des êtres réunis en société.\par
La politesse n’exclut nullement l’amitié sincère et la confiance, que la science des mœurs surtout doit établir. Les sciences n’ouvrent point la porte à mille vices\phantomsection
\label{footnote76}\footnote{Épicure disait au contraire que « la philosophie est la source de toutes les vertus, qui nous enseignent que la vie est sans agrément, si la prudence, l’honnêteté et la justice ne dirigent tous nos mouvements. Mais en suivant toujours la route qu’elles nous tracent, nos jours s’écoulent avec satisfaction dont le bonheur est inséparable ; car ces vertus sont le propre d’une vie pleine de félicité et d’agrément, qui ne peut jamais être sans leur excellente pratique. » Diogène Laërce, {\itshape Vies et Doctrines des Philosophes illustres}, livre X, section 132.} ; en occupant l’homme d’une façon utile ou agréable, elles le détournent de mille désordres qui sont les ressources ordinaires de l’ignorance et de la paresse. Les sciences ne produisent aucunement le luxe, elles le décrient, elles exhortent les hommes à s’en garantir, elles empêchent ceux qui savent s’en occuper de songer aux vanités dont les ignorants et les désœuvrés sont perpétuellement tourmentés.\par
L’envie de se distinguer n’est point une folle envie, c’est un sentiment naturel, très louable quand on se distingue par une conduite honnête, par des mœurs sages, par des talents avantageux au public.\par
Une folle envie de se distinguer, c’est celle qui cherche à s’illustrer en combattant de mauvaise foi les notions les plus évidentes et les plus raisonnables qui concourent à nous convaincre que l’ignorance est un mal et que la science est un bien sous quelque point de vue qu’on veuille l’envisager.\par
Toute science, comme on l’a dit ailleurs, est une suite d’expériences ou de faits ; les expériences mal faites constituent la fausse science ou l’erreur, dont les suites sont très funestes à l’homme. Les expériences constantes réitérées, réfléchies, constituent la vraie science et nous font connaître la vérité, toujours utile et nécessaire aux êtres de notre espèce. Prétendre que la science est inutile, c’est dire que les hommes n’ont besoin, pour se conduire en ce monde, ni d’expérience, ni de raison, ni de vérité, ce qui n’est pas remettre l’homme dans l’état sauvage ou dans l’état de nature mais le placer au-dessous des bêtes, qui ont du moins une dose d’expérience, de raison, de science et de vérité suffisantes pour se conserver et pour contenter leurs besoins.\par
Les besoins de l’homme, étant plus variés que ceux des autres animaux, demandent plus d’expériences, des connaissances plus étendues, une raison plus exercée, un plus grand nombre de vérités sans lesquelles il serait plus malheureux que les bêtes. L’homme ignorant et stupide n’a pas même les ressources que ce qu’on appelle {\itshape l’instinct} fournit à des castors.\par
Ce n’est que par une raison plus cultivée ou par des connaissances plus vastes que quelques hommes s’élèvent au-dessus de leurs semblables. Quelle différence prodigieuse la science et les talents de l’esprit ne mettent-ils pas entre les êtres de l’espèce humaine ! Les peuples les plus éclairés sont les plus florissants. L’Europe se trouve en état de faire la loi aux autres parties du monde par la supériorité des forces que la science lui donne ; parmi les nations qu’elle renferme, les plus puissantes, les plus actives, les plus industrieuses sont celles qui jouissent de plus de lumières. Un pays plongé dans l’ignorance est un royaume de ténèbres dont les habitants sont perpétuellement endormis.\par
L’homme naît en société et continue d’y vivre parce que la société lui est agréable et nécessaire ; il n’est aucunement destiné par sa nature à vivre dans les forêts, privé des secours de ses semblables. La vie sociale le forme, le modifie, le façonne, parce qu’il y jouit de ses propres expériences et de celles des autres ; ces expériences développent sa raison ou lui apprennent à distinguer le bien du mal. Déclamer contre la raison humaine et contre la science, c’est assurer que l’homme n’a nullement besoin de distinguer ce qui peut le conserver de ce qui peut le détruire, ce qui peut lui plaire de ce qui peut lui déplaire. L’homme {\itshape naturel} fabriqué par l’éloquent sophiste à qui l’on répond ici, serait un malheureux enfant qui n’aurait aucunes ressources ni pour se procurer le bien-être, ni pour éviter les maux dont il serait à tout moment menacé. Est-ce donc dans l’ignorance et la stupidité qu’il faut chercher des remèdes à la corruption, toujours enfantée par l’inexpérience et le délire\phantomsection
\label{footnote77}\footnote{Dacier (dans sa comparaison de Pyrrhus et de Marius) dit avec raison : « On ne hait point impunément les muses ; Marius fut comme les terres fortes qui, demeurant sans culture, produisent plus de mauvaises herbes que de bonnes. » Voyez sa traduction des {\itshape Vies des Hommes illustres} de Plutarque, tome 4, page 205, édition Amsterdam, 1734.} !\par
Une tradition très peu sensée fait croire à presque tous les peuples que leurs ancêtres grossiers ont dû jouir dans des temps éloignés d’un bien-être inconnu de leurs descendants. De là la fable de {\itshape l’âge d’or}, que l’on place toujours près du berceau des nations, c’est-à-dire à des époques où les hommes privés de toutes connaissances et ressources, ignorant même l’agriculture, vivaient comme les bêtes et se nourrissaient de racines et de glands. Il est bien difficile de croire que ces hommes, si dépourvus des moyens de satisfaire leurs besoins naturels, aient été ou plus sages ou plus heureux que nous. S’ils n’avaient point de luxe, ils manquaient souvent de tout ; s’ils n’avaient point de procès, ils se battaient et s’égorgeaient sans cesse pour la moindre dispute.\par
« L’ignorance du mieux est, suivant un Ancien, la cause de toutes les fautes. » La vie sociale, en éclairant l’homme, lui fournit des secours et lui découvre les motifs qui l’engagent à contenir ses passions ; plus il a de lumières, et plus il connaît ses véritables intérêts, toujours liés à ceux de ses semblables. Il n’est méchant que parce qu’il ignore ou parce qu’il perd de vue la façon dont il doit se conduire avec ses associés. Les princes, les grands, les riches ne font tant de mal sur la terre que parce qu’ils ne sont point éclairés. Quelques nations sont malheureuses et sans mœurs non parce qu’elles sont trop savantes, mais parce que ceux qui devraient les rendre sages ne veulent pas qu’on les éclaire, afin de pouvoir les conduire à la ruine.\par
Montaigne, conforme en cela aux idées des détracteurs de la science, dit « qu’il faut nous abêtir pour nous assagir, et nous éblouir pour nous guider\phantomsection
\label{footnote78}\footnote{Voyez {\itshape Essais}, livre II, chap. 12, p. 268.}\hyperref[bookmark92]{\dotuline{\textsuperscript{78}}} ». Il nous fait remarquer dans l’ancienne Rome la plus grande ignorance et les plus hautes vertus : mais quelles pouvaient être les vertus d’un peuple injuste et barbare dont les cruelles mains se baignaient continuellement dans le sang ? D’un peuple qui, sous prétexte d’amour pour la patrie, se permettait toutes sortes de crimes ! La modération et le désintéressement d’un Curius, la continence d’un Scipion et quelques vertus particulières peuvent-elles contrebalancer les horreurs dont une république de brigands affligea l’univers et les forfaits qui par la suite la détruisirent elle-même ? On nous dira que Rome plus éclairée n’en devint que plus méchante ; mais nous répondrons que les armes faibles de la philosophie romaine ne purent jamais combattre avec succès les vices introduits par le luxe, ni faire disparaître la sombre férocité qui toujours caractérisa le peuple romain : cette philosophie, souvent farouche et rebutante, n’était guère propre à lui donner des mœurs plus douces, surtout sous l’empire des tyrans, qui achevèrent de tout détruire\footnote{Il est évident que la philosophie enthousiaste et fanatique des stoïciens était celle qui convenait le mieux à des hommes qui vivaient sous des Tibère, des Néron, des Domitien, etc. Il fallait y apprendre à se passer de tout et à tout souffrir (abstine et sustine). Il fallait, à force d’imagination, se raidir contre les dangers dont on était entouré. Il fallait s’isoler et se concentrer en soi-même. Telle est la philosophie qui convient sous tout mauvais gouvernement.}.\par
Ce n’est pas de l’ignorance ou de la rupture de l’association humaine que nous devons attendre la félicité des peuples ; c’est, au contraire, de l’accroissement de leurs lumières, de leur raison plus cultivée, de leur expérience, de leur science que nous pouvons attendre le perfectionnement de la vie sociale et la réforme de tant d’institutions nuisibles, d’usages insensés, de préjugés puérils et de folles vanités qui s’opposent au bonheur des hommes. Cette réforme désirable ne peut être que l’ouvrage du temps, qui peu à peu guérit les hommes des folies de leur enfance pour les conduire à la maturité ; les efforts redoublés de l’esprit humain sont faits pour combattre les erreurs et pour dissiper les nuages qui ont empêché jusqu’ici les souverains et les peuples de donner une attention sérieuse aux objets les plus intéressants pour eux.\par
Quelques penseurs découragés nous diront peut-être qu’il est inutile de se flatter d’éclairer tout un peuple, et que la philosophie ni les principes de la morale ne sont pas à la portée du vulgaire. Nous répondrons que pour rendre une nation raisonnable il n’est pas besoin que tous les citoyens soient des savants ou de profonds philosophes : il suffit qu’elle soit gouvernée par des gens de bien. « Les peuples, suivant Platon, seront heureux quand ils seront gouvernés par des sages. » Toutes les sciences sont au-dessus de la capacité du vulgaire ; elles lui sont pourtant utiles, et les hommes les plus grossiers font journellement usage des principes et des règles dont la découverte n’est due qu’aux plus grands efforts du génie. Démocrite fut, dit-on, l’inventeur de la voûte ; cependant nous voyons aujourd’hui des voûtes construites suivant les règles par de simples manœuvres. Il faut du génie pour inventer et découvrir ; mais il ne faut que du bon sens pour profiter des découvertes qui ont le plus coûté. Les principes de la sagesse sont difficiles à découvrir mais tout gouvernement bien intentionné peut aisément les appliquer. La science n’est donc pas inutile au peuple même : les sages, les gens de lettres, les savants peuvent être considérés comme des citoyens destinés à fournir les esprits, à faciliter les travaux, à combattre les erreurs. Le génie le plus merveilleux peut s’égarer, sans doute, mais c’est aux lumières réunies de tous les êtres pensants qu’il appartient d’apprécier, de rectifier, de perfectionner les idées que chacun offre au public. Les vérités les plus intéressantes pour la félicité générale sont difficiles à trouver et ne peuvent être que le fruit tardif des recherches des hommes. Tout écrivain doit être clair, sincère, véridique ; c’est au public honnête, impartial, éclairé, qu’il appartient de juger ses idées ; des auteurs frivoles confondent communément un vain bruit avec la gloire et n’obtiennent les suffrages que de ceux qui leur ressemblent. Les hommes qui pensent, les personnes qui ont de la droiture, de la raison, de la vertu, voilà ceux qu’un auteur véridique reconnaît pour des juges compétents. « La philosophie, dit Cicéron, se contente d’un petit nombre de juges, elle récuse les jugements de la multitude, qui lui sont toujours suspects, et à qui elle doit déplaire\footnote{« Philosophiæ paucis est contenta judicibus, multitudinem consulto ipsa fugiens, eique ipsi et suspecta et invisa. »}. »\par
C’est pour les êtres pensants de tous les temps, de toutes les nations, qu’un philosophe doit écrire. Celui qui n’écrit que pour escroquer en passant les suffrages du public, la faveur des grands, les applaudissements de ses contemporains se rend communément l’esclave des opinions régnantes, auxquelles il sacrifie lâchement et sa raison et ses lumières, et l’intérêt du genre humain. « Il faut de l’audace, dit Événus, pour chercher la sagesse » ; il faut de la noblesse, du courage, de la franchise pour l’annoncer aux autres. La vérité seule rend durables les productions de l’esprit ; pour plaire à tous les siècles, il faut une âme exempte de préjugés, dont le règne est variable et de peu de durée. Aristote nous dit que « la plus nécessaire des sciences est de désapprendre le mal ». En un mot, pour éclairer les hommes il faut une âme forte, un cœur droit et pénétré d’amour pour l’humanité ; il faut de la liberté, de la vertu.\par
« Personne, dit un Ancien, ne voit ce que tu sais, mais chacun est à portée de voir ce que tu fais. » L’homme de lettres doit régler son intérieur avant de vouloir donner des préceptes aux autres\phantomsection
\label{footnote79}\footnote{Voyez, dans les {\itshape Caractéristiques} de Mylord Shaftesbury, deux traités, le {\itshape Soliloque} et l’{\itshape Avis à un Auteur}, qui n’ont pour objet que de former le cœur de ceux qui veulent écrire. Diogène comparait les savants dépourvus de mœurs aux instruments de musique qui n’entendent point les airs que l’on y exécute.}. On a très justement comparé le savant dont les mœurs sont déréglées à un aveugle qui tient un flambeau dont il éclaire les autres sans en être lui-même éclairé : sage et savant devraient être toujours des synonymes. Peut-on, en effet, se flatter d’être vraiment savant quand on ignore les devoirs qui nous lient aux êtres de notre espèce ? « La science, disait Thalès, nuit autant à ceux qui ne savent pas s’en servir, qu’elle est utile aux autres. » Il ne suffit pas de connaître ses devoirs si l’on ne prouve par ses actions que l’on en est persuadé. Peu de gens sont en état de juger les talents de l’esprit mais tout le monde est à portée de juger la conduite. Le savant, dans ses écrits, doit se proposer la gloire attachée aux vérités utiles qu’il expose à ses concitoyens ; mais ce n’est pas assez de les instruire : il faut encore leur plaire, afin de rendre plus convaincantes les instructions qu’on leur donne.\par
L’honneur est un ressort essentiel aux gens de lettres. « Les muses, dit Hésiode, sont filles de Jupiter ; elles ne doivent jamais oublier la noblesse de leur origine\footnote{Ce poète dit que Mnémosyne, ou la déesse de la mémoire, qui règne sur les hauteurs d’Éleuthère, c’est-à-dire dont l’empire est noble et libre, eut les muses de son commerce avec Jupiter. Par où il indique que les sciences et les arts ne peuvent naître que dans les pays libres. Voyez {\itshape Théogonie}, vers. 52 et suivants.}. » Que l’homme de lettres se respecte donc lui-même dans ses rivaux. Rien de plus avilissant pour les lettres que ces querelles déshonorantes, ces haines envenimées, ces basses jalousies que l’on voit trop souvent régner entre ceux qui les cultivent. La gloire n’a-t-elle donc pas des faveurs pour tous ses adorateurs ? L’envie n’est-elle pas un aveu formel de faiblesse et d’infériorité ? Que les savants soient émules, mais qu’ils ne soient ni envieux ni jaloux\footnote{« Le sage, dit Épicure, n’est point jaloux de la sagesse d’un autre. » Voyez Diogène Laërce, {\itshape Vies et Doctrines des Philosophes illustres}, livre X, section 121. comme un hypocrite, celui qui ne met point en pratique les préceptes qu’il donne aux autres.} ; qu’ils songent surtout que c’est se dégrader que de descendre dans l’arène pour amuser, par leurs combats, un vulgaire toujours prêt à déprimer des hommes dont il craint la supériorité.\par
Rien ne fait plus de tort aux lettres et aux sciences que l’arrogance et le ton méprisant que prennent quelquefois ceux qui les cultivent. La réflexion doit leur apprendre que le mépris et la hauteur sont insupportables et suffisent pour anéantir les sentiments de gratitude et de bienveillance que les talents les plus rares devraient exciter.\par
L’homme vraiment éclairé doit être juste ; qu’il rende à chacun ce qu’il lui doit, qu’il montre au rang, à la naissance, au pouvoir, les respects et la déférence que la société leur adjuge, qu’il honore les grands sans bassesse, qu’il mérite leur estime par une conduite réservée, qu’il ne fasse sentir à personne sa supériorité, qu’il ait de l’indulgence pour l’ignorant et le faible.\par
L’intolérance et l’orgueil ne peuvent que révolter. Chercher à se faire aimer et craindre de déplaire est un devoir qui oblige également tous les membres de la société. Il n’y a point de gloire à blesser, il n’y a point de bassesse à ménager l’amour-propre de ceux qui sont à portée de faire beaucoup de bien aux nations.\par
Les hommes les plus éclairés devraient le mieux connaître leurs véritables intérêts et, par conséquent, se distinguer par leur sociabilité, leur humanité envers tout le monde et leur union entre eux. La discorde, si commune entre les gens de lettres, n’est propre qu’à rendre méprisables des hommes dont le désir de l’estime, de la réputation, de la gloire, doit être le vrai mobile. Le public, souvent injuste, fait communément un crime, à tout le corps, des fautes ou des écarts de quelques individus ; les vices du philosophe rendent ses leçons suspectes. On est toujours tenté de regarder comme un charlatan,\par
Les talents de l’esprit sont des armes dangereuses entre les mains d’un méchant : il s’en sert pour blesser et les autres et lui-même. Épictète voulait avec raison que la philosophie fût réservée aux gens de bien ; voyant un débauché qui voulait s’y livrer : « À quoi penses-tu ? lui dit-il ; songe à rendre ton vase pur avant d’y rien verser. » Les plus grands talents se déshonorent et se prostituent lorsqu’ils sont possédés par des hommes sans mœurs et sans conduite. Aristote disait que l’avantage qu’il avait tiré de la philosophie était de faire sans être commandé ce que les autres ne font que par la crainte des lois. La conscience du sage est pour lui un frein plus puissant que la terreur. « Les gens de bien, dit Horace, s’abstiennent du mal par l’amour de la vertu\phantomsection
\label{footnote80}\footnote{Horace, {\itshape Épîtres}, livre I, 16, vers 52.} », c’est-à-dire dans la vue d’être contents d’eux-mêmes, de ne pas perdre le droit de s’aimer et d’être aimés des autres.\par
C’est par des mœurs plus honnêtes, plus sociables, plus décentes, que doivent se distinguer ceux qui par état se destinent à l’instruction des autres. L’habitude de penser, de rentrer en soi-même, de peser les conséquences des choses, devrait évidemment rendre les hommes plus vertueux à proportion qu’ils ont plus de lumières. Qu’un fat, qu’un étourdi qui jamais n’a réfléchi se rende incommode ou ridicule par sa vanité et ses impertinences, il ne faut pas s’en étonner ; mais la vanité, les petitesses ne sont-elles pas déplacées dans un homme qui ne doit s’annoncer que par l’élévation et la noblesse de sa façon de penser, et par la décence de ses mœurs ? L’étude doit apprendre à se défier des élans de l’imagination, à résister à ses impulsions fougueuses ; elle doit apprendre à raisonner, elle doit faire naître dans les âmes des sentiments plus délicats, plus nobles, plus distingués que dans les âmes vulgaires. L’homme d’esprit, doué d’un tact plus fin que les autres, doit sentir avec plus de promptitude ses devoirs envers ses semblables, ou ce qu’il faut faire pour mériter leur estime et leur affection. Le vrai savant devrait être le plus sociable des hommes.\par
Ne croyons pas néanmoins que cette sociabilité doive entraîner l’homme de lettres à chaque instant dans le tourbillon du monde, qui ne serait propre qu’à le dégoûter du travail et de la méditation. Sans être ni pédant ni farouche, l’homme dont le métier est de penser doit avoir de la dignité, de la réserve dans ses mœurs, et préférer le silence de la retraite aux assemblées bruyantes et dissipées. Le spectacle du monde et son mouvement varié ne doit être pour lui qu’un délassement passager, et non une occupation suivie ; il peut le rendre instructif s’il y puise des idées, des faits, des observations propres à fournir de la pâture à ses réflexions. Il est utile et nécessaire au philosophe, au moraliste, à l’homme de lettres, de voir les hommes de près, de les bien connaître, afin de donner à leurs ouvrages l’urbanité, à leurs peintures la ressemblance, à leurs préceptes les agréments capables de les faire réussir. Tout écrivain qui ne connaît pas le monde n’en peut parler pertinemment et n’en présente que des portraits ridicules et chimériques.\par
Mais il ne faut à l’homme de génie que des coups d’œil rapides pour saisir les objets et les peindre avec force. Un séjour continuel avec des êtres amollis et légers ferait perdre à ses tableaux les traits mâles et la teinte vigoureuse de la vérité. Les ouvrages dont les auteurs ne se proposent que de plaire aux grands, aux femmes, à un public frivole, ont rarement l’empreinte de l’immortalité.\par
En général, les savants et les gens de lettres ont plus à perdre qu’à gagner dans un commerce trop fréquent avec les gens du monde : s’ils y acquièrent du côté des grâces, de la diction, du {\itshape bon ton}, ils y perdent souvent du côté de la force, de la profondeur et surtout de la vérité, qui communément paraît trop austère et trop grave à des enfants volages qui ne veulent qu’être amusés et qui trouvent toute instruction inutile et ennuyeuse. Pour plaire aux gens du monde, l’homme de lettres doit être frivole, badin, superficiel et ne jamais parler raison.\par
C’est encore dans le grand monde que l’homme de lettres, ambitieux des vains suffrages d’une foule de personnages vains et légers, contracte l’habitude du faste, de la dépense, de l’arrogance, de la fatuité, du libertinage et des travers qui lui conviennent si peu. Il devient avide, envieux, intriguant, flatteur, pusillanime. Après lui avoir communiqué leurs vices et leurs folies, les gens du monde ne manquent pas de les lui reprocher avec aigreur et de le couvrir de ridicule.\par
Voilà comment des hommes faits pour instruire se rendent souvent méprisables en voulant plaire et amuser au lieu de se rendre utiles. Voilà comment les leçons de la sagesse deviennent infructueuses, par l’inconduite de ceux qui les débitent aux autres sans savoir s’y conformer eux-mêmes. Par un préjugé très commun dans le monde, la mauvaise conduite des savants rejaillit sur leur doctrine ; celle-ci est rejetée lorsque les mœurs de celui qui l’enseigne ne s’y trouvent pas conformes. Il y a loin, comme on dit, du cœur à l’esprit ; un homme peut raisonner très juste et se conduire très mal. « Les mœurs des philosophes, dit Sénèque, ne sont pas conformes à leurs préceptes ; ils ne vivent pas comme ils enseignent, mais ils enseignent comme il faut vivre. » Ainsi, ne vivons pas avec l’homme dont le cœur est mauvais : lisons ses ouvrages quand nous y trouverons des instructions utiles, rejetons et l’homme et ses ouvrages quand ils seront dangereux. « Un homme de bonnes mœurs, dit Montaigne, peut avoir des opinions fausses ; et un méchant peut prêcher la vérité, voire celui qui ne la croit pas. C’est sans doute une belle harmonie quand le faire et le dire vont ensemble\footnote{{\itshape Essais}, livre II, chap. 31.}. »\par
Le vrai savant, dont la conduite est sage, jouira d’une somme de bonheur plus grande que les autres hommes : toujours assuré de trouver en lui-même et dans la méditation des moyens de s’occuper agréablement, il sera peu sensible aux passions, aux fantaisies, aux vanités qui tourmentent les êtres frivoles dont le monde est rempli ; satisfait des plaisirs tranquilles du cabinet et des richesses que l’étude rassemble dans son sein, il peut à volonté se procurer des jouissances inconnues de la grandeur ignorante et superbe ou de l’épaisse opulence. L’ambition, la cupidité, les voluptés, la débauche ne toucheront point celui qui se suffit et qui, comme Bias, porte ses richesses en lui-même. « À la vérité, dit Épicure, le sage est sujet aux passions, mais leur impétuosité ne peut rien contre sa vertu\phantomsection
\label{footnote81}\footnote{Voyez Diogène Laërce, {\itshape Vies et Doctrines des Philosophes illustres}, livre X, 117.}. »\par
S’orner l’esprit, c’est acquérir par l’étude un ample fond d’idées que l’on peut à chaque instant contempler à son gré. La retraite, si pénible pour les hommes dissipés, fait les délices de l’homme de lettres qui, semblable à l’avare, augmente en secret son trésor à tout moment ; le tumulte du monde lui déplaît ; le vrai savant n’a qu’à perdre dans le commerce des êtres qu’il y rencontre. Ses livres, ses réflexions, la conversation de ses pairs suffisent au bonheur de celui qui s’est exercé l’esprit. Il jouit à chaque instant de la contemplation des richesses que chaque jour il dépose dans sa tête ; sans sortir de lui-même il considère le spectacle varié de la Nature, le jeu des passions et des actions des hommes, le tableau des vicissitudes de ce monde, les révolutions continuelles auxquelles les choses humaines sont exposées ; il possède des biens que ni l’injustice de la tyrannie, ni les caprices de la fortune ne peuvent lui enlever.\par
L’étude procure à l’homme qui pense une satisfaction douce que l’on peut comparer à celle de la bonne conscience ; elle le met toujours en état de rentrer avec plaisir en lui-même et de se passer des vains amusements si nécessaires aux personnes qui ne peuvent converser avec elles-mêmes.\par
Cependant, n’en croyons pas les maximes outrées d’une philosophie sauvage qui défendrait à l’homme de lettres de songer à sa fortune. N’écoutons pas les déclamations des cyniques qui font un devoir au sage de renoncer aux richesses sous prétexte que ce sont des biens trompeurs et périssables. L’aisance acquise par la science et les talents ne peut être blâmée\phantomsection
\label{footnote82}\footnote{Voyez Diogène Laërce, {\itshape Ut. Supra Sec.} 121.}. L’homme sensé doit éviter l’indigence qui, le mettant dans une trop grande dépendance, l’exposerait souvent à se déshonorer par des bassesses. La vraie sagesse ne consiste pas dans un mépris farouche pour ce que les hommes estiment et révèrent, elle consiste à ne s’y point attacher trop fortement et à conserver une constance qui fasse soutenir avec moins de peine les rigueurs de la fortune. La singularité, la négligence, la saleté, l’impolitesse, l’indécence n’annoncent point un philosophe mais un fanatique, un insensé, un esprit faible qui est la dupe de sa propre vanité ou un hypocrite qui veut tromper les autres par une grandeur d’âme simulée. Si l’utilité sociale est le fondement de la considération due aux talents de l’esprit, le savant doit se proposer de mériter les suffrages de ses concitoyens par des travaux dont il résulte des avantages réels pour la société. C’est en instruisant ou en amusant que l’homme de lettres peut se rendre cher et parvenir à la réputation qu’il désire.\par
« Rien n’est plus doux, dit Cicéron, que d’instruire et de former les esprits. » L’homme éclairé, l’homme de génie exercent dans le monde une autorité qui, fondée sur la vérité, devient irrésistible\footnote{Le fameux Swift dit quelque part « qu’il ne paraît guère dans un siècle que cinq ou six hommes de génie ; mais que s’ils réunissaient leurs forces diverses, le monde ne pourrait pas leur résister ». Voyez {\itshape The Adventurer}, tome I, page 234.}. Suivant Plutarque, le philosophe Ménédème comparait les gens de lettres qui se livrent à des études inutiles ou frivoles, aux amants de Pénélope qui, ne pouvant épouser la maîtresse, se livraient à la débauche avec les suivantes. « C’est ainsi, disait-il, que ceux qui n’ont pas la force d’atteindre à la philosophie se consument de travail sur des objets futiles et peu dignes de lui être comparés. » Dans les nations corrompues et soumises au despotisme, l’esprit est obligé de se porter sur des objets frivoles, et le génie ne s’exerce que sur des bagatelles. « La gloire, dit Phèdre, est une folie si nous croyons la trouver dans ce qui n’est point utile\footnote{Phèdre, {\itshape Fables}, livre III, 17, vers 12.}. »\par
Les opinions souvent nuisibles et fausses, ainsi que les mauvaises mœurs établies dans la société, contribuent quelquefois à pervertir les gens de lettres et tournent leurs esprits vers des objets inutiles ou dangereux. C’est ainsi que la dépravation publique fait éclore des productions obscènes et lubriques qui procurent à leurs auteurs une célébrité malheureuse faite pour les dégrader aux yeux des honnêtes gens. N’est-ce pas se rendre bien coupable que d’employer ses talents à la corruption de la jeunesse, à la propagation du vice ? Quels reproches ne devrait pas se faire un écrivain dont les ouvrages séduisants sont de nature à faire germer des passions funestes jusque dans la postérité la plus reculée ? Combien odieuse est une immortalité que l’on prétend acquérir par un empoisonnement perpétué du cœur humain !\par
La morale et l’équité ne permettent pas non plus de placer parmi les savants et les gens de lettres ces critiques impudents, de mauvaise foi, armés par une basse jalousie, qui semblent déclarer la guerre aux grands talents, qui déchirent les savants distingués et les immolent à la risée d’un public envieux et malin que le mérite offusque. Des écrivains de cet affreux caractère ne peuvent être regardés que comme des ennemis des sciences, des lettres, des progrès de l’esprit humain. Ce sont de vils complices de l’ignorance jalouse, de l’imposture inquiète, de la tyrannie alarmée qui, pour dominer sur la terre, voudraient y faire régner une nuit éternelle\phantomsection
\label{footnote83}\footnote{« ... Immensi fruitur caligine mundi. » Satii, {\itshape Thébaïde}, livre 3.}. Est-il une occupation plus infâme que celle d’amuser le public aux dépens des citoyens qui l’éclairent, qui le servent utilement, qui méritent toute sa reconnaissance ?\par
Pour être vraiment utile, la critique doit être juste, instructive, polie ; jamais il ne lui est permis de dégénérer en une satire offensante et personnelle. Les amusements que l’homme de lettres procure doivent être intéressants et contribuer sans cesse à la félicité publique. Ceux qui n’ont pour objet que de charmer les ennuis de quelques êtres légers, de flatter les vices du {\itshape bon ton}, d’exciter à la débauche, de favoriser les mauvaises mœurs, d’encenser la tyrannie, ne méritent que l’indignation et le mépris. Pour être en droit de prétendre à une estime fondée, les différentes classes de la république des lettres devraient, par des routes diverses, tendre invariablement à l’utilité générale ; c’est sur les droits de la vérité et sur les avantages qu’elle fournit aux hommes que la considération des gens de lettres peut être solidement établie.\par
La poésie, qui se propose de plaire par ses images, au lieu de nous peindre des passions efféminées, des amours méprisables, devrait intéresser l’imagination des hommes pour la vérité, en l’ornant de couleurs les plus capables de toucher.\par
La tragédie, pour être utile, doit inspirer de la frayeur pour les crimes des rois, dont les passions déchaînées produisent si souvent des catastrophes aussi cruelles que terribles ; elle devrait faire trembler les tyrans et rendre chères aux citoyens la liberté et la vertu, sans lesquelles nulle société ne peut être heureuse et florissante.\par
La satire, tant de fois employée pour immoler à la malignité publique des citoyens qui ne sont qu’à plaindre, devrait épargner les personnes et faire rougir le vice des désordres et des travers dont il se rend coupable. La satire générale est utile et louable ; la satire personnelle est inhumaine et punissable.\par
La comédie, destinée à faire sentir aux hommes le ridicule de leurs vices, de leurs défauts, de leurs travers, ne devrait jamais se permettre de les faire rire aux dépens de la raison, de la décence et des mœurs, pour lesquelles tout devrait inspirer le respect le plus profond\phantomsection
\label{footnote84}\footnote{On pourrait appliquer aux auteurs qui abusent de leur talent la malédiction de Démocrite qui s’écriait : « Malheur à vous qui des grâces pudiques et vierges, n’avez su faire que de viles prostituées ! » Combien de pièces de théâtre qui renferment des leçons de corruption que des gouvernements permettent qu’on donne publiquement à la jeunesse.}.\par
Les romans, qui trop communément ne servent qu’à faire germer et nourrir dans de jeunes cœurs des passions dangereuses, devraient au contraire mettre la jeunesse imprudente en garde contre des faiblesses capables d’influer sur le bonheur de la vie.\par
L’éloquence, dont trop souvent on abuse pour tromper et séduire, dans la bouche de l’homme de bien ne doit servir qu’à persuader la vérité, qu’à échauffer les cœurs des hommes de l’enthousiasme du bien public et des vertus sociables, qu’à leur inspirer de l’horreur pour le mal et du mépris pour les objets qui les détournent du chemin de la félicité.\par
Mais dans un monde occupé de futilités, la sagesse, la morale, la philosophie, la vertu même, deviennent souvent ridicules aux yeux d’une foule de beaux esprits : accoutumés à confirmer le public dans ses folies habituelles, ils semblent craindre les approches du règne de la raison. On pourrait comparer leur conduite à celle de ces femmes de mauvaise vie que l’on voit se désoler lorsque les dupes qu’elles amusaient autrefois commencent à songer à leurs affaires et renoncent à leurs folies pour prendre une conduite plus sensée. Les nations sont inondées de productions qui rarement ont pour objet les intérêts de l’homme. Emportés communément par l’imagination, les gens d’esprit dédaignent les études profondes, qui ne peuvent être que les fruits lents de la réflexion. Rien ne s’oppose plus aux progrès du {\itshape bon esprit} que le {\itshape bel esprit} ; la raison est souvent aux prises avec ceux qui pourraient le mieux seconder ses efforts. D’un autre côté, la république des lettres s’avilit quelquefois aux yeux des gens du monde par la conduite peu raisonnée de quelques-uns de ses membres, qui semblent prendre à tâche de persuader au public que la science et les talents sont incompatibles avec la bonté du cœur et le sang-froid de la raison.\par
Ainsi que les États libres, la république des lettres est souvent divisée en factions qui l’affaiblissent et l’exposent au mépris de ceux dont elle devrait se faire respecter. Que peuvent penser les grands, les gens du monde, quand ils voient les gens de lettres maladroitement occupés à se démolir les uns les autres et à contrarier les efforts de la raison lorsqu’elle tâche de détromper les hommes de leurs folies ? Tandis que le philosophe présentera des principes évidents, un bel esprit déclamera contre la vérité qui lui paraît trop triste, contre la morale qu’il traite de lugubre, contre la sagesse qu’il trouve trop sévère ; un autre exagérera l’incertitude de nos connaissances et consolera la sottise, en l’assurant que les meilleurs esprits n’en savent pas plus que les autres ; d’autres, enfin, jetteront du ridicule sur les découvertes les plus utiles. Les ouvrages profonds seront regardés comme ténébreux, comme les productions d’une métaphysique obscure et de quelques cerveaux creux. Enfin, les vérités les plus intéressantes demeureront ensevelies dans l’oubli si elles ne sont accompagnées des charmes du style et le plus souvent d’un faux brillant auquel le vulgaire attache le plus grand prix.\par
Les ornements du style ne doivent point être négligés : les grâces sont propres à rendre la vérité plus touchante ; mais ces ornements sont la forme, qui doit céder au fond. Le savant qui a profondément pensé n’a pas toujours le talent de bien écrire ; de même que celui qui possède ce talent si vanté n’a pas toujours péniblement médité. Quoi qu’il en soit, recevons le vrai avec reconnaissance, de quelque façon qu’il nous soit présenté, et souvenons-nous que le mépris de la vérité est le caractère distinctif des imposteurs, des charlatans, des ignorants, et surtout des tyrans, des ennemis du genre humain, personnages avec lesquels les gens de lettres ne devraient jamais souffrir qu’on les confondît. Ceux d’entre eux qui haïssent et décrient la vérité sont des insensés qui détruisent les fondements de leur propre gloire ; elle ne peut être solidement établie que sur l’utilité et sur la vérité, que tant d’aveugles ont la folie de décrier.\par
Gémissons de ce désordre et ne cessons point de répéter que les gens de lettres devraient se distinguer par leur concorde et s’unir pour concourir aux vues de la morale et de la saine philosophie, dont le but invariable ne peut être que de rendre les hommes meilleurs. Les connaissances et les lumières ne sont rien si elles ne contribuent au bien-être de la société ; la gloire qu’elles obtiennent n’est rien si elles ne nous procurent une félicité durable ; les sciences sont méprisables lorsqu’elles sont stériles, elles sont détestables quand elles contredisent la vraie morale, qui de toutes les sciences nous intéresse le plus\phantomsection
\label{footnote85}\footnote{« Débattons des sujets qui nous touchent plus directement et qu’il est mauvais d’ignorer. » Horace, {\itshape Satires}, livre II, 6, vers 72-73.}. « Il n’y a, dit Quintilien, que la sensibilité de l’âme qui rende vraiment éloquent et discret\phantomsection
\label{footnote86}\footnote{« Pectus est quod disertos facit, et vis mentis. » Quintilien, {\itshape Institutions oratoires}, livre 10, chap. 7, n°15, édition Cesner.}. » Un intérêt tendre pour l’humanité doit animer les gens de lettres : c’est l’homme qu’ils doivent éclairer, attendrir sur son propre sort, échauffer pour la vertu, parce que la vertu seule peut bannir les malheurs dont il est la victime et le mettre en possession du bonheur vers lequel il ne cesse de soupirer. « L’étude la plus importante pour l’homme, selon Pope, c’est l’homme. »\par
L’amour de la gloire, le désir de plaire et d’être estimé des gens de bien sont et doivent être les grands mobiles des gens de lettres et des savants ; leur faire un crime d’aimer la gloire et de courir après la renommée, c’est leur reprocher de ne point agir sans motifs. Rien de plus louable que de vouloir se faire considérer par des talents vraiment capables de contribuer au bien de tous. Mais l’homme de lettres manque son but dès qu’il n’est point utile ; il ne peut être utile s’il ne présente pas aux hommes des vérités dignes de les intéresser. Des riens brillants, des productions agréables, des ouvrages éphémères peuvent avoir des succès momentanés. Une réputation factice conservée par des cabales, des intrigues, des menées, des bassesses, des complaisances, peut durer quelque temps ; mais la gloire solide, la considération permanente, l’immortalité ne sont réservés qu’aux ouvrages dont le genre humain recueille en tout temps les fruits délicieux. Tout homme qui dans ses écrits ne cherche qu’à plaire à son siècle ou qui ne songe qu’à sa fortune, fera difficilement passer son nom à la postérité.\par
Hommes vraiment illustres et respectables quand vous travaillez au bonheur des nations ! savants et gens de lettres qui par des voies diverses cherchez la renommée ! songez qu’elle n’est que l’affection et l’estime publique, et que ces sentiments ne sont dus qu’à la vérité, à l’utilité, à la vertu. Que votre conduite apprenne donc à respecter les fonctions honorables que vos talents vous font remplir au milieu de vos concitoyens. Respectez-vous vous-mêmes ; souvenez-vous de votre propre dignité ; éloignez-vous de la bassesse et de la flatterie qui vous aviliraient aux yeux d’un public jaloux de vos prérogatives. Abjurez entre vous ces querelles déshonorantes qui ne peuvent amuser que la malignité de vos envieux. Unissez-vous pour combattre l’ignorance, les vices et les folies qui désolent la terre et s’opposent à la félicité sociale. Mais en attaquant les travers et les erreurs des hommes, ménagez leur amour-propre afin de rendre vos leçons plus efficaces. Craignez de blesser ceux que vous voulez guérir.\par
Philosophes ! Votre fonction sublime est de méditer l’homme, de lui découvrir les replis de son cœur, de lui montrer la vérité sans laquelle il ne peut obtenir le bonheur. Orateurs ! Que votre éloquence, nourrie par la philosophie, arrache l’homme à ses erreurs, à ses penchants vicieux, l’attendrisse sur lui-même et porte dans son cœur la compassion, l’humanité, l’affection qu’il doit à ses semblables. Historiens ! Servez-vous des recherches du sage et des couleurs de l’éloquence pour nous peindre avec vigueur et vérité l’intéressant tableau des vicissitudes humaines. Poètes ! Empruntez les lumières de la sagesse, la force de l’éloquence, les leçons de l’Histoire pour orner la vérité des charmes dont l’imagination est capable de l’embellir. Laissez-là ces chants frivoles et dangereux qui trop souvent n’ont eu pour objet que de rendre le vice aimable et d’inspirer du mépris pour la vertu. Érudits et savants ! Cessez de fouiller une Antiquité ténébreuse, pour n’y trouver que des choses inutiles aux races présentes. Penseurs ! Ne vous enfoncez plus dans l’affreux labyrinthe d’une métaphysique tortueuse dont il ne peut résulter aucun bien pour notre espèce : portez plutôt la subtilité de votre esprit sur des objets conformes à notre nature et que nous puissions saisir. Physiciens ! Naturalistes ! Médecins ! Renoncez aux vaines hypothèses, ne suivez que l’expérience : elle vous fournira des faits dont l’ensemble pourra former un système sûr, vraiment utile au genre humain. Jurisconsultes ! Abandonnez enfin les sentiers bourbeux de la routine, dégagez-vous des lisières de l’autorité ; cherchez dans la nature même de l’homme des lois conformes à son être : vous y trouverez une jurisprudence morale, juste, simple, facile, dont les peuples ont un si grand besoin.\par
Enfin, quelle que soit la route où vos talents vous jettent, que chacun de vous, ô savants ! se propose l’utilité de l’homme, le bien public, les intérêts de la société, le bonheur de l’univers à qui vos leçons sont destinées. Votre but étant le même, que personne ne dédaigne ou ne déprime les travaux de ses associés. Le champ de la science n’est-il pas assez vaste et fertile pour que chacun de vous puisse y cueillir des lauriers ? Bannissez donc, ô hommes utiles ! la discorde qui nuirait à vos succès, que vos âmes nobles et généreuses se mettent au-dessus des bassesses de l’envie, des petitesses de la vanité ; la jactance et le charlatanisme sont indignes de vous. C’est au public qu’il faut laisser le soin de vous louer. Souvenez-vous que les lettres et les sciences doivent rendre l’homme plus humain, plus doux, plus sociable et n’oubliez jamais que votre modestie, votre retenue, votre politesse et vos mœurs peuvent seules engager le public à vous pardonner vos talents, vos bienfaits, votre supériorité. En suivant ces maximes, vous mériterez l’amour, l’estime, les suffrages de vos contemporains, et vos travaux utiles feront passer votre gloire à la postérité qui jouira comme nous de vos travaux immortels.\par
L’espérance et le désir de l’immortalité, que tant de gens ont regardé comme une vaine chimère, une folie, une fumée, sont pourtant des motifs qui ont de tout temps aiguillonné puissamment les hommes de génie : ces passions sont fondées sur l’idée qu’ils se sont faite des droits que leurs travaux leur donneraient sur l’affection, l’estime et la reconnaissance des races futures. N’appelons donc point une chimère ce qui est un bien réel pour celui qui en jouit au dedans de lui-même à chaque instant de sa durée. La bonne conscience procure à l’homme de bien un bonheur très véritable et très solide, quoiqu’il n’en jouisse que par l’imagination, qui lui montre ses droits à l’affection des autres hommes. L’idée de l’immortalité n’est une chimère que pour ceux qui n’ont ni le courage ni le droit d’y prétendre.\par
L’affection et les louanges de la postérité sont des dettes qu’elle acquitte souvent pour ses injustes pères ; elle ne peut en priver ceux qui ont procuré de grands avantages, de grands plaisirs, de grandes vérités au genre humain. Par un privilège spécial attaché aux gens de lettres, l’écrivain distingué conserve tous ses droits au-delà même du trépas. Un ouvrage vraiment utile ou agréable est un bienfait perpétuel, il oblige les races les plus éloignées. La mort qui plonge souvent dans un oubli total tant de personnages superbes, ne détruit pas les rapports de l’homme de génie avec le genre humain et n’anéantit point nos devoirs envers celui qui a daigné nous instruire ou nous amuser. Ne serions-nous pas injustes, ingrats, insensés si nous refusions de chérir la mémoire de ceux qui chaque jour nous procurent d’heureux moments ?\par
Il subsiste encore un commerce tendre entre nous et les sages de l’Antiquité. Nous lisons avec reconnaissance les ouvrages immortels des Homère, des Cicéron, des Virgile, des Sénèque : nous leur payons fidèlement le tribut qu’ils ont dû se flatter d’obtenir de nous. Indépendamment du profit et du plaisir que nous retirons des écrits de ces illustres morts, l’intérêt actuel et permanent des nations veut que nous rendions des hommages aux bienfaiteurs du genre humain. C’est encourager les vivants que de louer les morts ; quoique leurs cendres froides soient insensibles à nos éloges présents, ils en ont joui pendant leur vie, et ils servent de siècle en siècle à conserver la flamme du génie, à la transmettre à ceux qui pourront les imiter.\par
Enfin, l’idée de l’immortalité ou de la reconnaissance future est faite pour consoler le grand homme de l’ingratitude, de l’injustice, de l’envie de ses contemporains. La conscience d’avoir bien fait le dédommage des louanges qu’on lui refuse ; il entend celles de l’avenir parce qu’il sait que les hommes sont toujours justes pour des bienfaiteurs dont ils ne craignent plus la supériorité.\par
Après avoir exposé les devoirs des hommes que leurs talents destinent à instruire leurs concitoyens, la morale ne peut pas omettre les devoirs de ceux qui exercent les beaux-arts, dont l’objet est d’agir sur les sens, de les remuer agréablement, d’amuser et de délasser les citoyens de leurs travaux, de porter des idées flatteuses à l’esprit. Il se trouve une affinité marquée entre les lettres et les productions des arts ; {\itshape la peinture}, dit Horace, {\itshape est comme la poésie}. Lorsqu’elle nous montre des actions, ne fait-elle pas la fonction de l’Histoire ? Lorsqu’elle les présente de manière à nous émouvoir vivement, n’agit-elle pas comme l’art oratoire dont le but est de remuer nos passions ?\par
Ainsi, de même que les gens de lettres, les artistes doivent dans leurs travaux divers se proposer un but moral. Qu’ils sentent leur pouvoir, qu’ils apprennent à se respecter eux-mêmes, qu’ils se regardent comme des citoyens non seulement faits pour amuser mais encore pour instruire. Qu’ils aient en vue un objet plus noble et plus grand que de flatter la vanité ou la dépravation de l’opulence, qu’ils éprouvent la louable ambition d’êtres utiles aux hommes et de les rendre meilleurs. Pourquoi l’artiste habile dont les ouvrages font penser et laissent dans les esprits des traces profondes et durables, ne chercherait-il pas à éclairer en même temps qu’il sait plaire ?\par
Les grands artistes chez les Grecs furent des citoyens considérés. Ils n’étaient point regardés comme de vils mercenaires : nourris dans les écoles de la philosophie, admis à la conversation des savants, ils avaient occasion de méditer leur art, de perfectionner leurs talents, et par là de les porter à ce degré de sublimité qui fait le désespoir des artistes modernes. Ceux-ci, trop souvent privés des lumières que procure une éducation soignée, étrangers à l’instruction, peu susceptibles de méditation, sont rarement capables de donner à leurs ouvrages cette noble simplicité, cette énergie, cette vie que l’on admire dans ceux des anciens.\par
Pour faire de belles choses l’artiste doit être instruit, doit avoir réfléchi sur son art, doit connaître les objets qu’il se propose d’imiter, enfin doit pressentir les effets qu’il peut produire. Sans ces connaissances il ne serait qu’un automate qui travaillerait au hasard ; dépourvu de principes, il ne pourrait jamais être sûr de réussir ou de plaire.\par
C’est sur les cœurs des hommes que l’artiste éclairé doit se proposer d’agir ; mais il ne se permettra jamais de les corrompre. Ainsi, au lieu de puiser ses sujets dans une mythologie souvent lascive et criminelle, au lieu de nous représenter sans cesse les amours d’une foule de divinités, de nymphes, de satyres impudiques, un peintre plus décent et plus moral nous retracera quelques traits mémorables de grandeur d’âme, de bonté, de justice, d’amour pour la patrie que lui fournit l’Histoire et dont il saisira les côtés les plus frappants. Les productions des arts deviendraient pour nous des leçons si elles ne nous offraient que des objets capables d’exciter à la vertu ; elles feraient alors bien plus d’honneur, sans doute, soit au pinceau du peintre, soit au ciseau du sculpteur, soit au burin du graveur, que les dérèglements consacrés par la religion impure des Grecs et des Romains, ou que des nudités indécentes que, sans respect pour les mœurs, nous voyons souvent étalées dans les palais ainsi que dans nos carrefours et nos rues. Quels reproches ne devraient pas se faire des artistes qui ne se servent de leurs talents que pour infecter les esprits d’images obscènes et faire éclore dans les cœurs des passions dangereuses ? Comment dans des nations policées où les mœurs de la jeunesse devraient être soigneusement garanties, souffre-t-on que tant de causes concourent à les empoisonner !\par
Mais dans les nations corrompues, les bonnes mœurs ne sont comptées pour rien. Des artistes privés eux-mêmes d’éducation, de lumières et de mœurs, ne peuvent plaire à une multitude dépravée qu’en lui présentant des objets conformes à ses goûts pervers. Dans une société sagement ordonnée tous les talents se donneraient la main pour exciter et nourrir les dispositions avantageuses au public et pour étouffer celles dont il peut résulter du désordre et des crimes. C’est alors que les arts deviendraient vraiment estimables. Ils s’honoreraient bien plus en transmettant à la postérité la reconnaissance publique pour les grands hommes, les vrais bienfaiteurs de la patrie, qu’en lui faisant passer les traits et la mémoire de tant de tyrans odieux, de prétendus héros, de conquérants détestables qu’elle devrait oublier.\par
Que les artistes apprennent donc à devenir des citoyens utiles, qu’ils sentent leur dignité, qu’ils s’associent avec les philosophes, les orateurs, les écrivains illustres, qu’ils méditent les ressources de l’art, qu’ils les fassent servir au bien public. D’accord avec le poète, que le musicien, au lieu d’amollir les âmes par les accents efféminés d’une passion rebattue, fasse entendre à ses concitoyens ces sons mâles, cette harmonie jadis si puissante dans la Grèce. Que la musique, par ses modes variés, excite tantôt le courage, la force, la grandeur d’âme, tantôt qu’elle porte la consolation, la pitié, le calme dans nos cœurs ; enfin, qu’unie à des paroles convenables elle leur prête une expression plus animée et les rende capables de faire naître des sentiments agréables conformes au bien de la société.\par
L’art du musicien montre une analogie très marquée avec celui de l’orateur et du poète. Pour rendre les paroles plus expressives et plus fortes, qu’il se pénètre lui-même des sentiments qu’il veut faire passer dans les autres. D’où l’on voit que l’instruction et la réflexion ne lui sont pas moins essentielles qu’aux peintres et aux autres artistes dont nous avons parlé. Faire de la bonne musique, c’est peindre à l’oreille, c’est y exciter des sensations que l’expérience et la réflexion ont montré capables de produire des sentiments désirés dans les auditeurs. Un musicien qui n’a pas la connaissance de l’homme et des moyens de le remuer n’est qu’une pure machine, un instrument sonore. Ainsi, ne soyons point surpris si les grands musiciens sont rares. Beaucoup de gens possèdent les règles de la musique mais ignorent les moyens de les appliquer. Bien des artistes, à force de travail, sont parvenus à vaincre les plus grandes difficultés et à s’attirer par là l’admiration du vulgaire ; mais cette musique purement mécanique ne suppose que des dispositions naturelles opiniâtrement exercées : elle n’annonce ni génie ni réflexion ; elle n’est pas faite pour produire sur les âmes les grands effets que l’on pourrait attendre du musicien qui a senti et médité le pouvoir de son art.\par
On met encore communément la danse au rang des arts libéraux. Indiquée par la nature des fluides de notre corps, dont les mouvements sont périodiques, nous la trouvons établie chez tous les peuples de la terre, tant sauvages que policés\footnote{Érophile, musicien grec, a remarqué que le battement des artères avait donné naissance à la mesure musicale. Voyez {\itshape Censorinus de Die nalali, Cum notis Havercamp}, page 57.}. Quelques-uns l’ont consacrée ou divinisée en l’alliant au culte religieux, d’autres religions la proscrivent comme un exercice contraire aux mœurs. Si nous considérons la danse comme exercice, elle est utile à la santé, elle rend l’homme plus dispos, elle lui enseigne à se mouvoir avec adresse, à se tenir d’une manière plus ferme, à marcher avec sûreté, à se montrer dans tout son avantage, à se présenter avec grâce, c’est-à-dire d’une façon qui annonce une éducation cultivée conforme aux manières adoptées par la société. Sous ce point de vue, la danse ne peut être blâmée ; utile pour nous-mêmes, elle nous rend plus agréables aux autres.\par
Mais la saine morale ne peut porter qu’un jugement défavorable de ces danses qui ne présentent aux yeux que des attitudes indécentes propres à faire germer dans l’esprit des deux sexes des pensées déshonnêtes, des désirs déréglés. Nous avons déjà fait voir ailleurs les dangers auxquels la jeunesse est trop souvent exposée dans ces assemblées confuses où l’innocence étourdie par le tumulte fait de très fréquents naufrages, où des passions criminelles cherchent et trouvent tant de moyens de se satisfaire. Les danses de ce genre sont des aventures périlleuses auxquelles des parents vertueux craindront de livrer une jeunesse imprudente ; ils sentiront que la raison ne peut les approuver. Conforme en cela aux règles de la morale la plus sévère, la morale de la Nature exhortera toujours les hommes à fuir les dangers. D’après la perversité des mœurs établies dans bien des nations, les gens même les plus corrompus seront forcés de convenir que la danse est un écueil auprès duquel la vertu vient souvent échouer.\par
Concluons de tout ce qui est dit dans ce chapitre que la science est utile et nécessaire aux nations, que ceux qui les instruisent sont des citoyens dignes d’être honorés, chéris, récompensés, que les détracteurs des connaissances humaines, les oppresseurs des lumières, les contempteurs des lettres sont des insensés qui méconnaissent et les biens qu’elles font aux hommes et les dangers de l’ignorance, qui fut toujours la source des malheurs de la terre. Tout a dû nous prouver que la méditation, la réflexion, l’étude sont nécessaires non seulement dans les sciences et les lettres, mais encore dans les arts. Enfin, tout a pu nous convaincre que les savants, les lettrés, les artistes, ne doivent jamais perdre de vue la morale et la vertu, dont pour être vraiment utiles ils devraient, chacun à sa manière, inculquer les leçons. C’est ainsi qu’en augmentant de jour en jour la masse des lumières ou des vérités, ils pourront se flatter de contribuer au bonheur de la vie sociale.
\subsection[{Chapitre XI. Devoirs des Commerçants, Manufacturiers, Artisans et Cultivateurs}]{Chapitre XI. Devoirs des Commerçants, Manufacturiers, Artisans et Cultivateurs}
\noindent Toute société est un assemblage d’hommes destinés à concourir, chacun à sa manière, au bien-être et à la conversation du corps dont ils sont membres. Quiconque travaille utilement pour tous les concitoyens devient dès lors un homme public que son pays doit protéger, honorer, favoriser proportionnellement aux avantages que le public en retire.\par
Cela posé, le commerçant est un membre estimable toutes les fois qu’il remplit dignement les fonctions auxquelles son état le destine. C’est lui qui débarrasse sa patrie des denrées et des productions superflues de la culture, des manufactures, de l’industrie, et qui lui procure en échange les objets, soit agréables soit nécessaires, dont elle peut manquer. Ainsi, le commerçant fait fleurir l’agriculture, qui languirait sans son secours ; c’est lui qui dans les temps de disette fait venir de l’étranger les subsistances dont l’intempérie des saisons a privé son pays. C’est le commerce qui donne la vie à tous les arts et métiers : il anime l’industrie et par là il occupe et nourrit une quantité prodigieuse d’hommes que sans lui leur indigence rendrait à charge aux nations. Combien de bras sont continuellement occupés pour la navigation, destinée à porter les ordres du négociant jusqu’aux extrémités de la terre ; ces ordres sont presque toujours plus ponctuellement exécutés que ceux du despote le plus absolu. Dans les pays les plus lointains des milliers de bras s’empressent à satisfaire ses désirs ; l’océan gémit sous le poids des navires qui des climats les plus éloignés viennent apporter à ses pieds des richesses et l’abondance à ses concitoyens. Le comptoir du négociant peut être comparé au cabinet d’un prince puissant qui met tout l’univers en mouvement.\par
Tel est le citoyen respectable que des préjugés gothiques et barbares ont l’impudence de flétrir au sein même des nations qui ne doivent qu’au commerce leurs richesses et leurs splendeurs ! Le commerçant pacifique paraît un objet méprisable aux yeux du guerrier stupide qui ne voit pas que cet homme qu’il dédaigne le vêtit, le nourrit, fait subsister son armée ! Une profession si utile n’est-elle donc pas plus honorable que l’oisiveté honteuse dans laquelle croupissent tant de nobles campagnards qui n’ont pour toute occupation que la chasse et le triste plaisir de vexer des paysans ? Jusqu’à quand la vanité des hommes leur fera-t-elle mépriser ceux mêmes dont ils reçoivent chaque jour les services les plus importants ? La considération sera-t-elle toujours exclusivement réservée aux destructeurs des hommes ? Ne devrait-elle pas se porter sur ceux qui s’occupent de leur bien-être, de leurs commodités, de leurs besoins !\par
Le préjugé dégradant pour le négoce, ainsi que pour les arts, date des temps de barbarie et de férocité où des sociétés naissantes ne connaissaient pas encore les avantages qu’elles pouvaient retirer du commerce. Aristote nous apprend que dans les anciennes républiques de la Grèce les marchands étaient exclus des charges de la magistrature. Par l’effet d’une pareille ignorance, les anciens Romains, uniquement occupés de l’agriculture et de la guerre, méprisèrent les marchands et les artisans. Mais enfin le temps et les besoins désabusèrent peu à peu les Grecs et les Romains de cette opinion ridicule et les personnes les plus distinguées de l’État ne rougirent plus d’exercer une profession lucrative pour elles-mêmes et très avantageuse à la patrie.\par
Lorsque des essaims de nations guerrières eurent partagé entre elles le vaste empire des Romains, le préjugé, qui toujours accompagne l’ignorance, vint de nouveau dégrader le commerce. L’Europe fut pendant des siècles plongée dans d’épaisses ténèbres et dans des guerres continuelles. Les peuples, asservis par des soldats licencieux, n’eurent aucune communication les uns avec les autres. Le commerce, qui ne peut fleurir sans liberté, fut exercé par des Juifs, des usuriers, qui se virent continuellement en butte à l’avarice d’une foule de tyrans. Ainsi, le négoce tomba dans des mains méprisables ; des malheureux, attirés par l’appas d’un gain démesuré pouvaient seuls entreprendre de le faire malgré tous les dangers dont ils étaient environnés. Telle est, sans doute, l’origine de l’injuste mépris que tant de nobles orgueilleux montrent encore pour une profession devenue très digne de la considération publique.\par
Cependant, quelques républiques, usant de leur liberté, firent le commerce avec succès et parvinrent par son moyen à un degré de puissance et de richesse qui causa la jalousie des autres peuples. Venise, Gênes, Florence apprirent à toute l’Europe les effets que pouvait produire le négoce {\itshape ;} des princes le favorisèrent. Un nouveau monde fut découvert {\itshape ;} ses richesses irritèrent la cupidité d’un grand nombre de nations. L’indifférence qu’elles avaient jusque-là témoignée pour le commerce se convertit dans un enthousiasme universel, et bientôt elles ne combattirent que pour s’arracher les unes aux autres quelques branches de commerce.\par
Voilà comment les passions et les folies des hommes les portent aux extrêmes. Tout fut sacrifié à la fureur du commerce : en sa faveur l’agriculture fut négligée, des royaumes furent dépeuplés pour former des colonies dans des contrées lointaines ; des torrents de richesses vinrent inonder l’Europe, sans la rendre plus heureuse. Elles amenèrent le luxe et tous les vices qu’il entraîne à sa suite, et ce luxe travailla sourdement à la destruction des États qu’une avidité sans bornes avait trop enrichis.\par
Le commerce, pour être utile, doit connaître des bornes et ne point nuire aux autres branches de l’administration. Rien de plus contraire au bien général que la passion de s’enrichir changée en épidémie. On voit quelquefois des nations, saisies de ce délire, négliger en sa faveur les objets les plus importants, recevoir leur principale impulsion de quelques marchands insatiables, se jeter pour leur complaire dans des guerres ruineuses, interminables, contracter des dettes immenses pour les soutenir et gémir ensuite pendant longtemps de leurs plus éclatants succès. Telle est, ô Bretons ! la cause de vos malheurs, de la misère que vous éprouvez malgré les richesses des deux mondes qui viennent sans interruption se rendre dans vos ports. Chez vous quelques négociants décident du sort de l’État, font entreprendre à tout moment des guerres insensées ; tandis qu’ils s’enrichissent, des impôts énormes accablent les autres citoyens et la nation épuisée se trouve dans la plus grande détresse. L’opulence de quelques individus ne prouve rien moins que l’opulence et l’aisance de l’État. Les dorures d’un palais ne l’empêcheront pas de tomber en ruines.\par
Le commerçant devrait chérir la paix et lui sacrifier sa propre avidité. Il est un très mauvais citoyen dès qu’il immole la félicité générale à ses vils intérêts. Un gouvernement sage, toujours guidé par la morale, doit contenir la passion des richesses, qui finit toujours par n’avoir plus de bornes ; il ne doit pas permettre qu’elle s’exerce aux dépens du cultivateur et du propriétaire, dont le négociant n’est fait que pour encourager les travaux. C’est l’intérêt du cultivateur qui constitue le véritable intérêt de l’État ; c’est lui que le législateur doit consulter préférablement à l’avarice de quelques marchands ou aux fantaisies indiscrètes de quelques opulents, qui jamais ne constituent la portion la plus nombreuse de la société. Enfin, tout nous prouve que la cupidité de l’homme doit être réprimée ; dès qu’on lui lâche la bride, elle anéantit les mœurs et la vertu. Les mœurs sont bien plus essentielles au bonheur d’une nation que des richesses qui rarement contribuent à sa force réelle, à son bien-être durable. Rome encore pauvre vint à bout de l’opulente Carthage.\par
La passion désordonnée de s’enrichir, devenue générale chez un peuple, y détruit communément le ressort de l’honneur, pour mettre en sa place un esprit {\itshape mercantile}, un amour sordide du gain directement opposée à tout sentiment noble et généreux. Possédé de cet esprit, le marchand ne rougit plus de rien dès qu’il peut en résulter du profit ; il ne connaît plus de patrie. Il fera, s’il y trouve quelque avantage, le commerce le plus contraire aux intérêts de sa nation ; enfin, accoutumé à regarder l’argent comme son idole, il s’y sacrifiera lui-même.\par
La vénalité n’est que le honteux trafic par lequel on consent à vendre son honneur, sa vertu, sa liberté, à celui qui veut les acheter. Ainsi que tous les excès, le commerce trop étendu finit par se punir lui-même. En augmentant dans un pays la masse des richesses, il augmente nécessairement le prix de toutes les denrées, par conséquent celui de la main-d’œuvre ou le salaire de l’ouvrier. Dès lors, les manufactures nationales perdent la concurrence avec celles des peuples moins riches qui travaillent à meilleur marché. D’ailleurs, c’est le propre des richesses de se concentrer dans les mains d’un petit nombre d’hommes qui ne souffrent pas de la cherté des denrées et marchandises ; mais l’ouvrier, l’artisan, l’homme du peuple souffrent de cette cherté et souvent périssent de faim à la porte du riche avare dont le cœur ne s’attendrit guère sur les besoins du malheureux. L’effet le plus commun de la richesse est d’endurcir le cœur.\par
Ainsi, la politique, toujours d’accord avec la morale, doit mettre un frein à la passion de s’enrichir, qui sans cela devient une contagion funeste à l’État. C’est de leur sol que les peuples doivent principalement faire sortir leurs richesses ; le commerce est fait pour en échanger le superflu contre les marchandises que ce sol ne peut pas produire. La terre est le fondement physique et moral de toute société. Le négociant est l’agent et le pourvoyeur du cultivateur, du propriétaire de la terre ; le fabriquant ou le manufacturier façonne les productions de la culture.\par
Tout ordre est renversé si les agents deviennent les arbitres et les maîtres de celui qu’ils doivent servir ; les mœurs se perdent quand ces agents le détournent de son travail par le luxe, par de vaines futilités, ou en lui faisant naître des besoins imaginaires qu’il ne peut satisfaire qu’aux dépens de ses mœurs et de son repos. Le commerce est utile, sans doute ; la politique doit le favoriser, la morale l’approuve, ceux qui le font sont des hommes utiles ; mais il doit avoir des bornes et ne point s’établir aux dépens des autres branches de l’économie politique. Le commerce n’est vraiment utile que lorsqu’il favorise l’agriculture, fait fleurir les manufactures, produit la population.\par
Dès qu’il nuit à ces objets essentiels, son utilité disparaît : il devient une manie funeste quand il ne sert qu’à faire éclore des guerres sanglantes et continuelles ; il est un dangereux poison quand il n’a pour but que d’alimenter le luxe et la vanité des hommes. Le négociant qui exporte les denrées superflues pour rapporter du blé, du vin, des huiles, de la laine ou d’autres denrées qui manquent à son pays est un citoyen très utile et mérite d’être considéré. Celui qui n’apporte à ses concitoyens que des objets capables d’allumer leurs passions, d’irriter leur vanité jalouse, d’exciter leur folie, est un homme dangereux. Presque tous les vains objets que l’Inde fournit à l’Europe n’ont de mérite que pour le caprice inconstant des femmes et la vanité de quelques hommes sottement dégoûtés des manufactures de leur pays. Les Européens ne se lasseront-ils jamais de sacrifier à des inutilités tant d’hommes et tant de sommes de cet argent qu’ils adorent\phantomsection
\label{footnote87}\footnote{On assure que le commerce des deux Indes coûte chaque année quarante mille hommes à la nation britannique. Le changement seul de climat est une cause de mort pour la plupart des Européens. Presque tout l’argent qui vient d’Amérique passe.} ? Toutes les futiles richesses que l’Europe va chercher aux extrémités du monde, sont-elles comparables aux trésors que l’agriculture pourrait tirer de son sol si elle était encouragée ?\par
Que dirons-nous de ce commerce affreux qui consiste à trafiquer du sang humain ? Acheter et vendre des hommes pour les jeter dans le plus dur esclavage est une barbarie qui fait frémir la justice et l’humanité. Mais l’avarice est cruelle de sang-froid ; elle réduit le crime en système, elle tâche de le couvrir du prétexte d’un grand intérêt national, et des nations affamées de richesses admettent ses excuses. Peuples avares et féroces ! Abandonnez l’Amérique, qui n’est pas faite pour vous, si vous ne pouvez la cultiver que par des forfaits odieux.\par
De pareils excès, si tous les commerçants s’en rendaient coupables, non seulement autoriseraient à les mépriser mais encore justifieraient la haine de tous les cœurs honnêtes. Mais distinguons ces affreux négociants de ceux qu’un commerce plus juste, plus légitime, rend utiles pour eux-mêmes et pour leur patrie. Ceux-ci, sans faire tort à personne, semblent mettre en commun les biens, les agréments, les découvertes de tout l’univers. En effet, la navigation et le commerce mettent en société tous les peuples de notre globe, établissent des rapports entre eux, les font jouir réciproquement d’un grand nombre d’avantages et servent surtout à étendre prodigieusement la sphère des connaissances humaines.\par
Si quelques nations ont cruellement abusé du commerce et, pour contenter leur avarice irritée, ont porté le carnage et le crime chez des peuples dont ils auraient dû s’attirer l’amitié, n’imputons point ces horreurs au commerce mais à l’ignorance, à la superstition farouche, qui rendirent en tout temps les hommes aveugles dans leurs passions et cruels sans remords. Les premiers conquérants de l’Amérique furent des brigands, des proscrits, des aventuriers que leurs crimes obligèrent de chercher fortune dans un autre monde, dont ils traitèrent les habitants de la façon que pouvaient faire des voleurs et des assassins.\par
Le vrai négociant, le commerçant estimable, est un homme juste. La probité, la bonne foi, l’amour de l’ordre, l’exactitude scrupuleuse à remplir ses engagements sont ses qualités distinctives. Une sage économie règle sa conduite ; l’on ne doit pas lui en faire un crime : c’est par elle qu’il peut garantir sa fortune, et souvent celle des autres, contre une infinité d’accidents que l’on ne peut ni prévenir ni prévoir. S’il n’y a qu’un insensé qui puisse légèrement hasarder son propre bien, il n’y a qu’un fripon qui puisse exposer la fortune des autres par des entreprises peu réfléchies. D’ailleurs, le négociant, étant un homme occupé, est communément à couvert des fantaisies, des passions et des vanités dont tant d’autres sont tourmentés. Tout commerçant éclairé est un homme d’honneur rempli de raison et de prudence ; jaloux de conserver l’estime qu’il a droit d’obtenir de ses concitoyens, il veut que sa réputation soit intacte, il a besoin de la confiance publique. Simple dans sa conduite et grave dans ses mœurs, il s’abstient des dépenses frivoles, du faste et des vices qui le conduiraient à sa ruine. Le négociant qui se livre aux extravagances du luxe finira communément par déranger ses affaires et ne ménagera pas avec plus de soin celles des imprudents qui lui ont accordé leur confiance. Les faillites si fréquentes et souvent si impunies que l’on voit arriver au sein des nations corrompues, annoncent une dépravation criminelle et déshonorante : ce sont des vols combinés avec la trahison et la perfidie. Le commerçant honnête et sage ne hasarde pas imprudemment son propre bien, et moins encore celui des autres.\par
Ainsi, ne confondons pas le vrai négociant, le commerçant estimable et prudent, avec ces hommes vicieux ou légers qui déshonorent une profession respectable : distinguons le pareillement de cette foule méprisable de trompeurs et de fourbes avides, qui, dépourvus d’éducation, de conscience et d’honneur, croient légitimes et permis tous les moyens de gagner, abusent indignement de la simplicité du public, ne se font aucun scrupule de surfaire et de tromper, soit sur la qualité, soit sur la quantité des marchandises. Des marchands de cette trempe sont bien coupables ; ils répandent sur le commerce un mépris qui ne devrait retomber que sur eux-mêmes.\par
La saine morale portera le même jugement de ces monopoleurs toujours prêts à profiter des calamités de leurs concitoyens, dont trop souvent ils sont les véritables auteurs. Il faut avoir des cœurs bien endurcis pour jouir tranquillement et sans pudeur d’une fortune acquise par la désolation publique ! Cette morale ferait en vain des reproches à ces traitants souvent si fiers qui négocient avec les despotes pour acheter le droit d’opprimer la société et de s’engraisser du sang des nations : des hommes de cette espèce sont des bourreaux privilégiés qui devraient rougir de la source impure d’une opulence fondée sur la ruine de la félicité générale. Il est pourtant des pays où ce trafic honteux n’est point déshonorant. Le financier enrichi par des extorsions est regardé comme un citoyen plus utile à l’État qu’il opprime, que le commerçant qui le fait prospérer.\par
Le vrai négociant, ainsi que le manufacturier, sont des êtres bienfaisants qui, en s’enrichissant eux-mêmes, donnent de l’activité, de la vie à toute la société, et par là méritent sa protection et son estime : ils font vivre et travailler le pauvre que le financier dépouille et réduit à mendier. Quelle foule innombrable d’artisans de toute espèce les manufactures et le commerce ne mettent-ils pas en mouvement ! Par eux il s’établit une liaison intime entre tous les membres de la société. En subsistant de son travail, l’artisan contribue sans relâche à la fortune de ceux qui l’emploient, ainsi qu’aux besoins, à la commodité, aux agréments, à la vanité même de ces riches ingrats qui le dédaignent en profitant de ses travaux, dont ils ne peuvent se passer un instant.\par
Rien de plus injuste et de plus bas que la manière insultante dont l’opulence altière regarde ces artisans qui chaque jour contribuent à lui fournir des besoins ou des plaisirs que sa faiblesse ne pourrait lui procurer. Cet artisan avili par la fierté dédaigneuse est pourtant un homme vraiment utile, doué quelquefois de talents rares {\itshape ;} et quand il est fidèle dans son travail, il est plus estimable que les fainéants qui le méprisent. Le souverain fastueux qui veut élever des monuments à sa vanité, n’a-t-il pas besoin du maçon, du charpentier, du serrurier et d’une foule d’hommes laborieux sans lesquels il ne pourrait se satisfaire ? Ces artisans divers ne sont-ils pas dignes d’estime, d’affection, de bienveillance, lorsqu’ils montrent du zèle dans leurs fonctions différentes ? Le monarque et le noble ne sont-ils pas forcés de recourir au manufacturier, au marchand pour meubler leur palais ? Ceux-ci mettent en jeu l’activité d’une foule d’hommes qui, du sein de l’indigence, contribuent à la magnificence des rois.\par
L’indigence, quand elle travaille, n’est jamais à mépriser. La pauvreté laborieuse est communément honnête et vertueuse ; elle n’est digne de mépris que lorsqu’elle se livre au désœuvrement et aux vices dont trop souvent l’opulence lui donne l’exemple. Ce sont très fréquemment les injustices et les mépris de la grandeur qui réduisent l’artisan au désespoir et au crime. De combien de forfaits, de vols, d’assassinats, ne se rendent pas complices tant de grands qui ont la cruauté de retenir le salaire de l’industrie laborieuse, du marchand qui les fournit, de l’artisan qui a travaillé fidèlement pour eux et qu’en récompense ils condamnent à mourir de faim ? Est-ce donc à des hommes de cette espèce qu’il appartient de mépriser d’honnêtes citoyens qui les ont bien servis ? L’opprobre et l’ignominie ne devraient-ils pas plutôt tomber sur ces ingrats assez cruels pour causer la ruine et le désespoir d’un grand nombre d’hommes qu’ils rendent inutiles ou dangereux pour la société ? Le voleur de grand chemin fait périr tout d’un coup celui qui a le malheur de tomber entre ses mains ; mais le voleur qui refuse de payer le salaire du pauvre le fait périr d’une mort lente avec sa famille entière. Les injustes mépris de la grandeur s’étendent, comme on l’a dit ailleurs, jusqu’au premier des arts, jusqu’à celui qui sert de base à la vie sociale. Par la plus étrange des folies, le riche méprise et dédaigne le laboureur, le cultivateur, le nourricier des nations, celui sans les travaux duquel il n’y aurait ni moissons, ni bétail, ni manufactures, ni commerce, ni aucuns des arts les plus indispensables à la société. N’apprendrez-vous jamais, ô riches stupides, et vous grands insensibles ! que c’est à l’agriculture que vous devez vos revenus, vos richesses, votre aisance, vos châteaux, ce luxe même dont l’ivresse vous étourdit ? Oui, c’est ce villageois dont les haillons et les manières vous dégoûtent, qui couvre vos tables de mets succulents, de vins délicieux. Ses brebis fournissent la laine qui vous habille, ses mains cultivent le lin pour vous si nécessaire ; sans lui vous n’auriez pas ces dentelles artistement tissées auxquelles votre vanité vous fait mettre un si grand prix : et vous avez pourtant l’audace de le mépriser !\par
La vie champêtre et le travail garantissent communément le cultivateur des vices et de la contagion dont les villes sont infectées : ce sont les injustices, les duretés et les désordres des riches qui corrompent son cœur et qui souvent altèrent l’innocence de ses mœurs. Les grands se plaignent fréquemment de la malice des paysans ; mais pour l’ordinaire c’est en eux-mêmes que ces hommes pervers devraient en chercher la cause. Perpétuellement dédaigné, opprimé, ravagé par la chasse et par des violences sans nombre, le paysan est forcé de haïr son seigneur, qui n’est communément pour lui qu’un tyran incommode. Le malheureux qu’un travail opiniâtre nourrit à peine, peut-il donc voir sans jalousie l’opulence nager dans l’abondance et le superflu, et rarement touchée de la misère du pauvre ? Enfin, l’éducation si négligée des habitants de la campagne, est-elle suffisante pour leur donner la force de résister aux impulsions, aux tentations, aux besoins même qui souvent les sollicitent au mal ? Les paysans ne sont voleurs, braconniers et fripons que parce que l’opulence les méprise, les maltraite et leur tend rarement une main secourable.\par
C’est ainsi que le défaut de reconnaissance, de justice et de bonté dans les riches et les puissants de la terre, anéantit la vertu dans les habitants des champs. Ceux-ci ne connaissent communément leurs supérieurs que par les vexations qu’on leur fait éprouver en leur nom. Si ces superbes seigneurs se montrent à leurs vassaux, ce n’est que pour les déprimer, les écraser, les fatiguer par leur luxe et leur vanité, les livrer aux outrages de leurs valets insolents. Faut-il être surpris que, d’après une conduite si révoltante, les riches ne trouvent dans les gens de la campagne que des envieux, des rebelles, des ennemis cachés toujours prêts à se venger des maux qu’on leur a faits ?\par
Tout est lié dans la vie sociale : c’est en rendant les grands meilleurs que l’on pourra corriger les petits. C’est en abolissant des lois gothiques, des privilèges injustes, des coutumes onéreuses, que l’on rappellera les uns et les autres à la vertu. Une bonne éducation surtout doit apprendre aux riches, aux nobles, aux puissants, qu’ils doivent se faire aimer de leurs inférieurs, qu’ils doivent se montrer reconnaissants pour les biens qu’ils en reçoivent, qu’ils ne peuvent s’acquitter envers eux qu’en leur montrant de l’équité, de la bienfaisance, de l’humanité.\par
Quand les grands de la terre seront imbus de ces maximes, ils cesseront de mépriser des citoyens dont l’existence est nécessaire à leur propre bonheur et sans lesquels ils ne jouiraient de rien. Ils sentiront ce qu’ils doivent à des hommes. Ils reconnaîtront que toute profession de laquelle la société recueille des fruits doit être plus estimée que celle qui ne produit aucuns biens désirables. Tout leur prouvera que ceux qui par divers moyens travaillent à leur procurer de l’aisance et des agréments ont droit à leur bienveillance, à leur affabilité.\par
Tout les convaincra que rien n’est plus contraire au but de la société que l’orgueil et la vanité. Enfin, tout leur fera voir que le vice seul déshonore et peut rendre méprisable, et que tout homme qui remplit fidèlement les devoirs de son état est digne des égards de ses concitoyens.\par
En se conformant dans leur conduite à des principes si clairement démontrés, les nobles et les opulents trouveront dans leurs inférieurs des dispositions plus favorables, des mœurs plus honnêtes, un attachement plus sincère, moins d’envie ou de malignité ; enfin, ils obtiendront d’eux ce dévouement, cette soumission du cœur que n’obtient jamais la crainte. Il n’est point d’hommes assez sauvages pour que la bonté ne parvienne pas à les toucher. Par une pente naturelle, les hommes sont portés à chérir ceux qu’ils sont accoutumés à respecter.\par
C’est toujours par la faute des grands qu’ils ne sont point aimés de ceux qui leur sont subordonnés. C’est en se rapprochant de ses vassaux qu’un noble deviendrait leur père, s’en ferait obéir et considérer, mériterait leur tendresse, sentiment que la hauteur ou la force ne peuvent point arracher.\par
Mais depuis longtemps les extravagances et les plaisirs bruyants du luxe ont attiré dans les villes ceux que leur état et leur fortune destinaient à être les protecteurs des habitants de la campagne et les soutiens de l’agriculture. Les vassaux sont devenus des étrangers pour leurs seigneurs ; ceux-ci, voulant paraître avec faste à la cour et dans la capitale, laissent honteusement dépérir les terres que leur présence pourrait fertiliser.\par
La vie champêtre et sa paisible uniformité sont odieuses à des êtres dont le fracas du vice est devenu l’élément. Le cultivateur n’a plus d’amis puissants ni de consolateurs dans ses peines. Le fermier est durement renvoyé à des gens d’affaires que les besoins multipliés du propriétaire rendent impitoyables. Bientôt la culture est abandonnée ou la terre ne fournit plus que de faibles moissons ; les villages désertés ne présentent que des solitudes et le chef lui-même se trouve endetté ou ruiné, méprisé de ceux mêmes qui ont le plus contribué à déranger sa fortune.\par
Tel est le sort que trop communément le luxe et la vanité préparent à ceux qu’ils parviennent à séduire. C’est aux champs que le noble serait vraiment respectable et puissant ; en demeurant dans ses terres, il conserverait sa fortune et ses mœurs, il se garantirait de l’air contagieux qu’on respire dans les cours.\par
En faisant travailler, il trouverait des moyens d’augmenter son aisance et celle des autres, plaisir plus solide et plus innocent que ceux du vice, que suit toujours la ruine et le repentir\footnote{La loi de Zoroastre met au nombre des plus grandes vertus de {\itshape semer les grains avec pureté, et de planter des arbres}. En effet, c’est pratiquer la vertu que d’être utile au public. D’après ces principes, défricher des terres, dessécher des marais, faire des chemins, établir des manufactures, etc., en un mot, faire travailler et subsister des hommes, sont des actions plus vertueuses que bien des pratiques auxquelles on attache vulgairement l’idée de vertu. Faire travailler est la meilleure des aumônes.}. C’est ainsi que tant de riches qui ne savent que dissiper sans profit ni pour eux-mêmes ni pour la société, se rendraient des citoyens utiles, chéris de leurs vassaux, dignes d’être considérés.\par
Ce qui a été dit dans toute cette section continue à nous prouver de la façon la plus claire que la politique ne peut jamais sans danger séparer ses maximes de celles de la morale. Les différents états ne sont que des moyens divers de servir la patrie ; la profession la plus noble est celle qui la sert le plus utilement. Dès que l’administration s’écarte de ces principes, tout tombe dans le désordre et la confusion. Un peuple sans probité devient le fléau des autres et se détruit bientôt lui-même. Un souverain sans justice est la ruine de son empire et n’exerce jamais qu’une puissance peu sûre. Les grands, les nobles, les magistrats, les prêtres, les riches ne peuvent être justement considérés qu’en tant qu’ils se montrent occupés de la félicité publique. Les sciences et les lettres ne méritent notre estime que lorsqu’elles éclairent la société sur les objets qui l’intéressent. Le commerce ne peut fleurir sans bonne foi. Enfin, l’agriculture, si nécessaire à la société, exige la protection et les secours des riches et des grands et, dûment encouragée, elle devient le soutien des bonnes mœurs.\par
Qu’est-ce donc qui empêche les citoyens des différentes classes de l’état de concourir fidèlement au but de la vie sociale ? C’est l’ignorance, qui fait que chacun d’entre eux ne voit pas assez clairement la liaison de son intérêt personnel avec l’intérêt de tous les autres. C’est une sotte vanité qui, enivrant les grands de folles chimères, leur fait croire que pour être heureux ils n’ont besoin de personne, erreur fatale à laquelle on peut attribuer ces divisions, ces haines et ces mépris réciproques, cette séparation d’intérêts que nous voyons subsister dans presque toutes les sociétés. C’est sur la vanité des hommes que la morale doit frapper lorsqu’elle voudra les ramener à l’union si nécessaire à la force, à la félicité des nations.\par
Aucun homme, aucun corps, aucun ordre de l’État n’est en droit de s’estimer qu’en vertu des avantages véritables dont il fait jouir la patrie.
\section[{Section V. Des Devoirs de la Vie privée}]{Section V. Des Devoirs de la Vie privée}\renewcommand{\leftmark}{Section V. Des Devoirs de la Vie privée}

\subsection[{Chapitre I. Devoirs des Époux}]{Chapitre I. Devoirs des Époux}
\noindent Nous avons examiné dans la section précédente les devoirs des personnes qui ont des rapports généraux et directs avec la société, ou de celles dont les fonctions et les facultés influent d’une façon plus ou moins marquée sur tous les citoyens. Nous allons considérer dans la section présente les devoirs résultants des rapports particuliers ou des liaisons plus intimes qui forment la vie privée. Et nous commencerons par les devoirs des époux.\par
Pour découvrir les devoirs de l’homme dans chaque état de la vie, il suffit d’examiner le but qu’il se propose dans l’état qu’il a choisi. Le mariage est une société entre l’homme et la femme dans laquelle les époux ont pour but de goûter légitimement les plaisirs de l’amour, d’où doivent résulter des êtres utiles à ceux qui leur ont donné l’existence et propres à les remplacer un jour dans la société.\par
Tel est le but que les hommes se proposent dans l’union conjugale ; les devoirs attachés à cet état en découlent nécessairement. Des êtres qui s’associent ne s’unissent que pour se procurer un bien-être dont ils seraient privés s’ils demeuraient séparés. Leurs engagements sont semblables parce que nul être n’en peut lier un autre sans lui être lié par des nœuds aussi forts. Toute société, pour être heureuse et stable, doit être soumise aux règles de l’équité ; cette équité, comme on a vu, remédie à l’inégalité que la Nature a mise entre les associés.\par
Dans toutes les nations l’homme fut reconnu pour le chef de la société conjugale et l’autorité sur la femme lui fut déférée : la supériorité du premier paraît même fondée sur la Nature : l’homme étant plus robuste, doit être le protecteur et le soutien de sa compagne et lui prescrire la subordination\footnote{Indépendamment de la faiblesse qui se montre dans les femmes, elles sont assujetties par la Nature à des infirmités que l’on peut regarder comme de vraies maladies, qui les affligent au moins pendant un quart de l’année.}. L’autorité maritale, ainsi que toute autorité sur la terre, n’est fondée que sur les avantages que l’époux est capable de procurer à celle avec qui son sort est lié. Si des lois injustes ou des usages peu raisonnables ont adjugé chez quelques peuples au mari un pouvoir illimité, s’il s’est trop souvent arrogé le droit d’exercer un empire trop dur, l’équité naturelle condamne ces usages et ces lois, met au néant ces droits comme évidemment usurpés, et, d’accord avec l’humanité, elle annonce aux époux que l’autorité déférée par la Nature à l’homme, loin de lui donner le pouvoir d’opprimer ou de maltraiter sa femme et d’en faire une esclave, l’oblige à l’aimer, à la défendre, à la garantir des dangers auxquels sa faiblesse la forcerait de succomber\footnote{Ceux qui nous vantent l’innocence et le bonheur de la vie des sauvages n’ont qu’à lire les relations des voyageurs pour se convaincre que leurs mœurs, loin d’être dignes d’envie, sont faits pour révolter toute âme sensible. Les sauvages traitent entre autres leurs femmes avec une cruauté, une tyrannie qui fait frémir : ils forcent ces malheureuses à s’occuper des travaux les plus pénibles tandis qu’ils se livrent à l’indolence. Dans la Guyane et sur les bords de l’Orénoque, le sauvage se met au lit lorsque sa femme est accouchée, et cette malheureuse est obligée de soigner son mari comme s’il était malade. Dans ce même pays, les mères, par pitié, sont dans l’usage de faire périr les filles qu’elles mettent au monde afin de leur épargner les peines et les chagrins dont leur sexe est menacé. Dans tout l’Orient les femmes sont renfermées et traitées en esclave. En un mot, presque en tout pays les lois, trop partiales pour les maris, leur donnent sur leurs femmes un pouvoir dont souvent ils abusent. Les vices et les défauts que l’on reproche aux femmes sont dus, en grande partie, à l’inégalité trop grande que les lois mettent entre elles et leurs superbes maîtres.}.\par
D’après ces principes incontestables, on voit que la Nature elle-même a fixé les limites de l’autorité d’un mari sur sa femme, et semble leur avoir prescrit à tous deux la tâche qu’ils ont à remplir dans la société conjugale. La protection, la vigilance, la prévoyance, les travaux les plus pénibles sont échus au mari, qui doit aimer sa femme, lui donner son appui et ses soins, soutenir sa faiblesse, et non pas en profiter pour la rendre malheureuse. Tout homme sensé veut rencontrer dans sa compagne un attachement habituel qui ne peut être que le fruit de l’affection qu’il lui montre. En échange de sa protection, de sa tendresse et de ses soins, la femme est obligée de lui marquer une juste déférence, une amitié tendre, des soins empressés faits pour cimenter de plus en plus leur union. D’où l’on voit que les devoirs des époux sont réciproques, c’est-à-dire lient également le mari et la femme ; ils les obligent sous peine de relâcher ou de briser des nœuds contractés pour leur bonheur mutuel. Telle est la sanction de la loi naturelle, à laquelle on ne peut se soustraire impunément. Il ne suffit pas à l’homme d’avoir donné le jour à des êtres de son espèce ; il faut encore, pour son bonheur, que ces êtres soient façonnés de manière à devenir les coopérateurs de sa félicité, les soutiens de sa vieillesse. Il a besoin de sa compagne pour élever leur enfance, pour les allaiter, pour leur apprendre à bégayer le doux nom de père. Il n’obtiendrait pas le but qu’il se propose si, semblable aux brutes, il ne songeait qu’à satisfaire en passant, avec une femelle quelconque, les besoins que la Nature lui fait éprouver. Tout lui montre qu’une femme à laquelle il ne tiendrait que par le lien du plaisir ne lui serait pas fermement attachée et pourrait également se livrer aux désirs de ceux qui la solliciteraient pareillement de contenter des besoins passagers. Perpétuellement entraînée par le goût de la volupté, elle ne se chargerait guère du soin pénible d’élever des enfants dont le sort l’intéresserait faiblement. D’ailleurs, des femmes abandonnées au premier venu ou sur lesquelles tous les citoyens auraient des droits égaux, ne manqueraient pas de faire naître des querelles, des rivalités, des combats funestes à la tranquillité publique.\par
L’amour, dans un être intelligent, prévoyant, raisonnable, ne doit point être traité à la façon des brutes. Celles-ci, en se propageant, ne cherchent qu’à satisfaire un besoin momentané ; leur union ne dure que jusqu’à ce que leurs petits soient en état de se passer de leurs soins. Mais l’homme, en cherchant le plaisir dans le mariage, porte encore ses vues plus loin : il veut posséder sa compagne exclusivement, non seulement parce que le besoin de l’amour se renouvelle en lui, mais encore parce qu’il a le besoin continuel de posséder un être qui contribue à lui rendre la vie douce par des dispositions étrangères à l’amour. Il veut donc trouver dans sa femme une amie constante et fidèle qui, indépendamment des plaisirs qu’elle procure à ses sens, soit disposée à lui faire goûter les plaisirs continus et durables de l’amitié, de la consolation, de la complaisance ; en un mot, il souhaite de se lier solidement avec un être sensible qui, après avoir partagé avec lui les agréments et les peines de la vie, continue à lui donner des soins dans sa vieillesse et dans ses infirmités. Il ne pourrait atteindre ce but désirable si, fermant les yeux sur l’avenir, il ne pensait qu’à satisfaire ses besoins momentanés avec une femme quelconque. Il doit donc désirer une union stable et permanente, propre à calmer son esprit par l’assurance des autres avantages dont il veut être à portée de jouir pendant le cours de sa vie. Cette union ne doit être dissoute que lorsque les époux sont animés d’une antipathie totalement contraire au but du mariage ; il ne peut lier pour la vie que des époux vertueux et raisonnables constamment disposés à remplir les engagements que leur pacte leur impose. Toute société qui n’apporterait que des chagrins et des peines à ceux qu’elle engage, devrait être rompue par la nature même des choses.\par
Ces réflexions peuvent nous mettre à portée de juger sainement les coutumes, les institutions et les lois observées chez les différentes nations relativement au mariage. Elles nous prouvent que l’union conjugale est le plus respectable des liens, le plus intéressant et pour ceux qu’il unit, et pour toute la société. Elles nous font voir que les époux ne doivent pas seulement se proposer d’assouvir leurs besoins et d’obéir à la volupté, mais qu’ils doivent encore songer aux jouissances plus durables que procurent la tendresse, la confiance, la cordialité. Nous dirons donc que tout ce qui contrarie ce but doit être condamné, que les préjugés, les mœurs et les lois qui tendraient à relâcher des nœuds si doux, sont faits pour être blâmés par tout homme raisonnable. Nous dirons que les peuples chez lesquels la corruption épidémique fait regarder l’adultère, la galanterie, la coquetterie, comme des choses indifférentes ou des bagatelles, n’ont aucune idée de la sainteté du mariage et du respect qui lui est dû. Nous dirons que les législateurs et les prétendus sages qui ont autorisé la polygamie, la prostitution, la communauté des femmes, ont été des insensés qui n’ont pas vu que leurs institutions anéantissaient le bonheur des époux et devenaient préjudiciables à la société.\par
En effet, n’en déplaise au {\itshape divin} Platon, des femmes communes à tous ne seraient véritablement estimées ni aimées de personne ; elles ne seraient d’ailleurs ni des compagnes attachées, ni des mères tendres et soigneuses : ce ne seraient que de viles prostituées. Enfin, tout est fait pour nous convaincre qu’un amour sans règles deviendrait un désordre capable de saper la société jusque dans ses fondements.\par
La polygamie, adoptée ou permise dans quelques nations, est, d’après la nature même des choses, un abus tyrannique introduit par une luxure effrénée et justement proscrit par des lois plus raisonnables. Une seule femme doit suffire aux besoins de tout homme qui n’est pas un débauché. Un mari peut-il donc partager son cœur également entre plusieurs femmes à la fois ? Ne rend-il pas malheureuses toutes celles qu’il néglige ? Son sérail ou son {\itshape harem} ne sont-ils pas exposés à des troubles continuels ? D’un autre côté, ce tyran peut-il être sincèrement aimé par des captives dont il est le geôlier et qu’il ne regarde que comme les instruments de son plaisir brutal ? Les sérails d’Orient ne sont remplis que d’esclaves dépourvues de sentiments, de raison et de mœurs, dont la sagesse ne tient qu’à des verrous ; la vertu, les sentiments du cœur peuvent seuls répandre des charmes sur les nœuds du mariage.\par
La saine morale n’approuvera pas davantage les maximes d’une morale lubrique et corrompue qui prétend justifier l’infidélité conjugale, ou du moins atténuer l’horreur qu’elle devrait inspirer. Si ces principes conviennent aux mœurs dépravées de quelques nations, ils sont évidemment contredits par la nature même du mariage, dont le bonheur dépend de l’union, de l’amitié, de l’estime, encore bien plus que des plaisirs passagers qu’il procure. Tout s’accorde à nous montrer que l’adultère est propre à bannir sans retour ces sentiments désirables et que rien ne peut justifier un crime qui doit par son essence anéantir le plus sacré des nœuds.\par
De quelque côté que vienne l’infidélité, elle est également condamnable. Un mari, parce qu’il est le plus fort, acquiert-il donc le droit d’être injuste envers celle à qui il doit exclusivement son amour et ses soins ? Si la femme est déshonorée aux yeux du public pour avoir violé les règles de la pudeur, pourquoi le mari coupable du même crime lève-t-il sa tête altière au milieu d’un public partial qui n’ose lui imprimer l’opprobre qu’il mérite ? Quelle étrange jurisprudence donne au mari la liberté de commettre impunément des injustices qu’il a le droit de punir avec rigueur lorsque sa femme se permet la même chose ? La faiblesse d’une femme donne-t-elle à son tyran le droit exclusif de lui ravir son cœur et de violer la foi qu’il lui avait jurée ? Gardons-nous de le croire ; les fautes d’un mari, à qui l’on doit supposer plus de force, de raison, de prudence, sont plus impardonnables que celles d’une femme dont la faiblesse est le partage. « Il y a, dit Plutarque, des maris assez injustes pour exiger de leurs femmes une fidélité qu’ils violent eux-mêmes ; ils ressemblent à ces généraux d’armée qui, fuyant lâchement devant l’ennemi, veulent pourtant que leurs soldats soutiennent ses efforts avec courage. »\par
C’est trop communément à la conduite injuste des maris, à leur inconstance, à leur vie déréglée, à leurs mauvaises manières, que l’on doit imputer les faiblesses de leurs femmes : il faudrait en effet supposer en elles une force et une grandeur d’âme bien rares si, trop souvent dédaignées, rebutées, outragées par des tyrans féroces, elles ne prêtaient jamais l’oreille aux discours des séducteurs, autant soumis, respectueux, complaisants, que leurs maris le sont peu. Un tyran n’est point fait pour fixer le cœur d’une femme ; en portant à d’autres la bonne humeur, les douceurs, l’amour qu’il lui doit, ne semble-t-il pas l’inviter à suivre son exemple ? Il faudrait du moins bien plus de vertu que l’on n’en rencontre dans des nations viciées pour qu’une infortunée accablée de chagrin et souvent baignée dans ses larmes se refusât aux consolations de celui qui met tout en œuvre pour lui faire oublier son devoir.\par
Nous voyons presque en tout pays l’opinion publique imprimer une sorte de honte ou de ridicule aux maris dont les femmes sont infidèles. Quoiqu’au premier coup d’œil cette façon de penser paraisse injuste, et le soit très souvent, quoiqu’elle semble blesser l’humanité, qui veut que l’on plaigne les malheureux, on pourrait néanmoins trouver à cette façon de penser un motif raisonnable. Le préjugé qui rend un mari responsable de la conduite de sa femme ne pourrait-il pas venir de ce que l’on a pensé qu’il n’y avait que la négligence, l’inconduite, les défauts ou les vices révoltants du premier qui pussent être la cause des dégoûts d’une femme, qu’il aurait dû contenir par sa vigilance, par son exemple et par son autorité ? L’opinion qui, souvent très mal à-propos, déshonore un mari dont la femme est sans mœurs, paraîtrait donc être de la même nature que celle qui rend un père responsable des désordres ou des crimes de son fils : l’on a pu croire que sans des qualités méprisables ou incommodes dans le mari, une femme honnête et bien élevée ne se serait jamais portée à des excès qui la déshonorent.\par
Quoi qu’il en soit de cette opinion défavorable au mari, la raison nous prouvera toujours que l’infidélité conjugale est un mal que la morale ne peut point traiter légèrement. Ce qui tend évidemment à faire disparaître la félicité domestique, la concorde, l’estime et la tendresse d’entre les époux, est une chose que la seule folie puisse faire regarder comme indifférente. En supposant même que de part et d’autre des époux s’accordassent à ne point se troubler dans leurs désordres, il en résultera toujours que la confiance et l’amitié sont totalement étrangères pour des êtres capables de prendre de pareils arrangements. D’ailleurs, le dérèglement des pères et mères n’est-il pas fait pour influer de la façon la plus fâcheuse sur les mœurs des enfants ? Nés de parents vicieux qui se méprisent ou se détestent, ces enfants recevront une éducation capable de les rendre à jamais malheureux. Quels citoyens peuvent former à la société des époux en discorde ou qui ne sont d’accord que dans leurs dérèglements ?\par
En général, l’homme est jaloux, il veut posséder sans partage ce qui lui appartient ; bien plus, il désire d’être aimé de ceux mêmes qu’il n’aime que faiblement. Les époux qui consentent à leurs infidélités mutuelles annoncent très clairement qu’il n’existe plus dans leurs âmes la moindre étincelle de l’attachement si nécessaire à leur état, ou qu’une affreuse antipathie a détruit en eux les sentiments les plus naturels. Cette haine ou cette indifférence doivent s’étendre sur des enfants dans lesquels un mari doit craindre de ne voir que les fruits des amours déshonnêtes de sa femme. Comment accorderait-il des soins paternels, une tendresse véritable, à des êtres qu’il peut supposer ne lui tenir par aucun lien ?\par
La raison nous montre que dans l’union conjugale le mari appartient à sa femme, de même que la femme appartient à son mari. L’un et l’autre ne peuvent, sans risquer leur bien-être, renoncer aux droits de cette propriété réciproque. L’un et l’autre doivent éviter avec soin ce qui peut altérer l’harmonie nécessaire à leur félicité domestique, que rien au monde ne pourra remplacer.\par
D’après ces principes, la coquetterie dans une femme est une disposition à laquelle la morale ne peut aucunement conniver : elle annonce une vanité méprisable, un désir de faire naître des passions déshonnêtes, afin d’exercer un despotisme auquel une femme vertueuse ne doit pas prétendre. N’est-ce pas un crime que d’allumer des feux criminels dans des cœurs qui ne doivent point les éprouver ? N’est-ce pas une cruauté que d’exciter des désirs dans l’espérance de faveurs que l’on ne peut ni ne veut point accorder ? N’y a-t-il pas de l’imprudence et de la légèreté à donner, soit au public qu’on doit respecter, soit à son époux dont on doit ménager la délicatesse, des soupçons capables de se déshonorer soi-même ? Sous quelque point de vue que l’on envisage la coquetterie, elle décèle toujours des dispositions très blâmables. Elle marque une volonté permanente de troubler la félicité des autres, elle indique une légèreté condamnable dans une matière importante, elle annonce une vanité que rien ne peut justifier. Une femme qui veut plaire à tout le monde, quand elle aurait le cœur pur, a du moins l’esprit gâté. Une femme vraiment honnête ne veut plaire qu’à son mari ; une femme vraiment sensée évite tout ce qui pourrait lui faire ombrage, parce qu’elle sait que son bonheur dépend des sentiments qu’elle trouvera dans son cœur. L’estime, la paix, la confiance sont des dispositions permanentes bien plus nécessaires au bonheur des époux que l’amour, sujet à s’exhaler dès qu’il est satisfait.\par
L’amour dans les deux sexes est, comme on l’a dit ailleurs, une passion naturelle excitée par le tempérament et nourrie par l’imagination, qui sollicite plus ou moins vivement les deux sexes à s’unir dans la vue de se procurer les plaisirs attachés à cette union. La beauté du corps fait pour l’ordinaire éclore subitement cette passion ou ce désir. Dans le choix d’une femme, la figure est souvent la première qualité à laquelle on s’arrête ; elle n’est, sans doute, aucunement à négliger. Mais comme l’expérience nous prouve que l’amour est une passion peu durable, que la possession le fait très promptement disparaître, la prudence et la prévoyance doivent faire sentir à ceux qui veulent s’unir, qu’il est des qualités plus solides que la beauté que l’on doit chercher dans le mariage. La beauté fut souvent comparée à une fleur passagère, et l’amour au papillon léger. La femme la plus belle devient en peu de temps une femme très ordinaire aux yeux du mari qui l’avait adorée\footnote{« Les Espagnols disent que la beauté est comme les odeurs, dont la force est de peu de durée, après quoi on s’y accoutume et on ne les sent plus. » Voyez {\itshape Réflexions sur les Femmes}, par Mme de Lambert. Bion le Borystenire disait que « la femme laide fait mal aux yeux, et que la belle fait mal à la tête ».}. {\itshape La beauté}, disait Socrate, {\itshape est une tyrannie de courte durée}.\par
Rien de plus rare que de voir réussir les mariages qui n’ont eu que l’amour aveugle et la beauté pour motifs. Les passions violentes n’ont que peu de durée : l’imprudence des époux enivrés leur fait bientôt abuser des plaisirs qu’ils auraient dû sagement économiser. Le mariage doit être chaste ; {\itshape la pudeur}, dit Madame de Lambert, {\itshape doit être conservée dans le temps même destiné à la perdre}. Les époux doivent respecter les liens sacrés qui les unissent et ne jamais se permettre la licence, presque toujours suivie du dégoût. D’ailleurs, un mari sage doit craindre d’allumer dans l’imagination d’une femme un goût pour des voluptés qu’elle ne pourrait satisfaire qu’aux dépens de sa vertu. Plutarque nous apprend que les Grecs avaient élevé un temple à Vénus {\itshape voilée} ; sur quoi il observe qu’on ne peut entourer cette déesse de trop d’ombre, d’obscurité et de mystères.\par
L’effet de la beauté est d’exciter des désirs ; elle expose communément les femmes à des séductions et à des dangers. Antisthène, consulté par un jeune homme sur le choix d’une femme, lui répondit : « Si vous la prenez très belle, vous ne la posséderez pas tout seul ; si vous la prenez trop laide, vous vous en dégoûterez promptement : il vaut donc mieux pour vous qu’elle ne soit ni trop belle, ni trop laide. » Les qualités du cœur, les agréments de l’esprit, la douceur, la sensibilité sont des dispositions que la raison nous dit de préférer soit à la beauté, sujette à se flétrir, soit aux richesses, incapables de remplacer la vertu et de procurer un vrai bonheur à des époux, surtout quand ils ignorent la façon de s’en servir. La beauté, disait un ancien sage, est le bien d’autrui. En effet, comme dit Juvénal, il est rare de rencontrer la pudeur et la beauté réunies dans un même sujet\phantomsection
\label{footnote88}\footnote{« Rara est adeo concordia formæ atque pudicitiæ. » Juvénal, {\itshape Satires}, X, vers 297.}. Les charmes de la figure, qui par un effet naturel saisissent et frappent ceux qui les considèrent, empêchent très souvent une femme de cultiver ou d’acquérir les dispositions les plus nécessaires à la félicité conjugale. Une belle femme n’est pas la dernière à s’apercevoir du pouvoir de ses charmes : cette idée la rend vaine. Elle est communément trop occupée d’elle-même pour songer au bonheur des autres ; elle s’aime exclusivement, elle a l’ambition d’exercer son empire, il lui faut une cour. Idolâtre d’elle-même, elle veut être adorée ; elle est perpétuellement entourée d’ennemis qui, sans cesse occupés à lui plaire, conspirent contre son cœur, que sa vertu n’est guère en état de défendre. Rien de plus rare qu’une femme d’une grande beauté qui ne se croie point dispensée de montrer à son mari l’attachement et les soins que son état lui prescrit. Accoutumée à régner, elle consent rarement à se prêter aux volontés de celui à qui elle doit de la déférence et des complaisances : son empire finit en présence de l’époux ; conséquemment, elle ne tarde point à le fuir, à le haïr, et souvent à se livrer à quelque adorateur soumis qui bientôt règne en maître.\par
Ainsi, cet empire qui paraît si flatteur à la vanité des femmes, n’a nulle solidité ; elles finissent le plus souvent par être méprisées de ceux même à qui elles font les plus grands sacrifices. Mais leur sort devient plus déplorable encore quand leurs appas flétris ne leur permettent plus de jouer un rôle dans la société. Abandonnées pour lors de leurs esclaves affranchis, vous les voyez communément livrées à une sombre mélancolie. Une dévotion chagrine est une faible ressource pour remplacer les plaisirs auxquels elles s’étaient accoutumées ; elles vivent dans l’oubli et passent leurs tristes jours à regretter un pouvoir anéanti. Tel est le sort de ces imprudentes que le vice a dégradées. La vertu seule donne des droits imprescriptibles, une puissance que rien ne peut ébranler. « Le règne de la vertu est pour toute la vie. Il y a peu de temps à être belle, et beaucoup à ne l’être plus… Des mœurs pures, un esprit juste et fin, un cœur droit et sensible sont des beautés renaissantes et toujours nouvelles\footnote{{\itshape Réflexions sur les Femmes}. Solon voulut qu’une nouvelle mariée mangeât quelques fruits de bonne odeur avant d’habiter avec son mari, pour apprendre qu’elle devait toujours lui parler avec douceur et se rendre agréable.}. » Elles sont faites pour fixer la tendresse et l’amitié de tout mari sensé et pour attirer à tout âge l’admiration et les respects des autres, sentiments plus durables et plus flatteurs que les fleurettes dont se repaît une sotte vanité.\par
Nonobstant les opinions reçues parmi des nations sans mœurs, la morale ne cessera de répéter aux maris d’être justes, de ne point se prévaloir de leur autorité pour exercer la tyrannie sur des êtres pour qui leur faiblesse même devrait intéresser. Elle leur dira d’aimer leurs femmes, de ne point rougir aux yeux du public d’un attachement qui doit les rendre estimables aux yeux des personnes sensées : leur suffrage est sans doute préférable à celui d’un tas de libertins qui n’ont aucune idée ni de l’importance ni de la sainteté des nœuds faits pour unir les époux. Le mari qui se rend le tyran de sa femme est un lâche, un homme sans cœur, un barbare dont les lois devraient châtier la férocité. Tout époux infidèle qui prive sa femme des marques de sa tendresse est un homme injuste qui, en lui ravissant la récompense qu’il doit à sa vertu, semble l’inviter au désordre.\par
Il n’est point de vice qui dans une société corrompue ne trouve des apologistes, il n’est point de désordre que des exemples fréquents ne semblent ennoblir ou du moins justifier. Cependant, nul exemple criminel ne peut autoriser le crime\footnote{« Nulli unquam vitio advocatus defuit. » Cicéron.}. La raison ne cessera donc de représenter à une femme que son intérêt le plus cher est de ménager la tendresse de celui que la Nature et les lois rendent l’arbitre de son sort. Cette raison lui recommandera de le ramener à son devoir par une grande indulgence, d’opposer la patience à son délire, de le forcer de rougir de ses injustices et de ses mépris. La patience et la douceur ont quelque chose de sublime et d’imposant pour le vice lui-même. Quelle supériorité une femme vertueuse ne prend-elle pas sur un homme dépourvu de raison et de mœurs ! Est-il rien de plus noble et de plus généreux qu’une beauté que les dérèglements de son mari ne peuvent écarter du sentier de la vertu !\par
Une femme qui par des infidélités se venge des outrages qu’elle reçoit de son époux, est sans doute moins coupable que celle qui la première provoque sa colère et sa jalousie par une conduite déréglée ; cependant, elle pèche toujours contre ses propres intérêts, elle ne fait qu’augmenter la discorde, elle se prive de la considération d’un public qui, malgré la dépravation générale des mœurs, veut toujours que la vertu ne se démente pas au milieu des épreuves.\par
La force, la grandeur d’âme sont des qualités tellement admirées qu’on désire de les trouver même dans le sexe le plus faible. Quoiqu’au premier coup d’œil ce sentiment paraisse injuste, il est pourtant fondé : on suppose qu’une femme bien élevée doit avoir de la fermeté quand il s’agit de la pudeur, dans laquelle dès l’enfance on lui apprend à faire consister son honneur et sa gloire. L’on croit que, parvenue une fois à franchir cette barrière que l’éducation avait pris soin de fortifier, il n’en est plus d’assez puissante pour la contenir dans les choses les plus importantes de la vie.\par
En effet, si par un hasard peu commun quelques femmes, nonobstant leurs faiblesses, conservent encore les vertus sociales, ces vertus sont anéanties dans la plupart de celles qui ont franchi les limites de l’honneur. On les voit pour l’ordinaire dépourvues de franchise, perpétuellement occupées à tromper, se faire une habitude du mensonge, de la trahison, de la fausseté. Rien de moins sûr que le commerce de la plupart des femmes galantes, dont la vie ne devient le plus souvent qu’une intrigue continue, une imposture perpétuelle. Toute conduite qui doit être cachée demande une vigilance, un manège et des soins incroyables pour se soustraire à la censure médisante. D’ailleurs, le goût de la débauche oblige la femme qui s’y livre à tromper la foule de ceux dont elle reçoit les hommages. Enfin, toute femme corrompue, pour avoir des complices, cherche à corrompre les autres.\par
Joignez à ces dispositions dangereuses dans le commerce de la vie, la longue suite d’extravagances dans lesquelles une femme galante est continuellement entraînée : toute occupation utile lui paraît odieuse, sa maison lui devient insupportable. Il lui faut un tourbillon, une dissipation perpétuelle pour l’étourdir sur les reproches de sa conscience et sur ses chagrins domestiques. Ses folles dépenses se multiplient ; les enfants équivoques qu’elle donne à son mari sont totalement négligés ; ils n’éprouvent jamais les caresses ou les tendres sollicitudes d’une mère évaporée, que d’ailleurs ses vices rendraient totalement incapable de leur former le cœur et l’esprit.\par
Des époux désunis par le caractère ou par le vice ne peuvent pas mettre dans l’éducation de leurs enfants cet accord, cette heureuse harmonie des sentiments et des préceptes nécessaire pour les faire fructifier. Si l’un des parents est vertueux, l’imprudence, l’humeur et l’exemple de l’autre rendront à tout moment ses leçons inutiles. Un père déréglé peut frustrer par son exemple tous les soins de la mère la plus tendre. Une femme légère, vaine et sans conduite, peut déranger à chaque instant tous les projets d’un mari raisonnable sur ses enfants.\par
Voilà comment les désordres des époux, après avoir banni d’entre eux la concorde, influent encore de la façon la plus terrible sur leur postérité ; celle-ci, destituée d’instructions et de bons exemples, ne manquera pas d’imiter à son tour les dérèglements qu’elle a vu pratiquer à ses parents. Tels sont les effets déplorables que produisent dans la société la galanterie, la coquetterie, les infidélités, que quelques moralistes relâchés ont traitées avec tant de légèreté, tandis que l’on en voit à tout moment résulter des mariages malheureux, des fortunes dissipées, des enfants qui se trouvent corrompus dès l’âge le plus tendre.\par
Ces effets doivent être attribués à l’imprudence avec laquelle les mariages sont communément contractés. Si c’est l’amour aveugle qui forme les nœuds des époux, cet amour enivré par la beauté ne songe aucunement aux qualités de l’esprit ou du cœur si nécessaires pour rendre ces nœuds durables. Désenchantés par la jouissance, les époux ne tardent pas à se voir tels qu’ils sont et se deviennent incommodes par des défauts qui, à la longue, les rendent réciproquement insupportables.\par
Mais dans les nations livrées au luxe et aux préjugés, c’est rarement l’amour qui préside au mariage : un intérêt sordide, la vanité de la naissance, des idées fausses de convenance sont uniquement consultées dans les alliances. Les talents, les sentiments, la conformité des humeurs et des caractères, la bonne éducation, la douceur, la complaisance, le bon sens, la raison n’entrent point dans les calculs de ces êtres mercenaires et vains qui ne cherchent qu’à combiner l’opulence et la naissance. Quel bonheur peut-il résulter de ce trafic honteux de la richesse et de la vanité ? Au sortir du couvent, c’est-à-dire d’une prison dans laquelle une fille sans expérience a tristement végété, sans consulter son inclination, des parents inhumains la font passer dans les bras d’un homme qu’elle n’a jamais vu, dont ils ne connaissent souvent eux-mêmes que le nom ou la fortune et dont les qualités intérieures ne les occupent nullement. Ainsi, des époux se trouvent liés sans se connaître ; ils se méprisent dès qu’ils se sont connus ; ils finissent communément par se haïr et s’éviter autant qu’il est possible. À ces causes déjà très suffisantes pour faire du mariage une source de désagréments, il faut joindre encore la jeunesse, l’inexpérience, la déraison de ceux qui s’y engagent. Une sage législation ne devrait-elle pas mettre obstacle à ces mariages précoces qui n’unissent d’ordinaire que des enfants peu mûrs et pour le corps, et pour l’esprit ? On ne peut attendre de ces alliances inconsidérées ou dictées par des intérêts mal entendus, que des unions malheureuses, des imprudences continuelles, des désordres fréquents et une race sans vigueur. Les grands ne se marient que pour perpétuer leur race ; follement occupés de transmettre leur nom à la postérité, ils semblent tout oublier pour de vaines chimères.\par
Faut-il après cela s’étonner de voir, surtout dans un rang élevé et dans une fortune brillante, si peu d’époux heureux, contre une foule d’imprudents qui passent leur vie soit à se tourmenter sans relâche, soit à se fuir incessamment ? Privés presque toujours des consolations et des charmes que le mariage est fait pour procurer, nous voyons communément les grands et les riches chercher dans des dépenses énormes, dans des plaisirs coûteux, dans des dissipations continuelles, dans des voluptés coupables, des moyens de remplacer la paix et les douceurs que la vie domestique leur refuse. Combien de dépenses, d’inquiétudes, de mouvements pour suppléer au bonheur paisible, à la sérénité continue dont la raison et la vertu feraient jouir à tout moment des époux unis par les liens de l’affection, de l’estime, de la confiance ! Mais des êtres inconsidérés n’ont pas même l’idée de ces avantages inestimables ; ils ne sont faits pour être sentis que par des êtres raisonnables, qui seuls en connaissent le prix. Peut-il y avoir un renversement plus complet dans les idées, que l’opinion dépravée qui, dans un rang distingué, fait que des époux rougissent de la tendresse que par état ils se doivent l’un à l’autre ? Est-il rien de plus insensé qu’une corruption capable d’étouffer dans les cœurs les sentiments les plus essentiels, les plus légitimes, les plus faits pour être avoués ? Ceux qui s’annoncent dans le monde par de semblables travers ne devraient-ils pas être accablés d’opprobre et d’infamie ? L’ignorance et les préjugés sont la source des maux qui troublent continuellement la félicité publique et particulière. Que dirons-nous de la folle vanité de ces hommes nouvellement enrichis qui ont la manie de faire contracter à leurs enfants des alliances avec des familles illustres, où leurs filles, ainsi qu’eux-mêmes, n’éprouveront par la suite que des mépris insultants ? Les nobles et les grands ne se regardent pas comme unis par le sang à des êtres inférieurs par la naissance ; orgueilleux et vains au sein même de l’indigence, ils s’imaginent que la richesse est trop payée par l’honneur de leur alliance. Mais l’expérience la plus réitérée ne peut guérir des hommes enivrés de leurs préjugés : tout conspire à les y maintenir, tout contribue à leur persuader que la richesse et la grandeur sont les seuls biens désirables, tandis qu’elles ne seront jamais que les moyens de se procurer le bien-être par l’usage sensé que la vertu seule en peut faire. L’éducation des riches et des grands ne leur fournit aucunement les lumières dont ils auraient besoin pour se rendre heureux ; elle les rend avares et vains et ne développe nullement en eux ni les sentiments du cœur, ni l’art de bien raisonner.\par
Nous aurons lieu de parler dans la suite de celle que l’on donne à ce sexe que la Nature avait fait pour le bonheur du nôtre. Nous verrons que loin de cultiver et d’orner l’esprit fin, l’imagination vive, le cœur sensible que cette Nature accorde aux femmes, loin de leur inspirer les idées, les sentiments et les goûts qui contribueraient à leur félicité véritable et à celle des époux que le sort leur destine, l’éducation ne semble se proposer que d’en faire des êtres totalement incapables de songer à leur propre bonheur et à celui de leur famille.\par
Dans des nations dépravées par le luxe et par l’oisiveté, une femme d’un certain ordre se trouve complètement désœuvrée ; elle se croirait avilie si elle prenait quelque soin de sa maison. Elle n’a donc pour s’occuper d’autre ressource que des amusements continuels qui tendent tous à l’écarter de ses devoirs : ils consistent dans un jeu habituel dont la manie peut avoir les plus fâcheuses conséquences, dans des bals où la vanité déploie toutes les ressources de la coquetterie, dans des spectacles où tout respire la volupté et semble exciter les femmes à mépriser les vertus faites pour les rendre chères à leurs maris. Enfin, ces passe-temps consistent dans la lecture des romans dont le but est d’allumer sans cesse l’imagination pour des plaisirs que la vertu défend\phantomsection
\label{footnote89}\footnote{Les Anciens faisaient tant de cas d’une vie laborieuse et occupée dans les femmes, que leurs poètes nous représentent les princesses, les reines, les déesses comme travaillant à des ouvrages utiles. Les Perses ne pouvaient concevoir qu’Alexandre portât des habits tissés par sa propre sœur. Parmi les femmes du grand monde, plus un travail est inutile, et plus on montre d’ardeur à s’y livrer. On rougirait de faire quelque chose d’utile.}. Comment une conduite si déraisonnable formerait-elle des épouses vertueuses, attentives, occupées du soin de plaire à leurs maris ? Des femmes dont la tête n’est remplie que de frivolités, d’images déshonnêtes, d’amusements pernicieux, deviendront-elles des compagnes sédentaires, des mères économes et réglées, des amies assidues et sincères capables de consoler et de conseiller des époux dont la présence seule les effarouche et les ennuie ? Des êtres que tout ramène sans cesse au jeu, à la volupté, à la dissipation, à la coquetterie, donneront-ils à leurs enfants les soins et la vigilance que leur état leur impose ? Enfin, des êtres ennemis de toute réflexion travailleront-ils à l’ouvrage sérieux de leur propre bonheur, intimement lié à celui de tous ceux qui les entourent\footnote{« Pour vous, ô femmes ! dit Périclès dans Thucydide, le but constant de votre sexe doit être d’éviter que le public parle de vous ; et le plus grand éloge que vous puissiez mériter, c’est de n’être l’objet ni de la critique ni de l’admiration. » Voyez Thucydide, {\itshape Histoire}, livre II. Mais il est bon d’observer en passant que chez les Grecs les femmes se tenaient renfermées dans leurs maisons et ne prenaient aucune part à la société, au lieu que chez les nations modernes de l’Europe les femmes vivent dans la société et devraient bien plus que les femmes des Grecs acquérir les qualités propres à s’y faire estimer. Une femme qui vit dans la retraite n’a pas besoin des vertus nécessaires pour bien vivre dans le monde.} ?\par
Grâce au peu de soin que l’on donne à l’instruction des grands et des riches, au lieu d’être des maris tendres, humains et sensibles, ils ne sont pour l’ordinaire que d’indignes despotes, méprisés et détestés par des femmes que sous les beaux dehors de la décence ils traitent souvent secrètement en esclaves et sur lesquelles ils croient pouvoir impunément exercer leur injustice, leur humeur, leurs caprices. Des parents guidés par leur avarice ou leurs indignes préjugés ont livré à ces lâches tyrans des victimes que la loi rigoureuse force presque en tout pays de gémir dans l’affliction pendant tout le cours de leur vie. On ne consulte dans les alliances, comme on a vu, que l’ambition, l’orgueil, la cupidité, que l’on décore du nom de {\itshape convenance.} Par là des mariages mal assortis ne font que rapprocher des ennemis qui se font éprouver à tout moment des contrariétés et des déboires, qui soupirent après le moment qui déliera leurs chaînes ou qui, lorsque les choses ne sont pas portées à cet excès, vivent dans une indifférence complète, sont séparés d’intérêts, ne s’occupent aucunement de leur félicité réciproque, non plus que de celle des enfants auxquels ils n’ont donné le jour que pour n’y plus songer.\par
Rien dans le mariage ne peut suppléer à l’union des cœurs, à cet heureux accord si nécessaire au bien-être des époux. La fortune la plus ample est toujours insuffisante pour fournir aux dépenses, aux amusements, aux caprices sans nombre par lesquels on tâche de remplacer le contentement solide qu’on devrait trouver chez soi. Un mari peu attaché à sa femme, livré à la dissipation, au jeu, au libertinage, lui refuse souvent le nécessaire. De son côté, une femme dépourvue de raison et d’économie est perpétuellement irritée de celle que son mari plus sage oppose à ses désirs insatiables ; elle le regarde comme l’ennemi de son bonheur.\par
Quant à l’homme du peuple qui, faute de culture, conserve presque toujours des moeurs sauvages, incapable de mettre un frein à ses passions, il regarde sa femme comme une victime destinée à souffrir ses violences.\par
Les lois, dans presque tous les pays, guidées par des préjugés barbares, ne donnent aux époux aucun moyen de rompre les liens cruels des mariages mal assortis ; ils sont communément obligés de tramer pendant la vie des chaînes qui les accablent. La femme, surtout, ne peut aucunement se soustraire à la tyrannie domestique d’un mari qui lui fait en secret sentir le poids affreux de son autorité ; d’un autre côté, celui-ci est forcé de vivre même avec une femme qui chaque jour le déshonore et dont le cœur corrompu brûle d’une flamme adultère. Si des époux veulent s’ôter de devant les yeux les objets qui les affligent, ils sont contraints de révéler leurs infortunes au public, de faire retentir sans pudeur les tribunaux de leurs disputes et des détails scandaleux de leurs malheurs privés.\par
Une législation plus équitable, plus conforme à la Nature, devrait briser pour toujours des nœuds qui ne servent qu’à lier des malheureux. Le mariage n’est fait que pour procurer aux époux des plaisirs honnêtes, des consolations, des douceurs ; dès qu’il ne leur produit que des peines, la loi ne devrait-elle pas anéantir une société si contraire à son but et à son institution ?\par
On nous dira peut-être que les lois ne doivent point se prêter à l’inconstance des hommes, que les nœuds du mariage sont respectables et sacrés et ne peuvent être rompus sans danger pour la société ; enfin, on nous dira que le sort des enfants deviendrait trop incertain s’il était permis à leurs parents de se séparer à volonté.\par
Nous répondrons à ces objections spécieuses que les hommes, nonobstant leur inconstance, sont fortement retenus par les liens de l’habitude, de la décence publique, par la crainte des embarras et du blâme, par la complication des affaires, en sorte qu’il n’y a pas lieu d’appréhender que des époux longtemps unis se séparent à la légère. Rome, où le divorce était permis, ne nous en fournit en cinq cents ans qu’un seul exemple. Les divorces n’y devinrent fréquents que lorsque le luxe eut corrompu totalement les mœurs. Des époux raisonnables se supporteront réciproquement et ne chercheront point à se séparer ; mais il est utile que des êtres dépourvus de raison soient éloignés les uns des autres : les enfants élevés au sein des dissensions domestiques, ne peuvent être que malheureux et négligés, ils doivent nécessairement se pervertir, au lieu de devenir des citoyens utiles à la patrie. Les époux indigents ou d’une fortune médiocre ne songeront guère à se séparer ; les divorces n’auraient lieu qu’entre les riches, qui sont en état de pourvoir aux enfants provenus de l’union qu’ils ont dessein de rompre.\par
Rien de plus respectable et de plus saint que l’union conjugale quand les époux remplissent fidèlement l’objet qu’elle doit leur proposer ; alors, de l’observation réciproque des devoirs qu’elle impose, il résulte un bien réel pour les époux, pour leurs enfants, pour la société toute entière. Si l’amour a formé ces nœuds si doux, l’estime, la tendresse, la concorde les resserrent à tout moment ; ils empêchent l’inconstance de les rompre. L’inconstance n’est que le fruit du vice inquiet et mécontent ; la vertu, toujours tranquille et modérée, fortifie les liens qui subsistent entre les époux. Elle leur apprend qu’ils doivent se montrer du moins une indulgence réciproque ; la raison leur prouvera que, faits pour vivre ensemble, la familiarité qui règne entre eux ne doit nullement exclure les prévenances, les attentions, les soins si propres à réveiller et cimenter l’affection ; ils éviteront donc tout ce qui peut blesser ou choquer l’objet dont chacun d’eux voudra toujours mériter l’estime et l’affection.\par
Le monde est rempli d’époux qui ne semblent réserver leurs attentions, leurs complaisances, leurs soins et leur belle humeur que pour des étrangers et des inconnus, et qui regardent leurs femmes et leurs enfants comme des esclaves faits pour essuyer à tout moment leur brutalité et leur mauvaise humeur ; ils ne voient pas, les insensés, que c’est chez soi qu’il faut établir le repos et le bien-être ! L’intimité ne dispense nullement les époux de se montrer de bons procédés, de la complaisance, des égards. Au contraire même, la fréquentation continuelle les rend plus nécessaires entre des êtres qui se voient incessamment.\par
La raison prescrit au mari d’adoucir son empire par sa tendresse ; elle recommande à la femme la soumission, la patience ; céder, pour elle, c’est remporter la victoire. La douceur est l’arme la plus forte qu’elle puisse opposer aux passions d’un mari que la contradiction ne ferait qu’aliéner ou rendre plus intraitable. Quel cœur assez féroce pour n’être point désarmé par la patience et par les larmes touchantes d’une femme douce, aimable, vertueuse !\par
Faute d’observer ces règles importantes, on voit souvent dans le mariage des dégoûts réciproques succéder quelquefois à l’amour le plus vif. Une conduite sage et mesurée est surtout nécessaire dans une association faite pour durer toujours ; les égards et la complaisance ne sont point des gênes quand on sent l’intérêt que l’on a de se plaire sans cesse. L’attention sur soi, le soin d’éviter tout ce qui peut altérer l’harmonie ou refroidir la tendresse deviennent faciles quand on en a contracté l’habitude. Par un abus trop commun, la familiarité des époux fait qu’ils sont très peu soigneux de ménager leur délicatesse : la femme coquette veut plaire à tout le monde, hormis à son mari.\par
Il n’est point de bonheur comparable à celui de deux êtres sincèrement unis par les liens de l’amour, de la fidélité, de la cordialité, et chez qui ces sentiments, se succédant tour à tour, se varient sans jamais s’épuiser. Quoi de plus attendrissant que le spectacle d’un époux occupé du bonheur d’une femme chérie qu’il ne quitte qu’avec peine, qu’il ne retrouve jamais sans un nouveau plaisir ! Est-il une félicité plus grande pour ces heureux époux que de lire à tout moment dans leurs yeux le contentement que chacun s’applaudit d’y faire éclore ? Leur propre maison a pour eux des charmes qu’ils chercheraient vainement au dehors ou dans le tumulte des plaisirs. La solitude, un désert n’ont rien d’affligeant pour des êtres qui se suffisent, qui trouvent l’un dans l’autre les charmes de la conversation, les douceurs de l’amitié. Est-il une joie plus pure pour eux que de se voir entourés d’enfants qui, formés par leurs soins réunis, seront sages et vertueux et serviront un jour de consolation et de support à leur vieillesse !\par
C’est en effet de l’union des époux que dépendent les vertus de leur postérité. Un père vicieux et tyran ne formera que des esclaves remplis de vices. Une mère frivole, galante, dissipée, ne saurait former des filles sages, modestes, retenues ; une mère de famille incapable de s’occuper, dépourvue de prévoyance et d’économie, ne peut élever que des êtres qui porteront le désordre dans les maisons où ils présideront un jour. C’est à l’extravagance et à la dépravation de tant de mauvais mariages que l’on doit attribuer les maux dont des nations entières sont affligées.\par
C’est encore à cette corruption que l’on doit attribuer la multitude des célibataires que l’on trouve surtout dans les pays où le luxe et la débauche ont fixé leur domicile. Des hommes dissipés et dominés par le goût du plaisir craignent des liens gênants pour l’inconstance ; ils trouvent dans la corruption générale des moyens de satisfaire aux demandes de leur tempérament sans se charger des embarras du ménage. D’ailleurs, ils regardent les femmes comme un bien commun, ou du moins dont la conquête devient aisée dès qu’on veut l’entreprendre. Les désordres ou la facilité des femmes doivent nécessairement multiplier le nombre des amants et des célibataires.\par
D’un autre côté, les hommes les plus sensés sont faits pour craindre des liens capables de les rendre malheureux pour la vie. La mauvaise éducation des femmes, leur passion effrénée pour la dépense et les plaisirs, la rareté des bons mariages sont des raisons propres à faire préférer le célibat à des engagements qui semblent souvent exclure le repos et le bien-être. La plus grande opulence suffit à peine dans un pays de luxe pour faire face aux besoins que ce luxe se plaît à créer. On craint de s’appauvrir en donnant le jour à des enfants.\par
Néanmoins, il est certain que le célibataire se prive d’un grand nombre d’avantages que l’union conjugale est capable de procurer. Un vieux garçon est un être isolé qui dans sa vieillesse et ses infirmités se trouve communément abandonné et livré à la rapacité de ses domestiques ; il n’éprouve point dans ses peines les soins d’une femme attentive ou de ses enfants ; il languit dans ses vieux jours, entouré de collatéraux avides qui soupirent après sa succession.\par
Bien des moralistes ont déclamé contre le célibat, qu’ils ont regardé comme une source de corruption, des législateurs l’ont voulu punir comme contraire à la population ; ils n’ont point vu que le célibat multiplié était lui-même l’effet de la corruption publique, autorisée ou tolérée par de mauvais gouvernements ou par des institutions vicieuses. En vain Auguste fit-il des lois contre les célibataires, qu’il regardait comme des conjurés qui tramaient la perte de l’empire. C’est en déracinant le luxe, en réformant les mœurs, en gouvernant les nations selon les règles de l’équité, que l’on peut inviter les hommes à se multiplier. Le despotisme, le luxe, le mépris des bonnes mœurs sont des fléaux dont la réunion ne peut qu’accélérer la ruine d’un État. Un mauvais gouvernement anéantit jusqu’aux races futures ; il ne fait que des malheureux, des esclaves incertains de leur sort, qui vivent au hasard et qui ne peuvent sans crainte songer à se multiplier : des enfants ne feraient que redoubler et leurs besoins présents et leurs inquiétudes sur l’avenir. La population n’est que trop grande sous un gouvernement qui ne fait que des malheureux et dans les nations où le vice marche la tête levée.\par
C’est en réprimant le luxe, en corrigeant les mœurs, en punissant l’adultère, en châtiant la prostitution publique, qu’un législateur vertueux peut parvenir à diminuer le nombre des célibataires, à rendre les mariages plus heureux et plus capables de former des citoyens à l’État. On se plaint des effets et l’on ne remonte pas à leurs causes : sous un mauvais gouvernement, sous des princes sans mœurs et sans vigilance, la masse entière de la société doit nécessairement se corrompre et se dissoudre.\par
La politique et la morale sont également intéressées à détourner du célibat. Le mariage unit l’homme plus intimement à son pays, à la société ; il le force de montrer plus d’activité : le père de famille est semblable à un arbre vigoureux qui s’attache à la terre par un grand nombre de racines. L’effet du célibat, au contraire, est de détacher de la chose publique, de concentrer l’homme en lui-même, de le rendre personnel, de lui donner une profonde indifférence pour les autres. Le célibataire ne s’occupe que du présent et s’embarrasse fort peu de l’avenir ; en un mot, il devient communément plus âpre et moins sociable, parce qu’il n’est point adouci par les sentiments multipliés que les tendres noms d’{\itshape époux} et de {\itshape père} doivent faire éprouver.
\subsection[{Chapitre II. Devoirs des Pères et Mères, et des Enfants}]{Chapitre II. Devoirs des Pères et Mères, et des Enfants}
\noindent Le principal objet du mariage est de faire naître des enfants qui deviennent un jour des membres utiles de la société, ainsi que les consolateurs, les appuis de leurs parents. L’amour des pères et mères pour leurs enfants est un sentiment qui se trouve même dans les animaux les plus sauvages : nous les voyons remplis de la plus tendre sollicitude pour leur progéniture. Ce sentiment doit être encore plus vif dans l’homme, qui voit dans sa postérité des coopérateurs de ses travaux, des amis liés d’intérêts avec lui, des soutiens de sa vieillesse. Un père peut espérer de voir dans la suite ses soins payés par les êtres à qui il les donne, au lieu que les animaux accordent les leurs à des êtres incapables de reconnaissance qui les abandonneront dès que leurs forces leur permettront de se passer de leurs secours. D’où l’on voit que les parents ont moins de sentiment ou d’instinct que les bêtes lorsque après avoir donné la vie à des enfants, ils négligent de s’occuper de leur bien-être.\par
L’existence n’est un bien qu’autant qu’elle est heureuse ; la vie serait un présent fatal si elle était continuellement misérable. Ce n’est donc pas pour avoir reçu la vie de ses parents qu’un enfant leur doit de la reconnaissance ; cette vie peut n’être que l’effet de la volupté ou d’un appétit aveugle qui ne cherche qu’à se satisfaire. La tendresse, la piété filiale, la gratitude de l’enfant ne peuvent être solidement établies que sur le soin que ses parents ont pris de son bonheur.\par
L’autorité paternelle fondée sur la Nature, sur les besoins de l’homme faible dans son enfance, est très juste puisqu’elle n’a pour objet que la conservation et le bonheur d’un être qui, sans les secours continuels de ses parents, serait à chaque instant exposé à périr et ne pourrait écarter aucun des dangers qui l’environnent. L’homme étant, au moment de sa naissance, de tous les animaux le plus incapable de se défendre et de se procurer la subsistance, se trouve dans la dépendance de ceux qui, en lui donnant la vie, se sont engagés à la lui conserver et à lui fournir les moyens de satisfaire ses besoins.\par
L’enfant, par sa naissance, se trouve en société avec ses pères et mères dont, à son insu, il reçoit pendant longtemps les services et les secours gratuits. Ce n’est que par la suite qu’il apprend les engagements qu’il a contractés avec eux, la reconnaissance qu’il leur doit, la façon dont il peut s’acquitter. Sa raison venant à se développer lui montre la nécessité de remplir ses devoirs ou de payer ses dettes. L’opinion publique, la crainte du blâme, les notions de vertu, l’habitude d’obéir à ses parents concourent à lui indiquer et à lui faciliter la conduite qu’il est obligé de tenir, et à confirmer en lui les sentiments qu’il doit à des êtres bienfaisants et secourables qui se sont constamment occupés de son bien-être.\par
C’est ainsi que tout conspire à graver dans les cœurs la {\itshape piété filiale}, c’est-à-dire cette tendresse soumise, timide, respectueuse, que les enfants convenablement élevés se sentent obligés de montrer à leurs pères et mères, dont ils ne peuvent jamais assez payer l’affection. Enfin, les enfants doivent songer qu’ils deviendront pères à leur tour, et que pour acquérir de justes droits sur l’attachement et la reconnaissance de leur postérité, ils doivent témoigner ces sentiments à ceux desquels ils ont reçu le jour. « Il faut, disait Thalès, attendre de son fils ce que l’on a fait à son père. »\par
D’un autre côté, la tendresse paternelle, ou l’amour que les parents ont pour leurs enfants, est fondée sur des motifs raisonnés et non, comme on l’a cru communément, sur une prétendue {\itshape force du sang}, ou sur une sympathie occulte que l’ignorance a gratuitement imaginée. Cet amour a pour base l’espoir de trouver dans les enfants qu’on a fait naître des êtres disposés à reconnaître un jour les soins qu’ils ont reçus par un dévouement respectueux, par un zèle à toute épreuve, par des soins empressés. D’ailleurs, l’amour-propre d’un père est flatté d’avoir produit, pour ainsi dire, un autre lui-même, d’avoir donné l’existence à quelqu’un qui perpétuera son nom, qui rappellera sa mémoire aux autres, qui le représentera dans la société. Telle est évidemment la cause des chagrins que ressentent les grands de la terre lorsqu’ils ne peuvent avoir de postérité : ils craignent alors de voir leurs noms totalement oubliés, au lieu qu’ils s’imaginent perpétuer leur propre existence et se survivre en laissant des enfants après eux. C’est ainsi que l’imagination des hommes, s’élançant dans l’avenir, les fait à tout moment jouir d’avance de ce qui se passera dans le monde, lors même qu’ils ne seront plus qu’un amas de poussière.\par
D’après ces dispositions, les parents forment souvent des projets pour leurs descendants, jettent les fondements de leur grandeur, s’occupent de leur fortune, veulent par des testaments régler leurs destinées, et quelquefois font des sacrifices réels et pénibles à l’idée du bonheur futur de leur race, quoiqu’ils sachent très bien qu’ils n’en seront pas les témoins. Tout homme croit voir déjà ce qui se passera lorsqu’il ne sera plus ; l’imagination parvient souvent à nous créer des chimères auxquelles nous tenons plus fortement qu’à des réalités ; celles qu’enfante la tendresse paternelle sont utiles à la société, c’est pour elles que souvent un bon père se prive de mille jouissances, dans l’idée de faire jouir des êtres qui n’existent point encore. Que deviendraient les familles si l’esprit de chaque citoyen se renfermait dans les bornes de son existence présente sans jamais porter ses regards sur l’avenir ? Les parents sans prévoyance ou qui pour satisfaire leurs passions ou leurs plaisirs négligent les soins qu’ils doivent à leur postérité, sont justement blâmés par leurs contemporains. Tout homme qui ne songe qu’à lui est regardé comme un mauvais père et comme un mauvais citoyen.\par
Néanmoins, il faut convenir que ce soin de l’avenir, réel ou prétendu, rend souvent les parents injustes ou cruels à l’égard de leurs enfants. Un père avare ne veut point se dépouiller de son vivant ; sous prétexte du plus grand bien de ses enfants, à qui il laissera ses trésors, il leur refusera quelquefois le nécessaire. L’avare n’est bon qu’après sa mort ; il est détesté tant qu’il vit. Un père prévoyant se garde bien d’abandonner sa fortune à une jeunesse bouillante qui méconnaîtrait presque toujours les règles d’une sage économie ; d’ailleurs, il sait qu’il serait imprudent de se dépouiller totalement lui-même et de se mettre dans la dépendance de ceux qui doivent dépendre de lui. Mais dès qu’il aime véritablement ses enfants, il les met autant qu’il peut à portée de jouir sous ses yeux ; il jouit alors lui-même du plaisir qu’il cause à des êtres si chers.\par
Des idées fausses, des notions vagues et peu fondées sur l’expérience n’ont fait qu’obscurcir en tout temps la morale ; on a regardé la tendresse paternelle et la piété filiale comme des sentiments {\itshape innés}, que les hommes apportaient en naissant, qui se trouvaient inhérents au sang. Néanmoins, la réflexion la plus légère aurait pu détromper de ce préjugé si flatteur. Un père, dans son fils, aime un autre lui-même, un être dont il attend du contentement, du plaisir, des secours. Un fils bien élevé aime son père lorsqu’il voit en lui l’ami le plus sûr, l’auteur de son bien-être, la source de sa félicité. Ces sentiments de part et d’autre deviennent habituels et passent alors pour des effets de {\itshape l’instinct} ou de la Nature. Cependant on ne les trouve guère dans les nations corrompues et dans les familles mal organisées.\par
Ce serait se tromper que d’attendre de la Nature, de l’instinct ou de la force du sang, des sentiments que les soins et la tendresse des parents n’auraient pas semés et cultivés dans les cœurs des enfants. Il ne suffit pas d’être père pour exciter en eux l’affection et le retour auxquels la paternité met à portée de prétendre. Pour être aimé il faut se rendre aimable ; c’est une loi dont nul homme ne peut être exempté. L’existence, comme on vient de le dire, n’est pas un bien par elle-même : elle ne le devient que par les avantages que l’on y trouve attachés. Les parents ont reçu de la Nature une autorité légitime sur leurs enfants, mais nulle autorité sur la terre ne donne le droit de nuire ou de rendre malheureux : toute dépendance, toute soumission ne peut avoir pour motif que le bien qui résulte de l’autorité à laquelle on se soumet. La paternité ne peut pas dispenser de cette loi primitive. Un père qui abuse de son pouvoir, qui ne montre ni tendresse ni soins à ses enfants, qui, au contraire, exerce sur eux un empire déraisonnable, qui s’oppose à leur félicité, qui néglige même de leur procurer tout le bonheur dont il est capable, se rend indigne du nom de père et ne doit pas s’attendre à trouver en eux les sentiments d’un amour bien sincère : il ne peut être que le prix de la bonté. La piété filiale ne peut être fondée que sur la tendresse paternelle ; ces sentiments naturels disparaissent dès qu’ils ne sont pas appuyés, parce que la première loi de la Nature veut que l’homme n’éprouve de l’affection que pour ce qui contribue à son bonheur, vers lequel sa nature le fait tendre sans cesse.\par
Combien ne voit-on pas de pères transformés en tyrans qui ne regardent leurs enfants que comme des esclaves destinés par la Nature à se soumettre sans réserve à leurs caprices despotiques ? Les aveugles s’imaginent donc que pour avoir donné le jour à des êtres qu’ils doivent aimer, ils ont acquis le droit d’en faire les jouets de leur humeur et de leurs volontés arbitraires ! Le nom de {\itshape père}, qui renferme l’idée de l’affection et de l’intérêt le plus tendre, est-il donc fait pour ne présenter à l’esprit d’un enfant que l’idée d’un maître impitoyable des coups duquel il ne peut se défendre ? Peut-on donner le nom de pères à ces ambitieux\footnote{Tout homme qui n’est pas aveuglé par le préjugé doit sentir la perversité des lois et des usages de certains pays où, pour favoriser la sotte vanité de quelques nobles, l’aîné doit emporter lui seul tous les biens de la famille tandis que ses frères et sœurs sont condamnés à l’indigence. N’est-il pas honteux que dans des nations qui se disent policées, la législation laisse subsister des coutumes si folles et si dénaturées ? Des enfants ainsi déshérités par la loi ont-ils donc de grandes obligations à ceux qui leur ont donné la naissance ?} injustes pour tous leurs enfants, qui les sacrifient cruellement à la fortune d’un aîné sous prétexte qu’il est chargé de soutenir dans le monde la splendeur de sa famille ? Est-il une barbarie plus féroce que celle de ces indignes parents qui, pour mieux doter une fille, forcent sa sœur à se condamner à une prison perpétuelle qu’elle arrosera toute sa vie de ses larmes ? Des êtres de cet affreux caractère ne peuvent point être nommés des parents ; ils ne méritent pas même le nom d’{\itshape hommes}, et les lois devraient soustraire leurs enfants infortunés à une autorité dont ils font un abus si détestable.\par
C’est surtout dans l’établissement des enfants que des parents déraisonnables font souvent paraître leur cruauté. Guidés communément soit par une avarice sordide, soit par la vanité, vous ne les voyez guère consulter les inclinations de leurs enfants. Nous avons fait remarquer ci-devant les conséquences déplorables de ces mariages dont l’intérêt seul forme les tristes nœuds et dont les époux sont les victimes, mais où l’on voit principalement éclater la dureté des parents, c’est lorsque par hasard, séduits par l’amour, leurs enfants contre leur gré ont eu le malheur de contracter une alliance : pour lors, ces parents implacables pardonnent rarement le mépris de leur autorité. Au lieu de s’apaiser avec le temps et d’oublier des fautes sans remède, vous les voyez quelquefois pousser leur affreuse vengeance par-delà le tombeau et par des exhérédations inhumaines dévouer leur propre sang à la misère et au désespoir.\par
Le cœur d’un père devrait-il jamais être fermé pour toujours à la pitié ? Il n’y a que le vice incorrigible ou le crime endurci qui puissent autoriser sa partialité pour ses enfants ; s’il est l’auteur de leur existence, il leur doit le bonheur à tous. Juge dans sa famille, qu’il tienne une juste balance. La difformité du corps est-elle une raison pour prendre en haine un enfant que son état même doit rendre un objet de compassion ? Quels cœurs que ceux de tant de parents qui, parce qu’un enfant est déjà malheureux, se plaisent à lui faire sentir encore plus le poids de sa misère ? Un enfant contrefait doit être plaint, et l’on doit plus soigner son esprit afin de réparer le caprice du sort\phantomsection
\label{footnote90}\footnote{On dit qu’un magistrat en France déshérita sa fille par son testament, uniquement parce qu’elle était laide. Son testament fut cassé par un arrêt du Parlement de Paris.}.\par
Que dirons-nous de la faiblesse de ces pères qui ne voient dans leurs enfants que des héritiers dont la présence importune leur rappelle à tout moment leur propre fin ? Mais ces hommes qui semblent redouter si fort leur fin, se flatteraient-ils de ne point finir s’ils n’avaient point d’enfants ou d’héritiers ? « Les hommes, dit Homère, sont faits pour se succéder comme les feuilles sur les arbres\footnote{Montaigne dit très bien en parlant des enfants : « Voire il semble que la jalousie que nous avons de les voir paraître et jouir du monde, quand nous sommes à même de le quitter, nous rendre plus épargnants et restreints envers eux. Il nous fâche lorsqu’il nous marche sur les talons, comme pour nous solliciter de sortir. Et si nous avons à craindre cela, puisque l’ordre des choses porte qu’ils ne peuvent, à dire vérité, être ni vivre qu’aux dépens de notre être et de notre vie, nous ne devions pas nous mêler d’être pères... » Il ajoute plus loin : « C’est injustice de voir qu’un père vieil, cassé et demi-mort, jouisse seul à un coin du foyer des biens qui suffiraient à l’avancement et entretien de plusieurs enfants. » Voyez {\itshape Essais}, livre II, chap. 8, p. 65.}. »\par
Les sentiments de la tendresse paternelle sont étouffés ou méconnus par l’avarice, ainsi que par la prodigalité. Dans des nations infectées par le luxe, par la vanité, par l’amour de la dépense et de la représentation, et surtout par la contagion du vice, peut-on donner le nom respectable de {\itshape père} à des hommes frivoles, dissipés et qui prodiguent tout à leurs plaisirs honteux, qui, occupés à satisfaire leurs fantaisies extravagantes ou criminelles, ne font rien pour leurs enfants ou les regardent comme un fardeau ? Ces aveugles que leurs désordres et leurs folies rendent ennemis de leur propre sang, se flattent-ils qu’en dépensant leurs richesses pour nourrir des étrangers, des inconnus, des parasites, des femmes perdues, ils s’attacheront des amis plus solides et plus constants qu’ils ne s’en feraient de leurs enfants que la Nature leur unit par les plus étroits liens ? Ces étrangers ou ces inconnus viendront-ils dans la vieillesse ou dans les infirmités, donner des consolations et des soins à ces pères qui auront négligé de se faire des amis domestiques dans la personne de leurs enfants ? Mais la vanité et le luxe étouffent tellement dans les cœurs les sentiments les plus naturels, que la femme, les enfants, les proches d’un libertin prodigue sont plus éloignés de son cœur que des inconnus, des flatteurs, des femmes sans mœurs, qui jamais ne lui seront utiles !\par
Avec une conduite si cruelle et si peu conforme à la tendresse paternelle, ne soyons pas surpris que l’amour des enfants pour leurs pères soit si rare et même paraisse un phénomène dans bien des nations. Des pères dépourvus d’entrailles et de bonté exercent une autorité révoltante sur des infortunés qui souvent ne peuvent voir dans les auteurs de leurs jours que des tyrans pour lesquels la décence les force de cacher toute leur haine, ou des hommes méprisables qui par leur existence mettent de longs obstacles aux jouissances et aux désordres que ces enfants désireraient d’imiter. Des parents vicieux communiquant leurs vices à leur postérité, lui font désirer avec ardeur le temps où elle pourra librement se livrer aux dérèglements dont elle a reçu l’exemple : des parents dépourvus de sensibilité sont-ils en droit d’attendre des sentiments qu’ils n’ont jamais fait naître ou qu’ils ont étouffés ?\par
Les mauvais pères ne peuvent souffrir que leurs enfants les imitent. « Ceux, dit Plutarque, qui reprennent leurs enfants des fautes qu’ils commettent eux-mêmes, ne voient pas que sous le nom de leurs enfants ils se condamnent eux-mêmes\footnote{Voyez Plutarque au traité {\itshape Comment il faut nourrir les enfants}.}. »\par
En effet, les enfants attachent une idée de bien-être à tout ce qu’ils voient faire à leurs parents ; ils veulent les imiter, nonobstant toutes les défenses. Jamais on ne leur persuadera qu’il n’y a point de plaisir dans les actions qu’ils voient faire soit à leurs pères, soit aux personnes qui règlent leur conduite ; les défenses alors ne font qu’irriter leur curiosité et leur faire désirer le temps où ils pourront sans obstacles mettre en pratique les exemples qui les ont frappés dans la maison paternelle. Juvénal a grande raison de dire que « l’on doit le plus grand respect à l’enfance\footnote{Maxima debetur puero reverentia. » {\itshape Satires}, 14, vers 47.} ». C’est en ne faisant devant les enfants que des choses louables qu’on les rend vertueux, c’est en ne louant en leur présence que des actions vraiment estimables qu’on leur inspire le goût du bon et du beau.\par
Celui qui veut mériter le nom de père et jouir des prérogatives attachées à ce titre respectable, doit remplir soigneusement les devoirs que son état lui impose. Un bon père aime ses enfants et tâche d’en faire des amis ; il veut leur plaire, il craint d’aliéner leur tendresse et d’étouffer leur reconnaissance par d’injustes rigueurs, il s’arme de patience, parce qu’il sait qu’un âge privé de raison et d’expérience est moins digne de colère que d’indulgence et de pitié, il ne se montre point l’ennemi jaloux des plaisirs innocents dont il ne saurait jouir désormais lui-même, il consent à ceux que l’enfance ou la jeunesse sont faits pour désirer, il ne s’oppose qu’à ces plaisirs dangereux qui tendraient à corrompre et l’esprit et le cœur. Des enfants sans jugement regarderont peut-être ces obstacles comme une tyrannie, leur déraison actuelle les révoltera contre un joug incommode pour leurs aveugles désirs ; mais leurs esprits plus mûrs se rappelleront un jour avec reconnaissance l’inflexibilité qui résistait prudemment à leurs folies.\par
Ce n’est point une indulgence aveugle et souvent très cruelle qui constitue la vraie bonté d’un père, c’est une indulgence équitable et raisonnée. Des parents trop faciles ne sont pas bons : ils sont faibles ; cette faiblesse, leur fermant les yeux sur les vices de leurs enfants, en fait des êtres incommodes par la suite, et pour les parents mêmes et pour la société. Un bon père est celui qui, indulgent pour les fautes inséparables d’un âge dépourvu de prudence, s’arme de son autorité et emploie, s’il le faut, la verge de fer pour réprimer les dispositions criminelles du cœur, pour dompter les passions insociables, pour arrêter des mouvements qui, devenus habituels, rendraient un jour son fils odieux dans le monde et par là même très malheureux.\par
La rigueur injuste et déplacée ne fait que des esclaves tremblants ou des rebelles. Tout père que la raison guide, doit la montrer à ses enfants et les forcer de reconnaître qu’il les punit justement. Un gouvernement arbitraire ou tyrannique produit en petit dans les familles les mêmes inconvénients que dans les grandes sociétés : un père de famille qui veut régner en despote sur les siens, gouverne par la terreur et ne méritera jamais l’affection de ses sujets. Des parents ont la folie d’exiger que leurs enfants, dans un âge tendre, aient les mêmes idées, les mêmes amusements, les mêmes goûts qu’eux ! Il est assez rare que les enfants aient les inclinations de leurs pères, parce que ceux-ci ont eu soin pour l’ordinaire de les faire beaucoup souffrir pour les rendre conformes à leurs propres fantaisies et n’ont fait réellement que les en dégoûter.\par
Quoi de plus ridicule que le vain orgueil de ces parents qui se rendent inaccessibles à leurs enfants, qui ne leur montrent qu’un front sévère, qui jamais ne les approchent de leur sein ! Le bon père se montre à ses enfants, se prête à leurs jeux innocents ; il leur fait contracter l’habitude de vivre avec lui dans une juste confiance. Il récompense par des caresses les efforts qu’ils font pour lui plaire ; il sait que sa tendresse est le ressort le plus capable d’exciter au bien des âmes flexibles qu’une sévérité habituelle ne ferait que repousser et dégoûter. Il ne craindra pas qu’une familiarité mesurée lui fasse perdre ses droits ou son autorité ; il sait qu’elle n’est jamais plus sûre et plus fidèlement obéie que lorsqu’elle est juste et fondée sur la tendresse. Enfin, il s’abstiendra de ces duretés qui deviennent inhumaines dès qu’on les exerce à contretemps sur des êtres auxquels la défense est interdite. Tout père qui exige de la bassesse de ses enfants ne peut guère se flatter d’en faire d’honnêtes gens ; il n’en fera que des êtres faux, dissimulés, menteurs, qui auront tous les vices des valets ou des esclaves. Un bon père doit traiter ses enfants en amis, ménager leur délicatesse, craindre d’affaiblir le ressort de leurs âmes : on ne peut rien attendre de bon des cœurs qu’on avilit. La paternité ne donne pas le droit de contrister mal à propos ceux qu’elle veut corriger. Combien de parents sont assez injustes pour excéder leurs enfants par des outrages, afin de les punir ensuite de leur colère ! Enfin, combien de parents sont plus déraisonnables que les enfants, auxquels ils devraient apprendre à contenir leurs passions !\par
Si l’autorité paternelle, quelque respectable qu’elle soit, ne donne jamais le droit d’être injuste, on ne doit pas non plus lui obéir quand elle exige des choses contraires à la vertu. Le père d’Agésilas roi de Sparte, le sollicitant de juger contre les lois : « Ô mon père, lui dit-il, tu m’as dit dans ma jeunesse d’obéir aux lois ; je veux donc encore maintenant t’obéir en ne jugeant pas contre les lois\footnote{Voyez Plutarque, {\itshape De la mauvaise Honte}.}. »\par
Une bonne éducation est le plus important des devoirs que la morale impose aux parents, pour leur bonheur propre, pour l’avantage de leurs enfants, pour le bien général de la société. C’est par l’éducation seule que ces parents peuvent se promettre de former des êtres dociles et qui deviennent un jour des citoyens utiles. Si des occupations nécessaires ou une incapacité totale empêchent souvent les pères et mères de cultiver convenablement l’esprit de leurs enfants, rien ne devrait au moins les dispenser de veiller sur l’éducation qu’ils leur font donner, de s’occuper de leurs mœurs et de leur inspirer l’amour de la vertu. Si les talents nécessaires pour enseigner des sciences sublimes et difficiles sont le partage de très peu de personnes, tout homme de bien qui a de l’expérience est en état d’enseigner à son fils les devoirs de la décence, de la politesse, de la probité, de l’humanité, de la justice. Des parents honnêtes peuvent, par leur exemple encore plus que par leurs leçons, indiquer à leurs enfants le chemin de la vertu, qui seule peut les rendre estimables et leur apprendre à faire un bon usage et des talents de l’esprit et des dons de la fortune\footnote{« L’exemple, dit un moraliste moderne, est un tableau vivant qui peint la vertu en action et communique l’impression qui la meut à tous les cœurs qu’il atteint. » Voyez un livre intitulé {\itshape Les Mœurs}, partie II, chap. I, art. III, §. I.}.\par
Par une convention tacite de la société, les pères lui sont responsables des vices et des crimes de leurs enfants, de même que les enfants portent souvent la peine des iniquités de leurs pères. L’opinion publique, qui dégrade et condamne à une sorte d’ignominie le père d’un fils coupable, semble supposer que ce fils ne se serait pas livré au crime et n’aurait pas encouru le châtiment infligé par les lois s’il eût reçu de son père des principes honnêtes ou des exemples louables. En punissant le fils des crimes de son père, cette opinion semble pareillement supposer qu’on ne doit pas se fier à l’enfant d’un tel père qui n’a pu lui donner des sentiments estimables. Voilà comme les préjugés, souvent injustes dans leurs effets, sont pourtant quelquefois fondés sur des raisons. L’expérience nous montre cependant que les parents les plus honnêtes et les plus vertueux peuvent quelquefois donner le jour à des monstres, et qu’un fils très digne d’affection peut être né d’un père très méprisable. Mais le public, qui rarement se donne le soin d’approfondir les choses, condamne indistinctement et les parents et les enfants qui s’annoncent par des crimes ; il lui suffit de savoir en gros que les pères négligents ou méchants ne forment communément que des enfants pervers et que ceux-ci pour l’ordinaire ont reçu de bonne heure des impressions fâcheuses de leurs parents. Le fils d’un concussionnaire, d’un usurier, d’un méchant homme, est souvent forcé de rougir d’être né d’un tel père. C’est un fatal héritage pour des enfants honnêtes que le nom d’un père décrié par ses vices et ses crimes.\par
Rien n’est donc plus intéressant pour les parents que de présenter à leurs enfants des exemples honnêtes et de les habituer de bonne heure à les suivre. Une bonne éducation est le meilleur héritage que l’on puisse laisser à sa postérité : elle répare quelquefois une fortune délabrée, elle tient souvent lieu d’une naissance illustre, elle va même jusqu’à faire oublier les iniquités des pères. C’est surtout par une éducation vertueuse que les parents peuvent mériter la reconnaissance, la tendresse, le dévouement respectueux et les soins empressés qu’ils sont en droit d’attendre de leurs enfants\phantomsection
\label{footnote91}\footnote{Solon, par une loi, ordonna qu’un fils ne serait point obligé de nourrir son père dans la vieillesse si le père, ayant eu les moyens de faire apprendre un métier à son fils, avait négligé ce devoir.}.\par
Ceux-ci, formés par les préceptes d’une bonne morale, apprendront ce qu’ils doivent à des êtres qui, après leur avoir donné le jour, se sont tendrement occupés du soin de les conserver à la vie. Ils apprendront à vénérer celle qui les a portés dans son sein, qui les a nourris de son lait, ou du moins qui a montré la sollicitude la plus tendre pour écarter d’eux les dangers et les maladies, qui leur a peu à peu appris à exprimer leurs désirs, qui a supporté les infirmités et les dégoûts de leur âge imbécile. Ils sentiront que ces soins continus et multipliés ne se peuvent jamais payer d’une trop longue reconnaissance, d’une trop grande soumission, d’une tendresse trop assidue, d’un respect trop profond. Enfin, tout leur prouvera que les sentiments justes d’une reconnaissance illimitée ne doivent être effacés ni par l’humeur chagrine, ni par les longues infirmités, ni par les faiblesses de l’âge.\par
Cette morale ne leur laissera pas ignorer les sentiments de respect et de tendresse qu’ils doivent également à un père vigilant et bienfaisant qui s’est occupé des moyens de leur procurer ou de leur conserver une fortune ou les talents nécessaires pour subsister avec honneur, pour occuper un état estimable dans la société. Ils auront lieu de s’honorer d’être descendus d’un père estimé par ses concitoyens ; ils s’applaudiront d’avoir reçu de lui la vie ainsi que l’éducation et les talents dont il a pris soin de les orner. Le nom d’un père aimable par sa bonté, respectable par ses lumières et ses vertus, qui s’est rendu cher par ses bienfaits, excitera toujours dans des âmes bien façonnées un attendrissement capable d’étouffer les impulsions d’un intérêt sordide. Un fils bien élevé peut-il être avide au point de désirer la mort d’un père qu’il ne peut regarder que comme son plus grand bienfaiteur, son ami le plus sincère ! Des sentiments si bas et si cruels ne sont faits que pour les âmes dépravées de ces enfants sans mœurs dont les vices insatiables ont besoin de la mort d’un père pour s’assouvir en liberté\footnote{Un fils de cette trempe, racontant un jour son père à ses camarades, leur disait : « Voyez-vous ce coquin-là ? Il me retient depuis longtemps mon bien, dont je ferais un si bon usage s’il voulait s’en aller. »}. Ces vœux indignes ne peuvent se former que dans ces esclaves irrités par la tyrannie ou dans ces enfants négligés ou abandonnés par des parents déréglés. De pareils désirs n’entreront point dans le cœur d’un enfant vertueux, ou du moins y seront très promptement étouffés : l’éducation, la morale, l’opinion publique, toujours favorable aux parents, s’accorderont pour lui faire sentir que le père le plus injuste, le plus chagrin, le plus incommode est pourtant son père, est l’auteur de ses jours, a des moments heureux dans lesquels sa tendresse parle. Si son âme ulcérée par les mauvais traitements ne lui permet pas de sentir une tendresse réelle, il se respectera du moins lui-même : il craindra de se déshonorer par des procédés capables de lui attirer le blâme de la société, il se fera un mérite de pardonner les traitements qu’il reçoit d’une main respectable, il supportera en silence des maux auxquels il ne peut remédier, il se soumettra avec courage à la destinée rigoureuse qui voulut pour un temps le rendre malheureux ; enfin, il s’applaudira des triomphes réitérés que la vertu lui fera remporter sur les impulsions subites dont il se sent agité et qu’il sacrifie à son pénible devoir. Est-il rien de plus noble et de plus beau que d’exercer le pardon des injures sur son père ? Est-il rien qui rende un fils bien né plus digne des applaudissements de sa propre conscience que de savoir vaincre les mouvements d’un cœur que tout sollicite à la vengeance ? D’ailleurs, cette vengeance aurait-elle quelque charme puisqu’elle serait condamnée par toute la société ? Un fils malheureux par l’injustice de son père est comme le citoyen malheureux par la tyrannie de son roi : il n’est permis ni à l’un ni à l’autre de se faire justice à lui-même et de violer dans sa colère les droits de la société. « La soumission, dit Addison, des enfants à leurs parents, est la base de tout gouvernement et la mesure de celle que le citoyen doit à ses chefs : à qui obéira-t-on si l’on n’est pas soumis à son père\footnote{Voyez {\itshape Le Mentor moderne}.} ? »\par
Ainsi, la saine politique, toujours d’accord avec la saine morale, veut que les enfants soient soumis à leurs pères : l’intérêt des sociétés l’exige, de même que celui des familles ; chaque père de famille est un roi dans la sienne mais il ne lui est jamais permis d’en devenir le tyran. Le gouvernement chinois a pris l’autorité paternelle pour modèle de la sienne, mais, ainsi que les lois romaines, il donne très injustement aux pères le droit de faire périr ses enfants ; par les mêmes principes, le gouvernement chinois est arbitraire et despotique et produit très souvent des tyrans. Des lois plus raisonnables, fondées sur une morale plus sage, ne permettent ni aux souverains ni aux parents d’exercer la tyrannie ; elles permettent aux peuples de réclamer contre la tyrannie des pères des peuples, elles défendent aux pères de famille d’user de leur pouvoir d’une façon injuste et cruelle, elles ordonnent aux enfants de supporter les injustices de leurs pères\footnote{Les lois de la Chine, en favorisant l’autorité paternelle jusqu’à l’excès et en la rendant toujours sacrée, ont en quelque façon remédié au despotisme du gouvernement. Nonobstant ce despotisme, la Chine est, dit-on, très peuplée, parce que chacun est intéressé à devenir père de famille ou roi dans sa maison. Au contraire, parmi les nations européennes, la subordination des enfants pour leurs parents n’est peut-être pas assez marquée lorsqu’ils cessent d’en dépendre par les liens de l’intérêt ou de la fortune. Parmi les grands surtout, les pères et les enfants se traitent presque comme des étrangers qui n’ont rien de commun. Des enfants plaideront indécemment contre leurs parents et les traiteront à la rigueur. Des êtres dépourvus de sentiments et de mœurs ne craignent pas de se déshonorer dans des nations où l’argent fait tout pardonner, jusqu’à la violation de la tendresse paternelle et de la piété filiale ! {\itshape Virtus post nummios} (la vertu après l’argent) est la devise des pays où le luxe s’est établi sur la ruine des mœurs.}.\par
Tels sont les principes et les devoirs que la morale enseigne aux parents, tels sont les préceptes qu’elle donne à leurs enfants, à qui une éducation honnête doit les inculquer pour les leur rendre familiers. Si ces principes sont souvent oubliés ou méconnus, c’est que des pères négligents, dissipés ou pervers, sont incapables de faire naître dans leurs enfants des sentiments honnêtes. C’est que, trop souvent, des pères injustes mettent tout en œuvre pour fixer la haine dans des âmes dans lesquelles ils auraient dû n’établir que le respect et l’amour.\par
On se plaint communément que les enfants n’ont pas pour leurs parents une tendresse égale à celle que les parents ont pour leurs enfants : l’amour paternel l’emporte communément, dit-on, sur la piété filiale. Rien de plus aisé que de se rendre compte de ce phénomène moral. Il est rare et presque impossible que le père le plus tendre ne fasse quelquefois sentir son autorité ; il le peut, il le doit. La jeunesse, presque toujours inconsidérée, force à tout moment un père à se souvenir qu’il est le maître ; il se trouve obligé de contrarier les goûts, les fantaisies, les inclinations de ses enfants. Dès lors, ceux-ci ne voient le plus souvent en lui qu’un maître, un censeur occupé à gêner leurs volontés et qui met des entraves à leur liberté. Or, l’homme étant par sa nature amoureux de sa liberté, la moindre gêne lui déplaît. La supériorité d’un père en impose presque toujours à son fils ; les bienfaits les plus grands et les plus réitérés sont à peine capables de contrebalancer en lui l’amour de l’indépendance, l’une des plus fortes passions du cœur humain. D’un autre côté, le bon père est un bienfaiteur, et les bienfaits ne font des ingrats que par la supériorité qu’ils donnent à ceux qui les font sur ceux qui les reçoivent. Voilà pourquoi les enfants sont sujets à l’ingratitude ; ils la font bientôt éclater quand l’éducation n’a pas fait disparaître à temps les symptômes de ce vice odieux.
\subsection[{Chapitre III. De l’Éducation}]{Chapitre III. De l’Éducation}
\noindent Après avoir prouvé que l’éducation des enfants est le plus important devoir des pères et mères, arrêtons-nous un moment sur cet objet essentiel. Nous avons vu que la plus grande partie du bonheur des parents dépendait nécessairement des sentiments qu’ils inspirent à leurs enfants ; d’un autre côté, il n’est pas douteux que rien n’est plus intéressant pour un être sociable que d’avoir des dispositions propres à lui mériter la bienveillance des autres ; enfin, toute société demande que ses membres contribuent à son bien-être.\par
L’{\itshape éducation} est l’art de modifier, de façonner et d’instruire les enfants de manière à devenir des hommes utiles et agréables à leur famille, à leur patrie, et capables de se procurer le bonheur à eux-mêmes.\par
« Il est, dit Théognis, bien plus facile de donner la vie à un enfant, que de lui donner une belle âme. » C’est ce que l’éducation doit pourtant se proposer. Tout a dû nous convaincre que l’homme n’apporte en naissant ni bonté ni méchanceté : il apporte la faculté de sentir ses besoins, qu’il est incapable de satisfaire par lui-même, des passions plus ou moins vives suivant l’organisation et le tempérament dont la Nature l’a doué. Élever un enfant, c’est se servir de ses dispositions naturelles, de son tempérament, de sa sensibilité, de ses besoins, de ses passions, pour le modifier ou le rendre tel que l’on désire. C’est lui montrer ce qu’il doit aimer ou craindre et lui fournir les moyens de l’obtenir ou de l’éviter ; c’est exciter ses désirs pour certains objets et les réprimer pour d’autres. Les passions bien dirigées, c’est-à-dire réglées d’une façon avantageuse et pour lui-même et pour les autres, conduisent l’enfant à la vertu ; ces passions abandonnées à leur fougue ou mal dirigées, le rendent vicieux et méchant.\par
Un moraliste célèbre\footnote{M. Helvétius. Voyez son livre {\itshape De l’Esprit}, discours III.} a cru que l’éducation pouvait tout faire sur les hommes et qu’ils étaient tous également susceptibles d’être modifiés de la façon qu’on désire, pourvu que l’on sût mettre leur intérêt en jeu. Mais l’expérience nous prouve qu’il est des enfants dans l’âme desquels on ne peut allumer aucun intérêt puissant : il en est qui n’aiment rien fortement ; il en est de timides et d’audacieux ; il en est qu’il faut pousser et d’autres que l’on peut à peine retenir ; il en est qu’un naturel stupide, une organisation fâcheuse, un tempérament rebelle rendent très peu susceptibles d’être modifiés. Nous voyons des âmes volatiles et légères que l’on ne peut aucunement fixer, tandis que d’autres sont tellement engourdies que l’on ne peut les animer par aucun moyen. C’est donc se tromper de croire que l’éducation puisse tout faire dans l’homme : elle ne peut qu’employer les matériaux que la Nature lui présente ; elle ne peut semer avec succès que dans un terrain préparé par la Nature, de façon à répondre aux soins que la culture lui donnera.\par
La première éducation s’occupe principalement à façonner, former, fortifier le corps de l’enfant, lui apprend à faire usage de ses membres, l’habitue à régler ses besoins, réprime les mouvements de ses passions lorsqu’elles se trouvent contraires à son propre bien : cette première éducation modifie déjà dans un enfant ses facultés intellectuelles d’une façon qui souvent influe sur le reste de sa vie.\par
Les parents ne paraissent pas faire assez d’attention à cette première partie de l’enfance ; on l’abandonne à des nourrices, puis à des gouvernantes, qui commencent par remplir les esprits de leurs élèves des craintes, des idées fausses, des vices et des folies dont elles sont imbues elles-mêmes. Entre leurs mains un enfant contracte l’habitude du mensonge, de la fausseté, de la pusillanimité, de la gourmandise, de la mollesse. Tantôt gâté par des caresses et des flatteries, tantôt corrigé mal à propos, il se trouve déjà rempli de passions opiniâtres qui n’ont pas été combattues, d’une foule d’erreurs et de préjugés tenaces qui le tourmenteront jusqu’au dernier soupir et que la seconde éducation, quand même elle serait plus raisonnable, ne pourra point déraciner. Les premiers moments de la vie, que l’on néglige trop communément, mériteraient une attention particulière : ils décident quelquefois pour toujours du caractère d’un enfant. Platon attribue la décadence où l’empire de Cyrus tomba depuis sa mort, à l’éducation de ses enfants confiée à des femmes qui flattaient leurs passions naissantes et ne leur inspiraient que des vertus dignes d’elles.\par
« Tu es homme, dit Ménandre, c’est-à-dire l’animal le plus sujet aux caprices du sort. » Cela posé, une éducation molle et efféminée ne convient pas même aux femmes, que l’on devrait fortifier au lieu de les rendre encore plus faibles que la Nature ne les a formées. Les vicissitudes auxquelles la vie humaine est sujette imposent aux parents les plus riches le devoir de ne point accoutumer l’enfance à la paresse, à l’indolence, au luxe, à la vanité ; il faut de bonne heure affermir le corps par l’exercice et la fatigue, et prémunir l’esprit contre les coups de la fortune. Rien de plus malheureux que les enfants dont les parents les ont rendus vains, sensuels, gourmands, délicats. Une pareille éducation peut un jour redoubler pour eux toutes les peines qu’ils seront forcés d’éprouver ; elle ôte aux hommes cette énergie, cette activité, cette force du corps qui convient à leur sexe. La mollesse, l’oisiveté et la volupté en font des membres inutiles à la société et fatigants pour eux-mêmes. Des enfants accoutumés au faste, à la délicatesse, à être toujours servis, se trouveront très souvent malheureux dans le cours de la vie lorsqu’ils se verront privés des commodités et des secours que l’habitude leur aura rendu nécessaires.\par
Les femmes devraient recevoir une éducation plus mâle : elle les rendrait plus robustes, capables de produire des enfants mieux constitués ; elle les garantirait d’une foule d’infirmités, de vapeurs, de faiblesses dont elles sont communément affligées.\par
Mais dès l’âge le plus tendre, l’éducation semble se proposer d’affaiblir le corps des enfants et de leur gâter l’esprit et le cœur par des idées fausses, par des passions dangereuses, et surtout par des vanités que trop souvent tout contribue à fixer en eux pour toujours. L’éducation subséquente, au lieu de détruire les impressions fâcheuses qu’ils ont reçues de leurs nourrices, des gouvernantes et des valets auxquels ils ont été livrés, les confirme pour l’ordinaire et les rend habituelles et permanentes. Comment des parents ou des instituteurs imbus eux-mêmes d’erreurs, de préjugés, de passions, de folles vanités, songeraient-ils à rectifier les vices de la première éducation ? Comment des pères et mères remplis de l’orgueil de la naissance, rongés d’ambition et de l’amour des richesses, épris des extravagances du luxe, de la parure, de la mode, anéantiraient-ils dans l’esprit de leurs enfants les idées fausses qu’on leur a données de ces choses dès l’âge le plus tendre ? L’éducation n’est pour l’ordinaire que l’art d’inspirer à la jeunesse les passions et les folies dont les hommes faits sont eux-mêmes tourmentés ; il faudrait avoir reçu soi-même une éducation raisonnable pour être en état de guider ses enfants dans le chemin de la vertu.\par
L’exemple des parents, comme nous l’avons fait sentir, contribue surtout à rendre leurs enfants vertueux ou vicieux. Cet exemple est pour eux une instruction indirecte et continuelle plus efficace que les leçons les plus réitérées. Un père est aux yeux de son enfant l’être le plus grand, le plus puissant, le plus libre, celui à qui il voudrait le plus ressembler.\par
Que sera-ce si les parents sont déréglés et sans mœurs ! « Les exemples domestiques, dit Juvénal, quand ils sont vicieux, corrompent d’autant plus vite que ceux qui les donnent en imposent davantage. Un ou deux enfants dont Prométhée forma le cœur d’une meilleure argile savent peut-être résister ; mais le reste obéit à l’impulsion fatale qu’il reçut en naissant. Que toutes nos actions soient donc irréprochables, de crainte que nos enfants ne s’autorisent de nos crimes ; car nous sommes tous imitateurs dociles de la perversité\phantomsection
\label{footnote92}\footnote{Juvénal, {\itshape Satires}, XIV, vers 31 et suivants.}. » Un enfant conçoit promptement le désir d’imiter ce qu’il voit faire aux personnes qui le gouvernent, parce qu’il les suppose plus instruites des moyens de se procurer du plaisir ; imiter, c’est essayer de se rendre heureux par les moyens qu’on voit employés par les autres. En vain des pères dissolus diront-ils à leur fils : « Faites ce qu’on vous dit, et ne faites pas ce que vous nous voyez faire. » L’élève, dans le fond de son âme, leur répliquera toujours, « Vous êtes libres dans vos actions et vous agiriez autrement s’il n’en résultait pour vous quelque plaisir que vous prétendez me cacher ; mais malgré vos leçons, je vous imiterai. »\par
À l’éducation particulière et aux exemples domestiques, souvent très dangereux vient se joindre par la suite l’opinion publique, communément très viciée. En sortant des mains de ses parents et de ses maîtres, le jeune homme n’est frappé dans le monde que d’exemples pervers, n’entend que des maximes fausses, trouve que la conduite de tous ceux qui l’entourent est dans une contradiction perpétuelle avec les principes qu’on a pu lui enseigner. Dès lors, il se croit obligé de {\itshape faire comme les autres} ; les idées saines que l’éducation aurait par hasard consignées dans sa tête sont bientôt effacées ; il suit le torrent et renonce à des maximes qui ne serviraient qu’à le faire passer pour ridicule ou singulier et qui lui fermeraient le chemin de la fortune.\par
Lycurgue regardait l’éducation des enfants comme la plus importante affaire du législateur. Néanmoins, le gouvernement, en tout pays, semble très peu s’occuper de celle des citoyens : cet objet essentiel pour la félicité publique est pour l’ordinaire totalement négligé. On dirait que ceux qui gouvernent les nations ne s’embarrassent aucunement de former des membres utiles à la société ; la morale est par eux regardée comme une science spéculative dont la pratique est parfaitement indifférente. Bien plus, des mauvais gouvernements n’ont ni la volonté ni la capacité de rendre leurs sujets vertueux ; la vertu déplaît aux tyrans et aux despotes : elle n’a pas la souplesse qu’ils demandent. Les idées de la justice et de l’humanité répandues dans les cœurs nuiraient aux intentions d’une politique perverse qui veut régner sur des automates.\par
Si, comme on l’a suffisamment prouvé, la justice est la vertu fondamentale sur laquelle la morale doit s’établir, il est clair que toute morale est bannie des nations soumises au despotisme ou à la tyrannie. En vain l’intérêt général dirait aux hommes d’être justes, tandis que la voix plus forte de l’intérêt personnel, appuyée par les maîtres de la terre, par les dispensateurs des dignités, des faveurs, des rangs et des richesses, leur crie à tout moment qu’avec la morale et la vertu on ne parvient à rien, on languit dans la misère et dans l’obscurité, et même on s’expose très souvent aux coups de la puissance. En un mot, tout fait voir qu’en suivant la voie de la justice on n’obtient aucun bonheur, et l’on risque à chaque pas d’être écrasé par la foule, qui suit un chemin directement opposé.\par
Conséquemment à ces principes et aux remarques qu’on est à portée de faire journellement dans les contrées soumises à de mauvais gouvernements, la vraie morale ne doit entrer pour rien dans l’éducation des citoyens : elle mettrait des obstacles invincibles et continuels à leur félicité, ou du moins elle les priverait des vains objets dans lesquels le commun des hommes la fait consister faussement. Ainsi, les maximes que dans chaque État l’on peut insinuer à la jeunesse seront très contraires à celles que la morale pourrait leur proposer. Quels avantages à la cour pourrait promettre à son fils le courtisan qui lui dirait d’être juste, de ne nuire à personne, de se montrer fermement attaché à la vertu, de placer en elle son honneur, de préférer cet honneur à sa fortune, à son avancement, à la faveur du prince et de ses ministres ? Il est évident que sous un mauvais gouvernement, de pareilles maximes conduiraient à la disgrâce et paraîtraient dictées par le délire. Le courtisan et le grand qui voudront ouvrir le chemin de la fortune à leurs enfants, leur donneront des instructions diamétralement opposées ; ils leur diront : « Ne connaissez d’autre règle que la volonté du maître ; qu’elle soit toujours juste à vos yeux ; ne lui résistez jamais ; sacrifiez-lui un honneur qui n’est rien s’il ne conduit à la puissance, au crédit, aux richesses auxquels votre rang doit vous faire prétendre ; l’unique honneur pour vous est d’être distingué par le prince ; apprenez qu’un bon courtisan doit être {\itshape sans honneur et sans humeur}\phantomsection
\label{footnote93}\footnote{Ce mot est attribué au duc d’Orléans, régent de France durant la minorité de Louis XV. On dit qu’un ministre moderne, fameux par ses ravages, voulant enseigner à ses fils la manière de se conduire dans le monde, se contenta de leur dire que l’on distinguait des hommes de deux espèces : les fripons et les honnêtes gens ; c’est-à-dire, disait-il, {\itshape les gens d’esprit et les sots} : qu’ils n’avaient qu’à choisir la classe à laquelle ils aimaient mieux appartenir.} ; l’honneur et la vertu ne sont point faits pour des esclaves destinés à recevoir toutes les impulsions de leur maître. »\par
L’éducation du jeune homme d’une illustre naissance lui apprendra que la noblesse transmise par ses aïeux doit lui suffire pour parvenir à tout, qu’il n’a besoin ni de science, ni de mérite personnel, ni de vertu ; que ces choses utiles à l’avancement de quelques citoyens obscurs et méprisables ne sont nullement nécessaires à celui que son nom seul doit porter aux grandeurs ; que la morale n’est bonne que pour amuser les loisirs de quelques vains spéculateurs ; que la justice, faite pour les faibles et le vulgaire, ne doit aucunement servir de règle aux grands, qui n’ont nul intérêt de se soumettre à ses lois trop gênantes. Si le noble se destine aux armes, il n’aura besoin ni de lumières ni de raison. Il faudra bien se garder de lui développer les principes de l’équité naturelle, qui trop souvent contrediraient les ordres des chefs, auxquels son métier l’obligera d’obéir en aveugle et sans jamais hésiter. Dès que le despote commande, le guerrier ne doit entendre ni les lois de la justice, ni le cri de la pitié, ni les gémissements de sa nation ; il est fait pour s’élancer les yeux fermés sur ses amis, ses concitoyens, ses parents mêmes. Tels sont les principes que l’éducation doit de bonne heure inspirer à des esclaves destinés à retenir d’autres esclaves dans les fers.\par
Un gouvernement pervers souffrira-t-il qu’on donne une éducation plus morale au jeune homme que l’on destine à la magistrature ? Celui dont l’état est de rendre la justice à ses concitoyens doit-il montrer pour elle un attachement inviolable ? Hélas ! Lui conseiller de s’attacher fermement aux lois de l’équité, ce serait le mettre dans une guerre continuelle avec le despote et ses ministres, qui voudraient les détruire, ce serait l’exposer à des avanies, à des exils, à des prisons, à des fers, ce serait le mettre en danger d’être enseveli sous les ruines du temple de Thémis, qui ne peut résister aux assauts furieux du dieu terrible de la guerre. Sous un gouvernement arbitraire, l’éducation ne peut enseigner aux gardiens, aux dépositaires des lois, que de les livrer aux caprices de la tyrannie, aux séductions de la faveur, aux violences du pouvoir. Pour réussir ou pour vivre tranquille, le magistrat doit être souple et faire plier la justice sous la volonté changeante du maître et de ses favoris. Il doit avoir deux balances : l’une pour l’homme riche et puissant, l’autre pour le faible et le pauvre.\par
Dans les pays où l’avidité du maître et les besoins de ses courtisans insatiables ont fait éclore la finance et multiplier les traitants, quelle éducation, quels principes des hommes accoutumés à s’enrichir par d’injustes rapines donneront-ils à leurs enfants ? Leur diront-ils d’être justes, humains, sensibles à la pitié, modérés dans leurs désirs ? Non, sans doute ; le financier recommandera au fils qu’il destine à son métier cruel, d’être dur, inhumain, impitoyable, d’avoir un cœur de fer, de sacrifier tout sentiment honnête ou généreux au désir d’augmenter sa fortune {\itshape ;} il lui dira de s’engraisser du sang des malheureux ; il lui fera voir que dans des richesses sans bornes consistent et l’honneur et la gloire d’un véritable financier\footnote{L’instituteur des enfants d’un financier s’étant plaint à leur père que les fils ne faisaient aucun progrès dans leurs études : « Apprenez-leur, dit ce père, l’arithmétique et la politesse, et ils en sauront assez pour vivre dans le monde. » Si le traitant doit être dur envers les malheureux, il doit être bas, prévenant, généreux envers ses protecteurs et les grands.}.\par
Le riche n’apprendra point à sa postérité la manière louable d’user de ses richesses. Ses descendants dépourvus d’instruction, de mœurs et de bienveillance, dissiperont follement les trésors amassés par l’injustice en débauches, en festins, en parures, en extravagances. Ils penseront n’être au monde que pour se livrer sans cesse à de vains amusements ; ils ne se croiront obligés de rien faire pour les autres ; ils tomberont dans l’ennui, qui toujours accompagne ou suit la paresse et le dérèglement ; ils se ruineront pour s’en tirer et n’auront jamais éprouvé la félicité pure que la vertu réserve à ceux qui dès la jeunesse ont appris à la goûter.\par
Enfin, les gens du peuple, toujours abrutis et privés de raison sous des gouvernements négligents ou pervers, n’auront aucune idée de la vertu ni des mœurs. Dépravé par l’exemple de ses supérieurs ou tourmenté par leurs vexations, l’homme du peuple devient méchant et peu capable d’inspirer à ses enfants des sentiments honnêtes qu’il n’a pu acquérir par lui-même et que ses parents malheureux ne lui ont point transmis.\par
On nous dira peut-être que dans toutes les nations les ministres de la religion sont chargés d’enseigner la morale et d’inculquer ses préceptes à la jeunesse, mais l’expérience nous fait voir l’impuissance de leurs leçons contre le torrent impétueux qui entraîne sans cesse les hommes au mal. Les motifs que la religion leur présente sont souvent trop relevés, trop spirituels, trop au-dessus de l’intelligence des mortels grossiers pour les déterminer au bien. Les moralistes religieux se plaignent eux-mêmes de l’inutilité, de l’inefficacité de leurs préceptes répétés à tout moment {\itshape ;} s’ils agissent sur quelques âmes tranquilles, timorées, capables de les méditer, ils ne peuvent rien sur le grand nombre que des forces irrésistibles semblent pousser au vice.\par
Indépendamment de la dépravation innée que la religion révélée impute à la nature humaine, on peut expliquer le penchant si marqué qui porte les hommes au mal par des causes naturelles et sensibles que nous voyons agir sous nos yeux. Ces causes sont : l’ignorance profonde dans laquelle on voit croupir les nations, les exemples funestes des riches et des grands, imités par les pauvres, la négligence des législateurs, qui paraissent communément s’être très peu souciés de donner des mœurs aux peuples ou qu’on leur fît connaître leurs intérêts, leurs vrais rapports et les devoirs les plus essentiels à la vie sociale.\par
Enfin, la plus puissante de ces causes, c’est la fausse politique de tant de princes eux-mêmes aveugles qui trop souvent semblent vouloir anéantir toute idée de justice ou de vertu dans leurs États, et qui croient n’être grands qu’en régnant sur des sujets stupides, vicieux, en discorde pour de futiles intérêts.\par
Les peuples sont des pupilles dans lesquels leurs tuteurs paraissent craindre que la raison ne vienne à se développer. L’art de gouverner les hommes n’est pour la plupart des souverains de la terre que l’art de les tromper, de les tenir dans l’aveuglement, afin de les dépouiller et de les sacrifier impunément à toutes leurs fantaisies. Les passions effrénées des tyrans, la corruption des cours, voilà les causes visibles et naturelles de l’ignorance, de la dépravation et des calamités qui font gémir les habitants du monde.\par
En vain les ministres de la religion continueront d’inculquer à la jeunesse les préceptes d’une morale divine appuyée sur les récompenses et les punitions d’une autre vie\footnote{Voyez section V, chap. 9.}. En vain la philosophie présenterait aux hommes une morale humaine fondée sur les avantages sensibles que la vertu peut procurer dans la vie présente. Les promesses, les menaces et les motifs surnaturels de la religion seront toujours trop faibles pour rendre les hommes meilleurs ; les motifs humains du philosophe et les biens qu’il promet en ce monde, paraîtront des chimères tant que la morale aura pour ennemis les princes qui tiennent dans leurs puissantes mains les mobiles les plus capables de faire agir les mortels sur la terre.\par
Il ne faut donc pas s’étonner si l’éducation est négligée, découragée, méprisée ou même très inutile dans des nations abruties, corrompues et mal gouvernées. Les maximes les plus évidentes de la morale se trouvent à chaque instant contredites par des exemples, par des usages, par des institutions, par des lois, par des intérêts assez puissants pour contrebalancer l’intérêt général. Tout le monde est sollicité au mal et personne ne trouve d’intérêt à faire le bien. De là ces embarras infinis dans lesquels se sont jeté tous ceux qui ont essayé de donner des plans d’éducation propres à former des citoyens. Ils n’ont pas vu, sans doute, que les meilleurs systèmes en ce genre ne pouvaient aucunement se concilier avec les préjugés du vulgaire et les vues sinistres de ceux qui règlent les destinées des peuples : ils ne se sont pas aperçu que les États despotiques ne voulaient pas qu’on formât de bons citoyens, ils n’ont pas senti que la saine morale est incompatible avec une fausse politique et que pour élever les hommes d’une manière conforme aux intérêts de la société, il fallait commencer par faire goûter la saine morale à ceux qui gouvernent le monde, leur faire connaître leurs intérêts véritables, afin de les porter à seconder cette morale par les lois, par les récompenses et les châtiments dont ils sont dépositaires. En un mot, ces philosophes n’ont pas senti que la réforme de l’éducation dépendait nécessairement de la réforme des mœurs publiques, qui ne peut être l’ouvrage que d’un gouvernement éclairé, vigilant, équitable et bien intentionné.\par
Le gouvernement seul peut faire régner dans un État les vertus générales et les mœurs publiques. C’est du temps et du progrès des lumières que l’on peut attendre cette révolution si désirable dans les esprits des maîtres de la terre. Jusqu’à ce temps fortuné, les hommes, pour leur bonheur particulier, seront réduits à se contenter de la pratique des vertus convenables à la vie privée, dont la morale leur montrera l’utilité, même au sein des nations les plus dépravées, et qu’une bonne éducation inspirera dès l’enfance à ceux qui pourront en connaître les avantages inestimables. Plus la société est corrompue, plus le gouvernement exerce de rigueurs, et plus les citoyens honnêtes se trouvent obligés de se concentrer en eux-mêmes pour y chercher le bien-être que la patrie est alors incapable de leur procurer.\par
L’éducation, à proprement parler, ne devrait être que la morale inculquée à la jeunesse et rendue familière dès l’âge le plus tendre. Élever un jeune homme, c’est lui apprendre ses devoirs envers tous ceux avec lesquels il aura des rapports, c’est lui enseigner la conduite qu’il doit tenir envers ses parents, c’est lui faire sentir l’intérêt qu’il a de mériter leurs bontés, c’est lui montrer comment il doit se comporter avec les grands et les petits, les riches et les pauvres, ses amis et ses ennemis. Les devoirs d’un État ne sont que les règles indiquées par la morale dans les diverses positions de la vie. L’éducation d’un prince devrait se proposer de lui faire connaître ses devoirs envers son peuple et les différentes nations dont il est entouré ; elle devrait le rendre juste, humain, tempérant, modéré, et lui présenter les intérêts qui l’invitent à pratiquer les mêmes vertus sociales que les particuliers. C’est, comme on l’a prouvé, faute d’élever les princes dans ces maximes, que tourmentés toute leur vie de passions et de vices, ils rendent malheureuses les nations dont ils sont obligés de faire le bonheur.\par
L’éducation des riches et des grands devrait avoir pour objet de les mettre à portée de faire un bon usage des richesses et des emplois qu’ils posséderont un jour ; elle devrait leur montrer les devoirs que la morale leur prescrit envers leurs concitoyens comme les seuls moyens de mériter l’estime, la considération, les respects, qui ne sont dus qu’à la bienfaisance, à l’équité, à l’affabilité, à la noblesse des sentiments.\par
Mais les enfants destinés à jouer les rôles les plus importants dans la société sont communément ceux dont l’éducation est la plus mauvaise et la plus honteusement négligée. On ne songe aucunement à briser l’humeur, à dompter le caractère, à combattre les caprices, à réprimer les passions des enfants de race illustre : ils apprennent dès le berceau qu’ils sont faits pour commander, qu’ils sont au-dessus des règles et des lois, que tout doit plier devant eux, qu’ils n’ont besoin ni de sciences ni de talents pour obtenir les distinctions auxquelles leur naissance les appelle. Ce seront pourtant ces enfants volontaires qui régleront un jour les destinées des peuples ! Les enfants nés dans l’opulence ne sont pas moins gâtés : ils savent dès l’âge le plus tendre la distance que la richesse met entre les hommes, ils deviennent insolents ; les faiblesses des parents aussi bien que leurs négligences leur laissent prendre des plis qui ne s’effaceront jamais. Rien de plus important que d’apprendre de bonne heure à l’homme à fléchir sous la nécessité et à se conformer aux vues de la société dont un jour il doit être un membre utile et agréable.\par
En effet, l’éducation ne peut avoir pour objet que de faire connaître aux hommes la manière dont ils doivent agir dans tous les états de la vie : comme rois, comme nobles, comme ministres, comme magistrats, comme parents, comme amis, comme associés. Ainsi, l’éducation n’est jamais que la morale présentée aux hommes dans leur enfance pour leur enseigner leurs devoirs dans les rapports divers qu’ils auront un jour les uns avec les autres.\par
Quelque variés que paraissent ces rapports ou ces circonstances, une éducation vraiment sociale enseignera la même morale à tous les hommes dans tous les états de la vie. Elle leur fera sentir qu’ils doivent être justes et bienfaisants envers tous les êtres de l’espèce humaine : c’est à quoi se bornent, comme on a vu, tous les devoirs de l’homme, qui se réduisent à la justice envisagée sous tous ses points de vue. L’éducation ne peut se proposer que d’habituer les hommes dès leur enfance à réprimer les passions contraires à leur propre bonheur et à celui des autres, et à leur fournir les motifs capables de les y porter. En montrant leurs esclaves dans le délire de l’ivresse, les Lacédémoniens se proposaient d’exciter de bonne heure dans leurs enfants de l’horreur pour un vice qui dégrade l’homme et le met au-dessous des bêtes. En punissant un enfant d’une faute ou d’une impertinence, on lui montre qu’en commettant certaines actions il déplaît et par là même devient malheureux ; ainsi, l’on oppose la crainte à ses désirs inconsidérés, et cette crainte, changée en habitude, se trouve assez forte pour contenir sa témérité, à laquelle, sans la correction, il donnerait un libre cours, ce qui le rendrait insupportable un jour dans la société.\par
L’éducation, pour être efficace, devrait être une suite d’expériences qui prouveraient sans cesse aux enfants que le mal qu’ils font aux autres finit toujours par retomber sur eux-mêmes. Dès qu’ils se montreraient injustes envers leurs camarades, on devrait aussitôt leur faire éprouver une injustice pareille, dès qu’ils frapperaient quelqu’un on les frapperait à leur tour, dès qu’ils montreraient de la hauteur on aurait soin de les humilier et de leur faire sentir qu’un valet mérite des égards, comme homme, de la part de ceux qui ont droit d’exiger ses services mais qui n’ont jamais celui de le mépriser parce qu’il est pauvre ou malheureux. Cette éducation expérimentale soigneusement observée serait plus imposante que des préceptes stériles que l’on se contente pour l’ordinaire de jeter vaguement, ou même que l’on ne donne jamais aux enfants gâtés de la fortune. Faute d’observer ces règles si naturelles, la société se trouve remplie d’hommes injustes, vains, opiniâtres, fougueux. Ils portent dans la société des vices et des défauts qui, n’ayant pas été réprimés à temps, les rendent incommodes, désagréables pour les autres et font que souvent ils essuient mille désagréments qu’ils auraient évités s’ils eussent reçu une éducation plus soignée.\par
Mais pour inspirer de bonne heure à l’enfance ou à la jeunesse des idées de justice, il est très important que les parents et les instituteurs se montrent justes à l’égard de leurs élèves. Une éducation capricieuse, despotique et guidée par l’humeur, révolterait les disciples, les dégoûterait de ses leçons et ne servirait qu’à confondre dans leur esprit les notions de l’équité. Des personnes emportées, impatientes, d’un caractère variable, ne sont point propres à former la jeunesse et à fixer ses idées. L’éducation demande de la douceur, du sang-froid et surtout une conduite ferme et soutenue. Il faut que l’enfant reconnaisse lui-même la justice dans les châtiments qu’on lui inflige ainsi que dans les récompenses qu’il reçoit. Il faut qu’il sente l’équité et l’utilité des motifs qui déterminent les maîtres soit à la sévérité, soit à la tendresse : une rigueur injuste les fait regarder comme des tyrans odieux, des caresses déplacées seront prises pour des marques de faiblesse. Il est difficile de bien élever des enfants qui se voient alternativement les jouets, soit de la mauvaise humeur non motivée, soit de la tendresse aveugle de leurs parents ou de leurs maîtres : entre de pareilles mains leurs esprits ne prennent point de fixité. Voilà pourquoi les femmes, communément dominées par des humeurs et des sentiments variables, sont peu capables d’élever les enfants, de leur inspirer des principes constants propres à régler uniformément la conduite de la vie. C’est à l’éducation que l’on doit attribuer l’inconstance, la faiblesse, l’instabilité du caractère et des idées que l’on trouve dans la plupart des hommes.\par
Une éducation négligée laisse dans les hommes des impressions ineffaçables. C’est dans l’âge tendre qu’il faut empêcher les passions, les vices et les défauts de naître, ou qu’il faut du moins forcer les enfants de les contenir : par là ils prennent l’habitude de les maîtriser. C’est surtout à l’orgueil, si souvent caressé dans les enfants des princes et des grands, qu’il faut déclarer la guerre. Une éducation très différente de celle qu’on leur donne communément devrait effacer jusqu’aux dernières traces de ce mépris insultant que l’enfance conçoit de si bonne heure pour l’indigence. Elle devrait lui faire sentir à chaque instant le besoin que l’opulence et la grandeur ont de ces hommes qu’elles ont l’ingratitude de mépriser et de repousser durement. Elle devrait apprendre à ne jamais dédaigner quiconque travaille soit pour satisfaire les besoins des grands, soit pour leur fournir les commodités et les plaisirs de la vie. Ainsi formé, l’élève deviendrait juste, il respecterait l’utilité, il serait reconnaissant, il trouverait que le cultivateur ou l’artisan, sous des haillons, couvrent souvent des hommes plus intéressants, plus nécessaires à leurs concitoyens et par conséquent plus estimables, que le courtisan inutile ou méchant qu’il voit chargé de titres, de dorures, de broderies, de rubans.\par
En réprimant ainsi l’orgueil de son élève, en lui faisant sentir sa propre faiblesse et le besoin continuel qu’il a des hommes qui lui paraissaient les plus abjects, on fera naître en lui la sensibilité, disposition si précieuse dans la vie sociale ; il s’intéressera au sort du malheureux, qu’il voit si nécessaire à son propre bien-être.\par
On aura soin de cultiver en lui cette bienveillance humaine et tendre ; on remuera son cœur par des secousses fréquentes, par des tableaux touchants présentés à ses yeux et capables d’agir sur l’imagination. On le conduira dans la cabane du pauvre, près du lit du malade ; on lui montrera les détails de la misère de l’homme utile qui, souvent entouré d’une famille désolée, manque de tout pour mettre le riche dans l’aisance. On le fera méditer sur les infortunes sans nombre sous lesquelles gémissent tant de mortels, ses semblables. On lui fera contempler surtout ceux que les coups du sort ont précipités dans la misère ; on lui dira que leurs malheurs sont les effets du hasard, dont les caprices en font des victimes innocentes tandis que ces mêmes caprices placent les grands et les riches dans l’abondance et les honneurs.\par
Ainsi, l’élève ne s’enorgueillira point de cette aveugle préférence ; il éprouvera le sentiment de la pitié ; il partagera les peines des infortunés, elles passeront en lui-même ; il se félicitera de se voir en état de les soulager {\itshape \textbackslash} il goûtera le doux plaisir de la bienfaisance ; il verra couler les larmes de la gratitude ; il se félicitera de les avoir méritées ; enfin, il reconnaîtra que le véritable avantage qu’un homme puisse avoir sur les autres consiste uniquement dans le pouvoir de les rendre heureux.\par
C’est ainsi que la vertu s’apprend. Voilà comment l’éducation peut donner un cœur sensible ; elle peut ainsi jeter dans les esprits des semences salutaires, les nourrir, les faire éclore et former des citoyens honnêtes, modestes, compatissants. C’est par des leçons de cette espèce que l’on devrait façonner l’enfance et la jeunesse de ces hommes faits pour occuper un rang distingué dans le monde. Quelle que fût la position où la fortune dût les placer, ils n’oublieraient pas qu’ils sont hommes et qu’ils ont besoin des hommes pour leur propre félicité. Mais faute d’avoir appris à connaître les infortunes de leurs semblables et d’avoir éprouvé le plaisir de les faire cesser, les hommes à la prospérité desquels rien ne devrait manquer sont communément gonflés d’un orgueil insociable ; pleins d’estime pour eux-mêmes, à peine laissent-ils tomber leurs regards dédaigneux sur des êtres qu’ils supposent inutiles pour eux-mêmes et d’une espèce inférieure. Ils n’ont point appris à aimer, à s’attendrir sur les misères, à sentir les charmes de la bienfaisance. L’on ne voit partout que des riches et des grands orgueilleux, injustes, insensibles, inhumains, qui, dépourvus de tout sentiment d’affection, ne peuvent transmettre à leur postérité que l’indifférence, l’apathie, la vanité qui les endurcissent contre les malheureux.\par
S’il est peu de parents qui sentent l’importance d’une bonne éducation, il en est encore bien moins qui soient capables de la donner eux-mêmes ou d’y veiller attentivement. Un père est trop occupé de ses affaires et souvent de ses plaisirs pour penser à former le cœur de son fils. Une mère dissipée ne songe qu’à sa parure, à ses amusements, et quelquefois à ses galanteries ; elle croirait s’avilir si elle songeait à ses enfants\footnote{« Qui ne voit, dit Montaigne, qu’en un état tout dépend de son éducation et nourriture ? Et cependant, sans aucune discrétion, on le laisse à la merci des parents, tant fols et méchants qu’ils soient. » Voyez {\itshape Essais}, livre II, chap. XXXI, vers le commencement.}. Par là les enfants des grands et des riches sont communément abandonnés à des domestiques qui ne leur apprennent rien de bon. C’est surtout dans leur commerce que les enfants se plaisent. Dans l’antichambre ou la cuisine ils jouent un rôle qui flatte leur vanité naissante ; ils n’y sont point contrariés, ils y exercent librement une sorte d’empire sur des êtres subordonnés. Il n’est rien qu’ils apprennent plus promptement que les prérogatives que la naissance et l’opulence donnent à ceux qui les posséderont un jour ; les premières leçons qu’ils reçoivent sont des leçons de hauteur, d’impertinence, de vice, que rien ne pourra par la suite effacer. En sortant des mains des valets et des gouvernantes, l’enfant d’un homme riche est mis dans celles d’un instituteur qui souvent n’a lui-même aucune des qualités nécessaires pour former le cœur et l’esprit de son élève. Quand même un heureux hasard l’aurait pourvu des talents les plus rares, il ne pourrait les employer utilement pour corriger un disciple indocile et déjà perverti de longue main. La douceur est déplacée avec un enfant hautain ; la rigueur le révolte et déplaît souvent à ses parents, assez vains pour exiger que l’on respecte leur sang jusque dans les sottises de leurs enfants. Ainsi, l’instituteur contredit est bientôt découragé ; il devient indifférent et finit par ne s’embarrasser nullement des progrès de son élève, qu’il abandonne à son mauvais sort. Voilà comment l’éducation particulière forme si peu de sujets remarquables.\par
D’ailleurs, comment les grands et les riches trouveraient-ils des instituteurs éclairés et vertueux tandis que le mérite n’est point senti par eux ou devient même souvent l’objet de leurs dédains ? Le noble ne fait cas que de la naissance, le riche n’estime que l’opulence : ils ne peuvent concevoir qu’un savant pauvre puisse mériter les égards des personnes de leur sorte. Celui qu’ils ont chargé de l’instruction de leurs enfants n’est à leurs yeux qu’un mercenaire, un valet renforcé qu’ils ne distinguent souvent des autres que par des mépris humiliants. Il n’y a qu’un père éclairé lui-même qui sente vraiment l’importance du dépôt qu’il confie aux soins d’un autre ; il voit dans le gouverneur de son fils un ami respectable qui veut bien se charger de contribuer avec zèle à son bonheur et à celui de sa postérité. L’insensé qui méprise l’instituteur de son fils ne sait donc pas que c’est de lui que dépend le bien-être et l’honneur de sa famille ? « Vous donnez votre fils à élever à un esclave, disait un philosophe à un père opulent et avare ; eh bien ! au lieu d’un esclave, vous en aurez deux. » Pour rendre l’éducation utile, il faut que celui qui s’en charge se respecte lui-même et soit respecté des autres : un enfant qui s’aperçoit que ses parents ont peu d’égards pour son maître ne tarde pas à le mépriser ; d’ailleurs, il le hait comme un censeur continuel ou comme son ennemi. Les bons instituteurs sont rares parce que rien n’est plus rare que des parents qui sachent démêler le mérite obscur, l’apprécier équitablement, lui montrer les sentiments qui lui sont dus. Cette équité reconnaissante suppose des réflexions et des vues qui ne se trouvent guère dans les êtres superbes et dissipés entre les mains desquels la fortune va communément se placer. Chez les Grecs et les Romains la science était considérée ; les souverains, les généraux d’armée, les hommes d’État la cultivaient eux-mêmes et montraient une profonde vénération à ceux qui se livraient aux soins de former la jeunesse. Mais par une suite des préjugés barbares qui subsistent encore chez la plupart des nations modernes, la noblesse dédaigne de s’instruire ; elle se glorifie de son ignorance, qui ne l’empêchera nullement de parvenir aux honneurs militaires qu’elle ambitionne. L’exercice du cheval, l’escrime, la danse, une démarche assurée, un maintien libre et gracieux, une politesse verbale et souvent peu sincère, un jargon propre à plaire aux femmes, voilà les perfections que l’éducation des grands se propose de leur donner. La culture de l’esprit et la science des mœurs n’entre pour rien dans les calculs de la noblesse ; le métier de la guerre dispense d’avoir des lumières et des vertus. Les grands suppléent au défaut de connaissances et d’étude par des vices, des amusements, des dépenses qui communément ne tardent pas à déranger leur fortune. Quant à cette noblesse engourdie qui végète dans le fond de ses terres, elle ne s’occupe que de la chasse ou du jeu et n’a pour toute étude que la connaissance futile de sa généalogie et de celle de ses voisins.\par
Le riche qui par ses travaux pénibles ou par ses injustices et ses bassesses est parvenu à s’enrichir, s’embarrasse fort peu que son fils ait des connaissances et des vertus ; il regarde l’étude comme un temps perdu, les mœurs comme inutiles et la probité sévère comme un obstacle à la fortune. L’éducation qu’il trouve la plus intéressante pour son fils est celle qui apprend la bassesse, la souplesse, l’art de plaire aux grands pour acquérir le droit de dépouiller le pauvre.\par
Il est peu de parents et d’instituteurs qui soient doués des qualités requises pour élever la jeunesse. Ceux qui se chargent de ce soin important, indépendamment de la science et de l’esprit, devraient connaître l’homme, étudier le caractère, les facultés, les penchants des élèves qu’ils ont dessein de former. L’expérience nous prouve que tous les enfants n’ont pas les mêmes dispositions naturelles et ne sont pas toujours propres à répondre aux vues qu’on a sur eux. À quoi bon tourmenter et punir un enfant à qui la Nature a souvent refusé l’activité, la pénétration, la mémoire, et presque toujours le pouvoir de prêter une attention suivie aux objets qu’on lui présente ? La violence, la rigueur, des châtiments réitérés sont-ils des moyens propres à exciter l’amour de l’étude dans des âmes que l’on afflige et qu’on dégrade ? La douceur, la patience, la persuasion, l’indulgence, la bonne humeur sont des moyens bien plus sûrs de gagner la jeunesse, que la colère et la dureté.\par
Bien des pères instruits eux-mêmes et remplis d’enthousiasme pour la science, voudraient faire de leurs enfants des prodiges ; mais ne savent-ils pas que l’éducation ne fait des prodiges que lorsque la Nature lui fournit des matériaux nécessaires pour les exécuter ? Les enfants précoces ou prodigieux finissent le plus souvent par devenir des hommes très médiocres ; il ne faut pas s’en étonner : pour s’exercer avec succès, il faut que les organes aient pris de la consistance et de la vigueur ; exiger qu’un enfant montre une application suivie, c’est vouloir qu’il soit plus fort que son âge ne le comporte. Les disciples que l’on veut faire trop promptement avancer dans la carrière des sciences, ou se rebutent, ou sont bientôt épuisés par les efforts qu’on leur demande. Ceux dont on prétend faire des prodiges n’ont d’ordinaire que beaucoup de mémoire et très peu de jugement ; ce sont des machines frêles dont on a trop tendu les ressorts. Quant à ceux qui réfléchissent avant d’être parvenus à la maturité, ils sont communément d’une santé délicate qui les fait périr de très bonne heure. « Ne serre point, dit Phocylide, trop fortement la main d’un tendre enfant\phantomsection
\label{footnote94}\footnote{Voyez {\itshape Phocylidis carm.}}. »\par
Que les pères sensés ou les instituteurs de la jeunesse, par une sotte vanité, ne s’obstinent donc pas à forcer la Nature : qu’ils la consultent et la secondent, sans jamais la traverser. Dans l’âge tendre, l’esprit affamé de sensations a besoin de voltiger ; il ne peut ni se fixer, ni mettre de la suite dans ses travaux. Plus l’imagination est active, moins elle souffre la contrainte. Au lieu de l’amortir, il est bon de profiter de cette curiosité remuante qui, quand on la dirige sagement, est une disposition très favorable. Il est donc important de ne point occuper la jeunesse trop longtemps des mêmes objets. En variant les études on en fait un amusement, et les maîtres sont à portée de démêler les penchants qui s’annoncent dans leurs élèves ; ils se garderont bien de les contrarier. Un des plus grands défauts de l’éducation ordinaire, c’est d’être despotique, avilissante, capable d’étouffer les plus puissants ressorts de l’âme. Les parents et les maîtres ne parlent à leurs disciples que comme à des esclaves ; ils ne s’adressent qu’à leur crédulité ; ils jugent au-dessous de leur dignité de raisonner avec eux, de leur exposer les motifs de leurs préceptes, de leur faire reconnaître l’équité de leurs demandes et l’intérêt que le disciple doit trouver à s’y rendre. Cette éducation servile ne peut faire que des automates dépourvus de raison, étrangers à tous principes, toujours incertains et flottants, incapables de juger par eux-mêmes, guidés pendant le reste de leur vie par les lisières de l’habitude et de l’autorité. Ou bien cette éducation peu raisonnée rencontre, dans les têtes actives, des rebelles en garde contre des leçons qu’ils croient n’avoir pour base que les caprices des tyrans qu’ils détestent.\par
C’est en compatissant à la faiblesse du jeune âge, c’est en se proportionnant à sa force, c’est en se rapetissant, pour ainsi dire, en sa faveur, que consiste le grand art d’élever la jeunesse. Voilà comment le père ou l’instituteur, dépouillant la doctrine de ce qu’elle a de farouche, lui concilieront l’amitié de leurs élèves. Il faut raisonner avec son disciple si l’on veut en faire un être raisonnable. Il faut ne jamais le tromper si l’on veut mériter sa confiance et son respect ; une éducation despotique ne peut former que des méchants ou des sots.\par
Des parents raisonnables iront-ils se désoler parce que leurs enfants n’ont pas les penchants, l’esprit et les goûts qu’ils ont eux-mêmes ? Haïront-ils leurs descendants parce que le destin ne leur a pas donné ni les mêmes traits du visage, ni les mêmes facultés intellectuelles ? Loin de tout père équitable ces sentiments dénaturés ! S’il ne peut faire un savant de son fils, il peut du moins se promettre d’en faire un honnête homme. Les grands talents sont le partage d’un petit nombre de mortels mais tout être susceptible de raison peut apprendre à chérir la vertu, à connaître ses avantages, à sentir la force des motifs qui doivent la faire pratiquer. Il n’est pas d’élève en qui, si l’on s’accommodait à son âge, on ne pût dès sa plus tendre enfance semer les germes de la sagesse. Il est plus important pour un père que son fils devienne un jour juste, reconnaissant, sensible à ses bienfaits, compatissant pour sa vieillesse, que de le voir devenir un homme de goût, un érudit, un géomètre, un jurisconsulte, un métaphysicien. Il importe plus à la société d’être peuplée de gens de bien que de gens de lettres méchants, de savants sans probité, de poètes adulateurs, de gens d’esprit sans mœurs. Il faut aux familles des cœurs honnêtes, il faut aux nations des citoyens vertueux. Les riches et les grands éprouvent très rarement le plaisir d’être pères. Ce n’est qu’en donnant aux enfants une bonne éducation qu’on acquiert pleinement les droits de la paternité. L’éducation pose les fondements de la félicité future et des parents, et des enfants, et des familles, et des sociétés. Pour bien des gens, la qualité de père ne paraît les obliger à rien ; pour d’autres elle n’est qu’un pénible fardeau dont ils veulent se décharger à tout prix.\par
Il serait néanmoins plus prudent qu’un père ne perdît point ses enfants de vue : nul être n’est plus intéressé que lui à leur former le cœur de manière à les faire contribuer un jour à son propre bien-être. C’est sous les yeux de parents soigneux et tendres que les enfants contracteront cet attachement mêlé de crainte et de respect qui constitue la piété filiale. En éloignant d’eux leurs enfants pour les abandonner totalement à une autorité étrangère, les parents semblent renoncer à leurs droits les plus chers ; ils deviennent, pour ainsi dire, des inconnus pour leur postérité. Qu’ils ne soient point étonnés s’ils ne retrouvent un jour dans des enfants ainsi abandonnés que des sujets rebelles peu façonnés au joug qu’ils doivent porter sans cesse. Durant leur exil de la maison paternelle, ils auront appris bien des choses qu’ils devraient ignorer, ils auront contracté des passions, des défauts, des habitudes que leurs parents voudront en vain combattre et déraciner. Pour lors, ces enfants indociles ne verront dans les nouveaux maîtres, à l’autorité desquels ils ne sont pas accoutumés, que des usurpateurs, des censeurs, des tyrans, des ennemis. Tels sont les fruits que recueillent communément tant de pères qui n’ont pas eu le soin de semer et de cultiver la vertu dans les cœurs de leurs enfants. Ceux-ci causent à leurs parents des chagrins aussi longs que la vie et qui souvent les conduisent au tombeau\footnote{Bien des pères négligents pourraient s’approprier la sentence d’un Arabe qui dit : « Tout ce que tu plantes dans ton jardin te sera de quelque utilité ; mais si tu plantes un homme, il te déracinera peut-être un jour. » Voyez {\itshape Sentences arabes}.}. Si l’éducation domestique ou particulière est souvent défectueuse et négligée, l’éducation publique fut jusqu’ici très peu capable de procurer des avantages plus réels à la société. Elle est communément confiée à des hommes qui n’ont ni les lumières, ni les qualités nécessaires pour faire ni des époux vertueux, ni des pères de familles, ni des hommes d’État, ni même de bons citoyens. Dans presque toutes les nations, l’éducation n’est qu’un despotisme exercé par des pédants sans expérience du monde, sur une jeunesse qu’ils tourmentent sans fruit : leur projet semblerait être de faire perdre tristement le temps à des enfants dont les parents cherchent à se débarrasser. Ces instituteurs font communément débuter leurs élèves par l’étude abstraite d’une grammaire inintelligible, qui les mène à la connaissance de quelques langues mortes, que très peu d’entre eux, au sortir de leurs études, possèdent passablement. Mais la routine, qui jamais ne raisonne, est la loi qui gouverne ces maîtres ; ce serait pour eux un crime d’oser s’en écarter.\par
Les lettres, la poésie, l’éloquence, les écrits sublimes des Anciens sont sans doute très capables de remplir agréablement les moments de ceux qui de bonne heure ont goûté les charmes de l’étude ; mais ces plaisirs sont stériles s’ils ne sont accompagnés d’utilité. Qu’un homme ait appris à sentir toutes les beautés d’Homère, de Virgile et d’Horace, quel bien en résulte-t-il pour la société s’il n’a point en même temps appris à être bon père, bon ami, bon citoyen ? L’esprit le plus orné est inutile aux autres s’il ne s’est habitué à la vertu, toujours inséparable de l’amour du genre humain. Une éducation qui ne fait que des savants ne peut pas être comparée à celle qui ferait des gens de bien, beaucoup plus nécessaires à la vie sociale que des érudits dont souvent les recherches ne mènent à rien, ou des beaux esprits quelquefois très étrangers aux devoirs de la société. C’est par le cœur que l’éducation devrait toujours commencer : l’utilité de l’homme est le vrai but de toutes les connaissances humaines. C’est vers elle, comme vers un centre commun, que les sciences, les lettres et les arts devraient se rapporter. Rien de plus facile dans notre siècle que de procurer à la jeunesse une éducation qui la mettre à portée de s’orner l’esprit à l’aide des chefs-d’œuvre de la Grèce et de Rome, et de se former le goût ; mais rien de plus difficile que de lui donner des mœurs honnêtes.\par
Le défaut le plus grand de l’éducation publique, c’est d’être banale, ou de n’être adaptée ni aux caractères, ni aux dispositions naturelles, ni aux penchants des enfants qui la reçoivent, ni aux professions diverses auxquelles les parents les destinent. Le noble et le roturier, l’enfant du militaire et du magistrat, les fils des grands et des pauvres, les disciples pénétrants et stupides reçoivent les mêmes leçons que des élèves destinés à faire des cénobites, des théologiens et des prêtres. Ce sont en effet ces derniers qui sont chargés en tout pays de former des citoyens et partout ils ne les forment que pour les connaissances dont ils ont besoin eux-mêmes dans leur profession. Ceux qui ont le mieux profité de l’éducation publique possèdent du grec et du latin, ont parcouru l’Antiquité tant sacrée que profane, ils ont la mémoire chargée de mots ; mais ils n’ont rien appris de ce qu’il faudrait savoir pour remplir les devoirs de l’état qu’ils auront dans le monde. Que dirons-nous de cette science abstraite et ténébreuse qui, usurpant impudemment le nom de {\itshape philosophie}, termine ordinairement l’éducation publique ! On dirait que bien loin d’instruire la jeunesse, cette prétendue philosophie ne se propose que de jeter l’esprit humain dans des pièges dont il ne puisse se tirer ; par son moyen tout devient problème, obscurité. L’art de raisonner enveloppé de termes barbares ne semble fait que pour dégoûter les bons esprits de la raison et de la recherche de la vérité. Cette vaine logique hérissée de subtilités sert d’introduction à une métaphysique escarpée, aérienne, dans laquelle l’imagination perpétuellement égarée cherche à sonder péniblement des profondeurs impénétrables complètement étrangères au bien-être de la société. Cette éducation nationale toujours guidée par la routine, qui lui paraît sacrée, ne donne à ses élèves que de faibles notions de la Nature. La physique entre ses mains ne suit que rarement la marche de la raison, qui ne peut reconnaître que l’expérience pour son guide, et qui, mûrie par le temps, est faite pour s’élever au-dessus des vaines hypothèses que le préjugé et l’ignorance prennent pour la science.\par
Nous ne parlerons point ici de cette morale stoïque, monastique, anti-sociale, que l’éducation montre aux hommes comme le chemin de la perfection. Pour peu qu’on l’examine, on reconnaîtra que cette morale farouche qui ne convient qu’à des anachorètes n’est nullement faite pour des citoyens, et que si elle était praticable, elle finirait par dissoudre la société, par séparer les hommes et peupler les déserts. C’est pourtant de cette morale que l’éducation publique repaît communément ses élèves, qui l’admirent comme merveilleuse, sans avoir jamais la force de la mettre en pratique. Que peut penser un bon esprit de cette scolastique révérée qui ne semble s’être emparée de la morale que pour le rendre problématique, obscure, impossible à saisir\footnote{Nous rapporterons ici le jugement qu’a porté de cette morale un écrivain célèbre et non suspect, qui, parlant des siècles d’ignorance dont les institutions subsistent néanmoins aujourd’hui, nous dit : « On traitait la morale dans les écoles comme le reste de la théologie, par raisonnement plus que par autorité et problématiquement, mettant tout en question jusqu’aux vérités les plus claires : d’où sont venues avec les temps tant de décisions de casuistes, éloignées non seulement de pureté de l’Évangile mais de la droite raison. Car où ne va-t-on point en ces matières quand on se donne toute liberté de raisonner ? Or ces casuistes se sont plus appliqués à faire connaître les péchés qu’à en montrer les remèdes. Il se sont principalement occupés à décider ce qui est péché mortel et à distinguer à quelle vertu est contraire chaque péché, si c’est la justice, la prudence ou la tempérance. Ils se sont étudiés à mettre, pour ainsi dire, les péchés au rabais et à justifier plusieurs actions que les Anciens, moins subtils, mais plus sincères, jugeaient très criminelles. » D’où l’on voit que les subtilités vaines et les arguties puériles de la philosophie sont encore la base de la morale intelligible que l’on enseigne à ceux mêmes qui sont destinés à l’instruction des peuples ; Voyez M. Fleury, VI, {\itshape Discours sur l’Histoire ecclésiastique}, §. 9. Dans une grande partie de l’Europe, l’éducation de la jeunesse fut pendant près de deux siècles presque exclusivement confiée à des jésuites, décriés par des principes aussi contraires à la politique qu’aux bonnes mœurs, et qui ont fait tous leurs efforts pour empêcher les lumières de la science de pénétrer dans les écoles dont ils avaient la direction.} ?\par
On dirait, en général, qu’en livrant leurs enfants à l’éducation publique les parents ne veulent que s’en débarrasser et leur faire employer bien ou mal les années les plus précieuses, les plus importantes de la vie.\par
On dirait encore que, conformément aux vues politiques que nous avons reprochées aux anciens prêtres d’Égypte et d’Assyrie, ceux qui président chez les modernes à l’éducation publique se proposent d’environner toutes les sciences de ténèbres et d’obstacles pour retarder la marche de l’esprit humain. Tout homme qui cherche à s’éclairer est continuellement arrêté par les nuages dont des sophistes ont artistement entouré la vérité ; il trouve à combattre et l’autorité des philosophes anciens, communément guidés par un vain enthousiasme, et les préjugés des modernes, égarés par un respect aveugle pour l’Antiquité, qui rarement dans ses recherches consulta l’expérience ou la raison, auxquelles on persiste encore à préférer l’autorité.\par
Quiconque veut découvrir la vérité, que l’éducation publique ainsi que d’autres causes s’efforcent à dérober de ses regards, est obligé de voler de ses propres ailes et de renoncer à des guides qui ne feraient que l’égarer. La morale, si nécessaire aux hommes, évidemment fondée sur leur nature, dont les principes sont si clairs pour tous ceux qui daigneront la consulter, est encore pour bien des gens au fond du puits de Démocrite et ne peut être connue que de ceux qui oseront y descendre. Pour peu que l’on ait fait attention aux principes établis dans cet ouvrage et aux devoirs généraux et particuliers destinés à régler la conduite des citoyens dans chaque état, on reconnaîtra sans peine qu’une bonne éducation n’est dans le vrai et ne peut être que la morale rendue familière à la jeunesse ou dont les principes lui sont inculqués de bonne heure afin que par la suite ils lui servent dans tout le cours de la vie.\par
Qu’est-ce qu’élever un jeune prince ? C’est lui inspirer de bonne heure les idées, les dispositions, les désirs, les volontés, les passions qu’il doit avoir pour bien gouverner un jour le peuple à la prospérité duquel son propre bien-être sera lié par des nœuds indissolubles ; c’est lui montrer l’intérêt qu’il a d’être juste, afin d’être aimé, défendu, obéi de bon cœur par une nation nombreuse et florissante dont le bonheur influera nécessairement sur son chef ; c’est faire naître dans celui qui doit un jour commander à des hommes les sentiments capables de mériter leur attachement inviolable ; c’est accoutumer ce jeune prince à trembler en voyant dans l’Histoire les malheurs des nations et les trônes renversés soit par les passions, soit par la négligence et la faiblesse de tant de souverains qui n’ont pas connu l’art de gouverner.\par
D’où l’on voit que l’éducation d’un prince consiste à lui inculquer d’être juste afin de jouir d’un pouvoir assuré, de travailler au bonheur de ses sujets afin d’être heureux lui-même, de craindre de les opprimer ou d’abuser du pouvoir suprême afin de ne point s’attirer des malheurs inévitables. L’équité, la fermeté, l’amour de l’ordre, la vigilance, le goût du travail, la passion de la vraie gloire, des sentiments profonds d’humanité, voilà les dispositions que l’on devrait faire éclore et cultiver dans les cœurs qui régleront les destinées des empires.\par
Élever un jeune homme destiné à occuper un jour de grandes places, c’est lui inspirer de bonne heure l’ambition de plaire à ses concitoyens, de mériter leur reconnaissance et leurs applaudissements par le bien qu’on leur fera, par les talents qu’on leur montrera. C’est enflammer son imagination par l’idée de la gloire ou de l’estime de tout un peuple. C’est lui apprendre à seconder les vues sages du souverain dont il doit quelque jour partager l’autorité. C’est lui faire sentir que pour être flatteuse et durable cette autorité doit être bienfaisante, équitable, éclairée. C’est lui montrer dans l’Histoire et dans des ouvrages utiles les ressources des hommes de génie pour contribuer à la félicité des peuples. C’est, enfin, lui faire envisager avec frayeur les chutes si fréquentes de tant d’indignes favoris qui par l’abus qu’ils ont fait du pouvoir se sont vu précipités du faîte de la grandeur dans l’abîme de l’opprobre et de la misère, et dont les jours ont été quelquefois terminés par une mort infamante.\par
L’éducation du noble ou de celui que l’on destine au métier de la guerre doit se proposer de lui donner une force, une fermeté d’âme qui l’accoutume dès l’âge le plus tendre à contempler sans crainte les dangers et la mort. Pour exciter en lui ce courage généreux, il faut semer dans son jeune cœur le sentiment de l’honneur, l’amour de la patrie, le désir d’acquérir des droits à l’estime de ses concitoyens, la crainte de la perdre par une conduite abjecte et lâche. Cette éducation doit s’appliquer à combattre, ou plutôt à prévenir, le sot orgueil de la naissance, qui persuaderait aux nobles que leur sang est plus pur que celui des citoyens qu’ils doivent un jour défendre pour en être justement considérés. Cette éducation doit tempérer un courage qui dégénérerait peut-être un jour en férocité, par des sentiments d’humanité, qui doivent accompagner le guerrier même dans les combats. Tout devrait inspirer à l’homme vraiment noble une noble fierté, l’horreur de la servitude, le véritable amour de la patrie, la crainte de la voir tomber sous la tyrannie, qui réduirait le guerrier lui-même à l’état méprisable d’un esclave déshonoré. Enfin, l’éducation militaire devrait fournir à ses élèves l’expérience et les connaissances nécessaires pour remplir avec honneur les fonctions de leur état et pour diminuer les périls auxquels une valeur non dirigée est souvent exposée. L’étude de l’Histoire, de la géographie, de la tactique, etc. est indispensable à tout militaire qui veut faire son métier d’une façon distinguée et non comme un sauvage farouche ou comme un automate qui ne sait que se faire imprudemment égorger. Quel amas prodigieux de connaissances ne faut-il pas pour former un ingénieur, un homme de mer, un général qui ne veut pas livrer inutilement ses soldats à la mort !\par
Celui que l’on destine à devenir un jour l’organe des lois, le protecteur du citoyen, le ministre de l’équité, doit se pénétrer de bonne heure d’un saint respect pour la justice et pour la fonction auguste qu’il remplira dans la société. Il apprendra qu’il doit placer son honneur et sa gloire dans ses lumières et son intégrité ; il étudiera les lois, et surtout il méditera les règles constantes et sûres de l’équité naturelle ou de la vraie morale qui guideront ses pas dans le dédale tortueux de la jurisprudence ténébreuse, dont on a souvent tant de peine à se dégager.\par
Le jeune homme qui doit jouir d’une grande fortune doit être remué fortement dans son enfance par des sentiments de bienfaisance, d’humanité, de pitié pour tous ceux que le sort n’a pas autant favorisés que lui. Il doit apprendre de bonne heure que les richesses ne donnent des avantages réels à ceux qui les possèdent que par les moyens qu’ils leur fournissent de se rendre eux-mêmes heureux par le bonheur qu’ils répandent sur d’autres. L’éducation des enfants destinés à l’opulence devrait les prémunir contre les vices et les vanités, qui ne sont propres qu’à les tourmenter et à les conduire sans vrais plaisirs à la ruine. Elle devrait encore leur orner l’esprit afin d’échapper aux ennuis que produisent constamment la satiété et l’oisiveté.\par
L’éducation de celui qui se destine au sacerdoce consiste à lui inspirer les sentiments et à lui fournir les lumières convenables à son état. Les ministres de la religion se trouvant, comme on a vu, presque partout en possession d’élever la jeunesse, devraient surtout s’occuper du soin d’étudier et de simplifier la morale, se la rendre familière, afin d’en semer les premiers germes dans les cœurs de leurs disciples et pour la prêcher avec fruit aux nations dont l’instruction leur est confiée. Réservant pour ses membres des spéculations trop abstraites, des controverses obscures, des discussions épineuses peu faites pour le commun des mortels, le clergé ne devrait annoncer aux peuples que des vérités relatives aux mœurs et vraiment nécessaires au bonheur de la vie.\par
C’est de leurs méditations que les hommes sont en droit d’attendre un {\itshape catéchisme moral et social} dont on pourrait espérer des fruits, que ne produiront jamais des notions inaccessibles à la raison. Quelle reconnaissance le genre humain entier n’aurait-il pas pour des prêtres citoyens qui emploieraient leurs études et leur temps à rendre la morale assez claire pour être également entendue et des grands et des petits, et des souverains et des sujets !\par
Quand on se propose de former des savants et des gens de lettres, on devrait profiter des dispositions naturelles de la jeunesse pour tourner les esprits vers des objets vraiment avantageux pour la vie sociale. Si l’on consultait sagement les penchants des disciples, si l’on cultivait les talents auxquels on les verrait portés, les nations ne manqueraient pas de philosophes, de géomètres, de physiciens, d’astronomes, de chimistes, de botanistes et de médecins qui, par des routes diverses, contribueraient aux progrès des connaissances utiles au genre humain.\par
Une éducation plus morale et plus sociale détournerait l’imagination bouillante des jeunes gens de ces pénibles futilités auxquelles on les voit trop souvent se livrer.\par
La poésie perdrait-elle donc ses charmes si, laissant là ses fables et ses fictions surannées, elle s’occupait à nous montrer une Nature plus vraie, si au lieu de nous corrompre par les peintures du vice, elle nous rendait enfin les vertus plus aimables ? L’éloquence en deviendrait-elle moins forte ou moins animée si on ne l’employait qu’à porter dans les esprits des vérités intéressantes et dans les cœurs des sentiments honnêtes ? Démosthène et Cicéron sont-ils jamais plus grands que lorsqu’ils parlent à leurs concitoyens des objets vraiment dignes de les occuper\footnote{Plutarque, dans la {\itshape Vie de Cicéron}, en fait un grand éloge, disant : « C’est de tous les orateurs celui qui a le mieux montré aux Romains quel charme et quel puissant attrait l’éloquence ajoute à ce qui est beau et honnête, et combien ce qui est juste est invincible quand il est bien dit. »} ? Que la jeunesse étudie donc ces modèles, qu’elle puise dans les écrits immortels de l’Antiquité l’amour de la patrie, de la liberté, de la vertu, et non l’art futile d’orner des bagatelles, de prêter au vice des charmes et d’inventer des fictions. Les nations, suffisamment amusées par les jouets de leur enfance, demandent enfin à être instruites, éclairées. La vérité n’est-elle pas assez riche pour fournir un champ vaste aux recherches de l’esprit ? L’homme social et la Nature ne sont-ils pas un fond que l’on ne peut jamais épuiser ?\par
Tout prouve donc que la morale devrait être la pierre angulaire de l’éducation sociale : elle doit se proposer de ramener tous les états de la vie à la raison, à l’utilité générale, à la vertu. Elle fera sentir à celui qui doit jouir de la grandeur, de l’opulence, de l’autorité, que ces avantages sont perdus pour ceux qui ne savent les employer au bonheur de la société. Cette éducation consolera le pauvre et lui montrera dans mille travaux divers, dans l’industrie, dans la probité, des moyens sûrs de se soustraire à la misère et au crime, et de se procurer soit une subsistance honnête, soit une aisance honorable.\par
Au lieu de remplir les enfants des grands d’une sotte vanité, au lieu d’entêter le fils du noble de sa vaine généalogie et du mérite très douteux de ses pères, au lieu de repaître le magistrat futur des vaines prétentions de sa place, au lieu de gonfler le prêtre de l’orgueil de son ministère, une éducation vraiment sociale doit inspirer à tous une modestie, une justice, une humanité, en un mot les vertus sans lesquelles nulle société ne peut être unie et fortunée.\par
Rien ne rend les hommes moins sociables que leur vanité. Sans déplacer les rangs divers, une éducation nationale devrait donc combattre sans relâche les vanités et détruire ces indignes préjugés qui rendent si souvent les hommes les plus élevés orgueilleux, injustes, haïssables pour leurs concitoyens. Cette éducation devrait inculquer dès la jeunesse, non pas que tous les hommes sont égaux, mais que tous les hommes doivent être justes et bienfaisants. Elle ne doit pas enseigner que le fils d’un grand seigneur devrait se placer sur la même ligne que le fils d’un artisan, mais que le premier doit tendre une main secourable à l’indigent et ne peut avoir jamais le droit de maltraiter ou de mépriser celui qu’il voit dans la misère. Les hommes ne sont égaux que par l’obligation d’être bons, utiles à leurs semblables, unis les uns aux autres, qui leur est à tous également imposée.\par
La vraie morale ne confond pas tous les ordres d’un État : elle prescrit aux citoyens de remplir fidèlement les devoirs attachés à leurs sphères. Elle enjoint à tous d’être équitables, de s’unir d’intérêts, de se prêter des secours mutuels, de s’aimer comme des proches, dont les uns sont favorisés et les autres disgraciés par l’aveugle fortune ; elle leur défend de se haïr ou de se mépriser parce que la haine et le mépris anéantissent l’harmonie sociale. Toute société est un concert dont le charme dépend de l’accord des parties qui le composent. L’instruction la plus importante pour les hommes, considérés soit comme individus soit comme en masse ou en corps, serait de leur faire sentir que séparés d’intérêts ils ne peuvent point travailler efficacement à l’ouvrage de leur félicité durable, qui ne peut être l’effet que des travaux réunis de tous les membres et de tous les corps de la société. Dans toute nation la justice impose à tous les hommes une chaîne de devoirs, qui lie ensemble le souverain et le dernier des sujets et à laquelle personne ne peut se soustraire sans danger.\par
Ainsi, l’éducation publique devrait jeter les fondements de l’harmonie sociale, aussi nécessaire au bonheur de la vie privée qu’à celui de la vie publique. Les instituteurs de la jeunesse ne devraient donc pas négliger, comme ils font, d’enseigner à leurs élèves les devoirs auxquels les engageront quelque jour la société conjugale, l’état d’un père et d’une mère de famille, les liaisons du sang qui subsistent entre des proches, les nœuds faits pour unir des amis, enfin les devoirs de maîtres et de serviteurs, objets qui vont nous occuper dans le reste de cet ouvrage.\par
C’est ainsi que l’éducation pourrait remplir peu à peu l’esprit des citoyens de connaissances bien plus utiles, sans doute, que celles que l’on puise dans des études souvent stériles et pour le cœur et pour l’esprit. À quoi bon avoir appris tous les faits de l’Histoire ancienne ou moderne si l’on ne sait en tirer des instructions utiles pour la race présente ? Quel fruit a-t-on recueilli de la lecture des philosophes et des sages de l’Antiquité si l’on n’applique leurs maximes et leurs leçons à sa propre conduite ? Enfin, à quoi peuvent servir les talents de l’esprit s’ils ne contribuent ni à notre propre félicité ni à celle des autres ? L’éducation publique dans les nations les plus éclairées fait assez de savants, de gens de lettres, de poètes légers, d’hommes amusants, mais elle fait très peu de bons citoyens ; elle ne forme des hommes ni pour la patrie, ni pour les familles, ni même des individus assez sages pour se conserver.\par
Si l’éducation publique laisse parmi nous la jeunesse dans une ignorance complète de ce qu’elle devrait savoir, elle ne la garantit pas de la connaissance des vices qu’elle devrait à jamais ignorer. Les collèges, ces sanctuaires destinés à conserver l’innocence et la pureté du jeune âge, servent communément à lui faire contracter des habitudes funestes et capables d’influer sur le bien-être de la vie : un sujet corrompu suffit quelquefois pour corrompre la masse entière de ses camarades. Rien de plus commun que de voir une jeunesse énervée déjà par la débauche et confirmée dans le vice, même dans les asiles faits pour la mettre à l’abri de ces dangers.\par
Sans une réforme totale, que les gouvernements seuls sont en état d’opérer, la jeunesse, dans les pays même les plus policés, sera longtemps privée d’une éducation conforme aux vrais intérêts de la société. Les pères de famille qui voudront conserver les mœurs de leurs enfants, les former à la sagesse, à la vraie science, à la probité, seront réduits à les soigner eux-mêmes s’ils en sont capables, ou du moins à chercher des instituteurs dignes de leur confiance, de leur attachement et de leur reconnaissance.\par
Ceux-ci, pour répondre à leurs vues, se garderont bien de prendre avec les enfants qu’ils veulent attirer à la science et à la vertu, le ton impérieux de la pédanterie. Ils sauront que la tyrannie ne fait que des esclaves, que les châtiments arbitraires ne servent qu’à révolter, qu’il ne faut pas rendre les devoirs rebutants quand on veut les faire aimer. Ils verront que les fautes avouées méritent de l’indulgence, afin d’encourager la candeur et la franchise. Ils reconnaîtront que la raison, bien présentée, se fait entendre dès l’âge le plus tendre, et qu’elle est plus propre à convaincre que des ordres non motivés qui ne font des enfants que de pures machines. « Un homme bien né, dit Cicéron, n’obéit qu’à ceux qui lui donnent des préceptes utiles, qui l’instruisent de ce qu’il doit apprendre, qui lui commandent en vertu d’une autorité dont il reconnaît l’utilité pour lui-même. »\par
Les bons instituteurs trouveront que l’enfance est sensible à l’estime et à la honte, et que ces mobiles peuvent être employés avec succès dans l’âge même le plus tendre. Ils s’apercevront facilement qu’une application trop longue et trop suivie est contraire à la santé et ne sert qu’à rendre le travail odieux. Enfin, tout les invitera à tempérer l’autorité. Est-il rien de plus lâche que cette pédanterie si commune qui s’enorgueillit d’un pouvoir exercé sur un enfant, dans un âge surtout dont les fautes méritent plus de pitié que de colère ? Les châtiments redoublés ne sont propres qu’à faire des âmes basses, des menteurs dépourvus des sentiments de l’honneur ; ils perdent tout leur effet quand ils deviennent habituels. Ils ne doivent être rigoureux que lorsqu’il s’agit d’étouffer dans leurs germes des qualités qui annonceraient un mauvais cœur. La malice noire, la hauteur, le mensonge, l’injustice, l’ingratitude, la cruauté, doivent être soigneusement réprimés ; les fautes qui ne sont dues qu’à l’étourderie, à la légèreté, doivent être facilement pardonnées.\par
Telles sont les routes que la raison propose aux instituteurs de la jeunesse, telle est en général la conduite qu’ils doivent tenir pour rendre leurs instructions efficaces. Des maîtres de cette trempe sont faits pour être honorés, chéris, dignement récompensés ; ils acquerront des droits assurés sur la reconnaissance éternelle des parents équitables et sur celle des enfants. Ceux-ci sentiront tôt ou tard ce qu’ils doivent à des hommes qui, sans se rebuter de leurs fautes, de leur indocilité, de leurs folies, de leur paresse, sont parvenus, à force de soins et de travaux, à les rendre des citoyens estimables et à leur faire aimer l’étude, dans laquelle ils trouveront pendant le reste de leur vie des ressources assurées contre l’ennui qui tourmente tous les hommes désœuvrés. Ils reconnaîtront qu’une bonne éducation est le plus grand des bienfaits et que les soins de ceux de qui on l’a reçue ne peuvent être payés d’assez de reconnaissance.\par
Si l’éducation des hommes est souvent négligée, soit par des parents imprudents, soit par des gouvernements peu sages, celle du sexe destiné à faire des épouses et des mères semble avoir été parfaitement oubliée dans presque toutes les nations. La danse, la musique, l’aiguille : voilà pour l’ordinaire toute la science que l’on enseigne à de jeunes personnes qui gouverneront un jour des familles\footnote{On ne peut se dispenser de rapporter ici la façon dont un moraliste moderne fait sentir le ridicule de l’éducation des filles. « Tenez-vous droites ; vous vous penchez d’un côté ; vous marchez comme un {\itshape z}. Votre bouche fait peur, vous ne touchez point à votre visage, levez donc votre tête, où sont vos mains ? Tournez les pieds en dehors, effacez vos épaules, etc. Voilà pendant douze ou quinze ans la morale du matin ; le soir on la répète. Aussi le premier en date pour une éducation si distinguée est le maître à danser. » M. Champion.}, voilà les perfections et les talents que l’on demande à un sexe duquel dépend le bonheur du nôtre ! Une mère se croit attentive parce qu’elle tourmente impitoyablement sa fille pour des minuties qu’elle devrait mépriser elle-même et lui apprendre à dédaigner. Ces bagatelles paraissent pourtant si graves aux yeux de la plupart des mères qu’elles deviennent chaque jour pour elles une source intarissable d’humeur et de colère, et pour leurs filles une source de chagrins et de pleurs. Au lieu de former leurs cœurs à la vertu, au lieu de leur faire connaître les devoirs qu’elles auront à remplir un jour, au lieu d’orner l’esprit qu’elles ont reçu de la Nature par des connaissances capables de les soustraire à l’ennui auquel, plus que les hommes encore, elles seront exposées dans le cours de la vie, l’éducation qu’elles reçoivent ne semble avoir pour but que de leur rétrécir la tête, de leur inspirer, dans les bras même de leurs nourrices, le goût de la parure et de la vanité, de leur faire attacher la plus grande importance à des misères, de ne les occuper que des grâces du corps, de leur faire entièrement négliger les ornements intérieurs de l’esprit\footnote{Il est évident que les femmes, que tout entretient dans une sorte d’enfance, ne sont pas la cause qui contribue le moins aux progrès du luxe et de la vanité nationale. On dit que dans un pays très livré au luxe, où un homme {\itshape comme il faut} ne pouvait se présenter dans des compagnies du bon ton sans avoir des dentelles, une femme ivre de son opulence se plaignit hautement de son mari pour lui avoir présenté un ami qui n’avait à sa chemise que des manchettes brodées.}. On dirait que cette éducation se propose d’en faire des idoles destinées à se repaître d’encens et à vivre dans une ignorance totale de ce qu’elles doivent à la patrie.\par
Ainsi que les princes, les femmes sont gâtées et méconnaissent les devoirs de la vie sociale ; la manière dont elles sont communément élevées ferait croire que l’on craint d’en faire des êtres raisonnables. On ne les occupe que d’ajustements et de modes ; on ne leur parle que d’amusements, de spectacles, de bals, d’assemblées ; on leur donne des leçons de coquetterie ; on les dispose d’avance à l’empire qu’elles doivent exercer un jour ; on leur suggère les moyens d’irriter les passions pour lesquelles on devrait leur inspirer de l’horreur.\par
Il ne faut pas s’étonner si des femmes nourries dans ces principes n’ont souvent aucune des qualités nécessaires pour contribuer au bonheur des autres ou pour se rendre elles-mêmes solidement heureuses. Il ne faut pas être surpris de les voir si souvent tomber dans les pièges que leur tend la galanterie et de les trouver incapables de fixer par les qualités de l’âme les adorateurs que leurs charmes ont séduits pour quelques instants. Une fille à qui son éducation ne montre rien de plus important que l’art de la séduction ne tarde pas à mettre ces leçons en pratique dès qu’elle en a la liberté {\itshape ;} de là les intrigues et les dérèglements qui, comme on l’a remarqué, mettent à jamais la discorde et le trouble entre les époux, de là ce désœuvrement des femmes dont la fatigue les pousse vers des amusements ruineux ou des plaisirs coupables, de là ce vide dans l’esprit qui, lorsque leurs charmes se sont flétris, les rend inutiles, chagrines, incommodes dans la société, et les oblige de chercher soit dans l’esprit de cabale, soit dans une sombre dévotion, des remèdes contre l’ennui dont elles sont dévorées.\par
Indépendamment des leçons et des exemples dangereux que peut donner une mère coquette et déréglée, il n’est pas de situation plus douloureuse que celle de sa fille, surtout si la Nature l’a douée de quelques charmes : elle ne tarde pas alors à déplaire à cette mère. Chagrine de voir ses charmes éclipsés par des appas naissants, celle-ci ne regarde sa fille que comme une rivale, une ennemie nuisible à ses propres prétentions ; en conséquence, elle la force d’essuyer à tout moment une mauvaise humeur continue et les effets souvent barbares de la vanité furieuse. Malheureuse par la dureté de sa mère, elle n’a rien de plus pressé que de suivre la première voie qui peut la délivrer de la tyrannie maternelle ; elle ne s’y soustrait souvent que pour tomber sous la tyrannie maritale, qui durera pendant toute sa vie.\par
L’éducation publique que l’on donne aux jeunes filles n’est pas de nature à les garantir de ces inconvénients. Pour se débarrasser d’elles lorsqu’elles les gênent dans leurs plaisirs, des parents insensés les remettent entre les mains de quelques recluses qui, totalement séparées du monde, n’en ont aucune idée. Des personnes vouées au célibat sont-elles donc faites pour instruire une fille dans les devoirs de la vie conjugale ? Des femmes dépourvues d’expérience pourront-elles la prémunir contre des séductions et des dangers qu’elles-mêmes ne doivent point connaître ? Si elles leur donnent quelques leçons de morale, elles sont communément défigurées par des rêveries superstitieuses et font pour l’ordinaire consister la vertu dans des pratiques minutieuses totalement étrangères aux intérêts de la société. Une pareille éducation ne sert qu’à remplir l’esprit de vains scrupules, de terreurs paniques, de petitesses capables d’inquiéter pendant toute la vie, sans mettre un frein réel aux passions que le monde fait éclore.\par
Élevée de cette manière, une fille sans expérience, sans talents, sans idées, est tout à coup tirée de sa prison pour passer dans les bras d’un inconnu dont elle doit faire le bonheur, ainsi que de la postérité à laquelle elle va donner le jour. Mais dépourvue de principes, elle ne connaît aucun devoir, elle erre à l’aventure ; et si elle ne trouve pas dans son mari, par un heureux hasard, des sentiments et des lumières propres à la guider, elle est bientôt entraînée dans tous les pièges et les travers dont une société corrompue est remplie.\par
C’est visiblement à l’éducation funeste que l’on donne aux femmes que l’on doit attribuer leurs faiblesses, leurs imprudences, leur frivolité, les désordres qu’elles produisent si souvent dans le monde ; enfin, les chagrins et les ennuis qui finissent par les punir un jour de leurs folies. Rien de plus triste que le sort d’une femme qui, survivant à ses attraits, dans l’abandon où le monde la laisse, ne trouve en elle-même qu’un vide affreux pour remplacer les adorations, les amusements bruyants et les plaisirs continuels dont elle s’était fait une habitude. C’est pourtant à ce sort si cruel que l’éducation semble les condamner. Des parents ignorants et sans vues négligent d’instruire ces êtres si sensibles, de les fortifier contre les dangers de leur propre cœur, de leur inspirer le courage de la vertu : on dirait qu’ils craignent que les ornements de l’esprit et du cœur ne fassent tort aux agréments du corps. Ne voit-on pas qu’un esprit cultivé prête à la beauté plus d’empire, et que la vertu rendra cette beauté plus estimable et la remplacera lorsqu’elle n’existera plus ? Comme des fleurs passagères, les femmes ne se croient faites que pour plaire quelques instants. Ne devraient-elles pas se proposer de perpétuer les hommages qu’on leur rend ? Combien la beauté a-t-elle de charmes quand elle est accompagnée de pudeur, de talents, de raison, de vertus ! Une femme belle et vertueuse est le spectacle le plus enchanteur que la Nature puisse offrir à nos regards.\par
Que ce sexe charmant fait pour répandre tant d’agréments et de douceur dans la vie ne craigne donc point de cultiver son esprit : des connaissances utiles ne nuiront point à ses grâces. Qu’il songe surtout à cultiver un cœur que la Nature a rendu susceptible des vertus les plus sociables. Par là les femmes plairont toujours, elles exerceront un empire plus flatteur que ce pouvoir éphémère qui n’est dû qu’à des appas sujets à se flétrir, elles fixeront des sentiments qu’elles auront pu légitimement exciter, elles s’attireront des hommages plus sincères, plus constants, plus désirables que ceux que leur prodiguent des trompeurs qui ne veulent qu’abuser de leur faiblesse et de leur crédulité, elles seront honorées et recherchées pendant toute leur vie. Jusque dans la vieillesse et dans la solitude, elles retrouveront en elles-mêmes les connaissances dont elles se seront ornées, elles jouiront et de l’estime publique, et d’une sérénité préférable au tumulte des plaisirs et à ces vains amusements qui ne font d’ordinaire qu’une diversion momentanée à des ennuis continuels.\par
L’on ne peut aucunement douter que la conduite des femmes n’influe de la façon la plus marquée sur les mœurs des hommes. Ainsi, tout doit convaincre qu’une meilleure éducation donnée à la moitié la plus aimable du genre humain produirait un changement heureux dans l’autre. On dit avec raison que le commerce des femmes contribue à rendre les mœurs plus douces et plus sociables, mais dans des nations frivoles et corrompues, il est à craindre que ce qu’on qualifie de douceur dans les mœurs ne dégénère trop souvent en mollesse, en légèreté, en incurie, en oubli même de ses devoirs. Pour complaire à des femmes vaines et peu réfléchies, les hommes s’occupent de parures, d’équipages, de bagatelles : ils deviennent efféminés. La force d’âme, la fermeté, la vertu mâle font place à l’indolence, au luxe, à la frivolité, à la galanterie. Dans les contrées où des femmes inconsidérées ont le droit de donner le ton et de régler les goûts, la société se remplit de soupirants oisifs, de complaisants, d’amusants ; mais on n’y trouve guère d’hommes vertueux et raisonnables. L’éducation que l’on donne aux femmes en fait des enfants gâtés qu’il faut toujours amuser pour les tenir en belle humeur.\par
Nonobstant ces fâcheuses influences de la conduite des femmes sur les mœurs nationales, n’écoutons point les déclamations chagrines de quelques moralistes, soit anciens, soit modernes, qui voudraient faire croire que la raison, la solidité, le bon sens ne sont point le partage de cette portion si précieuse de la société. Une éducation molle et complètement défectueuse est la vraie cause qui fait que tant de femmes possèdent dans des corps faibles des âmes plus faibles encore. Cette frivolité, cette espèce d’enfance continuée, l’inhabitude de réfléchir les livrent sans défense à la flatterie, aux pièges du vice, aux vanités du luxe, à toutes les extravagances introduites soit par la négligence des législateurs, soit par le faste et la corruption des cours que des êtres imprudents trouvent beau d’imiter.\par
Ce n’est pas la Nature qui donne à tant de femmes cette mollesse, cette aversion du travail, cette faiblesse du corps, ces infirmités habituelles si communes parmi celles qui sont nées dans l’opulence et la grandeur ; ces effets sont dus au défaut d’exercice, à une vie trop sensuelle qui, dès l’âge le plus tendre, empêchent les corps de prendre la vigueur dont ils auraient besoin et contribuent à augmenter leur débilité naturelle.\par
La vie dissipée et les désordres que produit le luxe font que les femmes d’un certain ordre plongées dans une langueur continuelle n’ont ni la volonté ni le pouvoir d’allaiter leurs enfants elles-mêmes ; elles sont forcées de violer le premier devoir que la Nature impose aux mères. Cette faiblesse n’est pourtant pas inhérente à tout le sexe : les femmes du peuple nous prouvent qu’elles ont non seulement la force de remplir les devoirs des mères mais encore que l’habitude les rend capables de supporter les travaux les plus durs.\par
Quant à la force de l’esprit, les exemples des citoyennes de Lacédémone et de Rome suffisent pour nous convaincre que les femmes dirigées par une éducation plus mâle et par une législation convenable sont susceptibles de grandeur d’âme, de patriotisme, d’enthousiasme pour la gloire, de fermeté, de courage ; en un mot, de passions généreuses qui doivent faire rougir tant d’hommes amollis que l’on voit dans les contrées énervées par le luxe et le despotisme\phantomsection
\label{footnote95}\footnote{Cornélie, mère des Gracques, se contenta de montrer ses deux fils à une dame qui lui demandait à voir ses bijoux et ses parures. Selon Plutarque, les femmes de Sparte étaient très affligées quand après une défaite elles voyaient arriver leurs fils au lieu que celles dont les fils avaient été tués, en allaient rendre grâce aux Dieux et s’en félicitaient. Voyez Plutarque, {\itshape Vie d’Agésilas}.} : ces deux fléaux dégradent les âmes et les détournent des objets vraiment utiles et nobles. Corrompue toujours elle-même, la tyrannie ne veut régner que sur des êtres sans activité, sans élévation, sans force et sans vertus.\par
C’est donc, on ne peut trop le répéter, d’un gouvernement attentif et bienfaisant que les nations peuvent attendre une éducation légale plus favorable aux bonnes mœurs, plus conforme au bien de la société. Sans recourir à des impôts onéreux, les états policés trouveront des moyens abondants de procurer aux différentes classes des citoyens l’éducation qui leur convient, dans les amples revenus de tant de maisons déjà destinées à cet usage et qui remplissent si mal l’attente du public. En attachant de la considération et des récompenses à la profession utile de former la jeunesse, les peuples ne manqueront ni de savants ni de gens de bien qui seconderont les vues des souverains. Les connaissances en tout genre se simplifient, se facilitent, se perfectionnent de jour en jour ; les principes de la morale, comme tout doit en convaincre, sont si clairs qu’on peut les mettre à la portée du peuple même : il n’est si grossier que parce qu’on néglige de l’instruire et qu’on l’oblige à végéter dans une ignorance imbécile et sauvage. Les enfants des gens du peuple sont presque en tout pays totalement abandonnés à leurs propres fantaisies ; on les voit dans les carrefours et dans les rues contracter dès la plus tendre jeunesse des habitudes et des vices qui les conduiront un jour au gibet.\par
Quoique, comme on l’a dit plus haut, tous les hommes ne soient pas susceptibles de la même éducation, quoiqu’il soit presque impossible de modifier deux individus précisément de la même manière, cependant il est et possible et facile de modifier les hommes en masse, de porter les esprits vers de certains objets, de donner un ton uniforme aux passions d’un peuple. Il n’est pas dans une nation deux hommes parfaitement semblables, soit pour le corps soit pour les facultés de l’esprit\footnote{« Mille hominum species, et rerum discolor usus : velle suum cuique est, nec voto vivitur uno. » Pers. {\itshape Satyr.} V, vers 52, 53.} ; on trouve néanmoins une ressemblance générale dans les traits et dans les idées du plus grand nombre des individus. Quoiqu’il n’y ait pas deux Français qui se ressemblent parfaitement, néanmoins le caractère général de la nation française est la gaieté, l’activité, la politesse, la sociabilité, l’étourderie, la vanité, l’amour du luxe. Quoique deux Espagnols ne soient pas les mêmes, nous trouvons que la masse de leur nation est grave, taciturne, superstitieuse, ennemie du travail. Le caractère et les mœurs des nations dépendent en premier lieu de la nature du climat, qui influe sur le corps, et ensuite du gouvernement, de l’éducation, des opinions, des usages qui influent sur les esprits et décident des mœurs nationales ; ces mœurs ne sont jamais que les habitudes contractées par le plus grand nombre des hommes dont les nations sont composées.\par
Sans avoir les lumières que l’éducation procure aux personnes d’un ordre plus relevé, le peuple serait pourtant susceptible de recevoir facilement la dose d’instruction et de morale nécessaire à sa conduite ou pour diminuer du moins les vices dont il est communément infecté. Par une négligence déplorable de presque tous les gouvernements, l’enfance de l’homme du peuple, de l’artisan, du pauvre, est totalement abandonnée, les premières années des indigents sont entièrement perdues. Des souverains plus vigilants parviendraient aisément à donner des mœurs plus raisonnables à ceux même que le préjugé en fait croire le moins susceptibles. On nous dit que le gouvernement chinois est parvenu à rendre la politesse populaire {\itshape ;} sans corriger les mœurs, il a corrigé les manières, tandis qu’avec aussi peu de peine il eût pu rendre la vertu populaire. Des voyageurs nous apprennent que l’on voit dès l’âge le plus tendre la gravité s’établir sur le front des enfants arabes : on les trouve aussi posés dans l’enfance que les hommes faits sont ailleurs étourdis et pétulants pendant toute leur vie.\par
Indépendamment de la négligence du gouvernement qui trop souvent ferme les yeux sur les mœurs du peuple, l’état d’avilissement où ce peuple est tenu, sa dépendance excessive, les oppressions et les dédains qu’il est forcé d’essuyer de la part de ses supérieurs, contribuent encore à le corrompre. Tout homme qui se méprise lui-même ne craint plus le mépris des autres ; celui qui a perdu l’espoir d’être estimé s’abandonne au vice et ne rougit plus de rien. Voilà sans doute pourquoi l’on trouve tant de bassesse, tant de friponneries, tant de rapines, si peu de probité, de décence et de bonne foi dans les petits marchands, les artisans, les valets, en un mot, dans les dernières classes du peuple. Les personnes de cet ordre se permettent tout ce qui ne conduit pas directement au gibet.\par
En dégradant les hommes, on anéantit pour eux le sentiment de l’honneur et ils perdent dès lors toute idée de vertu. Le despotisme, qui ne fait que des esclaves oppresseurs et des esclaves opprimés, doit visiblement détruire l’honneur dans toutes les âmes. Le courtisan avili par son maître avilit à son tour ceux qui se trouvent placés au-dessous de lui ; ceux-ci finissent par se livrer à toutes sortes d’infamies. Il n’y a qu’une liberté légitime et honnête qui puisse faire naître le sentiment de l’honneur. Un esclave n’aura jamais sincèrement une haute idée de lui-même ; il sera fat, vain, impudent, impertinent, mais jamais il n’aura la fierté noble que la liberté et la sécurité peuvent seules donner.\par
Dans les nations où règne le luxe, tout contribue, comme on l’a souvent répété, à pervertir les mœurs du peuple : il lui faut des amusements et des plaisirs analogues à ceux de ses supérieurs, il lui faut des spectacles, des tréteaux, des {\itshape parades}, des tavernes, des guinguettes, qui non seulement lui font perdre son temps et son argent mais encore qui lui font perdre ses mœurs et le déterminent au crime. C’est dans le gouvernement une très grande imprudence que d’accoutumer le peuple à des amusements continuels ; ceux qui s’imaginent par là le rendre plus tranquille et détourner son attention de l’idée de sa misère se trompent très lourdement : ils ne font, en amusant des hommes indigents, que redoubler leurs infortunes, les inviter à la licence ainsi qu’à la révolte. Le peuple doit travailler ; pour le rendre tranquille et bon il faut l’instruire et le soulager. Des écoles de mœurs adaptées à la capacité des enfants les plus grossiers mettraient une politique attentive au moins à portée d’essayer si l’on ne pourrait pas rendre les gens du peuple un peu meilleurs, un peu plus sociables qu’ils ne sont communément. Des établissements de cette espèce, convenablement encouragés, changeraient peut-être en peu de temps les mœurs d’un vaste empire.\par
Mais les tentatives les plus faciles paraissent entourées de difficultés insurmontables à la paresse ou déplaisent à la mauvaise volonté. Les souverains seront toujours les maîtres des mœurs des peuples ; ils ont entre leurs mains tout ce qui peut remuer les volontés des hommes, ils peuvent à leur gré les porter vers le vice ou la vertu. S’ils donnaient à la réforme de l’éducation publique la moitié des secours et des soins qu’ils donnent à l’appui d’une foule d’institutions inutiles, les peuples auraient bientôt l’instruction dont ils ont tant de besoin. Si les leçons de la morale étaient soutenues par des honneurs et des récompenses, les nations ne manqueraient pas d’hommes disposés à les instruire. Enfin, si les bonnes mœurs conduisaient à des distinctions honorables et à la fortune, on ne peut pas douter qu’il ne se fît promptement une révolution désirable dans les mœurs des nations. Si des princes amis des arts les ont fait éclore en un instant dans leurs États, pourquoi douterait-on que des princes vertueux n’y fissent naître des vertus avec la même facilité.\par
N’est-il pas bien étrange que dans de vastes royaumes il n’y ait aucune école propre à former des politiques, des négociateurs, des ministres, des hommes capables de soulager les souverains dans les soins divers de l’administration ? La faveur, communément méritée par des bassesses et des intrigues, suffit-elle donc pour conférer les qualités que demandent les emplois importants desquels dépend le destin des empires ? Ne soyons donc pas surpris de voir le despotisme, perpétuellement dupe de ses propres folies, renverser les États soit par sa maladresse, soit par l’incapacité des agents qu’il emploie.\par
Il ne faut pas non plus être étonné de voir le vice et le crime régner sur des nations dont les gouvernements sont tellement aveuglés qu’ils semblent ignorer qu’une bonne éducation, une saine morale, de bonnes lois appuyées par des récompenses et des châtiments empêcheraient les vices et les crimes d’éclore et dispenseraient de recourir à tant de supplices cruels, et toujours inutiles tant qu’on ne portera pas le remède à la source du mal. « Occupe-toi, dit Confucius, du soin de prévenir les crimes, afin de t’épargner le soin de les punir. »\par
[Pour peu que l’on réfléchisse, l’on sera forcé de reconnaître qu’il n’est à proprement parler qu’une seule science vraiment intéressante pour les habitants de ce monde, à laquelle toutes les connaissances humaines sont faites pour aboutir et contribuer. Cette science, c’est la morale, qui embrasse toutes les actions et les devoirs de l’homme en société. Ce n’est donc, dans le vrai, que la morale appliquée ou adaptée aux différents états de la vie, que l’éducation devrait enseigner à la jeunesse. Qu’est-ce en effet qu’élever un jeune homme ? C’est lui communiquer de bonne heure les connaissances nécessaires à l’état qu’on veut lui faire embrasser, c’est l’habituer à tenir la conduite la plus propre à se faire estimer et chérir de ceux avec lesquels il aura des rapports, c’est lui indiquer les moyens d’être heureux en contribuant d’une façon quelconque à l’utilité, aux plaisirs, au contentement des autres. L’enfant à qui la nourrice enseigne à bégayer ses premières idées lui fait contracter l’habitude de converser avec les hommes, de leur communiquer des choses qui le feront estimer un jour en raison de leur utilité ou de leur agrément. En apprenant à lire, cet enfant amasse peu à peu des faits, des connaissances, des exemples, des expériences qui serviront par la suite à sa propre instruction et à celle des autres. La religion, que dès les plus tendres années l’on tâche d’inculquer aux enfants, ne doit avoir pour objet que de les rendre justes, humains, sociables, bienfaisants, par la crainte de déplaire à l’auteur de la Nature, qu’on montre comme rempli de bienveillance pour notre espèce. L’Histoire n’est utile que parce qu’elle nous fournit les preuves multipliées des effets redoutables qu’ont produit sur la terre les passions et les délires des hommes. L’érudition, la lecture des Anciens, l’étude des langues mortes seraient des occupations bien stériles si elles ne nous mettaient pas à portée de profiter des préceptes de la sagesse antique et d’appliquer la raison des siècles antérieurs à notre conduite présente. La jurisprudence est la connaissance des règles établies pour le maintien de la justice et de la paix dans la société. Ce qu’on appelle le {\itshape Droit de la Nature et des Gens} n’est, comme on l’a fait voir, que la morale qui doit régler la conduite des nations entre elles. La politique est-elle donc autre chose que la connaissance des devoirs mutuels qui lient les souverains et les sujets, c’est-à-dire la morale des rois ?\par
La morale devrait être le but unique de toutes les sciences qu’on enseigne à la jeunesse : toutes à leur manière doivent contribuer à rendre les hommes utiles, toutes doivent par des moyens divers concourir à procurer la félicité générale par le bien-être des individus. En s’occupant utilement pour tous, le savant acquiert des droits très légitimes à sa propre subsistance, à son salaire, à sa gloire, à la reconnaissance du public. Le mérite de la physique, de la médecine, de la chimie, de la mécanique, de l’astronomie, etc. ne peut être fondé que sur le bien que ces sciences font aux hommes. Les arts, les manufactures, le commerce, l’agriculture, les différents métiers fournissent aux gens du peuple mille moyens de subsister, de faire une fortune honnête ; en contribuant au bien-être social, ils travaillent à leur propre félicité. La morale, si honteusement négligée dans l’éducation, est évidemment le lien de la société. Elle oblige à leur insu des ingrats qui la dédaignent. Apprends à être utile afin de vivre heureux en ce monde, voilà ce que l’éducation, d’accord avec la vraie morale, doit inculquer à l’homme\footnote{[Passage entre crochets ajouté par l’auteur après la fin du tome III et dernier, comme devant être placé à la fin de ce chapitre III de la section V.]}.]\phantomsection
\label{footnote96}
\subsection[{Chapitre IV. Devoirs des Proches ou des Membres d’une même Famille}]{Chapitre IV. Devoirs des Proches ou des Membres d’une même Famille}
\noindent Toute famille est une société dont les membres peuvent être comparés à des rameaux partis d’une souche commune et qui pour leur intérêt doivent contribuer à maintenir entre eux l’union nécessaire à la conservation et au bonheur du tout dont ils font partie. Les parents ou les proches sont des amis donnés par la Nature, qui nous rappellent une origine commune avec nous, qui peignent à notre esprit des ancêtres dont la mémoire doit nous inspirer de la tendresse et du respect, qui nous font souvenir que c’est le même sang qui coule dans nos veines ; enfin, qui nous font sentir que notre bien-être exige que nous demeurions unis avec des êtres capables de contribuer à notre félicité, intéressés à notre prospérité, disposés à prendre part à nos plaisirs et à nos peines, à nous secourir dans l’adversité, à nous aider à parer les coups de la fortune. Toutes ces considérations suffisent pour nous faire connaître ce que les membres d’une même famille se doivent réciproquement.\par
Si la morale nous prescrit la pratique de la justice, de l’humanité, de la pitié, de la bienfaisance et de toutes les vertus sociales à l’égard de tous les hommes avec lesquels nous avons des rapports, on ne peut pas douter qu’elle ne nous fasse un devoir plus strict encore de montrer ces dispositions à ceux qui nous sont plus étroitement attachés par les liens du sang. Ainsi, tout confirme les droits de la parenté, tout prouve que nous devons à nos proches l’affection, les bienfaits, la compassion et les secours que nous exigerions d’eux si nous en avions besoin nous-mêmes. Des parents sont des personnes auxquelles, indépendamment des nœuds de la consanguinité, nous tenons encore par les liens de l’habitude, de la familiarité, de la fréquentation ; ils connaissent notre situation, ils sont dépositaires d’une partie de nos secrets, de nos vues, de nos intérêts, et par là sont plus capables de nous aider de leurs conseils, de favoriser les projets que nous pouvons former. Une famille bien unie, c’est-à-dire composée de personnes honnêtes, doit avoir une force que l’on ne peut rencontrer dans ces familles divisées dont les membres en discorde sont comme étrangers les uns aux autres.\par
Les parents que la fortune favorise deviennent naturellement les bienfaiteurs de ceux qu’elle oublie ; ceux qui ont du crédit, du pouvoir, des places éminentes s’attirent les regards des autres et deviennent les protecteurs et les soutiens des plus faibles. Ceux qui se distinguent par leurs lumières et leur prudence deviennent des conseillers dont on prend les avis ; ils peuvent, en raison des avantages qu’ils procurent aux autres, exercer une sorte d’autorité qu’on est obligé de reconnaître. Dans les familles, ainsi que dans toute autre société, les hommes qui sont à portée de faire plus de bien doivent, pour l’intérêt de tous, jouir d’une supériorité légitime.\par
Malgré les grands avantages attachés à l’union des familles, rien de plus rare que de voir des parents bien unis. Les frères mêmes nous donnent quelquefois des marques d’une discorde infiniment déshonorante\phantomsection
\label{footnote97}\footnote{Plutarque rapporte que deux frères Spartiates ayant eu querelle, les magistrats nommés {\itshape éphores} condamnèrent leur père à l’amende pour avoir manqué de leur inspirer dans leur enfance des sentiments plus convenables. Voyez Plutarque, {\itshape Dits notables des Lacédémoniens}.}. Faute de réfléchir, les hommes perdent continuellement de vue le but qu’ils devraient se proposer ; des intérêts personnels les séparent de l’intérêt général, qui ne touche jamais d’une façon bien sensible les personnes dont l’esprit ne s’est pas habitué à raisonner. L’orgueil, la vanité, la colère et la brutalité, que la familiarité met souvent trop à l’aise, sont les causes fréquentes de la division des parents, qui se trouvent quelquefois plus éloignés les uns des autres que les indifférents.\par
En effet, cette familiarité trop grande, qui semblerait au premier coup d’œil resserrer les nœuds des familles, contribue le plus souvent à les brouiller irrévocablement ; elle met les parents à portée de s’incommoder par leurs défauts mutuels, qui à la longue finissent par produire des divisions mortelles. De là viennent souvent ces haines invétérées qui remplacent l’harmonie nécessaire aux familles et que l’on voit pourtant quelquefois s’allumer entre des frères, entre les parents les plus proches. {\itshape La familiarité}, dit-on, {\itshape engendre le mépris} ; à quoi l’on peut ajouter que le mépris engendre la haine. Le mépris engendré par la familiarité ne vient que de ce qu’en rapprochant des hommes peu raisonnables, elle met leurs vices combinés en état de fermenter et de produire un venin dangereux.\par
Cela posé, des parents devraient non seulement redoubler d’égards les uns pour les autres mais encore s’armer d’une patience et d’une indulgence plus fortes afin de prévenir les ruptures que la trop grande familiarité peut causer. La familiarité ne dispense pas les personnes qui se fréquentent le plus des égards qu’elles se doivent ; elle les invite même à fuir avec plus de soin les occasions de se blesser. Il semble à bien des gens que la liaison fréquente et la familiarité doit leur donner le droit de manquer à ceux dont ils se croient les amis les plus intimes. Les parents, devant s’aimer, doivent craindre de se blesser et de rompre par là la bonne intelligence faite pour régner entre eux.\par
Faute de faire des réflexions si simples, les parents se croient souvent autorisés à se fatiguer de leurs passions diverses. Les plus distingués par leur rang ou leurs richesses accablent les autres sous le poids de leur vanité, de leur supériorité ; ils ne voient que des esclaves dans leurs parents moins fortunés. En général, on trouve communément que des collatéraux usent avec hauteur des avantages dont ils jouissent. Rien de plus ordinaire que des oncles qui par de longues souffrances font acheter à leurs neveux des bienfaits toujours mêlés de reproches et de duretés ; dans l’espérance qu’ils leur laissent entrevoir d’une succession opulente, ils se croient en droit de les traiter avec une tyrannie dont l’effet nécessaire est d’étouffer jusqu’aux germes de la reconnaissance. Rien de plus dur, surtout, que l’empire de ces nouveaux parvenus que la fortune enivre et qui se croient tout permis à l’égard des parents indigents qui vivent dans leur dépendance. {\itshape Ne soyez pas un oncle pour moi} fut un proverbe dans Rome ; il peut être adopté dans tout pays\footnote{« Ne fis patrum mihi. »}. Des parents de cette trempe ne doivent guère s’attendre que leurs cendres soient jamais arrosées de larmes bien sincères : leur mort est pour leurs collatéraux la fin d’un esclavage odieux. La reconnaissance est impossible quand elle est anéantie par une tyrannie continuelle. En bonne foi, est-ce donc être bienfaisant que de laisser à quelqu’un des biens que l’on ne peut emporter sous sa tombe ? L’homme bienfaisant fait jouir et jouit délicieusement lui-même du bien qu’il fait aux autres ; voilà celui qui mérite une reconnaissance véritable et qui peut se flatter que sa mémoire sera chère à ses collatéraux.\par
La vanité ferme souvent le cœur aux malheurs de ses parents. L’opulence, toujours hautaine, rougit de tenir à des indigents et à des infortunés ; elle n’est flattée que d’appartenir à des parents illustres dont elle croit sottement que la gloire rejaillit sur ceux qui l’environnent. Ainsi, les parents les plus dignes de pitié sont précisément ceux à qui l’orgueil refuse d’en montrer ! N’est-ce pas violer la loi la plus sacrée que la Nature impose aux membres d’une famille, que de refuser des secours et de l’appui à ceux qui en ont le plus pressant besoin ?\par
Enfin, un intérêt sordide est la cause la plus ordinaire des divisions fréquentes qui séparent des proches. Des hommes avides ne connaissent rien au monde de comparable à l’argent ; vous les voyez lui sacrifier à tout moment et l’union des familles et les égards qu’ils doivent à leur propre sang. Sous prétexte de la justice de leurs droits, vous les trouvez inflexibles au point de ne plus entendre le cri de l’humanité. On verra quelquefois un parent opulent se prévaloir de la loi pour dépouiller sans remords des parents qui languissent dans l’indigence et dans la misère.\par
Quoi qu’il en soit des raisons ou des prétextes qui divisent des proches, ils sont toujours plus ou moins blâmables et déshonorants. Une famille bien unie annonce des âmes sensibles, honnêtes, généreuses, dégagées d’un vil intérêt ; une famille divisée montre des âmes intéressées, insociables, injustes et sans pitié. Une famille composée de gens de cette trempe ne prévient nullement le public en sa faveur. Des chicaneurs acharnés toujours en procès les uns avec les autres annoncent des âmes ignobles et dignes de mépris. Enfin, une famille dont les membres sont perpétuellement en guerre ne peut jouir des fruits de la parenté : elle est privée des secours mutuels que devraient se prêter des personnes attachées par les liens du même sang.\par
En réfléchissant sur la nature humaine on trouvera, indépendamment des causes que nous avons rapportées, la source des divisions et des inimitiés que l’on voit trop souvent régner entre parents et qui font que souvent ils se refusent les secours qu’ils accordent quelquefois plus volontiers même à des étrangers. L’homme veut être libre dans ses actions : ses proches ne sont pas des êtres de son choix, les services qu’il leur rend sont des dettes à ses propres yeux ainsi qu’aux leurs ; il n’y satisfait qu’à regret, soit parce qu’il croit sa liberté gênée, soit parce qu’il s’imagine que ses bienfaits ne seront pas reconnus. Mais la justice et la bonté du cœur doivent anéantir ces calculs, et la grandeur d’âme nous porte à faire du bien même aux ingrats.
\subsection[{Chapitre V. Devoirs des Amis}]{Chapitre V. Devoirs des Amis}
\noindent L’amitié est une association formée entre des personnes qui éprouvent les unes pour les autres une affection plus particulière que pour le reste des hommes.\par
Quoique la morale nous excite à la bienveillance pour tous les membres de la société, quoique l’humanité nous fasse un devoir de montrer de l’affection à tous les êtres de notre espèce, cependant nous éprouvons pour quelques personnes les sentiments d’une prédilection plus forte, fondée sur l’idée du bien-être que nous espérions trouver dans un commerce intime avec elles. L’affection qui lie des amis entre eux ne peut avoir pour base qu’une conformité dans les penchants, les goûts et les caractères qui les rend nécessaires à leur bonheur réciproque.\par
Aimer quelqu’un, c’est en avoir besoin, c’est le trouver capable de contribuer à notre félicité. L’amitié sincère est un des plus grands avantages dont l’homme puisse jouir dans la vie\footnote{« Nil ego contulerim jucundo sanus amico. », {\itshape Satires}, livre I, 5, vers 44.} ; rien de plus malheureux que ces cœurs avides qui, concentrés en eux-mêmes, ne s’attachent à personne.\par
« Il n’y a point, dit Bacon, de solitude plus désolante que celle d’un homme privé d’amis, sans lesquels le monde n’est qu’un vaste désert ; celui qui est incapable d’amitié tient plus de la bête que de l’homme. »\par
Par l’amitié, l’homme double, pour ainsi dire, son être : elle suppose, en effet, un pacte en vertu duquel les amis s’engagent à se témoigner une confiance réciproque, à se donner en toute occasion des consolations, des conseils, des secours, à mettre leurs intérêts en commun, à partager leurs plaisirs et leurs peines.\par
Est-il rien de plus doux que de trouver quelqu’un dans le sein de qui l’on puisse déposer sans crainte ses pensées les plus secrètes, ses sentiments les plus cachés, et dans le cœur duquel on soit toujours sûr de rencontrer une volonté permanente de s’intéresser à nous, de soulager nos douleurs, d’essuyer nos larmes, de calmer nos inquiétudes, de faire cesser nos chagrins, de nous aider à supporter les orages de la vie ?\par
Par l’amitié, notre sort, notre bonheur, notre être, deviennent ceux de notre ami ; nous nous identifions avec lui, il devient un autre nous-même.\par
Sa raison, sa prudence, sa sagesse, sa fortune, sa personne, sont à nous ; nos affections et nos joies se confondent\phantomsection
\label{footnote98}\footnote{« L’amitié, dit un moraliste moderne, est un mariage spirituel qui établit entre deux âmes un commerce général et une correspondance parfaite. » Voyez un livre intitulé {\itshape Les Mœurs}, partie III, chap. 2. M. Dacier va encore plus loin : « Tel est, dit-il, l’effet de la véritable amitié, que l’on se trouve dans son ami plus que dans soi-même ; et l’on peut dire de l’amitié ce qu’un poète a dit de l’amour : “Et mira prorsum res foret, ut ad me fierem mortuus, ad puerum ut intus viverem.” » Voyez ses notes sur la satire d’Horace, livre II, 6.}. Fortifiés l’un par l’autre, nous marchons avec plus d’assurance dans les routes incertaines de ce monde. « Un ami, dit Aristote, est une âme qui vit dans deux corps. »\par
Tels sont les engagements contenus dans l’amitié, qui n’est que le pacte fait pour lier deux cœurs réunis par les mêmes besoins ou les mêmes intérêts. D’où l’on voit que l’amitié n’est point désintéressée : elle a visiblement pour objet le bien-être réciproque de ceux qui forment ses doux nœuds. L’intérêt qui lie entre eux des amis est louable quand il se propose la jouissance des agréments qu’ils peuvent se procurer par leurs qualités personnelles, qui seules peuvent donner de la solidité aux attachements des hommes. Il n’y a qu’une amitié fondée sur les dispositions habituelles du cœur qui puisse être permanente ; celle qui n’aurait pour fondement et pour but que le désir de partager avec un ami les avantages de sa fortune serait un sentiment abject, un intérêt sordide et digne d’être blâmé. « Quelle est, dit Plutarque, la monnaie de l’amitié ? C’est la bienveillance et le plaisir joints avec la vertu. L’amitié parfaite et véritable exige trois choses : la vertu comme honnête, la conversation comme agréable, et l’utilité comme nécessaire\phantomsection
\label{footnote99}\footnote{Voyez Plutarque, {\itshape De la Pluralité des Amis}.}. »\par
Il suffit d’avoir énoncé les engagements du pacte qui lie deux amis pour connaître tous les devoirs que l’amitié leur impose et les moyens d’entretenir une association si douce, si nécessaire à leur félicité. Ces devoirs consistent évidemment dans une confiance mutuelle, dans des attentions réciproques, dans une constance que rien ne puisse ébranler, dans une disposition invariable de contribuer au bien-être de celui qu’on a choisi pour ami. La confiance ne peut être fondée que sur des qualités dont on ait lieu de présumer la durée ; il n’y a que les dispositions cimentées par l’habitude sur qui l’on puisse compter. Ces dispositions doivent être utiles à l’association que l’on forme et, par conséquent, vertueuses.\par
D’où il suit que la vertu seule peut donner à l’amitié une base inébranlable ou faire les vrais amis. L’homme de bien est seul en droit de compter sur le cœur de l’homme qui lui ressemble. « Les méchants, dit un illustre moderne, n’ont que des complices ; les voluptueux ont des compagnons de débauche ; les gens intéressés ont des associés ; les politiques assemblent des factieux ; les princes ont des courtisans ; les hommes vertueux sont les seuls qui aient des amis\phantomsection
\label{footnote100}\footnote{M. de Voltaire. Voyez {\itshape La Raison par alphabet}, ou {\itshape Dictionnaire philosophique}, article « amitié ». « Hoc primum sentio, dit Cicéron, nisi in bonis amicitiam esse non posse. » {\itshape De l’Amitié}, chap. V.}. »\par
De tout temps l’on s’est plaint de la rareté des amis et, par la même raison, de tout temps l’on s’est plaint de la rareté de la vertu. Dans des sociétés frivoles et corrompues, l’amitié véritable doit être presque entièrement ignorée : elle n’est pas faite pour des hommes pervers toujours prêts à la sacrifier aux intérêts de leurs vices ou de leurs passions, elle n’est pas faite pour les princes dont le cœur isolé n’a besoin de s’attacher à personne, elle n’est point faite pour les grands, presque toujours divisés entre eux par leur ambition, elle n’est pas faite pour les riches, qui ne demandent que des parasites, des flatteurs, des complaisants, elle n’est point faite pour des êtres légers accoutumés à ne s’arrêter jamais sur les objets ; elle est presque totalement bannie du commerce des femmes, chez lesquelles l’amitié n’est d’ordinaire qu’un engouement passager que l’intérêt le plus léger fait promptement disparaître.\par
Rien de plus commun, en effet, que de prendre l’engouement pour de l’amitié : il en a très souvent les symptômes ; mais sa vivacité le décèle et semble annoncer qu’il n’est pas fait pour durer. Plutarque, parlant des nouvelles connaissances, dit : « Elles nous font faire plusieurs commencements d’amitié et de familiarité qui jamais ne viennent à perfection. Il faut, dit-il ailleurs\footnote{Au traité {\itshape De la Pluralité des Amis}.}, avoir mangé un minot de sel avec celui qu’on veut aimer. » Mais séduits par quelques qualités, soit de l’esprit, soit même du corps, bien des gens au premier coup d’œil croient avoir trouvé un ami ; bientôt l’illusion cesse et l’on ne voit dans cet ami prétendu qu’un homme qui n’a rien de ce qui peut constituer l’amitié véritable. Un ami, pour la plupart des hommes, est un complaisant qui les amuse, qui se prête à leurs goûts, à leurs caprices, qui partage habituellement leurs plaisirs, qui les admire, qui veut bien les aider à dissiper leur fortune. Faut-il être surpris de voir disparaître des amis de cette trempe dès que la fortune est disparue\phantomsection
\label{footnote101}\footnote{« Ceux, dit Plutarque, qui croient avoir beaucoup d’amis se croient bienheureux, bien qu’ils voient encore plus grand nombre de mouches en leur cuisine ; mais ni elles n’y demeurent point si la viande y défaut, ni eux s’ils n’y sentent du profit. » Voyez Plutarque, {\itshape De la Pluralité des Amis}. Il dit encore : « L’amitié est bien, par manière de dire, bête de compagnie, mais non pas de troupe. » Aristote s’écriait souvent : « Ô mes amis, il n’est point d’amis. » Ovide dit avec assez de raison : « Donec eris sospes, multos numerabis amicos : tempora si fuerint nubila, solus eris. »} ?\par
Tout le monde veut des amis et très peu de gens ont le discernement nécessaire pour les choisir ou les qualités propres à les fixer. O hommes, qui vous plaignez sans cesse de la rareté des amis, avez-vous donc bien réfléchi sur la force d’un titre que vous prodiguez à tous ceux qui flattent votre vanité ? Avez-vous bien songé aux dispositions sur lesquelles l’amitié doit se fonder ? Avez-vous sérieusement pesé les engagements renfermés dans ce contrat des cœurs honnêtes ? Vous prétendez inspirer à ces hommes qui vous entourent des sentiments vifs et permanents : montrez-leur donc des qualités qu’ils puissent toujours aimer. Riches et grands ! Vous ne leur montrez que de la hauteur, du faste, de la vanité ; eh bien, vous aurez autour de vous des âmes basses et rampantes mais vous n’aurez point d’amis. Si vous voulez des Pylade, soyez donc des Oreste. Vous voulez des amis qui se sacrifient pour vous dans des occasions périlleuses : songez que l’enthousiasme de l’amitié est très rare et que des milliers d’années n’en offrent que peu d’exemples.\par
L’enthousiasme, qui toujours porte les choses à l’extrême, est visiblement cause que bien des moralistes ont fait de l’amitié véritable une chimère, un être de raison, une vertu si sublime que sa perfection merveilleuse n’est propre qu’à décourager la faiblesse des mortels. On croit lire des romans ou des rêves quand on voit dans Platon, dans Cicéron, dans Lucien, les effets miraculeux qu’ils attribuent à l’amitié. Notre imagination flattée par ces riantes peintures les réalise pour nous et par là nous nous formons une fausse mesure et des principes exagérés sur l’amitié.\par
Pour nous en faire des idées véritables, souvenons-nous toujours que nous ne sommes que des hommes, c’est-à-dire des êtres remplis d’imperfections, de faiblesse, sujets à varier dans nos penchants et nos goûts ; nous sommes quelquefois très promptement fatigués des qualités qui d’abord nous promettaient les plaisirs les plus durables. Les amitiés les plus vives sont communément de très courte durée : elles partent d’un enthousiasme qui s’exhale avec rapidité. Très peu d’hommes ont une quantité suffisante de la chaleur d’âme nécessaire pour alimenter toujours un sentiment si violent. Au bout de quelques années, on balance quelquefois à faire à l’amitié des sacrifices qu’on lui eût faits sans hésiter dans ses premiers instants. D’ailleurs, dans un monde corrompu, frivole et dissipé, il est très peu d’âmes {\itshape aimantes}, et encore moins d’esprits solides. Rien de plus rare que la chaleur continue de l’âme, combinée avec la solidité, qui toujours suppose du sang-froid. C’est entre les hommes honnêtes et de sang-froid que l’on rencontre l’amitié la moins sujette à varier.\par
L’amitié véritable est sans doute en droit d’exiger des sacrifices : ce ne serait point aimer quelqu’un que de ne vouloir lui rien sacrifier. Mais, comme on l’a dit ailleurs, sacrifier quelque chose à un objet, c’est préférer cet objet à la chose qu’on lui sacrifie ou dont on se prive pour lui. Jusqu’où doit-on pousser les sacrifices dans l’amitié ? Il n’y a que la force de l’amitié qui puisse fixer la mesure de ces sacrifices. Des exemples nous prouvent que des amis ont poussé l’héroïsme jusqu’à s’immoler l’un pour l’autre ; nous devons en conclure que l’amitié était en eux si forte, était pour eux un besoin aussi grand, un intérêt aussi puissant que l’amour de la patrie et de la gloire l’a été pour quelques citoyens illustres ou que l’amour d’une maîtresse l’est pour un amant bien épris. Toute passion forte fait que l’homme qui en est remué s’oublie lui-même pour ne voir que l’objet dont son âme est occupée. Sacrifier sa fortune à son ami, c’est préférer l’indigence à la perte de cet ami.\par
Toujours épris d’eux-mêmes, les hommes, pour la plupart, sont peu disposés à se rendre justice ; ils se croient des objets tellement faits pour intéresser le monde qu’ils s’imaginent qu’il n’est rien qu’on ne doive leur sacrifier. En amitié, on veut des enthousiastes sans avoir aucune des qualités nécessaires pour allumer cet enthousiasme dans les cœurs. On exige l’attachement le plus sincère de la part d’une foule de flatteurs, de sycophantes, de complaisants, dont souvent on a fait les jouets de sa vanité, et l’on veut que des hommes de ce caractère soient des amis assez fidèle pour s’immoler à l’amitié !\par
D’un autre côté, un grand nombre de moralistes, séduits par les exemples sublimes et rares d’une amitié héroïque, n’en ont parlé qu’avec une sorte d’enthousiasme. Ils ont supposé que ce sentiment, pour être véritable, ne devait jamais mettre de bornes à ses sacrifices. Ils n’ont sans doute pas vu que très peu d’hommes sur la terre sont des héros, que très peu d’âmes sont assez exaltées pour se sacrifier elles-mêmes à l’amitié, qui pour l’ordinaire est un sentiment plus tranquille et plus réfléchi que l’amour et qui, par conséquent, permet des retours plus fréquents sur soi-même.\par
Enfin, ces moralistes n’ont pas vu qu’il y avait des degrés dans l’amitié et qu’il était possible d’aimer quelqu’un sans porter l’affection jusqu’aux derniers termes de l’enthousiasme. La morale, pour être vraie, doit voir les hommes tels qu’ils sont ; une morale enthousiaste n’est faite que pour des hommes extraordinaires et ne fait souvent que des hypocrites qui feignent des sentiments généreux dont ils se font honneur. Chacun veut se faire passer pour un ami à toutes épreuves, chacun exige de l’enthousiasme dans ses amis, tandis que tout le monde convient que rien n’est plus rare sur la terre que cette amitié sublime que l’on prétend avoir et qu’on voudrait rencontrer dans les autres.\par
Soyons justes et disons que pour mériter des amis fidèles il faut être fidèle soi-même aux devoirs de l’amitié. Avez-vous soigneusement rempli tous ces devoirs ? Avez-vous partagé les plaisirs et les peines de votre ami ? L’avez-vous consolé dans ses afflictions ? Lui avez-vous prêté dans son infortune le secours qu’il était en droit d’attendre de votre attachement ? Avez-vous défendu avec chaleur les intérêts de sa réputation quand elle était attaquée ? Avez-vous été au-devant de ses besoins quand il était dans la détresse ? Avez-vous, dans vos bienfaits, ménagé la délicatesse de son cœur ? Eh bien, vous avez acquis le droit d’attendre de sa part un attachement inviolable ; vous avez celui de vous plaindre dès qu’il a la bassesse de vous abandonner. S’il se trouve si peu d’amis constants, c’est qu’il est très peu d’hommes qui connaissent les engagements de l’amitié. Celle-ci communément paraît engager à peu de chose : à des égards, des complaisances, des procédés auxquels le cœur n’a souvent point de part. Dans le langage du monde, des amis sont des hommes associés par le plaisir, que la conformité de quelques goûts, de quelques intérêts momentanés et quelquefois de quelques vices\phantomsection
\label{footnote102}\footnote{« Magna inter molles concordia. » Juvénal, {\itshape Satires}, II, vers 47.} rassemble, met dans l’habitude de se voir plus souvent et de vivre dans une intimité plus grande qu’avec les autres. Les amis de cette espèce sont utiles ou nécessaires à leurs amusements réciproques ; tels sont les amis de table, les amis du jeu, les amis de débauche et la plupart des amis de société, dont l’objet pour l’ordinaire est de se rassembler pour jouir en commun des avantages qu’elle procure et qui ne tardent point à s’éclipser dès que les motifs qui les portaient à se fréquenter viennent à disparaître. Vainement attendrait-on des prodiges d’attachement, de constance, de fidélité de ces sortes d’amis : ils ne sont constants que dans leur attachement au plaisir, ils ne sont les amis que de ceux qu’ils croient en état de leur fournir un passe-temps agréable ; l’indifférence remplace l’amitié dès qu’ils ne trouvent plus les moyens de s’amuser.\par
C’est ainsi que par un honteux abus des mots, on donne vulgairement le nom d’amis à des personnes qui n’ont rien de ce qu’il faut pour prétendre à ce titre respectable. Pour avoir périodiquement et pendant longtemps fréquenté une maison, pour avoir pris régulièrement part aux amusements qu’elle procure, pour avoir joui de la société qu’elle réunit, des hommes se qualifient d’{\itshape amis intimes} et semblent exiger rigoureusement tous les droits attachés à cette qualité si auguste et si rare. Un illustre moderne a dit avec raison « qu’en ouvrant l’entrée de toutes les maisons, le luxe, et ce qu’on appelle l’esprit de société, a soustrait une infinité de gens au besoin de l’amitié\phantomsection
\label{footnote103}\footnote{Voyez le livre {\itshape De l’Esprit}, discours III, chap. 14, pag. 356, édition in-4°. Plutarque dit « qu’il n’est pas possible d’aimer ni d’être aimé de plusieurs… L’affection, étant départie à plusieurs, s’en affaiblit et revient presque au néant. » Voyez Plutarque, {\itshape De la Pluralité des Amis}.}. »\par
Au milieu du tumulte qu’on voit régner dans les sociétés où le luxe et la vanité ont fixé leur séjour, il est presque impossible de connaître les hommes mêmes que l’on a fréquentés le plus longtemps : ils se perdent à tout moment dans la foule, ils n’ont jamais le temps de se connaître eux-mêmes. Le tourbillon du monde éloigne et rapproche sans cesse des êtres qui s’unissent et se séparent avec la plus grande facilité. Ceux que l’on nomme des {\itshape connaissances} sont communément des êtres parfaitement inconnus ; les {\itshape liaisons} sont des attachements passagers qui ne lient personne, et ce qu’on appelle ses {\itshape amis} sont des gens que l’on voit très souvent mais dont on est rarement en état de démêler les dispositions véritables.\par
Ne soyons donc pas étonnés de la légèreté singulière avec laquelle l’amitié se traite dans la société. Contents de se montrer extérieurement quelques égards, les amis vulgaires, dont le monde est rempli, non seulement n’ont les uns pour les autres aucun attachement véritable mais encore sont souvent les premiers à médire de leurs prétendus amis, à dévoiler leurs défauts, à s’en amuser avec d’autres, et même avec des indifférents. Pour des personnes de ce caractère, l’amitié est un lien si faible, qu’elles ne s’imaginent pas même devoir à ceux qu’ils appellent leurs amis l’indulgence et l’équité que l’on doit à tous les hommes. On dirait que la plupart des gens du monde ne se lient que pour s’immoler les uns les autres.\par
Il faut se connaître pour s’aimer\footnote{La première règle en fait d’amitié, dit l’auteur du livre sur {\itshape Les Mœurs}, c’est de ne point aimer sans connaître. Une autre, qui n’est pas moins importante, c’est de ne choisir des amis que dans la classe des gens de bien. — Les plantes les plus vivaces ne sont pas celles qui croissent le plus vite. L’amitié n’est de même, pour l’ordinaire, ferme et durable que quand elle s’est formée lentement. Aimer précipitamment, c’est s’exposer à des ruptures. Voyez {\itshape partie III, chap. II}.}. L’amitié est un sentiment sérieux, réfléchi, fondé sur les besoins du cœur. Des esprits agités par une dissipation continuelle n’ont nul besoin d’amis : ils ne veulent qu’être amusés. L’amitié vraie, toujours produite par l’estime, veut trouver des qualités propres à la fixer, il lui faut des vertus auxquelles elle puisse s’attacher avec constance. Elle ne s’engage point à la légère parce qu’elle connaît l’étendue de ses engagements, elle ne trouve point à se placer dans ces âmes évaporées qui se font un jeu des liens les plus sacrés ; elle craint la dissipation, la frivolité l’importune. Les vrais amis se suffisent : pour être complètement heureux, ils n’ont besoin que d’être ensemble. Le tourbillon du monde les empêcherait de goûter les charmes des épanchements de cœur, de la confiance, des consolations, des conseils qui font la douceur de l’amitié. L’ami sincère aime à se reposer dans le sein de son ami ; il jouit avec lui d’une liberté, d’un repos que le tumulte troublerait. L’amitié, ainsi que l’amour heureux, est une passion solitaire qui pour jouir en paix fuit les regards des hommes. Comme l’amour, elle est jalouse ; comme lui, elle aime les ombres du mystère. L’indiscrétion, la vanité, la légèreté, l’étourderie lui déplaisent : elle veut de la constance, de la gravité, de la solidité.\par
L’amitié sincère, étant un besoin du cœur qui doit souvent renaître, veut être alimentée par la présence de son objet. Les attachements les plus vifs s’affaiblissent par l’absence, ainsi que par les distractions fréquentes. L’amitié est peu forte lorsqu’elle peut longtemps se priver sans douleur de celui qui l’a fait naître. C’est une maxime très sage que celle qui dit : ne laisse point croître l’herbe sur le sentier qui conduit chez ton ami. Qu’est-ce en effet qu’un ami qui ne se sent aucunement pressé de voir celui qui le chérit, qui le console, et dont la vue seule, lors même qu’il se tait, est propre à réjouir son cœur ? La vue d’un ami, dit un Arabe, rafraîchit comme la rosée du matin. Une maxime ancienne\phantomsection
\label{footnote104}\footnote{Cicéron l’attribue à Bias. Voyez {\itshape De l’Amitié}, chap. XVI.} conseille aux amis de {\itshape s’aimer comme pouvant un jour se haïr}. Elle serait odieuse dans l’amitié sincère, qui ne peut admettre la défiance après avoir bien connu l’objet de son attachement, mais cette maxime est très bien placée dans les liaisons futiles que l’on qualifie très faussement du nom d’{\itshape amitié} ; elle est très prudente dans ces amitiés qui n’ont pour fondement que le vice et la débauche. Elle devrait être sans cesse devant les yeux de ces prétendus amis qui ne se lient que pour des cabales méprisables, des intrigues criminelles, des intérêts sujets à mettre la discorde entre les associés : l’indiscrétion, le vertige, la trahison, la noirceur, accompagnent si souvent des liaisons de ce genre, que l’on ne peut trop conseiller à ceux qui s’y livrent de prévoir les suites de leurs engagements dangereux.\par
Ne point croire à l’amitié serait une extrémité plus malheureuse que de s’y fier aveuglément ou que de s’en faire des idées romanesques ou trop sublimes. S’il existe dans le monde des âmes arides et peu capable d’aimer, si l’on y trouve une multitude d’êtres frivoles et légers sur lesquels il serait fort imprudent de compter, on y rencontre des cœurs honnêtes, sensibles, solides, auxquels l’homme de bien s’attachera par sympathie parce qu’il y trouvera des sentiments conformes à ceux dont il est animé. L’univers ne serait pour nous qu’une affreuse solitude si une défiance continuelle nous empêchait d’y rien aimer. D’un autre côté, nous passerions toute la vie à chercher sans succès si nous ne voulions nous attacher qu’à des hommes parfaits. Les maximes peu favorables à l’amitié ou capables de la rendre suspecte sont dues à des penseurs qui vivaient à la cour ou sous des gouvernements despotiques, d’où il est très naturel que la confiance et l’amitié soient bannies. Ces auteurs n’ont pas décrié l’amitié ; ils ont voulu faire entendre qu’elle n’existait pas dans les pays qu’ils habitaient\phantomsection
\label{footnote105}\footnote{Voyez {\itshape Les Poésies de Saadi}, le livre {\itshape De l’Esprit}, les {\itshape Maximes} de La Rochefoucault.}. Ce n’est pas dans ces pays que l’on trouve des amis bien sincères, ni de quoi peindre l’espèce humaine sous ses traits les plus beaux. Il n’y a, je le répète, que la vertu qui puisse donner la confiance nécessaire dans l’amitié, il n’y a que l’homme de bien qui soit un sûr dépositaire des secrets qu’on lui confie. Il n’y a que l’homme vertueux dont les intérêts ne changent pas et sur la discrétion duquel on puisse compter en sûreté. Le vice est imprudent lorsqu’il se confie au vice, dont les intérêts variables changent à tout moment. C’est être aveugle que de confier un secret important à l’homme faible, vain et léger, qui ne pourra le garder ; un tel homme n’est pas fait pour l’amitié. Trahir son ami par faiblesse ou par légèreté peut avoir des suites aussi fâcheuses que le trahir par la méchanceté la plus noire.\par
« La première loi de l’amitié, dit Cicéron, veut que les amis n’exigent pas des choses déshonnêtes, et que l’on refuse de s’y prêter. Car, dit-il ailleurs, si l’on était obligé de faire tout ce que des amis peuvent demander, une telle amitié devrait être regardée comme une conjuration\footnote{« Hæ igitur prima lex in amicitiâ sanciatur, ut neque rogemus res turpes, nec faciamus rogati. » Cicéron, {\itshape De l’Amitié}, XII, 40. « Nam si omnia facienda sint, quæ amicitiæ taies, sed conjurationes putandæ sunt. » {\itshape De Officiis}, livre III, 10.}. » Enfin, ce grand orateur nous apprend que « la Nature a donné l’amitié pour prêter ses secours aux vertus et non pour être la compagne du vice\footnote{« Virtutum amicitia adjutrix a natura data est, non vitiorum cornes. » Cicéron, {\itshape De l’Amitié}.} ». Si la vertu seule peut consolider les liens de l’amitié sincère, cette amitié doit disparaître dès que le crime se montre. Un ami véritable ne peut exiger de son ami des complaisances injustes et déshonorantes. Il n’y a que des hommes sans vertu, de faux amis, des complaisants avilis qui puissent se prêter au crime. L’ami vertueux, en trouvant son ami criminel, gémit et reconnaît qu’il s’est trompé. Rutilius ayant refusé de commettre une injustice pour obliger son ami, celui-ci, très mécontent, lui dit : « À quoi donc me sert ton amitié ? Mais à quoi me servira la tienne, si elle me rend injuste ? » lui répliqua Rutilius\footnote{Voyez Valère Maxime, {\itshape Des Faits et des Paroles mémorables}.}. Phocion disait au roi Antipater : « Vous ne pouvez m’avoir en même temps pour flatteur et pour ami. » Telle est la conduite que la morale propose à l’amitié qui, comme tout concourt à le prouver, ne peut être sûre et constante que lorsqu’elle unit des êtres réfléchis, raisonnables, vertueux ; le meilleur des amis, dit un sage d’Orient, « est celui qui avertit son ami quand il s’égare et qui le remet dans son chemin\footnote{Voyez {\itshape Sentent. Arab.}} ».\par
Néanmoins, plus la corruption est grande, plus les gens de bien ont besoin des consolations de l’amitié ; elle est faite pour les dédommager des rigueurs de la tyrannie, de l’injustice des hommes, de la dépravation des mœurs ; elle leur fait trouver dans son sein une félicité particulière et secrète qu’ils préfèrent à celle qu’ils chercheraient vainement dans le tumulte des plaisirs ou dans les désordres de la société. « L’amitié, dit Démophile, est le port de la vie ».\par
L’homme est-il soumis à des devoirs envers ses ennemis ? Oui, sans doute ; on leur doit la justice et l’humanité. Rien n’annonce plus l’équité, que de reconnaître le mérite dans ceux même desquels on a sujet de se plaindre. Rien ne montre plus de vraie grandeur dans l’âme, que d’oublier les injures et de faire du bien à ceux qui nous ont fait du mal. C’est, comme on l’a dit ailleurs, le moyen le plus sûr de désarmer la colère, l’envie, l’inimitié. Diogène disait « que l’on pouvait se venger de ses ennemis en se rendant soi-même homme de bien et vertueux ». « Nous devons, dit-il encore, tâcher d’avoir de bons amis pour nous apprendre à faire le bien, et de méchants ennemis pour nous empêcher de mal faire. »\par
Xénophon dit que l’homme sage sait tirer un grand profit de ses ennemis. Un ennemi sensé, dit un poète d’Orient, vaut mieux qu’un sot ami. Un flatteur ayant exhorté Philippe de Macédoine à se venger des discours insolents que Nicanor avait tenu sur son compte, ne vaudrait-il pas mieux, lui répondit ce prince, examiner si je n’y aurais pas donné lieu ? Le même prince disait que les harangueurs d’Athènes, en parlant mal de lui, lui fournissaient le moyen de se corriger de ses fautes\phantomsection
\label{footnote106}\footnote{Voyez Plutarque, {\itshape Dits notables des Princes}, et dans le traité {\itshape De l’Utilité des Ennemis}.}.\par
Nous pouvons donc tirer des fruits utiles du sein même de nos ennemis, à l’égard desquels rien ne nous doit dispenser d’être humains et justes. Disons avec Théognis : « Je ne mépriserai aucun de mes ennemis, s’il est bon ; je ne louerai aucun de mes amis, s’il est pervers\phantomsection
\label{footnote107}\footnote{Voyez {\itshape Poëta græci minores}.}. »
\subsection[{Chapitre VI. Devoirs des Maîtres et des Serviteurs}]{Chapitre VI. Devoirs des Maîtres et des Serviteurs}
\noindent Les riches, comme on a vu, mettent les pauvres dans leur dépendance et par les avantages qu’ils leur font obtenir, exercent sur eux une autorité légitime, c’est-à-dire avouée, consentie par ceux-ci lorsqu’elle les met à portée de jouir d’un bien-être qu’ils ne pourraient pas se procurer par eux-mêmes.\par
Tel est le fondement naturel de l’autorité que les maîtres exercent sur leurs domestiques. Cette autorité, comme toutes les autres, devient une usurpation tyrannique lorsqu’elle s’exerce d’une façon injuste et cruelle. Nul homme, comme on ne peut trop souvent le répéter, ne peut acquérir le droit de commander d’autres pour les rendre malheureux {\itshape ;} les mauvais traitements d’un maître dépourvu de justice et d’humanité sont des violences manifestes que les lois devraient réprimer.\par
Chez les Romains, dont les lois étaient aussi féroces qu’eux, les esclaves n’étaient point réputés des hommes ; il semblait à ces brigands que la captivité les eût dénaturés. Leurs maîtres ont pu longtemps disposer de leur vie même et les traitaient comme un bétail destiné à servir de jouet à leurs caprices les plus barbares. Mais par la suite, des lois plus humaines arrachèrent aux maîtres la faculté d’exercer une tyrannie si détestable ; elles voulurent que les esclaves fussent traités comme des hommes. Enfin, l’esclavage fut aboli en Europe ; les chefs des familles furent servis par des hommes libres qui, sous de certaines conditions, consentirent à leur rendre les services qu’ils pouvaient désirer ou à les exempter des travaux qui leur paraissaient trop pénibles.\par
Ainsi, la raison humaine se développant avec le temps, guérit peu à peu les nations de leur barbarie et les ramène à des usages plus équitables, plus conformes à la morale, à l’intérêt du genre humain. Cette morale crie à tous les habitants du monde que riches ou pauvres, puissants ou faibles, heureux ou malheureux, ils sont de la même espèce, qu’ils ont des droits égaux à l’équité, à la bienfaisance, à la pitié de leurs semblables.\par
Mais sa voix ne se fait point entendre à ces mêmes Européens quand leur avidité les a transplantés dans un nouveau monde. Vous les voyez dans ces climats commander en vrais tyrans à des nègres malheureux qu’un commerce odieux achète comme de vils animaux pour les revendre à des maîtres impitoyables qui leur font éprouver les cruautés et les caprices dont l’insolence, l’impunité, l’avarice peuvent rendre capable. Ce trafic abominable est pourtant autorisé par les lois de nations qui se disent humaines et policées, tandis qu’un intérêt sordide leur fait évidemment méconnaître les droits les plus saints de l’humanité.\par
Elle devrait leur faire sentir que des hommes noirs sont des hommes sur la liberté desquels des hommes blancs n’ont pas le droit d’attenter, ou du moins qu’ils devraient traiter avec bonté lorsque le destin les soumet à leur puissance\footnote{Les papiers anglais ont récemment dénoncé à l’exécration publique l’insolente cruauté d’un habitant de la Jamaïque qui s’est mis dans l’usage de faire traîner sa voiture, qu’il conduit lui-même, par six nègres auxquels, pendant la plus grande chaleur, il fait parcourir une lieue et demie par heure à grands coups de fouet. Une relation précédente de la même contrée assure qu’un habitant eut un jour la cruauté de faire mettre à la broche un de ses nègres. De pareilles horreurs prouvent à quels excès d’insolence et de barbarie l’opulence peut porter quand elle n’est pas réprimée par l’éducation et les lois. Comment le peuple anglais, si jaloux de sa propre liberté, abandonne-t-il des Africains malheureux aux caprices furieux des colons américains ? Mais l’intérêt sordide du commerce étouffe dans des marchands le cri de l’humanité. Le sensible marquis de Beccaria, dans son traité célèbre, {\itshape Des Délits et des Peines}, dit que « dans toutes les sociétés humaines il subsiste un effort continuel qui tend à conférer le pouvoir et le bonheur à une portion des associés et à réduire l’autre portion dans la faiblesse et la misère. Les bonnes lois sont faites pour s’opposer à cet effort, etc. » Mais les lois, faites par des oppresseurs et des maîtres, se sont rarement occupées des intérêts du malheureux.}.\par
Les hommes n’obéissent volontairement à d’autres que lorsque l’obéissance leur est utile. Les maîtres forment avec leurs domestiques une société dont les conditions portent que les premiers s’engagent de leur donner des soins, de leur fournir un bien-être et des moyens de subsister qu’ils ne seraient pas en état de se procurer eux-mêmes.\par
En échange, les serviteurs s’engagent à servir leurs maîtres, c’est-à-dire à travailler pour eux, à recevoir leurs ordres, à les accomplir fidèlement, à veiller sur leurs intérêts.\par
D’où l’on voit clairement que la justice veut que les conditions de ce contrat soient de part et d’autre fidèlement exécutées, vu que nul homme ne peut obliger les autres par des conventions qu’il se permettrait de violer.\par
Mais, comme une malheureuse expérience ne le prouve que trop souvent, la grandeur, la puissance, les richesses font communément oublier l’équité ; les personnes qui jouissent de ces avantages se persuadent pour l’ordinaire qu’ils ne doivent rien à ceux qui s’en trouvent privés. Ces malheureux, loin d’exciter la compassion ou la bienveillance dans les cœurs des heureux, semblent n’y faire naître qu’un orgueil insultant et leur faire croire que celui qu’ils voient abattu à leurs pieds est un être d’une espèce différente de la leur. Contents de se faire craindre, les hommes, pour la plupart, ne s’embarrassent aucunement de se faire aimer.\par
Une disposition si contraire à l’humanité devrait être soigneusement combattue et déracinée dans l’enfance. Rien de plus impérieux qu’un enfant que les moindres contradictions jettent souvent dans des colères convulsives : si l’éducation néglige de réprimer à temps ces premiers mouvements, ils se changent en habitudes et ne peuvent plus se détruire. La hauteur, la dureté, la colère habituelle d’un maître envers ses domestiques annonce toujours une éducation négligée.\par
« Accoutumez-vous, dit Madame de Lambert, à montrer de la bonté pour vos domestiques. Un Ancien (Sénèque) dit qu’il faut les regarder {\itshape comme des amis malheureux}. Songez que vous ne devez qu’au hasard l’extrême différence qu’il y a de vous à eux. Ne leur faites point sentir leur état, n’appesantissez point leurs peines, rien n’est si bas que d’être haut à qui nous est soumis.\par
« Aimez l’ordre et tempérez le sérieux qui vous convient comme maître, par la douceur et l’affabilité envers ceux qui vous servent. Souvenez-vous toujours que comme hommes ils sont vos égaux, et qu’il n’y a point de proportion entre le salaire, même le plus fort, et la dure nécessité dans laquelle se trouve celui qui rend à son semblable les offices de serviteur. »\par
On ne peut rien ajouter à des conseils si humains et si sages. Ce n’est aucunement par une conduite hautaine et dure que l’on pourra se faire servir avec zèle ; la colère du maître trouble le serviteur, l’irrite intérieurement et l’empêche de bien faire ce qu’on lui commande.\par
Si cette colère est habituelle, il s’y fait, la méprise et porte continuellement dans son cœur une haine comprimée qui peut souvent éclater d’une façon très funeste. Bien des maîtres, par leur conduite insensée, ressemblent à ces gardiens de bêtes dont ils excitent la férocité au risque d’en être tôt ou tard dévorés. Ils doivent regarder leurs domestiques comme des ennemis, puisqu’ils prennent soin d’étouffer dans leurs âmes tout sentiment d’affection et d’honneur.\par
Presque toujours, les mauvais maîtres font les mauvais serviteurs. « Sommes-nous en droit, dit la même personne, de vouloir nos domestiques sans défauts, nous qui leur en montrons tous les jours ? Il faut en souffrir. Quand vous leur montrez de l’humeur ou de la colère, quel spectacle n’offrez-vous point à leurs yeux ? Ne vous ôtez-vous pas le droit de les reprendre ? »\par
Un maître prudent doit se sentir intéressé à veiller sur la conduite et les mœurs de ses gens : sa sûreté, sa vie, dépendent de leur fidélité.\par
À quels dangers n’est point journellement exposé un maître dont le valet est ivrogne, joueur, livré à la crapule, à la débauche ? Ces vices dans des êtres sans raison, et sans principes surtout, peuvent avoir les plus terribles conséquences.\par
Si des maîtres ont eu le bonheur d’avoir une éducation plus raisonnable que les infortunés dont ils reçoivent les services, ils doivent au moins le leur prouver par leur conduite. « Donnez, dit la même dame, un bon exemple à vos domestiques, et pensez bien, mon fils, qu’un maître s’humilie de la façon la plus honteuse et se met au-dessous de ses domestiques quand ils sont les témoins ou les ministres de ses crimes, qu’ils ne trouvent pas en lui les qualités qui seules rendent un maître digne du respect et qui lui attachent le cœur de ses gens. » Un maître débauché, dissolu, obéré, qui par des escroqueries cherche à fournir à ses folies, est-il un homme bien respectable aux yeux de son valet ?\par
Une maîtresse qui a rendu ses femmes confidentes de ses intrigues criminelles, est-elle en droit d’exiger leur estime et leur soumission ? N’a-t-elle pas tout lieu de craindre à tout moment qu’elles ne publient les honteux secrets dont elles sont dépositaires ? Pour être aimé, il faut qu’un maître fasse éprouver à ses serviteurs des sentiments de bonté ; pour être respecté, il faut qu’il ne leur laisse apercevoir qu’une conduite décente dont il ne puisse rougir quand elle serait divulguée.\par
La bonté du maître ne consiste pas dans une familiarité souvent capable d’attirer le mépris : elle consiste à leur montrer de la bienveillance, à les secourir dans leurs infirmités, à les aider dans leurs entreprises honnêtes, à reconnaître leur bonne conduite, à les récompenser de leur attachement et de leur zèle. Une familiarité trop grande diminue le respect et la vigilance des domestiques ; rien de plus monstrueux qu’une maison où les valets sont les maîtres : ceux qui sont faits pour commander y deviennent des esclaves et le désordre est la suite de cette démocratie.\par
Combien de familles divisées et ruinées par la facilité que des maîtres ont de prêter l’oreille aux discours de leurs gens ? Les femmes sont surtout sujettes à cette faiblesse, de laquelle il résulte souvent des brouilleries entre les époux, les parents, les enfants, les amis. Quand même ces tracasseries ne feraient que diviser les domestiques entre eux, elles nuiraient toujours à l’harmonie nécessaire dans toute maison bien ordonnée. Les valets sont pour l’ordinaire trop sujets à leurs passions pour être écoutés par des maîtres prudents ; leurs querelles cessent très promptement dès qu’on refuse de les écouter ; ces procès deviennent interminables quand des maîtres ont la faiblesse de vouloir les juger.\par
L’état heureux ou malheureux d’une maison annonce le caractère de ceux qui la gouvernent. Une maison bien réglée, une famille bien unie, des domestiques soumis et tranquilles annoncent un maître sage et respectable ; une maison, au contraire, dépourvue de règle, désunie, remplie de valets mutins, annonce dans son chef une conduite désordonnée, des vices, ou du moins une indolence digne d’être blâmée.\par
Rien de moins commun qu’une maison bien réglée, parce que rien n’est plus rare que des maîtres capables d’établir chez eux l’ordre et de l’y maintenir. Le maître honnête et vigilant ne veut être servi que par des serviteurs honnêtes ; il les rend tels par sa propre conduite. Les fripons, s’y trouvant déplacés, ne tardent pas à s’éloigner. Des valets insolents annoncent pour l’ordinaire des maîtres gonflés d’un sot orgueil.\par
Rien de plus révoltant dans la société que l’impertinence si commune aux valets des riches et des grands\footnote{« Maxima quæque domus servis est plena superbis. » Juvénal, {\itshape Satires}, V, vers 66.}. La manière arrogante dont ces esclaves hautains reçoivent si fréquemment le mérite timide et l’infortune tremblante, est un des malheurs les plus cuisants pour la vertu réduite à solliciter.\par
Un maître, s’il a des sentiments, doit punir sévèrement ses gens quand ils s’oublient : la haine que produit leur insolence retomberait sur lui-même. Est-il rien de plus bas que la vanité de ces hommes altiers qui croient leur grandeur intéressée à soutenir l’impertinence de leurs valets ? L’impunité honteuse dont les grands et les riches jouissent dans bien des nations s’étend communément à leurs valets et devient une source de maux cruels pour le pauvre dénué de protection.\par
Dans des capitales immenses et très peuplées, rien de plus fréquent que de voir des personnes écrasées ou blessées soit par l’imprudence, la méchanceté des cochers, soit par la négligence et la vanité des maîtres. Quelles sottes idées de gloire doivent avoir des {\itshape petits maîtres} qui, de même que leurs valets, se font un plaisir d’inspirer la terreur aux passants. Quels cœurs doivent posséder des furieux qui se font un jeu de la vie de leurs concitoyens ? Un artisan, un père, une mère estropiés réduisent souvent une famille nombreuse à la misère, et de pareils accidents sont des amusements pour l’opulence insolente et pour ses valets !\par
Des lois sévères devraient réprimer l’impétuosité de ces riches désœuvrés dont l’objet le plus pressant n’est souvent que de promener rapidement leur ennui et leur oisiveté. Une police exacte et rigoureuse devrait châtier exemplairement ces valets qui, protégés par des maîtres puissants, se croient en droit d’insulter ou de blesser des gens honnêtes qu’ils devraient respecter. Les âmes basses sont arrogantes et cruelles quand elles se sentent appuyées. D’ailleurs ceux qui font les lois, ainsi que ceux qui les font observer, étant eux-mêmes à couvert des dangers dont le pauvre est entouré, ne songent guère à les écarter et montrent une indulgence funeste à la grandeur ou à l’opulence.\par
Rien dans la société ne devrait être plus sacré que la vie du moindre citoyen, souvent plus utile à l’État que le riche qui l’écrase. Il n’est pas d’affaires assez pressées pour disculper un téméraire qui, dans sa course inconsidérée, blesse ou fait périr son semblable. Faut-il que la vie d’un homme ne soit comptée pour rien dans des états policés ?\par
Dans les pays où règne le luxe, les grands, par une sotte vanité, semblent inviter leurs domestiques à s’oublier. En habillant trop richement ces hommes grossiers, ils s’imaginent valoir mieux que des citoyens modestes auxquels ils devraient des égards et du respect. Trop souvent, le vulgaire imbécile ne juge les personnes que par leurs habits, l’homme de mérite est souvent exposé aux mépris d’un valet qui se croit au dessus de lui parce qu’il se voit mieux vêtu. Un domestique doit être habillé d’une façon conforme à son état mais les lois devraient réprimer un faste qui tend à confondre les rangs divers dans lesquels les citoyens doivent être partagés.\par
On voit quelquefois les valets d’un traitant ou d’un grand plus richement vêtus qu’un guerrier sans fortune qui a longtemps exposé sa vie pour le service du prince ! Le citoyen peu riche et qui a besoin de protection est obligé de faire souvent une dépense qui surpasse ses facultés pour n’être point rebuté par des valets impertinents.\par
Un maître est responsable au public de la conduite de ses gens. C’est à lui qu’il appartient de réprimer en eux les vices incommodes à la société. Si nous la voyons infectée de tant de valets arrogants, corrompus, libertins, nous pouvons en conclure que les exemples de leurs maîtres contribuent à faire naître en eux les vices qu’on y trouve. Des maîtres sans mœurs font de leurs valets les confidents et les ministres de leurs débauches, leurs âmes avilies par cet indigne métier deviennent étrangères à tout sentiment d’honneur ; bientôt le serviteur veut imiter son maître et pour y parvenir il a recours au larcin, à la rapine.\par
C’est ainsi que des maîtres pervers corrompent leurs serviteurs et souvent sont assez injustes pour se plaindre de leur bassesse et de leur rapacité, dont ils sont les premières causes. C’est ainsi qu’en leur apprenant par leur exemple à mépriser les mœurs, ils les conduisent au crime.\par
D’un autre côté, le luxe, en multipliant les domestiques dans les villes, remplit la société de fainéants que tout invite au désordre pour remplir le vide d’un temps qu’ils ne savent point employer. Le désœuvrement des valets est pour eux, ainsi que pour les autres, une source féconde de dérèglements. Une politique prévoyante devrait remédier aux inconvénients de ce luxe qui prive les campagnes de ses cultivateurs, qui fait refluer dans les cités une foule de paresseux sans principes et sans mœurs dont la principale occupation n’est souvent que de propager la corruption jusque dans les dernières classes du peuple. L’enfant d’un laboureur, utile à la campagne, devient nuisible à la ville. Il y fait souvent du mal, lors même qu’il a des mœurs. S’il se marie pour les conserver, il peuple la société d’enfants qu’il ne peut guère élever et soutenir sans recourir à des voies préjudiciables à son maître.\par
D’un autre côté, ses enfants devenus grands sont souvent obligés de chercher dans la débauche et même dans le crime des moyens de se tirer de l’indigence où ils sont nés\phantomsection
\label{footnote108}\footnote{« Personne, selon Bayle, ne fait des enfants de meilleur cœur que les pauvres, sachant bien qu’ils ne les nourriront pas. »}. C’est évidemment dans les mariages des valets qu’on peut trouver la source de tant de prostituées, de filous, de fainéants, de malfaiteurs de toute espèce dont les nations opulentes sont inondées. Les pauvres à la campagne se livrent au travail mais les pauvres à la ville se livrent soit au crime, soit à la mendicité, moyens presque également pernicieux à la société. Si la multiplicité des domestiques paraît flatteuse à la vanité de quelques maîtres, elle n’est pas moins contraire à leurs intérêts qu’à ceux du public ; ils en seront moins bien servis en remplissant leurs maisons d’une foule de fainéants dont ils ne peuvent employer utilement les bras.\par
Une maison trop nombreuse devient une machine trop compliquée pour pouvoir en diriger les mouvements avec facilité. La multiplicité des valets fait naître dans les maisons opulentes des abus, des rapines, des vols d’usage déguisés sous le nom de {\itshape droits}, auxquels des maîtres faciles ont la faiblesse de conniver. Mais cette facilité ne fait que des ingrats et cette prétendue générosité ou bonté ne fait que des fripons qui se croient autorisés à voler toutes les fois qu’ils s’imaginent pouvoir le faire sans danger.\par
Tout nous prouve qu’un nombreux domestique, par les désordres qu’il entraîne, est une des principales causes de la ruine des grandes maisons et du peu d’aisance que l’on trouve assez souvent chez les grands : faute d’avoir le temps ou la capacité de s’occuper de leurs propres affaires, ils s’en rapportent communément à quelques mercenaires qui, profitant de leurs désordres et de leur négligence, ne font qu’accélérer leur destruction. {\itshape L’œil du maître} est un mot que chacun a dans la bouche mais dont la dissipation, la frivolité et le vice font à tout moment négliger la pratique.\par
Il n’y a qu’une vanité bien puérile qui ait pu persuader à des grands qu’il était au-dessous d’eux d’être au fait de leurs affaires, de s’en occuper eux-mêmes, et que la grandeur consistait à n’y rien entendre, à se laisser dévorer par une troupe de valets inutiles, à souffrir le désordre chez eux, à se laisser accabler de dettes, à se faire sans cesse importuner par des créanciers. Une façon de penser si étrange ne peut être qu’une suite des préjugés gothiques de la noblesse, qui lui faisaient croire qu’excepté le métier de la guerre il lui était honorable d’ignorer tout le reste.\par
Aux yeux de la raison, rien n’est plus déshonorant qu’une négligence et une impéritie dont l’effet est d’être sans cesse dupé par des fripons. Rien de plus ignoble que de se réduire par son incurie dans une sorte de misère : quelle différence y a-t-il entre un indigent et un riche malaisé ? Est-il rien de plus injuste et de plus bas que de se mettre par sa faute et ses folies dans le cas de frustrer ses créanciers de ce qui leur est dû et d’accumuler des dettes sans dessein de les payer ? Si la grandeur consistait dans une pareille conduite, les grands devraient être regardés comme les plus fous et les plus méprisables des hommes. « Il est, dit Plutarque, honnête et convenable de veiller sur son bien, afin de s’abstenir du bien des autres\phantomsection
\label{footnote109}\footnote{Voyez Plutarque, {\itshape Vie de Philopœmen}. Xénophon fait dire à Socrate qu’il convient à tout homme sensé et à tout bon citoyen d’améliorer son bien.}. »\par
Tout chef de famille se doit à lui-même et à sa postérité de veiller à ses affaires {\itshape ;} sa vigilance est un devoir, et sa négligence serait un vice impardonnable.\par
Un maître sensé doit trouver une occupation agréable dans les soins qu’il donne à ses propres affaires, il ne dédaignera point une sage économie qui seule fera régner l’abondance chez lui. Il veut être le maître de sa félicité ; il sait que le désordre et l’indigence plongent les grands dans la dépendance et le mépris, et que l’insensé qui s’est ruiné est forcé de recourir à des expédients indignes de toute âme honnête et noble. Les bassesses et les infamies, qui souvent déshonorent les grands sont évidemment dues au défaut d’économie, aux dépenses énormes dans lesquelles leur vanité, leur paresse, leurs dérèglements les entraînent. Il faut ramper quand on veut réparer une fortune dérangée par l’extravagance.\par
Est-il une position plus heureuse que celle d’un chef de famille vertueux et sagement occupé de ses devoirs ? Les soins qu’il se donne sont récompensés à tout moment par les sentiments qu’il éprouve de la part de tous ceux qu’il voit autour de lui. Il jouit de ses biens, dont les grands savent si rarement jouir, il fait sortir l’abondance des terrains même les plus stériles, il encourage l’industrie de ses fermiers, il a le plaisir de créer, de commander à la Nature, de la forcer d’obéir à ses volontés.\par
Sous ses yeux, tout prospère, ses vassaux travaillent et s’enrichissent, ses domestiques secondent ses vues et partagent avec leur maître les avantages de l’opulence, qui le met à portée de récompenser et de faire des heureux.\par
Tel est le but que devraient se proposer pour leur propre intérêt les seigneurs, les grands, les possesseurs de terre. Une vie ainsi occupée ne serait-elle pas préférable à cette vie inquiète et fastidieuse qu’ils mènent dans les cours ou dans les villes où, à force d’amusements et de plaisirs, on finit communément par ne plus jouir de rien ?\par
C’est en répandant le bien-être sur un grand nombre d’hommes que l’on peut vraiment montrer sa grandeur et sa puissance. C’est en occupant les hommes qu’on peut les enrichir et se procurer à soi-même une opulence légitime. C’est en s’occupant utilement que l’on se dérobe à l’ennui, au désordre, et que l’on prévient les dérèglements des serviteurs. C’est en les rendant heureux par des bienfaits réels que l’on fait naître en eux le respect, la soumission, la fidélité, l’amour de leurs devoirs.\par
Le serviteur doit respecter dans son maître celui de qui dépend sa propre félicité ; son intérêt l’invite donc à lui montrer invariablement la déférence que son état lui impose. Il doit craindre de lui déplaire par des manières arrogantes ou par des murmures indiscrets ; il doit s’armer de patience, parce que la patience est la vertu de son état, qui le destine à souffrir les variations auxquelles sont sujets les hommes même du meilleur caractère ; il se promettra par là de désarmer la colère. Tout lui prouve, en effet, que la fureur la plus enflammée s’éteint par la douceur ; il obéira donc sans réplique aux ordres qu’on lui donne.\par
Un maître juste n’ordonne rien que de juste ; un maître injuste doit être abandonné. Le serviteur remplira soigneusement la tâche qui lui sera prescrite et cherchera le moyen de s’en acquitter de son mieux. Il évitera la maladresse, qui n’est due pour l’ordinaire qu’à la précipitation, au défaut d’attention. Il en apportera même dans les petites choses, afin de s’épargner des reproches toujours humiliants. Il sera exact et ponctuel, afin de ne point s’attirer la mauvaise humeur de celui dont il doit rechercher la bienveillance à tout moment. Il observera surtout les règles de la plus exacte fidélité ; il se souviendra qu’en entrant au service de son maître il s’est engagé non seulement à respecter sa propriété, mais encore à la défendre contre les autres et à confondre ses intérêts avec les siens.\par
Par un abus contraire à toute justice, les domestiques s’accoutument souvent à retirer des rétributions de ceux qui fournissent des denrées ou des marchandises à la maison de leurs maîtres.\par
Mais un serviteur fidèle reconnaîtra sans peine que ces prétendus {\itshape profits} ou {\itshape droits}, quoique autorisés par l’usage des valets corrompus ou des maîtres négligents, ne peuvent être réputés légitimes et ne sont dans le fait que des vols déguisés.\par
Enfin, le serviteur honnête craindra l’oisiveté comme pouvant devenir le chemin du vice et du crime ; il cherchera donc à remplir par quelque travail utile les intervalles que lui laissera le service de son maître.\par
Par là il emploiera d’une façon avantageuse pour lui-même des moments que des valets paresseux donnent au jeu, à l’intempérance, à la débauche. En tenant cette conduite, un domestique aura droit de prétendre à l’amitié, à la tendresse de tout maître en qui la vanité n’aura pas étouffé tout sentiment de gratitude et de justice.\par
Mépriser un serviteur de ce caractère, ce serait se montrer dépourvu de raison et d’équité. Un serviteur attaché est un ami bien plus sûr que la plupart de ceux que l’on rencontre dans le monde. Un maître qui n’aurait pas pour lui des égards et de la reconnaissance, serait ennemi de lui-même et se rendrait digne de mépris.\par
Combien d’esclaves que l’opinion et le préjugé dédaignent, ont fait éclater pour leurs maîtres un zèle, une générosité si nobles, qu’ils méritent d’être célébrés à bien plus juste titre que tant de héros que l’univers admire\phantomsection
\label{footnote110}\footnote{Valère Maxime rapporte plusieurs exemples d’esclaves qui ont généreusement sacrifié leur vie pour sauver celle de leurs maîtres. Tacite cite l’esclave de Piton : celui-ci étant proscrit, son valet prit son nom et se laissa tuer en sa place. Sous Caligula, une femme esclave endura courageusement la torture la plus cruelle sans vouloir rien avouer qui pût faire tort à son maître. L’illustre Catinat, disgracié et dépourvu d’argent, trouva dans son valet de chambre un ami généreux qui lui remit avec joie toute sa petite fortune. Combien d’officiers et de généraux dans les combats ont été redevables de la vie au courage de leurs domestiques qui se sont exposés aux plus grands dangers pour eux ? Tels sont pourtant les gens que des maîtres hautains daignent à peine regarder comme des êtres de leur espèce ! Il est des maîtres qui prennent leurs domestiques pour des bêtes de somme ; ils leur permettent à peine de manger, de dormir, d’être fatigués ou malades, d’être sensibles à la douleur, de s’apercevoir des outrages et des duretés qu’on leur fait éprouver. Des sybarites amollis, des femmes pour qui le moindre mouvement est insupportable, oubliant leur propre misère, leur ineptie, leur faiblesse, exigent une force, une promptitude, une adresse inconcevable pour les infortunés qui les servent. En Amérique et en Asie, où la chaleur du climat redouble l’indolence et la paresse, une femme trop délicate pour ramasser un mouchoir fait impitoyablement fustiger une esclave pour les fautes les plus légères. On peut en général remarquer qu’il n’y a point de service plus dur que celui des parvenus, des gens de rien enrichis ; étonnés d’un pouvoir qui n’était pas fait pour eux, ils exercent un empire cruel sur leurs malheureux serviteurs. « Il n’y a, dit Claudien, rien de plus dur qu’un homme de rien qui s’est élevé bien haut. » La hauteur et la dureté envers ses domestiques annonce l’injustice, l’ingratitude, le mauvais cœur, et surtout une très grande lâcheté. Est-il rien de plus lâche que d’exercer un pouvoir cruel sur des malheureux qu’on voit enchaînés sans défense à ses pieds ? Cependant, ces hommes dédaignés dont on fait les jouets des caprices les plus barbares ont montré quelquefois des sentiments d’honneur et d’héroïsme dont leurs indignes maîtres seraient totalement incapables. Dans un établissement européen du Nouveau Monde, on manquait d’un bourreau pour faire mourir des nègres fugitifs que l’on avait repris ; pour suppléer à ce défaut, un Créole ordonne à l’un de ses esclaves de pendre ces infortunés. Celui-ci disparaît un instant mais il revient bientôt avec une hache dont il s’était servi pour se couper une main ; offrant alors son bras sanglant et tronqué à son maître : « Force-moi donc à présent, lui dit-il, de devenir le bourreau de mes frères. »} !\par
Que des hommes superbes cessent donc de mépriser et de traiter avec dureté des serviteurs nécessaires à leur propre félicité, sans lesquels ils seraient obligés de se servir eux-mêmes, qu’un maître respecte dans celui qui le sert l’humanité malheureuse, qu’il ne l’outrage jamais, qu’il voie toujours en lui et son semblable et l’homme utile à son propre bonheur. Quand il aura éprouvé son attachement, ses soins assidus, sa fidélité, qu’il le chérisse, qu’il le traite comme un ami sincère, qu’il se souvienne que le salaire qu’il lui donne ne dispense pas de la reconnaissance et qu’il est toujours fort au-dessous de ce qu’il lui doit. Est-il rien de plus honteux que de voir tant de maîtres regarder comme des dettes les services les plus pénibles d’un domestique que souvent ils ne paient que par des hauteurs et par la plus noire ingratitude ? Des gages souvent très modiques pourraient-ils donc les acquitter pleinement envers un serviteur attentif et fidèle, des soins assidus et dégoûtants qu’exigent de longues maladies, des travaux que demandent des voyages fatigants, enfin du renoncement total et continuel à ses propres volontés qui rend la servitude si fâcheuse ? Des hommes qui se dévouent ainsi pour leurs maîtres acquièrent sur leur tendresse des droits si justes qu’il n’y a que la dureté et l’orgueil le plus détestable qui puissent les méconnaître. C’est évidemment l’injustice et la fierté de tant de maîtres inhumains qui sont cause que leurs valets sont communément leurs plus grands ennemis. On dirait, à voir leur conduite, qu’ils regardent leurs domestiques comme des bêtes, ou plutôt comme des automates dépourvus de toute sensibilité contre lesquels ils peuvent librement exercer leurs passions, leurs caprices, leur humeur la plus bizarre. Après cela, l’on reproche à des malheureux perpétuellement aigris et rebutés d’être très indifférents pour leurs maîtres, de les servir machinalement et surtout de n’être animés que par l’intérêt. Ainsi, l’on travaille continuellement à repousser les cœurs de ses domestiques, on les dégrade par une hauteur insultante, on les récompense très mal, et l’on se plaint ensuite de les trouver peu attachés, vils, intéressés ! Que les maîtres apprennent donc, et qu’ils n’oublient jamais, que la bonté seule attire les cœurs, que c’est en traitant leurs serviteurs avec les égards dus à des hommes qu’on peut leur inspirer des sentiments d’honneur, que c’est en les récompensant convenablement qu’on leur apprendra à penser avec plus de noblesse. Enfin, tout leur prouvera que les bons maîtres sont seuls en état de former des domestiques fidèles, et que ceux-ci, malgré leur servitude, sont dignes d’être estimés.\par
Si la servitude était un titre pour mépriser les hommes, de quel œil devrait-on regarder la servitude volontaire de tant de courtisans, d’autant plus humiliante que ceux qui s’y soumettent n’y sont pas forcés par la nécessité de subsister et devraient par état avoir le cœur trop haut pour s’abaisser ? Cependant, poussés souvent par le plus vil intérêt, vous les voyez ramper servilement aux pieds du crédit et du pouvoir, s’empresser de rendre à la puissance les services les plus bas, souffrir sans se rebuter des outrages et des avanies que souvent un valet ne pourrait pas supporter.\par
Plaignons donc les hommes quand ils sont malheureux mais ne méprisons que ceux qui par leur conduite avilissante se rendent méprisables.\par
Chapitre VII. De la Conduite dans le Monde De la Politesse ; de la Décence ; de l’Esprit ; de la Gaieté ; du Goût
\subsection[{Chapitre VII. De la Conduite dans le Monde. De la Politesse ; de la Décence , de l’Esprit ; de la Gaieté ; du Goût}]{Chapitre VII. De la Conduite dans le Monde. De la Politesse ; de la Décence , de l’Esprit ; de la Gaieté ; du Goût}
\noindent Après avoir considéré les devoirs que chaque état impose aux hommes dans les différentes positions où ils peuvent se trouver, il nous reste encore à examiner ce qu’ils se doivent les uns aux autres dans la vie commune du monde, c’est-à-dire la conduite qu’ils sont obligés de suivre pour rendre le commerce de la vie agréable et tranquille, les qualités qu’ils doivent acquérir ou posséder pour mériter et conserver l’estime et l’affection des êtres avec lesquels ils peuvent avoir des rapports permanents ou passagers.\par
Le commerce de la vie nous apprend plus ou moins promptement les moyens que nous devons employer pour mériter la bienveillance des personnes avec qui nous vivons habituellement ou que le mouvement de la société nous présente. En réfléchissant sur ce que nous exigeons des autres pour en être contents, nous découvrons bientôt ce que nous devons faire pour qu’ils soient contents de nous. Voilà l’origine naturelle de la {\itshape politesse} qui, comme on l’a déjà fait entrevoir ci-dessus, est l’habitude de montrer aux personnes avec qui nous vivons les sentiments et les égards que nous leur devons.\par
L’homme ne naît pas poli, il le devient par l’éducation, par les préceptes, par l’exemple, par sa propre expérience, par ses réflexions sur les caractères des hommes ; en un mot, par l’usage du monde.\par
Tout lui prouve que pour être heureux il faut plaire ; il s’aperçoit bientôt que pour y parvenir il faut se conformer aux idées, aux conventions de ceux avec qui l’on vit, ménager leur amour-propre ou leur vanité toujours active, leur montrer de l’estime, ou du moins des égards. Tout homme s’aimant et s’estimant lui-même veut voir ces sentiments adoptés par les autres ; c’est sur ces prétentions bien ou mal fondées qu’ils jugent des êtres avec lesquels ils ont des rapports.\par
La politesse a été très bien définie par un moraliste moderne : « L’expression ou l’imitation des vertus sociales. C’en est, dit-il, l’expression si elle est vraie et l’imitation si elle est fausse. Les vertus sociales sont celles qui nous rendent utiles et agréables à ceux avec qui nous avons à vivre ; un homme qui les posséderait toutes aurait nécessairement la politesse au souverain degré\footnote{Voyez les {\itshape Considérations sur les Mœurs}, par M. Duclos.}. »\par
Quelques penseurs chagrins confondent la politesse vraie avec la fausse, ou bien, la faisant uniquement consister dans des formalités incommodes et minutieuses, dans des signes d’attachement ou d’estime équivoques et peu sincères, dans des expressions hyperboliques introduites par l’usage, ils l’ont proscrite injustement et lui ont préféré une rudesse grossière et sauvage qu’ils ont qualifiée de {\itshape franchise}. Mais dans la vie sociale la politesse est une qualité nécessaire puisqu’elle sert à rappeler aux hommes les sentiments qu’ils se doivent, les ménagements avec lesquels, pour leurs intérêts mutuels, sont obligés de se traiter des êtres qui ont un besoin continuel de converser ensemble.\par
Gardons-nous donc de blâmer imprudemment des usages, des formules, des conventions, des signes toujours utiles dès qu’ils retracent à notre mémoire ce que nous devons à nos semblables et ce qui peut nous concilier leur bienveillance. Conformons-nous à ces coutumes lorsqu’elles ne choquent point la probité, soumettons-nous à des pratiques que l’on ne peut violer sans indécence et dont l’omission nous ferait accuser de vanité, de rusticité, de singularité et nous rendrait déplaisants ou ridicules.\par
Le mépris des règles de la politesse et des usages du monde annonce en effet un sot orgueil toujours fait pour blesser. Le refus de se soumettre à des coutumes adoptées par la société est une révolte impertinente et digne d’être blâmée. Chaque homme est en droit de penser comme il voudra mais il ne peut, sans manquer à ses associés, s’exempter des règles imposées par tous et se soustraire à l’autorité publique quand elle ne prescrit rien de contraire aux bonnes mœurs. Respectons le public, suivons ses usages, craignons de lui déplaire en négligeant les signes extérieurs auxquels on est convenu d’attacher les idées de bienveillance, d’attachement, d’estime, de respect ou, si l’on veut, d’indulgence et d’humanité, que nous devons même aux faiblesses de nos semblables.\par
Si nous devons des égards à tous les êtres de notre espèce, la politesse n’est qu’un acte de justice et d’humanité. L’inconnu, l’étranger est en droit d’attendre de nous des marques de la bienveillance universelle qui est due à tous les hommes, puisque si le hasard nous transportait à notre tour dans un pays inconnu, nous souhaiterions de trouver dans ses habitants des signes de bienveillance, d’hospitalité, d’humanité. Cependant, bien des personnes qui passent même pour polies et bien élevées semblent oublier ou méconnaître ces principes : les inconnus leur paraissent n’avoir aucun droit à leurs égards.\par
Dans les spectacles, les promenades, les fêtes, les lieux publics, on voit bien des gens se conduire avec une rudesse, une impolitesse, une grossièreté très déplaisantes et dont on a souvent lieu de se repentir par les querelles et les conséquences quelquefois très funestes qu’elles entraînent. On ne doit ni négliger ni mépriser les signes dont les hommes sont convenus pour marquer la bienveillance et les attentions qui sont dues à tout le monde. Si ces sortes de signes ne sont pas toujours sincères, ils prouvent au moins qu’il existe dans toutes les nations civilisées des idées de ce que les êtres de la même espèce se doivent les uns aux autres même lorsqu’ils ne sont pas intimement liés.\par
La politesse franche et vraie est celle qui part des sentiments d’attachement, de considération, de respect qu’excitent en nous les qualités éminentes que nous trouvons dans les personnes à qui nous les marquons. Nous ne pouvons, il est vrai, éprouver ces sentiments pour tout le monde mais nous devons à tout le monde de la bienveillance, de la bonté, de l’humanité. Nous sommes quelquefois forcés de montrer du respect même à la méchanceté puissante parce que notre conservation veut que nous évitions de blesser ceux qui pourraient nous nuire ; alors les égards que nous leur montrons sont des effets de la crainte, et celle-ci exclut l’amour.\par
L’{\itshape estime} est un sentiment favorable fondé sur des qualités que nous jugeons utiles et louables, et d’après lesquelles nous attachons du prix à ceux qui les possèdent ; c’est une disposition à les aimer, à nous lier avec eux. Le mépris est un sentiment d’aversion fondé sur des qualités inutiles et peu louables. Le mépris est insupportable à ceux qui s’en trouvent les objets parce qu’il semble en quelque sorte les exclure de la société comme inutiles. On peut être estimé sans être aimé mais on ne peut être aimé solidement sans être estimé. Les attachements les plus durables sont ceux dont l’estime est la base.\par
La {\itshape considération} est un sentiment d’estime mêlé de respect, excité par des qualités peu communes, par des actions grandes et nobles, par des talents rares et sublimes. Considérer quelqu’un, c’est lui témoigner une attention marquée en faveur des qualités qui le distinguent des autres. D’où l’on voit que la considération n’est due qu’à la grandeur d’âme, aux grands talents, à la vertu.\par
Il y a, nous dit-on, de la fausseté à marquer de la politesse, de l’estime, de la considération à des hommes à qui ces sentiments ne sont point dus.\par
Mais nous devons des ménagements et des égards à tous ceux que la société s’accorde à respecter : nous ne sommes point leurs juges. Il serait imprudent de montrer du mépris à la méchanceté quand elle a le pouvoir de nuire ; il faut éviter autant qu’on peut les méchants mais quand le hasard ou la nécessité nous les présente, il ne faut point les provoquer par sa conduite : il faut les craindre, et lorsque nous plions devant eux, notre conduite n’est que l’expression de notre crainte. Il n’y a que l’homme de bien qui ait droit de prétendre aux hommages du cœur, à l’affection sincère, à l’estime et à la considération véritable ; les méchants en pouvoir doivent se contenter d’en recevoir les signes extérieurs.\par
Le mépris est insupportable aux hommes les plus méprisables. Plus les méchants ont la conscience du mépris qu’ils méritent, plus ils sont irrités de celui qu’on leur montre.\par
Les signes du respect sont dus à la puissance ; les égards que la crainte, ou les conventions de la société, ou notre devoir, nous obligent d’avoir pour nos supérieurs ou pour les personnes qui exercent sur nous une autorité bien ou mal fondée, se nomment {\itshape respect}. Un fils doit respecter son père, même lorsqu’il est injuste. Un citoyen respecte les princes, les grands, les gens en place, lors même qu’ils sont méchants, parce qu’il s’exposerait par une sotte vanité aux effets de leurs ressentiments. Le respect, étant mêlé de crainte, coûte toujours beaucoup à l’amour-propre des hommes, communément blessés ou gênés par la supériorité des autres. Si les signes du respect sont flatteurs pour celui qui les reçoit parce qu’ils lui rappellent sa puissance et sa grandeur, ils déplaisent à celui qui les donne parce qu’ils lui rappellent sa faiblesse et son infériorité. Voilà pourquoi rien de plus rare que des inférieurs sincèrement attachés à leurs supérieurs : ceux-ci font communément sentir à leurs protégés toute la distance que mettent entre eux le rang et la puissance.\par
Les égards que nous montrons à nos égaux se nomment {\itshape politesse, bons procédés}, lors même que nous n’éprouvons pas pour eux les sentiments d’un attachement véritable : c’est une monnaie courante que chacun donne et reçoit pour ce qu’elle vaut. La vie sociale demande que l’on ait de bons procédés pour les indifférents ; et d’ailleurs, nous en exigeons même des personnes avec lesquelles nous sommes peu liés, d’où l’on voit que cette conduite est fondée en justice.\par
Les signes de considération sont dus au mérite, aux talents rares et utiles, aux vertus. Les signes de la tendresse sont dus à l’amitié. Les égards que nous avons pour nos inférieurs s’appellent {\itshape bonté, affabilité}. Nous devons leur en donner des marques parce que c’est le moyen de nous concilier leur affection, qui jamais ne peut être indifférente à l’homme de bien : il rougirait de ne devoir qu’à la crainte les respects et les hommages qu’il veut obtenir du cœur. Les signes de la bienveillance universelle sont dus à tous les hommes parce qu’ils sont nos semblables. Enfin, pour un cœur sensible il n’est rien de plus digne de ménagement et de respect que la misère ; c’est une sorte de consolation que nous devons aux malheureux.\par
En saluant un inférieur, un homme du peuple, un malheureux, les riches ou les grands lui annoncent qu’ils ont de l’humanité, qu’ils ne le dédaignent pas, qu’ils le comptent pour quelque chose, qu’ils lui veulent du bien. Rien ne serait plus conforme à la saine morale que d’apprendre aux enfants nés dans l’opulence à ne jamais montrer du mépris à leurs inférieurs. Ils se rendraient par là plus dignes de leur amour, ils affaibliraient la haine ou l’envie que l’indigence doit naturellement concevoir contre les heureux, sentiments que l’orgueil ne peut qu’envenimer. N’est-ce donc pas assez que des hommes soient misérables sans encore le leur faire sentir à tout moment ?\par
L’éducation devrait encore garantir les grands de cette politesse hautaine et dédaigneuse qui, bien loin d’inspirer de l’amour et de la confiance à ceux qui l’essuient, semble les écarter, les repousser, leur annoncer la distance à laquelle l’orgueil veut les tenir ; la politesse de ce genre est souvent plus révoltante qu’une insulte avérée. « Les grands, dit un moderne, qui écartent les hommes à force de politesse sans bonté, ne sont bons qu’à être écartés eux-mêmes à force de respect sans attachement… La politesse des grands doit être l’humanité ; celle des inférieurs, de la reconnaissance, si les grands le méritent ; celle des égaux, de l’estime et des services mutuels\phantomsection
\label{footnote111}\footnote{Voyez les {\itshape Considérations sur les Mœurs}.}. »\par
Les habitants de la cour sont d’ordinaire les plus polis des hommes parce qu’ils sont accoutumés à craindre de blesser l’amour-propre de tous ceux qui peuvent les servir ou les desservir dans leurs projets divers : ils savent que quelquefois l’homme le plus abject peut mettre des obstacles à leurs désirs. D’un autre côté, les grands sont communément très polis afin d’être eux-mêmes plus respectés ou pour avertir leurs inférieurs de la soumission qu’ils en attendent.\par
Le désir d’obliger doit être mis au rang des qualités les plus propres à nous concilier l’affection dans la vie sociale. Cette disposition est visiblement émanée de la bienveillance et des secours que nous devons aux êtres de notre espèce. Rendre service à quelqu’un, c’est exercer envers lui la bienfaisance. Ainsi, l’homme {\itshape obligeant} acquiert des droits sur l’affection des autres et sur sa propre estime. Celui qui se sert de son crédit pour faire sortir de l’oubli le mérite ignoré, pour réparer les injustices du sort, pour fournir des secours à la vertu, est un vrai bienfaiteur digne de la reconnaissance de tout bon citoyen. Sans produire toujours des effets si marqués, le désir d’obliger est toujours agréable dans le commerce de la vie ; il part de la complaisance et de la politesse, qui nous portent à nous prêter gaiement aux vœux de ceux à qui nous voulons plaire. Ainsi que la bienfaisance, l’humeur obligeante ne doit jamais s’exercer aux dépens de la vertu. C’est nuire à la société et souvent à soi-même que d’obliger les méchants. C’est faire du mal aux vicieux que de les servir dans leurs dérèglements. C’est se rendre coupable que de prêter ses secours à l’iniquité. On est un lâche, un flatteur quand on a la faiblesse de servir ou d’obliger des gens inutiles ou nuisibles. Une politesse excessive, une complaisance banale, un désir aveugle d’obliger produisent souvent autant de maux dans la vie de ce monde que l’impolitesse et la brutalité.\par
Dans quelque familiarité que les hommes vivent entre eux, la politesse ne devrait jamais être totalement bannie : l’amour-propre est si prompt à s’alarmer, la vanité est si facile à irriter, que l’on devrait toujours craindre de les réveiller. Nos amis nous dispensent volontiers des formalités incommodes et banales de la politesse et de l’étiquette, mais nos amis ne peuvent jamais consentir à se voir méprisés. Rien de plus cruel que le mépris de la part de ceux que l’on aime et dont on voudrait être aimé. Ainsi, l’amitié, en bannissant les {\itshape compliments} ou les signes extérieurs de la politesse, ne peut cesser d’exiger les sentiments réels dont ces marques sont les annonces. Les railleries piquantes, les discours peu mesurés que la familiarité semble souvent autoriser, sont les causes les plus communes des ruptures et des brouilleries qu’on voit dans la société.\par
L’amour-propre, qui toujours flatte, et l’étourderie, qui ne voit guère les choses telles qu’elles sont, font que bien des gens présument trop de l’amitié des personnes qu’ils fréquentent et ne savent pas mesurer jusqu’où l’on peut aller avec elles. On suppose assez souvent que l’on peut tout se permettre avec ceux que l’on croit ses {\itshape intimes amis}, tandis que très souvent ces prétendus {\itshape amis intimes} n’ont pour nous que les sentiments très faibles d’une bienveillance générale que l’on ne doit pas confondre avec la véritable amitié. Le monde est rempli de maladroits présomptueux qui se rendent désagréables à ceux dont ils n’ont pas suffisamment approfondi les dispositions. « Je ne savais pas être si fort de vos amis », disait un homme à un indiscret qui présumait trop de son attachement ; {\itshape faites un peu de façon}, disait un autre à quelqu’un qui en usait avec lui d’une façon trop familière. Un peu de réflexion ne devrait-il pas nous montrer qu’il est des positions où l’ami le plus cher peut devenir incommode à son ami ?\par
L’union conjugale même, pour être maintenue dans toute sa force, ne dispense pas les époux de ces attentions qui annoncent l’estime et le désir de plaire. En public, des époux raisonnables respecteront leur amour-propre ou ne négligeront pas les égards mutuels faits pour annoncer qu’ils ont les sentiments convenables à des êtres qui s’aiment. Il est des gens assez mal avisés pour refuser tout signe de bienveillance et d’attachement aux personnes dont ils ont le plus d’intérêt d’entretenir l’affection. La société est remplie d’époux qui ne se distinguent que par leurs mauvaises manières, de pères qui traitent leurs enfants sans aucuns ménagements, d’amis qui croient que tout leur est permis avec leurs amis, enfin de maîtres qui ne peuvent parler avec bonté ou de sang-froid à leurs domestiques. C’est ainsi que les hommes qui vivent le plus familièrement finissent très souvent par se détester.\par
Les égards et les bonnes manières ne sont jamais ni déplacés ni perdus. Les différentes façons de les exprimer, par sa conduite et ses discours, servent à nourrir dans les cœurs des hommes les dispositions nécessaires à leur contentement réciproque. Jamais nous ne sommes contents de ceux qui nous montrent qu’ils n’ont pas pour nous les sentiments que nous en exigeons.\par
Nous devons certains égards même aux personnes qui nous sont totalement inconnues. Un être vraiment sociable doit s’abstenir d’offenser ceux même qu’un pur hasard vient offrir à sa vue. Cet inconnu peut être un homme d’un mérite rare ou d’un rang distingué ; l’on peut se repentir de ne lui avoir pas montré les sentiments qu’il a droit d’exiger. Il n’est personne qui ne rougisse d’avoir traité d’une façon trop légère ou peu respectueuse un inconnu, lorsqu’on vient à découvrir par la suite que ce même inconnu est un personnage considérable. D’ailleurs, l’homme de bien, toujours animé du sentiment de la bienveillance universelle, désire de la témoigner même à ceux qu’il ne voit qu’en passant.\par
Ainsi, les égards dus à la société nous prescrivent des ménagements et de la politesse pour les personnes mêmes avec lesquelles nous n’avons point eu ou nous n’aurons jamais de liaison particulière. Rien de plus impoli ni de plus impertinent que ces regards curieux, effrontés, embarrassants, que des hommes qui se croient bien élevés jettent souvent sur des femmes dans les promenades ou dans les lieux où se rend le public. Une bonne éducation, ainsi que la bienséance, devrait sans doute nous apprendre que nos regards sont faits pour ménager la délicatesse et la pudeur d’un sexe que le nôtre doit respecter, ou du moins ne point obliger de rougir.\par
En général, l’homme bien né contractera l’habitude de ne blesser personne. Faute de faire attention à cette règle si simple, à combien d’inconvénients fâcheux une foule d’imprudents ne se trouve-t-elle pas à tout moment exposée ? En voyant la façon dont bien des gens se comportent en public avec ceux que le sort leur présente, on croirait que tout inconnu est pour eux un ennemi avec lequel ils veulent entrer en guerre. De là naissent mille rencontres imprévues dont les suites sont souvent très sérieuses entre des personnes peu disposées à souffrir, soit les regards insultants, soit les manières peu mesurées de ceux qui se trouvent sur leur chemin. Eh ! Quoi ! Tous les hommes, tous les habitants d’une même ville ne devraient-ils pas se donner des signes de bienveillance ? A-t-on à rougir des égards que l’on montre à ses concitoyens ?\par
Le moyen le plus sûr de bien vivre avec les hommes est de leur témoigner, autant qu’il est possible, que nous avons pour eux les sentiments et l’opinion qu’ils veulent trouver en nous. Nous ne sommes point blâmables de leur sacrifier souvent une portion de notre amour-propre ; il vaut mieux, en général, pécher par le trop que par le trop peu dans les égards que nous leur témoignons. Mais la vanité de l’homme est si chétive et si pauvre qu’elle craint de se priver elle-même de tout ce qu’elle accorde aux autres. Sous prétexte d’éviter la bassesse et la flatterie, on se refuse souvent à des condescendances innocentes pour les faiblesses humaines, auxquelles une grandeur d’âme véritable se prêterait sans répugnance. On n’est point bas pour montrer de l’indulgence ; elle est au contraire une marque de grandeur quand il ne résulte aucun mal de sa facilité. Il y a de la raison à céder à la force\phantomsection
\label{footnote112}\footnote{Les Lacédémoniens, qui n’étaient pas des hommes bas, nous ont donné un bel exemple de l’indulgence qu’on peut avoir pour la folie des grands. Alexandre le Grand ayant eu la petitesse de se faire passer pour le fils de Jupiter et pour un dieu, voulut être reconnu tel par tous les États de la Grèce. Sur quoi les Lacédémoniens rendirent ce décret vraiment laconique : {\itshape puisque Alexandre veut être Dieu, qu’il soit Dieu}.} ; il y a de la générosité à faire plier son amour-propre sous celui d’un homme de mérite d’ailleurs, sous celui d’un ami qui peut avoir de légers défauts compensés par un grand nombre de qualités louables. Si dans le commerce de la vie on s’obstinait à ne mettre jamais les hommes qu’à leur vraie place, on se verrait bientôt brouillé avec tout le monde.\par
Bien des gens se font un point d’honneur de mettre dans le commerce de la vie une raideur qui les rend désagréables sans les faire estimer. Ils disent qu’ils sont francs, qu’ils ne sont point flatteurs, tandis que dans le fond ils ne sont que vains, grossiers, remplis de petitesse, de malice et d’envie. « La vertu, dit Horace, tient le milieu entre ces deux vices opposés, et en est également éloignée. » En effet, une âme vraiment noble et généreuse ne craint pas de s’avilir par sa facilité ; elle ne rougit même pas de rendre aux autres plus qu’ils n’ont droit d’exiger.\par
Il n’y a qu’une vanité inquiète sur ses propres prétentions, souvent suspectes pour elle-même, qui fasse tenir sans cesse la balance pour peser à toute rigueur ce qu’elle veut accorder ou refuser. Tout sacrifice de l’amour-propre coûte infiniment aux petits esprits : ils n’attachent de l’importance qu’à des bagatelles ; par la crainte d’être trop polis, ils se rendent impertinents.\par
De là ce conflit perpétuel des vanités que nous voyons à tout moment en guerre dans la société. Des hommes vains craignent toujours d’en trop faire et de se dégrader par l’indulgence qu’ils montreraient aux autres. Les grands affectent du mépris pour le savant ou l’homme de lettres, dont ils veulent bien s’amuser sans jamais consentir que leurs talents divers les mettent trop à leur niveau. L’homme de qualité prétend que l’homme de mérite sans naissance {\itshape se tienne toujours à sa place}.\par
Le commerce qui s’établit assez souvent entre la noblesse indigente et la bourgeoisie opulente n’est ordinairement qu’un combat de deux vanités également ridicules. Le financier, ainsi que l’homme de lettres, ont quelquefois la vanité de fréquenter les grands qui les méprisent. Ils pensent s’illustrer par une liaison qui les dégrade, et ces grands dont ils ont la folie de se croire les amis ne les regardent que comme des protégés, des inférieurs qu’ils daignent honorer par leur condescendance. « Les grands, disait Diogène, sont comme le feu, dont il ne faut ni trop s’éloigner, ni s’approcher de trop près. »\par
Rien de plus sensé ni de plus avantageux dans la vie que de rester dans sa sphère. Un Arabe a dit très sagement : {\itshape ne va point au marché pour n’y vendre qu’à perte}. Le commerce des grands ne peut être que désavantageux aux petits.\par
Tous les talents de l’esprit et du cœur ne sont rien aux yeux d’un homme de qualité qui ne connaît rien de comparable à la naissance ; la vertu paraît très inutile au courtisan qui ne fait cas que de ce qui mène à la fortune. Le mérite perd tout son prix auprès de ceux qui n’en ont pas.\par
L’homme de génie n’est qu’un sot auprès d’un sot titré ; l’homme à talents doit être bas s’il veut plaire à la grandeur. La fréquentation des grands ôte communément à l’esprit cette noble fierté, ce courage, cette liberté qui le rendraient capable de faire des choses utiles et grandes\phantomsection
\label{footnote113}\footnote{La vanité a communément plus de part que le goût ou que l’amour des sciences aux faveurs que les princes montrent aux savants et aux gens de lettres. Les {\itshape Mémoires de Brandebourg} nous parlent d’un souverain fastueux qui eut une Académie qu’il jugea nécessaire à sa gloire comme d’avoir une {\itshape ménagerie}. Denys le Jeune, tyran de Syracuse, s’expliquait assez franchement à cet égard. Il disait qu’il entretenait à sa cour des philosophes et des gens de lettres, non qu’il les estimât, mais parce qu’il voulait être estimé à cause de la faveur qu’il leur montrait. Voyez Plutarque, {\itshape Dits notables}. Plusieurs tyrans et despotes ont favorisé les lettres dans les mêmes vues que Denys ; par là ils se sont assurés des panégyristes et quelquefois des apologistes de leurs actions les plus blâmables. Des princes ont honoré et distingué des astronomes, des géomètres, des antiquaires, et surtout des poètes. Mais on n’en voit guère qui aient aimé des philosophes véridiques et sincères. Les bienfaits des despotes furent même souvent un obstacle aux vrais progrès de l’esprit humain.}.\par
L’homme dont la fortune est médiocre ne gagne dans la fréquentation de l’opulence que le désir de s’enrichir, le goût du luxe, l’amour de la dépense, la tentation de se ruiner pour ne le point céder à celui dont le faste l’éblouit. L’homme sage ne devrait point sortir de son état : c’est le moyen d’éviter les dégoûts que produiraient en lui les hauteurs, les prétentions, la vanité des autres. La manie des grands est une source de ruine pour les indigents ou les personnes dont la fortune est bornée. Il serait plus prudent de rester plutôt en deçà que de vouloir aller au delà de ses facultés.\par
En général, il ne peut y avoir d’agréments réciproques et durables dans les {\itshape mésalliances} de société ou dans les liaisons entre des personnes qui diffèrent trop, soit par le rang, l’état, la fortune, soit par les talents, l’esprit et le caractère. Ceux qui sentent leur supériorité, en quelque genre que ce soit, ne tardent pas communément à s’en prévaloir contre leurs inférieurs ; de là naissent des discordes et des haines, fruits nécessaires des hauteurs, des mépris, des railleries que l’on fait communément éprouver à ceux qu’on voit au-dessous de soi. Les petits n’ont à gagner que des mépris avec les grands ; les personnes d’un esprit médiocre sont bientôt dédaignées par ceux qui ont quelque avantage de ce côté.\par
On trouve des gens qui, par une sotte ambition, veulent {\itshape primer} dans les sociétés qu’ils fréquentent. Pour y réussir, vous les verrez quelquefois préférer le commerce de leurs inférieurs à celui de leurs égaux, qui ne leur laisseraient pas prendre les mêmes avantages. C’est ainsi que des gens d’esprit ont quelquefois la faiblesse de fuir leurs pareils et de se plaire avec des sots qu’ils peuvent impunément dominer ; pouvoir peu glorieux, sans doute, que celui qu’on exerce sur des hommes faibles et méprisables ! Il n’y a qu’une vanité bien puérile qui puisse être flattée des hommages de ceux même qu’elle méprise.\par
Quels qu’en soient les motifs, il y a de la bassesse, de la lâcheté, de la sottise à fréquenter ceux qu’on ne peut ni aimer ni estimer. Rien n’est plus vil que la conduite de ces grands qui vont {\itshape piquer} la table d’un parvenu pour avoir l’occasion de rire à ses dépens. L’homme dont le cœur est bien placé s’abstient de voir familièrement des personnes dépourvues de qualités aimables. Il n’ira point chez l’homme vain, parce qu’il aurait à souffrir de sa vanité ; personne n’est en effet plus sujet à s’oublier qu’un sot qui s’est enrichi. Rien de plus insolent que lui lorsqu’il se voit entouré de ses flatteurs et parasites. L’homme de bien ne fréquentera point le prodigue, parce qu’il rougirait de contribuer à sa ruine ou de tirer parti de sa folie. Enfin, il ne fréquentera point des personnes décriées ou dignes de mépris, parce qu’il se respecte lui-même et craint de se déshonorer aux yeux des autres.\par
Le monde est plein de gens que l’on ne peut fréquenter sans apologie ou sans se croire obligé d’expliquer les motifs des liaisons qu’on forme avec eux.\par
Il ne faut, autant qu’on peut, se lier qu’avec des personnes estimables dont on n’ait point à rougir ; et pour lors, il n’y aura ni apologie à faire, ni explications à donner. Le hasard, nos circonstances, nos besoins peuvent nous forcer de rencontrer quelquefois des personnes peu dignes de notre attachement vrai, de notre estime sincère, mais il y a de la bassesse et de la fausseté à vivre dans l’intimité avec des gens pour qui l’on ne peut éprouver aucun sentiment favorable. Le bas flatteur est le seul qui puisse se soumettre à une pareille contrainte, l’homme vil peut seul consentir à vivre longtemps sous le masque.\par
Quelque parti que l’on suive, celui qui veut vivre dans le monde doit se prêter, autant qu’il peut, à l’amour-propre bien ou mal fondé de ceux qu’il fréquente ; s’il n’en a pas le courage, qu’il s’abstienne d’un commerce qui ne lui convient pas. Le misanthrope est toujours un orgueilleux ou bien un envieux dont la vanité et l’envie sont irritées de tout. Vivre avec des hommes, c’est vivre avec des êtres remplis d’amour-propre et de préjugés auxquels il faut souscrire, ou se condamner à vivre en solitaire. Notre amour-propre doit nous apprendre que nous devons fermer les yeux sur celui des autres {\itshape ;} l’homme prudent et sociable est toujours occupé à réprimer le sien. Il y a de la force, de la grandeur, de la noblesse à vaincre ses propres faiblesses et à supporter celles des autres. Le grand art de vivre est d’exiger fort peu et d’accorder beaucoup. Pour être content de tout le monde, il faut rendre les personnes avec qui nous vivons contentes et d’elles-mêmes et de nous ; cet objet mérite assurément qu’on lui sacrifie quelque chose.\par
Pour le bien de la paix, il est bon de consentir quelquefois à être dupe et de ne point tirer parti de sa propre supériorité. Les hommes sont perpétuellement en guerre non parce qu’ils ont de la grandeur d’âme, mais parce qu’ils n’ont pas le courage de céder. Les corps, comme les individus, se haïssent ou se méprisent parce qu’ils n’ont pas les mêmes passions, les mêmes goûts, les mêmes façons de voir, les mêmes préjugés. Un courtisan ambitieux, un prince, un conquérant regardent avec mépris les spéculations d’un philosophe qui contrarient leurs goûts et leurs préjugés. De son côté, le sage regarde leurs folies en pitié et trouve qu’un esprit élevé ne voit rien de grand sur la terre que la vertu : les cèdres ne paraissent que des herbes à l’aigle qui plane au haut des airs.\par
Mais pour vivre avec les hommes il faut se prêter à leurs opinions, sous peine d’en être détesté. Ivre de son amour-propre et de ses propres idées, chacun oublie l’amour-propre des autres et refuse de se conformer à l’opinion qu’ils ont d’eux-mêmes ; telle est la source d’où l’on voit perpétuellement découler tous les désagréments de la vie. Le monde est une assemblée dans laquelle chacun se montre à son avantage ; pour bien jouer son rôle, il est utile de laisser chacun jouer le sien.\par
Le rôle de l’homme de bien est d’être patient, indulgent, généreux, et de contenir au fond de son cœur les mouvements de colère qui, sans corriger personne, ne feraient que le rendre malheureux. L’humeur noire ne ferait que porter le trouble au dedans de nous-mêmes et nous faire haïr de ceux avec qui nous sommes destinés à vivre en paix. Il n’y a point, dans les folies des hommes, de quoi se brouiller sans retour avec l’espèce humaine. Le sage en rit intérieurement mais il se prête quelquefois aux jeux enfantins de ces êtres en qui la raison ne s’est pas encore montrée : il sait qu’une censure amère ne peut rien contre le torrent de la mode et des préjugés.\par
Soumis aux conventions honnêtes de la société, dont nous ne sommes ni les arbitres ni les réformateurs, en attendant que l’esprit humain se développe et se dégage des bandelettes du préjugé, laissons à chacun le rang que l’opinion lui décerne. Pleins d’égards pour nos semblables, ne les affligeons point par une conduite arrogante qui rendrait inutiles les leçons de la sagesse.\par
Que le philosophe sincère dans ses écrits présente la vérité sans nuages, parce qu’elle est utile à la société ; mais s’il vit dans le monde, qu’il épargne la faiblesse des individus. Indulgent pour ses concitoyens, qu’il n’entre point en guerre avec leurs prétentions ; poli avec ses égaux, respectueux pour ses supérieurs, affable pour ceux qu’il voit au-dessous de lui, qu’il ne s’arroge pas le droit de choquer les personnes que le hasard lui fait rencontrer.\par
Qu’il fréquente le monde et n’attache aucun mérite à le fuir, qu’il ne vive dans l’intimité qu’avec des personnes choisies dont les dispositions, les idées et les mœurs sont à l’unisson des siennes : c’est là qu’il peut ouvrir son cœur et se plaindre des travers et des tristes folies dont sa patrie est souvent la victime.\par
Il déplore avec eux les opinions insensées auxquelles tant de gens attachent follement leur bien-être mais il sait que le cynisme, la misanthropie, l’humeur, la singularité, ne sont aucunement propres à détromper les hommes. « Ne frappez pas, dit Pythagore\footnote{C’est le onzième des symboles de ce philosophe dans la traduction de Dacier, page 183, tome I, édition de Paris, 1706. On le trouve aussi dans le traité de Plutarque {\itshape De la Pluralité des Amis}, au tome I de ses {\itshape Œuvres morales} de la version d’Amyot, page 263, verso, édition de Vascosan, in-8°.}, indifféremment dans la main de tout le monde. » Ce précepte si sage paraît totalement ignoré dans les assemblages bigarrés que l’on rencontre partout. Quoique l’homme sociable ne se croie pas en droit de jouer dans la société le rôle d’improbateur, il évitera néanmoins le commerce des méchants, parmi lesquels il serait totalement déplacé.\par
Un des inconvénients les plus fâcheux des villes opulentes et peuplées vient du mélange des compagnies ; l’on y trouve à tout moment les personnes les plus estimables indignement confondues avec les hommes les plus décriés et les plus méprisables. Que dis-je ! Ceux-ci sont quelquefois non seulement tolérés mais encore recherchés pour des qualités amusantes ou des talents aimables que trop souvent on préfère aux qualités du cœur. Au défaut d’une censure publique qui devrait flétrir tous les pervers, les honnêtes gens feraient très bien de se liguer pour exclure de leurs cercles ces hommes notés qui, parce que les lois ont oublié de les punir, se présentent effrontément dans la bonne compagnie.\par
Rien de plus étrange et même de plus dangereux que la facilité avec laquelle des personnages méprisables, des joueurs, des aventuriers, des fripons, des escrocs, trouvent souvent le moyen de pénétrer dans ce qu’on appelle la bonne compagnie ; elle se trouve fréquemment forcée de rougir des membres dont elle s’est composée. On y voit quelquefois admettre des hommes les plus décriés. Les gens du monde, peu difficiles dans leurs liaisons, perpétuellement ennuyés, ne cherchant qu’à passer le temps, semblent dire de la plupart de ceux qui les fréquentent : « Ce sont des fripons, de malhonnêtes gens, on le sait {\itshape ;} mais il faut bien s’amuser. »\par
En général, on pardonne très aisément aux méchants le mal qu’ils font aux autres {\itshape ;} dans le tumulte du monde on ne craint point assez les gens sans mœurs et sans vertu. On écoute avec plaisir celui qui dit des méchancetés, des calomnies, des médisances sur le compte des autres, pourvu qu’il ait le soin de les débiter avec esprit et gaieté. C’est ainsi que l’homme du plus mauvais cœur passe quelquefois pour {\itshape charmant.} L’amour-propre des auditeurs leur persuade que le méchant qui les amuse changera pour eux de ton, de caractère et n’osera jamais les traiter eux-mêmes comme il traite les autres. C’est néanmoins ce qui arrive assez souvent, et pour lors l’homme {\itshape charmant} devient un monstre abominable.\par
Chacun connaît le danger des liaisons en théorie et l’oublie dans la pratique. Rien de moins agréable et de moins sûr que les maisons ouvertes, pour ainsi dire, à tous ceux qui s’y présentent. Tant de gens dont la vanité se repaît de l’idée de recevoir beaucoup de monde, devraient naturellement s’attendre à voir souvent chez eux des personnes suspectes et dangereuses.\par
Quand on ne reçoit un homme que sur son nom, son titre, son esprit, son état, ses talents agréables et quelquefois son habit, on risque de se repentir un jour de l’avoir admis chez soi. Ce sont les qualités du cœur et le caractère d’un homme qu’il faudrait s’efforcer de connaître avant de se lier avec lui.\par
Mais on dirait que les gens du monde s’embarrassent fort peu des honnêtes gens, qui souvent les ennuient. Assez semblables aux enfants, ils se soucient fort peu du commerce des personnes sensées, qu’ils ne croient propres qu’à les troubler dans leurs vains amusements.\par
C’est un inconvénient assez commun dans le monde que la légèreté avec laquelle les hommes se présentent les uns les autres dans les sociétés. Les personnes sensées ne veulent pas admettre indifféremment tout le monde, et tout homme qui pense devrait se défendre de présenter, même à ses amis intimes, des personnes qu’il ne connaît que faiblement ou qui n’ont rien de conforme aux goûts, au caractère, aux mœurs de ceux à qui il les présente. On se trompe très fréquemment en ce genre ; chacun s’imagine que l’homme qui lui plaît a les qualités requises pour plaire à tout le monde, tandis que fort souvent les endroits mêmes par lesquels un homme nous plaît le rendent désagréable pour d’autres.\par
Le talent d’assortir les hommes est très rare, comme nous le verrons bientôt ; cependant, il contribue beaucoup à l’agrément de la société et répandrait bien plus de plaisirs sur le commerce de la vie.\par
La vie sociale demande que, sans blesser la justice, tout homme sensé se conforme aux lois de la {\itshape décence}, qui n’est que la conformité de sa conduite avec ce que la société où l’on vit a jugé convenable. Conséquemment, la décence prescrit de ne point heurter de front les coutumes, les manières généralement adoptées, lorsqu’elles n’ont rien de contraire à la vertu, c’est-à-dire à la décence naturelle, toujours faite pour l’emporter sur la décence de convention.\par
La raison condamne donc la conduite impudente et révoltante du cynisme antique qui se faisait un mérite de braver toute décence dans les mœurs. Elle blâme cette philosophie qui ne se plaît qu’à contrarier avec chagrin les usages les plus innocents et qui se fait remarquer par sa singularité. On a loué Pythagore de s’être sagement accommodé à tout le monde ; sa maxime était de {\itshape ne point sortir du grand chemin}. Tout homme qui affecte la singularité annonce une tête occupée de minuties auxquelles elle attache la plus grande importance. Ce tour d’esprit, par sa nouveauté, semble d’abord intéresser mais, revenu de sa surprise, le public punit communément par le mépris l’homme singulier, dans lequel il ne découvre bientôt qu’une sotte vanité.\par
« Il me semble, dit Montaigne, que toutes façons écartées et particulières partent plutôt de folie ou d’affectation ambitieuse que de vraie raison. » Il n’est juste et permis de s’écarter des usages prescrits par les conventions que lorsqu’ils sont évidemment contraires à la droite raison, à l’équité naturelle, et par là même au bien de la société. Caton fit très sagement de sortir d’un spectacle où l’on allait exposer une femme nue aux regards impudiques d’un peuple corrompu.\par
L’on peut et l’on doit être décent au milieu d’une société dont les mœurs sont criminelles et vicieuses. Tout homme honnête doit refuser de prendre part à la dissolution générale parce qu’il sait qu’elle est essentiellement nuisible ; il ne paraît alors singulier ou ridicule qu’à des hommes dont il est fait pour mépriser les jugements.\par
La décence naturelle est fondée sur les convenances nécessaires des êtres vivants en société, sur l’intérêt constant des hommes, sur la vertu. Cette décence nous interdit les actions approuvées du public quand elles sont évidemment opposées aux bonnes mœurs ; ses lois doivent être en tout temps préférées à des coutumes, des opinions, des conventions arbitraires autorisées par la déraison des peuples, qui souvent ont des idées très fausses de la décence. Il y a, dit-on, des nations sauvages où les femmes sont dans l’usage de se prostituer aux étrangers et se croient outragées par ceux qui se refusent à leurs faveurs. L’Anglais qui, se rappelant qu’il avait une femme en son pays, ne voulut pas se conformer à cette coutume impudique, put bien paraître ridicule ou singulier à ces femmes sans pudeur, mais il n’en fut pas moins estimable aux yeux de tous les êtres raisonnables. Les nations les plus corrompues rendent souvent hommage à la décence et montrent de l’indignation quand on cesse de la respecter. Cette sorte d’hypocrisie nous prouve que les hommes les plus vicieux sont forcés de rougir de leurs désordres et ne peuvent consentir à se voir tels qu’ils sont. Une femme déréglée se trouve elle-même à la gêne lorsqu’elle voit en public un spectacle licencieux ou quand on lui fait entendre des discours obscènes\footnote{Dans des nations policées et sans mœurs, il est presque impossible de mettre sur la scène les vices et les désordres qui règnent le plus dans le monde : le public alors crierait à l’indécence, et les personnes les plus coupables ne seraient pas les dernières à se plaindre qu’on leur manque. La stérilité des bons sujets de comédies et l’uniformité des pièces de théâtre viennent de décences {\itshape gazées}, afin de n’avoir pas l’air de pécher grossièrement contre la décence qu’ils prétendent respecter. Un grand nombre de pièces de Molière, applaudies dans le siècle passé, seraient aujourd’hui rejetées avec indignation. Cela prouve-t-il que le public actuel a plus de mœurs et de vertus qu’autrefois ? Non, sans doute. Cela prouve que ce public est moins grossier ou moins franc, et qu’il sait mieux qu’autrefois qu’il est honteux d’approuver des choses contraires à la décence.}. La {\itshape bienséance} est la convenance de notre conduite avec les temps, les lieux, les mœurs, les circonstances, les personnes avec qui nous vivons ; elle consiste à mettre les hommes et les choses en leur place, à rendre à chacun ce que nous lui devons. D’où l’on voit qu’elle est fondée sur l’équité, qui jamais ne peut approuver des choses injustes et déshonnêtes. Manquer aux bienséances, c’est donc manquer à la justice. L’éducation, l’exemple, l’usage du monde nous donnent des idées vraies ou fausses de la bienséance ; c’est à la raison éclairée qu’il appartient d’en juger en dernier ressort.\par
La bienséance nous défend de choquer par nos actions ou nos discours les personnes avec lesquelles nous vivons. Conséquemment, elle nous fait un devoir d’éviter tout ce qui peut exciter dans les autres des idées peu favorables de nous-mêmes ou peindre à leur imagination des objets capables de leur déplaire. Est-il rien de plus contraire à la bienséance que les paroles déshonnêtes et les propos contraires à la pudeur dont souvent les conversations sont remplies ? Quoique l’usage semble autoriser, du moins parmi les hommes, les conversations de ce genre, elles paraîtront toujours très peu séantes à ceux qui ont pour les mœurs le respect qui leur est dû. Si les personnes bien élevées contractent l’habitude de la propreté extérieure, qui est fondée sur la crainte d’offrir aux yeux des objets capables de causer du dégoût, elles doivent avoir pour les oreilles les mêmes ménagements. L’on ne peut donc s’empêcher de blâmer et de proscrire de la conversation ces détails dégoûtants de maladies et d’infirmités que se permettent des personnes que leur éducation semblerait avoir accoutumées à se montrer plus réservées. Nous nous contenterons de leur représenter que les discours ne doivent tracer dans l’esprit des auditeurs que des images sur lesquelles ils puissent s’arrêter avec plaisir.\par
Les {\itshape manières} sont les façons extérieures de se comporter dans le monde introduites par l’usage et les conventions de la société ; elles consistent dans le maintien, dans les mouvements du corps, dans la façon de se présenter, etc. L’éducation et l’exemple nous en font contracter l’habitude ; indifférentes en elles-mêmes, nous sommes obligés de nous y conformer sous peine d’être regardés comme impolis et mal élevés. Il faut dans les manières éviter l’affectation, qui rend toujours les hommes ridicules. Pour se rendre agréable dans le monde, il ne suffit pas de posséder de la science, des talents, des vertus : il faut encore savoir les produire d’une façon qui plaise. L’homme de bien ne doit point dédaigner le titre d’homme aimable. Il y a de la négligence, de la sottise ou de la présomption, et non pas du mérite, à rejeter les moyens propres à concilier la bonne opinion du public. Des façons ridicules, des manières inusitées, un extérieur maussade, un ton brusque et grossier, une franchise déplacée, une ignorance rustique des usages reçus sont faits pour indisposer ou pour exciter la risée. Il y a tout autant d’impertinence que de stupidité à mépriser ou ignorer les manières consacrées par la convention. Les bonnes manières sont le vernis du mérite. La vertu se ferait tort si elle refusait des ornements propres à la rendre plus attrayante. Le sage n’a point à rougir de sacrifier aux grâces. Faute de faire ces réflexions, l’on voit bien des gens de mérite paraître ridicules et déplacés dans le monde. Ce monde, souvent pervers, se croit en droit de mépriser la science et la vertu quand il les trouve destituées des agréments auxquels il attache communément une très haute idée. D’un autre côté, le monde ne peut pour l’ordinaire juger que sur l’extérieur ; ses jugements superficiels ne sont sans doute rien moins qu’infaillibles, cependant ils ne laissent pas d’avoir quelques fondements. L’ignorance des bonnes manières annonce une éducation négligée, une absence de réflexion, une incurie blâmable. Un extérieur délabré semble indiquer un défaut d’ordre dans l’esprit, de même qu’une heureuse physionomie prévient favorablement dès le premier abord. Des manières décentes, faciles, naturelles, engageantes, découvrent des dispositions louables, telles que le désir d’être aimé, la crainte de blesser, l’habitude de traiter avec les hommes, la connaissance des égards qu’on doit à la société, une attention constante à ne point la choquer.\par
Le véritable {\itshape savoir-vivre} n’est que la connaissance et la pratique des manières propres à nous concilier l’estime et l’amitié des personnes avec qui nous vivons. Ces manières sont bonnes dès qu’elles n’ont rien de contraire à la vertu, qu’elles ne servent qu’à rendre plus agréable et plus insinuante. Quoique rien ne soit plus sujet à tromper que les signes extérieurs, il n’en est pas moins sûr qu’un extérieur prévenant, simple, décent, annonce communément un intérieur bien réglé. Les bonnes manières sont l’expression d’une belle âme. La vertu même peut rebuter lorsqu’elle se présente sous une forme agreste et sauvage.\par
Quand nous parlons des manières que la morale prescrit au sage d’adopter, nous ne lui disons pas de se conformer à ces façons impertinentes, à ces modes variables, à ce jargon éphémère, à des vaines grimaces dans lesquelles des fats et des femmes frivoles font souvent consister le {\itshape bon ton}. Les manières de cette espèce sont des effets d’une sotte vanité faite pour déplaire aux personnes sensées, les seules dont l’homme sensé doit rechercher les suffrages.\par
Ainsi, distinguons ce qu’un monde futile appelle de {\itshape belles manières} de ce qu’on peut justement appeler de {\itshape bonnes manières} : celles-ci partent d’une affection sociale, du respect que nous devons à la société. Est-il rien de plus insultant pour elle que les airs insolemment aisés du {\itshape petit maître}, que les étourderies affectées de la coquette, que la négligence étudiée d’un tas d’êtres importants qui, croyant se faire estimer par leurs façons impertinentes, ne font que se rendre odieux ou méprisables ? Si des façons abjectes et grossières sont capables de nuire au mérite, les manières affectées de la fatuité ne lui font pas moins de tort. L’homme de bien ne doit jamais se couvrir des livrées de la folie ; il doit chercher à plaire à des personnes raisonnable et non à une troupe sans cervelle qu’il devrait éviter. Une lâche complaisance pour les travers accrédités dégraderait un homme sage et le ferait mépriser ; c’est d’un monde estimable, et non d’un monde frivole, qu’il doit ambitionner l’estime et l’amitié. Des airs légers, étourdis, évaporés, ne conviennent point à l’homme sociable, qui doit toujours par son maintien montrer qu’il s’occupe du soin de plaire à ses associés. Des airs arrogants et suffisants ne vont point à celui qui veut mériter la bienveillance des autres ; ce n’est qu’aux sots qu’il appartient de se donner bien de la peine pour se rendre insupportable ou ridicule. Un fat avantageux, par toutes ses belles manières, ne fait que tourner le dos à la considération dont il se croit assuré.\par
Pour nous faire aimer, nos manières doivent annoncer aux autres la modestie, la complaisance, la douceur, l’envie de plaire, la déférence, la politesse, la bonne éducation, la crainte de manquer aux égards. Les manières usitées dans le monde ne sont le plus souvent que des grimaces peu sincères, parce que les hommes, peu difficiles sur leurs liaisons, ne fréquentent pas toujours des personnes à qui ces sentiments sont dus. La politesse et les manières vraies ne peuvent se trouver qu’entre ceux qui s’aiment et s’estiment sincèrement.\par
En un mot, le commerce de la vie demande que nous contractions l’habitude de ne faire que ce qui peut plaire et d’éviter avec soin tout ce qui peut aliéner ceux avec lesquels notre destin nous unit. L’homme vraiment sociable doit s’observer même dans les petites choses : les fautes souvent réitérées ne laissent pas à la longue de choquer ceux avec qui nous vivons. L’attention et l’exactitude sont des qualités louables dans la société ; elles cessent d’être gênantes pour ceux à qui l’habitude les a rendu familières.\par
Néanmoins, aux yeux de bien des gens, {\itshape l’exactitude est la vertu des sots} ; mais ce qui contribue à nous concilier la bienveillance ne doit jamais être traité de sottise ; nous ne devons aucunement mépriser une qualité dont l’absence nous rend souvent désagréables même à nos amis les plus intimes. L’inexactitude annonce communément légèreté ou vanité. L’attention scrupuleuse à ne point blesser les autres est une disposition estimable puisqu’elle prouve la crainte de leur déplaire. Toute la vie sociale ne doit-elle pas avoir pour but de chercher à se faire aimer ? L’exactitude ne peut donc être dédaignée que dans des sociétés frivoles où l’homme perpétuellement distrait et tiraillé en sens contraire par des plaisirs passagers ou des fantaisies inopinées ne suit jamais dans sa marche aucune direction constante\footnote{Un homme d’esprit conseillait à un ami de ne jamais se faire attendre, de peur que celui qui l’attendrait n’eût le temps de faire l’énumération de ses défauts. {\itshape Aspettare, e non venire} est, suivant les Italiens, la source d’une impatience mortelle.}. Si l’incurie, l’inadvertance, la légèreté, l’étourderie, l’indifférence sur ce qu’on doit aux personnes avec lesquelles on vit, sont des dispositions capables d’altérer à la longue ou même d’anéantir leur bienveillance, il est bon de ne pas négliger dans le commerce de la vie les {\itshape attentions} par lesquelles nous prouvons aux autres que nous nous occupons d’eux, que nous ne les oublions pas, que nous ne perdons point de vue ce que nous leur devons.\par
L’homme attentif est assuré de plaire ; on lui sait gré de ses soins, chacun éprouve pour lui le sentiment de la reconnaissance. Les attentions {\itshape délicates} sont celles qui préviennent les désirs ; elles supposent qu’on a pris la peine d’étudier nos penchants et de nous éviter celle de les manifester. Elles annoncent un tact fin, une pénétration qui fait deviner la pensée des personnes que l’on veut obliger, une adresse qui leur sauve l’embarras du bienfait.\par
En général, il faut de l’attention quand on veut marcher avec agrément et sûreté dans le sentier étroit et raboteux de la vie. Il en faut dans le physique comme dans le moral : l’adresse est le fruit de l’attention ; la maladresse déplaît et nuit parce qu’elle nous rend souvent inutiles à nous-mêmes et aux autres. La {\itshape gaucherie} nous expose à la risée. L’homme qui veut plaire dans le monde doit se garantir du ridicule, dont le propre est toujours de diminuer l’estime. Avec de l’attention sur soi-même, on se corrige peu à peu et l’habitude nous rend facile ce qui d’abord nous paraissait difficile ou même impossible. Un fat, un présomptueux, un sot sont incapables de se corriger.\par
Ces détails qui paraîtront peut-être minutieux à bien des gens, ne doivent pourtant pas être totalement négligés quand on veut vivre agréablement dans le monde. Tout ce qui contribue à resserrer les liens de l’affection entre les hommes n’est nullement à dédaigner. Il y a de l’arrogance, de la hauteur, de la sottise à se croire dispensé de faire ce qui peut attirer la bienveillance générale, au-dessus de laquelle nul homme ne doit se mettre, quelque idée qu’il se fasse de ses propres talents ou de sa supériorité.\par
Parmi les qualités qui distinguent les hommes dans le commerce de la vie ou qui les font désirer, on doit placer les talents de l’esprit, l’enjouement, la gaieté, la science, les connaissances utiles et agréables, le goût, etc.\par
L’esprit nous plaît par son activité ; ses saillies subites nous surprennent, nous remuent, nous offrent des idées neuves, présentent à notre imagination des tableaux capables de l’amuser. On peut le définir comme la facilité de saisir les rapports des choses et de les présenter avec grâce. L’esprit juste est celui qui saisit avec précision la vérité. Le bon esprit est celui qui saisit les rapports, les convenances de la conduite ; celui qui le possède est l’homme de bien éclairé.\par
La plus grande gloire de l’esprit est de connaître la vérité. Il ne peut mériter l’estime qu’autant qu’il est utile ; c’est une arme cruelle dans la main d’un méchant. L’esprit d’un être sociable doit être sociable, c’est-à-dire contenu par l’équité, l’humanité, la modestie, la crainte de blesser ; l’esprit qui se fait haïr est dès lors une sottise. La crainte fut toujours incompatible avec l’amour, et l’estime est l’amour des qualités de l’homme.\par
L’esprit qui ne sait briller qu’aux dépens des autres est un esprit dangereux propre à troubler la douceur de la vie. La plupart des sociétés ressemblent à ces sacrifices barbares dans lesquels on immolait des victimes humaines.\par
Faute de faire attention à ces vérités, les gens d’esprit portent souvent l’alarme dans la société. La vanité que leur donne l’idée d’être craints leur persuade que tout leur est permis, qu’ils peuvent impunément abuser de leurs talents et faire sentir aux autres toute leur supériorité. Assurés des suffrages de quelques admirateurs peu délicats, ils s’embarrassent très peu de l’inimitié de ceux qu’ils blessent par leurs sarcasmes. Applaudis par des envieux et des méchants dont l’univers abonde, les gens d’esprit ont souvent la folie de préférer leurs suffrages à ceux des gens de bien. Enfin, par un étrange renversement des idées, le mot {\itshape esprit} devient souvent un synonyme de noirceur, de pétulance, de malignité, de folie.\par
Rien ne produit plus de ravages et de désagréments que la médisance, la critique impitoyable, l’esprit improbateur, talents funestes par lesquels bien des gens prétendent se distinguer ! L’envie, la jalousie et surtout la vanité sont, comme on l’a fait remarquer, les vraies causes de cette conduite. On critique les autres, on expose leurs défauts, on les relève afin de faire parade de sa pénétration, de son goût ; et pour se procurer un plaisir si futile, on risque de se faire un grand nombre d’ennemis. Les propos indiscrets font éclore à tout moment des haines immortelles dont tout homme raisonnable doit craindre de se rendre l’objet. Simonide disait « qu’on se repent souvent d’avoir parlé, et jamais de s’être tu ». Un homme se rend bien plus aimable en fermant les yeux sur les défauts des autres qu’il ne se rend estimable par sa promptitude à les pénétrer. « Taisez-vous, disait Pythagore, ou dites quelque chose qui vaille mieux que le silence. »\par
L’esprit ne peut être aimable s’il n’est assaisonné de bonté ; l’honnête homme, avec un esprit ordinaire, est préférable dans le commerce de la vie au génie le plus sublime empoisonné par la méchanceté. Les grands talents sont rares ; la société n’en a pas un besoin continuel mais elle ne peut se passer de vertus sociales. La douce {\itshape bonhomie} est préférable à l’esprit et au génie, qu’elle rend bien plus aimables quand elle les accompagne.\par
Lisons avec plaisir les ouvrages de l’homme d’esprit et du savant qui nous procurent soit du délassement, soit de l’instruction, mais vivons avec l’homme honnête et sensible sur la bonté duquel nous pouvons toujours compter. Choisissons pour ami l’homme de bien qui craint de nous déplaire et qui nous aime ; préférons-le à ces esprits redoutables qui sacrifient l’amitié même à leurs bons mots. Mais par un aveuglement très commun, l’on est bien plus jaloux de passer pour homme d’esprit que pour homme sensible et vertueux : on aime mieux se faire craindre que de se faire aimer dans des sociétés où tout le monde est en guerre.\par
Nul homme, s’il n’est bon, n’est longtemps agréable dans le commerce de la vie. L’homme de génie, s’il est vain ou méchant, efface le plaisir qu’il a fait par ses écrits et dispense le public de la reconnaissance. Un génie malfaisant ne fait du bien qu’aux envieux ; il porte la désolation dans les cœurs qu’il immole et l’indignation dans les âmes honnêtes. Il n’est pas de monstre plus à craindre que celui qui réunit un mauvais cœur et de très grands talents.\par
C’est, comme on a dit ailleurs, sur l’utilité seule que peuvent se fonder légitimement le mérite et la gloire attachés aux talents divers de l’esprit, aux lettres, aux sciences, aux arts, dont le but doit être de tirer des objets divers dont ils s’occupent, des moyens d’augmenter la somme du bien-être social et de mériter par là l’estime et la reconnaissance du public. La gloire n’est que l’estime universelle méritée par des talents qui plaisent et qui sont utiles. C’est ternir cette gloire, c’est la rendre équivoque que de nuire à ses semblables, dont l’homme, quelque supérieur qu’il soit, doit toujours ambitionner l’affection.\par
Nonobstant les préceptes affligeants d’une morale austère et sauvage qui semble vouloir insinuer qu’une vie bien réglée doit être triste et mélancolique, nous dirons que l’enjouement, la gaieté, la belle humeur sont des qualités louables et faites pour plaire dans le monde ; elles ne peuvent choquer que des misanthropes envieux et jaloux du contentement des autres. Mais cette gaieté devient blâmable quand elle s’exerce d’une façon inhumaine aux dépens du bien-être des citoyens. Quelle étrange gaieté que celle qui consiste dans des railleries piquantes, des sarcasmes offensants, des satires désolantes ? Est-ce donc être sociable ou gai que d’aller dans un repas immoler une partie des convives à la risée de l’autre ? La méchanceté, toujours inquiète et soupçonneuse, peut-elle être compatible avec la gaieté véritable, qui ne part jamais que d’une imagination riante, de la sécurité de l’âme, de la bonté du caractère.\par
La vertu seule donne à l’esprit une sérénité constante ; la vraie gaieté ne peut être le partage que de l’homme de bien. Pour être franche et pure, elle doit être soutenue par une bonne conscience, qui seule peut procurer la paix, le contentement intérieur, la joie que rien ne trouble. La gaieté est toujours plus vive dans la compagnie des personnes que l’on sait favorablement disposées. La présence d’un inconnu ou d’un homme qui déplaît suffit souvent pour dérouter l’enjouement et pour convertir en tristesse les parties dans lesquelles on se promettait le plus de joie. On n’est point gai quand on est obligé d’user de circonspection, d’avoir de la défiance : ces dispositions sont propres à priver l’esprit de la liberté de s’épanouir. Épicure disait, qu’« il ne faut pas tant regarder ce qu’on mange, que ceux avec qui l’on mange. » Connaître les hommes avec qui l’on vit et bien assortir les personnes que l’on rassemble, est un art trop négligé\textsuperscript{8}.\par
L’ennui, la satiété, l’oisiveté, qui communément tourmentent les gens du monde, font que pour se procurer quelque activité ils ont besoin d’un grand mouvement, d’un changement de scène perpétuel. Bientôt, fatigué des personnes qu’on a vu souvent, on espère trouver\par
N8 Plutarque loue le philosophe Chilon de n’avoir pas voulu promettre de se trouver au festin de Périandre avant d’avoir su les noms de tous les autres convives. Il ajoute que se mêler indifféremment avec toutes sortes de gens dans un banquet, c’est agir en homme dépourvu de jugement. Voyez Plutarque, {\itshape Banquet des sept Sages}.\par
dans des connaissances nouvelles des plaisirs nouveaux. Toujours trompé dans son attente, on voit beaucoup de monde et l’on ne s’attache à personne ; au milieu d’un tourbillon continuel, on ignore les douceurs de l’amitié, de l’intimité, de la confiance. Par un abus ridicule la sociabilité dégénère en cohue, et l’on dirait que les personnes les plus favorisées de la fortune ne se servent de leur opulence que pour s’étourdir elles-mêmes : vous les voyez toujours en mouvement sans jamais jouir de rien, l’inquiétude les poursuit jusqu’au sein des plaisirs, ils pensent incessamment à s’en procurer d’autres. Voilà sans doute pourquoi la gaieté franche et vraie se rencontre si peu à la table des riches et des grands : uniquement occupés du soin d’étaler leur faste, ils ne rassemblent que des convives dont les mœurs, les idées, les caractères, les états sont très peu compatibles. L’ennui préside à tant de soupers brillants et fastidieux parce que les compagnies les plus illustres ne sont communément composées que de combattants sous les armes toujours prêts à faire la guerre aux prétentions des autres. Le jeu est le lien ordinaire de ces assemblées de gens qui n’ont rien d’utile ou d’agréable à se dire.\par
D’un autre côté, comme les grands et les riches, par une idée fausse de grandeur, tiennent pour ainsi dire {\itshape maison ouverte}, ils ne se rendent aucunement difficiles, ils s’embarrassent fort peu de connaître ceux dont ils composent leur société. Des gens qui vivent dans une dissipation continuelle n’ont pas le temps d’approfondir les caractères. Pour peu qu’un homme ait un nom, des titres, des manières, l’art d’amuser, le jargon insipide du grand monde, il a toutes les qualités requises pour être reçu dans les meilleures compagnies ; voilà pourquoi nous les voyons si souvent composées de gens qui ne s’aiment ni ne s’estiment lorsqu’ils se connaissent, ou qui le plus souvent ne se connaissent point du tout. Rien de moins amusant que ces sociétés banales où tout homme prudent est obligé de vivre avec une réserve continuelle.\par
« La confiance, dit le duc de La Rochefoucault, fournit plus à la conversation que l’esprit. » La vraie gaieté suppose de l’affection, de l’amitié, une exemption totale de soupçons et de craintes. En vain chercherait-on ces dispositions dans des cercles et des banquets où chacun représente, où chacun, occupé des intérêts de son amour-propre, épie celui des autres, le mesure des yeux, est bien plus disposé à prendre de l’humeur ou à nuire qu’à communiquer du plaisir ou contribuer de bonne foi à l’amusement de tous. La vanité n’est point gaie ; toujours inquiète et soupçonneuse, concentrée en elle-même, elle craint de s’échapper. La gaieté n’est communément le partage que des personnes simples et droites qui se trouvent en liberté, qui vivent avec cordialité, qui se communiquent réciproquement le plaisir d’être ensemble. Nulle société agréable entre les hommes sans l’assurance de trouver dans leurs associés des égards, de la politesse, de la bienveillance, de la sincérité, de l’indulgence, de l’amitié.\par
Le contentement vrai ne semble aucunement fait pour les cours des princes ; l’orgueil de l’étiquette doit l’en bannir absolument, pour faire place à la réserve et à l’ennui majestueux. Il est exclu des assemblées des grands, toujours trop occupés de leurs menées et de leurs intérêts cachés. Il n’assiste pas aux festins de l’opulence, qui ne connaît de plaisir que dans son luxe et son faste. On le rencontre peu dans les compagnies mêlées et dans les cabales littéraires. Enfin, on le chercherait vainement dans la plupart des sociétés brillantes, qui sont les théâtres où de fiers champions viennent se livrer des combats continuels et où les différents acteurs sont toujours sous le masque. Quiconque veut être gai doit, en entrant dans une compagnie, oublier lui-même et faire oublier aux autres son amour-propre, ses petitesses, ses titres, ses prétentions.\par
Rien de moins sociable et de moins gai que la société dédaigneuse et hautaine qui s’arroge exclusivement le titre de {\itshape bonne compagnie} par excellence : les personnes dont elle est composée sont des courtisans par état, ennemis les uns des autres, qui sous les dehors d’une politesse affectée couvrent des âmes ulcérées. Ce sont des nobles entêtés de leurs prérogatives, toujours prêts à faire sentir aux autres la hauteur de leurs prétentions, ce sont des femmes occupées d’intrigues, de cabales, de galanteries criminelles, perpétuellement jalouses les unes des autres. Des Protées sans esprit et sans caractère qui n’ont que l’art de se prêter aux fantaisies et au jargon de la frivolité, passent pour des gens du {\itshape bon ton}. Aux yeux de l’homme de bien, la {\itshape bonne compagnie} est celle qui est composée de personnes honnêtes, vertueuses et bien unies. Le {\itshape bon ton} est celui qui maintient l’harmonie sociale.\par
Par une juste compensation, les indigents, le peuple, les jeunes gens, les personnes d’une fortune médiocre, en un mot ceux que la grandeur dédaigneuse et le bel esprit appellent {\itshape gens du mauvais ton}, trouvent le secret de s’amuser et de rire de meilleur cœur que tant d’êtres superbes qui rarement savent jouir de la vie. Tout plaisir est neuf pour la jeunesse et pour l’homme laborieux ; la joie franche se livre, s’abandonne sans contrainte. L’artisan a d’ailleurs acheté par du travail le droit de se délasser, tandis que l’homme désœuvré a communément épuisé tous les amusements. Enfin, des hommes simples vivent {\itshape bonnement} entre eux et sont bien disposés à l’égard de leurs égaux, au lieu que les personnes d’un ordre plus relevé n’apportent le plus souvent dans leurs parties que les sentiments tristes et cachés de l’envie, de la haine, de la contrainte et de l’ennui. Ce qu’on appelle {\itshape le grand monde} est ordinairement composé de gens qui s’ennuient réciproquement, qui souvent se détestent et qui pourtant ne peuvent se passer les uns des autres.\par
La gaieté vraie ne peut être l’effet que de la bonté du cœur, de la complaisance mutuelle, du contentement intérieur répandu sur les autres. On ne doit pas la confondre avec la joie bruyante de l’intempérance, ni avec la dissipation tumultueuse et les orgies de la débauche. L’homme de bien est un homme de goût qui met du choix, de la décence, de la retenue dans ses plaisirs ; il ne trouve rien de piquant dans ceux qui ne sont pas assaisonnés par la raison.\par
Le {\itshape goût} est l’habitude de juger promptement les beautés et les défauts des productions de l’esprit ou des arts. L’homme de goût plaît dans la société parce qu’il présente à l’esprit des autres des idées choisies capables de flatter leur imagination. Dans la poésie, notre imagination est remuée par un heureux choix d’images, de similitudes, de circonstances capables de fixer agréablement l’attention. Dans la peinture, le goût nous plaît parce qu’il rassemble les situations les plus propres à nous faire une impression agréable et vive. Le goût moral, de même que celui qui a les beaux-arts pour objet, est l’habitude de juger sainement et promptement des beautés et des défauts, des convenances et des disconvenances des actions humaines, c’est-à-dire de connaître les degrés de l’estime ou du blâme que mérite la conduite de l’homme. Ce goût est le fruit de l’expérience, de la réflexion, de la raison. En morale, un homme de goût est un homme d’un tact fin et suffisamment exercé qui juge avec facilité ce qui mérite l’approbation ou le mépris, d’où l’on voit que ce que plusieurs moralistes ont appelé un {\itshape instinct moral}, bien loin d’être une faculté {\itshape innée}, est une disposition acquise et dont peu de gens sont doués. Il n’y a donc que l’homme de bien, l’homme sociable et vertueux qui ait véritablement un bon esprit, la science vraiment utile, la gaieté vraie, enfin un goût sûr dans les choses les plus intéressantes à la vie\phantomsection
\label{footnote115}\hyperref[bookmark131]{\dotuline{\textsuperscript{115}}}.\par
Les méchants et les vicieux ne sont réellement que des hommes sans jugement, sans esprit et sans goût, qui mènent dans la société une vie inquiète et troublée sans jamais y jouir des plaisirs purs réservés à la sagesse. En un mot, tout nous prouve que si la félicité peut être le partage de quelque être de l’espèce humaine, elle doit exclusivement appartenir à l’homme vertueux, qui toujours a le droit d’être content de lui-même et de se flatter de contenter les autres.
\subsection[{Chapitre VIII. De la Félicité}]{Chapitre VIII. De la Félicité}
\noindent La morale, comme tout a dû le prouver, est l’art de rendre l’homme heureux par la connaissance et la pratique de ses devoirs. « Ce ne sont pas, dit Marc Aurèle, les raisonnements, ce ne sont pas les richesses, la gloire, ni les plaisirs qui rendent l’homme heureux ; ce sont ses actions. Pour qu’elles soient bonnes il faut connaître et le bien et le mal : il faut savoir pourquoi l’homme est né et quels sont ses devoirs… Être heureux, c’est se faire un sort agréable à soi-même ; et ce sort agréable consiste dans les bonnes dispositions de l’âme, dans la pratique du bien, dans l’amour de la vertu\textsuperscript{1}. »\par
La félicité est un état constant, inaltérable que l’on ne peut trouver ni dans ce qu’on désire, ni dans ce qui nous manque, mais dans ce qu’on possède. Les plaisirs ne sont que des bonheurs instantanés ; ils ne peuvent procurer cette continuité, cette permanence nécessaire à notre félicité. Ainsi, les dons de la fortune, la gloire, les avantages que donne le préjugé, dépendant du caprice du sort ou de la fantaisie des hommes, ne peuvent donner à l’esprit cette fixité de laquelle son bonheur doit dépendre, ni bannir les inquiétudes qui peuvent le troubler. Les plaisirs des sens sont encore moins capables de nous fournir le contentement et la sécurité de l’âme ; quelque variés qu’on les suppose, ils finissent toujours par s’émousser avec promptitude et par nous plonger ensuite dans la langueur de l’ennui. En un mot, les objets extérieurs ne peuvent donner à l’homme une félicité continue, qui serait impossible et par la nature de l’homme et par la nature des choses\textsuperscript{2}.\par
C’est donc en lui que l’homme doit établir un bonheur inaltérable, et la vertu seule peut y produire non une insensibilité morne et nuisible mais une activité réglée qui occupe agréablement\par
N1 Aristote, dans ses livres moraux adressés à Nicomaque, dit qu’« être heureux, bien agir et bien vivre, sont une seule et même chose… que le bon, l’honnête et l’agréable sont étroitement liés, au point de ne pouvoir jamais être séparés. » Cicéron a dit que la vie heureuse est l’objet unique de toute la philosophie. Voyez Cicéron, {\itshape De Finibus}, livre II. Il serait bien inutile de parler aux hommes de morale et de vertu s’il n’en résultait pas le plus grand bien pour eux : une vertu totalement {\itshape gratuite} est une chimère peu séduisante pour des êtres qui désirent le bonheur par une impulsion constante de leur nature. Platon disait le philosophe « l’ami de la Nature et le parent de la vérité. » Suivant Aristote, (livre I, chap. 1 de sa {\itshape Morale}), « tout art et toute science, ainsi que toute action et tout projet, doit voir quelque bien pour objet ».\par
N2 Plutarque (traduction d’Amyot) dit : « Là où le vivre doucement et joyeusement ne procède point du dehors de l’homme, ainsi, au contraire, de l’homme départ et donne à toutes choses qui sont autour de lui joie et plaisir, quand son naturel et ses mœurs sont au-dedans bien composés, parce que c’est la fontaine et source vive dont tout le contentement procède. » Voyez Plutarque, {\itshape Du Vice et de la Vertu}. l’esprit sans le fatiguer ou lui causer du dégoût. La vertu n’étant que la disposition habituelle de contribuer au bien-être de nos semblables et l’homme vertueux étant celui qui met cette disposition en usage, il suit que l’homme sociable ne peut se faire un bonheur isolé et que sa félicité dépend toujours du bien qu’il fait aux autres.\par
Un ancien poète a dit avec raison que « l’homme de bien double la durée de sa vie, et que c’est vivre deux fois que de jouir de la vie passée ». Est-il rien de plus satisfaisant que de vivre sans reproche, de pouvoir à chaque instant repasser dans sa mémoire le bien qu’on a fait à ses semblables, de ne trouver dans sa conduite que des objets agréables dont on ait droit de s’applaudir ! Toute la vie de l’homme vertueux et bienfaisant n’est pour lui qu’une suite d’images délicieuses et de tableaux riants. « Lorsque l’on a cultivé la vertu, dit Cicéron, dans toute la suite de la vie, on en recueille de merveilleux fruits dans la vieillesse ; et non seulement ces fruits sont toujours présents jusqu’au dernier moment de la vie, ce qui serait toujours beaucoup quand il n’y aurait que cela seul, mais ils sont accompagnés d’une joie perpétuelle que produit le témoignage d’une bonne conscience et le souvenir de tous les biens que nous avons faits\textsuperscript{3}. » Diogène disait que « pour l’homme de bien, tous les jours doivent être des jours de fêtes ». Procurer à l’homme une félicité durable que rien ne puisse altérer, et lier cette félicité à celle des êtres avec lesquels il vit, voilà le problème dont la morale doit s’occuper et qu’on a tenté de résoudre dans cet ouvrage. Notre but était de prouver que le vrai bonheur consiste dans le témoignage invariable d’une bonne conscience, ce juge incorruptible établi pour toujours au dedans de nous-mêmes pour nous applaudir du bien que nous faisons et dont les décrets sont confirmés par ceux sur qui nous agissons. « Il n’y a point, dit Cicéron, de plus grand théâtre pour la vertu que la conscience\textsuperscript{4}. » Quintilien a dit depuis que {\itshape la conscience vaut mille témoins}\textsuperscript{5}.\par
Quel pouvoir sur la terre peut ravir à l’homme de bien le plaisir toujours nouveau de rentrer satisfait en lui-même, d’y contempler en paix l’harmonie de son cœur, d’y sentir la réaction des cœurs de ses semblables, d’y voir l’amour et l’estime de soi confirmés par les autres ? Telle est la félicité que la morale propose à tous les hommes dans tous les états de la vie ; c’est à ce bien-être permanent qu’elle leur conseille de sacrifier des passions aveugles, des fantaisies indiscrètes, des plaisirs d’un moment.\par
La morale, pour avoir une base invariable, doit être établie sur un principe évidemment commun à tous les êtres de l’espèce humaine, inhérent à leur nature, mobile unique de toutes leurs actions. Ce principe, comme on l’a déjà fait voir ailleurs, est le désir de se\par
N3 Cicéron, {\itshape De la Vieillesse}, chap. III.\par
N4 {\itshape Tusculanes}, II, §. 26.\par
N5 {\itshape Institutions oratoires}, livre 5, chap. XI, n°41, édition Gesner. conserver, de jouir d’une existence heureuse, d’être bien dans tous les moments de notre durée sur la terre ; c’est ce désir toujours présent, toujours actif, toujours constant dans l’homme, que l’on désigne sous le nom d’{\itshape amour de soi}, d’{\itshape intérêt}.\par
Pour être persuasive, la morale, au lieu de combattre ou d’étouffer cet amour ou cet intérêt inséparable de nous et nécessaire à notre conservation, doit le guider, l’éclairer et le fortifier. Elle manquerait son but si elle voulait empêcher l’homme de s’aimer, de chercher son bonheur, de travailler à ses intérêts ; elle est faite pour lui montrer comment doit s’aimer un être raisonnable et sociable, comment il doit se conserver, comment il peut mériter l’estime et l’affection des autres. Elle lui enseignera quels sont les intérêts qu’il doit écouter et distinguer de ceux qu’il doit sacrifier à des intérêts bien plus chers et plus solides. La morale n’est que l’art de s’aimer véritablement soi-même en vivant avec des hommes ; la raison n’est que la connaissance de la route qui conduit à la félicité.\par
Faute de réfléchir, les hommes ont la plus grande peine à sentir la liaison de leur intérêt personnel avec celui des êtres dont ils sont environnés. Cette ignorance de nos rapports entraîne l’ignorance de tous les devoirs de la vie. Au sein des sociétés, on ne voit que des hommes isolés à qui l’on ne peut faire concevoir qu’ils se rendent odieux et misérables en séparant leurs intérêts de ceux des êtres dont ils ont besoin pour leur propre bonheur. Par une suite de cette ignorance, le tyran n’a plus d’intérêts communs avec son peuple, qu’il craint et pour lequel il est un objet d’horreur. Les grands rougissent de confondre leurs intérêts avec ceux des vils citoyens qu’ils méprisent. Les magistrats, orgueilleux d’avoir le droit de juger, ne s’occupent que des intérêts futiles de leur vanité. Les ministres de la religion, contents des droits qu’ils ont reçu du Ciel, dédaignent de s’occuper des intérêts frivoles du reste des mortels. Les soldats, payés et favorisés par le prince, n’ont plus rien qui les attache à la patrie. Autorisé par la loi, le mari ne se met guère en peine de contribuer au bonheur de sa femme ; celle-ci de son côté ne croit rien devoir au despote qui la néglige ou l’outrage. Le père occupé de son avarice ou de ses plaisirs oublie qu’il doit l’éducation et le bien-être à des enfants forcés de désirer sa mort. Des maîtres hautains traitent avec dureté des serviteurs dont ils se font des ennemis cruels. Enfin, il n’est presque point d’amis sincères et constants, parce que la société n’est remplie que d’hommes indifférents qui se font une existence isolée ou qui se font la guerre.\par
De cette malheureuse division d’intérêts naissent évidemment tous les inconvénients publics et particuliers, les discordes, les rapines, les trahisons, les perfidies dont les sociétés civiles et domestiques deviennent les théâtres. Voilà sans doute pourquoi tant de moralistes ont avec grande raison regardé l’amour aveugle de soi, l’intérêt personnel, comme une disposition odieuse et méprisable sur laquelle il serait insensé et dangereux de fonder la morale. Voilà pourquoi des philosophes ont prétendu que la vertu consistait dans une lutte continuelle avec une nature essentiellement dépravée. Ils ont cru que dire à l’homme de s’aimer lui-même, c’était l’exciter à s’aimer exclusivement sans songer aucunement aux autres. En un mot, ils se sont imaginé qu’établir les devoirs de la morale sur l’amour de soi, c’était lâcher la bride à toutes les passions suggérées par une Nature aveugle et privée de raison.\par
Les moralistes qui invitent les hommes à suivre leurs passions ressemblent à ces médecins qui permettent à leurs malades désespérés de satisfaire leurs fantaisies les plus nuisibles. Si quelques sophistes imprudents ont prétendu que l’homme, en s’aimant lui-même, en suivant sa nature, en consultant son intérêt, pouvait impunément se livrer à ses passions, ils se sont grossièrement trompés. La médecine, avec la morale, devrait suffire pour les convaincre que celui qui s’aime véritablement et qui veut se procurer une existence agréable, doit, pour son propre intérêt, résister fortement aux penchants dont tout lui montre les dangers. Est-ce donc s’aimer soi-même que de n’opposer aucun remède à la fièvre que produisent les excès de l’intempérance, les ardeurs impudiques, les emportements de la colère, les mouvements de la haine, les morsures de l’envie, les délires de l’ambition, les fureurs du jeu, les angoisses de l’avarice ? Est-ce s’aimer vraiment soi-même que de séparer son cœur des êtres avec lesquels notre intérêt et nos besoins nous lient et sans l’estime et l’affection desquels la vie serait désagréable ? L’homme {\itshape personnel}, concentré en lui-même, qui ne voit que lui seul en ce monde, peut-il donc se flatter que quelqu’un s’intéresse sincèrement à son sort ? Celui qui n’aime que lui-même n’est aimé de personne.\par
« Je ne puis, dit Marc Aurèle, être touché d’un bonheur qui n’est fait que pour moi. » Un être sociable ne peut se rendre heureux tout seul, ne peut se suffire à lui-même, éprouve le besoin de communiquer aux autres un bien-être qui toujours rejaillit sur son propre cœur. Quelqu’un a dit, avec grande raison : « Si vous voulez être heureux tout seul, vous ne le serez jamais ; tout le monde vous contestera votre bonheur ; si vous voulez que tout le monde soit heureux avec vous, chacun vous aidera… Si vous voulez être heureux en sûreté, il faut l’être avec innocence ; il n’y a de bonheur certain et durable que celui de la vertu\textsuperscript{6}. »\par
Aristote compare l’homme vertueux à un bon musicien qui écoute avec plaisir les sons harmonieux qu’il tire de son instrument et qui, même tout seul, s’en applaudit. L’homme de bien est le seul qui sache comment il faut s’aimer, qui connaisse son véritable intérêt, qui distingue les impulsions de la Nature qu’il doit suivre ou réprimer ; enfin, il a seul un {\itshape amour-propre} légitime, un droit fondé sur sa propre\par
N6 Voyez {\itshape Lettre d’un Mère à son Fils sur la vraie Gloire}, tome II du recueil du r.p. Desmolets, pages 295 et 296. estime, parce qu’il sait avoir droit à l’estime des autres. Ne condamnons pas ce sentiment honnête, ne le confondons pas avec l’orgueil ou la vanité. Nul homme ne peut être estimé des autres s’il ne se respecte lui-même. Le renoncement à l’estime publique est une source féconde de vices et de crimes. La conscience ou la connaissance de sa propre valeur ne peut être blâmée que lorsqu’elle est injuste ou lorsqu’elle n’a point égard à la valeur des autres. « L’amour de l’estime est l’âme de la société ; il nous unit les uns aux autres. J’ai besoin de votre approbation, vous avez besoin de la mienne… Il est aussi honnête d’être glorieux avec soi-même que ridicule de l’être avec les autres\textsuperscript{7}. »\par
Privé par l’injustice du rang que l’homme de bien sait devoir occuper, il n’est point avili pour cela, il ne cesse pas de s’estimer, il connaît sa propre dignité et se console par la justice de ses droits. Son bonheur est en lui-même, il l’y retrouve toujours. Le cœur d’un honnête homme est un asile où il jouit en sûreté d’un bien-être immuable qu’on ne peut lui arracher.\par
Cette félicité n’est point idéale et chimérique, elle est réelle ; son existence est démontrée pour tout homme qui se plaît à rentrer quelquefois en lui-même. Est-il un mortel sur la terre qui ne se soit applaudi toutes les fois qu’il a fait une action vertueuse ? Qui est-ce qui n’a pas senti son cœur se dilater après avoir soulagé un malheureux ? Qui est-ce qui n’a pas contemplé avec transport l’image du bonheur tracé sur le visage de ceux dont il avait réjoui les âmes par ses bienfaits ? Est-il quelqu’un qui ne se soit félicité de sa bonté généreuse même lorsque l’ingratitude lui refusait le retour que méritait sa bienfaisance ? Enfin, est-il un homme qui n’ait point éprouvé un sentiment de complaisance, un redoublement d’affection pour lui-même quand il a fait des sacrifices à la vertu ? En contemplant alors la force de son âme, ne se trouve-t-il pas aussi heureux qu’un héros qui repasse ses victoires dans son esprit ? « Le sage, dit Horace, ne connaît que Jupiter au-dessus de lui ; il est riche, libre, beau, comblé d’honneurs, il est le roi des rois\textsuperscript{8}. » Marius n’était-il pas bien content au milieu de ses malheurs, quand un Romain le vit assis sur les ruines de Carthage ?\par
Que l’on ne nous dise donc plus que la vertu demande des sacrifices douloureux. L’estime juste de soi, les applaudissements légitimes de la conscience, l’idée de sa grandeur et de sa propre dignité ne sont-ils pas des récompenses assez amples pour dédommager l’homme de bien des vanités, des frivolités, des avantages futiles qu’il sacrifie au plaisir d’être constamment estimé de lui-même et des autres ? Les motifs naturels de l’amour de soi, de l’intérêt bien entendu, ne sont-ils donc pas plus réels, plus puissants, plus dignes de l’homme de bien que les motifs romanesques d’une morale enthousiaste que l’on admire sans jamais s’y rendre ? Faut-il autre chose pour exciter les hommes à la vertu que leur faire sentir que l’estime, l’affection, la tendresse et la félicité intérieure l’accompagnent toujours ? Pour leur inspirer l’horreur du vice, peut-on leur présenter des motifs plus pressants que les remords, les infirmités, les malheurs sans nombre dont la Nature, au défaut des lois, punit fidèlement les égarements des peuples et des individus ?\par
N7 {\itshape Ibidem}, pages 296 et 311.\par
\par
Quelle que soit la dépravation des mœurs, est-il une seule vertu à laquelle les méchants mêmes ne rendent incessamment hommage ? Est-il un vice qui dans les autres ne leur paraisse incommode et méprisable ? Le concert unanime de tous les habitants de la terre, bons ou méchants, sages ou insensés, justes ou injustes, s’accorde donc à nous crier que la vertu est le souverain bien et que le vice est un mal que tous sont forcés de haïr. Tous les vices sont ennemis des vices ; la société des méchants est composée de membres qui s’incommodent à tout moment les uns et les autres.\par
Dira-t-on que les décrets par lesquels la Nature adjuge des récompenses à la vertu et décerne des châtiments contre les transgresseurs de la morale sont supposés imaginaires ? Ne les voyons-nous pas s’exécuter sous nos yeux de la façon la plus marquée ? En vertu de ces arrêts irrévocables, nous voyons les peuples justes et tranquilles jouir durant une profonde paix d’une prospérité digne d’envie, tandis que des peuples ambitieux expient par de longues misères les maux qu’ils se sont faits à eux-mêmes et aux autres. Nous voyons des souverains équitables et vigilants goûter le plaisir si doux d’être chéris de leurs sujets rendus heureux par leurs soins, tandis que nous voyons les tyrans agités et tremblants sur les débris des nations désolées. Nous voyons les grands et les riches bienfaisants jouir des respects et de l’amour de ceux que leur crédit protège ou que leurs bienfaits soulagent, tandis que le courtisan odieux ne se console de la haine publique que par son impudente vanité ou tandis que des héritiers avides attendent impatiemment la mort de l’avare qui s’oppose à leurs jouissances. Nous voyons l’abondance et la concorde régner chez les époux vertueux, chez le père de famille économe et bienfaisant, tandis que nous ne trouvons que division et désordres chez ces époux en discorde et ces chefs de familles à qui la règle est inconnue. Enfin, nous voyons les bonnes mœurs, la tempérance et la vertu récompensées par la santé, la vigueur, l’estime publique, et la dissolution cruellement punie par de longues infirmités et par le mépris universel. « Les méchants, dit Plutarque, n’ont besoin d’aucun dieu ni d’aucun homme qui les punisse, parce que leur vie corrompue et tourmentée est pour eux un châtiment continuel. »\par
Que l’on ne dise donc plus que la Nature n’a point de récompenses suffisantes à donner aux observateurs de ses lois, ni de peines à infliger à ceux qui les violent. Il n’est point sur la terre de vertu qui ne trouve son salaire, il n’est point de vice ou de folie qui ne soient sévèrement punis. La morale est la science du bonheur pour tous les hommes, soit qu’on les considère en masse, soit qu’on les regarde comme partagés en sociétés particulières, en liaisons, en familles, soit enfin qu’on ne s’occupe que du bien-être des individus, abstraction faite des êtres qui les environnent.\par
La félicité des peuples dépend d’une sage politique qui, comme nous l’avons prouvé, n’est que la morale appliquée au gouvernement des empires. Un gouvernement juste rend les peuples heureux ; personne n’y sent la verge de l’oppression ; chaque citoyen y travaille en paix à sa subsistance, à celle de sa famille ; la terre, soigneusement cultivée, y porte l’abondance ; l’industrie, dégagée des chaînes de l’exacteur, y prend un libre essor ; le commerce y fleurit au sein de la liberté ; la population suit toujours l’abondance ou la facilité de subsister. Une patrie qui rend ses enfants heureux trouve en eux des défenseurs actifs, prêts à sacrifier leur vie et leurs trésors à la félicité publique partagée par chacun des citoyens.\par
La félicité des rois dépend de leur fidélité à remplir les devoirs de leur état. Un prince fermement attaché à la justice la fait régner sur son peuple ; celui-ci regarde son chef comme un dieu tutélaire, comme l’auteur de tous les biens dont il jouit. Protégé par ses bienfaits, le sujet travaille avec ardeur et pour lui-même et pour son maître, dont il sait que les vues ont toujours le bien de tous pour but invariable. Que manque-t-il à la gloire, à la puissance, à la sûreté, au contentement d’un souverain qui voit dans tous ses sujets des enfants réunis d’intérêts avec lui et prêts à tout entreprendre pour contribuer au bonheur d’une famille dont le chef a su gagner tous les cœurs ? Est-il sur la terre une félicité plus grande que celle d’un monarque que ses vertus mettent en droit de compter sur la tendresse de tout son peuple, sur la vénération de ses voisins, sur l’admiration de la postérité la plus reculée ? Le bonheur d’un bon roi n’est le plus grand des bonheurs que parce qu’il est à portée de faire un plus grand nombre d’heureux.\par
La félicité des grands et des riches consiste dans la faculté de prêter une main secourable et bienfaisante à ceux que le destin afflige ; ce bonheur disparaît pour eux quand ils ne font pas de leur pouvoir ou de leur opulence le seul usage qui puisse les rendre eux-mêmes heureux. Le crédit, la puissance, la richesse ne sont rien dès qu’elles ne contribuent en rien à la félicité de ceux qui les possèdent {\itshape ;} elles ne peuvent y contribuer qu’en répandant le bien-être.\par
La félicité des familles dépend de la fidélité de leurs chefs à remplir leurs devoirs. En les observant avec exactitude, des époux bien unis conspirent à élever des enfants destinés à devenir un jour les supports et les consolateurs de leur vieillesse. Leurs exemples et leurs bienfaits identifient avec leur famille des serviteurs sincèrement attachés, qui deviennent des amis zélés, des coopérateurs de leurs entreprises. « Peu d’hommes, dit Plutarque, sont appelés à gouverner des villes et des empires, mais chacun est à portée de gouverner sagement sa famille et sa maison. »\par
La félicité du pauvre (car la Nature marâtre ne l’exclut pas du bonheur, à l’exemple de ces hommes hautains qui le supposent plus malheureux qu’eux-mêmes), la félicité, dis-je, du pauvre consiste dans les moyens de subsister par un travail modéré. Ce travail, qui paraît un si grand mal à l’oisive opulence, est pour lui un bien réel ; l’habitude l’y accoutume, le besoin le lui rend cher, il l’exempte d’une foule d’infirmités, de désirs, de besoins, d’inquiétudes dont le riche est travaillé. Le pauvre n’est-il pas plus heureux que le despote ou le tyran que la terreur poursuit jusqu’au fond de son sérail ? Gygès, roi de Lydie, enivré de ses richesses et de sa puissance, consulta l’oracle pour savoir s’il existait au monde un mortel plus heureux que lui ; l’oracle lui indiqua un laboureur d’Arcadie\phantomsection
\label{footnote116}\hyperref[bookmark132]{\dotuline{\textsuperscript{116}}}.\par
La félicité du savant et de l’homme de lettres consiste dans la jouissance des connaissances utiles dont leur esprit s’est enrichi ; l’étude est pour eux un plaisir habituel qui les garantit des chimères qui font l’objet des désirs du vulgaire abusé. D’ailleurs, une vie agréablement occupée les dispense de recourir à des vices et à des folies sans nombre, ressources ordinaires de ceux dont l’esprit n’est point cultivé. Rien n’égale les plaisirs que la retraite procure à celui qui a contracté l’habitude de converser avec lui-même, rien ne manque à son bonheur et à la considération qu’il mérite par ses talents s’il y joint une âme vertueuse, sans laquelle les talents mêmes perdent tout leur prix. Les études des savants, les fruits de leurs méditations doivent se montrer dans leurs mœurs : les plus instruits des hommes sont obligés à être les plus humains, les meilleurs, les plus honnêtes, et bientôt ils jouiront de la considération et de la gloire dans lesquelles ils placent tout leur bonheur. Ménandre a dit « que les mœurs de celui qui nous parle nous persuadent bien mieux que tous ses raisonnements ».\par
Enfin, la félicité de l’homme qui vit dans le monde consiste à jouir des plaisirs honnêtes que la société lui procure, à mériter par sa complaisance, ses attentions et ses égards, la bienveillance et l’estime des personnes desquelles le destin le rapproche, à goûter avec un petit nombre d’amis choisis les douceurs de la confiance, à pratiquer dans sa sphère les devoirs de son état, à contenter les autres afin de se mettre lui-même en droit de jouir du contentement, qui fut et sera toujours la récompense de la vertu.\par
C’est évidemment à l’ignorance ou au mépris des règles de la morale qu’est due la plus grande partie des malheurs de la terre. Partout on voit les hommes séparés par l’intérêt personnel mal entendu, presque entièrement étrangers les uns pour les autres, former des associations non pour se rendre réciproquement la vie douce et agréable mais pour se nuire de plus près, pour se tourmenter sans relâche. Ces aveugles mortels peuvent être comparés à des voyageurs engagés dans une foule qui s’avanceraient inconsidérément sans jamais songer à ceux qui les précèdent ou les suivent, non plus qu’à ceux qui marchent à leurs côtés. De ces dispositions il résulte un mécontentement général ; personne n’est satisfait ni de ses compagnons de voyage, ni de lui-même.\par
Les malheurs attachés au mépris de la morale se font sentir aux sociétés comme aux individus. Les nations pour lesquelles une fausse politique forgea presque toujours un code fondé sur leurs aveugles intérêts mais très contraire à la justice, à la vertu, furent et seront toujours les victimes de leur perversité. Pourquoi voyons-nous des peuples enrichis par le commerce, jouissant d’un bon gouvernement et de la liberté, possesseurs de contrées immenses, et néanmoins toujours avides, inquiets, mécontents, tourmentés sans relâche de mouvements convulsifs ? C’est qu’on ne jouit de rien sans la vertu, c’est que tout devient poison pour des hommes sans mœurs dont le propre est d’abuser des biens les plus précieux. Sous un embonpoint trompeur, les nations corrompues cachent souvent les maladies les plus cruelles.\par
Pourquoi des princes tout-puissants au bonheur desquels rien ne devrait manquer, passent-ils leurs tristes jours dans les alarmes ou dans les langueurs de l’ennui ? C’est qu’imbus dès leur enfance des maximes empoisonnées de la flatterie, ils s’imaginent ne rien devoir aux autres hommes : ils se croient des divinités immobiles faites pour recevoir l’encens et les hommages des mortels méprisés. Les infortunés ne connaissent que le plaisir d’être craint, ils ignorent le doux plaisir d’être aimé ! Les aveugles ne sentent pas qu’un prince n’est vraiment heureux qu’à la tête d’un peuple heureux. Quel mobile peut agir sur le cœur d’un monarque lorsqu’il est insensible au bonheur d’être chéri de ses sujets ? Enorgueillis dès le berceau ou nourris dans l’ignorance de leurs devoirs, les grands et les riches ne savent pas que le pouvoir de faire du bien est la seule source légitime des distinctions établies entre les hommes. Plongés dans une mollesse fastidieuse, rassasiés de vains amusements, étrangers aux plaisirs du cœur, peu touchés de la tendresse de leurs inférieurs, qu’ils dédaignent, ils ne jouissent qu’en idée d’une grandeur que l’on redoute et que leur morgue fait détester. Rarement on voit la sérénité ou la joie pure habiter sur le front de ceux que le vulgaire croit des êtres bien fortunés. Les aiguillons secrets de l’ambition, les inquiétudes de la vanité, les supplices lents de l’ennui vengent cruellement l’indigent de ceux qui le méprisent et l’oppriment.\par
Perpétuellement écrasé sous les vexations et les dédains des hommes puissants, l’homme du peuple est aigre, brutal et sans mœurs. Il gémit dans la misère et fait à tout moment une comparaison chagrine de son état laborieux et pénible avec celui de ces oisifs qu’il suppose très heureux. Il imite autant qu’il peut leurs vanités et leurs travers, et par ses efforts impuissants, il ne fait que redoubler son malheur.\par
Communément étrangers à la raison, à la morale, l’homme du peuple et l’indigent suivent en aveugles les impulsions de leur nature inculte et cherchent souvent dans le vice ou dans le crime le bonheur dont ils se voient privés par leurs supérieurs. Ce sont, comme on l’a dit ailleurs, les riches et les grands qui sont la cause primitive des vices et des désordres des pauvres. Faute de connaître les vrais principes de la morale ou les moyens d’arriver au but qu’en cette vie tout homme doit se proposer, les familles ne sont très souvent composées que de malheureux. On n’y voit que des époux sans tendresse journellement occupés à se rendre la vie insupportable, des pères tyrans, des mères dissipées et déréglées, des enfants corrompus par des exemples funestes, des proches en querelles, des maîtres impérieux et durs, des serviteurs sans attachement et sans probité. Tous ces associés divers ne semblent se rapprocher les uns des autres que pour travailler de concert à se rendre misérables. Dans le commerce du monde, chacun, par inadvertance ou par folie, paraît vouloir renoncer à l’affection, à l’estime, à la considération, qui sont pourtant les objets de ses vœux les plus ardents. Une vanité présomptueuse, des manières offensantes, un orgueil inflexible, des jalousies inquiètes bannissent des assemblées destinées à la joie, l’amitié sincère, la cordialité, la gaieté véritable, qui seules peuvent répandre des charmes sur la vie. En voyant la conduite de bien des gens, on dirait qu’ils ne s’assemblent que pour avoir occasion de se haïr et de s’attrister mutuellement.\par
Ce serait fermer ses yeux à l’expérience que de ne point reconnaître les influences du vice ou du mal moral sur le physique des hommes. Combien de nations et de contrées florissantes n’ont pas été presque anéanties et rendues incultes par l’ignorance, les vices, la négligence des rois ? En vain la Nature a-t-elle doué des empires peuplés de la plus grande fertilité : des souverains dépourvus de mœurs et de lumières viennent à bout de les convertir en déserts. L’ambition toujours cruelle et la vanité dispendieuse des princes dépouillent et font périr sans pitié les peuples qu’elles immolent à leurs aveugles caprices ; ces despotes si fiers sont ensuite tout surpris de ne trouver dans leurs États qu’une solitude effrayante et des sujets incapables de leur fournir les secours qu’ils ne cessent de leur demander. Mais les besoins continuels d’une cour affamée ont découragé l’agriculture, ont banni le commerce, ont fait languir les manufactures, ont arrêté les travaux de tous les citoyens ; ceux-ci ont été livrés aux vexations des grands ou aux extorsions ingénieuses et réitérées des traitants altérés du sang des peuples. C’est ainsi que la négligence, les passions et les vices des puissants sont une malédiction pour la terre : ils la forcent d’être stérile, ils condamnent à l’infortune, à la faim, à la contagion, à la mort, ceux qui devraient la cultiver.\par
Indépendamment de ces effets généraux et marqués du vice ou du mépris de la morale sur toute une nation, qui peut douter de ses effets sur les individus ? Combien de maladies contractées par les fatales habitudes de la débauche, de l’intempérance, de l’oisiveté, de la trop grande ardeur dans la poursuite des plaisirs ? À ces causes, qui détruisent chaque jour la santé et l’existence d’une foule d’êtres imprudents, joignez l’ennui cruel, les peines d’esprit, les vapeurs, les chagrins, les remords, les mécontentements habituels qui minent peu à peu les corps et les conduisent à pas lents au tombeau. Le suicide, effet terrible soit d’une maladie de langueur, soit d’un délire subit, n’est point rare chez les peuples dont les mœurs sont corrompues. Des sybarites affaiblis par le luxe et le vice n’ont pas la force de soutenir les coups du sort. Voilà comment le moral influe sur le physique, voilà comment, faute de raison et de vertu, tant d’hommes ne semblent vivre sur la terre que pour souffrir eux-mêmes et faire des malheureux. Par une loi constante de la Nature, nul homme dans la vie sociale n’est fort que par sa réunion avec ses associés ; personne n’est estimé et considéré qu’en se rendant utile, personne ne peut être aimé qu’en faisant du bien aux autres, personne ne peut être heureux qu’en faisant des heureux ; enfin, personne ne peut jouir de la paix du cœur, du contentement de soi-même, de la tranquillité constante si favorable à la conservation de son être, qu’en se rendant témoignage qu’il a fidèlement accompli les devoirs de la morale dans le poste qu’il occupe parmi les hommes. La morale, on ne peut trop le répéter, est la seule route qui mène à la félicité véritable : elle influe sur le physique : le visage même de l’homme de bien annonce le repos dont il jouit.\par
Nous voyons donc que le bonheur n’est le partage exclusif d’aucun état. La Nature invite également tous ses enfants à travailler pour l’obtenir, mais dans quelque position qu’ils se trouvent, elle l’a toujours attaché à la vertu. Rien n’est donc moins fondé que les vaines déclamations d’une sombre philosophie qui décrie indistinctement les grandeurs, les richesses, le désir de la gloire et qui les interdit à tous ceux qui cherchent la sagesse. Est-il rien de plus désirable pour les peuples que de voir la vertu sur le trône travaillant également à la félicité commune des souverains et des sujets ? Quel bien pour les hommes si ceux qui sous les rois jouissent de l’autorité voulaient en faire usage pour s’illustrer par la vigilance à remplir leurs nobles fonctions ! Le riche ne serait-il pas un citoyen respectable si, au lieu de dissiper ses trésors sans profit pour lui-même, il s’en servait pour ranimer l’indigence découragée, pour soulager les malheurs publics, pour réveiller l’industrie ? Enfin, cette gloire que l’on traite de fumée, n’est-elle pas un objet réel et désirable puisqu’elle n’est que l’estime universelle faite pour exciter l’esprit et le génie à contribuer au bien-être et aux agréments de la vie ?\par
N’écoutons pas non plus les conseils fanatiques d’une morale farouche qui voudrait nous montrer la perfection sublime et la félicité complète dans une apathie insociable, dans une indifférence totale pour le genre humain. Toute morale qui se propose d’isoler l’homme, de le concentrer en lui-même, de le séparer des êtres parmi lesquels la\par
Nature l’a placé, est une morale dictée par la misanthropie, qui ne doit point en imposer à des êtres sociables. Comment celui qui romprait tous les liens faits pour l’unir à ses semblables pourrait-il avoir des vertus ? Qu’est-ce que des vertus qui n’ont pas le genre humain pour objet ? Quelle estime les hommes doivent-ils à des sauvages effarouchés qui, pour se dispenser de leur être utiles, vont s’enfoncer dans des déserts ? Est-ce travailler à la félicité de l’homme vivant en société que de lui conseiller de rentrer dans l’état d’un sauvage et de renoncer aux avantages sans nombre que la vie sociale lui procure ? Le sauvage est-il vraiment heureux ? En quoi peut consister le bonheur merveilleux d’un être vivant avec les bêtes, perpétuellement occupé à leur disputer sa nourriture, exposé à l’inclémence des saisons, privé des ressources, des commodités, des lumières, des secours que la société fournit à ses membres ? Le sauvage est-il un être vertueux ? Peut-on appeler des vertus l’absence des désirs pour des objets dont on n’a point d’idées ?\par
Enfin, trouvons-nous dans les hordes sauvages répandues encore dans le Nouveau Monde que des vertus bien réelles remplacent les vices que les nations nombreuses et policées communiquent à leurs citoyens ? Non, sans doute ! Si ces sauvages sont exempts de la soif des richesses, des besoins immodérés du luxe, des chaînes du despotisme, des entraves du grand monde, nous les voyons faire un usage affreux de leur liberté naturelle, ou plutôt de leur folie, pour s’égorger les uns les autres. Sur les plus légers prétextes, ils portent la désolation et le carnage chez leurs voisins, ils exercent sur leurs captifs des cruautés qui font frémir la nature, ils traitent leurs femmes avec une férocité révoltante. Leurs enfants ne sont pas à l’abri de leurs fureurs soudaines ; en place des vices dont les nations civilisées sont agitées, nous trouverons que les peuplades sauvages ont une cruauté, une soif de la vengeance, une déraison qui ne sait mettre aucun frein aux passions les plus terribles. Des hommes de cet affreux caractère peuvent-ils être des modèles de vertu ? Leur genre de vie déplorable annonce-t-il aucunement la félicité ? Leur franchise même n’est que le signe de leur tempérament indompté ; leurs vertus sont souvent des crimes, leur innocence n’est que l’ignorance grossière de ce qui constitue le bonheur de la vie\phantomsection
\label{footnote117}\hyperref[bookmark133]{\dotuline{\textsuperscript{117}}}.\par
Vivons donc avec les hommes ; fermons les yeux sur leurs défauts, cherchons à les servir, ne les haïssons jamais. Si les nations civilisées sont malheureuses, c’est qu’elles conservent encore trop de vestiges de leur barbarie primitive. C’est à cet esprit sauvage que l’on doit attribuer la plupart des guerres que la déraison des princes secondée par les préjugés des grands et des peuples rend encore si fréquentes sur la terre. Par la folie des souverains, les peuples les plus policés vivent encore comme des hordes sauvages et sont perpétuellement occupés à se détruire. Par une suite des opinions fausses transmises par nos barbares ancêtres, le métier fatal de la guerre est réputé la profession la plus noble, l’art d’exterminer les hommes est celui qui conduit le plus sûrement aux honneurs, aux récompenses, à la gloire, dans des nations qui auraient bien plus besoin des arts de la paix pour devenir heureuses et florissantes. Mais l’esprit insociable et sauvage maintenu presque en tous lieux par l’ambition des princes, s’oppose à la guérison des préjugés mêmes dont on reconnaît les affreuses conséquences. Ce sont des cours sauvages, ignorantes, corrompues, qui donnent le ton aux nations et qui entretiennent chez elles les erreurs, le mépris de la science, les usages déraisonnables, les vanités puériles dont tant de peuples sont encore infectés. Enfin, dans l’examen que nous avons fait des vices des hommes, tout nous prouve qu’ils viennent de leur inexpérience, de leur légèreté qui, contribuant à les tenir dans une longue enfance, les rendent encore très insociables et très sauvages.\par
Malgré la puissance des forces qui s’obstinent à retenir les hommes dans un état si contraire à leur véritable nature, rien ne nous autorise à désespérer de la guérison des esprits et de la réforme des mœurs. L’expérience et le malheur sont les grands maîtres des hommes ; ils les forceront tôt ou tard à renoncer à des préjugés qui partout s’opposent à leur félicité. Des souverains plus éclairés connaîtront enfin leurs intérêts, ils renonceront un jour à cette politique injuste aussi contraire à leur bien-être qu’à celui de leurs sujets. Ils sentiront que ces guerres interminables, ces conquêtes ruineuses, ces triomphes sanglants ne font que saper les fondements de la félicité nationale et que la politique ne peut jamais s’écarter impunément des règles de la morale. À force de calamités, les princes s’instruiront de leurs devoirs et reconnaîtront que le pouvoir arbitraire ne procure à celui qui l’exerce que le triste avantage de régner en tremblant sur des esclaves chagrins et découragés.\par
Ainsi, n’affligeons pas les hommes par une morale désespérante, ne les renvoyons pas dans les forêts, ne les séparons pas les uns des autres ; disons-leur d’être plus justes, plus modérés, plus sociables, montrons-leur les motifs capables de les convaincre et de les toucher, gardons-nous de leur dire que la félicité n’est point faite pour eux, faisons-leur sentir que la vertu seule peut donner un bien-être dont leurs vanités, leurs vices et leurs folies les écartent à tout moment.\par
Nous conviendrons sans peine que la réforme si désirable des mœurs des nations et des souverains ne se montre encore que dans le lointain. Elle ne peut être que le fruit tardif des expériences et des lumières répandues peu à peu sur les hommes et des circonstances que le destin seul peut amener. Cela même n’est pas fait pour décourager le sage : il sait que ce n’est qu’avec lenteur que la vérité se propage mais qu’elle est faite pour produire son effet tôt ou tard. Les égarements des hommes, toujours punis par la Nature, les forceront de recourir à la raison, à la morale, à la vertu, dans le sein de laquelle ils trouveront ce bonheur que des penseurs chagrins ont supposé n’être point fait pour la terre.\par
Que les amis de la sagesse continuent donc de semer des vérités, qu’ils se tiennent assurés qu’elles germeront un jour. Si leurs leçons paraissent inutiles à leurs contemporains, elles serviront à la postérité, dont le bien-être ne doit pas être indifférent aux gens de bien qui pensent. La vérité est un bien commun à tous les habitants de ce monde ; rejetée dans une contrée, elle fructifie dans une autre, repoussée dans un siècle, elle sera mieux accueillie dans un temps plus heureux, dédaignée par les pères, elle fera le bonheur de leurs descendants rendus plus sages par les folies de leurs ancêtres.\par
Enfin, quand même un heureux changement dans les mœurs des peuples ne serait qu’une flatteuse chimère, les conseils d’une sage morale ne seraient point inutiles pour cela : ils serviraient du moins à fortifier l’homme de bien dans la pratique de la vertu, à la lui rendre plus chère, à le confirmer de plus en plus dans les sentiments habituels à son cœur.\par
L’espoir d’un avenir plus heureux, les peintures touchantes de la vertu contribuent, pour ainsi dire, à rafraîchir, à ranimer les âmes honnêtes et sensibles souvent flétries par le spectacle affligeant des calamités qui désolent le monde.\par
Au défaut du bonheur public que la société lui refuse, le citoyen vertueux est réduit à se procurer une félicité particulière. Dans le sein de sa famille, dans le sein de l’amitié, il trouvera des consolations, des douceurs, un bien-être que la tyrannie ne pourra lui ravir ; en pratiquant fidèlement les vertus sociales, il jouira de la sérénité constante de son cœur. Sur le visage de sa femme, de ses enfants, de ses amis, de ses serviteurs, il lira le contentement, il s’applaudira d’y contribuer ; il jouira de la confiance, de l’estime, de la tendresse de tous les êtres avec lesquels il aura des rapports, il sera content de lui par la certitude d’être chéri de tous ceux qui l’entourent.\par
Le méchant, au contraire, toujours mécontent de lui-même, ne rencontre partout que des ennemis ; il ne voit en tous lieux que des accusateurs qui lui reprochent sa conduite odieuse et ses traitements cruels, il voudrait pouvoir les anéantir. Semblable à Caligula, il désirerait que tous n’eussent qu’une seule tête afin de pouvoir l’abattre d’un seul coup. Dans la société, dans la maison, dans lui, il ne trouve qu’un spectacle effrayant dont l’idée le poursuit même lorsqu’il est sans nul témoin\phantomsection
\label{footnote118}\hyperref[bookmark134]{\dotuline{\textsuperscript{118}}}.\par
En promettant à l’homme une félicité complète, la morale ne lui fait point espérer l’exemption des malheurs de ce monde ; elle ne le garantira pas des calamités publiques, des coups de la fortune, de la méchanceté des hommes, de l’indigence qui souvent accompagne le mérite et la vertu, des maladies cruelles, des maux physiques, de la mort.\par
Mais du moins la morale prépare son esprit aux événements de la vie ; elle lui apprend à supporter avec courage les revers imprévus, à ne point s’en laisser abattre, à se soumettre aux décrets du sort. Dans les peines les plus cuisantes, elle offre à l’homme de bien une retraite agréable en lui-même où la paix d’une bonne conscience lui fournira des consolations inconnues des méchants qui, aux malheurs qu’ils éprouvent, sont forcés de joindre encore la honte et les remords de leurs vices et de leurs actions criminelles.\par
Le plus cruel tourment d’un méchant dans l’infortune, c’est la conscience de son affreux caractère, de la haine qu’il est fait pour exciter, de la justice du châtiment qu’il éprouve. « Il vaut mieux, dit Épicure, être malheureux et raisonnable, qu’être heureux et dépourvu de raison. »\par
Le vrai sage n’est point un homme impassible ; il n’a point les prétentions de ce stoïcien insensé qui, au milieu des tourments, criait à la douleur qu’elle n’était point un mal. Il n’est point insensible à la perte de la fortune, de la santé, de ses proches, de ses amis ; il ne fait pas consister la vertu à contempler d’un œil sec la privation des objets les plus chers à son cœur.\par
Il gémit comme un autre de la rigueur du destin mais il trouve dans la vertu des ressources et des forces ; il sent qu’avec elle l’on ne peut être complètement malheureux\phantomsection
\label{footnote119}\footnote{« Est etiam quiete et pur e, et eleganter acta ætatis placida, ac lenis senectus. Cicéron, {\itshape De Senect.}, chap. 5. « C’est, dit M. Dacier, une vérité constante que l’heureuse vieillesse est une couronne de gloire et de sécurité qui ne se trouve que dans le sentier de la vertu. » Voyez {\itshape Comparaison de Pyrrhus et de Marius}, à la fin.}, et que sans elle la puissance, la grandeur, l’opulence, la santé, sont incapables de procurer la vraie félicité. Enfin, dans la vieillesse et jusqu’au bord du tombeau, l’homme vertueux est soutenu par le souvenir consolant d’une vie paisible, pure et bien ordonnée\phantomsection
\label{footnote120}\footnote{« On n’est point sous le malheur, disait Démocrite, tant qu’on est loin de l’injustice. »}.
\subsection[{Chapitre IX. De la Mort}]{Chapitre IX. De la Mort}
\noindent Non seulement une conduite réglée par la morale nous procure une paix inaltérable, une félicité pure pendant notre séjour en ce monde, non seulement elle fait jouir d’une vieillesse heureuse et considérée, mais encore elle affermit contre les craintes de la mort, si terribles pour les coupables.\par
Si, comme on l’a dit ailleurs, la religion, soit naturelle, soit révélée, ne peut jamais contredire les devoirs que la Nature impose à l’être sociable, si cette religion n’est vraie que par sa conformité avec les lois de la saine morale ou par le bonheur qu’elle procure aux hommes, enfin si la religion ne fait que joindre des motifs surnaturels aux motifs naturels, humains et connus, dont la morale universelle peut se servir pour exciter à la vertu, rien n’est fait pour troubler la sécurité de l’honnête homme prêt à sortir de cette vie pour en commencer une autre.\par
Persuadé que l’univers est sous l’empire d’un monarque rempli de bienveillance pour ses sujets, il ne peut en mourant éprouver aucune inquiétude sur son sort. Quelle raison l’homme de bien aurait-il de se défier des caprices ou de redouter la colère d’un dieu dont la bonté et la justice constituent le caractère essentiel et immuable ? L’idée d’une vie future, qui sert de base à toute religion, n’est elle-même fondée que sur les récompenses que la vertu doit attendre tôt ou tard d’un dieu plein d’équité.\par
Un dieu juste peut-il ne point aimer l’homme juste ? Un dieu bon peut-il haïr celui qui dans ce monde a fait du bien à ses semblables ? Un dieu rempli de miséricorde peut-il rejeter celui dont les entrailles se sont émues sur les infortunes de ses frères ? Enfin, celui qui a tâché d’être utile à la société craindrait-il de rencontrer au terme de ses jours un juge inexorable dans le souverain de la Nature, dans le créateur, le conservateur et le père de la race humaine, dans ce législateur de la volonté duquel la religion fait dériver les règles de la morale ? Non, sans doute ; ce serait contredire toutes les perfections morales attribuées à la divinité que de croire un instant que l’homme de bien pût lui déplaire.\par
Il est vrai que la religion exige encore d’autres vertus dans l’homme pour mériter la faveur divine. Mais dans le cours de cet ouvrage on s’est uniquement proposé de présenter à tous les habitants de la terre les motifs humains, sensibles, naturels, qui peuvent les porter à faire le bien dans le monde actuel, même en faisant abstraction de leurs idées religieuses. On ne leur a parlé que des moyens d’obtenir un bonheur aussi durable que la vie présente. C’est aux théologiens qu’il appartient exclusivement de montrer aux mortels les motifs divins, invisibles, surnaturels, qui doivent les conduire un jour à la félicité permanente que la religion fait espérer au-delà des bornes de la vie. Quoique rien ne dût paraître plus efficace pour exciter les hommes à la vertu et les détourner du mal, que l’idée d’un bonheur éternel, spirituel, ineffable ou que la crainte de châtiments rigoureux et sans fin, néanmoins l’expérience nous fait voir que ces motifs présentés chaque jour par les ministres de la religion ne peuvent rien, ou du moins n’agissent que faiblement sur la multitude. Dominés par le présent, les hommes, pour la plupart, ne pensent guère à l’avenir, qui leur paraît toujours fort éloigné.\par
Le monde est rempli d’êtres vicieux qui font profession de se soumettre à la religion, de croire les récompenses et les châtiments qu’elle annonce, sans pourtant que ces idées produisent aucun bien réel dans leur conduite ici-bas.\par
En effet, lorsqu’on voit les vices, les désordres et les crimes que se permettent tant d’hommes qui se disent très convaincus de la réalité des récompenses et des châtiments éternels que la religion annonce, on serait tenté de croire que ce ne sont que de vaines chimères qui n’en imposent à personne ou que ces idées séduisantes et terribles sont un frein beaucoup trop faible pour contenir les passions. Tant de souverains religieux et dévots, par leurs guerres cruelles, inutiles et fréquentes, par leurs injustes conquêtes, par la tyrannie et les extorsions qu’ils font éprouver à leurs peuples, par les dérèglements auxquels on les voit se livrer journellement, semblent faire entendre que la religion, qu’ils feignent de croire, qu’ils protègent, qu’ils affectent de respecter, n’est pas faite pour eux et n’est qu’un épouvantail destiné à contenir leurs crédules sujets. Ceux-ci néanmoins ne sont pas, pour la plupart, mieux contenus que leurs maîtres.\par
Les nations les plus religieuses nous montrent une foule d’hommes qui allient chaque jour la croyance et la pratique extérieure de la religion avec l’injustice, l’inhumanité, la rapine, la fraude, la débauche. L’on y voit des voleurs publics, des traitants, des fripons, des prostituées, des libertins, et parmi le peuple, des ivrognes et des crapuleux qui jamais n’ont eu de doutes sur l’autre vie et qui pourtant n’agissent point en conséquence ; leurs désordres habituels sont l’objet continuel des remontrances inutiles de nos orateurs sacrés.\par
Mais si la religion effraie par ses menaces les transgresseurs de la morale, quelques philosophes reprochent à ses ministres de les confirmer dans leurs dérèglements et de les rassurer, par les moyens faciles qu’ils leur donnent de calmer leurs consciences, d’expier leurs iniquités et d’apaiser la colère divine. « À quoi servent, disent-ils, ces terreurs d’une autre vie, s’il suffit pour les faire disparaître de se soumettre à des pratiques stériles, à des confessions humiliantes pour le moment, à des cérémonies, à des formules, à des aumônes et des prières\textsuperscript{1} ? N’est-ce pas, demandent-ils, anéantir l’effet des craintes que la religion inspire, que d’assurer qu’un repentir tardif à l’article de la mort est capable d’effacer toutes les taches d’une vie criminelle ? » Ils trouvent que ses ministres, souvent trop indulgents pour les grands de la terre, aplanissent la voie du Ciel à ces illustres coupables, dont ils devraient plutôt aiguiser les remords. Quoiqu’il en soit de ces reproches, de l’aveu même des prêtres de la divinité, rien n’est plus rare que de voir la religion opérer sur des cœurs corrompus un changement sincère et propre à mériter la félicité future.\par
D’un autre côté, l’on trouve les théologiens souvent peu d’accord entre eux sur les moyens de satisfaire à la justice divine et d’obtenir le bonheur éternel. Les uns exigent trop peu des hommes ou leur procurent des expiations faciles, les autres, par un rigorisme excessif, les rebutent, leur montrent la route du Ciel remplie de tant de difficultés qu’ils les jettent dans le désespoir ou dans un fanatisme farouche, insociable, aussi contraire à la vraie morale que les désordres les plus funestes. Rien de moins fait pour la vie sociale que le superstitieux sombre et mélancolique qui, devenu l’ennemi de lui-même, se croit obligé de se tourmenter sans cesse, de renoncer aux plaisirs les plus innocents, de se séparer des vivants, de méditer sa fin au milieu des tombeaux. Quel bien pour l’espèce humaine peut résulter de cette conduite insociable ? L’homme continuellement abreuvé de ses larmes, nourri de mélancolie, agité de vains scrupules et de terreurs imaginaires, aigri par la solitude et les privations, peut-il être un membre utile ou agréable pour la société ? Est-ce donc accomplir les devoirs de la morale que de se faire du mal à soi-même sans faire du bien à personne ? C’est sans doute se former des idées bien sinistres et bien contradictoires d’un dieu rempli d’amour pour les hommes, que de croire qu’on ne lui plaît qu’en s’affligeant sans relâche ou en demeurant séquestré du reste des humains. Si des casuistes trop faciles ouvrent le Ciel aux plus grands scélérats, des rigoristes outrés le ferment à tout le monde : peu de gens savent trouver un juste milieu entre ces deux extrêmes.\par
Des inconséquences si frappantes sont cause que bien des gens ont osé douter de l’utilité ou du pouvoir qu’on attribuait à la religion. D’un autre côté, l’histoire ancienne et moderne montrant à chaque page les excès, les ravages, les haines immortelles, les persécutions atroces, les massacres lamentables qu’ont souvent produits sur la terre\par
N1 Rien de plus ridicule que les cérémonies extravagantes que la superstition a fait imaginer chez quelques peuples pour rassurer les hommes contre les craintes de la mort. Un Banian se tient assuré que tous ses péchés lui seront remis s’il peut en expirant tenir la queue d’une vache et recevoir son urine sur le visage. D’autres se croient très sûrs d’être sauvés s’ils peuvent mourir sur les bords du Gange. Les Parsis ne doutent aucunement que leurs fautes ne soient expiées si un prêtre fait pour eux des prières et des cérémonies auprès du feu sacré. Pour assurer le salut du mahométan, on lui met en mourant dans la main quelque passage de l’{\itshape Alcoran}. Le prêtre russe, moyennant de l’argent, expédie au mourant un passeport pour l’autre monde, etc. l’ambition du sacerdoce et le zèle furieux de ses partisans fanatiques, quelques penseurs en ont conclu que cette religion qui servait tant de fois de prétexte à des crimes, était non seulement inutile mais encore incompatible avec la saine morale, la vraie politique, le bien-être et le repos des sociétés. Conséquemment, quelques philosophes se sont crus suffisamment autorisés à secouer le joug d’une religion qui leur paraissait incommode et dangereuse. L’existence d’une autre vie, dont ils voyaient que l’idée ne contenait aucunement les passions de ceux mêmes qui auraient dû en être le plus fortement convaincus, leur parut chimérique ou douteuse. En un mot, on ne peut disconvenir que l’insociabilité, l’intolérance, l’ambition et l’avarice de plusieurs ministres de la religion ne leur aient en tout temps suscité un grand nombre d’ennemis, même parmi des hommes éclairés et vertueux.\par
C’est aux théologiens qu’il appartient de concilier cette conduite avec les principes soit de la morale naturelle, soit de la religion, ou du moins de se laver d’accusations si graves. Qu’ils ramènent les égarés de bonne foi par des raisonnements capables de les détromper de leurs idées peu favorables sur l’utilité du système de l’autre vie. Bornés dans cet ouvrage à faire connaître les motifs humains d’une morale destinée à tous les hommes (quelles que puissent être d’ailleurs leurs opinions vraies ou fausses), nous dirons à ceux qui rejettent la religion révélée et ses dogmes sur l’autre vie qu’ils n’en sont pas moins obligés de se conformer durant la vie présente aux préceptes humains et naturels de la morale universelle, sous peine de s’attirer le mépris et la haine de la société, châtiments assurés et dont l’incrédulité la plus forte ne pourra jamais douter.\par
Bien plus, si c’est véritablement l’intérêt de la morale et le bien-être de la vie sociale qui ont déterminé le philosophe à faire divorce avec la religion, il est obligé plus que tout autre de montrer au public des mœurs plus sociables, plus douces, plus honnêtes, en un mot, une conduite moins blâmable que celle qu’il impute aux partisans de cette religion. Il ne convient point à celui qui s’écarte des principes religieux sous prétexte du mal qu’ils ont produit sur la terre, de se permettre l’intolérance, l’opiniâtreté, la haine envers ceux qui ne pensent pas comme lui. Il ne lui est pas plus permis de se livrer à des vices que la raison condamne. La vraie philosophie doit toujours annoncer des mœurs innocentes et sévères ; grave sans être ni triste ni farouche, elle ne doit jamais se prêter aux dérèglements des hommes.\par
Nous dirons donc à tous ceux qui ne renoncent à la religion que parce qu’elle gêne leurs passions, qu’ils ne doivent point pour cela se croire des philosophes ou des amis de la sagesse. La vraie sagesse fut et sera toujours incompatible avec le vice et le dérèglement ; ses préceptes ne peuvent jamais être opposés à ceux de la morale. Des philosophes sans mœurs et sans vertus seraient des imposteurs, des charlatans méprisables. Ces prétendus amis de la sagesse, ces apôtres de la raison seraient des insensés, des aveugles, des ignorants, s’ils se rendaient les apologistes du vice et les contempteurs de la vertu, qui seule peut faire notre bonheur en ce monde. Ce serait alors qu’on pourrait à juste titre regarder des philosophes de cette trempe comme des libertins, des corrupteurs, des ennemis du genre humain. Ils sont aussi coupables que ces casuistes relâchés qui, par une lâche complaisance pour les vices et les passions des hommes, affaiblissent leurs scrupules ou leurs remords et leur rendent le chemin du Ciel plus facile que la religion ne le permet.\par
Tout homme qui aura médité la nature humaine et les vrais intérêts de la société, quelles que puissent être d’ailleurs ses idées religieuses, est forcé de reconnaître que la vertu est utile et nécessaire en ce monde, que sans elle nulle société ne peut ni prospérer ni subsister, que sans elle nul individu ne peut se faire aimer et considérer, que le vice est destructeur pour les nations ainsi que pour les familles et pour chacun de leurs membres ; en un mot, tout homme qui pense doit sentir qu’il n’est point de désordre qui ne trouve son châtiment, même en cette vie, qu’il n’est point de vertu qui n’y trouve quelque consolation ou récompense et qui ne contribue au bonheur de celui qui l’exerce. Un philosophe qui méconnaîtrait des vérités si claires serait un stupide, un ignorant, un homme peu susceptible d’expérience et de réflexion. Étrange philosophie, sans doute, que celle qui ne verrait pas les effets si marqués du désordre, du libertinage, du vice, et leur influence funeste sur les nations et les individus, ou qui ne sentirait pas les avantages inestimables que la vertu procure à tous ceux qui la cultivent au sein même des sociétés corrompues !\par
D’un autre côté, il suffit de connaître et pratiquer des vérités si simples pour vivre heureusement sur la terre. Ainsi, quel que puisse être son sort dans l’autre vie, l’incrédule, s’il est honnête homme ou vraiment philosophe, peut dans cette vie passagère, en observant fidèlement les devoirs de la morale humaine, se procurer tout le bonheur dont il s’est fait l’idée. S’il exerce avec soin les vertus sociales, s’il évite les vices, les imperfections, les défauts qui peuvent déplaire aux autres et lui nuire à lui-même, s’il contribue par ses talents et ses travaux à l’utilité générale, il se rendra cher à tous ceux qui auront des rapports avec lui, il sera bon père, époux fidèle, ami sincère, citoyen estimable, et quelle que soit la place que la religion lui assigne dans l’autre monde, il jouira dans celui-ci de l’affection et de la considération qui sont dues au mérite.\par
Borné dans ses espérances, il ne se flattera point d’obtenir les joies ineffables d’une autre vie : il se contentera de celles que l’on trouve ici-bas. Lorsqu’il aura bien mérité du genre humain par ses services, au défaut de l’espoir d’une immortalité surnaturelle (que l’homme religieux a seul droit de se promettre), il se flattera d’obtenir une immortalité naturelle, ou d’exister après sa mort dans la mémoire des hommes. Ainsi, satisfait de son sort en ce monde, privé d’espérances et de craintes pour l’avenir, plein de confiance dans ses droits sur la tendresse de la postérité, l’incrédule honnête et vertueux peut vivre très heureux et voir sa fin d’un œil plus tranquille que tant d’hommes soumis à la religion, qu’ils pratiquent si peu.\par
Quelles que soient les opinions vraies ou fausses des hommes, les lois inflexibles de leur nature les obligent également, leur morale doit être la même et tout leur prouvera que dans le monde qu’ils habitent, la vertu conduit à la félicité et le vice au malheur. Si l’on peut aisément s’égarer en matière de spéculations, on ne s’égarera jamais dans sa conduite en vivant conformément à la nature d’un être sociable, intelligent, raisonnable, qui connaît son vrai bonheur et les moyens de l’obtenir. En suivant la route indiquée par la morale, l’homme de bien vivra content et mourra sans alarmes. Le moment du trépas, si cruel pour tant d’êtres inutiles ou nuisibles à la terre, ne peut effrayer l’homme vertueux qui, satisfait du rôle qu’il a joué, se retire de la scène avec tranquillité et dit avec le poète : « J’ai vécu, j’ai bien fourni la carrière que le sort m’avait tracée\textsuperscript{2}. »\par
Il n’y a que l’homme de bien, l’homme raisonnable, l’homme utile aux autres hommes qui puisse dire avec vérité {\itshape j’ai vécu}. Ce n’est point vivre, c’est végéter que de ne point contribuer au bonheur de ses semblables ; c’est exister comme les plantes venimeuses ou les minéraux empoisonnés que d’être sur la terre pour n’y faire que du mal. Il n’y a que celui dont l’esprit s’est orné par la sagesse, dont le cœur s’est fortifié par la raison, qui ait acquis le droit de mourir avec courage et de se mettre au-dessus des terreurs de la mort, si accablantes pour tant d’êtres pusillanimes qui tiennent follement à la vie sans pourtant en savoir tirer aucun profit.\par
C’est au moment de la mort que le pauvre et l’infortuné ont un avantage marqué sur ces hommes que le vulgaire croit les possesseurs exclusifs de la félicité. L’indigent, l’artisan, le laboureur, l’homme du peuple ne quittent point la vie avec cette répugnance que l’on remarque pour l’ordinaire dans ceux qui meurent sur le duvet. Le malheureux ne voit dans la mort que la fin de ses peines ; l’homme de bien, trop souvent exposé aux rigueurs de la fortune dans un monde pervers où il n’a d’autre secours que celui de sa vertu, envisage sa fin comme le port qui va le mettre en sûreté.\par
Bien plus, il y eut dans tous les temps des hommes qui, pour se soustraire aux chagrins de la vie, en ont volontairement accéléré le terme. L’Antiquité admira leur action et la prit pour une marque d’un courage héroïque. Les modernes, à cet égard, ont changé d’opinion. La religion condamne le suicide comme une désobéissance formelle à la volonté divine, comme une désertion lâche qui nous fait abandonner le poste où Dieu nous a placé, enfin comme une faiblesse qui fait que nous ne pouvons soutenir les coups de la fortune.\par
N2 « Vixi, et quem dederat cursum fortuna, peregi. » Virgile.\par
En effet, le suicide, comme on l’a déjà fait entendre, est l’effet d’une vraie maladie, d’un dérangement subit ou lent dans la machine. Pour être totalement dégoûté de la vie qui, malgré ses traverses, offre des plaisirs si variés à tous les hommes, pour étouffer en eux le désir de se conserver inséparable de leur nature, pour éteindre entièrement l’espérance qui reste au fond des cœurs même au milieu des plus grands malheurs, il faut une révolution terrible, un renversement général dans les idées d’où résulte une aversion forte pour l’existence, devenue à notre imagination le plus fâcheux de tous les maux, le plus irréparable. Des effets si cruels ne peuvent être produits que par une véritable maladie, que l’on pourrait comparer, soit à un transport de folie ou de rage qui nous aveugle, soit à une maladie de langueur qui nous mine sourdement et nous conduit à la mort. Ainsi que les insensés ou les fous décidés, les hommes qui finissent par se détruire sont uniquement occupés d’un seul objet, sans la possession duquel ils ne voient plus rien d’agréable dans la vie. Dans Caton d’Utique, cet objet fut la liberté de son pays ; dans un avare, ce sera la perte de son or ; dans un amant, ce sera la perte de celle qu’il aime avec fureur ; dans un homme vain, ce sera la privation des choses qui flattaient sa vanité. L’absence de ces objets divers agit différemment sur les hommes en raison de leurs tempéraments ou de leurs caractères. Les uns, plus emportés, se livrent subitement au désespoir ; les autres, d’un tempérament moins bouillant ou plus mélancolique, couvent très longtemps dans leur sein le projet de mourir. Dans ces différentes façons de se détruire, il n’y a proprement ni force ni faiblesse, ni courage ni lâcheté : il y a maladie, soit {\itshape aiguë}, soit {\itshape chronique}. Les hommes, accoutumés à juger les actions par les motifs qui les font naître, ont admiré le suicide produit par l’amour de la patrie, de la liberté, de la vertu, et ils l’ont blâmé quand il n’eut pour motif que l’avarice, un fol amour, une vanité puérile. Le suicide est une folie ; c’est à la religion à décider si elle rend coupable aux yeux de la divinité.\par
Si le suicide est l’effet d’une maladie, il serait peu sensé de prétendre le combattre par des raisonnements. Mais la morale peut au moins fournir les moyens de se garantir d’un mal si étrange qui devient épidémique dans les nations mal gouvernées, livrées au luxe, à la vanité, à l’avarice, à la corruption des mœurs, à des plaisirs déréglés. Une vie sage, des désirs modérés, l’économie dans les plaisirs, la fuite du luxe et des objets capables d’irriter les passions et la vanité, enfin le travail, sont des préservatifs contre une maladie dont l’effet est de nous dégoûter de la vie et de nous armer contre nous-mêmes. La vraie force consiste à résister aux passions dangereuses ; en réformant les mœurs, un bon gouvernement rendra les hommes plus contents de leur sort et les suicides moins fréquents. L’homme de bien éclairé est le seul qui puisse avoir un vrai courage et contempler de sang-froid les approches de la mort. L’ignorance et le vice sont toujours lâches, incertains et timides ; les étourdis et les méchants n’ont jamais eu le temps d’envisager leur fin. La résignation du sage dans ses derniers moments ne peut être l’effet que de la réflexion et du calme que procure une bonne conscience. Une vie pure, une conduite raisonnable et réfléchie, voilà la meilleure la seule préparation à la mort.\par
Enfin, il n’y a que l’homme juste, bienfaisant, estimable, qui voie dans ses derniers instants sa couche entourée d’amis fidèles et dont l’urne soit arrosée de larmes bien sincères. Quoi de plus propre à consoler de la nécessité de mourir, que l’idée de subsister dans la mémoire des autres et de produire encore longtemps des sentiments de tendresse dans les cœurs de tous ceux qu’on laisse après soi dans le monde ?\par
Combien de gens meurent sans avoir jamais su profiter de la vie ! Vivre, c’est agir. Jouir, c’est goûter le plaisir d’être aimé en faisant des heureux, c’est rendre les autres contents afin d’être soi-même content. Mais ces plaisirs réservés aux âmes honnêtes et sensibles sont inconnus des méchants endurcis qui, après avoir vécu dans le trouble, meurent désespérés. Ils ne sont point faits pour les hommes livrés aux vices, à la dissipation, à des plaisirs criminels ou frivoles, que la mort vient toujours surprendre et ne trouve aucunement affermis contre ses coups. Enfin, les plaisirs consolants de la vertu, si propres à fortifier les cœurs, sont ignorés de la plupart des princes, des grands, des riches qui, placés sur la terre pour la rendre heureuse, ne font communément que redoubler ses maux. Tout nous montre que les hommes que le rang et la fortune met à portée de faire le plus de bien, sont très souvent inutiles ou nuisibles pendant toute leur vie, ne savent jouir de rien et n’emportent en mourant les regrets de personne. Faute de connaître le contentement attaché à la vertu bienfaisante, les mortels qui pourraient se rendre les plus heureux vivent ou dans la stupeur de l’ennui, ou dans une agitation fatigante, soit pour eux-mêmes, soit pour les autres. Leur mort, désirée par ceux qui les entourent, est pour ceux-ci un moment de délivrance et de joie. Par quel droit, en effet, celui qui n’a fait aucun bien sur la terre, qui n’a vécu que pour lui seul, qui même n’aura fait qu’affliger les malheureux dont il est environné, pourrait-il prétendre qu’on le regrette ? Les pleurs et les regrets des vivants sont des hommages du cœur qui ne sont dus qu’à l’homme de bien sensible et tendre. La vie heureuse et la mort tranquille ne peuvent être les effets que de l’utilité, des talents, de la bonté, de la vertu.\par
Reconnaissez donc, ô hommes ! que dans la vertu seule réside ce bonheur qu’on désire et qu’on cherche si vainement ailleurs. Ce n’est qu’en vous montrant utiles et bons que vous pourrez prétendre à l’amour de vos semblables et que vous aurez le droit de vous aimer vous-mêmes.\par
Apprenez enfin à connaître votre {\itshape intérêt} le plus cher, le plus réel : apprenez la manière dont chacun de vous doit s’aimer. Cet amour de soi est nécessaire, naturel, inséparable de l’homme, approuvé par la morale. Mais il vous impose le devoir d’aimer les autres, de contribuer à leur bien-être, si vous voulez mériter leur tendresse et leurs secours.\par
Occupez-vous donc de ceux qui font route avec vous dans le sentier difficile de la vie. Prêtez-leur une main secourable afin de les engager à vous assister à leur tour. Ce serait se haïr que de se concentrer en soi-même et d’oublier les égards, la bienveillance, les soins que l’on doit montrer aux autres ; ce serait une entreprise aussi folle qu’inutile que celle de vivre heureux dans la société sans le secours de ses associés. Hélas ! Nul d’entre vous, ô mortels, n’est à l’abri des traits du sort. Nul d’entre vous n’est sûr de ne pas boire quelque jour dans la coupe de l’infortune. Nul d’entre vous, dans quelque rang qu’il se trouve, ne peut se passer un instant de l’assistance des autres, soit pour écarter le mal, soit pour obtenir quelque plaisir.\par
{\scshape Aimez, pour être aimé.} Voilà le précepte simple auquel peut se réduire la morale universelle\textsuperscript{3}.\par
{\scshape Peuples que la Nature a répandu sur les différentes contrées de la terre, aimez-vous donc les uns les autres et terminez des combats éternels qui détruisent} à tout moment votre félicité !\par
{\scshape Souverains} ! Aimez vos peuples, et vous trouverez dans leur amour un soutien que rien ne peut ébranler.\par
{\scshape Grands, nobles, riches et puissants de ce monde !} Faites du bien aux hommes, et vous serez vraiment chéris et distingués.\par
{\scshape Sages} et savants ! Éclairez les nations, soyez vraiment utiles ; vous serez considérés et vos illustres noms se transmettront à la postérité.\par
 {\scshape Époux ! Parents ! Amis et maîtres ! Aimez, pour obtenir la tendresse, qui peut seule répandre des charmes sur vos associations diverses.} \par
{\scshape Citoyens} ! Dans vos liaisons habituelles, ne perdez jamais de vue le désir d’être aimé ou de plaire. En vous conformant à des règles si claires, vous jouirez en ce monde de toute la félicité dont la nature humaine est susceptible. Chacun de vous, ô mortels, vivra content sur la terre et n’éprouvera point d’alarmes quand il se verra forcé de la quitter.\par
N3 « Si vis amari, ama. » Sénèque.\par
 \hyperref[footnote107]{\dotuline{ \textsuperscript{107} }} \par
\par
 \hyperref[footnote108]{\dotuline{ \textsuperscript{108} }} \par
\par
 \hyperref[footnote109]{\dotuline{ \textsuperscript{109} }} \par
\par
 \hyperref[footnote110]{\dotuline{ \textsuperscript{110} }} \par
\par
 \hyperref[footnote111]{\dotuline{ \textsuperscript{111} }} \par
\par
 \hyperref[footnote112]{\dotuline{ \textsuperscript{112} }} \par
\par
 \hyperref[footnote113]{\dotuline{ \textsuperscript{113} }} \par
\par
 \hyperref[footnote114]{\dotuline{ \textsuperscript{114} }} \par
\par
 \hyperref[footnote115]{\dotuline{ \textsuperscript{115} }} \par
N9 Quelques anciens philosophes de la secte académique ont reconnu une liaison entre le goût du beau physique et du beau moral, entre l’amour de l’ordre physique et l’amour de la vertu. En effet, l’un et l’autre de ces goûts semblent dépendre de la finesse des organes, qui constitue la sensibilité. Il y a communément lieu de présumer qu’un homme qui néglige l’ordre dans les choses extérieures ou qui est insensible aux beautés physiques, n’a pas une tête bien arrangée. Tout dans la Nature est lié par des chaînons imperceptibles. Il est bien difficile que le bon goût subsiste longtemps sous un gouvernement despotique.\par
 \hyperref[footnote116]{\dotuline{ \textsuperscript{116} }} \par
Voyez Valère Maxime, {\itshape Des Faits et des Paroles mémorables}, livre 7, chap. I, art. 2, édition Torren, Leide, 1726.\par
 \hyperref[footnote117]{\dotuline{ \textsuperscript{117} }} \par
N10 Aristote, dans ses livres moraux, livre VIII, chap. I, dit « qu’une vie solitaire et privée d’associés est contraire à la félicité de l’homme et répugne à sa nature, vu que l’homme par sa nature est un animal {\itshape sociable et politique} ». Il ajoute « qu’un homme qui se plaît dans la solitude et qui fuit le commerce de ses semblables n’est pas un homme, mais un monstre. La solitude doit l’empêcher d’exercer aucune vertu. » Un anonyme très estimable, partant des mêmes principes, a dit : « En s’éloignant des hommes, on s’éloigne des vertus nécessaires à la société. Quand on vit seul on se néglige, on devient farouche, on se livre à son humeur. Le monde nous force à nous observer. » Voir {\itshape Lettres d’une Mère à son Fils sur la vraie Gloire}. Le même Aristote, au I\textsuperscript{er} livre de sa {\itshape Politique}, dit que « celui qui aime une vie complètement isolée n’est pas un homme, mais doit être ou un dieu, ou une brute ».\par
 \hyperref[footnote118]{\dotuline{ \textsuperscript{118} }} \par
N11 Tous les méchants voudraient bien être bons, parce qu’ils éprouvent à tout moment les désagréments attachés à la méchanceté ou au vice. Platon (au V livre des {\itshape Lois}) dit que « tout homme injuste est injuste malgré lui ». Le même philosophe dit dans le {\itshape Timée} : « Personne n’est méchant de plein gré, il l’est par {\itshape une suite de quelque vice de conformation dans son corps ou par l’effet d’une mauvaise éducation}. » On peut dire, d’un autre côté, que l’homme de bien est un être bien constitué et bien élevé qui suit sans résistance une nature bien réglée, qui a contracté sans peine l’habitude d’être bon et qui l’exerce avec promptitude et facilité. Aristote observe très justement que « nous ne recevons aucune des vertus morales de la Nature ; nous devenons, dit-il, bons et justes de la même manière qu’on devient bon architecte ou bon musicien. » La Nature ne nous donne que des dispositions à l’aide desquelles nous sommes plus ou moins propres à devenir bons, justes, bienfaisants, etc. Un homme né sans finesse dans l’oreille, sans agilité dans les doigts, ne deviendra jamais un musicien habile. Le méchant est un être mal organisé, mal élevé, ou en qui l’éducation n’a pu rectifier le vice de sa conformation. De même qu’un mauvais musicien, un mauvais peintre, un sculpteur maladroit voudraient bien exceller dans leurs professions, le méchant rend souvent hommage au mérite de la vertu qu’il n’a pas la force de suivre. Il voudrait être bon mais l’habitude le ramène au vice dont il sent les inconvénients. Ces réflexions peuvent servir à jeter du jour sur la morale et à nous expliquer la conduite de bien des hommes qui font souvent le mal en dépit d’eux-mêmes.\par
 \hyperref[footnote119]{\dotuline{ \textsuperscript{119} }} \par
N12
 


% at least one empty page at end (for booklet couv)
\ifbooklet
  \pagestyle{empty}
  \clearpage
  % 2 empty pages maybe needed for 4e cover
  \ifnum\modulo{\value{page}}{4}=0 \hbox{}\newpage\hbox{}\newpage\fi
  \ifnum\modulo{\value{page}}{4}=1 \hbox{}\newpage\hbox{}\newpage\fi


  \hbox{}\newpage
  \ifodd\value{page}\hbox{}\newpage\fi
  {\centering\color{rubric}\bfseries\noindent\large
    Hurlus ? Qu’est-ce.\par
    \bigskip
  }
  \noindent Des bouquinistes électroniques, pour du texte libre à participation libre,
  téléchargeable gratuitement sur \href{https://hurlus.fr}{\dotuline{hurlus.fr}}.\par
  \bigskip
  \noindent Cette brochure a été produite par des éditeurs bénévoles.
  Elle n’est pas faîte pour être possédée, mais pour être lue, et puis donnée.
  Que circule le texte !
  En page de garde, on peut ajouter une date, un lieu, un nom ; pour suivre le voyage des idées.
  \par

  Ce texte a été choisi parce qu’une personne l’a aimé,
  ou haï, elle a en tous cas pensé qu’il partipait à la formation de notre présent ;
  sans le souci de plaire, vendre, ou militer pour une cause.
  \par

  L’édition électronique est soigneuse, tant sur la technique
  que sur l’établissement du texte ; mais sans aucune prétention scolaire, au contraire.
  Le but est de s’adresser à tous, sans distinction de science ou de diplôme.
  Au plus direct ! (possible)
  \par

  Cet exemplaire en papier a été tiré sur une imprimante personnelle
   ou une photocopieuse. Tout le monde peut le faire.
  Il suffit de
  télécharger un fichier sur \href{https://hurlus.fr}{\dotuline{hurlus.fr}},
  d’imprimer, et agrafer ; puis de lire et donner.\par

  \bigskip

  \noindent PS : Les hurlus furent aussi des rebelles protestants qui cassaient les statues dans les églises catholiques. En 1566 démarra la révolte des gueux dans le pays de Lille. L’insurrection enflamma la région jusqu’à Anvers où les gueux de mer bloquèrent les bateaux espagnols.
  Ce fut une rare guerre de libération dont naquit un pays toujours libre : les Pays-Bas.
  En plat pays francophone, par contre, restèrent des bandes de huguenots, les hurlus, progressivement réprimés par la très catholique Espagne.
  Cette mémoire d’une défaite est éteinte, rallumons-la. Sortons les livres du culte universitaire, cherchons les idoles de l’époque, pour les briser.
\fi

\ifdev % autotext in dev mode
\fontname\font — \textsc{Les règles du jeu}\par
(\hyperref[utopie]{\underline{Lien}})\par
\noindent \initialiv{A}{lors là}\blindtext\par
\noindent \initialiv{À}{ la bonheur des dames}\blindtext\par
\noindent \initialiv{É}{tonnez-le}\blindtext\par
\noindent \initialiv{Q}{ualitativement}\blindtext\par
\noindent \initialiv{V}{aloriser}\blindtext\par
\Blindtext
\phantomsection
\label{utopie}
\Blinddocument
\fi
\end{document}
