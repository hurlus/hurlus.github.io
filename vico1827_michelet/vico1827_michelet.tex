%%%%%%%%%%%%%%%%%%%%%%%%%%%%%%%%%
% LaTeX model https://hurlus.fr %
%%%%%%%%%%%%%%%%%%%%%%%%%%%%%%%%%

% Needed before document class
\RequirePackage{pdftexcmds} % needed for tests expressions
\RequirePackage{fix-cm} % correct units

% Define mode
\def\mode{a4}

\newif\ifaiv % a4
\newif\ifav % a5
\newif\ifbooklet % booklet
\newif\ifcover % cover for booklet

\ifnum \strcmp{\mode}{cover}=0
  \covertrue
\else\ifnum \strcmp{\mode}{booklet}=0
  \booklettrue
\else\ifnum \strcmp{\mode}{a5}=0
  \avtrue
\else
  \aivtrue
\fi\fi\fi

\ifbooklet % do not enclose with {}
  \documentclass[french,twoside]{book} % ,notitlepage
  \usepackage[%
    papersize={105mm, 297mm},
    inner=12mm,
    outer=12mm,
    top=20mm,
    bottom=15mm,
    marginparsep=0pt,
  ]{geometry}
  \usepackage[fontsize=9.5pt]{scrextend} % for Roboto
\else\ifav
  \documentclass[french,twoside]{book} % ,notitlepage
  \usepackage[%
    a5paper,
    inner=25mm,
    outer=15mm,
    top=15mm,
    bottom=15mm,
    marginparsep=0pt,
  ]{geometry}
  \usepackage[fontsize=12pt]{scrextend}
\else% A4 2 cols
  \documentclass[twocolumn]{report}
  \usepackage[%
    a4paper,
    inner=15mm,
    outer=10mm,
    top=25mm,
    bottom=18mm,
    marginparsep=0pt,
  ]{geometry}
  \setlength{\columnsep}{20mm}
  \usepackage[fontsize=9.5pt]{scrextend}
\fi\fi

%%%%%%%%%%%%%%
% Alignments %
%%%%%%%%%%%%%%
% before teinte macros

\setlength{\arrayrulewidth}{0.2pt}
\setlength{\columnseprule}{\arrayrulewidth} % twocol
\setlength{\parskip}{0pt} % classical para with no margin
\setlength{\parindent}{1.5em}

%%%%%%%%%%
% Colors %
%%%%%%%%%%
% before Teinte macros

\usepackage[dvipsnames]{xcolor}
\definecolor{rubric}{HTML}{800000} % the tonic 0c71c3
\def\columnseprulecolor{\color{rubric}}
\colorlet{borderline}{rubric!30!} % definecolor need exact code
\definecolor{shadecolor}{gray}{0.95}
\definecolor{bghi}{gray}{0.5}

%%%%%%%%%%%%%%%%%
% Teinte macros %
%%%%%%%%%%%%%%%%%
%%%%%%%%%%%%%%%%%%%%%%%%%%%%%%%%%%%%%%%%%%%%%%%%%%%
% <TEI> generic (LaTeX names generated by Teinte) %
%%%%%%%%%%%%%%%%%%%%%%%%%%%%%%%%%%%%%%%%%%%%%%%%%%%
% This template is inserted in a specific design
% It is XeLaTeX and otf fonts

\makeatletter % <@@@


\usepackage{blindtext} % generate text for testing
\usepackage[strict]{changepage} % for modulo 4
\usepackage{contour} % rounding words
\usepackage[nodayofweek]{datetime}
% \usepackage{DejaVuSans} % seems buggy for sffont font for symbols
\usepackage{enumitem} % <list>
\usepackage{etoolbox} % patch commands
\usepackage{fancyvrb}
\usepackage{fancyhdr}
\usepackage{float}
\usepackage{fontspec} % XeLaTeX mandatory for fonts
\usepackage{footnote} % used to capture notes in minipage (ex: quote)
\usepackage{framed} % bordering correct with footnote hack
\usepackage{graphicx}
\usepackage{lettrine} % drop caps
\usepackage{lipsum} % generate text for testing
\usepackage[framemethod=tikz,]{mdframed} % maybe used for frame with footnotes inside
\usepackage{pdftexcmds} % needed for tests expressions
\usepackage{polyglossia} % non-break space french punct, bug Warning: "Failed to patch part"
\usepackage[%
  indentfirst=false,
  vskip=1em,
  noorphanfirst=true,
  noorphanafter=true,
  leftmargin=\parindent,
  rightmargin=0pt,
]{quoting}
\usepackage{ragged2e}
\usepackage{setspace} % \setstretch for <quote>
\usepackage{tabularx} % <table>
\usepackage[explicit]{titlesec} % wear titles, !NO implicit
\usepackage{tikz} % ornaments
\usepackage{tocloft} % styling tocs
\usepackage[fit]{truncate} % used im runing titles
\usepackage{unicode-math}
\usepackage[normalem]{ulem} % breakable \uline, normalem is absolutely necessary to keep \emph
\usepackage{verse} % <l>
\usepackage{xcolor} % named colors
\usepackage{xparse} % @ifundefined
\XeTeXdefaultencoding "iso-8859-1" % bad encoding of xstring
\usepackage{xstring} % string tests
\XeTeXdefaultencoding "utf-8"
\PassOptionsToPackage{hyphens}{url} % before hyperref, which load url package

% TOTEST
% \usepackage{hypcap} % links in caption ?
% \usepackage{marginnote}
% TESTED
% \usepackage{background} % doesn’t work with xetek
% \usepackage{bookmark} % prefers the hyperref hack \phantomsection
% \usepackage[color, leftbars]{changebar} % 2 cols doc, impossible to keep bar left
% \usepackage[utf8x]{inputenc} % inputenc package ignored with utf8 based engines
% \usepackage[sfdefault,medium]{inter} % no small caps
% \usepackage{firamath} % choose firasans instead, firamath unavailable in Ubuntu 21-04
% \usepackage{flushend} % bad for last notes, supposed flush end of columns
% \usepackage[stable]{footmisc} % BAD for complex notes https://texfaq.org/FAQ-ftnsect
% \usepackage{helvet} % not for XeLaTeX
% \usepackage{multicol} % not compatible with too much packages (longtable, framed, memoir…)
% \usepackage[default,oldstyle,scale=0.95]{opensans} % no small caps
% \usepackage{sectsty} % \chapterfont OBSOLETE
% \usepackage{soul} % \ul for underline, OBSOLETE with XeTeX
% \usepackage[breakable]{tcolorbox} % text styling gone, footnote hack not kept with breakable


% Metadata inserted by a program, from the TEI source, for title page and runing heads
\title{\textbf{ Principes de la philosophie de l’histoire (trad. Michelet) }}
\date{1827}
\author{Giambattista Vico}
\def\elbibl{Giambattista Vico. 1827. \emph{Principes de la philosophie de l’histoire (trad. Michelet)}}
\def\elsource{Giambattista Vico, jules Michelet, \emph{Principes de la philosophie de l’histoire} ; traduits de la \emph{{\itshape Scienza nuova}} de J. B. Vico et précédés d’un \emph{{\itshape Discours sur le système et la vie de l’auteur}}, par Jules Michelet, Paris, J. Renouard, 1827, VIII-LXXII-392 p. Source : \href{http://gallica.bnf.fr/ark:/12148/bpt6k91444s}{\dotuline{Gallica}}\footnote{\href{http://gallica.bnf.fr/ark:/12148/bpt6k91444s}{\url{http://gallica.bnf.fr/ark:/12148/bpt6k91444s}}}.}

% Default metas
\newcommand{\colorprovide}[2]{\@ifundefinedcolor{#1}{\colorlet{#1}{#2}}{}}
\colorprovide{rubric}{red}
\colorprovide{silver}{lightgray}
\@ifundefined{syms}{\newfontfamily\syms{DejaVu Sans}}{}
\newif\ifdev
\@ifundefined{elbibl}{% No meta defined, maybe dev mode
  \newcommand{\elbibl}{Titre court ?}
  \newcommand{\elbook}{Titre du livre source ?}
  \newcommand{\elabstract}{Résumé\par}
  \newcommand{\elurl}{http://oeuvres.github.io/elbook/2}
  \author{Éric Lœchien}
  \title{Un titre de test assez long pour vérifier le comportement d’une maquette}
  \date{1566}
  \devtrue
}{}
\let\eltitle\@title
\let\elauthor\@author
\let\eldate\@date


\defaultfontfeatures{
  % Mapping=tex-text, % no effect seen
  Scale=MatchLowercase,
  Ligatures={TeX,Common},
}


% generic typo commands
\newcommand{\astermono}{\medskip\centerline{\color{rubric}\large\selectfont{\syms ✻}}\medskip\par}%
\newcommand{\astertri}{\medskip\par\centerline{\color{rubric}\large\selectfont{\syms ✻\,✻\,✻}}\medskip\par}%
\newcommand{\asterism}{\bigskip\par\noindent\parbox{\linewidth}{\centering\color{rubric}\large{\syms ✻}\\{\syms ✻}\hskip 0.75em{\syms ✻}}\bigskip\par}%

% lists
\newlength{\listmod}
\setlength{\listmod}{\parindent}
\setlist{
  itemindent=!,
  listparindent=\listmod,
  labelsep=0.2\listmod,
  parsep=0pt,
  % topsep=0.2em, % default topsep is best
}
\setlist[itemize]{
  label=—,
  leftmargin=0pt,
  labelindent=1.2em,
  labelwidth=0pt,
}
\setlist[enumerate]{
  label={\bf\color{rubric}\arabic*.},
  labelindent=0.8\listmod,
  leftmargin=\listmod,
  labelwidth=0pt,
}
\newlist{listalpha}{enumerate}{1}
\setlist[listalpha]{
  label={\bf\color{rubric}\alph*.},
  leftmargin=0pt,
  labelindent=0.8\listmod,
  labelwidth=0pt,
}
\newcommand{\listhead}[1]{\hspace{-1\listmod}\emph{#1}}

\renewcommand{\hrulefill}{%
  \leavevmode\leaders\hrule height 0.2pt\hfill\kern\z@}

% General typo
\DeclareTextFontCommand{\textlarge}{\large}
\DeclareTextFontCommand{\textsmall}{\small}

% commands, inlines
\newcommand{\anchor}[1]{\Hy@raisedlink{\hypertarget{#1}{}}} % link to top of an anchor (not baseline)
\newcommand\abbr[1]{#1}
\newcommand{\autour}[1]{\tikz[baseline=(X.base)]\node [draw=rubric,thin,rectangle,inner sep=1.5pt, rounded corners=3pt] (X) {\color{rubric}#1};}
\newcommand\corr[1]{#1}
\newcommand{\ed}[1]{ {\color{silver}\sffamily\footnotesize (#1)} } % <milestone ed="1688"/>
\newcommand\expan[1]{#1}
\newcommand\foreign[1]{\emph{#1}}
\newcommand\gap[1]{#1}
\renewcommand{\LettrineFontHook}{\color{rubric}}
\newcommand{\initial}[2]{\lettrine[lines=2, loversize=0.3, lhang=0.3]{#1}{#2}}
\newcommand{\initialiv}[2]{%
  \let\oldLFH\LettrineFontHook
  % \renewcommand{\LettrineFontHook}{\color{rubric}\ttfamily}
  \IfSubStr{QJ’}{#1}{
    \lettrine[lines=4, lhang=0.2, loversize=-0.1, lraise=0.2]{\smash{#1}}{#2}
  }{\IfSubStr{É}{#1}{
    \lettrine[lines=4, lhang=0.2, loversize=-0, lraise=0]{\smash{#1}}{#2}
  }{\IfSubStr{ÀÂ}{#1}{
    \lettrine[lines=4, lhang=0.2, loversize=-0, lraise=0, slope=0.6em]{\smash{#1}}{#2}
  }{\IfSubStr{A}{#1}{
    \lettrine[lines=4, lhang=0.2, loversize=0.2, slope=0.6em]{\smash{#1}}{#2}
  }{\IfSubStr{V}{#1}{
    \lettrine[lines=4, lhang=0.2, loversize=0.2, slope=-0.5em]{\smash{#1}}{#2}
  }{
    \lettrine[lines=4, lhang=0.2, loversize=0.2]{\smash{#1}}{#2}
  }}}}}
  \let\LettrineFontHook\oldLFH
}
\newcommand{\labelchar}[1]{\textbf{\color{rubric} #1}}
\newcommand{\milestone}[1]{\autour{\footnotesize\color{rubric} #1}} % <milestone n="4"/>
\newcommand\name[1]{#1}
\newcommand\orig[1]{#1}
\newcommand\orgName[1]{#1}
\newcommand\persName[1]{#1}
\newcommand\placeName[1]{#1}
\newcommand{\pn}[1]{\IfSubStr{-—–¶}{#1}% <p n="3"/>
  {\noindent{\bfseries\color{rubric}   ¶  }}
  {{\footnotesize\autour{ #1}  }}}
\newcommand\reg{}
% \newcommand\ref{} % already defined
\newcommand\sic[1]{#1}
\newcommand\surname[1]{\textsc{#1}}
\newcommand\term[1]{\textbf{#1}}

\def\mednobreak{\ifdim\lastskip<\medskipamount
  \removelastskip\nopagebreak\medskip\fi}
\def\bignobreak{\ifdim\lastskip<\bigskipamount
  \removelastskip\nopagebreak\bigskip\fi}

% commands, blocks
\newcommand{\byline}[1]{\bigskip{\RaggedLeft{#1}\par}\bigskip}
\newcommand{\bibl}[1]{{\RaggedLeft{#1}\par\bigskip}}
\newcommand{\biblitem}[1]{{\noindent\hangindent=\parindent   #1\par}}
\newcommand{\dateline}[1]{\medskip{\RaggedLeft{#1}\par}\bigskip}
\newcommand{\labelblock}[1]{\medbreak{\noindent\color{rubric}\bfseries #1}\par\mednobreak}
\newcommand{\salute}[1]{\bigbreak{#1}\par\medbreak}
\newcommand{\signed}[1]{\bigbreak\filbreak{\raggedleft #1\par}\medskip}

% environments for blocks (some may become commands)
\newenvironment{borderbox}{}{} % framing content
\newenvironment{citbibl}{\ifvmode\hfill\fi}{\ifvmode\par\fi }
\newenvironment{docAuthor}{\ifvmode\vskip4pt\fontsize{16pt}{18pt}\selectfont\fi\itshape}{\ifvmode\par\fi }
\newenvironment{docDate}{}{\ifvmode\par\fi }
\newenvironment{docImprint}{\vskip6pt}{\ifvmode\par\fi }
\newenvironment{docTitle}{\vskip6pt\bfseries\fontsize{18pt}{22pt}\selectfont}{\par }
\newenvironment{msHead}{\vskip6pt}{\par}
\newenvironment{msItem}{\vskip6pt}{\par}
\newenvironment{titlePart}{}{\par }


% environments for block containers
\newenvironment{argument}{\itshape\parindent0pt}{\vskip1.5em}
\newenvironment{biblfree}{}{\ifvmode\par\fi }
\newenvironment{bibitemlist}[1]{%
  \list{\@biblabel{\@arabic\c@enumiv}}%
  {%
    \settowidth\labelwidth{\@biblabel{#1}}%
    \leftmargin\labelwidth
    \advance\leftmargin\labelsep
    \@openbib@code
    \usecounter{enumiv}%
    \let\p@enumiv\@empty
    \renewcommand\theenumiv{\@arabic\c@enumiv}%
  }
  \sloppy
  \clubpenalty4000
  \@clubpenalty \clubpenalty
  \widowpenalty4000%
  \sfcode`\.\@m
}%
{\def\@noitemerr
  {\@latex@warning{Empty `bibitemlist' environment}}%
\endlist}
\newenvironment{quoteblock}% may be used for ornaments
  {\begin{quoting}}
  {\end{quoting}}

% table () is preceded and finished by custom command
\newcommand{\tableopen}[1]{%
  \ifnum\strcmp{#1}{wide}=0{%
    \begin{center}
  }
  \else\ifnum\strcmp{#1}{long}=0{%
    \begin{center}
  }
  \else{%
    \begin{center}
  }
  \fi\fi
}
\newcommand{\tableclose}[1]{%
  \ifnum\strcmp{#1}{wide}=0{%
    \end{center}
  }
  \else\ifnum\strcmp{#1}{long}=0{%
    \end{center}
  }
  \else{%
    \end{center}
  }
  \fi\fi
}


% text structure
\newcommand\chapteropen{} % before chapter title
\newcommand\chaptercont{} % after title, argument, epigraph…
\newcommand\chapterclose{} % maybe useful for multicol settings
\setcounter{secnumdepth}{-2} % no counters for hierarchy titles
\setcounter{tocdepth}{5} % deep toc
\markright{\@title} % ???
\markboth{\@title}{\@author} % ???
\renewcommand\tableofcontents{\@starttoc{toc}}
% toclof format
% \renewcommand{\@tocrmarg}{0.1em} % Useless command?
% \renewcommand{\@pnumwidth}{0.5em} % {1.75em}
\renewcommand{\@cftmaketoctitle}{}
\setlength{\cftbeforesecskip}{\z@ \@plus.2\p@}
\renewcommand{\cftchapfont}{}
\renewcommand{\cftchapdotsep}{\cftdotsep}
\renewcommand{\cftchapleader}{\normalfont\cftdotfill{\cftchapdotsep}}
\renewcommand{\cftchappagefont}{\bfseries}
\setlength{\cftbeforechapskip}{0em \@plus\p@}
% \renewcommand{\cftsecfont}{\small\relax}
\renewcommand{\cftsecpagefont}{\normalfont}
% \renewcommand{\cftsubsecfont}{\small\relax}
\renewcommand{\cftsecdotsep}{\cftdotsep}
\renewcommand{\cftsecpagefont}{\normalfont}
\renewcommand{\cftsecleader}{\normalfont\cftdotfill{\cftsecdotsep}}
\setlength{\cftsecindent}{1em}
\setlength{\cftsubsecindent}{2em}
\setlength{\cftsubsubsecindent}{3em}
\setlength{\cftchapnumwidth}{1em}
\setlength{\cftsecnumwidth}{1em}
\setlength{\cftsubsecnumwidth}{1em}
\setlength{\cftsubsubsecnumwidth}{1em}

% footnotes
\newif\ifheading
\newcommand*{\fnmarkscale}{\ifheading 0.70 \else 1 \fi}
\renewcommand\footnoterule{\vspace*{0.3cm}\hrule height \arrayrulewidth width 3cm \vspace*{0.3cm}}
\setlength\footnotesep{1.5\footnotesep} % footnote separator
\renewcommand\@makefntext[1]{\parindent 1.5em \noindent \hb@xt@1.8em{\hss{\normalfont\@thefnmark . }}#1} % no superscipt in foot
\patchcmd{\@footnotetext}{\footnotesize}{\footnotesize\sffamily}{}{} % before scrextend, hyperref


%   see https://tex.stackexchange.com/a/34449/5049
\def\truncdiv#1#2{((#1-(#2-1)/2)/#2)}
\def\moduloop#1#2{(#1-\truncdiv{#1}{#2}*#2)}
\def\modulo#1#2{\number\numexpr\moduloop{#1}{#2}\relax}

% orphans and widows
\clubpenalty=9996
\widowpenalty=9999
\brokenpenalty=4991
\predisplaypenalty=10000
\postdisplaypenalty=1549
\displaywidowpenalty=1602
\hyphenpenalty=400
% Copied from Rahtz but not understood
\def\@pnumwidth{1.55em}
\def\@tocrmarg {2.55em}
\def\@dotsep{4.5}
\emergencystretch 3em
\hbadness=4000
\pretolerance=750
\tolerance=2000
\vbadness=4000
\def\Gin@extensions{.pdf,.png,.jpg,.mps,.tif}
% \renewcommand{\@cite}[1]{#1} % biblio

\usepackage{hyperref} % supposed to be the last one, :o) except for the ones to follow
\urlstyle{same} % after hyperref
\hypersetup{
  % pdftex, % no effect
  pdftitle={\elbibl},
  % pdfauthor={Your name here},
  % pdfsubject={Your subject here},
  % pdfkeywords={keyword1, keyword2},
  bookmarksnumbered=true,
  bookmarksopen=true,
  bookmarksopenlevel=1,
  pdfstartview=Fit,
  breaklinks=true, % avoid long links
  pdfpagemode=UseOutlines,    % pdf toc
  hyperfootnotes=true,
  colorlinks=false,
  pdfborder=0 0 0,
  % pdfpagelayout=TwoPageRight,
  % linktocpage=true, % NO, toc, link only on page no
}

\makeatother % /@@@>
%%%%%%%%%%%%%%
% </TEI> end %
%%%%%%%%%%%%%%


%%%%%%%%%%%%%
% footnotes %
%%%%%%%%%%%%%
\renewcommand{\thefootnote}{\bfseries\textcolor{rubric}{\arabic{footnote}}} % color for footnote marks

%%%%%%%%%
% Fonts %
%%%%%%%%%
\usepackage[]{roboto} % SmallCaps, Regular is a bit bold
% \linespread{0.90} % too compact, keep font natural
\newfontfamily\fontrun[]{Roboto Condensed Light} % condensed runing heads
\ifav
  \setmainfont[
    ItalicFont={Roboto Light Italic},
  ]{Roboto}
\else\ifbooklet
  \setmainfont[
    ItalicFont={Roboto Light Italic},
  ]{Roboto}
\else
\setmainfont[
  ItalicFont={Roboto Italic},
]{Roboto Light}
\fi\fi
\renewcommand{\LettrineFontHook}{\bfseries\color{rubric}}
% \renewenvironment{labelblock}{\begin{center}\bfseries\color{rubric}}{\end{center}}

%%%%%%%%
% MISC %
%%%%%%%%

\setdefaultlanguage[frenchpart=false]{french} % bug on part


\newenvironment{quotebar}{%
    \def\FrameCommand{{\color{rubric!10!}\vrule width 0.5em} \hspace{0.9em}}%
    \def\OuterFrameSep{\itemsep} % séparateur vertical
    \MakeFramed {\advance\hsize-\width \FrameRestore}
  }%
  {%
    \endMakeFramed
  }
\renewenvironment{quoteblock}% may be used for ornaments
  {%
    \savenotes
    \setstretch{0.9}
    \normalfont
    \begin{quotebar}
  }
  {%
    \end{quotebar}
    \spewnotes
  }


\renewcommand{\headrulewidth}{\arrayrulewidth}
\renewcommand{\headrule}{{\color{rubric}\hrule}}

% delicate tuning, image has produce line-height problems in title on 2 lines
\titleformat{name=\chapter} % command
  [display] % shape
  {\vspace{1.5em}\centering} % format
  {} % label
  {0pt} % separator between n
  {}
[{\color{rubric}\huge\textbf{#1}}\bigskip] % after code
% \titlespacing{command}{left spacing}{before spacing}{after spacing}[right]
\titlespacing*{\chapter}{0pt}{-2em}{0pt}[0pt]

\titleformat{name=\section}
  [block]{}{}{}{}
  [\vbox{\color{rubric}\large\raggedleft\textbf{#1}}]
\titlespacing{\section}{0pt}{0pt plus 4pt minus 2pt}{\baselineskip}

\titleformat{name=\subsection}
  [block]
  {}
  {} % \thesection
  {} % separator \arrayrulewidth
  {}
[\vbox{\large\textbf{#1}}]
% \titlespacing{\subsection}{0pt}{0pt plus 4pt minus 2pt}{\baselineskip}

\ifaiv
  \fancypagestyle{main}{%
    \fancyhf{}
    \setlength{\headheight}{1.5em}
    \fancyhead{} % reset head
    \fancyfoot{} % reset foot
    \fancyhead[L]{\truncate{0.45\headwidth}{\fontrun\elbibl}} % book ref
    \fancyhead[R]{\truncate{0.45\headwidth}{ \fontrun\nouppercase\leftmark}} % Chapter title
    \fancyhead[C]{\thepage}
  }
  \fancypagestyle{plain}{% apply to chapter
    \fancyhf{}% clear all header and footer fields
    \setlength{\headheight}{1.5em}
    \fancyhead[L]{\truncate{0.9\headwidth}{\fontrun\elbibl}}
    \fancyhead[R]{\thepage}
  }
\else
  \fancypagestyle{main}{%
    \fancyhf{}
    \setlength{\headheight}{1.5em}
    \fancyhead{} % reset head
    \fancyfoot{} % reset foot
    \fancyhead[RE]{\truncate{0.9\headwidth}{\fontrun\elbibl}} % book ref
    \fancyhead[LO]{\truncate{0.9\headwidth}{\fontrun\nouppercase\leftmark}} % Chapter title, \nouppercase needed
    \fancyhead[RO,LE]{\thepage}
  }
  \fancypagestyle{plain}{% apply to chapter
    \fancyhf{}% clear all header and footer fields
    \setlength{\headheight}{1.5em}
    \fancyhead[L]{\truncate{0.9\headwidth}{\fontrun\elbibl}}
    \fancyhead[R]{\thepage}
  }
\fi

\ifav % a5 only
  \titleclass{\section}{top}
\fi

\newcommand\chapo{{%
  \vspace*{-3em}
  \centering % no vskip ()
  {\Large\addfontfeature{LetterSpace=25}\bfseries{\elauthor}}\par
  \smallskip
  {\large\eldate}\par
  \bigskip
  {\Large\selectfont{\eltitle}}\par
  \bigskip
  {\color{rubric}\hline\par}
  \bigskip
  {\Large TEXTE LIBRE À PARTICPATION LIBRE\par}
  \centerline{\small\color{rubric} {hurlus.fr, tiré le \today}}\par
  \bigskip
}}

\newcommand\cover{{%
  \thispagestyle{empty}
  \centering
  {\LARGE\bfseries{\elauthor}}\par
  \bigskip
  {\Large\eldate}\par
  \bigskip
  \bigskip
  {\LARGE\selectfont{\eltitle}}\par
  \vfill\null
  {\color{rubric}\setlength{\arrayrulewidth}{2pt}\hline\par}
  \vfill\null
  {\Large TEXTE LIBRE À PARTICPATION LIBRE\par}
  \centerline{{\href{https://hurlus.fr}{\dotuline{hurlus.fr}}, tiré le \today}}\par
}}

\begin{document}
\pagestyle{empty}
\ifbooklet{
  \cover\newpage
  \thispagestyle{empty}\hbox{}\newpage
  \cover\newpage\noindent Les voyages de la brochure\par
  \bigskip
  \begin{tabularx}{\textwidth}{l|X|X}
    \textbf{Date} & \textbf{Lieu}& \textbf{Nom/pseudo} \\ \hline
    \rule{0pt}{25cm} &  &   \\
  \end{tabularx}
  \newpage
  \addtocounter{page}{-4}
}\fi

\thispagestyle{empty}
\ifaiv
  \twocolumn[\chapo]
\else
  \chapo
\fi
{\it\elabstract}
\bigskip
\makeatletter\@starttoc{toc}\makeatother % toc without new page
\bigskip

\pagestyle{main} % after style

  \section[{Avis du traducteur}]{Avis du traducteur}\phantomsection
\label{avis}\renewcommand{\leftmark}{Avis du traducteur}


\byline{}
\noindent  Les Principes de la Philosophie de l’Histoire dont nous donnons une traduction abrégée, ont pour titre original : {\itshape Cinq Livres sur les principes d’une Science nouvelle, relative à la nature commune des nations, par Jean-Baptiste Vico, ouvrage dédié à S. S. (Clément XII)}. Trois éditions ont été faites du vivant de l’auteur, dans les années 1725, 1730, et 1744. La dernière est celle qu’on a réimprimée le plus souvent, et que nous avons suivie.\par
\emph{« Ce livre, disait Monti, est une montagne aride et sauvage qui recèle des mines d’or. »} La comparaison manque de justesse. Si l’on voulait la suivre, on pourrait accuser dans la Science nouvelle, non pas l’aridité, mais bien  un luxe de végétation. Le génie impétueux de Vico l’a surchargée à chaque édition d’une foule de répétitions sous lesquelles disparaît l’unité du dessein de l’ouvrage. Rendre sensible cette unité, telle devait être la pensée de celui qui au bout d’un siècle venait offrir à un public français un livre si éloigné par la singularité de sa forme des idées de ses contemporains. Il ne pouvait atteindre ce but qu’en supprimant, abrégeant ou transposant les passages qui en reproduisaient d’autres sous une forme moins heureuse, ou qui semblaient appelés ailleurs par la liaison des idées. Il a fallu encore écarter quelques paradoxes bizarres, quelques étymologies forcées, qui ont jusqu’ici décrédité les vérités innombrables que contient la Science nouvelle. Mais on a indiqué dans l’appendice du discours préliminaire les passages de quelque importance qui ont été abrégés ou retranchés. Le jour n’est pas loin sans doute où, le nom de Vico ayant pris enfin la place qui lui est due, un intérêt historique s’étendra sur tout ce qu’il a écrit, et où ses  erreurs ne pourront faire tort à sa gloire ; mais ce temps n’est pas encore venu.\par
\par
On trouvera dans le discours et dans l’appendice qui le suit une vie complète de Vico. Le mémoire qu’il a lui-même écrit sur sa vie ne va que jusqu’à la publication de son grand ouvrage. Nous avons abrégé ce morceau, en élaguant toutes les idées qu’on devait retrouver dans la {\itshape Science nouvelle}, mais nous y avons ajouté de nouveaux détails, tirés des opuscules et des lettres de Vico, ou conservés par la tradition.\par
Plusieurs personnes nous ont prodigué leurs secours et leurs conseils. Nous regrettons qu’il ne nous soit pas permis de les nommer toutes.\par
M. le chevalier de Angelis, auteur de travaux inédits sur Vico, a bien voulu nous communiquer la plupart des ouvrages italiens que nous avons extraits ou cités ; exemple trop rare de cette libéralité d’esprit qui met tout en commun  entre ceux qui s’occupent des mêmes matières. On ne peut reconnaître une bonté si désintéressée, mais rien n’en efface le souvenir.\par
Des avocats distingués, MM. Renouard, Cœuret de Saint-Georges et Foucart, ont éclairé le traducteur sur plusieurs questions de droit. Mais il a été principalement soutenu dans son travail par M. Poret, professeur au collège de Sainte-Barbe. Si cette première traduction française de la Science nouvelle, résolvait d’une manière satisfaisante les nombreuses difficultés que présente l’original, elle le devrait en grande partie au zèle infatigable de son amitié.
\section[{Discours sur le système et la vie de Vico}]{Discours sur le système et la vie de Vico}\phantomsection
\label{discours}\renewcommand{\leftmark}{Discours sur le système et la vie de Vico}

\noindent  Dans la rapidité du mouvement critique imprimé à la philosophie par Descartes, le public ne pouvait remarquer quiconque restait hors de ce mouvement. Voilà pourquoi le nom de Vico est encore si peu connu en-deçà des Alpes. Pendant que la foule suivait ou combattait la réforme cartésienne, un génie solitaire fondait la philosophie de l’histoire. N’accusons pas l’indifférence des contemporains de Vico ; essayons plutôt de l’expliquer, et de montrer que {\itshape La Science nouvelle} n’a été si négligée pendant le dernier siècle que parce qu’elle s’adressait au nôtre.\par
Telle est la marche naturelle de l’esprit humain : connaître d’abord et ensuite juger, s’étendre dans le monde extérieur et rentrer plus tard en soi-même, s’en rapporter au sens commun et le soumettre à l’examen du sens individuel. Cultivé dans la première période par la religion, par la poésie et les  arts, il accumule les faits dont la philosophie doit un jour faire usage. Il a déjà le sentiment de bien des vérités, il n’en a pas encore la science. Il faut qu’un Socrate, un Descartes, viennent lui demander de quel droit il les possède, et que les attaques opiniâtres d’un impitoyable scepticisme l’obligent de se les approprier en les défendant. L’esprit humain, ainsi inquiété dans la possession des croyances qui touchent de plus près son être, dédaigne quelque temps toute connaissance que le sens intime ne peut lui attester ; mais dès qu’il sera rassuré, il sortira du monde intérieur avec des forces nouvelles pour reprendre l’étude des faits historiques : en continuant de chercher le vrai il ne négligera plus le vraisemblable, et la philosophie, comparant et rectifiant l’un par l’autre le sens individuel et le sens commun, embrassera dans l’étude de l’homme celle de l’humanité tout entière.\par
Cette dernière époque commence pour nous. Ce qui nous distingue éminemment, c’est, comme nous disons aujourd’hui, notre {\itshape tendance historique}. Déjà nous voulons que les faits soient vrais dans leurs moindres détails ; le même amour de la vérité doit nous conduire à en chercher les rapports, à observer les lois qui les régissent, à examiner enfin si l’histoire ne peut être ramenée à une forme scientifique.\par
 Ce but dont nous approchons tous les jours, le génie prophétique de Vico nous l’a marqué longtemps d’avance. Son système nous apparaît au commencement du dernier siècle, comme une admirable protestation de cette partie de l’esprit humain qui se repose sur la sagesse du passé conservée dans les religions, dans les langues et dans l’histoire, sur cette sagesse vulgaire, mère de la philosophie, et trop souvent méconnue d’elle. Il était naturel que cette protestation partît de l’Italie. Malgré le génie subtil des Cardan et des Jordano Bruno, le scepticisme n’y étant point réglé par la Réforme dans son développement, n’avait pu y obtenir un succès durable ni populaire. Le passé, lié tout entier à la cause de la religion, y conservait son empire. L’Église catholique invoquait sa perpétuité contre les protestants, et par conséquent recommandait l’étude de l’histoire et des langues. Les sciences qui, au moyen âge, s’étaient réfugiées et confondues dans le sein de la religion, avaient ressenti en Italie moins que partout ailleurs les bons et les mauvais effets de la division du travail ; si la plupart avaient fait moins de progrès, toutes étaient restée unies. L’Italie méridionale particulièrement conservait ce goût d’universalité, qui avait caractérisé le génie de la Grande-Grèce. Dans l’antiquité, l’école pythagoricienne avait allié la métaphysique  et la géométrie, la morale et la politique, la musique et la poésie. Au treizième siècle, l’{\itshape ange de l’école} avait parcouru le cercle des connaissances humaines pour accorder les doctrines d’Aristote avec celles de l’Église. Au dix-septième enfin, les jurisconsultes du royaume de Naples restaient seuls fidèles à cette définition antique de la jurisprudence : \emph{{\itshape scientia rerum divinarum atque humanarum}}. C’était dans une telle contrée qu’on devait tenter pour la première fois de fondre toutes les connaissances qui ont l’homme pour objet dans un vaste système, qui rapprocherait l’une de l’autre l’histoire des faits et celle des langues, en les éclairant toutes deux par une critique nouvelle, et qui accorderait la philosophie et l’histoire, la science et la religion.\par
\par
Néanmoins, on aurait peine à comprendre ce phénomène, si Vico lui-même ne nous avait fait connaître quels travaux préparèrent la conception de son système ({\itshape Vie de Vico écrite par lui-même}). Les détails que l’on va lire sont tirés de cet inestimable monument ; ceux qui ne pouvaient entrer ici ont été rejetés dans l’appendice du discours.\par
{\scshape Jean-Baptiste Vico}, né à Naples, d’un pauvre libraire, en 1668, reçut l’éducation du temps ; c’était l’étude des langues anciennes, de la scholastique,  de la théologie et de la jurisprudence. Mais il aimait trop les généralités, pour s’occuper avec goût de la pratique du droit. Il ne plaida qu’une fois, pour défendre son père, gagna sa cause, et renonça au barreau ; il avait alors seize ans. Peu de temps après, la nécessité l’obligea de se charger d’enseigner le droit aux neveux de l’évêque d’Ischia. Retiré pendant neuf années dans la belle solitude de Vatolla, il suivit en liberté la route que lui traçait son génie, et se partagea entre la poésie, la philosophie et la jurisprudence. Ses maîtres furent les jurisconsultes romains, le divin Platon, et ce Dante avec lequel il avait lui-même tant de rapport par son caractère mélancolique et ardent. On montre encore la petite bibliothèque d’un couvent où il travaillait, et où il conçut peut-être la première idée de {\itshape La Science nouvelle}.\par

\begin{quoteblock}
 \noindent « Lorsque Vico revint à Naples (c’est lui-même qui parle), il se vit comme étranger dans sa patrie. La philosophie n’était plus étudiée que dans les {\itshape Méditations} de Descartes, et dans son {\itshape Discours sur la méthode}, où il désapprouve la culture de la poésie, de l’histoire et de l’éloquence. Le platonisme, qui au seizième siècle les avait si heureusement inspirées, qui pour ainsi dire, avait alors ressuscité la Grèce antique en Italie, était relégué dans la poussière des cloîtres. Pour le droit, les  commentateurs modernes étaient préférés aux interprètes anciens. La poésie corrompue par l’afféterie, avait cessé de puiser aux torrents de Dante, aux limpides ruisseaux de Pétrarque. On cultivait même peu la langue latine. Les sciences, les lettres étaient également languissantes. »
 \end{quoteblock}

\noindent C’est que les peuples, pas plus que les individus, n’abdiquent impunément leur originalité. Le génie italien voulait suivre l’impulsion philosophique de la France et de l’Angleterre, et il s’annulait lui-même. Un esprit vraiment italien ne pouvait se soumettre à cette autre invasion de l’Italie par les étrangers. Tandis que tout le siècle tournait des yeux avides vers l’avenir, et se précipitait dans les routes nouvelles que lui ouvrait la philosophie, Vico eut le courage de remonter vers cette antiquité si dédaignée, et de s’identifier avec elle. Il ferma les commentateurs et les critiques, et se mit à étudier les originaux, comme on l’avait fait à la renaissance des lettres.\par
Fortifié par ces études profondes, il osa attaquer le cartésianisme, non-seulement dans sa partie dogmatique qui conservait peu de crédit, mais aussi dans sa méthode que ses adversaires même avaient embrassée, et par laquelle il régnait sur l’Europe. Il faut voir dans le discours où il compare la méthode d’enseignement suivie par les  modernes à celle des anciens\footnote{Il y propose le problème suivant : \emph{{\itshape Ne pourrait-on pas animer d’un même esprit tout le savoir divin et humain, de sorte que les sciences se donnassent la main, pour ainsi dire, et qu’une université d’aujourd’hui représentât un Platon ou un Aristote, avec tout le savoir que nous avons de plus que les anciens} ?}}, avec quelle sagacité il marque les inconvénients de la première. Nulle part les abus de la nouvelle philosophie n’ont été attaqués avec plus de force et de modération : l’éloignement pour les études historiques, le dédain du sens commun de l’humanité, la manie de réduire en art ce qui doit être laissé à la prudence individuelle, l’application de la méthode géométrique aux choses qui comportent le moins une démonstration rigoureuse, etc. Mais en même temps ce grand esprit, loin de se ranger parmi les détracteurs aveugles de la réforme cartésienne, en reconnaît hautement le bienfait : il voyait de trop haut pour se contenter d’aucune solution incomplète : \emph{« Nous devons beaucoup à Descartes qui a établi le sens individuel pour règle du vrai ; c’était un esclavage trop avilissant, que de faire tout reposer sur l’autorité. Nous lui devons beaucoup pour avoir voulu soumettre la pensée à la méthode ; l’ordre des scolastiques n’était qu’un désordre. Mais vouloir que le jugement de l’individu règne seul, vouloir tout assujettir à la méthode géométrique, c’est tomber dans l’excès opposé. Il serait temps  désormais de prendre un moyen terme ; de suivre le jugement individuel, mais avec les égards dus à l’autorité ; d’employer la méthode, mais une méthode diverse selon la nature des choses\footnote{{\itshape Réponse à un article du journal littéraire d’Italie} où l’on attaquait le livre {\itshape De antiquissimâ Italorum sapientiâ ex originibus linguæ latinæ cruendâ}. 1711.}. »}\par
Celui qui assignait à la vérité le double {\itshape criterium} du sens individuel et du sens commun, se trouvait dès lors dans une route à part. Les ouvrages qu’il a publiés depuis, n’ont plus un caractère polémique. Ce sont des discours publics, des opuscules, où il établit séparément les opinions diverses qu’il devait plus tard réunir dans son grand système. L’un de ces opuscules est intitulé : {\itshape Essai d’un système de jurisprudence, dans lequel le droit civil des Romains serait expliqué par les révolutions de leur gouvernement}. Dans un autre, il entreprend de prouver que {\itshape la sagesse italienne des temps les plus reculés peut se découvrir dans les étymologies latines}. C’est un traité complet de métaphysique, trouvé dans l’histoire d’une langue\footnote{Cet ouvrage est le seul dont Vico n’ait point transporté les idées dans la {\itshape Science nouvelle}. Nous en donnerons prochainement une traduction.}. On peut néanmoins faire sur ces premiers travaux de Vico une observation qui montre tout le chemin qu’il avait encore à parcourir pour arriver à {\itshape La Science nouvelle} : c’est qu’il rapporte la sagesse de la jurisprudence romaine,  et celle qu’il découvre dans la langue des anciens Italiens, au génie des jurisconsultes ou des philosophes, au lieu de l’expliquer, comme il le fit plus tard, par la sagesse instinctive que Dieu donne aux nations. Il croit encore que la civilisation italienne, que la législation romaine, ont été importées en Italie, de l’Égypte ou de la Grèce.\par
Jusqu’en 1719, l’unité manqua aux recherches de Vico ; ses auteurs favoris avaient été jusque-là Platon, Tacite et Bacon, et aucun d’eux ne pouvait la lui donner : \emph{« Le second considère l’homme tel qu’il est, le premier tel qu’il doit être ; Platon contemple l’honnête avec la sagesse spéculative, Tacite observe l’utile avec la sagesse pratique. Bacon réunit ces deux caractères ({\itshape cogitare, videre}). Mais Platon cherche dans la sagesse vulgaire d’Homère, un ornement plutôt qu’une base pour sa philosophie ; Tacite disperse la sienne à la suite des événements ; Bacon dans ce qui regarde les lois ne fait pas assez abstraction des temps et des lieux pour atteindre aux plus hautes généralités. Grotius a un mérite qui leur manque ; il enferme dans son système de droit universel la philosophie et la théologie, en les appuyant toutes deux sur l’histoire des faits, vrais ou fabuleux, et sur celle des langues. »}\par
La lecture de Grotius fixa ses idées et détermina  la conception de son système. Dans un discours prononcé en 1719, il traita le sujet suivant : \emph{« Les éléments de tout le savoir divin et humain peuvent se réduire à trois, {\itshape connaître, vouloir, pouvoir}. Le principe unique en est l’intelligence. L’œil de l’intelligence, c’est-à-dire la raison, reçoit de Dieu la lumière du vrai éternel. Toute science vient de Dieu, retourne à Dieu, est en Dieu\footnote{{\itshape Omnis divinæ atque humanæ eruditionis elementa tria, nosse, velle, posse : quorum principium unum mens ; cujus oculus ratio, cui æterni veri lumen præbet Deus…… — Hæc tria elementa, quæ tam existere, et nostra esse, quàm nos vivere certò scimus, unâ illâ re, de quâ omninò dubitare non possumus, nimirùm cogitatione explicemus : quod quò faciliùs faciamus, hanc tractationem universam divido in partes tres : in quarum primâ omnia scientiarum principia à Deo esse : in secundâ, divinum lumen, sive æternum verum per hæc tria, quæ proposuimus elementa omnes scientias permeare : easque omnes unâ arctissimâ complexione colligatas alias in alias dirigere, et cunctas ad Deum ipsarum principium revocare : in tertiâ, quidquid usquàm de divinæ ac humanæ eruditionis principiis scriptum, dictumve sit, quod cum his principiis congruerit, verum ; quod dissenserit, falsum esse demonstremus. Atque adeò de divinarum atque humanarum rerum notitiâ hæc agam tria, de origine, de circulo, de constantiâ ; et ostendam, origine, omnes à Deo provenire ; circulo, ad Deum redire omnes ; constantiâ, omnes constare in Deo, omnesque eas ipsas præter Deum tenebras esse et errores.}}. »} Et il se chargeait de prouver la fausseté de tout ce qui s’écarterait de cette doctrine. C’était, disaient quelques-uns, promettre plus que Pic de la Mirandole, quand il afficha ses thèses {\itshape de omni scibili}. En effet Vico n’avait pu dans un discours montrer que la partie philosophique de son système, et avait été obligé  d’en supprimer les preuves, c’est-à-dire toute la partie philologique. S’étant mis ainsi dans l’heureuse nécessité d’exposer toutes ses idées, il ne tarda pas à publier deux essais intitulés : {\itshape Unité de principe du droit universel}, 1720 ; — {\itshape Harmonie de la science du jurisconsulte} ({\itshape De constantiâ jurisprudentis}), c’est-à-dire, accord de la philosophie et de la philologie, 1721. Peu après (1722) il fit paraître des notes sur ces deux ouvrages, dans lesquels il appliquait à Homère la critique nouvelle dont il y avait exposé les principes.\par
Cependant ces opuscules divers ne formaient pas un même corps de doctrine ; il entreprit de les fondre en un seul ouvrage qui parut, en 1725, sous le titre de : {\itshape Principes d’une science nouvelle, relative à la nature commune des nations, au moyen desquels on découvre de nouveaux principes du droit naturel des gens}. Cette première édition de {\itshape La Science nouvelle}, est aussi le dernier mot de l’auteur, si l’on considère le fond des idées. Mais il en a entièrement changé la forme dans les autres éditions publiées de son vivant. Dans la première, il suit encore une marche analytique\footnote{Vico a très bien marqué lui-même les progrès de sa méthode : \emph{« Ce qui me déplaît dans mes livres sur le droit universel ({\itshape De juris uno principio}, et {\itshape De constantiâ jurisprudentis}), c’est que j’y pars des idées de Platon et d’autres grands philosophes, pour descendre à l’examen des intelligences bornées et stupides des premiers hommes qui fondèrent l’humanité païenne ; tandis que j’aurais dû suivre une marche toute contraire. De là les erreurs où je suis tombé dans certaines matières... — Dans la première édition de la Science nouvelle, j’errais, sinon dans la matière, au moins dans l’ordre que je suivais. Je traitais des principes des idées, en les séparant des principes des langues, qui sont naturellement unis entre eux. Je parlais de la méthode propre à la Science nouvelle, en la séparant des principes des idées et des principes des langues. »} {\itshape Additions à une préface de la Science nouvelle, publiées avec d’autres pièces inédites de Vico, par M. Antonio Giordano}, 1818. Ajoutons à cette critique, que, dans la première édition, il conçoit pour l’humanité l’espoir d’une perfection stationnaire. Cette idée, que tant d’autres philosophes devaient reproduire, ne reparaît plus dans les éditions suivantes.}. Elle est infiniment  supérieure pour la clarté. Néanmoins c’est dans celles de 1730 et de 1744 que l’on a toujours cherché de préférence le génie de Vico. Il y débute par des axiomes, en déduit toutes les idées particulières et s’efforce de suivre une méthode géométrique que le sujet ne comporte pas toujours. Malgré l’obscurité qui en résulte, malgré l’emploi continuel d’une terminologie bizarre que l’auteur néglige souvent d’expliquer, il y a dans l’ensemble du système, présenté de cette manière, une grandeur imposante, et une sombre poésie qui fait penser à celle de Dante. Nous avons traduit en l’abrégeant l’édition de 1744 ; mais, dans l’exposé du système que l’on va lire, nous nous sommes souvent rapprochés de la méthode que l’auteur avait suivie dans la première, et qui nous a paru convenir davantage à un public français.\par
\par
 Dans cette variété infinie d’actions et de pensées, de mœurs et de langues que nous présente l’histoire de l’homme, nous retrouvons souvent les mêmes traits, les mêmes caractères. Les nations les plus éloignées par les temps et par les lieux suivent dans leurs révolutions politiques, dans celles du langage, une marche singulièrement analogue. Dégager les phénomènes réguliers des accidentels, et déterminer les lois générales qui régissent les premiers ; tracer l’histoire universelle, éternelle, qui se produit dans le temps sous la forme des histoires particulières, décrire le cercle idéal dans lequel tourne le monde réel, voilà l’objet de la nouvelle science. Elle est tout à la fois la philosophie et l’histoire de l’humanité.\par
Elle tire son unité de la religion, principe producteur et conservateur de la société. Jusqu’ici on n’a parlé que de théologie naturelle ; la Science nouvelle est une théologie sociale, une démonstration historique de la Providence, une histoire des décrets par lesquels, à l’insu des hommes et souvent malgré eux, elle a gouverné la grande cité du genre humain. Qui ne ressentira un divin plaisir en ce corps mortel, lorsque nous contemplerons ce monde des nations, si varié de caractères, de temps  et de lieux, dans l’uniformité des idées divines ?\par
Les autres sciences s’occupent de diriger l’homme et de le perfectionner ; mais aucune n’a encore pour objet la connaissance des principes de la civilisation d’où elles sont toutes sorties. La science qui nous révélerait ces principes, nous mettrait à même de mesurer la carrière que parcourent les peuples dans leurs progrès et leur décadence, de calculer les âges de la vie des nations. Alors on connaîtrait les moyens par lesquels une société peut s’élever ou se ramener au plus haut degré de civilisation dont elle soit susceptible, alors seraient accordées la théorie et la pratique, les savants et les sages, les philosophes et les législateurs, la sagesse de réflexion avec la sagesse instinctive ; et l’on ne s’écarterait des principes de cette science de l’{\itshape humanisation}, qu’en abdiquant le caractère d’homme, et se séparant de l’humanité.\par
\par
La Science nouvelle puise à deux sources : la philosophie, la philologie. La philosophie contemple le vrai par la raison ; la philologie observe le réel ; c’est la science des faits et des langues. La philosophie doit appuyer ses théories sur la certitude des faits ; la philologie emprunter à la philosophie ses théories pour élever les faits au caractère de vérités universelles éternelles.\par
Quelle philosophie sera féconde ? celle qui relèvera,  qui dirigera l’homme déchu et toujours débile, sans l’arracher à sa nature, sans l’abandonner à sa corruption. Ainsi nous fermons l’école de la Science nouvelle aux stoïciens qui veulent la mort des sens, aux épicuriens qui font des sens la règle de l’homme ; ceux-là s’enchaînent au destin, ceux-ci s’abandonnent au hasard ; les uns et les autres nient la Providence. Ces deux doctrines isolent l’homme, et devraient s’appeler philosophies {\itshape solitaires}. Au contraire, nous admettons dans notre école les philosophes politiques, et surtout les platoniciens, parce qu’ils sont d’accord avec tous les législateurs sur nos trois principes fondamentaux : existence d’une Providence divine, nécessité de modérer les passions et d’en faire des vertus humaines, immortalité de l’âme. Ces trois vérités philosophiques répondent à autant de faits historiques : institution universelle des religions, des mariages et des sépultures. Toutes les nations ont attribué à ces trois choses un caractère de sainteté ; elles les ont appelées \emph{{\itshape humanitatis commercia}} (Tacite), et par une expression plus sublime encore, \emph{{\itshape fœdera generis humani}}.\par
La philologie, science du réel, science des faits historiques et des langues, fournira les matériaux à la science du vrai, à la philosophie. Mais le réel, ouvrage de la liberté de l’individu, est incertain de sa nature. Quel sera le {\itshape criterium}, au moyen duquel  nous découvrirons dans sa mobilité le caractère immuable du vrai ?… le sens commun, c’est-à-dire le jugement irréfléchi d’une classe d’homme, d’un peuple, de l’humanité ; l’accord général du sens commun des peuples constitue la sagesse du genre humain. Le sens commun, la sagesse vulgaire, est la règle que Dieu a donnée au monde social.\par
Cette sagesse est une sous la double forme des actions et des langues, quelque variées qu’elles puissent être par l’influence des causes locales, et son unité leur imprime un caractère analogue chez les peuples les plus isolés. Ce caractère est surtout sensible dans tout ce qui touche le droit naturel. Interrogez tous les peuples sur les idées qu’ils se font des rapports sociaux, vous verrez qu’ils les comprennent tous de même sous des expressions diverses ; on le voit dans les proverbes qui sont les maximes de la sagesse vulgaire. N’essayons pas d’expliquer cette uniformité du droit naturel en supposant qu’un peuple l’a communiqué à tous les autres. Partout il est indigène, partout il a été fondé par la Providence dans les mœurs des nations.\par
Cette identité de la pensée humaine, reconnue dans les actions et dans le langage, résout le grand problème de la sociabilité de l’homme, qui a tant embarrassé les philosophes ; et si l’on ne trouvait point le nœud délié, nous pourrions le trancher  d’un mot : {\itshape Nulle chose ne reste longtemps hors de son état naturel ; l’homme est sociable, puisqu’il reste en société}.\par
Dans le développement de la société humaine, dans la marche de la civilisation, on peut distinguer trois âges, trois périodes ; âge divin ou théocratique, âge héroïque, âge humain ou civilisé. À cette division répond celle des temps obscur, fabuleux, historique. C’est surtout dans l’histoire des langues que l’exactitude de cette classification est manifeste. Celle que nous parlons a dû être précédée par une langue métaphorique et poétique et celle-ci par une langue hiéroglyphique ou sacrée.\par
Nous nous occuperons principalement des deux premières périodes. Les causes de cette civilisation dont nous sommes si fiers, doivent être recherchées dans les âges que nous nommons barbares, et qu’il serait mieux d’appeler religieux et poétiques ; toute la sagesse du genre humain y était déjà, dans son ébauche et dans son germe. Mais lorsque nous essayons de remonter vers des temps si loin de nous, que de difficultés nous arrêtent ! La plupart des monuments ont péri, et ceux mêmes qui nous restent ont été altérés, dénaturés par les préjugés des âges suivants. Ne pouvant expliquer les origines de la société, et ne se résignant point à les ignorer, on s’est représenté la barbarie antique  d’après la civilisation moderne. Les vanités nationales ont été soutenues par la vanité des savants qui mettent leur gloire à reculer l’origine de leurs sciences favorites. Frappé de l’heureux instinct qui guida les premiers hommes, on s’est exagéré leurs lumières, et on leur a fait honneur d’une sagesse qui était celle de Dieu. Pour nous, persuadés qu’en toute chose les commencements sont simples et grossiers, nous regarderons les Zoroastre, les Hermès et les Orphées moins comme les auteurs que comme les produits et les résultats de la civilisation antique, et nous rapporterons l’origine de la société païenne au sens commun qui rapprocha les uns des autres les hommes encore stupides des premiers âges.\par
Les fondateurs de la société sont pour nous ces cyclopes dont parle Homère, ces géants par lesquels commence l’histoire profane aussi bien que l’histoire sacrée. Après le déluge, les premiers hommes, excepté les patriarches ancêtres du peuple de Dieu, durent revenir à la vie sauvage, et par l’effet de l’éducation la plus dure, reprirent la taille gigantesque des hommes antédiluviens. (\emph{{\itshape Nudi ac sordidi in hos artus, in hæc corpora, quæ miramur, excrescunt.}{\scshape Taciti} {\itshape  Germania.}})\par
Ils s’étaient dispersés dans la vaste forêt qui couvrait la terre, tout entiers aux besoins physiques,  farouches, sans loi, sans Dieu. En vain la nature les environnait de merveilles ; plus les phénomènes étaient réguliers, et par conséquent dignes d’admiration, plus l’habitude les leur rendait indifférents. Qui pouvait dire comment s’éveillerait la pensée humaine ?… Mais le tonnerre s’est fait entendre, ses terribles effets sont remarqués ; les géants effrayés reconnaissent la première fois une puissance supérieure, et la nomment Jupiter ; ainsi dans les traditions de tous les peuples, {\itshape Jupiter terrasse les géants}. C’est l’origine de l’idolâtrie, fille de la crédulité, et non de l’imposture, comme on l’a tant répété.\par
L’idolâtrie fut nécessaire au monde, {\itshape sous le rapport social} : quelle autre puissance que celle d’une religion pleine de terreurs, aurait dompté le stupide orgueil de la force, qui jusque-là isolait les individus ? — {\itshape sous le rapport religieux} : ne fallait-il pas que l’homme passât par cette religion des sens, pour arriver à celle de la raison, et de celle-ci à la religion de la foi ?\par
Mais comment expliquer ce premier pas de l’esprit humain, ce passage critiqué de la brutalité à l’humanité ? Comment dans un état de civilisation aussi avancé que le nôtre, lorsque les esprits ont acquis par l’usage des langues, de l’écriture et du calcul, une habitude invincible d’abstraction, nous replacer dans l’imagination de ces premiers hommes  plongés tout entiers dans les sens, et comme ensevelis dans la matière ? Il nous reste heureusement sur l’enfance de l’espèce et sur ses premiers développements le plus certain, le plus naïf de tous les témoignages : c’est l’enfance de l’individu.\par
L’enfant admire tout, parce qu’il ignore tout. Plein de mémoire, imitateur au plus haut degré, son imagination est puissante en proportion de son incapacité d’abstraire. Il juge de tout d’après lui-même, et suppose la volonté partout où il voit le mouvement.\par
Tels furent les premiers hommes. Ils firent de toute la nature un vaste corps animé, passionné comme eux. Ils parlaient souvent par signes ; ils pensèrent que les éclairs et la foudre étaient les signes de cet être terrible. De nouvelles observations multiplièrent les signes de Jupiter, et leur réunion composa une langue mystérieuse, par laquelle il daignait faire connaître aux hommes ses volontés. L’intelligence de cette langue devint une science, sous les noms de divination, théologie mystique, mythologie, muse.\par
Peu à peu tous les phénomènes de la nature, tous les rapports de la nature à l’homme, ou des hommes entre eux devinrent autant de divinités. Prêter la vie aux êtres inanimés, prêter un corps aux choses immatérielles, composer des êtres qui n’existent complètement dans aucune réalité, voilà  la triple création du monde fantastique de l’idolâtrie. Dieu dans sa pure intelligence, crée les êtres par cela qu’il les connaît ; les premiers hommes, puissants de leur ignorance, créaient à leur manière par la force d’une imagination, si je puis le dire, toute matérielle. {\itshape Poète} veut dire {\itshape créateur} ; ils étaient donc poètes, et telle fut la sublimité de leurs conceptions qu’ils s’en épouvantèrent eux-mêmes, et tombèrent tremblants devant leur ouvrage. (\emph{{\itshape Fingunt simul creduntque.}{\scshape Tacite}.})\par
C’est pour cette poésie {\itshape divine} qui créait et expliquait le monde invisible, qu’on inventa le nom de {\itshape sagesse}, revendiqué ensuite par la philosophie. En effet la poésie était déjà pour les premiers âges une philosophie sans abstraction, toute d’imagination et de sentiment. Ce que les philosophes {\itshape comprirent} dans la suite, les poètes l’avaient {\itshape senti} ; et si, comme le dit l’école, {\itshape rien n’est dans l’intelligence qui n’ait été dans le sens}, les poètes furent le {\itshape sens} du genre humain, les philosophes en furent l’{\itshape intelligence}\footnote{\emph{{\itshape Philosophie est une poésie sophistiquée.}} Montaigne ; III v., p. 216 édit. Lefebvre.}.\par
Les signes par lesquels les hommes commencèrent à exprimer leurs pensées, furent les objets mêmes qu’ils avaient divinisés. Pour dire {\itshape la mer}, ils la montraient de la main ; plus tard ils dirent  {\itshape Neptune}. C’est la {\itshape langue des dieux} dont parle Homère. Les noms des trente mille dieux latins recueillis par Varron, ceux des Grecs non moins nombreux, formaient le vocabulaire {\itshape divin} de ces deux peuples. Originairement la langue {\itshape divine} ne pouvant se parler que par actions, presque toute action était consacrée ; la vie n’était pour ainsi dire qu’une suite d’{\itshape actes muets de religion}. De là restèrent dans la jurisprudence romaine, les {\itshape acta legitima}, cette pantomime qui accompagnait toutes les transactions civiles. Les hiéroglyphes furent l’écriture propre à cette langue imparfaite, loin qu’ils aient été inventés par les philosophes pour y cacher les mystères d’une sagesse profonde. Toutes les nations barbares ont été forcées de commencer ainsi, en attendant qu’elles se formassent un meilleur système de langage et d’écriture. Cette langue muette convenait à un âge où dominaient les religions ; elles veulent être respectées, plutôt que {\itshape raisonnées}.\par
Dans l’âge {\itshape héroïque}, la langue {\itshape divine} subsistait encore, la langue {\itshape humaine} ou articulée commençait ; mais cet âge en eut de plus une qui lui fut propre ; je parle des emblèmes, des devises, nouveau genre de signes qui n’ont qu’un rapport indirect à la pensée. C’est cette langue que {\itshape parlent} les armes des héros ; elle est restée celle de la discipline  militaire. Transportée dans la langue articulée, elle dut donner naissance aux comparaisons, aux métaphores, etc. En général la métaphore fait le fond des langues.\par
Le premier principe qui doit nous guider dans la recherche des étymologies, c’est que la marche des idées correspond à celle des choses. Or les degrés de la civilisation peuvent être ainsi indiqués : {\itshape Forêts, cabanes, villages, cités} ou sociétés de citoyens, {\itshape académies} ou sociétés de savants ; les hommes habitent d’abord les {\itshape montagnes}, ensuite les {\itshape plaines}, enfin les {\itshape rivages}. Les idées, et les perfectionnements du langage ont dû suivre cet ordre. Ce principe étymologique suffit pour les langues indigènes, pour celles des pays barbares qui restent impénétrables aux étrangers, jusqu’à ce qu’ils leur soient ouverts par la guerre ou par le commerce. Il montre combien les philologues ont eu tort d’établir que la signification des langues est arbitraire. Leur origine fut naturelle, leur signification doit être fondée en nature. On peut l’observer dans le latin, langue {\itshape plus héroïque}, moins raffinée que le grec ; tous les mots y sont tirés par figures d’objets agrestes et sauvages.\par
La langue {\itshape héroïque} employa pour noms communs des noms propres ou des noms de peuples. Les anciens Romains disaient un {\itshape Tarentin} pour un  homme parfumé. Tous les peuples de l’antiquité dirent un {\itshape Hercule} pour un héros. Cette création des caractères idéaux qui semblerait l’effort d’un art ingénieux, fut une nécessité pour l’esprit humain. Voyez l’enfant ; les noms des premières personnes, des premières choses qu’il a vues, il les donne à toutes celles en qui il remarque quelqu’analogie. De même les premiers hommes, incapables de former l’idée abstraite du {\itshape poète}, du {\itshape héros}, nommèrent tous les héros du nom du premier héros, tous les poètes, etc. Par un effet de notre amour instinctif de l’uniformité, ils ajoutèrent à ces premières idées des fictions singulièrement en harmonie avec les réalités, et peu à peu les noms de {\itshape héros}, de {\itshape poète}, qui d’abord désignaient tel individu, comprirent tous les caractères de perfection qui pouvaient entrer dans le type idéal de l’{\itshape héroïsme}, de la {\itshape poésie}. Le {\itshape vrai poétique}, résultat de cette double opération, fut plus vrai que le {\itshape vrai réel} ; quel héros de l’histoire remplira le {\itshape caractère héroïque} aussi bien que l’Achille de l’Iliade ?\par
Cette tendance des hommes à placer des types idéaux sous des noms propres, a rempli de difficultés et de contradictions apparentes les commencements de l’histoire. Ces types ont été pris pour des individus. Ainsi toutes les découvertes des anciens Égyptiens appartiennent à un Hermès ; la  première constitution de Rome, même dans cette partie morale qui semble le produit des habitudes, sort tout armée de la tête de Romulus ; tous les exploits, tous les travaux de la Grèce héroïque composent la vie d’Hercule ; Homère enfin nous apparaît seul sur le passage des temps héroïques à ceux de l’histoire, comme le représentant d’une civilisation tout entière. Par un privilège admirable, ces hommes prodigieux ne sont pas lentement enfantés par le temps et par les circonstances ; ils naissent d’eux-mêmes, et ils semblent créer leur siècle et leur patrie. Comment s’étonner que l’antiquité en ait fait des dieux ?\par
Considérez les noms d’Hermès, de Romulus, d’Hercule et d’Homère, comme les expressions de tel caractère national à telle époque, comme désignant les types de l’esprit inventif chez les Égyptiens, de la société romaine dans son origine, de l’héroïsme grec, de la poésie populaire des premiers âges chez la même nation, les difficultés disparaissent, les contradictions s’expliquent ; une clarté immense luit dans la ténébreuse antiquité.\par
Prenons Homère, et voyons comment toutes les invraisemblances de sa vie et de son caractère deviennent, par cette interprétation, des convenances, des nécessités. {\itshape Pourquoi tous les peuples grecs se sont-ils disputé sa naissance}, l’ont-ils revendiqué  pour citoyen ? c’est que chaque tribu retrouvait en lui son caractère, c’est que la Grèce s’y reconnaissait, c’est qu’elle était elle-même Homère. — {\itshape Pourquoi des opinions si diverses sur le temps où il vécut} ? c’est qu’il vécut en effet pendant les cinq siècles qui suivirent la guerre de Troie, dans la bouche et dans la mémoire des hommes. — {\itshape Jeune, il composa l’Iliade…} La Grèce, jeune alors, toute ardente de passions sublimes, violentes, mais généreuses, fit son héros d’Achille, le héros de la force. {\itshape Dans sa vieillesse, il composa l’Odyssée…} La Grèce plus mûre, conçut longtemps après le caractère d’Ulysse, le héros de la sagesse. — {\itshape Homère fut pauvre et aveugle…} dans la personne des rapsodes, qui recueillaient les chants populaires, et les allaient répétant de ville en ville, tantôt sur les places publiques, tantôt dans les fêtes des dieux. Alors comme aujourd’hui les aveugles devaient mener le plus souvent cette vie mendiante et vagabonde ; d’ailleurs la supériorité de leur mémoire les rendait plus capables de retenir tant de milliers de vers.\par
Homère n’étant plus un homme, mais désignant l’ensemble des chants improvisés par tout le peuple et recueillis par les rapsodes, se trouve justifié de tous les reproches qu’on lui a faits, et de la bassesse d’images, et des licences, et du mélange des dialectes.  Qui pourrait s’étonner encore qu’il ait élevé les hommes à la grandeur des dieux, et rabaissé les dieux aux faiblesses humaines ? le vulgaire ne fait-il pas les dieux a son image ?\par
Le génie d’Homère s’explique aussi sans peine ; l’incomparable puissance d’invention qu’on admire dans ses caractères, l’originalité sauvage de ses comparaisons, la vivacité de ses peintures de morts et de batailles, son pathétique sublime, tout cela n’est pas le génie d’un homme, c’est celui de l’âge héroïque. Quelle force de jeunesse n’ont pas alors l’imagination, la mémoire, et les passions qui inspirent la poésie ?\par
Les trois principaux titres d’Homère sont désormais mieux motivés : c’est bien le fondateur de la civilisation en Grèce, le père des poètes, la source de toutes les philosophies grecques. Le dernier titre mérite une explication : les philosophes ne tirèrent point leurs systèmes d’Homère, quoiqu’ils cherchassent à les autoriser de ses fables ; mais ils y trouvèrent réellement une occasion de recherches, et une facilité de plus pour exposer et populariser leurs doctrines.\par
Cependant on peut insister : {\itshape en supposant qu’un peuple entier ait été poète, comment put-il inventer les artifices du style, ces épisodes, ces tours heureux, ce nombre poétique....} ? et comment eût-il pu  ne pas les inventer ? les tours ne vinrent que de la difficulté de s’exprimer ; les épisodes de l’inhabileté qui ne sait pas distinguer et écarter les choses qui ne vont pas au but. Quant au nombre musical et poétique, il est naturel à l’homme ; les bègues s’essaient à parler en chantant ; dans la passion, la voix s’altère et approche du chant. Partout les vers précédèrent la prose.\par
Passer de la poésie à la prose, c’était abstraire et généraliser ; car le langage de la première est tout concret, tout particulier. La poésie elle-même, quoiqu’elle sortît alors de l’usage vulgaire, reçut aussi les expressions générales ; aux noms propres, qui, dans l’indigence des langues, lui avaient servi à désigner les caractères, elle substitua des noms imaginaires, et conçut des caractères purement idéaux ; ce fut là le commencement de son troisième âge, de l’âge {\itshape humain} de la poésie.\par
\par
L’origine de la religion, de la poésie et des langues étant découverte, nous connaissons celle de la société païenne. Les poèmes d’Homère en sont le principal monument. Joignez-y l’histoire des premiers siècles de Rome, qui nous présente le meilleur commentaire de l’histoire fabuleuse des Grecs ; en effet Rome ayant été fondée lorsque les langues vulgaires du Latium avaient fait de grands progrès,  l’héroïsme romain jeune encore, au milieu de peuples déjà mûrs, s’exprima en langue vulgaire, tandis que celui des Grecs s’était exprimé en langue héroïque.\par
Le commencement de la religion fut celui de la société. Les géants, effrayés par la foudre qui leur révèle une puissance supérieure, se réfugient dans les cavernes. L’état bestial finit avec leurs courses vagabondes ; ils s’assurent d’un asile régulier, ils y retiennent une compagne par la force, et la famille a commencé. Les premiers pères de famille sont les premiers prêtres ; et comme la religion compose encore toute la sagesse, les premiers sages ; maîtres absolus de leur famille, ils sont aussi les premiers rois ; de là le nom de {\itshape patriarches} (pères et princes). Dans une si grande barbarie, leur joug ne peut être que dur et cruel ; le Polyphème d’Homère est aux yeux de Platon l’image des premiers pères de famille. Il faut bien qu’il en soit ainsi pour que les hommes domptés par le gouvernement de la famille se trouvent préparés à obéir aux lois du gouvernement civil qui va succéder. Mais ces rois absolus de la famille sont eux-mêmes soumis aux puissances divines, dont ils interprètent les ordres à leurs femmes et à leurs enfants ; et comme alors il n’y a point d’action qui ne soit soumise à un Dieu, le gouvernement est en effet théocratique.\par
 Voilà l’âge d’or, tant célébré par les poètes, l’âge où les dieux règnent sur la terre. Toute la vertu de cet âge, c’est une superstition barbare qui sert pourtant à contenir les hommes, malgré leur brutalité et leur orgueil farouche. Quelque horreur que nous inspirent ces religions sanguinaires, n’oublions pas que c’est sous leur influence que se sont formées les plus illustres sociétés du monde ; l’athéisme n’a rien fondé.\par
Bientôt la famille ne se composa pas seulement des individus liés par le sang. Les malheureux qui étaient restés dans la promiscuité des biens et des femmes, et dans les querelles qu’elle produisait, voulant échapper aux insultes des violents, recoururent aux autels des forts, situés sur les hauteurs. Ces autels furent les premiers asiles, \emph{{\itshape vetus urbes condentium consilium}}, dit Tite-Live. Les forts tuaient les violents et protégeaient les réfugiés. Issus de Jupiter, c’est-à-dire, nés sous ses auspices, ils étaient héros par la naissance et par la vertu. Ainsi se forma le caractère idéal de l’Hercule antique ; les héros étaient {\itshape héraclides}, enfants d’Hercule, comme les sages étaient appelés enfants de la sagesse, etc.\par
Les nouveaux venus, conduits dans la société par l’intérêt, non par la religion, ne partagèrent pas les prérogatives des héros, particulièrement celle du mariage solennel. Ils avaient été reçus à condition  de servir leurs défenseurs comme esclaves ; mais, devenus nombreux, ils s’indignèrent de leur abaissement, et demandèrent une part dans ces terres qu’ils cultivaient. Partout où les héros furent vaincus, ils leur cédèrent des terres qui devaient toujours relever d’eux ; ce fut la première {\itshape loi agraire}, et l’origine des {\itshape clientèles} et des {\itshape fiefs}.\par
Ainsi s’organisa la cité : les pères de famille formèrent une classe de {\itshape nobles}, de {\itshape patriciens}, conservant le triple caractère de rois de leur maison, de prêtres et de sages, c’est-à-dire, de dépositaires des auspices. Les réfugiés composèrent une classe de {\itshape plébéiens, compagnons, clients, vassaux}, sans autre droit que la jouissance des terres, qu’ils tenaient des nobles.\par
Les cités héroïques furent toutes gouvernées aristocratiquement ; les rois des familles soumirent leur empire domestique à celui de leur ordre. Les principaux de l’ordre héroïque furent appelés {\itshape rois} de la cité, et administrèrent les affaires communes, en ce qui touchait la guerre et la religion.\par
Ces petites sociétés étaient essentiellement guerrières (πόλις, πόλεμος). {\itshape Étranger} ({\itshape hostis}), dans leur langage, est synonyme d’{\itshape ennemi}. Les héros s’honoraient du nom de brigands ({\itshape Voy.} Thucydide), et exerçaient en effet le brigandage ou la piraterie. À l’intérieur, les cités héroïques n’étaient pas plus  tranquilles. Les anciens nobles, dit Aristote ({\itshape Politique}), juraient une éternelle inimitié aux plébéiens. L’histoire romaine nous le confirme : les plébéiens combattaient pour l’intérêt des nobles, à leurs propres dépens, et ceux-ci les ruinaient par l’usure, les enfermaient dans leurs cachots particuliers, les déchiraient de coups de fouets. Mais l’amour de l’honneur, qui entretient dans les républiques aristocratiques cette violente rivalité des ordres, cause en récompense dans la guerre une généreuse émulation. Les nobles se dévouent au salut de la patrie, auquel tiennent tous les privilèges de leur ordre ; les plébéiens, par des exploits signalés, cherchent à se montrer dignes de partager les privilèges des nobles. Ces querelles, qui tendent à établir l’égalité, sont le plus puissant moyen d’agrandir les républiques.\par
\par
Pour compléter ce tableau des âges divin et héroïque, nous rapprocherons l’histoire du droit civil de celle du droit politique. Dans la première, nous retrouvons toutes les vicissitudes de la seconde. Si les gouvernements résultent des mœurs, la jurisprudence varie selon la forme du gouvernement. C’est ce que n’ont vu ni les historiens, ni les jurisconsultes ; ils nous expliquent les lois, nous en rappellent l’institution sans en marquer les rapports  avec les révolutions politiques ; ainsi ils nous présentent les faits isolés de leurs causes. Demandez-leur pourquoi la jurisprudence antique des Romains fut entourée de tant de solennités, de tant de mystères ; ils ne savent qu’accuser l’imposture des patriciens.\par
Au premier âge, le droit et la raison, c’est ce qui est ordonné d’en haut, c’est ce que les dieux ont révélé par les auspices, par les oracles et autres signes matériels. Le droit est fondé sur une autorité divine. Demander la moindre explication serait un blasphème. Admirons la Providence qui permit qu’à une époque où les hommes étaient incapables de discerner le droit, la raison véritable, ils trouvassent dans leur erreur un principe d’ordre et de conduite. La jurisprudence, la science de ce droit divin, ne pouvait être que la connaissance des rites religieux ; la justice était tout entière dans l’observation de certaines pratiques, de certaines cérémonies. De là le respect superstitieux des Romains pour les {\itshape acta legitima} ; chez eux, les noces, le testament étaient dits {\itshape justa}, lorsque les cérémonies requises avaient été accomplies.\par
Le premier tribunal fut celui des dieux ; c’est à eux qu’en appelaient ceux qui recevaient quelque tort, ce sont eux qu’ils invoquaient comme témoins et comme juges. Quand les jugements de la religion  se régularisèrent, les coupables furent dévoués, anathématisés ; sur cette sentence, ils devaient être mis à mort. On la prononçait contre un peuple aussi bien que contre un individu ; les guerres ({\itshape pura et pia bella}) étaient des jugements de Dieu. Elles avaient toutes un caractère de religion ; les hérauts qui les déclaraient, dévouaient les ennemis, et appelaient leurs dieux hors de leurs murs ; les vaincus étaient considérés comme sans dieux ; les rois traînés derrière le char des triomphateurs romains étaient offerts au Capitole à Jupiter Férétrien, et de là immolés.\par
Les duels furent encore une espèce de jugement des dieux. \emph{{\itshape Les républiques anciennes}, dit Aristote dans sa{\itshape  Politique, n’avaient pas de lois judiciaires pour punir les crimes et réprimer la violence.}} Le duel offrait seul un moyen d’empêcher que les guerres individuelles ne s’éternisassent. Les hommes, ne pouvant distinguer la cause réellement juste, croyaient juste celle que favorisaient les dieux. Le {\itshape droit héroïque} fut celui de la force.\par
La violence des héros ne connaissait qu’un seul frein : le respect de la parole. Une fois prononcée, la parole était pour eux sainte comme la religion, immuable comme le passé ({\itshape fas, fatum}, de {\itshape fari}). Aux actes religieux qui composaient seuls toute la justice de l’âge divin, et qu’on pourrait appeler  {\itshape formules d’actions}, succédèrent des {\itshape formules parlées}. Les secondes héritèrent du respect qu’on avait eu pour les premières, et la superstition de ces formules fut inflexible, impitoyable : \emph{{\itshape Uti linguâ nuncupassit, ita jus esto}} (Douze Tables) : Agamemnon a prononcé qu’il immolerait sa fille ; il faut qu’il l’immole. Ne crions pas comme Lucrèce, \emph{{\itshape tantum relligio potuit suadere malorum}} !… Il fallait cette horrible fidélité à la parole dans ces temps de violence ; la faiblesse soumise à la force avait à craindre de moins ses caprices. — L’équité de cet âge n’est donc pas l’{\itshape équité naturelle}, mais l’{\itshape équité civile} ; elle est dans la jurisprudence ce que la {\itshape raison d’état} est en politique, un principe d’utilité, de conservation pour la société.\par
La sagesse consiste alors dans un usage habile des paroles, dans l’application précise, dans l’appropriation du langage à un but d’intérêt. C’est là la sagesse d’Ulysse ; c’est celle des anciens jurisconsultes romains avec leur fameux {\itshape cavere. Répondre sur le droit}, ce n’était pour eux autre chose que précautionner les consultants, et les préparer à circonstancier devant les tribunaux le cas contesté, de manière que les formules d’actions s’y rapportassent de point en point, et que le préteur ne pût refuser de les appliquer. — Imitées des formules religieuses, les formules légales de l’âge héroïque  furent enveloppées des mêmes mystères : le secret, l’attachement aux choses établies sont l’âme des républiques aristocratiques.\par
Les formules religieuses, étant toutes en action, n’avaient rien de général ; les formules légales dans leurs commencements n’ont rapport qu’à un fait, à un individu ; ce sont de simples exemples d’après lesquels on juge ensuite les faits analogues. La loi, toute particulière encore, n’a pour elle que l’autorité ({\itshape dura est, sed scripta est}) ; elle n’est pas encore fondée en principe, en {\itshape vérité}. Jusque-là, il n’y a qu’un droit civil ; avec l’âge {\itshape humain} commence le droit naturel, le droit de l’humanité raisonnable. La justice de ce dernier âge considère le mérite des faits et des personnes ; une justice aveugle serait faussement impartiale ; son égalité apparente serait en effet inégalité. Les exceptions, les privilèges sont souvent demandés par l’équité naturelle ; aussi les gouvernements humains savent faire plier la loi dans l’intérêt de l’égalité même.\par
À mesure que les démocraties et les monarchies remplacent les aristocraties héroïques, l’importance de la loi civile domine de plus en plus celle de la loi politique. Dans celles-ci tous les intérêts privés des citoyens étaient renfermés dans les intérêts publics ; sous les gouvernements {\itshape humains}, et surtout sous les monarchies, les intérêts publics n’occupent  les esprits qu’à propos des intérêts privés ; d’ailleurs les mœurs s’adoucissant, les affections particulières en prennent d’autant plus de force, et remplacent le patriotisme.\par
Sous les gouvernements {\itshape humains}, l’égalité que la nature a mise entre les hommes en leur donnant l’intelligence, caractère essentiel de l’humanité, est consacrée dans l’égalité civile et politique. Les citoyens sont dès lors égaux, d’abord comme souverains de la cité, ensuite comme sujets d’un monarque qui, distingué seul entre tous, leur dicte les mêmes lois.\par
Dans les républiques populaires bien ordonnées, la seule inégalité qui subsiste est déterminée par le cens : Dieu veut qu’il en soit ainsi, pour donner l’avantage à l’économie sur la prodigalité, à l’industrie et à la prévoyance sur l’indolence et la paresse. — Le peuple pris en général veut la justice ; lorsqu’il entre ainsi dans le gouvernement, il fait des lois justes, c’est-à-dire généralement bonnes.\par
Mais peu à peu les états populaires se corrompent. Les riches ne considèrent plus leur fortune comme un moyen de supériorité légale, mais comme un instrument de tyrannie ; le peuple qui sous les gouvernements héroïques ne réclamait que l’égalité, veut maintenant dominer à son tour ; il ne manque pas de chefs ambitieux qui lui présentent des lois  populaires, des lois qui tendent à enrichir les pauvres. Les querelles ne sont plus légales ; elles se décident par la force. De là des guerres civiles au-dedans, des guerres injustes au-dehors. Les puissants s’élèvent dans le désordre ; et l’anarchie, la pire des tyrannies, force le peuple de se réfugier dans la domination d’un seul. Ainsi le besoin de l’ordre et de la sécurité fonde les monarchies. Voilà la {\itshape loi royale} (pour parler comme les jurisconsultes) par laquelle Tacite légitime la monarchie romaine sous Auguste : \emph{{\itshape Qui cuncta discordiis fessa sub imperium unius accepit.}}\par
Fondées sur la protection des faibles, les monarchies doivent être gouvernées d’une manière populaire. Le prince établit l’égalité, au moins dans l’obéissance ; il humilie les grands, et leur abaissement est déjà une liberté pour les petits. Revêtu d’un pouvoir sans bornes, il consulte non la loi, mais l’équité naturelle. Aussi la monarchie est-elle le gouvernement le plus conforme à la nature, dans les temps de la civilisation la plus avancée.\par
Les monarques se glorifient du titre de cléments, et rendent les peines moins sévères ; ils diminuent cette terrible puissance paternelle des premiers âges. La bienveillance de la loi descend jusqu’aux esclaves ; les ennemis même sont mieux traités, les vaincus conservent des droits. Celui de citoyen,  dont les républiques étaient si avares, est prodigué ; et le pieux Antonin veut, selon le mot d’Alexandre, que le monde soit une seule cité.\par
\par
Voilà toute la vie politique et civile des nations, tant qu’elles conservent leur indépendance. Elles passent successivement sous trois gouvernements. La législation divine fonde la monarchie domestique, et commence l’{\itshape humanité} ; la législation héroïque ou aristocratique forme la cité, et limite les abus de la force ; la législation populaire consacre dans la société l’égalité naturelle ; la monarchie enfin doit arrêter l’anarchie, et la corruption publique qui l’a produite.\par
Quand ce remède est impuissant, il en vient inévitablement du dehors un autre plus efficace. Le peuple corrompu était esclave de ses passions effrénées ; il devient esclave d’une nation meilleure qui le soumet par les armes, et le sauve en le soumettant. Car ce sont deux lois naturelles : {\itshape Qui ne peut se gouverner, obéira}, — et, {\itshape aux meilleurs l’empire du monde}.\par
Que si un peuple n’était secouru dans ce misérable état de dépravation ni par la monarchie ni par la conquête, alors, au dernier des maux, il faudrait bien que la Providence appliquât le dernier des remèdes. Tous les individus de ce peuple se  sont isolés dans l’intérêt privé ; on n’en trouvera pas deux qui s’accordent, chacun suivant son plaisir ou son caprice. Cent fois plus barbares dans cette dernière période de la civilisation qu’ils ne l’étaient dans son enfance ! la première barbarie était de nature, la seconde est de réflexion ; celle-là était féroce, mais généreuse ; un ennemi pouvait fuir ou se défendre ; celle-ci, non moins cruelle, est lâche et perfide ; c’est en embrassant qu’elle aime à frapper. Aussi ne vous y trompez pas ; vous voyez une foule de corps, mais si vous cherchez des {\itshape âmes humaines}, la solitude est profonde ; ce ne sont plus que des bêtes sauvages.\par
Qu’elle périsse donc cette société par la fureur des factions, par l’acharnement désespéré des guerres civiles ; que les cités redeviennent forêts, que les forêts soient encore le repaire des hommes, et qu’à force de siècles, leur ingénieuse malice, leur subtilité perverse disparaissent sous la rouille de la barbarie. Alors stupides, abrutis, insensibles aux raffinements qui les avaient corrompus, ils ne connaissent plus que les choses indispensables à la vie ; peu nombreux, le nécessaire ne leur manque pas ; ils sont de nouveau susceptibles de culture ; avec l’antique simplicité l’on verra bientôt reparaître la piété, la véracité, la bonne foi, sur lesquelles est fondée la justice, et qui font toute  la beauté de l’ordre éternel établi par la Providence.\par
\par
C’est après ces épurations sévères que Dieu renouvela la société européenne sur les ruines de l’empire romain. Dirigeant les choses humaines dans le sens des décrets ineffables de sa grâce, il avait établi le christianisme en opposant la vertu des martyrs à la puissance romaine, les miracles et la doctrine des pères à la vaine sagesse des Grecs ; mais il fallait arrêter les nouveaux ennemis qui menaçaient de toutes parts la foi chrétienne et la civilisation, au nord les Goths ariens, au midi les Arabes mahométans, qui contestaient également à l’auteur de la religion son divin caractère.\par
On vit renaître l’âge {\itshape divin} et le gouvernement théocratique. On vit les rois catholiques revêtir les habits de diacre, mettre la croix sur leurs armes, sur leurs couronnes, et fonder des ordres religieux et militaires pour combattre les infidèles. Alors revinrent les guerres pieuses de l’antiquité ({\itshape pura et pia bella}) ; mêmes cérémonies pour les déclarer : on appelait hors des murs d’une ville assiégée les saints, protecteurs de l’ennemi ; et l’on cherchait à dérober leurs reliques. — Les jugements divins reparurent sous le nom de {\itshape purgations canoniques} ; les duels en furent une espèce, quoique non reconnue par les canons. — Les brigandages et les représailles  de l’antiquité, la dureté des servitudes héroïques se renouvelèrent, surtout entre les infidèles et les chrétiens. — Les {\itshape asiles} du monde ancien se rouvrirent chez les évêques, chez les abbés ; c’est le besoin de cette protection qui motive la plupart des constitutions de fiefs. Pourquoi tant de lieux escarpés ou retirés portent-ils des noms de saints ? c’est que des chapelles y servaient d’asiles. — L’{\itshape âge muet} des premiers temps du monde se représenta, les vainqueurs et les vaincus ne s’entendaient point ; nulle écriture en langue vulgaire. Les signes hiéroglyphiques furent employés pour marquer les droits seigneuriaux sur les maisons et sur les tombeaux, sur les troupeaux et sur les terres. Ainsi, nous retrouvons au moyen âge la plupart des caractères observés déjà dans la plus haute antiquité.\par
\par
Quand toutes les observations qui précèdent sur l’histoire du genre humain, ne seraient point appuyées par le témoignage des philosophes et des historiens, des grammairiens et des jurisconsultes, ne nous conduiraient-elles pas à reconnaître dans ce monde {\itshape la grande cité des nations fondée et gouvernée par Dieu même} ? — On élève jusqu’au ciel la sagesse législative des Lycurgue, des Solon, et des décemvirs, auxquels on rapporte la police tant célébrée des trois plus glorieuses cités, des plus signalées  par la vertu civile ; et pourtant combien ne sont-elles pas inférieures en grandeur et en durée à la république de l’univers !\par
Le miracle de sa constitution, c’est qu’à chacune de ses révolutions, elle trouve dans la corruption même de l’état précédent les éléments de la forme nouvelle qui peut la sauver. Il faut bien qu’il y ait là une sagesse au-dessus de l’homme....\par
Cette sagesse ne nous force pas par des lois positives, mais elle se sert pour nous gouverner des usages que nous suivons librement. Répétons donc ici le premier principe de la Science nouvelle : les hommes ont fait eux-mêmes le monde social, tel qu’il est ; mais ce monde n’en est pas moins sorti d’une intelligence, souvent contraire, et toujours supérieure aux fins particulières que les hommes s’étaient proposées. Ces fins d’une vue bornée sont pour elle les moyens d’atteindre des fins plus grandes et plus lointaines. Ainsi les hommes isolés encore veulent le plaisir brutal, et il en résulte la sainteté des mariages et l’institution de la famille ; — les pères de famille veulent abuser de leur pouvoir sur leurs serviteurs, et la cité prend naissance ; — l’ordre dominateur des nobles veut opprimer les plébéiens, et il subit la servitude de la loi, qui fait la liberté du peuple ; — le peuple libre tend à secouer le frein de la loi, et il est assujetti à un monarque ; — le  monarque croit assurer son trône en dégradant ses sujets par la corruption, et il ne fait que les préparer à porter le joug d’un peuple plus vaillant ; — enfin quand les nations cherchent à se détruire elles-mêmes, elles sont dispersées dans les solitudes… et le phénix de la société renaît de ses cendres.\par
\par
Tel est l’exposé bien incomplet sans doute de ce vaste système ; nous l’abandonnons aux méditations de nos lecteurs. Il serait trop long de suivre Vico dans les applications ingénieuses qu’il a faites de ses principes. Nous ajouterons seulement quelques mots pour faire connaître quel fut le sort de l’auteur et de l’ouvrage.\par
La Science nouvelle eut quelque succès en Italie, et la première édition fut épuisée en trois ans. Plusieurs grands personnages, entre autres le pape Clément XII, écrivirent à Vico des lettres flatteuses. Des savants de Venise qui voulaient réimprimer {\itshape La Science nouvelle} dans cette ville, lui persuadèrent d’écrire lui-même sa vie pour qu’on l’insérât, dans un {\itshape Recueil des Vies des littérateurs les plus distingués de l’Italie}. Mais dans le reste de l’Europe le grand ouvrage de Vico ne produisit aucune sensation. Leclerc, qui avait rendu compte du livre {\itshape De uno universi juris principio} dans la {\itshape Bibliothèque universelle}, ne parla point de {\itshape La Science nouvelle}.  Le {\itshape Journal de Trévoux} en fit une simple mention. Le {\itshape Journal de Leipzig}\footnote{Orthographié « Leipsik » [NdE].} inséra un article calomnieux qui lui avait été envoyé de Naples.\par
Employé fréquemment par les vice-rois espagnols ou autrichiens à composer des discours, des vers, des inscriptions pour les occasions solennelles, Vico n’en resta pas moins dans l’indigence où il était né. Il ne suppléait à l’insuffisance des appointements de la chaire de rhétorique qu’il occupait à l’université de Naples, qu’en donnant chez lui des leçons de langue latine. Au moment même où il achevait {\itshape La Science nouvelle}, il concourut pour une chaire de droit, et il échoua.\par
Dans cette position pénible, il faisait toute sa consolation du soin d’élever ses deux filles, qu’il aimait beaucoup, et dont l’aînée réussit dans la poésie italienne. C’était, dit l’éditeur des opuscules de Vico, auquel un fils du grand homme a transmis ces détails, c’était un spectacle touchant de voir le philosophe jouer avec ses filles aux heures que lui laissaient d’ennuyeux devoirs. Un ami qui le trouvait un jour avec elles, ne put s’empêcher de répéter ce passage du Tasse : \emph{{\itshape C’est Alcide qui, la quenouille en main, amuse de récits fabuleux les filles de Méonie.}} Ce bonheur domestique était lui-même mêlé d’amertume. Un de ses enfants fut atteint d’une maladie longue et cruelle. Un autre  devint par sa mauvaise conduite la honte de sa famille, et Vico fut obligé de demander qu’il fût enfermé.\par
À l’avènement de la maison de Bourbon, sa condition sembla s’améliorer, il fut nommé historiographe du roi, et obtint que son fils, Gennaro Vico, dont on connaissait le mérite et la probité, lui succédât comme professeur ; mais ces faveurs venaient bien tard. Il languissait déjà sous le poids de l’âge et des plus douloureuses infirmités. Enfin ses forces diminuant tous les jours, il resta quatorze mois sans parler et sans reconnaître ses propres enfants. Il ne sortit de cet état que pour s’apercevoir de sa mort prochaine, et, après avoir rempli le devoir d’un chrétien, il expira en récitant les psaumes de David, le 20 janvier 1744. Il avait 76 ans accomplis.\par
Ne quittons point cet homme rare sans apprendre de lui-même comment il supporta ses malheurs : \emph{« Qu’elle soit à jamais louée, dit-il dans une lettre, cette Providence qui, lors même qu’elle semble à nos faibles yeux une justice sévère, n’est qu’amour et que bonté. Depuis que j’ai fait mon grand ouvrage, je sens que j’ai revêtu un nouvel homme. Je n’éprouve plus la tentation de déclamer contre le mauvais goût du siècle, puisqu’en me repoussant de la place que je demandais, il m’a donné l’occasion de composer la Science nouvelle. Le  dirai-je ? je me trompe peut-être, mais je voudrais bien ne pas me tromper : la composition de cet ouvrage m’a animé d’un esprit héroïque qui me met au-dessus de la crainte de la mort et des calomnies de mes rivaux. Je me sens assis sur une roche de diamant, quand je songe au jugement de Dieu qui fait justice au génie par l’estime du sage !… 1726. »}\par
Nous rapporterons encore, quoi qu’il en coûte, les dernières lignes qui soient sorties de sa plume :\par

\begin{quoteblock}
 \noindent « Maintenant Vico n’a plus rien à espérer au monde. Accablé par l’âge et les fatigues, usé par les chagrins domestiques, tourmenté de douleurs convulsives dans les cuisses et dans les jambes, en proie à un mal rongeur qui lui a déjà dévoré une partie considérable de la tête, il a renoncé entièrement aux études, et a envoyé au père Louis-Dominique, si recommandable par sa bonté et par son talent dans la poésie élégiaque, le manuscrit des notes sur la première édition de la Science nouvelle, avec l’inscription suivante :\par
 
\begin{center}
\noindent AU TIBULLE CHRÉTIEN \\
AU PÈRE LOUIS DOMINIQUE \\
JEAN BAPTISTE VICO \\
POURSUIVI ET BATTU \\
PAR LES ORAGES CONTINUELS D’UNE FORTUNE ENNEMIE \\
ENVOIE CES DÉBRIS INFORTUNÉS DE LA SCIENCE NOUVELLE \\
PUISSENT ILS TROUVER CHEZ LUI UN PORT UN LIEU DE REPOS\par
\end{center}

 \noindent  [Après avoir rappelé les obstacles, les contradictions qu’il rencontra, il ajoute ce qui suit :] « Vico bénissait ces adversités qui le ramenaient à ses études. Retiré dans sa solitude comme dans un fort inexpugnable, il méditait, il écrivait quelque nouvel ouvrage, et tirait une noble vengeance de ses détracteurs. C’est ainsi qu’il en vint à trouver la {\itshape Science nouvelle}… Depuis ce moment il crut n’avoir rien à envier à ce Socrate, dont Phèdre disait :\par
 « L’envie le condamna vivant, mais sa cendre est absoute. Que l’on m’assure sa gloire, et je ne refuse point sa mort\footnote{\emph{{\itshape Cujus non fugio mortem, si famam assequar,Et cedo invidiæ, dum modo absolvar cinis.}}} ! »
 \end{quoteblock}

\subsection[{Appendice du discours}]{Appendice du discours}\phantomsection
\label{appdisc}

\begin{argument}\noindent  Cet appendice renferme la vie de Vico, la liste de tous ses ouvrages et celle des auteurs qui l’ont imité, attaqué, ou simplement mentionné ; enfin l’indication des principaux ouvrages qui ont été écrits sur la philosophie de l’histoire.
\end{argument}


\astertri

\noindent Nous ne répéterons pas ici les détails relatifs à la vie de Vico, que nous avons déjà donnés au commencement et à la fin du discours.\par
Vico naquit en 1668, et non en 1670, comme on le lit dans sa Vie écrite par lui-même. L’éditeur de ses {\itshape Opuscules} a rectifié cette date d’après les registres de naissance. À l’âge de sept ans, il perdit beaucoup de sang par suite d’une chute, et le chirurgien décida qu’il mourrait ou resterait imbécile ; la prédiction ne fut point vérifiée. \emph{« Cet accident ne fit qu’altérer son humeur, et le rendit mélancolique et ardent, caractère ordinaire des hommes qui unissent la vivacité d’esprit et la profondeur. »} Après avoir fait ses humanités et surpassé ses maîtres, il se livra avec ardeur à la dialectique ; mais les subtilités de la scholastique le rebutèrent : il faillit perdre l’esprit, et demeura découragé pour dix-huit mois.\par
Un jour qu’il était entré par hasard dans une école de droit, le professeur louait un célèbre jurisconsulte ; ce moment décida de sa vie…… \emph{« Dès ces premières études, Vico était charmé en lisant les maximes dans lesquelles les interprètes anciens ont résumé et généralisé les motifs particuliers du législateur. Il aimait aussi à observer le soin avec lequel les jurisconsultes  pèsent les termes des lois qu’ils expliquent. Il vit dès lors dans les interprètes anciens les philosophes de l’équité naturelle ; dans les interprètes érudits les historiens du droit romain : double présage de ses recherches sur le principe d’un droit universel, et du bonheur avec lequel il devait éclairer l’étude de la jurisprudence romaine par celle de la langue latine. »}\par
Il nous a fait connaître la marche de ses études pendant les neuf années qui suivirent cette époque. Ce n’est point ici un de ces romans où les philosophes exposent leurs idées dans une forme historique ; la route de Vico est trop sinueuse pour qu’on puisse la supposer tracée d’avance.\par
D’abord la nécessité d’embrasser toute la science qu’il enseignait, l’obligea de s’occuper du droit canonique. Pour mieux comprendre ce droit, il entra dans l’étude du dogme ; cette étude devait le conduire plus tard à \emph{« chercher un principe du droit naturel qui pût expliquer les origines historiques du droit romain et en général du droit des nations païennes, et qui, sous le rapport moral, n’en fût pas moins conforme à la saine doctrine de la Grâce »}.\par
Vers le même temps, la lecture de Laurent Valla, qui accuse de peu d’élégance les jurisconsultes romains, celle d’un autre critique qui comparait la versification savante de Virgile avec celle des modernes, le déterminèrent à se livrer à l’étude de la littérature latine qu’il associa à celle de l’italienne. Il lisait alternativement Cicéron et Boccace, Dante et Virgile, Horace et Pétrarque. Chaque ouvrage était lu trois fois ; la première pour en saisir l’unité, la seconde pour en observer la suite et pour étudier l’artifice de la composition, la troisième pour en noter les expressions remarquables, ce qu’il faisait sur le livre même.\par
Lisant ensuite, dans l’{\itshape Art poétique} d’Horace, que l’étude des moralistes ouvre à la poésie la source de richesses la plus abondante, il s’y livra avec ardeur, en commençant par Aristote, qu’il avait vu citer le plus souvent dans les livres élémentaires de droit. \emph{« Dans cette étude, il observa bientôt que la jurisprudence romaine n’était qu’un art de décider les cas particuliers selon l’équité,  art dont les jurisconsultes donnaient d’innombrables préceptes conformes à la justice naturelle, et tirés de l’intention du législateur ; mais que la science du juste enseignée par les philosophes est fondée sur un petit nombre de vérités éternelles, dictées par une justice métaphysique qui est comme l’architecte de la cité ; qu’ainsi l’on n’apprend dans les écoles que la moitié de la science du droit. »}\par
La morale le ramena à la métaphysique ; mais comme il tirait peu de profit de celle d’Aristote, il se mit à lire Platon, sur sa réputation de prince des philosophes. Il comprit alors pourquoi la métaphysique du premier ne lui avait servi de rien pour appuyer la morale. \emph{« Celle du second conduit à reconnaître pour principe physique l’idée éternelle qui tire d’elle-même et crée la matière. Conformément à cette métaphysique, Platon donne pour base à sa morale l’idéal de la justice ; et c’est de là qu’il part pour fonder sa république, sa législation idéales. La lecture de Platon éveilla dans l’esprit de Vico la première conception d’un droit idéal éternel, en vigueur dans la cité universelle, qui est renfermée dans la pensée de Dieu, et dans la forme de laquelle sont instituées les cités de tous les temps et de tous les pays. Voilà la république que Platon devait déduire de sa métaphysique ; mais il ne le pouvait, ignorant la chute du premier homme. »}\par
Les ouvrages philosophiques de Platon, d’Aristote et de Cicéron, dont le but est de diriger l’homme social, l’éloignèrent également \emph{« et des épicuriens, toujours renfermés dans la molle oisiveté de leurs jardins, et des stoïciens qui, tout entiers dans les théories, se proposent l’impassibilité ; ce sont morales de solitaires. Mais il admira la physique des stoïciens qui composent l’univers de points, comme les platoniciens le composent de nombres. Il rejeta également les physiques {\itshape mécaniques} d’Épicure et de Descartes. La physique expérimentale des Anglais lui parut devoir être utile à la médecine ; mais il se garda bien de s’occuper d’une science qui ne servait de rien à la philosophie de l’homme, et dont la langue était barbare »}.\par
 Comme Aristote et Platon tirent souvent leurs preuves des mathématiques, il étudia la géométrie pour les mieux entendre ; mais il ne poussa pas loin cette étude, pensant qu’il suffisait de connaître la méthode des géomètres ; \emph{« pourquoi mettre dans de pareilles entraves un esprit habitué à parcourir le champ sans bornes des généralités, et à chercher d’heureux rapprochements dans la lecture des orateurs, des historiens et des poètes ? »}\par
De retour à Naples, Vico y trouva cette décadence universelle dont on a vu le tableau. Combien il se félicita de n’avoir pas eu de maître dont les paroles fussent pour lui des lois ; combien il remercia la solitude de ses forêts, où il avait pu suivre une carrière toute indépendante ! Voyant qu’on négligeait surtout la langue latine, il se détermina à en faire un des principaux objets de ses études ; pour mieux s’y livrer, il abandonna le grec, et ne voulut jamais apprendre le français. Il croyait avoir remarqué que ceux qui savent tant de langues, n’en possèdent jamais une parfaitement. Il abandonna les critiques, les commentateurs, et ferma même les dictionnaires. Les premiers n’arrivent guère à sentir les beautés d’une langue étrangère, par l’habitude qu’ils ont de chercher toujours les défauts. La décadence de la langue latine date de l’époque où commencèrent à paraître les seconds. Il ne conserva d’autre lexique que le {\itshape Nomenclateur} de Junius pour l’intelligence des termes techniques. Il lut les auteurs dans des éditions sans notes, en cherchant à pénétrer dans leur esprit avec une critique philosophique. Aussi ses amis l’appelaient-ils, comme on nommait autrefois Épicure, \emph{αὐτοδιδάσκαλος, {\itshape le maître de soi-même}}.\par
On commençait dès lors à connaître son mérite, et les théatins cherchaient à le faire entrer dans leur ordre ; comme il n’était point gentilhomme, ils offraient de lui obtenir une dispense du pape. Vico refusa, et se maria, à ce qu’il paraît, peu de temps après. Vers la même époque, la chaire de rhétorique étant venue à vaquer, il refusait de concourir, parce qu’il avait échoué peu auparavant dans la demande d’une autre place ; mais ses amis se moquèrent de sa simplicité dans les choses d’intérêt ; il concourut et réussit (1697 ou 98).\par
 Cette place lui donna l’occasion d’exposer partiellement, dans une suite de discours d’ouverture, les idées qu’il devait réunir dans son grand ouvrage (1699-1720). Ce sont toujours des sujets généraux « où la philosophie descend aux applications de la vie civile ; il y traite du but des études et de la méthode qu’on doit y suivre, des fins de l’homme, du citoyen, du chrétien. »\par
Ces discours, généralement admirables par la hauteur des vues, ont une forme paradoxale et quelquefois bizarrement dramatique. L’homme, dit-il dans celui de 1699, doit embrasser le cercle des sciences ; qui ne le fait pas, ne le veut pas sérieusement. Nous ignorons toute la puissance de nos facultés. De même que Dieu est l’esprit du monde, l’esprit humain est un dieu dans l’homme. Ne vous est-il pas arrivé de faire, dans l’élan d’une volonté forte, des choses que vous admiriez ensuite, et que vous étiez tentés d’attribuer à un dieu plutôt qu’à vous-mêmes ? — Dans le discours de 1700, Dieu, juge de la grande cité, prononce cette sentence dans la forme des lois romaines : L’homme naîtra pour la vérité et pour la vertu, c’est-à-dire pour moi ; la raison commandera, les passions obéiront. Si quelque insensé, par corruption, par négligence ou par légèreté, enfreint cette loi, criminel au premier chef, qu’il se fasse à lui-même une guerre cruelle…… puis vient la description pathétique de cette guerre intérieure.\par
1701. Tout artifice, toute intrigue doivent être bannis de la république des lettres, si l’on veut acquérir de véritables lumières. — 1704. Quiconque veut trouver dans l’étude le profit et l’honneur, doit travailler pour la gloire, c’est-à-dire pour le bien général. — 1705. Les époques de gloire et de puissance pour les sociétés, ont été celles où elles ont fleuri par les lettres. — 1707. La connaissance de notre nature déchue doit nous exciter à embrasser dans nos études l’universalité des arts et des sciences, et nous indiquer l’ordre naturel dans lequel nous les devons apprendre. — Les discours de 1699 et de 1700 sont les seuls qu’on ait conservés en entier ; ils se trouvent dans le quatrième volume du {\itshape Recueil des Opuscules} de Vico.\par
 Nous avons parlé déjà de deux discours plus remarquables encore ({\itshape De nostri temporis studiorum ratione}, 1708. — {\itshape Omnis divinæ atque humanæ eruditionis elementa tria, nosse, velle, posse}, etc. 1719). Le second a été fondu par Vico dans son livre sur l’{\itshape Unité de principe du droit}, qui lui-même a fourni les matériaux de {\itshape La Science nouvelle}.\par
Le premier ouvrage considérable de Vico, est le traité : {\itshape De antiquissimâ Italorum sapientiâ ex linguæ latinæ originibus eruendâ}, 1710. La lecture du traité plus ingénieux que solide de Bacon, {\itshape De sapientiâ veterum}, lui fit naître l’idée de chercher les principes de la sagesse antique, non dans les fables des poètes, mais dans les étymologies de la langue latine, comme Platon les avait cherchés dans celles de la langue grecque ({\itshape Voy.} le {\itshape Cratyle}). Ce travail devait avoir deux parties, l’une métaphysique, l’autre physique. La première seule a été imprimée, sous le titre indiqué ci-dessus. Vico paraît n’avoir pas achevé la seconde ; il dit seulement en avoir dédié à Aulisio un morceau considérable, intitulé : {\itshape De æquilibrio corporis animantis}. Il y traitait de l’ancienne médecine des Égyptiens. Je n’ai pu me procurer cet opuscule, qui peut-être n’a pas été imprimé. Dans le peu qu’il en cite, on voit qu’il avait soupçonné l’analogie du calorique et du magnétisme.\par
Le livre {\itshape De antiquissimâ Italorum sapientiâ}, est de tous les ouvrages de Vico celui dont il a le moins profité dans la Science nouvelle. Rien de plus ingénieux que ses réflexions sur la signification identique des mots {\itshape verum} et {\itshape factum} dans l’ancienne langue latine, sur le sens d’{\itshape intelligere, cogitare, dividere, minuere, genus} et {\itshape forma, verum} et {\itshape æquum, causa} et {\itshape negotium}, etc. Nous avons fait connaître dans Vico le fondateur de la philosophie de l’histoire ; peut-être, dans un second volume, montrerons-nous en lui le métaphysicien subtil et profond, l’antagoniste du cartésianisme, l’adversaire le plus éclairé et le plus éloquent de l’esprit du dix-huitième siècle. La traduction de l’ouvrage dont nous venons de parler entrerait dans cette nouvelle publication.\par
\par
 Vico s’occupa bientôt d’un travail tout différent. Le duc de Traetto, Adrien Caraffe, le pria de se charger d’écrire la vie du maréchal Antoine Caraffe, son oncle, d’après les Mémoires qu’il avait laissés. Il y consacra une partie de ses nuits pendant deux ans \emph{« et s’efforça d’y concilier le respect dû aux princes avec celui que réclame la vérité »}. L’ouvrage parut en un volume, 1716, et concilia à l’auteur l’estime et l’amitié de Gravina, avec lequel il entretint dès lors une correspondance assidue. Nous n’avons pu trouver ni l’histoire ni les lettres.\par
Pour se préparer à écrire cette vie, Vico lut le grand ouvrage de Grotius. Nous avons vu quelle révolution cette lecture opéra dans ses idées. On lui avait demandé des notes pour une nouvelle édition du {\itshape Droit de la guerre et de la paix}, et il en avait déjà écrit sur le premier livre et sur la moitié du second, lorsqu’il s’arrêta, \emph{« réfléchissant qu’il convenait peu à un catholique d’orner de notes l’ouvrage d’un hérétique\footnote{On voit pourtant ({\itshape Recueil des Opuscules}, t. I, p. 118) qu’il correspondait avec un Juif, dont il fait l’éloge, et qui, dit-il, était son ami.} »}.\par
Lorsque Vico eut fait paraître ses deux ouvrages, {\itshape De uno universi juris principio}, et {\itshape De constantiâ jurisprudentis} (1721), l’importance de ces travaux et son ancienneté dans l’université de Naples, l’encouragèrent à concourir pour une chaire de droit qui se trouvait vacante. Plusieurs de ses adversaires comptaient bien qu’il vanterait longuement ses services envers l’université ; plusieurs espéraient qu’il s’en tiendrait à l’érudition vulgaire des principaux auteurs qui avaient traité la matière ; d’autres, qu’il se jetterait sur ses principes du droit universel. Il les trompa tous : après une invocation courte, grave et touchante, il lut le commencement de la loi, et suivit une méthode familière aux anciens jurisconsultes, mais toute nouvelle dans les concours. Les applaudissements unanimes de l’auditoire lui faisaient croire qu’il avait réussi ;  il en fut autrement. \emph{« Mais voici ce qui prouve que Vico est né pour la gloire de Naples et de l’Italie ; il venait de perdre tout espoir d’avancement dans sa patrie ; un autre aurait dit adieu aux lettres, se serait repenti peut-être de les avoir cultivées ; pour lui il ne songea qu’à compléter son système. »}\par
Nous ajouterons peu de choses à ce que nous avons dit sur les dernières années de Vico, et sur les malheurs qui attristèrent la fin de sa carrière. Une seule anecdote montrera l’état de gêne où il se trouvait, et l’indifférence de ses protecteurs. On a trouvé la note suivante au dos d’une lettre adressée à Vico par le cardinal Laurent Corsini, son Mécène, depuis pape sous le nom de Clément XII. \emph{« Réponse de Son Éminence le cardinal Corsini qui n’a pas eu le moyen de m’aider à imprimer mon ouvrage. Ce refus m’a forcé de penser à ma pauvreté. Il a fallu que j’employasse le prix d’un beau diamant, que je portais au doigt, à payer l’impression et la reliure. J’ai dédié l’ouvrage au seigneur cardinal, parce que je l’avais promis. »} L’amitié d’un simple gentilhomme, nommé Pietro Belli, fut plus utile à Vico, qui reconnut ses bienfaits en mettant une préface à sa traduction de la {\itshape Siphilis} de Frascator.\par
Dans une situation si pénible, il ne laissait échapper aucune plainte. Seulement il lui arrivait quelquefois de dire à un ami {\itshape que le malheur le poursuivrait jusqu’au tombeau}. Cette triste prophétie fut réalisée. À sa mort, les professeurs de l’université s’étaient rassemblés chez lui, selon l’usage, pour accompagner leur collègue à sa dernière demeure. La confrérie de Sainte-Sophie, à laquelle tenait Vico, devait porter le corps. Il était déjà descendu dans la cour et exposé. Alors commença une vive altercation entre les membres de la congrégation et les professeurs, qui prétendaient également au droit de porter les coins du drap mortuaire. Les deux partis s’obstinant, la congrégation se retira et laissa le cadavre. Les professeurs ne pouvant l’enterrer seuls, il fallut le remonter dans la maison. Son malheureux fils, l’âme navrée, s’adressa au chapitre de l’église métropolitaine, et le fit enterrer enfin dans l’église des pères de l’Oratoire ({\itshape detta de’ Gerolamini}),  qu’il fréquentait de son vivant, et qu’il avait choisie lui-même pour le lieu de sa sépulture.\par
Les restes de Vico demeurèrent négligés et ignorés jusqu’en 1789. Alors son fils Gennaro lui fit graver, dans un coin écarté de l’église, une simple épitaphe. L’Arcadie de Rome, dont Vico était membre, lui avait érigé un monument. Le possesseur actuel du château de Cilento, a mis une inscription à sa mémoire dans une bibliothèque peu considérable du couvent de Sainte-Marie de la Pitié, où il travaillait ordinairement pendant son séjour à Vatolla.\par
\par
Nous avons parlé du peu d’impression que produisit sur le public l’apparition du système de Vico. Lorsque parurent les livres {\itshape De uno juris principio} et {\itshape De constantiâ jurisprudentis}, l’ouvrage, dit-il lui-même, n’éprouva qu’une critique, c’est qu’on ne le comprenait pas. Cependant le fameux Leclerc le comprit, car il écrivit à l’auteur une lettre flatteuse, et témoigna une haute estime pour l’ouvrage, dans la {\itshape Bibliothèque ancienne et moderne}, 2\textsuperscript{e} partie du volume XVIII, article 8.\par
Lorsque les idées de Vico s’étendirent, et qu’il sentit la nécessité de réunir les deux ouvrages pour les appuyer l’un par l’autre, il entreprit d’abord d’établir son système en montrant l’invraisemblance de tout ce qu’on avait dit sur le même sujet ; l’ouvrage devait avoir deux volumes in-4º. Mais il sentit les inconvénients de cette méthode négative : d’ailleurs un revers de fortune l’avait mis hors d’état de faire des frais d’impression si considérables. Il concentra toutes ses facultés dans la méditation la plus profonde pour donner à son ouvrage une forme positive, et le réduire à de plus étroites proportions. Le résultat de ce nouveau travail fut la première édition de la {\itshape Science nouvelle}, qui parut en 1725.\par
{\itshape La Science nouvelle} fut attaquée par les protestants et par les catholiques. Tandis qu’un Damiano Romano, accusait le système de Vico d’être contraire à la religion, le journal de Leipzig\footnote{Orthographié « Leipsig » [NdE].} insérait  un article envoyé par un autre compatriote de Vico, dans lequel on lui reprochait d’avoir \emph{{\itshape approprié son système au goût de l’église romaine}}. Vico accepte ce dernier reproche, mais il ajoute un mot remarquable : \emph{{\itshape N’est-ce pas un caractère commun à toute religion chrétienne, et même à toute religion, d’être fondée sur le dogme de la Providence.}} {\itshape Recueil des Opuscules}, t. I, p. 141. — L’accusation de Damiano a été reproduite en 1821, par M. Colangelo\footnote{\noindent Damiano Romano. {\itshape Défense historique des lois grecques venues à Rome contre l’opinion moderne de M. Vico}, 1736, in-4º. — {\itshape Quatorze lettres sur le troisième principe de la science nouvelle, relatif à l’origine du langage ; ouvrage dans lequel on montre par des preuves tirées tant de la philosophie que de l’histoire sacrée et profane, que toutes les conséquences de ce principe sont fausses et erronées}, 1749. — Dans la préface de son premier ouvrage, il reconnaît que Vico a mérité l’immortalité ; dans le second, fait après la mort de Vico, il l’appelle plagiaire, etc. — Il croit prouver d’abord que le système de Vico n’est pas nouveau, et dans cette partie, malgré la diffusion et le pédantisme, l’ouvrage est assez curieux, en ce qu’il rapproche de Vico les auteurs qui ont pu le mettre sur la voie. — Il soutient ensuite que ce système est erroné, et particulièrement contraire à la religion chrétienne. Le critique bienveillant rappelle à cette occasion l’hérésie d’un Alméricus (p. 139), dont on jeta, les cendres au vent.\par
M. Colangelo. {\itshape Essai de quelques considérations sur la Science nouvelle}, dédié à M. Louis de Médicis, ministre des finances. 1821.\par
Quelques admirateurs de Vico ont appuyé ces injustes accusations, qu’ils regardaient comme autant d’éloges. Dans le désir d’ajouter Vico à la liste des philosophes du 18\textsuperscript{e} siècle, ils ont prétendu qu’il avait obscurci son livre à dessein, pour le faire passer à la censure. Cette tradition, dont on rapporte l’origine à Genovesi, a passé de lui à Galanti son biographe, et ensuite à M. de A. Les personnes qui ont le plus étudié Vico, MM. de A. et Jannelli n’y ajoutent aucune foi, et la lecture du livre suffit pour la réfuter.
}.\par
On a vu dans le discours, comment Vico abandonna la méthode analytique qu’il avait suivie d’abord pour donner à son livre une forme synthétique. Dans la seconde édition (1730), il part souvent des idées de la première comme de principes établis, et les exprime en formules qu’il emploie ensuite sans les expliquer.\par
 Dans la dernière édition (1744), l’obscurité et la confusion augmentent. On ne peut s’en étonner lorsqu’on sait comment elle fut publiée. L’auteur arrivait au terme de sa vie et de ses malheurs ; depuis plusieurs mois il avait perdu connaissance. Il paraît que son fils Gennaro Vico rassembla les notes qu’il avait pu dicter depuis l’édition de 1730, et les intercala à la suite des passages auxquels elles se rapportaient le mieux, sans entreprendre de les fondre avec le texte auquel il n’osait toucher.\par
La plupart des retranchements que nous nous sommes permis, portent sur ces additions.\par
Quoique nous n’ayons point traduit le morceau considérable, intitulé : {\itshape Idée de l’ouvrage}, et que nous ayons abrégé de moitié la {\itshape Table chronologique}, nous n’avons réellement rien retranché du 1\textsuperscript{er} livre. Tout ce que nous avons passé dans la table, se trouve placé ailleurs, et plus convenablement. Quant à l’{\itshape Idée de l’ouvrage}, Vico avoue lui-même, en tête de l’édition de 1730, qu’il y avait mis d’abord une sorte de préface qu’il supprima, et qu’il écrivit cette explication du frontispice pour remplir exactement le même nombre de pages. Ce frontispice est une sorte de représentation allégorique de {\itshape La Science nouvelle}. Debout sur le globe terrestre, la Métaphysique en extase contemple l’œil divin dans le mystérieux triangle ; elle en reçoit un rayon qui se réfléchit sur la statue d’Homère (des poèmes duquel l’auteur doit tirer une grande partie de ses preuves). Le globe pose sur un autel qui porte aussi le feu sacré et le bâton augural, la torche nuptiale et l’urne funéraire, symboles des premiers principes de la société. Sur le devant, le tableau de l’alphabet, les faisceaux, les balances, etc., désignent autant de parties du système.\par
C’est sur le second livre que portent les principaux retranchements. Le plus considérable des morceaux que nous n’avons pas cru devoir traduire, est une explication historique de la mythologie grecque et latine. Il comprend, dans le deuxième volume de l’édition de Milan (1803), les pages 101-107, 120-138, 147-156, 159, 165-171, 179, 182-185, 216-223, 235-238, 239-240, 254-268. Nous en avons rejeté l’extrait à la fin de la traduction.  Pour ne point juger cette partie du système avec une injuste sévérité, il faut se rappeler qu’au temps de Vico, la science mythologique était encore frappée de stérilité par l’opinion ancienne qui ne voyait que des démons dans les dieux du paganisme, ou renfermée dans le système presque aussi infécond de l’apothéose. Vico est un des premiers qui aient considéré ces divinités comme autant de symboles d’idées abstraites.\par
Les autres retranchements du livre II comprennent les pages 7-12, 40-46, 49, 69-71, 90-92, 188-192, 210, et en grande partie 286-288. Ceux des derniers livres ne portent que sur les pages 78-9, 81-2, 84, 133, 138-140, 143-4.\par
\par
Nous avons mentionné, à l’époque de leur publication, tous les ouvrages importants de Vico. 1708. {\itshape De nostri temporis studiorum ratione}. — 1710. {\itshape De antiquissimâ Italorum sapientiâ ex originibus linguæ latinæ eruendâ} ; trad. en italien, 1816, Milan. — 1716. {\itshape Vita di Marcesciallo Antonio Caraffa}. — 1721. {\itshape De uno juris universi principio. De constantiâ jurisprudentis}. — Enfin les trois éditions de {\itshape La Scienza nuova}, 1725, 1730, 1744. La première a été réimprimée, en 1817, à Naples, par les soins de M. Salvatore Galotti. La dernière l’a été, en 1801, à Milan ; à Naples, en 1811 et en 1816, ou 1818 ? 1821 ? Elle a été traduite en allemand par M. W. E. Weber, Leipzig, 1822. — Pour compléter cette liste, nous n’aurons qu’à suivre l’éditeur des {\itshape Opuscules} de Vico. M. Carlantonio de Rosa, marquis de Villa-Rosa, les a recueillis en quatre volumes in-8º (Naples, 1818). Nous n’avons trouvé qu’une omission dans ce recueil. C’est celle de quelques notes faites par Vico sur l’{\itshape Art poétique} d’Horace. Ces notes peu remarquables ne portent point de date. Elles ont été publiées récemment. — Les pièces inédites publiées, en 1818, par M. Antonio Giordano, se trouvent aussi dans le recueil de M. de Rosa.\par
Le premier volume du {\itshape Recueil des Opuscules} contient plusieurs écrits en prose italienne. Le plus curieux est le mémoire de Vico sur sa vie. L’estimable éditeur, descendant d’un  protecteur de Vico, y a joint une addition de l’auteur qu’il a retrouvée dans ses papiers, et a complété la vie de Vico d’après les détails que lui a transmis le fils même du grand homme. Rien de plus touchant que les pages XV et 158-168 de ce volume. Nous en avons donné un extrait. Les autres pièces sont moins importantes. — 1715. Discours sur les repas somptueux des Romains, prononcé en présence du duc de Medina-Celi, vice-roi. — Oraison funèbre d’Anne-Marie d’Aspremont, comtesse d’Althann, mère du vice-roi. Beaucoup d’originalité. Comparaison remarquable entre la guerre de la succession d’Espagne et la seconde guerre punique. — 1727. Oraison funèbre d’Angiola Cimini, marquise de la Petrella. L’argument est très beau : \emph{{\itshape Elle a enseigné par l’exemple de sa vie la douceur et l’austérité} (il soave austero){\itshape  de la vertu.}}\par
\par
Le second volume renferme quelques opuscules et un grand nombre de lettres, en italien. Le principal opuscule est la {\itshape Réponse à un article du journal littéraire d’Italie}. C’est là qu’il juge Descartes avec l’impartialité que nous avons admirée plus haut. Dans deux lettres que contient aussi ce volume (au père de Vitré, 1726, et à D. Francesco Solla, 1729), il attaque la réforme cartésienne, et l’esprit du \textsc{xviii}\textsuperscript{e} siècle, souvent avec humeur, mais toujours d’une manière éloquente. — Deux morceaux sur Dante ne sont pas moins curieux. On y trouve l’opinion reproduite depuis par Monti, que l’auteur de la {\itshape Divine Comédie} est plus admirable encore dans le purgatoire et le paradis que dans cet enfer si exclusivement admiré. — 1730. {\itshape Pourquoi les orateurs réussissent mal dans la poésie}. — {\itshape De la grammaire}. — 1720. {\itshape Remercîment à un défenseur de son système}. Dans cette lettre curieuse, Vico explique le peu de succès de {\itshape La Science nouvelle}. On y trouve le passage suivant : \emph{« Je suis né dans cette ville, et j’ai eu affaire à bien des gens pour mes besoins. Me connaissant dès ma première jeunesse, ils se rappellent mes faiblesses et mes erreurs. Comme le mal que nous  voyons dans les autres nous frappe vivement, et nous reste profondément gravé dans la mémoire, il devient une règle d’après laquelle nous jugeons toujours ce qu’ils peuvent faire ensuite de beau et de bon. D’ailleurs je n’ai ni richesses ni dignité ; comment pourrais-je me concilier l’estime de la multitude ? etc. »} — 1725. Lettre dans laquelle il se félicite de n’avoir pas obtenu la chaire de droit, ce qui lui a donné le loisir de composer {\itshape La Science nouvelle} ({\itshape Voy.} l’avant-dernière page du discours.) — Lettre fort belle sur un ouvrage qui traitait de la morale chrétienne, à M\textsuperscript{gr} Muzio Gaëta. — Lettre au même, dans laquelle il donne une idée de son livre {\itshape De antiquâ sapientiâ Italorum}. \emph{« Il y a quelques années que j’ai travaillé à un système complet de métaphysique. J’essayais d’y démontrer que l’homme est Dieu dans le monde des grandeurs abstraites, et que Dieu est géomètre dans le monde des grandeurs concrètes, c’est-à-dire dans celui de la nature et des corps. En effet, dans la géométrie l’esprit humain part du point, chose qui n’a point de parties, et qui, par conséquent, est infinie ; ce qui faisait dire à Galilée que quand nous sommes réduits au point, il n’y a plus lieu ni à l’augmentation, ni à la diminution, ni à l’égalité… Non-seulement dans les problèmes, mais aussi dans les théorèmes, connaître et faire, c’est la même chose pour le géomètre comme pour Dieu. »}\par
Les réponses des hommes de lettres auxquels écrit Vico, donnent une haute idée du public philosophique de l’Italie à cette époque. Les principaux sont Muzio Gaëta, archevêque de Bari ; un prédicateur célèbre, Michelangelo, capucin ; Nicoló Concina, de l’ordre des Prêcheurs, professeur de philosophie et de droit naturel, à Padoue, qui enseignait plusieurs parties de la doctrine de Vico ; Tommaso Marin Alfani, du même ordre, qui assure avoir été comme ressuscité après une longue maladie, par la lecture d’un nouvel ouvrage de Vico ; le duc de Laurenzano, auteur d’un ouvrage sur le bon usage des passions humaines ; enfin l’abbé Antonio Conti, noble vénitien, auteur d’une tragédie de César, et qui était lié avec Leibnitz et Newton. Vico était aussi en correspondance avec le célèbre Gravina, avec Paolo  Doria, philosophe cartésien, et avec ce prodigieux Aulisio, professeur de droit, à Naples, qui savait neuf langues, et qui écrivit sur la médecine, sur l’art militaire et sur l’histoire. D’abord ennemi de Vico, Aulisio se réconcilia avec lui après la lecture du discours {\itshape De nostri temporis studiorum ratione}. Nous n’avons ni les lettres qu’il écrivit à ces trois derniers ni leurs réponses.\par
\par
Dans le troisième volume des {\itshape Opuscules}, Vico offre une preuve nouvelle que le génie philosophique n’exclut point celui de la poésie. Ainsi sont dérangées sans cesse les classifications rigoureuses des modernes. Quoi de plus subtil, et en même temps de plus poétique que le génie de Platon ? Vico présente, par ce double caractère, une analogie remarquable avec l’auteur de la Divine comédie.\par
Mais, c’est dans sa prose, c’est dans son grand poème philosophique de la {\itshape Science nouvelle}, que Vico rappelle la profondeur et la sublimité de Dante. Dans ses poésies, proprement dites, il a trop souvent sacrifié au goût de son siècle. Trop souvent son génie a été resserré par l’insignifiance des sujets officiels qu’il traitait. Cependant plusieurs de ces pièces se font remarquer par une grande et noble facture. Voyez particulièrement, l’exaltation de Clément XII, le panégyrique de l’électeur de Bavière, Maximilien Emmanuel ; la mort d’Angela Cimini ; plusieurs sonnets, pages 7, 9, 190, 195 ; enfin un épithalame dans lequel il met plusieurs des idées de {\itshape La Science nouvelle}, dans la bouche de Junon.\par
Nous ne nous arrêterons que sur les poésies où Vico a exprimé un sentiment personnel. La première est une élégie qu’il composa à l’âge de vingt-cinq ans (1693) ; elle est intitulée {\itshape Pensées de mélancolie}. À travers les {\itshape concetti} ordinaires aux poètes de cette époque, on y démêle un sentiment vrai : \emph{« Douces images du bonheur, venez encore aggraver ma peine ! Vie pure et tranquille, plaisirs honnêtes et modérés, gloire et trésors acquis  par le mérite, paix céleste de l’âme, (et ce qui est plus poignant à mon cœur) amour dont l’amour est le prix, douce réciprocité d’une foi sincère !… »} Longtemps après, sans doute de 1720 à 1730, il répond par un sonnet à un ami qui déplorait l’ingratitude de la patrie de Vico. \emph{« Ma chère patrie m’a tout refusé !… Je la respecte et la révère. Utile et sans récompense, j’ai trouvé déjà dans cette pensée une noble consolation. Une mère sévère ne caresse point son fils, ne le presse point sur son sein, et n’en est pas moins honorée... »} La pièce suivante, la dernière du recueil de ses poésies, présente une idée analogue à celle du dernier morceau qu’il a écrit en prose ({\itshape Voy.} la fin du {\itshape Discours}). C’est une réponse au cardinal Filippo Pirelli, qui avait loué {\itshape La Science nouvelle} dans un sonnet. \emph{« Le destin s’est armé contre un misérable, a réuni sur lui seul tous les maux qu’il partage entre les autres hommes, et a abreuvé son corps et ses sens des plus cruels poisons. Mais la Providence ne permet pas que l’âme qui est à elle soit abandonnée à un joug étranger. Elle l’a conduit, par des routes écartées, à découvrir son œuvre admirable du monde social, à pénétrer dans l’abîme de sa sagesse les lois éternelles par lesquelles elle gouverne l’humanité. Et grâce à vos louanges, ô noble poète, déjà fameux, déjà antique de son vivant, il vivra aux âges futurs, l’infortuné Vico ! »}\par
\par
Le quatrième volume renferme ce que Vico a écrit en latin. La vigueur et l’originalité avec lesquelles il écrivait en cette langue eût fait la gloire d’un savant ordinaire.\par
1696. {\itshape Pro auspicatissimo in Hispaniam reditu Francisci Benavidii S. Stephani comitis atque in regno Neap. Pro rege oratio}. — 1697. {\itshape In funere Catharinæ Aragoniæ Segorbiensium ducis oratio}. — 1702. {\itshape Pro felici in Neapolitanum solium aditu Philippi V, Hispaniarum novique orbis monarchæ oratio}. — 1708. {\itshape De nostri temporis studiorum ratione oratio ad litterarum studiosam juventutem, habita in R. Neap. Academiâ}. — 1738. {\itshape In}  {\itshape Caroli et Mariæ Amaliæ utriusque Siciliæ regum nuptiis oratio. — Oratiuncula pro adsequendâ laureâ in utroque jure}. — {\itshape Carolo Borbonio utriusque Siciliæ Regi R. Neap. Academia}. — {\itshape Carolo Borbonio utriusque Siciliæ Regi epistola.}\par
1729. {\itshape Vici vindiciæ sive notæ in acta eruditorum Lipsiensia mensis augusti A. 1727, ubi inter nova literaria unum extat de ejus libro, cui titulus : Principi d’una scienza nuova d’intorno alla commune natura delle nazioni}. Cet article, où l’on reproche à Vico d’avoir \emph{{\itshape approprié son système au goût de l’Église romaine}}, avait été envoyé par un Napolitain. La violence avec laquelle Vico répond à un adversaire obscur, ferait quelquefois sourire, si l’on ne connaissait la position cruelle où se trouvait alors l’auteur. \emph{« Lecteur impartial, dit-il en terminant, il est bon que tu saches que j’ai dicté cet opuscule au milieu des douleurs d’une maladie mortelle, et lorsque je courais les chances d’un remède cruel qui, chez les vieillards, détermine souvent l’apoplexie. Il est bon que tu saches que depuis vingt ans j’ai fermé tous les livres, afin de porter plus d’originalité dans mes recherches sur le droit des gens ; le seul livre où j’ai voulu lire c’est le sens commun de l’humanité. »} Ce qui rend cet opuscule précieux, c’est qu’en plusieurs endroits Vico déclare que le sujet propre de la Science nouvelle, c’est {\itshape la nature commune aux nations}, et que son système du droit des gens n’en est que le principal corollaire.\par
1708. {\itshape Oratio cujus argumentum, hostem hosti infensiorem infestioremque quam stultum sibi esse neminem}. Nul n’a d’ennemi plus cruel et plus acharné que l’insensé ne l’est de lui-même. — 1732. {\itshape De mente heroicâ oratio habita in R. Neap. academiâ}. L’héroïsme dont parle Vico est celui d’une grande âme, d’un génie courageux qui ne craint point d’embrasser dans ses études l’universalité des connaissances, et qui veut donner à sa nature le plus haut développement qu’elle comporte. Nulle part il ne s’est plus abandonné à l’enthousiasme qu’inspire la science considérée dans son ensemble et dans son harmonie. Cet ouvrage, qui semble porter l’empreinte d’une composition très rapide,  est surtout remarquable par la chaleur et la poésie du style. L’auteur avait cependant soixante-quatre ans.\par
Ajoutez à cette liste des ouvrages latins de Vico, un grand nombre de belles inscriptions. Voici l’indication des plus considérables : Inscriptions funéraires en l’honneur de D. Joseph Capece et D. Carlo de Sangro, 1707, faites par ordre du comte de Daun, général des armées impériales dans le royaume de Naples. — Autre en l’honneur de l’empereur Joseph, 1711, faite par ordre du vice-roi, Charles Borromée. — Autre en l’honneur de l’impératrice Éléonore, faite par ordre du cardinal Wolfgang de Scratembac, vice-roi.\par
\par
Nous avons déjà nommé la plupart des auteurs qui ont mentionné Vico ({\itshape Journal de Trévoux}, 1726, septembre ; page 1742). — {\itshape Journal de Leipzig}, 1727, août, page 383. — {\itshape Bibliothèque ancienne et moderne} de Leclerc, tome XVIII, partie II, pag. 426. — Damiano Romano. — Duni ? {\itshape Governo civile}. — Cesarotti (sur Homère). — Parini (dans ses cours à Milan). — Joseph de Cesare. {\itshape Pensées de Vico sur. …} 18… ? — Signorelli. — Romagnosi (de Parme). — L’abbé Talia. {\itshape Lettres sur la philosophie morale}, 1817, Padoue. — Colangelo — ({\itshape Biblioteca analitica, passim}). — Joignez-y Herder, dans ses opuscules, et Wolf dans son {\itshape Musée des sciences de l’antiquité} (tome I, page 555). Ce dernier n’a extrait que la partie de {\itshape La Science nouvelle} relative à Homère. — Aucun Anglais, aucun Écossais, que je sache, n’a fait mention de Vico, si ce n’est l’auteur d’une brochure récemment publiée sur l’état des études en Allemagne et en Italie. — En France, M. Salfi est le premier qui ait appelé l’attention du public sur la Science nouvelle, dans son {\itshape Éloge de Filangieri}, et dans plusieurs numéros de la {\itshape Revue encyclopédique}, t. II, p. 540 ; t. VI, p. 364 ; t. VII, p. 343. — {\itshape Voy.} aussi {\itshape Mémoires du comte Orloff sur Naples}, 1821, t. IV, p. 439, et t. V, p. 7.\par
Vico n’a point laissé d’école ; aucun philosophe italien n’a  saisi son esprit dans tout le siècle dernier ; mais un assez grand nombre d’écrivains ont développé quelques-unes de ses idées. Nous donnons ici la liste des principaux.\par
Genovesi (né en 1712, mort en 1769). N’ayant pu me procurer que deux des nombreux ouvrages de ce disciple illustre de Vico (les {\itshape Institutions} et la {\itshape Diceosina}), je donne les titres de tous les livres qu’il a faits, en faveur de ceux qui seraient à même de faire de plus amples recherches. — {\itshape Leçons d’économie politique et commerciale}. — {\itshape Méditations philosophiques} (sur la religion et la morale), 1758. — {\itshape Institutions de métaphysique à l’usage des commençants}. — {\itshape Lettre académique} (sur l’utilité des sciences, contre le paradoxe de J.-J. Rousseau), 1764. — {\itshape Logique à l’usage des jeunes gens}, 1766 (divisée en cinq parties : {\itshape emendatrice, inventrice, giudicatrice, ragionatrice, ordonatrice}. On estime le dernier chapitre, {\itshape Considérations sur les sciences et les arts}). — {\itshape Traité des sciences métaphysiques}, 1764 (divisé en cosmologie, théologie, anthropologie). — {\itshape Dicéosine, ou science des droits et des devoirs de l’homme}, 1767 ; ouvrage inachevé. C’est surtout dans le troisième volume de la {\itshape Dicéosine} que Genovesi expose des idées analogues à celles de Vico.\par
Filangieri (né en 1752, mort en 1788). Quoique cet homme célèbre n’ait rien écrit qui se rattache au système de Vico, nous croyons devoir le placer dans cette liste. À l’époque de sa mort prématurée, il méditait deux ouvrages ; le premier eût été intitulé : {\itshape Nouvelle science des sciences} ; le second : {\itshape Histoire civile, universelle et perpétuelle}. Il n’est resté qu’un fragment très court du premier, et rien du second. J’ai cherché inutilement ce fragment.\par
Cuoco (mort en 1822). {\itshape Voyage de Platon en Italie}. Ouvrage très superficiel et qui exagère tous les défauts du {\itshape Voyage d’Anacharsis}. Les hypothèses historiques de Vico ont souvent chez Cuoco un air plus paradoxal encore, parce qu’on n’y voit plus les principes dont elles dérivent. Ce sont à peu près les mêmes idées sur l’{\itshape Histoire éternelle}, sur l’Histoire romaine en particulier sur les douze tables, sur l’âge et la patrie d’Homère, etc.  Au moment où les persécutions égarèrent la raison du malheureux Cuoco, il détruisit un travail fort remarquable, dit-on, sur le système de la Science nouvelle.\par
L’infortuné Mario Pagano (né en 1750, mort en 1800), est de tous les publicistes celui qui a suivi de plus près les traces de Vico. Mais quel que soit son talent, on peut dire que, dans ses {\itshape Saggi politici}, les idées de Vico ont autant perdu en originalité que gagné en clarté. Il ne fait point marcher de front, comme Vico, l’histoire des religions, des gouvernements, des lois, des mœurs, de la poésie, etc. Le caractère religieux de la Science nouvelle a disparu. Les explications physiologiques qu’il donne à plusieurs phénomènes sociaux, ôtent au système sa grandeur et sa poésie, sans l’appuyer sur une base plus solide. Néanmoins les {\itshape Essais politiques} sont encore le meilleur commentaire de la Science nouvelle. Voici les points principaux dans lesquels il s’en écarte. 1º Il pense avec raison que la {\itshape seconde barbarie}, celle du moyen âge, n’a pas été aussi semblable à la première que Vico paraît le croire. 2º Il estime davantage la sagesse orientale. 3º Il ne croit pas que {\itshape tous} les hommes après le déluge soient tombés dans un état de brutalité complète. 4º Il explique l’origine des mariages, non par un sentiment religieux, mais par la jalousie. Les plus forts auraient enlevé les plus belles, auraient ainsi formé les premières familles et fondé la première noblesse. 5º Il croit qu’à l’origine de la société, les hommes furent, non pas agriculteurs, comme l’ont cru Vico et Rousseau, mais chasseurs et pasteurs.\par
Chez tous les écrivains que nous venons d’énumérer, les idées de Vico sont plus ou moins modifiées par l’esprit français du dernier siècle. Un philosophe de nos jours me semble mieux mériter le titre de disciple légitime de Vico. C’est M. Cataldo Jannelli, employé à la bibliothèque royale de Naples, qui a publié, en 1817, un ouvrage intitulé : {\itshape Essai sur la nature et la nécessité de la science des choses et histoires humaines}. Nous n’entreprendrons pas de juger ce livre remarquable. Nous observerons seulement que l’auteur ne semble pas tenir assez de compte  de la perfectibilité de l’homme. Il compare trop rigoureusement l’humanité à un individu, et croit qu’elle aura sa vieillesse comme sa jeunesse et sa virilité (page 58).\par
\par
Il ne nous reste qu’à donner la liste des principaux auteurs français, anglais et allemands qui ont écrit sur la philosophie de l’histoire. Lorsque nous n’étions pas sûr d’indiquer avec exactitude le titre de l’ouvrage, nous avons rapporté seulement le nom de l’auteur.\par
{\scshape France.} Bossuet. {\itshape Discours sur l’histoire universelle}, 1681. — Voltaire. {\itshape Philosophie de l’histoire. Essai sur l’esprit et les mœurs des nations}, commencé en 1740, imprimé en 1785. — Turgot. {\itshape Discours sur les avantages que l’établissement du christianisme a procurés au genre humain. Autre sur les progrès de l’esprit humain. Essais sur la géographie politique. Plan d’histoire universelle. Progrès et décadences alternatives des sciences et des arts. Pensées détachées}. Ces divers morceaux sont ce que nous avons de plus original et de plus profond sur la philosophie de l’histoire. L’auteur les a écrits à l’âge de vingt-cinq ans, lorsqu’il était au séminaire, de 1750 à 1754. {\itshape Voy.} le second volume des {\itshape Œuvres complètes}, 1810. — Condorcet. {\itshape Esquisse d’un tableau historique des progrès de l’esprit humain} ; écrit en 1793, publié en 1799. — M\textsuperscript{me} de Staël, {\itshape passim}, et surtout dans son ouvrage sur la {\itshape Littérature considérée dans ses rapports avec les institutions politiques}. — Walckenaer. {\itshape Essai sur l’histoire de l’espèce humaine}. — Cousin. {\itshape De la philosophie de l’histoire} ; très court, mais très éloquent, dans ses {\itshape Fragments philosophiques} ; écrit en 1818, imprimé en 1826.\par
{\scshape Angleterre.} Ferguson. {\itshape Essai sur l’histoire de la société civile}, 1767 ; trad. — Millar. {\itshape Observations sur les distinctions de rang dans la société}, 1771. — Kames. {\itshape Essais sur l’histoire de l’homme}, 1773. — Dunbar, {\itshape Essais sur l’histoire de l’humanité}, 1780. — Price… 1787. — Priestley. {\itshape Discours sur l’histoire} ; traduits.\par
{\scshape Allemagne.} Iselin. {\itshape Histoire du genre humain}, 1764. — Herder.  {\itshape Idées philosophiques sur l’histoire de l’humanité}, 1772 (traduit par M. Edgard Quinette, 1837). — Kant. {\itshape Idée de ce que pourrait être une histoire universelle, considérée dans les vues d’un citoyen du monde} (traduit par Villiers dans {\itshape Le Conservateur}, tome II, an VIII). Autres opuscules du même, sur l’identité de la race humaine, sur le commencement de l’histoire du genre humain, sur la théorie de la pure religion morale, etc. (traduits dans le même volume du {\itshape Conservateur}, ou dans les {\itshape Archives philosophiques et littéraires}, tome VIII). — Lessing. {\itshape Éducation du genre humain}, 1786. — Meiners. {\itshape Histoire de l’humanité}, 1786. Voyez aussi ses autres ouvrages {\itshape passim}. — Carus. {\itshape Idées pour servir à l’histoire du genre humain}. — Ancillon. {\itshape Essais philosophiques, ou nouveaux mélanges}, etc., 1817. {\itshape Voy.} philosophie de l’histoire, dans le premier volume ; perfectibilité, dans le second (écrit en français).\par
Ajoutez à cette liste un nombre infini d’ouvrages dont le sujet est moins général, mais qui n’en sont pas moins propres à éclairer la philosophie de l’histoire ; tels que l’{\itshape Histoire de la culture et de la littérature en Europe}, par Eichhorn ; la {\itshape Symbolique} de Creuzer\footnote{Orthographié « Creutzer » [NdE].}, etc.\par

\byline{}

\chapteropen
\part[{Principes de la philosophie de l’histoire}]{Principes de la philosophie de l’histoire}\renewcommand{\leftmark}{Principes de la philosophie de l’histoire}


\byline{}

\chaptercont

\chapteropen
\part[{Livre premier. Des principes}]{Livre premier. \\
Des principes}

\chaptercont

\chapteropen
\chapter[{Argument}]{Argument}

\chaptercont
\noindent  {\itshape On ne peut déterminer quelles lois observe la civilisation dans son développement, sans remonter à son origine. L’auteur prouve d’abord la nécessité de suivre dans cette recherche une nouvelle méthode, par l’insuffisance et la contradiction de tout ce qu’on a dit sur l’histoire ancienne jusqu’à la seconde guerre punique} (chap. I.){\itshape  — Il expose ensuite sous la forme d’axiomes, les vérités générales qui font la base de son système} (chap. II.){\itshape  — Il indique enfin les trois grands principes d’où part la science nouvelle, et la méthode qui lui est propre} (chap. III et IV.)\par
\par
 Chapitre I\textsuperscript{er}. {\scshape Table chronologique.} {\itshape Vaines prétentions des Égyptiens à une science profonde et à une antiquité exagérée. Le peuple hébreu est le plus ancien de tous. Division de l’histoire des premiers siècles en trois périodes.} — 1. {\itshape Déluge. Géants. Âge d’or. Premier Hermès.} — 2. {\itshape Hercule et les Héraclides. Orphée. Second Hermès. Guerre de Troie. Colonies grecques de l’Italie et de la Sicile.} — 3. {\itshape Jeux olympiques. Fondation de Rome. Pythagore. Servius Tullius. Hésiode, Hippocrate et Hérodote. Thucydide ; guerre du Péloponnèse}\footnote{Orthographié « Péloponèse » [NdE].}{\itshape . Xénophon ; Alexandre. Lois Publilia et Petilia. Guerre de Tarente et de Pyrrhus. Seconde guerre punique.}\par
{\itshape Dans ce chapitre, l’auteur jette en passant les fondements d’une critique nouvelle} : 1º {\itshape La civilisation de chaque peuple a été son propre ouvrage, sans communication du dehors} ; 2º {\itshape On a exagéré la sagesse ou la puissance des premiers peuples} ; 3º {\itshape On a pris pour des individus des êtres allégoriques ou collectifs (Hercule, Hermès.)}\par
\par
Chapitre {\scshape II. Axiomes}. 1-22.{\itshape  Axiomes généraux.} 23-114.{\itshape  Axiomes particuliers. == 1-4. Réfutation des opinions que l’on s’est formées jusqu’ici sur les commencements de la civilisation. —} 5-15{\itshape . Fondements du vrai. Méditer le monde social dans son idée éternelle. —} 16-22{\itshape . Fondements du certain. Apercevoir le monde social dans sa réalité.} 23-28{\itshape . Division des peuples anciens en hébreux et gentils. Déluge}  {\itshape universel. Géants. —} 28-30{\itshape . Principes de la théologie poétique. —} 31-40{\itshape . Origine de l’idolâtrie, de la divination, des sacrifices. —} 41-46{\itshape . Principes de la mythologie historique. —} 47-62{\itshape . Poétique. —} 47-49{\itshape . Principe des caractères poétiques. —} 50-62{\itshape . Suite de la poétique. Fable, convenance, pensée, expression, chant, vers. —} 63-65{\itshape . Principes étymologiques. —} 66-96{\itshape . Principes de l’histoire idéale. == —} 70-84{\itshape . Origine des sociétés. — 84-96. Ancienne histoire romaine. —} 97-103{\itshape . Migrations des peuples. —} 104-114{\itshape . Principes du droit naturel.}\par
Chapitre {\scshape III. Trois principes fondamentaux.} — Religions et croyance à une Providence, mariages et modération des passions, sépultures et croyance à l’immortalité de l’âme.\par
Chapitre {\scshape IV. De la méthode.} — Le point de départ de la science nouvelle est la première pensée humaine que les hommes durent concevoir, à savoir, l’idée d’un Dieu. == Cette science emploie d’abord des preuves philosophiques, ensuite des preuves philologiques.\par
{\itshape Les preuves} philosophiques {\itshape elles-mêmes sont ou théologiques ou logiques. La science nouvelle est une} démonstration historique de la Providence ; {\itshape elle trace le cercle éternel d’une} histoire idéale {\itshape dans lequel tourne l’histoire réelle de toutes les nations. Elle s’appuie sur une} critique nouvelle, {\itshape dont le criterium est le} sens commun du genre humain. {\itshape Cette}  {\itshape critique est le fondement d’un nouveau système du} droit des gens.\par
Preuves philologiques, {\itshape tirées de l’interprétation des fables, de l’histoire des langues, etc.}
\chapterclose


\chapteropen
\chapter[{Chapitre premier. Table chronologique, ou préparation des matières. que doit mettre en œuvre la science nouvelle}]{Chapitre premier. \\
Table chronologique, ou préparation des matières \\
que doit mettre en œuvre la science nouvelle}

\chaptercont
\noindent  La table chronologique que l’on a sous les yeux embrasse l’histoire du monde ancien, depuis le déluge jusqu’à la seconde guerre punique, en commençant par les Hébreux, et continuant par les Chaldéens, les Scythes, les Phéniciens, les Égyptiens, les Grecs et les Romains. On y voit figurer des hommes ou des faits célèbres, lesquels sont ordinairement placés par les savants dans d’autres temps, dans d’autres lieux, ou qui même n’ont point existé. En récompense nous y tirons des ténèbres profondes où ils étaient restés ensevelis, des hommes et des faits remarquables, qui ont puissamment influé sur le cours des choses humaines ; et nous montrons combien les explications  qu’on a données sur l’{\itshape origine de la civilisation}, présentent d’incertitude, de frivolité et d’inconséquence.\par
\par
Mais toute étude sur la civilisation païenne doit commencer par un examen sévère des prétentions des nations anciennes, et surtout des Égyptiens, à une antiquité exagérée. Nous tirerons deux utilités de cet examen : celle de savoir à quelle époque, à quel pays il faut rapporter les commencements de cette civilisation ; et celle d’appuyer par des preuves, humaines à la vérité, tout le système de notre religion, laquelle nous apprend d’abord que le premier peuple fut le peuple hébreu, que le premier homme fut Adam, créé en même temps que ce monde par le Dieu véritable\footnote{V. p. 50, édition de Milan, 1801.}.\par
Notre chronologie se trouve entièrement contraire au système de Marsham, qui veut prouver que les Égyptiens devancèrent toutes les nations dans la religion et dans la politique, de sorte que leurs rites sacrés et leurs règlements civils, transmis aux autres peuples, auraient été reçus des Hébreux avec quelques changements. Avant d’examiner ce qu’on doit croire de cette antiquité, il faut avouer qu’elle ne paraît pas avoir profité beaucoup aux Égyptiens. Nous voyons dans les {\itshape Stromates} de saint Clément d’Alexandrie, que les livres du leurs prêtres,  au nombre de quarante-deux, couraient alors dans le public, et qu’ils contenaient les plus graves erreurs en philosophie et en astronomie. Leur médecine, selon Galien, {\itshape De Medicinâ mercuriali}, était un tissu de puérilités et d’impostures. Leur morale était dissolue, puisqu’elle permettait, qu’elle honorait même la prostitution. Leur théologie n’était que superstitions, prestiges et magie. Les arts du fondeur et du sculpteur restèrent chez eux dans l’enfance ; et quant à la magnificence de leurs pyramides, on peut dire que la grandeur n’est point inconciliable avec la barbarie.\par
C’est la fameuse Alexandrie qui a ainsi exalté l’antique sagesse des Égyptiens. La cité d’Alexandre unit la subtilité africaine à l’esprit délicat des Grecs, et produisit des philosophes profonds dans les choses divines. Célébrée comme la {\itshape mère des sciences}, désignée chez les Grecs par le nom de πόλις, {\itshape la ville} par excellence, elle vit son Musée aussi célèbre que l’avaient été à Athènes l’académie, le lycée et le portique. Là s’éleva le grand prêtre Manéthon\footnote{Orthographié « Manéton » [NdE].}, qui donna à toute l’histoire de l’Égypte l’interprétation d’une sublime théologie naturelle, précisément comme les philosophes grecs avaient donné à leurs fables nationales un sens tout philosophique. ({\itshape Voy.} le commencement du livre II.) Dans ce grand entrepôt du commerce de la Méditerranée et de l’Orient, un peuple si vaniteux\footnote{\emph{{\itshape Gloriæ animalia}}, et dans Tacite : \emph{{\itshape Gens novarum religionum avida}}.}, avide de superstitions  nouvelles, imbu du préjugé de son antiquité prodigieuse et des vastes conquêtes de ses rois, ignorant enfin que les autres nations païennes avaient pu, sans rien savoir l’une de l’autre, concevoir des idées uniformes sur les dieux et sur les héros, ce peuple, dis-je, ne put s’empêcher de croire que tous les dieux des navigateurs qui venaient commercer chez lui, étaient d’origine égyptienne. Il voyait que toutes les nations avaient leur Jupiter et leur Hercule ; il décida que son Jupiter Ammon était le plus ancien de tous, que tous les Hercule avaient pris leur nom de l’Hercule Égyptien.\par
Diodore de Sicile, qui vivait du temps d’Auguste, et qui traite les Égyptiens trop favorablement, ne leur donne que deux mille ans d’antiquité, encore a-t-il été réfuté victorieusement par Giacomo Cappello dans son {\itshape Histoire sacrée et égyptienne}. Cette antiquité n’est pas mieux prouvée par le Pimandre. Ce livre que l’on a vanté comme contenant la doctrine d’Hermès, est l’œuvre d’une imposture évidente. Casaubon n’y trouve pas une doctrine plus ancienne que le platonisme, et Saumaise ne le considère que comme une compilation indigeste.\par
L’intelligence humaine, étant infinie de sa nature, exagère les choses qu’elle ignore, bien au-delà de la réalité. Enfermez un homme endormi dans un lieu très étroit, mais parfaitement obscur, l’horreur des ténèbres le lui fait croire certainement plus grand qu’il ne le trouvera en touchant les murs  qui l’environnent. Voilà ce qui a trompé les Égyptiens sur leur antiquité.\par
Même erreur chez les Chinois, qui ont fermé leur pays aux étrangers, comme le firent les Égyptiens jusqu’à Psammétique, et les Scythes jusqu’à l’invasion de Darius, fils d’Hystaspe. Quelques jésuites ont vanté l’antiquité de Confucius, et ont prétendu avoir lu des livres imprimés avant Jésus-Christ ; mais d’autres auteurs mieux informés ne placent Confucius que cinq cents ans avant notre ère, et assurent que les Chinois n’ont trouvé l’imprimerie que deux siècles avant les Européens. D’ailleurs la philosophie de Confucius, comme celle des livres sacrés de l’Égypte, n’offre qu’ignorance et grossièreté dans le peu qu’elle dit des choses naturelles. Elle se réduit à une suite de préceptes moraux dont l’observance est imposée à ces peuples par leur législation.\par
Dans cette dispute des nations sur la question de leur antiquité, une tradition vulgaire veut que les Scythes aient l’avantage sur les Égyptiens. Justin commence l’histoire universelle par placer même avant les Assyriens deux rois puissants, Tanaïs le scythe, et l’égyptien Sésostris. D’abord Tanaïs part avec une armée innombrable pour conquérir l’Égypte, ce pays si bien défendu par la nature contre une invasion étrangère. Ensuite Sésostris, avec une armée non moins nombreuse, s’en va subjuguer la Scythie, laquelle n’en reste pas moins inconnue jusqu’à ce qu’elle soit envahie par Darius. Encore  à cette dernière époque qui est celle de la plus haute civilisation des Perses, les Scythes se trouvent-ils si barbares que leur roi ne peut répondre à Darius qu’en lui envoyant des signes matériels sans pouvoir même écrire sa pensée en hiéroglyphes. Les deux conquérants traversent l’Asie avec leurs prodigieuses armées sans la soumettre ni aux Scythes ni aux Égyptiens. Elle reste si bien indépendante, qu’on y voit s’élever ensuite la première des quatre monarchies les plus célèbres, celle des Assyriens.\par
La prétention de ces derniers à une haute antiquité est plus spécieuse. En premier lieu leur pays est situé dans l’intérieur des terres, et nous démontrerons dans ce livre que les peuples habitèrent d’abord les contrées méditerranées et ensuite les rivages. Ajoutez qu’on regarde généralement les Chaldéens comme les premiers sages du paganisme, en plaçant Zoroastre à leur tête. De la tribu chaldéenne, se forma sous Ninus la grande nation des Assyriens, et le nom de la première se perdit dans celui de la seconde. Mais les Chaldéens ont été jusqu’à prétendre qu’ils avaient conservé des observations astronomiques d’environ vingt-huit mille ans. Josèphe\footnote{Orthographié « Josephe » [NdE].} a cru à ces observations antédiluviennes, et a prétendu qu’elles avaient été inscrites sur deux colonnes, l’une de marbre, l’autre de brique, qui devaient les préserver du déluge ou du l’embrasement du monde. On peut placer les deux colonnes dans le {\itshape Musée de la crédulité}.\par
Les Hébreux au contraire, étrangers aux nations  païennes, comme l’attestent Josèphe et Lactance, n’en connurent pas moins le nombre exact des années écoulées depuis la création ; c’est le calcul de Philon, approuvé par les critiques les plus sévères, et dont celui d’Eusèbe ne s’écarte d’ailleurs que de quinze cents ans, différence bien légère en comparaison des altérations monstrueuses qu’ont fait subir à la chronologie les Chaldéens, les Scythes, les Égyptiens et les Chinois. Il faut bien reconnaître que les Hébreux ont été le premier peuple, et qu’ils ont conservé sans altération les monuments de leur histoire depuis le commencement du monde.\par
Après les {\itshape Hébreux}, nous plaçons les {\itshape Chaldéens} et les {\itshape Scythes}, puis les {\itshape Phéniciens}. Ces derniers doivent précéder les {\itshape Égyptiens}, puisque, selon la tradition, ils leur ont transmis les connaissances astronomiques qu’ils avaient tirées de la Chaldée, et qu’ils leur ont donné en outre les caractères alphabétiques, comme nous devons le démontrer.\par
\par
Si nous ne donnons aux Égyptiens que la cinquième place dans cette table, nous ne profiterons pas moins de leurs antiquités. Il nous en reste deux grands débris, aussi admirables que leurs pyramides. Je parle de deux vérités historiques, dont l’une nous a été conservée par Hérodote : 1º Ils divisaient tout le temps antérieurement écoulé en trois âges, {\itshape âge des dieux, âge des héros, âge des hommes} ; 2º pendant ces trois âges, trois langues  correspondantes se parlèrent, langue hiéroglyphique ou {\itshape sacrée}, langue symbolique ou {\itshape héroïque}, langue {\itshape vulgaire}, celle dans laquelle les hommes expriment par des signes convenus les besoins ordinaires de la vie. De même, Varron dans ce grand ouvrage {\itshape Rerum divinarum et humanarum}, dont l’injure des temps nous a privés, divisait l’ensemble des siècles écoulés en trois périodes, {\itshape temps obscur}, qui répond à l’âge divin des Égyptiens, {\itshape temps fabuleux}, qui est leur âge héroïque, enfin {\itshape temps historique}, l’âge des hommes, dans la nomenclature égyptienne.\par
\emph{{\itshape Des nations civilisées ou barbares, il n’en est aucune}, selon l’observation de Diodore, {\itshape  qui ne se regarde comme la plus ancienne, et qui ne fasse remonter ses annales jusqu’à l’origine du monde.}} Les Égyptiens nous fourniront encore à l’appui de ce principe deux traditions de vanité nationale, savoir, que Jupiter Ammon était le plus ancien de tous les Jupiter, et que les Hercule des autres nations avaient pris leur nom de l’Hercule Égyptien.\par
\par
(An du monde, 1656.) Le {\itshape déluge universel} est notre point de départ. La confusion des langues qui suivit eut lieu chez les enfants de Sem, chez les peuples orientaux. Mais il en fut sans doute autrement chez les nations sorties de Cham et de Japhet (ou Japet) ; les descendants de ces deux fils de Noé durent se disperser dans la vaste forêt qui couvrait la terre. Ainsi errants  et solitaires, ils perdirent bientôt les mœurs humaines, l’usage de la parole, devinrent semblables aux animaux sauvages, et reprirent la taille gigantesque des hommes antédiluviens. Mais lorsque la terre desséchée put de nouveau produire le tonnerre par ses exhalaisons, les géants épouvantés rapportèrent ce terrible phénomène à un Dieu irrité. Telle est l’origine de tant de Jupiter, qui furent adorés des nations païennes. De là la divination appliquée aux phénomènes du tonnerre, au vol de l’aigle, qui passait pour l’oiseau de Jupiter. Les Orientaux se firent une divination moins grossière ; ils observèrent le mouvement des planètes, les divers aspects des astres, et leur premier sage fut Zoroastre (selon Bochart, \emph{{\itshape le contemplateur des astres}}.) — Ce système ruine nécessairement celui des étymologistes qui cherchent dans l’Orient l’origine de toutes les langues. Selon nous, toutes les nations sorties de Cham et de Japhet se créèrent leurs langues dans les contrées méditerranées où elles s’étaient fixées d’abord ; puis descendant vers les rivages, elles commencèrent à commercer avec les Phéniciens, peuple navigateur qui couvrit de ses colonies les bords de la Méditerranée et de l’Océan.\par
(Ans du monde, 2000-2500.) Dès que les géants, quittant leur vie vagabonde, se mettent à cultiver les champs, nous voyons commencer l’{\itshape âge d’or} ou {\itshape âge divin} des Grecs, et quelques siècles après celui du Latium, l’{\itshape âge de Saturne}, dans lequel les dieux vivaient sur la terre avec les hommes.\par
 Dans cet âge divin paraît d’abord le premier Hermès. \emph{{\itshape Les Égyptiens}, dit Jamblique, {\itshape  rapportaient à cet Hermès toutes les inventions nécessaires ou utiles à la vie sociale.}} C’est qu’Hermès ne fut point un sage, un philosophe divinisé après sa mort, mais le caractère idéal des premiers hommes de l’Égypte, qui sans autre sagesse que celle de l’instinct naturel, y formèrent d’abord des familles, puis des tribus, et fondèrent enfin une grande nation\footnote{Est-il vrai que, dans cette période, Hermès ait porté d’Égypte en Grèce la connaissance des lettres et les premières lois ? ou bien Cadmus aurait-il enseigné aux Grecs l’alphabet de la Phénicie ? Nous ne pouvons admettre ni l’une ni l’autre opinion. — Les Grecs ne se servirent point d’hiéroglyphes comme les Égyptiens, mais d’une écriture alphabétique, encore ne l’employèrent-ils que bien des siècles après. — Homère confia ses poèmes à la mémoire des Rapsodes, parce que de son temps les lettres alphabétiques n’étaient point trouvées, ainsi que le soutient Josèphe contre le sentiment d’Appion. — Si Cadmus eût porté les lettres phéniciennes en Grèce, la Béotie qui les eût reçues la première n’eût-elle pas dû se distinguer par sa civilisation entre toutes les parties de la Grèce ? — D’ailleurs quelle différence entre les lettres grecques et les phéniciennes ? == Quant à l’introduction simultanée des lois et des lettres, les difficultés sont plus grandes encore. — D’abord le mot \foreign{νόμος} ne se trouve nulle part dans Homère. — Ensuite, est-il indispensable que des lois soient écrites ? n’en existait-il pas en Égypte avant Hermès, inventeur des lettres ? dira-t-on qu’il n’y eut pas de lois à Sparte où Lycurgue avait défendu aux citoyens l’étude des lettres ? ne voit-on pas dans Homère un Conseil des héros, \foreign{βουλή}, où l’on délibérait de vive voix sur les lois, et un Conseil du peuple, \foreign{ἀγορά}, où on les publiait de la même manière. La Providence a voulu que les sociétés qui n’ont point encore la connaissance des lettres se fondent d’abord sur les usages et les coutumes, pour se gouverner ensuite par des lois, quand elles sont plus civilisées. Lorsque la barbarie antique reparut au moyen âge, ce fut encore sur des coutumes que se fonda le droit chez toutes les nations européennes.}. D’après la division des trois âges que reconnaissaient les Égyptiens,  Hermès devait être un dieu, puisque sa vie embrassait tout ce qu’on appelait l’{\itshape âge des dieux} dans cette nomenclature\footnote{Les héros investis du triple caractère de chefs des peuples, de guerriers et de prêtres, furent désignés dans la Grèce par le nom d’{\itshape Héraclides}, ou enfants d’Hercule ; dans la Crète, dans l’Italie et dans l’Asie mineure, par celui de {\itshape Curètes} ({\itshape quirites}, de l’inusité {\itshape quir, quiris}, lance).}.\par
(An du monde, 3223-3223.) L’{\itshape âge héroïque} qui suit celui des dieux, est caractérisé par Hercule, Orphée et le second Hermès. L’Occident a ses Hercule, l’Orient ses Zoroastre qui présentent le même caractère. Autant de types idéaux des fondateurs des sociétés, et des poètes théologiens. Si l’on s’obstine à ne voir que des hommes dans ces êtres allégoriques, que de difficultés se présentent\footnote{\noindent Orphée surtout, si on le considère comme un individu, offre aux yeux de la critique l’assemblage de mille monstres bizarres. — D’abord il vient de Thrace, pays plus connu comme la patrie de Mars, que comme le berceau de la civilisation. — Ce Thrace sait si bien le grec qu’il compose en cette langue des vers d’une poésie admirable. — Il ne trouve encore que des bêtes farouches dans ces Grecs, auxquels tant de siècles auparavant Deucalion a enseigné la piété envers les dieux, dont Hellen a formé une même nation en leur donnant une langue commune, chez lesquels enfin règne depuis trois cents ans la maison d’Inachus. — Orphée trouve la Grèce sauvage, et en quelques années elle fait assez de progrès pour qu’il puisse suivre Jason à la conquête de la Toison d’or ; remarquez que la marine n’est point un des premiers arts dont s’occupent les peuples. — Dans cette expédition il a pour compagnons Castor et Pollux, frères d’Hélène, dont l’enlèvement causa la fameuse guerre de Troie. Ainsi, la vie d’un seul homme nous présente plus de faits qu’il ne s’en passerait en mille années !… Ce sont peut-être de semblables observations qui ont fait conjecturer à Cicéron, dans son livre sur la Nature des Dieux, qu’{\itshape Orphée n’a jamais existé}. Elles s’appliquent, pour la plupart, avec la même force à Hercule, à Hermès et à Zoroastre.À ces difficultés chronologiques, joignez-en d’autres, morales ou politiques. Orphée, voulant améliorer les mœurs de la Grèce, lui propose l’exemple d’un Jupiter adultère, d’une Junon implacable qui persécute la vertu dans la personne d’Hercule, d’un Saturne qui dévore ses enfants ! et c’est par ces fables capables de corrompre et d’abrutir le peuple le plus civilisé, le plus vertueux, qu’Orphée élève les hommes encore bruts à l’humanité et à la civilisation.\par
Guidés par les principes de la science nouvelle, nous éviterons ces terribles écueils de la {\itshape mythologie} ; nous verrons que ces fables, détournées de leur sens par la corruption des hommes, ne signifiaient dans l’origine rien que de vrai, rien qui ne fût digne des fondateurs des sociétés. La découverte des caractères poétiques, des types idéaux, que nous venons d’exposer, fera luire un jour pur et serein à travers ces nuages sombres dont s’était voilée la {\itshape chronologie}.
} !\par
(An du monde, 2820.) D’habiles critiques ont porté plus loin le scepticisme : ils ont pensé que la {\itshape guerre de Troie} n’avait  jamais eu lieu, du moins telle qu’Homère la raconte ; et ils ont renvoyé à la {\itshape Bibliothèque de l’Imposture} les Dictys de Crète, et les Darès de Phrygie, qui en ont écrit l’histoire en prose, comme s’ils eussent été contemporains.\par
\par
(Vers 2950.) Dans le siècle qui suit immédiatement la guerre de Troie, et à la suite des courses errantes d’Énée et d’Anténor\footnote{Orthographié « Antenor » [NdE].}, de Diomède et d’Ulysse, nous plaçons {\itshape la fondation des colonies grecques de l’Italie et de la Sicile}. C’est trois siècles avant l’époque adoptée par les chronologistes ; mais ont-ils le droit de s’en étonner, eux qui varient de quatre cent soixante ans sur le temps où vécut Homère, l’auteur  le plus voisin de ces événements. La fondation de ces colonies est du petit nombre des faits dans lesquels nous nous écartons de la chronologie ordinaire, mais nous y sommes contraints par une raison puissante. C’est que Syracuse et tant d’autres villes n’auraient pas eu assez de temps pour s’élever au point de richesse et de splendeur où elles parvinrent. Pendant ses guerres contre les Carthaginois, Syracuse n’avait rien à envier à la magnificence et à la politesse d’Athènes. Longtemps après, Crotone presque déserte fait pitié à Tite-Live, lorsqu’il songe au nombre prodigieux de ses anciens habitants.\par
(An du monde, 3223.) Le {\itshape temps certain}, l’{\itshape âge des hommes} commence à l’époque où les {\itshape jeux olympiques} fondés par Hercule, furent rétablis par Iphitus. Depuis le premier, on comptait les années par les récoltes ; depuis le second, on les compta par les révolutions du soleil.\par
La première {\itshape Olympiade} coïncide presque avec la {\itshape fondation de Rome} (776, 753 ans avant J.-C.) Mais Rome aura pendant longtemps bien peu d’importance. Toutes ces idées magnifiques que l’on s’est faites jusqu’ici sur les commencements de Rome et de toutes les autres capitales des peuples célèbres, disparaissent, comme le brouillard aux rayons du soleil, devant ce passage précieux de Varron rapporté par Saint-Augustin dans la Cité de Dieu : \emph{{\itshape pendant deux siècles et demi qu’elle obéit à ses rois, Rome soumit plus de vingt peuples, sans étendre son empire à plus de vingt milles}}.\par
 (An du monde, 3290 ; de Rome 37.) Nous plaçons {\itshape Homère} après la fondation de Rome. L’histoire grecque, dont il est le principal flambeau, nous a laissés dans l’incertitude sur son siècle et sur sa patrie. On verra au livre III pourquoi nous nous écartons de l’opinion reçue sur ces deux points, et sur le fait même de son existence. — Nous élèverons les mêmes doutes sur celle d’{\itshape Ésope} que nous considérons non comme un individu, mais comme un type idéal, et dont nous plaçons l’époque entre celle d’Homère et celle des sept sages de la Grèce.\par
(3468 ; 225.) {\itshape Pythagore} qui vient ensuite, est, selon Tite-Live, contemporain de Servius Tullius ; on voit s’il a pu enseigner la science des choses divines à Numa qui vivait près de deux siècles auparavant. Tite-Live dit aussi que pendant ce règne de Servius Tullius, où l’intérieur de l’Italie était encore barbare, il eût été impossible que le nom même de Pythagore pénétrât de Crotone à Rome à travers tant de peuples différents de langues et de mœurs. Ce dernier passage doit nous faire entendre combien devaient être faciles ces longs voyages dans lesquels Pythagore alla, dit-on, consulter en Thrace les disciples d’Orphée, en Perse les mages, les Chaldéens à Babylone, les Gymnosophistes dans l’Inde, puis en revenant, les prêtres de l’Égypte, les disciples d’Atlas dans la Mauritanie, et les Druides dans la Gaule, pour rentrer enfin dans sa patrie, riche de toute la {\itshape sagesse barbare}\footnote{Si nous en croyons ceux qui, aux applaudissements des savants, ont entrepris de nous faire connaître la succession des écoles de la {\itshape philosophie barbare}, Zoroastre fut le maître de Bérose et des Chaldéens, Bérose celui d’Hermès et des Égyptiens, Hermès celui d’Atlas et des Éthiopiens, Atlas celui d’Orphée, qui, de la Thrace, vint établir son école en Grèce. On sent ce qu’ont de sérieux ces communications entre les premiers peuples, qui, à peine sortis de l’état sauvage, vivaient ignorés même de leurs voisins, et n’avaient connaissance les uns des autres qu’autant que la guerre ou le commerce leur en donnait l’occasion.Ce que nous disons de l’isolement des premiers peuples s’applique particulièrement aux Hébreux. — Lactance assure que Pythagore n’a pu être disciple d’Isaïe. — Un passage de Josèphe prouve que les Hébreux, au temps d’Homère et de Pythagore, vivaient inconnus à leurs voisins de l’intérieur des terres, et à plus forte raison aux nations éloignées dont la mer les séparait. — Ptolémée Philadelphe s’étonnant qu’aucun poète, aucun historien n’eût fait mention des lois de Moïse, le juif Démétrius lui répondit que ceux qui avaient tenté de les faire connaître aux Gentils, avaient été punis miraculeusement, tels que Théopompe qui en perdit le sens, et Théodecte qui fut privé de la vue. — Aussi Josèphe ne craint point d’avouer cette longue obscurité des Juifs, et il l’explique de la manière suivante : \emph{{\itshape Nous n’habitons point les rivages ; nous n’aimons point à faire le négoce et à commercer avec les étrangers.}} Sans doute la Providence voulait, comme l’observe Lactance, empêcher que la religion du vrai Dieu ne fût profanée par les communications de son peuple avec les Gentils. — Tout ce qui précède est confirmé par le témoignage du peuple Hébreux lui-même, qui prétendait qu’à l’époque où parut la version des Septante, les ténèbres couvrirent le monde pendant trois jours, et qui, en expiation, observait un jeûne solennel, le 8 de tébet ou décembre. Ceux de Jérusalem détestaient les juifs hellénistes qui attribuaient une autorité divine à cette version.}.\par
 (An du monde, 3468 ; de Rome 225.) {\itshape Servius Tullius}, institue le cens, dans lequel on a vu jusqu’ici le fondement de la {\itshape liberté démocratique}, et qui ne fut dans le principe que celui de la {\itshape liberté aristocratique}.\par
(3500.) C’est l’époque où les Grecs trouvèrent leur écriture vulgaire ({\itshape Voyez} plus bas.) Nous y plaçons {\itshape Hésiode, Hérodote} et {\itshape Hippocrate}. — Les chronologistes déclarent sans hésiter qu’Hésiode vivait trente  ans avant Homère, quoiqu’ils diffèrent de quatre siècles et demi sur le temps où il faut placer l’auteur de l’Iliade. Mais Velleius Paterculus et Porphyre (dans Suidas), sont d’avis qu’Homère précéda de beaucoup Hésiode. Quant aux trépieds consacrés par ce dernier en mémoire de sa victoire sur Homère, ce sont des monuments tels qu’en fabriquent de nos jours les faiseurs de médailles, qui vivent de la simplicité des curieux. — Si nous considérons, d’un côté, que la vie d’Hippocrate est toute fabuleuse, et que, de l’autre, il est l’auteur incontestable d’ouvrages écrits en prose et en caractères vulgaires, nous rapporterons son existence au temps d’Hérodote qui écrivit de même en prose et dont l’histoire est pleine de fables.\par
\par
(An du monde, 3530.) {\itshape Thucydide} vécut à l’époque la mieux connue de l’histoire grecque, celle de la guerre du Péloponnèse ; et c’est afin de n’écrire que des choses certaines qu’il a choisi cette guerre pour sujet. Il était fort jeune, pendant la vieillesse d’Hérodote qui eût pu être son père ; or, il dit que, \emph{{\itshape jusqu’au temps de son père, les Grecs ne surent rien de leurs propres antiquités}}. Que devaient-ils donc savoir de celles des barbares qu’ils nous ont seuls fait connaître ?… et que penserons-nous de celles des Romains, peuple tout occupé de l’agriculture et de la guerre, lorsque Thucydide fait un tel aveu au nom de ses Grecs, qui devinrent sitôt philosophes ? Dira-t-on  que les Romains ont reçu de Dieu un privilège particulier ?\par
(An du monde, 3553 ; de Rome 303.) L’époque de Thucydide est celle où Socrate fondait la morale, où Platon cultivait avec tant de gloire la métaphysique ; c’est pour Athènes l’âge de la civilisation la plus raffinée. Et c’est alors que les historiens nous font venir d’Athènes à Rome ces lois des {\itshape douze tables} si grossières et si barbares. {\itshape Voy.} plus loin la réfutation de ce préjugé.\par
Les Grecs avaient commencé sous le règne de Psammétique à mieux connaître l’Égypte ; à partir de cette époque, les récits d’Hérodote sur cette contrée prennent un caractère de certitude. (3553.) Ce fut de {\itshape Xénophon} qu’ils reçurent les premières connaissances exactes qu’ils aient eues de la Perse ; la {\itshape nécessité} de la guerre fit pour la Perse ce qu’avait fait pour l’Égypte l’{\itshape utilité} du commerce. Encore Aristote nous assure-t-il qu’avant la {\itshape conquête d’Alexandre}, l’on avait débité bien des fables sur les mœurs et l’histoire des Perses. — (3660.) C’est ainsi que la Grèce commença à avoir quelques notions certaines sur les peuples étrangers.\par
Deux lois changent à cette époque la constitution de Rome.\par
(3658 ; 416.) La loi {\itshape Publilia} est le passage visible de l’aristocratie à la démocratie. On n’a point assez remarqué cette loi, faute d’en savoir comprendre le langage.\par
(3661 ; 419.) La loi {\itshape Petilia, de nexu}, n’est pas moins digne d’attention. Par cette loi, les nobles perdirent leurs droits sur la personne des Plébéiens dont ils étaient  créanciers. Mais le sénat conserva son empire souverain sur toutes les terres de la république, et le maintint jusqu’à la fin par la force des armes.\par
(An du monde 3708 ; 489.) {\itshape Guerre de Tarente}, où les Latins et les Grecs commencent à prendre connaissance les uns des autres. Lorsque les Tarentins maltraitèrent les vaisseaux des Romains, et ensuite leurs ambassadeurs, ils alléguèrent pour excuse, selon Florus, qu’\emph{{\itshape ils ne savaient qui étaient les Romains, ni d’où ils venaient}}. Tant les premiers peuples se connaissaient peu, à une distance si rapprochée, et lors même qu’aucune mer ne les séparait !\par
(3849 ; 552.) {\itshape Seconde guerre punique.} C’est en commençant le récit de cette guerre que Tite-Live déclare qu’\emph{{\itshape il va écrire désormais l’histoire romaine avec plus de certitude, parce que cette guerre est la plus mémorable de toutes celles que firent les Romains}}. Néanmoins il avoue son ignorance sur trois circonstances essentielles : d’abord il ne sait sous quels consuls, Annibal, vainqueur de Sagonte, quitta l’Espagne pour aller en Italie, ni par quelle partie des Alpes il exécuta son passage, ni quelles étaient alors ses forces ; il trouve sur ce dernier article la plus grande diversité d’opinions dans les anciennes annales.\par
\par
D’après toutes les observations que nous avons faites sur cette table, on voit que tout ce qui nous est parvenu de l’antiquité païenne jusqu’au temps où nous nous arrêtons, n’est qu’incertitude et obscurité.  Aussi nous ne craignons pas d’y pénétrer comme dans un champ sans maître, qui appartient au premier occupant ({\itshape res nullius, quæ occupanti conceduntur}). Nous ne craindrons point d’aller contre les droits de personne, lorsqu’en traitant ces matières nous ne nous conformerons pas, ou que même nous serons contraires, aux opinions que l’on s’est faites jusqu’ici sur les {\itshape origines de la civilisation}, et que par là nous les ramènerons à des {\itshape principes scientifiques}. Grâce à ces principes, {\itshape les faits de l’histoire certaine} retrouveront leurs {\itshape origines primitives}, faute desquelles ils semblent jusqu’ici n’avoir eu ni {\itshape fondement} commun, ni {\itshape continuité}, ni {\itshape cohérence}.
\chapterclose


\chapteropen
\chapter[{Chapitre II. Axiomes}]{Chapitre II. \\
Axiomes}

\chaptercont
\noindent  Maintenant pour donner une forme aux {\itshape matériaux} que nous venons de préparer dans la table chronologique, nous proposons les {\itshape axiomes} philosophiques et philologiques que l’on va lire, avec un petit nombre de {\itshape postulats} raisonnables, et de {\itshape définitions} où nous avons cherché la clarté. Ainsi que le sang parcourt le corps qu’il anime, de même ces idées générales, répandues dans la {\itshape science nouvelle}, l’animeront de leur esprit dans toutes ses déductions sur la {\itshape nature commune des nations}.\par
\section[{1-22. Axiomes généraux}]{1-22. Axiomes généraux}
\subsection[{1-4. Réfutation des opinions que l’on s’est formées jusqu’ici des commencements de la civilisation}]{ \textsc{1-4. } {\itshape Réfutation des opinions que l’on s’est formées jusqu’ici des commencements de la civilisation} }
\noindent 1. Par un effet de la nature infime de l’intelligence de l’homme, lorsqu’il se trouve arrêté par l’ignorance, il se prend lui-même pour règle de tout.\par
 De là deux choses ordinaires : {\itshape La renommée croit dans sa marche ; elle perd sa force pour ce qu’on voit de près} ({\itshape fama crescit eundo ; minuit præsentia famam}). La marche a été longue depuis le commencement du monde, et la renommée n’a cessé de produire les opinions magnifiques que l’on a conçues jusqu’à nous de ces antiquités que leur extrême éloigneraient dérobe à notre connaissance. Ce caractère de l’esprit humain a été observé par Tacite (Agricola) : \emph{{\itshape omne ignotum pro magnifico est}} ; l’inconnu ne manque pas d’être admirable.\par
\par
2. Autre caractère de l’esprit humain : s’il ne peut se faire aucune idée des choses lointaines et inconnues, il les juge sur les choses connues et présentes.\par
C’est là la source inépuisable des erreurs où sont tombés toutes les nations, tous les savants, au sujet des commencements de l’{\itshape humanité} ; les premières s’étant mises à observer, les seconds à raisonner sur ce sujet dans des siècles d’une brillante civilisation, ils n’ont pas manqué de juger d’après leur temps, des premiers âges de l’humanité, qui naturellement ne devaient être que grossièreté, faiblesse, obscurité.\par
3. \emph{{\itshape Chaque nation grecque ou barbare, a follement prétendu avoir trouvé la première, les commodités de la vie humaine, et conservé les traditions de son histoire depuis l’origine du monde.}} Ce mot précieux est de Diodore de Sicile.\par
 Par là sont écartées à la fois les vaines prétentions des Chaldéens, des Scythes, des Égyptiens et des Chinois, qui se vantent tous d’avoir fondé la civilisation antique. Au contraire, Josèphe met les Hébreux à l’abri de ce reproche en faisant l’aveu magnanime qu’\emph{{\itshape ils sont restés cachés à tous les peuples païens}}. Et en même temps l’histoire sainte nous représente le monde comme jeune, eu égard à la vieillesse que lui supposaient les Chaldéens, les Scythes, les Égyptiens, et que lui supposent encore aujourd’hui les Chinois. Preuve bien forte en faveur de la vérité de l’histoire sainte.\par
À la vanité des nations, joignez celle des savants ; ils veulent que ce qu’ils savent soit aussi ancien que le monde. Le mot de Diodore détruit tout ce qu’ils ont pensé de cette sagesse antique qu’il faudrait désespérer d’égaler ; prouve l’imposture des oracles de Zoroastre le Chaldéen, et d’Anacharsis le Scythe, qui ne nous sont pas parvenus, du Pimandre de Mercure trismégiste, des vers d’Orphée, des {\itshape vers dorés} de Pythagore (déjà condamnés par les plus habiles critiques) ; enfin découvre à la fois l’absurdité de tous les sens mystiques donnés par l’érudition aux hiéroglyphes égyptiens, et celle des allégories philosophiques par lesquelles on a cru expliquer les fables grecques.
\subsection[{5-15. Fondements du vrai. (Méditer le monde social dans son idéal éternel)}]{5-15. {\itshape Fondements du vrai. }\\
(Méditer le monde social dans son idéal éternel)}
\noindent  5. Pour être utile au genre humain, la philosophie doit relever et diriger l’homme déchu et toujours débile ; elle ne doit ni l’arracher à sa propre nature, ni l’abandonner à sa corruption.\par
Ainsi sont exclus de l’école de la nouvelle science les Stoïciens qui veulent la mort des sens, et les Épicuriens qui font des sens la règle de l’homme ; ceux-là s’enchaînant au destin, ceux-ci s’abandonnant au hasard et faisant mourir l’âme avec le corps ; les uns et les autres niant la Providence. Ces deux sectes isolent l’homme et devraient s’appeler philosophies {\itshape solitaires}. Au contraire nous admettons dans notre école les philosophes politiques, et surtout les Platoniciens, parce qu’ils sont d’accord avec tous les législateurs sur trois points capitaux : existence d’une Providence divine, nécessité de modérer les passions humaines et d’en faire des vertus {\itshape humaines}, immortalité de l’âme. Cet axiome nous donnera les trois principes de la nouvelle science\footnote{Le principe du droit naturel est {\itshape le juste dans son unité}, autrement dit, l’unité des idées du genre humain concernant les choses dont l’utilité ou la nécessité est commune à toute la nature humaine. Le pyrrhonisme détruit l’{\itshape humanité}, parce qu’il ne donne point l’unité. L’épicuréisme la dissipe, en quelque sorte, parce qu’il abandonne au sentiment individuel le jugement de l’utilité. Le stoïcisme l’anéantit, parce qu’il ne reconnaît d’utilité ou de nécessité que celles de l’âme, et qu’il méconnaît celles du corps ; encore le {\itshape Sage} seul peut-il juger de celles de l’âme. La seule doctrine de Platon nous présente le juste dans son unité ; ce philosophe pense qu’on doit suivre comme la règle du vrai ce qui semble un, ou le même à tous les hommes. Édition de 1725, réimprimée en 1817, page 74.}.\par
\par
 6. La philosophie considère l’homme tel qu’il doit être ; ainsi elle ne peut être utile qu’à un bien petit nombre d’hommes qui veulent vivre dans la république de Platon, et non ramper dans \emph{{\itshape la fange du peuple de Romulus}}\footnote{\emph{{\itshape Dicit enim} (Cato){\itshape  tanquam in Platonis} πολιτεία{\itshape , non tanquam in Romuli fæce sententiam.}} Cic. {\itshape ad Atticum}, lib. II ({\itshape Note du Traducteur}).}.\par
7. La législation considère l’homme tel qu’il est, et veut en tirer parti pour le bien de la société humaine. Ainsi de trois vices, l’orgueil féroce, l’avarice, l’ambition, qui égarent tout le genre humain, elle tire le métier de la guerre, le commerce, la politique ({\itshape la corte}), dans lesquels se forment le courage, l’opulence, la sagesse de l’homme d’état. Trois vices capables de détruire la race humaine produisent la félicité publique.\par
Convenons qu’il doit y avoir une Providence divine, une intelligence législatrice du monde : grâce à elle, les passions des hommes livrés tout entiers à l’intérêt privé, qui les ferait vivre en bêtes féroces dans les solitudes, ces passions mêmes ont formé la hiérarchie civile, qui maintient la société humaine.\par
8. Les choses, hors de leur état naturel, ne peuvent y rester, ni s’y maintenir.\par
 Si, depuis les temps les plus reculés dont nous parle l’histoire du monde, le genre humain a vécu, et vit tolérablement en société, cet axiome termine la grande dispute élevée sur la question de savoir {\itshape si la nature humaine est sociable}, en d’autres termes {\itshape s’il y a un droit naturel} ; dispute que soutiennent encore les meilleurs philosophes et les théologiens contre Épicure et Carnéade, et qui n’a point été fermée par Grotius lui-même.\par
Cet axiome, rapproché du septième et de son corollaire, prouve que l’homme a le libre arbitre, quoique incapable de changer ses passions en vertus, mais qu’il est aidé naturellement par la Providence de Dieu, et d’une manière surnaturelle par la Grâce.\par
9. Faute de savoir le {\itshape vrai}, les hommes tâchent d’arriver au {\itshape certain}, afin que si l’{\itshape intelligence} ne peut être satisfaite par la {\itshape science}, la {\itshape volonté} du moins se repose sur la {\itshape conscience}.\par
10. La {\itshape philosophie} contemple la {\itshape raison}, d’où vient la {\itshape science du vrai} ; la {\itshape philologie} étudie les actes de la liberté humaine, elle en suit l’{\itshape autorité} ; et c’est de là que vient la conscience du {\itshape certain}. — Ainsi nous comprenons sous le nom de {\itshape philologues} tous les grammairiens, historiens, critiques, lesquels s’occupent de la connaissance des {\itshape langues} et des {\itshape faits} (tant des faits {\itshape intérieurs} de l’histoire des peuples, comme lois et usages, que des faits {\itshape extérieurs},  comme guerres, traités de paix et d’alliance, commerce, voyages.)\par
Le même axiome nous montre que les {\itshape philosophes} sont restés à moitié chemin en négligeant de donner à leurs {\itshape raisonnements} une {\itshape certitude} tirée de l’{\itshape autorité} des {\itshape philologues} ; que les {\itshape philologues} sont tombés dans la même faute, puisqu’ils ont négligé de donner aux faits le caractère de {\itshape vérité} qu’ils auraient tiré des {\itshape raisonnements philosophiques}. Si les philosophes et les philologues eussent évité ce double écueil, ils eussent été plus utiles à la société, et ils nous auraient prévenus dans la recherche de cette nouvelle science.\par
11. L’étude des actes de la {\itshape liberté humaine}, si incertaine de sa nature, tire sa certitude et sa détermination du {\itshape sens commun} appliqué par les hommes aux {\itshape nécessités} ou {\itshape utilités} humaines, {\itshape double source du droit naturel des gens}\footnote{Le {\itshape droit naturel des gens} a, dans Vico, une signification très entendue. Il comprend non-seulement les rapports des sociétés entre elles, mais même tous les rapports des individus entre eux ({\itshape Note du Traducteur}).}.\par
12. Le {\itshape sens commun} est un {\itshape jugement} sans {\itshape réflexion}, partagé par tout un ordre, par tout un peuple, par toute une nation, ou par tout le genre humain.\par
Cet axiome (avec la définition suivante) nous ouvrira  une critique nouvelle relative aux {\itshape auteurs des peuples}, qui ont dû précéder de plus de mille ans les {\itshape auteurs de livres}, dont la critique s’est occupée jusqu’ici exclusivement.\par
13. Des idées uniformes nées chez des peuples inconnus les uns aux autres, doivent avoir un motif commun de vérité.\par
Grand principe, d’après lequel le sens commun du genre humain est le {\itshape criterium} indiqué par la Providence aux nations pour déterminer la certitude dans le droit naturel des gens. On arrive à cette certitude en connaissant l’unité, l’essence de ce droit auquel toutes les nations se conforment avec diverses modifications ({\itshape Voy.} le vingt-deuxième axiome.)\par
Le même axiome renferme toutes les idées qu’on s’est formées jusqu’ici du droit naturel des gens ; droit qui, selon l’opinion commune, serait sorti d’une nation pour être transmis aux autres. Cette erreur est devenue scandaleuse par la vanité des Égyptiens et des Grecs, qui, à les en croire, ont répandu la civilisation dans le monde.\par
C’était une conséquence naturelle qu’on fît venir de Grèce à Rome la loi des douze tables. Ainsi le droit civil aurait été communiqué aux autres peuples par une prévoyance humaine ; ce ne serait pas un droit mis par la divine Providence dans la nature, dans les mœurs de l’humanité, et ordonné par elle chez toutes les nations !\par
Nous ne cesserons dans cet ouvrage de tâcher de  démontrer que le droit naturel des gens naquit chez chaque peuple en particulier, sans qu’aucun d’eux sût rien des autres ; et qu’ensuite à l’occasion des guerres, ambassades, alliances, relations de commerce, ce droit fut reconnu commun à tout le genre humain.\par
14. La {\itshape nature} des choses consiste en ce qu’elles naissent en certaines circonstances, et de certaines manières. Que les circonstances se représentent les mêmes, les choses naissent les mêmes et non différentes.\par
15. Les {\itshape propriétés inséparables} du sujet doivent résulter de la modification avec laquelle, de la manière dont la chose est née ; ces propriétés {\itshape vérifient} à nos yeux que la nature de la chose même (c’est-à-dire la manière dont elle est née) est telle, et non pas autre.
\subsection[{16-22. Fondements du certain. (Apercevoir le monde social dans sa réalité)}]{16-22. {\itshape Fondements du certain. }\\
(Apercevoir le monde social dans sa réalité)}
\noindent 16. Les traditions vulgaires doivent avoir quelques {\itshape motifs publics de vérité}, qui expliquent comment elles sont nées, et comment elles se sont conservées longtemps chez des peuples entiers.\par
Assigner à ces traditions leurs véritables causes qui, à travers les siècles, à travers les changements de langues et d’usages, nous sont arrivées déguisées  par l’erreur, ce sera un des grands travaux de la nouvelle science.\par
17. Les façons de parler vulgaires sont les témoignages les plus graves sur les usages nationaux des temps où se formèrent les langues.\par
18. Une langue ancienne qui est restée en usage, doit, considérée avant sa maturité, être un grand monument des usages des premiers temps du monde.\par
Ainsi c’est du latin qu’on tirera les preuves philologiques les plus concluantes en matière de droit des gens ; les Romains ont surpassé sans contredit tous les autres peuples dans la connaissance de ce droit. Ces preuves pourront aussi être recherchées dans la langue allemande qui partage cette propriété avec l’ancienne langue romaine.\par
19. Si les lois des douze tables furent les coutumes en vigueur chez les peuples du Latium depuis l’âge de Saturne, coutumes qui, toujours mobiles chez les autres tribus, furent fixées par les Romains sur le bronze, et gardées religieusement par leur jurisprudence, ces lois sont un grand monument de l’ancien droit naturel des peuples du Latium.\par
20. Si les poèmes d’Homère peuvent être considérés comme l’histoire civile des anciennes coutumes grecques, ils sont pour nous deux grands trésors  du droit naturel des gens considéré chez les Grecs.\par
Cette vérité et la précédente ne sont encore que des {\itshape postulats}, dont la démonstration se trouvera dans l’ouvrage.\par
21. Les philosophes grecs précipitèrent la marche naturelle que devait suivre leur nation ; ils parurent dans la Grèce lorsqu’elle était encore toute barbare, et la firent passer immédiatement à la civilisation la plus raffinée ; en même temps les Grecs conservèrent entières leurs histoires fabuleuses, tant divines qu’héroïques. La civilisation marcha d’un pas plus réglé chez les Romains ; ils perdirent entièrement de vue leur histoire {\itshape divine} ; aussi l’{\itshape âge des dieux}, pour parler comme les Égyptiens ({\itshape Voy.} l’axiome 28), est appelé par Varron le {\itshape temps obscur} des Romains ; les Romains conservèrent dans la langue vulgaire leur histoire héroïque, qui s’étend depuis Romulus jusqu’aux lois Publilia et Petilia, et nous trouverons réfléchie dans cette histoire toute la suite de celle des héros grecs\footnote{La vérité de ces observations nous est confirmée par l’exemple de la nation française. Elle vit s’ouvrir au milieu de la barbarie du onzième siècle, cette fameuse école de Paris, où Pierre Lombard, {\itshape le maître des sentences}, enseignait la scholastique la plus subtile ; et d’un autre côté elle a conservé une sorte de poème homérique dans l’histoire de l’archevêque Turpin, ce recueil universel des {\itshape Fables héroïques} qui ont ensuite embelli tant de poèmes et de romans. Ce passage prématuré de la barbarie aux sciences les plus subtiles, a donné à la langue française une délicatesse supérieure à celle de toutes les langues vivantes ; c’est elle qui reproduit le mieux l’atticisme des Grecs. Comme la langue grecque, elle est aussi éminemment propre à traiter les sujets scientifiques.}.\par
 Nous trouvons encore, dans nos principes, une autre cause de cette marche des Romains, et peut-être cette cause explique plus convenablement l’effet indiqué. Romulus fonda Rome au milieu d’autres cités latines plus anciennes ; il la fonda en ouvrant un asile, \emph{{\itshape moyen}, dit Tite-Live, {\itshape  employé jadis par la sagesse des fondateurs de villes}} ; l’âge de la violence durant encore, il dut fonder sa ville sur la même base qui avait été donnée aux premières cités du monde. La civilisation romaine partit de ce principe ; et comme les langues vulgaires du Latium avaient fait de grands progrès, il dut arriver que les Romains expliquèrent en langue vulgaire les affaires de la vie civile, tandis que les Grecs les avaient exprimées en langue héroïque. Voilà aussi pourquoi les Romains furent les {\itshape héros du monde}, et soumirent les autres cités du Latium, puis l’Italie, enfin l’univers. Chez eux l’héroïsme était jeune, lorsqu’il avait commencé à vieillir chez les autres peuples du Latium, dont la soumission devait préparer toute la grandeur de Rome.\par
22. Il existe nécessairement dans la nature une {\itshape langue intellectuelle commune à toutes les nations} ; toutes les choses qui occupent l’activité de l’homme en société y sont uniformément comprises, mais  exprimées avec autant de modifications qu’on peut considérer ces choses sous divers aspects. Nous le voyons dans les proverbes ; ces maximes de la {\itshape sagesse vulgaire}, sont entendues dans le même sens par toutes les nations anciennes et modernes, quoique dans l’expression elles aient suivi la diversité des manières de voir. — Cette langue appartient à la {\itshape science nouvelle} ; guidés par elle, les philologues pourront se faire {\itshape un vocabulaire intellectuel commun à toutes les langues mortes et vivantes}.
\section[{23-114. Axiomes particuliers}]{23-114. Axiomes particuliers}
\subsection[{23-28. Division des peuples anciens en Hébreux et Gentils. — Déluge universel. — Géants}]{ \textsc{23-28. } {\itshape Division des peuples anciens en Hébreux et Gentils. — Déluge universel. — Géants} }
\noindent 23. L’histoire sacrée est plus ancienne que toutes les histoires profanes qui nous sont parvenues, puisqu’elle nous fait connaître, avec tant de détails et dans une période de huit siècles, l’état de nature sous les patriarches ({\itshape état de famille}, dans le langage de la {\itshape science nouvelle}). Cet état dont, selon l’opinion unanime des politiques, sortirent les peuples et les cités, l’histoire profane n’en fait point mention, ou en dit à peine quelques mots confus.\par
24. Dieu défendit la divination aux Hébreux ; cette défense est la base de leur religion ; la divination au contraire est le principe de la société chez  toutes les nations païennes. Aussi tout le monde ancien fut-il divisé en Hébreux et Gentils.\par
25. Nous démontrerons le {\itshape déluge universel}, non plus par les preuves philologiques de Martin Scoock ; elles sont trop légères ; ni par les preuves astrologiques du cardinal d’Alliac, suivi par Pic de la Mirandole ; elles sont incertaines et même fausses ; mais par les faits d’une {\itshape histoire physique} dont nous trouverons les vestiges dans les fables.\par
26. Il a existé des {\itshape géants} dans l’antiquité, tels que les voyageurs disent en avoir trouvé de très grossiers et de très féroces à l’extrémité de l’Amérique dans le pays des Patagons. Abandonnant les vaines explications que nous ont données les philosophes de leur existence, nous l’expliquerons par des causes en partie physiques, en partie morales, que César et Tacite ont remarquées en parlant de la stature gigantesque des anciens Germains. Nous rapportons ces causes à l’{\itshape éducation} sauvage, et pour ainsi dire {\itshape bestiale}, des enfants.\par
27. L’histoire grecque, qui nous a conservé tout ce que nous avons des antiquités païennes, en exceptant celles de Rome, prend son commencement du {\itshape déluge, et de l’existence des géants}.\par
Cette tradition nous présente la {\itshape division originaire du genre humain} en deux espèces, celle des géants et celle des hommes d’une stature naturelle,  celle des Gentils et celle des Hébreux. Cette différence ne peut être venue que de l’éducation {\itshape bestiale} des uns, de l’éducation {\itshape humaine} des autres ; d’où l’on peut conclure que les Hébreux ont eu une autre origine que celle des Gentils.
\subsection[{28-40. Principes de la théologie pratique. — Origine de l’idolâtrie, de la divination, des sacrifices}]{ \textsc{28-40. } {\itshape Principes de la théologie pratique. — Origine de l’idolâtrie, de la divination, des sacrifices} }
\noindent 28. Il nous reste deux grands débris des antiquités égyptiennes ; 1º Les Égyptiens divisaient tout le temps antérieurement écoulé en trois âges, {\itshape âge des dieux, âge des héros, âge des hommes} ; 2º Pendant ces trois âges, trois langues correspondantes se parlèrent, langue hiéroglyphique ou {\itshape sacrée}, langue symbolique ou {\itshape héroïque}, langue {\itshape vulgaire} ou {\itshape épistolaire}, celle dans laquelle les hommes expriment par des signes convenus les besoins ordinaires de la vie.\par
29. Homère parle dans cinq passages de ses poèmes d’une langue plus ancienne que l’héroïque dont il se servait, et il l’appelle langue des dieux. ({\itshape Voy.} livre 2, chap. 6.)\par
30. Varron a pris la peine de recueillir trente mille noms de divinités reconnues par les Grecs. Ces noms se rapportaient à autant de besoins de la vie {\itshape naturelle, morale, économique}, ou {\itshape civile} des premiers temps. — Concluons des trois traditions qui  viennent d’être rapportées que, {\itshape partout la société a commencé par la religion}. C’est le premier des trois principes de la science nouvelle.\par
31. Lorsque les peuples sont {\itshape effarouchés} par la violence et par les armes, au point que les lois humaines n’auraient plus d’action, il n’existe qu’un moyen puissant pour les dompter, c’est la religion.\par
Ainsi dans l’{\itshape état sans lois} ({\itshape stato eslege}), la Providence réveilla dans l’âme des plus violents et des plus fiers une idée confuse de la divinité, afin qu’ils entrassent dans la vie sociale et qu’ils y fissent entrer les nations. Ignorants comme ils étaient, ils appliquèrent mal cette idée, mais l’effroi que leur inspirait la divinité telle qu’ils l’imaginèrent, commença à ramener l’ordre parmi eux.\par
Hobbes ne pouvait voir la société commencer ainsi parmi {\itshape les hommes violents et farouches} de son système, lui qui, pour en trouver l’origine, s’adresse au hasard d’Épicure. Il entreprit de remplir la grande lacune laissée par la philosophie grecque, qui n’avait point considéré {\itshape l’homme dans l’ensemble de la société du genre humain}. Effort magnanime auquel le succès n’a pas répondu\footnote{La fin de cet alinéa est rejetée dans une note du chapitre III. ({\itshape Note du Traducteur.})} !\par
32. Lorsque les hommes ignorent les causes naturelles  des phénomènes, et qu’ils ne peuvent les expliquer par des analogies, ils leur attribuent leur propre nature ; par exemple, le vulgaire dit que {\itshape l’aimant aime le fer}. ({\itshape Voy.} l’axiome 1\textsuperscript{er}.)\par
33. La physique des ignorants est une métaphysique vulgaire, dans laquelle ils rapportent les causes des phénomènes qu’ils ignorent à la volonté de Dieu, sans considérer les moyens qu’emploie cette volonté.\par
34. L’observation de Tacite est très juste : \emph{{\itshape mobiles ad superstitionem perculsæ semel mentes}}. Dès que les hommes ont laissé surprendre leur âme par une superstition pleine de terreurs, ils y rapportent tout ce qu’ils peuvent imaginer, voir, ou faire eux-mêmes.\par
35. L’admiration est fille de l’ignorance.\par
36. L’imagination est d’autant plus forte que le raisonnement est plus faible.\par
37. Le plus sublime effort de la poésie est d’animer, de passionner les choses insensibles. — Il est ordinaire aux enfants de prendre dans leurs jeux les choses inanimées, et de leur parler comme à des personnes vivantes. — Les hommes du monde enfant durent être naturellement des poètes sublimes.\par
38. Passage précieux de Lactance, sur l’origine  de l’idolâtrie : \emph{{\itshape Rudes initio domines Deos appellarunt, sive ob miraculum virtutis (hoc verò putabant rudes adhuc et simplices) ; sive, ut fieri solet, in admirationem præsentis potentiæ ; sive ob beneficia, quibus erant ad humanitatem compositi}} ; au commencement, les hommes encore simples et grossiers divinisèrent de bonne foi ce qui excitait leur admiration, tantôt la vertu, tantôt une puissance secourable (la chose est ordinaire), tantôt la bienfaisance de ceux qui les avaient civilisés.\par
39. Dès que notre intelligence est éveillée par l’admiration, quel que soit l’effet extraordinaire que nous observions, comète, parélie, ou toute autre chose, la curiosité, fille de l’ignorance et mère de la science, nous porte à demander : Que signifie ce phénomène ?\par
40. La superstition qui remplit de terreur l’âme des magiciennes, les rend en même temps cruelles et barbares ; au point que souvent pour célébrer leurs affreux mystères, elles égorgent sans pitié et déchirent en pièces l’être le plus innocent et le plus aimable, un enfant.\par
Voilà l’origine des sacrifices, dans lesquels la férocité des premiers hommes faisait couler le sang humain. Les Latins eurent leurs {\itshape victimes de Saturne} ({\itshape Saturni hostiæ}) ; les Phéniciens faisaient passer à travers les flammes les enfants consacrés à Moloch ; et les douze tables conservent quelques traces de  semblables consécrations. — Cette explication nous fera mieux entendre le vers fameux : \emph{{\itshape La crainte seule a fait les premiers dieux.}} Les fausses religions sont nées de la crédulité, et non de l’imposture. — Elle répond aussi à l’exclamation impie de Lucrèce au sujet du sacrifice d’Iphigénie (\emph{{\itshape tant la religion put enfanter de maux} !}). Ces religions cruelles étaient le premier degré par lequel la Providence amenait les hommes encore farouches, {\itshape les fils des Cyclopes et des Lestrigons}, à la civilisation des âges d’Aristide, de Socrate et de Scipion.
\subsection[{41-46. Principes de la Mythologie historique}]{ \textsc{41-46. } {\itshape Principes de la Mythologie historique} }
\noindent 41-42. Dans cette période qui suivit le déluge universel, les descendants impies des fils de Noé retournèrent à l’état sauvage, se dispersèrent comme des bêtes farouches dans la vaste forêt qui couvrait la terre, et par l’effet d’une éducation toute {\itshape bestiale}, redevinrent géants à l’époque où il tonna la première fois après le déluge. C’est alors que {\itshape Jupiter foudroie et terrasse les géants}. Chaque nation païenne eut son Jupiter. — Il fallut sans doute plus d’un siècle après le déluge pour que la terre moins humide pût exhaler des vapeurs capables de produire le tonnerre.\par
43. Toute nation païenne eut son Hercule, fils de Jupiter ; le docte Varron en a compté jusqu’à quarante. — Voilà l’origine de l’héroïsme chez les  premiers peuples, qui faisaient sortir leurs héros des dieux.\par
Cette tradition et la précédente qui nous montre d’abord tant de Jupiter, ensuite tant d’Hercule chez les nations païennes, nous indique que les premières sociétés ne purent se fonder sans religion, ni s’agrandir sans vertu. — En outre, si vous considérez l’isolement de ces peuples sauvages qui s’ignoraient les uns les autres, et si vous vous rappelez l’axiome : {\itshape Des idées uniformes nées chez des peuples inconnus entre eux, doivent avoir un motif commun de vérité}, vous trouverez un grand principe, c’est que les premières fables durent contenir des vérités relatives à l’état de la société, et par conséquent être l’histoire des premiers peuples.\par
44. Les premiers sages parmi les Grecs furent les {\itshape poètes théologiens}, lesquels sans aucun doute fleurirent avant les {\itshape poètes héroïques}, comme Jupiter fut père d’Hercule.\par
Des trois traditions précédentes, il résulte que les nations païennes avec leurs Jupiter et leurs Hercule, furent dans leurs commencements toutes poétiques, et que d’abord naquit chez elles la {\itshape poésie divine}, ensuite l’{\itshape héroïque}.\par
45. Les hommes sont naturellement portés à conserver dans quelque monument le souvenir des lois et institutions, sur lesquelles est fondée la société où ils vivent.\par
 46. Toutes les histoires des barbares commencent par des fables.
\subsection[{47-62. Poétique}]{ \textsc{47-62. } {\itshape Poétique} }
\subsubsection[{47-49. Principe des caractères poétiques}]{ \textsc{47-49. } {\itshape Principe des caractères poétiques} }
\noindent 47. L’esprit humain aime naturellement l’uniforme.\par
Cet axiome appliqué aux fables s’appuie sur une observation. Qu’un homme soit fameux en bien ou en mal, le vulgaire ne manque pas de le placer en telle ou telle circonstance, et d’inventer sur son compte des fables en harmonie avec son caractère ; {\itshape mensonges de fait}, sans doute, mais {\itshape vérités d’idées}, puisque le public n’imagine que ce qui est analogue à la réalité. Qu’on y réfléchisse, on trouvera que le {\itshape vrai poétique} est {\itshape vrai métaphysiquement}, et que le {\itshape vrai physique}, qui n’y serait pas conforme, devrait passer pour faux. Le véritable capitaine, par exemple, c’est le Godefroi du Tasse ; tous ceux qui ne se conforment pas en tout à ce modèle, ne méritent point le nom de capitaine. Considération importante dans la poétique.\par
48. Il est naturel aux enfants de transporter l’idée et le nom des premières personnes, des premières choses qu’ils ont vues, à toutes les personnes, à toutes les choses qui ont avec elles quelque ressemblance, quelque rapport.\par
 49. C’est un passage précieux que celui de Jamblique, {\itshape sur les mystères des Égyptiens} : les Égyptiens attribuaient à Hermès Trismégiste toutes les découvertes utiles ou nécessaires à la vie humaine.\par
Cet axiome et le précédent renverseront cette sublime théologie naturelle par laquelle ce grand philosophe interprète les mystères de l’Égypte.\par
Dans les axiomes 47, 48 et 49, nous trouvons le principe des caractères poétiques, lesquels constituent l’essence des fables. Le premier nous montre le penchant naturel du vulgaire à imaginer des fables et à les imaginer avec convenance. — Le second nous fait voir que les premiers hommes qui représentaient l’enfance de l’humanité, étant incapables d’abstraire et de généraliser, furent contraints de créer les caractères poétiques, pour y ramener, comme à autant de modèles, toutes les espèces particulières qui auraient avec eux quelque ressemblance. Cette ressemblance rendait infaillible la convenance des fables antiques. Ainsi les Égyptiens rapportaient au type du {\itshape sage dans les choses de la vie sociale} toutes les découvertes utiles ou nécessaires à la vie, et comme ils ne pouvaient atteindre cette abstraction, encore moins celle de {\itshape sagesse sociale}, ils personnifiaient le genre tout entier sous le nom d’Hermès Trismégiste. Qui peut soutenir encore qu’au temps où les Égyptiens enrichissaient le monde de leurs découvertes, ils étaient déjà philosophes, déjà capables de généraliser ?
\subsubsection[{50-62. Fable, convenance, pensée, expression, etc.}]{ \textsc{50-62. } {\itshape Fable, convenance, pensée, expression, etc.} }
\noindent  50. Dans l’enfance, la mémoire est très forte ; aussi l’imagination est vive à l’excès ; car l’imagination n’est autre chose que la mémoire avec extension, ou composition. — Voilà pourquoi nous trouvons un caractère si frappant de vérité dans les images poétiques, que dut former le monde enfant.\par
51. En tout les hommes suppléent à la nature par une étude opiniâtre de l’art ; en poésie seulement, toutes les ressources de l’art ne feront rien pour celui que la nature n’a point favorisé. — Si la poésie fonda la civilisation païenne qui devait produire tous les arts, il faut bien que la nature ait fait les premiers poètes.\par
52. Les enfants ont à un très haut degré la faculté d’imiter ; tout ce qu’ils peuvent déjà connaître, ils s’amusent à l’imiter. — Aux temps du monde enfant, il n’y eut que des peuples poètes ; la poésie n’est qu’imitation.\par
C’est ce qui peut faire comprendre, pourquoi tous les arts de nécessité, d’utilité, de commodité, et même la plupart des arts d’agrément, furent trouvés dans les siècles poétiques, avant qu’il se formât des philosophes : les arts ne sont qu’autant d’imitations de la nature, une {\itshape poésie réelle}, si je l’ose dire.\par
\par
 53. Les hommes sentent d’abord, sans remarquer les choses senties ; ils les remarquent ensuite, mais avec la confusion d’une âme agitée et passionnée ; enfin, éclairés par une pure intelligence, ils commencent à réfléchir.\par
Cet axiome nous explique la formation des pensées poétiques. Elles sont l’expression des passions et des sentiments, à la différence des pensées philosophiques qui sont le produit de la réflexion et du raisonnement. Plus les secondes s’élèvent aux généralités, plus elles approchent du {\itshape vrai} ; les premières au contraire deviennent {\itshape plus certaines} (c’est-à-dire qu’elles peignent plus fidèlement), à proportion qu’elles descendent dans les particularités.\par
54. Les hommes interprètent les choses douteuses ou obscures qui les touchent, conformément à leur propre nature, et aux passions et usages qui en dérivent.\par
Cet axiome est une règle importante de notre mythologie. Les fables imaginées par les premiers hommes furent sévères comme leurs farouches inventeurs, qui étaient à peine sortis de l’indépendance bestiale pour commencer la société. Les siècles s’écoulèrent, les usages changèrent, et les fables furent altérées, détournées de leur premier sens, obscurcies dans les temps de corruption et de dissolution qui précédèrent même l’existence d’Homère. Les Grecs, craignant de trouver les dieux aussi contraires à leurs vœux, qu’ils devaient l’être à leurs  mœurs, attribuèrent ces mœurs aux dieux eux-mêmes, et donnèrent souvent aux fables un sens honteux et obscène.\par
55. Étendez à tous les Gentils, le passage suivant où Eusèbe parle des seuls Égyptiens, il devient précieux : Originairement la théologie des Égyptiens ne fut autre chose qu’une histoire mêlée de fables ; les âges suivants qui rougissaient de ces fables, leur supposèrent peu à peu une signification mystique. C’est ce que fit Manéthon, grand-prêtre de l’Égypte, qui prêta à l’histoire de son pays le sens d’une sublime théologie naturelle.\par
Les deux axiomes précédents sont deux fortes preuves en faveur de notre mythologie historique et en même temps deux coups mortels pertes au préjugé qui attribue aux anciens une sagesse impossible à égaler ({\itshape innarrivabile}). Ils renferment en même temps deux puissants arguments en faveur de la vérité du christianisme, qui dans l’histoire sainte ne présente aucun récit dont il ait à rougir.\par
56. Les premiers auteurs parmi les Orientaux, les Égyptiens, les Grecs et les Latins, les premiers écrivains qui firent usage des nouvelles langues de l’Europe, lorsque la barbarie antique reparut au moyen âge, se trouvent avoir été des poètes.\par
57. Les muets s’expliquent par des gestes, ou par d’autres signes matériels, qui ont des rapports  naturels avec les idées qu’ils veulent faire entendre.\par
C’est le principe des langues hiéroglyphiques, en usage chez toutes les nations dans leur première barbarie. C’est celui du {\itshape langage naturel qui s’est parlé jadis dans le monde}, si l’on s’en rapporte à la conjecture de Platon ({\itshape Cratyle}), suivi par Jamblique, par les Stoïciens et par Origène ({\itshape contre Celse}). Mais comme ils avaient seulement deviné la vérité, ils trouvèrent des adversaires dans Aristote (Περί ερμηνείας), et dans Galien ({\itshape De decretis Hippocratis et Platonis}) ; Publius Nigidius parle de cette dispute dans Aulu-Gelle. À ce {\itshape langage naturel} dut succéder le {\itshape langage poétique}, composé d’images, de similitudes et de comparaisons, enfin de traits qui peignaient les propriétés naturelles des êtres.\par
58. Les muets émettent des sons confus avec une espèce de chant. Les bègues ne peuvent délier leur langue qu’en chantant.\par
59. Les grandes passions se soulagent par le chant, comme on l’observe dans l’excès de la douleur ou de la joie.\par
D’après ces deux axiomes, si les premiers hommes du monde païen retombèrent dans un état de brutalité où ils devinrent {\itshape muets} comme les bêtes, on doit croire que les plus violentes passions purent seules les arracher à ce silence, et qu’{\itshape ils formèrent leurs premières langues en chantant}.\par
\par
 60. Les langues durent commencer par des {\itshape monosyllabes}. Maintenant encore au milieu de tant de facilités pour apprendre le langage articulé, les enfants, dont les organes sont si flexibles, commencent toujours ainsi.\par
61. Le vers {\itshape héroïque} est le plus ancien de tous. Le vers spondaïque est le plus lent, et la suite prouvera que le vers héroïque fut originairement spondaïque.\par
62. Le vers {\itshape iambique} est celui qui se rapproche le plus de la prose, et l’iambe est un mètre rapide, comme le dit Horace.\par
Ces deux axiomes peuvent nous faire conjecturer que le développement des idées et des langues fut correspondant. Les sept axiomes précédents doivent nous convaincre que chez toutes les nations l’on parla d’abord en vers, puis en prose.
\subsection[{63-65. Principes étymologiques}]{ \textsc{63-65. } {\itshape Principes étymologiques} }
\noindent 63. {\itshape L’âme est portée} naturellement {\itshape à se voir au-dehors et dans la matière} ; ce n’est qu’avec beaucoup de peine, et par la réflexion, qu’elle en vient à se comprendre elle-même. — Principe universel d’étymologie ; nous voyons en effet dans toutes les langues les choses de l’âme et de l’intelligence exprimées par des métaphores qui sont tirées des corps et de leurs propriétés.\par
\par
 64. L’{\itshape ordre des idées} doit suivre l’{\itshape ordre des choses}.\par
65. Tel est l’ordre que suivent les choses humaines : d’abord les {\itshape forêts}, puis les {\itshape cabanes}, puis, les {\itshape villages}, ensuite les {\itshape cités}, ou réunions de citoyens, enfin les {\itshape académies}, ou réunions de savants. — Autre grand principe étymologique, d’après lequel l’histoire des langues indigènes doit suivre cette série de changements que subissent les choses. Ainsi dans la langue latine, nous pouvons observer que tous les mots ont des {\itshape origines sauvages et agrestes} : par exemple, {\itshape lex} ({\itshape legere}, cueillir) dut signifier d’abord {\itshape récolte de glands}, d’où l’arbre qui produit les glands fut appelé {\itshape illex, ilex} ; de même que {\itshape aquilex} est incontestablement {\itshape celui qui recueille les eaux}. Ensuite {\itshape lex} désigna la récolte des {\itshape légumes} ({\itshape legumina}) qui en dérivent leur nom. Plus tard, lorsqu’on n’avait pas de lettres pour écrire les lois, {\itshape lex} désigna nécessairement la réunion des citoyens, ou l’assemblée publique. La présence du peuple constituait {\itshape la loi} qui rendait les testaments authentiques, {\itshape calatis comitiis}. Enfin l’action de recueillir les lettres, et d’en faire comme un faisceau pour former chaque parole, fut appelée legere, lire.
\subsection[{66-96. Principes de l’histoire idéale}]{ \textsc{66-96. } {\itshape Principes de l’histoire idéale} }
\noindent 66. Les hommes sentent d’abord le {\itshape nécessaire}, puis font attention à l’{\itshape utile}, puis cherchent la {\itshape commodité} ; plus tard aiment le {\itshape plaisir}, s’abandonnent  au {\itshape luxe}, et en viennent enfin à {\itshape tourmenter leurs richesses}\footnote{{\itshape Divitias suas trahunt, vexant.} Salluste. ({\itshape N. du T.})}.\par
\par
67. Le caractère des peuples est d’abord cruel, ensuite {\itshape sévère}, puis {\itshape doux} et bienveillant, puis {\itshape ami de la recherche}, enfin {\itshape dissolu}.\par
68. Dans l’histoire du genre humain, nous voyons s’élever d’abord des caractères {\itshape grossiers et barbares}, comme le Polyphème d’Homère ; puis il en vient d’{\itshape orgueilleux et de magnanimes}, tels qu’Achille ; ensuite de {\itshape justes et de vaillants}, des Aristides, des Scipions ; plus tard nous apparaissent avec de nobles images de {\itshape vertus}, et en même temps {\itshape avec de grands vices}, ceux qui au jugement du vulgaire obtiennent la véritable gloire, les Césars et les Alexandres ; plus tard des caractères {\itshape sombres, d’une méchanceté réfléchie}, des Tibères ; enfin des {\itshape furieux} qui s’abandonnent en même temps à une {\itshape dissolution sans pudeur}, comme les Caligulas, les Nérons, les Domitiens.\par
La dureté des premiers fut nécessaire, afin que l’homme, obéissant à l’homme dans l’{\itshape état de famille}, fût préparé à obéir aux lois dans l’{\itshape état civil} qui devait suivre ; les seconds incapables de céder à leurs égaux, servirent à établir à la suite de l’état de famille les {\itshape républiques aristocratiques} ; les troisièmes à frayer le chemin à la {\itshape démocratie} ; les quatrièmes  à élever les {\itshape monarchies} ; les cinquièmes à les affermir ; les sixièmes à les renverser.\par
69. Les gouvernements doivent être conformes à la nature de ceux qui sont gouvernés. — D’où il résulte que l’école des princes, c’est la science des mœurs des peuples.\par
\subsubsection[{70-82. Commencements des sociétés}]{ \textsc{70-82. } {\itshape Commencements des sociétés} }
\noindent 70. Qu’on nous accorde la proposition suivante (la chose ne répugne point en elle-même, et plus tard elle se trouve vérifiée par les faits) : du {\itshape premier état sans loi et sans religion} sortirent d’abord un petit nombre d’hommes supérieurs par la force, lesquels fondèrent les {\itshape familles}, et à l’aide de ces mêmes familles commencèrent à cultiver les champs ; la foule des autres hommes en sortit longtemps après en se {\itshape réfugiant} sur les terres cultivées par les premiers pères de famille.\par
71. {\itshape Les habitudes originaires}, particulièrement celle de l’indépendance naturelle, {\itshape ne se perdent point tout d’un coup}, mais par degrés et à force de temps.\par
72. Supposé que toutes les sociétés aient commencé par le culte d’une divinité quelconque, les {\itshape pères} furent sans doute, dans l’état de famille, les {\itshape sages} en fait de divination, les {\itshape prêtres} qui sacrifiaient pour connaître la volonté du ciel par les  auspices, et les {\itshape rois} qui transmettaient les lois divines à leur famille.\par
73 et 76. C’est une tradition vulgaire que le monde fut d’abord gouverné par des rois, — que la première forme de gouvernement fut la monarchie.\par
74. Autre tradition vulgaire : {\itshape les premiers rois qui furent élus, c’étaient les plus dignes}.\par
75. Autre : {\itshape les premiers rois furent des sages}. Le vain souhait de Platon était en même temps un regret de ces premiers âges pendant lesquels {\itshape les philosophes régnaient, ou les rois étaient philosophes}.\par
Dans la personne des premiers pères se trouvèrent donc réunis la sagesse, le sacerdoce et la royauté. Les deux dernières supériorités dépendaient de la première. Mais cette sagesse n’était point la sagesse {\itshape réfléchie} (riposta) celle des philosophes, mais la {\itshape sagesse vulgaire} des législateurs. Nous voyons que dans la suite chez toutes les nations les prêtres marchaient la couronne sur la tête.\par
77. Dans l’état de famille, les pères durent exercer un {\itshape pouvoir monarchique}, dépendant de Dieu seul, sur la personne et sur les biens de leurs {\itshape fils}, et, à plus forte raison, sur ceux des hommes qui s’étaient réfugiés sur leurs terres, et qui étaient devenus leurs {\itshape serviteurs}. Ce sont ces premiers monarques du monde que désigne l’Écriture Sainte en les  appelant {\itshape patriarches}, c’est-à-dire, {\itshape pères et princes}. Ce droit monarchique fut conservé par la loi des douze tables dans tous les âges de l’ancienne Rome : \emph{{\itshape Patri familias jus vitæ et necis in liberos esto}}, le père de famille a sur ses enfants droit de vie et de mort ; principe d’où résulte le suivant, \emph{{\itshape quidquid filius acquirit, patri acquirit}}, tout ce que le fils acquiert, il l’acquiert à son père.\par
78. Les {\itshape familles} ne peuvent avoir été nommées d’une manière convenable à leur origine, si l’on n’en fait venir le nom de ces {\itshape famuli}, ou serviteurs des premiers pères de famille.\par
79. Si les premiers {\itshape compagnons}, ou {\itshape associés}, eurent pour but une {\itshape société d’utilité}, on ne peut les placer antérieurement à ces réfugiés qui, ayant cherché la sûreté près des premiers pères de famille, furent obligés pour vivre de cultiver les champs de ceux qui les avaient reçus. — Tels furent les véritables {\itshape compagnons des héros}, dans lesquels nous trouvons plus tard les {\itshape plébéiens} des cités héroïques, et en dernier lieu les {\itshape provinces soumises} à des peuples souverains.\par
80. Les hommes s’engagent dans des rapports de bienfaisance, lorsqu’ils espèrent retenir une partie du {\itshape bienfait}, ou en tirer une grande utilité ; tel est le genre de bienfait que l’on doit attendre dans la vie sociale.\par
\par
 81. C’est un caractère des hommes courageux de ne point laisser perdre par négligence ce qu’ils ont acquis par leur courage, mais de ne céder qu’à la nécessité ou à l’intérêt, et cela peu à peu, et le moins qu’ils peuvent. Dans ces deux axiomes nous voyons les {\itshape principes éternels des fiefs}, qui se traduisent en latin avec élégance par le mot {\itshape beneficia}.\par
82. Chez toutes les nations anciennes nous ne trouvons partout que {\itshape clientèles} et {\itshape clients}, mots qu’on ne peut entendre convenablement que par {\itshape fiefs} et {\itshape vassaux}. Les feudistes ne trouvent point d’expressions latines plus convenables pour traduire ces derniers mots que {\itshape clientes} et {\itshape clientelæ}.\par
Les trois derniers axiomes avec les douze précédents (en partant du 70\textsuperscript{e}), nous font connaître l’{\itshape origine des sociétés}. Nous trouvons cette origine, comme on le verra d’une manière plus précise, dans la nécessité imposée aux pères de famille par leurs serviteurs. Ce premier gouvernement dut être {\itshape aristocratique}, parce que les pères de famille s’unirent en corps politique pour résister à leurs serviteurs mutinés contre eux, et furent cependant obligés pour les ramener à l’obéissance, de leur faire des concessions de terres analogues aux {\itshape feuda rustica} ({\itshape fiefs roturiers}) du moyen âge. Ils se trouvèrent eux-mêmes avoir assujetti leurs souverainetés domestiques (que l’on peut comparer aux {\itshape fiefs nobles}) à la {\itshape souveraineté de l’ordre} dont ils faisaient partie. Cette origine des sociétés sera prouvée par le fait ;  mais quand elle ne serait qu’une hypothèse, elle est si simple et si naturelle, tant de phénomènes politiques s’y rapportent d’eux-mêmes, comme à leur cause, qu’il faudrait encore l’admettre comme vraie. Autrement il devient impossible de comprendre comment l’{\itshape autorité civile} dériva de l’{\itshape autorité domestique} ; comment le patrimoine public se forma de la réunion des patrimoines particuliers ; comment à sa formation, la société trouva des éléments tout préparés dans un corps peu nombreux qui pût commander dans une multitude de plébéiens qui pût obéir. Nous démontrerons qu’en supposant les familles composées seulement {\itshape de fils}, et non {\itshape de serviteurs}, cette formation des sociétés a été impossible.\par
83. Ces concessions de terres constituèrent la première {\itshape loi agraire} qui ait existé, et la nature ne permet pas d’en {\itshape imaginer}, ni d’en {\itshape comprendre} une qui puisse offrir plus de précision.\par
Dans cette loi agraire furent distingués les trois genres de possession qui peuvent appartenir aux trois sortes de personnes : {\itshape domaine bonitaire} appartenant aux Plébéiens ; {\itshape domaine quiritaire} appartenant aux Pères, conservé par les armes, et par conséquent {\itshape noble} ; {\itshape domaine éminent}, appartenant au corps souverain. Ce dernier genre de possession n’est autre chose que la souveraine puissance dans les républiques aristocratiques.
\subsubsection[{84-96. Ancienne histoire romaine}]{ \textsc{84-96. } {\itshape Ancienne histoire romaine} }
\noindent  84. Dans un passage remarquable de sa {\itshape Politique}, où il énumère les diverses sortes de gouvernements, Aristote fait mention de la {\itshape royauté héroïque}, où les rois, chefs de la religion, administraient la justice au-dedans, et conduisaient les guerres au-dehors.\par
Cet axiome se rapporte précisément à la royauté héroïque de Thésée et de Romulus. {\itshape Voyez} la vie du premier dans Plutarque. Quant aux rois de Rome, nous voyons Tullus Hostilius juge d’Horace\footnote{Par l’intermédiaire des duumvix auxquels il délègue son pouvoir. ({\itshape Note du Traducteur.})}. Les rois de Rome étaient appelés rois des choses sacrées, {\itshape reges sacrorum}. Et même après l’expulsion des rois, de crainte d’altérer la forme des cérémonies, on créait un roi des choses sacrées ; c’était le chef des féciaux, ou hérauts de la république.\par
85. Autre passage remarquable de la {\itshape Politique} d’Aristote : \emph{{\itshape Les anciennes républiques n’avaient point de lois pour punir les offenses et redresser les torts particuliers ; ce défaut de lois est commun à tous les peuples barbares.}} En effet les peuples ne sont barbares dans leur origine que parce qu’ils ne sont pas encore adoucis par les lois. — De là la {\itshape nécessité des duels et des représailles personnelles} dans les temps barbares, où l’on manque de {\itshape lois judiciaires}.\par
 86. Troisième passage non moins précieux du même livre : \emph{{\itshape Dans les anciennes républiques, les nobles juraient aux plébéiens une éternelle inimitié.}} Voilà ce qui explique l’orgueil, l’avarice, et la barbarie des nobles à l’égard des plébéiens, dans les premiers siècles de l’histoire romaine. Au milieu de cette prétendue liberté populaire que l’imagination des historiens nous montre dans Rome, ils {\itshape pressaient}\footnote{Ce mot est pris dans le sens anglais, {\itshape to press. Angariarono}. ({\itshape Note du Traducteur.})} les plébéiens, et les forçaient de les servir à la guerre à leurs propres dépens ; ils les enfonçaient, pour ainsi dire, dans un abîme d’usures ; et lorsque ces malheureux n’y pouvaient satisfaire, ils les tenaient enfermés toute leur vie dans leurs prisons particulières, afin de se payer eux-mêmes par leurs travaux et leurs sueurs ; là, ces tyrans les déchiraient à coups de verges comme les plus vils esclaves.\par
87. Les républiques aristocratiques se décident difficilement à la guerre, de crainte d’aguerrir la multitude des plébéiens.\par
88. Les gouvernements aristocratiques conservent les richesses dans l’ordre des nobles, parce qu’elles contribuent à la puissance de cet ordre. — C’est ce qui explique la clémence avec laquelle les Romains traitaient les vaincus ; ils se contentaient de leur ôter  leurs armes, et leur laissaient la jouissance de leurs biens ({\itshape dominium bonitarium}), sous la condition d’un tribut supportable. — Si l’aristocratie romaine combattit toujours les lois agraires proposées par les Gracques, c’est qu’elle craignait d’enrichir le petit peuple.\par
89. L’{\itshape honneur} est le plus noble aiguillon de la valeur militaire.\par
90. Les peuples, chez lesquels les différents ordres se disputent les {\itshape honneurs} pendant la paix, doivent déployer à la guerre une {\itshape valeur héroïque} ; les uns veulent se conserver le privilège des honneurs, les autres mériter de les obtenir. Tel est le principe de l’{\itshape héroïsme} romain depuis l’expulsion des rois jusqu’aux guerres puniques. Dans cette période, les nobles se dévouaient pour leur patrie, dont le salut était lié à la conservation des privilèges de leur ordre ; et les plébéiens se signalaient par de brillants exploits pour prouver qu’ils méritaient de partager les mêmes honneurs.\par
91. Les querelles dans lesquelles les différents ordres cherchent {\itshape l’égalité des droits}, sont pour les républiques le plus puissant moyen d’agrandissement.\par
Autre principe de l’{\itshape héroïsme} romain, appuyé sur trois vertus civiles : {\itshape confiance magnanime des plébéiens}, qui veulent que les patriciens leur communiquent  les droits civils, en même temps que ces lois dont ils se réservent la connaissance mystérieuse ; {\itshape courage des patriciens}, qui retiennent dans leur ordre un privilège si précieux ; {\itshape sagesse des jurisconsultes}, qui interprètent ces lois, et qui peu à peu en étendent l’utilité en les appliquant à de nouveaux cas, selon ce que demande la raison. Voilà les trois caractères qui distinguent exclusivement la jurisprudence romaine.\par
92. Les faibles veulent les lois ; les puissants les repoussent ; les ambitieux en présentent de nouvelles pour se faire un parti ; les princes protègent les lois, afin d’égaler les puissants et les faibles.\par
Dans sa première et sa seconde partie, cet axiome éclaire l’histoire des querelles qui agitent les aristocraties. Les nobles font de la connaissance des lois le {\itshape secret} de leur ordre, afin qu’elles dépendent de leurs caprices, et qu’ils les appliquent {\itshape aussi arbitrairement que des rois}. Telle est, selon le jurisconsulte Pomponius, la raison pour laquelle les plébéiens désiraient la loi des douze tables : \emph{{\itshape gravia erant jus latens, incertum, et manus regia}}. C’est aussi la cause de la répugnance que montraient les sénateurs pour accorder cette législation : \emph{{\itshape mores patrios servandos ; leges ferri non oportere}}. Tite-Live dit au contraire, que les nobles ne repoussaient pas les vœux du peuple, \emph{{\itshape desideria plebis non aspernari}}. Mais Denis d’Halicarnasse, devait être mieux informé que Tite-Live des antiquités romaines, puisqu’il  écrivait d’après les mémoires de Varron, le plus docte des Romains\footnote{Nous rejetons une longue digression sur la question de savoir si les lois des douze tables ont été transportées d’Athènes à Rome, dans la note où nous citerons un passage plus considérable d’un autre ouvrage de Vico sur le même sujet. ({\itshape Note du Traducteur.})}.\par
Le troisième article du même axiome nous montre la route que suivent les ambitieux dans les états populaires pour s’élever au pouvoir souverain ; ils secondent le désir naturel du peuple, qui, ne pouvant s’élever aux idées générales, veut une loi pour chaque cas particulier. Aussi voyons-nous que Sylla, chef du parti de la noblesse, n’eut pas plus tôt vaincu Marius, chef du parti du peuple, et rétabli la république en rendant le gouvernement à l’aristocratie, qu’il remédia à la multitude des lois par l’institution des {\itshape quæstiones perpetuæ}.\par
Enfin le même axiome nous fait connaître dans sa dernière partie le secret motif pour lequel les Empereurs, en commençant par Auguste, firent des {\itshape lois innombrables pour des cas particuliers} ; et pourquoi chez les modernes tous les états monarchiques ou républicains ont reçu le corps du droit romain, et celui du droit canonique.\par
93. Dans les démocraties où domine une multitude avide, dès qu’une fois cette multitude s’est ouvert par les lois la porte des honneurs, la paix n’est plus qu’une lutte dans laquelle on se dispute la puissance, non plus avec les lois, mais avec les  armes ; et la puissance elle-même est un moyen de faire des lois pour enrichir le parti vainqueur ; telles furent à Rome les lois agraires proposées par les Gracques. De là résultent à la fois des guerres civiles au-dedans, des guerres injustes au-dehors.\par
Cet axiome confirme par son contraire ce qu’on a dit de l’{\itshape héroïsme} romain pour tout le temps antérieur aux Gracques.\par
94. Plus les biens sont attachés à la personne, au corps du possesseur, plus la liberté naturelle conserve sa fierté ; c’est avec le superflu que la servitude enchaîne les hommes.\par
Dans son premier article, cet axiome est un nouveau principe de l’{\itshape héroïsme} des premiers peuples ; dans le second, c’est le {\itshape principe naturel des monarchies}.\par
95. Les hommes aiment d’abord à sortir de sujétion et désirent l’{\itshape égalité} ; voilà les plébéiens dans les républiques aristocratiques, qui finissent par devenir des gouvernements populaires. Ils s’efforcent ensuite de {\itshape surpasser leurs égaux} ; voilà le petit peuple dans les états populaires qui dégénèrent en oligarchies. Ils veulent enfin {\itshape se mettre au-dessus des lois} ; et il en résulte une démocratie effrénée, une anarchie, qu’on peut appeler la pire des tyrannies, puisqu’il y a autant de tyrans qu’il se trouve d’hommes audacieux et dissolus dans la cité. Alors le petit peuple, éclairé par ses propres maux, y cherche un  remède en {\itshape se réfugiant dans la monarchie}. Ainsi nous trouvons dans la nature cette {\itshape loi royale} par laquelle Tacite légitime la monarchie d’Auguste : \emph{{\itshape qui cuncta bellis civilibus fessa nomine principis sub imperium}{\scshape accepit}}.\par
96. Lorsque la réunion des familles forma les premières cités, {\itshape les nobles} qui sortaient à peine de l’{\itshape indépendance de la vie sauvage}, ne voulaient point se soumettre au frein des lois, ni aux charges publiques ; voilà les {\itshape aristocraties} où les nobles sont seigneurs. Ensuite les plébéiens étant devenus nombreux et aguerris, les nobles se soumirent, comme les plébéiens, aux lois et aux charges publiques ; voilà les nobles dans les {\itshape démocraties}. Enfin pour s’assurer la vie commode dont ils jouissent, ils inclinèrent naturellement à se soumettre au gouvernement d’un seul ; voilà les nobles sous la {\itshape monarchie}.
\subsection[{97-103. Migration des peuples}]{ \textsc{97-103. } {\itshape Migration des peuples} }
\noindent 97. Qu’on m’accorde, et la raison ne s’y refuse pas, qu’après le déluge, les hommes habitèrent d’abord sur les {\itshape montagnes} ; il sera naturel de croire qu’ils descendirent quelque temps après dans les {\itshape plaines}, et qu’au bout d’un temps considérable, ils prirent assez de confiance pour aller jusqu’aux {\itshape rivages} de la mer.\par
98. On trouve dans Strabon un passage précieux  de Platon, où il raconte qu’après les déluges particuliers d’Ogygès et de Deucalion, les hommes habitèrent {\itshape dans les cavernes des montagnes}, et il les reconnaît dans ces cyclopes, ces Polyphèmes, qui lui représentent ailleurs les premiers pères de famille ; ensuite sur les {\itshape sommets} qui dominent les vallées, tels que Dardanus qui fonda Pergame, depuis la citadelle de Troie ; enfin dans les {\itshape plaines}, tels qu’Ilus qui fit descendre Troie jusqu’à la plaine voisine de la mer, et qui l’appela Ilion.\par
99. Selon une tradition ancienne, Tyr, fondée d’abord {\itshape dans les terres}, fut ensuite assise sur le {\itshape rivage} de la mer de Phénicie ; et l’histoire nous apprend que de là elle passa dans une {\itshape île} voisine, qu’Alexandre rattacha par une chaussée au continent.\par
Le postulat 97 et les deux traditions qui viennent à l’appui, nous apprennent que les peuples {\itshape méditerranés} se formèrent d’abord, ensuite les peuples {\itshape maritimes}.\par
Nous y trouvons aussi une preuve remarquable de l’antiquité du peuple hébreux, dont Noé plaça le berceau dans la Mésopotamie, contrée la plus {\itshape méditerranée} de l’ancien monde habitable. Là aussi se fonda la première monarchie, celle des Assyriens, sortis de la tribu chaldéenne, laquelle avait produit les premiers sages, et Zoroastre le plus ancien de tous.\par
100. Pour que les hommes se décident à {\itshape abandonner pour toujours la terre où ils sont nés}, et qui  naturellement leur est chère, il faut les plus extrêmes nécessités. Le désir d’acquérir par le commerce, ou de conserver ce qu’ils ont acquis, peut seul les décider à quitter leur patrie {\itshape momentanément}.\par
C’est le principe de la {\itshape Transmigration des peuples}, dont les moyens furent, ou les {\itshape colonies maritimes des temps héroïques}, ou les {\itshape invasions des barbares}, ou les {\itshape colonies} les plus lointaines {\itshape des Romains}, ou celles {\itshape des Européens dans les deux Indes}.\par
Le même axiome nous démontre que les descendants des fils de Noé durent {\itshape se perdre et se disperser} dans leurs courses vagabondes, comme les bêtes sauvages, soit pour {\itshape échapper} aux animaux farouches qui peuplaient la vaste forêt dont la terre était couverte ; soit en {\itshape poursuivant} les femmes rebelles à leurs désirs, soit en {\itshape cherchant} l’eau et la pâture. Ils se trouvèrent ainsi épais sur toute la terre, lorsque le tonnerre se faisant entendre pour la première fois depuis le déluge, les ramena à des pensées religieuses, et leur fit concevoir un Dieu, un Jupiter ; principe uniforme des sociétés païennes qui eurent chacune leur Jupiter. S’ils eussent conservé des mœurs {\itshape humaines}, comme le peuple de Dieu, ils seraient, comme lui, restés en Asie ; cette partie du monde est si vaste, et les hommes étaient alors si peu nombreux, qu’ils n’avaient aucune nécessité de l’abandonner ; il n’est point dans la nature que l’on quitte par caprice le pays de sa naissance.\par
\par
 101. Les Phéniciens furent les premiers navigateurs du monde ancien.\par
102. Les nations encore barbares {\itshape sont impénétrables} ; au-dehors, il faut la {\itshape guerre} pour les ouvrir aux étrangers, au-dedans l’intérêt du {\itshape commerce}, pour les déterminer à les admettre. Ainsi Psammétique ouvrit l’Égypte aux Grecs de l’Ionie et de la Carie, lesquels durent être célèbres après les Phéniciens par leur commerce maritime\footnote{C’est ce qui explique ces grandes richesses qui permirent aux Ioniens de bâtir le temple de Junon à Samos, et aux Cariens d’élever le tombeau de Mausole, qui furent placés au nombre des sept merveilles du monde. La gloire du commerce maritime appartint en dernier lieu à ceux de Rhodes qui élevèrent à l’entrée de leur port le fameux colosse du Soleil. ({\itshape Vico.})}. Ainsi dans les temps modernes les Chinois ont ouvert leur pays aux Européens.\par
Ces trois axiomes nous donnent le principe d’un {\itshape autre système d’étymologie pour les mots dont l’origine est certainement étrangère}, système différent de celui dans lequel nous trouvons l’{\itshape origine des mots indigènes}. Sans ce principe, nul moyen de connaître l’{\itshape histoire des nations transplantées par des colonies aux lieux où s’étaient établies déjà d’autres nations}. Ainsi Naples fut d’abord appelée {\itshape Sirène}, d’un mot syriaque, ce qui prouve que les Syriens, ou Phéniciens, y avaient d’abord fondé un comptoir. Ensuite elle s’appela {\itshape Parthenope}, d’un mot grec de la langue {\itshape héroïque}, et enfin {\itshape Neapolis} dans la langue grecque vulgaire ; ce qui prouve que les  Grecs s’y étaient établis ensuite, pour partager le commerce des Phéniciens. De même sur les rivages de Tarente il y eut une colonie syrienne appelée {\itshape Siri}, que les Grecs nommèrent ensuite {\itshape Polylée} ; Minerve, qui y avait un temple, en tira le surnom de {\itshape Poliade}.\par
103. Je demande qu’on m’accorde, et on sera forcé de le faire, qu’il y ait eu {\itshape sur le rivage du Latium une colonie grecque}, qui, {\itshape vaincue et détruite par les Romains}, sera restée ensevelie dans les ténèbres de l’antiquité.\par
Si l’on n’accorde point ceci, quiconque réfléchit sur les choses de l’antiquité et veut y mettre quelqu’ensemble, ne trouve dans l’histoire romaine que sujets de s’étonner ; elle nous parle d’{\itshape Hercule}, d’{\itshape Évandre}, d’{\itshape Arcadiens}, de {\itshape Phrygiens établis dans le Latium}, d’un {\itshape Servius Tullius} d’origine grecque, d’un {\itshape Tarquin l’Ancien}, fils du Corinthien Démarate, d’{\itshape Énée}, auquel le peuple romain rapporte sa première origine. \emph{{\itshape Les lettres latines}, comme l’observe Tacite, {\itshape  étaient semblables aux anciennes lettres grecques}} ; et pourtant Tite-Live pense qu’au temps de Servius Tullius, le nom même de Pythagore qui enseignait alors dans son école tant célébrée de Crotone n’avait pu pénétrer jusqu’à Rome. Les Romains ne commencèrent à connaître les Grecs d’Italie qu’à l’occasion de la guerre de Tarente, qui entraîna celle de Pyrrhus et des Grecs d’outre-mer ({\itshape Florus}).
\subsection[{104-114. Principes du droit naturel}]{ \textsc{104-114. } {\itshape Principes du droit naturel} }
\noindent  104. Elle est digne de nos méditations, cette pensée de Dion Cassius : \emph{{\itshape la coutume est semblable à un roi, la loi à un tyran}} : ce qui doit s’entendre de la coutume raisonnable, et de la loi qui n’est point animée de l’esprit de la raison naturelle.\par
Cet axiome termine par le fait la grande dispute à laquelle a donné lieu la question suivante : {\itshape le droit est-il dans la nature, ou seulement dans l’opinion des hommes} ? c’est la même que l’on a proposée dans le corollaire du 8\textsuperscript{e} axiome : {\itshape la nature humaine est-elle sociable} ? Si la coutume commande, comme un roi à des sujets qui veulent obéir, le droit naturel qui a été ordonné par la coutume, est né des mœurs humaines, résultant de la {\scshape nature commune des nations}. Ce droit conserve la société, parce qu’il n’y a chose plus agréable et par conséquent plus naturelle que de suivre les coutumes enseignées par la nature. D’après tout ce raisonnement, {\itshape la nature humaine} dont elles sont un résultat, {\itshape ne peut être que sociable}.\par
Cet axiome, rapproché du 8\textsuperscript{e} et de son corollaire, prouve que l’{\itshape homme n’est pas injuste par le fait de sa nature, mais par l’infirmité d’une nature déchue}. Il nous démontre le premier {\itshape principe du christianisme}, qui se trouve dans le caractère d’Adam, considéré avant le péché, et dans l’état de perfection où il dut avoir été conçu par son créateur. Il nous démontre par suite les {\itshape principes catholiques}  {\itshape de la grâce}. La grâce suppose le libre arbitre, auquel elle prête un secours {\itshape surnaturel}, mais qui est aidé {\itshape naturellement} par la {\itshape Providence} ({\itshape Voy.} le même axiome 8\textsuperscript{e} et son second corollaire.) Sur ce dernier article la religion chrétienne s’accorde avec toutes les autres. Grotius, Selden et Pufendorf devaient fonder leurs systèmes sur cette base, et se ranger à l’opinion des jurisconsultes romains, selon lesquels le {\itshape droit naturel a été ordonné par la divine Providence}.\par
\par
105. Le {\itshape droit naturel des gens est sorti des mœurs et coutumes} des nations, lesquelles se sont rencontrées dans {\itshape un sens commun}, ou manière de voir uniforme, et cela sans {\itshape réflexion}, sans prendre {\itshape exemple} l’une de l’autre.\par
Cet axiome, avec le mot de Dion Cassius qui vient d’être rapporté, établit que la Providence est {\itshape la législatrice du droit naturel des gens}, parce qu’elle est la {\itshape reine des affaires humaines}.\par
Le même axiome établit la différence qui existe entre le {\itshape droit naturel des Hébreux}, celui des {\itshape Gentils}, et celui des {\itshape philosophes}. Les Gentils eurent seulement les secours {\itshape ordinaires} de la Providence, les Hébreux eurent de plus les secours {\itshape extraordinaires} du vrai Dieu, et c’est le principe de la {\itshape division de tous les peuples anciens en Hébreux et Gentils}. Les philosophes par leurs raisonnements arrivèrent à l’idée d’un droit plus parfait que celui que pratiquaient les Gentils ; mais ils ne parurent que  deux mille ans après la fondation des sociétés païennes. Ces trois différences, inaperçues jusqu’ici, renversent les trois systèmes de Grotius, de Selden et de Pufendorf\footnote{Orthographié « Puffendorf » [NdE].}.\par
106. Les sciences doivent prendre pour point de départ l’époque où commence le sujet dont elles traitent\footnote{Cet axiome placé ici à cause de son rapport {\itshape particulier} avec le droit des gens, s’applique {\itshape généralement} tous les objets dont nous avons à parler. Il aurait dû être rangé parmi les {\itshape axiomes généraux} ; si nous l’avons mis en cet endroit, c’est qu’on voit mieux dans le droit des gens que dans toute autre matière particulière, combien il est conforme à la vérité, et important dans l’application ({\itshape Vico}).}.\par
107. Les {\itshape Gentes} (familles, tribus, clans) commencèrent avant les cités ; du moins celles que les Latins appelèrent {\itshape gentes majores}, c’est-à-dire, {\itshape maisons nobles anciennes}, comme celle des {\itshape Pères} dont Romulus composa le sénat, et en même temps la cité de Rome. Au contraire, on appela {\itshape gentes minores}, les {\itshape maisons nobles nouvelles} fondées après les cités, telles que celles des {\itshape Pères}, dont Junius Brutus, après avoir chassé les rois, remplit le sénat, devenu presque désert par la mort des sénateurs que Tarquin le Superbe avait fait périr.\par
108. Telle fut aussi la division des dieux : {\itshape dii majorum gentium}, ou dieux consacrés par les familles avant la fondation des cités ; et {\itshape dii minorum gentium}, ou dieux consacrés par les peuples, comme  Romulus, que le peuple romain appela après sa mort {\itshape Dius Quirinus}.\par
Ces trois axiomes montrent que les systèmes de Grotius, de Selden et de Pufendorf, manquent dans leurs principes mêmes. Ils commencent par les {\itshape nations déjà} formées et composant dans leur ensemble la {\itshape société du genre humain}, tandis que l’{\itshape humanité} commença chez toutes les nations primitives à l’{\itshape époque où les familles étaient les seules sociétés et où elles adoraient les dieux majorum gentium}.\par
109. Les hommes à courtes vues prennent pour la justice ce qu’on leur montre rentrer dans les termes de la loi.\par
110. Admirons la définition que donne Ulpien de l’{\itshape équité civile} : \emph{{\itshape c’est une présomption de droit, qui n’est point connue naturellement à tous les hommes} (comme l’équité naturelle){\itshape , mais seulement à un petit nombre d’hommes, qui réunissant la sagesse, l’expérience et l’étude, ont appris ce qui est nécessaire au maintien de la société}}. C’est ce que nous appelons {\itshape raison d’état}.\par
111. La {\itshape certitude de la loi} n’est qu’une {\itshape ombre effacée} de la raison ({\itshape obscurezza}) {\itshape appuyée sur l’autorité}. Nous trouvons alors les lois {\itshape dures} dans l’application, et pourtant nous sommes obligés de les appliquer en considération de leur {\itshape certitude. Certum}, en bon latin, signifie {\itshape particularisé} ({\itshape individuatum},  comme dit l’école) ; dans ce sens, {\itshape certum}, et {\itshape commune}, sont très bien opposés entre eux.\par
La {\itshape certitude} est le principe de la {\itshape jurisprudence inflexible}, naturelle aux âges barbares, et dont l’{\itshape équité civile} est la règle. Les barbares, n’ayant que des idées particulières, {\itshape s’en tiennent naturellement à cette certitude}, et sont satisfaits, pourvu que les termes de la loi soient appliqués avec précision. Telle est l’idée qu’ils se forment du droit. Aussi la phrase d’Ulpien, \emph{{\itshape lex dura est, sed scripta est}}, s’exprimerait plus élégamment selon la langue et selon la jurisprudence, par les mots : {\itshape lex dura est, sed certa est}.\par
112. Les hommes éclairés estiment conforme à la justice ce que l’impartialité reconnaît être utile dans chaque cause.\par
113. Dans les lois, le {\itshape vrai} est une lumière certaine dont nous éclaire la {\itshape raison naturelle}. Aussi les jurisconsultes disent-ils souvent {\itshape verum est}, pour {\itshape æquum est}. ({\itshape Voy.} les axiomes 9 et 10.)\par
114. L’{\itshape équité naturelle de la jurisprudence humaine} dans son plus grand développement est une {\itshape pratique}, une application {\itshape de la sagesse aux choses de l’utilité} ; car la {\itshape sagesse}, en prenant le mot dans le sens le plus étendu, n’est que la {\itshape science de faire des choses l’usage qu’elles ont dans la nature}.\par
Tel est le principe de la {\itshape jurisprudence humaine}, dont la règle est l’{\itshape équité naturelle}, et qui est inséparable  de la civilisation. Cette jurisprudence, ainsi que nous le démontrerons, est l’{\itshape école publique} d’où sont sortis les philosophes. ({\itshape Voyez} le livre IV, vers la fin.)\par
Les six dernières propositions établissent que la {\itshape Providence a été la législatrice du droit naturel des gens}. Les nations devant vivre pendant une longue suite de siècles encore incapables de connaître la {\itshape vérité} et l’{\itshape équité naturelle}, la Providence permit qu’en attendant elles s’attachassent à la {\itshape certitude} et à l’{\itshape équité civile} qui suit religieusement l’expression de la loi ; de façon qu’elles observassent la loi, même lorsqu’elle devenait {\itshape dure} et rigoureuse dans l’application, {\itshape pour assurer le maintien de la société humaine}.\par
C’est pour avoir ignoré les vérités énoncées dans ces derniers axiomes, que les trois principaux auteurs, qui ont écrit sur le droit naturel des gens, se sont égarés comme de concert dans la recherche des principes sur lesquels ils devaient fonder leurs systèmes. Ils ont cru que les nations païennes, dès leur commencement, avaient compris l’{\itshape équité naturelle} dans sa perfection idéale, sans réfléchir qu’il fallut bien deux mille ans pour qu’il y eût des philosophes, et sans tenir compte de l’assistance particulière que reçut du vrai Dieu un peuple privilégié.
\chapterclose


\chapteropen
\chapter[{Chapitre III. Trois principes fondamentaux}]{Chapitre III. \\
Trois principes fondamentaux}

\chaptercont
\noindent  Maintenant, afin d’éprouver si les propositions que nous avons présentées comme les {\itshape éléments} de la Science nouvelle, peuvent donner forme aux {\itshape matériaux} préparés dans la table chronologique, nous prions le lecteur de réfléchir à tout ce qu’on a jamais écrit sur les principes du savoir divin et humain des Gentils, et d’examiner s’il y trouvera rien qui contredise toutes ces propositions, ou plusieurs d’entre elles, ou même une seule ; chacune étant étroitement liée avec toutes les autres, en ébranler une, c’est les ébranler toutes. S’il fait cette comparaison, il ne verra certainement dans ce qu’on a écrit sur ces matières que des {\itshape souvenirs} confus, que les rêves d’une {\itshape imagination} déréglée ; la {\itshape réflexion} y est restée étrangère, par l’effet des deux vanités dont nous avons parlé (axiome 3). La {\itshape vanité des nations}, dont chacune veut être la plus ancienne de toutes, nous ôte l’espoir de trouver les principes  de la Science nouvelle dans les écrits des {\itshape philologues} ; la {\itshape vanité des savants}, qui veulent que leurs sciences favorites aient été portées à leur perfection dès le commencement du monde, nous empêche de les chercher dans les ouvrages des {\itshape philosophes} ; nous suivrons donc ces recherches, comme s’il n’existait point de livres.\par
Mais dans cette nuit sombre dont est couverte à nos yeux l’antiquité la plus reculée, apparaît une lumière qui ne peut nous égarer ; je parle de cette vérité incontestable : {\itshape le monde social est certainement l’ouvrage des hommes} ; d’où il résulte que l’on en peut, que l’on en doit trouver les principes dans les modifications mêmes de l’intelligence humaine. Cela admis, tout homme qui réfléchit, ne s’étonnera-t-il pas que les philosophes aient entrepris sérieusement de connaître le {\itshape monde de la nature} que Dieu a fait et dont il s’est réservé la science, et qu’ils aient négligé de méditer sur ce {\itshape monde social}, que les hommes peuvent connaître, puisqu’il est leur ouvrage ? Cette erreur est venue de l’infirmité de l’intelligence humaine : plongée et comme ensevelie dans le corps, elle est portée naturellement à percevoir les choses corporelles, et a besoin d’un grand travail, d’un grand effort pour se comprendre elle-même ; ainsi l’œil voit tous les objets extérieurs, et ne peut se voir lui-même que dans un miroir.\par
Puisque {\itshape le monde social est l’ouvrage des hommes}, examinons en quelle chose ils se sont rapportés et  {\itshape se rapportent toujours}. C’est de là que nous tirerons {\itshape les principes qui expliquent comment se forment, comment se maintiennent toutes les sociétés}, principes universels et éternels, comme doivent l’être ceux de toute science.\par
Observons toutes les nations barbares ou policées, quelque éloignées qu’elles soient de temps ou de lieu ; elles sont fidèles à trois coutumes {\itshape humaines} : toutes ont une {\itshape religion} quelconque, toutes contractent des {\itshape mariages solennels}, toutes {\itshape ensevelissent} leurs morts. Chez les nations les plus sauvages et les plus barbares, nul acte de la vie n’est entouré de cérémonies plus augustes, de solennités plus saintes, que ceux qui ont rapport à la {\itshape religion}, aux {\itshape mariages}, aux {\itshape sépultures}. Si des idées uniformes chez des peuples inconnus entre eux doivent avoir un principe commun de vérité, Dieu a sans doute enseigné aux nations que partout la civilisation avait eu cette triple base, et qu’elles devaient à ces trois institutions une fidélité religieuse, de peur que le monde ne redevînt sauvage et ne se couvrît de nouvelles forêts. C’est pourquoi nous avons pris ces trois coutumes éternelles et universelles pour les {\itshape trois premiers principes de la science nouvelle}.\par
\section[{I}]{I}
\noindent Qu’on n’oppose point au premier de nos principes le témoignage de quelques voyageurs modernes, selon lesquels les Cafres, les Brésiliens, quelques peuples des Antilles et d’autres parties du Nouveau-Monde, vivent en société sans avoir aucune  connaissance de Dieu\footnote{Bayle a sans doute été trompé par leurs rapports, lorsqu’il affirme, dans le Traité de la Comète, \emph{{\itshape que les peuples peuvent vivre dans la justice sans avoir besoin de la lumière de Dieu}}. Avant lui, Polybe avait dit : \emph{{\itshape si les hommes étaient philosophes, il n’y aurait plus besoin de religion}}. Mais s’il n’existait point de société, y aurait-il des philosophes ? Or, sans les religions, point de société. ({\itshape Vico.}) Les trois dernières lignes sont tirées du second corollaire de l’axiome 31. [{\itshape Note du Traducteur.}]}. Ce sont nouvelles de voyageurs, qui, pour faciliter le débit de leurs livres, les remplissent de récits monstrueux. Toutes les nations ont cru un Dieu, une Providence. Aussi dans toute la suite des temps, dans toute l’étendue du monde, on peut réduire à quatre le nombre des religions principales. Celles des Hébreux et des Chrétiens qui attribuent à la Divinité un esprit libre et infini ; celle des idolâtres qui la partagent entre plusieurs dieux composés d’un corps et d’un esprit libre ; enfin celle des Mahométans, pour lesquels Dieu est un esprit infini et libre dans un corps infini ; ce qui fait qu’ils placent les récompenses de l’autre vie dans les plaisirs des sens.\par
Aucune nation n’a cru à l’existence d’un Dieu tout matériel, ni d’un Dieu tout intelligence sans liberté. Aussi les Épicuriens qui ne voient dans le monde que matière et hasard, les Stoïciens qui, semblables en ceci aux Spinosistes, reconnaissent pour Divinité une intelligence infinie animant une matière infinie et soumise au destin, ne pourront raisonner de législation ni de politique. Spinoza\footnote{Orthographié « Spinosa » [NdE].} parle de la société civile comme d’une société de marchands. Cicéron disait à l’épicurien Atticus qu’il  ne pouvait raisonner avec lui sur la législation, à moins qu’il ne lui accordât l’existence d’une Providence divine. Dira-t-on encore que la secte stoïcienne et l’épicurienne s’accordent avec la jurisprudence romaine, qui prend l’existence de cette Providence pour premier principe ?
\section[{II}]{II}
\noindent L’opinion selon laquelle l’{\itshape union de l’homme et de la femme sans mariage solennel serait innocente}, est accusée d’erreur par les usages de toutes les nations. Toutes célèbrent religieusement les mariages, et semblent par là regarder les unions illégitimes comme une sorte de bestialité, quoique moins coupable. En effet les parents dont le lien des lois n’assure point l’union, {\itshape perdent} leurs enfants, autant qu’il est en eux ; le père et la mère pouvant toujours se séparer, l’enfant abandonné de l’un et de l’autre, doit rester exposé à devenir la proie des chiens ; et si l’humanité publique ou privée ne l’élevait, il croîtrait sans qu’on lui transmît ni religion, ni langue, ni aucun élément de civilisation. Ainsi, de ce monde social embelli et policé par tous les arts de l’humanité, ils tendent à en faire la grande forêt des premiers âges, où, avant Orphée, erraient les hommes à la manière des bêtes sauvages, suivant au hasard la coupable brutalité de leurs appétits, où un amour sacrilège unissait les fils à leurs mères, et les pères à leurs filles.
\section[{III}]{III}
\noindent Enfin pour apprécier l’importance du troisième  principe de la civilisation, qu’on imagine un état dans lequel les cadavres humains resteraient sur la terre sans {\itshape sépulture}, pour servir de pâture aux chiens et aux oiseaux de proie. Dès lors les cités se dépeupleraient, les champs resteraient sans culture, et les hommes chercheraient les glands mêlés et confondus avec la cendre des morts. Aussi c’est avec raison qu’on a désigné les sépultures par cette expression sublime \emph{{\itshape fœdera generis humani}}, et par cette autre expression moins élevée qu’emploie Tacite, \emph{{\itshape humanitatis commercia}}. Toutes les nations païennes se sont accordées à croire que les âmes allaient errantes autour des corps laissés sans sépulture, et demeuraient inquiètes sur la terre ; que par conséquent elles survivaient aux corps, et étaient {\itshape immortelles}. Les rapports des voyageurs modernes nous prouvent que maintenant encore plusieurs peuples barbares partagent cette croyance. La chose nous est attestée pour les Péruviens et les Mexicains par Acosta, pour les peuples de la Virginie par Thomas Aviot, pour ceux de la nouvelle Angleterre par Richard Waitborn ; pour ceux de la Guinée par Hugues Linschotan, et pour les Siamois par Joseph Scultenius. — Aussi Sénèque a-t-il dit : \emph{{\itshape Quum de immortalitate loquimur, non leve momentum apud nos habet consensus hominum aut timentium inferos, aut colentium ; hac persuasione publica utor.}}
\chapterclose


\chapteropen
\chapter[{Chapitre IV. De la méthode}]{Chapitre IV. \\
De la méthode}

\chaptercont
\noindent  Pour achever d’établir nos principes, il nous reste dans ce premier livre à examiner la méthode que doit suivre la Science nouvelle. Si, comme nous l’avons dit dans les axiomes, {\itshape la science doit prendre pour point de départ l’époque où commence le sujet de la science}, nous devons, pour nous adresser d’abord aux philologues, commencer aux cailloux de Deucalion, aux pierres d’Amphion, aux hommes nés des sillons de Cadmus, ou des chênes dont parle Virgile (\emph{{\itshape duro robore nati}}). Pour les philosophes, nous partirons des grenouilles d’Épicure, des cigales de Hobbes, des \emph{{\itshape hommes simples et stupides}} de Grotius, des \emph{{\itshape hommes jetés dans le monde sans soin ni aide de Dieu}}, dont parle Pufendorf, des géants grossiers et farouches, tels que les Patagons du détroit de Magellan ; enfin des {\itshape Polyphèmes} d’Homère, dans lesquels Platon reconnaît les premiers pères de famille. Nous devons commencer à  les observer dès le moment où ils ont commencé à penser {\itshape en hommes} ; et nous trouvons d’abord que, dans cette barbarie profonde, leur liberté bestiale ne pouvait être domptée et enchaînée que par l’{\itshape idée d’une divinité quelconque qui leur inspirât de la terreur}. Mais, lorsque nous cherchons comment cette première pensée {\itshape humaine} fut conçue dans le monde païen, nous rencontrons de graves difficultés. Comment descendre d’une nature cultivée par la civilisation à cette nature inculte et sauvage ; c’est à grand’peine que nous pouvons la {\itshape comprendre}, loin de pouvoir nous la {\itshape représenter} ?\par
Nous devons donc partir d’une notion quelconque de la divinité dont les hommes ne puissent être privés, quelque sauvages, quelque farouches qu’ils soient ; et voici comment nous expliquons cette connaissance : {\itshape l’homme déchu, n’espérant aucun secours de la nature, appelle de ses désirs quelque chose de surnaturel qui puisse le sauver} ; or cette chose surnaturelle n’est autre que Dieu. Voilà la lumière que Dieu a répandue sur tous les hommes. Une observation vient à l’appui de cette idée, c’est que les libertins qui vieillissent, et qui sentent les forces naturelles leur manquer, deviennent ordinairement religieux.\par
Mais des hommes tels que ceux qui commencèrent les nations païennes, devaient, comme les animaux, ne penser que sous l’aiguillon des passions les plus violentes. En suivant une métaphysique vulgaire qui fut la théologie des poètes, nous  rappellerons ({\itshape Voy.} les axiomes) {\itshape cette idée effrayante d’une divinité} qui borna et contint les {\itshape passions bestiales} de ces hommes perdus, et en fit des {\itshape passions humaines}. De cette idée dut naître le noble {\itshape effort propre à la volonté de l’homme}, de tenir en bride les mouvements imprimés à l’âme par le corps, de manière à les étouffer, comme il convient à l’{\itshape homme sage}, ou à les tourner à un meilleur usage, comme il convient à l’{\itshape homme social}, au membre de la société\footnote{Notre libre arbitre, notre volonté libre peut seule réprimer ainsi l’impulsion du corps… Tous les corps sont des agents nécessaires, et que les mécaniciens appellent {\itshape forces, efforts, puissances}, ne sont que les mouvements des corps, mouvements étrangers au sentiment. ({\itshape Vico.})}.\par
Cependant, par un effet de leur nature corrompue, les hommes toujours tyrannisés par l’égoïsme, ne suivent guère que leur intérêt ; chacun voulant pour soi tout ce qui est utile, sans en faire part à son prochain, ils ne peuvent {\itshape donner à leurs passions la direction salutaire qui les rapprocherait de la justice}. Partant de ce principe, nous établissons que l’homme {\itshape dans l’état bestial, n’aime que sa propre conservation} ; il prend femme, il a des enfants, et il aime sa conservation {\itshape en y joignant celle de sa famille} ; arrivé à la vie civile, il cherche à la fois sa propre conservation et celle {\itshape de la cité} dont il fait partie ; lorsque les empires s’étendent sur plusieurs peuples, il cherche avec sa conservation celle {\itshape des nations} dont il est membre ; enfin quand les nations sont liées par les rapports des traités, du  commerce, et de la guerre, il embrasse dans un même désir sa conservation et {\itshape celle du genre humain}. Dans toutes ces circonstances, l’homme est principalement attaché à son intérêt particulier. Il faut donc que ce soit {\itshape la Providence} elle-même qui le retienne dans cet ordre de choses, et {\itshape qui lui fasse suivre dans la justice la société de famille, de cité, et enfin la société humaine}. Ainsi conduit par elle, l’homme incapable d’atteindre toute l’utilité qu’il désire, obtient ce qu’il en doit prétendre, et c’est ce qu’on appelle {\itshape le juste}. La dispensatrice du juste parmi les hommes, c’est la {\itshape justice divine}, qui, appliquée aux affaires du monde par la Providence, conserve la {\itshape société humaine}.\par
La {\itshape science nouvelle} sera donc sous l’un de ses principaux aspects une {\itshape théologie civile de la Providence divine}, laquelle semble avoir manqué jusqu’ici. Les philosophes ont ou entièrement méconnu la Providence, comme les Stoïciens et les Épicuriens, ou l’ont considérée seulement dans l’ordre des choses physiques. Ils donnent le nom de {\itshape théologie naturelle} à la métaphysique, dans laquelle ils étudient cet attribut de Dieu, et ils appuient leurs raisonnements d’observations tirées du {\itshape monde matériel} ; mais c’était surtout dans l’{\itshape économie du monde civil} qu’ils auraient dû chercher les preuves de la Providence… La Science nouvelle sera, pour ainsi parler, {\itshape une démonstration de fait, une démonstration historique de la Providence}, puisqu’elle doit être une histoire des décrets par  lesquels cette Providence a gouverné, à l’insu des hommes, et souvent malgré eux, la grande cité du genre humain. Quoique ce monde ait été créé {\itshape particulièrement} et {\itshape dans le temps}, les lois qu’elle lui a données, n’en sont pas moins {\itshape universelles} et {\itshape éternelles}.\par
Dans la contemplation de cette Providence éternelle et infinie la Science nouvelle trouve des {\itshape preuves divines} qui la confirment et la démontrent. N’est-il pas naturel en effet que la Providence divine ayant pour instrument la {\itshape toute-puissance}, exécute ses décrets par des moyens aussi faciles que le sont les usages et coutumes suivis librement par les hommes… que, conseillée par la {\itshape sagesse infinie}, tout ce qu’elle dispose soit ordre et harmonie… qu’ayant pour fin son {\itshape immense bonté}, elle n’ordonne rien qui ne tende à un bien toujours supérieur à celui que les hommes se sont proposé ? Dans l’obscurité jusqu’ici impénétrable qui couvre l’origine des nations, dans la variété infinie de leurs mœurs et de leurs coutumes, dans l’immensité d’un sujet qui embrasse toutes les choses humaines, peut-on désirer des preuves plus sublimes que celles que nous offriront la {\itshape facilité} des moyens employés par la Providence, l’{\itshape ordre} qu’elle établit, la {\itshape fin} qu’elle se propose, laquelle fin n’est autre que la conservation du genre humain ? Voulons-nous que ces preuves deviennent distinctes et lumineuses ? Réfléchissons avec quelle {\itshape facilité} l’on voit naître les choses, par suite d’occasions lointaines, et  souvent contraires aux desseins des hommes ; et néanmoins elles viennent s’y adapter comme d’elles-mêmes ; autant de preuves que nous fournit la {\itshape toute-puissance}. Observons encore dans l’{\itshape ordre} des choses humaines, comme elles naissent au temps, au lieu où elles doivent naître, comme elles sont différées quand il convient qu’elles le soient\footnote{\noindent C’est en cela qu’Horace fait consister toute la beauté de l’ordre :\par

\begin{verse}
Ordinis hæc virtus erit et Venus, aut ego fallor,\\
Ut jam nunc dicat, jam nunc debentia dici\\
Pleraque differat, et præsens in tempus omittat.\\
\end{verse}

\bibl{{\itshape Art poétique}. ({\itshape Vico.})}
} ; c’est l’ouvrage de la {\itshape sagesse infinie}. Considérons en dernier lieu si nous pouvons concevoir dans telle occasion, dans tel lieu, dans tel temps, quelques {\itshape bienfaits divins} qui eussent pu mieux conduire et conserver la société humaine, au milieu des besoins et des maux éprouvés par les hommes ; voilà les preuves que nous fournit l’{\itshape éternelle bonté} de Dieu. — Ces trois sortes de preuves peuvent se ramener à une seule : Dans toute la série des choses possibles, notre esprit peut-il imaginer des causes plus nombreuses, moins nombreuses, ou autres, que celles dont le monde social est résulté ?… Sans doute le lecteur éprouvera un plaisir divin en ce corps mortel, lorsqu’il {\itshape contemplera dans l’uniformité des idées divines ce monde des nations, par toute l’étendue et la variété des lieux et des temps}. Ainsi nous aurons prouvé par le fait aux Épicuriens que leur hasard ne peut errer selon la folie de ses caprices,  et aux Stoïciens que leur chaîne éternelle des causes à laquelle ils veulent attacher le monde, est elle-même suspendue à la main puissante et bienfaisante du Dieu très grand et très bon.\par
Ces preuves {\itshape théologiques} seront appuyées par une espèce de preuves {\itshape logiques} dont nous allons parler. En réfléchissant sur les commencements de la religion et de la civilisation païennes, on arrive à ces premières origines, au-delà desquelles c’est une vaine curiosité d’en demander d’antérieures ; ce qui est le caractère propre des principes. Alors s’expliquera la manière particulière dont les choses sont nées, autrement dit, leur {\itshape nature} (axiome 14) ; or l’explication de la nature des choses est le propre de la science. Enfin cette explication de leur nature se confirmera par l’observation des {\itshape propriétés éternelles} qu’elles conservent ; lesquelles propriétés ne peuvent résulter que de ce qu’elles sont nées dans tel temps, dans tel lieu et de telle manière, en d’autres termes, de ce qu’elles ont une telle nature (axiomes 14, 15.)\par
Pour arriver à trouver cette nature des choses humaines, la Science nouvelle procède par une {\itshape analyse} sévère {\itshape des pensées humaines relatives aux nécessités ou utilités de la vie sociale, qui sont les deux sources éternelles du droit naturel des gens} (axiome 11). Ainsi considérée sous le second de ses principaux aspects, la Science nouvelle est une {\itshape histoire des idées humaines}, d’après laquelle semble devoir procéder la {\itshape métaphysique de l’esprit humain}. S’il est vrai  que {\itshape les sciences doivent commencer au point même où leur sujet a commencé} (axiome 104), la métaphysique, cette reine des sciences, commença à l’époque où les hommes se mirent à penser {\itshape humainement}, et non point à celle où les philosophes se mirent à réfléchir sur les idées humaines.\par
Pour déterminer l’époque et le lieu où naquirent ces idées, pour donner à leur histoire la certitude qu’elle doit tirer de la {\itshape chronologie et de la géographie métaphysiques} qui lui sont propres, la science nouvelle applique une {\itshape Critique} pareillement {\itshape métaphysique} aux fondateurs, aux {\itshape auteurs des nations}, antérieurs de plus de mille ans aux {\itshape auteurs de livres}, dont s’est occupé jusqu’ici la {\itshape critique philologique}. Le critérium dont elle se sert (axiome 13), est celui que la providence divine a enseigné également à toutes les nations, savoir : {\itshape le sens commun du genre humain}, déterminé par la convenance nécessaire des choses humaines elles-mêmes (convenance qui fait toute la beauté du monde social). C’est pourquoi le genre de preuve sur lequel nous nous appuyons principalement, c’est que, telles lois étant établies par la Providence, la destinée des nations {\itshape a dû, doit} et {\itshape devra} suivre le cours indiqué par la Science nouvelle, quand même des mondes infinis en nombre naîtraient pendant l’éternité ; hypothèse indubitablement fausse. De cette manière, la Science nouvelle trace le cercle éternel d’une {\itshape histoire idéale}, sur lequel tournent {\itshape dans le temps les histoires de toutes les nations}, avec leur naissance, leurs progrès, leur  décadence et leur fin. Nous dirons plus : celui qui étudie la Science nouvelle, se raconte à lui-même cette histoire idéale, en ce sens que {\itshape le monde social étant l’ouvrage de l’homme}, et {\itshape la manière} dont il s’est formé devant, par conséquent, {\itshape se retrouver dans les modifications de l’âme humaine}, celui qui médite cette science s’en crée à lui-même le sujet. Quelle histoire plus certaine que celle où la même personne est à la fois l’acteur et l’historien ? Ainsi la Science nouvelle procède précisément comme la géométrie, qui crée et contemple en même temps le monde idéal des grandeurs ; mais la Science nouvelle a d’autant plus de réalité que les lois qui régissent les affaires humaines en ont plus que les points, les lignes, les superficies et les figures. Cela même montre encore que les preuves dont nous avons parlé sont d’une espèce {\itshape divine}, et qu’elles doivent, ô lecteur, te donner un plaisir {\itshape divin} : car pour Dieu, connaître et faire, c’est la même chose.\par
Ce n’est pas tout ; d’après la définition du {\itshape vrai} et du {\itshape certain} que nous avons donnée plus haut, les hommes furent longtemps incapables de connaître le {\itshape vrai} et la {\itshape raison}, source de la {\itshape justice intérieure}\footnote{Cette justice intérieure, fut pratiquée par les Hébreux que le vrai Dieu éclairait de sa lumière, et auxquels sa loi défendait jusqu’aux pensées injustes, chose dont les législateurs mortels ne s’étaient jamais embarrassés. Les Hébreux croyaient en un Dieu tout esprit, qui scrute le cœur des hommes ; les gentils croyaient leurs dieux composés d’âme et de corps, et par conséquent incapables de pénétrer dans les cœurs. La justice intérieure ne fut connue chez eux que par les raisonnements des philosophes, lesquels ne parurent que deux mille ans après la formation des nations qui les produisirent. ({\itshape Vico.})},  qui peut seule suffire aux intelligences. Mais en attendant, ils se gouvernèrent par la {\itshape certitude de l’autorité}, par le {\itshape sens commun du genre humain} (critérium de notre Critique métaphysique), sur le témoignage duquel se repose la conscience de toutes les nations (axiome 9). Ainsi sous un autre aspect, la science nouvelle devient une {\itshape philosophie de l’autorité}, source de la justice {\itshape extérieure}, pour parler le langage de la théologie morale. Les trois principaux auteurs qui ont écrit sur le droit naturel (Grotius, Selden et Pufendorf), auraient dû tenir compte de cette autorité, plutôt que de celles qu’ils tirent de tant de citations d’auteurs. Elle a régné chez les nations plus de mille ans avant qu’elles eussent des écrivains ; ces écrivains n’ont donc pu en avoir aucune connaissance. Aussi Grotius, plus érudit et plus éclairé que les deux autres, combat les jurisconsultes romains presque sur tous les points ; mais les coups qu’il leur porte ne frappent que l’air, puisque ces jurisconsultes ont établi leurs principes de justice sur la {\itshape certitude de l’autorité du genre humain}, et non sur l’{\itshape autorité des hommes déjà éclairés}.\par
\par
Telles sont les preuves {\itshape philosophiques} qu’emploiera cette science. Les preuves {\itshape philologiques} doivent venir en dernier lieu ; elles peuvent se ramener toutes aux sept classes suivantes : 1º Notre {\itshape explication des fables} se rapporte à notre système d’une manière naturelle, et qui n’a rien de  pénible ou de forcé. Nous montrons dans les fables l’{\itshape histoire civile des premiers peuples}, lesquels se trouvent avoir été partout naturellement {\itshape poètes}. 2º Même accord avec les {\itshape locutions héroïques}, qui s’expliqueront dans toute la vérité du sens, dans toute la propriété de l’expression ; 3º et avec les {\itshape étymologies des langues indigènes}, qui nous donnent l’histoire des choses exprimées par les mots, en examinant d’abord leur sens propre et originaire, et en suivant le progrès naturel du sens figuré, conformément à l’ordre des idées dans lequel se développe l’histoire des langues (axiomes 64, 65). 4º Nous trouvons encore expliqué par le même système le {\itshape vocabulaire mental des choses relatives à la société}\footnote{{\itshape Voyez} l’axiome 22, et le second chapitre du II\textsuperscript{e} livre, corollaire relatif au mot {\itshape Jupiter}.}, qui, prises dans leur substance, ont été perçues d’une manière uniforme par le {\itshape sens} de toutes les nations, et qui dans leurs modifications diverses, ont été diversement {\itshape exprimées} par les langues. 5º Nous séparons le vrai du faux en tout ce que nous ont conservé les {\itshape traditions vulgaires} pendant une longue suite de siècles. Ces traditions ayant été suivies si longtemps, et par des peuples entiers, doivent avoir eu un motif commun de vérité (axiome 16). 6º Les {\itshape grands débris} qui nous restent de l’antiquité, jusqu’ici inutiles à la science, parce qu’ils étaient négligés, mutilés, dispersés, reprennent leur éclat, leur place et leur ordre naturels.  7º Enfin tous les faits que nous raconte l’{\itshape histoire certaine} viennent se rattacher à ces antiquités expliquées par nous, comme à leurs causes naturelles. — Ces {\itshape preuves philologiques} nous font voir dans la {\itshape réalité} les choses que nous avons aperçues dans la méditation du monde {\itshape idéal}. C’est la méthode prescrite par Bacon, {\itshape cogitare, videre}. Les preuves {\itshape philosophiques} que nous avons placées d’abord, confirment par la {\itshape raison l’autorité} des preuves {\itshape philologiques}, qui à leur tour prêtent aux premières l’appui de leur {\itshape autorité} (axiome 10.)\par
Concluons tout ce qui s’est dit en général pour {\itshape établir les principes de la Science nouvelle}. Ces principes sont {\itshape la croyance en une Providence divine, la modération des passions par l’institution du mariage}, et le dogme de l’{\itshape immortalité de l’âme} consacré par l’usage des {\itshape sépultures}. Son critérium est la maxime suivante : {\itshape ce que l’universalité ou la pluralité du genre humain sent être juste, doit servir de règle dans la vie sociale}. La sagesse {\itshape vulgaire} de tous les législateurs, la sagesse {\itshape profonde} des plus célèbres philosophes s’étant accordées pour admettre ces principes et ce critérium, on doit y trouver les bornes de la raison humaine ; et quiconque veut s’en écarter doit prendre garde de s’écarter de l’humanité tout entière.
\chapterclose

\chapterclose


\chapteropen
\part[{Livre second. De la sagesse poétique}]{Livre second. \\
De la sagesse poétique}

\chaptercont

\chapteropen
\chapter[{Argument}]{Argument}

\chaptercont
\noindent  {\itshape Frappé de l’idée que l’admiration exagérée pour la sagesse des premiers âges est le plus grand obstacle aux progrès de la philosophie de l’histoire, l’auteur examine comment les peuples des temps poétiques imaginèrent la Nature, qu’ils ne pouvaient connaître encore. Il appelle cet ensemble des croyances antiques, sagesse, et non pas science, parce qu’elles se rapportaient généralement à un but pratique. Dans ce livre, il passe en revue toutes les idées que les premiers hommes se firent sur la logique et la morale, sur l’économie domestique et politique, sur la physique, la cosmographie et l’astronomie, sur la chronologie et la géographie. C’est en quelque sorte l’encyclopédie des peuples barbares (M. Jannelli}, Delle cose humane).\par
Chapitre I\textsuperscript{er}. {\scshape Sujet de ce livre}. == § I. {\itshape Les fables n’ont point le sens mystérieux que les philosophes leur ont attribué. La Providence a mis dans l’instinct des premiers hommes les germes de civilisation que la réflexion devait ensuite développer.} — § II. {\itshape De la sagesse en général. Sens divers de ce mot à différentes époques.} — § III. {\itshape Exposition et division de la} sagesse poétique.\par
 Chapitre {\scshape II. De la métaphysique poétique.} == § I. {\itshape Origine de la poésie, de l’idolâtrie, de la divination et des sacrifices. Certitude du déluge universel et de l’existence des géants. Les premiers peuples furent poètes naturellement et nécessairement. La crédulité, et non l’imposture, fit les premiers dieux.} — § II. {\itshape Corollaires relatifs aux principaux aspects de la science nouvelle. Philosophie de la propriété, histoire des idées humaines, critique philosophique, histoire idéale éternelle, système du droit naturel des gens, origines de l’histoire universelle.}\par
Chapitre {\scshape III. De la Logique poétique.} — § I. {\itshape Définition et étymologie du mot logique. Les premiers hommes divinisèrent tous les objets, et prirent les noms de ces dieux pour signes ou symboles des choses qu’ils voulaient exprimer.} — § II. {\itshape Corollaires relatifs aux tropes, aux métamorphoses poétiques et aux monstres de la fable. Origine des principales figures. Ces figures du langage, ces créations de la poésie, ne sont point, comme on l’a cru, l’ingénieuse invention des écrivains, mais des formes nécessaires dont toutes les nations se sont servies à leur premier âge, pour exprimer leurs pensées.} — § III. {\itshape Corollaires relatifs aux} caractères poétiques{\itshape  employés comme signes du langage par les premières nations. Solon, Dracon, Ésope, Romulus et autres rois de Rome, les décemvirs, etc.} — § IV. {\itshape Corollaires relatifs à l’origine des langues et des lettres, dans laquelle nous}  {\itshape devons trouver celle des hiéroglyphes, des lois, des noms, des armoiries, des médailles, des monnaies. On n’a pu trouver jusqu’ici l’origine des langues, ni celle des lettres, parce qu’on les a cherchées séparément. Les premiers hommes ont dû parler successivement trois langues, l’}hiéroglyphique{\itshape , la} symbolique{\itshape  et la} vulgaire{\itshape . Les langues vulgaires n’ont point une signification arbitraire. Ordre dans lequel furent trouvées les parties du discours dans la langue articulée ou vulgaire.} — § V. {\itshape Corollaires relatifs à l’origine de l’élocution poétique, des épisodes, du tour, du nombre, du chant et du vers. Ces ornements du style naquirent, dans l’origine, de l’indigence du langage. La poésie a précédé la prose.} — § VI. {\itshape Corollaires relatifs à la logique des esprits cultivés. La topique naquit avant la critique. Ordre dans lequel les diverses méthodes furent employées par la philosophie. Incapacité des premiers hommes de s’élever aux idées générales, surtout en législation.}\par
Chapitre {\scshape IV. De la morale poétique}, {\itshape et de l’origine des vertus} vulgaires{\itshape  qui résultèrent de l’institution de la religion et des mariages. Caractère farouche et religions sanguinaires des hommes de l’âge d’or. Ces religions furent cependant nécessaires.}\par
Chapitre V. Du gouvernement de la famille, ou {\scshape Économie} dans les âges poétiques. == § I. {\itshape De la famille}  {\itshape composée des parents et des enfants, sans esclaves ni serviteurs. Éducation des âmes, éducation des corps. Les premiers pères furent à la fois les sages, les prêtres et les rois de leur famille. La sévérité du gouvernement de la famille prépara les hommes à obéir au gouvernement civil. Les premiers hommes, fixés sur les hauteurs, près des sources vives, perdirent par une vie plus douce la taille des géants. Communauté de l’eau, du feu, des sépultures.} — § II. {\itshape Des familles, en y comprenant non-seulement les parents, mais les} serviteurs (famuli{\itshape ). Cette composition des familles fut antérieure à l’existence des cités, et sans elle cette existence était impossible. Les hommes qui étaient restés sauvages se réfugient auprès de ceux qui avaient déjà formé des familles, et deviennent leurs} clients{\itshape  ou} vassaux{\itshape . Premiers} héros{\itshape . Origine des asiles, des fiefs, etc.} — § III. {\itshape Corollaires relatifs aux contrats qui se font par le simple consentement des parties. Les premiers hommes ne pouvaient connaître les engagements de} bonne foi{\itshape . — Chez eux, les seuls contrats étaient ceux de} cens territorial ; {\itshape point de} contrats de société{\itshape , point de} mandataires.\par
Chapitre {\scshape VI. De la politique poétique.} — § I. {\itshape Origine des premières républiques, dans la forme la plus rigoureusement aristocratique. Puissance sans borne des premiers pères de famille sur leurs enfants et sur leurs} serviteurs{\itshape . Ils sont forcés, par la révolte de ces derniers, de s’unir en corps}  {\itshape politique. Les rois ne sont d’abord que de simples chefs. Premiers comices. Les} serviteurs{\itshape , investis par les nobles ou} héros{\itshape  du} domaine bonitaire{\itshape  des champs qu’ils cultivaient, deviennent les premiers} plébéiens{\itshape , et aspirent à conquérir, avec le droit des mariages solennels, tous les privilèges de la cité.} — § II. {\itshape Les sociétés politiques sont nées toutes de certains principes éternels des fiefs. Différence des} domaines bonitaire, quiritaire, éminent{\itshape . Le corps souverain des nobles avait conservé le dernier, qui était, dans l’origine, un droit général sur tous les fonds de la cité. Opposition des nobles et des plébéiens, des sages et du vulgaire, des citoyens et des hôtes ou étrangers.} — § III. {\itshape De l’origine du cens et du trésor public. Le cens était d’abord une redevance territoriale que les plébéiens payaient aux nobles. Plus tard il fut payé au trésor ; cette institution aristocratique devint ainsi le principe de la démocratie. Observations sur l’histoire des} domaines. — § IV. {\itshape De l’origine des comices chez les Romains. Étymologie des mots} Curia, Quirites, Curètes{\itshape . Révolutions que subirent les comices.} — § V. {\itshape Corollaire : c’est la divine Providence qui règle les sociétés, et qui a ordonné le droit naturel des gens.} — § VI. {\itshape Suite de la politique} héroïque{\itshape . La navigation est l’un des derniers arts qui furent cultivés dans les temps héroïques. Pirateries et caractère inhospitalier des premiers peuples. Leurs guerres continuelles.} — § VII. {\itshape Corollaires relatifs aux antiquités romaines. Le gouvernement de Rome fut},  {\itshape dans son origine, plus aristocratique que monarchique, et malgré l’expulsion des rois, il ne changea point de caractère, jusqu’à l’époque où les plébéiens acquirent le droit des mariages solennels et participèrent aux charges publiques.} — § VIII. {\itshape Corollaire relatif à l’}héroïsme{\itshape  des premiers peuples. Il n’avait rien de la magnanimité, du désintéressement et de l’humanité, dont le mot d’}héroïsme{\itshape  rappelle l’idée dans les temps modernes.}\par
Chapitre {\scshape VII. De la physique poétique.} — § I. {\itshape De la physiologie poétique. Les premiers hommes rapportèrent à diverses parties du corps toutes nos facultés intellectuelles et morales. Note sur l’incapacité de généraliser, qui caractérisait les premiers hommes.} — § II.{\itshape  Corollaire relatif aux descriptions} héroïques{\itshape . Les premiers hommes rapportaient aux cinq sens les fonctions externes de l’âme.} — § III. {\itshape Corollaire relatif aux mœurs héroïques.}\par
Chapitre {\scshape VIII. De la cosmographie poétique.} {\itshape Elle fut proportionnée aux idées étroites des premiers hommes.}\par
Chapitre {\scshape IX. De l’astronomie poétique.} {\itshape Le ciel, que les hommes avaient placé d’abord au sommet des montagnes, s’éleva peu à peu dans leur opinion. Les dieux montèrent dans les planètes, les héros dans les constellations.}\par
 Chapitre {\scshape X. De la chronologie poétique.} {\itshape Son point de départ. Quatre espèces d’anachronismes. Canon chronologique, pour déterminer les commencements de l’histoire universelle, antérieurement au règne de Ninus, d’où elle part ordinairement. L’étude du développement de la civilisation humaine prête une certitude nouvelle aux calculs de la chronologie.}\par
Chapitre {\scshape XI. De la géographie poétique.} — § I. {\itshape Les diverses parties du monde ancien ne furent d’abord que les parties du petit monde de la Grèce. L’Hespérie en était la partie occidentale, etc. Il en dut être de même de la géographie des autres contrées. Les héros qui passent pour avoir fondé des colonies lointaines, Hercule, Évandre, Énée, etc., ne sont que des expressions symboliques du caractère des indigènes qui fondèrent ces villes.} — § II. {\itshape Des noms et descriptions des cités héroïques. Sens et dérivés du mot} ara.\par
{\scshape Conclusion de ce livre.} {\itshape Les poètes théologiens ont été le} sens{\itshape  (ou le} sentiment{\itshape ), les philosophes ont été l’}intelligence{\itshape  de l’humanité.}
\chapterclose


\chapteropen
\chapter[{Chapitre premier. Sujet de ce livre}]{Chapitre premier. \\
Sujet de ce livre}

\chaptercont
\section[{§ I}]{§ I}
\noindent  Nous avons dit dans les axiomes que {\itshape toutes les histoires des Gentils ont eu des commencements fabuleux}, que {\itshape chez les Grecs} qui nous ont transmis tout ce qui nous reste de l’antiquité païenne, {\itshape les premiers sages furent les poètes théologiens}, enfin que {\itshape la nature veut qu’en toute chose les commencements soient grossiers} : d’après ces données, nous pouvons présumer que tels furent aussi les commencements de la {\itshape sagesse poétique}. Cette haute estime dont elle a joui jusqu’à nous est l’effet de la {\itshape vanité des nations}, et surtout de celle {\itshape des savants}. De même que Manéthon, le grand prêtre d’Égypte, interpréta l’histoire fabuleuse des Égyptiens par une haute {\itshape théologie naturelle}, les philosophes grecs donnèrent à la leur une interprétation {\itshape philosophique}. Un  de leurs motifs était sans doute de déguiser l’infamie de ces fables, mais ils en eurent plusieurs autres encore. Le {\itshape premier} fut leur respect pour la religion : chez les Gentils, toute société fut fondée par les fables sur la religion. Le {\itshape second} motif fut leur juste admiration pour l’ordre social qui en est résulté et qui ne pouvait être que l’ouvrage d’une sagesse surnaturelle. En {\itshape troisième} lieu, ces fables tant célébrées pour leur sagesse et entourées d’un respect religieux ouvraient mille routes aux recherches des philosophes, et appelaient leurs méditations sur les plus hautes questions de la philosophie. {\itshape Quatrièmement}, elles leur donnaient la facilité d’exposer les idées philosophiques les plus sublimes en se servant des expressions des poètes, héritage heureux qu’ils avaient recueilli. Un {\itshape dernier} motif, assez puissant à lui seul, c’est la facilité que trouvaient les philosophes à consacrer leurs opinions par l’autorité de la sagesse poétique et par la sanction de la religion. De ces cinq motifs les deux premiers et le dernier impliquaient une louange de la sagesse divine, qui a ordonné le monde civil, et un témoignage que lui rendaient les philosophes, même au milieu de leurs erreurs. Le troisième et le quatrième étaient autant d’artifices salutaires que permettait la Providence, afin qu’il se formât des philosophes capables de la comprendre et de la reconnaître pour ce qu’elle est, un attribut du vrai Dieu. Nous verrons d’un bout à l’autre de ce livre que tout ce que les poètes avaient d’abord {\itshape senti}  relativement à la {\itshape sagesse vulgaire}, les philosophes le {\itshape comprirent} ensuite relativement à {\itshape une sagesse plus élevée} ({\itshape riposta}) ; de sorte qu’on appellerait avec raison les premiers le {\itshape sens}, les seconds l’{\itshape intelligence} du genre humain. On peut dire de l’espèce ce qu’Aristote dit de l’individu : {\itshape Il n’y a rien dans l’intelligence qui n’ait été auparavant dans le sens} ; c’est-à-dire que l’esprit humain ne comprend rien que les sens ne lui aient donné auparavant occasion de comprendre. L’{\itshape intelligence}, pour remonter au sens étymologique, {\itshape inter legere, intelligere}, l’intelligence agit lorsqu’elle tire de ce qu’on a {\itshape senti} quelque chose qui ne tombe point sous les {\itshape sens}.
\section[{§ II. De la sagesse en général}]{§ II. {\itshape De la sagesse en général}}
\noindent Avant de traiter {\itshape de la sagesse poétique}, il est bon d’examiner en général ce que c’est que {\itshape sagesse}. La sagesse est la faculté qui domine toutes les doctrines relatives aux sciences et aux arts dont se compose l’humanité. Platon définit la sagesse \emph{{\itshape la faculté qui perfectionne l’homme}}. Or l’homme, en tant qu’homme, a deux parties constituantes, l’esprit et le cœur, ou si l’on veut, l’intelligence et la volonté. La sagesse doit développer en lui ces deux puissances à la fois, la seconde par la première, de sorte que l’intelligence étant éclairée par la connaissance des choses les plus sublimes, la volonté fasse choix des choses les meilleures. Les choses les plus sublimes en ce monde, sont les connaissances que l’entendement  et le raisonnement peuvent nous donner relativement à Dieu ; les choses les meilleures sont celles qui concernent le bien de tout le genre humain ; les premières s’appellent divines, les secondes humaines ; la véritable sagesse doit donc donner la connaissance des choses divines pour conduire les choses humaines au plus grand bien possible. Il est à croire que Varron, qui mérita d’être appelé le plus docte des Romains, avait élevé sur cette base son grand ouvrage {\itshape Des choses divines et humaines}, dont l’injure des temps nous a privés. Nous essaierons dans ce livre de traiter le même sujet, autant que nous le permet la faiblesse de nos lumières et le peu d’étendue de nos connaissances.\par
La {\itshape sagesse} commença chez les Gentils par la {\itshape muse}, définie par Homère dans un passage très remarquable de l’{\itshape Odyssée}, \emph{{\itshape la science du bien et du mal}} ; cette science fut ensuite appelée {\itshape divination}, et c’est sur la défense de cette divination, de cette science du bien et du mal refusée à l’homme par la nature, que Dieu fonda la religion des Hébreux, d’où est sortie la nôtre. La {\itshape muse} fut donc proprement dans l’origine la science de la divination et des auspices, laquelle fut la {\itshape sagesse vulgaire} de toutes les nations, comme nous le dirons plus au long ; elle consistait à contempler Dieu dons l’un de ses attributs, dans sa Providence ; aussi, de {\itshape divination}, l’essence de Dieu a-t-elle été appelée {\itshape divinité}. Nous verrons dans la suite que dans ce genre de sagesse, les sages furent les {\itshape poètes théologiens}, qui, à n’en  pas douter, fondèrent la civilisation grecque. Les Latins tirèrent de là l’usage d’appeler {\itshape professeurs de sagesse} ceux qui professaient l’astrologie judiciaire. — Ensuite la {\itshape sagesse} fut attribuée aux hommes célèbres pour avoir donné des avis utiles au genre humain ; tels furent les sept sages de la Grèce. — Plus tard la {\itshape sagesse} passa dans l’opinion aux hommes qui ordonnent et gouvernent sagement les états, dans l’intérêt des nations. — Plus tard encore le mot {\itshape sagesse} vint à signifier la {\itshape science naturelle des choses divines}, c’est-à-dire la métaphysique, qui cherchant à connaître l’intelligence de l’homme par la contemplation de Dieu, doit tenir Dieu pour le régulateur de tout bien, puisqu’elle le reconnaît pour la source de toute vérité\footnote{En conséquence la métaphysique doit essentiellement travailler au bonheur du genre humain dont la conservation tient au sentiment universel qu’ont tous les hommes d’une divinité douce de providence. C’est peut-être pour avoir démontré cette providence que Platon a été surnommé {\itshape le divin}. La philosophie qui enlève à Dieu un tel attribut, mérite moins le nom du philosophie et de sagesse que celui de folie. ({\itshape Vico.})}. — Enfin la {\itshape sagesse} parmi les Hébreux et ensuite parmi les Chrétiens a désigné la {\itshape science des vérités éternelles révélées par Dieu} ; science qui, considérée chez les Toscans comme {\itshape science du vrai bien et du vrai mal}, reçut peut-être pour cette cause son premier nom, {\itshape science de la divinité}.\par
D’après cela, nous distinguerons à plus juste titre que Varron, trois espèces de {\itshape théologie : théologie poétique}, propre aux {\itshape poètes théologiens}, et qui fut la {\itshape théologie civile} de toutes les nations païennes ; {\itshape théologie naturelle}, celle des métaphysiciens ; la troisième, qui dans la classification de Varron est la  théologie poétique\footnote{La théologie {\itshape poétique} fut chez les Gentils la même que la théologie {\itshape civile}. Si Varron la distingue de la théologie {\itshape civile} et de la théologie {\itshape naturelle}, c’est que, partageant l’erreur vulgaire qui place dans les fables les mystères d’une philosophie sublime, il l’a crue mêlée de l’une et de l’autre. ({\itshape Vico.})}, est pour nous la {\itshape théologie chrétienne}, mêlée de la théologie civile, de la naturelle, et de la révélée, la plus sublime des trois. Toutes se réunissent dans la contemplation de la Providence divine ; cette Providence qui conduit la marche de l’humanité, voulut qu’elle partît de la {\itshape théologie poétique} qui réglait les actions des hommes d’après certains signes sensibles, pris pour des avertissements du ciel ; et que la {\itshape théologie naturelle}, qui démontre la Providence par des raisons d’une nature immuable et au-dessus des sens, préparât les hommes à recevoir la {\itshape théologie révélée}, par l’effet d’une foi surnaturelle et supérieure aux sens et à tous les raisonnements.
\section[{§ III. Exposition et division de la sagesse poétique}]{§ III. {\itshape Exposition et division de la sagesse poétique}}
\noindent Puisque la métaphysique est la science sublime qui répartit aux sciences subalternes les sujets dont elles doivent traiter, puisque la sagesse des anciens ne fut autre que celle des {\itshape poètes théologiens}, puisque les origines de toutes choses sont naturellement grossières, {\itshape nous devons chercher le commencement}  {\itshape de la sagesse poétique dans une métaphysique informe}. D’une seule branche de ce tronc sortirent, en se séparant, {\itshape la logique, la morale, l’économie et la politique poétiques} ; d’une autre branche sortit avec le même caractère poétique la {\itshape physique}, mère de la {\itshape cosmographie}, et par suite de l’{\itshape astronomie}, à laquelle la {\itshape chronologie} et la {\itshape géographie}, ses deux filles, doivent leur certitude. Nous ferons voir d’une manière claire et distincte comment les fondateurs de la civilisation païenne, guidés par leur théologie naturelle, ou {\itshape métaphysique}, imaginèrent les dieux ; comment par leur {\itshape logique} ils trouvèrent les langues, par leur {\itshape morale} produisirent les héros, par leur {\itshape économie} fondèrent les familles, par leur {\itshape politique} les cités ; comment par leur {\itshape physique}, ils donnèrent à chaque chose une origine divine, se créèrent eux-mêmes en quelque sorte par leur {\itshape physiologie}, se firent un univers tout de dieux par leur {\itshape cosmographie}, portèrent dans leur {\itshape astronomie} les planètes et les constellations de la terre au ciel, donnèrent commencement à la série des temps dans leur {\itshape chronologie}, enfin dans leur {\itshape géographie} placèrent tout le monde dans leur pays (les Grecs dans la Grèce, et de même des autres peuples). Ainsi la Science nouvelle pourra devenir une histoire des idées, coutumes et actions du genre humain. De cette triple source nous verrons sortir les principes de l’{\itshape histoire de la nature humaine}, principes identiques avec ceux de l’{\itshape histoire universelle} qui semblent manquer jusqu’ici.
\chapterclose


\chapteropen
\chapter[{Chapitre II. De la métaphysique poétique}]{Chapitre II. \\
De la métaphysique poétique}

\chaptercont
\section[{§ I. Origine de la poésie, de l’idolâtrie, de la divination et des sacrifices}]{§ I. {\itshape Origine de la poésie, de l’idolâtrie, de la divination et des sacrifices}}
\noindent  [L’auteur établit d’abord la certitude du déluge universel, et de l’existence des géants. Les preuves les plus fortes qu’il allègue ont été déjà énoncées dans les axiomes 25, 26, 27. {\itshape Voyez} aussi le Discours préliminaire.]\par
C’est dans l’état de stupidité farouche où se trouvèrent les premiers hommes, que tous les philosophes et les philologues devaient prendre leur point de départ pour raisonner sur la sagesse des Gentils. Ils devaient interroger d’abord la science qui cherche ses preuves, non pas dans le monde extérieur, mais dans l’âme de celui qui la médite, je veux dire, la métaphysique. Ce monde social étant indubitablement l’ouvrage des hommes, on pouvait en lire les principes dans les modifications de l’esprit humain.\par
 La {\itshape sagesse poétique}, la première sagesse du paganisme, dut commencer par une métaphysique, non point de raisonnement et d’abstraction, comme celle des esprits cultivés de nos jours, mais de sentiment et d’imagination, telle que pouvaient la concevoir ces premiers hommes, qui n’étaient que sens et imagination sans raisonnement. La métaphysique dont je parle, c’était leur {\itshape poésie}, faculté qui naissait avec eux. L’{\itshape ignorance est mère de l’admiration} ; ignorant tout, ils admiraient vivement. Cette poésie fut d’abord {\itshape divine} : ils rapportaient à des dieux la cause de ce qu’ils admiraient. Voyez le passage de Lactance (axiome 38). \emph{{\itshape Les anciens Germains}, dit Tacite, {\itshape  entendaient la nuit le soleil qui passait sous la mer d’occident en orient ; ils affirmaient aussi qu’ils voyaient les dieux.}} Maintenant encore les sauvages de l’Amérique divinisent tout ce qui est au-delà de leur faible capacité. Quelles que soient la simplicité et la grossièreté de ces nations, nous devons présumer que celles des premiers hommes du paganisme allaient bien au-delà. Ils donnaient aux objets de leur admiration une existence analogue à leurs propres idées. C’est ce que font précisément les enfants (axiome 37), lorsqu’ils prennent dans leurs jeux des choses inanimées et qu’ils leur parlent comme à des personnes vivantes. Ainsi ces premiers hommes, qui nous représentent l’enfance du genre humain, créaient eux-mêmes les choses d’après leurs idées. Mais cette création différait infiniment de celle de Dieu : Dieu dans sa pure  intelligence connaît les êtres, et les crée par cela même qu’il les connaît ; les premiers hommes, puissants de leur ignorance, créaient à leur manière par la force d’une imagination, si je puis dire, toute {\itshape matérielle}. Plus elle était matérielle, plus ses créations furent sublimes ; elles l’étaient au point de troubler à l’excès l’esprit même d’où elles étaient sorties. Aussi les premiers hommes furent appelés {\itshape poètes}, c’est-à-dire, {\itshape créateurs}, dans le sens étymologique du mot grec. Leurs créations réunirent les trois caractères qui distinguent la haute poésie dans l’invention des fables, la sublimité, la popularité, et la puissance d’émotion qui la rend plus capable d’atteindre le but qu’elle se propose, celui l’{\itshape enseigner au vulgaire à agir selon la vertu}. — De cette faculté originaire de l’esprit humain, il est resté une loi éternelle : les esprits une fois frappés de terreur, \emph{{\itshape fingunt simul credunt que}}, comme le dit si bien Tacite.\par
Tels durent se trouver les fondateurs de la civilisation païenne, lorsqu’un siècle ou deux après le déluge, la terre desséchée forma de nouveaux orages, et que la foudre se fit entendre. Alors sans doute un petit nombre de géants dispersés dans les bois, vers le sommet des montagnes, furent épouvantés par ce phénomène dont ils ignoraient la cause, levèrent les yeux, et remarquèrent le ciel pour la première fois. Or, comme en pareille circonstance, il est dans la nature de l’esprit humain d’attribuer au phénomène qui le frappe, ce qu’il  trouve en lui-même, ces premiers hommes, dont toute l’existence était alors dans l’énergie des forces corporelles, et qui exprimaient la violence extrême de leurs passions par des murmures et des hurlements, se figurèrent le ciel comme un grand corps animé, et l’appelèrent Jupiter\footnote{Avec l’idée d’un Jupiter, auquel ils attribuèrent bientôt une Providence, naquit le droit, {\itshape jus}, appelé {\itshape ious} par les Latins, et par les anciens Grecs Δίαιον, {\itshape céleste}, du mot \foreign{Διός} ; les Latins dirent également {\itshape sub dio}, et sub jove pour exprimer {\itshape sous le ciel}. Puis, si l’on en croit Platon dans son Cratyle, on substitua par euphonie Δίχαιον. Ainsi toutes les nations païennes ont contemplé le ciel, qu’elles considéraient comme Jupiter, pour en recevoir par les auspices des lois, des avis divins ; ce qui prouve que le principe commun des sociétés a été la {\itshape croyance à une Providence divine.} Et pour en commencer l’énumération, {\itshape Jupiter} fut le {\itshape ciel} chez les Chaldéens, en ce sens qu’ils croyaient recevoir de lui la connaissance de l’avenir par l’observation des aspects divers et des mouvements des étoiles, et on nomma {\itshape astronomie} et {\itshape astrologie} la science des lois qu’observent les astres, et celle de leur langage ; la dernière fut prise dans le sens d’astrologie judiciaire, et dans les lois romaines {\itshape Chaldéen} veut dire astrologue. — Chez les Perses, {\itshape Jupiter} fut le {\itshape ciel}, qui faisait connaître aux hommes les choses cachées ; ceux qui possédaient cette science s’appelaient Mages, et tenaient dans leurs rites une verge qui répond au bâton augural des Romains. Ils s’en servaient pour tracer des cercles astronomiques, comme depuis les magiciens dans leurs enchantemens. Le ciel était pour les Perses le temple de Jupiter, et leurs rois, imbus de cette opinion, détruisaient les temples construits par les Grecs. — Les Égyptiens confondaient aussi {\itshape Jupiter} et le ciel, sous le rapport de l’influence qu’il avait sur les choses sublunaires et des moyens qu’il donnait de connaître l’avenir ; de nos jours encore ils conservent une divination vulgaire. — Même opinion chez les Grecs qui tiraient du ciel des θεωρήματα et des μαθήματα, en les contemplant des yeux du corps, et en les observant, c’est-à-dire, en leur obéissant comme aux lois de Jupiter. C’est du mot μαθηματα, que les astrologues sont appelés {\itshape mathématiciens} dans les lois romaines. — Quant à la croyance des Romains, on connaît le vers d’Ennius, \emph{{\itshape Aspice hoc sublime cadens, quem omnes invocant jovem}} ; le pronom {\itshape hoc} est pris dans le sens de {\itshape cœlum}. Les Romains disaient aussi {\itshape templa cœli}, pour exprimer la région du ciel désigné par les augures pour prendre les auspices ; et par dérivation, {\itshape templum} signifia tout lieu découvert où la vue ne rencontre point d’obstacle (\emph{{\itshape neptunia templa}}, la mer dans Virgile). — Les anciens Germains, selon Tacite, adoraient leurs Dieux dans des lieux sacrés qu’il appelle \emph{{\itshape lucos et nemora}}, ce qui indique sans doute des clairières dans l’épaisseur des bois. L’église eut beaucoup de peine à leur faire abandonner cet usage (V. {\itshape Concilia Stanctense et Bracharense}, dans le recueil de Bouchard). On en trouve encore aujourd’hui des traces chez les Lapons et chez les Livoniens. — Les Perses disaient simplement le {\itshape Sublime} pour désigner {\itshape Dieu}. Leurs temples n’étaient que des collines découvertes où l’on montait de deux côtés par d’immenses escaliers ; c’est dans la hauteur de ces collines qu’ils faisaient consister leur magnificence. Tous les peuples placent la beauté des temples dans leur élévation prodigieuse. Le point le plus élevé s’appelait, selon Pausanias, αετος l’aigle, l’oiseau des auspices, celui dont le vol est le plus élevé. De là peut être {\itshape pinnæ templorum, pinnæ murorum}, et en dernier lieu, {\itshape aquilæ} pour les créneaux. Les Hébreux adoraient dans le tabernacle {\itshape le Très-Haut} qui est au-dessus des cieux ; et partout où le peuple de Dieu étendait ses conquêtes, Moïse ordonnait que l’on brûlât les bois sacrés, sanctuaires de l’idolâtrie. — Chez les chrétiens mêmes, plusieurs nations disent le {\itshape ciel} pour {\itshape Dieu}. Les Français et les Italiens disent {\itshape fasse le ciel, j’espère dans les secours du ciel} ; il en est de même en espagnol. Les français disent {\itshape bleu} pour {\itshape le ciel}, dans une espèce de serment {\itshape par bleu}, et dans ce blasphème impie {\itshape morbleu} (c’est-à-dire {\itshape meure le ciel}, en prenant ce mot dans le sens de {\itshape Dieu}.) Nous venons de donner un essai du vocabulaire dont on a parlé dans les axiomes 13 et 22. ({\itshape Vico.})}. Ils présumèrent que par le fracas du tonnerre, par les éclats de la foudre, Jupiter {\itshape voulait leur dire quelque chose} ; et ils  commencèrent à se livrer à la {\itshape curiosité, fille de l’ignorance et mère de la science} [qu’elle produit, lorsque l’admiration a ouvert l’esprit de l’homme]. Ce caractère est toujours le même dans le vulgaire ; voient-ils une comète, un parélie, ou tout autre  phénomène céleste, ils s’inquiètent et demandent {\itshape ce qu’il signifie} (axiome 39). Observent-ils les effets étonnants de l’aimant mis en contact avec le fer ; ils ne manquent pas, même dans ce siècle de lumières, de décider que l’aimant a pour le fer une sympathie mystérieuse, et ils font ainsi de toute la nature un vaste corps animé, qui a ses sentiments et ses passions. Mais, à une époque si avancée de la civilisation, les esprits, même du vulgaire, sont trop détachés des sens, trop spiritualisés par les nombreuses abstractions de nos langues, par l’art de l’écriture, par l’habitude du calcul, pour que nous puissions nous former cette image prodigieuse de la {\itshape nature passionnée} ; nous disons bien ce mot de la bouche, mais nous n’avons rien dans l’esprit. Comment pourrions-nous nous replacer dans la vaste imagination de ces premiers hommes dont l’esprit étranger à toute abstraction, à toute subtilité, était tout {\itshape émoussé} par les passions, {\itshape plongé} dans les sens, et comme {\itshape enseveli} dans la matière. Aussi, nous l’avons déjà dit, on {\itshape comprend} à peine aujourd’hui, mais on ne peut {\itshape imaginer} comment pensaient les premiers hommes qui fondèrent la civilisation païenne.\par
\par
C’est ainsi que les premiers {\itshape poètes théologiens} inventèrent la première fable {\itshape divine}, la plus sublime de toutes celles qu’on imagina ; c’est ce Jupiter {\itshape roi et père des hommes et des dieux}, dont la main lance la foudre ; image si populaire, si capable  d’émouvoir les esprits, et d’exercer sur eux une influence morale, que les inventeurs eux-mêmes crurent à sa réalité, la redoutèrent et l’honorèrent avec des rites affreux. Par un effet de ce caractère de l’esprit humain que nous avons remarqué d’après Tacite (\emph{{\itshape mobiles ad superstitionem perculsæ semel mentes}}, axiome 23), dans tout ce qu’ils apercevaient, imaginaient, ou faisaient eux-mêmes, ils ne virent que Jupiter, animant ainsi l’univers dans toute l’étendue qu’ils pouvaient concevoir. C’est ainsi qu’il faut entendre dans l’histoire de la civilisation le {\itshape Jovis omnia plena} ; c’est ce Jupiter que Platon prit pour l’éther, qui pénètre et remplit toutes choses ; mais les premiers hommes ne plaçaient pas leur Jupiter plus haut que la cime des montagnes, comme nous le verrons bientôt.\par
Comme ils parlaient par signes, ils crurent d’après leur propre nature que le tonnerre et la foudre étaient les signes de Jupiter. C’est de {\itshape nuere}, faire signe, que la volonté divine fut plus tard appelée {\itshape numen} ; Jupiter commandait par signes, idée sublime, digne expression de la majesté divine. Ces signes étaient, si je l’ose dire, des {\itshape paroles réelles}, et la nature entière était la langue de Jupiter. Toutes les nations païennes crurent posséder cette langue dans la divination, laquelle fut appelée par les Grecs {\itshape théologie}, c’est-à-dire, {\itshape science du langage des dieux}. Ainsi Jupiter acquit ce {\itshape regnum fulminis}, par lequel il est {\itshape le roi des hommes et des dieux}. Il reçut alors deux titres, {\itshape optimus} dans le sens de  très fort (de même que chez les anciens latins, {\itshape fortis} eut le même sens que {\itshape bonus} dans des temps plus modernes) ; et {\itshape maximus}, d’après l’étendue de son corps, aussi vaste que le ciel.\par
De là tant de Jupiters dont le nombre étonne les philologues ; chaque nation païenne eut le sien.\par
Originairement Jupiter fut en poésie un {\itshape caractère divin}, un {\itshape genre créé par l’imagination} plutôt que par l’intelligence ({\itshape universale fantastico}), auquel tous les peuples païens rapportaient les choses relatives aux auspices. Ces peuples, durent être tous poètes, puisque la {\itshape sagesse poétique} commença par cette {\itshape métaphysique poétique} qui contemple Dieu dans l’attribut de sa Providence, et les premiers hommes s’appelèrent {\itshape poètes théologiens}, c’est-à-dire {\itshape sages qui entendent le langage des dieux}, exprimé par les auspices de Jupiter. Ils furent surnommés {\itshape divins}, dans le sens du mot {\itshape devins}, qui vient de {\itshape divinari}, deviner, prédire. Cette science fut appelée {\itshape muse}, expression qu’Homère nous définit par {\itshape la science du bien et du mal}, qui n’est autre que la {\itshape divination}\footnote{La défense de la divination faite par Dieu à son peuple fut le fondement de la véritable religion. ({\itshape Vico.})}. C’est encore, d’après cette {\itshape théologie mystique} que les poètes furent appelés par les Grecs, μύσται, [qu’Horace traduit fort bien par \emph{{\itshape les interprètes des dieux}}], lesquels expliquaient les divins mystères des auspices et des oracles. Toute nation païenne eut une sibylle qui possédait cette science ; on en a compté jusqu’à douze. Les sibylles  et les oracles sont les choses les plus anciennes dont nous parle le paganisme.\par
\par
Tout ce qui vient d’être dit s’accorde donc avec le mot célèbre,\par

… La crainte seule a fait les premiers dieux ;\\

\noindent mais les hommes ne s’inspirèrent pas cette crainte les uns aux autres ; ils la durent à leur propre imagination (ce qui répond à l’axiome : {\itshape les fausses religions sont nées de la crédulité et non de l’imposture}). Cette origine de l’{\itshape idolâtrie} étant démontrée, celle de la {\itshape divination} l’est aussi ; ces deux sœurs naquirent en même temps. Les {\itshape sacrifices} en furent une conséquence immédiate, puisqu’on les faisait pour {\itshape procurare} (c’est-à-dire pour bien entendre) les auspices.\par
Ce qui nous prouve que la poésie a dû naître ainsi, c’est ce caractère éternel et singulier qui lui est propre : {\itshape le sujet propre à la poésie c’est l’impossible, et pourtant le croyable} ({\itshape impossibile credibile}). Il est impossible que la matière soit esprit, et pourtant l’on a cru que le ciel, d’où semblait partir la foudre, était Jupiter. Voilà encore pourquoi les poètes aiment tant à chanter les prodiges opérés par les magiciennes dans leurs enchantements ; cette disposition d’esprit peut être rapportée au sentiment instinctif de la toute-puissance de Dieu, qu’ont en eux les hommes de toutes les nations.\par
Les vérités que nous venons d’établir renversent tout ce qui a été dit sur l’{\itshape origine de la poésie}, depuis  Aristote et Platon jusqu’aux Scaliger et aux Castelvetro. Nous l’avons montré, c’est par un effet de la {\itshape faiblesse du raisonnement} de l’homme, que la poésie s’est trouvée si sublime à sa naissance, et qu’avec tous les secours de la philosophie, de la poétique et de la critique, qui sont venues plus tard, on n’a jamais pu, je ne dirai point surpasser, mais égaler son premier essor\footnote{Voilà pourquoi Homère se trouve le premier de tous les poètes du genre {\itshape héroïque}, le plus sublime de tous, dans l’ordre du mérite comme dans celui du temps. ({\itshape Vico.})}. Cette découverte de l’origine de la poésie détruit le préjugé commun sur la profondeur de la sagesse antique, à laquelle les modernes devraient désespérer d’atteindre, et dont tous les philosophes depuis Platon jusqu’à Bacon ont tant souhaité de pénétrer le secret. Elle n’a été autre chose qu’une {\itshape sagesse vulgaire de législateurs} qui fondaient l’ordre social, et non point une {\itshape sagesse mystérieuse sortie du génie de philosophes profonds}. Aussi, comme on le voit déjà par l’exemple tiré de Jupiter, tous les {\itshape sens mystiques d’une haute philosophie} attribués par les savants aux fables grecques et aux hiéroglyphes égyptiens, paraîtront aussi choquants que le {\itshape sens historique} se trouvera facile et naturel.
\section[{§ II. Corollaires relatifs aux principaux aspects de la science nouvelle}]{§ II. Corollaires relatifs aux principaux aspects de la science nouvelle}
\noindent  1. On peut conclure de tout ce qui précède que, conformément au premier principe de la Science nouvelle, développé dans le chapitre {\itshape de la Méthode} ({\itshape l’homme n’espérant plus aucun secours de la nature, appelle de ses désirs quelque chose de surnaturel qui puisse le sauver}), la Providence permit que les premiers hommes tombassent dans l’erreur de craindre une fausse divinité, un Jupiter auquel ils attribuaient le pouvoir de les foudroyer. Au milieu des nuées de ces premiers orages, à la lueur de ces éclairs, ils aperçurent cette grande vérité, {\itshape que la Providence veille à la conservation du genre humain}. Aussi, sous un de ses principaux aspects, la Science nouvelle est d’abord une {\itshape théologie civile}, une explication raisonnée de la marche suivie par la Providence ; et cette théologie commença par la sagesse {\itshape vulgaire} des législateurs qui fondèrent les sociétés, en prenant pour base la croyance d’un Dieu doué de providence ; elle s’acheva par la sagesse plus élevée ({\itshape riposta}) des philosophes qui démontrent la même vérité par des raisonnements, dans leur théologie naturelle.\par
2. Un autre aspect principal de la science nouvelle, c’est une {\itshape philosophie de la propriété} (ou {\itshape autorité}  dans le sens primitif où les douze tables prennent ce mot\footnote{On continua à appeler dans le droit, {\itshape nos auteurs}, ceux dont nous tenons un droit à une propriété. ({\itshape Vico.})}). La première propriété fut {\itshape divine} : Dieu s’appropria les premiers hommes peu nombreux, qu’il tira de la vie sauvage pour commencer la vie sociale. — La seconde propriété fut {\itshape humaine}, et dans le sens le plus exact ; elle consista pour l’homme dans la possession de ce qu’on ne peut lui ôter sans l’anéantir, dans le libre {\itshape usage de sa volonté}. Pour l’intelligence, ce n’est qu’une puissance passive sujette à la vérité. Les hommes commencèrent, dès ce moment, à exercer leur liberté en réprimant les impulsions passionnées du corps, de manière à les étouffer ou à les mieux diriger, effort qui caractérise les agents libres. Le premier acte libre des hommes fut d’abandonner la vie vagabonde qu’ils menaient dans la vaste forêt qui couvrait la terre, et de s’accoutumer à une vie sédentaire, si opposée à leurs habitudes. — Le troisième genre de propriété fut celle {\itshape de droit naturel}. Les premiers hommes qui abandonnaient la vie vagabonde occupèrent des terres et y restèrent longtemps ; ils en devinrent seigneurs par droit d’occupation et de longue possession. C’est l’origine de tous les {\itshape domaines}.\par
Cette {\itshape philosophie de la propriété} suit naturellement la {\itshape théologie civile} dont nous parlions. Éclairée par les preuves que lui fournit la théologie civile, elle éclaire elle-même avec celles qui lui sont propres, les preuves que la philologie tire de l’histoire  et des langues ; trois sortes de preuves qui ont été énumérées dans le chapitre de la méthode. Introduisant la certitude dans le domaine de la liberté humaine, dont l’étude est si incertaine de sa nature, elle éclaire les ténèbres de l’antiquité, et {\itshape donne forme de science à la philologie}.\par
3. Le troisième aspect est une {\itshape histoire des idées humaines}. De même que la {\itshape métaphysique poétique} s’est divisée en plusieurs sciences subalternes, {\itshape poétiques} comme leur mère, cette histoire des idées nous donnera l’origine informe des sciences pratiques cultivées par les nations, et des sciences spéculatives étudiées de nos jours par les savants.\par
4. Le quatrième aspect est une {\itshape critique philosophique} qui naît de l’histoire des idées mentionnée ci-dessus. Cette critique cherche ce que l’on doit croire sur les fondateurs, ou auteurs des nations, lesquels doivent précéder de plus de mille ans les auteurs de livres, qui est l’objet de la critique philologique.\par
5. Le cinquième aspect est une {\itshape histoire idéale éternelle} dans laquelle tournent les histoires réelles de toutes les nations. De quelque état de barbarie et de férocité que partent les hommes pour se civiliser par l’influence des religions, les sociétés commencent, se développent et finissent d’après des lois que nous examinerons dans ce second livre, et que nous retrouverons au livre IV où nous suivons {\itshape la marche des sociétés}, et au livre V où nous observons le {\itshape retour des choses humaines}.\par
 6. Le sixième aspect est un système du {\itshape droit naturel des gens}. C’était avec le commencement des peuples, que Grotius, Selden et Pufendorf devaient commencer leurs systèmes (axiome 106 : {\itshape les sciences doivent prendre pour point de départ l’époque où commence le sujet dont elles traitent}). Ils se sont égarés tous trois, parce qu’ils ne sont partis que du milieu de la route. Je veux dire qu’ils supposent d’abord un état de civilisation où les hommes seraient déjà éclairés par une {\itshape raison développée}, état dans lequel les nations ont produit les philosophes qui se sont élevés jusqu’à l’idéal de la justice. En premier lieu, Grotius procède indépendamment du principe d’une Providence, et prétend que son système donne un degré nouveau de précision à toute connaissance de Dieu. Aussi toutes ses attaques contre les jurisconsultes romains portent à faux, puisqu’ils ont pris pour principe la Providence divine, et qu’ils ont voulu traiter du {\itshape droit naturel des gens}, et non point du droit naturel des philosophes, et des théologiens moralistes. — Ensuite vient Selden, dont le système suppose la Providence. Il prétend que le droit des enfants de Dieu s’étendit à toutes les nations, sans faire attention au caractère inhospitalier des premiers peuples, ni à la division établie entre les Hébreux et les Gentils ; sans observer que les Hébreux ayant perdu de vue leur droit naturel dans la servitude d’Égypte, il fallut que Dieu lui-même le leur rappelât en leur donnant sa loi sur le mont Sinaï. Il oublie que Dieu,  dans sa loi, défend jusqu’aux pensées injustes, chose dont ne s’embarrassèrent jamais les législateurs mortels. Comment peut-il prouver que les Hébreux ont transmis aux Gentils leur droit naturel, contre l’aveu magnanime de Josèphe, contre la réflexion de Lactance cité plus haut ? Ne connaît-on pas enfin la haine des Hébreux contre les Gentils, haine qu’ils conservent encore aujourd’hui dans leur dispersion ? — Quant à Pufendorf, il commence son système par {\itshape jeter l’homme dans le monde, sans soin ni secours de Dieu}. En vain il essaie d’excuser dans une dissertation particulière cette hypothèse épicurienne. Il ne peut pas dire le premier mot en fait de droit, sans prendre la Providence pour principe\footnote{\noindent {\itshape Nous rapprocherons de ce passage celui qui y correspond dans la première édition} :\par
\noindent « Grotius prétend que son système peut se passer de l’idée de la Providence. Cependant sans religion les hommes ne seraient pas réunis en nations… Point de physique sans mathématique ; point de morale ni de politique sans métaphysique, c’est-à-dire sans démonstration de Dieu. — Il suppose le premier homme bon, parce qu’il n’était {\itshape pas mauvais}. Il compose le genre humain à sa naissance d’hommes {\itshape simples et débonnaires}, qui auraient été poussés par l’intérêt à la vie sociale ; c’est dans le fait l’hypothèse d’Épicure.Puis vient Selden, qui appuie son système sur le petit nombre de lois que Dieu dicta aux enfants de Noé. Mais Sem fut le seul qui persévéra dans la religion du Dieu d’Adam. Loin de fonder un droit commun à ses descendants et à ceux de Cham et de Japhet, on pourrait dire plutôt qu’il fonda un droit exclusif, qui fit plus tard distinguer les Juifs des Gentils…\par
« Pufendorf, en jetant l’homme dans le monde {\itshape sans secours de la Providence}, hasarde une hypothèse digne d’Épicure, ou plutôt de Hobbes…\par
« Écartant ainsi la Providence, ils ne pouvaient découvrir les sources de tout ce qui a rapport à l’économie du droit naturel des gens, ni celles des religions, des langues et des lois, ni celles de la paix et de la guerre, des traités, etc. De là deux erreurs capitales.\par
« 1. D’abord ils croient que leur droit naturel, fondé sur les théories des philosophes, des théologiens, et sur quelques-unes de celles des jurisconsultes, et qui est éternel dans son idée abstraite, a dû être aussi éternel dans l’usage et dans la pratique des nations. Les jurisconsultes romains raisonnent mieux en considérant ce droit naturel comme ordonné par la Providence, et comme éternel en ce sens, que sorti des mêmes origines que les religions, il passe comme elles par différens âges, jusqu’à ce que les philosophes viennent le perfectionner et le compléter par des théories fondées sur l’idée de la justice éternelle.\par
« 2. Leurs systèmes n’embrassent pas la moitié du droit naturel des gens. Ils parlent de celui qui regarde la conservation du genre humain, et ils ne disent rien de celui qui a rapport à la conservation des peuples en particulier. Cependant c’est le droit naturel établi séparément dans chaque cité qui a préparé les peuples à reconnaître, dès leurs premières communications, le sens commun qui les unit, de sorte qu’ils donnassent et redussent des lois conformes à toute la nature humaine, et les respectassent comme dictées par la Providence. » ({\itshape Vico.})
}. — Pour nous, persuadés que l’idée  du droit et l’idée d’une {\itshape Providence} naquirent en même temps, nous commençons à parler du {\itshape droit} en parlant de ce moment où les premiers auteurs des nations conçurent l’idée de Jupiter. Ce droit fut d’abord {\itshape divin}, dans ce sens qu’il était interprété par la {\itshape divination}, science des auspices de Jupiter ; les auspices furent les {\itshape choses divines}, au moyen desquelles les nations païennes réglaient toutes les {\itshape choses humaines}, et la réunion des unes et des autres forme le sujet de la jurisprudence.\par
7. Considérée sous le dernier de ses principaux aspects, la Science nouvelle nous donnera les {\itshape principes et les origines de l’histoire universelle}, en partant de l’âge appelé par les Égyptiens {\itshape âge des Dieux},  par les Grecs, {\itshape âge d’or}. Faute de connaître la {\itshape chronologie raisonnée de l’histoire poétique}, on n’a pu saisir jusqu’ici l’enchaînement de toute l’{\itshape histoire du monde païen}.
\chapterclose


\chapteropen
\chapter[{Chapitre III. De la logique poétique}]{Chapitre III. \\
De la logique poétique}

\chaptercont
\section[{§ I}]{§ I}
\noindent  La {\itshape métaphysique}, ainsi nommée lorsqu’elle contemple les choses dans tous les genres de l’être, devient {\itshape logique} lorsqu’elle les considère dans tous les genres d’expressions par lesquelles on les désigne ; de même la poésie a été considérée par nous comme une {\itshape métaphysique poétique}, dans laquelle les poètes théologiens prirent la plupart des choses matérielles pour des êtres divins ; la même poésie, occupée maintenant d’exprimer l’idée de ces divinités, sera considérée comme une {\itshape logique poétique}.\par
{\itshape Logique} vient de λόγος. Ce mot, dans son premier sens, dans son sens propre, signifia {\itshape fable} (qui a passé dans l’italien {\itshape favella}, langage, discours) ; la fable, chez les Grecs, se dit aussi μῦθος, d’où les latins tirèrent le mot {\itshape mutus} ; en effet, dans les {\itshape temps muets}, le discours fut {\itshape mental} ; aussi λόγος signifie {\itshape idée}  et {\itshape parole}. Une telle langue convenait à des âges religieux ({\itshape les religions veulent être révérées en silence, et non pas raisonnées}). Elle dut commencer par des signes, des gestes, des indications matérielles dans un rapport naturel avec les idées : aussi λόγος, {\itshape parole}, eut en outre chez les Hébreux le sens d’{\itshape action}, chez les Grecs celui de {\itshape chose}. Μῦθος a été aussi défini un {\itshape récit véritable}, un {\itshape langage véritable}\footnote{{\itshape C’est cette langue naturelle que les hommes ont parlée autrefois}, selon Platon et Jamblique. Platon a deviné plutôt que découvert cette vérité. De là l’inutilité de ses recherches dans le {\itshape Cratyle}, de là les attaques d’Aristote et de Gallen. ({\itshape Vico.})}. Par {\itshape véritable}, il ne faut pas entendre ici {\itshape conforme à la nature des choses}, comme dut l’être la {\itshape langue sainte}, enseignée à Adam par Dieu même.\par
La première langue que les hommes se firent eux-mêmes fut toute d’imagination, et eut pour signes les substances même qu’elle animait, et que le plus souvent elle divinisait. Ainsi Jupiter, Cybèle, Neptune, étaient simplement le ciel, la terre, la mer, que les premiers hommes, muets encore, exprimaient en les montrant du doigt, et qu’ils imaginaient comme des êtres animés, comme des dieux ; avec les noms de ces trois divinités, ils exprimaient toutes les choses relatives au ciel, à la terre, à la mer. Il en était de même des autres dieux : ils rapportaient toutes les fleurs à Flore, tous les fruits à Pomone.\par
Nous suivons encore une marche analogue à celle de ces premiers hommes, mais c’est à l’égard  des choses intellectuelles, telles que les facultés de l’âme, les passions, les vertus, les vices, les sciences, les arts ; nous nous en formons ordinairement l’idée comme d’autant de {\itshape femmes} (la justice, la poésie, etc.), et nous ramenons à ces êtres fantastiques toutes les causes, toutes les propriétés, tous les effets des choses qu’ils désignent. C’est que nous ne pouvons exposer au-dehors les choses intellectuelles contenues dans notre entendement, sans être secondés par l’imagination, qui nous aide à les expliquer et à les peindre sous une image humaine. Les premiers hommes (les {\itshape poètes théologiens}), encore incapables d’abstraire, firent une chose toute contraire, mais plus sublime : ils donnèrent des sentiments et des passions aux êtres matériels, et même aux plus étendus de ces êtres, au ciel, à la terre, à la mer. Plus tard, la puissance d’abstraire se fortifiant, ces vastes imaginations se resserrèrent, et les mêmes objets furent désignés par les signes les plus petits ; Jupiter, Neptune et Cybèle devinrent si petits, si légers, que le premier vola sur les ailes d’un aigle, le second courut sur la mer porté dans un mince coquillage, et la troisième fut assise sur un lion.\par
Les formes mythologiques ({\itshape mitologie}) doivent donc être, comme le mot l’indique, le {\itshape langage propre des fables} ; les fables étant autant de genres dans la langue de l’imagination ({\itshape generi fantastici}), les formes mythologiques sont des {\itshape allégories} qui y répondent. Chacune comprend sous elle plusieurs espèces  ou plusieurs individus. Achille est l’idée de la valeur, commune à tous les vaillants ; Ulysse, l’idée de la prudence commune à tous les sages.
\section[{§ II. Corollaires relatifs aux tropes, aux métamorphoses poétiques et aux monstres des poètes}]{§ II. {\itshape Corollaires relatifs aux tropes, aux métamorphoses poétiques et aux monstres des poètes}}
\noindent 1. Tous les premiers tropes sont autant de corollaires de cette logique poétique. Le plus brillant, et pour cela même le plus fréquent et le plus nécessaire, c’est la métaphore. Jamais elle n’est plus approuvée que lorsqu’elle prête du sentiment et de la passion aux choses insensibles, en vertu de cette métaphysique par laquelle les premiers poètes animèrent les corps sans vie, et les douèrent de tout ce qu’ils avaient eux-mêmes, de sentiment et de passion ; si les premières fables furent ainsi créées, toute métaphore est l’abrégé d’une fable. — Ceci nous donne un moyen de juger du temps où les métaphores furent introduites dans les langues. Toutes les métaphores tirées par analogie des objets corporels pour signifier des abstractions, doivent dater de l’époque où le jour de la philosophie a commencé à luire ; ce qui le prouve, c’est qu’en toute langue les mots nécessaires aux arts de la civilisation, aux sciences les plus sublimes, ont des origines agrestes. Il est digne d’observation que, dans toutes les langues, la plus grande partie des expressions relatives  aux choses inanimées sont tirées par métaphore, du corps humain et de ses parties, ou des sentiments et passions humaines. Ainsi {\itshape tête}, pour cime, ou commencement, {\itshape bouche} pour toute ouverture, {\itshape dents} d’une charrue, d’un râteau, d’une scie, d’un peigne, {\itshape langue} de terre, {\itshape gorge} d’une montagne, une {\itshape poignée} pour un petit nombre, {\itshape bras} d’un fleuve, {\itshape cœur} pour le milieu, {\itshape veine} d’une mine, {\itshape entrailles} de la terre, {\itshape côte} de la mer, {\itshape chair} d’un fruit ; le vent {\itshape siffle}, l’onde {\itshape murmure}, un corps {\itshape gémit} sous un grand poids. Les latins disaient {\itshape sitire agros, laborare fructus, luxuriari segetes} ; et les Italiens disent {\itshape andar in amore le piente, andar in pazzia le viti, lagrimare gli orni}, et {\itshape fronte, spalle, occhi, barbe, collo, gamba, piede, pianta}, appliqués à des choses inanimées. On pourrait tirer d’innombrables exemples de toutes les langues. Nous avons dit dans les axiomes, que l’{\itshape homme ignorant se prenait lui-même pour règle de l’univers} ; dans les exemples cités ci-dessus, il se fait de lui-même un univers entier. De même que la métaphysique de la raison nous enseigne que {\itshape par l’intelligence l’homme devient tous les objets} (\emph{{\itshape homo intelligendo fit omnia}}), la métaphysique de l’imagination nous démontre ici que l’{\itshape homme devient tous les objets faute d’intelligence} (\emph{{\itshape homo non intelligendo fit omnia}}) ; et peut-être le second axiome est-il plus vrai que le premier, puisque l’homme, dans l’exercice de l’intelligence, étend son esprit pour saisir les objets, et que, dans la privation de l’intelligence, il fait tous les objets de lui-même, et par cette  transformation devient à lui seul toute la nature.\par
2. Dans une telle logique, résultant elle-même d’une telle métaphysique, les premiers poètes devaient tirer les noms des choses d’{\itshape idées sensibles et plus particulières} ; voilà les deux sources de la métonymie et de la {\itshape synecdoque}. En effet, la métonymie du {\itshape nom de l’auteur pris pour celui de l’ouvrage}, vint de ce que l’auteur était plus souvent nommé que l’ouvrage ; celle {\itshape du sujet pris pour sa forme et ses accidents} vint de l’incapacité d’abstraire du sujet les accidents et la forme. Celles de {\itshape la cause pour l’effet} sont autant de petites fables ; les hommes s’imaginèrent les causes comme des {\itshape femmes} qu’ils revêtaient de leurs effets : ainsi l’{\itshape affreuse pauvreté}, la {\itshape triste vieillesse}, la {\itshape pâle mort}.\par
3. La {\itshape synecdoque} fut employée ensuite, à mesure que l’on s’éleva des particularités aux généralités, ou que l’on réunit les parties pour composer leurs entiers. Le nom de {\itshape mortel} fut d’abord réservé aux {\itshape hommes}, seuls êtres dont la condition mortelle dût se faire remarquer. Le mot {\itshape tête} fut pris pour l’{\itshape homme}, dont elle est la partie la plus capable de frapper l’attention. {\itshape Homme} est une abstraction qui comprend génériquement le corps et toutes ses parties, l’intelligence et toutes les facultés intellectuelles, le cœur et toutes les habitudes morales. Il était naturel que dans l’origine {\itshape tignum} et {\itshape culmen} signifiassent au propre une {\itshape poutre} et de la {\itshape paille} ; plus tard, lorsque les cités s’embellirent, ces mots signifièrent tout l’édifice. De même le {\itshape toit} pour la maison entière,  parce qu’aux premiers temps on se contentait d’un abri pour toute habitation. Ainsi {\itshape puppis}, la poupe, pour le vaisseau, parce que cette partie la plus élevée du vaisseau est la première qu’on voit du rivage ; et chez les modernes on a dit une {\itshape voile}, pour un {\itshape vaisseau. Mucro}, la {\itshape pointe}, pour l’{\itshape épée} ; ce dernier mot est abstrait et comprend génériquement la pomme, la garde, le tranchant et la pointe ; ce que les hommes remarquèrent d’abord, ce fut la pointe qui les effrayait. On prit encore la matière pour l’ensemble de la matière et de la forme : par exemple, le {\itshape fer} pour l’{\itshape épée} ; c’est qu’on ne savait pas encore abstraire la forme de la matière. Cette figure mêlée de métonymie et de synecdoque, {\itshape tertia messis erat}, c’était la troisième moisson, fut, sans aucun doute, employée d’abord naturellement et par nécessité ; il fallait plus de mille ans pour que le terme astronomique {\itshape année} pût être inventé. Dans le pays de Florence, on dit toujours, pour désigner un espace de dix ans, {\itshape nous avons moissonné dix fois}. — Ce vers, où se trouvent réunies une métonymie et deux synecdoques,\par

{\itshape Post aliquot mea regna videns mirabor aristas},\\

\noindent n’accuse que trop l’impuissance d’expression qui caractérisa les premiers âges. Pour dire {\itshape tant d’années}, on disait {\itshape tant d’épis}, ce qui est encore plus particulier que {\itshape moissons}. L’expression n’indiquait que l’indigence des langues, et les grammairiens y ont cru voir l’effort de l’art.\par
 4. L’{\itshape ironie} ne peut certainement prendre naissance que dans les temps où l’on réfléchit. En effet, elle consiste dans un mensonge {\itshape réfléchi} qui prend le masque de la vérité. Ici nous apparaît un grand principe qui confirme notre découverte de l’{\itshape origine de la poésie} ; c’est que les premiers hommes des nations païennes ayant eu la simplicité, l’ingénuité de l’enfance, {\itshape les premières fables ne purent contenir rien de faux}, et furent nécessairement, comme elles ont été définies, des {\itshape récits véritables}.\par
5. Par toutes ces raisons, il reste démontré que {\itshape les tropes}, qui se réduisent tous aux quatre espèces que nous avons nommées, ne sont point, comme on l’avait cru jusqu’ici, l’ingénieuse invention des écrivains, mais {\itshape des formes nécessaires dont toutes les nations se sont servies dans leur âge poétique pour exprimer leurs pensées}, et que ces expressions, à leur origine, ont été employées dans leur sens propre et naturel. Mais, à mesure que l’esprit humain se développa, à mesure que l’on trouva les paroles qui signifient des formes abstraites, ou des genres comprenant leurs espèces, ou unissant les parties en leurs entiers, les expressions des premiers hommes devinrent des figures. Ainsi, nous commençons à ébranler ces deux erreurs communes des grammairiens, qui regardent {\itshape le langage des prosateurs comme propre, celui des poètes comme impropre} ; et qui croient {\itshape que l’on parla d’abord en prose, et ensuite en vers}.\par
6. Les monstres, les {\itshape métamorphoses poétiques},  furent le résultat nécessaire de cette incapacité d’abstraire la forme et les propriétés d’un sujet, caractère essentiel aux premiers hommes, comme nous l’avons prouvé dans les axiomes. Guidés par leur logique grossière, ils devaient {\itshape mettre ensemble des sujets}, lorsqu’ils voulaient {\itshape mettre ensemble des formes}, ou bien {\itshape détruire un sujet pour séparer sa forme première de la forme opposée qui s’y trouvait jointe}.\par
7. La {\itshape distinction des idées} fit les {\itshape métamorphoses}. Entre autres phrases {\itshape héroïques} qui nous ont été conservées dans la jurisprudence antique, les Romains nous ont laissé celle de {\itshape fundum fieri}, pour {\itshape auctorem fieri} ; de même que le fonds de terre soutient et la couche superficielle qui le couvre, et ce qui s’y trouve semé, ou planté, ou bâti, de même l’approbateur soutient l’acte qui tomberait sans son approbation ; l’approbateur quitte le caractère d’un être qui se meut à sa volonté, pour prendre le caractère opposé d’une chose stable.
\section[{§ III. Corollaires relatifs aux caractères poétiques employés comme signes du langage par les premières nations}]{§ III. {\itshape Corollaires relatifs aux caractères poétiques employés comme signes du langage par les premières nations}}
\noindent Le langage poétique fut encore employé longtemps dans l’âge historique, à peu près comme les fleuves larges et rapides qui s’étendent bien loin dans la mer, et préservent, par leur impétuosité, la  douceur naturelle de leurs eaux. Si on se rappelle deux axiomes (48, {\itshape Il est naturel aux enfants de transporter l’idée et le nom des premières personnes, des premières choses qu’ils ont vues, à toutes les personnes, à toutes les choses qui ont avec elles quelque ressemblance, quelque rapport.} — 49. {\itshape Les Égyptiens attribuaient à Hermès Trismégiste toutes les découvertes utiles ou nécessaires à la vie humaine}), on sentira que la langue poétique peut nous fournir, relativement à ces {\itshape caractères} qu’elle employait, la matière de grandes et importantes découvertes dans les choses de l’antiquité.\par
1. Solon fut un {\itshape sage}, mais de {\itshape sagesse vulgaire} et non de {\itshape sagesse savante} ({\itshape riposta}). On peut conjecturer qu’il fut chef du parti du peuple, lorsque Athènes était gouvernée par l’aristocratie, et que ce conseil fameux qu’il donnait à ses concitoyens ({\itshape connaissez-vous vous-mêmes}), avait un sens politique plutôt que moral, et était destiné à leur rappeler l’égalité de leurs droits. Peut-être même {\itshape Solon n’est-il que le peuple d’Athènes, considéré comme reconnaissant ses droits, comme fondant la démocratie}. Les Égyptiens avaient rapporté à Hermès toutes les découvertes utiles ; les Athéniens rapportèrent à Solon toutes les institutions démocratiques. — De même, Dracon n’est que l’emblème de la sévérité du gouvernement aristocratique qui avait précédé\footnote{La plupart des lois dont les Athéniens et les Lacédémoniens font honneur à Solon et à Lycurgue, leur ont été attribuées à tort, puisqu’elles sont entièrement contraires au principe de leur conduite. Ainsi Solon institue l’aréopage, qui existait dès le temps de la guerre de Troie, et dans lequel Oreste avait été absous du meurtre de sa mère par la voix de Minerve (c’est-à-dire par le partage égal des voix). Cet aréopage, institué par Solon, le fondateur de la démocratie à Athènes, maintient de toute sa sévérité le gouvernement aristocratique jusqu’au temps de Périclès. Au contraire on attribue à Lycurgue, au fondateur de la république aristocratique de Sparte, une loi agraire analogue à celle que les Gracques proposèrent à Rome. Mais nous voyons que, lorsque Agis voulut réellement introduire à Sparte un partage égal des terres conforme aux principes de la démocratie, il fut étranglé par ordre des Éphores. {\itshape Édition de 1730, pag. 209.}}.\par
 2. Ainsi durent être attribuées à Romulus toutes les lois relatives à la division des ordres ; à Numa tous les règlements qui concernaient les choses saintes et les cérémonies sacrées ; à Tullus Hostilius toutes les lois et ordonnances militaires ; à Servius-Tullius le cens, base de toute démocratie\footnote{L’opinion de Montesquieu et de Vico sur le caractère des institutions de Servius-Tullius a été suivie par M. Niebuhr. ({\itshape N. du T.})}, et beaucoup d’autres lois favorables à la liberté populaire ; à Tarquin l’Ancien, tous les signes et emblèmes, qui, aux temps les plus brillants de Rome, contribuèrent à la majesté de l’empire.\par
3. Ainsi durent être attribuées aux décemvirs, et ajoutées aux douze tables un grand nombre de lois que nous prouverons n’avoir été faites qu’à une époque postérieure. Je n’en veux pour exemple que la défense d’imiter le luxe des Grecs dans les funérailles. Défendre l’abus avant qu’il se fût introduit, c’eût été le faire connaître, et comme l’enseigner. Or, il ne put s’introduire à Rome qu’après les  guerres contre Tarente et Pyrrhus, dans lesquelles les Romains commencèrent à se mêler aux Grecs. Cicéron observe que la loi est exprimée en latin, dans les mêmes termes où elle fut conçue à Athènes.\par
4. Cette découverte des caractères poétiques nous prouve qu’Ésope doit être placé dans l’ordre chronologique bien avant les sept sages de la Grèce. Les sept sages furent admirés pour avoir commencé à donner des préceptes de morale et de politique {\itshape en forme de maximes}, comme le fameux {\itshape Connaissez-vous vous-même} ; mais, auparavant, Ésope avait donné de tels préceptes {\itshape en forme de comparaisons et d’exemples}, exemples dont les poètes avaient emprunté le langage à une époque plus reculée encore. En effet, dans l’ordre des idées humaines, on observe les {\itshape choses semblables} pour les employer d’abord comme {\itshape signes}, ensuite comme {\itshape preuves}. On prouve d’abord par l’{\itshape exemple}, auquel une chose semblable suffit, et finalement par l’{\itshape induction}, pour laquelle il en faut plusieurs. Socrate, père de toutes les sectes philosophiques, introduisit la dialectique par l’{\itshape induction}, et Aristote la compléta avec le {\itshape syllogisme}, qui ne peut prouver qu’au moyen d’une idée générale. Mais pour les esprits peu étendus encore, il suffit de leur présenter une {\itshape ressemblance} pour les persuader : Ménénius Agrippa n’eut besoin, pour ramener le peuple romain à l’obéissance, que de lui conter une fable dans le genre de celles d’Ésope.\par
Le petit peuple des cités héroïques se nourrissait  de ces préceptes politiques dictés par la raison naturelle : {\itshape Ésope est le caractère poétique des plébéiens considérés sous cet aspect}. On lui attribua ensuite beaucoup de fables morales, et il devint le {\itshape premier moraliste}, de la même manière que Solon était devenu {\itshape le législateur} de la république d’Athènes. Comme Ésope avait donné ses préceptes {\itshape en forme de fables}, on le plaça avant Solon, qui avait donné les siens {\itshape en forme de maximes}. De telles fables durent être écrites d’abord {\itshape en vers héroïques}, comme plus tard, selon la tradition, elles le furent {\itshape en vers iambiques}, et enfin {\itshape en prose}, dernière forme sous laquelle elles nous sont parvenues. En effet, les vers iambiques furent pour les Grecs un langage intermédiaire entre celui des vers héroïques et celui de la prose.\par
5. De cette manière, on rapporta aux auteurs de la {\itshape sagesse vulgaire} les découvertes de la {\itshape sagesse} philosophique. Les Zoroastre en Orient, les Trismégiste en Égypte, les Orphée en Grèce, en Italie les Pythagore, devinrent, dans l’opinion, des {\itshape philosophes}, de {\itshape législateurs} qu’ils avaient été. En Chine, Confucius a subi la même métamorphose.
\section[{§ IV. Corollaires relatifs à l’origine des langues et des lettres, laquelle doit nous donner celle des hiéroglyphes, des lois, des noms, des armoiries, des médailles, des monnaies}]{§ IV. {\itshape Corollaires relatifs à l’origine des langues et des lettres, laquelle doit nous donner celle des hiéroglyphes, des lois, des noms, des armoiries, des médailles, des monnaies}}
\noindent  Après avoir examiné la théologie des poètes ou {\itshape métaphysique poétique}, nous avons traversé la {\itshape logique poétique} qui en résulte, et nous arrivons à la {\itshape recherche de l’origine des langues et des lettres}. Il y a autant d’opinions sur ce sujet difficile, qu’on peut compter de savants qui en ont traité. La difficulté vient d’une erreur dans laquelle ils sont tous tombés : ils ont regardé comme choses distinctes, l’origine des langues et celle des lettres, que la nature a unies. Pour être frappé de cette union, il suffisait de remarquer l’étymologie commune de γραμματική, {\itshape grammaire}, et de γράμματα, {\itshape lettres}, caractères (γράφω, {\itshape écrire}) ; de sorte que la {\itshape grammaire}, qu’on définit {\itshape l’art de parler}, devrait être définie l’{\itshape art d’écrire}, comme l’appelle Aristote. — D’un autre côté, {\itshape caractères} signifie {\itshape idées, formes, modèles} ; et certainement les {\itshape caractères poétiques} précédèrent {\itshape ceux des sons articulés}. Josèphe soutient contre Apion\footnote{Orthographié « Appion » [NdE].}, qu’au temps d’Homère les lettres vulgaires n’étaient pas encore inventées. — Enfin, si les lettres avaient été dans l’origine des {\itshape figures de}  {\itshape sons articulés} et non des signes arbitraires\footnote{Vico semble adopter une opinion très différente quelques pages plus loin. ({\itshape Note du Traducteur.})}, elles devraient être uniformes chez toutes les nations, comme les sons articulés. Ceux qui désespéraient de trouver cette origine, devaient toujours ignorer que les premières nations {\itshape ont pensé au moyen des symboles ou caractères poétiques, ont parlé en employant pour signes les fables, ont écrit en hiéroglyphes}, principes certains qui doivent guider la philosophie dans l’étude des {\itshape idées humaines}, comme la philologie dans l’étude des {\itshape paroles humaines}.\par
Avant de rechercher l’origine des langues et des lettres, les philosophes et les philologues devaient se représenter les premiers hommes du paganisme comme concevant les objets par l’idée que leur imagination en personnifiait, et comme s’exprimant, faute d’un autre langage, par des gestes ou par des {\itshape signes matériels} qui avaient des rapports naturels avec les idées\footnote{Par exemple, {\itshape trois épis}, ou l’{\itshape action de couper trois fois des épis}, pour signifier {\itshape trois années}. — Platon et Jamblique ont dit que cette langue, dont les expressions portaient avec elles leur sens naturel, s’était parlée autrefois. Ce fut sans doute cette langue {\itshape atlantique} qui, selon les savants, exprimait les idées par la nature même des choses, c’est-à-dire, par leurs propriétés naturelles. ({\itshape Vico.})}.\par
En tête de ce que nous ayons à dire sur ce sujet, nous plaçons la tradition égyptienne selon laquelle {\itshape trois langues} se sont parlées, correspondant, pour l’ordre comme pour le nombre, aux {\itshape trois âges} écoulés depuis le commencement du monde, {\itshape âges des}  {\itshape dieux, des héros et des hommes}. La première langue avait été la {\itshape langue hiéroglyphique}, ou {\itshape sacrée}, ou {\itshape divine} ; la seconde {\itshape symbolique}, c’est-à-dire employant pour caractères les {\itshape signes} ou {\itshape emblèmes héroïques} ; la troisième {\itshape épistolaire}, propre à faire communiquer entre elles les personnes éloignées, pour les besoins présents de la vie. — On trouve dans l’Iliade deux passages précieux qui nous prouvent que les Grecs partagèrent cette opinion des Égyptiens. \emph{{\itshape Nestor}, dit Homère, {\itshape  vécut trois âges d’hommes parlant diverses langues. Nestor a dû être un symbole de la chronologie}}, déterminée par les trois langues qui correspondaient aux trois âges des Égyptiens. Cette phrase proverbiale, {\itshape vivre les années de Nestor}, signifiait, vivre autant que le monde. Dans l’autre passage, Énée raconte à Achille que \emph{{\itshape des hommes parlant diverses langues commencèrent à habiter Ilion depuis le temps où Troie fut rapprochée des rivages de la mer, et où Pergame en devint la citadelle}}. — Plaçons à côté de ces deux passages la tradition égyptienne d’après laquelle {\itshape Thot} ou {\itshape Hermès aurait trouvé les lois et les lettres}.\par
À l’appui de ces vérités nous présenterons les suivantes : chez les Grecs, le mot {\itshape nom} signifia la même chose que {\itshape caractère}\footnote{Le besoin d’assurer les terres à leurs possesseurs fut un des motifs qui déterminèrent le plus puissamment l’invention des {\itshape caractères} ou {\itshape noms} (dans le sens originaire de {\itshape nomina}, maisons divisées en plusieurs familles ou {\itshape gentes}). Ainsi Mercure Trismégiste, symbole poétique des premiers fondateurs de la civilisation égyptienne, inventa les {\itshape lois} et les {\itshape lettres} ; et c’est du nom de Mercuro, regardé aussi comme le Dieu des marchands, {\itshape mercatorum}, que les Italiens disent {\itshape mercare} pour marquer de {\itshape lettres} ou de {\itshape signes} quelconques les bestiaux et les autres objets de commerce ({\itshape robe da mercantara}) pour la distinction et la sûreté des propriétés. Qui ne s’étonnerait de voir subsister jusqu’à nos jours une telle conformité de pensée et de langage entre les nations ? ({\itshape Vico.})}, et par analogie, les  pères de l’Église traitent indifféremment {\itshape de divinis caracteribus} et {\itshape de divinis nominibus. Nomen} et {\itshape definitio} signifient la même chose, puisqu’en termes de rhétorique, on dit {\itshape quæstio nominis} pour celle qui cherche la {\itshape définition} du fait, et qu’en médecine la partie qu’on appelle {\itshape nomenclature} est celle qui {\itshape définit} la nature des maladies. — Chez les Romains, {\itshape nomina} désigna d’abord et dans son sens propre les {\itshape maisons partagées en plusieurs familles}. Les Grecs prirent d’abord ce mot dans le même sens, comme le prouvent les noms patronymiques, les noms des pères, dont les poètes, et surtout Homère, font un usage si fréquent. De même, les patriciens de Rome sont définis dans Tite-Live de la manière suivante, \emph{{\itshape qui possunt nomine ciere patrem}}. Ces noms patronymiques se perdirent ensuite dans la Grèce, lorsqu’elle eut partout des gouvernements démocratiques ; mais à Sparte, république aristocratique, ils furent conservés par les Héraclides. — Dans la langue de la jurisprudence romaine, {\itshape nomen} signifie {\itshape droit} ; et en grec, νόμος, qui en est à peu près l’homonyme, a le sens de {\itshape loi}. De νόμος, vient νόμισμα, {\itshape monnaie}, comme le remarque Aristote ; et les étymologistes veulent que les Latins aient aussi  tiré de νόμος, leur {\itshape nummus}. Chez les Français, du mot {\itshape loi} vient {\itshape aloi}, titre de la monnaie. Enfin au moyen âge, la loi ecclésiastique fut appelée {\itshape canon}, terme par lequel on désignait aussi la redevance emphytéotique payée par l’emphytéote… Les Latins furent peut-être conduits par une idée analogue, à désigner par un même mot {\itshape jus}, le {\itshape droit} et l’{\itshape offrande} ordinaire que l’on faisait à Jupiter (les parties grasses des victimes). De l’ancien nom de ce dieu {\itshape Jous}, dérivèrent les génitifs {\itshape Jovis} et {\itshape juris}. — Les Latins appelaient les terres {\itshape prædia}, parce que, ainsi que nous le ferons voir, les premières terres cultivées furent les premières {\itshape prædæ} du monde. C’est à ces terres que le mot {\itshape domare}, dompter, fut appliqué d’abord. Dans l’ancien droit romain, on les disait {\itshape manucaptæ}, d’où est resté {\itshape manceps}, celui qui est obligé sur immeuble envers le trésor. On continua de dire dans les lois romaines, {\itshape jura prædiorum}, pour désigner les servitudes qu’on appelle {\itshape réelles}, et qui sont attachées à des immeubles. Ces terres {\itshape manucaptæ} furent sans doute appelées d’abord {\itshape mancipia}, et c’est certainement dans ce sens qu’on doit entendre l’article de la loi des douze tables, \emph{{\itshape qui nexum faciet mancipiumque}}. Les Italiens considérèrent la chose sous le même aspect que les anciens Latins, lorsqu’ils appelèrent les terres {\itshape poderi}, de {\itshape podere}, puissance ; c’est qu’elles étaient acquises par la force ; ce qui est encore prouvé par l’expression du moyen âge, {\itshape presas terrarum}, pour dire les {\itshape champs avec leurs limites}. Les Espagnols appellent  {\itshape prendas} les entreprises courageuses ; les Italiens disent {\itshape imprese} pour {\itshape armoiries}, et {\itshape termini} pour {\itshape paroles}, expression qui est restée dans la scholastique. Ils appellent encore les armoiries {\itshape insegne}, d’où leur vient le verbe {\itshape insegnare}. De même Homère, au temps duquel on ne connaissait pas encore les lettres alphabétiques, nous apprend que la lettre de Pretus contre Bellérophon fut écrite en {\itshape signes}, σήματα.\par
Pour compléter tout ceci, nous ajouterons trois vérités incontestables : 1º dès qu’il est démontré que les premières nations païennes furent {\itshape muettes} dans leurs commencements, on doit admettre qu’elles s’expliquèrent par des {\itshape gestes} ou des {\itshape signes matériels}, qui avaient un rapport naturel avec les idées ; 2º elles durent assurer par des {\itshape signes} les {\itshape limites de leurs champs}, et conserver des {\itshape monuments durables de leurs droits} ; 3º toutes employèrent la {\itshape monnaie}. — Toutes les vérités que nous venons d’énoncer nous donnent l’{\itshape origine des langues et des lettres}, dans laquelle se trouve comprise celle des {\itshape hiéroglyphes}, des {\itshape lois}, des {\itshape noms}, des {\itshape armoiries}, des {\itshape médailles}, des {\itshape monnaies}, et en général, de la {\itshape langue} que parla, de l’écriture qu’employa, dans son origine, le {\itshape droit naturel des gens}\footnote{\noindent Telle est l’origine des {\itshape armoiries}, et par suite des {\itshape médailles}. Les familles, puis les nations, les employèrent d’abord par nécessité. Elles devinrent plus tard un objet d’amusement et d’érudition. On a donné à ces {\itshape emblèmes} le nom d’{\itshape héroïques}, sans en bien sentir le motif. Les modernes ont besoin d’y inscrire des devises qui leur donnent un sens ; il n’en était pas de même des emblèmes employés naturellement dans les temps héroïques ; leur silence parlait assez. Ils portaient avec eux leur signification ; ainsi {\itshape trois épis}, ou le {\itshape geste de couper trois fois des épis}, signifiait naturellement {\itshape trois années} ; d’où il vint que {\itshape caractère} et {\itshape nom} s’employèrent indifféremment l’un pour l’autre, et que les mots {\itshape nom} et {\itshape nature} eurent la même signification, comme nous l’avons dit plus haut.Ces {\itshape armoiries}, ces {\itshape armes} et {\itshape emblèmes des familles}, furent employés au moyen âge, lorsque les nations, redevenues muettes, perdirent l’usage du langage vulgaire. Il ne nous reste aucune connaissance des langues que parlaient alors les Italiens, les Français, les Espagnols et les autres nations de ce temps. Les prêtres seuls savaient le latin et le grec. En français {\itshape clerc} voulait dire souvent {\itshape lettré} ; au contraire, chez les italiens, {\itshape laico} se disait pour {\itshape illettré}, comme on le voit dans un beau passage de Dante. Parmi les prêtres mêmes, il y avait tant d’ignorance, qu’on trouve des actes souscrits par des évêques, où ils ont mis simplement la marque d’une croix, faute de savoir écrire leur nom. Parmi les prélats instruits, il y en avait même peu qui eussent écrire. Le père Mabillon, dans son ouvrage {\itshape De re diplomaticâ}, a pris le soin de reproduire par la gravure les signatures apposées par des évêques et des archevêques aux actes des Conciles de ces temps barbares ; l’écriture en est plus informe que celle des hommes les plus ignorants d’aujourd’hui ; et pourtant ces prélats étaient les chanceliers des royaumes chrétiens, comme aujourd’hui encore les trois archevêques archichanceliers de l’Empire pour les langues allemande, française et italienne. Une loi anglaise accorde la vie au coupable digne de mort qui pourra prouver qu’il sait lire. C’est peut-être pour cette cause que plus tard le mot {\itshape lettré} a fini par avoir à peu près le même sens que celui de savant. — Il est encore résulté de cette ignorance de l’écriture, que dans les anciennes maisons il n’y a guères de mur où l’on n’ait gravé quelque figure, quelqu’emblème.\par
Concluons de tout ceci que ces {\itshape signes} divers, employés nécessairement par les nations {\itshape muettes} encore, pour assurer la distinction des propriétés furent ensuite appliqués aux usages publics, soit à ceux de la paix (d’où provinrent les médailles), soit à ceux de la guerre. Dans ce dernier cas, ils ont l’usage primitif des hiéroglyphes, puisqu’ordinairement les guerres ont lieu entre des nations qui parlent des langues différentes et qui par conséquent sont {\itshape muettes} l’une par rapport à l’autre.
}.\par
Pour établir ces principes sur une base plus solide  encore, nous devons attaquer l’opinion selon laquelle les hiéroglyphes auraient été inventés par les philosophes, pour y cacher les mystères d’une  sagesse profonde, comme on l’a cru des Égyptiens. Ce fut pour toutes les premières nations une nécessité naturelle de s’exprimer en hiéroglyphes. À ceux des Égyptiens et des Éthiopiens nous croyons pouvoir joindre les caractères magiques des Chaldéens ; les cinq présents, les {\itshape cinq paroles matérielles} que le roi des Scythes envoya à Darius fils d’Hystaspe ; les pavots que Tarquin le Superbe abattit avec sa baguette devant le messager de son fils ; les rébus de Picardie employés, au moyen âge, dans le nord de la France. Enfin les anciens Écossais (selon Boëce), les Mexicains et autres peuples indigènes de l’Amérique écrivaient en hiéroglyphes, comme les Chinois le font encore aujourd’hui.\par
1. Après avoir détruit cette grave erreur, nous reviendrons aux trois langues distinguées par les Égyptiens ; et pour parler d’abord de la première, nous remarquerons qu’Homère, dans cinq passages, fait mention d’une langue plus ancienne que la sienne, qui est l’{\itshape héroïque} ; il l’appelle \emph{{\itshape langue des dieux}}. D’abord dans l’Iliade : \emph{{\itshape Les dieux}, dit-il, {\itshape  appellent ce géant Briarée, les hommes Égéon}} ; plus loin, en parlant d’un oiseau, \emph{{\itshape son nom est Chalcis chez les dieux, Cymindis chez les hommes}} ; et au sujet du fleuve de Troie, \emph{{\itshape les dieux l’appellent Xanthe, et les hommes Scamandre}}. Dans l’Odyssée, il y a deux passages analogues :  \emph{{\itshape ce que les hommes appellent Charybde et Scylla, les dieux l’appellent les Rochers errants}} ; l’herbe qui doit prémunir Ulysse contre les enchantements de Circé \emph{{\itshape est inconnue aux hommes, les dieux l’appellent moly}}.\par
Chez les Latins, Varron s’occupa de la langue divine ; et les trente mille dieux dont il rassembla les noms, devaient former un riche vocabulaire\footnote{La plupart des langues ont à peu près trente mille mots. Si l’on peut ajouter foi aux calculs de Héron dans son ouvrage sur la Langue Anglaise, l’Espagnol en aurait trente mille, le Français trente-deux mille, l’Italien trente-cinq mille, l’Anglais trente-sept mille. ({\itshape Note du Traducteur.})}, au moyen duquel les nations du Latium pouvaient exprimer les besoins de la vie humaine, sans doute peu nombreux dans ces temps de simplicité, où l’on ne connaissait que le nécessaire. Les Grecs comptaient aussi trente mille dieux, et divinisaient les pierres, les fontaines, les ruisseaux, les plantes, les rochers, de même que les sauvages de l’Amérique déifient tout ce qui s’élève au-dessus de leur faible capacité. Les {\itshape fables divines} des Latins et des Grecs durent être pour eux les premiers hiéroglyphes, les caractères sacrés de cette langue divine dont parlent les Égyptiens.\par
2. La {\itshape seconde langue}, qui répond à l’{\itshape âge des héros}, se parla par symboles, au rapport des Égyptiens. À ces symboles peuvent être rapportés les {\itshape signes héroïques} avec lesquels écrivaient les héros, et qu’Homère appelle σήματα. Conséquemment, ces symboles durent être des métaphores, des images, des similitudes ou comparaisons qui, ayant passé  depuis dans la {\itshape langue articulée}, font toute la richesse du style poétique.\par
Homère est indubitablement {\itshape le premier auteur de la langue grecque} ; et puisque nous tenons des Grecs tout ce que nous connaissons de l’antiquité païenne, il se trouve aussi le premier auteur que puisse citer le paganisme. Si nous passons aux Latins, les premiers monuments de leur langue sont les fragments des {\itshape vers saliens}. Le premier écrivain latin dont on fasse mention est le {\itshape poète} Livius Andronicus. Lorsque l’Europe fut retombée dans la barbarie, et qu’il se forma deux nouvelles langues, la première, que parlèrent les Espagnols, fut la langue {\itshape romane} ({\itshape di romanzo}), langue de la poésie {\itshape héroïque}, puisque les {\itshape romanciers} furent les {\itshape poètes héroïques} du moyen âge. En France, le premier qui écrivit en langue vulgaire fut Arnauld Daniel Pacca, le plus ancien de tous les poètes provençaux ; il florissait au onzième siècle. Enfin l’Italie eut ses premiers écrivains dans les {\itshape rimeurs} de Florence et de la Sicile.\par
3. Le {\itshape langage épistolaire} [ou alphabétique], que l’on est convenu d’employer comme moyen de communication entre les personnes éloignées, dut être parlé originairement chez les Égyptiens, par les classes inférieures d’un peuple qui dominait en Égypte, probablement celui de Thèbes, dont le roi, Ramsès, étendit son empire sur toute cette grande nation. En effet, chez les Égyptiens, cette langue correspondait à l’âge des {\itshape hommes} ; et ce nom d’{\itshape hommes} désigne les classes inférieures, chez les peuples héroïques  (particulièrement au moyen âge, où {\itshape homme} devient synonyme de {\itshape vassal}), par opposition aux {\itshape héros}. Elle dut être adoptée {\itshape par une convention libre} ; car c’est une règle éternelle que le langage et l’écriture vulgaire sont un droit des peuples. L’empereur Claude ne put faire recevoir par les Romains trois lettres qu’il avait inventées, et qui manquaient à leur alphabet. Les lettres inventées par le Trissin n’ont pas été reçues dans la langue italienne, quelque nécessaires qu’elles fussent.\par
La {\itshape langue épistolaire} ou {\itshape vulgaire} des Égyptiens dut s’écrire avec des lettres également {\itshape vulgaires}. Celles de l’Égypte ressemblaient à l’alphabet vulgaire des Phéniciens, qui, dans leurs voyages de commerce, l’avaient sans doute porté en Égypte. Ces caractères n’étaient autre chose que les {\itshape caractères mathématiques} et les {\itshape figures géométriques}, que les Phéniciens avaient eux-mêmes reçus des Chaldéens, les premiers mathématiciens du monde. Les Phéniciens les transmirent ensuite aux Grecs, et ceux-ci, avec la supériorité de génie qu’ils ont eue sur toutes les nations, employèrent ces formes géométriques comme formes des sons articulés, et en tirèrent leur alphabet vulgaire, adopté ensuite par les Latins\footnote{Nous avons déjà rapporté le passage où Tacite nous apprend \emph{{\itshape que les lettres des Latins ressemblaient à l’ancien alphabet des Grecs}}. Ce qui le prouve, c’est que les Grecs employèrent pendant longtemps les lettres majuscules pour figurer les nombres, et que les Latins conservèrent toujours le même usage. ({\itshape Vico.})}. On ne peut croire que les Grecs aient  tiré des Hébreux ou des Égyptiens la {\itshape connaissance des lettres vulgaires}.\par
\par
Les philologues ont adopté sur parole l’opinion que la signification des {\itshape langues vulgaires} est arbitraire. Leurs {\itshape origines ayant été naturelles}, leur {\itshape signification dut être fondée en nature}. On peut l’observer dans la {\itshape langue vulgaire} des Latins, qui a conservé plus de traces que la grecque, de son origine {\itshape héroïque}, et qui lui est aussi supérieure pour la force, qu’inférieure pour la délicatesse. Presque tous les mots y sont des {\itshape métaphores} tirées des objets naturels, d’après leurs propriétés ou leurs effets sensibles. En général, la {\itshape métaphore} fait le fond des langues. Mais les grammairiens, s’épuisant en paroles qui ne donnent que des idées confuses, ignorant les origines des mots qui, dans le principe ne purent être que claires et distinctes, ont rassuré leur ignorance en décidant d’une manière générale et absolue {\itshape que les voix humaines articulées avaient une signification arbitraire}. Ils ont placé dans leurs rangs Aristote, Galien et d’autres philosophes, et les ont armés contre Platon et Jamblique.\par
Il reste cependant une difficulté. {\itshape Pourquoi y a-t-il autant de langues vulgaires qu’il existe de peuples} ? Pour résoudre ce problème, établissons d’abord une grande vérité : par un effet de la {\itshape diversité des climats}, les peuples ont {\itshape diverses natures.}  Cette variété de natures leur a fait voir sous {\itshape différents aspects} les choses utiles ou nécessaires à la vie humaine, et a produit la {\itshape diversité des usages}, dont {\itshape celle des langues} est résultée. C’est ce que les proverbes prouvent jusqu’à l’évidence. Ce sont des maximes pour l’usage de la vie, dont le {\itshape sens} est le même, mais dont l’{\itshape expression} varie sous autant de rapports divers qu’il y a eu et qu’il y a encore de nations\footnote{Les locutions {\itshape héroïques} conservées et abrégées dans la précision des langues plus récentes, ont bien étonné les commentateurs de la Bible, qui voient les noms des mêmes rois exprimés d’une manière dans l’Histoire Sacrée, et d’une autre dans l’Histoire profane. C’est que le même homme est envisagé dans l’une, je supposé, sous le rapport de la figure, de la puissance, etc. ; dans l’autre, sous le rapport de son caractère, des choses qu’il a entreprises. Nous observons de même qu’en Hongrie la même ville a un nom chez les Hongrois, un autre chez les Grecs, un troisième chez les Allemands, un quatrième chez les Turcs. L’allemand, qui est une langue {\itshape héroïque}, quoique vivante, reçoit tous les mots étrangers en leur faisant subir une transformation. On doit conjecturer que les Latins et les Grecs en font autant, lorsqu’ils expriment tant de choses particulières aux barbares, avec des mots qui sonnent si bien en latin et en grec. Voilà pourquoi on trouve tant d’obscurité dans la géographie et dans l’histoire naturelle des anciens. ({\itshape Vico.})}.\par
D’après ces considérations, nous avons médité un {\itshape vocabulaire mental}, dont le but serait d’{\itshape expliquer toutes les langues}, en ramenant la {\itshape multiplicité de leurs expressions} à certaines {\itshape unités d’idées}, dont les peuples ont conservé le fond en leur donnant des formes variées, et les modifiant diversement. Nous faisons dans cet ouvrage un usage continuel de ce vocabulaire. C’est, avec une méthode différente, le même sujet qu’a traité Thomas Hayme  dans ses dissertations {\itshape De linguarum cognatione}, et {\itshape De linguis in genere, et variarum linguarum harmoniâ}.\par
De tout ce qui précède, nous tirerons le corollaire suivant : plus les langues sont {\itshape riches en locutions héroïques, abrégées par les locutions vulgaires}, plus elles sont belles ; et elles tirent cette beauté de la {\itshape clarté avec laquelle elles laissent voir leur origine} : ce qui constitue, si je puis le dire, leur véracité, leur fidélité. Au contraire, plus elles présentent un grand nombre de mots dont l’origine est cachée, moins elles sont agréables, à cause de leur obscurité, de leur confusion, et des erreurs auxquelles elle peut donner lieu. C’est ce qui doit arriver dans les langues {\itshape formées d’un mélange de plusieurs idiomes barbares}, qui n’ont point laissé de traces de leurs origines, ni des changements que les mots ont subis dans leur signification.\par
\par
Maintenant, pour comprendre la formation de ces trois sortes de langues et d’alphabets, nous établirons le principe suivant : {\itshape les dieux, les héros et les hommes commencèrent dans le même temps}. Ceux qui imagineront les {\itshape dieux} étaient des {\itshape hommes}, et croyaient leur nature {\itshape héroïque} mêlée de la {\itshape divine} et de l’{\itshape humaine}. Les trois espèces de langues et d’écritures furent aussi contemporaines dans leur origine, mais avec trois différences capitales : la langue {\itshape divine} fut très peu articulée, et presque entièrement {\itshape muette} ; la langue des {\itshape héros, muette et articulée} par un mélange égal, et composée par conséquent de  paroles vulgaires et de caractères héroïques, avec lesquels écrivaient les héros (σήματα, dans Homère) ; la langue des {\itshape hommes} n’eut presque rien de muet, et fut à peu près entièrement {\itshape articulée}. Point de langue vulgaire qui ait autant d’expressions que de choses à exprimer. — Une conséquence nécessaire de tout ceci, c’est que, dans l’origine, la langue {\itshape héroïque} fut extrêmement confuse, cause essentielle de l’obscurité des fables.\par
La langue articulée commença par l’{\itshape onomatopée}, au moyen de laquelle nous voyons toujours les enfants se faire très bien entendre. Les premières paroles humaines furent ensuite les {\itshape interjections}, ces mots qui échappent dans le premier mouvement des passions violentes, et qui dans toutes les langues sont monosyllabiques. Puis vinrent les {\itshape pronoms}. L’interjection soulage la passion de celui à qui elle échappe, et elle échappe lors même qu’on est seul ; mais les pronoms nous servent à communiquer aux autres nos idées sur les choses dont les noms propres sont inconnus ou à nous, ou à ceux qui nous écoutent. La plupart des pronoms sont des monosyllabes dans presque toutes les langues. On inventa alors les {\itshape particules}, dont les {\itshape prépositions}, également monosyllabiques, sont une espèce nombreuse. Peu à peu se formèrent les {\itshape noms}, presque tous monosyllabiques dans l’origine. On le voit dans l’allemand, qui est une langue mère, parce que l’Allemagne n’a jamais été occupée par des conquérants étrangers.  Dans cette langue, toutes les racines sont des monosyllabes.\par
Le nom dut précéder le {\itshape verbe}, car le discours n’a point de sens s’il n’est régi par un nom, exprimé ou sous-entendu. En dernier lieu se formèrent les verbes. Nous pouvons observer en effet que les enfants disent des noms, des particules, mais point de verbes : c’est que les noms éveillent des idées qui laissent des traces durables ; il en est de même des particules qui signifient des modifications. Mais les verbes signifient des mouvements accompagnés des idées d’antériorité et de postériorité, et ces idées ne s’apprécient que par le point indivisible du présent, si difficile à comprendre, même pour les philosophes. J’appuierai ceci d’une observation physique. Il existe ici un homme qui, à la suite d’une violente attaque d’apoplexie, se souvenait bien des noms, mais avait entièrement oublié les verbes. — Les verbes qui sont des genres à l’égard de tous les autres, tels que : {\itshape sum}, qui indique l’existence, verbe auquel se rapportent toutes les essences, c’est-à-dire tous les objets de la métaphysique ; {\itshape sto, eo}, qui expriment le repos et le mouvement, auxquels se rapportent toutes les choses physiques ; {\itshape do, dico, facio}, auxquels se rapportent toutes les choses d’action, relatives soit à la morale, soit aux intérêts de la famille ou de la société, ces verbes, dis-je, sont tous des monosyllabes à l’impératif, {\itshape es, sta, i, da, dic, fac} ; et c’est par l’impératif qu’ils ont dû commencer.\par
Cette {\itshape génération du langage} est conforme aux lois  de la nature en général, d’après lesquelles les éléments, dont toutes les choses se composent et où elles vont se résoudre, sont indivisibles ; elle est conforme aux lois de la nature humaine en particulier, en vertu de cet axiome : {\itshape Les enfants, qui, dès leur naissance, se trouvent environnés de tant de moyens d’apprendre les langues, et dont les organes sont si flexibles, commencent par prononcer des monosyllabes.} À plus forte raison doit-on croire qu’il en a été ainsi chez ces premiers hommes, dont les organes étaient très durs, et qui n’avaient encore entendu aucune voix humaine. — Elle nous donne en outre {\itshape l’ordre dans lequel furent trouvées les parties du discours}, et conséquemment {\itshape les causes naturelles de la syntaxe}. Ce système semble plus raisonnable que celui qu’ont suivi Jules Scaliger et François Sanctius relativement à la langue latine : ils raisonnent d’après les principes d’Aristote, comme si les peuples qui trouvèrent les langues avaient dû préalablement aller aux écoles des philosophes.
\section[{§ V. Corollaires relatifs à l’origine de l’élocution poétique, des épisodes, du tour, du nombre, du chant et du vers}]{§ V. {\itshape Corollaires relatifs à l’origine de l’élocution poétique, des épisodes, du tour, du nombre, du chant et du vers}}
\noindent Ainsi se forma la {\itshape langue poétique}, composée d’abord de symboles ou {\itshape caractères divins} et {\itshape héroïques}, qui furent ensuite exprimés en {\itshape locutions vulgaires},  et finalement écrits en {\itshape caractères vulgaires}. Elle naquit de l’{\itshape indigence du langage}, et de la nécessité de s’exprimer ; ce qui se démontre par les ornements même dont se pare la poésie, je veux dire les images, les hypotyposes, les comparaisons, les métaphores, les périphrases, les tours qui expriment les choses par leurs propriétés naturelles, les descriptions qui les peignent par les détails ou par les effets les plus frappants, ou enfin par des accessoires emphatiques et même oiseux.\par
Les {\itshape épisodes} sont nés dans les premiers âges de la {\itshape grossièreté des esprits}, incapables de distinguer et d’écarter les choses qui ne vont pas au but. La même cause fait qu’on observe toujours les mêmes effets dans les idiots, et surtout dans les femmes.\par
Les {\itshape tours} naquirent de la {\itshape difficulté de compléter la phrase par son verbe}. Nous avons vu que le verbe fut trouvé plus tard que les autres parties du discours. Aussi les Grecs, nation ingénieuse, employèrent moins de tours que les Latins, les Latins moins que les Allemands.\par
Le {\itshape nombre} ne fut introduit que tard dans la prose. Les premiers qui l’employèrent furent, chez les Grecs, Gorgias de Léontium, et chez les Latins, Cicéron. Avant eux, c’est Cicéron lui-même qui le rapporte, on ne savait rendre le discours nombreux qu’en y mêlant certaines {\itshape mesures poétiques}. Il nous sera très utile d’avoir établi ceci, lorsque nous traiterons de l’{\itshape origine du chant et du vers}.\par
Tout ce que nous venons de dire semble prouver  que, par une loi nécessaire de notre nature, le {\itshape langage poétique} a précédé celui de la prose. Par suite de la même loi, les fables, {\itshape universaux de l’imagination}, durent naître avant ceux du raisonnement et de la philosophie. Ces derniers ne purent être créés qu’au moyen de la prose. En effet, les poètes ayant d’abord formé le langage poétique par l’{\itshape association des idées particulières}, comme on l’a démontré, les peuples formèrent ensuite la langue de la prose, en ramenant à un seul mot, comme les espèces au genre, les parties qu’avait mises ensemble le langage poétique. Ainsi cette phrase poétique usitée chez toutes les nations, {\itshape le sang me bout dans le cœur}, fut exprimée par un seul mot, στόμαχος, {\itshape ira}, colère. Les hiéroglyphes, et les lettres alphabétiques furent aussi comme autant de genres auxquels on ramena la variété infinie des sons articulés. Cette méthode abrégée, appliquée aux mots et aux lettres, donna plus d’activité aux esprits, et les rendit capables d’abstraire ; ensuite purent venir les philosophes, qui, préparés par cette classification vulgaire des mots et des lettres, travaillaient à celle des idées, et formèrent les {\itshape genres intelligibles}. Ne conviendra-t-on pas maintenant que pour trouver l’origine des {\itshape lettres}, il fallait chercher en même temps celle des {\itshape langues} ?\par
Quant au {\itshape chant} et au {\itshape vers}, nous avons dit dans nos axiomes, que, supposé que les hommes aient été d’abord muets, ils commencèrent par prononcer les voyelles en chantant, comme font les muets ; puis ils durent, comme les bègues, articuler aussi  les consonnes en chantant\footnote{Ce qui le prouve, ce sont les diphthongues qui restèrent dans les langues, et qui durent être bien plus nombreuses dans l’origine. Ainsi les Grecs et les Français qui ont passé d’une manière prématurée de la barbarie à la civilisation ont conservé beaucoup de diphthongues. Voyez la note de l’axiome 21. ({\itshape Vico.})}. Ces premiers hommes ne devaient s’essayer à parler que lorsqu’ils éprouvaient des passions très violentes. Or, de telles passions s’expriment par un ton de voix très élevé, qui multiplie les diphtongues, et devient une sorte de chant. Ce premier chant vint naturellement de la difficulté de prononcer, laquelle se démontre par la cause et par l’effet. {\itshape Par la cause}, les premiers hommes avaient une grande dureté dans l’organe de la voix, et d’ailleurs bien peu de mots pour l’exercer\footnote{Maintenant encore, au milieu de tant de moyens d’apprendre à parler, ne voyons-nous pas les enfants, malgré la flexibilité de leurs organes, prononcer les consonnes avec la plus grande peine. Les Chinois, qui avec un très petit nombre de signes diversement modifiés, expriment en langue vulgaire leurs cent vingt mille hiéroglyphes, parlent aussi en chantant. ({\itshape Vico.})}. {\itshape Par l’effet} : il y a dans la poésie italienne un grand nombre de retranchements ; dans les origines de la langue latine, on trouve aussi beaucoup de mots qui durent être syncopés, puis étendus avec le temps. Le contraire arriva pour les répétitions de syllabes. Lorsque les bègues tombent sur une syllabe qui leur est facile à prononcer, ils s’y arrêtent avec une sorte de chant, comme pour compenser celles qu’ils prononcent difficilement. J’ai connu un excellent musicien qui avait ce défaut  de prononciation ; lorsqu’il se trouvait arrêté, il se mettait à chanter d’une manière fort agréable, et parvenait ainsi à articuler. Les Arabes commencent presque tous les mots par {\itshape al}, et l’on dit que les Huns furent ainsi appelés parce qu’ils commençaient tous les mots par {\itshape hun}. Ce qui prouve encore que les langues furent d’abord un {\itshape chant}, c’est ce que nous avons dit, qu’avant Gorgias et Cicéron, les prosateurs grecs et latins employaient des nombres poétiques ; au moyen âge, les pères de l’Église latine en firent autant, et leur prose semble faite pour être chantée.\par
Le premier genre de {\itshape vers} dut être approprié à la langue, à l’âge des {\itshape héros} : tel fut le vers {\itshape héroïque}, le plus noble de tous. C’était l’expression des émotions les plus vives de la terreur ou de la joie. La poésie {\itshape héroïque} ne peint que les passions les plus violentes. Si le vers {\itshape héroïque} fut d’abord spondaïque, on ne peut l’attribuer, comme le fait la tradition vulgaire, à l’effroi inspiré par le serpent Python ; l’effroi précipite les idées et les paroles plutôt qu’il ne les ralentit. En latin, {\itshape sollicitus} et {\itshape festinans} expriment la frayeur. La lenteur des esprits, la difficulté du langage, voilà ce qui dut le rendre spondaïque ; et il a conservé quelque chose de ce caractère, en exigeant invariablement un spondée à son dernier pied. Plus tard, les esprits et les langues ayant plus de facilité, le dactyle entra dans la poésie ; un nouveau progrès détermina l’emploi de l’iambe, {\itshape pes citus}, comme dit Horace. Enfin l’intelligence et la prononciation  ayant acquis une grande rapidité, on commença de parler en prose, ce qui était une sorte de généralisation. Le vers iambique se rapproche tellement de la prose, qu’il échappait souvent aux prosateurs. Ainsi le chant uni aux vers devint de plus en plus rapide, en suivant exactement le progrès du langage et des idées. — Ces vérités philosophiques sont appuyées par la tradition suivante : l’histoire ne nous présente rien de plus ancien que les {\itshape oracles} et les {\itshape sibylles} ; l’antiquité de ces dernières a passé en proverbe. Nous trouvons partout des Sibylles chez les plus anciennes nations : or, on assure qu’elles chantaient leurs réponses en vers héroïques, et partout les oracles répondaient en vers de cette mesure. Ce vers fut appelé par les Grecs {\itshape pythien}, de leur fameux oracle d’Apollon Pythien. Les Latins l’appelèrent vers {\itshape saturnien}, comme l’atteste Festus. Ce vers dut être inventé en Italie dans l’{\itshape âge de Saturne}, qui répond à l’{\itshape âge d’or} des Grecs. Ennius, cité par le même Festus, nous apprend que les {\itshape faunes} de l’Italie rendaient en cette forme de vers leurs oracles, {\itshape fata}. Puis le nom de vers {\itshape saturnien} passa aux vers iambiques de six pieds, peut-être parce que ces derniers vers firent employés naturellement dans le langage, comme auparavant les vers {\itshape saturniens-héroïques}. — Les savants modernes sont aujourd’hui divisés sur la question de savoir si la poésie hébraïque a une mesure, ou simplement une sorte de rythme\footnote{Orthographié « rhythme » [NdE].} ; mais Josèphe, Philon, Origène et Eusèbe, tiennent pour la première opinion ;  et ce qui la favorise principalement, c’est que, selon saint Jérôme, le livre de Job, plus ancien que ceux de Moïse, serait écrit en vers héroïques depuis la fin du second chapitre jusqu’au commencement du quarante-deuxième. — Si nous en croyons l’auteur anonyme de l’{\itshape Incertitude des sciences}, les Arabes, qui ne connaissaient point l’écriture, conservèrent leur ancienne langue, en retenant leurs poèmes nationaux jusqu’au temps où ils inondèrent les provinces orientales de l’empire grec.\par
Les Égyptiens écrivaient leurs épitaphes en {\itshape vers}, et sur des colonnes appelées {\itshape siringi}, de {\itshape sir}, chant ou chanson. Du même mot vient sans doute le nom des {\itshape Sirènes}, êtres mythologiques célèbres par leur chant. Ce qui est plus certain, c’est que les fondateurs de la civilisation grecque furent les {\itshape poètes théologiens}, lesquels furent aussi {\itshape héros} et chantèrent en {\itshape vers héroïques}. Nous avons vu que les premiers auteurs de la langue latine furent les poètes sacrés appelés {\itshape saliens} ; il nous reste des fragments de leurs vers, qui ont quelque chose du {\itshape vers héroïque}, et qui sont les plus anciens monuments de la langue latine. À Rome, les triomphateurs laissèrent des inscriptions qui ont une apparence de vers {\itshape héroïques}, telles que celles de Lucius Emilius Regillus,\par

{\itshape Duello magno dirimendo, regibus subjugandis} ;\\

\noindent et celle d’Acilius Glabrion,\par

{\itshape Fudit, fugat, prosternit maximas legiones}.\\

\noindent Si on examine bien les fragments de la loi des douze  tables, on trouvera que la plupart des articles se terminent par un vers adonique, c’est-à-dire par une fin de vers {\itshape héroïque} ; c’est ce que Cicéron imita dans ses {\itshape Lois}, qui commencent ainsi :\par


\begin{verse}
Deos caste adeunto.\\
Pietatem adhibento.\\
\end{verse}

\noindent De là vint, chez les Romains, l’usage mentionné par le même Cicéron ; les enfants chantaient la loi des douze tables, \emph{{\itshape tanquam necessarium carmen}}. Ceux des Crétois chantaient de même la loi de leur pays, au rapport d’Élien. — À ces observations joignez plusieurs traditions vulgaires. Les lois des Égyptiens furent les {\itshape poèmes} de la déesse Isis (Platon). Lycurgue et Dracon donnèrent leurs lois en {\itshape vers} aux Spartiates et aux Athéniens (Plutarque et Suidas). Enfin Jupiter dicta en {\itshape vers} les lois de Minos (Maxime de Tyr).\par
Maintenant revenons des lois à l’histoire. Tacite rapporte dans les {\itshape Mœurs des Germains}, que ce peuple conservait en {\itshape vers} les souvenirs des premiers âges ; et dans sa note sur ce passage, Juste-Lipse dit la même chose des Américains. L’exemple de ces deux nations, dont la première ne fut connue que très tard par les Romains, et dont la seconde a été découverte par les Européens il y a seulement deux siècles, nous donne lieu de conjecturer qu’il en a été de même de toutes les nations barbares, anciennes et modernes. La chose est hors de doute pour les anciens Perses et pour les Chinois. Au rapport de Festus, les guerres puniques furent écrites  par Nævius en {\itshape vers héroïques}, avant de l’être par Ennius ; et Livius Andronicus, le premier écrivain latin, avait écrit dans un {\itshape poème héroïque} appelé {\itshape la Romanide}, les annales des anciens Romains. Au moyen âge, les historiens latins furent des {\itshape poètes héroïques}, comme Gunterus, Guillaume de Pouille, et autres. Nous avons vu que les premiers écrivains dans les nouvelles langues de l’Europe, avaient été des {\itshape versificateurs}. Dans la Silésie, province où il n’y a guère que des paysans, ils apportent en naissant le don de la {\itshape poésie}. En général, l’allemand conserve ses origines {\itshape héroïques}, et voilà pourquoi on traduit si heureusement en allemand les mots composés du grec, surtout ceux du langage poétique. Adam Rochemberg l’a remarqué, mais sans en comprendre la cause. Bernegger a fait de toutes ces expressions un catalogue, enrichi ensuite par Georges Christophe Peischer, dans son {\itshape Index de græcæ et germanicæ linguæ analogiâ}. La langue latine a aussi laissé des exemples nombreux de ces compositions formées de mots entiers ; et les poètes, en continuant à se servir de ces mots composés, n’ont fait qu’user de leur droit. Cette facilité de composition dut être une propriété commune à toutes les langues primitives. Elles se créèrent d’abord des noms, ensuite des verbes, et lorsque les verbes leur manquèrent, elles unirent les noms eux-mêmes. Voilà les principes de tout ce qu’a écrit Morhof dans ses recherches sur la langue et la poésie allemande\footnote{Nous trouvons ici une preuve de ce que nous avons avancé dans les axiomes. Si les savants s’appliquent à trouver les origines de la langue allemande en suivant nos principes, ils y feront d’étonnantes découvertes. ({\itshape Vico.})}.\par
 Nous croyons avoir victorieusement réfuté l’erreur commune des grammairiens qui prétendent que {\itshape la prose précéda les vers}, et avoir montré dans l’{\itshape origine de la poésie}, telle que nous l’avons découverte, l’{\itshape origine des langues} et celle {\itshape des lettres}.
\section[{§ VI. Corollaires relatifs à la logique des esprits cultivés}]{§ VI. {\itshape Corollaires relatifs à la logique des esprits cultivés}}
\noindent 1. D’après tout ce que nous venons d’établir en vertu de cette {\itshape logique poétique} relativement à l’origine des langues, nous reconnaissons que c’est avec raison que les premiers auteurs du langage furent réputés {\itshape sages} dans tous les âges suivants, puisqu’ils donnèrent aux choses {\itshape des noms conformes à leur nature}, et remarquables par la {\itshape propriété}. Aussi nous avons vu que chez les Grecs et les Latins, {\itshape nom} et {\itshape nature} signifièrent souvent la même chose.\par
2. La {\itshape topique} commença avec la {\itshape critique}. La topique est l’art qui conduit l’esprit dans sa première opération, qui lui enseigne les aspects divers ({\itshape les lieux}, τόποι) que nous devons épuiser, en les observant successivement, pour connaître dans son entier l’objet que nous examinons. Les fondateurs  de la civilisation humaine se livrèrent à une {\itshape topique sensible}, dans laquelle ils unissaient les propriétés, les qualités ou rapports des individus ou des espèces, et les employaient tout concrets à former leurs {\itshape genres poétiques} ; de sorte qu’on peut dire avec vérité que le {\itshape premier âge} du monde s’occupa de la première opération de l’esprit.\par
Ce fut dans l’intérêt du genre humain que la Providence fit naître la {\itshape topique} avant la {\itshape critique}. Il est naturel de {\itshape connaître} d’abord les choses, et ensuite de les {\itshape juger}. La topique rend les esprits {\itshape inventifs}, comme la {\itshape critique} les rend {\itshape exacts}. Or, dans les premiers temps, les hommes avaient à trouver, à {\itshape inventer} toutes les choses nécessaires à la vie. En effet, quiconque y réfléchira, trouvera que les choses utiles ou nécessaires à la vie, et même celles qui ne sont que de commodité, d’agrément ou de luxe, avaient déjà été trouvées par les Grecs, avant qu’il y eût parmi eux des philosophes. Nous l’avons dit dans un axiome : {\itshape Les enfants sont grands imitateurs ; la poésie n’est qu’imitation ; les arts ne sont que des imitations de la nature, qu’une poésie réelle.} Ainsi, les premiers peuples qui nous représentent l’{\itshape enfance} du genre humain, fondèrent d’abord le monde des arts ; les philosophes, qui vinrent longtemps après, et qui nous en représentent la {\itshape vieillesse}, fondèrent le monde des sciences, qui compléta le système de la civilisation humaine.\par
3. Cette {\itshape histoire des idées humaines} est confirmée d’une manière singulière par l’{\itshape histoire de la}  {\itshape philosophie} elle-même. La première méthode d’une philosophie grossière encore fut l’αὐτοψία, ou {\itshape évidence des sens} ; nous avons vu, dans l’origine de la poésie, quelle vivacité avaient les sensations dans les âges poétiques. Ensuite vint Ésope, symbole des moralistes que nous appellerons vulgaires ; Ésope, antérieur aux sept sages de la Grèce, employa des {\itshape exemples} pour raisonnements ; et comme l’âge poétique durait encore, il tirait ces exemples de quelque fiction analogue, moyen plus puissant sur l’esprit du vulgaire, que les meilleurs raisonnements abstraits\footnote{Comme le prouve le succès avec lequel Ménénius Agrippa ramena à l’obéissance le peuple romain. ({\itshape Vico.})}. Après Ésope vint Socrate : il commença la dialectique par l’{\itshape induction}, qui conclut de plusieurs choses certaines à la chose douteuse qui est en question. Avant Socrate, la médecine, fécondant l’observation par l’induction, avait produit Hippocrate, le premier de tous les médecins pour le mérite comme pour l’époque, Hippocrate, auquel fut si bien dû cet éloge immortel, \emph{{\itshape nec fallit quemquam, nec falsus ab ullo est}}. Au temps de Platon, les mathématiques avaient, par la méthode de composition dite {\itshape synthèse}, fait d’immenses progrès dans l’école de Pythagore, comme on peut le voir par le Timée. Grâce à cette méthode, Athènes florissait alors par la culture de tous les arts qui font la gloire du génie humain, par la poésie, l’éloquence et l’histoire, par la musique et les arts du dessin. Ensuite vinrent  Aristote et Zénon ; le premier enseigna le {\itshape syllogisme}, forme de raisonnement qui n’unit point les idées particulières pour former des idées générales, mais qui décompose les idées générales dans les idées particulières qu’elles renferment ; quant au second, sa méthode favorite, celle du {\itshape sorite}, analogue à celle de nos modernes philosophes, n’aiguise l’esprit qu’en le rendant trop subtil. Dès lors la philosophie ne produisit aucun fruit remarquable pour l’avantage du genre humain. C’est donc avec raison que Bacon, aussi grand philosophe que profond politique, recommande l’{\itshape induction} dans son {\itshape Organum}. Les Anglais, qui suivent ce précepte, tirent de l’{\itshape induction} les plus grands avantages dans la philosophie expérimentale.\par
4. Cette {\itshape histoire des idées humaines} montre jusqu’à l’évidence l’erreur de ceux qui attribuant, selon le préjugé vulgaire, une haute sagesse aux anciens, ont cru que Minos, Thésée, Lycurgue, Romulus et les autres rois de Rome, donnèrent à leurs peuples des lois {\itshape universelles}. Telle est la forme des lois les plus anciennes, qu’elles semblent s’adresser à un seul homme ; d’un premier cas, elles s’étendaient à tous les autres, car {\itshape les premiers peuples étaient incapables d’idées générales} ; ils ne pouvaient les concevoir avant que les faits qui les appelaient se fussent présentés. Dans le procès du jeune Horace, la loi de Tullus Hostilius n’est autre chose que la sentence portée contre l’illustre accusé par les duumvirs qui avaient été créés par le roi pour ce  jugement\footnote{Selon Tite-Live, Tullus ne voulut point juger lui-même Horace, parce qu’il craignait de prendre sur lui l’odieux d’un tel jugement ; explication tout à fait ridicule. Tite-Live n’a pas compris que dans un sénat {\itshape héroïque}, c’est-à-dire, aristocratique, un roi n’avait d’autre puissance que celle de créer des duumvirs ou commissaires pour juger les accusés ; le peuple des cités héroïques ne se composait que de nobles auxquels l’accusé déjà condamné pouvait toujours en appeler. ({\itshape Vico.})}. Cette loi de Tullus est un {\itshape exemple}, dans le sens où l’on dit {\itshape châtiments exemplaires}. S’il est vrai, comme le dit Aristote, que \emph{{\itshape les républiques héroïques n’avaient pas de lois pénales}}, il fallait que les {\itshape exemples} fussent d’abord réels ; ensuite vinrent les exemples {\itshape abstraits}. Mais lorsque l’on eut acquis des idées générales, on reconnut que la propriété essentielle de la loi devait être l’{\itshape universalité} ; et l’on établit cette maxime de jurisprudence : \emph{{\itshape legibus, non exemplis est judicandum}}.
\chapterclose


\chapteropen
\chapter[{Chapitre IV. De la morale poétique, et de l’origine des vertus vulgaires qui résultèrent de l’institution de la religion et des mariages}]{Chapitre IV. \\
De la morale poétique, et de l’origine des vertus vulgaires qui résultèrent de l’institution de la religion et des mariages}

\chaptercont
\noindent  La {\itshape métaphysique des philosophes} commence par éclairer l’âme humaine, en y plaçant l’idée d’un Dieu, afin qu’ensuite la logique, la trouvant préparée à mieux distinguer ses idées, lui enseigne les méthodes de raisonnement, par le secours desquelles la morale purifie le cœur de l’homme. De même la {\itshape métaphysique poétique} des premiers humains les frappa d’abord par la crainte de Jupiter, dans lequel ils reconnurent le pouvoir de lancer la foudre, et terrassa leurs âmes aussi bien que leurs corps, par cette fiction effrayante. Incapables d’atteindre encore une telle idée par le raisonnement, ils la conçurent par un sentiment faux dans la {\itshape matière}, mais vrai dans la {\itshape forme}. De cette {\itshape logique} conforme à leur nature sortit la {\itshape morale poétique}, qui d’abord les rendit {\itshape pieux}.  La {\itshape piété} était la base sur laquelle la Providence voulait fonder les sociétés. En effet, chez toutes les nations, la piété a été généralement la mère des vertus domestiques et civiles ; la religion seule nous apprend à les observer, tandis que la philosophie nous met plutôt en état d’en discourir.\par
{\itshape La vertu commença par l’effort.} Les géants enchaînés sous les monts par la terreur religieuse que la foudre leur inspirait, {\itshape s’abstinrent} désormais d’errer à la manière des bêtes farouches dans la vaste forêt qui couvrait la terre, et prirent l’habitude de mener une vie sédentaire dans leurs retraites cachées, en sorte qu’ils devinrent plus tard les fondateurs des sociétés. Voilà l’un de {\itshape ces grands bienfaits que dut au ciel le genre humain}, selon la tradition vulgaire, {\itshape quand il régna sur la terre} par la religion des auspices. Par suite de ce premier {\itshape effort}, la vertu commença à poindre dans les âmes. Ils continrent leurs passions brutales, ils évitèrent de les satisfaire à la face du ciel qui leur causait un tel effroi, et chacun d’eux s’efforça d’entraîner dans sa caverne une seule femme dont il se proposait de faire sa compagne pour la vie. Ainsi la {\itshape Vénus humaine} succédant à la {\itshape Vénus brutale}, ils commencèrent à connaître la pudeur, qui, après la religion, est le principal lien des sociétés. Ainsi s’établit le {\itshape mariage}, c’est-à-dire {\itshape l’union charnelle faite selon la pudeur, et avec la crainte d’un Dieu}. C’est le second principe de la Science nouvelle, lequel dérive du premier (la croyance à une Providence).\par
 Le {\itshape mariage} fut accompagné de trois solennités. — La première est celle des auspices de Jupiter, auspices tirés de la foudre qui avait décidé les géants à les observer. De cette divination, {\itshape sortes}, les Latins définirent le mariage, {\itshape omnis vitæ consortium}, et appelèrent le mari et la femme, {\itshape consortes}. En italien, on dit vulgairement que la fille qui se marie {\itshape prende sorte}. Aussi est-ce un principe du droit des gens, que {\itshape la femme suive la religion publique de son mari}. — La seconde solennité consiste dans le voile dont la jeune épouse se couvre, en mémoire de ce premier mouvement de pudeur qui détermina l’institution des mariages. — La troisième, toujours observée par les Romains, fut d’enlever l’épouse avec une feinte violence, pour rappeler la violence véritable avec laquelle les géants entraînèrent les premières femmes dans leurs cavernes.\par
Les hommes se créèrent, sous le nom de {\itshape Junon}, un symbole de ces {\itshape mariages solennels}. C’est le premier de tous les symboles divins après celui de Jupiter…\par
\par
Considérons le genre de vertu que la religion donna à ces premiers hommes : ils furent {\itshape prudents}, de cette sorte de prudence que pouvaient donner les auspices de Jupiter ; {\itshape justes}, envers Jupiter, en le redoutant (Jupiter, {\itshape jus} et {\itshape pater}), et envers les hommes, en ne se mêlant point des affaires d’autrui ; c’est l’état des géants, tels que Polyphème les représente  à Ulysse, isolés dans les cavernes de la Sicile : cette justice n’était au fond que l’isolement de l’état sauvage. Ils pratiquaient la {\itshape continence}, en ce qu’ils se contentaient d’une seule femme pour la vie. Ils avaient le {\itshape courage}, l’{\itshape industrie}, la {\itshape magnanimité}, les vertus de l’âge d’or, pourvu que nous n’entendions point par {\itshape âge d’or}, ce qu’ont entendu dans la suite les poètes efféminés. Les vertus du premier âge, à la fois {\itshape religieuses} et {\itshape barbares}, furent analogues à celles qu’on a tant louées dans les Scythes, qui enfonçaient un couteau en terre, l’adoraient comme un dieu, et justifiaient leurs meurtres par cette religion sanguinaire.\par
Cette morale des nations superstitieuses et farouches du paganisme produisit chez elles l’usage de {\itshape sacrifier aux dieux des victimes humaines}. Lorsque les Phéniciens étaient menacés par quelque grande calamité, leurs rois immolaient à Saturne leurs propres enfants (Philon, Quinte-Curce). Carthage, colonie de Tyr, conserva cette coutume. Les Grecs la pratiquèrent aussi, comme on le voit par le sacrifice d’Iphigénie\footnote{On s’étonnera peu de ce dernier évènement si l’on songe à l’étendue illimité de la {\itshape puissance paternelle} des premiers hommes du paganisme, de ces Cyclopes de la fable. Cette puissance fut sans borne chez les nations les plus éclairées, telles que la grecque, chez les plus sages, telles que la romaine ; jusqu’aux temps de la plus haute civilisation, les pères y avaient le droit de faire périr leurs enfants nouveau-nés. C’est ce qui doit diminuer l’horreur que nous inspire, dans la douceur de nos temps modernes, la sévérité de Brutus, condamnant ses fils, et de Manlius faisant périr le sien pour avoir combattu et vaincu au mépris de ses ordres. ({\itshape Vico.})}. Les sacrifices humains étaient en usage  chez les Gaulois (César) et chez les Bretons (Tacite). Ce culte sacrilège fut défendu par Auguste aux Romains qui habitaient les Gaules, et par Claude aux Gaulois eux-mêmes (Suétone).\par
Les Orientalistes veulent que ce soient les Phéniciens qui aient répandu dans tout le monde les sacrifices de leur Moloch. Mais Tacite nous assure que les sacrifices humains étaient en usage dans la Germanie, contrée toujours fermée aux étrangers ; et les Espagnols les retrouvèrent dans l’Amérique, inconnue jusque-là au reste du monde.\par
Telle était la barbarie des nations à l’époque même où les {\itshape anciens Germains voyaient les dieux sur la terre}, où les {\itshape anciens Scythes}, où les {\itshape Américains}, brillaient de ces {\itshape vertus de l’âge d’or} exaltées par tant d’écrivains. Les victimes humaines sont appelées dans Plaute, {\itshape victimes de Saturne}, et c’est sous Saturne que les auteurs placent l’âge d’or du Latium ; tant il est vrai que cet âge fut celui de la douceur, de la bénignité et de la justice ! Rien n’est plus vain, nous devons le conclure de tout ce qui précède, que les fables débitées par les savants sur l’{\itshape innocence de l’âge d’or} chez les païens. Cette innocence n’était autre chose qu’une superstition fanatique qui, frappant les premiers hommes de la crainte des dieux que leur imagination avait créés, leur faisait observer quelque devoir malgré leur brutalité et leur orgueil farouche. Plutarque, choqué de cette superstition, met en problème s’il n’eût pas mieux valu ne croire aucune divinité, que  de rendre aux dieux ce culte impie. Mais il a tort d’opposer l’athéisme à cette religion, quelque barbare qu’elle pût être. Sous l’influence de cette religion se sont formées les plus illustres sociétés du monde ; l’athéisme n’a rien fondé.\par
Nous venons de traiter de la morale du premier âge, ou {\itshape morale divine} ; nous traiterons plus tard de la {\itshape morale héroïque}.
\chapterclose


\chapteropen
\chapter[{Chapitre V. Du gouvernement de la famille, ou économie, dans les âges poétiques}]{Chapitre V. \\
Du gouvernement de la famille, ou économie, dans les âges poétiques}

\chaptercont
\section[{§ I. De la famille composée des parents et des enfants, sans esclaves ni serviteurs}]{§ I. {\itshape De la famille composée des parents et des enfants, sans esclaves ni serviteurs}}
\noindent  Les héros {\itshape sentirent}, par l’instinct de la nature humaine, les deux vérités qui constituent toute la science économique, et que les Latins conservèrent dans les mots {\itshape educere, educare}, relatifs, l’un à l’éducation de l’âme, l’autre à celle du corps. Nous parlerons d’abord de {\itshape la première de ces deux éducations}.\par
Les premiers {\itshape pères} furent à la fois les {\itshape sages}, les {\itshape prêtres} et les {\itshape rois} ou {\itshape législateurs} de leurs familles\footnote{C’est cette tradition vulgaire sur la sagesse des anciens qui a trompé Platon, et lui a fait regretter \emph{{\itshape les temps où les philosophes régnaient, où les rois étaient philosophes}}. ({\itshape Vico.})}. Ils durent être dans la famille des {\itshape rois absolus}, supérieurs à tous les autres membres, et soumis seulement  à Dieu. Leur pouvoir fut armé des terreurs d’une religion effroyable, et sanctionné par les peines les plus cruelles ; c’est dans le caractère de Polyphème que Platon reconnaît les premiers pères de famille\footnote{Cette tradition mal interprétée a jeté tous les politiques dans l’erreur de croire que la {\itshape première forme des gouvernemens civils aurait été la monarchie}. Partant de cette erreur, ils ont établi pour principe de leur fausse science que {\itshape la royauté tirait son origine de la violence, ou de la fraude qui aurait bientôt éclaté en violence}. Mais à cette époque où les hommes avaient encore tout l’orgueil farouche de la liberté {\itshape bestiale}, cette simplicité grossière où ils se contentaient des productions spontanées de la nature pour aliments, de l’eau des fontaines pour boisson, et des cavernes pour abri pendant leur sommeil ; dans cette égalité naturelle où tous les pères étaient souverains de leur famille, on ne peut comprendre comment la fraude ou la force eussent assujéti tous les hommes à un seul. ({\itshape Vico.})}. — Remarquons seulement ici que les hommes, sortis de leur liberté native, et domptés par la sévérité du {\itshape gouvernement de la famille}, se trouvèrent préparés à obéir aux lois du {\itshape gouvernement civil} qui devait lui succéder. Il en est resté cette loi éternelle, que les républiques seront plus heureuses que celle qu’imagina Platon, toutes les fois que les pères de famille n’enseigneront à leurs enfants que la religion, et qu’ils seront admirés des fils comme leurs {\itshape sages}, révérés comme leurs {\itshape prêtres}, et redoutés comme leurs {\itshape rois}.\par
Quant à la {\itshape seconde partie de la science économique}, l’éducation des corps, on peut conjecturer que, par l’effet des terreurs religieuses, de la dureté du gouvernement des pères de famille, et des ablutions sacrées, les fils perdirent peu à peu la taille  des géants, et prirent la stature convenable à des hommes. Admirons la Providence d’avoir permis qu’avant cette époque les hommes fussent des géants : il leur fallait, dans leur vie vagabonde, une complexion robuste pour supporter l’inclémence de l’air et l’intempérie des saisons ; il leur fallait des forces extraordinaires pour pénétrer la grande forêt qui couvrait la terre, et qui devait être si épaisse dans les temps voisins du déluge....\par
La grande idée de la {\itshape science économique} fut réalisée dès l’origine, savoir : qu’il faut que les pères, par leur travail et leur industrie, laissent à leurs fils un patrimoine où ils trouvent une subsistance facile, commode et sûre, quand même ils n’auraient plus aucun rapport avec les étrangers, quand même toutes les ressources de l’état social viendraient à leur manquer, quand même il n’y aurait plus de cités ; de sorte qu’en supposant les dernières calamités les {\itshape familles subsistent}, comme {\itshape origine de nouvelles nations}. Ils doivent laisser ce patrimoine dans des lieux qui jouissent d’un {\itshape air sain}, qui possèdent des {\itshape sources} d’eaux vives, et dont la {\itshape situation} naturellement {\itshape forte} leur assure un asile dans le cas où les cités périraient ; il faut enfin que ce patrimoine comprenne de {\itshape vastes campagnes} assez riches pour nourrir les malheureux qui, dans la ruine des cités voisines, viendraient s’y {\itshape réfugier}, les cultiveraient, et en reconnaîtraient le propriétaire pour {\itshape seigneur}. Ainsi la Providence ordonna l’état de famille, employant non {\itshape la tyrannie des lois, mais la}  {\itshape douce autorité des coutumes} ({\itshape Voy.} axiome 104 le passage cité de Dion Cassius). Les {\itshape forts}, les puissants des premiers âges, établirent leurs habitations au sommet des montagnes. Le latin {\itshape arces}, l’italien {\itshape rocce}, ont, outre leur premier sens, celui de {\itshape forteresses}.\par
Tel fut l’ordre établi par la {\itshape Providence} pour commencer la société païenne. Platon en fait honneur à la {\itshape prévoyance} des premiers fondateurs des cités. Cependant, lorsque la barbarie antique reparaissant au moyen âge détruisait partout les cités, le même ordre assura le salut des {\itshape familles}, d’où sortirent les nouvelles nations de l’Europe. Les Italiens ont continué à dire {\itshape castella}, pour {\itshape seigneuries}. En effet, on observe généralement que les cités les plus anciennes, et presque toutes les capitales, ont été bâties au sommet des montagnes, tandis que les villages sont répandus dans les plaines. De là vinrent sans doute ces phrases latines, {\itshape summo loco, illustri loco nati}, pour dire les nobles ; {\itshape imo, obscuro loco nati}, pour désigner les plébéiens : les premiers habitaient les cités, les seconds les campagnes.\par
C’est par rapport aux {\itshape sources vives} dont nous avons parlé, que les politiques regardent la {\itshape communauté des eaux} comme l’occasion de l’union des familles. De là les premières {\itshape associations} furent dites par les Grecs φρατρίαι, (peut-être de φρέαρ, puits), comme les premiers {\itshape villages} furent appelés {\itshape pagi} par les Latins, du mot {\itshape πάγη}, fontaine. Les Romains célébraient les {\itshape mariages} par l’emploi  solennel de l’{\itshape eau} et du {\itshape feu} : parce que les premiers mariages furent contractés naturellement par des hommes et des femmes qui avaient l’{\itshape eau et le feu en commun}, comme membres de la même famille, et dans l’origine comme frères et sœurs. Le dieu du foyer de chaque maison était appelé {\itshape lar} ; d’où {\itshape focus laris}. C’était là que le père de famille sacrifiait aux dieux de la maison, \emph{{\itshape deivei parentum}} (loi des douze tables, {\itshape de parricidio}) ; comme parle l’Histoire sainte, \emph{{\itshape le Dieu de nos pères, le Dieu d’Abraham, d’Isaac, de Jacob}}. De là encore la loi que propose Cicéron, \emph{{\itshape sacra familiaria perpetua manento}} ; et les expressions si fréquentes dans les lois romaines, {\itshape filius familias in sacris paternis, sacra patria} pour la {\itshape puissance paternelle}. Ce respect du foyer domestique était commun aux barbares du moyen âge, puisque même au temps de Boccace, qui nous l’atteste dans sa {\itshape Généalogie des dieux}, c’était l’usage à Florence, qu’au commencement de chaque année, le père de famille assis à son foyer près d’un tronc d’arbre auquel il mettait le feu, jetait de l’encens et versait du vin dans la flamme ; usage encore observé, par le bas peuple de Naples, le soir de la vigile de Noël. On dit aussi {\itshape tant de feux}, pour tant de familles.\par
\par
L’institution des {\itshape sépultures}, qui vint après celle des {\itshape mariages}, résulta de la nécessité de cacher des objets qui choquaient les sens. Ainsi commença la croyance universelle de l’{\itshape immortalité des âmes humaines},  appelées {\itshape dii manes}, et dans la loi des douze tables, {\itshape deivei parentum}...\par
Les {\itshape philologues} et les {\itshape philosophes} ont pensé communément que dans ce qu’on appelle l’{\itshape état de nature}, les familles n’étaient composées que de {\itshape fils} ; elles le furent aussi de {\itshape serviteurs} ou {\itshape famuli}, d’où elles tirèrent principalement ce nom. Sur cette {\itshape économie} incomplète ils ont fondé une fausse {\itshape politique}, comme la suite doit le démontrer. Pour nous, nous commencerons à traiter de la {\itshape politique} des premiers âges, en prenant pour point de départ ces {\itshape serviteurs} ou {\itshape famuli}, qui appartiennent proprement à l’étude de l’{\itshape économie}.
\section[{§ II. Des familles composées de serviteurs, antérieures à l’existence des cités, et sans lesquelles cette existence était impossible}]{§ II. {\itshape Des familles composées de serviteurs, antérieures à l’existence des cités, et sans lesquelles cette existence était impossible}}
\noindent Au bout d’un laps de temps considérable, plusieurs des géants impies qui étaient restés dans la {\itshape communauté des femmes et des biens}, et dans les querelles qu’elle produisait, {\itshape les hommes simples et débonnaires}, dans le langage de Grotius, les {\itshape abandonnés de Dieu} dans celui de Pufendorf, furent contraints, pour échapper aux {\itshape violents} de Hobbes, de se réfugier aux autels des {\itshape forts}. Ainsi un froid très vif contraint les bêtes sauvages à venir chercher un asile dans les lieux habités. Les chefs de famille, plus courageux parce qu’ils avaient déjà formé une première société, recevaient sous leur  protection ces malheureux réfugiés, et tuaient ceux qui osaient faire des courses sur leurs terres. Déjà {\itshape héros par leur naissance}, puisqu’ils étaient nés de Jupiter, c’est-à-dire nés sous ses auspices, ils devinrent {\itshape héros par la vertu}. Dans ce dernier genre d’héroïsme, les Romains se montrèrent supérieurs à tous les peuples de la terre, puisqu’ils surent également\par

{\itshape Parcere subjectis, et debellare superbos}.\\

\noindent Les premiers hommes qui fondèrent la civilisation avaient été conduits à la société par la {\itshape religion} et par l’{\itshape instinct naturel de propager la race humaine}, causes honorables qui produisirent le mariage, {\itshape la première et la plus noble amitié du monde}. Les seconds qui entrèrent dans la société y furent contraints par {\itshape la nécessité de sauver leur vie}. Cette société dont l’{\itshape utilité} était le but, fut d’une {\itshape nature servile}. Aussi les réfugiés ne furent protégés par les héros qu’à une condition juste et raisonnable, celle {\itshape de gagner eux-mêmes leur vie en travaillant pour les héros, comme leurs serviteurs}. Cette condition analogue à l’esclavage fut le modèle de celle où l’on réduisit les prisonniers faits à la guerre après la formation des cités.\par
Ces premiers serviteurs se nommaient chez les Latins {\itshape vernæ}, tandis que les fils des héros, pour se distinguer, s’appelaient {\itshape liberi}. Du reste, ces derniers n’avaient aucune autre distinction : \emph{{\itshape dominum ac servum nullis educationis deliciis dignoscas}}. Ce  que Tacite dit des Germains peut s’entendre de tous les premiers peuples barbares ; et nous savons que chez les anciens Romains le père de famille avait droit de vie et de mort sur ses fils, et la propriété absolue de tout ce qu’ils pouvaient acquérir, au point que jusqu’aux Empereurs les fils et les esclaves ne différaient en rien sous le rapport du {\itshape pécule}. Ce mot {\itshape liberi} signifia aussi d’abord {\itshape nobles} : les arts {\itshape libéraux} sont les arts nobles ; {\itshape liberalis} répond à l’italien {\itshape gentile}. Chez les Latins les maisons nobles s’appelaient {\itshape gentes} ; ces premières {\itshape gentes} se composaient des seuls {\itshape nobles}, et les seuls {\itshape nobles} furent libres dans les premières cités.\par
Les serviteurs furent aussi appelés {\itshape clientes}, et ces {\itshape clientèles} furent la première image des fiefs, comme nous le verrons plus au long.\par
\par
Sous le {\itshape nom} seul du {\itshape père de famille} étaient compris tous ses {\itshape fils}, tous ses {\itshape esclaves} et {\itshape serviteurs}. Ainsi, dans les temps héroïques on put dire avec vérité, comme Homère le dit d’Ajax, \emph{{\itshape le rempart des Grecs} (πύργος Αχαιών)}, que seul il combattait contre l’armée entière des Troyens\footnote{Orthographié « Troiens » [NdE].} : on put dire qu’Horace soutint seul sur un pont le choc d’une armée d’Étrusques ; par quoi l’on doit entendre {\itshape Ajax, Horace, avec leurs compagnons ou serviteurs}. Il en fut précisément de même dans la {\itshape seconde barbarie} [dans celle du moyen âge] ; quarante héros normands, qui revenaient de la terre sainte, mirent en fuite une armée de Sarrasins qui tenaient Salerne assiégée.\par
 C’est à cette {\itshape protection} accordée par les héros à ceux qui se {\itshape réfugièrent} sur leurs terres, qu’on doit rapporter l’origine des {\itshape fiefs}. Les premiers furent d’abord des {\itshape fiefs roturiers personnels}, pour lesquels les {\itshape vassaux} étaient {\itshape vades}, c’est-à-dire obligés personnellement à suivre les héros partout où ils les menaient pour cultiver leurs terres, et plus tard, de les suivre dans les jugements ({\itshape rei} ; et {\itshape actores}). Du {\itshape vas} des Latins, du βας dérivèrent le {\itshape was} et le {\itshape wassus} employés par les feudistes barbares pour signifier {\itshape vassal}. Ensuite durent venir les {\itshape fiefs roturiers réels}, pour lesquels les vassaux durent être les premiers {\itshape prædes} ou {\itshape mancipes} obligés sur biens immeubles ; le nom de {\itshape mancipes} resta propre à ceux qui étaient ainsi obligés envers le trésor public.\par
\par
Nous venons de donner la première origine des {\itshape asiles}. C’est en ouvrant un asile que Cadmus fonde Thèbes, la plus ancienne cité de la Grèce. Thésée fonde Athènes en élevant l’{\itshape autel des malheureux}, nom bien convenable à ceux qui erraient auparavant, dénués de tous les biens divins et humains que la société avait procurés aux hommes pieux. Romulus fonde Rome en ouvrant un asile dans un bois, \emph{{\itshape vetus urbes condentium consilium}}, dit Tite-Live. De là Jupiter reçut le titre d’{\itshape hospitalier. Étranger} se dit en latin {\itshape hospes}.
\section[{§ III. Corollaires relatifs aux contrats qui se font par le simple consentement des parties}]{§ III. {\itshape Corollaires relatifs aux contrats qui se font par le simple consentement des parties}}
\noindent  Les nations héroïques, ne s’occupant que des choses nécessaires à la vie, ne recueillant d’autres fruits que les productions spontanées de la nature, ignorant l’usage de la monnaie, et étant pour ainsi dire {\itshape tout corps}, toute matière, ne pouvaient certainement connaître les contrats qui, selon l’expression moderne, se font {\itshape par le seul consentement}. L’ignorance et la grossièreté sont naturellement soupçonneuses ; aussi les hommes ne pouvaient connaître les engagements {\itshape de bonne foi}. Ils assuraient toutes les {\itshape obligations}, en employant la {\itshape main}, soit en réalité, soit par fiction en ajoutant à l’acte la garantie des {\itshape stipulations solennelles} ; de là ce titre célèbre dans la loi des douze tables : \emph{{\itshape Si quis nexum faciet mancipiumque, uti linguâ nuncupassit, ita jus esto}}. Un tel état civil étant supposé, nous pouvons en inférer ce qui suit.\par
I. On dit que dans les temps les plus anciens, les {\itshape achats} et les {\itshape ventes} se faisaient par {\itshape échange}, lors même qu’il s’agissait d’immeubles. Ces échanges ne furent autre chose que les cessions de terres faites au moyen âge, à charge de cens seigneurial ({\itshape livelli}). Leur utilité consistait en ce que l’une des parties avait trop de terres riches en fruits dont l’autre partie manquait.\par
 II. Les {\itshape locations de maisons} ne pouvaient avoir lieu lorsque les {\itshape cités} étaient petites, et les habitations étroites. On doit croire plutôt que les propriétaires fonciers donnaient du terrain pour qu’on y bâtît ; toute location se réduisait donc à un cens territorial.\par
III. Les {\itshape locations de terres} durent être emphytéotiques. Les grammairiens ont dit, sans en comprendre le sens, que {\itshape clientes} était {\itshape quasi colentes}. Ces locations de terres répondent aux {\itshape clientèles} des Latins.\par
IV. Telle fut sans doute la raison pour laquelle on ne trouve dans les anciennes archives du moyen âge, d’autres contrats que des {\itshape contrats de cens seigneurial} pour des maisons ou pour des terres, soit perpétuel, soit à temps.\par
V. Cette dernière observation explique peut-être pourquoi l’emphytéose est un {\itshape contrat de droit civil}, c’est-à-dire {\itshape du droit héroïque des Romains}. À ce droit héroïque Ulpien oppose le \emph{{\itshape droit naturel des peuples civilisés} ({\itshape gentium humanarum})} ; il les appelle {\itshape civilisés} ou {\itshape humains}, par opposition aux barbares des premiers temps ; et il ne peut entendre parler des {\itshape barbares} qui de son temps se trouvaient hors de l’Empire, et dont par conséquent le droit n’importait point aux jurisconsultes romains.\par
VI. Les {\itshape contrats de société} étaient inconnus, par un effet de l’isolement naturel des premiers hommes. Chaque père de famille s’occupait uniquement de ses affaires, sans se mêler de celles des autres, comme Polyphème le dit à Ulysse dans l’Odyssée.\par
 VII. Pour la même raison, il n’y avait point de {\itshape mandataires}. De là cette maxime qui est restée dans le droit civil : {\itshape nous ne pouvons acquérir par une personne qui n’est point sous notre puissance}, \emph{{\itshape per extraneam personam acquiri nemini}}.\par
VIII. Le droit des nations {\itshape civilisées, humanarum}, comme dit Ulpien, ayant succédé au droit des nations {\itshape héroïques}, il se fit une telle révolution, que le {\itshape contrat de vente}, qui anciennement ne produisait point d’action de garantie, si on n’avait point stipulé en cas d’éviction la cause pénale appelée {\itshape stipulatio duplæ}, est aujourd’hui le plus favorable de tous les contrats appelés {\itshape de bonne foi}, parce que naturellement elle doit y être observée sans qu’elle ait été promise.
\chapterclose


\chapteropen
\chapter[{Chapitre VI. De la politique poétique}]{Chapitre VI. \\
De la politique poétique}

\chaptercont
\section[{§ I. Origine des premières républiques, dans la forme la plus rigoureusement aristocratique}]{§ I. {\itshape Origine des premières républiques, dans la forme la plus rigoureusement aristocratique}}
\noindent  Les {\itshape familles} se formèrent donc de ces serviteurs ({\itshape famuli}) reçus sous la protection des héros. Nous avons déjà vu en eux les premiers membres d’une société politique ({\itshape socii}). Leur vie dépendait de leurs seigneurs, et par suite tout ce qu’ils pouvaient acquérir ; droit terrible que les héros exerçaient aussi sur leurs enfants\footnote{Aristote définit les fils, \emph{{\itshape des instrumens animés de leurs pères}} ; et jusqu’au temps où la constitution de Rome devint entièrement démocratique, les pères de famille conservèrent dans son intégrité cette monarchie domestique. Dans les premiers siècles, ils pouvaient vendre leurs fils jusqu’à trois fois. Plus tard lorsque la civilisation eut adouci les esprits, l’émancipation se fit par trois ventes fictives. Mais les Gaulois et les Celtes conservèrent toujours le même pouvoir sur leurs enfants et leurs esclaves. On a retrouvé les mêmes mœurs dans les Indes occidentales : les pères y vendaient réellement leurs enfants ; et en Europe les Moscovites et les Tartares peuvent exercer quatre fois le même droit. Tout ceci prouve combien les modernes se sont mépris sur le sens du mot célèbre : {\itshape les barbares n’ont point sur leurs enfants le même pouvoir que les citoyens romains}. Cette maxime des jurisconsultes anciens se rapporte aux nations vaincues par le peuple romain. La victoire leur ôtant tout droit {\itshape civil}, ainsi que nous le démontrerons, les vaincus conservaient seulement la puissance paternelle, donnée par la {\itshape nature}, les liens naturels du sang, {\itshape cognationes}, et d’un autre côté le {\itshape domaine naturel} ou {\itshape bonitaire} ; en tout cela leurs obligations étaient simplement {\itshape naturelles, de jure naturali gentium}, en ajoutant, avec Ulpien, {\itshape humanarum}. Mais pour les peuples indépendants de l’Empire, ces droits furent {\itshape civils}, et précisément les mêmes que ceux des citoyens romains. ({\itshape Vico.})}. Mais {\itshape les fils de famille} se trouvaient,  à la mort de leurs pères, affranchis de ce despotisme domestique, et l’exerçaient à leur tour sur leurs enfants. Dans le droit romain, tout citoyen affranchi de la {\itshape puissance paternelle}, est lui-même appelé {\itshape père de famille}. Les {\itshape serviteurs}, au contraire, étaient obligés de passer leur vie dans le même état de dépendance. Après bien des années, ils durent naturellement se lasser de leur condition, et se révolter contre les {\itshape héros}. Nous avons déjà indiqué dans les axiomes, d’une manière générale, que {\itshape les serviteurs avaient fait violence aux héros dans l’état de famille, et que cette révolution avait occasionné la naissance des républiques}. Dans une telle nécessité, les héros devaient être portés à s’unir en {\itshape corps politique}, pour résister à la multitude de leurs serviteurs révoltés, en mettant à leur tête l’un d’entre eux distingué par son courage et par sa présence d’esprit ; de tels chefs furent appelés {\itshape rois}, du mot {\itshape regere}, diriger. De cette manière, on peut dire avec Pomponius, \emph{{\itshape rebus ipsis dictantibus regna condita}} ;  pensée profonde, qui s’accorde bien avec le principe établi par la jurisprudence romaine : {\itshape le droit naturel des gens a été fondé par la Providence divine} (\emph{{\itshape jus naturale gentium divinâ Providentiâ constitutum}}). Les pères étant {\itshape rois et souverains} de leurs familles, il était impossible, dans la fière égalité de ces âges barbares, qu’aucun d’entre eux cédât à un autre ; ils formèrent donc des {\itshape sénats régnants}, c’est-à-dire {\itshape composés d’autant de rois des familles}, et, sans être conduits par aucune sagesse humaine, ils se trouvèrent avoir uni leurs intérêts privés dans un intérêt commun, que l’on appela {\itshape patria}, sous-entendu {\itshape res}, c’est-à-dire {\itshape intérêt des pères}. Les nobles, seuls citoyens des premières {\itshape patries}, se nommèrent {\itshape patriciens}. Dans ce sens, on peut regarder comme vraie la tradition selon laquelle {\itshape on ne consultait que la nature dans l’élection des rois des premiers âges}. Deux passages précieux de Tacite, qu’on lit dans les {\itshape Mœurs des Germains}, appuient cette tradition et nous donnent lieu de conjecturer que l’usage dont il parle était celui de tous les premiers peuples : \emph{{\itshape Non casus, non fortuita conglobatio turmam aut cuneum facit, sed familiæ et propinquitates ; duces exemplo potius quàm imperio, si prompti, si conspicui, si ante aciem agant, admiratione præsunt.}} Tels furent les premiers {\itshape rois}. Ce qui le prouve, c’est que les poètes n’imaginèrent pas autrement Jupiter, {\itshape le roi des hommes et des dieux}. On le voit dans Homère s’excuser auprès de Thétis de n’avoir pu contrevenir à ce que les dieux avaient une fois  déterminé dans le grand conseil de l’Olympe. N’est-ce pas là le langage qui convient au roi d’une aristocratie ? En vain les stoïciens voudraient nous présenter ici {\itshape Jupiter} comme {\itshape soumis à leur destin} ; Jupiter et tous les dieux ont tenu conseil sur les choses humaines, et les ont par conséquent déterminées par l’effet d’une {\itshape volonté libre}. Ce passage nous en explique deux autres, où les politiques croient à tort qu’Homère désigne la {\itshape monarchie} : c’est lorsque Agamemnon veut abaisser la fierté d’Achille, et qu’Ulysse persuade aux Grecs, qui se soulèvent pour retourner dans leur patrie, de continuer le siège de Troie. Dans les deux passages, il est dit qu’{\itshape un seul est roi} : mais dans l’un et l’autre il s’agit de la {\itshape guerre}, dans laquelle il faut toujours un seul chef, selon la maxime de Tacite : \emph{{\itshape eam esse imperandi conditionem, ut non aliter ratio constet, quant si uni reddatur}}. Du reste, partout où Homère fait mention des héros, il leur donne l’épithète de {\itshape rois} ; ce qui se rapporte à merveille au passage de la Genèse où Moïse, énumérant les descendants d’Ésaü, les appelle tous rois, {\itshape duces} (c’est-à-dire capitaines) dans la Vulgate. Les ambassadeurs de Pyrrhus lui rapportèrent qu’ils avaient vu à Rome un {\itshape sénat de rois}.\par
Sans l’hypothèse d’une révolte de {\itshape serviteurs}, on ne peut comprendre comment les {\itshape pères} auraient consenti à assujettir leurs monarchies domestiques à la souveraineté de l’ordre dont ils faisaient partie. C’est la nature des hommes courageux (axiome 81) de sacrifier le moins qu’ils peuvent de ce qu’ils ont  acquis par leur courage, et seulement autant qu’il est nécessaire pour conserver le reste. Aussi voyons-nous souvent dans l’histoire romaine combien les héros rougissaient {\itshape virtute parta per flagitium amittere}. Du moment qu’il est établi (nous l’avons démontré et nous le démontrerons mieux encore) que les gouvernements ne sont point nés de la fraude, ni de la violence d’un seul, peut-on, en embrassant tous les cas humainement possibles, imaginer d’une autre manière comment le {\itshape pouvoir civil} se forma par la réunion du {\itshape pouvoir domestique} des pères de famille, et comment le {\itshape domaine éminent} des gouvernements résulta de l’ensemble des {\itshape domaines naturels}, que nous avons déjà indiqués comme ayant été {\itshape ex jure optimo}, c’est-à-dire libres de toute charge publique ou particulière ?\par
Les héros ainsi réunis en corps politique, et investis à la fois du pouvoir sacerdotal et militaire, nous apparaissent dans la Grèce sous le nom d’{\itshape Héraclides}, dans l’ancienne Italie, dans la Crète et dans l’Asie Mineure, sous celui de {\itshape Curètes}. Leurs réunions furent les comices {\itshape curiata}, les plus anciens dont fasse mention l’histoire romaine. Sans doute on y assistait d’abord les armes à la main. Dans la suite, on n’y délibérait plus que sur les choses sacrées, dont les choses profanes avaient elles-mêmes emprunté le caractère dans les premiers temps. Tite-Live s’étonne de ce qu’au passage d’Annibal, de pareilles assemblées se tenaient dans les Gaules ; mais nous voyons dans Tacite, que chez ce  peuple les prêtres tenaient des assemblées analogues, \emph{{\itshape dans lesquelles ils ordonnaient les punitions, comme si les dieux eussent été présents}}. Il était raisonnable que les héros se rendissent en armes à ces réunions, où l’on ordonnait le châtiment des coupables : la souveraineté des lois est une dépendance de la souveraineté des armes. Tacite dit aussi en général que les Germains traitaient tout armés des affaires publiques sous la présidence de leurs prêtres. On peut conjecturer qu’il en fut de même de tous les premiers peuples barbares.\par
D’après tout ce qu’on vient de dire, le droit des {\itshape Quirites} ou {\itshape Curètes} dut être le {\itshape droit naturel} des gens ou nations {\itshape héroïques} de l’Italie. Les Romains, pour distinguer leur droit de celui des autres peuples, l’appelèrent {\itshape jus Quiritium romanorum}. Si cette dénomination avait eu pour origine la convention des Sabins et des Romains, si les seconds eussent tiré leur nom de {\itshape Cure}, capitale des premiers, ce nom eût été {\itshape Cureti} et non {\itshape Quirites} ; et si cette capitale des Sabins se fût appelée {\itshape Cere}, comme le veulent les grammairiens latins, le mot dérivé eût été {\itshape Cerites}, expression qui désignait les citoyens condamnés par les censeurs à porter les charges publiques sans participer aux honneurs.\par
Ainsi les premières cités n’eurent pour citoyens que des nobles qui les gouvernaient. Mais ils n’auraient eu personne à qui commander, si l’intérêt commun ne les eût décidés à satisfaire leurs clients révoltés, et à leur accorder la {\itshape première loi agraire}  qu’il y ait eu au monde. Afin de ne sacrifier que le moins possible de leurs privilèges, les héros ne leur accordèrent que le {\itshape domaine bonitaire} des champs qu’ils leur assignaient. C’est une loi du droit naturel des gens, que le {\itshape domaine} suit la {\itshape puissance}. Or les serviteurs ne jouissant d’abord de la vie que d’une manière précaire dans les asiles ouverts par les héros, il était conforme au droit et à la raison qu’ils eussent aussi un {\itshape domaine} précaire, et qu’ils en jouissent tant qu’il plairait aux héros de leur conserver la possession des champs qu’ils leur avaient assignés. Ainsi les serviteurs devinrent les premiers plébéiens ({\itshape plebs}) des cités héroïques, où ils n’avaient aucun privilège de citoyen. Lorsque Achille se voit enlever Briséis par Agamemnon, \emph{{\itshape c’est}, dit-il, {\itshape  un outrage que l’on ne ferait pas à un journalier qui n’a aucun droit de citoyen}}. Tels furent les {\itshape plébéiens} de Rome jusqu’à l’époque de la lutte dans laquelle ils arrachèrent aux patriciens le {\itshape droit des mariages}. La loi des douze tables avait été pour eux une seconde loi agraire par laquelle les nobles leur accordaient le {\itshape domaine quiritaire} des champs qu’ils cultivaient ; mais, puisqu’en vertu du droit des gens, les étrangers étaient capables du {\itshape domaine civil}, les plébéiens qui avaient la même capacité n’étaient point encore citoyens, et à leur mort ils ne pouvaient laisser leurs champs à leurs familles, ni {\itshape ab intestat}, ni {\itshape par testament}, parce qu’ils n’avaient pas les droits de {\itshape suite}, d’{\itshape agnation}, de {\itshape gentilité}, qui dépendaient des {\itshape mariages solennels} ; les champs  assignés aux plébéiens retournaient à {\itshape leurs auteurs}, c’est-à-dire aux nobles. Aussi aspirèrent-ils à partager les privilèges des mariages solennels ; non que, dans cet état de misère et d’esclavage, ils élevassent leur ambition jusqu’à s’allier aux familles des nobles, ce qui se serait appelé {\itshape connubia cum patribus}. Ils demandèrent seulement {\itshape connubia patrum}, c’est-à-dire la faculté de contracter les mariages solennels, tels que ceux des {\itshape pères}. La principale solennité de ces mariages était les auspices publics (\emph{{\itshape auspicia majora}}, selon Messala et Varron), ces auspices que les {\itshape pères} revendiquaient comme leur privilège ({\itshape auspicia esse sua}). Demander le {\itshape droit des mariages}, c’était donc demander le {\itshape droit de cité}, dont ils étaient le principe naturel ; cela est si vrai, que le jurisconsulte Modestinus définit le mariage de la manière suivante : \emph{{\itshape omnis divini et humani juris communicatio}}. Comment définirait-on avec plus de précision le droit de cité lui-même ?
\section[{§ II. Les sociétés politiques sont nées toutes de certains principes éternels des fiefs}]{§ II. {\itshape Les sociétés politiques sont nées toutes de certains principes éternels des fiefs}}
\noindent Conformément aux principes éternels des fiefs que nous avons placés dans nos axiomes (80, 81), il y eut dès la naissance des sociétés trois espèces de propriétés ou {\itshape domaines}, relatives à trois espèces de {\itshape fiefs}, que trois classes de {\itshape personnes} possédèrent sur trois sortes de {\itshape choses} : 1º {\itshape domaine bonitaire} des fiefs  roturiers [ou {\itshape humains}, en prenant le mot d’{\itshape homme}, comme au moyen âge, dans le sens de {\itshape vassal}] ; c’est la propriété des fruits que les {\itshape hommes}, ou {\itshape plébéiens}, ou {\itshape clients}, ou {\itshape vassaux}, tiraient des terres des {\itshape héros, patriciens} ou {\itshape nobles}. 2º {\itshape Domaine quiritaire} des fiefs nobles, ou {\itshape héroïques}, ou militaires, que les héros se réservèrent sur leurs terres, comme droit de souveraineté. Dans la formation des républiques héroïques, ces fiefs souverains, ces souverainetés privées s’assujettirent naturellement à la {\itshape haute souveraineté des ordres héroïques régnants}. 3º {\itshape Domaine civil}, dans toute la propriété du mot. Les pères de famille avaient reçu les terres de la divine Providence, comme une sorte de fiefs {\itshape divins} ; {\itshape souverains} dans l’état de famille, ils formèrent par leur réunion les {\itshape ordres régnants} dans l’état de cités. Ainsi prirent naissance les {\itshape souverainetés civiles}, soumises à Dieu seul. Toutes les puissances souveraines reconnaissent la Providence, et ajoutent à leurs titres de majesté, {\itshape par la grâce de Dieu} ; elles doivent en effet avouer publiquement que c’est de lui qu’elles tiennent leur autorité, puisque, si elles défendaient de l’adorer, elles tomberaient infailliblement. Jamais il n’y eut au monde une nation d’{\itshape athées}, de {\itshape fatalistes}, ni d’{\itshape hommes qui rapportassent tous les événements au hasard}.\par
En vertu de ce droit de {\itshape domaine éminent} donné aux puissances civiles par la Providence, {\itshape elles sont maîtresses du peuple et de tout ce qu’il possède}. Elles peuvent disposer des personnes, des biens et  du travail, elles peuvent imposer des taxes et des tributs, lorsqu’elles ont à exercer ce droit que j’appelle {\itshape domaine du fond public} ({\itshape dominio de’ fundi}), et que les écrivains qui traitent du droit public appellent {\itshape domaine éminent}. Mais les souverains ne peuvent l’exercer que pour conserver l’état dans sa {\itshape substance}, comme dit l’École, parce qu’à sa conservation ou à sa ruine tiennent la ruine ou la conservation de tous les intérêts particuliers.\par
Les Romains ont connu, au moins par une sorte d’instinct, cette formation des républiques d’après les principes éternels des fiefs. Nous en avons la preuve dans la formule de la revendication : \emph{{\itshape aio hunc fundum meum esse ex jure Quiritium}}. Ils attachaient cette action {\itshape civile} au {\itshape domaine du fond} qui dépend de la {\itshape cité} et dérive de la {\itshape force} pour ainsi dire {\itshape centrale} qui lui est propre. C’est par elle que tout citoyen romain est seigneur de sa terre par un {\itshape domaine indivis} (par une pure {\itshape distinction de raison}, comme dirait l’École). De là l’expression {\itshape ex jure Quiritium ; Quirites}, ainsi qu’on l’a vu, signifiait d’abord les Romains armés de lances dans les réunions publiques qui constituaient la cité. Telle est la raison inconnue jusqu’ici pour laquelle les fonds et tous les biens vacants {\itshape reviennent} au fisc, c’est que tout patrimoine particulier est patrimoine public par indivis ; tout propriétaire particulier manquant, le patrimoine particulier n’est plus désigné comme {\itshape partie}, et se trouve confondu avec la masse du {\itshape tout}. D’après la loi {\itshape Papia Poppea} (Des déshérences), le patrimoine  du célibataire sans parents {\itshape revenait} au fisc, non comme héritage, mais comme pécule, {\itshape ad populum}, dit Tacite, \emph{{\itshape tanquam omnium parentem}}……\par
Les premières cités se composèrent d’un {\itshape ordre} de nobles et d’une {\itshape foule} de peuples. De l’opposition de ces éléments résulta une loi éternelle, c’est que les plébéiens veulent toujours {\itshape changer l’état des choses}, les nobles {\itshape le maintenir} ; aussi dans les mouvements politiques donne-t-on le nom d’{\itshape optimates} à tous ceux qui veulent maintenir l’ancien état des choses, (d’{\itshape ops}, secours, puissance, entraînant une idée de stabilité).\par
Ici nous voyons naître une double division : 1. La première, des {\itshape sages} et du {\itshape vulgaire}. Les héros avaient fondé les états par la {\itshape sagesse des auspices}. C’est relativement à cette division, que le vulgaire conserva l’épithète de {\itshape profane}, les nobles ou héros étant les prêtres des cités héroïques. Chez les premiers peuples, on ôtait le droit de cité par une sorte d’excommunication ({\itshape aquâ et igne interdicebantur}). 2. La seconde division fut celle de {\itshape civis}, citoyen, et {\itshape hostis}, hôte, étranger, ennemi ; les premières cités se composaient des héros et de ceux auxquels ils avaient donné asile. Les \emph{{\itshape héros}, selon Aristote, {\itshape  juraient une éternelle inimitié}} aux plébéiens, {\itshape hôtes} des cités héroïques\footnote{L’hospitalité héroïque entraîna aussi dans d’autres occasions l’idée d’inimitié : Pâris fut hôte d’Hélène, Thésée d’Ariane, Jason de Médée, Énée de Didon ; ces enlèvemens, ces trahisons étaient des actions {\itshape héroïques}. ({\itshape Vico.})}.
\section[{§ III. De l’origine du cens et du trésor public (ærarium, chez les Romains)}]{§ III. {\itshape De l’origine du cens et du trésor public (}ærarium{\itshape , chez les Romains)}}
\noindent  Dans les anciennes républiques, le {\itshape cens} consistait en une redevance que les plébéiens payaient aux nobles pour les terres qu’ils tenaient d’eux. Ainsi le cens des Romains, dont on rapporte l’établissement à Servius Tullius, fut dans le principe une institution aristocratique.\par
Les plébéiens avaient encore à supporter les usures intolérables des nobles, et les usurpations fréquentes qu’ils faisaient de leurs champs ; au point que, si l’on en croit les plaintes de Philippe, tribun du peuple, deux mille nobles finirent par posséder toutes les terres qui auraient dû être divisées entre trois cent mille citoyens. Environ quarante ans après l’expulsion de Tarquin le Superbe, la noblesse, rassurée par sa mort, commença à faire sentir sa tyrannie au pauvre peuple, et le sénat paraît avoir ordonné alors que les plébéiens paieraient au trésor public le {\itshape cens} qu’auparavant ils payaient à chacun des nobles, afin que le trésor pût fournir à leurs dépenses dans la guerre. Depuis cette époque, nous voyons le {\itshape cens} reparaître dans l’histoire romaine. Tite-Live prétend que les nobles \emph{{\itshape dédaignaient de présider au cens}} ; il n’a pas compris qu’ils repoussaient cette institution. Ce n’était plus le cens institué par Servius Tullius, lequel avait été le fondement de l’aristocratie. Les nobles, par leur propre  avarice, avaient déterminé l’institution du nouveau cens, qui devint, avec le temps, le principe de la démocratie.\par
L’inégalité des propriétés dut produire de grands mouvements, des révoltes fréquentes de la part du petit peuple. Fabius mérita le surnom de Maximus, pour les avoir apaisés par sa sagesse, en ordonnant que tout le peuple romain fût divisé en trois classes (sénateurs, chevaliers, et plébéiens), dans lesquelles les citoyens se placeraient selon leurs facultés. Auparavant, l’ordre des sénateurs, composé entièrement de nobles, occupait seul les magistratures ; les plébéiens riches purent entrer dans cet ordre. Ils oublièrent leurs maux en voyant que la route des honneurs leur était ouverte désormais. C’est ce changement, c’est la loi Publilia, qui établirent la démocratie dans Rome, et non la loi des douze tables, qu’on aurait apportée d’Athènes. Aussi Tite-Live, tout ignorant qu’il est de ce qui regarde la constitution ancienne de Rome, nous raconte que les nobles se plaignaient d’avoir plus perdu par la loi Publilia, que gagné par toutes les victoires qu’ils avaient remportées la même année\footnote{Bernardo Segni, traduit ce qu’Aristote appelle une république démocratique, par \emph{{\itshape republica per censo}}. ({\itshape Vico.})}.\par
Dans la démocratie, où le peuple entier constitue la cité, il arriva que le {\itshape domaine civil} ne fut plus ainsi appelé dans le sens de {\itshape domaine public}, quoiqu’il eût été appelé {\itshape civil} du mot de {\itshape cité}. Il se divisa entre tous les {\itshape domaines privés} des citoyens  romains dont la réunion constituait la cité romaine. {\itshape Dominium optimum} signifia bien une pleine propriété, mais non plus {\itshape domaine par excellence} (domaine {\itshape éminent}). Le {\itshape domaine quiritaire} ne signifia plus un {\itshape domaine} dont le plébéien ne pouvait être expulsé sans que le noble dont il le tenait vînt pour le défendre et le maintenir en possession ; il signifia un {\itshape domaine privé} avec faculté de {\itshape revendication}, à la différence du {\itshape domaine bonitaire}, qui se maintient par la seule possession.\par
Les mêmes changements eurent lieu au moyen âge, en vertu des lois qui dérivent de la {\itshape nature éternelle des fiefs}. Prenons pour exemple le royaume de France, dont les provinces furent alors autant de souverainetés appartenant aux seigneurs qui relevaient du roi. Les biens des seigneurs durent originairement n’être sujets à aucune charge publique. Plus tard, par successions, par déshérences ou par confiscation pour rébellion, ils furent incorporés au royaume, et cessant d’être {\itshape ex jure optimo}, devinrent sujets aux charges publiques. D’un autre côté, les châteaux et les terres qui composaient le domaine particulier des rois, ayant passé, par mariage ou par concession, à leurs vassaux, se trouvent aujourd’hui assujettis à des taxes et à des tributs. Ainsi, dans les royaumes soumis à la même loi de succession, le domaine {\itshape ex jure optimo} se confondit peu à peu avec le {\itshape domaine privé}, sujet aux charges publiques, de même que le {\itshape fisc}, patrimoine des Empereurs, alla se confondre avec le trésor ou {\itshape ærarium}.
\section[{§ IV. De l’origine des comices chez les Romains}]{§ IV. {\itshape De l’origine des comices chez les Romains}}
\noindent  Les deux sortes d’{\itshape assemblées héroïques} distinguées dans Homère, βουλή, ἀγορά, devaient répondre aux {\itshape comices par curies}, qui furent les premières assemblées des Romains, et à leurs comices {\itshape par tribus}. Les premiers furent dits {\itshape curiata} ({\itshape comitia}), de {\itshape quir, quiris}, lance\footnote{De même que les Grecs, du mot χείρ, la main, qui par extension signifie aussi puissance chez toutes les nations, tirèrent celui de χύρια, dans un sens analogue à celui du latin {\itshape curia}. ({\itshape Vico.})}. Les {\itshape quirites, cureti}, hommes armés de lances, et investis du droit sacerdotal des augures, paraissaient seuls aux comices {\itshape curiata}.\par
Depuis que Fabius Maximus eut distribué les citoyens selon leurs biens, en trois classes, {\itshape sénateurs, chevaliers}, et {\itshape plébéiens}, les nobles ne formèrent plus un ordre dans la cité, et se partagèrent, selon leur fortune, entre les trois classes. Dès lors on distingua le {\itshape patricien} du {\itshape sénateur} et du {\itshape chevalier}, le {\itshape plébéien} de l’{\itshape homme sans naissance} ({\itshape ignobilis}) ; {\itshape plébéien} ne fut plus opposé à {\itshape patricien}, mais à {\itshape sénateur} ou {\itshape chevalier} ; ce mot désigna un citoyen {\itshape pauvre}, quelque {\itshape noble} qu’il pût être ; {\itshape sénateur}, au contraire, ne fut plus synonyme de {\itshape patricien}, mais il désigna le citoyen {\itshape riche}, même {\itshape sans naissance}. Depuis cette époque, on appela {\itshape comices par centuries} les assemblées dans lesquelles tout le peuple romain se réunissait dans ses trois classes pour décider des affaires publiques, et particulièrement pour voter sur les {\itshape lois consulaires}. Dans les {\itshape comices par tribus}, le peuple  continua à voter sur les {\itshape lois tribunitiennes} ou {\itshape plébiscites} [ce qui pendant longtemps n’avait signifié que : lois communiquées au peuple, lois publiées devant les plébéiens, {\itshape plebi scita} ou {\itshape nota}, telle que la loi de l’éternelle expulsion des Tarquins, promulguée par Junius Brutus]. Pour la régularité des cérémonies religieuses, les comices par curies, où l’on traitait des choses sacrées, furent toujours les {\itshape assemblées des seuls chefs des curies} ; au temps des rois, où ces assemblées commencèrent, on y traitait de toutes les choses {\itshape profanes} en les considérant comme {\itshape sacrées}.
\section[{§ V. Corollaire. C’est la divine Providence qui règle les sociétés, et qui a fondé le droit naturel des gens}]{§ V. {\itshape Corollaire. C’est la divine Providence qui règle les sociétés, et qui a fondé le droit naturel des gens}}
\noindent En voyant les sociétés naître ainsi dans l’{\itshape âge divin}, avec le gouvernement {\itshape théocratique}, pour se développer sous le gouvernement {\itshape héroïque}, qui conserve l’esprit du premier, on éprouve une admiration profonde pour la sagesse avec laquelle la Providence conduisit l’homme à un but tout autre que celui qu’il se proposait, lui imprima la crainte de la Divinité, et {\itshape fonda la société sur la religion}. La religion arrêta d’abord les géants dans les terres qu’ils occupèrent les premiers, et cette prise de possession fut l’origine de tous les droits de propriété, de tous les {\itshape domaines}. Retirés au sommet des monts, ils y trouvèrent, pour fixer leur vie errante, des  lieux salubres, forts de situation, et pourvus d’eau, trois circonstances indispensables pour élever des cités. C’est encore la religion qui les détermina à former une union régulière et aussi durable que la vie, celle du {\itshape mariage}, d’où nous avons vu dériver le pouvoir paternel, et par suite tous les pouvoirs. Par cette union ils se trouvèrent avoir fondé les {\itshape familles}, berceau des sociétés politiques. Enfin, en ouvrant les {\itshape asiles}, ils donnèrent lieu aux {\itshape clientèles}, qui, par suite de la {\itshape première loi agraire} dont nous avons parlé, devaient produire les {\itshape cités}. Composées d’un ordre de nobles qui commandaient, et d’un ordre de plébéiens nés pour obéir, les cités eurent d’abord un gouvernement {\itshape aristocratique}. Rien ne pouvait être plus conforme à la nature sauvage et solitaire de ces premiers hommes, puisque l’esprit de l’aristocratie est la conservation des limites qui séparent les différents ordres au-dedans, les différents peuples au-dehors. Grâce à cette forme de gouvernement, les nations nouvellement entrées dans la civilisation, devaient rester longtemps sans communication extérieure, et oublier ainsi l’état sauvage et bestial d’où elles étaient sorties. Les hommes n’ayant encore que des idées très particulières, et ne pouvant comprendre ce que c’est que le {\itshape bien commun}, la Providence sut, au moyen de cette forme de gouvernement, les conduire à s’unir à leur patrie, dans le but de conserver un objet d’intérêt privé, aussi important pour eux que leur {\itshape monarchie domestique} ; de cette manière, sans aucun dessein, ils  s’accordèrent dans cette généralité du bien social, qu’on appelle {\itshape république}.\par
Maintenant recourons à ces {\itshape preuves divines} dont on a parlé dans le chapitre de la Méthode ; examinons combien sont naturels et simples les moyens par lesquels la Providence a dirigé la marche de l’humanité, rapprochons-en le nombre infini des phénomènes qui se rapportent aux quatre causes dans lesquelles nous verrons partout les éléments du monde social (les {\itshape religions}, les {\itshape mariages}, les {\itshape asiles} et la {\itshape première loi agraire}), et cherchons ensuite entre tous les cas humainement possibles, si des choses si nombreuses et si variées ont pu avoir des origines plus simples et plus naturelles. Au moment où les sociétés devaient naître, les {\itshape matériaux}, pour ainsi parler, n’attendaient plus que la {\itshape forme}. J’appelle {\itshape matériaux} les religions, les langues, les terres, les mariages, les noms propres et les armes ou emblèmes, enfin les magistratures et les lois. Toutes ces choses furent d’abord {\itshape propres} à l’individu, {\itshape libres} en cela même qu’elles étaient individuelles, et, parce qu’elles étaient libres, capables de constituer de véritables républiques. Ces religions, ces langues, etc., avaient été propres aux premiers hommes, monarques de leur famille. En formant par leur union des corps politiques, ils donnèrent naissance à la {\itshape puissance civile}, puissance souveraine, de même que dans l’état précédent celle des pères sur leurs familles n’avait relevé que de Dieu. Cette {\itshape souveraineté civile}, considérée comme  une personne, eut son {\itshape âme} et son {\itshape corps} : l’{\itshape âme} fut une compagnie de sages, tels qu’on pouvait en trouver dans cet état de simplicité, de grossièreté. Les plébéiens représentèrent le {\itshape corps}. Aussi est-ce une loi éternelle dans les sociétés, que les uns y doivent tourner leur esprit vers les travaux de la politique, tandis que les autres appliquent leur corps à la culture des arts et des métiers. Mais c’est aussi une loi que l’{\itshape âme} doit toujours y commander, et le {\itshape corps} toujours servir.\par
Une chose doit augmenter encore notre admiration. La Providence, en faisant naître les familles, qui, sans connaître le Dieu véritable, avaient au moins quelque notion de la Divinité, en leur donnant une religion, une langue, etc., qui leur fussent propres, avait déterminé l’existence d’un {\itshape droit naturel des familles}, que les {\itshape pères} suivirent ensuite dans leurs rapports avec leurs {\itshape clients}. En faisant naître les républiques sous une forme aristocratique, elle transforma le {\itshape droit naturel des familles}, qui s’était observé dans l’état de nature, en {\itshape droit naturel des gens}, ou des peuples. En effet, les pères de famille qui s’étaient réservé leur religion, leur langue, leur législation particulière à l’exclusion de leurs clients, ne purent se séparer ainsi sans attribuer ces privilèges aux ordres souverains dans lesquels ils entrèrent ; c’est en cela que consista la {\itshape forme si rigoureusement aristocratique des républiques héroïques}. De cette manière, le {\itshape droit des gens} qui s’observe maintenant entre les nations, fut, à  l’origine des sociétés, une sorte de privilège pour les puissances souveraines. Aussi le peuple où l’on ne trouve point une puissance souveraine investie de tels droits, n’est point un peuple à proprement parler, et ne peut traiter avec les autres d’après les lois du droit des gens ; une nation supérieure exercera ce droit pour lui.
\section[{§ VI. Suite de la politique héroïque}]{§ VI. {\itshape Suite de la politique héroïque}}
\noindent Tous les historiens commencent l’{\itshape âge héroïque} avec les courses navales de Minos et l’expédition des Argonautes ; ils en voient la continuation dans la guerre de Troie, la fin dans les courses errantes des héros, qu’ils terminent au retour d’Ulysse. C’est alors que dut naître Neptune, le dernier des douze grands dieux. La marine est, à cause de sa difficulté, l’un des derniers arts que trouvent les nations. Nous voyons dans l’Odyssée que, lorsque Ulysse aborde sur une nouvelle terre, il monte sur quelque colline pour voir s’il découvrira la fumée qui annonce les habitations des hommes. D’un autre côté, nous avons cité dans les axiomes ce que dit Platon sur l’\emph{{\itshape horreur que les premiers peuples éprouvèrent longtemps pour la mer}}. Thucydide en explique la raison en nous apprenant que \emph{{\itshape la crainte des pirates empêcha longtemps les peuples grecs d’habiter sur les rivages}}. Voilà pourquoi Homère arme la main de Neptune du \emph{{\itshape trident qui fait trembler la terre}}. Ce trident n’était qu’un croc pour arrêter les  barques ; le poète l’appelle {\itshape dent} par une belle métaphore, en ajoutant une particule qui donne au mot le sens superlatif.\par
Dans ces vaisseaux de pirates nous reconnaissons le {\itshape taureau}, sous la forme duquel Jupiter enlève Europe ; le {\itshape Minotaure}, ou taureau de Minos, avec lequel il enlevait les jeunes garçons et les jeunes filles des côtes de l’Attique. Les antennes s’appelaient {\itshape cornua navis}. Nous y voyons encore le {\itshape monstre} qui doit dévorer Andromède, et le {\itshape cheval ailé} sur lequel Persée vient la délivrer. Les {\itshape voiles} du vaisseau furent appelées ses {\itshape ailes, alarum remigium}. Le {\itshape fil} d’Ariane est l’art de la navigation, qui conduit Thésée à travers le {\itshape labyrinthe} des îles de la mer Égée.\par
Plutarque, dans sa Vie de Thésée, dit que les {\itshape héros} tenaient à grand honneur le nom de {\itshape brigand}, de même qu’au moyen âge, où reparut la barbarie antique, l’italien {\itshape corsale} était pris pour un {\itshape titre de seigneurie}. Solon, dans sa législation, permit, dit-on, les associations pour cause de {\itshape piraterie}. Mais ce qui étonne le plus, c’est que Platon et Aristote placent le {\itshape brigandage} parmi les espèces de {\itshape chasse}. En cela, les plus grands philosophes d’une nation si éclairée sont d’accord avec les barbares de l’ancienne Germanie, chez lesquels, au rapport de César, le {\itshape brigandage}, loin de paraître infâme, était regardé comme un {\itshape exercice de vertu}. Pour des peuples qui ne s’appliquaient à aucun art, c’était {\itshape fuir l’oisiveté}. Cette coutume barbare dura si longtemps chez les nations les plus policées, qu’au rapport de  Polybe, les Romains imposèrent aux Carthaginois, entre autres conditions de paix, celle de ne point passer le cap de Pélore pour cause de commerce ou de {\itshape piraterie}. Si l’on allègue qu’à cette époque les Carthaginois et les Romains n’étaient, de leur propre aveu, que des barbares\footnote{Plaute dit dans plusieurs endroits, qu’il a traduit, en {\itshape langue barbare}, les comédies grecques…, \emph{{\itshape Marcus vertit barbarè}}. ({\itshape Vico.})}, nous citerons les Grecs eux-mêmes qui, aux temps de leur plus haute civilisation, pratiquaient, comme le montrent les sujets de leurs comédies, ces mêmes coutumes qui font aujourd’hui donner le nom de {\itshape barbarie} à la côte d’Afrique opposée à l’Europe.\par
Le principe de cet ancien droit de la guerre fut le caractère inhospitalier des {\itshape peuples héroïques} que nous avons observé plus haut. Les {\itshape étrangers} étaient à leurs yeux d’{\itshape éternels ennemis}, et ils faisaient consister l’honneur de leurs empires à les tenir le plus éloignés qu’il était possible de leurs frontières ; c’est ce que Tacite nous rapporte des Suèves, le peuple le plus fameux de l’ancienne Germanie. Un passage précieux de Thucydide prouve que les {\itshape étrangers} étaient considérés comme des {\itshape brigands}. Jusqu’à son temps\footnote{Οὐκ ἔχοντός πω αἰσχύνην τούτου τοῦ ἔργου, φέροντος δέ τι καὶ δόξης μᾶλλον· δηλοῦσι δὲ τῶν τε ἠπειρωτῶν τινὲς ἔτι καὶ νῦν, οἷς κόσμος καλῶς τοῦτο δρᾶν, καὶ οἱ παλαιοὶ τῶν ποιητῶν τὰς πύστεις τῶν καταπλεόντων πανταχοῦ ὁμοίως ἐρωτῶντες εἰ λῃσταί εἰσιν, ὡς οὔτε ὧν πυνθάνονται ἀπαξιούντων τὸ ἔργον, οἷς τε ἐπιμελὲς εἴη εἰδέναι οὐκ ὀνειδιζόντων.}, les voyageurs qui se rencontraient sur terre ou sur mer, se demandaient réciproquement s’ils n’étaient point des {\itshape brigands} ou des {\itshape pirates}, en prenant  sans doute ce mot dans le sens d’{\itshape étrangers}. Nous retrouvons cette coutume chez toutes les nations barbares, au nombre desquels on est forcé de compter les Romains, lorsqu’on lit ces deux passages curieux de la loi des douze tables : \emph{{\itshape Adversus hostem æterna auctoritas esto.}}{\itshape  —} \emph{{\itshape Si status dies sit, cum hoste venito}}\footnote{On prend ordinairement dans ce passage le mot {\itshape hostis} dans le sens de l’{\itshape adverse partie} ; mais Cicéron observe précisément à ce sujet que {\itshape hostis} était pris par les anciens latins dans le sens du {\itshape peregrinus}. ({\itshape Vico.})}. Les peuples civilisés eux-mêmes n’admettent d’étrangers que ceux qui ont obtenu une permission expresse d’habiter parmi eux.\par
Les {\itshape cités}, selon Platon, \emph{{\itshape eurent en quelque sorte dans la guerre leur principe fondamental}} ; la {\itshape guerre} elle-même, πόλεμος, tira son nom de πόλις, {\itshape cité}… Cette éternelle inimitié des peuples jeta beaucoup de jour sur le récit qu’on lit dans Tite-Live, de la première guerre d’Albe et de Rome : \emph{{\itshape Les Romains}, dit-il, {\itshape  avaient longtemps fait la guerre contre les Albains}}, c’est-à-dire que les deux peuples avaient longtemps auparavant exercé réciproquement {\itshape ces brigandages} dont nous parlons. L’action d’{\itshape Horace} qui {\itshape tue sa sœur pour avoir pleuré Curiace}, devient plus vraisemblable si l’on suppose qu’il était non son {\itshape fiancé}, mais son ravisseur\footnote{Comment expliquer cette prétendue alliance, quand Romulus lui-même, sorti du sang des rois d’Albe, vengeur de Numitor auquel il avait rendu le trône, ne put trouver de femmes chez les Albains. ({\itshape Vico.})}. Il est bien digne de remarque, que, par ce genre de convention, {\itshape la victoire de l’un des deux peuples devait être décidée par l’issue du combat}  {\itshape des principaux intéressés}, tels que les trois Horaces et les trois Curiaces dans la guerre d’Albe, tels que Pâris et Ménélas dans la guerre de Troie. De même, quand la barbarie antique reparut au moyen âge, les princes décidaient eux-mêmes les querelles nationales par des combats singuliers, et les peuples se soumettaient à ces sortes de jugements. Albe ainsi considérée fut la Troie latine, et l’Hélène romaine fut la sœur d’Horace.\par
Les {\itshape dix ans} du siège de Troie célébrés chez les Grecs, répondent, chez les Latins, {\itshape aux dix ans} du siège de Véies ; c’est un nombre fini pour le nombre infini des années antérieures, pendant lesquelles les cités avaient exercé entre elles de continuelles hostilités\footnote{Le {\itshape nombre}, chose la plus abstraite de toutes, fut la dernière que comprirent les nations. Pour désigner un grand nombre, on se servit d’abord de celui de {\itshape douze}, de là les {\itshape douze} grands dieux, les {\itshape douze} travaux d’Hercule, les {\itshape douze} parties de l’as, les {\itshape douze} tables, etc. Les Latins ont conservé, d’une époque où l’on connaissait mieux les nombres, leur mot {\itshape sexcenti}, et les Italiens, {\itshape cento}, et ensuite {\itshape cento e mille}, pour dire un nombre innombrable. Les philosophes seuls peuvent arriver à comprendre l’idée d’{\itshape infini}. ({\itshape Vico.})}\footnote{Il est à croire qu’au temps de la guerre de Troie, le nom de αχαιοι, {\itshape achivi}, était restreint à une partie du peuple grec, qui fit cette guerre ; mais ce nom s’étant étendu à toute la nation, on dit au temps d’Homère {\itshape que toute la Grèce s’était liguée contre Troie}. Ainsi nous voyons dans Tacite que ce nom de {\itshape Germanie}, étendu depuis à une vaste contrée de l’Europe, n’avait désigné originalement qu’une tribu qui, passant le Rhin, chassa les Gaulois de ses bords ; la gloire de cette conquête fit adopter ce nom par toute la {\itshape Germanie}, comme la gloire du siège de Troie avait fait adopter celui d’{\itshape achivi} par tous les Grecs. ({\itshape Vico.})}.\par
Les guerres éternelles des cités anciennes, leur  éloignement pour former des ligues et des confédérations, nous expliquent pourquoi l’Espagne fut soumise par les Romains ; l’Espagne, dont César avouait que partout ailleurs il avait combattu pour l’empire, là seulement pour la vie ; l’Espagne, que Cicéron proclamait la mère des plus belliqueuses nations du monde. La résistance de Sagunte, arrêtant pendant huit mois la même armée qui, après tant de pertes et de fatigues, faillit triompher de Rome elle-même dans son Capitole ; la résistance de Numance, qui fit trembler les vainqueurs de Carthage, et ne put être réduite que par la sagesse et l’héroïsme du triomphateur de l’Afrique, n’étaient-elles pas d’assez grandes leçons pour que cette nation généreuse unît toutes ses cités dans une même confédération, et fixât l’empire du monde sur les bords du Tage ? Il n’en fut point ainsi : l’Espagne mérita le déplorable éloge de Florus : \emph{{\itshape sola omnium provinciarum vires suas, postquam victa est, intellexit}}. Tacite fait la même remarque sur les Bretons, que son Agricola trouva si belliqueux : \emph{{\itshape dum singuli pugnant, universi vincuntur}}.\par
Les historiens frappés de l’éclat des {\itshape entreprises navales des temps héroïques}, n’ont point remarqué {\itshape les guerres de terre} qui se faisaient aux mêmes époques, encore moins la {\itshape politique héroïque} qui gouvernait alors la Grèce. Mais Thucydide, cet écrivain plein de sens et de sagacité, nous en donne une indication précieuse : \emph{{\itshape Les cités héroïques}, dit-il, {\itshape  étaient toutes sans murailles}}, comme Sparte dans  la Grèce, comme Numance, la Sparte de l’Espagne ; \emph{{\itshape telle était}, ajoute-t-il, {\itshape  la fierté indomptable et la violence naturelle des héros, que tous les jours ils se chassaient les uns les autres de leurs établissements}}. Ainsi Amulius chassa Numitor, et fut chassé lui-même par Romulus, qui rendit Albe à son premier roi. Qu’on juge combien il est raisonnable de chercher un moyen de certitude pour la chronologie dans les généalogies héroïques de la Grèce, et dans cette suite non interrompue des quatorze rois latins ! Dans les siècles les plus barbares du moyen âge, on ne trouve rien de plus inconstant, de plus variable, que la fortune des maisons royales. \emph{{\itshape Urbem Romam principio reges}{\scshape habuere}}, dit Tacite à la première ligne des Annales. L’ingénieux écrivain s’est servi du plus faible des trois mots employés par les jurisconsultes pour désigner la possession, {\itshape habere, tenere, possidere}.
\section[{§ VII. Corollaires relatifs aux antiquités romaines, et particulièrement à la prétendue monarchie de Rome, à la prétendue liberté populaire qu’aurait fondée Junius Brutus}]{§ VII. {\itshape Corollaires relatifs aux antiquités romaines, et particulièrement à la prétendue monarchie de Rome, à la prétendue liberté populaire qu’aurait fondée Junius Brutus}}
\noindent En considérant ces rapports innombrables de l’histoire politique des Grecs et des Romains, tout homme qui consulte la réflexion plutôt que la mémoire ou l’imagination, affirmera sans hésiter que,  depuis les temps des rois jusqu’à l’époque où les plébéiens partagèrent avec les nobles le {\itshape droit des mariages solennels, le peuple de Mars se composa des seuls nobles}… On ne peut admettre que les plébéiens, que la tourbe des plus vils ouvriers, traités dès l’origine comme esclaves, eussent le droit d’élire les rois, tandis que les {\itshape Pères} auraient seulement sanctionné l’élection. C’est confondre ces premiers temps avec celui où les plébéiens étaient déjà une partie de la cité, et concouraient à élire les consuls, droit qui ne leur fut communiqué par les {\itshape Pères} qu’après celui des {\itshape mariages solennels}, c’est-à-dire au moins trois cents ans après la mort de Romulus.\par
Lorsque les philosophes ou les historiens parlent des {\itshape premiers temps}, ils prennent le mot {\itshape peuple} dans un sens {\itshape moderne}, parce qu’ils n’ont pu imaginer les {\itshape sévères aristocraties} des âges antiques ; de là deux erreurs dans l’acception des mots {\itshape rois} et {\itshape liberté}. Tous les auteurs ont cru que la {\itshape royauté romaine} était {\itshape monarchique}, que la {\itshape liberté} fondée par Junius Brutus était une {\itshape liberté populaire}. On peut voir à ce sujet l’inconséquence de Bodin.\par
Tout ceci nous est confirmé par Tite-Live, qui, en racontant l’institution du consulat par Junius Brutus, dit positivement qu’il n’y eut rien de changé dans la constitution de Rome (Brutus était trop sage pour faire autre chose que la ramener à la pureté de ses principes primitifs), et que l’existence de deux consuls annuels ne diminua rien de la puissance  royale, \emph{{\itshape nihil quicquam de regiâ potestate deminutum}}. Ces consuls étaient deux rois annuels d’une aristocratie, \emph{{\itshape reges annuos}}, dit Cicéron dans le livre des lois, de même qu’il y avait à Sparte des rois à vie, quoique personne ne puisse contester le caractère aristocratique de la constitution lacédémonienne. Les consuls, pendant leur {\itshape règne}, étaient, comme on sait, sujets à l’appel, de même que les rois de Sparte étaient sujets à la surveillance des éphores : leur {\itshape règne annuel} étant fini, les consuls pouvaient être accusés, comme on vit les éphores condamner à mort des rois de Sparte. Ce passage de Tite-Live nous démontre donc à la fois, et que la {\itshape royauté romaine fut aristocratique}, et que la {\itshape liberté fondée par Brutus ne fut point populaire}, mais particulière aux nobles ; elle n’affranchit pas le peuple des patriciens, ses maîtres, mais elle affranchit ces derniers de la tyrannie des Tarquins.\par
Si la variété de tant de causes et d’effets observés jusqu’ici dans l’histoire de la république romaine, si l’influence continue que ces causes exercèrent sur ces effets, ne suffisent pas pour établir que la royauté chez les Romains eut un caractère aristocratique, et que la liberté fondée par Brutus fut restreinte à l’ordre des nobles, il faudra croire que les Romains, peuple grossier et barbare, ont reçu de Dieu un privilège refusé à la nation la plus ingénieuse et la plus policée, à celle des Grecs ; qu’ils ont connu leurs antiquités, tandis que les Grecs, au rapport de Thucydide, ne surent rien des  leurs jusqu’à la guerre du Péloponnèse\footnote{Nous avons observé dans la table chronologique que cette époque est pour l’histoire grecque celle de la plus grande lumière, comme pour l’histoire romaine l’époque de la seconde guerre punique ; c’est alors que Tite-Live déclare qu’il écrit l’histoire avec plus de certitude ; et pourtant il n’hésite point d’avouer qu’il ignore les trois circonstances historiques les plus importantes. {\itshape Voyez la table chronologique.} ({\itshape Vico.})}. Mais quand on accorderait ce privilège aux Romains, il faudrait convenir que leurs traditions ne présentent que des souvenirs obscurs, que des tableaux confus, et qu’avec tout cela la raison ne peut s’empêcher d’admettre ce que nous avons établi sur les antiquités romaines.
\section[{§ VIII. Corollaire relatif à l’héroïsme des premiers peuples}]{§ VIII. {\itshape Corollaire relatif à l’héroïsme des premiers peuples}}
\noindent D’après les principes de la {\itshape politique héroïque} établis ci-dessus, l’{\itshape héroïsme des premiers peuples}, dont nous sommes obligés de traiter ici, fut bien différent de celui qu’ont imaginé les philosophes, imbus de leurs préjugés sur la sagesse merveilleuse des anciens, et trompés par les philologues sur le sens de ces trois mots, {\itshape peuple, roi} et {\itshape liberté}. Ils ont entendu par le premier mot, {\itshape des peuples où les plébéiens seraient déjà citoyens}, par le second, des {\itshape monarques}, par le troisième, {\itshape une liberté populaire}. Ils ont fait entrer dans l’héroïsme des premiers âges, trois idées naturelles à des esprits éclairés et adoucis par la civilisation : l’idée d’une {\itshape justice raisonnée},  et conduite par les maximes d’une morale socratique ; l’idée de cette {\itshape gloire} qui récompense les bienfaiteurs du genre humain ; enfin, l’idée d’un noble {\itshape désir de l’immortalité}. Partant de ces trois erreurs, ils ont cru que les rois et autres grands personnages des temps anciens s’étaient consacrés, eux, leurs familles, et tout ce qui leur appartenait, à adoucir le sort des malheureux qui forment la majorité dans toutes les sociétés du monde.\par
Cependant cet Achille, le plus grand des héros grecs, Homère nous le représente sous trois aspects entièrement contraires aux idées que les philosophes ont conçues de l’héroïsme antique. Achille est-il {\itshape juste} quand Hector lui demande la sépulture en cas qu’il périsse, et que, sans réfléchir au sort commun de l’humanité, il répond durement : \emph{{\itshape Quel accord entre l’homme et le lion, entre le loup et l’agneau ? Quand je t’aurai tué, je te dépouillerai, pendant trois jours je te traînerai lié à mon char autour des murs de Troie, et tu serviras ensuite de pâture à mes chiens.}} Aime-t-il la {\itshape gloire}, lorsque, pour une injure particulière, il accuse les dieux et les hommes, se plaint à Jupiter de son rang élevé, rappelle ses soldats de l’armée alliée, et que, ne rougissant point de se réjouir avec Patrocle de l’affreux carnage que fait Hector de ses compatriotes, il forme le souhait impie que tous les Troyens et tous les Grecs périssent dans cette guerre, et que Patrocle et lui survivent seuls à leur ruine ? Annonce-t-il le noble {\itshape amour de l’immortalité}, lorsqu’aux  enfers, interrogé par Ulysse s’il est satisfait de ce séjour, il répond qu’il aimerait mieux vivre encore, et être le dernier des esclaves ? Voilà le héros qu’Homère qualifie toujours du nom d’\emph{{\itshape irréprochable} (ἀμύμων)}, et qu’il semble proposer aux Grecs pour modèle de la vertu héroïque ? Si l’on veut qu’Homère instruise autant qu’il intéresse, ce qui est le devoir du poète, on ne doit entendre par ce héros {\itshape irréprochable}, que le plus orgueilleux, le plus irritable de tous les hommes ; la vertu célébrée en lui, c’est la susceptibilité, la délicatesse du point d’honneur, dans laquelle les duellistes faisaient consister toute leur morale, lorsque la barbarie antique reparut au moyen âge, et que les romanciers exaltent dans leurs chevaliers errants.\par
Quant à l’histoire romaine, on appréciera les héros qu’elle vante, si l’on réfléchit à l’\emph{{\itshape éternelle inimitié}} que, selon Aristote, les \emph{{\itshape nobles ou héros juraient aux plébéiens}}. Qu’on parcoure l’âge de la {\itshape vertu romaine}, que Tite-Live fixe au temps de la guerre contre Pyrrhus (\emph{{\itshape nulla ætas virtutum feracior}}), et que, d’après Salluste (saint Augustin, {\itshape Cité de Dieu}), nous étendons depuis l’expulsion des rois jusqu’à la seconde guerre punique. Ce Brutus, qui immole à la liberté ses deux fils, espoir de sa famille ; ce Scévola qui effraie Porsenna et détermine sa retraite en brûlant la main qui n’a pu l’assassiner ; ce Manlius qui punit de mort la faute glorieuse d’un fils vainqueur ; ces Décius qui se dévouent pour sauver leurs armées ; ces Fabricius, ces Curius, qui repoussent  l’or des Samnites, et les offres magnifiques du roi d’Épire ; ce Régulus enfin, qui, par respect pour la sainteté du serment, va chercher à Carthage la mort la plus cruelle ; que firent-ils pour l’avantage des infortunés plébéiens ? Tout l’héroïsme des maîtres du peuple ne servait qu’à l’épuiser par des guerres interminables, qu’à l’enfoncer dans un abîme d’usure, pour l’ensevelir ensuite dans les cachots particuliers des nobles, où les débiteurs étaient déchirés à coups de verges, comme les plus vils des esclaves. Si quelqu’un tentait de soulager les plébéiens par une loi agraire, l’ordre des nobles accusait et mettait à mort le bienfaiteur du peuple. Tel fut le sort (pour ne citer qu’un exemple) de ce Manlius qui avait sauvé le Capitole. Sparte, la ville {\itshape héroïque} de la Grèce, eut son Manlius dans le roi Agis ; Rome, la ville {\itshape héroïque} du monde, eut son Agis dans la personne de Manlius : Agis entreprit de soulager le pauvre peuple de Lacédémone, et fut étranglé par les éphores ; Manlius, soupçonné à Rome du même dessein, fut précipité de la roche Tarpéienne. Par cela seul que les nobles des premiers peuples se tenaient pour {\itshape héros}, c’est-à-dire pour des êtres d’une nature supérieure à celle des plébéiens, ils devaient maltraiter la multitude. En lisant l’histoire romaine, un lecteur raisonnable doit se demander avec étonnement que pouvait être cette {\itshape vertu} si vantée des Romains avec un orgueil si tyrannique ? cette {\itshape modération} avec tant d’avarice ? cette {\itshape douceur} avec un esprit si farouche ?  cette {\itshape justice} au milieu d’une si grande inégalité ?\par
Les principes qui peuvent faire cesser cet étonnement, et nous expliquer l’héroïsme des anciens peuples, sont nécessairement les suivants : I. En conséquence de l’éducation sauvage des géants dont nous avons parlé, l’{\itshape éducation des enfants} doit conserver chez les peuples héroïques cette sévérité, cette barbarie originaire ; les Grecs et les Romains pouvaient tuer leurs enfants nouveau nés ; les Lacédémoniens battaient de verges leurs enfants dans le temple de Diane, et souvent jusqu’à la mort. Au contraire, c’est la sensibilité paternelle des modernes, qui leur donne en toute chose cette délicatesse étrangère à l’antiquité. — II. {\itshape Les épouses doivent s’acheter, chez de tels peuples, avec les dots héroïques}, usage que les prêtres romains conservèrent dans la solennité de leurs mariages, qu’ils contractaient {\itshape coemptione et farre}. Tacite en dit autant des anciens Germains, auxquels cette coutume était probablement commune avec tous les peuples barbares. Chez eux, les femmes sont considérées par leurs maris comme nécessaires pour leur donner des enfants, mais du reste traitées comme esclaves. Telles sont les mœurs du nouveau monde et d’une grande partie de l’ancien. Au contraire, lorsque la femme apporte une dot, elle achète la liberté du mari, et obtient de lui un aveu public qu’il est incapable de supporter les charges du mariage. C’est peut-être l’origine des privilèges importants dont les Empereurs  romains favorisent les dots. — III. {\itshape Les fils acquièrent, les femmes épargnent pour leurs pères et leurs maris} ; c’est le contraire de ce qui se fait chez les modernes. — IV. {\itshape Les jeux et les plaisirs sont fatigants}, comme la lutte, la course. Homère dit toujours Achille \emph{{\itshape aux pieds légers}}. Ils sont en outre {\itshape dangereux} : ce sont des joutes, des chasses, exercices capables de fortifier l’âme et le corps, et d’habituer à mépriser, à prodiguer la vie. — V. {\itshape Ignorance complète du luxe, des commodités sociales, des doux loisirs.} — VI. {\itshape Les guerres sont toutes religieuses}, et par conséquent atroces. — VII. De telles guerres entraînent dans toute leur dureté {\itshape les servitudes héroïques} ; les vaincus sont regardés comme des hommes sans dieux, et perdent non-seulement la liberté civile, mais la liberté naturelle. — D’après toutes ces considérations, les républiques doivent être alors {\itshape des aristocraties naturelles}, c’est-à-dire {\itshape composées d’hommes qui soient naturellement les plus courageux} ; le gouvernement doit être de nature à réserver tous les honneurs civils à un petit nombre de nobles, de pères de famille, qui fassent consister le bien public dans la conservation de ce pouvoir absolu qu’ils avaient originairement sur leurs familles, et qu’ils ont maintenant dans l’état, de sorte qu’ils entendent le mot {\itshape patrie} dans le sens étymologique qu’on peut lui donner, {\itshape l’intérêt des pères} ({\itshape patria}, sous-entendu {\itshape res}).\par
Tel fut donc l’{\itshape héroïsme} des premiers peuples, telle la {\itshape nature morale} des héros, tels leurs {\itshape usages}, leurs {\itshape gouvernements} et leurs lois. Cet {\itshape héroïsme} ne  peut désormais se représenter, pour des causes toutes contraires à celles que nous avons énumérées, et qui ont produit deux sortes de gouvernements {\itshape humains}, les {\itshape républiques populaires} et les {\itshape monarchies}. Le héros digne de ce nom, caractère bien différent de celui des temps {\itshape héroïques}, est appelé par les souhaits des peuples affligés ; les philosophes en {\itshape raisonnent}, les poètes l’{\itshape imaginent}, mais la nature des sociétés ne permet pas d’espérer un tel bienfait du ciel.\par
Tout ce que nous avons dit jusqu’ici sur l’{\itshape héroïsme des premiers peuples}, reçoit un nouveau jour des axiomes relatifs à l’{\itshape héroïsme romain}, que l’on trouvera analogue à l’{\itshape héroïsme des Athéniens} encore gouvernés par le sénat aristocratique de l’aréopage, et à l’{\itshape héroïsme de Sparte}, république d’{\itshape héraclides}, c’est-à-dire de {\itshape héros}, ou {\itshape nobles}, comme on l’a démontré.
\chapterclose


\chapteropen
\chapter[{Chapitre VII. De la physique poétique}]{Chapitre VII. \\
De la physique poétique}

\chaptercont
\noindent  Après avoir observé quelle fut la sagesse des premiers hommes dans la logique, la morale, l’économie et la politique, passons au second rameau de l’arbre métaphysique, c’est-à-dire à la physique, et de là à la cosmographie, par laquelle nous parvenons à l’astronomie, pour traiter ensuite de la chronologie et de la géographie, qui en dérivent.\par
\section[{§ I. De la physiologie poétique}]{§ I. {\itshape De la physiologie poétique}}
\noindent Les {\itshape poètes théologiens}, dans leur physique grossière, considérèrent dans l’homme deux idées métaphysiques, {\itshape être, subsister}. Sans doute ceux du Latium conçurent bien grossièrement l’{\itshape être}, puisqu’ils le confondirent avec l’action de {\itshape manger}. Tel fut probablement le premier sens du mot {\itshape sum}, qui depuis eut les deux significations. Aujourd’hui même nous entendons nos paysans dire d’un malade, {\itshape il mange encore}, pour {\itshape il vit encore}. Rien de plus abstrait que l’idée d’{\itshape existence}. Ils conçurent aussi l’idée de {\itshape subsister}  c’est-à-dire {\itshape être debout, être sur ses pieds}. C’est dans ce sens que les destins d’Achille étaient attaches à ses talons.\par
Les premiers hommes réduisaient toute la machine du corps humain aux {\itshape solides} et aux {\itshape liquides}. Les {\scshape solides} eux-mêmes, ils les réduisaient aux chairs, {\itshape viscera} [{\itshape vesci} voulait dire {\itshape se nourrir}, parce que les aliments que l’on assimile font de la chair] ; aux os et articulations, {\itshape artus} [observons que {\itshape artus} vient du mot {\itshape ars}, qui chez les anciens Latins signifiait la force du corps ; d’où {\itshape artitus}, robuste ; ensuite on donna ce nom d’{\itshape ars} à tout système de préceptes propres à former quelques facultés de l’âme] ; aux nerfs, qu’ils prirent pour les {\itshape forces}, lorsque, usant encore du langage muet, ils parlaient avec des signes matériels [ce n’est pas sans raison qu’ils prirent {\itshape nerfs} dans ce sens, puisque les nerfs tendent les muscles, dont la tension fait la force de l’homme] ; enfin à la moelle, c’est dans la moelle qu’ils placèrent non moins sagement l’essence de la vie [l’amant appelait sa maîtresse {\itshape medulla}, et {\itshape medullitùs} voulait dire {\itshape de tout cœur} ; lorsque l’on veut désigner l’excès de l’amour, on dit qu’il brûle la moelle des os, {\itshape urit medullas}]. Pour les {\scshape liquides}, ils les réduisaient à une seule espèce, à celle du sang ; ils appelaient {\itshape sang} la liqueur spermatique, comme le prouve la périphrase {\itshape sanguine cretus}, pour {\itshape engendré} ; et c’était encore une expression juste, puisque cette liqueur semble formée du plus pur de notre sang. Avec la même justesse, ils appelèrent le sang {\itshape le suc}  {\itshape des fibres}, dont se compose la chair. C’est de là que les Latins conservèrent {\itshape succi plenus}, pour dire {\itshape charnu}, plein d’un sang abondant et pur.\par
Quant à l’autre partie de l’homme, qui est l’{\itshape âme}, les {\itshape poètes théologiens} la placèrent dans l’{\itshape air}, chez les Latins {\itshape anima} ; l’air fut pour eux le véhicule de la vie, d’où les Latins conservèrent la phrase {\itshape animâ vivimus}, et en poésie, {\itshape ferri ad vitales auras}, pour naître ; {\itshape ducere vitales auras}, pour vivre ; {\itshape vitam referre in auras}, pour mourir ; et en prose {\itshape animam ducere}, vivre ; {\itshape animam trahere}, être à l’agonie ; {\itshape animam efflare, emittere}, expirer ; ensuite les physiciens placèrent aussi dans l’air l’âme du monde. C’est encore une expression juste que {\itshape animus} pour la partie douée du sentiment : les Latins disent {\itshape animo sentimus}. Ils considérèrent {\itshape animus} comme mâle, {\itshape anima} comme femelle, parce que {\itshape animus} agit sur {\itshape anima} ; le premier est l’\emph{{\itshape igneus vigor}} dont parle Virgile ; de sorte qu’{\itshape animus} aurait son sujet dans les nerfs, {\itshape anima} dans le sang et dans les veines. L’{\itshape æther} serait le véhicule d’{\itshape animus}, l’air celui d’{\itshape anima} ; le premier circulant avec toute la rapidité des esprits animaux, la seconde plus lentement avec les esprits vitaux. {\itshape Anima} serait l’agent du mouvement ; {\itshape animus} l’agent et le principe des actes de la volonté. Les {\itshape poètes théologiens} ont senti, par une sorte d’instinct, cette dernière vérité ; et dans les poèmes d’Homère ils ont appelé l’âme ({\itshape animus}), une force {\itshape sacrée}, une {\itshape puissance mystérieuse}, un {\itshape dieu inconnu}. En général, lorsque les Grecs et les Latins rapportaient  quelqu’une de leurs paroles, de leurs actions à un principe supérieur, ils disaient {\itshape un dieu l’a voulu ainsi}. Ce principe fut appelé par les Latins {\itshape mens animi}. Ainsi, dans leur grossièreté, ils pénétrèrent cette vérité sublime que la théologie naturelle a établie par des raisonnements invincibles contre la doctrine d’Épicure, {\itshape les idées nous viennent de Dieu}.\par
Ils ramenaient toutes les fonctions de l’âme à trois parties du corps, {\itshape la tête, la poitrine, le cœur}. À la {\itshape tête}, ils rapportaient toutes les connaissances, et comme elles étaient chez eux toutes d’imagination, ils placèrent dans la tête la {\itshape mémoire}, dont les Latins employaient le nom pour désigner l’{\itshape imagination}. Dans le retour de la barbarie au moyen âge, on disait {\itshape imagination} pour {\itshape génie, esprit}. [Le biographe contemporain de Rienzi l’appelle \emph{{\itshape uomo fantastico}} pour {\itshape uomo d’ingegno}.] En effet, l’imagination n’est que le résultat des souvenirs ; le {\itshape génie} ne fait autre chose que travailler sur les matériaux que lui offre la {\itshape mémoire}. Dans ces premiers temps où l’esprit humain n’avait point tiré de l’art d’écrire, de celui de raisonner et de compter, la subtilité qu’il a aujourd’hui, où la multitude de mots abstraits que nous voyons dans les langues modernes, ne lui avait pas encore donné ses habitudes d’abstraction continuelle, il occupait toutes ses forces dans l’exercice de ces trois belles facultés qu’il doit à son union avec le corps, et qui toutes trois sont relatives à la première opération de l’esprit, l’{\itshape invention} ; il fallait trouver avant de juger, la {\itshape topique} devait précéder  la {\itshape critique}, ainsi que nous l’avons dit page 163. Aussi les {\itshape poètes théologiens} dirent que la {\itshape mémoire} (qu’ils confondaient avec l’{\itshape imagination}) était la {\itshape mère des muses}, c’est-à-dire des arts.\par
En traitant de ce sujet, nous ne pouvons omettre une observation importante qui jette beaucoup de jour sur celle que nous avons faite dans la {\itshape Méthode} ({\itshape il nous est} aujourd’hui {\itshape difficile de} comprendre, {\itshape impossible} d’imaginer {\itshape la manière de penser des premiers hommes qui fondèrent l’humanité païenne}\footnote{\noindent Les premiers hommes étant presque ainsi {\itshape incapables de généraliser} que les animaux, pour qui toute sensation nouvelle efface entièrement la sensation analogue qu’ils ont pu éprouver, ils ne pouvaient {\itshape combiner des idées et discourir}. Toutes les pensées ({\itshape sentenze}) devaient en conséquence être {\itshape particularisées} par celui qui les pensait, ou plutôt qui les {\itshape sentait}. Examinons le trait sublime que Longin admire dans l’ode de Sapho, traduite par Catulle : le poète exprime par une comparaison les transports qu’inspire la présence de l’objet aimé,\par

\begin{verse}
{\itshape Ille mi par esse deo videtur},\\
Celui-là est pour moi égal en bonheur aux dieux même....\\
\end{verse}
\par
\noindent la pensée n’atteint pas ici le plus haut degré du sublime, parce que l’amant ne la {\itshape particularise} point en la restreignant à lui-même ; c’est au contraire ce que fait Térence, lorsqu’il dit :\par

\begin{verse}
{\itshape Vitam deorum adepti sumus},\\
Nous avons atteint la félicité des dieux.\\
\end{verse}
\par
\noindent ce sentiment est propre à celui qui parle, le pluriel est pour le singulier ; cependant ce pluriel semble en faire un sentiment commun à plusieurs. Mais le même poète dans une autre comédie porte le sentiment au plus haut degré de sublimité en le singularisant et l’appropriant à celui qui l’éprouve,\par
{\itshape Deus factus sum}, je ne suis plus un homme, mais un Dieu.\\
\par
\noindent Les {\itshape pensées abstraites} regardant les généralités sont du domaine des philosophes, et les {\itshape réflexions sur les passions} sont d’une {\itshape fausse} et {\itshape froide poésie}.
}). Leur esprit précisait, particularisait toujours, de  sorte qu’à chaque changement dans la physionomie ils croyaient voir un nouveau visage, à chaque nouvelle passion un autre cœur, une autre âme ; de là ces expressions poétiques, commandées par une nécessité naturelle plus que par celle de la mesure, {\itshape ora, vultus, animi, pectora, corda}, employées pour leurs singuliers.\par
Ils plaçaient dans la {\itshape poitrine} le siège de toutes les passions, et au-dessous, les deux germes, les deux levains des passions : dans l’{\itshape estomac} la partie irascible, et la partie concupiscible surtout dans le {\itshape foie}, qui est défini {\itshape le laboratoire du sang} ({\itshape officina}). Les poètes appellent cette partie {\itshape præcordia} ; ils attachent au foie de Titan chacun des animaux remarquables par quelque passion ; c’était entendre d’une manière confuse, que {\itshape la concupiscence est la mère de toutes les passions}, et que {\itshape les passions sont dans nos humeurs}.\par
Ils rapportaient au {\itshape cœur} tous les conseils ; les héros roulaient leurs pensées, leurs inquiétudes dans leur cour ; {\itshape agitabant, versabant, volutabant corde curas}. Ces hommes encore stupides ne pensaient aux choses qu’ils avaient à faire, que lorsqu’ils étaient agités par les passions. De là les Latins appelaient les sages {\itshape cordati}, les hommes de peu de sens, {\itshape vecordes}. Ils disaient {\itshape sententiæ}, pour {\itshape résolutions}, parce que leurs jugements n’étaient que le résultat de leurs sentiments ; aussi les jugements des {\itshape héros} s’accordaient toujours avec la vérité dans leur {\itshape forme}, quoiqu’ils fussent souvent faux dans leur {\itshape matière}.
\section[{§ II. Corollaire relatif aux descriptions héroïques}]{§ II. {\itshape Corollaire relatif aux descriptions héroïques}}
\noindent  Les premiers hommes ayant peu ou point de raison, et étant au contraire tout imagination, rapportaient {\itshape les fonctions externes de l’âme aux cinq sens du corps}, mais considérés dans toute la finesse, dans toute la force et la vivacité qu’ils avaient alors. Les mots par lesquels ils exprimèrent l’action des sens le prouvent assez : ils disaient pour entendre, {\itshape audire}, comme on dirait {\itshape haurire}, puiser, parce que les oreilles semblent boire l’air, renvoyé par les corps qu’il frappe. Ils disaient pour voir distinctement, {\itshape cernere oculis} (d’où l’italien {\itshape scernere, discerner}), mot à mot {\itshape séparer par les yeux}, parce que les yeux sont comme un crible dont les pupilles sont les trous ; de même que du crible sortent les jets de poussière qui vont toucher la terre, ainsi des yeux semblent sortir par les pupilles les jets ou rayons de lumière qui vont frapper les objets que nous voyons distinctement ; c’est le {\itshape rayon visuel}, deviné par les stoïciens, et démontré de nos jours par Descartes. Ils disaient, pour {\itshape voir} en général, {\itshape usurpare oculis. Tangere}, pour {\itshape toucher} et {\itshape dérober}, parce qu’en touchant les corps nous en enlevons, nous en dérobons toujours quelque partie. Pour {\itshape odorer}, ils disaient {\itshape olfacere}, comme si, en recueillant les odeurs, nous les faisions nous-mêmes ;  et en cela ils se sont rencontrés avec la doctrine des cartésiens. Enfin, pour goûter, pour juger des saveurs, ils disaient {\itshape sapere}, quoique ce mot s’appliquât proprement aux choses douées de saveur, et non au sens qui en juge ; c’est qu’ils cherchaient dans les choses la saveur qui leur était propre : de là cette belle métaphore de {\itshape sapientia}, la sagesse, laquelle tire des choses leur usage naturel, et non celui que leur suppose l’opinion.\par
Admirons en tout ceci la Providence divine qui, nous ayant donné comme pour la garde de notre corps des {\itshape sens}, à la vérité bien inférieurs à ceux des brutes, voulut qu’à l’époque où l’homme était tombé dans un état de brutalité, il eût pour sa conservation les sens les plus actifs et les plus subtils, et qu’ensuite ces sens s’affaiblissent, lorsque viendrait l’âge de la {\itshape réflexion}, et que cette faculté prévoyante protégerait le corps à son tour.\par
On doit comprendre d’après ce qui précède, pourquoi les {\itshape descriptions héroïques}, telles que celles d’Homère, ont tant d’éclat, et sont si frappantes, que tous les poètes des âges suivants n’ont pu les imiter, bien loin de les égaler.
\section[{§ III. Corollaire relatif aux mœurs héroïques}]{§ III. {\itshape Corollaire relatif aux mœurs héroïques}}
\noindent De telles {\itshape natures héroïques}, animées de tels {\itshape sentiments héroïques}, durent créer et conserver des {\itshape mœurs} analogues à celles que nous allons esquisser.\par
 Les {\itshape héros}, récemment sortis des {\itshape géants}, étaient au plus haut degré {\itshape grossiers} et {\itshape farouches}, d’un entendement très borné, d’une vaste imagination, agités des passions les plus violentes ; ils étaient nécessairement {\itshape barbares, orgueilleux, difficiles, obstinés} dans leurs résolutions, et en même temps très {\itshape mobiles}, selon les nouveaux objets qui se présentaient. Ceci n’est point contradictoire ; vous pouvez observer tous les jours l’opiniâtreté de nos paysans, qui cèdent à la première raison que vous leur dites, mais qui, par faiblesse de réflexion, oublient bien vite le motif qui les avait frappés, et reviennent à leur première idée. — Par suite du même {\itshape défaut de réflexion}, les {\itshape héros} étaient {\itshape ouverts}, incapables de dissimuler leurs impressions, {\itshape généreux} et {\itshape magnanimes}, tels qu’Homère représente Achille, le plus grand de tous les héros grecs. Aristote part de ces mœurs {\itshape héroïques}, lorsqu’il veut dans sa {\itshape Poétique}, que le héros de la tragédie ne soit ni parfaitement bon, ni entièrement méchant, mais qu’il offre un mélange de grands vices et de grandes vertus. En effet, l’{\itshape héroïsme d’une vertu parfaite} est une conception qui appartient à la philosophie et non pas à la poésie.\par
L’{\itshape héroïsme galant} des modernes a été imaginé par les poètes qui vinrent bien longtemps après Homère, soit que l’invention des fables nouvelles leur appartienne, soit que les mœurs devenant efféminées avec le temps, ils aient altéré, et enfin corrompu entièrement les premières fables graves et sévères, comme il convenait aux fondateurs des  sociétés. Ce qui le prouve, c’est qu’Achille, qui fait tant de bruit pour l’enlèvement de Briséis, et dont la colère suffit pour remplir une Iliade, ne montre pas une fois dans tout ce poème un sentiment d’amour ; Ménélas, qui arme toute la Grèce contre Troie pour reconquérir Hélène, ne donne pas, dans tout le cours de cette longue guerre, le moindre signe d’{\itshape amoureux tourment} ou de jalousie.\par
Tout ce que nous avons dit sur les {\itshape pensées}, les {\itshape descriptions} et les {\itshape mœurs héroïques}, appartient à la {\scshape découverte du véritable Homère}, que nous ferons dans le livre suivant.
\chapterclose


\chapteropen
\chapter[{Chapitre VIII. De la cosmographie poétique}]{Chapitre VIII. \\
De la cosmographie poétique}

\chaptercont
\noindent  Les {\itshape poètes théologiens}, ayant pris pour principes de leur {\itshape physique} les êtres divinisés par leur imagination, se firent une {\itshape cosmographie} en harmonie avec cette {\itshape physique}. Ils composèrent le monde de dieux du ciel, de l’enfer ({\itshape dii superi, inferi}), et de dieux intermédiaires (qui furent probablement ceux que les anciens Latins appelaient {\itshape medioxumi}).\par
Dans le monde, ce fut le {\itshape ciel} qu’ils contemplèrent d’abord. Les choses du ciel durent être pour les Grecs les premiers μαθήματα, {\itshape connaissances par excellence}, les premiers θεωρήματα, objets {\itshape divins de contemplation}. Le mot {\itshape contemplation}, appliqué à ces choses, fut tiré par les Latins de ces espaces du ciel désignés par les augures pour y observer les présages, et appelés {\itshape templa cœli}. — Le {\itshape ciel} ne fut pas d’abord plus haut pour les poètes, que {\itshape le sommet des montagnes} ; ainsi les enfants s’imaginent que les montagnes sont les {\itshape colonnes} qui soutiennent la voûte du ciel, et les Arabes admettent ce principe de cosmographie  dans leur Coran ; de ces {\itshape colonnes}, il resta {\itshape les deux colonnes d’Hercule}, qui remplacèrent Atlas fatigué de porter le ciel sur ses épaules. {\itshape Colonne} dut venir d’abord de {\itshape columen} ; ce n’était que des {\itshape soutiens}, des {\itshape étais} arrondis dans la suite par l’architecture.\par
La fable des géants faisant la guerre aux dieux et entassant {\itshape Ossa sur Pélion, Olympe sur Ossa}, doit avoir été trouvée depuis Homère. Dans l’Iliade, les dieux se tiennent toujours \emph{{\itshape sur la cime du mont Olympe}}. Il suffisait donc que l’Olympe s’écroulât pour en faire tomber les dieux. Cette fable, quoique rapportée dans l’Odyssée, y est peu convenable : dans ce poème, l’{\itshape enfer} n’est pas plus profond que la {\itshape fosse} où Ulysse voit les ombres des héros et converse avec elles. Si l’Homère de l’Odyssée avait cette idée bornée de l’{\itshape enfer}, il devait concevoir du {\itshape ciel} une idée analogue, une idée conforme à celle que s’en était faite l’Homère de l’Iliade.
\chapterclose


\chapteropen
\chapter[{Chapitre IX. De l’astronomie poétique}]{Chapitre IX. \\
De l’astronomie poétique}

\begin{argument}\noindent Démonstration astronomique, fondée sur des preuves physico-philologiques, de l’uniformité des principes ci-dessus établis chez toutes les nations païennes.
\end{argument}


\chaptercont
\noindent  La force indéfinie de l’esprit humain se développant de plus en plus, et la contemplation du ciel, nécessaire pour prendre les augures, obligeant les peuples à l’observer sans cesse, {\itshape le ciel s’éleva} dans l’opinion des hommes, {\itshape et avec lui s’élevèrent les dieux et les héros}.\par
Pour retrouver l’{\itshape astronomie poétique}, nous ferons usage de {\itshape trois vérités philologiques} : I. L’astronomie naquit chez les Chaldéens. II. Les Phéniciens apprirent des Chaldéens, et communiquèrent aux Égyptiens, l’usage du cadran, et la connaissance de l’élévation du pôle. III. Les Phéniciens, instruits par les mêmes Chaldéens, portèrent aux Grecs la connaissance des divinités qu’ils plaçaient dans les étoiles. — Avec ces trois vérités philologiques s’accordent  {\itshape deux principes philosophiques} : le premier est tiré de la nature sociale des peuples ; ils {\itshape admettent difficilement les dieux étrangers}, à moins qu’ils ne soient parvenus au dernier degré de liberté religieuse, ce qui n’arrive que dans une extrême décadence. Le second est {\itshape physique} ; l’erreur de nos yeux nous fait paraître {\itshape les planètes plus grandes que les étoiles fixes}.\par
Ces principes établis, nous dirons que chez toutes les nations païennes, de l’Orient, de l’Égypte, de la Grèce et du Latium, l’astronomie naquit uniformément d’une croyance vulgaire ; {\itshape les planètes paraissant beaucoup plus grandes que les étoiles fixes, les dieux montèrent dans les planètes, et les héros furent attachés aux constellations}. Aussi les Phéniciens trouvèrent les dieux et les héros de la Grèce et de l’Égypte déjà préparés à jouer ces deux rôles ; et les Grecs, à leur tour, trouvèrent dans ceux du Latium la même facilité. Les {\itshape héros}, et les {\itshape hiéroglyphes} qui signifiaient leurs caractères ou leurs entreprises, furent donc placés dans le {\itshape ciel}, ainsi qu’un grand nombre des {\itshape dieux principaux}, et servirent {\itshape l’astronomie des savants}, en donnant des noms aux étoiles. Ainsi, en partant de cette {\itshape astronomie vulgaire}, les premiers peuples écrivirent au {\itshape ciel} l’histoire de leurs dieux et de leurs héros……
\chapterclose


\chapteropen
\chapter[{Chapitre X. De la chronologie poétique}]{Chapitre X. \\
De la chronologie poétique}

\chaptercont
\noindent  Les {\itshape poètes théologiens} donnèrent à la {\itshape chronologie} des commencements conformes à une telle {\itshape astronomie}. Ce {\itshape Saturne}, qui chez les Latins tira son nom {\itshape à satis}, des semences, et qui fut appelé par les Grecs Κρόνος de Χρόνος, {\itshape le temps}, doit nous faire comprendre que les premières nations, toutes composées d’agriculteurs, commencèrent à compter les années par les récoltes de froment. C’est en effet la seule, ou du moins la principale chose dont la production occupe les agriculteurs toute l’année. Usant d’abord du langage muet, ils montrèrent autant d’{\itshape épis} ou de {\itshape brins de paille}, ou bien encore firent autant de fois {\itshape le geste de moissonner}, qu’ils voulaient indiquer d’{\itshape années}…\par
Dans la chronologie ordinaire, on peut remarquer quatre espèces d’anachronismes. 1º Temps {\itshape vides} de faits, qui devraient en être remplis ; tels que l’âge des dieux, dans lequel nous avons trouvé  les origines de tout ce qui touche la société, et que pourtant le savant Varron place dans ce qu’il appelle le {\itshape temps obscur}. 2º Temps {\itshape remplis} de faits, et qui devaient en être vides, tels que l’âge des héros, où l’on place tous les événements de l’âge des dieux, dans la supposition que toutes les fables ont été l’invention des poètes héroïques, et surtout d’Homère. 3º Temps {\itshape unis}, qu’on devait diviser ; pendant la vie du seul Orphée, par exemple, les Grecs, d’abord semblables aux bêtes sauvages, atteignent toute la civilisation qu’on trouve chez eux à l’époque de la guerre de Troie. 4º Temps {\itshape divisés} qui devaient être unis ; ainsi on place ordinairement la fondation des colonies grecques dans la Sicile et dans l’Italie, plus de trois siècles après les courses errantes des héros qui durent en être l’occasion.\par

\labelblock{Canon chronologique. \\
{\itshape Pour déterminer les commencements de l’histoire universelle, antérieurement au règne de Ninus d’où elle part ordinairement.}}

\noindent Nous voyons d’abord les hommes, en exceptant quelques-uns des enfants de Sem, dispersés à travers la vaste forêt qui couvrait la terre un siècle dans l’Asie orientale, et deux siècles dans le reste du monde. Le culte de Jupiter, que nous retrouvons partout chez les premières nations païennes, fixe les fondateurs des sociétés dans les lieux où les ont conduits leurs courses vagabondes, et alors commence l’âge des dieux qui dure neuf siècles. Déterminés dans le choix de leurs premières demeures par le besoin de trouver de l’eau et des aliments, ils ne peuvent se fixer d’abord sur le rivage de la mer, et les premières sociétés s’établissent dans l’intérieur des terres. Mais vers la fin du premier {\itshape âge}, les  peuples descendent plus près de la mer. Ainsi chez les Latins, il s’écoule plus de neuf cents ans depuis le {\itshape siècle} d’or du Latium, depuis l’{\itshape âge de Saturne} jusqu’au temps où Ancus Martius vient sur les bords de la mer s’emparer d’Ostie. — L’âge héroïque qui vient ensuite, comprend deux cents années pendant lesquelles nous voyons d’abord les courses de Minos, l’expédition des Argonautes, la guerre de Troie et les longs voyages des héros qui ont détruit cette ville. C’est alors, plus de mille ans après le déluge, que Tyr, capitale de la Phénicie, descend de l’intérieur des terres sur le rivage, pour passer ensuite dans une île voisine. Déjà elle est célèbre par la navigation et par les colonies qu’elle a fondées sur les côtes de la Méditerranée et même au-delà du détroit, avant les temps héroïques de la Grèce.\par
Nous avons prouvé l’uniformité du développement des nations, en montrant comment elles s’accordèrent à {\itshape élever leurs dieux jusqu’aux étoiles}, usage que les Phéniciens portèrent de l’Orient en Grèce et en Égypte. D’après cela, les Chaldéens durent régner dans l’Orient autant de siècles qu’il s’en écoula depuis Zoroastre jusqu’à Ninus, qui fonda la monarchie assyrienne, la plus ancienne du monde ; autant qu’on dut en compter depuis Hermès Trismégiste jusqu’à Sésostris, qui fonda aussi en Égypte une puissante monarchie. Les Assyriens et les Égyptiens, nations méditerranées, durent suivre dans les révolutions de leurs gouvernements la marche générale que nous avons indiquée. Mais les Phéniciens, nation maritime, enrichie par le commerce, durent s’arrêter dans la démocratie, le premier des gouvernements {\itshape humains}. (Voyez le 4\textsuperscript{e} liv.)\par
Ainsi par le simple secours de l’intelligence, et sans avoir besoin de celui de la mémoire, qui devient inutile lorsque les faits manquent pour frapper nos sens, nous avons rempli la lacune que présentait l’histoire universelle dans ses origines, tant pour l’ancienne Égypte que pour l’Orient plus ancien encore.\par
\par
De cette manière l’étude du {\itshape développement de la civilisation humaine}, prête une certitude nouvelle aux {\itshape calculs} de la chronologie. Conformément à l’axiome 106, {\itshape elle part du point même où commence le sujet qu’elle traite} : elle part de χρόνος, {\itshape le temps}, ou Saturne, ainsi appelé {\itshape a satis}, parce que l’on comptait les années par les récoltes ; d’{\itshape Uranie}, la muse qui contemple le ciel pour prendre les augures ; de Zoroastre, {\itshape contemplateur des astres}, qui rend des oracles d’après la direction des étoiles tombantes. Bientôt Saturne monte dans la septième sphère, Uranie contemple les planètes et les étoiles fixes, et les Chaldéens favorisés par l’immensité  de leurs plaines deviennent astronomes et astrologues, en mesurant le cercle que ces astres décrivent, en leur supposant diverses influences sur les corps sublunaires, et même sur les libres volontés de l’homme ; sous les noms d’{\itshape astronomie}, d’{\itshape astrologie} ou de {\itshape théologie} cette science ne fut autre que la {\itshape divination}. Du ciel les mathématiques descendirent pour mesurer la terre, sans toutefois pouvoir le faire avec certitude à moins d’employer les mesures fournies par les cieux. Dans leur partie principale elles furent nommées avec propriété {\itshape géométrie}.\par
C’est à tort que les chronologistes ne prennent point leur science au point même où commence le sujet qui lui est propre. Ils commencent avec l’année astronomique, laquelle n’a pu être connue qu’au bout de dix siècles au moins. Cette méthode pouvait leur faire connaître les conjonctions et les oppositions qui avaient pu avoir lieu dans le ciel entre les planètes ou les constellations ; mais ne pouvait leur rien apprendre de la succession des choses de la terre. Voilà ce qui a rendu impuissants les nobles efforts du cardinal Pierre d’Alliac. Voilà pourquoi l’histoire universelle a tiré si peu d’avantages pour éclairer son origine et sa suite du génie admirable et de l’étonnante érudition de Pétau et de Joseph Scaliger.
\chapterclose


\chapteropen
\chapter[{Chapitre XI. De la géographie poétique}]{Chapitre XI. \\
De la géographie poétique}

\chaptercont
\noindent  La {\itshape géographie poétique}, l’autre œil de l’{\itshape histoire fabuleuse}, n’a pas moins besoin d’être éclaircie que la {\itshape chronologie poétique}. En conséquence d’un de nos axiomes ({\itshape les hommes qui veulent expliquer aux autres des choses inconnues et lointaines dont ils n’ont pas la véritable idée, les décrivent en les assimilant à des choses connues et rapprochées}), la {\itshape géographie poétique}, prise dans ses parties et dans son ensemble, naquit dans l’enceinte de la Grèce, sous des proportions resserrées. Les Grecs sortant de leur pays pour se répandre dans le monde, la géographie alla s’étendant jusqu’à ce qu’elle atteignit les limites que nous lui voyons aujourd’hui. Les géographes anciens s’accordent à reconnaître une vérité dont ils n’ont point su faire usage : c’est que {\itshape les anciennes nations, émigrant dans des contrées étrangères et lointaines, donnèrent des noms tirés de leur ancienne patrie, aux cités, aux montagnes et aux fleuves, aux isthmes et aux détroits, aux îles et aux promontoires}.\par
 C’est dans l’enceinte même de la Grèce que l’on plaça d’abord la partie {\itshape orientale} appelée {\itshape Asie} ou {\itshape Inde}, l’{\itshape occidentale} appelée {\itshape Europe} ou {\itshape Hespérie}, la {\itshape septentrionale}, nommée {\itshape Thrace} ou {\itshape Scythie}, enfin la {\itshape méridionale}, dite {\itshape Lybie} ou {\itshape Mauritanie}. Les parties du {\itshape monde} furent ainsi appelées du nom des parties du {\itshape petit monde de la Grèce}, selon la situation des premières relativement à celle des dernières. Ce qui le prouve, c’est que les {\itshape vents cardinaux} conservent dans leur géographie les noms qu’ils durent avoir originairement dans l’intérieur de la Grèce.\par
D’après ces principes, la grande péninsule située à l’orient de la Grèce conserva le nom d’{\itshape Asie Mineure}, après que le nom d’{\itshape Asie} eut passé à cette vaste partie {\itshape orientale} du monde, que nous appelons ainsi dans un sens absolu. Au contraire, la Grèce, qui était à l’{\itshape occident} par rapport à l’Asie, fut appelée {\itshape Europe}, et ensuite ce nom s’étendit au grand continent, que limite l’Océan occidental. — Ils appelèrent d’abord {\itshape Hespérie} la partie {\itshape occidentale} de la Grèce, sur laquelle se levait le soir l’étoile {\itshape Hesperus}. Ensuite, voyant l’Italie dans la même situation, ils la nommèrent {\itshape Grande Hespérie}. Enfin, étant parvenus jusqu’à l’Espagne, ils la désignèrent comme la {\itshape dernière Hespérie}. — Les Grecs d’Italie, au contraire, durent appeler {\itshape Ionie} la partie de la Grèce qui était {\itshape orientale} relativement à eux, et la mer qui sépare la Grande-Grèce de la Grèce proprement dite, en garde le nom d’Ionienne ; ensuite l’analogie de situation entre la Grèce proprement dite et  la Grèce Asiatique, fit appeler {\itshape Ionie}, par les habitants de la première, la partie de l’Asie Mineure qui se trouvait à leur orient. [Il est probable que Pythagore vint en Italie de Samé, partie du royaume d’Ulysse, située dans la {\itshape première Ionie}, plutôt que de Samos, située dans la seconde.] — De la {\itshape Thrace grecque} vinrent Mars et Orphée ; ce dieu et ce poète théologien ont évidemment une origine grecque. De la {\itshape Scythie grecque} vint Anacharsis avec ses oracles scythiques non moins faux que les vers d’Orphée. De la même partie de la Grèce sortirent les Hyperboréens, qui fondèrent les oracles de Delphes et de Dodone. C’est dans ce sens que Zamolxis fut {\itshape Gète}, et Bacchus {\itshape Indien}. — Le nom de {\itshape Morée}, que le Péloponnèse conserve jusqu’à nos jours, nous prouve assez que Persée, héros d’une origine évidemment grecque, fit ses exploits célèbres dans la {\itshape Mauritanie grecque} ; le royaume de Pélops ou Péloponnèse a l’Achaïe au nord, comme l’Europe est au nord de l’Afrique. Hérodote raconte qu’autrefois les {\itshape Maures furent blancs}, ce qu’on ne peut entendre que des {\itshape Maures de la Grèce}, dont le pays est appelé encore aujourd’hui {\itshape la Morée blanche}. — Les Grecs avaient d’abord appelé {\itshape Océan} toute mer d’un aspect sans bornes, et Homère avait dit que l’île d’Éole était ceinte par l’{\itshape Océan}. Lorsqu’ils arrivèrent à l’{\itshape Océan} véritable, ils étendirent cette idée étroite, et désignèrent par le nom d’{\itshape Océan} la mer qui embrasse toute la terre comme une grande île\footnote{\noindent {\itshape Ces principes de géographie} peuvent justifier {\itshape Homère} d’erreurs très graves qui lui sont imputées à tort. Par exemple les {\itshape Cimmériens} durent avoir, comme il le dit, des nuits plus longues que tous les peuples de la {\itshape Grèce}, parce qu’ils étaient placés dans sa partie la plus septentrionale ; ensuite on a reculé l’habitation des {\itshape Cimmériens} jusqu’aux {\itshape Palus-Méotides}. On disait à cause de leurs longues nuits qu’ils habitaient près des enfers, et les habitants de {\itshape Cumes}, voisins de la grotte de la Sibylle qui conduisait aux enfers, reçurent, à cause de cette prétendue analogie de situation, le nom de {\itshape Cimmériens}. Autrement il ne serait point croyable qu’Ulysse, voyageant sans le secours des enchantements (contre lesquels Mercure lui avait donné un préservatif), fût allé en un jour voir l’enfer chez les {\itshape Cimmériens des Palus-Méotides}, et fût revenu le même jour à {\itshape Circéi}, maintenant le mont Circello, près de Cumes. — Les {\itshape Lotophages} et les {\itshape Lestrigons} durent aussi être voisins de la Grèce.Les mêmes {\itshape principes de géographie poétique} peuvent résoudre de grandes difficultés dans l’{\itshape Histoire ancienne de l’Orient}, où l’on éloigne beaucoup vers le {\itshape nord} ou le {\itshape midi} des peuples qui durent être placés d’abord dans l’{\itshape orient} même.\par
Ce que nous disons de la {\itshape Géographie des Grecs} se représente dans celle des {\itshape Latins}. Le {\itshape Latium} dut être d’abord bien resserré, puisqu’en deux siècles et demi, Rome, sous ses rois, soumit à peu près {\itshape vingt peuples} sans étendre son empire à plus de {\itshape vingt milles}. L’{\itshape Italie} fut certainement circonscrite par la Gaule Cisalpine et par la Grande-Grèce ; ensuite les conquêtes des Romains étendirent ce nom à toute la Péninsule. La {\itshape mer d’Étrurie} dut être bien limitée lorsqu’Horatius-Coclès arrêtait seul toute l’Étrurie sur un pont ; ensuite ce nom s’est étendu par les victoires de Rome à toute cette mer qui baigne la côte inférieure de l’Italie. De même le {\itshape Pont} où Jason conduisit les Argonautes, dut être la terre la plus voisine de l’Europe, celle qui n’en est séparée que par l’étroit bassin appelé {\itshape Propontide} ; cette terre dut donner son nom à la mer du {\itshape Pont}, et ce nom s’étendit à tout le golfe que présente l’Asie, dans cette partie de ses rivages où fut depuis le royaume de Mithridates ; le père de Médée, selon la même fable, était né à Chalcis, dans cette ville grecque de l’Eubée qui s’appelle maintenant {\itshape Négrepont}. — La première {\itshape Crète} dut être une île dans cet Archipel où les Cyclades forment une sorte de {\itshape labyrinthe} ; c’est de là probablement que Minos allait en course contre les Athéniens ; dans la suite, la {\itshape Crète} sortit de la mer Égée pour se fixer dans celle où nous la plaçons.\par
Puisque des Latins nous sommes revenus aux Grecs, remarquons que cette nation vaine en se répandant dans le monde, y célébra partout {\itshape la guerre de Troie} et {\itshape les voyages des héros errants} après sa destruction, des héros grecs, tels que Ménélas, Diomède, Ulysse, et des héros troyens, tels que Anténor, Capys, Énée. Les Grecs ayant retrouvé dans toutes les contrées du monde un {\itshape caractère de fondateurs des sociétés} analogue à celui de leur {\itshape Hercule de Thèbes}, ils placèrent partout son nom et le firent voyager par toute la terre qu’il purgeait de monstres sans en rapporter dans sa patrie autre chose que de la gloire. Varron compte environ quarante {\itshape Hercules}, et il affirme que celui des Latins s’appelait {\itshape Dius Fidius} ; les Égyptiens, aussi vains que les Grecs, disaient que leur {\itshape Jupiter Ammon} était le plus ancien des {\itshape Jupiter}, et que les {\itshape Hercules} des autres nations avaient pris leur nom de l’{\itshape Hercule Égyptien}. Les Grecs observèrent encore qu’il y avait eu partout un {\itshape caractère poétique de bergers parlant en vers} ; chez eux c’était {\itshape Évandre l’Arcadien} ; Évandre ne manqua pas de passer de l’Arcadie dans le {\itshape Latium}, où il donna l’hospitalité à l’{\itshape Hercule grec}, son compatriote, et prit pour femme {\itshape Carmenta}, ainsi nommée de {\itshape carmina, vers} ; elle trouva chez les Latins {\itshape les lettres}, c’est-à-dire, les {\itshape formes} des sons articulés qui sont la {\itshape matière} des vers. Enfin ce qui confirme tout ce que nous venons de dire, c’est que les Grecs observèrent ces {\itshape caractères poétiques} dans le Latium, en même temps qu’ils trouvèrent leurs {\itshape Curètes} répandus dans la {\itshape Saturnie}, c’est-à-dire dans l’ancienne Italie, dans la Crète et dans l’Asie.\par
Mais comme ces mots et ces idées passèrent des {\itshape Grecs} aux {\itshape Latins} dans un temps où les nations, encore très {\itshape sauvages}, étaient {\itshape fermées aux étrangers*}, nous avons demandé plus haut qu’on nous passât la conjecture suivante : {\itshape Il peut avoir existé sur le rivage du Latium une cité grecque, ensevelie depuis dans les ténèbres de l’antiquité, laquelle aurait donné aux Latins les lettres de l’alphabet.} Tacite nous apprend que les lettres latines furent d’abord semblables \emph{{\itshape aux plus anciennes}} des Grecs, ce qui est une forte preuve que les Latins ont reçu l’alphabet grec de ces {\itshape Grecs du Latium}, et non de la Grande-Grèce, encore moins de la Grèce proprement dite ; car s’il en eût été ainsi, ils n’eussent connu ces lettres qu’au temps de la guerre de Tarente et de Pyrrhus, et alors ils se seraient servis {\itshape des plus modernes}, et non pas {\itshape des anciennes}.\par
Les noms d’{\itshape Hercule}, d’{\itshape Évandre} et d’{\itshape Énée} passèrent donc de la Grèce dans le Latium, par l’effet de quatre causes que nous trouverons {\itshape dans les mœurs et le caractère des nations} : 1º les peuples encore barbares sont attachés aux coutumes de leur pays, mais à mesure qu’ils commencent à se civiliser, ils prennent du goût pour {\itshape les façons de parler des étrangers}, comme pour leurs marchandises et leurs manières ; c’est ce qui explique pourquoi les Latins changèrent leur {\itshape Dius Fidius} pour l’Hercule des Grecs, et leur jurement national {\itshape Medius Fidius} pour {\itshape Mehercule, Mecastor, Edepol}. 2º La vanité des nations, nous l’avons souvent répété, les porte à se donner l’{\itshape illustration d’une origine étrangère}, surtout lorsque les traditions de leurs âges barbares semblent favoriser cette croyance ; ainsi, au moyen âge, Jean Villani nous raconte que Fiesole fut fondé par Atlas, et qu’un roi troyen du nom de Priam régna en Germanie ; ainsi les Latins méconnurent sans peine leur véritable fondateur, pour lui substituer {\itshape Hercule}, fondateur de la société chez les Grecs, et changèrent le {\itshape caractère de leurs bergers-poètes} pour celui de l’{\itshape Arcadien Évandre}. 3º Lorsque les nations remarquent des {\itshape choses étrangères}, qu’elles ne peuvent bien expliquer avec des mots de leur langue, {\itshape elles ont} nécessairement {\itshape recours aux mots des langues étrangères}. 4º Enfin, les premiers peuples, incapables d’abstraire d’un sujet les qualités qui lui sont propres, {\itshape nomment les sujets pour désigner les qualités}, c’est ce que prouvent d’une manière certaine plusieurs expressions de la langue latine. Les Romains ne savaient ce que c’était que {\itshape luxe} ; lorsqu’ils l’eurent observé dans les Tarentins, ils dirent {\itshape un Tarentin} pour {\itshape un homme parfumé}. Ils ne savaient ce que c’était que {\itshape stratagème militaire} ; lorsqu’ils l’eurent observé dans les Carthaginois, ils appelèrent les stratagèmes {\itshape punicas artes}, les arts puniques ou carthaginois. Ils n’avaient point l’idée du {\itshape faste} ; lorsqu’ils le remarquèrent dans les Capouans, ils dirent {\itshape supercilium campanicum}, pour {\itshape fastueux, superbe}. C’est de cette manière que Numa et Ancus furent {\itshape Sabins} ; les Sabins étant remarquables par leur piété, les Romains dirent {\itshape Sabin}, faute de pouvoir exprimer {\itshape religieux}. Servius Tullius fut {\itshape Grec} dans le langage des Romains, parce qu’ils ne savaient pas dire {\itshape habile et rusé}.\par
Peut-être doit-on comprendre de cette manière les {\itshape Arcadiens d’Évandre}, et les {\itshape Phrygiens d’Énée}. Comment des {\itshape bergers}, qui ne savaient ce que c’est que la mer, seraient-ils sortis de l’Arcadie, contrée toute méditerranée de la Grèce, pour tenter une si longue navigation et pénétrer jusqu’au milieu du Latium ? Cependant toute tradition vulgaire doit avoir originairement quelque cause publique, quelque fondement de vérité.… Ce sont les Grecs qui, chantant par tout le monde leur guerre de Troie et les aventures de leurs héros, {\itshape ont fait d’Énée le fondateur de la nation romaine}, tandis que, selon Bochart, il ne mit jamais le pied en Italie, que Strabon assure qu’il ne sortit jamais de Troie, et qu’Homère, dont l’autorité a plus de poids ici, raconte qu’il y mourut et qu’il laissa le trône à sa postérité. Cette fable, inventée par la vanité des Grecs et adoptée par celle des Romains, ne put naître qu’{\itshape au temps de la guerre de Pyrrhus}, époque à laquelle les Romains commencèrent à accueillir ce qui venait de la Grèce.\par
Il est plus naturel de croire qu’il exista sur le rivage du Latium une cité grecque qui, vaincue par les Romains, fut détruite en vertu du droit héroïque des nations barbares, que les vaincus furent reçus à Rome dans la classe des plébéiens, et que, dans le langage poétique, on appela dans la suite {\itshape Arcadiens} ceux d’entre les vaincus qui avaient d’abord erré dans les forêts, {\itshape Phrygiens} ceux qui avaient erré sur mer.\par
* Tite-Live assure qu’à l’époque de Servius Tullius, le nom si célèbre de Pythagore n’aurait pu parvenir de Crotone à Rome à travers tant de nations séparées par la diversité de leurs langues et de leurs mœurs. ({\itshape Vico.})
}\footnote{\noindent La {\itshape géographie} comprenant la {\itshape nomenclature} et la {\itshape chorographie} ou description des lieux, principalement des cités, il nous reste à la considérer sous ce double aspect pour achever ce que nous avions à dire de la {\itshape sagesse poétique}.Nous avons remarqué plus haut que les {\itshape cités héroïques} furent fondées par la Providence dans des lieux d’une forte position, désignés par les Latins, dans la langue sacrée de leur âge divin, par le nom d’{\itshape Ara}, ou bien d’{\itshape Arces} (de là, au moyen âge, l’italien {\itshape rocche}, et ensuite {\itshape castella} pour {\itshape seigneuries}). Ce nom d’{\itshape Ara} dut s’étendre à tout le pays dépendant de chaque cité héroïque, lequel s’appelait aussi {\itshape Ager}, lorsqu’on le considérait sous le rapport des limites communes avec les cités étrangères, et {\itshape territorium} sous le rapport de la juridiction de la cité sur les citoyens. Il y a sur ce sujet un passage remarquable de Tacite ; c’est celui où il décrit l’{\itshape Ara maxima} d’Hercule à Rome \emph{{\itshape  : Igitur à foro boario, ubi œneum bovis simulacrum adspicimus, quia id genus animalium aratro subditur, sulcus designandi oppidi captus, ut magnam Herculis aram complecteretur, ara Herculis erat.}} Joignez-y le passage curieux où Salluste parle de la fameuse {\itshape Ara} des frères Philènes, qui servait de limites à l’empire carthaginois et à la Cyrénaïque. Toute l’ancienne géographie est pleine de semblables {\itshape aræ} ; et pour commencer par l’Asie, Cellarius observe que toutes les cités de la Syrie prenaient le nom d’{\itshape Are}, avant ou après leurs noms particuliers ; ce qui faisait donner à la Syrie elle-même celui d’{\itshape Aramea} ou {\itshape Aramia}. Dans la Grèce, Thésée fonda la cité d’Athènes en érigeant le fameux {\itshape autel des malheureux}. Sans doute il comprenait avec raison sous cette dénomination les vagabonds sans lois et sans culte qui, pour échapper aux rixes continuelles de l’état bestial, cherchaient un asile dans les lieux forts occupés par les premières sociétés, faibles qu’ils étaient par leur isolement, et manquant de tous les biens que la civilisation assurait déjà aux hommes réunis par la religion.\par
Les Grecs prenaient encore αρα dans le sens de {\itshape vœu, action de dévouer}, parce que les premières victimes {\itshape saturni hostiæ}, les premiers αναθήματα, {\itshape diris devoti}, furent immolés sur les premières {\itshape Aræ}, dans le sens où nous prenons ce mot. Ces premières victimes furent les hommes encore sauvages qui osèrent poursuivre sur les terres labourées par les forts, les faibles qui s’y réfugiaient ({\itshape campare} en italien, du latin {\itshape campus}, pour {\itshape se sauver}). Ils y étaient consacrés à {\itshape Vesta} et immolés. Les Latins en ont conservé {\itshape supplicium}, dans les deux sens de {\itshape supplice} et de {\itshape sacrifice}. En cela la langue grecque répond à la langue latine : ἄρα, {\itshape vœu, action de dévouer} veut dire aussi {\itshape noxa}, la personne ou la chose coupable, et de plus {\itshape diræ}, les Furies. Les premiers coupables qu’on dévoua, {\itshape primæ noxæ}, étaient consacrés aux Furies, et ensuite sacrifiés sur les premières {\itshape aræ} dont nous avons parlé. Le mot {\itshape hara} dut signifier chez les anciens Latins, non pas le lieu où l’on élève les troupeaux, mais la {\itshape victime}, d’où vint certainement {\itshape haruspex}, celui qui tire les présages de l’examen des entrailles des victimes immolées devant les autels.\par
D’après ce que nous avons vu relativement à l’{\itshape Ara maxima} d’Hercule, c’est sur une {\itshape ara} semblable à celle de Thésée que Romulus dut fonder Rome, en ouvrant un asile dans un bois. Jamais les Latins ne parlent d’un bois sacré, {\itshape lucus}, sans faire mention d’un autel, {\itshape ara}, élevé dans ce bois à quelque divinité. Aussi lorsque Tite-Live nous dit en général que les asiles furent le moyen employé d’ordinaire par les anciens fondateurs des villes, \emph{{\itshape vetus urbes condentium consilium}}, il nous indique la raison pour laquelle on trouve dans l’ancienne géographie tant de cités avec le nom d’{\itshape Aræ}. Nous avons parlé de l’Asie et de l’Afrique, mais il en est de même en Europe, particulièrement en Grèce, en Italie, et maintenant encore en Espagne. Tacite mentionne en Germanie l’{\itshape Ara Ubiorum}. De nos jours on donne ce nom en Transilvanie à plusieurs cités.\par
C’est aussi de ce mot {\itshape Ara}, prononcé et entendu d’une manière si uniforme par tant de nations séparées par les temps, les lieux et les usages, que les Latins durent tirer le mot {\itshape aratrum}, charrue, dont la courbure se disait {\itshape urbs} (le sens le plus ordinaire de ce mot est celui de {\itshape ville}) ; du même mot vinrent enfin {\itshape arx}, forteresse, {\itshape arceo}, repousser ({\itshape ager arcifinius}, chez les auteurs qui ont écrit {\itshape sur les limites des champs}), et {\itshape arma, arcus}, armes, arc ; c’était une idée bien sage de faire ainsi consister le courage à arrêter et repousser l’injustice. Ἄρης, {\itshape Mars} vint sans doute de la défense des {\itshape aræ}. ({\itshape Vico.})
}.
\chapterclose


\chapteropen
\chapter[{Conclusion de ce livre}]{Conclusion de ce livre}

\chaptercont
\noindent  Nous avons démontré que la {\scshape sagesse poétique} mérite deux magnifiques éloges, dont l’un lui a été constamment attribué. I. C’est elle qui {\itshape fonda l’humanité chez les Gentils}, gloire que la vanité des nations et des savants a voulu lui assurer, et lui aurait plutôt enlevée. II. L’autre gloire lui a été attribuée jusqu’à nous par une tradition vulgaire ; c’est que la {\itshape sagesse antique}, par une même inspiration, {\itshape rendait ses sages également grands comme philosophes, comme législateurs et capitaines, comme historiens, orateurs et poètes}. Voilà pourquoi elle a été tant regrettée ; cependant, dans la réalité, elle ne fit que les {\itshape ébaucher}, tels que nous les avons trouvés dans les fables ; ces germes féconds nous ont laissé voir dans l’imperfection de sa forme primitive la {\itshape science} de réflexion, la science de recherches, ouvrage tardif de la philosophie. On peut dire en effet que dans les {\itshape fables, l’instinct} de l’humanité avait marqué d’avance les principes de la science moderne, que les {\itshape méditations} des savants ont depuis éclairée par des {\itshape raisonnements}, et résumée dans des {\itshape maximes}. Nous pouvons conclure par le principe dont la démonstration était l’objet de ce livre : {\itshape Les poètes théologiens furent le sens, les philosophes furent l’}intelligence {\itshape de la sagesse humaine}.
\chapterclose

\chapterclose


\chapteropen
\part[{Livre troisième. Découverte du véritable Homère}]{Livre troisième. \\
Découverte du véritable Homère}

\chaptercont

\chapteropen
\chapter[{Argument}]{Argument}

\chaptercont
\noindent  {\itshape Ce livre n’est qu’un appendice du précédent. C’est une application de la méthode qu’on y a suivie, au plus ancien auteur du paganisme, à celui qu’on a regardé comme le fondateur de la civilisation grecque, et par suite de celle de l’Europe. L’auteur entreprend de prouver} : 1º {\itshape qu’Homère n’a pas été philosophe} ; 2º {\itshape qu’il a vécu pendant plus de quatre siècles} ; 3º {\itshape que toutes les villes de la Grèce ont eu raison de le revendiquer pour citoyen} ; 4º {\itshape qu’il a été, par conséquent, non pas un individu, mais un être collectif, un} symbole du peuple grec racontant sa propre histoire dans des chants nationaux.\par
Chapitre I\textsuperscript{er}. {\scshape De la sagesse philosophique que l’on attribue à Homère.} {\itshape La force et l’originalité avec lesquelles il a peint des mœurs barbares, prouvent qu’il partageait les passions de ses héros. Un philosophe n’aurait pu, ni voulu peindre si naïvement de telles mœurs.}\par
Chapitre II. {\scshape De la patrie d’Homère.} {\itshape Vico conjecture que l’auteur ou les auteurs de l’Odyssée eurent pour patrie les contrées occidentales de la Grèce ; ceux de l’Iliade, l’Asie Mineure. Chaque ville grecque revendiqua Homère pour citoyen},  {\itshape parce qu’elle reconnaissait quelque chose de son dialecte vulgaire dans l’Iliade ou l’Odyssée.}\par
Chapitre III. {\scshape Du temps où vécut Homère.} {\itshape Un grand nombre de passages indiquent des époques de civilisation très diverses, et portent à croire que les deux poèmes ont été travaillés par plusieurs mains, et continués pendant plusieurs âges.}\par
Chapitre IV. {\scshape Pourquoi le génie d’Homère dans la poésie héroïque ne peut jamais être égalé.} {\itshape C’est que les caractères des héros qu’il a peints ne se rapportent pas à des êtres individuels, mais sont plutôt des symboles populaires de chaque caractère moral. Observations sur la comédie et la tragédie.}\par
Chapitres V et VI. {\scshape Observations philosophiques et philologiques}, {\itshape qui doivent servir à la découverte du véritable Homère. La plupart des observations philosophiques rentrent dans ce qui a été dit au second livre, sur l’origine de la poésie.}\par
Chapitre VII. § I. {\scshape Découverte du véritable Homère.} — § II. {\itshape Tout ce qui était absurde et invraisemblable dans l’Homère que l’on s’est figuré jusqu’ici, devient dans notre Homère convenance et nécessité.} — § III. {\itshape On doit trouver dans les poèmes d’Homère les deux principales sources des faits relatifs au droit naturel des gens, considéré chez les Grecs.}\par
Appendice. {\scshape Histoire raisonnée des poètes dramatiques et lyriques.} {\itshape Trois âges dans la poésie lyrique, comme dans la tragédie.}
\chapterclose


\chapteropen
\chapter[{[Introduction]}]{[Introduction]}

\chaptercont
\noindent  Avoir démontré, comme nous l’avons fait dans le livre précèdent, que la {\itshape sagesse poétique} fut la {\itshape sagesse vulgaire} des peuples grecs, d’abord {\itshape poètes théologiens}, et ensuite {\itshape héroïques}, c’est avoir prouvé d’une manière implicite la même vérité relativement à la {\itshape sagesse d’Homère}. Mais Platon prétend au contraire qu’Homère posséda \emph{{\itshape la sagesse réfléchie} ({\itshape riposta}){\itshape  des âges civilisés}} ; et il a été suivi dans cette opinion par tous les philosophes, spécialement par Plutarque, qui a consacré à ce sujet un livre tout entier. Ce préjugé est trop profondément enraciné dans les esprits, pour qu’il ne soit pas nécessaire d’examiner particulièrement {\itshape si Homère a jamais été philosophe}. Longin avait cherché à résoudre ce problème dans un ouvrage dont fait mention Diogène Laërce dans la vie de Pyrrhon.
\chapterclose


\chapteropen
\chapter[{Chapitre I. De la sagesse philosophique que l’on a attribuée à Homère}]{Chapitre I. \\
De la sagesse philosophique que l’on a attribuée à Homère}

\chaptercont
\noindent  Nous accorderons, d’abord, comme il est juste, qu’{\itshape Homère a dû suivre les sentiments vulgaires}, et par conséquent {\itshape les mœurs vulgaires de ses contemporains} encore barbares ; de tels sentiments, de telles mœurs fournissent à la poésie les sujets qui lui sont propres. Passons-lui donc d’avoir présenté {\itshape la force} comme la mesure de la grandeur des dieux ; laissons Jupiter démontrer, par la force avec laquelle il enlèverait {\itshape la grande chaîne} de la fable, qu’il est {\itshape le roi des dieux et des hommes} ; laissons {\itshape Diomède, secondé par Minerve, blesser Vénus et Mars} ; la chose n’a rien d’invraisemblable dans un pareil système ; laissons Minerve, dans le combat des dieux, {\itshape dépouiller Vénus et frapper Mars d’un coup de pierre}, ce qui peut faire juger si elle était la déesse de la philosophie dans la croyance vulgaire ; passons encore au poète de nous avoir rappelé fidèlement l’usage d’{\itshape empoisonner les flèches}\footnote{Usage barbare dont les nations se seraient constamment abstenues si l’on en croyait les auteurs qui ont écrit sur le droit des gens, et qui pourtant était alors pratiqué par ces Grecs auxquels on attribue la gloire d’avoir répandu la civilisation dans le monde. ({\itshape Vico.})}, comme  le fait le héros de l’Odyssée, qui va exprès à Éphyre pour y trouver des herbes vénéneuses ; l’usage enfin de {\itshape ne point ensevelir les ennemis tués dans les combats}, mais de les laisser {\itshape pour être la pâture des chiens et des vautours}.\par
Cependant, la fin de la poésie {\itshape étant d’adoucir la férocité du vulgaire}, de l’esprit duquel les poètes disposent en maîtres, {\itshape il n’était point d’un homme sage} d’inspirer au vulgaire de l’admiration pour des {\itshape sentiments et des coutumes si barbares}, et de le confirmer dans les uns et dans les autres par le plaisir qu’il prendrait à les voir si bien peints. {\itshape Il n’était point d’un homme sage} d’amuser le peuple {\itshape grossier}, de la {\itshape grossièreté} des héros et des dieux. Mars, en combattant Minerve, l’appelle \emph{κυνόμυια ({\itshape musca canina})} ; Minerve donne un coup de poing à Diane ; Achille et Agamemnon, le premier des héros et le roi des rois, se donnent l’épithète de {\itshape chien}, et se traitent comme le feraient à peine des valets de comédie.\par
Comment appeler autrement que {\itshape sottise} la prétendue {\itshape sagesse} du général en chef Agamemnon, qui a besoin d’être forcé par Achille à restituer Chryséis au prêtre d’Apollon, son père, tandis que le dieu, pour venger Chryséis, ravage l’armée des Grecs par une peste cruelle ? Ensuite le roi des rois, se regardant comme outragé, croit rétablir son honneur  en déployant une {\itshape justice} digne de la {\itshape sagesse} qu’il a montrée. Il enlève Briséis à Achille, sans doute afin que ce héros, {\itshape qui portait avec lui le destin de Troie}, s’éloigne avec ses guerriers et ses vaisseaux, et qu’Hector égorge le reste des Grecs que la peste a pu épargner… Voilà pourtant le poète qu’on a jusqu’ici regardé comme le {\itshape fondateur de la civilisation des Grecs}, comme l’{\itshape auteur de la politesse de leurs mœurs}. C’est du récit que nous venons de faire qu’il déduit toute l’Iliade ; ses principaux acteurs sont un tel capitaine, un tel héros ! Voilà le poète {\itshape incomparable dans la conception des caractères poétiques} ! Sans doute il mérite cet éloge, mais dans un autre sens, comme on le verra dans ce livre. Ses caractères les plus sublimes choquent en tout les idées d’un âge civilisé, mais ils sont {\itshape pleins de convenance}, si on les rapporte à la nature {\itshape héroïque} des hommes {\itshape passionnés et irritables} qu’il a voulu peindre.\par
Si Homère est un {\itshape sage}, un {\itshape philosophe}, que dire de la passion de ses héros pour le {\itshape vin} ? Sont-ils affligés, leur consolation c’est de s’{\itshape enivrer}, comme fait particulièrement le sage Ulysse. Scaliger s’indigne de voir toutes ces {\itshape comparaisons tirées des objets les plus sauvages, de la nature la plus farouche}. Admettons cependant qu’Homère a été forcé de les choisir ainsi pour se faire mieux entendre du vulgaire, alors si {\itshape farouche} et si {\itshape sauvage} ; cependant le bonheur même de ces comparaisons, leur mérite incomparable, n’indique pas certainement un esprit  {\itshape adouci} et {\itshape humanisé par la philosophie}. Celui en qui les leçons des {\itshape philosophes} auraient développé les sentiments de l’{\itshape humanité} et de la {\itshape pitié} n’aurait pas eu non plus ce {\itshape style si fier et d’un effet si terrible} avec lequel il décrit dans toute la variété de leurs accidents, les plus sanglants {\itshape combats}, avec lequel il diversifie de cent manières bizarres les tableaux de {\itshape meurtre} qui font la sublimité de l’Iliade. La {\itshape constance d’âme} que donne et assure l’étude de la {\itshape sagesse philosophique} pouvait-elle lui permettre de supposer tant de {\itshape légèreté}, tant de {\itshape mobilité} dans les dieux et les héros ; de montrer les uns, sur le moindre motif, passant du plus grand trouble à un calme subit ; les autres, dans l’accès de la plus violente colère, se rappelant un souvenir touchant, et fondant en larmes\footnote{Au moyen âge, dont l’{\itshape Homère toscan} (Dante) n’a chanté que des {\itshape faits réels}, nous voyons que Rienzi, exposant aux Romains l’oppression dans laquelle ils étaient tenus par les nobles, fut interrompu par ses sanglots et par ceux de tous les assistants. La vie de Rienzi par un auteur contemporain nous représente au naturel les {\itshape mœurs héroïques} de la Grèce, telles qu’elles sont peintes dans Homère. ({\itshape Vico.}) {\itshape Voy.} dans la note du discours le jugement sur Dante.} ; d’autres au contraire, navrés de douleur, oubliant tout-à-coup leurs maux, et s’abandonnant à la joie, à la première distraction agréable, comme le sage Ulysse au banquet d’Alcinoüs ; d’autres enfin, d’abord calmes et tranquilles, s’irritant d’une parole dite sans intention de leur déplaire, et s’emportant au point de menacer de la mort celui qui l’a prononcée. Ainsi Achille reçoit dans sa tente l’infortuné Priam, qui est  venu seul pendant la nuit à travers le camp des Grecs, pour racheter le cadavre d’Hector ; il l’admet à sa table, et pour un mot que lui arrache le regret d’avoir perdu un si digne fils, Achille oublie les saintes lois de l’hospitalité, les droits d’une confiance généreuse, le respect dû à l’âge et au malheur ; et dans le transport d’une fureur aveugle, il menace le vieillard de lui arracher la vie. Le même Achille refuse, dans son obstination impie, d’oublier en faveur de sa patrie l’injure d’Agamemnon, et ne secourt enfin les Grecs massacrés indignement par Hector, que pour venger le ressentiment particulier que lui inspire contre Pâris la mort de Patrocle. Jusque dans le tombeau, il se souvient de l’enlèvement de Briséis ; il faut que la belle et malheureuse Polixène soit immolée sur son tombeau, et apaise par l’effusion du sang innocent ses cendres altérées de vengeance.\par
Je n’ai pas besoin de dire qu’on ne peut guère comprendre comment {\itshape un esprit grave, un philosophe habitué à combiner ses idées d’une manière raisonnable}, se serait occupé à imaginer ces contes de vieilles, bons pour amuser les enfants, et dont Homère a rempli l’Odyssée.\par
Ces mœurs {\itshape sauvages} et {\itshape grossières, fières} et {\itshape farouches}, ces caractères {\itshape déraisonnables} et {\itshape déraisonnablement obstinés}, quoique souvent {\itshape d’une mobilité et d’une légèreté puériles}, ne pouvaient appartenir, comme nous l’avons démontré ({\scshape livre} II, {\itshape Corollaires de la nature héroïque}), qu’à des hommes {\itshape faibles}  {\itshape d’esprit} comme des enfants, {\itshape doués d’une imagination vive} comme celle des femmes, {\itshape emportés dans leurs passions} comme les jeunes gens les plus violents. Il faut donc refuser à Homère toute {\itshape sagesse philosophique}.\par
Voilà l’origine des {\itshape doutes} qui nous forcent de rechercher quel fut le {\scshape véritable Homère}.
\chapterclose


\chapteropen
\chapter[{Chapitre II. De la patrie d’Homère}]{Chapitre II. \\
De la patrie d’Homère}

\chaptercont
\noindent  Presque toutes les cités de la Grèce se disputèrent la gloire d’avoir donné le jour à Homère. Plusieurs auteurs ont même cherché sa patrie dans l’Italie, et Léon Allacci ({\itshape De patriâ Homeri}) s’est donné une peine inutile pour la déterminer. S’il est vrai qu’il n’existe point d’écrivain plus ancien qu’Homère, comme Josèphe le soutient contre Apion le grammairien, si les écrivains que nous pourrions consulter ne sont venus que longtemps après lui, il faut bien que nous employions notre {\itshape critique métaphysique} à trouver dans Homère lui-même et son siècle et sa patrie, en le considérant moins comme {\itshape auteur de livre}, que comme {\itshape auteur} ou fondateur de {\itshape nation} ; et en effet, il a été considéré comme le fondateur de la civilisation grecque.\par
L’{\itshape auteur de l’Odyssée} naquit sans doute dans les parties occidentales de la Grèce, en tirant vers le midi. Un passage précieux justifie cette conjecture : Alcinoüs, roi de l’île des Phéaciens, maintenant Corfou,  offre à Ulysse un vaisseau bien équipé, pour le ramener dans son pays, et lui fait remarquer que ses sujets, \emph{{\itshape experts dans la marine, seraient en état, s’il le fallait, de le conduire jusqu’en Eubée}} ; c’était, au rapport de ceux que le hasard y avait conduits, la contrée la plus lointaine, la Thulé du monde grec ({\itshape ultima Thule}). L’Homère de l’Odyssée qui avait une telle idée de l’Eubée, ne fut pas sans doute le même que celui de l’Iliade, car l’Eubée n’est pas très éloignée de Troie et de l’Asie Mineure, {\itshape où naquit sans doute le dernier}.\par
On lit dans Sénèque, que c’était une question célèbre que débattaient les grammairiens grecs, de savoir si {\itshape l’Iliade et l’Odyssée étaient du même auteur}.\par
Si les villes grecques se disputèrent l’honneur d’avoir produit Homère, c’est que chacune reconnaissait dans l’Iliade et l’Odyssée {\itshape ses mots, ses phrases et son dialecte vulgaires}. Cette observation nous servira à {\itshape découvrir} le {\scshape véritable Homère}.
\chapterclose


\chapteropen
\chapter[{Chapitre III. Du temps où vécut Homère}]{Chapitre III. \\
Du temps où vécut Homère}

\chaptercont
\noindent  L’âge d’Homère nous est indiqué par les remarques suivantes, tirées de ses poèmes : — 1. Aux funérailles de Patrocle, Achille donne tous les {\itshape jeux} que la Grèce civilisée célébrait à Olympie. — 2. L’{\itshape art de fondre} des bas-reliefs et de {\itshape graver} les métaux était déjà inventé, comme le prouve, entre autres exemples, le bouclier d’Achille. La {\itshape peinture} n’était pas encore trouvée, ce qui s’explique naturellement : {\itshape l’art du fondeur} abstrait les superficies, mais il en conserve une partie par le relief ; {\itshape l’art du graveur} ou {\itshape ciseleur} en fait autant dans un sens opposé ; mais la {\itshape peinture} abstrait les superficies d’une manière absolue ; c’est, dans les arts du dessin, le dernier effort de l’invention. Aussi, ni Homère ni Moïse ne font mention d’aucune peinture ; preuve de leur antiquité ! — 3. Les délicieux {\itshape jardins} d’Alcinoüs, la magnificence de son {\itshape palais}, la somptuosité de sa {\itshape table}, prouvent que les Grecs admiraient déjà le luxe et le faste. — 4. Les Phéniciens portaient  déjà sur les côtes de la Grèce l’{\itshape ivoire}, la {\itshape pourpre} et cet {\itshape encens} d’Arabie dont la grotte de Vénus exhale le parfum ; en outre, du lin ou {\itshape byssus} le plus fin, de riches {\itshape vêtements}. Parmi les présents offerts à Pénélope par ses amants, nous remarquons un voile ou manteau dont l’ingénieux travail ferait honneur au luxe recherché des temps modernes\footnote{\emph{
\begin{verse}
…………… μέγαν περικαλλέα πέπλον\\
ποικίλον ἐν δ’ ἀρ’ ἔσαν περόναι δυο καίδεχα πᾶσαι\\
χρύσειαι, κληῖσιν ἐυγνἀμπτοις ἀραρυῖαι. Od. Σ.\\
\end{verse}
}}. — 5. Le char sur lequel Priam va trouver Achille est de bois de {\itshape cèdre} ; l’antre de Calypso en exhala l’agréable odeur. Cette délicatesse de bon goût fut ignorée des Romains aux époques où les Néron et les Héliogabale aimaient à anéantir les choses les plus précieuses, comme par une sorte de fureur. — 6. Descriptions des {\itshape bains} voluptueux de Circé. — 7. Les {\itshape jeunes esclaves} des amants de Pénélope, avec leur beauté, leurs grâces et leurs blondes chevelures, nous sont représentés tels que les recherche la délicatesse moderne. — 8. Les hommes soignent leur {\itshape chevelure} comme les femmes ; Hector et Diomède en font un reproche à Pâris. — 9. Homère nous montre toujours ses héros se nourrissant de {\itshape chair rôtie}, nourriture la plus simple de toutes, celle qui demande le moins d’apprêt, puisqu’il suffit de braises pour la préparer\footnote{L’usage en resta dans les sacrifices, et les Romains appelèrent toujours {\itshape prosficia} les chairs des victimes rôties sur les autels que l’on partageait entre les convives ; dans la suite les victimes, comme les viandes profanes, furent rôties avec des broches. Lorsqu’Achille reçoit Priam à sa table, il ouvre l’agneau, et ensuite Patrocle le rôtit, prépare la table, et sert le pain dans des corbeilles ; les héros ne célébraient point de banquets qui ne fussent des sacrifices, où ils étaient eux-mêmes les prêtres. Les Latins en conservèrent {\itshape epulæ}, banquets somptueux, le plus souvent donnés par les grands ; {\itshape epulum}, repas donné au peuple par la république ; {\itshape epulones}, prêtres qui prenaient part au repas sacré. Agamemnon tue lui-même les deux agneaux dont le sang doit consacrer le traité fait avec Priam ; tant on attachait alors une idée magnifique à une action qui nous semble maintenant celle d’un boucher ! ({\itshape Vico.})}. Les {\itshape viandes bouillies}  ne durent venir qu’ensuite, car elles exigent, outre le feu, de l’eau, un chaudron et un trépied ; Virgile nourrit ses héros de viandes bouillies, et leur en fait aussi rôtir avec des broches. Enfin vinrent les {\itshape aliments assaisonnés}. — Homère nous présente comme l’aliment le plus délicat des héros, {\itshape la farine mêlée de fromage et de miel} ; mais il tire de la {\itshape pêche} deux de ses comparaisons ; et lorsqu’Ulysse, rentrant dans son palais sous les habits de l’indigence, demande l’aumône à l’un des amants de Pénélope, il lui dit que \emph{{\itshape les dieux donnent aux rois hospitaliers et bienfaisants des mers abondantes en poissons qui font les délices des festins}}. — 10. Les {\itshape héros} contractent mariage avec des {\itshape étrangères} ; les {\itshape bâtards succèdent} au trône ; observation importante qui prouverait qu’Homère a paru à l’époque où le {\itshape droit héroïque} tombait en désuétude dans la Grèce, pour faire place à la {\itshape liberté populaire}.\par
En réunissant toutes ces observations, recueillies pour la plupart dans l’Odyssée, ouvrage de la vieillesse d’Homère au sentiment de Longin, nous partageons l’opinion de ceux qui placent l’âge d’Homère  {\itshape longtemps après la guerre de Troie}, à une distance de quatre siècles et demi, et nous le croyons contemporain de Numa. Nous pourrions même le rapprocher encore, car Homère parle de l’Égypte, et l’on dit que Psammétique, dont le règne est postérieur à celui de Numa, fut le premier roi d’Égypte qui ouvrit cette contrée aux Grecs ; mais une foule de passages de l’Odyssée montrent que la Grèce était depuis longtemps ouverte aux marchands phéniciens, dont les Grecs aimaient déjà les récits non moins que les marchandises, à peu près comme l’Europe accueille maintenant tout ce qui vient des Indes. Il n’est donc point contradictoire qu’Homère n’ait pas vu l’Égypte, et qu’il raconte tant de choses de l’Égypte et de la Lybie, de la Phénicie et de l’Asie en général, de l’Italie et de la Sicile, d’après les rapports que les Phéniciens en faisaient aux Grecs.\par
Il n’est pas si facile d’accorder {\itshape cette recherche et cette délicatesse dans la manière de vivre}, que nous observions tout à l’heure, avec les {\itshape mœurs sauvages et féroces} qu’il attribue à ses héros, particulièrement dans l’Iliade. Dans l’impuissance d’accorder ainsi la douceur et la férocité, {\itshape ne placidis coeant immitia}, on est tenté de croire que les deux poèmes ont été travaillés par plusieurs mains, et continués pendant plusieurs âges. Nouveau pas que nous faisons dans la {\itshape recherche du} {\scshape véritable Homère}.
\chapterclose


\chapteropen
\chapter[{Chapitre IV. Pourquoi le génie d’Homère dans la poésie héroïque ne peut jamais être égalé. Observations sur la comédie et la tragédie}]{Chapitre IV. \\
Pourquoi le génie d’Homère dans la poésie héroïque ne peut jamais être égalé. Observations sur la comédie et la tragédie}

\chaptercont
\noindent  L’absence {\itshape de toute philosophie} que nous avons remarquée dans Homère, et nos {\itshape découvertes sur sa patrie et sur l’âge} où il a vécu, nous font soupçonner fortement qu’il pourrait bien n’avoir été qu’{\itshape un homme tout à fait vulgaire}. À l’appui de ce soupçon viennent deux observations.\par
1. Horace, dans son Art poétique, trouve qu’il est trop difficile d’imaginer de nouveaux {\itshape caractères} après Homère, et conseille aux poètes tragiques de les emprunter plutôt à l’Iliade (\emph{{\itshape Rectiùs iliacum carmen deducis in actus, Quàm si…}}). Il n’en est pas de même pour la {\itshape comédie} : les caractères de la nouvelle comédie à Athènes furent tous imaginés par les poètes du temps, auxquels une loi défendait de jouer des personnages réels, et ils le furent avec tant de bonheur, que les Latins, avec tout leur orgueil, reconnaissent la supériorité des Grecs dans la comédie. (Quintilien).\par
 2. Homère, venu si longtemps avant les philosophes, les critiques et les auteurs d’{\itshape Arts poétiques}, fut et reste encore {\itshape le plus sublime des poètes} dans le genre le plus sublime, {\itshape dans le genre héroïque} ; et la {\itshape tragédie} qui naquit après fut toute {\itshape grossière} dans ses commencements, comme personne ne l’ignore.\par
La première de ces difficultés eût dû suffire pour exciter les recherches des Scaliger, des Patrizio, des Castelvetro, et pour engager tous les maîtres de l’{\itshape art poétique} à chercher la raison de cette différence… Cette raison ne peut se trouver que dans l’{\itshape origine de la poésie} (v. le livre précédent), et conséquemment dans la {\itshape découverte des caractères poétiques}, qui font toute l’essence de la poésie.\par
1. L’ancienne comédie prenait des {\itshape sujets véritables} pour les mettre sur la scène, tels qu’ils étaient ; ainsi ce misérable Aristophane joua Socrate sur le théâtre, et prépara la ruine du plus vertueux des Grecs. La {\itshape nouvelle comédie peignit les mœurs des âges civilisés}, dont les philosophes de l’école de Socrate avaient déjà fait l’objet de leurs méditations ; éclairés par les {\itshape maximes} dans lesquelles cette philosophie avait résumé toute la morale, Ménandre et les autres comiques grecs purent se former des {\itshape caractères idéaux}, propres à frapper l’attention du vulgaire, si docile aux {\itshape exemples}, tandis qu’il est si incapable de profiter des {\itshape maximes}.\par
2. La {\itshape tragédie}, bien différente dans son objet, met sur la scène les {\itshape haines}, les {\itshape fureurs}, les {\itshape ressentiments},  les {\itshape vengeances héroïques}, toutes passions des {\itshape natures sublimes}. Les sentiments, le langage, les actions qui leur sont appropriés, ont, par leur violence et leur atrocité même, quelque chose de {\itshape merveilleux}, et toutes ces choses sont au plus haut degré {\itshape conformes entre elles}, et {\itshape uniformes dans leurs sujets}. Or, ces tableaux passionnés ne furent jamais faits avec plus d’avantage que par les Grecs des {\itshape temps héroïques}, à la fin desquels vint Homère…… Aristote dit avec raison dans sa Poétique, qu’Homère est {\itshape un poète unique pour les fictions}. C’est que les {\itshape caractères poétiques} dont Horace admire dans ses ouvrages l’incomparable vérité, se rapportèrent à {\itshape ces genres créés par l’imagination} ({\itshape generi fantastici}), dont nous avons parlé dans la {\itshape métaphysique poétique}. À chacun de ces {\itshape caractères} les peuples grecs attachèrent toutes les {\itshape idées particulières} qu’on pouvait y rapporter, en considérant chaque caractère comme un genre. Au caractère d’Achille, dont la peinture est le principal sujet de l’Iliade, ils rapportèrent toutes les qualités propres à la {\itshape vertu héroïque}, les sentiments, les mœurs qui résultent de ces qualités, l’irritabilité, la colère implacable, la violence {\itshape qui s’arroge tout par les armes} (Horace). Dans le caractère d’Ulysse, principal sujet de l’Odyssée, ils firent entrer tous les traits distinctifs de la {\itshape sagesse héroïque}, la prudence, la patience, la dissimulation, la duplicité, la fourberie, cette attention à sauver l’exactitude du langage, sans égard à la réalité des actions, qui fait que ceux qui écoutent,  se trompent eux-mêmes. Ils attribuèrent à ces deux {\itshape caractères} les actions {\itshape particulières} dont la célébrité pouvait assez frapper l’attention d’un peuple encore stupide, pour qu’il les rangeât dans l’un ou dans l’autre genre. Ces deux {\itshape caractères}, ouvrages d’une nation tout entière, devaient nécessairement présenter dans leur conception une heureuse {\itshape uniformité} ; c’est dans cette {\itshape uniformité}, d’accord avec le sens commun d’une nation entière, que consiste toute la {\itshape convenance}, toute la grâce d’une fable. Créés par de si puissantes imaginations, ces caractères ne pouvaient être que {\itshape sublimes}. De là deux lois éternelles en poésie : d’après la première, le {\itshape sublime poétique} doit toujours avoir quelque chose de {\itshape populaire} ; en vertu de la seconde, les peuples qui se firent d’abord eux-mêmes les {\itshape caractères héroïques}, ne peuvent observer leurs contemporains {\itshape civilisés} [et par conséquent si différents], sans leur transporter les idées qu’ils empruntent à ces caractères si renommés.
\chapterclose


\chapteropen
\chapter[{Chapitre V. Observations philosophiques devant servir à la découverte du véritable Homère}]{Chapitre V. \\
Observations philosophiques devant servir à la découverte du véritable Homère}

\chaptercont
\noindent  1. Rappelons d’abord cet axiome : {\itshape Les hommes sont portés naturellement à consacrer le souvenir des lois et institutions qui font la base des sociétés auxquelles ils appartiennent.} — 2. L’{\itshape histoire} naquit d’abord, ensuite la {\itshape poésie}. En effet, l’histoire est la simple {\itshape énonciation du vrai}, dont la poésie est une {\itshape imitation exagérée}. Castelvetro a aperçu cette vérité, mais cet ingénieux écrivain n’a pas su en profiter pour trouver la véritable {\itshape origine de la poésie} ; c’est qu’il fallait combiner ce principe avec le suivant : — 3. Les {\itshape poètes} ayant certainement précédé les {\itshape historiens vulgaires}, la première {\itshape histoire} dut être la {\itshape poétique}. — 4. Les {\itshape fables} furent à leur origine des récits véritables et d’un caractère sérieux, et (μῦθος, {\itshape fable}, a été définie par {\itshape vera narratio}). Les fables naquirent, pour la plupart, {\itshape bizarres}, et devinrent successivement {\itshape moins appropriées} à leurs sujets primitifs, {\itshape altérées, invraisemblables, obscures, d’un effet choquant} et surprenant, enfin {\itshape incroyables} ; voilà les sept sources de la difficulté des fables. — 5.  Nous avons vu dans le second livre comment Homère reçut les fables déjà {\itshape altérées} et {\itshape corrompues}. — 6. Les {\itshape caractères poétiques}, qui sont l’essence des {\itshape fables}, naquirent d’une impuissance naturelle des premiers hommes, incapables d’{\itshape abstraire du sujet ses formes et ses propriétés} ; en conséquence, nous trouvons dans ces {\itshape caractères} une {\itshape manière de penser commandée par la nature aux nations entières}, à l’époque de leur plus profonde barbarie. — C’est le propre des barbares d’agrandir et d’étendre toujours les {\itshape idées particulières. Les esprits bornés}, dit Aristote dans sa {\itshape Morale, font une maxime}, une règle générale, {\itshape de chaque idée particulière}. La raison doit en être que l’esprit humain, infini de sa nature, étant resserré dans la grossièreté de ses sens, ne peut exercer ses facultés presque divines qu’en {\itshape étendant les idées particulières} par l’imagination. C’est pour cela peut-être que dans les poètes grecs et latins les images des dieux et des héros apparaissent toujours plus grandes que celles des hommes, et qu’aux siècles barbares du moyen âge, nous voyons dans les tableaux les figures du Père, de Jésus-Christ et de la Vierge, d’une grandeur colossale. — 7. La {\itshape réflexion}, détournée de son usage naturel, est {\itshape mère du mensonge} et de la fiction. Les barbares en sont dépourvus ; aussi les premiers poètes héroïques des Latins chantèrent des histoires véritables, c’est-à-dire les guerres de Rome. Quand la barbarie de l’antiquité reparut au moyen âge, les poètes latins de cette époque, les Gunterius, les  Guillaume de Pouille, ne chantèrent que des faits réels. Les romanciers du même temps s’imaginaient écrire des histoires véritables, et le Boiardo, l’Arioste, nés dans un siècle éclairé par la philosophie, tirèrent les sujets de leur poème de la chronique de l’archevêque Turpin. C’est par l’effet de ce {\itshape défaut de réflexion}, qui rend les barbares incapables de {\itshape feindre}, que Dante, tout profond qu’il était dans la {\itshape sagesse philosophique}, a représenté dans sa {\itshape Divine Comédie}, des personnages réels et des faits historiques. Il a donné à son poème le titre de {\itshape comédie}, dans le sens de l’{\itshape ancienne comédie} des Grecs, qui prenait pour sujet des personnages réels. Dante ressembla sous ce rapport à l’Homère de l’Iliade, que Longin trouve toute dramatique, toute en actions, tandis que l’Odyssée est toute en récits. Pétrarque, avec toute sa science, a pourtant chanté dans un poème latin la seconde guerre punique ; et dans ses poésies italiennes, les {\itshape Triomphes}, où il prend le ton héroïque, ne sont autre chose qu’un {\itshape recueil d’histoires}. — Une preuve frappante que les premières {\itshape fables} furent des {\itshape histoires}, c’est que la {\itshape satire} attaquait non-seulement des personnes {\itshape réelles}, mais les personnes les plus connues ; que la {\itshape tragédie} prenait pour sujets des {\itshape personnages de l’histoire poétique} ; que l’{\itshape ancienne comédie} jouait sur la scène {\itshape des hommes} célèbres encore {\itshape vivants}. Enfin la {\itshape nouvelle comédie}, née à l’époque où les Grecs étaient le plus capables de {\itshape réflexion, créa} des personnages tout d’{\itshape invention} ; de même, dans l’Italie moderne, la  {\itshape nouvelle comédie} ne reparut qu’au commencement de ce quinzième siècle, déjà si éclairé. Jamais les Grecs et les Latins ne prirent un {\itshape personnage imaginaire} pour sujet principal d’une tragédie. Le public moderne, d’accord en cela avec l’ancien, veut que les opéras dont les sujets sont tragiques, soient {\itshape historiques} pour le fond ; et s’il supporte les {\itshape sujets d’invention} dans la comédie, c’est que ce sont des aventures particulières qu’il est tout simple qu’on ignore, et que pour cette raison l’on croit véritables. — 8. D’après cette explication des {\itshape caractères poétiques}, les allégories poétiques qui y sont rattachées, ne doivent avoir qu’un sens relatif à l’{\itshape histoire} des premiers temps de la Grèce. — 9. De telles {\itshape histoires durent se conserver naturellement dans la mémoire} des peuples, en vertu du premier principe observé au commencement de ce chapitre. Ces premiers hommes, qu’on peut considérer comme représentant l’enfance de l’humanité, durent posséder à un degré merveilleux la faculté de la {\itshape mémoire}, et sans doute il en fut ainsi par une volonté expresse de la Providence ; car, au temps d’Homère, et quelque temps encore après lui, l’écriture vulgaire n’avait pas encore été trouvée (Josèphe, {\itshape Contre Apion}). Dans ce travail de l’esprit, les peuples, qui à cette époque étaient pour ainsi dire tout {\itshape corps} sans {\itshape réflexion}, furent tout {\itshape sentiment} pour {\itshape sentir} les particularités, toute {\itshape imagination} pour les saisir et les agrandir, toute {\itshape invention} pour les rapporter aux genres que l’imagination avait créés ({\itshape generi fantastici}),  enfin toute {\itshape mémoire} pour les retenir. Ces facultés appartiennent sans doute à l’esprit, mais tirent du corps leur origine et leur vigueur. Chez les Latins, {\itshape mémoire} est synonyme d’{\itshape imagination} ({\itshape memorabile}, imaginable, dans Térence) ; ils disent {\itshape comminisci} pour feindre, imaginer ; {\itshape commentum} pour une {\itshape fiction}, et en italien {\itshape fantasia} se prend de même pour {\itshape ingegno}. La {\itshape mémoire} rappelle les objets, l’{\itshape imagination} en imite et en altère la forme réelle, le {\itshape génie} ou faculté d’inventer leur donne un tour nouveau, et en forme des assemblages, des compositions nouvelles. Aussi les {\itshape poètes théologiens} ont-ils appelé la {\itshape mémoire} la {\itshape mère des Muses}. — 10. Les {\itshape poètes} furent donc sans doute les premiers {\itshape historiens} des nations. Ceux qui ont cherché l’{\itshape origine de la poésie}, depuis Aristote et Platon, auraient pu remarquer sans peine que toutes les {\itshape histoires} des nations païennes ont des commencements {\itshape fabuleux}. — 11. Il est impossible d’être à la fois et au même degré {\itshape poète} et {\itshape métaphysicien sublimes}. C’est ce que prouve tout examen de la nature de la poésie. La {\itshape métaphysique} détache l’{\itshape âme} des {\itshape sens} ; la {\itshape faculté poétique} l’y plonge pour ainsi dire et l’y ensevelit ; la {\itshape métaphysique} s’élève aux {\itshape généralités}, la {\itshape faculté} poétique descend aux {\itshape particularités}. — 12. En poésie, l’art est inutile sans la nature : la poétique, la critique, peuvent faire des esprits {\itshape cultivés}, mais non pas leur donner de la {\itshape grandeur} ; la {\itshape délicatesse} est un talent pour les petites choses, et la {\itshape grandeur d’esprit} les dédaigne naturellement. Le torrent impétueux  peut-il rouler une eau limpide ? ne faut-il pas qu’il entraîne dans son cours des arbres et des rochers ? {\itshape Excusons} donc {\itshape les choses basses et grossières qui se trouvent dans Homère}. — 13. Malgré ces défauts, Homère n’en est pas moins {\itshape le père, le prince de tous les poètes sublimes}. Aristote trouve qu’il est impossible d’{\itshape égaler les mensonges poétiques d’Homère} ; Horace dit {\itshape que ses caractères sont inimitables} ; deux éloges qui ont le même sens. — Il semble s’élever jusqu’au ciel par le {\itshape sublime de la pensée} ; nous avons expliqué déjà ce mérite d’Homère, {\scshape livre} II, page 225.\par
Joignez à ces réflexions celles que nous avons faites un peu plus haut (pages 252-257), et qui prouvent à la fois combien il est poète, et {\itshape combien peu il est philosophe}. — 14. Les {\itshape inconvenances}, les {\itshape bizarreries} qu’on pourrait lui reprocher, furent l’effet naturel de l’impuissance, de la {\itshape pauvreté de la langue} qui se formait alors. Le {\itshape langage} se composait encore d’{\itshape images}, de {\itshape comparaisons}, faute de {\itshape genres} et {\itshape d’espèces qui pussent définir les choses avec propriété} ; ce langage était le produit naturel d’une {\itshape nécessité, commune à des nations entières}. — C’était encore une {\itshape nécessité} que les premières nations parlassent {\itshape en vers héroïques} ({\scshape livre} II, page 158). — 15. De telles {\itshape fables}, de telles {\itshape pensées} et de telles {\itshape mœurs}, un tel {\itshape langage} et de tels {\itshape vers} s’appelèrent également {\itshape héroïques}, furent {\itshape communs à des peuples entiers}, et par conséquent {\itshape aux individus} dont se composaient ces peuples.
\chapterclose


\chapteropen
\chapter[{Chapitre VI. Observations philologiques, qui serviront à la découverte de véritable Homère}]{Chapitre VI. \\
Observations philologiques, qui serviront à la découverte de véritable Homère}

\chaptercont
\noindent  1. Nous avons déjà dit plus haut que toutes les anciennes {\itshape histoires} profanes commencent par des {\itshape fables} ; que les peuples barbares, sans communication avec le reste du monde, comme les anciens Germains et les Américains, conservaient {\itshape en vers l’histoire} de leurs premiers temps ; que l’{\itshape histoire romaine} particulièrement fut d’abord écrite par des {\itshape poètes}, et qu’au moyen âge celle de l’Italie le fut aussi par des poètes latins. — 2. Manéthon, grand {\itshape pontife} d’Égypte, avait donné à l’{\itshape histoire} des premiers âges de sa nation, écrite en hiéroglyphes, l’interprétation d’une sublime {\itshape théologie naturelle} ; les {\itshape philosophes} grecs donnèrent une explication {\itshape philosophique} aux {\itshape fables} qui contenaient l’{\itshape histoire} des âges les plus anciens de la Grèce. Nous avons, dans le livre précédent, tenu une marche tout à fait contraire : nous avons ôté aux {\itshape fables} leurs sens {\itshape mystique} ou {\itshape philosophique} pour leur rendre leur véritable sens {\itshape historique}. — 3. Dans l’Odyssée, on veut louer quelqu’un d’avoir bien raconté une {\itshape histoire}, et l’on dit qu’{\itshape il l’a racontée comme un chanteur} ou {\itshape un musicien}. Ces {\itshape chanteurs} n’étaient sans doute autres  que les {\itshape rapsodes}, ces hommes du peuple qui savaient chacun par cœur quelque morceau d’Homère, et conservaient ainsi dans leur mémoire ses poèmes, qui n’étaient point encore écrits. ({\itshape Voy.} Josèphe, {\itshape Contre Apion}.) Ils allaient isolément de ville en ville en chantant les vers d’Homère dans les fêtes et dans les foires. — 4. D’après l’étymologie, les {\itshape rapsodes} (de ῥάπτειν, {\itshape coudre}, ᾠδάς, {\itshape des chants}), ne faisaient que {\itshape coudre}, arranger les {\itshape chants} qu’ils avaient recueillis, sans doute dans le peuple même. Le mot {\itshape Homère} présente dans son étymologie un sens analogue, ὅμοῦ, {\itshape ensemble}, εἰρεῖν, {\itshape lier}. Ὅμηρος signifie {\itshape répondant}, parce que le {\itshape répondant lie} ensemble le créancier et le débiteur. Cette étymologie, appliquée à l’Homère que l’on a conçu jusqu’ici, est aussi éloignée et aussi forcée qu’elle est convenable et facile relativement à notre Homère, qui {\itshape liait, composait}, c’est-à-dire mettait ensemble {\itshape les fables}. — 5. {\itshape Les Pisistratides divisèrent et disposèrent les poèmes d’Homère en Iliade et en Odyssée.} Ceci doit nous faire entendre que ces poèmes n’étaient auparavant qu’un amas confus de traditions poétiques. On peut remarquer d’ailleurs combien diffère le style des deux poèmes. — Les mêmes Pisistratides ordonnèrent qu’à l’avenir ces poèmes {\itshape seraient chantés par les rapsodes} dans la fête des Panathénées (Cicéron, {\itshape De naturâ deorum} ; Élien). — 6. Mais les Pisistratides furent chassés d’Athènes peu de temps avant que les Tarquins le fussent de Rome, de sorte qu’en plaçant Homère au temps de Numa,  comme nous l’avons fait, les {\itshape rapsodes conservèrent longtemps encore ses poèmes dans leur mémoire}. Cette tradition ôte tout crédit à la précédente, d’après laquelle les poèmes d’Homère auraient été {\itshape corrigés, divisés et mis en ordre} du temps des Pisistratides. Tout cela eût supposé l’écriture vulgaire, et si cette écriture eût existé dès cette époque, on n’aurait plus eu besoin de rapsodes pour retenir et pour chanter des morceaux de ces poèmes\footnote{\noindent Rien n’indique qu’Hésiode qui laissa ses ouvrages écrits ait été appris par cœur, comme Homère, par les rapsodes. Les chronologistes ont donc pris un soin puéril en le plaçant trente ans avant Homère, tandis qu’il dut venir après les Pisistratides.On pourrait cependant attaquer cette opinion en considérant Hésiode comme un de ces poètes cycliques, qui chantèrent toute l’{\itshape histoire fabuleuse} des Grecs, depuis l’origine de leur théogonie jusqu’au retour d’Ulysse à Itaque, et en les plaçant dans la même classe que les rapsodes homériques. Ces poètes dont le nom vient de κύκλος, {\itshape cercle}, ne purent être que des hommes du peuple qui, les jours de fêtes, chantaient les fables à la multitude rassemblée en cercle autour d’eux. On les désigne ordinairement eux-mêmes par l’épithète de κύκλιοι, ἐγκύκλιοι, et les recueils de leurs ouvrages par κύκλος ἐπικός, κύκλια ἔπη, ποίημα εγκύκλικον, ou simplement κύκλος. Hésiode, considéré comme un {\itshape poète cyclique}, qui raconte toutes les {\itshape fables relatives aux dieux} de la Grèce, aurait précédé Homère.\par
Ce que nous disions d’abord d’Hésiode, nous le dirons d’Hippocrate. Il laissa des ouvrages considérables écrits, non en vers, mais en {\itshape prose}, et par conséquent {\itshape incapables d’être retenus par cœur} ; nous le placerons au temps d’Hérodote. ({\itshape Vico.})
}.\par
Ce qui achève de prouver qu’Homère est {\itshape antérieur à l’usage de l’écriture}, c’est qu’{\itshape il ne fait mention nulle part des lettres de l’alphabet}. La lettre écrite par Prétus pour perdre Bellérophon, le fut, dit-il, {\itshape par des signes}, σήματα. — 7. Aristarque {\itshape corrigea} les poèmes d’Homère, et pourtant, sans parler de  cette foule de {\itshape licences} dans la mesure, on trouve encore dans la variété de ses dialectes, {\itshape ce mélange discordant d’expressions hétérogènes}, qui étaient sans doute autant d’{\itshape idiotismes} des divers peuples de la Grèce. — 8. Voyez plus haut ce que nous avons dit sur la patrie et sur l’âge d’Homère. Longin, ne pouvant dissimuler la grande {\itshape diversité de style} qui se trouve dans les deux poèmes, prétend qu’{\itshape Homère fit l’Iliade lorsqu’il était jeune encore, et qu’il composa l’Odyssée dans sa vieillesse}. Sans doute la colère d’Achille lui semble un sujet plus convenable pour un jeune homme, les aventures du prudent Ulysse pour un vieillard. Mais comment savoir ces particularités de l’histoire d’un homme, lorsqu’on en ignore les deux circonstances les plus importantes, le temps et le lieu ? C’est ce qui doit ôter toute confiance à la {\itshape Vie d’Homère} qu’a composée Plutarque, et à celle qu’on attribue souvent à Hérodote, et dans laquelle l’auteur a rempli un volume de tant de détails minutieux et de tant de belles aventures. — 9. La tradition veut qu’Homère ait été {\itshape aveugle}, et qu’il ait tiré de là son nom (c’était le sens d’Ὅμηρος dans le dialecte ionien). Homère lui-même nous représente {\itshape toujours aveugles} les poètes qui chantent à la table des grands ; c’est un {\itshape aveugle} qui paraît au banquet d’Alcinoüs et à celui des amants de Pénélope. — {\itshape Les aveugles ont une mémoire étonnante.} — Enfin, selon la même tradition, Homère était {\itshape pauvre, et allait dans les marchés de la Grèce en chantant ses poèmes}.
\chapterclose


\chapteropen
\chapter[{Chapitre VII}]{Chapitre VII}

\chaptercont
\section[{§ I. Découverte du véritable Homère}]{§ I. {\itshape Découverte du véritable Homère}}
\noindent  Ces observations philosophiques et philologiques nous portent à croire qu’il en est d’{\itshape Homère} comme de la {\itshape guerre de Troie}, qui fournit à l’histoire une fameuse époque chronologique, et dont cependant les plus sages critiques révoquent en doute la réalité. Certainement, s’il ne restait pas plus de traces d’{\itshape Homère} que de la {\itshape guerre de Troie}, nous ne pourrions y voir, après tant de difficultés, qu’{\itshape un être idéal}, et non pas un homme. Mais {\itshape ces deux poèmes} qui nous sont parvenus, nous forcent de n’admettre cette opinion qu’à demi, et de dire qu’{\itshape Homère a été l’idéal ou le} caractère héroïque {\itshape du peuple de la Grèce racontant sa propre histoire dans des chants nationaux}.
\section[{§ II. Tout ce qui était absurde et invraisemblable dans l’Homère que l’on s’est figuré jusqu’ici, devient dans notre Homère convenance et nécessité}]{§ II. {\itshape Tout ce qui était absurde et invraisemblable dans l’Homère que l’on s’est figuré jusqu’ici, devient dans notre Homère convenance et nécessité}}
\noindent — 1. D’abord l’incertitude de la {\itshape patrie} d’Homère nous oblige de dire que si les peuples de la Grèce se disputèrent l’honneur de lui avoir donné le jour, et  le revendiquèrent tous pour concitoyen, c’est qu’ils {\itshape étaient eux-mêmes Homère}. — S’il y a une telle diversité d’opinion sur l’époque où il a vécu, c’est qu’il vécut en effet dans la bouche et dans la mémoire des mêmes peuples, depuis la guerre de Troie jusqu’au temps de Numa, ce qui fait quatre cent soixante ans. — 2. La {\itshape cécité}, la {\itshape pauvreté} d’Homère furent celles des rapsodes, qui, étant aveugles (d’où leur venait le nom d’ὅμηροι), avaient une plus forte mémoire. C’étaient de pauvres gens qui gagnaient leur vie à chanter par les villes les {\itshape poèmes homériques}, dont ils étaient auteurs, en ce sens qu’ils faisaient partie des peuples qui y avaient consigné leur histoire. — 3. De cette manière, Homère composa l’Iliade {\itshape dans sa jeunesse}, c’est-à-dire dans celle de la Grèce. Elle se trouvait alors tout ardente de passions sublimes, d’orgueil, de colère et de vengeance. Ces sentiments sont ennemis de la dissimulation, et n’excluent point la générosité ; elle devait admirer Achille, le {\itshape héros de la force}. Homère déjà {\itshape vieux} composa l’Odyssée, lorsque les passions des Grecs commençaient à être refroidies par la réflexion, mère de la prudence. La Grèce devait alors admirer Ulysse, le {\itshape héros de la sagesse}. Au temps de la jeunesse d’Homère, la fierté d’Agamemnon, l’insolence et la barbarie d’Achille plaisaient aux peuples de la Grèce. Lors de sa vieillesse, ils aimaient déjà le luxe d’Alcinoüs, les délices de Calypso, les voluptés de Circé, les chants des Sirènes et les amusements des amants de Pénélope. Comment en effet rapporter au même  âge des mœurs absolument opposées ? Cette difficulté a tellement frappé Platon, que, ne sachant comment la résoudre, il prétend que dans les divins transports de l’enthousiasme poétique, Homère put voir dans l’avenir ces mœurs efféminées et dissolues. Mais n’est-ce pas attribuer le comble de l’imprudence à celui qu’il nous présente comme le fondateur de la civilisation grecque ? Peindre d’avance de telles mœurs, tout en les condamnant, n’est-ce pas enseigner à les imiter ? Convenons plutôt que l’auteur de l’Iliade dut précéder de longtemps celui de l’Odyssée ; que le premier, originaire du nord-est de la Grèce, chanta la guerre de Troie qui avait eu lieu dans son pays ; et que l’autre, né du côté de l’Orient et du Midi, célèbre Ulysse qui régnait dans ces contrées. — 4. Le caractère individuel d’Homère, disparaissant ainsi dans la foule des peuples grecs, il se trouve justifié de tous les reproches que lui ont faits les critiques, et particulièrement de la bassesse des pensées, de la grossièreté des mœurs, de ses comparaisons sauvages, des idiotismes, des licences de versification, de la variété des dialectes qu’il emploie ; enfin d’avoir élevé les hommes à la grandeur des dieux, et fait descendre les dieux au caractère d’hommes. Longin n’ose défendre de telles fables qu’en les expliquant par des allégories philosophiques ; c’est dire assez que, prises dans leur premier sens, elles ne peuvent assurer à Homère la gloire d’avoir fondé la civilisation grecque. — Toutes ces imperfections de la poésie homérique que l’on a  tant critiquées répondent à autant de caractères des peuples grecs eux-mêmes. — 5. Nous assurons à Homère le privilège d’avoir eu seul la puissance d’inventer les {\itshape mensonges poétiques} (Aristote), {\itshape les caractères héroïques} (Horace) ; le privilège d’une incomparable éloquence dans ses comparaisons sauvages, dans ses affreux tableaux de morts et de batailles, dans ses peintures sublimes des passions, enfin le mérite du style le plus brillant et le plus pittoresque. Toutes ces qualités appartenaient à l’âge héroïque de la Grèce. C’est le génie de cet âge qui fit d’Homère un {\itshape poète} incomparable. Dans un temps où la mémoire et l’imagination étaient pleines de force, où la puissance d’invention était si grande, il ne pouvait être {\itshape philosophe}. Aussi ni la philosophie, ni la poétique ou la critique, qui vinrent plus tard, n’ont pu jamais faire un poète qui approchât seulement d’Homère. — 6. Grâces à notre découverte, Homère est assuré désormais des trois titres immortels qui lui ont été donnés, d’avoir été le {\itshape fondateur de la civilisation grecque}, le {\itshape père de tous les autres poètes}, et la {\itshape source des diverses philosophies} de la Grèce. Aucun de ces trois titres ne convenait à Homère, tel qu’on se l’était figuré jusqu’ici. Il ne pouvait être regardé comme le {\itshape fondateur de la civilisation grecque}, puisque, dès l’époque de Deucalion et Pyrrha, elle avait été fondée avec l’institution des mariages, ainsi que nous l’avons démontré en traitant de la {\itshape sagesse poétique} qui fut le principe de cette civilisation. Il ne pouvait être regardé  comme le {\itshape père des poètes}, puisqu’avant lui avaient fleuri les {\itshape poètes théologiens}, tels qu’Orphée, Amphion, Linus et Musée ; les chronologistes y joignent Hésiode en le plaçant trente ans avant Homère. Il fut même devancé par plusieurs poètes héroïques, au rapport de Cicéron (Brutus) ; Eusèbe les nomme dans sa {\itshape Préparation évangélique} ; ce sont Philamon, Thémiride, Démodocus, Épiménide, Aristée, etc. — Enfin, on ne pouvait voir en lui la {\itshape source des diverses philosophies} de la Grèce, puisque nous avons démontré dans le second Livre que les philosophes ne trouvèrent point leurs doctrines dans les fables homériques, mais qu’ils les y rattachèrent. La {\itshape sagesse poétique} avec ses fables fournit seulement aux philosophes l’occasion de méditer les plus hautes vérités de la métaphysique et de la morale, et leur donna en outre la facilité de les expliquer.
\section[{§ III. On doit trouver dans les poèmes d’Homère les deux principales sources des faits relatifs au droit naturel des gens, considéré chez les Grecs}]{§ III. {\itshape On doit trouver dans les poèmes d’Homère les deux principales sources des faits relatifs au droit naturel des gens, considéré chez les Grecs}}
\noindent Aux éloges que nous venons de donner à Homère, ajoutons celui d’avoir été le {\itshape plus ancien historien du paganisme}, qui nous soit parvenu. Ses poèmes sont comme {\itshape deux grands trésors où se trouvent conservées les mœurs des premiers âges de la Grèce}. Mais le destin des {\itshape poèmes d’Homère} a été le même que celui des {\itshape lois des douze tables}. On a rapporté ces lois au législateur d’Athènes, d’où elles seraient passées à Rome, et l’on n’y a point vu  l’{\itshape histoire du droit naturel des peuples héroïques du Latium} ; on a cru que les {\itshape poèmes d’Homère} étaient la création du rare génie d’un individu, et l’on n’y a pu découvrir l’{\itshape histoire du droit naturel des peuples héroïques de la Grèce}.
\chapterclose


\chapteropen
\chapter[{Appendice. Histoire raisonnée des poètes dramatiques et lyriques}]{Appendice. \\
{\itshape Histoire raisonnée des poètes dramatiques et lyriques}}

\chaptercont
\noindent Nous avons déjà montré qu’antérieurement à Homère il y avait eu trois âges de poètes : celui des {\itshape poètes théologiens}, dans les chants desquels les fables étaient encore des histoires véritables et d’un caractère sévère ; celui des {\itshape poètes héroïques}, qui altérèrent et corrompirent ces fables ; enfin l’{\itshape âge d’Homère}, qui les reçut altérées et corrompues. Maintenant la même {\itshape critique métaphysique} peut, en nous montrant la cours d’idées que suivirent les anciens peuples, jeter un jour tout nouveau sur l’{\itshape histoire des poètes dramatiques et lyriques}.\par
Cette histoire a été traitée par les philologues avec bien de l’obscurité et de la confusion. Ils placent parmi les {\itshape lyriques} Amphion de Méthymne, poète très ancien des temps héroïques. Ils disent qu’il trouva le {\itshape dithyrambe}\footnote{Orthographié « dithyrambe » [NdE].}, et aussi le {\itshape chœur} ; qu’il introduisit des {\itshape satyres} qui chantaient des vers ; que le {\itshape dithyrambe} était un {\itshape chœur} qui dansait en rond, en chantant des vers en l’honneur de Bacchus. À les entendre, le temps des {\itshape poètes lyriques} vit aussi fleurir des {\itshape poètes tragiques} distingués, et Diogène Laërce assure que la première tragédie fut représentée par le {\itshape chœur} seulement. Ils disent encore qu’Eschyle fut le premier poète tragique, et Pausanias raconte qu’il reçut de Bacchus l’ordre d’écrire des tragédies ; d’un autre côté, Horace qui dans son art poétique commence à traiter de la tragédie en parlant de la satire\footnote{Orthographié « satyre » [NdE].}, en attribue l’invention à Thespis, qui au temps des vendanges fit jouer la première satire sur des tombereaux. Après serait venu Sophocle, que Palémon a proclamé l’{\itshape Homère des tragiques} ; enfin la carrière eût été fermée par Euripide qu’Aristote appelle le tragique par excellence, τραγικώτατος. Ils placent dans le même âge Aristophane, premier auteur de la {\itshape vieille comédie}, dont les {\itshape nuées} perdirent le vertueux Socrate. Cet abus ouvrit la route de la nouvelle comédie que Ménandre suivit plus tard.\par
Pour résoudre ces difficultés, il faut reconnaître qu’il y eut deux sortes de {\itshape poètes tragiques}, et autant de {\itshape lyriques}. Les anciens lyriques furent sans doute les auteurs des hymnes en l’honneur des dieux, analogues à  ceux que l’on attribue à Homère, et écrits aussi en vers héroïques. Chez les Latins les premiers poètes furent les auteurs des vers saliens, sorte d’hymnes chantés dans les fêtes des dieux par les prêtres saliens. Ce dernier mot vient peut-être de {\itshape salire, saltare} danser, de même que chez les Grecs le premier chœur avait été une danse en rond. Tout ceci s’accorde avec nos principes : les hommes des premiers siècles qui étaient essentiellement religieux, ne pouvaient louer que les dieux. Au moyen âge, les prêtres qui seuls alors étaient lettrés, ne composèrent d’autres poésies que des hymnes.\par
Lorsque l’âge héroïque succéda à l’âge divin, on n’admira, on ne célébra que les exploits des héros. Alors parurent les poètes lyriques semblables à l’Achille de l’Iliade, lorsqu’il chante sur sa lyre les {\itshape louanges des héros qui ne sont plus}\footnote{Amphion dut appartenir à cette classe. Il fut en outre l’inventeur du dithyrambe, première ébauche de la tragédie écrite en vers héroïques (nous avons démontré que ce vers fut le premier chez les Grecs). Ainsi le dithyrambe d’Amphion aurait été la première satire ; on vient de voir que c’est en parlant de la satire qu’Horace commence à traiter de la tragédie. ({\itshape Vico.})}. Les nouveaux lyriques furent ceux qu’on appelait {\itshape melici}, ceux qui écrivirent ce genre de vers que nous appelons {\itshape arie per musica} ; le prince de ces lyriques est Pindare. Ce genre de vers dut venir après l’iambique, qui lui-même, ainsi que nous l’avons vu, succéda à l’héroïque. Pindare vint au temps où la vertu grecque éclatait dans les pompes des jeux olympiques au milieu d’un peuple admirateur ; là chantaient les poètes lyriques. De même Horace parut à l’époque de la plus haute splendeur de Rome ; et chez les Italiens ce genre de poésie n’a été connu qu’à l’époque où les mœurs se sont adoucies et amollies.\par
Quant aux {\itshape tragiques} et aux {\itshape comiques}, on peut tracer ainsi la route qu’ils suivirent. Thespis et Amphion, dans deux parties différentes de la Grèce, inventèrent pendant la saison des vendanges\footnote{Il peut être vrai en ce sens que Bacchus, dieu de la vendange, ait commandé à Eschyle de composer des tragédies. ({\itshape Vico.})} la {\itshape satire}, ou tragédie antique jouée par des satyres. Dans cet âge de grossièreté, le premier déguisement consista à se couvrir de peaux de chèvres\footnote{Aussi a-t-on lieu de conjecturer que la tragédie a tiré son nom de ce genre de déguisement, plutôt que du bouc Τράγος, qu’on donnait en prix au vainqueur. ({\itshape Vico.})} les jambes et les cuisses, à se rougir de lie de vin le visage et la poitrine, et à s’armer le front de cornes\footnote{C’est de là peut-être que chez nous les vendangeurs sont encore appelés vulgairement {\itshape cornuti}. ({\itshape Vico.})}. La tragédie dut commencer par un chœur de satyres ; et la satire conserva pour caractère originaire la licence des injures et des insultes, {\itshape villanie}, parce que les villageois grossièrement déguisés se tenaient sur les tombereaux qui portaient la vendange, et  avaient la liberté de dire de là toute sorte d’injures aux honnêtes gens, comme le font encore aujourd’hui les vendangeurs de la {\itshape Campanie} appelée proverbialement {\itshape le séjour de Bacchus}. Le mot {\itshape satyre} signifiaient originairement en latin, \emph{{\itshape mets composés de divers aliments}} ({\itshape Festus})\footnote{{\itshape Lex per satyram} signifiait une loi qui comprenait des matières diverses. ({\itshape Vico.})}. Dans la satire dramatique, on voyait paraître, selon Horace, divers genres de personnages, héros et dieux, rois et artisans, enfin esclaves. La satire, telle qu’elle resta chez les Romains, ne traitait point de sujets divers.\par
Grâces au génie d’Eschyle, la {\itshape tragédie} antique fit place à la tragédie moyenne, et les chœurs de satyre aux chœurs d’hommes. La {\itshape tragédie moyenne} dut être l’origine de la {\itshape vieille comédie}, dans laquelle les grands personnages étaient traduits sur la scène ; et voilà pourquoi le chœur s’y plaçait naturellement. Ensuite vint Sophocle et après lui Euripide qui nous laissèrent la {\itshape tragédie nouvelle}, dans le même temps où la {\itshape vieille comédie} finissait avec Aristophane. Ménandre fut le père de la {\itshape comédie nouvelle}, dont les personnages sont de simples particuliers, et en même temps imaginaires ; c’est précisément parce qu’ils sont pris dans une condition privée, qu’ils pouvaient passer pour réels sans l’être en effet. Dès lors on ne devait plus placer le chœur dans la comédie ; le chœur est un {\itshape public} qui raisonne, et qui ne raisonne que de choses {\itshape publiques}.
\chapterclose

\chapterclose


\chapteropen
\part[{Livre quatrième. Du cours que suit l’histoire des nations}]{Livre quatrième. \\
Du cours que suit l’histoire des nations}

\chaptercont

\chapteropen
\chapter[{Argument}]{Argument}

\chaptercont
\noindent  {\itshape L’auteur récapitule ce qu’il a dit au second Livre, en ajoutant quelques développements. Dans ses recherches philosophiques sur la} sagesse poétique, {\itshape on a vu ses opinions sur l’âge des} dieux {\itshape et sur celui des} héros. {\itshape Il les présente ici sous une forme toute historique, il ajoute l’indication générale des caractères de l’âge des} hommes, {\itshape et trace ainsi une esquisse complète de l’}histoire idéale {\itshape indiquée dans les axiomes.}\par
Chapitre I\textsuperscript{er}. {\scshape Introduction. Trois sortes de natures, de mœurs, de droits naturels, de gouvernements}. — § I. {\itshape Introduction.} — § II. {\itshape Nature divine, poétique ou créatrice, héroïque, humaine et intelligente.} — § III. {\itshape Mœurs religieuses, violentes, réglées par le devoir.} — § IV. {\itshape Droits divin, héroïque, humain.} — § V. {\itshape Gouvernements théocratique, aristocratique, démocratique ou monarchique.}\par
Chapitre {\scshape II. Trois espèces de langues et de caractères.} — {\itshape Langues et caractères hiéroglyphiques, symboliques et emblématiques, vulgaires.}\par
Chapitre {\scshape III. Trois espèces du jurisprudence, d’autorité, de raison.} — {\itshape Corollaires relatifs à la}  {\itshape politique et au droit des Romains. — § I. Jurisprudence divine, qui se confondait avec la divination ; jurisprudence héroïque ou aristocratique, attachée rigoureusement aux formules ; jurisprudence humaine, dont la règle est l’équité naturelle.} — § II. {\itshape Autorité dans le sens de propriété ; autorité de tutelle}\footnote{Orthographié « tutèle » [NdE].}{\itshape  ; autorité de conseil.} — § III. {\itshape Raison divine, connue par les auspices ; raison d’état ; raison populaire, d’accord avec l’équité naturelle.} — § IV. {\itshape Corollaire relatif à la sagesse politique des anciens Romains.} — § V. {\itshape Corollaire relatif à l’histoire fondamentale du droit romain.}\par
Chapitre {\scshape IV. Trois espèces de jugements}. — § I. {\itshape Jugements divins et duels. Ce droit imparfait fut nécessaire au repos des nations. Il en est de même des jugements héroïques, rigoureusement conformes aux formules consacrées. Jugements humains, ou discrétionnaires.} — § II. {\itshape Trois périodes dans l’histoire des mœurs et de la jurisprudence (}sectæ temporum).\par
Chapitre {\scshape V. Autres preuves} {\itshape tirées des caractères propres aux aristocraties héroïques.} — § I. {\itshape De la garde et conservation des limites.} — § II. {\itshape De la conservation et distinction des ordres politiques. Jalousie avec laquelle les aristocraties primitives prohibaient les mariages entre les nobles et les plébéiens. On a mal entendu les connubia patrum que demandait le peuple romain. Pourquoi les empereurs romains favorisèrent la confusion des ordres.} — § III. {\itshape De la garde}  {\itshape des lois. Elle est plus ou moins sévère selon la forme du gouvernement. L’attachement des Romains à leur ancienne législation fut une des principales causes de leur grandeur.}\par
Chapitre VI. — § {\scshape I. Autres preuves} {\itshape tirées de la manière dont chaque état nouveau de la société se combine avec le gouvernement de l’état précédent. La démocratie conserve quelque chose de l’état aristocratique qui a précédé, etc.} — § II. {\itshape C’est une loi naturelle que les nations terminent leur carrière politique par la monarchie.} — § III. {\itshape Réfutation de Bodin, qui veut que les gouvernements aient été d’abord monarchiques, en dernier lieu aristocratiques.}\par
Chapitre VII. — § {\scshape I. Dernières preuves.} — § II. {\itshape Corollaire : que l’ancien droit romain à son premier âge fut un poème sérieux, et l’ancienne jurisprudence une poésie sévère, dans laquelle on trouve la première ébauche de la métaphysique légale. Les formules antiques étaient des espèces de drames. Les jurisconsultes ont remarqué l’indivisibilité des droits, mais non pas leur éternité.}\par
Note. {\itshape Comment chez les Grecs la philosophie sortit de la législation.}
\chapterclose


\chapteropen
\chapter[{Chapitre I. Introduction. Trois sortes de natures, de mœurs, de droits naturels, de gouvernements}]{Chapitre I. \\
Introduction. Trois sortes de natures, de mœurs, de droits naturels, de gouvernements}

\chaptercont
\section[{§ I. Introduction}]{§ I. {\itshape Introduction}}
\noindent  Nous avons au livre premier établi les {\itshape principes} de la Science nouvelle ; au livre second, nous avons recherché et découvert dans la {\itshape sagesse poétique l’origine de toutes les choses divines et humaines} que nous présente l’histoire du paganisme ; au troisième, nous avons trouvé que les {\itshape poèmes d’Homère} étaient pour l’histoire de la Grèce, comme les lois des douze tables pour celle du Latium, {\itshape un trésor de faits relatifs au droit naturel des gens}. Maintenant, éclairés sur tant de points par la philosophie et par la philologie, nous allons dans ce quatrième livre esquisser l’{\itshape histoire idéale} indiquée dans les axiomes, et exposer {\itshape la marche que suivent éternellement les nations}. Nous les montrerons, malgré la variété infinie de leurs mœurs, tourner sans en sortir jamais dans ce cercle des {\scshape trois âges}, {\itshape divin, héroïque et humain}.\par
 Dans cet ordre immuable, qui nous offre un étroit enchaînement de causes et d’effets, nous distinguerons trois sortes de {\itshape natures} desquelles dérivent trois sortes de {\itshape mœurs} ; de ces mœurs elles-mêmes découlent trois espèces de {\itshape droits naturels} qui donnent lieu à autant de {\itshape gouvernements}. Pour que les hommes déjà entrés dans la société pussent se communiquer les mœurs, droits et gouvernements dont nous venons de parler, il se forma trois sortes de {\itshape langues} et de {\itshape caractères}. Aux trois âges répondirent encore trois espèces de {\itshape jurisprudences} appuyées d’autant d’{\itshape autorités} et de {\itshape raisons} diverses, donnant lieu à autant d’espèces de {\itshape jugements}, et suivies dans trois {\itshape périodes} ({\itshape sectæ temporum}). Ces trois {\itshape unités d’espèces} avec beaucoup d’autres qui en sont une suite, se rassemblent elles-mêmes dans une {\itshape unité générale}, celle de {\itshape la religion honorant une Providence} ; c’est là l’{\itshape unité d’esprit} qui donne la {\itshape forme} et la {\itshape vie} au monde social.\par
Nous avons déjà traité séparément de toutes ces choses dans plusieurs endroits de cet ouvrage ; nous montrerons ici l’ordre qu’elles suivent dans le cours des affaires humaines.
\section[{§ II. Trois espèces de natures}]{§ II. {\itshape Trois espèces de natures}}
\noindent Maîtrisée par les illusions de l’imagination, faculté d’autant plus forte que le raisonnement est plus faible, la première nature fut {\itshape poétique} ou {\itshape créatrice}. Qu’on nous permette de l’appeler {\itshape divine} ; elle anima en effet et divinisa les êtres matériels selon  l’idée qu’elle se formait des dieux. Cette nature fut celle des {\itshape poètes-théologiens}, les plus anciens sages du paganisme, car toutes les sociétés païennes eurent chacune pour base sa croyance en ses dieux particuliers. Du reste, la nature des premiers hommes était {\itshape farouche} et {\itshape barbare} ; mais la même erreur de leur imagination leur inspirait une profonde terreur des dieux qu’ils s’étaient faits eux-mêmes, et la religion commençait à dompter leur farouche indépendance. ({\itshape Voy.} l’axiome 31.)\par
La seconde nature fut {\itshape héroïque} ; les héros se l’attribuaient eux-mêmes, comme un privilège de leur divine origine. Rapportant tout à l’action des dieux, ils se tenaient pour {\itshape fils de Jupiter} ; c’est-à-dire pour engendrés sous les auspices de Jupiter, et ce n’était pas sans raison, qu’ils se regardaient comme supérieurs par cette noblesse naturelle à ceux qui pour échapper aux querelles sans cesse renouvelées par la promiscuité infâme de l’état bestial se réfugiaient dans leurs asiles, et qui, arrivant sans religion, sans dieux, étaient regardés par les héros comme de vils animaux.\par
Le troisième âge fut celui de la nature {\itshape humaine intelligente}, et par cela même {\itshape modérée, bienveillante et raisonnable} ; elle reconnaît pour lois la conscience, la raison, le devoir.
\section[{§ III. Trois sortes de mœurs}]{§ III. {\itshape Trois sortes de mœurs}}
\noindent Les premières mœurs eurent ce caractère de {\itshape piété} et de {\itshape religion} que l’on attribue à Deucalion et Pyrrha,  à peine échappés aux eaux du déluge. — Les secondes furent celles d’hommes {\itshape irritables et susceptibles sur le point d’honneur}, tels qu’on nous représente Achille. — Les troisièmes furent {\itshape réglées par le devoir} ; elles appartiennent à l’époque où l’on fait consister l’honneur dans l’accomplissement des devoirs civils.
\section[{§ IV. Trois espèces de droits naturels}]{§ IV. {\itshape Trois espèces de droits naturels}}
\noindent {\itshape Droit divin.} Les hommes voyant en toutes choses les dieux ou l’action des dieux, se regardaient, eux et tout ce qui leur appartenait, comme dépendant immédiatement de la divinité.\par
{\itshape Droit héroïque}, ou droit de la force, mais de la force maîtrisée d’avance par la religion qui seule peut la contenir dans le devoir, lorsque les lois humaines n’existent pas encore, ou sont impuissantes pour la réprimer. La Providence voulut que les premiers peuples naturellement fiers et féroces trouvassent dans leur croyance religieuse un motif de se soumettre à la force, et qu’incapables encore de raison, ils jugeassent du droit par le succès, de la raison par la fortune ; c’était pour prévoir les événements que la fortune amènerait qu’ils employaient la divination. Ce droit de la force est le droit d’Achille, qui place toute raison à la pointe de son glaive.\par
En troisième lieu vint le {\itshape droit humain}, dicté par la raison humaine entièrement développée.
\section[{§ V. Trois espèces de gouvernements}]{§ V. {\itshape Trois espèces de gouvernements}}
\noindent  {\itshape Gouvernements divins}, ou {\itshape théocraties}. Sous ces gouvernements, les hommes croyaient que toute chose était commandée par les dieux. Ce fut l’âge des oracles, la plus ancienne institution que l’histoire nous fasse connaître.\par
{\itshape Gouvernements héroïques} ou {\itshape aristocratiques}. Le mot {\itshape aristocrates} répond en latin à {\itshape optimates}, pris pour {\itshape les plus forts} ({\itshape ops}, puissance) ; il répond en grec à {\itshape Héraclides}, c’est-à-dire, issus d’une race d’Hercule pour dire une race noble. Ces {\itshape Héraclides} furent répandus dans toute l’ancienne Grèce, et il en resta toujours à Sparte. Il en est de même des {\itshape curètes} que les Grecs retrouvèrent dans l’ancienne Italie ou {\itshape Saturnie}, dans la Crète et dans l’Asie. Ces {\itshape curètes} furent à Rome les {\itshape quirites}, ou citoyens investis du caractère sacerdotal, du droit de porter les armes, et de voter aux assemblées publiques.\par
{\itshape Gouvernements humains}, dans lesquels l’égalité de la nature intelligente, caractère propre de l’humanité se retrouve dans l’égalité civile et politique. Alors tous les citoyens naissent libres, soit qu’ils jouissent d’un gouvernement populaire dans lequel la totalité ou la majorité des citoyens constitue la force légitime de la cité, soit qu’un monarque place tous ses sujets sous le niveau des mêmes lois, et qu’ayant seul en main la force militaire, il s’élève au-dessus des citoyens par une distinction purement civile.
\chapterclose


\chapteropen
\chapter[{Chapitre II. Trois espèces de langues et de caractères}]{Chapitre II. \\
Trois espèces de langues et de caractères}

\chaptercont
\section[{§ I. Trois espèces de langues}]{§ I. {\itshape Trois espèces de langues}}
\noindent  {\itshape Langue divine mentale}, dont les signes sont des cérémonies sacrées, des actes muets de religion. Le droit romain en conserva ses {\itshape acta legitima}, qui accompagnaient toutes les transactions civiles. Une telle langue convient aux religions pour la raison que nous avons déjà dite, c’est qu’elles ont plus besoin d’être révérées que {\itshape raisonnées}. Cette langue fut nécessaire aux premiers âges, où les hommes ne pouvaient encore articuler.\par
La seconde {\itshape langue} fut celle {\itshape des signes héroïques} ; c’est le {\itshape langage des armes}, pour ainsi parler ; et il est resté celui de la discipline militaire.\par
La troisième est le {\itshape langage articulé}, que parlent aujourd’hui toutes les nations.
\section[{§ II. Trois espèces de caractères}]{§ II. {\itshape Trois espèces de caractères}}
\noindent {\itshape Caractères divins}, proprement {\itshape hiéroglyphes}. Nous avons prouvé qu’à leur premier âge, toutes les nations se servirent de tels caractères. À Jupiter on rapporta tout ce qui regardait les auspices ; à Junon tout ce qui était relatif aux mariages. En effet {\itshape c’est}  {\itshape une propriété innée de l’âme humaine d’aimer l’uniformité} ; lorsqu’elle est encore incapable de trouver par l’{\itshape abstraction} des expressions générales, elle y supplée par l’{\itshape imagination} ; elle choisit certaines images, certains modèles, auxquels elle rapporte toutes les espèces particulières qui appartiennent à chaque genre ; ce sont pour emprunter le langage de l’école, des {\itshape universaux poétiques}.\par
{\itshape Caractères héroïques}, analogues aux précédents. C’étaient encore des {\itshape universaux poétiques} qui servaient à désigner les diverses espèces d’objets qui occupaient l’esprit des héros ; ils attribuaient à Achille tous les exploits des guerriers vaillants, à Ulysse tous les conseils des sages\footnote{Lorsque l’esprit humain s’habitua à abstraire les {\itshape formes} et les {\itshape propriétés} des {\itshape sujets}, ces {\itshape universaux poétiques}, ces genres créés par l’imagination ({\itshape generi fantastici}), firent place à ceux que la raison créa ({\itshape generi intelligibili}), c’est alors que vinrent les philosophes ; et plus tard encore, les auteurs de la nouvelle comédie, dont l’époque est pour la Grèce celle de la plus haute civilisation, prirent des philosophes l’idée de ces derniers genres et les personnifièrent dans leurs comédies. ({\itshape Vico.})}.\par
Les {\itshape caractères vulgaires} parurent avec les {\itshape langues vulgaires}. Les langues vulgaires se composent de paroles qui sont comme des genres relativement aux expressions particulières dont se composaient les langues héroïques\footnote{Ainsi comme nous l’avons dit plus haut, la phrase héroïque, {\itshape le sang me bout dans le cœur}, fut résumée dans la langue vulgaire par ce mot abstrait et général, {\itshape je suis en colère}. ({\itshape Vico.})}. Les lettres remplacèrent aussi les hiéroglyphes d’une manière plus simple et plus générale ; à cent vingt mille caractères hiéroglyphiques, que les Chinois emploient encore aujourd’hui,  on substitua les lettres si peu nombreuses de l’alphabet.\par
Ces langues, ces lettres peuvent être appelées {\itshape vulgaires}, puisque le vulgaire a sur elles une sorte de souveraineté. Le pouvoir absolu du peuple sur les langues s’étend sous un rapport à la législation : le peuple donne aux lois le sens qui lui plaît, et il faut, bon gré malgré, que les puissants en viennent à observer les lois dans le sens qu’y attache le peuple. Les monarques ne peuvent ôter aux peuples cette souveraineté sur les langues ; mais elle est utile à leur puissance même. Les grands sont obligés d’observer les lois par lesquelles les rois fondent la monarchie, dans le sens ordinairement favorable à l’autorité royale que le peuple donne à ces lois. C’est une des raisons qui montrent que la démocratie précède nécessairement la monarchie\footnote{Voyez dans Tacite comment la monarchie s’établit à Rome à la faveur des titres républicains que privent les empereurs, et auxquels le peuple donna peu-à-peu un nouveau sens. ({\itshape Note du Traducteur.})}.
\chapterclose


\chapteropen
\chapter[{Chapitre III. Trois espèces de jurisprudences, d’autorités, de raisons ; corollaires relatifs à la politique et au droit des Romains}]{Chapitre III. \\
Trois espèces de jurisprudences, d’autorités, de raisons ; corollaires relatifs à la politique et au droit des Romains}

\chaptercont
\section[{§ I. Trois espèces de jurisprudences ou sagesses}]{§ I. {\itshape Trois espèces de jurisprudences ou sagesses}}
\noindent  {\itshape Sagesse divine} appelée {\itshape théologie mystique}, mots qui dans leur sens étymologique veulent dire, science du langage divin, connaissance des mystères de la {\itshape divination}. Cette science de la divination était la {\itshape sagesse vulgaire} de laquelle étaient {\itshape sages} les {\itshape poètes théologiens}, premiers sages du paganisme ; de cette théologie {\itshape mystique}, ils s’appelaient eux-mêmes {\itshape mystæ}, et Horace traduit ce mot d’une manière heureuse par {\itshape interprètes des dieux}… Cette sagesse ou jurisprudence plaçait la justice dans l’accomplissement des cérémonies solennelles de la religion ; c’est de là que les Romains conservèrent ce respect superstitieux pour les {\itshape acta legitima} ; chez eux les noces, le testament étaient dits {\itshape justa} lorsque les cérémonies requises avaient été accomplies.\par
La {\itshape jurisprudence héroïque} eut pour caractère de s’entourer de garantie par l’emploi de paroles précises.  C’est la sagesse d’Ulysse qui dans Homère approprie si bien son langage au but qu’il se propose, qu’il ne manque point de l’atteindre. La réputation des jurisconsultes romains était fondée sur leur {\itshape cavere} ; {\itshape répondre sur le droit}, ce n’était pour eux autre chose que précautionner les consultants, et les préparer à circonstancier devant les tribunaux le cas contesté de manière que les formules d’action s’y rapportassent de point en point, et que le préteur ne pût refuser de les appliquer. Il en fut des docteurs du moyen âge comme des jurisconsultes romains.\par
La {\itshape jurisprudence humaine} ne considère dans les faits que leur conformité avec la justice et la vérité ; sa {\itshape bienveillance} plie les lois à tout ce que demande l’intérêt égal des causes. Cette jurisprudence est observée sous les {\itshape gouvernements humains}, c’est-à-dire, dans les états populaires, et surtout dans la monarchie. La jurisprudence {\itshape divine et l’héroïque} propres aux âges de barbarie, s’attachent au {\itshape certain} ; la jurisprudence {\itshape humaine} qui caractérise les âges civilisés, ne se règle que sur le {\itshape vrai}. Tout ceci découle de la définition du {\itshape certain} et du {\itshape vrai} que nous avons donnée (axiomes 9 et 10).
\section[{§ II. Trois espèces d’autorités}]{§ II. {\itshape Trois espèces d’autorités}}
\noindent La première est {\itshape divine} ; elle ne comporte point d’explications ; comment demander à la Providence compte de ses décrets ? La deuxième, l’autorité {\itshape héroïque}, appartient tout entière aux formules solennelles  des lois. La troisième est l’autorité {\itshape humaine}, laquelle n’est autre que le crédit des personnes expérimentées, des hommes remarquables par une haute sagesse dans la spéculation ou par une prudence singulière dans la pratique.\par
À ces trois autorités civiles répondent trois autorités politiques.\par
Au premier âge, {\itshape autorité} et {\itshape propriété} furent synonymes. C’est dans ce sens que la loi des douze tables prend toujours le mot {\itshape autorité} ; {\itshape auteur} signifie toujours en terme de droit celui de qui on tient un {\itshape domaine}. Cette autorité était {\itshape divine}, parce qu’alors la propriété comme tout le reste était rapportée aux dieux. Cette autorité qui appartient aux {\itshape pères} dans l’état de famille, appartient aux {\itshape sénats souverains} dans les aristocraties héroïques. Le sénat autorisait ce qui avait été délibéré dans les assemblées du peuple.\par
Depuis la loi de Publilius Philo qui assura au peuple romain la liberté et la souveraineté, le sénat n’eut plus qu’une {\itshape autorité de tutelle}, analogue à ce droit des tuteurs, d’autoriser en affaires légales le pupille maître de ses biens. Le sénat assistait le peuple de sa présence dans les assemblées législatives, de peur qu’il ne résultât quelque dommage public de son peu de lumières.\par
Enfin l’état populaire faisant place à la monarchie, l’{\itshape autorité de tutelle} fut aussi remplacée par l’{\itshape autorité de conseil}, par celle que donne la réputation de sagesse ; c’est dans ce sens que les jurisconsultes  de l’empire s’appelèrent {\itshape autores}, auteurs de conseils. Telle aussi doit être l’autorité d’un sénat sous un monarque, lequel a pleine liberté de suivre ou de rejeter ce qui a été conseillé par le sénat.
\section[{§ III. Trois espèces de raisons}]{§ III. {\itshape Trois espèces de raisons}}
\noindent La première est la {\itshape raison divine}, dont Dieu seul a le secret, et dont les hommes ne savent que ce qui en a été révélé aux Hébreux et aux Chrétiens, soit au moyen d’un langage {\itshape intérieur} adressé à l’intelligence par celui qui est lui-même tout intelligence, soit par le langage {\itshape extérieur} des prophètes, langage que le Sauveur a parlé aux apôtres, qui ont ensuite transmis à l’église ses enseignements. Les Gentils ont cru aussi recevoir les conseils de cette {\itshape raison divine} par les auspices, par les oracles, et autres signes matériels, tels qu’ils pouvaient en recevoir de dieux qu’ils croyaient {\itshape corporels}. Dieu étant toute raison, la {\itshape raison} et l’{\itshape autorité} sont en lui une même chose, et pour la saine théologie l’{\itshape autorité divine} équivaut à la {\itshape raison}. — Admirons la Providence, qui dans les premiers temps où les hommes encore idolâtres étaient incapables d’entendre la {\itshape raison}, permit qu’à son défaut ils suivissent l’{\itshape autorité} des auspices, et se gouvernassent par les avis divins qu’ils croyaient en recevoir. En effet c’est une loi éternelle que lorsque les hommes ne voient point la {\itshape raison} dans les choses humaines, ou que même ils les voient {\itshape contraires à la raison}, ils se reposent  sur les conseils impénétrables de la Providence.\par
La seconde sorte de raison fut la {\itshape raison d’état}, appelée par les Romains {\itshape civilis æquitas}. C’est d’elle qu’Ulpien dit qu’\emph{{\itshape elle n’est point connue naturellement à tous les hommes}} (comme l’équité naturelle), \emph{{\itshape mais seulement à un petit nombre d’hommes qui ont appris par la pratique du gouvernement ce qui est nécessaire au maintien de la société}}. Telle fut la sagesse des sénats {\itshape héroïques}, et particulièrement celle du sénat romain, soit dans les temps où l’aristocratie décidait seule des intérêts publics, soit lorsque le peuple déjà maître se laissait encore guider par le sénat, ce qui eut lieu jusqu’au tribunal des Gracques.
\section[{§ IV. Corollaire relatif à la sagesse politique des anciens Romains}]{§ IV. {\itshape Corollaire relatif à la sagesse politique des anciens Romains}}
\noindent Ici se présente une question à laquelle il semble bien difficile de répondre : lorsque Rome était encore peu avancée dans la civilisation, ses citoyens passaient pour de sages politiques ; et dans le siècle le plus éclairé de l’empire, Ulpien se plaint qu’\emph{{\itshape un petit nombre d’hommes expérimentés possèdent la science du gouvernement}}.\par
Par un effet des mêmes causes qui firent l’{\itshape héroïsme} des premiers peuples, les anciens Romains qui ont été {\itshape les héros du monde}, se sont montrés naturellement fidèles à l’{\itshape équité civile}. Cette équité s’attachait religieusement aux paroles de la loi, les  suivait avec une sorte de superstition, et les appliquait aux faits d’une manière inflexible, quelque {\itshape dure}, quelque cruelle même que pût se trouver la loi. Ainsi agit encore de nos jours la {\itshape raison d’état}. L’{\itshape équité civile} soumettait naturellement toute chose à cette loi, reine de toutes les autres, que Cicéron exprime avec une gravité digne de la matière : \emph{{\itshape la loi suprême c’est le salut du peuple}}, \emph{{\itshape suprema lex populi salus esto}}. Dans les temps {\itshape héroïques} où les gouvernements étaient aristocratiques, les héros avaient dans l’intérêt public une grande part d’intérêt privé, je parle de leur {\itshape monarchie domestique} que leur conservait la société civile. La grandeur de cet intérêt particulier leur en faisait sacrifier sans peine d’autres moins importants. C’est ce qui explique le courage qu’ils déployaient en défendant l’état, et la prudence avec laquelle ils réglaient les affaires publiques. Sagesse profonde de la Providence ! Sans l’attrait d’un tel intérêt privé identifié avec l’intérêt public, comment ces pères de famille à peine sortis de la vie sauvage, et que Platon reconnaît dans le Polyphème d’Homère, auraient-ils pu être déterminés à suivre l’ordre civil ?\par
Il en est tout au contraire dans les temps {\itshape humains}, où les états sont démocratiques ou monarchiques. Dans les démocraties, les citoyens règnent sur la chose publique qui, se divisant à l’infini, se répartit entre tous les citoyens qui composent le peuple souverain. Dans les monarchies, les sujets sont obligés de s’occuper exclusivement de leurs  intérêts particuliers, en laissant au prince le soin de l’intérêt public. Joignez à cela les causes naturelles qui produisent les gouvernements {\itshape humains}, et qui sont toutes contraires à celles qui avaient produit l’{\itshape héroïsme}, puisqu’elles ne sont autres que désir du repos, amour paternel et conjugal, attachement à la vie. Voilà pourquoi les hommes d’aujourd’hui sont portés naturellement à considérer les choses d’après les circonstances les plus particulières qui peuvent rapprocher les intérêts privés d’une justice égale ; c’est l’{\itshape æquum bonum}, l’intérêt égal, que cherche la troisième espèce de raison, la raison naturelle, {\itshape æquitas naturalis} chez les jurisconsultes. La multitude n’en peut comprendre d’autre, parce qu’elle considère les motifs de justice dans leurs applications directes aux causes selon l’espèce individuelle des faits. Dans les monarchies il faut peu d’hommes d’état pour traiter des affaires publiques dans les cabinets en suivant l’équité civile ou raison d’état ; et un grand nombre de jurisconsultes pour régler les intérêts privés des peuples d’après l’{\itshape équité naturelle}.
\section[{§ V. Corollaire. Histoire fondamentale du Droit romain}]{§ V. {\itshape Corollaire. Histoire fondamentale du Droit romain}}
\noindent Ce que nous venons de dire sur les trois espèces de raisons peut servir de base à l’histoire du Droit romain. En effet {\itshape les gouvernements doivent être conformes à la nature des gouvernés} (axiome 69) ; les gouvernements sont même un résultat de cette nature, et les lois doivent en conséquence être appliquées  et interprétées d’une manière qui s’accorde avec la forme de ce gouvernement. Faute d’avoir compris cette vérité, les jurisconsultes et les interprètes du droit sont tombés dans la même erreur que les historiens de Rome, qui nous racontent que telles lois ont été faites à telle époque, sans remarquer les rapports qu’elles devaient avoir avec les différents états par lesquels passa la république. Ainsi les faits nous apparaissent tellement séparés de leurs causes, que Bodin, jurisconsulte et politique également distingué, montre tous les caractères de l’aristocratie dans les faits que les historiens rapportent à la prétendue démocratie des premiers siècles de la république. — Que l’on demande à tous ceux qui ont écrit sur l’histoire du Droit romain, pourquoi la jurisprudence {\itshape antique}, dont la base est la loi des douze tables, s’y conforme rigoureusement ; pourquoi la jurisprudence {\itshape moyenne}, celle que réglaient les édits des préteurs, commence à s’adoucir, en continuant toutefois de respecter le même code ; pourquoi enfin la jurisprudence {\itshape nouvelle}, sans égard pour cette loi, eut le courage de ne plus consulter que l’équité naturelle ? Ils ne peuvent répondre qu’en calomniant la générosité romaine, qu’en prétendant que ces rigueurs, ces solennités, ces scrupules, ces subtilités verbales, qu’enfin le mystère même dont on entourait les lois, étaient autant d’impostures des nobles qui voulaient conserver avec le privilège de la jurisprudence le pouvoir civil qui y est naturellement attaché. Bien  loin que ces pratiques aient eu aucun but d’imposture, c’étaient des usages sortis de la nature même des hommes de l’époque ; une telle nature devait produire de tels usages, et de tels usages devaient entraîner nécessairement de telles pratiques.\par
Dans le temps où le genre humain était encore extrêmement farouche, et où la religion était le seul moyen puissant de l’adoucir et de le civiliser, la Providence voulut que les hommes vécussent sous les gouvernements {\itshape divins}, et que partout régnassent des lois {\itshape sacrées}, c’est-à-dire {\itshape secrètes}, et cachées au vulgaire des peuples. Elles restaient d’autant plus facilement cachées dans l’état de famille, qu’elles se conservaient dans un {\itshape langage muet}, et ne s’expliquaient que par des cérémonies saintes, qui restèrent ensuite dans les {\itshape acta legitima}. Ces esprits grossiers encore croyaient de telles cérémonies indispensables, pour s’assurer de la volonté des autres, dans les rapports d’intérêt, tandis qu’aujourd’hui que l’intelligence des hommes est plus ouverte, il suffit de simples paroles et même de signes.\par
Sous les gouvernements {\itshape aristocratiques} qui vinrent ensuite, les mœurs étant toujours religieuses, les lois restèrent entourées du mystère de la religion et furent observées avec la sévérité et les scrupules qui en sont inséparables ; le secret est l’âme des aristocraties, et la rigueur de l’{\itshape équité civile} est ce qui fait leur salut. Puis, lorsque se formèrent les démocraties, sorte de gouvernement dont le caractère est plus ouvert et plus généreux et dans lequel commande la multitude  qui a l’instinct de l’{\itshape équité naturelle}, on vit paraître en même temps les langues et les lettres vulgaires, dont la multitude est, comme nous l’avons dit, souveraine absolue. Ce langage et ces caractères servirent à promulguer, à écrire les lois dont le secret fut peu à peu dévoilé. Ainsi le peuple de Rome ne souffrit plus le droit caché, \emph{{\itshape jus latens}} dont parle Pomponius ; et voulut avoir des lois écrites sur des tables, lorsque les caractères vulgaires eurent été apportés de Grèce à Rome.\par
Cet ordre de choses se trouva tout préparé pour la monarchie. Les monarques veulent suivre l’{\itshape équité naturelle} dans l’application des lois, et se conforment en cela aux opinions de la multitude. Ils égalent en droit les puissants et les faibles, ce que fait la seule monarchie. L’{\itshape équité civile}, ou {\itshape raison d’état}, devient le privilège d’un petit nombre de politiques et conserve dans le cabinet des rois son caractère mystérieux.
\chapterclose


\chapteropen
\chapter[{Chapitre IV. Trois espèces de jugements. — Corollaire relatif au duel et aux représailles. — Trois périodes dans l’histoire des mœurs et de la jurisprudence}]{Chapitre IV. \\
Trois espèces de jugements. — Corollaire relatif au duel et aux représailles. — Trois périodes dans l’histoire des mœurs et de la jurisprudence}

\chaptercont
\section[{§ I. Trois espèces de jugements}]{§ I. {\itshape Trois espèces de jugements}}
\noindent  Les premiers furent les {\itshape jugements divins}. Dans l’état qu’on appelle {\itshape état de nature}, et qui fut celui {\itshape des familles}, les pères de familles ne pouvant recourir à la protection des lois qui n’existaient point encore, en appelaient aux dieux des torts qu’ils souffraient, {\itshape implorabant deorum fidem} ; tel fut le premier sens, le sens propre de cette expression. Ils appelaient les dieux en témoignage de leur bon droit, ce qui était proprement {\itshape deos obtestari}. Ces invocations pour accuser, ou se défendre, furent les premières {\itshape orationes}, mot qui chez les Latins est resté pour signifier {\itshape accusation} ou {\itshape défense} ; on peut voir à ce sujet plusieurs beaux passages de Plaute et de Térence, et deux mots de la loi des douze tables : {\itshape furto orare}, et {\itshape pacto orare} (et non point {\itshape adorare}, selon la leçon de Juste Lipse), pour {\itshape agere, excipere}. D’après ces {\itshape orationes}, les Latins appelèrent  {\itshape oratores} ceux qui défendent les causes devant les tribunaux. Ces appels aux dieux étaient faits d’abord par des hommes simples et grossiers qui croyaient s’en faire entendre sur la cime des monts où l’on plaçait leur séjour. Homère raconte qu’ils habitaient sur celle de l’Olympe. À propos d’une guerre entre les Hermundures et les Cattes, Tacite dit en parlant des sommets des montagnes : dans l’opinion de ces peuples \emph{{\itshape preces mortalium nusquàm propiùs audiuntur}}. Les droits que les premiers hommes faisaient valoir dans ces {\itshape jugements divins} étaient divinisés eux-mêmes, puisqu’ils voyaient des dieux dans tous les objets. {\itshape Lar} signifiait la propriété de la maison, {\itshape dii hospitales} l’hospitalité, {\itshape dii penates} la puissance paternelle, {\itshape deus genius} le droit du mariage, {\itshape deus terminus} le domaine territorial, {\itshape dii manes} la sépulture. On retrouve dans les douze tables une trace curieuse de ce langage, \emph{{\itshape jus deorum manium}}.\par
Après avoir employé ces invocations ({\itshape orationes, obsecrationes, implorationes}, et encore {\itshape obtestationes}), ils finissaient par dévouer les coupables. Il y avait à Argos, et sans doute aussi dans d’autres parties de la Grèce, des temples de l’{\itshape exécration}. Ceux qui étaient ainsi dévoués étaient appelés αναθήματα nous dirions {\itshape excommuniés} ; ensuite on les mettait à mort. C’était le culte des Scythes qui enfonçaient un couteau en terre, l’adoraient comme un Dieu, et immolaient ensuite une victime humaine. Les Latins exprimaient cette idée par le verbe {\itshape mactare}, dont  on se servait toujours dans les sacrifices, comme d’un terme consacré. Les Espagnols en ont tiré leur {\itshape matar}, et les Italiens leur {\itshape ammazzare}. Nous avons déjà vu que chez les Grecs, ἄρα signifiait la chose ou la personne qui porte dommage, le vœu ou action de dévouer, et la furie à laquelle on dévouait ; chez les Latins {\itshape ara} signifiait l’autel et la victime. Ainsi toutes les nations eurent toujours une espèce d’excommunication. César nous a laissé beaucoup de détails sur celle qui avait lieu chez les Gaulois. Les Romains eurent leur {\itshape interdiction de l’eau et du feu}. Plusieurs consécrations de ce genre passeront dans la loi des douze tables : quiconque violait la personne d’un tribun du peuple était dévoué, consacré à Jupiter ; le fils dénaturé, aux dieux paternels ; à Cérès, celui qui avait mis le feu à la moisson de son voisin ; ce dernier était brûlé vif. Rappelons-nous ici ce qui a été dit de l’atrocité des peines dans l’âge divin (axiome 40). Les hommes ainsi dévoués furent sans doute ce que Plaute appelle \emph{{\itshape Saturni hostiæ}}.\par
On trouve le caractère tout religieux de ces jugements privés dans les guerres qu’on appelait {\itshape pura et pia bella}. Les peuples y combattaient {\itshape pro aris et focis}, expression qui désignait tout l’ensemble des rapports sociaux, puisque toutes les choses humaines étaient considérées comme {\itshape divines}. Les hérauts qui déclaraient la guerre appelaient les dieux de la cité ennemie hors de ses murs, et dévouaient le peuple attaqué. Les rois vaincus étaient présentés au capitole à Jupiter Férétrien, et ensuite immolés.  Les vaincus étaient considérés comme des {\itshape hommes sans Dieu} ; aussi les esclaves s’appelaient en latin {\itshape mancipia}, comme choses inanimées, et étaient tenus en jurisprudence {\itshape loco rerum}.\par
Les {\itshape duels} durent être chez les nations barbares une espèce de {\itshape jugements divins}, qui commencèrent sous les {\itshape gouvernements divins} et furent longtemps en usage sous les {\itshape gouvernements héroïques} ; on se rappelle ce passage de la {\itshape Politique} d’Aristote (cité dans les axiomes) où il dit que les \emph{{\itshape républiques héroïques n’avaient point de lois qui punissent l’injustice et réprimassent les violences particulières}}\footnote{On ne pouvait jusqu’ici ajouter foi à cette vérité tant que l’on attribuait aux premiers peuples ce parfait héroïsme imaginé par les philosophes ; préjugé qui résultait d’une opinion exagérée que l’on s’était formée de la sagesse des anciens. ({\itshape Vico.})}. Il est certain que dans la législation romaine ce ne sont que les préteurs qui introduisirent la loi prohibitive contre la violence, et les actions {\itshape de vi bonorum raptorum}. Aux temps de la seconde barbarie (celle du moyen âge), les représailles particulières durèrent jusqu’au temps de Barthole.\par
C’est par erreur que quelques-uns ont écrit que les duels s’étaient introduits {\itshape par défauts de preuves} ; ils devaient dire {\itshape par défauts de lois judiciaires}. Frotho, roi de Danemark\footnote{Orthographié « Danemarck » [NdE].}, ordonna que toutes les contestations se terminassent par le moyen du duel : c’était défendre qu’on les terminât par des jugements selon le droit. On ne voit qu’ordonnances du duel dans les lois des Lombards, des Francs, des Bourguignons,  des Allemands, des Anglais, des Normands et des Danois.\par
On n’a pas cru que la {\itshape barbarie antique} eût aussi connu l’usage du duel. Mais doit-on penser que ces premiers hommes, que ces {\itshape géants}, ces {\itshape cyclopes}, aient su endurer l’injustice. L’absence de lois dont parle Aristote devait les forcer de recourir aux duels. D’ailleurs deux traditions fameuses de l’antiquité grecque et latine prouvent que les peuples commençaient souvent les guerres ({\itshape duella} chez les anciens Latins), en décidant par un duel la querelle particulière des principaux intéressés ; je parle du combat de Ménélas contre Pâris, et des trois Horaces contre les trois Curiaces ({\itshape Voy.} page 208) si le combat restait indécis, comme dans le premier cas, la guerre commençait.\par
Dans ces jugements par les armes, ils estimaient la raison et le bon droit, d’après le hasard de la victoire. Ils durent tomber dans cette erreur par un conseil exprès de la Providence : chez des peuples barbares, encore incapables de raisonnement, les guerres auraient toujours produit des guerres, s’ils n’eussent jugé que le parti auquel les dieux se montraient contraires, était le parti injuste. Nous voyons que les Gentils insultaient au malheur du saint homme Job, parce que Dieu s’était déclaré contre lui. Lorsque la barbarie antique reparut au moyen âge, on coupait la main droite au vaincu, quelque juste que fût sa cause. C’est cette justice présumée du plus fort qui à la longue légitime les conquêtes ;  ce droit imparfait est nécessaire au repos des nations.\par
Les jugements {\itshape héroïques}, récemment dérivés des jugements {\itshape divins} ne faisaient point acception de causes ou de personnes, et s’observaient avec un respect scrupuleux des paroles. Des jugements {\itshape divins} resta ce qu’on appelait la religion des paroles, {\itshape religio verborum} ; généralement les choses divines sont exprimées par des formules consacrées dans lesquelles on ne peut changer une lettre ; aussi dans les anciennes formules de la jurisprudence romaine, imitée des formules sacrées, on disait : une virgule de moins, la cause est perdue ; {\itshape qui cadit virgulâ, caussâ cadit}. Cette rigueur des formules d’actions eût empêché les duumvirs, nommés pour juger Horace, d’absoudre le vainqueur des Albains quand même il se serait trouvé innocent. Le peuple le renvoya absous, \emph{{\itshape plutôt par admiration pour son courage, que pour la bonté de sa cause}}. (Tite-Live.)\par
Ces jugements inflexibles étaient nécessaires dans des temps où les héros plaçaient dans la force la raison et le bon droit, où ils justifiaient le mot ingénieux de Plaute : \emph{{\itshape pactum non pactum, non pactum pactum}}. Pour prévenir des plaintes, des rixes et des meurtres, la Providence voulut qu’ils fissent consister toute la justice dans l’expression précise des formules solennelles. Ce droit naturel des nations héroïques a fourni le sujet de plusieurs comédies de Plaute ; on y voit souvent un marchand  d’esclaves dépouillé injustement par un jeune homme, qui en lui dressant un piège le fait tomber à son insu, dans quelque cas prévu par la loi, et lui enlève ainsi une esclave qu’il aime. Loin de pouvoir intenter contre le jeune homme une action de dol, le marchand se trouve obligé à lui rembourser le prix de l’esclave vendue ; dans une autre pièce, il le prie de se contenter de la moitié de la peine qu’il a encourue comme coupable de vol {\itshape non manifeste} ; dans une troisième enfin, le marchand s’enfuit du pays, dans la crainte d’être convaincu d’avoir corrompu l’esclave d’autrui. Qui peut soutenir encore qu’au temps de Plaute l’équité naturelle régnait dans les jugements ?\par
Ce droit rigoureux fondé sur la lettre même de la loi, n’était pas seulement en vigueur parmi les hommes ; ceux-ci jugeant les dieux d’après eux ; croyaient qu’ils l’observaient aussi, et même dans leurs serments. Junon, dans Homère, atteste Jupiter, témoin et arbitre des serments, qu’\emph{{\itshape elle n’a point sollicité Neptune d’exciter la tempête contre les Troyens}}, parce qu’elle ne l’a fait que par l’intermédiaire du Sommeil ; et Jupiter se contente de cette réponse. Dans Plaute, Mercure sous la figure de Sosie dit au Sosie véritable : \emph{{\itshape Si je te trompe, puisse Mercure être désormais contraire à Sosie.}} On ne peut croire que Plaute ait voulu mettre sur le théâtre des dieux qui enseignassent le parjure au peuple ; encore bien moins peut-on le croire de Scipion l’Africain et de Lélius, qui, dit-on, aidèrent Térence à composer  ses comédies ; et toutefois dans l’{\itshape Andrienne}, Dave fait mettre l’enfant devant la porte de Simon par les mains de Mysis, afin que si par aventure son maître l’interroge à ce sujet, il puisse en conscience nier de l’avoir mis à cette place. Mais la preuve la plus forte en faveur de notre explication du droit héroïque, c’est qu’à Athènes, lorsqu’on prononça sur le théâtre le vers d’Euripide, ainsi traduit par Cicéron,\par


\begin{verse}
{\itshape Juravi linguâ, mentem injuratam habui},\\
J’ai juré seulement de la bouche, ma conscience n’a pas juré,\\
\end{verse}

\noindent Les spectateurs furent scandalisés et murmurèrent ; on voit qu’ils partageaient l’opinion exprimée dans les douze tables : \emph{{\itshape uti linguâ nuncupassit, ita jus esto}}. Ce respect inflexible de la parole dans les temps héroïques montre bien qu’Agamemnon ne pouvait rompre le vœu téméraire qu’il avait fait d’immoler Iphigénie. C’est pour avoir méconnu le dessein de la Providence [qui voulut qu’aux temps héroïques la parole fût considérée comme irrévocable] que Lucrèce prononce, au sujet de l’action d’Agamemnon, cette exclamation impie,\par


\begin{verse}
Tantùm religio potuit suadere malorum !\\
Tant la religion peut enfanter de maux !\\
\end{verse}

\noindent Ajoutons à tout ceci deux preuves tirées de la jurisprudence et de l’histoire romaines : ce ne fut que vers les derniers temps de la république que Galius Aquilius introduisit dans la législation l’action ({\itshape de dolo}) contre le dol et la mauvaise foi. Auguste  donna aux juges la faculté d’absoudre ceux qui avaient été séduits et trompés.\par
Nous retrouvons la même opinion chez les peuples {\itshape héroïques} dans la guerre comme dans la paix. Selon les termes dans lesquels les traités sont conclus, nous voyons les vaincus être accablés misérablement, ou tromper heureusement le courroux du vainqueur. Les Carthaginois se trouvèrent dans le premier cas : le traité qu’ils avaient fait avec les Romains leur avait assuré la conservation de leur vie, de leurs biens et de leur cité ; par ce dernier mot ils entendaient la {\itshape ville matérielle}, les édifices, {\itshape urbs} dans la langue latine ; mais comme les Romains s’étaient servis dans le traité du mot {\itshape civitas}, qui veut dire la réunion des citoyens, la société, ils s’indignèrent que les Carthaginois refusassent d’abandonner le rivage de la mer pour habiter désormais dans les terres, ils les déclarèrent rebelles, prirent leur ville, et la mirent en cendres ; en suivant ainsi le droit {\itshape héroïque}, ils ne crurent point avoir fait une guerre injuste. Un exemple tiré de l’histoire du moyen âge confirme encore mieux ce que nous avançons. L’Empereur Conrad III ayant forcé à se rendre la ville de Veinsberg qui avait soutenu son compétiteur, permit aux femmes seules d’en sortir avec tout ce qu’elles pourraient emporter ; elles chargèrent sur leur dos leurs fils, leurs maris et leurs pères. L’Empereur était à la porte, les lances baissées, les épées nues, tout prêt à user de la victoire ; cependant malgré sa colère, il laissa échapper  tous les habitants qu’il allait passer au fil de l’épée. Tant il est peu raisonnable de dire que le droit naturel, tel qu’il est expliqué par Grotius, Selden et Pufendorf, a été suivi dans tous les temps, chez toutes les nations !\par
Tout ce que nous venons de dire, tout ce que nous allons dire encore, découle de cette définition que nous avons donnée dans les axiomes, du {\itshape vrai} et du {\itshape certain} dans les lois et conventions. Dans les temps barbares, on doit trouver une jurisprudence rigoureusement attachée aux paroles ; c’est proprement le droit des gens, {\itshape fas gentium}. Il n’est pas moins naturel qu’aux temps {\itshape humains} le droit devenu plus large et plus bienveillant, ne considère plus que {\itshape ce qu’un juge impartial reconnaît être utile dans chaque cause} (axiome 112) ; c’est alors qu’on peut l’appeler proprement le droit de la nature, {\itshape fas naturæ}, le droit de l’{\itshape humanité} raisonnable.\par
Les jugements {\itshape humains} (discrétionnaires) ne sont point aveugles et inflexibles comme les jugements {\itshape héroïques}. La règle qu’on y suit, c’est la vérité des faits. La loi toute bienveillante y interroge la conscience, et selon sa réponse se plie à tout ce que demande l’intérêt égal des causes. Ces jugements sont dictés par une sorte de {\itshape pudeur naturelle, de respect de nos semblables}, qui accompagnent les lumières ; ils sont garantis par la {\itshape bonne foi}, fille de la civilisation. Ils conviennent à l’esprit de franchise, qui caractérise les républiques populaires, ennemies des mystères dont l’aristocratie aime à  s’envelopper ; elles conviennent encore plus à l’esprit généreux des monarchies : les monarques dans ces jugements se font gloire d’être supérieurs aux lois et de ne dépendre que de leur conscience et de Dieu. — Des jugements {\itshape humains}, tels que les modernes les pratiquent pendant la paix, sont sortis les trois systèmes du droit de la guerre que nous devons à Grotius, à Selden, et à Pufendorf.
\section[{§ II. Trois périodes dans l’histoire des mœurs et de la jurisprudence (sectæ temporum)}]{§ II. {\itshape Trois périodes dans l’histoire des mœurs et de la jurisprudence (}sectæ temporum)}
\noindent Nous voyons les jurisconsultes justifier {\itshape sectâ suorum temporum} leurs opinions en matière de droit. Ces {\itshape sectæ temporum} caractérisent la jurisprudence romaine, d’accord en ceci avec tous les peuples du monde. Elles n’ont rien de commun avec les {\itshape sectes des philosophes} que certains interprètes érudits du Droit romain voudraient y voir bon gré malgré. Lorsque les Empereurs exposent les motifs de leurs lois et constitutions, ils disent que de telles constitutions leur ont été dictées \emph{{\itshape sectâ suorum temporum}} ; Brisson, {\itshape De formulis Romanorum}, a recueilli les passages où l’on trouve cette expression. C’est que l’étude des mœurs du temps est l’école des princes. Dans ce passage de Tacite : \emph{{\itshape corrumpere et corrumpi seculum vocant}}, corrompre et être corrompu, voilà ce qui s’appelle le train du siècle, {\itshape seculum} répond à peu près à {\itshape secta}. Nous dirions maintenant : c’est la mode.\par
Toutes les choses dont nous avons parlé se  sont pratiquées dans trois sectes de temps, {\itshape sectæ temporum}, dans le langage des jurisconsultes : celle des temps religieux pendant lesquels régnèrent les gouvernements divins ; celle des temps où les hommes étaient irritables et susceptibles, tels qu’Achille dans l’antiquité, et les duellistes au moyen âge ; celle des temps civilisés, où règne la modération, celle des temps du droit naturel des nations {\scshape humaines}, \emph{{\itshape jus naturale gentium humanorum}} (Ulpien). Chez les auteurs latins du temps de l’Empire, le devoir des sujets se dit {\itshape officium civile}, et toute faute dans laquelle l’interprétation des lois fait voir une violation de l’équité naturelle, est qualifiée de l’épithète {\itshape incivile}. C’est la dernière {\itshape secta temporum} de la jurisprudence romaine qui commença dès la république. Les préteurs trouvant que les caractères, que les mœurs et le gouvernement des Romains étaient déjà changés, furent obligés pour approprier les lois à ce changement d’adoucir la rigueur de la loi des douze tables, rigueur conforme aux mœurs des temps où elle avait été promulguée. Plus tard les Empereurs durent écarter tous les voiles dont les préteurs avaient enveloppé l’équité naturelle, et la laisser paraître tout à découvert, toute généreuse, comme il convenait à la civilisation où les peuples étaient parvenus.
\chapterclose


\chapteropen
\chapter[{Chapitre V. Autres preuves tirées des caractères propres aux aristocraties héroïques. — Garde des limites, des ordres politiques, des lois}]{Chapitre V. \\
Autres preuves tirées des caractères propres aux aristocraties héroïques. — Garde des limites, des ordres politiques, des lois}

\chaptercont
\noindent  La succession constante et non interrompue des révolutions politiques liées les unes aux autres par un si étroit enchaînement de causes et d’effets, doit nous forcer d’admettre comme vrais les principes de la Science nouvelle. Mais pour ne laisser aucun doute, nous y joignons l’explication de plusieurs autres phénomènes sociaux, dont on ne peut trouver la cause que dans la nature des républiques {\itshape héroïques}, telles que nous l’avons découverte. Les deux traits principaux qui caractérisent les aristocraties sont la {\itshape garde des limites}, et la {\itshape conservation} et distinction des {\itshape ordres politiques}.\par
\section[{§ I. De la garde et conservation des limites}]{§ I. {\itshape De la garde et conservation des limites}}
\noindent ({\itshape Voyez} Livre II, chap. V et VI, particulièrement § VI.)
\section[{§ II. De la conservation et distinction des ordres politiques}]{§ II. {\itshape De la conservation et distinction des ordres politiques}}
\noindent C’est l’esprit des gouvernements aristocratiques que les liaisons de parenté, les successions, et par  elles les richesses, et avec les richesses la puissance restent dans l’ordre des nobles. Voilà pourquoi vinrent si tard les lois {\itshape testamentaires}. Tacite nous apprend qu’il n’y avait point de testament chez les anciens Germains. À Sparte, le roi Agis voulant donner aux pères de famille le pouvoir de tester, fut étranglé par ordre des éphores, défenseurs du gouvernement aristocratique\footnote{Qu’on voie par là si les commentateurs de la loi des douze tables ont été bien avisés de placer dans la onzième le titre suivant, \emph{{\itshape auspicia incommunicata plebi sunto}}. Tous les droits civils, publics et privés, étaient une dépendance des auspices, et restaient le privilège des nobles. Les droits privés étaient les noces, la puissance paternelle, la suite, l’agitation, la gentilité, la succession légitime, le testament et la tutelle. Après avoir dans les premières tables établi les lois qui sont propres à une {\itshape démocratie} (particulièrement la loi {\itshape testamentaire}) en communiquant tous ces droits privés au peuple, ils rendent la forme du gouvernement entièrement {\itshape aristocratique} par un seul titre de la onzième table. Toutefois dans cette confusion, ils rencontrent par hasard une vérité, c’est que plusieurs coutumes anciennes des Romains reçurent le caractère de lois dans les deux dernières tables ; ce qui montre bien que Rome fut dans les premiers siècles une aristocratie. ({\itshape Vico.})}.\par
Lorsque les démocraties se formèrent, et ensuite les monarchies, les nobles et les plébéiens se mêlèrent au moyen des alliances et des successions par testament, ce qui fit que les richesses sortirent peu à peu des maisons nobles. Quant au droit des mariages solennels, nous avons déjà prouvé que le peuple romain demanda, non le droit de contracter des mariages avec les patriciens, mais des mariages semblables à ceux des patriciens, {\itshape connubia patrum}, et non {\itshape cum patribus}.\par
Si l’on considère ensuite les {\itshape successions légitimes}  dans cette disposition de la loi des douze tables par laquelle la succession du père de famille revient d’abord {\itshape aux siens, suis}, à leur défaut aux agnats, et s’il n’y en a point, à ses autres parents, la loi des douze tables semblera avoir été précisément une {\itshape loi salique} pour les Romains. La Germanie suivit la même règle dans les premiers temps, et l’on peut conjecturer la même chose des autres nations primitives du moyen âge. En dernier lieu, elle resta dans la France et dans la Savoie. Baldus favorise notre opinion en appelant ce droit de succession, \emph{{\itshape jus gentium Gallorum}} ; chez les Romains il peut très bien s’appeler {\itshape jus gentium Romanarum}, en ajoutant l’épithète {\itshape heroïcarum}, et avec plus de précision {\itshape jus Romanum}. Ce droit répondrait tout à fait au {\itshape jus quiritium Romanorum}, que nous avons prouvé avoir été le droit naturel commun à toutes les nations héroïques. Nous avons les plus fortes raisons de douter que dans les premiers siècles de Rome, les filles succédassent. Nulle probabilité que les pères de famille de ces temps eussent connu la tendresse paternelle. La loi des douze tables appelait un agnat, même au septième degré, à exclure le fils émancipé de la succession de son père. Les pères de famille avaient un droit souverain de vie et de mort sur leurs fils, et la propriété absolue de leurs {\itshape acquêts}. Ils les mariaient pour leur propre avantage, c’est-à-dire, pour faire entrer dans leurs maisons les femmes qu’ils en jugeaient dignes. Ce caractère historique des premiers pères de famille  nous est conservé par l’expression {\itshape spondere}, qui dans son propre sens, veut dire, promettre pour autrui ; de ce mot fut dérivé celui de {\itshape sponsalia}, les fiançailles. Ils considéraient de même les {\itshape adoptions}, comme des moyens de soutenir des familles près de s’éteindre, en y introduisant les rejetons généreux des familles étrangères. Ils regardaient l’émancipation comme une peine et un châtiment. Ils ne savaient ce que c’était que la {\itshape légitimation}, parce qu’ils ne prenaient pour concubines que des affranchies ou des étrangères, avec lesquelles on ne contractait point de mariages solennels dans les temps héroïques, de peur que les fils ne dégénérassent de la noblesse de leurs aïeux. Pour la cause la plus frivole les {\itshape testaments} étaient nuls, ou s’annulaient, ou se rompaient, ou n’atteignaient point leur effet, ({\itshape nulla, irrita, rupta, destituta}), afin que les successions légitimes reprissent leur cours. Tant ces patriciens, des premiers siècles, étaient passionnés pour la gloire de leur nom ; passion qui les enflammait encore pour la gloire du nom romain ! tout ce que nous venons de dire caractérise les mœurs des cités {\itshape aristocratiques} ou {\itshape héroïques}.\par
Une erreur digne de remarque est celle des commentateurs de la loi des douze tables : ils prétendent qu’avant que cette loi eût été portée d’Athènes à Rome, et qu’elle eût réglé les successions testamentaires et légitimes, les successions {\itshape ab intestat} rentraient dans la classe des choses {\itshape quæ sunt nullius}. Il n’en fut pas ainsi : la Providence empêcha  que le monde ne retombât dans la communauté des biens qui avait caractérisé la barbarie de premiers âges, en assurant par la forme même du gouvernement aristocratique la certitude et la distinction des propriétés. Les successions légitimes durent naturellement avoir lieu chez toutes les premières nations avant qu’elles connussent les testaments. Cette dernière institution appartient à la législation des démocraties, et surtout des monarchies. Le passage de Tacite que nous avons cité plus haut, nous porte à croire qu’il en fut de même chez tous les peuples barbares de l’antiquité, et par suite, à conjecturer que la {\itshape loi salique} qui était certainement en vigueur dans la Germanie, fut aussi observée généralement par les peuples du moyen âge.\par
Jugeant de l’antiquité par leur temps (axiome 2), les jurisconsultes romains du dernier âge ont cru que la loi des douze tables avait appelé les filles à hériter du père mort {\itshape intestat}, et les avait comprises sous le mot {\itshape sui}, en vertu de la règle d’après laquelle le genre masculin désigne aussi les femmes. Mais on a vu combien la jurisprudence héroïque s’attachait à la propriété des termes ; et si l’on doutait que {\itshape suus} ne désignât pas exclusivement le fils de famille, on en trouverait une preuve invincible dans la formule de l’{\itshape institution des posthumes}, introduite tant de siècles après par {\itshape Gallus Aquilius} : \emph{{\itshape si quis natus natave erit}}. Il craignait que dans le mot {\itshape natus} on ne comprit point la fille posthume. C’est pour avoir ignoré ceci que Justinien prétend  dans les institutes que la loi des douze tables aurait désigné par le seul mot {\itshape adgnatus} les agnats des deux sexes, et qu’ensuite la jurisprudence {\itshape moyenne} aurait ajouté à la rigueur de la loi en la restreignant aux sœurs consanguines. Il dut arriver tout le contraire. Cette jurisprudence dut étendre d’abord le sens de {\itshape suus} aux filles, et plus tard le sens d’{\itshape adgnatus} aux sœurs consanguines. Elle fut appelée {\itshape moyenne}, précisément pour avoir ainsi adouci la rigueur de la loi des douze tables.\par
Lorsque l’Empire passa des nobles au peuple, les plébéiens qui faisaient consister toutes leurs forces, toutes leurs richesses, toute leur puissance dans la multitude de leurs fils, commencèrent à sentir la tendresse paternelle. Ce sentiment avait dû rester inconnu aux plébéiens des cités héroïques qui n’engendraient des fils que pour les voir esclaves des nobles. Autant la multitude des plébéiens avait été dangereuse aux aristocraties, aux gouvernements {\itshape du petit nombre}, autant elle était capable d’agrandir les démocraties et les monarchies. De là tant de faveurs accordées aux femmes par les lois impériales pour compenser les dangers et les douleurs de l’enfantement. Dès le temps de la république, les préteurs commencèrent à faire attention aux droits du sang, et à leur prêter secours au moyen des {\itshape possessions de biens}. Ils commencèrent à remédier aux {\itshape vices}, aux {\itshape défauts} des testaments, afin de favoriser la division des richesses qui font toute l’ambition du peuple.\par
 Les Empereurs allèrent bien plus loin. Comme l’éclat de la noblesse leur faisait ombrage, ils se montrèrent favorables aux {\itshape droits de la nature humaine}, commune aux nobles et aux plébéiens. Auguste commença à protéger les fidéicommis, qui auparavant ne passaient aux personnes incapables d’hériter que grâce à la délicatesse des héritiers grevés ; il fit tant pour les fidéicommis, qu’avant sa mort ils donnèrent le droit de contraindre les héritiers à les exécuter. Puis vinrent tant de sénatus-consultes, par lesquels les cognats furent mis sur la ligne des agnats. Enfin Justinien ôta la différence des legs et des fidéicommis, confondit {\itshape les quartes Falcidianienne} et {\itshape Trebellianique}, mit peu de distinction entre les testaments et les codicilles, et dans les successions {\itshape ab intestat} égala les agnats et les cognats en tout et pour tout. Ainsi les lois romaines de l’Empire se montrèrent si attentives à favoriser les {\itshape dernières volontés}, que, tandis qu’autrefois le plus léger défaut les annulait, elles doivent aujourd’hui être toujours interprétées de manière à les rendre valables s’il est possible.\par
Les démocraties sont bienveillantes pour les fils, les monarchies veulent que les pères soient occupés par l’amour de leurs enfants ; aussi les progrès de l’{\itshape humanité} ayant aboli le droit barbare des premiers pères de familles sur la personne de leurs fils, les Empereurs voulurent abolir aussi le droit qu’ils conservaient sur leurs acquêts, et introduisirent d’abord le {\itshape peculium castrense}, pour inviter les fils  de famille au service militaire ; puis ils en étendirent les avantages au {\itshape peculium quasi castrense}, pour les inviter à entrer dans le service du palais ; enfin pour contenter les fils qui n’étaient ni soldats ni lettrés, ils introduisirent le {\itshape peculium adventitium}. Ils ôtèrent les effets de la puissance paternelle à l’{\itshape adoption} qui n’est pas faite par un des ascendants de l’adopté. Ils approuvèrent universellement les {\itshape adrogations}, difficiles en ce qu’un citoyen, de père de famille, devient dépendant de celui dans la famille duquel il passe. Ils regardèrent les {\itshape émancipations} comme avantageuses ; donnèrent aux {\itshape légitimations} par mariage subséquent tout l’effet du mariage solennel. Enfin, comme le terme d’{\itshape imperium paternum} semblait diminuer la majesté impériale, ils introduisirent le mot de {\itshape puissance} paternelle, {\itshape patria potestas}\footnote{\noindent En cela l’habileté d’Auguste leur avait donné l’exemple. De crainte d’éveiller la jalousie du peuple en lui enlevant le privilège nominal de l’empire, {\itshape imperium}, il prit le titre de la puissance tribunitienne, {\itshape potestas tribunitia}, se déclarant ainsi le protecteur de la liberté romaine. Le tribunat avait été simplement une puissance de fait ; les tribuns n’eurent jamais dans la république ce qu’on appelait {\itshape imperium}. Sous le même Auguste, un tribun du peuple ayant ordonné à Labéon de comparaître devant lui, ce jurisconsulte célèbre, le chef d’une des deux écoles de la jurisprudence romaine, refusa d’obéir ; et il était dans son droit, puisque les tribuns n’avaient point l’{\itshape imperium}.\par
Une observation a échappé aux grammairiens, aux politiques et aux jurisconsultes, c’est que dans la lutte des plébéiens contre les patriciens pour obtenir le consulat, ces derniers voulant satisfaire le peuple sans établir de précédents relativement au partage de l’{\itshape empire}, créèrent des tribuns militaires en partie plébéiens, {\itshape cum consulari potestate}, et non point cum {\scshape imperio} {\itshape consulari}. Aussi tout le système de la république romaine fut compris dans cette triple formule : {\itshape Senatus autoritas, populi} {\scshape imperium, plebis potestas}. {\itshape Imperium} s’entend des grandes magistratures, du consulat, de la préture qui donnaient le droit de condamner à mort ; {\itshape potestas}, des magistratures inférieures, telles que l’édilité, et {\itshape modicâ coercitione continetur}. ({\itshape Vico.})
}.\par
 En dernier lieu, la bienveillance des Empereurs détendant à toute l’humanité, ils commencèrent à favoriser les esclaves. Ils réprimèrent la cruauté des maîtres. Ils étendirent les effets de l’affranchissement, en même temps qu’ils en diminuaient les formalités. Le droit de cité ne s’était donné dans les temps anciens qu’à d’illustres étrangers qui avaient bien mérité du peuple romain ; ils l’accordèrent à quiconque était né à Rome d’un père esclave, mais d’une mère libre, ne le fût-elle que par affranchissement. La loi reconnaissant libre quiconque {\itshape naissait} dans la cité ; sous de telles circonstances, le {\itshape droit naturel} changea de dénomination ; dans les aristocraties, il était appelé {\scshape droit des gens}, dans le sens du latin {\itshape gentes}, maisons nobles [pour lesquelles ce droit était une sorte de propriété] ; mais lorsque s’établirent les démocraties, où les nations entières sont souveraines, et ensuite les monarchies, où les monarques représentent les nations entières dont leurs sujets sont les membres, il fut nommé {\scshape droit naturel des nations}.
\section[{§ III. De la conservation des lois}]{§ III. {\itshape De la conservation des lois}}
\noindent La conservation {\itshape des ordres} entraîne avec elle celle des magistratures et des sacerdoces, et par suite celle des lois et de la jurisprudence. Voilà  pourquoi nous lisons dans l’histoire romaine que tant que le gouvernement de Rome fut aristocratique, le droit des mariages solennels, le consulat, le sacerdoce ne sortaient point de l’ordre des sénateurs, dans lequel n’entraient que les nobles ; et que la science des lois restait {\itshape sacrée} ou {\itshape secrète} (car c’est la même chose) dans le collège des pontifes, composé des seuls nobles chez toutes les nations {\itshape héroïques}. Cet état dura un siècle encore après la loi des douze tables, au rapport du jurisconsulte Pomponius. La connaissance des lois fut le dernier privilège que les patriciens cédèrent aux plébéiens.\par
Dans l’âge {\itshape divin}, les lois étaient gardées avec scrupule et sévérité. L’observation des {\itshape lois divines} a continué de s’appeler {\itshape religion}. Ces lois doivent être observées, en suivant certaines {\itshape formules inaltérables de paroles consacrées et de cérémonies solennelles}. — Cette observation sévère {\itshape des lois} est l’essence de l’aristocratie. Voulons-nous savoir pourquoi Athènes et presque toutes les cités de la Grèce passèrent si promptement à la démocratie ? Le mot connu des Spartiates nous en apprend la cause : {\itshape les Athéniens conservent par écrit des lois innombrables ; les lois de Sparte sont peu nombreuses, mais elles s’observent}. — Tant que le gouvernement de Rome fut aristocratique, les Romains se montrèrent observateurs rigides de la loi des douze tables, en sorte que Tacite l’appelle \emph{{\itshape finis omnis æqui juris}}. En effet, après celles qui furent jugées suffisantes  pour assurer la liberté et l’égalité civile\footnote{Ces lois doivent avoir été postérieures aux décemvirs, auxquels les anciens peuples les ont rapportées, comme au type idéal du législateur. ({\itshape Vico.})}, les lois consulaires relatives au droit privé furent peu nombreuses, si même il en exista. Tite-Live dit que la loi des douze tables fut la source de toute la jurisprudence. — Lorsque le gouvernement devint démocratique, le petit peuple de Rome, comme celui d’Athènes, ne cessait de faire des lois d’intérêt privé, incapable qu’il était de s’élever à des idées générales. Sylla, le chef du parti des nobles, après sa victoire sur Marius, chef du parti du peuple, remédia un peu au désordre par l’établissement des {\itshape quæstiones perpetuæ} ; mais dès qu’il eut abdiqué la dictature, les lois d’intérêt privé recommencèrent à se multiplier comme auparavant (Tacite). La multitude des lois est, comme le remarquent les politiques, la route la plus prompte qui conduise les états à la monarchie ; aussi Auguste pour l’établir en fit un grand nombre ; et les princes qui suivirent, employèrent surtout le sénat à faire des sénatus-consultes d’intérêt privé. Néanmoins dans le temps même où le gouvernement romain était déjà devenu démocratique, les {\itshape formules d’actions} étaient suivies si rigoureusement qu’il fallut toute l’éloquence de Crassus (que Cicéron appelait le Démosthène\footnote{Orthographié « Démosthènes » [NdE].} romain), pour que la {\itshape substitution pupillaire expresse} fût regardée comme contenant la {\itshape vulgaire} qui n’était pas exprimée. Il fallut tout le talent de  Cicéron pour empêcher Sextus Ebutius de garder la terre de Cecina, parce qu’il manquait une lettre à la formule. Mais avec le temps les choses changèrent au point que Constantin abolit entièrement les formules, et qu’il fut reconnu que {\itshape tout motif particulier d’équité prévaut sur la loi}. Tant les esprits sont disposés à reconnaître docilement l’équité naturelle sous les gouvernements {\itshape humains} ! Ainsi tandis que sous l’aristocratie, l’on avait observé si rigoureusement le {\itshape privilegia ne irroganto}, de la loi des douze tables, on fit sous la démocratie une foule de lois d’intérêt privé, et sous la monarchie les princes ne cessèrent d’accorder des {\itshape privilèges}. Or rien de plus conforme à l’équité naturelle que les {\itshape privilèges} qui sont mérités. On peut même dire avec vérité que toutes les exceptions faites aux lois chez les modernes, sont des {\itshape privilèges} voulus par le mérite particulier des faits, qui les sort de la disposition commune.\par
Peut-être est-ce pour cette raison que les nations barbares du moyen âge repoussèrent les lois romaines. En France on était puni sévèrement, en Espagne mis à mort, lorsqu’on osait les alléguer. Ce qui est sûr, c’est qu’en Italie, les nobles auraient rougi de suivre les rois romaines, et se faisaient honneur de n’être soumis qu’à celles des Lombards ; les gens du peuple au contraire qui ne quittent point facilement leurs usages, observaient plusieurs lois romaines qui avaient conservé force de coutumes. C’est ce qui explique comment furent  en quelque sorte ensevelies dans l’oubli chez les Latins les lois de Justinien, chez les Grecs les Basiliques. Mais lorsqu’ensuite se formèrent les monarchies modernes, lorsque reparut dans plusieurs cités la liberté populaire, le droit romain compris dans les livres de Justinien fut reçu généralement, en sorte que Grotius affirme que c’est {\itshape un droit naturel des gens} pour les Européens.\par
Admirons la sagesse et la gravité romaines, en voyant au milieu de ces révolutions politiques les préteurs et les jurisconsultes employer tous leurs efforts pour que les termes de la loi des douze tables, ne perdent que lentement et le moins possible le sens qui leur était propre. Ainsi en changeant de forme de gouvernement, Rome eut l’avantage de s’appuyer toujours sur les mêmes principes, lesquels n’étaient autres que ceux de la société humaine. Ce qui donna aux Romains la plus sage de toutes les jurisprudences, est aussi ce qui fit de leur Empire le plus vaste, le plus durable du monde. Voilà la principale cause de la grandeur romaine que Polybe et Machiavel expliquent d’une manière trop générale, l’un par l’esprit religieux des nobles, l’autre par la magnanimité des plébéiens, et que Plutarque attribue par envie à la fortune de Rome. La noble réponse du Tasso à l’ouvrage de Plutarque le réfute moins directement que nous ne le faisons ici.
\chapterclose


\chapteropen
\chapter[{Chapitre VI. Autres preuves tirées de la manière dont chaque forme de la société se combine avec la précédente. — Réfutation de Bodin}]{Chapitre VI. \\
Autres preuves tirées de la manière dont chaque forme de la société se combine avec la précédente. — Réfutation de Bodin}

\chaptercont
\section[{§ I}]{§ I}
\noindent  Nous avons montré dans ce Livre jusqu’à l’évidence que dans toute leur vie politique les nations passent par trois sortes d’états civils (aristocratie, démocratie, monarchie), dont l’origine commune est le gouvernement {\itshape divin}. \emph{{\itshape Une quatrième forme}, dit Tacite, {\itshape  soit distincte, soit mêlée des trois, est plus désirable que possible, et si elle se rencontre, elle n’est point durable.}} Mais pour ne point laisser de doute sur cette succession naturelle, nous examinerons comment chaque état se combine avec le gouvernement de l’état précédent ; mélange fondé sur l’axiome : lorsque les hommes changent, ils conservent quelque temps l’impression de leurs premières habitudes.\par
Les pères de familles desquels devaient sortir les nations païennes, ayant passé de la vie {\itshape bestiale} à la vie {\itshape humaine}, gardèrent dans l’{\itshape état de nature},  où il n’existait encore d’autre gouvernement que celui {\itshape des dieux}, leur caractère originaire de férocité et de barbarie ; et conservèrent à la formation des {\itshape premières aristocraties} le souverain empire qu’ils avaient eu sur leurs femmes et leurs enfants dans l’état de nature. Tous égaux, trop orgueilleux pour céder l’un à l’autre, ils ne se soumirent qu’à l’empire souverain des corps aristocratiques dont ils étaient membres ; leur {\itshape domaine} privé, jusque-là {\itshape éminent}, forma en se réunissant le {\itshape domaine} public également {\itshape éminent} du sénat qui gouvernait, de même que la réunion de leurs {\itshape souverainetés} privées composa la {\itshape souveraineté} publique des ordres auxquels ils appartenaient. Les cités furent donc dans l’origine des {\itshape aristocraties mêlées à la monarchie domestique des pères de famille}. Autrement, il est impossible de comprendre comment la société civile sortit de la société de la famille.\par
Tant que les pères conservèrent le domaine {\itshape éminent} dans le sein de leurs compagnies souveraines, tant que les plébéiens ne leur eurent pas arraché le droit d’acquérir des propriétés, de contracter des mariages solennels, d’aspirer aux magistratures, au sacerdoce, enfin de connaître les lois (ce qui était encore un privilège du sacerdoce), {\itshape les gouvernements furent aristocratiques}. Mais lorsque les plébéiens des cités héroïques devinrent assez nombreux, assez aguerris pour effrayer les pères (qui dans une {\itshape oligarchie} devaient être peu nombreux, comme le mot l’indique), et que, forts de leur  nombre, ils commencèrent à faire des lois sans l’autorisation du sénat, les républiques devinrent {\itshape démocratiques}. Aucun état n’aurait pu subsister avec deux {\itshape pouvoirs législatifs} souverains, sans se diviser en deux états. Dans cette révolution, l’autorité de {\itshape domaine} devint naturellement autorité de {\itshape tutelle} ; le peuple souverain, faible encore sous le rapport de la sagesse politique se confiait à son sénat, comme un roi dans sa minorité à un tuteur. Ainsi {\itshape les états populaires furent gouvernés par un corps aristocratique}.\par
Enfin lorsque les puissants dirigèrent le conseil public dans l’intérêt de leur puissance, lorsque le peuple corrompu par l’intérêt privé consentit à assujettir la liberté publique à l’ambition des puissants, et que du choc des partis résultèrent les guerres civiles, {\itshape la monarchie s’éleva sur les ruines de la démocratie}.
\section[{§ II. D’une loi royale, éternelle et fondée en nature, en vertu de laquelle les nations vont se reposer dans la monarchie}]{§ II. {\itshape D’une loi royale, éternelle et fondée en nature, en vertu de laquelle les nations vont se reposer dans la monarchie}}
\noindent Cette loi a échappé aux interprètes modernes du droit romain. Ils étaient préoccupés par cette fable de la {\itshape loi royale} de Tribonien, qu’il attribue à Ulpien dans les Pandectes, et dont il s’avoue l’auteur dans les Institutes. Mais les jurisconsultes romains avaient bien compris la {\itshape loi royale} dont nous parlons. Pomponius dans son histoire abrégée du droit romain caractérise cette loi par un mot plein  de sens, \emph{{\itshape rebus ipsis dictantibus regna condita}}. — Voici la formule éternelle dans laquelle l’a conçue la nature : lorsque les citoyens des démocraties ne considèrent plus que leurs intérêts particuliers, et que, pour atteindre ce but, ils tournent les forces nationales à la ruine de leur patrie, alors il s’élève un seul homme, comme Auguste chez les Romains, qui se rendant maître par la force des armes, prend pour lui tous les soins publics, et ne laisse aux sujets que le soin de leurs affaires particulières. Cette révolution fait le salut des peuples qui autrement marcheraient à leur destruction. — Cette vérité semble admise par les docteurs du droit moderne, lorsqu’ils disent : \emph{{\itshape universitates sub rege habentur loco privatorum}} ; c’est qu’en effet la plus grande partie des citoyens ne s’occupe plus du bien public. Tacite nous montre très bien dans ses annales le progrès de cette funeste indifférence ; lorsqu’Auguste fut près de mourir, quelques-uns discouraient vainement sur le bonheur de la liberté, \emph{{\itshape pauci bona libertatis incassum disserere}} ; Tibère arrive au pouvoir, et tous, les yeux fixés sur le prince, attendent pour obéir, \emph{{\itshape omnes principis jussa adspectare}}. Sous les trois Césars qui suivent, les Romains d’abord indifférents pour la république, finissent par ignorer même ses intérêts, comme s’ils y étaient étrangers, \emph{{\itshape incuriâ et ignorantiâ reipublicæ, tanquam alienæ}}. Lorsque les citoyens sont ainsi devenus étrangers à leur propre pays, il est nécessaire que les monarques les dirigent et les représentent.  Or comme dans les républiques, un puissant ne se fraie le chemin à la monarchie, qu’en se faisant un parti, il est naturel qu’{\itshape un monarque gouverne d’une manière populaire}. D’abord il veut que tous ses sujets soient égaux, et il humilie les puissants de façon que les petits n’aient rien à craindre de leur oppression. Ensuite il a intérêt à ce que la multitude n’ait point à se plaindre en ce qui touche la subsistance et la liberté naturelle. Enfin il accorde des privilèges ou à des ordres entiers (ce qu’on appelle des {\itshape privilèges de liberté}), ou à des individus d’un mérite extraordinaire qu’il tire de la foule pour les élever aux honneurs civils. Ces privilèges sont des {\itshape lois d’intérêt privé}, dictées par l’équité naturelle. Aussi la monarchie est-elle le gouvernement le plus conforme à la nature humaine, aux époques où la raison est le plus développée.
\section[{§ III. Réfutation des principes de la politique de Bodin}]{§ III. {\itshape Réfutation des principes de la politique de Bodin}}
\noindent Bodin suppose que les gouvernements, d’abord {\itshape monarchiques}, ont passé par la {\itshape tyrannie} à la {\itshape démocratie} et enfin à l’{\itshape aristocratie}. Quoique nous lui ayons assez répondu indirectement, nous voulons, {\itshape ad exuberantiam}, le réfuter par l’{\itshape impossible} et par l’{\itshape absurde}.\par
Il ne disconvient point que les familles n’aient été les éléments dont se composèrent les cités. Mais d’un autre côté il partage le préjugé vulgaire selon lequel les familles auraient été composées seulement  des parents et des enfants [et non en outre des serviteurs, {\itshape famuli}]. Maintenant nous lui demandons comment la {\itshape monarchie} put sortir d’un tel {\itshape état de famille}. Deux moyens se présentent seuls, la force et la ruse. La force ? Comment un père de famille pouvait-il soumettre les autres ? On conçoit que dans les démocraties les citoyens aient consacré à la patrie et leur personne et leur famille dont elle assurait la conservation, et que par là ils aient été apprivoisés à la monarchie. Mais ne doit-on pas supposer que, dans la fierté originaire d’une liberté farouche, les pères de famille auraient plutôt péri tous avec les leurs, que de supporter l’inégalité ? Quant à la ruse, elle est employée par les démagogues, lorsqu’ils promettent à la multitude la {\itshape liberté}, la {\itshape puissance} ou la {\itshape richesse}. Aurait-on promis la {\itshape liberté} aux premiers pères de famille ? ils étaient tous non-seulement {\itshape libres}, mais {\itshape souverains} dans leur domestique… La {\itshape puissance} ? à des solitaires, qui, tels que le Polyphème d’Homère, se tenaient dans leurs cavernes avec leur famille, sans se mêler des affaires d’autrui ? La {\itshape richesse} ? on ne savait ce que c’était que richesses, dans un tel état de simplicité. — La difficulté devient plus grande encore, lorsqu’on songe que dans la haute antiquité il n’y avait point de {\itshape forteresse}, et que les cités {\itshape héroïques} formées par la réunion des familles n’eurent point de murs pendant longtemps, comme nous le certifie Thucydide\footnote{La jalousie aristocratique empêchait qu’on en élevât. On sait que Valérius Publicola ne se justifia du reproche d’avoir construit une maison dans un lieu élevé, qu’en la rasant en une nuit. — Les nations les plus belliqueuses et les plus farouches sont celles qui conservèrent le plus longtemps l’usage de ne point fortifier les villes. En Allemagne, ce fut, dit-on, Henri l’Oiseleur qui le premier réunit dans des cités le peuple dispersé jusque-là dans les villages, et qui entoura les villes de murs. — Qu’on dise après cela que les premiers fondateurs des villes furent ceux qui marquèrent par un sillon le contour des murs ; qu’on juge si les étymologistes ont raison de faire venir le mot porte, {\itshape a portando aratro}, de la charrue qu’on portait pour interrompre le sillon à l’endroit où devaient être les portes. ({\itshape Vico.})}. Mais elle est vraiment insurmontable,  si l’on considère avec Bodin les familles comme composées seulement des fils. Dans cette hypothèse, qu’on explique l’établissement de la monarchie par la force ou par la ruse, les fils auraient été les instruments d’une ambition étrangère, et auraient trahi ou mis à mort leurs propres pères ; en sorte que ces gouvernements eussent été moins des monarchies, que des tyrannies impies et parricides.\par
Il faut donc que Bodin, et tous les politiques avec lui, reconnaissent les {\itshape monarchies domestiques} dont nous avons prouvé l’existence dans l’état de famille, et conviennent que les familles se composèrent non-seulement des fils, mais encore des serviteurs ({\itshape famuli}), dont la condition était une image imparfaite de celle des esclaves, qui se firent dans les guerres après la fondation des cités. C’est dans ce sens que l’on peut dire, comme lui, {\itshape que les républiques se sont formées d’hommes libres et d’un caractère sévère}. Les premiers citoyens de Bodin ne peuvent présenter ce caractère.\par
Si, comme il le prétend, l’aristocratie est la dernière  forme par laquelle passent les gouvernements, comment se fait-il qu’il ne nous reste du moyen âge qu’un si petit nombre de républiques aristocratiques ? On compte en Italie Venise, Gênes et Lucques, Raguse en Dalmatie, et Nuremberg en Allemagne. Les autres républiques sont des états populaires avec un gouvernement aristocratique.\par
Le même Bodin qui veut conformément à son système, que la royauté romaine ait été monarchique, et qu’à l’expulsion des tyrans la liberté populaire ait été établie à Rome, ne voyant pas les faits répondre à ses principes, dit d’abord que Rome fut un état populaire gouverné par une aristocratie ; plus loin, vaincu par la force de la vérité, il avoue, sans chercher à pallier son inconséquence, que la constitution et le gouvernement de Rome étaient également aristocratiques. L’erreur est venue de ce qu’on n’avait pas bien défini les trois mots {\itshape peuple, royauté, liberté}\footnote{Voyez livre II, pag. 214.}.
\chapterclose


\chapteropen
\chapter[{Chapitre VII. Dernières preuves à l’appui de nos principes sur la marche des sociétés}]{Chapitre VII. \\
Dernières preuves à l’appui de nos principes sur la marche des sociétés}

\chaptercont
\section[{§ I}]{§ I}
\noindent  1. Dans l’{\itshape état de famille} les peines furent atroces. C’est l’âge des Cyclopes et du Polyphème d’Homère. C’est alors qu’Apollon écorche tout vivant le satyre Marsyas. — La même barbarie continua dans les républiques aristocratiques ou {\itshape héroïques}. Au moyen âge on disait {\itshape peine ordinaire} pour peine de mort. Les lois de Sparte sont accusées de cruauté par Platon et par Aristote. À Rome, le vainqueur des Curiaces fut condamné à être battu de verges et attaché à l’arbre de malheur ({\itshape arbori infelici}). Métius Suffetius, roi d’Albe, fut écartelé, Romulus lui-même mis en pièces par les sénateurs. La loi des douze tables condamne à être brûlé vif celui qui met le feu à la moisson de son voisin ; elle ordonne que le faux témoin soit précipité de la Roche Tarpéienne ; enfin que le débiteur insolvable soit mis en quartiers. — Les peines s’adoucissent sous la {\itshape démocratie}. La faiblesse même de la multitude la  rend plus portée à la compassion. Enfin dans les {\itshape monarchies}, les princes s’honorent du titre de {\itshape cléments}.\par
2. Dans les guerres barbares des temps {\itshape héroïques}, les cités vaincues étaient ruinées, et leurs habitants, réduits à un état de servage, étaient dispersés par troupeaux dans les campagnes pour les cultiver au profit du peuple vainqueur. Les {\itshape démocraties} plus généreuses n’ôtèrent aux vaincus que les droits politiques, et leur laissèrent le libre usage du droit naturel (\emph{{\itshape jus naturale gentium humanarum}}, Ulpien). Ainsi les conquêtes s’étendant, tous les droits qui furent désignés plus tard comme {\itshape rationes propriæ civium Romanorum}, devinrent le privilège des citoyens romains (tels que le mariage, la puissance paternelle, le domaine {\itshape quiritaire}, l’émancipation, etc.) Les nations vaincues avaient aussi possédé ces droits au temps de leur indépendance. — Enfin vient la {\itshape monarchie}, et Antonin veut faire une seule Rome de tout le monde romain. Tel est le vœu des plus grands monarques\footnote{Alexandre le Grand disait que le monde n’était pour lui qu’une cité, dont la citadelle était sa phalange. ({\itshape Vico.})}. Le droit naturel des nations, appliqué et autorisé dans les provinces par les préteurs romains, finit, avec le temps, par gouverner Rome elle-même. Ainsi fut aboli le droit {\itshape héroïque} que les Romains avaient eu sur les provinces ; les monarques veulent que tous les sujets soient égaux sous leurs lois. La jurisprudence romaine, qui dans les temps {\itshape héroïques} n’avait eu pour base que la loi  des douze tables, commença dès le temps de Cicéron\footnote{{\itshape De legibus}.}, à suivre dans la pratique l’édit du préteur. Enfin, depuis Adrien, elle se régla sur l’{\itshape édit perpétuel}, composé presqu’entièrement des {\itshape édits provinciaux} par Salvius Julianus.\par
3. Les territoires bornés dans lesquels se resserrent les {\itshape aristocraties} pour la facilité du gouvernement, sont étendus par l’esprit conquérant de la {\itshape démocratie} ; puis viennent les monarchies, qui sont plus belles et plus magnifiques à proportion de leur grandeur.\par
4. Du gouvernement soupçonneux de l’{\itshape aristocratie} les peuples passent aux orages de la {\itshape démocratie}, pour trouver le repos sous la {\itshape monarchie}.\par
5. Ils partent de l’{\itshape unité} de la monarchie domestique, pour traverser les gouvernements du plus {\itshape petit nombre}, du {\itshape plus grand nombre}, et {\itshape de tous}, et retrouver l’{\itshape unité} dans la monarchie civile.
\section[{§ II. Corollaire. Que l’ancien droit romain à son premier âge fut un poème sérieux, et l’ancienne jurisprudence une poésie sévère, dans laquelle on trouve la première ébauche de la métaphysique légale. — Comment chez les Grecs la philosophie sortit de la législation}]{§ II. {\itshape Corollaire. Que l’ancien droit romain à son premier âge fut un poème sérieux, et l’ancienne jurisprudence une poésie sévère, dans laquelle on trouve la première ébauche de la métaphysique légale. — Comment chez les Grecs la philosophie sortit de la législation}}
\noindent Il y a bien d’autres effets importants, surtout dans  la jurisprudence romaine, dont on ne peut trouver la cause que dans nos principes, et surtout dans le 9\textsuperscript{e} axiome [lorsque les hommes ne peuvent atteindre le {\itshape vrai}, ils s’en tiennent au {\itshape certain}].\par
Ainsi les {\itshape mancipations} ({\itshape capere manu}) se firent d’abord {\itshape verâ manu}, c’est-à-dire, {\itshape avec une force réelle}. La {\itshape force} est un mot abstrait, la {\itshape main} est chose sensible, et chez toutes les nations elle a signifié la {\itshape puissance}\footnote{De là les χειροθεσίαι et les χειροτονίαι des Grecs : le premier mot désigne l’{\itshape imposition des mains} sur la tête du magistrat qu’on allait élire ; le second les acclamations des électeurs qui {\itshape élevaient les mains}. ({\itshape Vico.})}. Cette {\itshape mancipation} réelle n’est autre que l’{\itshape occupation}, source naturelle de tous les {\itshape domaines}. Les Romains continuèrent d’employer ce mot pour l’{\itshape occupation} d’une chose par la guerre ; les esclaves furent appelés {\itshape mancipia}, le butin et les conquêtes furent pour les Romains {\itshape res mancipi}, tandis qu’elles devenaient pour les vaincus {\itshape res nec mancipi}. Qu’on voie donc combien il est raisonnable de croire que la {\itshape mancipation} prit naissance dans les murs de la seule ville de Rome, comme un mode d’acquérir le {\itshape domaine civil} usité dans les affaires privées des citoyens !\par
Il en fut de même de la véritable {\itshape usucapion}, autre manière d’acquérir le {\itshape domaine}, mot qui répond à {\itshape capio cum vero usu}, en prenant {\itshape usus} pour possession. D’abord on prit possession en couvrant de son corps la chose possédée ; {\itshape possessio} fut dit pour {\itshape porro sessio}. — Dans les républiques {\itshape héroïques} qui selon Aristote  \emph{{\itshape n’avaient point de lois pour redresser les torts particuliers}}, nous avons vu que les {\itshape revendications} s’exerçaient {\itshape par une force}, par une violence {\itshape véritable}. Ce furent là les premiers duels, ou guerres privées. Les {\itshape actions personnelles} ({\itshape condictiones}) durent être les {\itshape représailles privées}, qui au moyen âge durèrent jusqu’au temps de Barthole.\par
Les mœurs devenant moins farouches avec le temps, les violences particulières commençant à être réprimées par les lois judiciaires, enfin la réunion des forces particulières ayant formé la force publique, les premiers peuples, par un effet de l’instinct poétique que leur avait donné la nature, durent imiter cette {\itshape force réelle} par laquelle ils avaient auparavant défendu leurs droits. Au moyen d’une fiction de ce genre, la {\itshape mancipation} naturelle devint la {\itshape tradition civile} solennelle, qui se représentait en simulant un nœud. Ils employèrent cette fiction dans les {\itshape acta legitima} qui consacraient tous leurs rapports légaux, et qui devaient être les cérémonies solennelles des peuples avant l’usage des langues vulgaires. Puis lorsqu’il y eut un langage articulé, les contractants s’assurèrent de la volonté l’un de l’autre en joignant au nœud des paroles solennelles qui exprimassent d’une manière certaine et précise les stipulations du contrat.\par
Par suite, les conditions ({\itshape leges}) auxquelles se rendaient les villes, étaient exprimées par des formules analogues, qui se sont appelées {\itshape paces} (de {\itshape pacio}) mot qui répond à celui de {\itshape pactum}. Il en est resté un vestige  remarquable dans la formule du traité par lequel se rendit Collatie. Tel que Tite-Live le rapporte, c’est une véritable stipulation ({\itshape contratto recettizio}) fait avec les interrogations et les réponses solennelles ; aussi ceux qui se rendaient étaient appelés, dans toute la propriété du mot, {\itshape recepti} ; {\itshape et ego recipio}, dit le héraut romain aux députés de Collatie. Tant il est peu exact de dire que dans les temps {\itshape héroïques} la {\itshape stipulation} fut particulière aux citoyens romains ! On jugera aussi si l’un a eu raison de croire jusqu’ici que Tarquin l’Ancien prétendit donner aux nations dans la formule dont nous venons de parler, un modèle pour les cas semblables. — Ainsi le {\itshape droit des gens héroïques} du Latium resta gravé dans ce titre de la loi des douze tables : \emph{{\scshape si quis nexum faciet mancipiumque uti lingua nuncupassit ita jus esto}}. C’est la grande source de tout l’ancien droit romain, et ceux qui ont rapproché les lois athéniennes de celle des douze tables, conviennent que ce titre n’a pu être importé d’Athènes à Rome.\par
L’{\itshape usucapion} fut d’abord une {\itshape prise de possession} au moyen du corps, et fut censée continuer par la seule intention. En même temps on porta la même fiction de l’emploi de la force dans les {\itshape revendications}, et les {\itshape représailles héroïques} se transformèrent en {\itshape actions personnelles} ; on conserva l’usage de les dénoncer solennellement aux débiteurs. Il était impossible que l’enfance de l’humanité suivît une marche différente ; on a remarqué dans un axiome que les enfants ont au plus haut degré la faculté  d’imiter {\itshape le vrai} dans les choses qui ne sont point au-dessus de leur portée ; c’est en quoi consiste la poésie, laquelle n’est qu’imitation.\par
Par un effet du même esprit, toutes les {\itshape personnes} qui paraissaient au forum, étaient distinguées par des {\itshape masques} ou {\itshape emblèmes} particuliers ({\itshape personæ}). Ces emblèmes propres aux familles étaient, si je puis le dire, des {\itshape noms réels}, antérieurs à l’usage des langues vulgaires. Le signe distinctif du père de famille désignait collectivement tous ses enfants, tous ses esclaves. Aux exemples déjà cités (page 181), joignons les prodigieux exploits des paladins français, et surtout de Roland, qui sont ceux d’une armée plutôt que ceux d’un individu ; ces paladins étaient des souverains, comme le sont encore les {\itshape palatins} d’Allemagne. Ceci dérive des principes de notre poétique. Les fondateurs du droit romain ne pouvant s’élever encore par l’abstraction aux idées générales, créèrent pour y suppléer des caractères poétiques, par lesquels ils désignaient les genres. De même que les poètes guidés par leur art portèrent les personnages et les masques sur le théâtre, les fondateurs du droit, conduits par la nature, avaient dans des temps plus anciens, porté sur le forum les {\itshape personnes} ({\itshape personas}) et les emblèmes\footnote{La quantité prouve que {\itshape persona} ne vient point, comme on le prétend, de {\itshape personare}. ({\itshape Vico.})}. — Incapables de se créer par l’intelligence des {\itshape formes abstraites}, ils en imaginèrent de {\itshape corporelles}, et les supposèrent {\itshape animées} d’après leur propre nature. Ils  réalisèrent dans leur imagination l’hérédité, {\itshape hereditas}, comme souveraine des héritages, et ils la placèrent tout entière dans chacun des effets dont ils se composaient ; ainsi quand ils présentaient aux juges une motte de terre dans l’acte de la {\itshape revendication}, ils disaient {\itshape hunc fundum}, etc. Ainsi ils {\itshape sentirent} imparfaitement, s’ils ne purent le {\itshape comprendre}, que {\itshape les droits sont indivisibles}. Les hommes étant alors naturellement poètes, la première jurisprudence fut toute {\itshape poétique} ; par une suite de fictions, elle supposait {\itshape que ce qui n’était pas fait l’était déjà}, que ce {\itshape qui était né, était à naître}, que le {\itshape mort était vivant}, et {\itshape vice versa}. Elle introduisait une foule de déguisements, de voiles qui ne couvraient rien, {\itshape jura imaginaria} ; de droits traduits en fable par l’imagination. Elle faisait consister tout son mérite à trouver des fables assez heureusement imaginées pour sauver la gravité de la loi, et appliquer le droit au fait. Toutes les fictions de l’ancienne jurisprudence furent donc des vérités sous le masque, et les formules dans lesquelles s’exprimaient les lois, furent appelées {\itshape carmina}, à cause de la mesure précise de leurs paroles auxquelles on ne pouvait ni ajouter, ni retrancher\footnote{Tite-Live dit en parlant de la sentence prononcée contre Horace : \emph{{\itshape Lex horrendi carminis erat}}. — Dans l’{\itshape Asinaria} de Plaute, Diabolus dit que le parasite \emph{{\itshape est un grand poète}}, parce qu’il sait mieux que tout autre trouver ces subtilités verbales qui caractérisaient les formules, ou {\itshape carmina}. ({\itshape Vico.})}. Ainsi tout l’{\itshape ancien} droit romain fut un {\itshape poème sérieux} que les Romains représentaient sur le forum, et l’ancienne jurisprudence fut une {\itshape poésie sévère}. Dans l’introduction  des Institutes, Justinien parle des fables du droit antique, \emph{{\itshape antiqui juris fabulas}} ; son but est de les tourner en ridicule, mais il doit avoir emprunté ce mot à quelque ancien jurisconsulte qui aura compris ce que nous exposons ici. C’est à ces {\itshape fables antiques} que la jurisprudence romaine rapporte ses premiers principes. De ces {\itshape personæ}, de ces {\itshape masques} qu’employaient les fables dramatiques si vraies et si sévères du droit, dérivent les premières origines de la doctrine du {\itshape droit personnel}.\par
Lorsque vinrent les âges de civilisation avec les gouvernements populaires, l’intelligence s’éveilla dans ces grandes assemblées\footnote{S’il est certain qu’il y eut des lois avant qu’il existât des philosophes, on doit en inférer que le spectacle des citoyens d’Athènes s’unissant par l’acte de la législation dans l’idée d’un intérêt égal qui fût commun à tous, aida Socrate à former les {\itshape genres intelligibles}, ou les {\itshape universaux abstraits}, au moyen de l’{\itshape induction}, opération de l’esprit qui recueille les particularités uniformes capables de composer un genre sous le rapport de leur uniformité. Ensuite Platon remarqua que, dans ces assemblées, les esprits des individus, passionnés chacun pour son intérêt, se réunissaient dans l’idée non passionnée de l’utilité commune. On l’a dit souvent, les hommes, pris séparément, sont conduits par l’intérêt personnel ; pris en masse, ils veulent la justice. C’est ainsi qu’il en vint à méditer les idées intelligibles et parfaites des esprits (idées distinctes de ces esprits, et qui ne peuvent se trouver qu’en Dieu même), et s’éleva jusqu’à la conception du {\itshape héros de la philosophie}, qui commande avec plaisir aux passions. Ainsi fut préparée la définition vraiment divine qu’Aristote nous a laissée de la loi : \emph{{\itshape Volonté libre de passion}} ; ce qui est le caractère de la volonté {\itshape héroïque}. Aristote comprit la {\itshape justice, reine} des vertus, qui habite dans le cœur du {\itshape héros}, parce qu’il avait vu la {\itshape justice légale}, qui habite dans l’âme du législateur et de l’homme d’état, commander à la {\itshape prudence} dans le sénat, au {\itshape courage} dans les armées, à la {\itshape tempérance} dans les fêtes, à la {\itshape justice particulière}, tantôt {\itshape commutative}, comme au forum, tantôt {\itshape distributive}, comme au trésor public, {\itshape ærarium} [où les impôts répartis équitablement donnent des droits proportionnels aux honneurs]. D’où il résulte que c’est de la place d’Athènes que sortirent les principes de la métaphysique, de la logique et de la morale. La liberté fit la législation, et de la législation sortit la philosophie.Tout ceci est une nouvelle réfutation du mot de Polybe que nous avons déjà cité (\emph{{\itshape Si les hommes étaient philosophes, il n’y aurait plus besoin de religion}}). Sans religion point de société, sans société point de philosophes. Si la {\itshape Providence} n’eût ainsi conduit les choses humaines, on n’aurait pas eu la moindre idée ni de {\itshape science} ni de vertu. ({\itshape Vico.})}. Les droits abstraits  et généraux furent dits {\itshape consistere in intellectu juris}. L’{\itshape intelligence} consiste ici à comprendre l’intention que le législateur a exprimée dans la loi, intention que désigne le mot {\itshape jus}. En effet cette intention fut celle {\itshape des citoyens qui s’accordaient dans la conception d’un intérêt raisonnable qui leur fût commun à tous}. Ils durent comprendre que cet intérêt était {\itshape spirituel} de sa nature, puisque tous les droits qui ne s’exercent point sur des choses corporelles, {\itshape nuda jura}, furent dits par eux {\itshape in intellectu juris consistere}. Puis donc que les droits sont des modes de la substance spirituelle, ils sont {\itshape indivisibles}, et par conséquent {\itshape éternels} ; car la corruption n’est autre chose que la division des parties. Les interprètes du droit romain ont fait consister toute la gloire de la métaphysique légale dans l’examen de l’indivisibilité des droits en traitant la fameuse matière {\itshape de dividuis et individuis}. Mais ils n’ont point considéré l’autre caractère des droits, non moins important que le premier, leur éternité. Il aurait dû pourtant les frapper dans ces deux règles qu’ils établissent 1º {\itshape cessante fine legis},  {\itshape cessat lex} ; ils ne disent point {\itshape cessante ratione} ; en effet le but, la fin de la loi, c’est l’intérêt des causes traité avec égalité ; cette fin peut changer, mais {\itshape la raison de la loi} étant une conformité de la loi au fait entouré de telles circonstances, toutes les fois que les mêmes circonstances se représentent, la {\itshape raison de la loi} les domine, vivante, impérissable ; 2º {\itshape tempus non est modus constituendi, vel dissolvendi juris} ; en effet le temps ne peut commencer ni finir ce qui est éternel. Dans les usucapions, dans les prescriptions, le temps ne finit point les droits, pas plus qu’il ne les a produits, il prouve seulement que celui qui les avait a voulu s’en dépouiller. Quoiqu’on dise que l’{\itshape usufruit prend fin}, il ne faut pas croire que le droit finisse pour cela, il ne fait que se dégager d’une servitude pour retourner à sa liberté première. — De là nous tirerons deux corollaires de la plus haute importance. Premièrement les droits étant {\itshape éternels} dans l’intelligence, autrement dit dans leur idéal, et les hommes existant {\itshape dans le temps}, les droits ne peuvent venir aux hommes que de Dieu. En second lieu, tous les droits qui ont été, qui sont ou seront, dans leur nombre, dans leur variété {\itshape infinis}, sont les modifications diverses de la {\itshape puissance} du premier homme, et du {\itshape domaine}, du droit de propriété, qu’il eut sur toute la terre.\par
Sous les gouvernements aristocratiques, la {\itshape cause} (c’est-à-dire la forme extérieure) des obligations consistait dans une formule où l’on cherchait une  garantie dans la précision des paroles et la propriété des termes\footnote{{\itshape A cavendo, cavissæ} ; puis, par contraction, {\itshape caussæ}. ({\itshape Vico.})}. Mais dans les temps civilisés où se formèrent les démocraties et ensuite les monarchies, la {\itshape cause} du contrat fut prise pour la volonté des parties et pour le contrat même. Aujourd’hui c’est la volonté qui rend le pacte obligatoire, et par cela seul qu’on a voulu contracter, la convention produit une action. Dans les cas où il s’agit de transférer la propriété, c’est cette même volonté qui valide la tradition naturelle et opère l’aliénation ; ce ne fut que dans les contrats verbaux, comme la stipulation, que la garantie du contrat conserva le nom de {\itshape cause} pris dans son ancienne acception. Ceci jette un nouveau jour sur les principes des obligations qui naissent des pactes et contrats, tels que nous les avons établis plus haut.\par
Concluons : l’homme n’étant proprement qu’{\itshape intelligence, corps} et {\itshape langage}, et le langage étant comme l’intermédiaire des deux substances qui constituent sa nature, le {\scshape certain} en matière de justice fut déterminé par {\itshape des actes du corps} dans les temps qui précédèrent l’invention du langage articulé. Après cette invention, il le fut par des {\itshape formules verbales}. Enfin la raison humaine ayant pris tout son développement, le certain alla se confondre avec le {\scshape vrai} des idées relatives à la justice, lesquelles furent déterminées par la raison d’après les circonstances les plus particulières des faits ;  {\itshape formule éternelle qui n’est sujette à aucune forme particulière}, mais qui éclaire toutes les formes diverses des faits, comme la lumière qui n’a point de figure, nous montre celle des corps opaques dans les moindres parties de leur superficie. C’est elle que le docte Varron appelait la {\scshape formule de la nature}.
\chapterclose

\chapterclose


\chapteropen
\part[{Livre cinquième. Retour des mêmes révolutions lorsque les sociétés détruites se relèvent de leurs ruines}]{Livre cinquième. \\
Retour des mêmes révolutions lorsque les sociétés détruites se relèvent de leurs ruines}

\chaptercont

\chapteropen
\chapter[{Argument}]{Argument}

\chaptercont
\noindent  {\itshape La plupart des preuves historiques données jusqu’ici par l’auteur à l’appui de ses principes, étant empruntées à l’antiquité, la Science nouvelle ne mériterait pas le nom d’}histoire éternelle de l’humanité, {\itshape si l’auteur ne montrait que les caractères observés dans les temps antiques se sont reproduits, en grande partie, dans ceux du moyen âge. Il suit dans ces rapprochements sa division des âges divin, héroïque et humain. Il conclut en démontrant que c’est la Providence qui conduit les choses humaines, puisque dans tout gouvernement ce sont les} meilleurs {\itshape qui ont dominé}. ({\itshape Il prend le mot} meilleurs {\itshape dans un sens très général.})\par
Chapitre {\scshape I. Objet de ce livre. — Retour de l’âge divin.} — {\itshape Pourquoi Dieu permit qu’un ordre de choses analogue à celui de l’antiquité reparût au moyen âge. Ignorance de l’écriture ; caractère religieux des guerres et des jugements, asiles, etc.}\par
Chapitre {\scshape II. Comment les nations parcourent de nouveau la carrière qu’elles ont fournie conformément à la nature éternelle des fiefs. Que l’ancien droit politique des romains se renouvela dans le droit féodal. (Retour de l’âge héroïque.}) —  {\itshape Comparaison des vassaux du moyen âge avec les clients de l’antiquité, des parlements avec les comices. Remarques sur les mots} hommage, baron, {\itshape sur les précaires, sur la recommandation personnelle, et sur les alleux}.\par
Chapitre {\scshape III. Coup d’œil sur le monde politique, ancien et moderne}, {\itshape considéré relativement au but de la Science nouvelle}. (Â{\scshape ge humain}.) — {\itshape Rome, n’étant arrêtée par aucun obstacle extérieur, a fourni toute la carrière politique que suivent les nations, passant de l’aristocratie à la démocratie, et de la démocratie à la monarchie. — Conformément aux principes de la Science nouvelle, on trouve aujourd’hui dans le monde beaucoup de monarchies, quelques démocraties, presque plus d’aristocraties.}\par
Chapitre {\scshape IV. Conclusion. D’une république éternelle fondée dans la nature par la Providence divine, et qui est la meilleure possible dans chacune de ses formes diverses.} — {\itshape C’est le résumé de tout le système, et son explication morale et religieuse.}
\chapterclose


\chapteropen
\chapter[{Chapitre I. Objet de ce livre. — Retour de l’âge divin}]{Chapitre I. \\
Objet de ce livre. — Retour de l’âge divin}

\chaptercont
\noindent  D’après les rapports innombrables que nous avons indiqués dans cet ouvrage entre les temps barbares de l’antiquité et ceux du moyen âge, on a pu sans peine en remarquer la merveilleuse correspondance, et saisir les lois qui régissent les sociétés, lorsque sortant de leurs ruines elles recommencent une vie nouvelle. Néanmoins nous consacrerons à ce sujet un livre particulier, afin d’éclairer les temps de la {\itshape barbarie moderne}, qui étaient restés plus obscurs que ceux de la {\itshape barbarie antique}, appelés eux-mêmes {\itshape obscurs} par le docte Varron dans sa division des temps. Nous montrerons en même temps comment le Tout-Puissant a fait servir les conseils de sa {\itshape Providence}, qui dirigeaient la marche des sociétés, aux décrets ineffables de sa {\itshape grâce}.\par
Lorsqu’il eut par des voies {\itshape surnaturelles} éclairé et affermi la vérité du christianisme, contre la puissance romaine par la vertu des martyrs, contre la  vaine sagesse des Grecs par la doctrine des Pères et par les miracles des Saints, alors s’élevèrent des nations armées, au nord les barbares Ariens, au midi les Sarrasins mahométans, qui attaquaient de toutes parts la divinité de Jésus-Christ. Afin d’établir cette vérité d’une manière inébranlable selon le cours {\itshape naturel} des choses humaines, Dieu permit qu’un nouvel ordre de choses naquît parmi les nations.\par
Dans ce conseil éternel, il ramena les mœurs du premier âge qui méritèrent mieux alors le nom de {\itshape divines}. Partout les rois catholiques, protecteurs de la religion, revêtaient les habits de diacres et consacraient à Dieu leurs personnes royales\footnote{Ils en ont conservé le titre de {\itshape sacrée majesté}. ({\itshape Vico.})}. Ils avaient des dignités ecclésiastiques : Hugues Capet s’intitulait comte et abbé de Paris, et les annales de Bourgogne remarquent en général que dans les actes anciens les princes de France prenaient souvent les titres de ducs et abbés, de comtes et abbés. — Les premiers rois chrétiens fondèrent des ordres religieux et militaires pour combattre les infidèles. — Alors revinrent avec plus de vérité le {\itshape pura et pia bella} des peuples héroïques. Les rois mirent la croix sur leurs bannières, et maintenant encore ils placent sur leurs couronnes un globe surmonté d’une croix. — Chez les anciens, le héraut qui déclarait la guerre, invitait les dieux à quitter la cité ennemie ({\itshape evocabat deos}). De même au moyen âge, on cherchait toujours à enlever les reliques des  cités assiégées. Aussi les peuples mettaient-ils leurs soins à les cacher, à les enfouir sous terre ; on voit dans toutes les églises que le lieu où on les conserve est le plus reculé, le plus secret.\par
À partir du commencement du cinquième siècle, où les barbares inondèrent le monde romain, les vainqueurs ne s’entendent plus avec les vaincus. Dans cet âge de fer, on ne trouve d’écriture en langue vulgaire ni chez les Italiens, ni chez les Français, ni chez les Espagnols. Quant aux Allemands, ils ne commencent à écrire d’actes dans leur langue qu’au temps de Frédéric de Souabe, et, selon quelques-uns, seulement sous Rodolphe de Habsbourg. Chez toutes ces nations on ne trouve rien d’écrit qu’en latin barbare, langue qu’entendaient seuls un bien petit nombre de nobles qui étaient ecclésiastiques. Faute de caractères vulgaires, les hiéroglyphes des anciens reparurent dans les emblèmes, dans les armoiries. Ces signes servaient à assurer les propriétés, et le plus souvent indiquaient les droits seigneuriaux sur les maisons et sur les tombeaux, sur les troupeaux et sur les terres.\par
Certaines espèces de {\itshape jugements divins} reparurent sous le nom de {\itshape purgations canoniques} ; les {\itshape duels} furent une espèce de ces jugements, quoique non autorisés par les canons. On revit aussi les brigandages héroïques. Les anciens héros avaient tenu à honneur d’être appelés {\itshape brigands} ; le nom de {\itshape corsale} fut un titre de seigneurie. Les {\itshape représailles} de l’antiquité, la dureté des {\itshape servitudes héroïques} se  renouvelèrent, et durent encore entre les infidèles et les chrétiens. La victoire passant pour le jugement du ciel, les vainqueurs croyaient {\itshape que les vaincus n’avaient point de Dieu}, et les traitaient comme de vils animaux.\par
Un rapport plus merveilleux encore entre l’antiquité et le moyen âge, c’est que l’on vit se rouvrir les {\itshape asiles}, qui, selon Tite-Live, avaient été l’\emph{{\itshape origine de toutes les premières cités}}. Partout avaient recommencé les violences, les rapines, les meurtres, et comme {\itshape la religion est le seul moyen de contenir des hommes affranchis du joug des lois humaines} (axiome 31), les hommes moins barbares qui craignaient l’oppression se réfugiaient chez les évêques, chez les abbés, et se mettaient sous leur protection, eux, leur famille et leurs biens ; c’est le besoin de cette protection qui motive la plupart des constitutions de fiefs. Aussi dans l’Allemagne, pays qui fut au moyen âge le plus barbare de toute l’Europe, il est resté, pour ainsi dire, plus de souverains ecclésiastiques que de séculiers. — De là le nombre prodigieux de cités et de forteresses qui portent des noms de saints. — Dans des lieux difficiles ou écartés, l’on ouvrait de petites chapelles où se célébrait la messe, et s’accomplissaient les autres devoirs de la religion. On peut dire que ces chapelles furent les {\itshape asiles} naturels des chrétiens ; les fidèles élevaient autour leurs habitations. Les monuments les plus anciens qui nous restent du moyen âge, sont des chapelles situées ainsi, et le  plus souvent ruinées. Nous en avons chez nous un illustre exemple dans l’abbaye de Saint-Laurent d’Averse, à laquelle fut incorporée l’abbaye de Saint-Laurent de Capoue. Dans la Campanie, le Samnium, l’Apulie\footnote{Orthographié « Appulie » [NdE].} et dans l’ancienne Calabre, du Vulture au golfe de Tarente, elle gouverna cent dix églises, soit immédiatement, soit par des abbés ou moines qui en étaient dépendants, et dans presque tous ces lieux les abbés de Saint-Laurent étaient en même temps les barons.
\chapterclose


\chapteropen
\chapter[{Chapitre II. Comment les nations parcourent de nouveau la carrière qu’elles ont fournie, conformément à la nature éternelle des fiefs. Que l’ancien droit politique des romains se renouvela dans le droit féodal. (Retour de l’âge héroïque.)}]{Chapitre II. \\
Comment les nations parcourent de nouveau la carrière qu’elles ont fournie, conformément à la nature éternelle des fiefs. Que l’ancien droit politique des romains se renouvela dans le droit féodal. (Retour de l’âge héroïque.)}

\chaptercont
\noindent  À l’âge {\itshape divin} ou théocratique dont nous venons de parler, succéda l’âge {\itshape héroïque} avec la même distinction de {\itshape natures} qui avait caractérisé dans l’antiquité les {\itshape héros} et les {\itshape hommes}. C’est ce qui explique pourquoi les vassaux roturiers s’appellent {\itshape homines} dans la langue du droit féodal. D’{\itshape homines} vinrent {\itshape hominium} et {\itshape homagium}. Le premier est pour {\itshape hominis dominium}, le domaine du seigneur sur la personne du vassal ; {\itshape homagium} est pour {\itshape hominis agium}, le droit qu’a le seigneur de mener le vassal où il veut. Les feudistes traduisent élégamment le mot barbare {\itshape homagium} par {\itshape obsequium}, qui dans le principe dut avoir le même sens en latin. Chez les anciens Romains, l’{\itshape obsequium} était inséparable de ce qu’ils appelaient {\itshape opera militaris}, et de ce que nos feudistes appellent {\itshape militare servitium} ; longtemps les plébéiens romains servirent à  leurs dépens les nobles à la guerre. Cet {\itshape obsequium} avec les charges qui en étaient la suite, fut vers la fin la condition des affranchis, {\itshape liberti}, qui restaient à l’égard de leur patron dans une sorte de dépendance ; mais il avait commencé avec Rome même, puisque l’institution fondamentale de cette cité fut le {\itshape patronage}, c’est-à-dire, la protection des malheureux qui s’étaient réfugiés dans l’asile de Romulus, et qui cultivaient, comme journaliers, les terres des patriciens. Nous avons déjà remarqué que dans l’histoire ancienne, le mot {\itshape clientela} ne peut mieux se traduire que par celui de {\itshape fiefs}. L’origine du mot {\itshape opera} nous prouve la vérité de ces principes. {\itshape Opera} dans sa signification primitive est le travail d’un paysan pendant un jour. Les Latins appellent {\itshape operarius} ce que nous entendons par {\itshape journalier}. — On disait chez les Latins {\itshape greges operarum}, comme {\itshape greges servorum}, parce que de tels ouvriers, ainsi que les esclaves des temps plus récents étaient regardés comme les bêtes de somme que l’on disait {\itshape pasci gregatim}. Par analogie on appelait les héros {\itshape pasteurs} ; Homère ne manque jamais de leur donner l’épithète de {\itshape pasteurs des peuples}. Νόμος, νομός, signifient {\itshape loi} et {\itshape pâturage}.\par
L’{\itshape obsequium} des affranchis, ayant peu à peu disparu, et la puissance des patrons ou seigneurs s’étant en quelque sorte {\itshape dispersée} dans les guerres civiles, {\itshape où les puissants deviennent dépendants des peuples}, cette puissance se {\itshape réunit} sans peine dans la personne des monarques, et il ne resta plus que  l’\emph{{\itshape obsequium principis}}, dans lequel, selon Tacite, consiste tout le \emph{{\itshape devoir des sujets d’une monarchie}}. Par opposition à leurs vassaux ou {\itshape homines}, les seigneurs des fiefs furent appelés {\itshape barons} dans le sens où les Grecs prenaient {\itshape héros}, et les anciens Latins {\itshape viri} ; les Espagnols disent encore {\itshape baron} pour signifier le {\itshape vir} des Latins. Cette dénomination d’{\itshape hommes}, leur fut donnée sans doute par opposition à la faiblesse des vassaux, faiblesse dont l’idée était dans les temps héroïques jointe à celle du sexe {\itshape féminin}. Les barons furent appelés {\itshape seigneurs}, du latin {\itshape seniores}. Les anciens parlements du moyen âge durent se composer des {\itshape seigneurs}, précisément comme le sénat de Rome avait été composé par Romulus des nobles les plus âgés. De ces {\itshape patres}, on dut appeler {\itshape patroni} ceux qui affranchissaient des esclaves, de même que chez nous {\itshape patron} signifie {\itshape protecteur} dans le sens le plus élégant et le plus conforme à l’étymologie. À cette expression répond celle de {\itshape clientes} dans le sens de {\itshape vassaux roturiers}, tels que purent être les {\itshape clients}, lorsque Servius Tullius par l’institution du cens, leur permit de tenir des terres en fiefs. ({\itshape Voy.} la pag. suivante.)\par
Les fiefs roturiers du moyen âge, d’abord {\itshape personnels} représentèrent les clientèles de l’antiquité. Au temps où brillait de tout son éclat la liberté populaire de Rome, les plébéiens vêtus de toges allaient tous les matins faire leur cour aux grands. Ils les saluaient du titre des anciens héros, {\itshape ave rex}, les menaient au forum, et les ramenaient le soir à la  maison. Les grands, conformément à l’ancien titre héroïque de {\itshape pasteurs des peuples}, leur donnaient à souper. Ceux qui étaient soumis à cette sorte de vasselage {\itshape personnel}, furent sans doute chez les anciens Romains les premiers {\itshape vades}, nom qui resta à ceux qui étaient obligés de suivre leurs {\itshape actores} devant les tribunaux ; cette obligation s’appelait {\itshape vadimonium}. En appliquant nos principes aux étymologies latines, nous trouvons que ce mot dut venir du nominatif {\itshape vas}, chez les Grecs Βας, et chez les barbares {\itshape was}, d’où {\itshape wassus}, et enfin {\itshape vassalus}.\par
À la suite des fiefs roturiers {\itshape personnels}, vinrent les {\itshape réels}. Nous les avons vus commencer chez les Romains avec l’institution du {\itshape cens}. Les plébéiens qui reçurent alors le domaine bonitaire des champs que les nobles leur avaient assignés, et qui furent dès lors sujets à des charges non-seulement {\itshape personnelles}, mais {\itshape réelles}, durent être désignés les premiers par le nom de {\itshape mancipes}, lequel resta ensuite à ceux qui sont {\itshape obligés sur biens immeubles envers le trésor public}. Ces plébéiens qui furent ainsi liés, {\itshape nexi}, jusqu’à la loi Petilia, répondent précisément aux {\itshape vassaux} que l’on nommait {\itshape hommes liges, ligati}. L’homme {\itshape lige} est, selon la définition des feudistes, {\itshape celui qui doit reconnaître pour amis et pour ennemis tous les amis et ennemis de son seigneur}. Cette forme de serment est analogue à celle que les anciens vassaux germains prêtaient à leur chef, au rapport de Tacite ; ils juraient \emph{{\itshape de se dévouer à sa gloire}}. Les rois vaincus auxquels le peuple romain  {\itshape regna dono dabat} (ce qui équivaut à {\itshape beneficio dabat}), pouvaient être considérés comme ses {\itshape hommes liges} ; s’ils devenaient ses alliés, c’était de cette sorte d’alliance que les Latins appelaient {\itshape fœdus inæquale}. Ils étaient {\itshape amis du peuple romain} dans le sens où les Empereurs donnaient le nom d’{\itshape amis} aux nobles qui composaient leur cour. Cette alliance inégale n’était autre chose que l’{\itshape investiture d’un fief souverain}. Cette investiture était donnée avec la formule que nous a laissée Tite-Live, savoir, que le roi allié \emph{{\itshape servaret majestatem populi Romani}} ; précisément de la même manière que le jurisconsulte Paulus dit que le préteur rend la justice \emph{{\itshape servatâ majestate populi Romani}}. Ainsi ces alliés étaient {\itshape seigneurs de fiefs souverains soumis à une plus haute souveraineté}.\par
On vit reparaître les {\itshape clientèles} des Romains sous le nom de {\itshape recommandation personnelle}. — Les {\itshape cens seigneuriaux} n’étaient pas sans analogie avec le {\itshape cens} institué par Servius Tullius, puisqu’en vertu de cette dernière institution les plébéiens furent longtemps assujettis à servir les nobles dans la guerre à leurs propres dépens, comme dans les temps modernes les vassaux appelés {\itshape angarii} et {\itshape perangarii}. — Les {\itshape précaires} du moyen âge étaient encore renouvelés de l’antiquité. C’était dans l’origine des terres accordées par les seigneurs aux prières des {\itshape pauvres} qui vivaient du produit de la culture. — ({\itshape Voy.} aussi pag. 183.)\par
Nous avons dit que ceux qui par l’institution du {\itshape cens} obtinrent le domaine bonitaire des champs  qu’ils cultivaient, furent les premiers {\itshape mancipes} des Romains. La {\itshape mancipation} revint au moyen âge ; le vassal mettait ses mains entre celles du seigneur pour lui jurer foi et obéissance. Dans l’acte de la {\itshape mancipation} les stipulations se représentèrent {\itshape sous la forme des infestucations} ou {\itshape investitures}, ce qui était la même chose. Avec les stipulations revint ce qui dans l’ancienne jurisprudence romaine avait été appelé proprement {\itshape cavissæ}, par contraction {\itshape caussæ} ; au moyen âge, on tira de la même étymologie le mot {\itshape cautelæ}. Avec ces {\itshape cautelæ} reparurent dans l’acte de la {\itshape mancipation}, les pactes que les jurisconsultes romains appelaient {\itshape stipulata}, de {\itshape stipula}, la paille qui revêt le grain ; c’est dans le même sens que les docteurs du moyen âge dirent d’après les {\itshape investitures} ou {\itshape infestucations, pacta vestita}, et {\itshape pacta nuda}. — On retrouve encore au moyen âge les deux sortes de domaines, {\itshape direct} et {\itshape utile}, qui répondent au domaine {\itshape quiritaire}, et {\itshape bonitaire} des anciens Romains. On y retrouve aussi les biens {\itshape ex jure optimo} que les feudistes érudits définissent de la manière suivante : {\itshape biens allodiaux, libres de toute charge publique et privée}. Cicéron remarque que de son temps il restait à Rome bien peu de choses qui fussent {\itshape ex jure optimo} ; et dans les lois romaines du dernier âge, il ne reste plus de connaissance des biens de ce genre. De même il est impossible maintenant de trouver de pareils alleux. Les biens {\itshape ex jure optimo} des Romains, les alleux du moyen âge, ont fini également par être des {\itshape biens immeubles libres de toute}  {\itshape charge privée}, mais sujets aux charges publiques.\par
Dans les premiers parlements, dans les {\itshape cours armées}, composées de barons, de pairs, on revoit les assemblées héroïques, où les {\itshape quirites} de Rome paraissaient en armes. L’histoire de France nous raconte que dans l’origine les rois étaient les chefs du parlement, et qu’ils commettaient des pairs au jugement des causes. Nous voyons de même chez les Romains qu’au premier jugement où, selon Cicéron, il s’agit de la vie d’un citoyen, le roi Tullus Hostilius nomma des commissaires ou duumvirs pour juger Horace. Ils devaient employer contre le fratricide la formule que cite Tite-Live, \emph{{\itshape in Horatium perduellionem dicerent}}. C’est que dans la sévérité des temps héroïques où la cité se composait des seuls héros, tout meurtre de citoyen était un acte d’hostilité contre la patrie, {\itshape perduellio}. Tout meurtre était appelé {\itshape parricidium}, meurtre d’un père, c’est-à-dire, d’un noble. Mais lorsque les plébéiens, les {\itshape hommes} dans la langue féodale, commencèrent à faire partie de la cité, le meurtre de tout homme fut appelé {\itshape homicide}.\par
Lorsque les universités d’Italie commencèrent à enseigner les lois romaines d’après les livres de Justinien, qui les présente d’une manière conforme au {\itshape droit naturel des peuples civilisés}, les esprits déjà plus ouverts s’attachèrent aux règles de l’équité naturelle dans l’étude de la jurisprudence, cette équité égale les nobles et les plébéiens dans la société, comme ils sont égaux dans la nature. Depuis que  Tibérius Coruncanius eut commencé à Rome d’enseigner publiquement la science des lois, la jurisprudence jusqu’alors secrète échappa aux nobles, et leur puissance s’en trouva peu à peu affaiblie. La même chose arriva aux nobles des nouveaux royaumes de l’Europe dont les gouvernements avaient été d’abord aristocratiques, et qui devinrent successivement populaires et monarchiques\footnote{Ces deux dernières formes, convenant également aux gouvernements des âges civilisés, peuvent sans peine se changer l’une pour l’autre. Mais revenir à l’aristocratie, c’est ce qui est inconciliable avec la nature sociale de l’homme. Le vertueux Dion de Syracuse, l’ami du divin Platon, avait délivré sa patrie de la tyrannie d’un monstre ; il n’en fut pas moins assassiné pour avoir essayé de rétablir l’aristocratie. Les pythagoriciens, qui composaient toute l’aristocratie de la Grande-Grèce, tentèrent d’opérer la même révolution, et furent massacrés ou brûlés vifs. En effet, dès qu’une fois les plébéiens ont reconnu qu’ils sont égaux en nature aux nobles, ils ne se résignent point à leur être inférieurs sous le rapport des droits politiques, et ils obtiennent cette égalité dans l’état populaire, ou sous la monarchie. Aussi voyons-nous le peu de gouvernements aristocratiques qui subsistent encore, s’attacher, avec un soin inquiet et une sage prévoyance, à contenir la multitude et à prévenir de dangereux mécontentemens. ({\itshape Vico.})}\footnote{Bodin avoue que le royaume de France eut, non pas un gouvernement, comme nous le prétendons, mais au moins une constitution {\itshape aristocratique} sous les races mérovingienne et carlovingienne. Nous demanderons alors à Bodin comment ce royaume s’est trouvé soumis, comme il l’est, à une monarchie pure. Sera-ce en vertu d’une {\itshape loi royale} par laquelle les paladins français se sont dépouillés de leur puissance en faveur des Capétiens, de même que le peuple romain abdiqua la sienne en faveur d’Auguste, si nous en croyons la fable de la {\itshape loi royale} débitée par Tribonien ? Ou bien dira-t-il que la France a été conquise par quelqu’un des Capétiens ?… Il faut plutôt que Bodin, et avec lui tous les politiques, tous les jurisconsultes, reconnaissent cette {\itshape loi royale, fondée en nature sur un principe éternel} ; c’est que la puissance libre d’un état, par cela même qu’elle est libre, doit en quelque sorte se réaliser. Ainsi, toute la force que perdent les nobles, le peuple la gagne, jusqu’à ce qu’il devienne libre ; toute celle que perd le peuple libre tourne au profit des rois, qui finissent par acquérir un pouvoir monarchique. Le droit naturel des moralistes est celui de la {\itshape raison} ; le droit naturel des gens est celui de l’{\itshape utilité} et de la {\itshape force}. Ce droit, comme disent les jurisconsultes, a été suivi par les nations, \emph{{\itshape usu exigente humanisque necessitatibus expostulantibus}}. ({\itshape Vico.})}.\par
 Après les remarques diverses que nous avons faites dans ce chapitre sur tant d’expressions élégantes de l’ancienne jurisprudence romaine, au moyen desquelles les feudistes corrigent la barbarie de la langue féodale, Oldendorp et tous les autres écrivains de son opinion doivent voir si le droit féodal est sorti, comme ils le disent, {\itshape des étincelles de l’incendie dans lequel les barbares détruisirent le droit romain}. Le droit romain au contraire est né de la féodalité ; je parle de cette féodalité primitive que nous avons observée particulièrement dans la barbarie antique du Latium, et qui a été la base commune de toutes les sociétés humaines.
\chapterclose


\chapteropen
\chapter[{Chapitre III. Coup d’œil sur le monde politique, ancien et moderne, considéré relativement au but de la science nouvelle}]{Chapitre III. \\
Coup d’œil sur le monde politique, ancien et moderne, considéré relativement au but de la science nouvelle}

\chaptercont
\noindent  La marche que nous avons tracée ne fut point suivie par Carthage, Capoue et Numance, ces trois cités qui firent craindre à Rome d’être supplantée dans l’empire du Monde. Les Carthaginois furent arrêtés de bonne heure dans cette carrière par la subtilité naturelle de l’esprit africain, encore augmentée par les habitudes du commerce maritime. Les Capouans le furent par la mollesse de leur beau climat, et par la fertilité de la Campanie {\itshape heureuse}. Enfin Numance commençait à peine son âge {\itshape héroïque}, lorsqu’elle fut accablée par la puissance romaine, par le génie du vainqueur de Carthage, et par toutes les forces du monde. Mais les Romains ne rencontrant aucun de ces obstacles, marchèrent d’un pas égal, guidés dans cette marche par la Providence qui se sert de l’instinct des peuples pour les conduire. Les trois formes de gouvernement se succédèrent chez eux conformément à l’ordre naturel ; l’aristocratie dura jusqu’aux lois {\itshape Publilia} et {\itshape Petilia}, la liberté populaire jusqu’à Auguste, la  monarchie tant qu’il fut humainement possible de résister aux causes intérieures et extérieures qui détruisent un tel état politique.\par
Aujourd’hui la plus complète civilisation semble répandue chez les peuples, soumis la plupart à un petit nombre de grands monarques. S’il est encore des nations barbares dans les parties les plus reculées du nord et du midi, c’est que la nature y favorise peu l’espèce humaine, et que l’instinct naturel de l’humanité y a été longtemps dominé par des religions farouches et bizarres. — Nous voyons d’abord au septentrion le czar de Moscovie qui est à la vérité chrétien, mais qui commande à des hommes d’un esprit lent et paresseux. — Le kan de Tartarie, qui a réuni à son vaste empire celui de la Chine, gouverne un peuple efféminé, tels que le furent les {\itshape seres} des anciens. — Le négus d’Éthiopie, et les rois de Fez et de Maroc règnent sur des peuples faibles et peu nombreux.\par
Mais sous la zone tempérée, où la nature a mis dans les facultés de l’homme un plus heureux équilibre, nous trouvons, en partant des extrémités de l’Orient, l’empire du Japon, dont les mœurs ont quelque analogie avec celles des Romains pendant les guerres puniques ; c’est le même esprit belliqueux, et si l’on en croit quelques savants voyageurs la langue japonaise présente à l’oreille une certaine analogie avec le latin. Mais ce peuple est en partie retenu dans l’état {\itshape héroïque} par une religion  pleine de croyances effrayantes, et dont les dieux tout couverts d’armes menaçantes inspirent la terreur. Les missionnaires assurent que le plus grand obstacle qu’ils aient trouvé dans ce pays à la foi chrétienne, c’est qu’on ne peut persuader aux nobles que les gens du peuple sont hommes comme eux. — L’empire de la Chine avec sa religion douce et sa culture des lettres, est très policé. — Il en est de même de l’Inde, vouée en général aux arts de la paix. — La Perse et la Turquie ont mêlé à la mollesse de l’Asie les croyances grossières de leur religion. Chez les Turcs particulièrement, l’orgueil du caractère national, est tempéré par une libéralité fastueuse, et par la reconnaissance.\par
L’Europe entière est soumise à la religion chrétienne, qui nous donne l’idée la plus pure et la plus parfaite de la divinité, et qui nous fait un devoir de la charité envers tout le genre humain. De là sa haute civilisation. — Les principaux états européens sont de grandes monarchies. Celles du Nord, comme la Suède et le Danemark il y a un siècle et demi, et comme aujourd’hui encore la Pologne et l’Angleterre, semblent soumises à un gouvernement aristocratique ; mais si quelque obstacle extraordinaire n’arrête la marche naturelle des choses, elles deviendront des monarchies pures. — Cette partie du monde plus éclairée a aussi plus d’états populaires que nous n’en voyons dans les trois autres. Le retour des mêmes besoins politiques y a renouvelé  la forme du gouvernement des Achéens et des Étoliens. Les Grecs avaient été amenés à concevoir cette forme de gouvernement par la nécessité de se prémunir contre l’ambition d’une puissance colossale. Telle a été aussi l’origine des cantons Suisses et des Provinces-Unies. Ces ligues perpétuelles d’un grand nombre de cités libres ont formé deux aristocraties. L’Empire germanique est aussi un système composé d’un grand nombre de cités libres et de princes souverains. La tête de ce corps est l’Empereur, et dans ce qui concerne les intérêts communs de l’Empire il se gouverne aristocratiquement. Du reste il n’y a plus en Europe que cinq aristocraties proprement dites, en Italie Venise, Gênes et Lucques, Raguse en Dalmatie, et Nuremberg en Allemagne ; elles n’ont pour la plupart qu’un territoire peu étendu\footnote{Si nous traversons l’Océan pour passer dans le Nouveau-Monde, nous trouverons que l’Amérique eût parcouru la même carrière sans l’arrivée des Européens. ({\itshape Vico.})}.\par
Notre Europe brille d’une incomparable civilisation ; elle abonde de tous les biens qui composent la félicité de la vie humaine ; on y trouve toutes les jouissances intellectuelles et morales. Ces avantages, nous les devons à la religion. La religion nous fait un devoir de la charité envers tout le genre humain ; elle admet à la seconder dans l’enseignement de ses préceptes sublimes les plus doctes philosophies de l’antiquité païenne ; elle a adopté, elle cultive trois langues, la plus ancienne, la  plus délicate et la plus noble, l’hébreu, le grec, et le latin. Ainsi, même pour les fins humaines, le christianisme est supérieur à toutes les religions : il unit la sagesse de l’autorité à celle de la raison, et cette dernière, il l’appuie sur la plus saine philosophie et sur l’érudition la plus profonde.\par
Après avoir observé dans ce Livre comment les sociétés recommencent la même carrière, réfléchissons sur les nombreux rapprochements que nous présente cet ouvrage entre l’antiquité et les temps modernes, et nous y trouverons expliquée non plus l’histoire particulière et temporelle des lois et des faits des Romains ou des Grecs, mais l’{\itshape histoire idéale} des lois éternelles que suivent toutes les nations dans leurs commencements et leurs progrès, dans leur décadence et leur fin, et qu’elles suivraient toujours quand même (ce qui n’est point) des mondes infinis naîtraient successivement dans toute l’éternité. À travers la diversité des formes extérieures, nous saisirons l’{\itshape identité de substance} de cette histoire. Aussi ne pouvons-nous refuser à cet ouvrage le titre orgueilleux peut-être de {\itshape Science nouvelle}. Il y a droit par son sujet : {\itshape la nature commune des nations} ; sujet vraiment universel, dont l’idée embrasse toute science digne de ce nom. Cette idée est indiquée dans la vaste expression de Sénèque : \emph{{\itshape Pusilla res hic mundus est, nisi id, quod quæerit, omnis mundus habeat.}}
\chapterclose


\chapteropen
\chapter[{Chapitre IV. Conclusion. — D’une république éternelle fondée dans la nature par la providence divine, et qui est la meilleure possible dans chacune de ses formes diverses}]{Chapitre IV. \\
Conclusion. — D’une république éternelle fondée dans la nature par la providence divine, et qui est la meilleure possible dans chacune de ses formes diverses}

\chaptercont
\noindent  Concluons en rappelant l’idée de Platon, qui ajoute aux trois formes de républiques une quatrième, dans laquelle régneraient les meilleurs, ce qui serait la véritable aristocratie naturelle. Cette république que voulait Platon, elle a existé dès la première origine des sociétés. Examinons en ceci la conduite de la Providence.\par
D’abord elle voulut que les géants qui erraient dans les montagnes, effrayés des premiers orages qui eurent lieu après le déluge, cherchassent un refuge dans les cavernes, que malgré leur orgueil ils s’humiliassent devant la divinité qu’ils se créaient, et s’assujettissent à une force supérieure qu’ils appelèrent Jupiter. C’est à la lueur des éclairs qu’ils virent cette grande vérité, {\itshape que Dieu gouverne le genre humain}. Ainsi se forma une première société que j’appellerai {\itshape monastique} dans le sens de l’étymologie, parce qu’elle était en effet composée de  {\itshape souverains solitaires} sous le gouvernement d’un être très bon et très puissant, {\scshape optimus maximus}. Excités ensuite par les plus puissants aiguillons d’une passion brutale, et retenus par les craintes superstitieuses que leur donnait toujours l’aspect du ciel, ils commencèrent à réprimer l’impétuosité de leurs désirs et à faire usage de la liberté humaine. Ils retinrent par force dans leurs cavernes des femmes, dont ils firent les compagnes de leur vie. Avec ces premières unions {\itshape humaines}, c’est-à-dire conformes à la pudeur et à la religion, commencèrent les mariages qui déterminèrent les rapports d’époux, de fils et de pères. Ainsi ils fondèrent les familles, et les gouvernèrent avec la dureté des cyclopes dont parle Homère ; la dureté de ce premier gouvernement était nécessaire, pour que les hommes se trouvassent préparés au gouvernement civil, lorsque s’élèveraient les cités. La première république se trouve donc dans la famille ; la forme en est monarchique, puisqu’elle est soumise aux pères de famille, qui avait la supériorité du sexe, de l’âge et de la vertu.\par
Aussi vaillants que chastes et pieux, ils ne fuyaient plus comme auparavant, mais, fixant leurs habitations, ils se défendaient, eux et les leurs, tuaient les bêtes sauvages qui infestaient leurs champs, et au lieu d’errer pour trouver leur pâture, ils soutenaient leurs familles en cultivant la terre ; toutes choses qui assurèrent le salut du genre humain. Au bout d’un long temps, ceux qui étaient restés dans  les plaines, sentirent les maux attachés à la communauté des biens et des femmes, et vinrent se réfugier dans les asiles ouverts par les pères de famille. Ceux-ci les recevant sous leur protection, la monarchie domestique s’étendit par les clientèles. C’était encore les meilleurs qui régnaient, {\scshape optimi}. Les réfugiés, impies et sans dieu, obéissaient à des hommes pieux, qui adoraient la divinité, bien qu’ils la divisassent par leur ignorance, et qu’ils se figurassent les dieux d’après la variété de leurs manières de voir ; étrangers à la pudeur, ils obéissaient à des hommes qui se contentaient pour toute leur vie d’une compagne que leur avait donnée la religion ; faibles et jusque-là errants au hasard, ils obéissaient à des hommes prudents qui cherchaient à connaître par les auspices la volonté des dieux, à des héros qui {\itshape domptaient la terre} par leurs travaux, tuaient les bêtes farouches, et secouraient le faible en danger.\par
Les pères de famille devenus puissants par la piété et la vertu de leurs ancêtres et par les travaux de leurs clients, oublièrent les conditions auxquelles ceux-ci s’étaient livrés à eux, et au lieu de les protéger, ils les opprimèrent. Sortis ainsi de l’{\itshape ordre naturel} qui est celui de la justice, ils virent leurs clients se révolter contre eux. Mais comme la société humaine ne peut subsister un moment sans ordre, c’est-à-dire sans dieu, la Providence fit naître l’{\itshape ordre civil} avec la formation des cités. Les pères de famille s’unirent pour résister aux clients, et pour les apaiser, leur abandonnèrent le domaine bonitaire  des champs dont ils se réservaient le domaine éminent. Ainsi naquit la cité, fondée sur un corps souverain de nobles. Cette noblesse consistait à sortir d’un mariage solennel, et célébré avec les auspices. Par elle les nobles régnaient sur les plébéiens, dont les unions n’étaient pas ainsi consacrées. — Au gouvernement théocratique où les dieux gouvernaient les familles par les auspices, succéda le gouvernement héroïque où les héros régnaient eux-mêmes, et dont la base principale fut la religion, privilège du corps des pères qui leur assurait celui de tous les droits civils. Mais comme la noblesse était devenue un don de la fortune, du milieu des nobles même s’éleva l’ordre des {\itshape pères} qui par leur âge étaient les plus dignes de gouverner ; et entre les pères eux-mêmes, les plus courageux, les plus robustes furent pris pour {\itshape rois}, afin de conduire les autres, et d’assurer leur résistance contre leurs clients mutinés\footnote{Ces rois des aristocraties ne doivent pas être confondus avec les {\itshape monarques}. ({\itshape Note du Traducteur.})}.\par
Lorsque par la suite des temps, l’intelligence des plébéiens se développa, ils revinrent de l’opinion qu’ils s’étaient formée de l’héroïsme et de la noblesse, et comprirent qu’ils étaient hommes aussi bien que les nobles. Ils voulurent donc entrer aussi dans l’ordre des citoyens. Comme la souveraineté devait avec le temps être étendue à tout le peuple, la Providence permit que les plébéiens rivalisassent longtemps avec les nobles de piété et de  religion, dans ces longues luttes qu’ils soutenaient contre eux, avant d’avoir part au droit des auspices, et à tous les droits publics et privés, qui en étaient regardés comme autant de dépendances. Ainsi le zèle même du peuple pour la religion le conduisait à la souveraineté civile. C’est en cela que le peuple romain surpassa tous les autres, c’est par là qu’il mérita d’être le {\itshape peuple roi}. L’ordre naturel se mêlant ainsi de plus en plus à l’ordre civil, on vit naître les républiques populaires. Mais comme tout devait s’y ramener à l’urne du sort ou à la balance, la Providence empêcha que le hasard ou la fatalité n’y régnât en ordonnant que le cens y serait la règle des honneurs, et qu’ainsi les hommes industrieux, économes et prévoyants plutôt que les prodigues ou les indolents, que les hommes généreux et magnanimes plutôt que ceux dont l’âme est rétrécie par le besoin, qu’en un mot les riches doués de quelque vertu, ou de quelque image de vertu, plutôt que les pauvres remplis de vices dont ils ne savent point rougir, fussent regardés comme les plus dignes de gouverner, comme les meilleurs\footnote{Le peuple pris en général veut la justice. Lorsque le peuple tout entier constitue la cité, il fait des lois justes, c’est-à-dire {\itshape généralement bonnes}. Si donc, comme le dit Aristote, de bonnes lois sont des volontés sans passion, en d’autres termes, des volontés dignes du {\itshape sage}, du {\itshape héros de la morale} qui commande aux passions, c’est dans les républiques populaires que naquit la philosophie ; la nature même de ces républiques conduisait la philosophie à former le sage, et dans ce but à chercher la vérité. Les secours de la philosophie furent ainsi substitues par la Providence à ceux de la religion. Au défaut des {\itshape sentiments} religieux qui faisaient pratiquer la vertu aux hommes, les {\itshape réflexions} de la philosophie leur apprirent à considérer la vertu en elle-même, de sorte que, s’ils n’étaient pas vertueux, ils surent du moins rougir du vice.À la suite de la philosophie naquit l’éloquence, mais telle qu’il convient dans des états où se font des lois {\itshape généralement bonnes}, une éloquence passionnée pour la justice, et capable d’enflammer le peuple par des idées de vertu qui le portent à faire de telles lois. Voilà, à ce qu’il semble, le caractère de l’éloquence romaine au temps de Scipion-l’Africain ; mais les états populaires venant à se corrompre, la philosophie suit cette corruption, tombe dans le scepticisme, et se met, par un écart de la science, à calomnier la vérité. De là naît une fausse éloquence, prête à soutenir le pour et le contre sur tous les sujets. ({\itshape Vico.})}.\par
 Lorsque les citoyens, ne se contentant plus de trouver dans les richesses des moyens de distinction, voulurent en faire des instruments de puissance, alors, comme les vents furieux agitent la mer, ils troublèrent les républiques par la guerre civile, les jetèrent dans un désordre universel, et d’un état de liberté les firent tomber dans la pire des tyrannies ; je veux dire, dans l’anarchie. À cette affreuse maladie sociale, la Providence applique les trois grands remèdes dont nous allons parler. D’abord il s’élève du milieu des peuples, un homme tel qu’Auguste, qui y établit la monarchie. Les lois, les institutions sociales fondées par la liberté populaire n’ont point suffi à la régler ; le monarque devient maître par la force des armes de ces lois, de ces institutions. La forme même de la monarchie retient la volonté du monarque tout infinie qu’est sa puissance, dans les limites de l’ordre naturel, parce que son gouvernement n’est ni tranquille ni durable, s’il ne sait point satisfaire ses peuples  sous le rapport de la religion et de la liberté naturelle.\par
Si la Providence ne trouve point un tel remède au-dedans, elle le fait venir du dehors. Le peuple corrompu était devenu {\itshape par la nature} esclave de ses passions effrénées, du luxe, de la mollesse, de l’avarice, de l’envie, de l’orgueil et du faste. Il devient esclave {\itshape par une loi du droit des gens} qui résulte de sa nature même ; et il est assujetti à des peuples {\itshape meilleurs}, qui le soumettent par les armes. En quoi nous voyons briller deux lumières qui éclairent l’ordre naturel ; d’abord : {\itshape qui ne peut se gouverner lui-même se laissera gouverner par un autre qui en sera plus capable.} Ensuite : {\itshape ceux-là gouverneront toujours le monde qui sont d’une nature meilleure.}\par
Mais si les peuples restent longtemps livrés à l’anarchie, s’ils ne s’accordent pas à prendre un des leurs pour monarque, s’ils ne sont point conquis par une nation meilleure qui les sauve en les soumettant ; alors au dernier des maux, la Providence applique un remède extrême. Ces hommes se sont accoutumés à ne penser qu’à l’intérêt privé ; au milieu de la plus grande foule, ils vivent dans une profonde solitude d’âme et de volonté. Semblables aux bêtes sauvages, on peut à peine en trouver deux qui s’accordent, chacun suivant son plaisir ou son caprice. C’est pourquoi les factions les plus obstinées, les guerres civiles les plus acharnées changeront les cités en forêts et les forêts en repaires d’hommes, et les siècles couvriront de  la rouille de la barbarie leur ingénieuse malice et leur subtilité perverse. En effet ils sont devenus plus féroces par la {\itshape barbarie réfléchie}, qu’ils ne l’avaient été par {\itshape celle de nature}. La seconde montrait une férocité généreuse dont on pouvait se défendre ou par la force ou par la fuite ; l’autre barbarie est jointe à une lâche férocité, qui au milieu des caresses et des embrassements en veut aux biens et à la vie de l’ami le plus cher. Guéris par un si terrible remède, les peuples deviennent comme engourdis et stupides, ne connaissent plus les raffinements, les plaisirs ni le faste, mais seulement les choses les plus nécessaires à la vie. Le petit nombre d’hommes qui restent à la fin, se trouvant dans l’abondance des choses nécessaires, redeviennent naturellement sociables ; l’antique simplicité des premiers âges reparaissant parmi eux, ils connaissent de nouveau la religion, la véracité, la bonne foi, qui sont les bases naturelles de la justice, et qui font la beauté, la grâce éternelle de l’ordre établi par la Providence.\par
Après l’observation si simple que nous venons de faire sur l’histoire du genre humain, quand nous n’aurions point pour l’appuyer tout ce que nous en ont appris les philosophes et les historiens, les grammairiens et les jurisconsultes, on pourrait dire avec certitude que c’est bien là la grande cité des nations fondée et gouvernée par Dieu même. On a élevé jusqu’au ciel comme de sages législateurs les Lycurgue, les Solon, les décemvirs, parce qu’on a  cru jusqu’ici qu’ils avaient foulé par leurs institutions les trois cités les plus illustres, celles qui brillèrent de tout l’éclat des vertus civiles ; et pourtant, que sont Athènes, Sparte et Rome pour la durée et pour l’étendue, en comparaison de cette république de l’univers, fondée sur des institutions qui tirent de leur corruption même la forme nouvelle qui peut seule en assurer la perpétuité ? Ne devons-nous pas y reconnaître le conseil d’une sagesse supérieure à celle de l’homme ? Dion Cassius assimile la loi à un tyran, la coutume à un roi. Mais la sagesse divine n’a pas besoin de la force des lois ; elle aime mieux nous conduire par les coutumes que nous observons librement, puisque les suivre, c’est suivre notre nature. Sans doute {\itshape les hommes ont fait eux-mêmes le monde social}, c’est le principe incontestable de la science nouvelle ; mais ce monde n’en est pas moins sorti d’une intelligence qui souvent s’écarte des fins particulières que les hommes s’étaient proposées, qui leur est quelquefois contraire et toujours supérieure. Ces fins bornées sont pour elle des moyens d’atteindre les fins plus nobles, qui assurent le salut de la race humaine sur cette terre. Ainsi les hommes veulent jouir du plaisir brutal, au risque de perdre les enfants qui naîtront, et il en résulte la sainteté des mariages, première origine des familles. Les pères de famille veulent abuser du pouvoir paternel qu’ils ont étendu sur les clients, et la cité prend naissance. Les corps souverains des nobles veulent appesantir leur souveraineté sur les plébéiens, et ils  subissent la servitude des lois, qui établissent la liberté populaire. Les peuples libres {\itshape veulent} secouer le frein des lois, et ils tombent sous la sujétion des monarques. Les monarques {\itshape veulent} avilir leurs sujets en les livrant aux vices et à la dissolution, par lesquels ils croient assurer leur trône ; et ils les disposent à supporter le joug de nations plus courageuses. Les nations {\itshape tendent} par la corruption à se diviser, à se détruire elles-mêmes, et de leurs débris dispersés dans les solitudes, elles renaissent, et se renouvellent, semblables au phénix de la fable. — Qui put faire tout cela ? ce fut sans doute l’{\itshape esprit}, puisque les hommes le firent avec intelligence. Ce ne fut point la {\itshape fatalité}, puisqu’ils le firent avec choix. Ce ne fut point le {\itshape hasard}, puisque les mêmes faits se renouvelant produisent régulièrement les mêmes résultats.\par
Ainsi se trouvent réfutés par le fait Épicure, et ses partisans, Hobbes et Machiavel, qui abandonnent le monde au hasard. Zénon et Spinoza le sont aussi, eux qui livrent le monde à la fatalité. Au contraire nous établissons avec les philosophes politiques, dont le prince est le divin Platon, que {\itshape c’est la providence qui règle les choses humaines}. Pufendorf méconnaît cette providence ; Selden la suppose ; Grotius en veut rendre son système indépendant. Mais les jurisconsultes romains l’ont prise pour premier principe du droit naturel.\par
On a pleinement démontré dans cet ouvrage que les premiers gouvernements du monde, fondés sur  la croyance en une providence, ont eu la religion pour leur {\itshape forme entière}, et qu’elle fut la seule base de l’état de famille. La religion fut encore le fondement principal des gouvernements héroïques. Elle fut pour les peuples un moyen de parvenir aux gouvernements populaires. Enfin, lorsque la marche des sociétés s’arrêta dans la monarchie, elle devint comme le rempart, comme le bouclier des princes. Si la religion se perd parmi les peuples, il ne leur reste plus de moyen de vivre en société ; ils perdent à la fois le lien, le fondement, le rempart de l’état social, la {\itshape forme même} de peuple sans laquelle ils ne peuvent exister. Que Bayle voie maintenant s’il est possible qu’{\itshape il existe réellement des sociétés sans aucune connaissance de Dieu} ! et Polybe, s’il est vrai, comme il l’a dit, qu’{\itshape on n’aura plus besoin de religion, quand les hommes seront philosophes}. Les religions au contraire peuvent seules exciter les peuples à faire {\itshape par sentiment} des actions vertueuses. Les {\itshape théories} des philosophes relativement à la vertu fournissent seulement des motifs à l’éloquence pour enflammer le sentiment, et le porter à suivre le devoir\footnote{Mais il est une différence essentielle entre la vraie religion et les fausses. La première nous porte par la grâce aux actions vertueuses pour atteindre un bien infini et éternel, qui ne peut tomber sous les sens ; c’est ici l’intelligence qui commande aux sens des actions vertueuses. Au contraire dans les fausses religions qui nous proposent pour cette vie et pour l’autre des biens bornés et périssables, tels que les plaisirs du corps, ce sont les sens qui excitent l’âme à bien agir. ({\itshape Vico.})}.\par
La Providence se fait sentir à nous d’une manière  bien frappante dans le respect et l’admiration que tous les savants ont eus jusqu’ici pour la sagesse de l’antiquité, et dans leur ardent désir d’en chercher et d’en pénétrer les mystères. Ce sentiment n’était que l’instinct qui portait tous les hommes éclairés à admirer, à respecter la sagesse infinie de Dieu, à vouloir s’unir avec elle ; sentiment qui a été dépravé par la vanité des savants et par celle des nations (axiomes 3 et 4.)\par
On peut donc conclure de tout ce qui s’est dit dans cet ouvrage, que la Science nouvelle porte nécessairement avec elle le goût de la piété, et que sans la religion il n’est point de véritable sagesse.
\chapterclose

\chapterclose


\chapteropen
\chapter[{Addition au second livre. Explication historique de la Mythologie}]{Addition au second livre. \\
{\itshape Explication historique de la Mythologie}}

\chaptercont
\noindent ({\itshape Voyez} l’Appendice du Discours, p. LX.)\par
 Lorsque l’idée d’une puissance supérieure, maîtresse du ciel et armée de la foudre, a été personnifiée par les premiers hommes sous le nom de {\scshape Jupiter}, la seconde divinité qu’ils se créent est le symbole, l’expression poétique du mariage. {\scshape Junon} est sœur et femme de Jupiter, parce que les premiers mariages consacrés par les auspices eurent lieu entre frères et sœurs. Du mot Ἥρα, Junon, viennent ceux de ἥρως, héros, Ἡρακλῆς, Hercule, ἔρως, amour, {\itshape hereditas}, etc. Junon impose à Hercule de grands travaux ; cette phrase traduite de la langue héroïque en langue vulgaire signifie, que la piété accompagnée de la sainteté des mariages, forme les hommes aux grandes vertus.\par
{\scshape Diane} est le symbole de la vie plus pure que menèrent les premiers hommes depuis l’institution des mariages solennels. Elle cherche les ténèbres pour s’unir à Endymion. Elle punit Actéon d’avoir violé la religion des eaux sacrées (qui avec le feu constituent la solennité des mariages). Couvert de l’eau qu’elle lui a jetée, {\itshape lymphatus}, devenu {\itshape cerf}, c’est-à-dire le plus timide des animaux, il est déchiré par ses propres chiens, autrement dit, par ses remords. Les nymphes de la déesse, {\itshape nymphæ} ou {\itshape lymphæ}, ne sont autre chose que les eaux pures et cachées dont elle écarte le profane Actéon, {\itshape puri latices}, de {\itshape latere}.\par
Après l’institution des auspices et du mariage vient celle des sépultures ; après Jupiter, Junon et Diane, naissent les dieux {\scshape Manes}. φύλαξ, {\itshape cippus}, signifient tombeau ; de là {\itshape ceppo}, en italien, arbre généalogique, φυλή, tribu, {\itshape filius} (et par {\itshape filus}, et {\itshape temen, subtemen}), {\itshape stemmata}, généalogie, lignes généalogiques. La grossièreté des premiers monuments funéraires qui marquaient à la fois la possession des terres, et la  perpétuité des familles, donna lieu aux métaphores de {\itshape stirps}, de {\itshape propago}, de {\itshape lignage}. Les enfants des fondateurs de la société humaine pouvaient donc se dire {\itshape duro robore nati}, ou fils de la terre, géants, {\itshape ingenui} ({\itshape quasi indè geniti}), aborigènes, αυτοχθονες. — {\itshape Humanitas, ab humando.}\par
{\scshape Apollon} est le dieu de la lumière, de la lumière sociale, qui environne les héros nés des mariages solennels, des unions consacrées par les auspices. Aussi préside-t-il à la divination, à la {\itshape muse}, qu’Homère définit la science du bien et du mal. Apollon poursuit Daphné, symbole de l’humanité encore errante, mais c’est pour l’amener à la vie sédentaire et à la civilisation ; elle implore l’aide des dieux (qui président aux auspices et à l’hyménée). Elle devient laurier, plante qui conserve sa verdure en se renouvelant par ses légitimes rejetons, et jouit ainsi que son divin amant d’une éternelle jeunesse.\par
Dans l’état de famille, les fruits spontanés de la terre ne suffisant plus, les hommes mettent le feu aux forêts et commencent à cultiver la terre. Ils sèment le froment dont les grains brûlés leur ont semblé une nourriture agréable. Voilà le grand travail d’Hercule, c’est-à-dire, de l’héroïsme antique. Les serpents qu’étouffe Hercule au berceau, l’hydre, le lion de Némée, le tigre de Bacchus, la chimère de Bellérophon, le dragon de Cadmus, et celui des Hespérides, sont autant de métaphores que l’indigence du langage força les premiers hommes d’employer pour désigner {\itshape la terre}. Le serpent qui dans l’Iliade dévore les huit petits oiseaux avec leur mère est interprété par Calchas comme signifiant {\itshape la terre troyenne}. En effet les hommes durent se représenter la terre comme un grand dragon couvert d’écailles, c’est-à-dire d’épines ; comme une hydre sortie des eaux (du déluge), et dont les têtes, dont les forêts renaissent à mesure qu’elles sont coupées ; la peau changeante de cette hydre passe du noir au vert, et prend ensuite la couleur de l’or. Les dents du serpent que Cadmus enfonce dans la terre expriment poétiquement les instruments de bois durci dont on se servit pour le labourage avant l’usage du fer (comme {\itshape dente tenaci} pour une ancre, dans Virgile). Enfin Cadmus devient lui-même serpent ; les Latins auraient dit en terme de droit, {\itshape fundus factus est}.\par
Les pommes d’or de la fable ne sont autres que les épis ; le blé fut le premier or du monde. Entre les avantages de la haute fortune dont il est déchu, Job rappelle qu’il mangeait du pain de froment. On donnait du grain pour récompense aux soldats victorieux, {\itshape adorea}. [Le nom d’{\itshape or} passa ensuite aux belles laines. Sans parler de la toison d’or des Argonautes, Atrée se plaint dans Homère de ce que Thyeste lui a volé ses {\itshape brebis d’or}.  Le même poète donne toujours aux rois l’épithète de πολύμηλους, riches en troupeaux. Les anciens Latins appelaient le patrimoine, {\itshape pecunia, à pecude}. Chez les Grecs le même mot, μῆλον, signifie pomme et troupeau, peut-être parce qu’on attachait un grand prix à ce fruit.] L’or du premier âge n’étant plus un métal, on conçoit le rameau de Proserpine dont parle Virgile, et tous les trésors que roulaient dans leurs eaux le Nil, le Pactole, le Gange et le Tage.\par
Les premiers essais de l’agriculture furent exprimés symboliquement par trois nouveaux dieux, savoir : {\scshape Vulcain}, le feu qui avait fécondé la terre ; {\itshape Saturne}, ainsi nommé de {\itshape sata}, semences [ce qui explique pourquoi l’âge de Saturne du Latium, répond à l’âge d’or des Grecs] ; en troisième lieu {\scshape Cybèle}, ou la terre cultivée. On la représente ordinairement assise sur un lion, symbole de la terre qui n’est pas encore domptée par la culture. La même divinité fut pour les Romains {\scshape Vesta}, déesse des cérémonies sacrées. En effet le premier sens du mot {\itshape colere} fut {\itshape cultiver la terre} ; la terre fut le premier autel, l’agriculture fut le premier culte. Ce culte consista originairement à mettre le feu aux forêts et à immoler sur les terres cultivées les vagabonds, les impies qui en franchissaient les limites sacrées, {\itshape Saturni hostiæ}. Vesta, toujours armée de la religion farouche des premiers âges, continua de garder le feu et le froment. Les noces se célébraient {\itshape aquâ, igni et farre} ; les noces appelées {\itshape nuptiæ confarreatæ} devinrent particulières aux prêtres, mais dans l’origine il n’y avait eu que des familles de prêtres. — Les combats livrés par les pères de famille aux vagabonds qui envahissaient leurs terres, donnèrent lieu à la création du dieu {\scshape Mars}.\par
Mais les héros reçoivent ceux qui se présentent en suppliants. La comparaison des deux classes d’hommes qui composent ainsi la société naissante, fait naître l’idée de {\scshape Vénus}, déesse de la beauté civile, de la noblesse. {\itshape Honestas} signifie à la fois noblesse, beauté et vertu. Les enfants, nés hors les mariages solennels, étaient légalement parlant, des {\itshape monstres}.\par
Mais les plébéiens prétendent bientôt au droit des mariages qui entraîne tous les droits civils. On distingue alors Vénus patricienne et Vénus plébéienne : la première est traînée par des ciguës, l’autre par des colombes, symbole de la faiblesse, et pour cette raison souvent opposées par les poètes, à l’aigle, à l’oiseau de Jupiter. Les prétentions des plébéiens sont marquées par les fables d’Ixion, amoureux de Junon ; de Tantale toujours altéré au milieu des eaux ; de Marsyas et de Linus qui défient Apollon au combat du chant, c’est-à-dire qui lui disputent le privilège des auspices ({\itshape cancre}, chanter et prédire.) Le succès ne répond pas toujours à leurs efforts. Phaéton est précipité du char du soleil,  Hercule étouffe Antée, Ulysse tue Irus, et punit les amants de Pénélope. Mais selon une autre tradition Pénélope, se livre à eux, comme Pasiphaé à son taureau (les plébéiens obtiennent le privilège des mariages solennels), et de ces unions criminelles résultent des {\itshape monstres}, tels que Pan et le Minotaure. Hercule s’effémine et file sous Iole et Omphale ; il se souille du sang de Nessus, entre en fureur et expire.\par
La révolution qui termine cette lutte est aussi exprimée par le symbole de {\scshape Minerve}. Vulcain fend la tête de Jupiter, d’où sort la déesse, {\itshape minuit caput}, étymologie de {\itshape Minerva. Caput} signifie la tête, et la partie la plus élevée, {\itshape celle qui domine}. Les Latins dirent toujours {\itshape capitis deminutio} pour {\itshape changement d’état} ; Minerve substitue l’état civil à l’état de famille. Plus tard on donna un sens métaphysique à cette fable de la naissance de Minerve, et on y vit la découverte la plus sublime de la philosophie, savoir, que l’idée éternelle est engendrée en Dieu par Dieu même, tandis que les idées créées sont produites par Dieu dans l’intelligence humaine.\par
La transaction qui termine cette révolution, est caractérisée par {\scshape Mercure}, qui, dans l’orgueil du langage aristocratique, {\itshape porte aux hommes les messages des dieux……}\par


\begin{raggedleft}FIN.\end{raggedleft}
\chapterclose

\chapterclose

 


% at least one empty page at end (for booklet couv)
\ifbooklet
  \pagestyle{empty}
  \clearpage
  % 2 empty pages maybe needed for 4e cover
  \ifnum\modulo{\value{page}}{4}=0 \hbox{}\newpage\hbox{}\newpage\fi
  \ifnum\modulo{\value{page}}{4}=1 \hbox{}\newpage\hbox{}\newpage\fi


  \hbox{}\newpage
  \ifodd\value{page}\hbox{}\newpage\fi
  {\centering\color{rubric}\bfseries\noindent\large
    Hurlus ? Qu’est-ce.\par
    \bigskip
  }
  \noindent Des bouquinistes électroniques, pour du texte libre à participation libre,
  téléchargeable gratuitement sur \href{https://hurlus.fr}{\dotuline{hurlus.fr}}.\par
  \bigskip
  \noindent Cette brochure a été produite par des éditeurs bénévoles.
  Elle n’est pas faîte pour être possédée, mais pour être lue, et puis donnée.
  Que circule le texte !
  En page de garde, on peut ajouter une date, un lieu, un nom ; pour suivre le voyage des idées.
  \par

  Ce texte a été choisi parce qu’une personne l’a aimé,
  ou haï, elle a en tous cas pensé qu’il partipait à la formation de notre présent ;
  sans le souci de plaire, vendre, ou militer pour une cause.
  \par

  L’édition électronique est soigneuse, tant sur la technique
  que sur l’établissement du texte ; mais sans aucune prétention scolaire, au contraire.
  Le but est de s’adresser à tous, sans distinction de science ou de diplôme.
  Au plus direct ! (possible)
  \par

  Cet exemplaire en papier a été tiré sur une imprimante personnelle
   ou une photocopieuse. Tout le monde peut le faire.
  Il suffit de
  télécharger un fichier sur \href{https://hurlus.fr}{\dotuline{hurlus.fr}},
  d’imprimer, et agrafer ; puis de lire et donner.\par

  \bigskip

  \noindent PS : Les hurlus furent aussi des rebelles protestants qui cassaient les statues dans les églises catholiques. En 1566 démarra la révolte des gueux dans le pays de Lille. L’insurrection enflamma la région jusqu’à Anvers où les gueux de mer bloquèrent les bateaux espagnols.
  Ce fut une rare guerre de libération dont naquit un pays toujours libre : les Pays-Bas.
  En plat pays francophone, par contre, restèrent des bandes de huguenots, les hurlus, progressivement réprimés par la très catholique Espagne.
  Cette mémoire d’une défaite est éteinte, rallumons-la. Sortons les livres du culte universitaire, cherchons les idoles de l’époque, pour les briser.
\fi

\ifdev % autotext in dev mode
\fontname\font — \textsc{Les règles du jeu}\par
(\hyperref[utopie]{\underline{Lien}})\par
\noindent \initialiv{A}{lors là}\blindtext\par
\noindent \initialiv{À}{ la bonheur des dames}\blindtext\par
\noindent \initialiv{É}{tonnez-le}\blindtext\par
\noindent \initialiv{Q}{ualitativement}\blindtext\par
\noindent \initialiv{V}{aloriser}\blindtext\par
\Blindtext
\phantomsection
\label{utopie}
\Blinddocument
\fi
\end{document}
