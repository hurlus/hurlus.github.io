%%%%%%%%%%%%%%%%%%%%%%%%%%%%%%%%%
% LaTeX model https://hurlus.fr %
%%%%%%%%%%%%%%%%%%%%%%%%%%%%%%%%%

% Needed before document class
\RequirePackage{pdftexcmds} % needed for tests expressions
\RequirePackage{fix-cm} % correct units

% Define mode
\def\mode{a4}

\newif\ifaiv % a4
\newif\ifav % a5
\newif\ifbooklet % booklet
\newif\ifcover % cover for booklet

\ifnum \strcmp{\mode}{cover}=0
  \covertrue
\else\ifnum \strcmp{\mode}{booklet}=0
  \booklettrue
\else\ifnum \strcmp{\mode}{a5}=0
  \avtrue
\else
  \aivtrue
\fi\fi\fi

\ifbooklet % do not enclose with {}
  \documentclass[french,twoside]{book} % ,notitlepage
  \usepackage[%
    papersize={105mm, 297mm},
    inner=12mm,
    outer=12mm,
    top=20mm,
    bottom=15mm,
    marginparsep=0pt,
  ]{geometry}
  \usepackage[fontsize=9.5pt]{scrextend} % for Roboto
\else\ifav
  \documentclass[french,twoside]{book} % ,notitlepage
  \usepackage[%
    a5paper,
    inner=25mm,
    outer=15mm,
    top=15mm,
    bottom=15mm,
    marginparsep=0pt,
  ]{geometry}
  \usepackage[fontsize=12pt]{scrextend}
\else% A4 2 cols
  \documentclass[twocolumn]{report}
  \usepackage[%
    a4paper,
    inner=15mm,
    outer=10mm,
    top=25mm,
    bottom=18mm,
    marginparsep=0pt,
  ]{geometry}
  \setlength{\columnsep}{20mm}
  \usepackage[fontsize=9.5pt]{scrextend}
\fi\fi

%%%%%%%%%%%%%%
% Alignments %
%%%%%%%%%%%%%%
% before teinte macros

\setlength{\arrayrulewidth}{0.2pt}
\setlength{\columnseprule}{\arrayrulewidth} % twocol
\setlength{\parskip}{0pt} % classical para with no margin
\setlength{\parindent}{1.5em}

%%%%%%%%%%
% Colors %
%%%%%%%%%%
% before Teinte macros

\usepackage[dvipsnames]{xcolor}
\definecolor{rubric}{HTML}{800000} % the tonic 0c71c3
\def\columnseprulecolor{\color{rubric}}
\colorlet{borderline}{rubric!30!} % definecolor need exact code
\definecolor{shadecolor}{gray}{0.95}
\definecolor{bghi}{gray}{0.5}

%%%%%%%%%%%%%%%%%
% Teinte macros %
%%%%%%%%%%%%%%%%%
%%%%%%%%%%%%%%%%%%%%%%%%%%%%%%%%%%%%%%%%%%%%%%%%%%%
% <TEI> generic (LaTeX names generated by Teinte) %
%%%%%%%%%%%%%%%%%%%%%%%%%%%%%%%%%%%%%%%%%%%%%%%%%%%
% This template is inserted in a specific design
% It is XeLaTeX and otf fonts

\makeatletter % <@@@


\usepackage{blindtext} % generate text for testing
\usepackage[strict]{changepage} % for modulo 4
\usepackage{contour} % rounding words
\usepackage[nodayofweek]{datetime}
% \usepackage{DejaVuSans} % seems buggy for sffont font for symbols
\usepackage{enumitem} % <list>
\usepackage{etoolbox} % patch commands
\usepackage{fancyvrb}
\usepackage{fancyhdr}
\usepackage{float}
\usepackage{fontspec} % XeLaTeX mandatory for fonts
\usepackage{footnote} % used to capture notes in minipage (ex: quote)
\usepackage{framed} % bordering correct with footnote hack
\usepackage{graphicx}
\usepackage{lettrine} % drop caps
\usepackage{lipsum} % generate text for testing
\usepackage[framemethod=tikz,]{mdframed} % maybe used for frame with footnotes inside
\usepackage{pdftexcmds} % needed for tests expressions
\usepackage{polyglossia} % non-break space french punct, bug Warning: "Failed to patch part"
\usepackage[%
  indentfirst=false,
  vskip=1em,
  noorphanfirst=true,
  noorphanafter=true,
  leftmargin=\parindent,
  rightmargin=0pt,
]{quoting}
\usepackage{ragged2e}
\usepackage{setspace} % \setstretch for <quote>
\usepackage{tabularx} % <table>
\usepackage[explicit]{titlesec} % wear titles, !NO implicit
\usepackage{tikz} % ornaments
\usepackage{tocloft} % styling tocs
\usepackage[fit]{truncate} % used im runing titles
\usepackage{unicode-math}
\usepackage[normalem]{ulem} % breakable \uline, normalem is absolutely necessary to keep \emph
\usepackage{verse} % <l>
\usepackage{xcolor} % named colors
\usepackage{xparse} % @ifundefined
\XeTeXdefaultencoding "iso-8859-1" % bad encoding of xstring
\usepackage{xstring} % string tests
\XeTeXdefaultencoding "utf-8"
\PassOptionsToPackage{hyphens}{url} % before hyperref, which load url package

% TOTEST
% \usepackage{hypcap} % links in caption ?
% \usepackage{marginnote}
% TESTED
% \usepackage{background} % doesn’t work with xetek
% \usepackage{bookmark} % prefers the hyperref hack \phantomsection
% \usepackage[color, leftbars]{changebar} % 2 cols doc, impossible to keep bar left
% \usepackage[utf8x]{inputenc} % inputenc package ignored with utf8 based engines
% \usepackage[sfdefault,medium]{inter} % no small caps
% \usepackage{firamath} % choose firasans instead, firamath unavailable in Ubuntu 21-04
% \usepackage{flushend} % bad for last notes, supposed flush end of columns
% \usepackage[stable]{footmisc} % BAD for complex notes https://texfaq.org/FAQ-ftnsect
% \usepackage{helvet} % not for XeLaTeX
% \usepackage{multicol} % not compatible with too much packages (longtable, framed, memoir…)
% \usepackage[default,oldstyle,scale=0.95]{opensans} % no small caps
% \usepackage{sectsty} % \chapterfont OBSOLETE
% \usepackage{soul} % \ul for underline, OBSOLETE with XeTeX
% \usepackage[breakable]{tcolorbox} % text styling gone, footnote hack not kept with breakable


% Metadata inserted by a program, from the TEI source, for title page and runing heads
\title{\textbf{ La Cité de Dieu }}
\date{426}
\author{Augustin (354, 430)}
\def\elbibl{Augustin (354, 430). 426. \emph{La Cité de Dieu}}
\def\elsource{Saint Augustin, {\itshape La Cité de Dieu de saint Augustin}, traduction nouvelle avec une introduction et des notes par M. Émile Saisset, Paris, Charpentier, 1855, 4 vol. HTML : \href{http://www.abbaye-saint-benoit.ch/saints%5Caugustin%5Ccitededieu%5C}{\dotuline{Œuvres complètes de saint Augustin}}\footnote{\href{http://www.abbaye-saint-benoit.ch/saints%5Caugustin%5Ccitededieu%5C}{\url{http://www.abbaye-saint-benoit.ch/saints%5Caugustin%5Ccitededieu%5C}}}.}

% Default metas
\newcommand{\colorprovide}[2]{\@ifundefinedcolor{#1}{\colorlet{#1}{#2}}{}}
\colorprovide{rubric}{red}
\colorprovide{silver}{lightgray}
\@ifundefined{syms}{\newfontfamily\syms{DejaVu Sans}}{}
\newif\ifdev
\@ifundefined{elbibl}{% No meta defined, maybe dev mode
  \newcommand{\elbibl}{Titre court ?}
  \newcommand{\elbook}{Titre du livre source ?}
  \newcommand{\elabstract}{Résumé\par}
  \newcommand{\elurl}{http://oeuvres.github.io/elbook/2}
  \author{Éric Lœchien}
  \title{Un titre de test assez long pour vérifier le comportement d’une maquette}
  \date{1566}
  \devtrue
}{}
\let\eltitle\@title
\let\elauthor\@author
\let\eldate\@date


\defaultfontfeatures{
  % Mapping=tex-text, % no effect seen
  Scale=MatchLowercase,
  Ligatures={TeX,Common},
}


% generic typo commands
\newcommand{\astermono}{\medskip\centerline{\color{rubric}\large\selectfont{\syms ✻}}\medskip\par}%
\newcommand{\astertri}{\medskip\par\centerline{\color{rubric}\large\selectfont{\syms ✻\,✻\,✻}}\medskip\par}%
\newcommand{\asterism}{\bigskip\par\noindent\parbox{\linewidth}{\centering\color{rubric}\large{\syms ✻}\\{\syms ✻}\hskip 0.75em{\syms ✻}}\bigskip\par}%

% lists
\newlength{\listmod}
\setlength{\listmod}{\parindent}
\setlist{
  itemindent=!,
  listparindent=\listmod,
  labelsep=0.2\listmod,
  parsep=0pt,
  % topsep=0.2em, % default topsep is best
}
\setlist[itemize]{
  label=—,
  leftmargin=0pt,
  labelindent=1.2em,
  labelwidth=0pt,
}
\setlist[enumerate]{
  label={\bf\color{rubric}\arabic*.},
  labelindent=0.8\listmod,
  leftmargin=\listmod,
  labelwidth=0pt,
}
\newlist{listalpha}{enumerate}{1}
\setlist[listalpha]{
  label={\bf\color{rubric}\alph*.},
  leftmargin=0pt,
  labelindent=0.8\listmod,
  labelwidth=0pt,
}
\newcommand{\listhead}[1]{\hspace{-1\listmod}\emph{#1}}

\renewcommand{\hrulefill}{%
  \leavevmode\leaders\hrule height 0.2pt\hfill\kern\z@}

% General typo
\DeclareTextFontCommand{\textlarge}{\large}
\DeclareTextFontCommand{\textsmall}{\small}

% commands, inlines
\newcommand{\anchor}[1]{\Hy@raisedlink{\hypertarget{#1}{}}} % link to top of an anchor (not baseline)
\newcommand\abbr[1]{#1}
\newcommand{\autour}[1]{\tikz[baseline=(X.base)]\node [draw=rubric,thin,rectangle,inner sep=1.5pt, rounded corners=3pt] (X) {\color{rubric}#1};}
\newcommand\corr[1]{#1}
\newcommand{\ed}[1]{ {\color{silver}\sffamily\footnotesize (#1)} } % <milestone ed="1688"/>
\newcommand\expan[1]{#1}
\newcommand\foreign[1]{\emph{#1}}
\newcommand\gap[1]{#1}
\renewcommand{\LettrineFontHook}{\color{rubric}}
\newcommand{\initial}[2]{\lettrine[lines=2, loversize=0.3, lhang=0.3]{#1}{#2}}
\newcommand{\initialiv}[2]{%
  \let\oldLFH\LettrineFontHook
  % \renewcommand{\LettrineFontHook}{\color{rubric}\ttfamily}
  \IfSubStr{QJ’}{#1}{
    \lettrine[lines=4, lhang=0.2, loversize=-0.1, lraise=0.2]{\smash{#1}}{#2}
  }{\IfSubStr{É}{#1}{
    \lettrine[lines=4, lhang=0.2, loversize=-0, lraise=0]{\smash{#1}}{#2}
  }{\IfSubStr{ÀÂ}{#1}{
    \lettrine[lines=4, lhang=0.2, loversize=-0, lraise=0, slope=0.6em]{\smash{#1}}{#2}
  }{\IfSubStr{A}{#1}{
    \lettrine[lines=4, lhang=0.2, loversize=0.2, slope=0.6em]{\smash{#1}}{#2}
  }{\IfSubStr{V}{#1}{
    \lettrine[lines=4, lhang=0.2, loversize=0.2, slope=-0.5em]{\smash{#1}}{#2}
  }{
    \lettrine[lines=4, lhang=0.2, loversize=0.2]{\smash{#1}}{#2}
  }}}}}
  \let\LettrineFontHook\oldLFH
}
\newcommand{\labelchar}[1]{\textbf{\color{rubric} #1}}
\newcommand{\milestone}[1]{\autour{\footnotesize\color{rubric} #1}} % <milestone n="4"/>
\newcommand\name[1]{#1}
\newcommand\orig[1]{#1}
\newcommand\orgName[1]{#1}
\newcommand\persName[1]{#1}
\newcommand\placeName[1]{#1}
\newcommand{\pn}[1]{\IfSubStr{-—–¶}{#1}% <p n="3"/>
  {\noindent{\bfseries\color{rubric}   ¶  }}
  {{\footnotesize\autour{ #1}  }}}
\newcommand\reg{}
% \newcommand\ref{} % already defined
\newcommand\sic[1]{#1}
\newcommand\surname[1]{\textsc{#1}}
\newcommand\term[1]{\textbf{#1}}

\def\mednobreak{\ifdim\lastskip<\medskipamount
  \removelastskip\nopagebreak\medskip\fi}
\def\bignobreak{\ifdim\lastskip<\bigskipamount
  \removelastskip\nopagebreak\bigskip\fi}

% commands, blocks
\newcommand{\byline}[1]{\bigskip{\RaggedLeft{#1}\par}\bigskip}
\newcommand{\bibl}[1]{{\RaggedLeft{#1}\par\bigskip}}
\newcommand{\biblitem}[1]{{\noindent\hangindent=\parindent   #1\par}}
\newcommand{\dateline}[1]{\medskip{\RaggedLeft{#1}\par}\bigskip}
\newcommand{\labelblock}[1]{\medbreak{\noindent\color{rubric}\bfseries #1}\par\mednobreak}
\newcommand{\salute}[1]{\bigbreak{#1}\par\medbreak}
\newcommand{\signed}[1]{\bigbreak\filbreak{\raggedleft #1\par}\medskip}

% environments for blocks (some may become commands)
\newenvironment{borderbox}{}{} % framing content
\newenvironment{citbibl}{\ifvmode\hfill\fi}{\ifvmode\par\fi }
\newenvironment{docAuthor}{\ifvmode\vskip4pt\fontsize{16pt}{18pt}\selectfont\fi\itshape}{\ifvmode\par\fi }
\newenvironment{docDate}{}{\ifvmode\par\fi }
\newenvironment{docImprint}{\vskip6pt}{\ifvmode\par\fi }
\newenvironment{docTitle}{\vskip6pt\bfseries\fontsize{18pt}{22pt}\selectfont}{\par }
\newenvironment{msHead}{\vskip6pt}{\par}
\newenvironment{msItem}{\vskip6pt}{\par}
\newenvironment{titlePart}{}{\par }


% environments for block containers
\newenvironment{argument}{\itshape\parindent0pt}{\vskip1.5em}
\newenvironment{biblfree}{}{\ifvmode\par\fi }
\newenvironment{bibitemlist}[1]{%
  \list{\@biblabel{\@arabic\c@enumiv}}%
  {%
    \settowidth\labelwidth{\@biblabel{#1}}%
    \leftmargin\labelwidth
    \advance\leftmargin\labelsep
    \@openbib@code
    \usecounter{enumiv}%
    \let\p@enumiv\@empty
    \renewcommand\theenumiv{\@arabic\c@enumiv}%
  }
  \sloppy
  \clubpenalty4000
  \@clubpenalty \clubpenalty
  \widowpenalty4000%
  \sfcode`\.\@m
}%
{\def\@noitemerr
  {\@latex@warning{Empty `bibitemlist' environment}}%
\endlist}
\newenvironment{quoteblock}% may be used for ornaments
  {\begin{quoting}}
  {\end{quoting}}

% table () is preceded and finished by custom command
\newcommand{\tableopen}[1]{%
  \ifnum\strcmp{#1}{wide}=0{%
    \begin{center}
  }
  \else\ifnum\strcmp{#1}{long}=0{%
    \begin{center}
  }
  \else{%
    \begin{center}
  }
  \fi\fi
}
\newcommand{\tableclose}[1]{%
  \ifnum\strcmp{#1}{wide}=0{%
    \end{center}
  }
  \else\ifnum\strcmp{#1}{long}=0{%
    \end{center}
  }
  \else{%
    \end{center}
  }
  \fi\fi
}


% text structure
\newcommand\chapteropen{} % before chapter title
\newcommand\chaptercont{} % after title, argument, epigraph…
\newcommand\chapterclose{} % maybe useful for multicol settings
\setcounter{secnumdepth}{-2} % no counters for hierarchy titles
\setcounter{tocdepth}{5} % deep toc
\markright{\@title} % ???
\markboth{\@title}{\@author} % ???
\renewcommand\tableofcontents{\@starttoc{toc}}
% toclof format
% \renewcommand{\@tocrmarg}{0.1em} % Useless command?
% \renewcommand{\@pnumwidth}{0.5em} % {1.75em}
\renewcommand{\@cftmaketoctitle}{}
\setlength{\cftbeforesecskip}{\z@ \@plus.2\p@}
\renewcommand{\cftchapfont}{}
\renewcommand{\cftchapdotsep}{\cftdotsep}
\renewcommand{\cftchapleader}{\normalfont\cftdotfill{\cftchapdotsep}}
\renewcommand{\cftchappagefont}{\bfseries}
\setlength{\cftbeforechapskip}{0em \@plus\p@}
% \renewcommand{\cftsecfont}{\small\relax}
\renewcommand{\cftsecpagefont}{\normalfont}
% \renewcommand{\cftsubsecfont}{\small\relax}
\renewcommand{\cftsecdotsep}{\cftdotsep}
\renewcommand{\cftsecpagefont}{\normalfont}
\renewcommand{\cftsecleader}{\normalfont\cftdotfill{\cftsecdotsep}}
\setlength{\cftsecindent}{1em}
\setlength{\cftsubsecindent}{2em}
\setlength{\cftsubsubsecindent}{3em}
\setlength{\cftchapnumwidth}{1em}
\setlength{\cftsecnumwidth}{1em}
\setlength{\cftsubsecnumwidth}{1em}
\setlength{\cftsubsubsecnumwidth}{1em}

% footnotes
\newif\ifheading
\newcommand*{\fnmarkscale}{\ifheading 0.70 \else 1 \fi}
\renewcommand\footnoterule{\vspace*{0.3cm}\hrule height \arrayrulewidth width 3cm \vspace*{0.3cm}}
\setlength\footnotesep{1.5\footnotesep} % footnote separator
\renewcommand\@makefntext[1]{\parindent 1.5em \noindent \hb@xt@1.8em{\hss{\normalfont\@thefnmark . }}#1} % no superscipt in foot
\patchcmd{\@footnotetext}{\footnotesize}{\footnotesize\sffamily}{}{} % before scrextend, hyperref


%   see https://tex.stackexchange.com/a/34449/5049
\def\truncdiv#1#2{((#1-(#2-1)/2)/#2)}
\def\moduloop#1#2{(#1-\truncdiv{#1}{#2}*#2)}
\def\modulo#1#2{\number\numexpr\moduloop{#1}{#2}\relax}

% orphans and widows
\clubpenalty=9996
\widowpenalty=9999
\brokenpenalty=4991
\predisplaypenalty=10000
\postdisplaypenalty=1549
\displaywidowpenalty=1602
\hyphenpenalty=400
% Copied from Rahtz but not understood
\def\@pnumwidth{1.55em}
\def\@tocrmarg {2.55em}
\def\@dotsep{4.5}
\emergencystretch 3em
\hbadness=4000
\pretolerance=750
\tolerance=2000
\vbadness=4000
\def\Gin@extensions{.pdf,.png,.jpg,.mps,.tif}
% \renewcommand{\@cite}[1]{#1} % biblio

\usepackage{hyperref} % supposed to be the last one, :o) except for the ones to follow
\urlstyle{same} % after hyperref
\hypersetup{
  % pdftex, % no effect
  pdftitle={\elbibl},
  % pdfauthor={Your name here},
  % pdfsubject={Your subject here},
  % pdfkeywords={keyword1, keyword2},
  bookmarksnumbered=true,
  bookmarksopen=true,
  bookmarksopenlevel=1,
  pdfstartview=Fit,
  breaklinks=true, % avoid long links
  pdfpagemode=UseOutlines,    % pdf toc
  hyperfootnotes=true,
  colorlinks=false,
  pdfborder=0 0 0,
  % pdfpagelayout=TwoPageRight,
  % linktocpage=true, % NO, toc, link only on page no
}

\makeatother % /@@@>
%%%%%%%%%%%%%%
% </TEI> end %
%%%%%%%%%%%%%%


%%%%%%%%%%%%%
% footnotes %
%%%%%%%%%%%%%
\renewcommand{\thefootnote}{\bfseries\textcolor{rubric}{\arabic{footnote}}} % color for footnote marks

%%%%%%%%%
% Fonts %
%%%%%%%%%
\usepackage[]{roboto} % SmallCaps, Regular is a bit bold
% \linespread{0.90} % too compact, keep font natural
\newfontfamily\fontrun[]{Roboto Condensed Light} % condensed runing heads
\ifav
  \setmainfont[
    ItalicFont={Roboto Light Italic},
  ]{Roboto}
\else\ifbooklet
  \setmainfont[
    ItalicFont={Roboto Light Italic},
  ]{Roboto}
\else
\setmainfont[
  ItalicFont={Roboto Italic},
]{Roboto Light}
\fi\fi
\renewcommand{\LettrineFontHook}{\bfseries\color{rubric}}
% \renewenvironment{labelblock}{\begin{center}\bfseries\color{rubric}}{\end{center}}

%%%%%%%%
% MISC %
%%%%%%%%

\setdefaultlanguage[frenchpart=false]{french} % bug on part


\newenvironment{quotebar}{%
    \def\FrameCommand{{\color{rubric!10!}\vrule width 0.5em} \hspace{0.9em}}%
    \def\OuterFrameSep{\itemsep} % séparateur vertical
    \MakeFramed {\advance\hsize-\width \FrameRestore}
  }%
  {%
    \endMakeFramed
  }
\renewenvironment{quoteblock}% may be used for ornaments
  {%
    \savenotes
    \setstretch{0.9}
    \normalfont
    \begin{quotebar}
  }
  {%
    \end{quotebar}
    \spewnotes
  }


\renewcommand{\headrulewidth}{\arrayrulewidth}
\renewcommand{\headrule}{{\color{rubric}\hrule}}

% delicate tuning, image has produce line-height problems in title on 2 lines
\titleformat{name=\chapter} % command
  [display] % shape
  {\vspace{1.5em}\centering} % format
  {} % label
  {0pt} % separator between n
  {}
[{\color{rubric}\huge\textbf{#1}}\bigskip] % after code
% \titlespacing{command}{left spacing}{before spacing}{after spacing}[right]
\titlespacing*{\chapter}{0pt}{-2em}{0pt}[0pt]

\titleformat{name=\section}
  [block]{}{}{}{}
  [\vbox{\color{rubric}\large\raggedleft\textbf{#1}}]
\titlespacing{\section}{0pt}{0pt plus 4pt minus 2pt}{\baselineskip}

\titleformat{name=\subsection}
  [block]
  {}
  {} % \thesection
  {} % separator \arrayrulewidth
  {}
[\vbox{\large\textbf{#1}}]
% \titlespacing{\subsection}{0pt}{0pt plus 4pt minus 2pt}{\baselineskip}

\ifaiv
  \fancypagestyle{main}{%
    \fancyhf{}
    \setlength{\headheight}{1.5em}
    \fancyhead{} % reset head
    \fancyfoot{} % reset foot
    \fancyhead[L]{\truncate{0.45\headwidth}{\fontrun\elbibl}} % book ref
    \fancyhead[R]{\truncate{0.45\headwidth}{ \fontrun\nouppercase\leftmark}} % Chapter title
    \fancyhead[C]{\thepage}
  }
  \fancypagestyle{plain}{% apply to chapter
    \fancyhf{}% clear all header and footer fields
    \setlength{\headheight}{1.5em}
    \fancyhead[L]{\truncate{0.9\headwidth}{\fontrun\elbibl}}
    \fancyhead[R]{\thepage}
  }
\else
  \fancypagestyle{main}{%
    \fancyhf{}
    \setlength{\headheight}{1.5em}
    \fancyhead{} % reset head
    \fancyfoot{} % reset foot
    \fancyhead[RE]{\truncate{0.9\headwidth}{\fontrun\elbibl}} % book ref
    \fancyhead[LO]{\truncate{0.9\headwidth}{\fontrun\nouppercase\leftmark}} % Chapter title, \nouppercase needed
    \fancyhead[RO,LE]{\thepage}
  }
  \fancypagestyle{plain}{% apply to chapter
    \fancyhf{}% clear all header and footer fields
    \setlength{\headheight}{1.5em}
    \fancyhead[L]{\truncate{0.9\headwidth}{\fontrun\elbibl}}
    \fancyhead[R]{\thepage}
  }
\fi

\ifav % a5 only
  \titleclass{\section}{top}
\fi

\newcommand\chapo{{%
  \vspace*{-3em}
  \centering % no vskip ()
  {\Large\addfontfeature{LetterSpace=25}\bfseries{\elauthor}}\par
  \smallskip
  {\large\eldate}\par
  \bigskip
  {\Large\selectfont{\eltitle}}\par
  \bigskip
  {\color{rubric}\hline\par}
  \bigskip
  {\Large TEXTE LIBRE À PARTICPATION LIBRE\par}
  \centerline{\small\color{rubric} {hurlus.fr, tiré le \today}}\par
  \bigskip
}}

\newcommand\cover{{%
  \thispagestyle{empty}
  \centering
  {\LARGE\bfseries{\elauthor}}\par
  \bigskip
  {\Large\eldate}\par
  \bigskip
  \bigskip
  {\LARGE\selectfont{\eltitle}}\par
  \vfill\null
  {\color{rubric}\setlength{\arrayrulewidth}{2pt}\hline\par}
  \vfill\null
  {\Large TEXTE LIBRE À PARTICPATION LIBRE\par}
  \centerline{{\href{https://hurlus.fr}{\dotuline{hurlus.fr}}, tiré le \today}}\par
}}

\begin{document}
\pagestyle{empty}
\ifbooklet{
  \cover\newpage
  \thispagestyle{empty}\hbox{}\newpage
  \cover\newpage\noindent Les voyages de la brochure\par
  \bigskip
  \begin{tabularx}{\textwidth}{l|X|X}
    \textbf{Date} & \textbf{Lieu}& \textbf{Nom/pseudo} \\ \hline
    \rule{0pt}{25cm} &  &   \\
  \end{tabularx}
  \newpage
  \addtocounter{page}{-4}
}\fi

\thispagestyle{empty}
\ifaiv
  \twocolumn[\chapo]
\else
  \chapo
\fi
{\it\elabstract}
\bigskip
\makeatletter\@starttoc{toc}\makeatother % toc without new page
\bigskip

\pagestyle{main} % after style

  \section[{Livre premier. Les Goths à Rome}]{Livre premier. \\
Les Goths à Rome}\renewcommand{\leftmark}{Livre premier. \\
Les Goths à Rome}

\noindent En écrivant cet ouvrage dont vous m’avez suggéré la première pensée, Marcellinus, mon très cher fils, et que je vous ai promis d’exécuter, je viens défendre la Cité de Dieu contre ceux qui préfèrent à son fondateur leurs fausses divinités ; je viens montrer cette cité toujours glorieuse, soit qu’on la considère dans son pèlerinage à travers le temps, vivant de foi au milieu des incrédules, soit qu’on la contemple dans la stabilité du séjour éternel, qu’elle attend présentement avec patience, jusqu’à ce que la patience se change en force au jour de la victoire suprême et de la parfaite paix. Cette entreprise est, à la vérité, grande et difficile, mais Dieu est notre appui. Aussi bien de quelle force n’aurai-je pas besoin pour persuader aux superbes que l’humilité possède une vertu supérieure qui nous élève, non par une insolence toute humaine, mais par une grâce divine, au-dessus des grandeurs terrestres toujours mobiles et chancelantes ? C’est le sens de ces paroles de l’Écriture, où le roi et le fondateur de la cité que nous célébrons, découvrant aux hommes sa loi, déclare que « Dieu résiste aux superbes et donne sa grâce aux humbles ». Cette conduite toute divine, l’orgueil humain prétend l’imiter, et il aime à s’entendre donner cet éloge :\par
 {\itshape « Tu sais pardonner aux humbles et dompter les superbes. »} \par
C’est pourquoi nous aurons plus d’une fois à parler dans cet ouvrage, autant que notre plan le comportera, de cette cité terrestre dévorée du désir de dominer et qui est elle-même esclave de sa convoitise, tandis qu’elle croit être la maîtresse des nations.\par
\subsection[{Chapitre premier}]{Chapitre premier}

\begin{argument}\noindent Beaucoup d’adversaires du Christ épargnés par les barbares, à la prise de Rome, par respect pour le Christ.
\end{argument}

\noindent C’est contre cet esprit d’orgueil que j’entreprends de défendre la Cité de Dieu. Parmi ses ennemis, plusieurs, il est vrai, abandonnant leur erreur impie, deviennent ses citoyens ; mais un grand nombre sont enflammés contre elle d’une si grande haine et poussent si loin l’ingratitude pour les bienfaits signalés de son Rédempteur, qu’ils ne se souviennent plus qu’il leur serait impossible de se servir pour l’attaquer de leur langue sacrilège, s’ils n’avaient trouvé dans les saints lieux un asile pour échapper au fer ennemi et sauver une vie dont ils ont la folie de s’enorgueillir.\par
Ne sont-ce pas ces mêmes Romains, que les barbares ont épargnés par respect pour le Christ, qui sont aujourd’hui les adversaires déclarés du nom du Christ ? J’en puis attester les sépulcres des martyrs et les basiliques des Apôtres qui, dans cet horrible désastre de Rome, ont également ouvert leurs portes aux enfants de l’Église et aux païens. C’est là que venait expirer la fureur des meurtriers ; c’est là que les victimes qu’ils voulaient sauver étaient conduites pour être à couvert de la violence d’ennemis plus féroces, qui n’étaient pas touchés de la même compassion. En effet, lorsque ces furieux, qui partout ailleurs s’étaient montrés impitoyables, arrivaient à ces lieux sacrés, où ce qui leur était permis autre part par le droit de la guerre leur avait été défendu, l’on voyait se ralentir cette ardeur brutale de répandre le sang et ce désir avare de faire des prisonniers. Et c’est ainsi que plusieurs ont échappé à la mort, qui maintenant se font les détracteurs de la religion chrétienne, imputant au Christ les maux que Rome a soufferts, et n’attribuant qu’à leur bonne fortune la conservation de leur vie, dont ils sont pourtant redevables au respect des barbares pour le Christ. Ne devraient-ils pas plutôt, s’ils étaient un peu raisonnables, attribuer les maux qu’ils ont éprouvés à cette Providence divine qui a coutume de châtier les méchants pour les amender, et qui se plaît même quelquefois à exercer par ces sortes d’afflictions la patience des gens de bien, afin qu’étant éprouvés et purifiés, elle les fasse passer à une meilleure vie, ou les laisse encore sur la terre pour l’accomplissement de ses fins ? Ne devraient-ils pas reconnaître comme un des fruits du christianisme cette modération inouïe des barbares, d’ailleurs cruels et sanguinaires, qui les ont épargnés contre la loi de la guerre en considération du Christ, soit dans les lieux profanes, soit dans les lieux consacrés, lesquels semblaient avoir été choisis à dessein vastes et spacieux pour étendre la miséricorde à un plus grand nombre ? Et dès lors, que ne rendent-ils grâce à Dieu, et que n’adorent-ils sincèrement son nom pour éviter le feu éternel, eux qui se sont faussement servis de ce nom sacré pour éviter une mort temporelle ? Tout au contraire, parmi ceux que vous voyez aujourd’hui insulter avec tant d’insolence aux serviteurs du Christ, il en est plusieurs qui n’auraient jamais échappé au carnage, s’ils ne s’étaient déguisés en serviteurs du Christ. Et maintenant, dans leur superbe ingratitude et leur démence impie, ces cœurs pervers s’élèvent contre Je nom de chrétien, au risque d’être ensevelis dans des ténèbres éternelles, après s’être fait de ce nom une protection frauduleuse pour conserver la jouissance de quelques jours passagers.
\subsection[{Chapitre II}]{Chapitre II}

\begin{argument}\noindent Il est sans exemple dans les guerres antérieures que les vainqueurs aient épargné le vaincu par respect pour les dieux.
\end{argument}

\noindent On a écrit l’histoire d’un grand nombre de guerres qui se sont faites avant la fondation de Rome et depuis son origine et ses conquêtes ; eh bien ! qu’on en trouve une seule où les ennemis, après la prise d’une ville, aient épargné ceux qui avaient cherché un refuge dans le temple de leurs dieux ! qu’on cite un seul chef des barbares qui ait ordonné à ses soldats de ne frapper aucun homme réfugié dans tel ou tel lieu sacré ! Énée ne vit-il pas Priam traîné au pied des autels et\par
 {\itshape « Souillant de son sang les autels et les feux qu’il avait lui-même consacrés ? »} \par
Est-ce que Diomède et Ulysse, après avoir massacré les gardiens de la citadelle, n’osèrent pas\par
 {\itshape « Saisir l’effigie sacrée de Pallas, et de leurs mains ensanglantées profaner les bandelettes virginales de la déesse ? »} \par
Ce qu’ajoute Virgile n’est pas vrai :\par
 {\itshape « Dès ce moment disparut sans retour l’espérance des Grecs. »} \par
C’est depuis lors, en effet, qu’ils furent vainqueurs ; c’est depuis lors qu’ils détruisirent Troie par le fer et par le feu ; c’est depuis lors qu’ils égorgèrent Priam abrité près des autels. La perte de Minerve ne fut donc pas la cause de la chute de Troie. Minerve elle-même, pour périr, n’avait-elle rien perdu ? Elle avait, dira-t-on, perdu ses gardes. Il est vrai, c’est après le massacre de ses gardes qu’elle fut enlevée par les grecs. Preuve évidente que ce n’étaient pas les Troyens qui étaient protégés par la statue, mais la statue qui était protégée par les Troyens. Comment donc l’adorait-on pour qu’elle fût la sauvegarde de Troie et de ses enfants, elle qui n’a pas su défendre ses défenseurs ?
\subsection[{Chapitre III}]{Chapitre III}

\begin{argument}\noindent Les Romains s’imaginant que les dieux pénates qui n’avaient pu protéger Troie leur seraient d’efficaces protecteurs.
\end{argument}

\noindent Voilà les dieux à qui les Romains s’estimaient heureux d’avoir confié la protection de leur ville. Pitoyable renversement d’esprit ! Ils s’emportent contre nous, quand nous parlons ainsi de leurs dieux, et ils s’emportent si peu contre leurs écrivains, qui pourtant en parlent de même, qu’ils les font apprendre à prix d’argent et prodiguent les plus magnifiques honneurs aux maîtres que l’État salarie pour les enseigner. Ouvrez Virgile, qu’on fait lire aux petits enfants comme un grand poète, le plus illustre et le plus excellent qui existe ; Virgile, dont on fait couler les vers dans ces jeunes âmes, pour qu’elles n’en perdent jamais le souvenir, suivant le précepte d’Horace :\par
 {\itshape « Un vase garde longtemps l’odeur de la première liqueur qu’on y a versée. »} \par
Lisez Virgile, et vous le verrez introduire Junon ; l’ennemie des Troyens, qui pour animer contre eux Éole, roi des vents, s’écrie :\par
 {\itshape « Une nation qui m’est odieuse navigue sur la mer Tyrrhénienne, portant en Italie Troie et ses Pénates vaincus. »} \par
Des hommes sages devaient-ils mettre Rome sous la protection de ces Pénates vaincus, pour l’empêcher d’être vaincue à son tour ? On dira que Junon parle ainsi comme une femme en colère, qui ne sait trop ce qu’elle dit. Soit ; mais Énée, tant de fois appelé le Pieux, ne s’exprime-t-il pas en ces termes :\par
 {\itshape « Panthus, fils d’Othrys, prêtre de Pallas et d’Apollon, tenant dans ses mains les vases sacrés et ses dieux vaincus, entraîne avec lui son petit-fils et court éperdu vers mon palais. »} \par
Ces dieux, qu’il n’hésite pas à appeler vaincus, ne paraissent-ils pas mis sous la protection d’Énée, bien plus qu’Énée sous la leur, lorsque Hector lui dit\par
 {\itshape « Troie commet à ta garde les objets de son culte et ses Pénates. »} \par
Si donc Virgile ne fait point difficulté, en parlant de pareils dieux, de les appeler vaincus et de les montrer protégés par un homme qui les sauve du mieux qu’il peut, n’y a-t-il pas de la démence à croire qu’on ait sagement fait de confier Rome à de tels défenseurs, et à s’imaginer qu’elle n’aurait pu être saccagée si elle ne les eût perdus ? Que dis-je ! adorer des dieux vaincus comme des gardiens et des protecteurs, n’est-ce pas déclarer qu’on les tient, non pour des divinités bienfaisantes, mais pour des présages de malheurs ? N’est-il pas plus sage, en effet, de penser qu’ils auraient péri depuis longtemps, si Rome ne les eût conservés de tout son pouvoir, que de s’imaginer que Rome n’eût point été prise, s’ils n’eussent auparavant péri ? Pensez-y un instant, et vous verrez combien il est ridicule de prétendre qu’on eût été invincible sous la garde de défenseurs vaincus. La ruine des dieux, disent-ils, a fait celle de Rome : n’est-il pas plus croyable qu’il a suffi pour perdre Rome d’avoir adopté pour protecteurs des dieux condamnés à périr ?\par
Qu’on ne vienne donc pas nous dire que les poètes ont parlé par fiction, quand ils ont fait paraître dans leurs chants des dieux vaincus. Non, c’est la force de la vérité qui a arraché cet aveu à leur bonne foi. Au surplus, nous traiterons ce sujet ailleurs plus à propos et avec le soin et l’étendue convenables ; je reviens maintenant à ces hommes ingrats et blasphémateurs qui imputent au Christ les maux qu’ils souffrent eu juste punition de leur perversité. Ils ne daignent pas se souvenir qu’on leur a fait grâce par respect pour le Christ, et que la langue dont ils se servent dans leur démence sacrilège pour insulter son nom, ils l’ont employée à faire un mensonge pour conserver leur vie. Ils savaient bien la retenir, cette langue, quand réfugiés dans nos lieux sacrés, ils devaient leur salut au nom de chrétiens ; et maintenant, échappés au fer de l’ennemi, ils lancent contre le Christ la haine et la malédiction !
\subsection[{Chapitre IV}]{Chapitre IV}

\begin{argument}\noindent Le temple de Junon au sac de Troie, et les basiliques des Apôtres pendant le sac de Rome.
\end{argument}

\noindent Troie elle-même, cette mère du peuple romain, ne put, comme je l’ai déjà dit, mettre à couvert dans les temples de ses dieux ses propres habitants contre le fer et le feu des Grecs, qui adoraient pourtant les mêmes dieux. Écoutez Virgile :\par
 {\itshape « Dans le temple de Junon, deux gardiens choisis, Phénix et le terrible Ulysse, veillaient à la garde du butin ; on voyait entassés çà et là les trésors dérobés aux temples incendiés des Troyens et les tables des dieux et les cratères d’or et les riches vêtements. À l’entour, debout, se presse une longue troupe d’enfants et de mères tremblantes. »} \par
Ce lieu consacré à une si grande déesse fut évidemment choisi pour servir aux Troyens, non d’asile, mais de prison. Comparez maintenant, je vous prie, ce temple qui n’était pas consacré à un petit dieu, au premier venu du peuple des dieux, mais à la reine des dieux, sœur et femme de Jupiter, comparez ce temple avec les basiliques de nos apôtres. Là, on portait les dépouilles des dieux dont on avait brûlé les temples, non pour les rendre aux vaincus, mais pour les partager entre les vainqueurs ; ici, tout ce qui a été reconnu, même en des lieux profanes, pour appartenir à ces asiles sacrés, y a été rapporté religieusement, avec honneur et avec respect. Là, on perdait sa liberté ; ici, on la conservait. Là, on s’assurait de ses prisonniers ; ici, il était défendu d’en faire. Là, on était traîné par des dominateurs insolents, décidés à vous rendre esclaves ; ici, on était conduit par des ennemis pleins d’humanité, décidés à vous laisser libres. En un mot, du côté de ces Grecs fameux par leur politesse, l’avarice et la superbe semblaient avoir choisi pour demeure le temple de Junon ; du côté des grossiers barbares, la miséricorde et l’humilité habitaient les basiliques du Christ. On dira peut-être que, dans la réalité, les Grecs épargnèrent les temples des dieux troyens, qui étaient aussi leurs dieux, et qu’ils n’eurent pas la cruauté de frapper ou de rendre captifs les malheureux vaincus qui se réfugiaient dans ces lieux sacrés. À ce compte, Virgile aurait fait un tableau de pure fantaisie, à la manière des poètes ; mais point du tout, il a décrit le sac de Troie selon les véritables mœurs de l’antiquité païenne.
\subsection[{Chapitre V}]{Chapitre V}

\begin{argument}\noindent Sentiment de César touchant la coutume universelle de piller les temples dans les villes prises d’assaut.
\end{argument}

\noindent Au rapport de Salluste, qui a la réputation d’un historien véridique, César dépeignait ainsi le sort réservé aux villes prises de vive force, quand il donna son avis dans le sénat sur le sort des complices de Catilina : « On ravit les vierges et les jeunes garçons ; on arrache les enfants des bras de leurs parents ; les mères de famille sont livrées aux outrages des vainqueurs ; on pille les temples et les maisons ; partout le meurtre et l’incendie ; tout est plein d’armes, de cadavres, de sang et de cris plaintifs. » Si César n’eût point parlé des temples, nous croirions que la coutume était d’épargner les demeures des dieux ; or, remarquez bien que les temples des Romains avaient à craindre ces profanations, non pas d’un peuple étranger, mais de Catilina et de ses complices, c’est-à-dire de citoyens romains et des sénateurs les plus illustres ; mais on dira peut-être que c’étaient des hommes perdus et des parricides.
\subsection[{Chapitre VI}]{Chapitre VI}

\begin{argument}\noindent Les Romains eux-mêmes, quand ils prenaient une ville d’assaut, n’avaient point coutume de faire grâce aux vaincus réfugiés dans les temples des dieux.
\end{argument}

\noindent Laissons donc de côté cette infinité de peuples qui se sont fait la guerre et n’ont jamais épargné les vaincus qui se sauvaient dans les temples de leurs dieux : parlons des Romains, de ces Romains dont le plus magnifique éloge est renfermé dans le vers fameux du poète :\par
 {\itshape « Tu sais pardonner aux humbles et dompter les superbes. »} \par
Considérons ce peuple à qui un auteur a rendu ce témoignage, qu’il aimait mieux pardonner une injure que d’en tirer vengeance. Quand ils ont pris et saccagé tant de grandes villes pour étendre leur domination, qu’on nous dise quels temples ils avaient coutume d’excepter pour servir d’asile aux vaincus. S’ils en avaient usé de la sorte, est-ce que leurs historiens en auraient fait mystère ? Mais quelle apparence que des écrivains qui cherchaient avidement l’occasion de louer les Romains eussent passé sous silence des marques si éclatantes et à leurs yeux si admirables de respect envers leurs dieux ! Marcus Marcellus, l’honneur du nom romain, qui prit la célèbre ville de Syracuse, la pleura, dit-on, avant de la saccager, et répandit des larmes pour elle avant que de répandre le sang de ses habitants. Il fit plus : persuadé que les lois de la pudeur doivent être respectées même à l’égard d’un ennemi, il donna l’ordre avant l’assaut de ne violer aucune personne libre. La ville néanmoins fut saccagée avec toutes les horreurs de la guerre, et l’on ne lit nulle part qu’un capitaine si chaste et si clément ait commandé que ceux qui se réfugieraient dans tel ou tel temple eussent la vie sauve. Et certes, si un pareil commandement eût été donné, les historiens ne l’auraient point passé sous silence, eux qui n’ont oublié ni les larmes de Marcellus, ni ses ordres pour protéger la chasteté. Fabius, le vainqueur de Tarente, est loué pour s’être abstenu de toucher aux images des dieux. Un de ses secrétaires lui ayant demandé ce qu’il fallait faire d’un grand nombre de statues tombées sous la main des vainqueurs, il fit une réponse dont la modération est relevée de fine ironie. « Comment sont-elles ? » demanda-t-il. Et sur la réponse qu’on lui fit, qu’elles étaient fort grandes et même armées : « Laissons, dit-il, aux Tarentins leurs dieux irrités. » Puis donc que les historiens romains n’ont pas manqué de nous dire les larmes de celui-ci et le rire de celui-là, la chaste compassion du premier et la modération spirituelle du second, comment auraient-ils gardé le silence, si quelques généraux avaient ordonné de tel ou tel de leurs dieux que l’on ne fit dans son temple ni victimes ni prisonniers ?
\subsection[{Chapitre VII}]{Chapitre VII}

\begin{argument}\noindent Les cruautés qui ont accompagné la prise de Rome doivent être attribuées aux usages de la guerre, tandis que la clémence dont les barbares ont fait preuve vient de la puissance du nom du Christ.
\end{argument}

\noindent Ainsi donc, toutes les calamités qui ont frappé Rome dans cette récente catastrophe, dévastation, meurtre, pillage, incendie, violences, tout doit être imputé aux terribles coutumes de la guerre ; mais ce qui est nouveau, c’est que des barbares se soient adoucis au point de choisir les plus grandes églises pour préserver un plus grand nombre de malheureux, d’ordonner qu’on n’y tuât personne, qu’on n’en fit sortir personne, d’y conduire même plusieurs prisonniers pour les arracher à la mort et à l’esclavage ; et voilà ce qui ne peut être attribué qu’au nom du Christ et à l’influence de la religion nouvelle. Qui ne voit pas une chose si évidente est aveugle ; qui la voit et n’en loue pas Dieu est ingrat ; qui s’oppose à ces louanges est insensé. Loin de moi l’idée qu’aucun homme sage puisse faire honneur de cette clémence aux barbares. Celui qui a jeté l’épouvante dans ces âmes farouches et inhumaines, qui les a contenues, qui les a miraculeusement adoucies, est celui-là même qui a dit, dès longtemps, par la bouche du Prophète : « Je visiterai avec ma verge leurs iniquités, et leurs péchés avec mes fléaux ; mais je ne leur retirerai point ma miséricorde. »
\subsection[{Chapitre VIII}]{Chapitre VIII}

\begin{argument}\noindent Les biens et les maux de la vie sont généralement communs aux bons et aux méchants.
\end{argument}

\noindent Quelqu’un dira : Pourquoi cette miséricorde divine a-t-elle fait aussi sentir ses effets à des impies et à des ingrats ? Pourquoi ? c’est parce qu’elle émane de celui « qui fait chaque jour lever son soleil sur les bons et sur les méchants, et tomber sa pluie sur les justes et sur les injustes ». Si quelques-uns de ces impies, se rendant attentifs à ces marques de bonté, viennent à se repentir et à se détourner des sentiers de l’impiété, il en est d’autres qui, suivant la parole de l’Apôtre, « méprisant les trésors de la bonté et de la longanimité divines, s’amassent par leur dureté et l’impénitence de leur cœur un trésor de colère pour le jour de la colère et de la manifestation du juste châtiment de Dieu qui rendra à chacun selon ses œuvres ». Et cependant, il est toujours vrai de dire que la patience de Dieu invite les méchants au repentir, comme ses châtiments exercent les bons à la résignation, et que sa miséricorde protège doucement les bons, comme sa justice frappe durement les méchants. Il a plu, en effet, à la divine Providence de préparer aux bons, pour la vie future, des biens dont les méchants ne jouiront pas, et aux méchants des maux dont les bons n’auront point à souffrir ; mais quant aux biens et aux maux de cette vie, elle a voulu qu’ils fussent communs aux uns et aux autres, afin qu’on ne désirât point avec trop d’ardeur des biens dont on entre en partage avec les méchants ; et qu’on n’évitât point comme honteux des maux qui souvent éprouvent les bons.\par
Il y a pourtant une très grande différence dans l’usage que les uns et les autres font de ces biens et de ces maux ; car l’homme bon ne se laisse point enivrer par les biens de cette vie, ni abattre par ses disgrâces : le méchant, au contraire, considère la mauvaise fortune comme une très grande peine, parce qu’il s’est laissé corrompre par la bonne. Plus d’une fois cependant Dieu fait paraître plus clairement sa main dans cette distribution des biens et des maux ; et véritablement, si tout péché était frappé dès cette vie d’une punition manifeste, l’on croirait qu’il ne reste plus rien à faire au dernier jugement ; tout comme si Dieu n’infligeait à aucun péché un châtiment visible ; on croirait qu’il n’y a point de Providence. Il en est de même des biens temporels. Si Dieu, par une libéralité toute évidente, ne les accordait à quelques-uns de ceux qui les lui demandent, nous penserions qu’ils ne dépendent point de sa volonté ; et s’il les donnait à tous ceux qui les lui demandent, nous nous accoutumerions à ne le servir qu’en vue de ces récompenses, et le culte que nous lui rendrions n’entretiendrait pas en nous la piété, mais l’avarice et l’intérêt. Or, puisqu’il en est ainsi, il ne faut point s’imaginer, quand les bons et les méchants sont également affligés, qu’il n’y ait point entre eux de différence parce que leur affliction est commune. La différence de ceux qui sont frappés demeure dans la ressemblance des maux qui les frappent ; et pour être exposés aux mêmes tourments, la vertu et le vice ne se confondent pas. Car, comme un même feu fait briller l’or et noircir la paille, comme un même fléau écrase le chaume et purifie le froment, ou encore, comme le marc ne se mêle pas avec l’huile, quoiqu’il soit tiré de l’olive par le même pressoir, ainsi un même malheur, venant à tomber sur les bons et sur les méchants, éprouve, purifie et fait resplendir les uns, tandis qu’il damne, écrase et anéantit les autres. C’est pour cela qu’en une même affliction, les méchants blasphèment contre Dieu, les bons, au contraire, le prient et le bénissent : tant il importe de considérer, non les maux qu’on souffre, mais l’esprit dans lequel on les subit ; car le même mouvement qui tire de la boue une odeur fétide, imprimé à un vase de parfums, en fait sortir les plus douces exhalaisons.
\subsection[{Chapitre IX}]{Chapitre IX}

\begin{argument}\noindent Les sujets de réprimande pour lesquels les gens de bien sont châtiés avec les méchants.
\end{argument}

\noindent Quels maux ont donc souffert les chrétiens, dans ces temps de désolation universelle, qui ne leur soient avantageux, s’ils savent les accepter dans l’esprit de la foi ? Qu’ils considèrent d’abord, en pensant humblement aux péchés qui ont allumé la colère de Dieu et attiré tant de calamités sur le monde, que si leur conduite est meilleure que celle des grands pécheurs et des impies, ils ne sont pas néanmoins tellement purs de toutes fautes qu’ils n’aient besoin, pour les expier, de quelques peines temporelles. En effet, outre qu’il n’y a personne, si louable que soit sa vie, qui ne cède quelquefois à l’attrait charnel de la concupiscence, et qui, sans se précipiter dans les derniers excès du vice et dans le gouffre de l’impiété, parvienne à se garantir de quelques pêchés, ou rares, ou d’autant plus fréquents qu’ils sont plus légers ; quel est celui qui se conduit aujourd’hui comme il le devrait à l’égard de ces méchants dont l’orgueil, l’avarice, les débauches et les impiétés, ont décidé Dieu à répandre la désolation sur la terre, ainsi qu’il en menace les hommes par la bouche de ses prophètes ? En effet, il arrive souvent que, par une dangereuse dissimulation, nous feignons de ne pas voir leurs fautes, pour n’être point obligés de les instruire, de les avertir, de les reprendre et quelquefois même de les corriger, et cela, soit parce que notre paresse ne veut pas s’en donner le soin, soit parce que nous n’avons pas le courage de leur rompre en visière, soit enfin parce que nous craignons de les offenser et par suite de compromettre des biens temporels que notre convoitise veut acquérir ou que notre faiblesse a peur de perdre. Et de la sorte bien que les gens honnêtes aient en horreur la vie des méchants, et qu’à cause de cela ils ne tombent pas dans la damnation réservée aux pécheurs après cette vie ; toutefois, de cela seul qu’ils se sont montrés indulgents pour les vices damnables dont les méchants sont souillés, par la seule crainte de perdre des biens passagers, c’est justement qu’ils sont châtiés avec eux dans le temps, sans être punis comme eux dans l’éternité ; c’est justement qu’ils sentent l’amertume de la vie, pour en avoir trop aimé la douceur et s’être montrés trop doux envers les méchants.\par
Je ne blâme pourtant pas la conduite de ceux qui ne reprennent pas et ne corrigent pas les pécheurs, parce qu’ils attendent une occasion plus favorable, ou parce qu’ils craignent, soit de les rendre pires, soit de les porter à mettre obstacle à la bonne éducation des faibles et aux progrès de la foi ; car alors c’est plutôt l’effet d’une charité prudente que d’un calcul intéressé. Mais le mal est que ceux qui vivent tout autrement que les impies et qui abhorrent leur conduite, leur sont indulgents au lieu de leur être sévères, de peur de s’en faire des ennemis et d’en être traversés dans la possession de biens-fort légitimes, il est vrai, mais auxquels devraient être moins attachés des chrétiens, voyageurs en ce monde et qui font profession de regarder le ciel comme leur patrie. Je ne parle pas seulement de ces personnes naturellement plus faibles, qui sont engagées dans le mariage, ont des enfants ou veulent en avoir, et possèdent des maisons et des serviteurs, de toutes celles enfin à qui l’Apôtre s’adresse, quand il donne des préceptes sur la manière dont les femmes doivent vivre avec leurs maris et les maris avec leurs femmes, sur les devoirs mutuels des pères et des enfants, des maîtres et des serviteurs ; ces personnes, dis-je, ne sont pas les seules qui soient très aises d’acquérir plusieurs biens temporels et très fâchées de les perdre, et qui n’osent par cette raison choquer des hommes dont elles détestent les mœurs ; je parle aussi de celles qui font profession d’une vie plus parfaite, qui ne sont point engagées dans le mariage et se contentent de peu pour leur subsistance ; je dis que celles-là même ne peuvent souvent se résoudre à reprendre les méchants, parce qu’elles craignent de hasarder contre eux leur réputation et leur vie, et redoutent leurs embûches et leurs violences. Et quoique cette crainte et les menaces mêmes des impies n’aillent pas jusqu’à décider ces personnes timides à imiter leurs exemples, c’est cependant une chose déplorable qu’elles n’aient point le courage, en présence de désordres dont la complicité leur ferait horreur, de les frapper d’un blâme qui serait pour plusieurs une correction salutaire. Pourquoi cette réserve ? est-ce afin de conserver leur considération et leur vie pour l’utilité du prochain ? Non, c’est par amour pour leur considération même et pour leur vie ; c’est par cette complaisance dans les paroles flatteuses et dans les opinions du jour, qui fait redouter le jugement du vulgaire, les tourments et la mort de la chair ; en un mot, c’est l’esclavage de l’intérêt personnel qu’on subit, au lieu de s’affranchir par la charité.\par
Voilà donc, ce me semble, une raison d’assez grand poids pour que les bons soient châtiés avec les méchants, lorsqu’il plaît à Dieu de punir par de simples maux temporels les mœurs corrompues des pécheurs. Ils sont châtiés ensemble, non pour mener avec eux une mauvaise vie, mais pour être comme eux, moins qu’eux cependant, attachés à la vie, à cette vie temporelle que les bons devraient mépriser, afin d’entraîner sur leurs pas les méchants blâmés et corrigés au séjour de la vie éternelle. Perd-on l’espoir de s’en faire ainsi des compagnons ? qu’on se résigne alors à les avoir pour ennemis et à les aimer comme tels ; car, tant qu’ils vivent, on ne peut savoir s’ils ne viendront pas à se convertir. Et ceux-là sont encore plus coupables dont parle ainsi le Prophète « Cet homme mourra dans son péché ; mais je demanderai compte de sa vie à qui dut veiller sur lui. » Car ceux qui veillent, c’est-à-dire ceux qui ont dans l’Église la conduite des peuples, sont établis pour faire au péché une guerre implacable. Et il ne faut pas croire cependant que celui-là soit exempt de toute faute, qui, n’ayant pas le caractère de pasteur, se montre indifférent pour la conduite des personnes que le commerce de la vie rapproche de lui, et néglige de les reprendre de peur d’encourir leur disgrâce et de compromettre des intérêts peut-être légitimes, mais dont il est charmé plus qu’il ne convient. Il y a là une faiblesse répréhensible et que Dieu punit justement par des maux temporels. Je signalerai une dernière explication de ces épreuves subies par les justes ; c’est Job qui me la fournit : il est bon que l’âme humaine s’estime à fond ce qu’elle vaut, et qu’elle sache bien si elle a pour Dieu un amour désintéressé.
\subsection[{Chapitre X}]{Chapitre X}

\begin{argument}\noindent Les saints ne perdent rien en perdant les choses temporelles.
\end{argument}

\noindent Pesez bien toutes ces raisons, et dites-moi s’il peut arriver aucun mal aux hommes de foi et de piété qui ne se tourne en bien pour eux. Serait-elle vaine, par hasard, cette parole de l’Apôtre : « Nous savons que tout concourt au bien de ceux qui aiment Dieu » ? — Mais ils ont perdu tout ce qu’ils avaient. Ont-ils perdu la foi, la piété ? Ont-ils perdu les biens de l’homme intérieur, riche devant Dieu ? Voilà l’opulence des chrétiens, comme parle le très opulent apôtre : « C’est une grande richesse que la piété et la modération d’un esprit qui se contente de ce qui suffit. Car nous n’avons rien apporté en ce monde, et il est sans aucun doute que nous ne pouvons aussi en rien emporter. Ayant donc de quoi nous nourrir et de quoi nous couvrir, nous devons être contents. Mais ceux qui veulent devenir riches tombent dans la tentation et dans le piège du diable, et en divers désirs inutiles et pernicieux qui précipitent les hommes dans l’abîme de la perdition et de la damnation. Car l’amour des richesses est la racine de tous les maux, et quelques-uns, pour en avoir été possédés, se sont détournés de la foi et embarrassés en une infinité d’afflictions et de peines. »\par
Ceux donc qui, dans le sac de Rome, ont perdu les richesses de la terre, s’ils les possédaient de la façon que recommande l’Apôtre, pauvres au dehors, riches au dedans, c’est-à-dire s’ils en usaient comme n’en usant pas, ils ont pu dire avec un homme fortement éprouvé, mais nullement vaincu : « Je suis sorti nu du ventre de ma mère, et je retournerai nu dans la terre. Le Seigneur m’avait tout donné, le Seigneur m’a tout ôté. Il n’est arrivé que ce qui lui a plu ; que le nom du Seigneur soit béni ! » Job pensait donc que la volonté du Seigneur était sa richesse, la richesse de son âme, et il ne s’affligeait point de perdre pendant la vie ce qu’il faut nécessairement perdre à la mort. Quant aux âmes plus faibles, qui, sans préférer ces biens terrestres au Christ, avaient pour eux quelque attachement profane, elles ont senti dans la douleur de les perdre le péché de les avoir aimés. Suivant la parole de l’Apôtre, que je rappelais tout à l’heure, elles ont d’autant plus souffert qu’elles avaient donné plus de prise à la douleur en s’embarrassant dans ses voies. Après avoir si longtemps fermé l’oreille aux leçons de la parole divine, il était bon qu’elles fussent rendues attentives à celles de l’expérience ; car lorsque l’Apôtre a dit : « Ceux qui veulent devenir « riches tombent dans la tentation, etc. », ce qu’il blâme dans les richesses, ce n’est pas de les posséder, mais de les convoiter ; aussi donne-t-il ailleurs des règles pour leur usage : « Recommandez », dit-il à Timothée, « aux riches de ce monde de n’être point orgueilleux, de ne mettre point leur confiance dans les richesses incertaines et périssables, mais dans le Dieu vivant qui nous fournit avec abondance tout ce qui est nécessaire à la vie ; ordonnez-leur d’être charitables et bienfaisants, de se rendre riches en bonnes œuvres, de donner l’aumône de bon cœur, de faire part de leurs biens, de se faire un trésor et un fondement solide pour l’avenir, afin d’arriver à la véritable vie. » Ceux qui faisaient un tel usage de leurs biens ont été consolés de pertes légères par de grands bénéfices, et ils ont tiré plus de satisfaction des biens qu’ils ont mis en sûreté, en les employant en aumônes, qu’ils n’ont ressenti de tristesse de ceux qu’ils ont perdus en voulant les retenir par avarice. Tout ce qu’ils n’ont pas eu la force d’enlever à la terre, la terre le leur a ravi pour jamais.\par
Il en est tout autrement de ceux qui ont écouté ce commandement de leur Seigneur : « Ne vous faites point des trésors dans la terre, où la rouille et les vers les dévorent, et où les voleurs les déterrent et les dérobent ; mais faites-vous des trésors dans le ciel, où les voleurs ne peuvent les dérober, ni la rouille et les vers les corrompre ; car, où est votre trésor, là est aussi votre cœur. » Ceux qui ont écouté cette voix ont éprouvé, dans les jours d’affliction, combien ils ont été sages de ne point mépriser le conseil d’un maître si véridique et d’un gardien si puissant et si fidèle de leur trésor. Si plusieurs se sont applaudis d’avoir caché leurs richesses en des lieux que le hasard a préservés pour un jour des atteintes de l’ennemi, quelle joie plus solide et plus sûre d’elle-même ont dû éprouver ceux qui, fidèles à l’avertissement de leur Dieu, ont cherché un asile à jamais inviolable à toutes les atteintes !\par
C’est ainsi que notre cher Paulin, évêque de Noie, de très riche qu’il était, devenu volontairement très pauvre, et d’autant plus opulent en sainteté, quand il fut fait prisonnier des barbares, à la prise de Nole, adressait en son cœur (c’est lui-même qui nous l’a confié) cette prière à Dieu. : « Seigneur, ne permettez pas que je sois torturé pour de l’or et de l’argent ; car où sont toutes mes richesses, vous le savez. » Elles étaient, en effet, aux lieux où nous recommande de les recueillir et de thésauriser le Prophète qui avait prédit au monde toutes ces calamités. Ainsi, ceux qui avaient obéi à leur Seigneur et thésaurisé suivant ses conseils, n’ont pas même perdu leurs richesses terrestres dans cette invasion des barbares ; et pour ceux qui ont eu à se repentir de leur désobéissance, ils ont appris le véritable usage de ces biens, non par une sagesse qui ait prévenu leur perte, mais par l’expérience qui l’a suivie.\par
Mais, dit-on, parmi les bons, il s’en est trouvé plusieurs, même chrétiens, qu’on a mis à la torture pour leur faire livrer leurs biens. Je réponds que le bien qui les rendait bons, ils n’ont pu ni le livrer, ni le perdre. S’ils ont préféré supporter les tourments que de livrer ces richesses, tristes gages d’iniquité, je dis qu’ils n’étaient pas vraiment bons. Ils avaient donc besoin d’être avertis par les souffrances que l’amour de l’or leur a fait subir, de celles que l’amour du Christ doit nous faire surmonter, afin d’apprendre ainsi à aimer celui qui enrichit d’une félicité éternelle les fidèles qui souffrent pour lui, de préférence à l’or et à l’argent, biens misérables qui ne sont pas dignes qu’on souffre pour eux, soit qu’on les conserve par un mensonge, soit qu’on les perde en avouant la vérité. Au surplus, nul dans les tortures n’a perdu le Christ en le confessant ; nul n’a conservé sa fortune qu’en la niant. Aussi, je dirai que les tourments leur étaient peut-être plus utiles, en leur apprenant à aimer un bien qui ne se corrompt pas, que ces biens temporels, dont l’amour ne servait qu’à tourmenter leurs possesseurs d’agitations sans fruit. Mais, dit-on encore, quelques-uns, qui n’avaient aucun trésor à livrer, n’ont pas laissé d’être mis à la torture, parce qu’on ne les en croyait pas sur parole. Je réponds que, s’ils n’avaient rien, ils désiraient peut-être avoir ; ils n’étaient point saintement pauvres dans leur volonté ; il a donc fallu leur montrer que ce ne sont point les richesses, mais la passion d’en avoir, qui rendent dignes de pareils châtiments. En est-il maintenant qui, ayant embrassé une vie meilleure, ne possédant ni or ni argent cachés, aient été torturés à cause des trésors qu’on leur supposait ? Je n’en sais rien, mais en serait-il ainsi, je dirais encore que celui qui, au milieu des tourments, confessait la pauvreté sainte, celui-là, certes, confessait Jésus-Christ. Or, un confesseur de la pauvreté sainte a bien pu être méconnu par les barbares, mais il n’a pu souffrir sans recevoir du ciel le prix de sa vertu.\par
J’entends dire que plusieurs chrétiens ont eu à subir une longue famine. Mais c’est encore une épreuve que les vrais fidèles ont tournée à leur avantage en la souffrant pieusement. Pour ceux, en effet, que la faim a tués, elle les a délivrés des maux de la vie, comme aurait pu faire une maladie ; pour ceux qu’elle n’a pas tués, elle leur a appris à mener une vie plus sobre et à faire des jeûnes plus longs.
\subsection[{Chapitre XI}]{Chapitre XI}

\begin{argument}\noindent S’il importe que la vie temporelle dure un peu plus ou un peu moins.
\end{argument}

\noindent On ajoute : Plusieurs chrétiens ont été massacrés, plusieurs ont été emportés par divers genres de morts affreuses. Si c’est là un malheur, il est commun à tous les hommes ; du moins, suis-je assuré qu’il n’est mort personne qui ne dût mourir un jour. Or, la mort égale la plus longue vie à la plus courte : car, ce qui n’est plus n’est ni pire, ni meilleur, ni plus court, ni plus long. Et qu’importe le genre de mort, puisqu’on ne meurt pas deux fois ? Puisqu’il n’est point de mortel que le cours des choses de ce monde ne menace d’un nombre infini de morts, je demande si, dans l’incertitude où l’on est de celle qu’il faudra endurer, il ne vaut pas mieux en souffrir une seule et mourir que de vivre en les craignant toutes. Je sais que notre lâcheté préfère vivre sous la crainte de tant de morts que de mourir une fois pour n’en plus redouter aucune ; mais autre chose est l’aveugle horreur de notre chair infirme et la conviction éclairée de notre raison. Il n’y a pas de mauvaise mort après une bonne vie ; ce qui rend la mort mauvaise, c’est l’événement qui la suit. Ainsi donc qu’une créature faite pour la mort vienne à mourir, il ne faut pas s’en mettre en peine ; mais où va-t-elle après la mort ? Voilà la question. Or, puisque les chrétiens savent que la mort du bon pauvre de l’Évangile, au milieu des chiens qui léchaient ses plaies, est meilleure que celle du mauvais riche dans la pourpre, je demande en quoi ces horribles trépas ont pu nuire à ceux qui sont morts, s’ils avaient bien vécu ?
\subsection[{Chapitre XII}]{Chapitre XII}

\begin{argument}\noindent Le défaut de sépulture ne cause aux chrétiens aucun dommage.
\end{argument}

\noindent Je sais que dans cet épouvantable entassement de cadavres plusieurs chrétiens n’ont pu être ensevelis. Eh bien ! est-ce un si grand sujet de crainte pour des hommes de foi, qui ont appris de l’Évangile que la dent des bêtes féroces n’empêchera pas la résurrection des corps, et qu’il n’y a pas un seul cheveu de leur tête qui doive périr ? Si les traitements que l’ennemi fait subir à nos cadavres pouvaient faire obstacle à la vie future, la vérité nous dirait-elle : « Ne craignez pas ceux qui tuent le corps, et ne peuvent tuer l’âme » ? À moins qu’il ne se rencontre un homme assez insensé pour prétendre que si les meurtriers du corps ne sont point à redouter avant la mort, ils deviennent redoutables après la mort, en ce qu’ils peuvent priver le corps de sépulture. À ce compte, elle serait fausse cette parole du Christ : « Ne craignez point ceux qui tuent le corps et ne peuvent rien faire de plus contre vous » ; car il resterait à sévir contre nos cadavres. Mais loin de nous de soupçonner de mensonge la parole de vérité ! S’il est dit, en effet, que les meurtriers font quelque chose lorsqu’ils tuent, c’est que le corps ressent le coup dont il est frappé ; une fois mort, il n’y a plus rien à faire contre lui, parce qu’il a perdu tout sentiment. Il est donc vrai que la terre n’a pas recouvert le corps d’un grand nombre de chrétiens ; mais aucune puissance n’a pu leur ravir le ciel, ni cette terre elle-même que remplit de sa présence le maître de la création et de la résurrection des hommes. On m’opposera cette parole du Psalmiste : « Ils ont exposé les corps morts de vos serviteurs pour servir de nourriture aux oiseaux du ciel et les chairs de vos saints pour être la proie des bêtes de la terre. Ils ont répandu leur sang comme l’eau autour de Jérusalem, et il n’y avait personne qui leur donnât la sépulture. » Mais le Prophète a plutôt pour but de faire ressortir la cruauté des meurtriers que les souffrances des victimes. Ce tableau de la mort paraît horrible aux yeux des hommes ; « mais elle est précieuse aux yeux du Seigneur, la mort des saints ». Ainsi donc, toute cette pompe des funérailles, sépulture choisie, cortège funèbre, ce sont là des consolations pour les vivants, mais non un soulagement véritable pour les morts. Autrement, si une riche sépulture était de quelque secours aux impurs, il faudrait croire que c’est un obstacle à la gloire du juste d’être enseveli simplement ou de ne pas l’être du tout. Certes, cette multitude de serviteurs qui suivait le corps du riche voluptueux de l’Évangile composait aux yeux des hommes une pompe magnifique, mais elles furent bien autrement éclatantes aux yeux de Dieu les funérailles de ce pauvre couvert d’ulcères que les anges portèrent, non dans un tombeau de marbre, mais dans le sein d’Abraham.\par
Je vois sourire les adversaires contre qui j’ai entrepris de défendre la Cité de Dieu. Et cependant leurs philosophes ont souvent marqué du mépris pour les soins de la sépulture. Plus d’une fois aussi, des armées entières, décidées à mourir pour leur patrie terrestre, se sont mises peu en peine de ce que deviendraient leurs corps et à quelles bêtes ils serviraient de pâture. C’est ce qui fait applaudir ce vers d’un poète :\par
 {\itshape « Le ciel couvre celui qui n’a point de tombeau. »} \par
Pourquoi donc tirer un sujet d’insulte contre les chrétiens de ces corps non ensevelis ? N’a-t-il pas été promis aux fidèles que tous leurs membres et leur propre chair sortiront un jour de la terre et du plus profond abîme des éléments, pour leur être rendus dans leur première intégrité ?
\subsection[{Chapitre XIII}]{Chapitre XIII}

\begin{argument}\noindent Pourquoi il faut ensevelir les corps des fidèles.
\end{argument}

\noindent Toutefois il ne faut pas négliger et abandonner la dépouille des morts, surtout les corps des justes et des fidèles qui ont servi d’instrument et d’organe au Saint-Esprit pour toutes sortes de bonnes œuvres. Si la robe d’un père ou son anneau ou telle autre chose semblable sont d’autant plus précieux à ses enfants que leur affection est plus grande, à plus forte raison devons-nous prendre soin du corps de ceux que nous aimons, car le corps est uni à l’homme d’une façon plus étroite et plus intime qu’aucun vêtement ; ce n’est point un secours ou un ornement étranger, c’est un élément de notre nature. Aussi voyons-nous qu’on a rendu aux justes des premiers temps ces suprêmes devoirs de piété, qu’on a célébré leurs funérailles et pourvu à leur sépulture, et qu’eux-mêmes durant leur vie ont donné des ordres à leurs enfants pour faire ensevelir ou transférer leurs dépouilles. Je citerai Tobie qui s’est rendu agréable à Dieu, au témoignage de l’ange, en faisant ensevelir les morts. Notre-Seigneur lui-même, qui devait ressusciter au troisième jour, approuve hautement et veut qu’on loue l’action de cette sainte femme qui répand sur lui un parfum précieux, comme pour l’ensevelir par avance. L’Évangile parle aussi avec éloge de ces fidèles qui reçurent le corps de Jésus à la descente de la croix, le couvrirent d’un linceul et le déposèrent avec respect dans un tombeau. Ce qu’il faut conclure de tous ces exemples, ce n’est pas que le corps garde après la mort aucun sentiment, mais c’est que la providence de Dieu s’étend jusque sur les restes des morts, et que ces devoirs de piété lui sont agréables comme témoignages de foi dans la résurrection. Nous en pouvons tirer aussi cet enseignement salutaire, que si les soins pieux donnés à la dépouille inanimée de nos frères ne sont point perdus devant Dieu, l’aumône qui soulage des hommes pleins de vie doit nous créer des droits bien autrement puissants à la rémunération céleste. Il y a encore sous ces ordres que les saints patriarches donnaient à leurs enfants pour la sépulture ou la translation de leurs derniers restes, des choses mystérieuses qu’il faut entendre dans un sens prophétique ; mais ce n’est pas ici le lieu de les approfondir, et nous en avons assez dit sur cette matière. Si donc la privation soudaine des choses les plus nécessaires à la vie, comme la nourriture et le vêtement, ne triomphe pas de la patience des hommes de bien, et, loin d’ébranler leur piété, ne sert qu’à l’éprouver et à la rendre plus féconde, pouvons-nous croire que l’absence des honneurs funèbres soit capable de troubler le repos des saints dans l’invisible séjour de l’éternité ? Concluons que si les derniers devoirs n’ont pas été rendus aux chrétiens lors du désastre de Rome ou à la prise d’autres villes, ni les vivants n’ont commis un crime, puisqu’ils n’ont rien pu faire, ni les morts n’ont éprouvé une peine, puisqu’ils n’ont rien pu sentir.
\subsection[{Chapitre XIV}]{Chapitre XIV}

\begin{argument}\noindent Les consolations divines n’ont jamais manqué aux saints dans la captivité.
\end{argument}

\noindent On se plaint que des chrétiens aient été emmenés captifs. Affreux malheur, en effet, si les barbares avaient pu les emmener quelque part où ils n’eussent point trouvé leur Dieu ! Ouvrez les saintes Écritures, vous y apprendrez comment on se console dans de pareilles extrémités. Les trois enfants de Babylone furent captifs ; Daniel le fut aussi, et comme lui d’autres prophètes ; le divin consolateur leur a-t-il jamais fait défaut ? Comment eut-il abandonné ses fidèles tombés sous la domination des hommes, celui qui n’abandonne pas le Prophète jusque dans les entrailles de la baleine ? Nos adversaires aiment mieux rire de ce miracle que d’y ajouter foi ; et cependant ils croient sur le témoignage de leurs auteurs qu’Arion de Méthymne, le célèbre joueur de lyre, jeté de son vaisseau dans la mer, fut reçu et porté au rivage sur le dos d’un dauphin. Mais, diront-ils, l’histoire de Jonas est plus incroyable. Soit, elle est plus incroyable, parce qu’elle est plus merveilleuse, et elle est plus merveilleuse, parce qu’elle trahit un bras plus puissant.
\subsection[{Chapitre XV}]{Chapitre XV}

\begin{argument}\noindent La piété de Régulus, souffrant volontairement la captivité pour tenir sa parole envers les dieux, ne le préserva pas de la mort.
\end{argument}

\noindent Les païens ont parmi leurs hommes illustres un exemple fameux de captivité volontairement subie par esprit de religion. Marcus Attilius Régulus, général romain, avait été pris par les Carthaginois. Ceux-ci, tenant moins à conserver leurs prisonniers qu’à recouvrer ceux qui leur avaient été faits par les Romains, envoyèrent Régulus à Rome avec leurs ambassadeurs, après qu’il se fut engagé par serment à revenir à Carthage, s’il n’obtenait pas ce qu’ils désiraient. Il part, et convaincu que l’échange des captifs n’était pas avantageux à la république, il en dissuade le sénat ; puis, sans y être contraint autrement que par sa parole, il reprend volontairement le chemin de sa prison. Là, les Carthaginois lui réservaient d’affreux supplices et la mort. On l’enferma dans un coffre de bois garni de pointes aigües, de sorte qu’il était obligé de se tenir debout, ou, s’il se penchait, de souffrir des douleurs atroces ; ce fut ainsi qu’ils le tuèrent en le privant de tout sommeil. Certes, voilà une vertu admirable et qui a su se montrer plus grande que la plus grande infortune ! Et cependant quels dieux avait pris à témoin Régulus, sinon ces mêmes dieux dont on s’imagine que le culte aboli est la cause de tous les malheurs du monde ? Si ces dieux qu’on servait pour être heureux en cette vie ont voulu ou permis le supplice d’un si religieux observateur de son serment, que pouvait faire de plus leur colère contre un parjure ? Mais je veux tirer de mon raisonnement une double conclusion nous avons vu que Régulus porta le respect pour les dieux jusqu’à croire qu’un serment ne lui permettait pas de rester dans sa patrie, ni de se réfugier ailleurs, mais lui faisait une loi de retourner chez ses plus cruels ennemis. Or, s’il croyait qu’une telle conduite lui fût avantageuse pour la vie présente, il était évidemment dans l’illusion, puisqu’il n’en recueillit qu’une affreuse mort. Voilà donc un homme dévoué au culte des dieux qui est vaincu et fait prisonnier ; le voilà qui, pour ne pas violer un serment prêté en leur nom, périt dans le plus affreux et le plus inouï des supplices ! Preuve certaine que le culte des dieux ne sert de rien pour le bonheur temporel. Si vous dites maintenant qu’il nous donne après la vie la félicité pour récompense, je vous demanderai alors pourquoi vous calomniez le christianisme, pourquoi vous prétendez que le désastre de Rome vient de ce qu’elle a déserté les autels de ses dieux, puisque, malgré le culte le plus assidu, elle aurait pu être aussi malheureuse que le fut Régulus ? Il ne resterait plus qu’à pousser l’aveuglement et la démence jusqu’à prétendre que si un individu a pu, quoique fidèle au culte des dieux, être accablé par l’infortune, il n’en saurait être de même d’une cité tout entière, la puissance des dieux étant moins faite pour se déployer sur un individu que sur un grand nombre. Comme si la multitude ne se composait pas d’individus !\par
Dira-t-on que Régulus, au milieu de sa captivité et de ses tourments, a pu trouver le bonheur dans le sentiment de sa vertu ? Que l’on se mette alors à la recherche de cette vertu véritable qui seule peut rendre un État heureux. Car le bonheur d’un État et celui d’un individu viennent de la même source, un État n’étant qu’un assemblage d’individus vivant dans un certain accord. Au surplus, je ne discute pas encore la vertu de Régulus ; qu’il me suffise, par l’exemple mémorable d’un homme qui aime mieux renoncer à la vie que d’offenser les dieux, d’avoir forcé mes adversaires de convenir que la conservation des biens corporels et de tous les avantages extérieurs de la vie n’est pas le véritable objet de la religion. Mais que peut-on attendre d’esprits aveuglés qui se glorifient d’un semblable citoyen et qui craignent d’avoir un État qui lui ressemble ? S’ils ne le craignent pas, qu’ils avouent donc que le malheur de Régulus a pu arriver à une ville aussi fidèle que lui au culte des dieux, et qu’ils cessent de calomnier le christianisme. Mais puisque nous avons soulevé ces questions au sujet des chrétiens emmenés en captivité, je dirai à ces hommes qui sans pudeur et sans prudence prodiguent l’insulte à notre sainte religion : Que l’exemple de Régulus vous confonde ! Car si ce n’est point une chose honteuse à vos dieux qu’un de leurs plus fervents admirateurs, pour garder la foi du serment, ait dû renoncer à sa patrie terrestre, sans espoir d’en trouver une autre, et mourir lentement dans les tortures d’un supplice inouï, de quel droit viendrait-on tourner à la honte du nom chrétien la captivité de nos fidèles, qui, l’œil fixé sur la céleste patrie, se savent étrangers jusque dans leurs propres foyers.
\subsection[{Chapitre XVI}]{Chapitre XVI}

\begin{argument}\noindent Le viol subi par les vierges chrétiennes dans la captivité, sans que leur volonté y fût pour rien, a-t-il pu souiller la vertu de leur âme ?
\end{argument}

\noindent On s’imagine couvrir les chrétiens de honte, quand pour rendre plus horrible le tableau de leur captivité, on nous montre les barbares violant les femmes ; les filles et même les vierges consacrées à Dieu. Mais ni la foi, ni la piété, ni la chasteté, comme vertu, ne sont ici le moins du monde intéressées ; le seul embarras que nous éprouvions, c’est de mettre d’accord avec la raison ce sentiment qu’on nomme pudeur. Aussi, ce que nous dirons sur ce sujet aura moins pour but de répondre à nos adversaires que de consoler des cœurs amis. Posons d’abord ce principe inébranlable que la vertu qui fait la bonne vie a pour siège l’âme, d’où elle commande aux organes corporels, et que le corps tire sa sainteté du secours qu’il prête à une volonté sainte. Tant que cette volonté ne faiblit pas, tout ce qui arrive au corps parle fait d’une volonté étrangère, sans qu’on puisse l’éviter autrement que par un péché, tout cela n’altère en rien notre innocence. Mais, dira-t-on, outre les traitements douloureux que peut souffrir le corps, il est des violences d’une autre nature, celles que le libertinage fait accomplir. Si une chasteté ferme et sûre d’elle-même en sort triomphante, la pudeur en souffre cependant, et on a lieu de craindre qu’un outrage qui ne peut être subi sans quelque plaisir de la chair ne se soit pas consommé sans quelque adhésion de la volonté.
\subsection[{Chapitre XVII}]{Chapitre XVII}

\begin{argument}\noindent Du suicide par crainte du châtiment et du déshonneur.
\end{argument}

\noindent S’il est quelques-unes de ces vierges qu’un tel scrupule ait portées à se donner la mort, quel homme ayant un cœur leur refuserait le pardon ? Quant à celles qui n’ont pas voulu se tuer, de peur de devenir criminelles en épargnant un crime à leurs ravisseurs, quiconque les croira coupables ne sera-t-il pas coupable lui-même de folle légèreté ? S’il n’est pas permis, en effet, de tuer un homme, même criminel, de son autorité privée, parce qu’aucune loi n’y autorise, il s’ensuit que celui qui se tue est homicide ; d’autant plus coupable en cela qu’il est d’ailleurs plus innocent du motif qui le porte à s’ôter la vie. Pourquoi détestons-nous le suicide de Judas ? Pourquoi la Vérité elle-même a-t-elle déclaré qu’en se pendant il a plutôt accru qu’expié le crime de son infâme trahison ? C’est qu’en désespérant de la miséricorde de Dieu, il s’est fermé la voie à un repentir salutaire. À combien plus forte raison faut-il donc rejeter la tentation du suicide quand on n’a aucun crime à expier ! En se tuant, Judas tua un coupable, et cependant il lui sera demandé compte, non seulement de la vie du Christ, mais de sa propre vie, parce qu’en se tuant à cause d’un premier crime, il s’est chargé d’un crime nouveau. Pourquoi donc un homme qui n’a point fait de mal à autrui s’en ferait-il à lui-même ? Il tuerait donc un innocent dans sa propre personne, pour empêcher un coupable de consommer son dessein, et il attenterait criminellement à sa vie, de peur qu’elle ne fût l’objet d’un attentat étranger !
\subsection[{Chapitre XVIII}]{Chapitre XVIII}

\begin{argument}\noindent Des violences que l’impureté d’autrui peut faire subir à notre corps, sans que notre volonté y participe.
\end{argument}

\noindent On alléguera la crainte qu’on éprouve d’être souillé par l’impureté d’autrui. Je réponds : Si l’impureté reste le fait d’un autre que vous, elle ne vous souillera pas ; si elle vous souille, c’est qu’elle est aussi votre fait. La pureté est une vertu de l’âme ; elle a pour compagne la force qui nous rend capables de supporter les plus grands maux plutôt que de consentir au mal. Or, l’homme le plus pur et le plus ferme est maître, sans doute, du consentement et du refus de sa volonté, mais il ne l’est pas des accidents que sa chair peut subir ; comment donc pourrait-il croire, s’il a l’esprit sain, qu’il a perdu la pureté parce que son corps violemment saisi aura servi à assouvir une impureté dont il n’est pas complice ? Si la pureté peut être perdue de la sorte, elle n’est plus une vertu de l’âme ; il faut cesser de la compter au nombre des biens qui sont le principe de la bonne vie, et le ranger parmi les biens du corps, avec la vigueur, la beauté, la santé et tous ces avantages qui peuvent souffrir des altérations, sans que la justice et la vertu en soient aucunement altérées. Or, si la pureté n’est rien de mieux que cela, pourquoi s’en mettre si fort en peine au péril même de la vie ? Rendez-vous à cette vertu de l’âme son vrai caractère, elle ne peut plus être détruite par la violence faite au corps. Je dirai plus s’il est vrai qu’en faisant des efforts pour ne pas céder à l’attrait des concupiscences charnelles, la sainte continence sanctifie le corps lui-même, j’en conclus que tarit que l’intention de leur résister se maintient ferme et inébranlable, le corps ne perd pas sa sainteté, car la volonté de s’en servir saintement persévère, et, autant qu’il dépend de lui, il nous en laisse la faculté.\par
La sainteté du corps ne consiste pas à préserver nos membres de toute altération et de tout contact : mille accidents peuvent occasionner de graves blessures, et souvent, pour nous sauver la vie, les chirurgiens nous font subir d’horribles opérations. Une sage-femme, soit malveillance, soit maladresse, soit pur hasard, détruit la virginité d’une jeune fille en voulant la constater, y a-t-il un esprit assez mal fait pour s’imaginer que cette jeune fille par l’altération d’un de ses organes, ait perdu quelque chose de la pureté de son corps ? Ainsi donc, tant que l’âme garde ce ferme propos qui fait la sainteté du corps, la brutalité d’une convoitise étrangère ne saurait ôter au corps le caractère sacré que lui imprime une continence persévérante. Voici une femme au cœur perverti qui, trahissant les vœux contractés devant Dieu, court se livrer à son amant. Direz-vous que pendant le chemin elle est encore pure de corps, après avoir perdu la pureté de l’âme, source de l’autre pureté ? Loin de nous cette erreur ! Disons plutôt qu’avec une âme pure, la sainteté du corps ne saurait être altérée, alors même que le corps subirait les derniers outrages ; et pareillement, qu’une âme corrompue fait perdre au corps sa sainteté, alors même qu’il n’aurait éprouvé aucune souillure matérielle. Concluons qu’une femme n’a rien à punir en soi par une mort volontaire, quand elle a été victime passive du péché d’autrui ; à plus forte raison, avant l’outrage : car alors elle se charge d’un homicide certain pour empêcher un crime encore incertain.
\subsection[{Chapitre XIX}]{Chapitre XIX}

\begin{argument}\noindent De Lucrèce, qui se donna la mort pour avoir été outragée.
\end{argument}

\noindent Nous soutenons que lorsqu’une femme, décidée à rester chaste, est victime d’un viol sans aucun consentement de sa volonté, il n’y a de coupable que l’oppresseur. Oseront-ils nous contredire, ceux contre qui nous défendons la pureté spirituelle et aussi la pureté corporelle des vierges chrétiennes outragées dans leur captivité ? Nous leur demanderons pourquoi la pudeur de Lucrèce, cette noble dame de l’ancienne Rome, est en si grand honneur auprès d’eux ? Quand le fils de Tarquin eut assouvi sa passion infâme, Lucrèce dénonça le crime à son mari, Collatin, et à son parent, Brutus, tous deux illustres par leur rang et par leur courage, et leur fit prêter serment de la venger ; puis, l’âme brisée de douleur et ne voulant pas supporter un tel affront, elle se tua. Dirons-nous qu’elle est morte chaste ou adultère ? Poser cette question c’est la résoudre. J’admire beaucoup cette parole d’un rhéteur qui déclamait sur Lucrèce : « Chose admirable ! » s’écriait-il ; « ils étaient deux ; et un seul fut adultère ! » Impossible de dire mieux et plus vrai. Ce rhéteur a parfaitement distingué dans l’union des corps la différence des âmes, l’une souillée par une passion brutale, l’autre fidèle à la chasteté, et exprimant à la fois cette union toute matérielle et cette différence morale, il a dit excellemment : « Ils étaient deux, un seul fut adultère. »\par
Mais d’où vient que la vengeance est tombée plus terrible sur la tête innocente que sur la tête coupable ? Car Sextus n’eut à souffrir que l’exil avec son père, et Lucrèce perdit la vie. S’il n’y a pas impudicité à subir la violence, y a-t-il justice à punir la chasteté ? C’est à vous que j’en appelle, lois et juges de Rome ! Vous ne voulez pas que l’on puisse impunément faire mourir un criminel, s’il n’a été condamné. Eh bien ! supposons qu’on porte ce crime à votre tribunal : une femme a été tuées non seulement elle n’avait pas été condamnée, mais elle était chaste et innocente ne punirez-vous pas sévèrement cet assassinat ? Or, ici, l’assassin c’est Lucrèce. Oui, cette Lucrèce tant célébrée a tué la chaste, l’innocente Lucrèce, l’infortunée victime de Sextus. Prononcez maintenant. Que si vous ne le faites point, parce que la coupable s’est dérobée à votre sentence, pourquoi tant célébrer la meurtrière d’une femme chaste et innocente ? Aussi bien ne pourriez-vous la défendre devant les juges d’enfer, tels que vos poètes nous les représentent, puisqu’elle est parmi ces infortunés\par
 {\itshape « Qui se sont donné la mort de leur propre main, et sans avoir commis aucun crime, on haine de l’existence, ont jeté leurs âmes au loin… »} \par
Veut-elle revenir au jour ?\par
 {\itshape « Le destin s’y oppose et elle est arrêtée par l’onde lugubre du marais qu’on ne traverse pas. »} \par
Mais peut-être n’est-elle pas là ; peut-être s’est-elle tuée parce qu’elle se sentait coupable ; peut-être (car qui sait, elle exceptée, ce qui se passait en son âme), touchée en secret par la volupté, a-t-elle consenti au crime, et puis, regrettant sa faute, s’est-elle tuée pour l’expier, mais, dans ce cas même, son devoir était, non de se tuer, mais d’offrir à ses faux jeux une pénitence salutaire. Au surplus, si les choses se sont passées ainsi, si on ne peut pas dire « Ils étaient deux, un seul fut adultère » ; si tous deux ont commis le crime, l’un par une brutalité ouverte, l’autre par un secret consentement, il n’est pas vrai alors qu’elle ait tué une femme innocente, et ses savants défenseurs peuvent soutenir qu’elle n’habite point cette partie des enfers réservée à ces infortunés « qui, purs de tout crime, se sont « arraché la vie ». Mais il y a ici deux extrémités inévitables : veut-on l’absoudre du crime d’homicide ? on la rend coupable d’adultère ; l’adultère est-il écarté ? il faut qu’elle soit homicide ; de sorte qu’on ne peut éviter cette alternative : si elle est adultère, pourquoi la célébrer ? si elle est restée chaste, pourquoi s’est-elle donné la mort ?\par
Quant à nous, pour réfuter ces hommes étrangers à toute idée de sainteté qui osent insulter les vierges chrétiennes outragées dans la captivité, qu’il nous suffise de recueillir cet éloge donné à l’illustre Romaine : « Ils étaient deux, un seul fut adultère. » On n’a pas voulu croire, tant la confiance était grande dans la vertu de Lucrèce, qu’elle se fût souillée par la moindre complaisance adultère. Preuve certaine que, si elle s’est tuée pour avoir subi un outrage auquel elle n’avait pas consenti, ce n’est pas l’amour de la chasteté qui a armé son bras, mais bien la faiblesse de la honte. Oui, elle a senti la honte d’un crime commis sur elle, bien que sans elle. Elle a craint, là fière Romaine, dans sa passion pour la gloire, qu’on ne pût dire, en la voyant survivre à son affront, qu’elle y avait consenti. À défaut de l’invisible secret de sa conscience, elle a voulu que sa mort fût un témoignage écrasant de sa pureté, persuadée que la patience serait contre elle un aveu de complicité.\par
Telle n’a point été la conduite des femmes chrétiennes qui ont subi la même violence. Elles ont voulu vivre, pour ne point venger sur elles le crime d’autrui, pour ne point commettre un crime de plus, pour ne point ajouter l’homicide à l’adultère ; c’est en elles-mêmes qu’elles possèdent l’honneur de la chasteté, dans le témoignage de leur conscience ; devant Dieu, il leur suffit d’être assurées qu’elles ne pouvaient rien faire de plus sans mal faire, résolues avant tout à ne pas s’écarter de la loi de Dieu, au risque même de n’éviter qu’à grand’peine les soupçons blessants de l’humaine malignité.
\subsection[{Chapitre XX}]{Chapitre XX}

\begin{argument}\noindent La loi chrétienne ne permet en aucun cas la mort volontaire.
\end{argument}

\noindent Ce n’est point sans raison que dans les livres saints on ne saurait trouver aucun passage où Dieu nous commande ou nous permette, soit pour éviter quelque mal, soit même pour gagner la vie éternelle, de nous donner volontairement la mort. Au contraire, cela nous est interdit par le précepte : « Tu ne tueras point. » Remarquez que la loi n’ajoute pas : « Ton prochain », ainsi qu’elle le fait quand elle défend le faux témoignage : « Tu ne porteras point faux témoignage contre ton prochain. » Cela ne veut pas dire néanmoins que celui qui porte faux témoignage contre soi-même soit exempt de crime ; car c’est de l’amour de soi-même que la règle de l’amour du prochain tire sa lumière, ainsi qu’il est écrit : « Tu aimeras ton prochain comme toi-même. » Si donc celui qui porte faux témoignage contre soi-même n’est pas moins coupable que s’il le portait contre son prochain, bien qu’en cette défense il ne soit parlé que du prochain et qu’il puisse paraître qu’il n’est pas défendu d’être faux témoin contre soi-même, à combien plus forte raison faut-il regarder comme interdit de se donner la mort, puisque ces termes « Tu ne tueras point », sont absolus, et que la loi n’y ajoute rien qui les limite ; d’où il suit que la défense est générale, et que celui-là même à qui il est commandé de ne pas tuer ne s’en trouve pas excepté. Aussi plusieurs cherchent-ils à étendre ce précepte jusqu’aux bêtes mêmes, s’imaginant qu’il n’est pas permis de les tuer. Mais que ne l’étendent-ils donc aussi aux arbres et aux plantes ? car, bien que les plantes n’aient point de sentiment, on ne laisse pas de dire qu’elles vivent, et par conséquent elles peuvent mourir, et même, quand la violence s’en mêle, être tuées. C’est ainsi que l’Apôtre, parlant des semences, dit : « Ce que tu sèmes ne peut vivre, s’il ne meurt auparavant » et le Psalmiste : « Il a tué leurs vignes par la grêle. » Est-ce à dire qu’en vertu du précepte : « Tu ne tueras point », ce soit un crime d’arracher un arbrisseau, et serons-nous assez fous pour souscrire, en cette rencontre, aux erreurs des Manichéens ? Laissons de côté ces rêveries, et lorsque nous lisons : « Tu ne tueras point », si nous ne l’entendons pas des plantes, parce qu’elles n’ont point de sentiment, ni des bêtes brutes, qu’elles volent dans l’air, nagent dans l’eau, marchent ou rampent sur terre, parce qu’elles sont privées de raison et ne forment point avec l’homme une société, d’où il suit que par une disposition très juste du Créateur, leur vie et leur mort sont également faites pour notre usage, il reste que nous entendions de l’homme seul ce précepte : « Tu ne tueras point », c’est-à-dire, tu ne tueras ni un autre ni toi-même, car celui qui se tue, tue un homme.
\subsection[{Chapitre XXI}]{Chapitre XXI}

\begin{argument}\noindent Des meurtres qui, par exception, n’impliquent point crime d’homicide.
\end{argument}

\noindent Dieu lui-même a fait quelques exceptions à la défense de tuer l’homme, tantôt par un commandement général, tantôt par un ordre temporaire et personnel. En pareil cas, celui qui tue ne fait que prêter son ministère à un ordre supérieur ; il est comme un glaive entre les mains de celui qui frappe, et par conséquent il ne faut pas croire que ceux-là aient violé le précepte : « Tu ne tueras point », qui ont entrepris des guerres par l’inspiration de Dieu, ou qui, revêtus du caractère de la puissance publique et obéissant aux lois de l’État, c’est-à-dire à des lois très justes et très raisonnables, ont puni de mort les malfaiteurs. L’Écriture est si loin d’accuser Abraham d’une cruauté coupable pour s’être déterminé, par pur esprit d’obéissance, à tuer son fils, qu’elle loue sa piété. Et l’on a raison de se demander si l’on peut considérer Jephté comme obéissant à un ordre de Dieu, quand, voyant sa fille qui venait à sa rencontre, il la tue pour être fidèle au vœu qu’il avait fait d’immoler le premier être vivant qui s’offrirait à ses regards son retour après la victoire. De même, comment justifie-t-on Samson de s’être enseveli avec les ennemis sous les ruines d’un édifice ? en disant qu’il obéissait au commandement intérieur de l’Esprit, qui se servait de lui pour faire des miracles. Ainsi donc, sauf les deux cas exceptionnels d’une loi générale et juste ou d’un ordre particulier de celui qui est la source de toute justice, quiconque tue un homme, soi-même ou son prochain, est coupable d’homicide.
\subsection[{Chapitre XXII}]{Chapitre XXII}

\begin{argument}\noindent La mort volontaire n’est jamais une preuve de grandeur d’âme.
\end{argument}

\noindent On peut admirer la grandeur d’âme de ceux qui ont attenté sur eux-mêmes, mais, à coup sûr, on ne saurait louer leur sagesse. Et même, à examiner les choses de plus près et de l’œil de la raison, est-il juste d’appeler grandeur d’âme cette faiblesse qui rend impuissant à supporter son propre mal ou les fautes d’autrui ? Rien ne marque mieux une âme sans énergie que de ne pouvoir se résigner à l’esclavage du corps et à la folie de l’opinion. Il y a plus de force à endurer une vie misérable qu’à la fuir, et les lueurs douteuses de l’opinion, surtout de l’opinion vulgaire, ne doivent pas prévaloir sur les pures clartés de la conscience. Certes, s’il y a quelque grandeur d’âme à se tuer, personne n’a un meilleur droit à la revendiquer que Cléombrote, dont on raconte qu’ayant lu le livre où Platon discute l’immortalité de l’âme, il se précipita du haut d’un mur pour passer de cette vie dans une autre qu’il croyait meilleure ; car il n’y avait ni calamité, ni crime faussement ou justement imputé dont le poids pût lui paraître insupportable ; si donc il se donna la mort, s’il brisa ces liens si doux de la vie, ce fut par pure grandeur d’âme. Eh bien ! je dis que si l’action de Cléombrote est grande, elle n’est du moins pas bonne ; et j’en atteste Platon lui-même, Platon, qui n’aurait pas manqué de se donner la mort et de prescrire le suicide aux autres, si ce même génie qui lui révélait l’immortalité de l’âme, ne lui avait fait comprendre que cette action, loin d’être permise, doit être expressément défendue.\par
Mais, dit-on, plusieurs se sont tués pour ne pas tomber en la puissance des ennemis. Je réponds qu’il ne s’agit pas de ce qui a été fait, mais de ce qu’on doit faire. La raison est au-dessus des exemples, et les exemples eux-mêmes s’accordent avec la raison, quand on sait choisir ceux qui sont le plus dignes d’être imités, ceux qui viennent de la plus haute piété. Ni les Patriarches, ni les Prophètes, ni les Apôtres ne nous ont donné l’exemple du suicide. Jésus-Christ, Notre-Seigneur, qui avertit ses disciples, en cas de persécution, de fuir de ville en ville, ne pouvait-il pas leur conseiller de se donner la mort, plutôt que de tomber dans les mains de leurs persécuteurs ? Si donc il ne leur a donné ni le conseil, ni l’ordre de quitter la vie, lui qui leur prépare, suivant ses promesses, les demeures de l’éternité, il s’ensuit que les exemples invoqués par les Gentils, dans leur ignorance de Dieu, ne prouvent rien pour les adorateurs du seul Dieu véritable.
\subsection[{Chapitre XXIII}]{Chapitre XXIII}

\begin{argument}\noindent De l’exemple de Caton, qui s’est donné la mort pour n’avoir pu supporter la victoire de César.
\end{argument}

\noindent Après l’exemple de Lucrèce, dont nous avons assez parlé plus haut, nos adversaires ont beaucoup de peine à trouver une autre autorité que celle de Caton, qui se donna la mort à Utique : non qu’il soit le seul qui ait attenté sur lui-même, mais il semble que l’exemple d’un tel homme, dont les lumières et la vertu sont incontestées, justifie complètement ses imitateurs. Pour nous, que pouvons-nous dire de mieux sur l’action de Caton, sinon que ses propres amis, hommes éclairés tout autant que lui, s’efforcèrent de l’en dissuader, ce qui prouve bien qu’ils voyaient plus de faiblesse que de force d’âme dans cette résolution, et l’attribuaient moins à un principe d’honneur qui porte à éviter l’infamie qu’à un sentiment de pusillanimité qui rend le malheur insupportable. Au surplus, Caton lui-même s’est trahi par le conseil donné en mourant à son fils bien-aimé. Si en effet c’était une chose honteuse de vivre sous la domination de César, pourquoi le père conseille-t-il au fils de subir cette honte, en lui recommandant de tout espérer de la clémence du vainqueur ? Pourquoi ne pas l’obliger plutôt à périr avec lui ? Si Torquatus a mérité des éloges pour avoir fait mourir son fils, quoique vainqueur, parce qu’il avait combattu contre ses ordres, pourquoi Caton épargne-t-il son fils, comme lui vaincu, alors qu’il ne s’épargne pas lui-même ? Y avait-il plus de honte à être vainqueur en violant la discipline, qu’à reconnaître un vainqueur en subissant l’humiliation ? Ainsi donc Caton n’a point pensé qu’il fût honteux de vivre sous la loi de César triomphant, puisque autrement il se serait servi, pour sauver l’honneur de son fils, du même fer dont il perça sa poitrine. Mais la Vérité est qu’autant il aima son fils, sur qui ses vœux et sa volonté appelaient la clémence de César, autant il envia à César (comme César l’a dit lui-même, à ce qu’on assure), la gloire de lui pardonner ; et si ce ne fut pas de l’envie, disons, en termes plus doux, que ce fut de la honte.
\subsection[{Chapitre XXIV}]{Chapitre XXIV}

\begin{argument}\noindent La vertu des chrétiens l’emporte sur celle de Régulus, supérieure elle-même à celle de Caton.
\end{argument}

\noindent Nos adversaires ne veulent pas que nous préférions à Caton le saint homme Job, qui aima mieux souffrir dans sa chair les plus cruelles douleurs, que de s’en délivrer par la mort, sans parler des autres saints que l’Écriture, ce livre éminemment digne d’inspirer confiance et de faire autorité, nous montre résolus à supporter la captivité et la domination des ennemis plutôt que d’attenter à leurs jours. Eh bien ! prenons leurs propres livres, et nous y trouverons des motifs de préférer quelqu’un à Marcus Caton : c’est Marcus Régulus. Caton, en effet, n’avait jamais vaincu César ; vaincu par lui, il dédaigna de se soumettre et préféra se donner la mort. Régulus, au contraire, avait vaincu les Carthaginois. Général romain, il avait remporté, à la gloire de Rome, une de ces victoires qui, loin de contrister les bons citoyens, arrachent des louanges à l’ennemi lui-même. Vaincu à son tour, il aima mieux se résigner et rester captif que s’affranchir et devenir meurtrier de lui-même. Inébranlable dans sa patience à subir le joug de Carthage, et dans sa fidélité à aimer Rome, il ne consentit pas plus à dérober son corps vaincu aux ennemis, qu’à sa patrie son cœur invincible. S’il ne se donna pas la mort, ce ne fut point par amour pour la vie. La preuve, c’est que pour garder la foi de son serment, il n’hésita point à retourner à Carthage, plus irritée contre lui de son discours au sénat romain que de ses victoires. Si donc un homme qui tenait si peu à la vie a mieux aimé périr dans les plus cruels tourments que se donner la mort, il fallait donc que le suicide fût à ses yeux un très grand crime. Or, parmi les citoyens de Rome les plus vertueux et les plus dignes d’admiration, en peut-on citer un seul qui soit supérieur à Régulus ? Ni la prospérité ne put le corrompre, puisqu’après de si grandes victoires il resta pauvre ; ni l’adversité ne put le briser, puisqu’en face de si terribles supplices il accourut intrépide. Ainsi donc, ces courageux et illustres personnages, mais qui n’ont après tout servi que leur patrie terrestre, ces religieux observateurs de la foi jurée, mais qui n’attestaient que de faux dieux, ces hommes qui pouvaient, au nom de la coutume et du droit de la guerre, frapper leurs ennemis vaincus, n’ont pas voulu, même vaincus par leurs ennemis, se frapper de leur propre main ; sans craindre la mort, ils ont préféré subir la domination du vainqueur que s’y soustraire par le suicide. Quelle leçon pour les chrétiens, adorateurs du vrai Dieu et amants de la céleste patrie ! avec quelle énergie ne doivent-ils pas repousser l’idée du suicide, quand la Providence divine, pour les éprouver ou les châtier, les soumet pour un temps au joug ennemi ! Qu’ils ne craignent point, dans cette humiliation passagère, d’être abandonnés par celui qui a voulu naître humble, bien qu’il s’appelle le Très-Haut ; et qu’ils se souviennent enfin qu’il n’y a plus pour eux de discipline militaire, ni de droit de la guerre qui les autorise ou leur commande la mort du vaincu. Si donc un vrai chrétien ne doit pas frapper même un ennemi qui a attenté ou qui est sur le point d’attenter contre lui, quelle peut donc être la source de cette détestable erreur que l’homme peut se tuer, soit parce qu’on a péché, soit de peur qu’on ne pèche à son détriment ?
\subsection[{Chapitre XXV}]{Chapitre XXV}

\begin{argument}\noindent Il ne faut point éviter un péché par un autre.
\end{argument}

\noindent Mais il est à craindre, dit-on, que soumis à un outrage brutal, le corps n’entraîne l’âme, par le vif aiguillon de la volupté, à donner au péché un coupable contentement ; et dès lors, le chrétien doit se tuer, non pour éviter le péché à autrui, mais pour s’en préserver lui-même. Je réponds que celui-là ne laissera point son âme céder à l’excitation d’une sensualité étrangère qui vit soumis à Dieu et à la divine sagesse, et non à la concupiscence de la chair. De plus, s’il est vrai et évident que c’est un crime détestable et digne de la damnation de se donner la mort, y a-t-il un homme assez insensé pour parler de la sorte : Péchons maintenant, de crainte que nous ne venions à pécher plus tard. Soyons homicides, de crainte d’être plus tard adultères. Quoi donc ! si l’iniquité est si grande qu’il n’y ait plus à choisir entre le crime et l’innocence, mais à opter entre deux crimes, ne vaut-il pas mieux préférer un adultère incertain et à venir à un homicide actuel et certain ; et le péché, qui peut être expié par la pénitence n’est-il point préférable à celui qui ne laisse aucune place au repentir ? Ceci soit dit pour ces fidèles qui se croient obligés à se donner la mort, non pour épargner un crime à leur prochain, mais de peur que la brutalité qu’ils subissent n’arrache à leur volonté un consentement criminel. Mais loin de moi, loin de toute âme chrétienne, qui, ayant mis sa confiance en Dieu, y trouve son appui, loin de nous tous cette crainte de céder à l’attrait honteux de la volupté de la chair ! Et si cet esprit de révolte sensuelle, qui reste attaché à nos membres, même aux approches de la mort, agit comme par sa loi propre en dehors de la loi de notre volonté, peut-il y avoir faute, quand la volonté refuse, puisqu’il n’y en a pas, quand elle est suspendue par le sommeil ?
\subsection[{Chapitre XXVI}]{Chapitre XXVI}

\begin{argument}\noindent Il n’est point permis de suivre l’exemple des saints en certains cas où la foi nous assure qu’ils ont agi par des motifs particuliers.
\end{argument}

\noindent On objecte l’exemple de plusieurs saintes femmes qui, au temps de la persécution, pour soustraire leur pudeur à une brutale violence, se précipitèrent dans un fleuve où elles devaient infailliblement être entraînées et périr. L’Église catholique, dit-on, célèbre leur martyre avec une solennelle vénération. Ici je dois me défendre tout jugement téméraire. L’Église a-t-elle obéi à une inspiration divine, manifestée par des signes certains, en honorant ainsi la mémoire de ces saintes femmes ? Je l’ignore ; mais cela peut être. Qui dira si ces vertueuses femmes, loin d’agir humainement, n’ont pas été divinement inspirées, et si, loin d’être égarées par le délire, elles n’ont pas exécuté un ordre d’en haut, comme fit Samson, dont il n’est pas permis de croire qu’il ait agi autrement ? Lorsque Dieu parle et intime un commandement précis, qui oserait faire un crime de l’obéissance et accuser la piété de se montrer trop docile ? Ce n’est point à dire maintenant que le premier venu ait le droit d’immoler son fils à Dieu, sous prétexte d’imiter l’exemple d’Abraham. En effet, quand un soldat tue un homme pour obéir à l’autorité légitime, il n’est coupable d’homicide devant aucune loi civile ; au contraire, s’il n’obéit pas, il est coupable de désertion et de révolte. Supposez, au contraire, qu’il eût agi de son autorité privée, il eût été responsable du sang versé ; de sorte que, pour une même action, ce soldat est justement puni, soit quand il la fait sans ordre, soit quand ayant ordre de la faire, il ne la fait pas. Or, si l’ordre d’un général a une si grande autorité, que dire d’un commandement du Créateur ? Ainsi donc, permis à celui qui sait qu’il est défendu d’attenter sur soi-même, de se tuer, si c’est pour obéir à celui dont il n’est pas permis de mépriser les ordres ; mais qu’il prenne garde que l’ordre ne soit pas douteux. Nous ne pénétrons, nous, dans les secrets de la conscience d’autrui que par ce qui est confié à notre oreille, et nous ne prétendons pas au jugement des choses cachées : « Nul ne sait ce qui se passe dans l’homme, si ce n’est l’esprit de l’homme qui est en lui. » Ce que nous disons, ce que nous affirmons, ce que nous approuvons en toutes manières, c’est que personne n’a le droit de se donner la mort, ni pour éviter les misères du temps, car il risque de tomber dans celles de l’éternité, ni à cause des péchés d’autrui, car, pour éviter un péché qui ne le souillait pas, il commence par se charger lui-même d’un péché qui lui est propre, ni pour ses péchés passés, car, s’il a péché, il a d’autant plus besoin de vivre pour faire pénitence, ni enfin, par le désir d’une vie meilleure, car il n’y a point de vie meilleure pour ceux qui sont coupables de leur mort.
\subsection[{Chapitre XXVII}]{Chapitre XXVII}

\begin{argument}\noindent Si la mort volontaire est désirable comme un refuge contre le péché.
\end{argument}

\noindent Reste un dernier motif dont j’ai déjà parlé, et qui consiste à fonder le droit de se donner la mort sur la crainte qu’on éprouve d’être entraîné au péché par les caresses de la volupté ou par les tortures de la douleur. Admettez ce motif comme légitime, vous serez conduits par le progrès du raisonnement à conseiller aux hommes de se donner la mort au moment où, purifiés par l’eau régénératrice du baptême, ils ont reçu la rémission de tous leurs péchés. Le vrai moment, en effet, de se mettre à couvert des péchés futurs, c’est quand tous les anciens sont effacés. Or, si la mort volontaire est légitime, pourquoi ne pas choisir ce moment de préférence ? quel motif peut retenir un nouveau baptisé ? pourquoi exposerait-il encore son âme purifiée à tous les périls de la vie, quand il lui est si facile d’y échapper, selon ce précepte : « Celui qui aime le péril y tombera » ? pourquoi aimer tant et de si grands périls, ou, si on ne les aime pas, pourquoi s’y exposer en conservant une vie dont on a le droit de s’affranchir ? est-il possible d’avoir le cœur assez pervers et l’esprit assez aveuglé pour se créer ces deux obligations contradictoires : l’une, de se donner la mort, de peur que la domination d’un maître ne nous fasse tomber dans le péché ; l’autre, de vivre, afin de supporter une existence pleine à chaque heure de tentations, de ces mêmes tentations que l’on aurait à craindre sous la domination d’un maître, et de mille autres qui sont inséparables de notre condition mortelle ? à ce compte, pourquoi perdrions-nous notre temps à enflammer le zèle des nouveaux baptisés par de vives exhortations, à leur inspirer l’amour de la pureté virginale, de la continence dans le veuvage, de la fidélité au lit conjugal, quand nous avons à leur indiquer un moyen de salut beaucoup plus sûr et à l’abri de tout péril, c’est de se donner la mort aussitôt après la rémission de leurs péchés, afin de paraître ainsi plus sains et plus purs devant Dieu ? Or, s’il y a quelqu’un qui s’avise de donner un pareil conseil, je ne dirai pas : Il déraisonne je dirai : Il est fou. Comment donc serait-il permis de tenir à un homme le langage que voici : « Tuez-vous, de crainte que, vivant sous la domination d’un maître impudique, vous n’ajoutiez à vos fautes vénielles quelque plus grand péché », si c’est évidemment un crime abominable de lui dire : « Tuez-vous, aussitôt après l’absolution de vos péchés, de crainte que vous ne veniez par la suite à en commettre d’autres et de plus grands, vivant dans un monde plein de voluptés attrayantes, de cruautés furieuses, d’illusions et de terreurs. » Puisqu’un tel langage serait criminel, c’est donc aussi une chose criminelle de se tuer. On ne saurait, en effet, invoquer aucun motif qui fût plus légitime ; celui-là ne l’étant pas, nul ne saurait l’être.
\subsection[{Chapitre XXVIII}]{Chapitre XXVIII}

\begin{argument}\noindent Pourquoi Dieu a permis que les barbares aient attenté à la pudeur des femmes chrétiennes.
\end{argument}

\noindent Ainsi donc, fidèles servantes tic Jésus-Christ, que la vie ne vous soit point à charge parce que les ennemis se sont fait un jeu de votre chasteté. Vous avez une grande et solide consolation, si votre conscience vous rend ce témoignage que vous n’avez point consenti au péché qui a été permis contre vous. Demanderez-vous pourquoi il a été permis ? qu’il vous suffise de savoir que la Providence, qui a créé le monde et qui le gouverne, est profonde en ses conseils ; « impénétrables sont ses jugements et insondables ses voies ». Toutefois descendez au fond de votre conscience, et demandez-vous sincèrement si ces dons de pureté, de continence, de chasteté n’ont pas enflé votre orgueil, si, trop charmées par les louanges des hommes, vous n’avez point envié à quelques-unes de vos compagnes ces mêmes vertus. Je n’accuse point, ne sachant rien, et je ne puis entendre la réponse de votre conscience ; mais si elle est telle que je le crains, ne vous étonnez plus d’avoir perdu ce qui vous faisait espérer les empressements des hommes, et d’avoir conservé ce qui échappe à leurs regards. Si vous n’avez pas consenti au mal, c’est qu’un secours d’en haut est venu fortifier la grâce divine que vous alliez perdre, et l’opprobre subi devant les hommes a remplacé pour vous cette gloire humaine que vous risquiez de trop aimer. Âmes timides, soyez deux fois consolées ; d’un côté, une épreuve, de l’autre, un châtiment ; une épreuve qui vous justifie, un châtiment qui vous corrige. Quant à celles d’entre vous dont la conscience ne leur reproche pas de s’être enorgueillies de posséder la pureté des vierges, la continence des veuves, la chasteté des épouses, qui, le cœur plein d’humilité, se sont réjouies avec crainte de posséder le don de Dieu, sans porter aucune envie à leurs émules en sainteté, qui dédaignant enfin l’estime des hommes, d’autant plus grande pour l’ordinaire que la vertu qui les obtient est plus rare, ont souhaité l’accroissement du nombre des saintes âmes plutôt que sa diminution qui les eût fait paraître davantage ; quant à celles-là, qu’elles ne se plaignent pas d’avoir souffert la brutalité des barbares qu’elles n’accusent point Dieu de l’avoir permise, qu’elles ne doutent point de sa providence, qui laisse faire ce que nul ne commet impunément. Il est en effet certains penchants mauvais qui pèsent secrètement sur l’âme, et auxquels la justice de Dieu lâche les rênes à un certain jour pour en réserver la punition au dernier jugement. Or, qui sait si ces saintes femmes, dont la conscience est pure de tout orgueil et qui ont eu à subir dans leur corps la violence des barbares, qui sait si elles ne nourrissaient pas quelque secrète faiblesse, qui pouvait dégénérer en faste ou en superbe, au cas où, dans le désordre universel, cette humiliation leur eût été épargnée ? De même que plusieurs ont été emportés par la mort, afin que l’esprit du mal ne pervertît pas leur volonté, ces femmes ont perdu l’honneur par la violence, afin que la prospérité ne pervertît pas leur modestie. Ainsi donc, ni celles qui étaient trop fières de leur pureté, ni celles que le malheur seul a préservées de l’orgueil, n’ont perdu la chasteté ; seulement elles ont gagné l’humilité ; celles-là ont été guéries d’un mal présent, celles-ci préservées d’un mal à venir.\par
Ajoutons enfin que, parmi ces victimes de la violence des barbares, plus d’une peut-être s’était imaginée que la continence est un bien corporel que l’on conserve tant que le corps n’est pas souillé, tandis qu’elle est un bien du corps et de l’âme tout ensemble, lequel réside dans la force de la volonté, soutenue par la grâce divine, et ne peut se perdre contre le gré de son possesseur. Les voilà maintenant délivrées de ce faux préjugé ; et quand leur conscience les assure du zèle dont elles ont servi Dieu, quand leur solide foi les persuade que ce Dieu ne peut abandonner qui le sert et l’invoque de tout son cœur, sachant du reste, de science certaine, combien la chasteté lui est agréable, elles doivent nécessairement conclure qu’il eût jamais permis l’outrage souffert par des âmes saintes, si cet outrage eût pu leur ravir le don qu’il leur a fait lui-même et qui les lui rend aimables, la sainteté.
\subsection[{Chapitre XXIX}]{Chapitre XXIX}

\begin{argument}\noindent Réponse que les enfants du Christ doivent faire aux infidèles, quand ceux-ci leur reprochent que le Christ ne les a pas mis a couvert de la fureur des ennemis.
\end{argument}

\noindent Toute la famille du Dieu véritable et souverain a donc un solide motif de consolation établi sur un meilleur fondement que l’espérance de biens chancelants et périssables ; elle doit accepter sans regret la vie temporelle elle-même, puisqu’elle s’y prépare à la vie éternelle, usant des biens de ce monde sans s’y attacher, comme fait un voyageur, et subissant les maux terrestres comme une épreuve ou un châtiment. Si on insulte à sa résignation, si on vient lui dire, aux jours d’infortune : « Où est ton Dieu ? » qu’elle demande à son tour à ceux qui l’interrogent, où sont leurs dieux, alors qu’ils endurent ces mêmes souffrances dont la crainte est le seul principe de leur piété. Pour nous, enfants du Christ, nous répondrons : Notre Dieu est partout présent et tout entier partout ; exempt de limites, il peut être présent en restant invisible et s’absenter sans se mouvoir. Quand ce Dieu m’afflige, c’est pour éprouver ma vertu ou pour châtier mes péchés ; et en échange de maux temporels, si je les souffre avec piété, il me réserve une récompense éternelle. Mais vous, dignes à peine qu’on vous parle de vos dieux, qui êtes-vous en face du mien, « plus redoutable que tous les dieux ; car tous les dieux des nations sont des démons, et le Seigneur a fait les cieux » ?
\subsection[{Chapitre XXX}]{Chapitre XXX}

\begin{argument}\noindent Ceux qui s’élèvent contre la religion chrétienne ne sont avides que de honteuses prospérités.
\end{argument}

\noindent Si cet illustre Scipion Nasica, autrefois votre souverain Pontife, qui dans la terreur de la guerre punique fut choisi d’une voix unanime par le sénat, comme le meilleur citoyen de Rome, pour aller recevoir de Phrygie l’image de la mère des dieux, si ce grand homme, dont vous n’oseriez affronter l’aspect, pouvait revenir à la vie, c’est lui qui se chargerait de rabattre votre impudence. Car enfin, qu’est-ce qui vous pousse à imputer au christianisme les maux que vous souffrez ? C’est le désir de trouver la sécurité dans le vice, et de vous livrer sans obstacle à tout le dérèglement de vos mœurs. Si vous souhaitez la paix et l’abondance, ce n’est pas pour en user honnêtement, c’est-à-dire avec mesure, tempérance et piété, mais pour vous procurer, au prix de folles prodigalités, une variété infinie de voluptés, et répandre ainsi dans les mœurs, au milieu de la prospérité apparente, une corruption mille fois plus désastreuse que toute la cruauté des ennemis. C’est ce que craignait Scipion, votre grand pontife, et, au jugement de tout le sénat, le meilleur citoyen de Rome, quand il s’opposait à la ruine de Carthage, cette rivale de l’empire romain, et combattait l’avis contraire de Caton. Il prévoyait les suites d’une sécurité fatale à des âmes énervées et voulait qu’elles fussent protégées par la crainte, comme des pupilles par un tuteur. Il voyait juste, et l’événement prouva qu’il avait raison. Carthage une fois détruite, la république romaine fut délivrée sans doute d’une grande terreur ; mais combien de maux naquirent successivement de cette prospérité ! la concorde entre les citoyens affaiblie et détruite, bientôt des séditions sanglantes, puis, par un enchaînement de causes funestes, la guerre civile avec ses massacres, ses flots de sang, ses proscriptions, ses rapines ; enfin, un tel déluge de calamités que ces Romains, qui, au temps de leur vertu, n’avaient rien à redouter que de l’ennemi, eurent beaucoup plus à souffrir, après l’avoir perdue, de la main de leurs propres concitoyens. La fureur de dominer, passion plus effrénée chez le peuple romain que tous les autres vices de notre nature, ayant triomphé dans un petit nombre de citoyens puissants, tout le reste, abattu et lassé, se courba sous le joug.
\subsection[{Chapitre XXXI}]{Chapitre XXXI}

\begin{argument}\noindent Par quels degrés s’est accrue chez les Romains la passion de la domination.
\end{argument}

\noindent Comment, en effet, cette passion se serait-elle apaisée dans ces esprits superbes, avant que de s’élever par des honneurs incessamment renouvelés jusqu’à la puissance royale ? Or, pour obtenir le renouvellement de ces honneurs, la brigue était indispensable ; et la brigue elle-même ne pouvait prévaloir que chez un peuple corrompu par l’avarice et la débauche. Or, comment le peuple devint-il avare et débauché ? par un effet de cette prospérité dont s’alarmait si justement Scipion, quand il s’opposait avec une prévoyance admirable à la ruine de la plus redoutable et de la plus opulente ennemie de Rome. Il aurait voulu que la crainte servit de frein à la licence, que la licence comprimée arrêtât l’essor de la débauche et de l’avarice, et qu’ainsi la vertu pût croître et fleurir pour le salut de la république, et avec la vertu, la liberté ! Ce fut par le même principe et dans un même sentiment de patriotique prévoyance que Scipion, je parle toujours de l’illustre pontife que le sénat proclama par un choix unanime le meilleur citoyen de Rome, détourna ses collègues du dessein qu’ils avaient formé de construire un amphithéâtre. Dans un discours plein d’autorité, il leur persuada de ne pas souffrir que la mollesse des Grecs vînt corrompre la virile austérité des antiques mœurs et souiller la vertu romaine de la contagion d’une corruption étrangère. Le sénat fut si touché par cette grave éloquence qu’il défendit l’usage des sièges qu’on avait coutume de porter aux représentations scéniques. Avec quelle ardeur ce grand homme eût-il entrepris d’abolir les jeux mêmes, s’il eût osé résister à l’autorité de ce qu’il appelait des dieux ! car il ne savait pas que ces prétendus dieux ne sont que de mauvais démons, ou s’il le savait, il croyait qu’on devait les apaiser plutôt que de les mépriser. La doctrine céleste n’avait pas encore été annoncée aux Gentils, pour purifier leur cœur par la foi, transformer en eux la nature humaine par une humble piété, les rendre capables des choses divines et les délivrer enfin de la domination des esprits superbes.
\subsection[{Chapitre XXXII}]{Chapitre XXXII}

\begin{argument}\noindent De l’établissement des jeux scéniques.
\end{argument}

\noindent Sachez donc, vous qui l’ignorez, et vous aussi qui feignez l’ignorance, n’oubliez pas, au milieu de vos murmures contre votre libérateur, que ces jeux scéniques, spectacles de turpitude, œuvres de licence et de vanité, ont été établis à Rome, non par la corruption des hommes, muais par le commandement de vos dieux. Mieux eût valu accorder les honneurs divins à Scipion que de rendre un culte à des dieux de cette sorte, qui n’étaient certes pas meilleurs que leur pontife. Écoutez-moi un instant avec attention, si toutefois votre esprit, longtemps enivré d’erreurs, est capable d’entendre la voix de la raison : Les dieux commandaient que l’on célébrât des jeux de théâtre pour guérir la peste des corps, et Scipion, pour prévenir la peste des âmes, ne voulait pas que le théâtre même fût construit. S’il vous reste encore quelque lueur d’intelligence pour préférer l’âme au corps, dites-moi qui vous devez honorer, de Scipion ou de vos dieux. Au surplus, si la peste vint à cesser, ce ne fut point parce que la folle passion des jeux plus raffinés de la scène s’empara d’un peuple belliqueux qui n’avait connu jusqu’alors que les jeux du cirque ; mais ces démons méchants et astucieux, prévoyant que la peste allait bientôt finir, saisirent cette occasion pour en répandre une autre beaucoup plus dangereuse et qui fait leur joie parce qu’elle s’attaque, non point au corps, mais aux mœurs. Et de fait, elle aveugla et corrompit tellement l’esprit des Romains que dans ces derniers temps (la postérité aura peine à le croire), parmi les malheureux échappés au sac de Rome et qui ont pu trouver un asile à Carthage, on en a vu plusieurs tellement possédés de cette étrange maladie qu’ils couraient chaque jour au théâtre s’enivrer follement du spectacle des histrions.
\subsection[{Chapitre XXXIII}]{Chapitre XXXIII}

\begin{argument}\noindent La ruine de Rome n’a pas corrigé les vices des Romains.
\end{argument}

\noindent Quelle est donc votre erreur, insensés, ou plutôt, quelle fureur vous transporte ! Quoi ! au moment où, si l’on en croit les récits des voyageurs, le désastre de Rome fait jeter un cri de douleur jusque chez les peuples de l’Orient, au moment où les cités les plus illustres dans les plus lointains pays font de votre malheur un deuil public, c’est alors que vous recherchez les théâtres, que vous y courez, que vous les remplissez, que vous en envenimez encore le poison. C’est cette souillure et cette perte des âmes, ce renversement de toute probité et de tout sentiment honnête que Scipion redoutait pour vous, quand il s’opposait à la construction d’un amphithéâtre, quand il prévoyait que vous pourriez aisément vous laisser corrompre par la bonne fortune, quand il ne voulait pas qu’il ne vous restât plus d’ennemis à redouter. Il n’estimait pas qu’une cité fût florissante, quand ses murailles sont debout et ses mœurs ruinées. Mais le séducteur des démons a eu plus de pouvoir sur vous que la prévoyance des sages. De là vient que vous ne voulez pas qu’on vous impute le mal que vous faites et que vous imputez aux chrétiens celui que vous souffrez. Corrompus par la bonne fortune, incapables d’être corrigés par la mauvaise, vous ne cherchez pas dans la paix la tranquillité de, l’État, mais l’impunité de vos vices. Scipion vous souhaitait la crainte de l’ennemi pour vous retenir sur la pente de la licence, et vous, écrasés par l’ennemi, vous ne pouvez pas même contenir vos dérèglements ; tout l’avantage de votre calamité, vous l’avez perdu ; vous êtes devenus misérables, et vous êtes restés vicieux.
\subsection[{Chapitre XXXIV}]{Chapitre XXXIV}

\begin{argument}\noindent La clémence de Dieu a adouci le désastre de Rome.
\end{argument}

\noindent Et cependant si vous vivez, vous le devez à Dieu, à ce Dieu qui ne vous épargne que pour vous avertir de vous corriger et de faire pénitence, à ce Dieu qui a permis que malgré votre ingratitude vous ayez évité la fureur des ennemis, soit en vous couvrant du nom de ses serviteurs, soit en vous réfugiant dans les églises de ses martyrs.\par
On dit que Rémus et Romulus, pour peupler leur ville, établirent un asile où les plus grands criminels étaient assurés de l’impunité. Exemple remarquable et qui s’est renouvelé de nos jours à l’honneur du Christ ! Ce qu’avaient ordonné les fondateurs de Rome, ses destructeurs l’ont également ordonné. Mais quelle merveille que ceux-là aient fait pour augmenter le nombre de leurs citoyens ce que ceux-ci ont fait pour augmenter le nombre de leurs ennemis ?
\subsection[{Chapitre XXXV}]{Chapitre XXXV}

\begin{argument}\noindent L’Église a des enfants cachés parmi ses ennemis et de faux amis parmi ses enfants.
\end{argument}

\noindent Tels sont les moyens de défense (et il y en a peut-être de plus puissants encore) que nous pouvons opposer à nos ennemis, nous enfants du Seigneur Jésus, rachetés de son sang et membres de la cité ici-bas étrangère, de la cité royale du Christ. N’oublions pas toutefois qu’au milieu de ces ennemis mêmes se cache plus d’un concitoyen futur, ce qui doit nous faire voir qu’il n’est pas sans avantage de supporter patiemment comme adversaire de notre foi celui qui peut en devenir confesseur. De même, au sein de la cité de Dieu, pendant du moins qu’elle accomplit son voyage à travers ce monde, plus d’un qui est uni à ses frères par la communion des mêmes sacrements, sera banni un jour de la société des saints. De ces faux amis, les uns se tiennent dans l’ombre, les autres osent mêler ouvertement leur voix à celle de nos adversaires, pour murmurer contre le Dieu dont ils portent la marque sacrée, jouant ainsi deux rôles contraires et fréquentant également les théâtres et les lieux saints. Faut-il cependant désespérer de leur conversion ? Non, certes, puisque parmi nos ennemis les plus déclarés, nous avons des amis prédestinés encore inconnus à eux-mêmes. Les deux cités, en effet, sont mêlées et confondues ensemble pendant cette vie terrestre jusqu’à ce qu’elles se séparent au dernier jugement. Exposer leur naissance, leur progrès et leur fin, c’est ce que je vais essayer de faire, avec l’assistance du ciel et pour la gloire de la cité de Dieu, qui tirera de ce contraste mi plus vif éclat.
\subsection[{Chapitre XXXVI}]{Chapitre XXXVI}

\begin{argument}\noindent Des sujets qu’il conviendra de traiter dans les livres suivants.
\end{argument}

\noindent Mais avant d’aborder cette entreprise, j’ai encore quelque chose à répondre à ceux qui rejettent les malheurs de l’empire romain sur notre religion, sous prétexte qu’elle défend de sacrifier aux dieux. Il faut pour cela que je rapporte (autant du moins que ma mémoire et le besoin de mon sujet le permettront) tous les maux qui sont arrivés à l’empire ou aux provinces qui en dépendent avant que cette défense n’eût été faite : calamités qu’ils ne manqueraient pas de nous attribuer, si notre religion eût paru dès ce temps-là et interdit leurs sacrifices impies. Je montrerai ensuite pourquoi le vrai Dieu, qui tient en sa main tous les royaumes de la terre, a daigné accroître le leur, et je ferai voir que leurs prétendus dieux, loin d’y avoir contribué, y ont plutôt nui, au contraire, par leurs fourberies et leurs prestiges. Je terminerai en réfutant ceux qui, convaincus sur ce dernier point par des preuves si claires, se retranchent à soutenir qu’il faut servir les dieux, non pour les biens de la vie présente, mais pour ceux de la vie future. Ici la question, si je ne me trompe, devient plus difficile et monte vers les régions sublimes. Nous avons affaire à des philosophes, non pas aux premiers venus d’entre eux, mais aux plus illustres et aux plus excellents, lesquels sont d’accord avec nous sur plusieurs choses, puisqu’ils reconnaissent l’âme immortelle et le vrai Dieu, auteur et providence de l’univers. Mais comme ils ont aussi beaucoup d’opinions contraires aux nôtres, nous devons les réfuter et nous ne faillirons pas à ce devoir. Nous combattrons donc leurs assertions impies dans toute la force qu’il plaira à Dieu de nous départir, pour l’affermissement de la cité sainte, de la vraie piété et du culte de Dieu, sans lequel on ne saurait parvenir à la félicité promise. Je termine ici ce livre, afin de passer au nouveau sujet que je me propose de traiter.
\section[{Livre deuxième. Rome et les faux dieux}]{Livre deuxième. \\
Rome et les faux dieux}\renewcommand{\leftmark}{Livre deuxième. \\
Rome et les faux dieux}

\subsection[{Chapitre premier}]{Chapitre premier}

\begin{argument}\noindent Il est nécessaire de ne point prolonger les discussions au-delà d’une certaine mesure.
\end{argument}

\noindent Si le faible esprit de l’homme, au lieu de résister à l’évidence de la vérité, voulait se soumettre aux enseignements de la saine doctrine, comme un malade aux soins du médecin, jusqu’à ce qu’il obtînt de Dieu par sa foi et sa piété la grâce nécessaire pour se guérir, ceux qui ont des idées justes et qui savent les exprimer convenablement n’auraient pas besoin d’un long discours pour réfuter l’erreur. Mais comme l’infirmité dont nous parlons est aujourd’hui plus grande que jamais, à ce point que l’on voit des insensés s’attacher aux mouvements déréglés de leur esprit comme à la raison et à la vérité même, tantôt par l’effet d’un aveuglement qui leur dérobe la lumière, tantôt par suite d’une opiniâtreté qui la leur fait repousser, on est souvent obligé, après leur avoir déduit ses raisons autant qu’un homme le doit attendre de son semblable, de s’étendre beaucoup sur des choses très claires, non pour les montrer à ceux qui les regardent, mais pour les faire toucher à ceux qui ferment les yeux de peur de les voir. Et cependant, si on se croyait tenu de répondre toujours aux réponses qu’on reçoit, quand finiraient les discussions ?\par
Ceux qui ne peuvent comprendre ce qu’on dit, ou qui, le comprenant, ont l’esprit trop dur et trop rebelle pour y souscrire, répondent toujours ; mais, comme dit l’Écriture : « Ils ne parlent que le langage de l’iniquité » ; et leur opiniâtreté infatigable est vaine. Si donc nous consentions à les réfuter autant de fois qu’ils prennent avec un front d’airain la résolution de ne pas se mettre en peine de ce qu’ils disent, pourvu qu’ils nous contredisent n’importe comment, vous voyez combien notre labeur serait pénible, infini et stérile, C’est pourquoi je ne souhaiterais pas avoir pour juges de cet ouvrage, ni vous-même, Marcellinus, mon cher fils, ni aucun de ceux à qui je l’adresse dans un esprit de discussion utile et loyale et de charité chrétienne, s’il vous fallait toujours des réponses, dès que vous verriez paraître un argument nouveau ; j‘aurais trop peur alors que vous ne devinssiez semblables à ces malheureuses femmes dont parle l’Apôtre, « qui incessamment apprennent sans jamais savoir la vérité ».
\subsection[{Chapitre II}]{Chapitre II}

\begin{argument}\noindent Récapitulation de ce qui a été traité dans le premier livre.
\end{argument}

\noindent Ayant commencé, dans le livre précédent, de traiter de la Cité de Dieu, à laquelle j’ai résolu, avec l’assistance d’en haut, de consacrer tout cet ouvrage, mon premier soin a été de répondre à ceux qui imputent les guerres dont l’univers est en ce moment désolé, et surtout le dernier malheur de Rome, à la religion chrétienne, sous prétexte qu’elle interdit les sacrifices abominables qu’ils voudraient faire aux démons. J’ai donc fait voir qu’ils devraient bien plutôt attribuer à l’influence du Christ le respect que les barbares ont montré pour son nom, en leur laissant, contre l’usage de la guerre, de vastes églises pour lieu de refuge, et en honorant à tel point leur religion (celle du moins qu’ils feignaient de professer), qu’ils ne se sont pas cru permis contre eux ce que leur permet contre tous le droit de la victoire. Delà s’est élevée une question nouvelle : pourquoi cette faveur divine s’est-elle étendue à des impies et à des ingrats, et pourquoi, d’un autre côté, les désastres de la guerre ont-ils également frappé les impies et les hommes pieux ? Je me suis quelque peu arrêté sur ce point, d’abord parce que cette répartition ordinaire des bienfaits de la Providence et des misères de l’humanité tombant indifféremment sur les bons et sur les méchants, porte le trouble dans plus d’une conscience ; puis j’ai voulu, et ç’a été mon principal objet, consoler de saintes femmes, chastes et pieuses victimes d’une violence qui a pu attrister leur pudeur, mais non souiller leur pureté, de peur qu’elles ne se repentent de vivre, elles qui n’ont rien dans leur vie dont elles aient à se repentir. J’ai ajouté ensuite quelques réflexions contre ceux qui osent insulter aux infortunes subies par les chrétiens et en particulier par ces malheureuses femmes restées chastes et saintes dans l’humiliation de leur pudeur ; adversaires sans bonne foi et sans conscience, indignes enfants de ces Romains renommés par tant de belles actions dont l’histoire conservera le souvenir, mais qui ont trouvé dans leurs descendants dégénérés les plus grands ennemis de leur gloire. Rome, en effet, fondée par leurs aïeux et portée à un si haut point de grandeur, ils l’avaient plus abaissée par leurs vices qu’elle ne l’a été par sa chute ; car cette chute n’a fait tomber que des pierres et du bois, au lieu que leurs vices avaient ruiné leurs mœurs, fondement et ornement des empires, et allumé dans les âmes des passions mille fois plus dévorantes que les feux qui ont consumé les palais de Rome. C’est par là que j’ai terminé le premier livre. Mon dessein maintenant est d’exposer les maux que Rome a soufferts depuis sa naissance, soit dans l’intérieur de l’empire, soit dans les provinces, soumises ; longue suite de calamités que nos adversaires ne manqueraient pas d’attribuer à la religion chrétienne, si, dès ce temps-là, la doctrine de l’Évangile eût fait librement retentir sa voix contre leurs fausses et trompeuses divinités.
\subsection[{Chapitre III}]{Chapitre III}

\begin{argument}\noindent Il suffit de consulter l’histoire pour voir quels maux sont arrivés aux Romains pendant qu’ils adoraient les dieux et avant l’établissement de la religion chrétienne.
\end{argument}

\noindent En lisant le récit que je vais tracer, il faut se souvenir que parmi les adversaires à qui je m’adresse il y a des ignorants qui ont fait naître ce proverbe : « La pluie manque, c’est la faute des chrétiens. » Il en est d’autres, je le sais, qui, munis d’études libérales, aiment l’histoire et connaissent les faits que j’ai dessein de rappeler ; mais afin de nous rendre odieux à la foule ignorante, ils feignent de ne pas les savoir et s’efforcent de faire croire au vulgaire que les désastres qui, selon l’ordre de la nature, affligent les hommes à certaines époques et dans certains lieux, n’arrivent présentement qu’à cause des progrès du christianisme qui se répand partout avec un éclat et une réputation incroyables, au détriment du culte des dieux. Qu’ils se souviennent donc avec nous de combien de calamités Rome a été accablée avant que Jésus-Christ ne se fût incarné, avant que son nom n’eût brillé parmi les peuples de cette gloire dont ils sont vainement jaloux. Comment justifieront-ils leurs dieux sur ce point, puisque, de leur propre aveu, ils ne les servent que pour se mettre à couvert de ces calamités qu’il leur plaît maintenant de nous imputer ? Je les prie de me dire pourquoi ces dieux ont permis que de si grands désastres arrivassent à leurs adorateurs avant que le nom de Jésus-Christ, partout proclamé, ne vînt offenser leur orgueil et mettre un terme à leurs sacrifices.
\subsection[{Chapitre IV}]{Chapitre IV}

\begin{argument}\noindent Les idolâtres n’ont jamais reçu de leurs dieux aucun précepte de vertu, et leur culte a été souillé de toutes sortes d’infamies.
\end{argument}

\noindent Et d’abord pourquoi ces dieux ne se sont-ils point mis en peine d’empêcher le dérèglement des mœurs ? Que le Dieu véritable se soit détourné des peuples qui ne le servaient pas, ç’a été justice ; mais d’où vient que les dieux, dont on regrette que le culte soit aujourd’hui interdit, n’ont établi aucune loi pour porter leurs adorateurs à la vertu ? La justice aurait voulu qu’ils eussent des soins pour les actions des hommes, en échange de ceux que les hommes rendaient à leurs autels. On dira que nul n’est méchant que par le fait de sa volonté propre. Qui le nie ? mais ce n’en était pas moins l’office des dieux de ne pas laisser ignorer à leurs adorateurs les préceptes d’une vie honnête, de les promulguer au contraire avec le plus grand éclat, de dénoncer les pécheurs par la bouche des devins et des oracles, d’accuser, de menacer hautement les méchants et de promettre des récompenses aux bons. Or, a-t-on jamais entendu rien prêcher de semblable dans leurs temples ? Quand j’étais jeune, je me souviens d’y être allé plus d’une fois ; j’assistais à ces spectacles et à ces jeux sacrilèges ; je contemplais les prêtres en proie à leur délire démoniaque, j’écoutais les musiciens, je prenais plaisir à ces jeux honteux qu’on célébrait en l’honneur des dieux, des déesses, de la vierge Célestis, de Cybèle, mère de tous les dieux. Le jour où on lavait solennellement dans un fleuve cette dernière divinité, de misérables bouffons chantaient devant son char des vers tellement infâmes qu’il n’eût pas été convenable, je ne dis pas à la mère des dieux, mais à la mère d’un sénateur, d’un, honnête homme, d’un de ces bouffons même, de prêter l’oreille à ces turpitudes. Car enfin tout homme a un sentiment de respect pour ses parents que la vie la plus dégradante ne saurait étouffer. Ainsi ces baladins auraient rougi de répéter chez eux et devant leurs mères, ne fût-ce que pour s’exercer, ces paroles et ces gestes obscènes dont ils honoraient la mère des dieux, en présence d’une multitude immense où les deux sexes étaient confondus. Et je ne doute pas que ces spectateurs qui s’empressaient à la fête, attirés par la curiosité, ne rentrassent à la maison, révoltés par l’infamie. Si ce sont là des choses sacrées, qu’appellerons-nous choses sacrilèges ? et qu’est-ce qu’une souillure, si c’est là une purification ? Ne donnait-on pas à ces fêtes le nom de {\itshape Services} ({\itshape Fercula}), comme si on eût célébré un festin où les démons pussent venir se repaître de leurs mets favoris ? Chacun sait, en effet, combien ces esprits immondes sont avides de telles obscénités ; il faudrait, pour en douter, ignorer l’existence de ces démons qui trompent les hommes en se faisant passer pour des dieux, ou bien vivre de telle sorte que leur protection parût plus à désirer que celle du vrai Dieu, et leur colère plus à craindre.
\subsection[{Chapitre V}]{Chapitre V}

\begin{argument}\noindent Des cérémonies obscènes qu’on célébrait en l’honneur de la mère des dieux.
\end{argument}

\noindent Je voudrais avoir ici pour juges, non ces hommes corrompus qui aiment mieux prendre du plaisir à des coutumes infâmes, que se donner de la peine pour les combattre, mais cet illustre Scipion Nasica, autrefois choisi par le sénat, comme le meilleur citoyen de Rome, pour aller recevoir Cybèle, et promener solennellement dans la ville la statue de ce démon. Je lui demanderais s’il ne souhaiterait pas que sa mère eût assez bien mérité de la république pour qu’on lui décernât les honneurs divins, comme à ces mortels privilégiés, devenus immortels et rangés au nombre des dieux par l’admiration et la reconnaissance des Grecs, des Romains et d’autres peuples°. Sans aucun doute, il souhaiterait un pareil bonheur à sa mère, si la chose était possible ; mais supposons qu’on lui demande après cela s’il voudrait que parmi ces honneurs divins on mêlât les chants obscènes de Cybèle. Ne s’écriera-t-il pas qu’il aimerait mieux pour sa mère qu’elle fût morte et privée de tout sentiment que d’être déesse pour se complaire à ces infamies ? Quelle apparence, en effet, qu’un sénateur romain, assez sévère de mœurs pour avoir empêché qu’on ne bâtît un théâtre dans une ville qu’il voulait peuplée d’hommes forts, souhaitât pour sa mère un culte qui fait accueillir avec faveur par une déesse des paroles dont une matrone se regarderait comme offensée ? Assurément il ne croirait point qu’une femme d’honneur, en devenant déesse, eût perdu à ce point la modestie, ni qu’elle pût écouter avec plaisir, de la bouche de ses adorateurs, des mots tellement impurs que si elle en eût entendu de pareils de son vivant, sans se boucher les oreilles et se retirer, ses proches, son mari et ses enfants eussent été obligés d’en rougir pour elle. Ainsi, cette mère des dieux, que le dernier des hommes refuserait d’avouer pour sa mère, voulant capter l’esprit des Romains, désigna pour venir au-devant d’elle le premier des citoyens, non pour le confirmer dans sa vertu par ses conseils et son assistance, mais pour le tromper par ses artifices, semblable à cette femme dont il est écrit : « Elle s’efforce de dérober aux hommes leur bien le plus précieux, qui est leur âme. » Que désirait-elle autre chose, en effet, en désignant Scipion, si ce n’est que ce grand homme, exalté par le témoignage d’une déesse, et se croyant arrivé au comble de la perfection, vînt à négliger désormais la vraie piété et la vraie religion, sans lesquelles pourtant le plus noble caractère tombe dans l’orgueil et se perd ? Et comment ne pas attribuer le choix fait par cette déesse à un dessein insidieux, quand on la voit se complaire dans ses fêtes à des obscénités que les honnêtes gens auraient horreur de supporter dans leurs festins ?
\subsection[{Chapitre VI}]{Chapitre VI}

\begin{argument}\noindent Les dieux des païens ne leur ont jamais enseigné les préceptes d’une vie honnête.
\end{argument}

\noindent C’est pour cela que ces divinités n’ont pris aucun soin pour régler les mœurs des cités et des peuples qui les adoraient, ni pour les préserver par de terribles et salutaires défenses de ces maux effroyables qui ont leur siège, non dans les champs et les vignes, non dans les maisons et les trésors, non dans le corps, qui est soumis à l’esprit ; mais dans l’esprit même qui gouverne le corps. Dira-t-on que les dieux défendaient de mal vivre ? Qu’on le montre, qu’on le prouve. Et il ne s’agit pas ici de nous vanter je ne sais quelles traditions secrètes murmurées à l’oreille d’un petit nombre d’initiés par une religion mystérieuse, amie prétendue de la chasteté et de la vertu ; qu’on nous cite, qu’on désigne les lieux, les assemblées, ou, à la place de ces fêtes impudiques, de ces chants et de ces postures d’histrions obscènes, à la place de ces {\itshape Fugalies} honteuses (vraiment faites pour mettre en fuite la pudeur et l’honnêteté), en un mot, à la place de toutes ces turpitudes, on ait enseigné au peuple, au nom des dieux, à réprimer l’avarice, à contenir l’ambition, à brider l’impudicité, à suivre enfin tous les préceptes que rappelle Perse en ces vers énergiques :\par
 {\itshape « Instruisez-vous, misérables mortels, et apprenez les raisons des choses, ce que nous sommes, le but de la vie et sa loi, la pente glissante qui nous entraîne au mal, la modération dans l’amour des richesses, les désirs légitimes, l’usage utile de l’argent, la générosité qui sied à l’honnête homme envers la patrie et ses proches, enfin ce que chacun doit être dans le poste où Dieu l’a placé.   »} \par
Qu’on nous dise en quels lieux on faisait entendre ces préceptes comme émanés de la bouche des dieux, en quels lieux on habituait le peuple à les écouter, comme cela se fait dans nos églises partout où la religion chrétienne a pénétré.
\subsection[{Chapitre VII}]{Chapitre VII}

\begin{argument}\noindent Les maximes inventées par les philosophes ne pouvaient servir à rien, étant dépourvues d’autorité divine et s’adressant à un peuple plus porté à suivre les exemples des dieux que les maximes des raisonneurs.
\end{argument}

\noindent On nous alléguera peut-être les systèmes et les controverses des philosophes. Je répondrai d’abord que ce n’est point Rome, mais la Grèce qui leur a donné naissance ; et si l’on persiste à vouloir en faire honneur à Rome, sous prétexte que la Grèce a été réduite en province romaine ; je dirai alors que les systèmes philosophiques ne sont point l’ouvrage des dieux, mais de quelques hommes doués d’un esprit rare et pénétrant, qui ont entrepris de découvrir par la raison la nature des choses, la règle des mœurs, enfin les conditions de l’usage régulier de la raison elle-même, tantôt fidèle et tantôt infidèle à ses propres lois. Aussi bien, parmi ces philosophes, quelques-uns ont découvert de grandes choses, soutenus qu’ils étaient par l’appui divin ; mais, arrêtés dans leur essor par la faiblesse humaine, ils sont tombés dans l’erreur ; juste répression de la divine Providence, qui a voulu surtout punir leur orgueil, et montrer, par l’exemple de ces esprits puissants, que la véritable voie pour monter aux régions supérieures, c’est l’humilité. Mais le moment viendra plus tard, s’il plaît au vrai Dieu notre Seigneur, de traiter cette matière et de la discuter à fond. Quoi qu’il en soit, s’il est vrai que, les philosophes aient découvert des vérités capables de donner à l’homme la vertu et le bonheur, n’est-ce point à eux qu’il eût fallu, pour être plus juste, décerner les honneurs divins ? Combien serait-il plus convenable et plus honnête de lire les livrés de Platon, dans un temple consacré à ce philosophe, que de voir des prêtres de Cybèle se mutiler dans le temple des démons, des efféminés s’y faire consacrer, des insensés s’y inciser le corps, cérémonies cruelles, honteuses, cruellement honteuses, honteusement cruelles, qui sont chaque jour célébrées en l’honneur des dieux ? Combien aussi serait-il plus utile, pour former la jeunesse à la vertu, de lire publiquement de bonnes lois, au nom des dieux, que de louer vainement celles des ancêtres ! En effet, tous les adorateurs de dieux pareils, lorsque {\itshape le poison brûlant de la passion}, comme dit Perse, s’est insinué dans leur âme, peu leur importe ce qu’enseignait Platon ou ce que Platon censurait, ils regardent ce que faisait Jupiter. De là ce jeune débauché de Térence qui, jetant les yeux sur le mur de la salle, et y voyant une peinture où Jupiter fait couler une pluie d’or dans le sein de Danaé, se sert d’un si grand exemple pour autoriser ses désordres, et se vanter d’imiter Dieu\par
 {\itshape « Et quel Dieu ? Celui qui ébranle de son tonnerre les temples du ciel. Certes, je n’en ferais pas autant, moi, chétif mortel, mais, pour le reste, je l’ai fait, et de grand cœur.   »} 
\subsection[{Chapitre VIII}]{Chapitre VIII}

\begin{argument}\noindent Les jeux scéniques, où sont étalées toutes les turpitudes des dieux, loin de leur déplaire, servent à les apaiser.
\end{argument}

\noindent Mais, dira-t-on, ce sont là des inventions de poules, et non les enseignements de la religion. Je ne veux pas répondre que ces enseignements sont encore plus scandaleux ; je me contente de prouver, l’histoire à la main, que ces jeux solennels, où l’on représente les fictions des poètes, n’ont pas été introduits dans les fêtes des dieux par l’ignorance et la superstition des Romains, mais que ce sont les dieux eux-mêmes, comme je l’ai indiqué au livre précédent, qui ont prescrit de les célébrer, et les ont pour ainsi dire violemment imposés par la menace. C’est, en effet, au milieu des ravages croissants d’une peste que les jeux scéniques furent institués à Rome pour la première fois par l’autorité des pontifes. Or, quel est celui qui, pour la conduite de sa vie, ne se conformera pas de préférence aux exemples donnés par les dieux dans les cérémonies consacrées par la religion, qu’aux préceptes inscrits dans les lois par une sagesse toute profane ? Si les poules ont menti, quand ils ont représenté Jupiter adultère, des dieux vraiment chastes auraient dû se courroucer et se venger d’un pareil scandale, au lieu de l’encourager et de le prescrire. Et cependant, ce qu’il y a de plus supportable dans ces jeux scéniques, ce sont les comédies et les tragédies, c’est-à-dire ces pièces imaginées par les poètes, où l’immoralité des actions n’est pas du moins aggravée par l’obscénité des paroles, ce qui fait comprendre qu’on leur donne place dans l’étude des belles-lettres, et que des personnes d’âge en imposent la lecture aux enfants.
\subsection[{Chapitre IX}]{Chapitre IX}

\begin{argument}\noindent Les anciens Romains jugeaient nécessaire de réprimer la licence des poètes, à la différence des Grecs qui ne leur imposaient aucune limite, se conformant en ce point a la volonté des dieux.
\end{argument}

\noindent Si l’on veut savoir ce que pensaient à cet égard les anciens Romains, il faut consulter Cicéron qui, dans son traité {\itshape De la République}, fait parler Scipion en ces termes : « Jamais la comédie, si l’habitude des mœurs publiques ne l’avait autorisée, n’aurait pu faire goûter les infamies qu’elle étalait sur le théâtre. Les Grecs du moins étaient conséquents dans leur extrême licence, puisque leurs lois permettaient à la comédie de tout dire sur tout citoyen et en l’appelant par son nom. » Aussi, comme dit encore Scipion dans le même ouvrage : « Qui n’a-t-elle pas atteint ? Ou plutôt, qui n’a-t-elle pas déchiré ? À qui fit-elle grâce ? Qu’elle ait blessé des flatteurs populaires, des citoyens malfaisants, séditieux, Cléon, Cléophon, Hyperbolus, à la bonne heure ; bien que, pour de tels hommes, la censure du magistrat vaille mieux que celle du poète. Mais que Périclès, gouvernant la république depuis tant d’années avec le plus absolu crédit, dans la paix ou dans la guerre, soit outragé par des vers, et qu’on les récite sur la scène, cela n’est pas moins étrange que si, parmi nous, Plaute et Névius se fussent avisés de médire de Publius et de Cnéus Scipion, ou Cécilius de Caton. » Et il ajoute un peu après « Nos lois des douze Tables, au contraire, si attentives à ne porter la peine de mort que pour un bien petit nombre de faits, ont compris dans cette classe le délit d’avoir récité publiquement ou d’avoir composé des vers qui attireraient sur autrui le déshonneur et l’infamie ; et elles ont sagement décidé ; car notre vie doit être soumise à la sentence des tribunaux, à l’examen légitime des magistrats, et non pas aux fantaisies des poètes ; et nous ne devons être exposés à entendre une injure qu’avec le droit d’y répondre et de nous défendre devant la justice. » Il est aisé de voir combien tout ce passage du quatrième livre de la République de Cicéron, que je viens de citer textuellement (sauf quelques mots omis ou modifiés), se rattache étroitement à la question que je veux éclaircir. Cicéron ajoute beaucoup d’autres réflexions, et conclut en montrant fort bien que les anciens Romains ne pouvaient souffrir qu’on louât ou qu’on blâmât sur la scène un citoyen vivant. Quant aux Grecs, qui autorisèrent cette licence, je répète, tout en la flétrissant, qu’on y trouve une sorte d’excuse, quand on considère qu’ils voyaient leurs dieux prendre plaisir au spectacle de l’infamie des hommes et de leur propre infamie, soit que les actions qu’on leur attribuait fussent de l’invention des poètes, soit qu’elles fussent véritables ; et plût à Dieu que les spectateurs n’eussent fait qu’en rire, au lieu de les imiter ! Au fait, c’eût été un peu trop superbe d’épargner la réputation des principaux de la ville et des simples citoyens, pendant que les dieux sacrifiaient la leur de si bonne grâce.
\subsection[{Chapitre X}]{Chapitre X}

\begin{argument}\noindent C’est un trait de la profonde malice des démons, de vouloir qu’on leur attribue des crimes, soit véritables, soit supposés.
\end{argument}

\noindent On allègue pour excuse que ces actions attribuées aux dieux ne sont pas véritables, mais supposées. Le crime alors n’en serait que plus énorme, si l’on consulte les notions de la vraie piété et de la vraie religion ; et si l’on considère la malice des démons, quel art profond pour tromper les hommes ! Quand on diffame un des premiers de l’État qui sert honorablement son pays, cette attaque n’est-elle pas d’autant plus inexcusable qu’elle est plus éloignée de la vérité ? Quel supplice ne méritent donc pas ceux qui font à Dieu une injure si atroce et si éclatante ! Au reste, ces esprits du mal, que les païens prennent pour des dieux, n’ont d’autre but, en se laissant attribuer de faux crimes, que de prendre les âmes dans ces fictions comme dans des filets, et de les entraîner avec eux dans le supplice où ils sont prédestinés ; soit que des hommes qu’ils se plaisent à faire passer pour des dieux, afin de recevoir à leur place par mille artifices les adorations des mortels, aient en effet commis ces crimes, soit qu’aucun homme n’en étant coupable, ils prennent plaisir à les voir imputer aux dieux, pour donner ainsi aux actions les plus méchantes elles plus honteuses l’autorité du ciel. C’est ainsi que les Grecs, esclaves de ces fausses divinités, n’ont pas cru que les poètes dussent les épargner eux-mêmes sur la scène, ou par le désir de se rendre en cela semblables à leurs dieux, ou par la crainte de les offenser, s’ils se montraient jaloux d’avoir une renommée meilleure que la leur.
\subsection[{Chapitre XI}]{Chapitre XI}

\begin{argument}\noindent Les Grecs admettaient les comédiens à l’exercice des fonctions publiques, convaincus qu’il y avait de l’injustice à mépriser des hommes dont l’art apaisait la colère des dieux.
\end{argument}

\noindent Les Grecs furent encore très conséquents avec eux-mêmes quand ils jugèrent les comédiens dignes des plus hautes charges de l’État. Nous apprenons, en effet, par Cicéron, dans ce même traité {\itshape De la République}, que l’Athénien Eschine, homme très éloquent, après avoir joué la tragédie dans sa jeunesse, brigua la suprême magistrature, et que les Athéniens envoyèrent souvent le comédien Aristodème en ambassade vers Philippe, pour traiter les affaires les plus importantes de la paix et de la guerre. Voyant leurs dieux accueillir avec complaisance les pièces de théâtre, il ne leur paraissait pas raisonnable de mettre au rang des personnes infâmes ceux qui servaient à les représenter. Nul doute que tous ces usages des Grecs ne fussent très scandaleux, mais nul doute aussi qu’ils ne fussent en harmonie avec le caractère de leurs dieux ; car comment auraient-ils empêché les poètes et les acteurs de déchirer les citoyens, quand ils les entendaient diffamer leurs dieux avec l’approbation de ces dieux mêmes ? Et comment auraient-ils méprisé, ou plutôt comment n’auraient-ils pas élevé aux premiers emplois ceux qui représentaient sur le théâtre des pièces qu’ils savaient agréables aux dieux ? Eût-il été raisonnable, tandis qu’on avait les prêtres en honneur, parce qu’ils attirent sur les hommes la protection des dieux en leur immolant des victimes, de noter d’infamie les comédiens qui, en jouant des pièces de théâtre, ne faisaient autre chose que satisfaire au désir des dieux et prévenir l’effet de leurs menaces, d’après la déclaration expresse des prêtres eux-mêmes ? Car nous savons que Labéon, dont l’érudition fait autorité en cette matière, distingue les bonnes divinités d’avec les mauvaises, et veut qu’on leur rende un culte différent, conseillant d’apaiser les mauvaises par des sacrifices sanglants et par des prières funèbres, et de se concilier les bonnes par des offrandes joyeuses et agréables, comme les jeux, les festins et les lectisternes. Nous discuterons plus tard, s’il plaît à Dieu, cette distinction de Labéon ; mais, pour n’en dire en ce moment que ce qui touche à notre sujet, soit que l’on offre indifféremment toutes choses à tous les dieux comme étant tous bons (car des dieux ne sauraient être mauvais, et ceux des païens ne sont tels que parce qu’ils sont tous des esprits immondes), soit que l’on mette quelque différence, comme le veut Labéon, dans les offrandes qu’on présente aux différents dieux, c’est toujours avec raison que les Grecs honorent les comédiens qui célèbrent les jeux, à l’égal des prêtres qui offrent des victimes, de peur de faire injure à tous les dieux, si tous aiment les jeux du théâtre, ou, ce qui serait plus grave encore, aux dieux réputés bons, s’il n’y a que ceux-là qui les voient avec plaisir.
\subsection[{Chapitre XII}]{Chapitre XII}

\begin{argument}\noindent Les Romains, en interdisant aux poètes d’user contre les hommes d’une liberté qu’ils leur donnaient contre les dieux, ont eu moins bonne opinion des dieux que d’eux-mêmes.
\end{argument}

\noindent Les Romains ont tenu à cet égard une conduite toute différente, comme s’en glorifie Scipion dans le dialogue déjà cité {\itshape De la République}. Loin de consentir à ce que leur vie et leur réputation fussent exposées aux injures et aux médisances des poètes, ils prononcèrent la peine capitale contre ceux qui oseraient composer des vers diffamatoires. C’était pourvoir à merveille au soin de leur honneur, mais c’était aussi se conduire envers les dieux d’une façon bien superbe et bien impie ; car enfin ils voyaient ces dieux supporter avec patience et même écouter volontiers les injures et les sarcasmes que leur adressaient les poètes, et, malgré cet exempte, ils ne crurent pas de leur dignité de supporter des insultes toutes pareilles ; de sorte qu’ils établirent des lois pour s’en garantir au moment même où ils permettaient que l’outrage fît partie des solennités religieuses. Ô Scipion ! comment pouvez-vous louer les Romains d’avoir défendu aux poètes d’offenser aucun citoyen, quand vous voyez que ces mêmes poètes n’ont épargné aucun de vos dieux ! Avez-vous estimé si haut la gloire du sénat comparée à celle du dieu du Capitole, que dis-je ? la gloire de Rome seule mise en balance avec celle de tout le ciel, que vous ayez lié par une loi expresse la langue médisante des poètes, si elle était dirigé contre un de vos concitoyens, tandis que vous la laissiez libre de lancer l’insulte à son gré contre tous vos dieux, sans que personne, ni sénateur, ni censeur, ni prince du sénat, ni pontife, eût le droit de s’y opposer ? Quoi il vous a paru scandaleux que Plaute ou Névius pussent attaquer les Scipions, ou que Caton fût insulté par Cécilius, et vous avez trouvé bon que votre Térence excitât les jeunes gens au libertinage par l’exemple du grand Jupiter !
\subsection[{Chapitre XIII}]{Chapitre XIII}

\begin{argument}\noindent Les Romains auraient dû comprendre que des dieux capables de se complaire à des jeux infâmes n’étaient pas dignes des honneurs divins.
\end{argument}

\noindent Scipion, s’il vivait, me répondrait peut-être : Comment ne laisserions-nous pas impunies des injures que les dieux eux-mêmes ont consacrées, puisque ces jeux scéniques, où on les fait agir et parler d’une manière si honteuse, ont été institués en leur honneur et sont entrés dans les mœurs de Rome par leur commandement formel ? — À quoi je réplique en demandant à mon tour comment cette conduite des dieux n’a pas fait comprendre aux Romains qu’ils n’avaient point affaire à des dieux véritables, mais à des démons indignes de recevoir d’une telle république les honneurs divins ? Assurément, il n’eût point été convenable, ni le moins du monde obligatoire de leur rendre un culte, s’ils eussent exigé des cérémonies injurieuses à la gloire des Romains ; comment dès lors, je vous prie, a-t-on pu juger dignes d’adoration ces esprits de mensonge dont la méprisable impudence allait jusqu’à demander que le tableau de leurs crimes fit partie de leurs honneurs ? Aussi, quoique assez aveuglés par la superstition pour adorer ces divinités étranges qui prétendaient donner un caractère sacré aux infamies du théâtre, les Romains, par un sentiment de pudeur et de dignité, refusèrent aux comédiens les honneurs que leur accordaient les Grecs. C’est ce que déclare Cicéron par la bouche de Scipion : « Regardant, dit-il, l’art des comédiens et le théâtre en général comme infâmes, les Romains ont interdit aux gens de cette espèce l’honneur des emplois publics ; bien plus, ils les ont fait exclure de leur tribu par une note du censeur. » Voilà, certes, un règlement d’une de la sagesse des Romains ; mais j’aurais voulu que tout le reste y eût répondu et qu’ils eussent été conséquents avec eux-mêmes. Qu’un citoyen romain, quel qu’il fût, du moment qu’il se faisait comédien, fût exclu de tout honneur public, que le censeur ne souffrît même pas qu’il demeurât dans sa tribu, cela est admirable, cela est digne d’un peuple dont la grande âme adorait la gloire, cela est vraiment romain ! Mais qu’on me dise s’il y avait quelque raison et quelque conséquence à exclure les comédiens de tout honneur, tandis que les comédies faisaient partie des honneurs des dieux. Longtemps la vertu romaine n’avait pas connu ces jeux du théâtre, et s’ils eussent été recherchés par goût du plaisir, on aurait pu en expliquer l’usage par le relâchement des mœurs ; mais non, ce sont les dieux qui ont ordonné de les célébrer. Comment donc flétrir le comédien par qui l’on honore le dieu ? et de quel droit noter d’infamie l’acteur d’une scène honteuse si l’on en adore le promoteur ? Voilà donc la dispute engagée entre les Grecs et les Romains. Les Grecs croient qu’ils ont raison d’honorer les comédiens, puisqu’ils adorent des dieux avides de comédies ; les Romains, au contraire, pensent que la présence d’un comédien serait une injure pour une tribu de plébéiens, et à plus forte raison pour le sénat. La question ainsi posée, voici un syllogisme qui termine tout. Les Grecs en fournissent la majeure : si l’on doit adorer de tels dieux, il faut honorer de tels hommes. La mineure est posée par les Romains : or, il ne faut point honorer de tels hommes. Les chrétiens tirent la conclusion : donc, il ne faut point adorer de tels dieux.
\subsection[{Chapitre XIV}]{Chapitre XIV}

\begin{argument}\noindent Platon, en excluant les poètes d’une cité bien gouvernée, s’est montré supérieur à ces dieux qui veulent être honorés par des jeux scéniques.
\end{argument}

\noindent Je demande encore pourquoi les auteurs de pièces de théâtre, à qui la loi des douze Tables défend de porter atteinte à la réputation des citoyens et qui se permettent de lancer l’outrage aux dieux, ne partagent point l’infamie des comédiens. Quelle raison et quelle justice y a-t-il, quand on couvre d’opprobre les acteurs de ces pièces honteuses et impies, à en honorer les auteurs ? C’est ici qu’il faut donner la palme à un Grec, à Platon, qui, traçant le modèle idéal d’une république parfaite, en a chassé les poètes, comme des ennemis de la vérité. Ce philosophe ne pouvait souffrir ni les injures qu’ils osent prodiguer aux dieux, ni le dommage que leurs fictions causent aux mœurs. Comparez maintenant Platon, qui n’était qu’on homme, chassant les poètes de sa république pour la préserver de l’erreur, avec ces dieux, dont la divinité menteuse voulait être honorée par des jeux scéniques. Celui-là s’efforce, quoique inutilement, de détourner les Grecs légers et voluptueux de la composition de ces honteux ouvrages ; ceux-là en extorquent la représentation à la pudeur des graves Romains. Et il n’a pas suffi aux dieux du paganisme que les pièces du théâtre fussent représentées, il a fallu les leur dédier, les leur consacrer, les célébrer solennellement en leur honneur. À qui donc, je vous prie, serait-il plus convenable de décerner les honneurs divins : à Platon, qui s’est opposé au scandale, ou aux démons qui l’ont voulu, abusant ainsi les hommes que Platon s’efforça vainement de détromper ?\par
Labéon a cru devoir inscrire ce philosophe au rang des demi-dieux, avec Hercule et Romulus. Or, les demi-dieux sont supérieurs aux héros, bien que les uns et les autres soient au nombre des divinités. Pour moi, je n’hésite pas à placer celui qu’il appelle un demi-dieu non seulement au-dessus des héros, mais au-dessus des dieux mêmes. Quoi qu’il en soit, les lois romaines approchent assez des sentiments de Platon ; si, en effet, Platon condamne les poètes et toutes leurs fictions, les Romains leur ôtent du moins la liberté de médire des hommes ; si celui-là les bannit de la cité, ceux-ci excluent du nombre des citoyens ceux qui représentent leurs pièces, et les chasseraient probablement tout à fait s’ils ne craignaient la colère de leurs dieux. Je conclus de là que les Romains ne peuvent recevoir de pareilles divinités ni même en espérer des lois propres à former les bonnes mœurs et à corriger les mauvaises, puisque les institutions qu’ils ont établies par une sagesse tout humaine surpassent et accusent celle des dieux. Les dieux, en effet, demandent des représentations théâtrales : les Romains excluent de tout honneur civil les hommes de théâtre. Ceux-là commandent qu’on étale sur la scène leur propre infamie : ceux-ci défendent de porter atteinte à la réputation des citoyens. Quant à Platon, il paraît ici comme un vrai demi-dieu, puisqu’il s’oppose au caprice insensé des divinités païennes et fait voir en même temps aux Romains ce qui manquait à leurs lois ; convaincu, en effet, que les poètes ne pouvaient être que dangereux, soit en défigurant la vérité dans leurs fictions, soit en proposant à l’imitation des faibles humains les plus détestables exemples donnés par les dieux, il déclara qu’il fallait les bannir sans exception d’un État réglé selon la sagesse. S’il faut dire ici le fond de notre pensée, nous ne croyons pas que Platon soit un dieu ni un demi-dieu ; nous ne le comparons à aucun des saints anges ou des vrais prophètes de Dieu, ni à aucun apôtre ou martyr de Jésus-Christ, ni même à aucun chrétien ; et nous dirons ailleurs, avec la grâce de Dieu, sur quoi se fonde notre sentiment ; mais puisqu’on en veut faire un demi-dieu, nous déclarons volontiers que nous le croyons supérieur, sinon à Hercule et à Romulus (bien qu’il n’ait pas tué son frère et qu’aucun poète ou historien ne lui impute aucun autre crime), du moins à Priape, ou à quelque Cynocéphale, ou enfin à la Fièvre, divinités ridicules que les Romains ont reçues des étrangers ou dont le culte est leur propre ouvrage. Comment donc de pareils dieux seraient-ils capables de détourner ou de guérir les maux qui souillent les âmes et corrompent les mœurs, eux qui prennent soin de répandre et de cultiver la semence de tous les désordres en ordonnant de représenter sur la scène leurs crimes véritables ou supposés, comme pour enflammer à plaisir les passions mauvaises et les autoriser de l’exemple du ciel ! C’est ce qui fait dire à Cicéron, déplorant vainement la licence des poètes : « Ajoutez à l’exemple des dieux les cris d’approbation du peuple, ce grand maître de vertu et de sagesse, quelles ténèbres vont se répandre dans les âmes ! quelles frayeurs les agiter ! quelles passions s’y allumer. »
\subsection[{Chapitre XV}]{Chapitre XV}

\begin{argument}\noindent Les Romains se sont donné certains dieux, non par raison, mais par vanité.
\end{argument}

\noindent Mais n’est-il pas évident que c’est la vanité plutôt que la raison qui les a guidés dans le choix de leurs fausses divinités ? Ce grand Platon, dont ils font un demi-dieu, qui a consacré de si importants ouvrages à combattre les maux les plus funestes, ceux de l’âme qui corrompent les mœurs, Platon n’a pas été jugé digne d’une simple chapelle ; mais pour leur Romulus, ils n’ont pas manqué de le mieux traiter que les dieux, bien que leur doctrine secrète le place au simple rang de demi-dieu. Ils sont allés jusqu’à lui donner un flamine, c’est-à-dire un de ces prêtres tellement considérés chez les Romains, comme le marquait le signe particulier de leur coiffure, que trois divinités seulement en avaient le privilège, savoir : Jupiter, Mars et Romulus ou Quirinus, car ce fut le nom que donnèrent à Romulus ses concitoyens quand ils lui ouvrirent en quelque façon la porte du ciel. Ainsi, ce fondateur de Rome a été préféré à Neptune et à Pluton, frères de Jupiter, et même à Saturne, père de ces trois dieux ; on lui a décerné le même honneur qu’à Jupiter ; et si cet honneur a été étendu à Mars, c’est probablement parce qu’il était père de Romulus.
\subsection[{Chapitre XVI}]{Chapitre XVI}

\begin{argument}\noindent Si les dieux avaient eu le moindre souci de faire régner la justice, ils auraient donné aux romains des préceptes et des lois, au lieu de les leur laisser emprunter aux nations étrangères.
\end{argument}

\noindent Si les Romains avaient pu recevoir des lois de leurs dieux, auraient-ils emprunté aux Athéniens celles de Solon, quelques années après la fondation de Rome ? Et encore ne les observèrent-ils pas telles qu’ils les avaient reçues, mais ils s’efforcèrent de les rendre meilleures. Je sais que Lycurgue avait feint d’avoir reçu les siennes d’Apollon, pour leur donner plus d’autorité sur l’esprit des Spartiates ; mais les Romains eurent la sagesse de n’en rien croire et de ne point puiser à cette source. On rapporte à Numa Pompilius, successeur de Romulus, l’établissement de plusieurs lois, parmi lesquelles un certain nombre qui réglaient beaucoup de choses religieuses ; mais ces lois étaient loin de suffire à la conduite de l’État, et d’ailleurs on ne dit pas que Numa les eût reçues des dieux. Ainsi donc, pour ce qui regarde les maux de l’âme, les maux de la conduite humaine, les maux qui corrompent les mœurs, maux si graves que les plus éclairés parmi les païens ne croient pas qu’un État y puisse résister, même quand les villes restent debout, pour tous les maux de ce genre, les dieux n’ont pris aucun souci d’en préserver leurs adorateurs ; bien au contraire, comme nous l’avons établi plus haut, ils ont tout fait pour les aggraver.
\subsection[{Chapitre XVII}]{Chapitre XVII}

\begin{argument}\noindent De l’enlèvement des Sabines, et des autres iniquités commises par les Romains aux temps les plus vantés de la république.
\end{argument}

\noindent On dira peut-être que si les dieux n’ont pas donné de lois aux Romains, c’est que « le caractère de ce peuple, autant que ses lois, comme dit Salluste, le rendait bon et équitable ». Un trait de ce caractère, ce fut, j’imagine, l’enlèvement des Sabines. Qu’y a-t-il, en effet, de plus équitable et de meilleur que de ravir par force, au gré de chacun, des filles étrangères, après les avoir attirées par l’appât trompeur d’un spectacle ? Parlons sérieusement : si les Sabins étaient injustes en refusant leurs filles, combien les Romains étaient-ils plus injustes en les prenant sans qu’on les leur accordât ? Il eût été plus juste de faire la guerre au peuple voisin pour avoir refusé d’accorder ses filles, que pour avoir redemandé ses filles ravies. Mieux eût donc valu que Romulus se fût conduit de la sorte ; car il n’est pas douteux que Mars n’eût aidé son fils à venger un refus injurieux et à parvenir ainsi à ses fins. La guerre lui eût donné une sorte de droit de s’emparer des filles qu’on lui refusait injustement, au lieu que la paix ne lui en laissait aucun de mettre la main sur des filles qu’on ne lui accordait pas ; et ce fut une injustice de faire la guerre à des parents justement irrités. Heureusement pour eux, les Romains, tout en consacrant par les jeux du cirque le souvenir de l’enlèvement des Sabines, ne pensèrent pas que ce fût un bon exemple à proposer à la république. Ils firent, à la vérité, la faute d’élever au rang des dieux Romulus, l’auteur de cette grande iniquité ; mais on ne peut leur reprocher de l’avoir autorisée par leurs lois ou par leurs mœurs.\par
Quant à l’équité et à la bonté naturelles de leur caractère, je demanderai s’ils en donnèrent une preuve après l’exil de Tarquin. Ce roi, dont le fils avait violé Lucrèce, ayant été chassé de Rome avec ses enfants, le consul Junius Brutus força le mari de Lucrèce, Tarquin Collatin, qui était son collègue et l’homme le plus excellent et le plus innocent du monde, à se démettre de sa charge et même à quitter la ville, par cela seul qu’il était parent des Tarquins et en portait le nom. Et le peuple favorisa ou souffrit cette injustice, quoique ce fût lui qui eût fait Collatin consul aussi bien que Brutus. Je demanderai encore si les Romains montrèrent cette équité et cette bonté tant vantées dans leur conduite à l’égard de Camille. Après avoir vaincu les Véïens, les plus redoutables ennemis de Rome, ce héros qui termina, après dix ans, par la prise de la capitale ennemie, une guerre sanglante où Rome avait été mise à deux doigts de sa perte, fut appelé en justice par la haine de ses envieux et par l’insolence des tribuns du peuple, et trouva tant d’ingratitude chez ses concitoyens qu’il s’en alla volontairement en exil, et fut même condamné en son absence à dix mille as d’amende, lui qui allait devenir bientôt pour la seconde fois, en chassant les Gaulois, le vengeur de son ingrate patrie. Mais il serait trop long de rapporter ici toutes les injustices et toutes les bassesses dont Rome fut le théâtre, à cette époque de discorde, où les patriciens s’efforçant de dominer sur le peuple, et le peuple s’agitant pour secouer le joug, les chefs des deux partis étaient assurément beaucoup plus animés par le désir de vaincre que par l’amour du bien et de l’équité.
\subsection[{Chapitre XVIII}]{Chapitre XVIII}

\begin{argument}\noindent Témoignage de Salluste sur les mœurs du peuple romain, tour à tour contenues par la crainte et relâchées par la sécurité.
\end{argument}

\noindent Au lieu donc de poursuivre, j’aime mieux rapporter le témoignage de ce même Salluste, qui m’a donné occasion d’aborder ce sujet en disant du peuple romain « que son caractère, autant que ses lois, le rendait bon et équitable ». Salluste veut ici glorifier ce temps où Rome, après la chute des rois, prit en très peu d’années d’incroyables accroissements, et cependant il ne laisse pas d’avouer, dès le commencement du premier livre de son {\itshape Histoire}, que dans ce même temps, quand l’autorité passa des rois aux consuls, les patriciens ne tardèrent pas à opprimer le peuple, ce qui occasionna la séparation du peuple et du sénat et une foule de dissensions civiles. En effet, après avoir rappelé qu’entre la seconde et la troisième guerre punique, les bonnes mœurs et la concorde régnaient parmi le peuple romain, heureux état de choses qu’il attribue, non à l’amour de la justice, mais à cette crainte salutaire de l’ennemi que Scipion Nasica voulait entretenir en s’opposant à la ruine de Carthage, l’historien ajoute ces paroles : « Mais, Carthage prise, la discorde, la cupidité, l’ambition, et tous les vices qui naissent d’ordinaire de la prospérité se développèrent rapidement. » D’où l’on doit conclure qu’auparavant ils avaient commencé de paraître et de grandir. Salluste ajoute, pour appuyer son sentiment : « Car les violences des citoyens puissants, qui amenèrent la séparation du peuple et du sénat, et une foule de dissensions civiles, troublèrent Rome dès le principe, et l’on n’y vit fleurir la modération et l’équité qu’au temps où les rois furent expulsés, alors qu’on redoutait les Tarquins et la guerre avec l’Étrurie. » On voit ici Salluste chercher la cause de cette modération et de cette équité qui régnèrent à Rome pendant un court espace de temps après l’expulsion des Tarquins. Cette cause, à ses yeux, c’est la crainte ; on redoutait, en effet, la guerre terrible que le roi Tarquin, appuyé sur ses alliés d’Étrurie, faisait au peuple qui l’avait chassé de son trône et de ses États. Mais ce qu’ajoute l’historien mérite une attention particulière : « Après cette époque, dit-il, les patriciens traitèrent les gens du peuple en esclaves, condamnant celui-ci à mort et celui-là aux verges, comme avaient fait les rois, chassant le petit propriétaire de son champ, et imposant à celui qui n’avait rien la plus dure tyrannie. Accablé de ces vexations, écrasé surtout par l’usure, le bas peuple, sur qui des guerres continuelles faisaient peser avec le service militaire les plus lourds impôts, prit les armes et se retira sur le mont Sacré et sur l’Aventin ; ce fut ainsi qu’il obtint ses tribuns et d’autres prérogatives. Mais la lutte elles dissensions ne furent entièrement éteintes qu’à la seconde guerre punique. » Voilà ce que devinrent, au bout de quelque temps, peu après l’expulsion des rois, ces Romains dont Salluste nous dit : « Que leur caractère, autant que leurs lois, les rendait justes et équitables. » Or, si telle a été la république romaine aux jours de sa vertu et de sa beauté, que dirons-nous du temps qui a suivi, où, comme dit Salluste : « Changeant peu à peu, de belle et vertueuse qu’elle était, elle devint laide et corrompue », et cela, comme il a soin de le remarquer, depuis la ruine de Carthage ? On peut voir, dans son Histoire, le tableau rapide qu’il trace de ces tristes temps, et par quels degrés la corruption, née des prospérités de Rome, aboutit enfin à la guerre civile : « Depuis cette époque, dit-il, les antiques mœurs, au lieu de s’altérer insensiblement, s’écoulèrent comme un torrent ; car le luxe et la cupidité avaient tellement dépravé la jeunesse que nul ne pouvait plus conserver son propre patrimoine ni souffrir la conservation de celui d’autrui. » Salluste parle ensuite avec quelque étendue des vices de Sylla et des autres hontes de la république, et tous les historiens sont ici d’accord avec lui, quoiqu’ils n’aient pas son éloquence. Voilà, ce me semble, des témoignages suffisants pour faire voir à quiconque voudra y prendre garde dans quel abîme de corruption Rome était tombée avant l’avènement de Notre-Seigneur, car tous ces désordres avaient éclaté, non seulement avant que Jésus-Christ revêtu d’un corps eût commencé à enseigner sa doctrine, mais avant qu’il fût né d’une vierge. Si donc les païens n’osent imputer à leurs dieux les maux de ces temps antérieurs, tolérables avant la ruine de Carthage, intolérables depuis, bien que leurs dieux seuls, dans leur méchanceté et leur astuce, en jetassent la semence dans l’esprit des hommes par les folles opinions qu’ils y répandaient, pourquoi imputent-ils les maux présents à Jésus-Christ, dont la doctrine salutaire défend d’adorer ces dieux faux et trompeurs, et qui, condamnant par une autorité divine ces dangereuses et criminelles convoitises du cœur humain, retire peu à peu sa famille d’un monde corrompu et qui tombe, pour établir, non sur les applaudissements de la vanité, mais sur le jugement de la vérité même, son éternelle et glorieuse cité !
\subsection[{Chapitre XIX}]{Chapitre XIX}

\begin{argument}\noindent De la corruption ou était tombée la république romaine avant que le Christ vînt abolir le culte des dieux.
\end{argument}

\noindent Voilà donc comment la république romaine, « changeant peu à peu, de belle et vertueuse qu’elle était, devint laide et corrompue ». Et ce n’est pas moi qui le dis le premier ; leurs auteurs, dont nous l’avons appris pour notre argent, l’ont dit longtemps avant l’avènement du Christ. Voilà comment depuis la ruine de Carthage, « les antiques mœurs, au lieu de s’altérer insensiblement, s’écoulèrent comme un torrent : tant le luxe et la cupidité avaient corrompu la jeunesse ! ». Où sont les préceptes donnés au peuple romain par ses dieux contre le luxe et la cupidité ? et plût au ciel qu’ils se fussent contentés de se taire sur la chasteté et la modestie, au lieu d’exiger des pratiques indécentes et honteuses auxquelles ils donnaient une autorité pernicieuse par leur fausse divinité ! Qu’on lise nos Écritures, on y verra cette multitude de préceptes sublimes et divins contre l’avarice et l’impureté, partout répandus dans les Prophètes, dit le saint Évangile, dans les Actes et les Épîtres des Apôtres, et qui font éclater à l’oreille des peuples assemblés non pas le vain bruit des disputes philosophiques, mais le tonnerre des divins oracles roulant dans les nuées du ciel. Les païens n’ont garde d’imputer à leurs dieux le luxe, la cupidité, les mœurs cruelles et dissolues qui avaient si profondément corrompu la république avant la venue de Jésus-Christ ; et ils osent reprocher à la religion chrétienne toutes les afflictions que leur orgueil et leurs débauches attirent aujourd’hui sur elle. Et pourtant, si les rois et les peuples, si tous les princes et les juges de la terre, si les jeunes hommes et les jeunes filles, les vieillards et les enfants, tous les âges, tous les sexes, sans oublier ceux à qui s’adresse saint Jean-Baptiste, publicains et soldats, avaient soin d’écouter et d’observer les préceptes de la vie chrétienne, la république serait ici-bas éclatante de prospérité et s’élèverait sans effort au comble de la félicité promise dans le royaume éternel ; mais l’un écoute et l’autre méprise, et comme il s’en trouve plus qui préfèrent la douceur mortelle des vices à l’amertume salutaire des vertus, il faut bien que les serviteurs de Jésus-Christ, quelle que soit leur condition, rois, princes, juges, soldats, provinciaux, riches et pauvres, libres ou esclaves de l’un ou de l’autre sexé, supportent cette république terrestre, fût-elle avilie, fût-elle au dernier degré de la corruption, pour mériter par leur patience un rang glorieux dans la sainte et auguste cour des anges, dans cette république céleste où la volonté de Dieu est l’unique loi.
\subsection[{Chapitre XL. De l’espèce de félicité et du genre de vie qui plairaient le plus aux ennemis de la religion chrétienne}]{Chapitre XL. \\
De l’espèce de félicité et du genre de vie qui plairaient le plus aux ennemis de la religion chrétienne}
\noindent Mais qu’importe aux adorateurs de ces méprisables divinités, aux ardents imitateurs de leurs crimes et de leurs débauches, que la république soit vicieuse et corrompue ? Qu’elle demeure debout, disent-ils ; que l’abondance y règne ; qu’elle soit victorieuse, pleine de gloire, ou mieux encore, tranquille au sein de la paix ; que nous fait tout le reste ? Ce qui nous importe, c’est que chacun accroisse tous les jours ses richesses pour suffire à ses profusions continuelles et s’assujettir les faibles. Que les pauvres fassent la cour aux riches pour avoir de quoi vivre, et pour jouir d’une oisiveté tranquille à l’ombre de leur protection ; que les riches fassent des pauvres les instruments de leur vanité et de leur fastueux patronage. Que les peuples saluent de leurs applaudissements, non les tuteurs de leurs intérêts, mais les pourvoyeurs de leurs plaisirs ; que rien de pénible ne soit commandé, rien d’impur défendu ; que les rois s’inquiètent de trouver dans leurs sujets, non la vertu, mais la docilité ; que les sujets obéissent aux rois, non comme aux directeurs de leurs mœurs, mais comme aux arbitres de leur fortune et aux intendants de leurs voluptés, ressentant pour eux, à la place d’un respect sincère, une crainte servile ; que les lois veillent plutôt à conserver à chacun sa vigne que son innocence ; que l’on n’appelle en justice que ceux qui entreprennent sur le bien ou sur la vie d’autrui, et qu’au reste il soit permis de faire librement tout ce qu’on veut des siens ou avec les siens, ou avec tous ceux qui veulent y consentir ; que les prostituées abondent dans les rues pour quiconque désire en jouir, surtout pour ceux qui n’ont pas le moyen d’entretenir une concubine ; partout de vastes et magnifiques maisons, des festins somptueux, où chacun, pourvu qu’il le veuille ou qu’il le puisse, trouve jour et nuit le jeu, le vin, le vomitoire, la volupté ; qu’on entende partout le bruit de la danse ; que le théâtre frémisse des transports d’une joie dissolue et des émotions qu’excitent les plaisirs les plus honteux et les plus cruels. Qu’il soit déclaré ennemi public celui qui osera blâmer ce genre de félicité ; et si quelqu’un veut y mettre obstacle, qu’on ne l’écoute pas, que le peuple l’arrache de sa place et le supprime du nombre des vivants ; que ceux-là seuls soient regardés comme de vrais dieux qui ont procuré au peuple ce bonheur et qui le l’ai conservent ; qu’on les adore suivant leurs désirs ; qu’ils exigent les jeux qui leur plaisent et les reçoivent de leurs adorateurs ou avec eux ; qu’ils fassent seulement que ni la guerre, ni la peste, ni aucune autre calamité, ne troublent un état si prospère ! Est-ce là, je le demande à tout homme en possession de sa raison, est-ce là l’empire romain ? ou plutôt, n’est-ce pas la maison de Sardanapale, de ce prince livré aux voluptés, qui fit écrire sur son tombeau qu’il ne lui restait plus après la mort que ce que les plaisirs avaient déjà consumé de lui pendant sa vie ? Si nos adversaires avaient un roi comme celui-là, complaisant pour toute débauche et désarmé contre tout excès, ils lui consacreraient, je n’en doute pas, et de plus grand cœur que les anciens Romains à Romulus, un temple et un flamine.
\subsection[{Chapitre XXI}]{Chapitre XXI}

\begin{argument}\noindent Sentiment de Cicéron sur la république romaine.
\end{argument}

\noindent Si nos adversaires récusent le témoignage de l’historien qui nous a dépeint la république romaine comme déchue de sa beauté et de sa vertu, s’ils s’inquiètent peu d’y voir abonder les crimes, les désordres et les souillures de toute espèce, pourvu qu’elle se maintienne et subsiste, qu’ils écoutent Cicéron, qui ne dit plus seulement, comme Salluste, que la république était déchue, mais qu’elle avait cessé d’être et qu’il n’en restait plus rien. Il introduit Scipion, le destructeur de Carthage, discourant sur la république en un temps où la corruption décrite par Salluste faisait pressentir sa ruine prochaine. C’est le moment qui suivit la mort de l’aîné des Gracques, le premier, au témoignage du même Salluste, qui ait excité de grandes séditions ; et il est question de sa fin tragique, dans la suite du dialogue. Or, sur la fin du second livre, Scipion s’exprime en ces termes : « Si dans un concert il faut maintenir un certain accord entre les sons différents qui sortent de la flûte, de la lyre et des voix humaines, sous peine de blesser par la moindre discordance les oreilles exercées, si ce parfait accord ne peut s’obtenir qu’en soumettant les accents les plus divers à une même mesure, de même, dans l’État, un certain équilibre est nécessaire entre les diverses classes, hautes, basses et moyennes, et l’harmonie résulte ici, comme dans la musique, d’un accord entre des éléments très divers ; cette harmonie, dans l’État, c’est la concorde, le plus fort et le meilleur gage du salut public, mais qui, sans la justice, ne peut exister. » Scipion développe quelque temps cette thèse, pour montrer combien la justice est avantageuse à un État, et combien tout est compromis quand elle disparaît. Alors l’un des interlocuteurs, Philus prend la parole et demande que la question soit traitée plus à fond, et que par de nouvelles recherches sur la nature du juste, on fixe la valeur de cette maxime qui commençait alors à se répandre : qu’il est impossible de gouverner la république sans injustice. Scipion consent que l’on discute ce problème, et fi ajoute qu’à son avis tout ce qu’on a dit sur la république n’est rien et qu’il est impossible de passer outre, si on n’a pas établi, non seulement qu’il n’est pas impossible de gouverner sans injustice, mais qu’il est impossible de gouverner sans prendre la justice pour règle souveraine. Cette question, remise au lendemain, est agitée avec grande chaleur et-fait le sujet du troisième livre. Philus prend le parti de ceux qui soutiennent qu’une république ne peut être gouvernée sans injustice, après avoir déclaré toutefois que ce sentiment n’est pas le sien. Il plaide de son mieux pour l’injustice contre la justice, tâchant de montrer par des raisons vraisemblables et par des exemples que la première est aussi avantageuse à la république que la seconde lui est inutile. Alors Lélius, sur la prière de tous, entreprend la défense de la justice et fait tous ses efforts pour démontrer qu’il n’y a rien de plus contraire à un État que l’injustice, et que sans une justice sévère il n’y a ni gouvernement, ni sécurité possibles.\par
Cette question paraissant suffisamment traitée, Scipion reprend son discours et recommande cette courte définition qu’il avait donnée : La république, c’est la chose du peuple. Or, le peuple n’est point un pur assemblage d’individus, mais une société fondée sur des droits reconnus et sur la communauté des intérêts. Ensuite il fait voir combien une bonne définition est utile dans tout débat, et il conclut de la sienne que la république, la chose du peuple, n’existe effectivement que lorsqu’elle est administrée selon le bien et la justice, soit par un roi, soit par un petit nombre de grands, soit par le peuple entier. Mais quand un roi est injuste et devient un tyran, comme disent les Grecs, quand les grands sont injustes et deviennent une faction, ou enfin quand le peuple est injuste et devient, lui aussi, un tyran, car Scipion ne voit pas d’autre nom à lui donner, alors, non seulement la république est corrompue, comme on l’avait reconnu la veille, mais, aux termes de la définition établie, la république n’est plus, puisqu’elle a cessé d’être la chose du peuple pour devenir celle d’un tyran ou d’une faction, le peuple lui-même, du moment qu’il devient injuste, cessant d’être le peuple, c’est-à-dire une société fondée sur des droits reconnus el sur la communauté des intérêts.\par
Lors donc que la république romaine était telle que la décrit Salluste, elle n’était pas seulement déchue de sa beauté et de sa vertu, comme le dit l’historien, mais elle avait cessé d’être, suivant le raisonnement de ces grands hommes. C’est ce que Cicéron prouve au commencement du cinquième livre, où il ne parle plus au nom de Scipion, mais en son propre nom. Après avoir rappelé ce vers d’Ennius :\par
{\itshape Rome a pour seul appui ses mœurs et ses grands hommes}, \par
« Ce vers, dit-il, par la vérité comme par la précision, me semble un oracle émané du sanctuaire. Ni les hommes, en effet, si l’État n’avait eu de telles mœurs, ni les mœurs publiques, s’il ne s’était montré de tels hommes, n’auraient pu fonder ou maintenir pendant si longtemps une si vaste domination. Aussi voyait-on, avant notre siècle, la force des mœurs héréditaires appeler naturellement les hommes supérieurs, et ces hommes éminents retenir les vieilles coutumes et les institutions des aïeux. Notre siècle, au contraire, recevant la république comme un chef-d’œuvre d’un autre âge, qui déjà commençait à vieillir et à s’effacer, non seulement a négligé de renouveler les couleurs du tableau primitif, mais ne s’est pas même occupé d’en conserver au moins le dessin et comme les derniers contours. »\par
« Que reste-t-il, en effet, de ces mœurs antiques, sur lesquelles le poète appuyait la république romaine ? Elles sont tellement surannées et mises en oubli, que, loin de les pratiquer, on ne les connaît même plus. Parlerai-je des hommes ? Les mœurs elles-mêmes n’ont péri que par le manque de grands hommes ; désastre qu’il ne suffit pas d’expliquer, et dont nous aurions besoin de nous faire absoudre, comme d’un crime capital ; car c’est grâce à nos vices, et non par quelque coup du sort que, conservant encore la république de nom, nous en avons dès longtemps perdu la réalité. »\par
Voilà quels étaient les sentiments de Cicéron, longtemps, il est vrai, après la mort de Scipion l’Africain, mais enfin avant l’avènement de Jésus-Christ. Certes, si un pareil état de choses eût existé et eût été signalé depuis l’établissement de la religion du Christ, quel est celui de nos adversaires qui ne l’eût imputé à son influence ? Je demande donc pourquoi leurs dieux ne se sont pas mis en peine de prévenir cette ruine de la république romaine que Cicéron, bien longtemps avant l’incarnation de Jésus-Christ, déplore avec de si pathétiques accents ? Maintenant c’est aux admirateurs des antiques mœurs et de la vieille Rome d’examiner s’il est bien vrai que la justice régnât dans ce temps-là ; peut-être, à la place d’une vivante réalité, n’y avait-il qu’une surface ornée de couleurs brillantes, suivant l’expression échappée à Cicéron. Mais nous discuterons ailleurs cette question, s’il plaît à Dieu. Car je m’efforcerai de prouver, en temps et lieu, que selon les définitions de la république et du peuple, données par Scipion avec l’assentiment de ses amis, jamais il n’y a eu à Rome de république, parce que jamais il n’y a eu de vraie justice. Si l’on veut se relâcher de cette sévérité et prendre des définitions plus généralement admises, je veux bien convenir que la république romaine a existé, surtout à mesure qu’on s’enfonce dans les temps primitifs ; mais il n’en demeure pas moins établi que la véritable justice n’existe que dans cette république dont le Christ est le fondateur et le gouverneur. Je puis, en effet, lui donner le nom de république, puisqu’elle est incontestablement la chose du peuple ; mais si ce mot, pris ailleurs dans un autre sens, s’écarte trop ici de notre langage accoutumé, il faut au moins reconnaître que le seul siège de la vraie justice, c’est cette cité dont il est dit dans l’Écriture sainte : « On a publié de toi des choses glorieuses, ô cité de Dieu ! »
\subsection[{Chapitre XXII}]{Chapitre XXII}

\begin{argument}\noindent Les dieux des Romains n’ont jamais pris soin d’empêcher que les mœurs ne fissent périr la république.
\end{argument}

\noindent Mais, pour revenir à la question, qu’on célèbre tant qu’on voudra la république romaine, telle qu’elle a été ou telle qu’elle est, il est certain que, selon leurs plus savants écrivains, elle était déchue bien avant l’avènement du Christ ; que dis-je ? n’ayant plus de mœurs, elle n’était déjà plus. Pour l’empêcher de périr, qu’auraient dû faire les dieux protecteurs ? lui donner les préceptes qui règlent la vie et forment les mœurs, en échange de tant de prêtres, de temples, de sacrifices, de cérémonies, de fêtes et de jeux solennels. Mais en tout cela les démons ne songeaient qu’à leur intérêt, se mettant fort peu en peine de la manière dont le peuple vivait, le portant au contraire à mal vivre, pourvu qu’asservi par la crainte il continuât de les honorer. Si on répond qu’ils lui ont donné des préceptes, qu’on les cite, qu’on les montre ; qu’on nous dise à quel commandement des dieux ont désobéi les Gracques en troublant l’État par leurs séditions ; Marius, Cinna et Carbon, en allumant des guerres civiles injustes dans leurs commencements, cruelles dans leur progrès, sanglantes dans leur terme ; Sylla enfin, dont on ne saurait lire la vie, les mœurs, les actions dans Salluste et dans les autres historiens, sans frémir d’horreur. Qui n’avouera qu’une telle république avait cessé d’exister ? Dira-t-on, pour la défense de ces dieux, qu’ils ont abandonné Rome à cause de cette corruption même, selon ces vers de Virgile :\par
 {\itshape « Les dieux protecteurs de cet empire ont tous abandonné leurs temples et leurs autels. »} \par
Mais d’abord, s’il en est ainsi, les païens n’ont pas le droit de se plaindre que la religion chrétienne leur ait fait perdre la protection de leurs dieux, puisque déjà les mœurs corrompues de leurs ancêtres avaient chassé des autels de Rome, comme des mouches, tout cet essaim de petites divinités. Où était d’ailleurs cette armée de dieux, lorsque Rome, longtemps avant la corruption des mœurs antiques, fut prise et brûlée par les Gaulois ? S’ils étaient là, ils dormaient sans doute ; car de toute la ville tombée au pouvoir de l’ennemi, il ne restait aux Romains que le Capitole, qui aurait été pris comme tout le reste, si les oies n’eussent veillé pendant le sommeil des dieux. Et de là, l’institution de la fête des oies, qui fit presque tomber Rome dans les superstitions des Égyptiens, adorateurs des bêtes et des oiseaux. Mais mon dessein n’est pas de parler présentement de ces maux extérieurs qui se rapportent au corps plutôt qu’à l’esprit et qui ont pour cause la guerre ou tout autre fléau ; je ne parle que de la décadence des mœurs, d’abord insensiblement altérées, puis s’écoulant comme un torrent et entraînant si rapidement la république dans leur ruine qu’il n’en restait plus, au jugement de graves esprits, que les murailles et les maisons. Certes, les dieux auraient eu raison de se retirer d’elle pour la laisser périr, et, comme dit Virgile, d’abandonner leurs temples et leurs autels, si elle eût méprisé leurs préceptes de vertu et de justice ; mais que dire de ces dieux, qui ne veulent plus vivre avec un peuple qui les adore, sous prétexte qu’il vit mal, quand ils ne lui ont pas appris à bien vivre ?
\subsection[{Chapitre XXIII}]{Chapitre XXIII}

\begin{argument}\noindent Les vicissitudes des choses temporelles ne dépendent point de la faveur ou de l’inimitié des démons, mais du conseil du vrai dieu.
\end{argument}

\noindent J’irai plus loin ; je dirai que les dieux ont paru aider leurs adorateurs à contenter leurs convoitises, et n’ont jamais rien fait pour les contenir. C’est en effet par leur assistance que Marius, homme nouveau et obscur, fauteur cruel de guerres civiles, fut porté sept fois au consulat et mourut, chargé d’années, échappant aux mains de Sylla vainqueur ; pourquoi donc cette même assistance ne l’a-t-elle pas empêché d’accomplir tant de cruautés ? Si nos adversaires répondent que les dieux ne sont pour rien dans sa fortune, ils nous font une grande concession ; car ils nous accordent qu’on peut se passer des dieux pour jouir de cette prospérité terrestre dont ils sont si épris, qu’on peut avoir force, richesses, honneurs, santé, grandeur, longue vie, comme Marins, tout en ayant les dieux contraires, et qu’on peut souffrir, comme Régulus, la captivité, l’esclavage, la misère, les veilles, les douleurs, les tortures et la mort enfin, tout en ayant les dieux propices. Si on accorde cela, on avoue en somme que les dieux ne servent à rien et que c’est en vain qu’on les adore. Si les dieux, en effet, loin de former les hommes à ces vertus de l’âme et à cette vie honnête qui les autorise à espérer le bonheur après la mort, leur donnent des leçons toutes contraires, et si d’ailleurs, quand il s’agit des biens passagers et temporels, ils ne peuvent nuire à ceux qu’ils détestent, ni être utiles à ceux qu’ils aiment, pourquoi les adorer ? pourquoi s’empresser autour de leurs autels ? pourquoi, dans les mauvais jours, murmurer contre eux, comme s’ils avaient par colère retiré leur protection ? et pourquoi en prendre occasion pour outrager et maudire la religion chrétienne ? Si, au contraire, dans l’ordre des choses temporelles, ils peuvent nuire ou servir, pourquoi ont-ils accordé au détestable Marius leur protection, et l’ont-ils refusée au vertueux Régulus ? Cela ne fait-il pas voir qu’ils sont eux-mêmes très injustes et très pervers ? Que si, par cette raison même, on est porté à les craindre et à les adorer, on se trompe, puisque rien ne prouve que Régulus les ait moins adorés que Marius. Et qu’on ne s’imagine pas non plus qu’il faille mener une vie criminelle à cause que les dieux semblent avoir favorisé Marius plutôt que Régulus. Je rappellerais alors que Métellus, un des plus excellents hommes parmi les Romains, qui eut cinq fils consulaires, fut un homme très heureux, au lieu que Catilina, vrai scélérat, périt misérablement dans la guerre criminelle qu’il avait excitée. Enfin, la véritable et certaine félicité n’appartient qu’aux gens de bien adorant le Dieu qui seul peut la donner.\par
Lors donc que cette république périssait par ses mauvaises mœurs, les dieux ne firent rien pour l’empêcher de périr, en accroissant ses mœurs ou en les corrigeant ; au contraire, ils travaillaient à la faire périr en accroissant la décadence et la corruption des mœurs. Et qu’ils ne viennent pas se faire passer pour bons, sous prétexte qu’ils abandonnèrent Rome en punition de ses iniquités. Non, ils restèrent là ; leur imposture est manifeste ; ils n’ont pu ni aider les hommes par de bons conseils, ni se cacher par leur silence. Je ne rappellerai pas que les habitants de Minturnes, touchés de l’infortune de Marius, le recommandèrent à la déesse Marica, et que cet homme cruel, sauvé contre toute espérance, rentra à Rome plus puissant que jamais à la tête d’hommes non moins cruels que lui et se montra, au témoignage des historiens, plus atroce et plus impitoyable que ne l’eût été le plus barbare ennemi. Mais encore une fois, je laisse cela de côté, et je n’attribue point cette sanglante félicité de Marius à je ne sais quelle Marica, mais à une secrète providence de Dieu, qui a voulu par là fermer la bouche à nos ennemis et retirer de l’erreur ceux qui, au lieu d’agir par passion, réfléchissent sérieusement sur les faits. Car bien que les démons aient quelque puissance en ces sortes d’événements, ils n’en ont qu’à condition de la recevoir du Tout-Puissant, et cela pour plusieurs raisons : d’abord pour que nous n’estimions pas à un trop haut prix la félicité temporelle, puisqu’elle est souvent accordée aux méchants, témoin Marins ; puis, pour que nous ne la considérions pas non plus comme un mal, puisque nous en voyons également jouir un grand nombre de bons et pieux serviteurs du seul et vrai Dieu, malgré les démons ; enfin pour que nous ne soyons pas tentés de craindre ces esprits immondes ou de chercher à nous les rendre propices, comme arbitres souverains des biens et des maux temporels, puisqu’il en est des démons comme des méchants en ce monde, qui ne peuvent faire que ce qui leur est permis par celui dont les jugements sont aussi justes qu’incompréhensibles.
\subsection[{Chapitre XXIV}]{Chapitre XXIV}

\begin{argument}\noindent Des proscriptions de Sylla auxquelles les démons se vantent d’avoir prêté leur assistance.
\end{argument}

\noindent Il est certain que lorsque Sylla, dont le gouvernement fut si atroce qu’en se portant le vengeur des cruautés de Marius il le fit regretter, se fût approché de Rome pour combattre son rival, les entrailles des victimes parurent si favorables, suivant le rapport de Tite-Live, que l’aruspice Postumius, convaincu qu’avec l’aide des dieux Sylla ne pouvait manquer de réussir dans ses desseins, répondit du succès sur sa tête. Vous voyez bien que les dieux ne s’étaient point retirés de leurs temples et de leurs autels, puisqu’ils prédisaient l’avenir, sans se mettre en peine du reste de rendre Sylla meilleur. Ils avaient des présages pour lui promettre une grande félicité et n’avaient point de menaces pour réprimer son ambition coupable. Ce n’est pas tout : comme il faisait la guerre en Asie contre Mithridate, Jupiter lui fit dire par Lucius Titius qu’il serait vainqueur, ce qui arriva. Plus tard, quand Sylla méditait de retourner à Rome pour venger par les armes ses injures et celle de ses amis, le même Jupiter lui fit dire par un soldat de la sixième légion que, lui ayant déjà présagé sa victoire contre Mithridate, il lui promettait encore de lui donner la puissance nécessaire pour s’emparer de la république, non toutefois sans répandre beaucoup de sang. Sylla voulut savoir du soldat sous quelle forme il avait vu Jupiter, et reconnut que c’était la même que le dieu avait déjà revêtue pour lui faire annoncer une première fois qu’il serait vainqueur. Comment justifier les dieux du soin qu’ils ont pris de prédire à Sylla le succès de ses entreprises, et de leur négligence à lui donner d’utiles avertissements pour détourner les maux qu’allait déchaîner sur Rome une guerre impie, honte et ruine de la république ? Il faut conclure de là, comme je l’ai dit plusieurs fois et comme les saintes Écritures et l’expérience même nous le font assez connaître, que les démons n’ont d’autre but que de passer pour dieux, de se faire adorer comme tels, et de porter les hommes à leur offrir un culte qui les associe à leurs crimes, afin qu’étant unis avec eux dans une même cause, ils soient condamnés comme eux par un même jugement de Dieu.\par
Quelque temps après, Sylla vint à Tarente, et ayant sacrifié, il aperçut au haut du foie de la victime la forme d’une couronne d’or. Sur ce présage, l’aruspice Postumius lui promit une grande victoire et ordonna que Sylla seul mangeât de ce foie. Presque au même instant l’esclave d’un certain Lucius Pontius s’écria, d’un ton inspiré : Je suis le messager de Bellone, la victoire est à toi, Sylla ! Puis il ajouta que le Capitole serait brûlé. Là-dessus étant sorti du camp, il revint le lendemain encore plus ému, et s’écria : Le Capitole est brûlé ! et, en effet, il l’était. On sait qu’il est facile à un démon de prévoir un tel événement et d’en apporter très — promptement la nouvelle ; mais considérez ici, ce qui importe fort à notre sujet, sous quels dieux veulent vivre ceux qui blasphèment le Sauveur venu pour les délivrer de la domination des démons. Cet homme s’écria, comme inspiré : La victoire est à toi, Sylla ! et pour faire croire qu’il était animé de l’esprit divin, il annonça comme prochain un événement qui s’accomplit en effet, tout éloigné qu’il fût de celui qui le prédisait ; mais il ne cria point : Sylla, garde-toi d’être cruel ! de manière à prévenir les horribles cruautés que commit à Rome cet illustre vainqueur à qui fut annoncé son triomphe par une couronne d’or empreinte sur le foie d’un veau ! Certes, si c’étaient des dieux justes et non des démons impies qui fissent paraître de tels présages, ils auraient bien plutôt révélé à Sylla, par l’inspection des entrailles, les maux que sa victoire devait causer à l’État et à lui-même. Car il est certain qu’elle ne fut pas si avantageuse à sa gloire que fatale à son ambition, puisque enivré par la prospérité, il lâcha la bride à ses passions et fit plus de mal à son âme en la perdant de mœurs qu’il n’en fit à ses ennemis en les tuant. Cependant ces malheurs si réels et si lamentables, les dieux ne les lui annoncèrent ni par les entrailles des victimes, ni par des augures, ni par quelque songe ou quelque prophétie. Ils n’appréhendaient pas qu’il fût vaincu, mais qu’il sa-vainquît lui-même ; ou plutôt ils travaillaient à faire que ce vainqueur de ses concitoyens devînt esclave de ses vices et d’autant plus asservi, par là même, au joug des démons.
\subsection[{Chapitre XXV}]{Chapitre XXV}

\begin{argument}\noindent Les démons ont toujours excité les hommes au mal en donnant aux crimes l’autorité de leur exemple.
\end{argument}

\noindent Qui ne reconnaît donc par là, si ce n’est celui qui aime mieux imiter de tels dieux que d’être préservé de leur commerce par la grâce du vrai Dieu, qui ne sent et ne comprend que tout leur effort est de donner au crime par leur exemple une autorité divine ? On les a même vus se battre les uns contre les autres dans une grande plaine de la Campanie, où peu après se donna une bataille entre les deux partis qui divisaient la république. Un bruit formidable se fit d’abord entendre, et plusieurs rapportèrent bientôt qu’ils avaient vu pendant quelques jours deux armées qui étaient aux prises. Le combat fini, on trouva des espèces de vestiges d’hommes et de chevaux, autant qu’il pouvait en rester après une telle mêlée. Si donc les dieux se sont véritablement battus ensemble, il n’en faut pas davantage pour excuser les guerres civiles ; et, dans cette hypothèse, je vous prie de considérer quelle est la méchanceté ou la misère de ces dieux ; si, au contraire, ce combat n’était qu’une vaine apparence, quel autre dessein ont-ils pu avoir que de justifier les guerres civiles des Romains et de leur faire croire qu’elles étaient innocentes, puisque les dieux les autorisaient par leur exemple ? Ces guerres, en effet, avaient déjà commencé, et déjà elles étaient signalées par des événements tragiques ; on se racontait avec émotion l’histoire de ce soldat qui, voulant dépouiller un mort, après la bataille, reconnut son frère et se tua sur son cadavre, en maudissant les discordes civiles. De peur donc qu’on ne fût trop affligé de ces malheurs, et afin que l’ardeur criminelle des partis allât toujours croissant, ces démons, qui se faisaient passer pour des dieux et adorer comme tels, eurent l’idée de se montrer aux hommes en état de guerre les uns contre les autres, afin que l’autorité d’un exemple divin étouffât dans les âmes les restes de l’affection patriotique. C’est par une ruse pareille qu’ils ont fait instituer ces jeux scéniques dont j’ai déjà beaucoup parlé, et où le drame et le chant attribuent aux dieux de telles infamies, qu’il suffit de les en croire capables ou de penser qu’ils les voient représenter avec plaisir pour les imiter en toute sécurité. Or, de crainte qu’on ne vînt à révoquer en doute ces combats entre les dieux, que nous lisons dans les poètes, et à les regarder comme d’injurieuses fictions, les dieux ne se sont pas bornés à les faire représenter sur le théâtre, ils ont voulu se donner eux-mêmes en représentation sur un champ de bataille.\par
J’ai dû insister sur ce point, parce que les auteurs païens n’ont pas fait difficulté de déclarer que la république romaine était morte de corruption, et qu’il n’en restait déjà plus rien avant l’avènement de Notre-Seigneur Jésus-Christ. Or, cette corruption, nos adversaires ne l’imputent point à leurs dieux, et cependant ils prétendent imputer à notre Sauveur ces maux passagers qui ne sauraient perdre les bons, ni dans cette vie, ni dans l’autre. Chose étrange ! Ils accusent le Christ, qui a donné tant de préceptes pour la purification des mœurs et contre la corruption des vices, et ils n’accusent point leurs dieux, qui, loin de préserver par de semblables préceptes le peuple qui les servait, ont fait tous leurs efforts pour le précipiter plus avant dans le mal par leur exemple et leur autorité. J’espère donc qu’il ne se rencontrera plus personne qui ose expliquer la chute de l’empire romain en disant avec Virgile :\par
 {\itshape « Tous les dieux se sont retirés de leurs temples et ont abandonné leurs autels. »} \par
Comme si ces dieux étaient des amis de la vertu, irrités contre les vices des hommes ! Non ; car ces présages tirés des entrailles des victimes, ces augures, ces prédictions, par lesquelles les dieux païens se complaisaient à faire croire qu’ils connaissaient l’avenir et influaient sur le destin des combats, tout cela témoigne qu’ils n’avaient pas cessé d’être présents. Et plût à Dieu qu’ils se fussent retirés ! la fureur des guerres civiles eût été moins excitée par les passions romaines qu’elle ne le fut par leurs instigations détestables.
\subsection[{Chapitre XXVI}]{Chapitre XXVI}

\begin{argument}\noindent Les faux dieux donnaient en secret des préceptes pour les bonnes mœurs, et en public des exemples d’impudicité.
\end{argument}

\noindent Après avoir mis au grand jour les cruautés et les turpitudes des dieux, lesquelles, feintes ou véritables, sont proposées en exemple au public, et consacrées dans des fêtes solennelles qu’on a établies sur leur demande et par crainte d’encourir leur vengeance en cas de refus, la question est de savoir comment il se fait que ces mêmes démons, qui confessent assez par là leur caractère d’esprits immondes, partisans de tous ces crimes dont ils demandent la représentation à l’impudicité des uns et à la faiblesse des autres, comment, dis-je, ces amis d’une vie criminelle et souillée passent pour donner dans le secret de leurs sanctuaires quelques préceptes de vertu à un certain nombre d’initiés. Si le fait est vrai, je n’y vois qu’une preuve de plus de l’excès de leur malice. Car tel est l’ascendant de la droiture et de la chasteté, qu’il n’est presque personne qui ne soit bien aise d’être loué pour ces vertus, dont le sentiment ne se perd jamais dans les natures les plus corrompues. Si donc les démons ne se transformaient pas quelquefois, comme dit l’Écriture, en anges de lumière, ils ne pourraient pas séduire les hommes. Ainsi l’impudicité s’étale à grand bruit devant la foule, et la chasteté murmure à peine quelques paroles hypocrites à l’oreille d’un petit nombre d’initiés. On expose en public ce qui est honteux, et on tient secret ce qui est honnête ; la vertu se cache et le vice s’affiche ; le mal a des spectateurs par milliers, et le bien trouve à peine quelques disciples, comme si l’on devait rougir de ce qui est honnête et faire gloire de ce qui ne l’est pas. Mais où enseigne-t-on ces beaux préceptes ? où donc, sinon dans les temples des démons, dans les retraites de l’imposture ? C’est que les préceptes secrets sont pour surprendre la bonne foi des honnêtes gens, qui sont toujours en petit nombre, et les spectacles publics pour empêcher les méchants, qui sont toujours en grand nombre, de se corriger.\par
Quant à nous, si on nous demandait où et quand les initiés de la déesse Célestis entendaient des préceptes de chasteté, nous ne pourrions le dire ; mais ce que nous savons, c’est que, lorsque nous étions devant son temple, en présence de sa statue, au milieu d’une foule de spectateurs qui ne savaient où trouver place, nous regardions les jeux avec une attention extrême, considérant tour à tour, d’un côté, le cortège des courtisanes, de l’autre, la déesse vierge, devant laquelle on jouait des scènes infâmes en manière d’adoration. Pas un mime qui ne fût obscène, pas une comédienne qui ne fût impudique ; chacun remplissait de son mieux son office d’impureté. On savait très bien ce qui était fait pour plaire à cette divinité virginale, et la matrone qui assistait à ces exhibitions retournait du temple à sa demeure plus savante qu’elle n’était venue. Les plus sages détournaient la vue des postures lascives des comédiens, mais un furtif regard leur apprenait l’art de faire le mal. Elles n’osaient pas, devant des hommes, regarder d’un œil libre des gestes impudiques, mais elles osaient moins encore condamner d’un cœur chaste un spectacle réputé divin. Et pourtant, ce qui s’enseignait ainsi publiquement dans le temple, on n’osait le faire qu’en secret dans la maison, comme si un reste de pudeur eût empêché les hommes de se livrer en toute liberté à des actions enseignées par la religion, et dont la représentation était même prescrite, sous peine d’irriter les dieux. Et maintenant, quel est cet esprit qui agit sur le cœur des méchants par des impressions secrètes, qui les pousse à commettre des adultères, et y trouve, pendant qu’on les commet, un spectacle agréable, sinon le même qui se complaît à ces représentations impures, qui consacre dans les temples les images des démons, et sourit dans les jeux aux images des vices, qui murmure en secret quelques paroles de justice pour surprendre le petit nombre des bons, et étale en public les appâts du vice pour attirer sous son joug le nombre infini des méchants ?
\subsection[{Chapitre XXVII}]{Chapitre XXVII}

\begin{argument}\noindent Quelle funeste influence ont exercée sur les mœurs publiques les jeux obscènes que les Romains consacraient à leurs dieux pour les apaiser.
\end{argument}

\noindent Un grave personnage, et qui se piquait de philosophie, Cicéron, sur le point d’être édile, criait à qui voulait l’entendre, qu’entre autres devoirs de sa magistrature, il avait à apaiser la déesse Flore par des jeux solennels. Or, ces jeux marquaient d’autant plus de dévotion qu’ils étaient plus obscènes. Il dit ailleurs (et alors il était consul, et la république courait le plus grand danger) que l’on avait célébré des jeux pendant dix jours et que rien n’avait été négligé pour apaiser les dieux ; comme s’il n’eût pas mieux valu irriter de tels dieux par la tempérance, que les apaiser par la luxure, et provoquer même leur inimitié par la pudeur que leur agréer. En effet, les partisans de Catilina ne pouvaient, si cruels qu’ils fussent, causer autant de mal aux Romains que leur en faisaient les dieux en leur imposant ces jeux sacrilèges. Pour détourner le dommage dont l’ennemi menaçait les corps, on recourait à des moyens mortellement pernicieux pour les âmes, et les dieux ne consentaient à se porter au secours des murailles de Rome qu’après avoir travaillé à la ruine de ses mœurs. Cependant, ces cérémonies si effrontées et si impures, si impudentes et si criminelles, ces scènes tellement immondes que l’instinctive honnêteté des Romains les porta à en mépriser les acteurs, à les exclure de toute dignité, à les chasser de la tribu, à les déclarer infâmes, ces fables scandaleuses et impies qui flattaient les dieux en les déshonorant, ces actions honteuses, si elles étaient réelles, et non moins honteuses, si elles étaient imaginaires, tout cela composait l’enseignement public de la cité. Le peuple voyait les dieux se complaire à ces turpitudes, et il en concluait qu’il était bon, non seulement de les représenter, mais aussi de les imiter, de préférence à ces prétendus préceptes de vertu qui enseignaient à si peu d’élus (supposé qu’on les enseignât) et avec tant de mystère, comme si on eût craint beaucoup plus de les voir divulgués que mal pratiqués.
\subsection[{Chapitre XXVIII}]{Chapitre XXVIII}

\begin{argument}\noindent De la sainteté de la religion chrétienne.
\end{argument}

\noindent Il n’y a donc que des méchants, des ingrats et des esprits obsédés et tyrannisés par le démon, qui murmurent de ce que les hommes ont été délivrés par le nom de Jésus-Christ du joug infernal de ces puissances impures et de la solidarité de leur châtiment ; eux seuls peuvent se plaindre de voir succéder aux ténèbres de l’erreur l’éclatante lumière de la vérité ; eux seuls ne sauraient souffrir que les peuples courent avec le zèle le plus pur vers des églises où de chastes barrières séparent les deux sexes, où l’on apprend ce qu’il faut faire pour bien vivre dans ce monde, afin d’être éternellement heureux dans l’autre, et où l’Écriture sainte, cette doctrine de justice, est annoncée d’un lieu éminent en présence de tout le monde, afin que ceux qui observent ses enseignements l’entendent pour leur salut, et ceux qui les violent, pour leur condamnation. Que si quelques moqueurs viennent se mêler aux fidèles, ou bien leur légèreté impie tombe par un changement soudain, ou bien elle est tenue en respect par la crainte et par la honte. Là, en effet, rien d’impur ne s’offre au regard, rien de déshonnête n’est proposé en exemple ; on enseigne les préceptes du vrai Dieu, on raconte ses miracles, on le loue de ses dons, on lui demande ses grâces.
\subsection[{Chapitre XXIX}]{Chapitre XXIX}

\begin{argument}\noindent Exhortation aux Romains pour qu’ils rejettent le culte des dieux.
\end{argument}

\noindent Voilà la religion digne de tes désirs, race glorieuse des Romains, race des Régulus, des Scévola, des Scipions, des Fabricius ! voilà le culte digne de toi et que tu ne peux mettre en balance avec les vanités impures et les pernicieux mensonges des démons ! S’il est en ton âme un principe naturel de vertu, songe que la véritable piété peut seule le maintenir dans sa pureté et le porter à sa perfection, tandis que l’impiété le corrompt et en fait une nouvelle cause des châtiments. Choisis donc la route que tu veux suivre ; afin de conquérir une gloire sans illusion et des éloges qui ne s’arrêtent pas à toi, mais qui remontent jusqu’à Dieu. Tu étais jadis en possession de la gloire humaine, mais par un secret conseil de la Providence, tu n’avais pas su choisir la véritable religion. Réveille-toi, il est grand jour ; fais comme quelques-uns de tes enfants dont les souffrances pour la vraie foi sont l’honneur de l’Église, combattants intrépides qui, en triomphant au prix de leur vie des puissances infernales, nous ont enfanté par leur sang une nouvelle patrie. C’est à cette patrie que nous te convions ; viens grossir le nombre de ses citoyens, viens-y chercher l’asile où les fautes sont véritablement effacées. N’écoute point ceux des tiens qui, dégénérés de la vertu de leurs pères, calomnient le Christ et les chrétiens, et leur imputent toutes les agitations de notre temps ; ce qu’il leur faut à eux, ce n’est pas le repos d’une vie douce, c’est la sécurité d’une vie mauvaise. Mais Rome n’a jamais convoité un pareil loisir, même en vue du seul bonheur de la vie présente. Or maintenant, c’est vers la vie future qu’il faut marcher ; la conquête en sera plus aisée et la victoire y sera sans illusion et sans terme. Tu n’y honoreras ni le feu de Vesta, ni la pierre du Capitole, mais le Dieu unique et véritable,\par
 {\itshape « Qui ne te mesurant ni l’espace ni la durée, te donnera un empire sans fin. »} \par
Ne cours plus après des dieux faux et trompeurs ; mais plutôt rejette-les, méprise-les, et prends ton essor vers la liberté véritable. Ces dieux ne sont pas des dieux, mais des esprits malfaisants dont ton bonheur éternel sera le supplice. Junon n’a jamais tant envié aux Troyens, dont tu es la fille selon la chair, la gloire de la cité romaine, que ces démons, que tu prends encore pour des dieux, n’envient à tous les hommes la gloire de l’éternelle cité. Toi-même, tu as jugé selon leur mérite les objets de ton culte, lorsqu’en leur conservant des jeux de théâtre pour les rendre propices, tu as condamné les acteurs à l’infamie. Souffre qu’on t’affranchisse de la domination de ces esprits impurs qui t’ont imposé comme un joug la consécration de leur propre ignominie. Tu as éloigné de tes honneurs ceux qui représentaient les crimes des dieux ; prie le vrai Dieu d’éloigner de toi ces dieux qui se complaisent dans le spectacle de leurs crimes, spectacle honteux, si ces crimes sont réels, spectacle perfide, si ces crimes sont imaginaires. Tu as exclu spontanément de la cité les comédiens et les histrions, c’est bien, mais achève d’ouvrir les yeux, et songe que la majesté divine ne saurait être honorée par tes fêtes, quand la dignité humaine en est avilie. Comment peux-tu croire que des dieux qui prennent plaisir à un culte et à des jeux obscènes soient au nombre des puissances du ciel, du moment que tu refuses de mettre les acteurs de ces jeux au nombre des derniers membres de la cité ? N’y a-t-il pas une cité incomparablement supérieure à toutes les autres, celle qui donne pour victoire la vérité, pour honneurs la sainteté, pour paix la félicité, pour vie l’éternité ? Elle ne peut compter de tels dieux parmi ses enfants, puisque tu as refusé de compter parmi les tiens de tels hommes. Si donc tu veux parvenir à cette cité bienheureuse, évite la société des démons. Ils ne peuvent être servis par d’honnêtes gens, ceux qui se laissent apaiser par des infâmes. Que la sainteté du christianisme retranche à ces dieux tes hommages, comme la sévérité du censeur retranchait à ces hommes tes dignités.\par
Quant aux biens et aux maux de l’ordre charnel, c’est-à-dire aux seuls biens dont les méchants désirent jouir et aux seuls maux qu’ils ne veuillent pas supporter, nous montrerons dans le livre suivant que les démons n’en disposent pas aussi souverainement qu’on se l’imagine ; et quand il serait vrai qu’ils distribuent à leur gré les vains avantages de la terre, ce ne serait pas une raison de les adorer et de perdre en les adorant les biens réels que leur malice nous envie.
\section[{Livre troisième. Les Romains et leurs faux dieux}]{Livre troisième. \\
Les Romains et leurs faux dieux}\renewcommand{\leftmark}{Livre troisième. \\
Les Romains et leurs faux dieux}

\subsection[{Chapitre premier}]{Chapitre premier}

\begin{argument}\noindent Des seuls maux que redoutent les méchants et dont le culte des dieux n’a jamais préservé le monde.
\end{argument}

\noindent Je crois en avoir assez dit sur les maux qui sont le plus à redouter, c’est-à-dire sur ceux qui regardent les mœurs et les âmes, et je tiens pour établi que les faux dieux, loin d’en alléger le poids à leurs adorateurs, ont servi au contraire à l’aggraver. Je vais parler maintenant des seuls maux que les idolâtres ne veulent point souffrir, tels que la faim, les maladies, la guerre, le pillage, la captivité, les massacres, et autres déjà énumérés au premier livre. Car le méchant ne met au rang des maux que ceux qui ne rendent pas l’homme mauvais, et il ne rougit pas, au milieu des biens qu’il loue, d’être mauvais lui-même ; en les louant, il est plus peiné d’avoir une mauvaise villa qu’une mauvaise vie comme si le plus grand bien de l’homme était d’avoir tout bon hormis soi-même. Or, je ne vois pas que les dieux du paganisme, au temps où leur culte florissait en toute liberté, aient garanti leurs adorateurs de ces maux qu’ils redoutent uniquement. En effet, avant l’avènement de notre Rédempteur, quand le genre humain s’est vu affligé en divers temps et en divers lieux d’une infinité de calamités, dont quelques-unes même sont presque incroyables, quels autres dieux adorait-il que les faux dieux ? à l’exception toutefois du peuple juif et d’un petit nombre d’âmes d’élite qui, en vertu d’un jugement de Dieu, aussi juste qu’impénétrable, ont été dignes, en quelque lieu que ce fût, de recevoir sa grâce. Je passe, pour abréger, les grands désastres survenus chez les autres peuples et ne veux parler ici que de l’empireromain, par où j’entends Rome elle-même et les provinces qui, réunies par alliance ou par soumission avant la naissance du Christ, faisaient déjà partie du corps de l’État.
\subsection[{Chapitre II}]{Chapitre II}

\begin{argument}\noindent Si les dieux que servaient en commun les Romains et les Grecs ont eu des raisons pour permettre la ruine de Troie.
\end{argument}

\noindent Et d’abord pourquoi Troie ou Ilion, berceau du peuple romain (car il n’y a plus rien à taire ou à dissimuler sur cette question, déjà touchée dans le premier livre), pourquoi Troie a-t-elle été prise et brûlée par les Grecs, dont les dieux étaient ses dieux ? C’est, dit-on, que Priam a expié le parjure de son père Laomédon. Il est donc vrai qu’Apollon et Neptune louèrent leurs bras à Laomédon pour bâtir les murailles de Troie, sur la promesse qu’il leur fit, et qu’il ne tint pas, de les payer de leurs journées. J’admire qu’Apollon, surnommé le divin, ait entrepris une si grande besogne sans prévoir qu’il n’en serait point payé. Et l’ignorance de Neptune, son oncle, frère de Jupiter et roi de la mer, n’est pas moins surprenante ; car Homère (qui vivait, suivant l’opinion commune, avant la naissance de Rome) lui fait faire au sujet des enfants d’Énée, fondateurs de cette ville, les prédictions les plus magnifiques. Il ajoute même que Neptune couvrit Énée d’un nuage pour la dérober à la fureur d’Achille, bien que ce Dieu désirât, comme il l’avoue dans Virgile :\par
 {\itshape « Renverser de fond en comble ces murailles de Troie construites de ses propres mains pour le parjure Laomédon. »} \par
Voilà donc des dieux aussi considérables que Neptune et Apollon qui, ne prévoyant pas que Laomédon retiendrait leur salaire, se sont faits constructeurs de murailles gratuitement et pour des ingrats. Prenez garde, car c’est peut-être une chose plus grave d’adorer des dieux si crédules que de leur manquer de parole. Homère lui-même n’a pas l’air de s’en rapporter à la fable, puisqu’en faisant de Neptune l’ennemi des Troyens, il leur donne pour ami Apollon, que le grief commun aurait dû mettre dans l’autre parti. Si donc vous croyez aux fables, rougissez d’adorer de pareils dieux ; si vous n’y croyez pas, ne parlez plus du parjure Laomédon ; ou bien alors expliquez-nous pourquoi ces dieux si sévères pour les parjures de Troie sont si indulgents pour ceux de Rome ; car autrement comment la conjuration de Catilina, même dans une ville aussi vaste et aussi corrompue que Rome, eût-elle trouvé un si grand nombre de partisans nourris de parjures et de sang romain ? Que faisaient chaque jour dans les jugements les sénateurs vendus, que faisait le peuple dans ses comices et dans les causes plaidées devant lui, que se parjurer sans cesse ? On avait conservé l’antique usage du serment au milieu de la corruption des mœurs, mais c’était moins pour arrêter les scélérats par une crainte religieuse que pour ajouter le parjure à tous les autres crimes.
\subsection[{Chapitre III}]{Chapitre III}

\begin{argument}\noindent Les dieux n’ont pu s’offenser de l’adultère de Pâris, ce crime étant commun parmi eux.
\end{argument}

\noindent C’est donc mal expliquer la ruine de Troie que de supposer les dieux indignés contre un roi parjure, puisqu’il est prouvé que ces dieux, dont la protection avait jusque-là maintenu l’empire troyen, à ce que Virgile assure, n’ont pu la défendre contre les Grecs victorieux. L’explication tirée de l’adultère de Pâris n’est pas plus soutenable ; car les dieux sont trop habitués à conseiller et à enseigner le crime pour s’en être faits les vengeurs. « La ville de Rome, dit Salluste, eut, selon la tradition, pour fondateurs et pour premiers habitants des Troyens fugitifs qui erraient çà et là sous la conduite d’Énée. »\par
Je conclus de là que si les dieux avaient cru devoir punir l’adultère de Pâris, ils auraientdû à plus forte raison, ou tout au moins au même titre, étendre leur vengeance sur les Romains, puisque cet adultère fut l’œuvre de la mère d’Énée. Mais pouvaient-ils détester dans Pâris un crime qu’ils ne détestaient point dans sa complice Vénus, devenue d’ailleurs mère d’Énée par son union adultère avec Anchise ? On dira peut-être que Ménélas fut indigné de la trahison de sa femme, au lieu que Vénus avait affaire à un mari complaisant. Je conviens que les dieux ne sont point jaloux de leurs femmes, à ce point même qu’ils daignent en partager la possession avec les habitants de la terre. Mais, pour qu’on ne m’accuse pas de tourner la mythologie en ridicule et de ne pas discuter assez gravement une matière de si grande importance, je veux bien ne pas voir dans Énée le fils de Vénus. Je demande seulement que Romulus ne soit pas le fils de Mars. Si nous admettons l’un de ces récits, pourquoi rejeter l’autre ? Quoi ! il serait permis aux dieux d’avoir commerce avec des femmes, et il serait défendu aux hommes d’avoir commerce avec les déesses ? En vérité, ce serait faire à Vénus une condition trop dure que de lui interdire en fait d’amour ce qui est permis au dieu Mars. D’ailleurs, les deux traditions ont également pour elles l’autorité de Rome, et César s’est cru descendant de Vénus tout autant que Romulus s’est cru fils du dieu de la guerre.
\subsection[{Chapitre IV}]{Chapitre IV}

\begin{argument}\noindent Sentiment de Varron sur l’utilité des mensonges qui font naître certains hommes du sang des dieux.
\end{argument}

\noindent Quelqu’un me dira : Est-ce que vous croyez à ces légendes ? Non, vraiment, je n’y crois pas ; et Varron même, le plus docte des Romains, n’est pas loin d’en reconnaître la fausseté, bien qu’il hésite à se prononcer nettement. Il dit que c’est une chose avantageuse à l’État que les hommes d’un grand cœur se croient du sang des dieux. Exaltée par le sentiment d’une origine si haute, l’âme conçoit avec plus d’audace de grands desseins, les exécute avec plus d’énergie et les conduit à leur terme avec plus de succès. Cette opinion de Varron, que j’exprime de mon mieux en d’autres ternies que les siens, vous voyez quelle large porte elle ouvre au mensonge,et il est aisé de comprendre qu’il a dû se fabriquer bien des faussetés touchant les choses religieuses, puisqu’on a jugé que le mensonge, même appliqué aux dieux, avait son utilité.
\subsection[{Chapitre V}]{Chapitre V}

\begin{argument}\noindent Il n’est point croyable que les dieux aient voulu punir l’adultère dans Pâris, l’ayant laissé impuni dans la mère de Romulus.
\end{argument}

\noindent Quant à savoir si Vénus a pu avoir Énée de son commerce avec Anchise, et Mars avoir Romulus de son commerce avec la fille de Numitor, c’est ce que je ne veux point présentement discuter ; car une difficulté analogue se rencontre dans nos saintes Écritures, quand il s’agit d’examiner si en effet les anges prévaricateurs se sont unis avec les filles des hommes et en ont eu ces géants, c’est-à-dire ces hommes prodigieusement grands et forts dont la terre fut alors remplie. Je me bornerai donc à ce dilemme : Si ce qu’on dit de la mère d’Énée et du père de Romulus est vrai, comment l’adultère chez les hommes peut-il déplaire aux dieux, puisqu’ils le souffrent chez eux avec tant de facilité ? Si cela est faux, il est également impossible que les dieux soient irrités des adultères véritables, puisqu’ils se plaisent au récit de leurs propres adultères supposés. Ajoutez que si l’on supprime l’adultère de Mars, afin de retrancher du même coup celui de Vénus, voilà l’honneur de la mère de Romulus bien compromis ; car elle était vestale, et les dieux ont dû venger plus sévèrement sur les Romains le crime de sacrilège que celui de parjure sur les Troyens. Les anciens Romains allaient même jusqu’à enterrer vives les vestales convaincues d’avoir manqué à la chasteté, au lieu que les femmes adultères subissaient une peine toujours plus douce que la mort ; tant il est vrai qu’ils étaient plus sévères pour la profanation des lieux sacrés que pour celle du lit conjugal.
\subsection[{Chapitre VI}]{Chapitre VI}

\begin{argument}\noindent Les dieux n’ont pas vengé le fratricide de Romulus.
\end{argument}

\noindent Il y a plus : si les crimes des hommesdéplaisaient tellement aux dieux qu’ils eussent abandonné Troie au carnage et à l’incendie pour punir l’adultère de Pâris, le meurtre du frère de Romulus aurait dû les irriter beaucoup plus contre les Romains que ne l’avait fait contre les Troyens l’injure d’un mari grec, et ils se seraient montrés plus sensibles au fratricide d’une ville naissante qu’à l’adultère d’un empire florissant. Et peu importe à la question que Romulus ait seulement donné l’ordre de tuer son frère, ou qu’il l’ait massacré de sa propre main, violence que les uns nient impudemment, tandis que d’autres la mettent en doute par pudeur, ou par douleur la dissimulent. Sans discuter sur ce point les témoignages de l’histoire, toujours est-il que le frère de Romulus fut tué, et ne le fut point par les ennemis, ni par des étrangers. C’est Romulus qui commit ce crime ou qui le commanda, et Romulus était bien plus le chef des Romains que Pâris ne l’était des Troyens. D’où vient donc que le ravisseur provoque la colère des dieux contre les Troyens, au lieu que le fratricide attire sur les Romains la faveur de ces mêmes dieux ? Que si Romulus n’a ni commis, ni commandé le crime, c’est toute la ville alors qui en est coupable, puisqu’en ne le vengeant pas elle a manqué à son devoir ; le crime est même plus grand encore ; car ce n’est plus un frère, mais un père qu’elle a tué, Rémus étant un de ses fondateurs, bien qu’une main criminelle l’ait empêché d’être un de ses rois. Je ne vois donc pas ce que Troie a fait de mal pour être abandonnée par les dieux et livrée à la destruction, ni ce que Rome a fait de bien pour devenir le séjour des dieux et la capitale d’un empire puissant, et il faut dire que les dieux, vaincus avec les Troyens, se sont réfugiés chez les Romains, afin de les tromper à leur tour, ou plutôt ils sont demeurés à Troie pour en séduire les nouveaux habitants, tout en abusant les habitants de Rome par de plus grands prestiges pour en tirer de plus grands honneurs.
\subsection[{Chapitre VII}]{Chapitre VII}

\begin{argument}\noindent De la seconde destruction de Troie par Fimbria, un des lieutenants de Marius.
\end{argument}

\noindent Quel nouveau crime en effet avait commis Troie pour mériter qu’au moment où éclatèrent les guerres civiles, le plus féroce des partisans de Marius, Fimbria, lui fît subir une destruction plus sanglante encore et plus cruelle que celle des Grecs ? Du temps de la première ruine, un grand nombre de Troyens trouva son salut dans la fuite, et d’autres en perdant la liberté conservèrent la vie ; mais Fimbria ordonna de n’épargner personne, et brûla la ville avec tous ses habitants. Voilà comment Troie fut traitée, non par les Grecs indignés de sa perfidie, mais par les Romains nés de son malheur, sans que les dieux, qu’elle adorait en commun avec ses bourreaux, se missent en peine de la secourir, ou pour mieux dire sans qu’ils en eussent le pouvoir. Est-il donc vrai que pour la seconde fois ils s’éloignèrent tous de leurs sanctuaires, et désertèrent leurs autels, ces dieux dont la protection maintenait une cité relevée de ses ruines ? Si cela est, j’en demande la raison car la cause des dieux me paraît ici d’autant plus mauvaise que je trouve meilleure celle des Troyens. Pour conserver leur ville à Sylla, ils avaient fermé leurs portes à Fimbria, qui, dans sa fureur, incendia et renversa tout. Or, à ce moment de la guerre civile, le meilleur parti était celui de Sylla ; car Sylla s’efforçait de délivrer la république opprimée. Les commencements de son entreprise étaient légitimes, et ses suites malheureuses n’avaient point encore paru. Qu’est-ce donc que les Troyens pouvaient faire de mieux, quelle conduite plus honnête, plus fidèle, plus convenable à leur parenté avec les Romains, que de conserver leur ville au meilleur parti, et de fermer leurs portes à celui qui portait sur la république ses mains parricides ? On sait ce que leur coûta cette fidélité ; que les défenseurs des dieux expliquent cela comme ils le pourront. Je veux que les dieux aient délaissé des adultères, et abandonné Troie aux flammes des Grecs, afin que Rome, plus chaste, naquît de ses cendres ; mais depuis, pourquoi ont-ils abandonné cette même ville, mère de Rome, et qui, loin de se révolter contre sa noble fille, gardait au contraire au parti le plus juste une sainte et inviolable fidélité ? pourquoi l’ont-ils laissée en proie, non pas aux Grecs généreux, mais au plus vil des Romains ? Que si le parti de Sylla, à qui ces infortunés avaient voulu conserver leur ville,déplaisait aux dieux, d’où vient qu’ils lui promettaient tant de prospérités ? cela ne prouve-t-il point qu’ils sont les flatteurs de ceux à qui sourit la fortune plutôt que les défenseurs des malheureux ? Ce n’est donc pas pour avoir été délaissée par les dieux que Troie a succombé. Les démons, toujours vigilants à tromper, firent ce qu’ils purent ; car au milieu des statues des dieux renversées et consumées, nous savons par Tite-Live qu’on trouva celle de Minerve intacte dans les ruines de son temple ; non sans doute afin qu’on pût dire à leur louange :\par
 {\itshape « Dieux de la patrie, dont la protection veille toujours sur Troie ! »} \par
mais afin qu’on ne dît pas à leur décharge\par
 {\itshape « Ils ont tous abandonné leurs sanctuaires et délaissé leurs autels. »} \par
Ainsi, il leur a été permis de faire ce prodige, non comme une consécration de leur pouvoir, mais comme une preuve de leur présence.
\subsection[{Chapitre VIII}]{Chapitre VIII}

\begin{argument}\noindent Rome devait-elle se mettre sous la protection des dieux de Troie ?
\end{argument}

\noindent Confier la protection de Rome aux dieux troyens après le désastre de Troie, quelle singulière prudence ! On dira peut-être que, lorsque Troie tomba sous les coups de Fimbria, les dieux s’étaient habitués depuis longtemps à habiter Rome. D’où vient donc que la statue de Minerve était restée debout dans les ruines d’Ilion ? Et puis, si les dieux étaient à Rome pendant que Fimbria détruisait Troie, ils étaient sans doute à Troie pendant que les Gaulois prenaient et brûlaient Rome ; mais comme ils ont l’ouïe très fine et les mouvements pleins d’agilité, ils accoururent au cri des oies, pour protéger du moins le Capitole ; quant à sauver le reste de la ville, ils ne le purent, ayant été avertis trop tard.
\subsection[{Chapitre IX}]{Chapitre IX}

\begin{argument}\noindent Faut-il attribuer aux dieux la paix dont jouirent les Romains sous le règne de Numa ?
\end{argument}

\noindent On s’imagine encore que si Numa Pompilius, successeur de Romulus, jouit de la paixpendant tout son règne et ferma les portes du temple de Janus qu’on a coutume de tenir ouvertes en temps de guerre, il dut cet avantage à la protection des dieux, en récompense des institutions religieuses qu’il avait établies chez les Romains. Et, sans doute, il y aurait à féliciter ce personnage d’avoir obtenu un si grand loisir, s’il avait su l’employer à des choses utiles et sacrifier une curiosité pernicieuse à la recherche et à l’amour du vrai Dieu ; mais, outre que ce ne sont point les dieux qui lui procurèrent ce loisir, je dis qu’ils l’auraient moins trompé, s’ils l’avaient trouvé moins oisif ; car moins ils le trouvèrent occupé, plus ils s’emparèrent de lui. C’est ce qui résulte des révélations de Varron, qui nous a donné la clef des institutions de Numa et des pratiques dont il se servit pour établir une société entre Rome et les dieux. Mais nous traiterons plus amplement ce sujet en son lieu, s’il plaît au Seigneur. Pour revenir aux prétendus bienfaits de ces divinités, je conviens que la paix est un bienfait, mais c’est un bienfait du vrai Dieu, et il en est d’elle comme du soleil, de la pluie et des autres avantages de la vie, qui tombent souvent sur les ingrats et les pervers. Supposez d’ailleurs que les dieux aient en effet procuré à Rome et à Numa un si grand bien, pourquoi ne l’ont-ils jamais accordé depuis à l’empire romain, même dans les meilleures époques ? est-ce que les rites sacrés de Numa avaient de l’influence, quand il les instituait, et cessaient d’en avoir, quand on les célébrait après leur institution ? Mais au temps de Numa, ils n’existaient pas encore, et c’est lui qui les fit ajouter au culte ; après Numa, ils existaient depuis longtemps, et on ne les conservait qu’en vue de leur utilité. Comment se fait-il donc que ces quarante-trois ans, ou selon d’autres, ces trente-neuf ans du règne de Numa se soient passés dans une paix continuelle, et qu’ensuite, une fois les rites établis et les dieux invoqués comme tuteurs et chefs de l’empire, il ne se soit trouvé, depuis la fondation de Rome jusqu’à Auguste, qu’une seule année, celle qui suivit la première guerre punique, où les Romains, car le fait est rapporté comme une grande merveille, aient pu fermer les portes du temple de Janus ?
\subsection[{Chapitre X}]{Chapitre X}\phantomsection
\label{\_chapitre10}

\begin{argument}\noindent S’il était désirable que l’empire romain s’accrut par de grandes et terribles guerres, alors qu’il suffisait, pour lui donner le repos et la sécurité, de la même protection qui l’avait fait fleurir sous Numa.
\end{argument}

\noindent Répondra-t-on que l’empire romain, sans cette suite continuelle de guerres, n’aurait pu étendre si loin sa puissance et sa gloire ? Mais quoi ! un empire ne saurait-il être grand sans être agité ? ne voyons-nous pas dans le corps humain qu’il vaut mieux n’avoir qu’une stature médiocre avec la santé que d’atteindre à la taille d’un géant avec des souffrances continuelles qui ne laissent plus un instant de repos et sont d’autant plus fortes qu’on a des membres plus grands ? quel mal y aurait-il, ou plutôt quel bien n’y aurait-il pas à ce qu’un État demeurât toujours au temps heureux dont parle Salluste, quand il dit : « Au commencement, les rois (c’est le premier nom de l’autorité sur la terre) avaient des inclinations différentes : les uns s’adonnaient aux exercices de l’esprit, les autres à ceux du corps. Alors la vie des hommes s’écoulait sans ambition ; chacun était content du sien. » Fallait-il donc, pour porter l’empire romain à ce haut degré de puissance, qu’il arrivât ce que déplore Virgile :\par
 {\itshape « Peu à peu le siècle se corrompt et se décolore ; bientôt surviennent la fureur de la guerre et l’amour de l’or.  »} \par
On dit, pour excuser les Romains d’avoir tant fait la guerre, qu’ils étaient obligés derésister aux attaques de leurs ennemis et qu’ils combattaient, non pour acquérir de la gloire,mais pour défendre leur vie et leur liberté. Eh bien ! soit ; car, comme dit Salluste : « Lorsque l’État, par le développement des lois, des mœurs et du territoire, eut atteint un certain degré de puissance, la prospérité, selon l’ordinaire loi des choses humaines, fit naître l’envie. Les rois et les peuples voisins de Rome lui déclarent la guerre ; ses alliés lui donnent peu de secours, la plupart saisis de crainte et ne cherchant qu’à écarter de soi le danger. Mais les Romains, attentifs au dehors comme au dedans, se hâtent, s’apprêtent, s’encouragent, vont au-devant de l’ennemi ; liberté, patrie,famille, ils défendent tout les armes à la main. Puis, quand le péril a été écarté par leur courage, ils portent secours à leurs « alliés, et se font plus d’amis à rendre des services qu’à en recevoir. » Voilà sans doute une noble manière de s’agrandir ; mais je serais bien aise de savoir si, sous le règne de Numa, où l’on jouit d’une si longue paix, les voisins de Rome venaient l’attaquer, ou s’ils demeuraient en repos, de manière à ne point troubler cet état pacifique ; car si Rome alors était provoquée, et si elle trouvait moyen, sans repousser les armes par les armes, sans déployer son impétuosité guerrière contre les ennemis, de les faire reculer, rien ne l’empêchait d’employer toujours le même moyen, et de régner en paix, les portes de Janus toujours closes. Que si cela n’a pas été en son pouvoir, il s’ensuit qu’elle n’est pas restée en paix tant que ses dieux l’ont voulu, mais tant qu’il a plu à ses voisins de la laisser en repos ; à moins que de tels dieux ne poussent l’impudence jusqu’à se faire un mérite de ce qui ne dépend que de la volonté des hommes. Il est vrai qu’il a été permis aux démons d’exciter ou de retenir les esprits pervers et de les faire agir par leur propre perversité ; mais ce n’est point d’une telle influence qu’il est question présentement ; d’ailleurs, si les démons avaient toujours ce pouvoir, s’ils n’étaient pas souvent arrêtés par une force supérieure et plus secrète, ils seraient toujours les arbitres de la paix et de la guerre, qui ont toujours leur cause dans les passions des hommes. Et cependant, il n’en est rien, comme on peut le prouver, non seulement par la fable, qui ment souvent et où l’on rencontre à peine quelque trace de vérité, mais aussi par l’histoire de l’empire romain.
\subsection[{Chapitre XI}]{Chapitre XI}

\begin{argument}\noindent De la statue d’Apollon de Cumes, dont on prétend que les larmes présagèrent la défaite des Grecs que le dieu ne pouvait secourir.
\end{argument}

\noindent Il n’y a d’autre raison que cette impuissance des dieux pour expliquer les larmes que versa pendant quatre jours Apollon de Cumes, au temps de la guerre contre les Achéens et le roi Aristonicus. Les aruspices effrayés furentd’avis qu’on jetât la statue dans la mer ; mais les vieillards de Cumes s’y opposèrent, disant que le même prodige avait éclaté pendant les guerres contre Antiochus et contre Persée, et que, la fortune ayant été favorable aux Romains, il avait été décrété par sénatus-consulte que des présents seraient envoyés à Apollon. Alors on fit venir d’autres aruspices plus habiles, qui déclarèrent que les larmes d’Apollon étaient de bon augure pour les Romains, parce que, Cumes étant une colonie grecque, ces larmes présageaient malheur au pays d’où elle tirait son origine. Peu de temps après on annonça que le roi Aristonicus avait été vaincu et pris : catastrophe évidemment contraire à la volonté d’Apollon, puisqu’il la déplorait d’avance et en marquait son déplaisir par les larmes de sa statue. On voit par là que les récits des poètes, tout fabuleux qu’ils sont, nous donnent des mœurs du démon une image qui ressemble assez à la vérité. Ainsi, dans Virgile, Diane plaint Camille, et Hercule pleure la mort prochaine de Pallas. C’est peut-être aussi pour cette raison que Numa, qui jouissait d’une paix profonde, mais sans savoir de qui il la tenait et sans se mettre en peine de le savoir, s’étant demandé dans son loisir à quels dieux il confierait le salut de Rome, Numa, dis-je, dans l’ignorance où il était du Dieu véritable et tout-puissant qui tient le gouvernement du monde, et se souvenant d’ailleurs que les dieux des Troyens apportés par Énée n’avaient pas longtemps conservé le royaume de Troie, ni celui de Lavinium qu’Énée lui-même avait fondé, Numa crut devoir ajouter d’autres dieux à ceux qui avaient déjà passé à Rome avec Romulus, comme on donne des gardes aux fugitifs et des aides aux impuissants.
\subsection[{Chapitre XII}]{Chapitre XII}\phantomsection
\label{\_chapitre12}

\begin{argument}\noindent Quelle multitude de dieux les Romains ont ajoutée à ceux de Numa, sans que cette abondance leur ait servi de rien.
\end{argument}

\noindent Et pourtant Rome ne daigna passe contenter des divinités déjà si nombreuses instituées par Numa. Jupiter n’avait pas encore son temple principal, et ce fut le roi Tarquin qui bâtit le Capitole. Esculape passa d’Épidaure à Rome, afin sans doute d’exercer sur un plus brillant théâtre ses talents d’habile médecin. Quant à la mère des dieux, elle vint je ne sais d’où, de Pessinunte. Aussi bien il n’était pas convenable qu’elle continuât d’habiter un lieu obscur, tandis que son fils dominait sur la colline du Capitole. S’il est vrai du reste qu’elle soit la mère de tous les dieux, on peut dire tout ensemble qu’elle a suivi à Rome certains de ses enfants et qu’elle en a précédé quelques autres. Je serais étonné pourtant qu’elle fût la mère de Cynocéphale, qui n’est venu d’Égypte que très tardivement. A-t-elle aussi donné le jour à la Fièvre ? c’est à son petit-fils Esculape de le décider ; mais quelle que soit l’origine de la Fièvre, je ne pense pas que des dieux étrangers osent regarder comme de basse condition une déesse citoyenne de Rome.\par
Voilà donc Rome sous la protection d’une foule de dieux ; car qui pourrait les compter ? indigènes et étrangers, dieux du ciel, de la terre, de la mer, des fontaines et des fleuves ; ce n’est pas tout, et il faut avec Varron y ajouter les dieux certains et les dieux incertains, dieux de toutes les espèces, les uns mâles, les autres femelles, comme chez les animaux. Eh bien ! avec tant de dieux, Rome devait-elle être en butte aux effroyables calamités qu’elle a éprouvées et dont je ne veux rapporter qu’un petit nombre ? Élevant dans les airs l’orgueilleuse fumée de ses sacrifices, elle avait appelé, comme par un signal, cette multitude de dieux à son secours, leur prodiguant les temples, les autels, les victimes et les prêtres, au mépris du Dieu véritable et souverain qui seul a droit à ces hommages. Et pourtant elle était plus heureuse quand elle avait moins de dieux ; mais à mesure qu’elle s’est accrue, elle a pensé qu’elle avait besoin d’un plus grand nombre de dieux, comme un plus vaste navire demande plus de matelots, s’imaginant sans doute que ces premiers dieux, sous lesquels ses mœurs étaient pures en comparaison de ce qu’elles furent depuis, ne suffisaient plus désormais à soutenir le poids de sa grandeur. Déjà en effet, sous ses rois mêmes, à l’exception de Numa dont j’ai parlé plus haut, il faut que l’esprit de discorde eût fait bien des ravages, puisqu’il poussa Romulus au meurtre de son frère.
\subsection[{Chapitre XIII}]{Chapitre XIII}

\begin{argument}\noindent Par quel moyen les Romains se procurèrent pour la première fois des épouses.
\end{argument}

\noindent Comment se fait-il que ni Junon, qui dès lors, d’accord avec son Jupiter,\par
 {\itshape « Couvrait de sa protection les Romains dominateurs du monde et le peuple vêtu de la toge »} \par
ni Vénus même, protectrice des enfants de son cher Énée, n’aient pu leur procurer de bons et honnêtes mariages ? car ils furent obligés d’enlever des filles pour les épouser, et de faire ensuite à leurs beaux-pères une guerre où ces malheureuses femmes, à peine réconciliées avec leurs maris, reçurent en dot le sang de leurs parents ? Les Romains, dit-on, sortirent vainqueurs du combat ; mais à combien de proches et d’alliés cette victoire coûta-t-elle la vie, et de part et d’autre quel nombre de blessés ! La guerre de César et de Pompée n’était que la lutte d’un seul beau-père contre un seul gendre, et encore, quand elle éclata, la fille de César, l’épouse de Pompée n’était plus ; et cependant, c’est avec un trop juste sentiment de douleur que Lucain s’écrie :\par
 {\itshape « Je chante cette guerre plus que civile, terminée aux champs de l’Émathie et où le crime fut justifié par la victoire. »} \par
Les Romains vainquirent donc, et ils purent dès lors, les mains encore toutes sanglantes du meurtre de leurs beaux-pères, obliger leurs filles à souffrir de funestes embrassements, tandis que celles-ci, qui pendant le combat ne savaient pour qui elles devaient faire des vœux, n’osaient pleurer leurs pères morts, de crainte d’offenser leurs maris victorieux. Ce ne fut pas Vénus qui présida à ces noces, mais Bellone, ou plutôt Alecto, cette furie d’enfer qui fit ce jour-là plus de mal aux Romains, en dépit de la protection que déjà leur accordait Junon, que lorsqu’elle fut déchaînée contre eux par cette déesse.\par
La captivité d’Andromaque fut plus heureuse que ces premiers mariages romains ; car, depuis que Pyrrhus fut devenu son époux, il ne fit plus périr aucun Troyen, au lieu que les Romains tuaient sur le champ de bataille ceux dont ils embrassaient les filles dans leurs lits. Andromaque, sous la puissance du vainqueur, avait sans doute à déplorer la mort de ses parents, mais elle n’avait plus à la craindre ; ces pauvres femmes, au contraire, craignaient la mort de leurs pères, quand leurs maris allaient au combat, et la déploraient en les voyant revenir, ou plutôt elles n’avaient ni la liberté de leur crainte ni celle de leur douleur. Comment, en effet, voir sans douleur la mort de leurs concitoyens, de leurs parents, de leurs frères, de leurs pères ? Et comment se réjouir sans cruauté de la victoire de leurs maris ? Ajoutez que la fortune des armes est journalière et que plusieurs perdirent en même temps leurs époux et leurs pères ; car les Romains ne furent pas sans éprouver quelques revers. On les assiégea dans leur ville, et après quelque résistance, les assaillants ayant trouvé moyen d’y pénétrer, il s’engagea dans le Forum même une horrible mêlée entre les beaux-pères et les gendres. Les ravisseurs avaient le dessous et se sauvaient à tout moment dans leurs maisons, souillant ainsi par leur lâcheté d’une honte nouvelle leur premier exploit déjà si honteux et si déplorable. Ce fut alors que Romulus, désespérant de la valeur des siens, pria Jupiter de les arrêter, ce qui fit donner depuis à ce dieu le surnom de Stator. Mais cela n’aurait encore servi de rien, si les femmes ne se fussent jetées aux genoux de leurs pères, les cheveux épars, et n’eussent apaisé leur juste colère par d’humbles supplications. Enfin, Romulus, qui n’avait pu souffrir à côté de lui son propre frère, et un frère jumeau, fut contraint de partager la royauté avec Tatius, roi des Sabins ; à la vérité il s’en défit bientôt, et demeura seul maître, afin d’être un jour un plus grand dieu. Voilà d’étranges contrats de noces, féconds en luttes sanglantes, et de singuliers actes de fraternité, d’alliance, de parenté, de religion ! voilà les mœurs d’une cité placée sous le patronage de tant de dieux ! On devine assez tout ce que je pourrais dire là-dessus, si mon sujet ne m’entraînait vers d’autres discours.
\subsection[{Chapitre XIV}]{Chapitre XIV}

\begin{argument}\noindent De la guerre impie que Rome fit aux Albains et du succès que lui valut son ambition.
\end{argument}

\noindent Qu’arriva-t-il ensuite après Numa, sous les autres rois, et quels maux ne causa point, aux Albains comme aux Romains, la guerre provoquée par ceux-ci, qui s’ennuyaient sans doute de la longue paix de Numa ? Que de sang répandu par les deux armées rivales, au grand dommage des deux États ! Albe, qui avait été fondée par Ascagne, fils d’Énée, et qui était de plus près que Troie la mère de Rome, fut attaquée par Tullus Hostilius ; mais si elle reçut du mal des Romains, elle ne leur en fit pas moins, au point qu’après plusieurs combats les deux partis, lassés de leurs pertes, furent d’avis de terminer leurs différends par le combat singulier de trois jumeaux de chaque parti. Les trois Horaces ayant été choisis du côté des Romains et les trois Curiaces du côté des Albains, deux Horaces furent tués d’abord par les trois Curiaces ; mais ceux-ci furent tués à leur tour par le seul Horace survivant. Ainsi Rome demeura victorieuse, mais à quel prix ? sur six combattants, un seul revint du combat. Après tout, pour qui fut le deuil et le dommage, si ce n’est pour les descendants d’Énée, pour la postérité d’Ascagne, pour la race de Vénus, pour les petits-fils de Jupiter ? Cette guerre ne fut-elle pas plus que civile, puisque la cité fille y combattit contre la cité mère ? Ajoutez à cela un autre crime horrible et atroce qui suivit ce combat des jumeaux. Comme les deux peuples étaient auparavant amis, à cause du voisinage et de la parenté, la sœur des Horaces avait été fiancée à l’un des Curiaces ; or, cette fille ayant aperçu son frère qui revenait chargé des dépouilles de son mari, ne put retenir ses larmes, et, pour avoir pleuré, son frère la tua. Je trouve qu’en cette rencontre cette fille se montra plus humaine que tout le peuple romain, et je ne vois pas qu’on la puisse blâmer d’avoir pleuré celui à qui elle avait déjà donné sa foi, que dis-je ? d’avoir pleuré peut-être sur un frère couvert du sang de l’homme à qui il avait promis sa sœur. On applaudit aux larmes que verse Énée, dans Virgile, sur son ennemi qu’il a tué de sa propre main et c’est encore ainsi que Marcellus, sur le point de détruire Syracuse, au souvenu de la splendeur où cette ville était parvenue avant de tomber sous ses coups, laissa couler des larmes de compassion. À mon tour, je demande au nom de l’humanité qu’on ne fasse point un crime à une femme d’avoir pleuré son mari, tué par son frère, alors que d’autres ont mérité des éloges pour avoir pleuré leurs ennemis par eux-mêmes vaincus. Dans le temps que cette fille pleurait la mort de son fiancé, que son frère avait tué, Rome se réjouissait d’avoir combattu avec tant de rage contre la cité sa mère, au prix de torrents de sang répandus de part et d’autre par des mains parricides.\par
À quoi bon m’alléguer ces beaux noms de gloire et de triomphe ? Il faut écarter ces vains préjugés, il faut regarder, peser, juger ces actions en elles-mêmes. Qu’on nous cite le crime d’Albe comme on nous parle de l’adultère de Troie, on ne trouvera rien de pareil, rien d’approchant. Si Albe est attaquée, c’est uniquement parce que\par
 {\itshape « Tullus veut réveiller les courages endormis des bataillons romains, qui se désaccoutumaient de la victoire ».} \par
Il n’y eut donc qu’un motif à cette guerre criminelle et parricide, ce fut l’ambition, vice énorme que Salluste ne manque pas de flétrir en passant, quand après avoir célébré les temps primitifs, où les hommes vivaient sans convoitise et où chacun était content du sien, il ajoute : « Mais depuis que Cyrus en Asie, les Lacédémoniens et les Athéniens en Grèce, commencèrent à s’emparer des villes et des nations, à prendre pour un motif de guerre l’ambition de s’agrandir, à mettre la gloire de l’État dans son étendue… », et tout ce qui suit sans que j’aie besoin de prolonger la citation. Il faut avouer que cette passion de dominer cause d’étranges désordres parmi les hommes. Rome était vaincue par elle quand elle se vantait d’avoir vaincu Albe et donnait le nom de gloire à l’heureux succès de son crime. Car, comme dit l’Écriture : « On loue le pécheur de ses mauvaises convoitises, et celui qui consomme l’iniquité est béni. » Écartons donc ces déguisements artificieux et ces fausses couleurs, afin depouvoir juger nettement les choses. Que personne ne me dise : Celui-là est un vaillant homme, car il s’est battu contre un tel et l’a vaincu. Les gladiateurs combattent aussi et triomphent, et leur cruauté trouve des applaudissements ; mais j’estime qu’il vaut mieux être taxé de lâcheté que de mériter de pareilles récompenses. Cependant, si dans ces combats de gladiateurs l’on voyait descendre dans l’arène le père contre le fils, qui pourrait souffrir un tel spectacle ? qui n’en aurait horreur ? Comment donc ce combat de la mère et de la fille, d’Albe et de Rome, a-t-il pu être glorieux à l’une et à l’autre ? Dira-t-on que la comparaison n’est pas juste, parce qu’Albe et Rome ne combattaient pas dans une arène ? Il est vrai ; mais au lieu de l’arène, c’était un vaste champ où l’on ne voyait pas deux gladiateurs, mais des armées entières joncher la terre de leurs corps. Ce combat n’était pas renfermé dans un amphithéâtre, mais il avait pour spectateurs l’univers entier et tous ceux qui dans la suite des temps devaient entendre parler de ce spectacle impie.\par
Cependant ces dieux tutélaires de l’empire romain, spectateurs de théâtre à ces sanglants combats, n’étaient pas complétement satisfaits ; et ils ne furent contents que lorsque la sœur des Horaces, tuée par son frère, fut allée rejoindre les trois Curiaces, afin sans doute que Rome victorieuse n’eût pas moins de morts qu’Albe vaincue. Quelque temps après, pour fruit de cette victoire, Albe fut ruinée, Albe, où ces dieux avaient trouvé leur troisième asile depuis qu’ils étaient sortis de Troie ruinée par les Grecs, et de Lavinium, où le roi Latinus avait reçu Énée étranger et fugitif. Mais peut-être étaient-ils sortis d’Albe, suivant leur coutume, et voilà sans doute pourquoi Albe succomba. Vous verrez qu’il faudra dire encore :\par
 {\itshape « Tous les dieux protecteurs de cet empire se sont retirés, abandonnant leurs temples et leurs autels. »} \par
Vous verrez qu’ils ont quitté leur séjour pour la troisième fois, afin qu’une quatrième Rome fût très sagement confiée à leur protection. Albe leur avait déplu, à ce qu’il paraît, parce qu’Amulius, pour s’emparer du trône, avait chassé son frère, et Rome ne leur déplaisait pas, quoique Romulus eût tué le sien. Mais, dit-on, avant de ruiner Albe, onen avait transporté les habitants à Rome pour ne faire qu’une ville des deux. Je le veux bien, mais cela n’empêche pas que la ville d’Ascagne, troisième retraite des dieux de Troie, n’ait été ruinée par sa fille. Et puis, pour unir en un seul corps les débris de ces deux peuples, combien de sang en coûta-t-il à l’un et à l’autre ? Est-il besoin que je rapporte en détail comment ces guerres, qui semblaient terminées par tant de victoires, ont été renouvelées sous les autres rois, et comment, après tant de traités conclus entre les gendres et les beaux-pères, leurs descendants ne laissèrent pas de reprendre les armes et de se battre avec plus de rage que jamais ? Ce n’est pas une médiocre preuve de ces calamités qu’aucun des rois de Rome n’ait fermé les portes du temple de Janus, et cela fait assez voir qu’avec tant de dieux tutélaires aucun d’eux n’a pu régner en paix.
\subsection[{Chapitre XV}]{Chapitre XV}

\begin{argument}\noindent Quelle a été la vie et la mort des rois de Rome.
\end{argument}

\noindent Et quelle fut la fin de ces rois eux-mêmes ? Une fable adulatrice place Romulus dans le ciel, mais plusieurs historiens rapportent au contraire qu’il fut mis en pièces par le sénat à cause de sa cruauté, et que l’on suborna un certain Julius Proculus pour faire croire que Romulus lui était apparu et l’avait chargé d’ordonner de sa part au peuple romain de l’honorer comme un dieu, expédient qui apaisa le peuple sur le point de se soulever contre le sénat. Une éclipse de soleil survint alors fort à propos pour confirmer cette opinion ; car le peuple, peu instruit des secrets de la nature, ne manqua pas de l’attribuer à la vertu de Romulus : comme si la défaillance de cet astre, à l’interpréter en signe de deuil, ne devait pas plutôt faire croire que Romulus avait été assassiné et que le soleil se cachait pour ne pas voir un si grand crime, ainsi qu’il arriva en effet lorsque la cruauté et l’impiété des Juifs attachèrent en croix Notre-Seigneur. Pour montrer que l’obscurcissement du soleil, lors de ce dernier événement, n’arriva pas suivant le cours ordinaire des astres, il suffit de considérer que les Juifs célébraient alors la pâque, ce qui n’a lieu que dans la pleine lune : or, les éclipses de soleil n’arrivent jamais naturellement qu’à la fin de la lunaison. Cicéron témoigne aussi que l’entrée de Romulus parmi les dieux est plutôt imaginaire que réelle, lorsque le faisant louer par Scipion dans ses livres {\itshape De la République}, il dit : « Romulus laissa de lui une telle idée, qu’étant disparu tout d’un coup pendant une éclipse de soleil, on crut qu’il avait été enlevé parmi les dieux : opinion qu’on n’a jamais eue d’un mortel sans qu’il n’ait déployé une vertu extraordinaire. » Et quant à ce que dit Cicéron que Romulus disparut tout d’un coup, ces paroles marquent ou la violence de la tempête qui le fit périr, ou le secret de l’assassinat : attendu que, suivant d’autres historiens, l’éclipse fut accompagnée de tonnerres qui, sans doute, favorisèrent le crime ou même consumèrent Romulus. En effet, Cicéron, dans l’ouvrage cité plus haut, dit, à propos de Tullus Hostilius, troisième roi de Rome, tué aussi d’un coup de foudre, qu’on ne crut pas pour cela qu’il eût été reçu parmi les dieux, comme on le croyait de Romulus, afin peut-être de ne pas avilir cet honneur en le rendant trop commun. Il dit encore ouvertement dans ses harangues : « Le fondateur de cette cité, Romulus, nous l’avons, par notre bienveillance et l’autorité de la renommée, élevé au rang des dieux immortels. » Par où il veut faire entendre que la divinité de Romulus n’est point une chose réelle, mais une tradition répandue à la faveur de l’admiration et de la reconnaissance qu’inspiraient ses grands services. Enfin, dans son {\itshape Hortensius}, il dit, au sujet des éclipses régulières du soleil : « Pour produire les mêmes ténèbres qui couvrirent la mort de Romulus, arrivée pendant une éclipse… » Certes, dans ce passage, il n’hésite point à parler de Romulus comme d’un homme réellement mort ; et pourquoi cela ? parce qu’il n’en parle plus en panégyriste, mais en philosophe.\par
Quant aux autres rois de Rome, si l’on excepte Numa et Ancus, qui moururent de maladie, combien la fin des autres a-t-elle été funeste ? Tullus Hostilius, ce destructeur de la ville d’Albe, fut consumé, comme j’ai dit, par le feu du ciel, avec toute sa maison. Tarquin l’Ancien fut tué par les enfants de son prédécesseur, et Servius Tullius par son gendre Tarquin le Superbe, qui lui succéda.\par
Cependant, après un tel assassinat, commis contre un si bon roi, les dieux ne quittèrent point leurs temples et leurs autels, eux qui, pour l’adultère de Pâris, sortirent de Troie et abandonnèrent cette ville à la fureur des Grecs. Bien loin de là, Tarquin succéda à Tullius, qu’il avait tué, et les dieux, au lieu de se retirer, eurent bien le courage de voir ce meurtrier de son beau-père monter sur le trône, remporter plusieurs victoires éclatantes sur ses ennemis et de leurs dépouilles bâtir le Capitole ; ils souffrirent même que Jupiter, leur roi, régnât du haut de ce superbe temple, ouvrage d’une main parricide ; car Tarquin n’était pas innocent quand il construisit le Capitole, puisqu’il ne parvint à la couronne que par un horrible assassinat. Quand plus tard les Romains le chassèrent du trône et de leur ville, ce ne fut qu’à cause du crime de son fils, et ce crime fut commis non seulement à son insu, mais en son absence. Il assiégeait alors la ville d’Ardée ; il combattait pour le peuple romain. On ne peut savoir ce qu’il eût fait si on se fût plaint à lui de l’attentat de son fils ; mais, sans attendre son opinion et son jugement à cet égard, le peuple lui ôta la royauté, ordonna aux troupes d’Ardée de revenir à Rome, et en ferma les portes au roi déchu. Celui-ci, après avoir soulevé contre eux leurs voisins et leur avoir fait beaucoup de mali forcé de renoncer à son royaume par la trahison des amis en qui il s’était confié, se retira à Tusculum, petite ville voisine de Rome, où il vécut de la vie privée avec sa femme l’espace de quatorze ans, et finit ses jours d’une manière plus heureuse que son beau-père, qui fut tué par le crime d’un gendre et d’une fille. Cependant les Romains ne l’appelèrent point le Cruel ou le Tyran, mais le Superbe, et cela peut-être parce qu’ils étaient trop orgueilleux pour souffrir son orgueil. En effet, ils tinrent si peu compte du crime qu’il avait commis en tuant son beau-père, qu’ils l’élevèrent à la royauté ; en quoi je me trompe fort si la récompense ainsi accordée à un crime ne fut pas un crime plus énorme. Malgré tout, les dieux ne quittèrent point leurs temples et leurs autels. À moins qu’on ne veuille dire pour les défendre qu’ils ne demeurèrent à Rome que pour punir les Romains en les séduisant par de vains triomphes et les accablant par des guerres sanglantes. Voilà quelle fut la fortune des Romains sous leurs rois, dans les plus beaux jours de l’empire, et jusqu’à l’exil de Tarquin le Superbe, c’est-à-dire l’espace d’environ deux cent quarante-trois ans, pendant lesquels toutes ces victoires, achetées au prix de tant de sang et de calamités, étendirent à peine cet empire jusqu’à vingt milles de Rome, territoire qui n’est pas comparable à celui de la moindre ville de Gétulie.
\subsection[{Chapitre XVI}]{Chapitre XVI}

\begin{argument}\noindent De Rome sous ses premiers consuls, dont l’un exila l’autre et fut tué lui-même par un ennemi qu’il avait blessé, après s’être souillé des plus horribles parricides.
\end{argument}

\noindent Ajoutons à cette époque celle où Salluste assure que Rome se gouverna avec justice et modération, et qui dura tant qu’elle eut à redouter le rétablissement de Tarquin et les armes des Étrusques. En effet, la situation de Rome fut très critique au moment où les Étrusques se liguèrent avec le roi déchu. Et c’est ce qui fait dire à Salluste que si la république fut alors gouvernée avec justice et modération, la crainte des ennemis y contribua plus que l’amour du bien. Dans ce temps si court, combien fut désastreuse l’année où les premiers consuls furent créés après l’expulsion des rois ! Ils n’achevèrent pas seulement le temps de leur magistrature, puisque Junius Brutus força son collègue Tarquin Collatin à se démettre de sa charge et à sortir de Rome, et que lui-même fut tué à peu de temps de là dans un combat où il s’enferra avec l’un des fils de Tarquin, après avoir fait mourir ses propres enfants et les frères de sa femme comme coupables d’intelligence avec l’ancien roi. Virgile ne peut se défendre de détester cette action, tout en lui donnant des éloges. À peine a-t-il dit :\par
{\itshape « Voilà ce père, qui, pour sauver la sainte liberté romaine, envoie au supplice ses enfants convaincus de trahison »}, \par
qu’il s’écrie aussitôt :\par
 {\itshape « Infortuné, quelque jugement que porte sur toi l’avenir ! »} \par
C’est-à-dire, malheureux père en dépit deslouanges de la postérité. Et, comme pour le consoler, il ajoute :\par
 {\itshape « Mais l’amour de la patrie et une immense passion de gloire triomphent de ton cœur ».} \par
Cette destinée de Brutus, meurtrier de ses enfants, tué par le fils de Tarquin qu’il vient de frapper à mort, ne pouvant survivre au fils et voyant le père lui survivre, ne semble-t-elle pas venger l’innocence de son collègue Collatin, citoyen vertueux, qui, après l’expulsion de Tarquin, fut traité aussi durement que le tyran lui-même ? Remarquez en effet que Brutus était, lui aussi, à ce qu’on assure, parent de Tarquin ; seulement il n’en portait pas le nom comme Collatin. On devait donc l’obliger à quitter son nom, mais non pas sa patrie ; il se fût appelé Lucius Collatin, et la perte d’un mot ne l’eût touché que très faiblement ; mais ce n’était pas le compte de Brutus, qui voulait lui porter un coup plus sensible en privant l’État de son premier consul et la patrie d’un bon citoyen. Fera-t-on cette fois encore un titre d’honneur à Brutus d’une action aussi révoltante et aussi inutile à la république ? Dira-t-on que :\par
 {\itshape « L’amour de la patrie et une immense passion de gloire ont triomphé de son cœur ? »} \par
Après qu’on eut chassé Tarquin le Superbe, Tarquin Collatin, mari de Lucrèce, fut créé consul avec Brutus. Combien le peuple romain se montra équitable, en regardant au nom d’un tel citoyen moins qu’à ses mœurs, et combien, au contraire, Brutus fut injuste, en ôtant à son collègue sa charge et sa patrie, quand il pouvait se borner à lui ôter son nom, si ce nom le choquait ! Voilà les crimes, voilà les malheurs de Rome au temps même qu’elle était gouvernée avec quelque justice et quelque modération. Lucrétius, qui avait été subrogé en la place de Brutus, mourut aussi avant la fin de l’année, Ainsi, Publius Valérius, qui avait succédé à Collatin, et Marcus Horatius, qui avait pris la place de Lucrétius, achevèrent cette année funeste et lugubre qui compta cinq consuls : triste inauguration de la puissance consulaire !
\subsection[{Chapitre XVII}]{Chapitre XVII}

\begin{argument}\noindent Des maux que la république romaine eut à souffrir après les commencements du pouvoir consulaire, sans que les dieux se missent en devoir de la secourir.
\end{argument}

\noindent Quand la crainte de l’étranger vint à s’apaiser, quand la guerre, sans être interrompue, pesa d’un poids moins lourd sur la république, ce fut alors que le temps de la justice et de la modération atteignit son terme, pour faire place à celui que Salluste décrit en ce peu de mots : « Les patriciens se mirent à traiter les gens du peuple en esclaves, condamnant celui-ci à mort, et celui-là aux verges, comme avaient fait les rois, chassant le petit propriétaire de son champ et imposant à celui qui n’avait rien la plus dure tyrannie. Accablé de ces vexations, écrasé surtout par l’usure, le bas peuple, sur qui des guerres continuelles faisaient peser, avec le service militaire, les plus lourds impôts, prit les armes et se retira sur le mont Sacré et sur l’Aventin ; ce fut ainsi qu’il obtint ses tribuns et d’autres prérogatives. Mais la lutte et les discordes ne furent entièrement éteintes qu’à la seconde guerre punique. » Mais à quoi bon arrêter mes lecteurs et m’arrêter moi-même au détail de tant de maux ? Salluste ne nous a-t-il pas appris en peu de paroles combien, durant cette longue suite d’années qui se sont écoulées jusqu’à la seconde guerre punique, Rome a été malheureuse, tourmentée au dehors par des guerres, agitée au dedans par des séditions ? Les victoires qu’elle a remportées dans cet intervalle ne lui ont point donné de joies solides ; elles n’ont été que de vailles consolations pour ses infortunes, et des amorces trompeuses à des esprits inquiets qu’elles engageaient de plus en plus dans des malheurs inutiles. Que les bons et sages Romains ne s’offensent point de notre langage ; et comment s’en offenseraient-ils, puisque nous ne disons rien de plus fort que leurs propres auteurs, qui nous laissent loin derrière eux par l’éclat de leurs tableaux composés à loisir, et dont les ouvrages sont la lecture habituelle des Romains et de leurs enfants ? À ceux qui viendraient à s’irriter contre moi, je demanderais comment donc ils me traiteraient, si je disais ce qu’on lit dans Salluste : « Les querelles, les séditions s’élevèrent et enfin les guerres civiles, tandis qu’un petit nombre d’hommes puissants, qui tenaient la plupart des autres dans leur dépendance, affectaient la domination sous le spécieux prétexte du bien du peuple et du sénat ; et l’on appelait bons citoyens, non ceux qui servaient les intérêts de la république (car tous étaient également corrompus), mais ceux qui par leur richesse et leur crédit maintenaient l’état présent des choses. » Si donc ces historiens ont cru qu’il leur était permis de rapporter les désordres de leur patrie, à laquelle ils donnent d’ailleurs tant de louanges, faute de connaître cette autre patrie plus véritable qui sera composée de citoyens immortels, que ne devons-nous point faire, nous qui pouvons parler avec d’autant plus de liberté que notre espérance en Dieu est meilleure et plus certaine, et que nos adversaires imputent plus injustement à Jésus-Christ les maux qui affligent maintenant le monde, afin d’éloigner les personnes faibles et ignorantes de la seule cité où l’on puisse vivre éternellement heureux ? Au reste, nous ne racontons pas de leurs dieux plus d’horreurs que ne font leurs écrivains les plus vantés et les plus répandus ; c’est dans ces écrivains mêmes que nous puisons nos témoignages, et encore ne pouvons-nous pas tout dire, ni dire les choses comme eux.\par
Où étaient donc ces dieux que l’on croit qui peuvent servir pour la chétive et trompeuse félicité de ce monde, lorsque les Romains, dont ils se faisaient adorer par leurs prestiges et leurs impostures, souffraient de si grandes calamités ? où étaient-ils, quand Valérius fut tué en défendant le Capitole incendié par une troupe d’esclaves et de bannis ? Il fut plus aisé à ce consul de secourir le temple qu’à cette armée de dieux et à leur roi très grand et très excellent, Jupiter, de venir au secours de leur libérateur. Où étaient-ils, quand Rome, fatiguée de tant de séditions et qui attendait dans un état assez calme le retour des députés qu’elle avait envoyés à Athènes pour en emprunter des lois, fut désolée par une famine et par une peste épouvantables ? Où étaient-ils, quand le peuple, affligé de nouveau par la disette, créa pour la première fois un préfet des vivres ; et quand Spurius Mélius, pour avoir distribué du blé au peuple affamé, fut accusé par ce préfet devant le vieux dictateur Quintius d’affecter la royauté et tué par Servilius, général de la cavalerie, au milieu du plus effroyable tumulte qui ait jamais alarmé la république ? Où étaient-ils, quand Rome, envahie par une terrible peste, après avoir employé tous les moyens de salut et imploré longtemps en vain le secours des dieux, s’avisa enfin de leur dresser des lits dans les temples, chose qui n’avait jamais été faite jusqu’alors, et qui fit donner le nom de Lectisternes à ces cérémonies sacrées ou plutôt sacrilèges ? Où étaient-ils, quand les armées romaines, épuisées par leurs défaites dans une guerre de dix ans contre les Véïens, allaient succomber sans l’assistance de Camille, condamné depuis par son ingrate patrie ? Où étaient-ils, quand les Gaulois prirent Rome, la pillèrent, la brûlèrent, la mirent à sac ? Où étaient-ils, quand une furieuse peste la ravagea et enleva ce généreux Camille, vainqueur des Véïens et des Gaulois ? Ce fut durant cette peste qu’on introduisit à Rome les jeux de théâtre, autre peste plus fatale, non pour les corps, mais pour les âmes. Où étaient-ils, quand un autre fléau se déclara dans la cité, je veux parler de ces empoisonnements imputés aux dames romaines des plus illustres familles, et qui révélèrent dans les mœurs un désordre pire que tous les fléaux ? Et quand l’armée romaine, assiégée par les Samnites avec ses deux consuls, aux Fourches-Caudines, fut obligée de subir des conditions honteuses et de passer sous le joug, après avoir donné en otage six cents chevaliers ? Et quand, au milieu des horreurs de la peste, la foudre vint tomber sur le camp des Romains ? Et quand Rome, affligée d’une autre peste non moins effroyable, fut contrainte de faire venir d’Épidaure Esculape à titre de médecin, faute de pouvoir réclamer les soins de Jupiter, qui depuis longtemps toutefois faisait sa demeure au Capitole, mais qui, ayant eu une jeunesse fort dissipée, n’avait probablement pas trouvé le temps d’apprendre la médecine ? Et quand les Laconiens, les Bruttiens, les Samnites et les Toscans, ligués avec les Gaulois Sénonais contre Rome, firent d’abord mourir ses ambassadeurs, mirent ensuite son armée en déroute et taillèrent en pièces treize mille hommes, avec le préteur et sept tribuns militaires ? Et quand enfin le peuple, après de longues et fâcheuses séditions, s’étant retiré sur le mont Aventin, on fut obligé d’avoir recours à une magistrature instituée pour les périls extrêmes et de nommer dictateur Hortensius, qui ramena le peuple à Rome et mourut dans l’exercice de ses fonctions : chose singulière, qui ne s’était pas encore vue et qui constitua un grief d’autant plus grave contre les dieux, que le médecin Esculape était alors présent dans la cité ?\par
Tant de guerres éclatèrent alors de toutes parts que, faute de soldats, on fut obligé d’enrôler les prolétaires, c’est-à-dire ceux qui, trop pauvres pour porter les armes, ne servaient qu’à donner des enfants à la république. Les Tarentins appelèrent à leur secours contre les Romains Pyrrhus, roi d’Épire, alors si fameux. Ce fut à ce roi qu’Apollon, consulté par lui sur le succès de son entreprise, répondit assez agréablement par un oracle si ambigu que le dieu, quoi qu’il arrivât, ne pouvait manquer d’avoir été bon prophète. Cet oracle, en effet, signifiait également que Pyrrhus vaincrait les Romains ou qu’il en serait vaincu, de sorte qu’Apollon n’avait qu’à attendre l’événement en sécurité. Quel horrible carnage n’y eut-il point alors dans l’une et l’autre armée ? Pyrrhus toutefois demeura vainqueur, et il aurait pu dès lors expliquer à son avantage la réponse d’Apollon, si, peu de temps après, dans un autre combat, les Romains n’avaient eu le dessus. À tant de massacres succéda une étrange maladie qui enlevait les femmes enceintes avant le moment de leur délivrance. Esculape, sans doute, s’excusait alors sur ce qu’il était médecin et non sage-femme. Le mal s’étendait même au bétail, qui périssait en si grand nombre qu’il semblait que la race allait s’en éteindre. Que dirai-je de cet hiver mémorable où le froid fut si rigoureux que les neiges demeurèrent prodigieusement hautes dans les rues de Rome l’espace de quinze jours et que le Tibre fut glacé ? si cela était arrivé de notre temps, que ne diraient point nos adversaires contre les chrétiens ? Parlerai-je encore de cette peste mémorable qui emporta tant de monde, et qui, prenant d’une année à l’autre plus d’intensité, sans que la présence d’Esculape servit de rien, obligea d’avoir recours aux livressibyllins, espèces d’oracles pour lesquels, suivant Cicéron, dans ses livres sur la divination, on s’en rapporte aux conjectures de ceux qui les interprètent comme ils peuvent ou comme ils veulent ? Les interprètes dirent donc alors que la peste venait de ce que plusieurs particuliers occupaient des lieux sacrés, réponse qui vint fort à propos pour sauver Esculape du reproche d’impéritie honteuse ou de négligence. Or, comment ne s’était-il trouvé personne qui s’opposât à l’occupation de ces lieux sacrés, sinon parce que tous étaient également las de s’adresser si longtemps et sans fruit à cette foule de divinités ? Ainsi ces lieux étaient peu à peu abandonnés par ceux qui les fréquentaient, afin qu’au moins, devenus vacants, ils pussent servir à l’usage des hommes. Les édifices mêmes qu’on rendit alors à leur destination pour arrêter la peste, furent encore depuis négligés et usurpés par les particuliers, sans quoi on ne louerait pas tant Varron de sa grande érudition pour avoir, dans ses recherches sur les édifices sacrés, exhumé tant de monuments inconnus. C’est qu’en effet on se servait alors de ce moyen plutôt pour procurer aux dieux une excuse spécieuse qu’à la peste un remède efficace.
\subsection[{Chapitre XVIII}]{Chapitre XVIII}

\begin{argument}\noindent Des malheurs arrivés aux Romains pendant la première guerre punique sans qu’ils aient pu obtenir l’assistance des dieux.
\end{argument}

\noindent Et durant les guerres puniques, lorsque la victoire demeura si longtemps en balance, dans cette lutte où deux peuples belliqueux déployaient toute leur énergie, combien de petits États détruits, combien de villes dévastées, de provinces mises au pillage, d’armées défaites, de flottes submergées, de sang répandu ! Si nous voulions raconter ou seule-nient rappeler tous ces désastres, nous referions l’histoire de Rome. Ce fut alors que les esprits effrayés eurent recours à des remèdes vains et ridicules. Sur la foi des livres sibyllins, on recommença les jeux séculaires, dont l’usage s’était perdu en des temps plus heureux. Les pontifes rétablirent aussi les jeux consacrés aux dieux infernaux, que la prospérité avait également fait négliger. Aussi bien je crois qu’en ce temps-là la joie devait être grande aux enfers, d’y voir arriver tant demonde, et il faut convenir que les guerres furieuses et les sanglantes animosités des hommes fournissaient alors aux démons de beaux spectacles et de riches festins. Mais ce qu’il y eut de plus déplorable dans cette première guerre punique, ce fut cette défaite des Romains dont nous avons parlé dans les deux livres précédents et où fut pris Régulus ; grand homme auquel il ne manqua, pour mettre fin à la guerre, après avoir vaincu les Carthaginois, que de résister à un désir immodéré de gloire, qui lui fit imposer des conditions trop dures à un peuple déjà épuisé. Si la captivité imprévue de cet homme héroïque, si l’indignité de sa servitude, si sa fidélité à garder son serment, si sa mort cruelle et inhumaine ne forcent point les dieux à rougir, il faut dire qu’ils sont d’airain comme leurs statues et n’ont point de sang dans les veines.\par
Au reste, durant ce temps, les calamités ne manquèrent pas à Rome au dedans de ses murailles. Un débordement extraordinaire du Tibre ruina presque toutes les parties basses de la ville ; plusieurs maisons furent renversées tout d’abord par la violence du fleuve, et les autres tombèrent ensuite à cause du long séjour des eaux. Ce déluge fut suivi d’un incendie plus terrible encore ; le feu, qui commença par les plus hauts édifices du Forum, n’épargna même pas son propre sanctuaire, le temple de Vesta, où des vierges choisies pour cet honneur, ou plutôt pour ce supplice, étaient chargées d’alimenter sa vie perpétuellement. Mais alors il ne se contentait pas de vivre, il sévissait, et les vestales épouvantées ne pouvaient sauver de l’embrasement cette divinité fatale qui avait déjà fait périr trois villes où elle était adorée. Alors le pontife Métellus, sans s’inquiéter de son propre salut, se jeta à travers les flammes et parvint à en tirer l’idole, étant lui-même à demi brûlé, car le feu ne sut pas le reconnaître. Étrange divinité, qui n’a seulement pas la force de s’enfuir, de sorte qu’un homme se montre plus capable de courir au secours d’une déesse que la déesse ne l’est d’aller au sien. Aussi bien si ces dieux ne savaient pas se défendre eux-mêmes du feu, comment en auraient-ils garanti la ville placée sous leur protection ? et en effet il parut bien qu’ils n’y pouvaient rien du tout. Nous ne parlerions pas ainsi à nos adversaires, s’ils disaient que leurs idoles sont les symboles des bienséternels et non les gages des biens terrestres, et qu’ainsi, quand ces symboles viennent à périr, comme toutes les choses visibles et corporelles, l’objet du culte subsiste et le dommage matériel peut toujours être réparé ; mais, par un aveuglement déplorable, on s’imagine que des idoles passagères peuvent assurer à une ville une félicité éternelle, et quand nous prouvons à nos adversaires que le maintien même des idoles n’a pu les garantir d’aucune calamité, ils rougissent de confesser une erreur qu’ils sont incapables de soutenir.
\subsection[{Chapitre XIX}]{Chapitre XIX}

\begin{argument}\noindent État déplorable de la république romaine pendant la seconde guerre punique, où s’épuisèrent les forces des deux peuples ennemis.
\end{argument}

\noindent Quant à la seconde guerre punique, il serait trop long de rapporter tous les désastres des deux peuples dont la lutte se développait sur de si vastes espaces, puisque, de l’aveu même de ceux qui n’ont pas tant entrepris de décrire les guerres de Rome que de les célébrer, le peuple à qui resta l’avantage parut moins vainqueur que vaincu. Quand Annibal, sorti d’Espagne, se fut jeté sur l’Italie comme un torrent impétueux, après avoir passé les Pyrénées, traversé les Gaules, franchi les Alpes et toujours accru ses forces dans une si longue marche en saccageant ou subjuguant tout, combien la guerre devint sanglante ! que de combats, d’armées romaines vaincues, de villes prises, forcées ou détachées du parti ennemi ! Que dirai-je de cette journée de Cannes où la rage d’Annibal, tout cruel qu’il était, fut tellement assouvie, qu’il ordonna la fin du carnage ? et de ces trois boisseaux d’anneaux d’or qu’il envoya aux Carthaginois après la bataille, pour faire entendre qu’il y était mort tant de chevaliers romains, que la perte était plus facile à mesurer qu’à compter, et pour laisser à penser quelle épouvantable boucherie on avait dû faire de combattants sans anneaux d’or ? Aussi le manque de soldats contraignit les Romains à promettre l’impunité aux criminels et à donner la liberté aux esclaves, moins pour recruter leur armée, que pour former une armée nouvelle avec ces soldats infâmes. Ce n’est pas tout : les armes mêmes manquèrent à ces esclaves, ou, pour les appeler d’un nom moins flétrissant, à ces nouveaux affranchis enrôlés pour la défense de la république. On en prit donc dans les temples, comme si les Romains eussent dit à leurs dieux : Quittez ces armes que vous avez si longtemps portées en vain, pour voir si nos esclaves n’en feront point un meilleur usage. — Cependant le trésor public manquant d’argent pour payer les troupes, les particuliers y contribuèrent de leurs propres deniers avec tant de zèle, qu’à l’exception de l’anneau et de la bulle, misérables marques de leur dignité, les sénateurs, et à plus forte raison les autres ordres et les tribuns, ne se réservèrent rien de précieux. Quels reproches les païens ne nous feraient-ils pas, s’ils venaient à être réduits à cette indigence, eux qui ne nous les épargnent pas dans ce temps où l’on donne plus aux comédiens pour un vain plaisir qu’on ne donnait autrefois aux légions pour tirer la république d’un péril extrême ?
\subsection[{Chapitre XX}]{Chapitre XX}

\begin{argument}\noindent De la ruine de Sagonte, qui périt pour n’avoir point voulu quitter l’alliance des Romains, sans que les dieux des romains vinssent à son secours.
\end{argument}

\noindent Mais de tous les malheurs qui arrivèrent pendant cette seconde guerre punique, il n’y eut rien de plus digne de compassion que la prise de Sagonte Cette ville d’Espagne, si attachée au peuple romain, fut en effet détruite pour lui être demeurée trop fidèle. Annibal, après avoir rompu la paix, uniquement occupé de trouver des occasions de pousser les Romains à la guerre, vint assiéger Sagonte avec une puissante armée. Dès que la nouvelle en parvint à Rome, on envoya des ambassadeurs à Annibal pour l’obliger à lever le siège, et sur son refus, ceux-ci passèrent à Carthage, où ils se plaignirent de cette infraction aux traités ; mais ils s’en retournèrent sans avoir rien pu obtenir. Cependant cette ville opulente, si chère à toute la contrée et à la république romaine, fut ruinée par les Carthaginois après huit ou neuf mois de siège. On n’en saurait lire le récit sans horreur, encore moins l’écrire ; j’y insisterai pourtant en quelques mots, parce que cela importe à mon sujet. D’abord elle fut tellement désolée parla famine que, suivant quelques historiens, les habitants furent obligés de se repaître de cadavres humains ; ensuite, accablés de toutes sortes de misères et ne voulant pas tomber entre les mains d’Annibal, ils dressèrent un grand bûcher où ils s’entrégorgèrent, eux et leurs enfants, au milieu des flammes. Je demande si les dieux, ces débauchés, ces gourmands, avides à humer le parfum des sacrifices, et qui ne savent que tromper les hommes par leurs oracles ambigus, ne devaient pas faire quelque chose en faveur d’une ville si dévouée aux Romains, et ne pas souffrir qu’elle pérît pour leur avoir gardé une inviolable fidélité, d’autant plus qu’ils avaient été les médiateurs de l’alliance qui unissait les deux cités. Et pourtant Sagonte, fidèle à la parole qu’elle avait donnée en présence des dieux, fut assiégée, opprimée, saccagée par un perfide, pour n’avoir pas voulu se rendre coupable de parjure. S’il est vrai que ces dieux épouvantèrent plus tard Annibal par des foudres et des tempêtes, quand il était sous les murs de Rome, d’où ils le forcèrent à se retirer, que n’en faisaient-ils autant pour Sagonte ? J’ose dire qu’il y aurait eu pour eux plus d’honneur à se déclarer en faveur des alliés de Rome, attaqués à cause de leur fidélité et dénués de tout secours, qu’à secourir Rome elle-même, qui combattait pour son propre intérêt et était en état de tenir tête à Annibal. S’ils étaient donc véritablement les protecteurs de la félicité et de la gloire de Rome, ils lui auraient épargné la honte ineffaçable de la ruine de Sagonte. Et maintenant, n’est-ce pas une folie de croire qu’on leur doit d’avoir sauvé Rome des mains d’Annibal victorieux, quand ils n’ont pas su garantir de ses coups une ville si fidèle aux Romains ? Si le peuple de Sagonte eût été chrétien, s’il eût souffert pour la foi de l’Évangile, sans toutefois se tuer et se brûler lui-même, il eût souffert du moins avec cette espérance que donne la foi et dont l’objet n’est pas une félicité passagère, mais une éternité bienheureuse ; au lieu que ces dieux que l’on doit, dit-on, servir et honorer afin de s’assurer la jouissance des biens périssables de cette vie, que pourront alléguer leurs défenseurs pour les excuser de la ruine de Sagonte ? à moins qu’ils ne reproduisent les arguments déjà invoqués à l’occasion de la mort de Régulus ; il n’y a d’autre différence, en effet, sinon que Régulus n’est qu’un seul homme, et que Sagonte est une ville entière ; mais ni Régulus, ni les Sagontins ne sont morts que pour avoir gardé leur foi. C’est pour le même motif que l’un voulut retourner aux ennemis et que les autres refusèrent de s’y joindre. Est-ce donc que la fidélité irrite les dieux, ou que l’on peut avoir les dieux favorables et ne pas laisser de périr, soit villes, soit particuliers ? Que nos adversaires choisissent. Si ces dieux s’offensent contre ceux qui gardent la foi jurée, qu’ils cherchent des perfides qui les adorent ; mais si avec toute leur faveur, villes et particuliers peuvent périr après avoir souffert une infinité de maux, alors certes c’est en vain qu’on les adore en vue de la félicité terrestre. Que ceux, donc qui se croient malheureux parce qu’il leur est interdit d’adorer de pareilles divinités, cessent de se courroucer contre nous, puisque enfin ils pourraient avoir leurs dieux présents, et même favorables, et ne pas laisser non seulement d’être malheureux, mais de souffrir les plus horribles tortures comme Régulus et les Sagontins.
\subsection[{Chapitre XXI}]{Chapitre XXI}

\begin{argument}\noindent De l’ingratitude de Rome envers Scipion, son libérateur, et de ses mœurs à l’époque réputée par Salluste la plus vertueuse.
\end{argument}

\noindent J’abrège afin de ne pas excéder les bornes que je me suis prescrites, et je viens au temps qui s’est écoulé entre la seconde et la dernière guerre contre Carthage, et où Salluste prétend que les bonnes mœurs et la concorde florissaient parmi les Romains. Or, en ces jours de vertu et d’harmonie, le grand Scipion, le libérateur de Rome et de l’Italie, qui avait achevé la seconde guerre punique, si funeste et si dangereuse, vaincu Annibal, dompté Carthage, et dont toute la vie avait été consacrée au service des dieux, Scipion se vit obligé, après le triomphe le plus éclatant, de céder aux accusations de ses ennemis, et de quitter sa patrie, qu’il avait sauvée et affranchie par sa valeur, pour passer le reste de ses jours dans la petite ville de Literne, si indifférent à son rappel qu’on dit qu’il ne voulut pas même qu’après sa mort on l’ensevelît dans cette ingrate cité. Ce fut dans ce même temps que le proconsul Manlius, après avoir subjugué les Galates, apporta à Rome les délices de l’Asie, pires pour elle que les ennemis les plus redoutables.\par
On y vit alors pour la première fois des lits d’airain et de riches tapis ; pour la première fois des chanteuses parurent dans les festins, et la porte fut ouverte à toutes sortes de dissolutions. Mais je passe tout cela sous silence, ayant entrepris de parler des maux que les hommes souffrent malgré eux, et non de ceux qu’ils font avec plaisir. C’est pourquoi il convenait beaucoup plus à mon sujet d’insister sur l’exemple de Scipion, qui mourut victime de la rage de ses ennemis, loin de sa patrie dont il avait été le libérateur, et abandonné de ces dieux qu’on ne sert que pour la félicité de la vie présente, lui qui avait protégé leurs temples contre la fureur d’Annibal. Mais comme Salluste assure que c’était le temps où florissaient les bonnes mœurs, j’ai cru devoir toucher un mot de l’invasion des délices de l’Asie, pour montrer que le témoignage de cet historien n’est vrai que par comparaison avec les autres époques où les mœurs furent beaucoup plus dépravées et les factions plus redoutables. Vers ce moment, en effet, entre la seconde et la troisième guerre punique, fut publiée la loi Voconia, qui défendait d’instituer pour héritière une femme, pas même une fille unique. Or, je ne vois pas qu’il se puisse rien imaginer de plus injuste que cette loi. Il est vrai que dans l’intervalle des deux guerres, les malheurs de la république furent un peu plus supportables ; car si Rome était occupée de guerres au dehors, elle avait pour se consoler, outre ses victoires, la tranquillité intérieure dont elle n’avait pas joui depuis longtemps. Mais, après la dernière guerre punique, la rivale de l’empire ayant été ruinée de fond en comble par un autre Scipion, qui en prit le surnom d’Africain, Rome, qui n’avait plus d’ennemis à craindre, fut tellement corrompue par la prospérité, et cette corruption fut suivie de calamités si désastreuses, que l’on peut dire que Carthage lui fit plus de mal par sa chute qu’elle ne lui en avait fait par ses armes au temps de sa plus grande puissance. Je ne dirai rien des revers et des malheurs sans nombre qui accablèrent les Romains depuis cette époque jusqu’à Auguste, qui leur ôta la liberté, mais, comme ils le reconnaissent eux-mêmes, une liberté malade et languissante, querelleuse et pleine de périls, et qui faisant tout plier sous une autorité toute royale, communiqua une vie nouvelle à cet empire vieillissant. Je ne dirai rien non plus du traité ignominieux fait avec Numance ; les poulets sacrés, dit-on, s’étaient envolés de leurs cages, ce qui était de fort mauvais augure pour le consul Mancinus ; comme si, pendant cette longue suite d’années où Numance tint en échec les armées romaines et devint la terreur de la république, les autres généraux ne l’eussent attaquée que sous des auspices défavorables !
\subsection[{Chapitre XXII}]{Chapitre XXII}

\begin{argument}\noindent De l’ordre donné par Mithridate de tuer tous les citoyens romains qu’on trouverait en Asie.
\end{argument}

\noindent Je passe, dis-je, tout cela sous silence ; mais puis-je taire l’ordre donné par Mithridate, roi de Pont, de mettre à mort le même jour tous les citoyens romains qui se trouveraient en Asie, où un si grand nombre séjournaient pour leurs affaires privées, ce qui fut exécuté ? Quel épouvantable spectacle ! Partout où se rencontre un Romain, à la campagne, par les chemins, à la ville, dans les maisons, dans les rues, sur les places publiques, au lit, à table, partout, à l’instant, il est impitoyablement massacré ! Quelles furent les plaintes des mourants, les larmes des spectateurs ou peut-être même des bourreaux ! et quelle cruelle nécessité imposée aux hôtes de ces infortunés, non seulement de voir commettre chez eux tant d’assassinats, mais encore d’en être eux-mêmes les exécuteurs, de quitter brusquement le sourire de la politesse et de la bienveillance pour exercer au milieu de la paix le terrible devoir de la guerre et recevoir intérieurement le contrecoup des blessures mortelles qu’ils portaient à leurs victimes ! Tous ces Romains avaient-ils donc méprisé les augures ? n’avaient-ils pas des dieux publics et des dieux domestiques à consulter avant que d’entreprendre un voyage si funeste ? S’ils ne l’ont pas fait, nos adversaires n’ont pas sujet de se plaindre de la religion chrétienne, puisque longtemps avant elle les Romains méprisaient ces vaines prédictions et s’ils l’ont fait, quel profit en ont-ils retiré alors que les lois, du moins les lois humaines, autorisaient ces superstitions ?
\subsection[{Chapitre XXIII}]{Chapitre XXIII}

\begin{argument}\noindent Des maux intérieurs qui affligèrent la république romaine à la suite d’une rage soudaine dont furent atteints tous les animaux domestiques.
\end{argument}

\noindent Rapportons maintenant le plus succinctement possible des maux d’autant plus profonds qu’ils furent plus intérieurs, je veux parler des discordes qu’on a tort d’appeler civiles, puisqu’elles sont mortelles pour la cité. Ce n’étaient plus des séditions, mais de véritables guerres où l’on ne s’amusait pas à répondre à un discours par un autre, mais où l’on repoussait le fer par le fer. Guerres civiles, guerres des alliés, guerres des esclaves, que de sang romain répandu parmi tant de combats ! quelle désolation dans l’Italie, chaque jour dépeuplée ! On dit qu’avant la guerre des alliés tous les animaux domestiques, chiens, chevaux, ânes, bœufs, devinrent tout à coup tellement farouches qu’ils sortirent de leurs étables et s’enfuirent çà et là, sans que personne pût les approcher autrement qu’au risque de la vie. Quel mal ne présageait pas un tel prodige, qui était déjà un grand mal, même s’il n’était pas un présage ! Supposez qu’un pareil accident arrivât de nos jours ; vous verriez les païens plus enragés contre nous que ne l’étaient contre eux leurs animaux.
\subsection[{Chapitre XXIV}]{Chapitre XXIV}

\begin{argument}\noindent De la discorde civile qu’alluma l’esprit séditieux des Gracques.
\end{argument}

\noindent Le signal des guerres civiles fut donné par les séditions qu’excitèrent les Gracques à l’occasion des lois agraires. Ces lois avaient pour objet de partager au peuple les terres que la noblesse possédait injustement ; mais vouloir extirper une injustice si ancienne, c’était une entreprise non seulement périlleuse, mais encore, comme l’événement l’a prouvé, des plus pernicieuses pour la république. Quelles funérailles suivirent la mort violente du premier des Gracques, et, peu après, celle du second ! Au mépris des lois et de la hiérarchie des pouvoirs, c’étaient la violence et les armes qui frappaient tour à tour les plébéiens et les patriciens. On dit qu’après la mort du second des Gracques, le consul Lucius Opimus,qui avait soulevé la ville contre lui et entassé les cadavres autour du tribun immolé, poursuivit les restes de son parti selon les formes de la justice et fit condamner à mort jusqu’à trois mille hommes d’où l’on peut juger combien de victimes avaient succombé dans la chaleur de la sédition, puisqu’un si grand nombre fut atteint par l’instruction régulière du magistrat. Le meurtrier de Caïus Gracchus vendit sa tête au consul son pesant d’or ; c’était le prix fixé avant ce massacre, où périt aussi le consulaire Marcus Fulvius avec ses enfants.
\subsection[{Chapitre XXV}]{Chapitre XXV}

\begin{argument}\noindent Du temple élevé à la Concorde par décret du sénat, dans le lieu même signalé par la sédition et le carnage.
\end{argument}

\noindent Ce fut assurément une noble pensée du sénat que le décret qui ordonna l’érection d’untemple à la Concorde dans le lieu même où une sédition sanglante avait fait périr tant de citoyens de toute condition, afin que ce monument du supplice des Gracques parlât auxyeux et à la mémoire des orateurs. Et cependant n’était-ce pas se moquer des dieux que de construire un temple à une déesse qui, si elle eût été présente à Rome, l’eût empêchée de se déchirer et de périr par les dissensions ? à moins qu’on ne dise que la Concorde, coupable de ces tumultes pour avoir abandonné le cœur des citoyens, méritait bien d’être enfermée dans ce temple comme dans une prison. Si l’on voulait faire quelque chose qui eût du rapport à ce qui s’était passé, pourquoi ne bâtissait-ou pas plutôt un temple à la Discorde ? Y a-t-il des raisons pour que la Concorde soit une déesse, et la Discorde non ? celle-là bonne et celle-ci mauvaise, selon la distinction de Labéon, suggérée sans doute par la vue du temple que les Romains avaient érigé à la Fièvre aussi bien qu’à la Santé. Pour être conséquents, ils devaient en dédier un non seulement à la Concorde, mais aussi à la Discorde. Ils s’exposaient à de trop grands périls en négligeant d’apaiser la colère d’une si méchante déesse, et ils ne se souvenaient plus que son indignation avait été le principe de la ruine de Troie. Ce fut elle, en effet, qui, pour se venger de ce qu’on ne l’avait point invitée avec les autres dieux aux noces de Pélée et de Thétis, mit la division entre les trois déesses, en jetant dans l’assemblée la fameuse pomme d’or, d’où prit naissance le différend de ces divinités, la victoire de Vénus, le ravissement d’Hélène et enfin la destruction de Troie. C’est pourquoi si elle s’était offensée de ce que Rome n’avait pas daigné lui donner un temple comme elle avait fait à tant d’autres, et si ce fut pour cela qu’elle y excita tant de troubles et de désordres, son indignation dut encore s’accroître quand elle vit que dans le lieu même où le massacre était arrivé, c’est-à-dire dans le lieu où elle avait montré de ses œuvres, on avait construit un temple à son ennemie. Les savants et les sages s’irritent contre nous quand nous tournons en ridicule toutes ces superstitions ; et toutefois, tant qu’ils resteront les adorateurs des mauvaises comme des bonnes divinités, ils n’auront rien à répondre à notre dilemme sur la Concorde et la Discorde. De deux choses l’une, en effet : ou ils ont négligé le culte de ces deux déesses, et leur ont préféré la Fièvre et la Guerre, qui ont eu des temples à Rome de toute antiquité ; ou ils les ont honorées, et alors je demande pourquoi ils ont été abandonnés par la Concorde et poussés par la Discorde jusqu’à la fureur des guerres civiles.
\subsection[{Chapitre XXVI}]{Chapitre XXVI}

\begin{argument}\noindent Des guerres qui suivirent la construction du temple de la Concorde.
\end{argument}

\noindent Ils crurent donc, en mettant devant les yeux des orateurs un monument de la fin tragique des Gracques, avoir an merveilleux obstacle contre les séditions ; mais les événements qui suivirent, plus déplorables encore, firent paraître l’inutilité de cet expédient. À partir de cette époque, en effet, les orateurs, loin de songer à éviter l’exemple des Gracques, s’étudièrent à les surpasser. C’est ainsi que Saturninus, tribun du peuple, le préteur Caïus Servilius, et, quelques années après, Marcus Drusus, excitèrent d’horribles séditions, d’où naquirent les guerres sociales qui désolèrent l’Italie et la réduisirent à un état déplorable. Puis vint la guerre des esclaves, suivie elle-même des guerres civiles pendant lesquelles il se livra tant de combats et qui coûtèrent tant de sang. On eût dit que tous ces peuples d’Italie, dont se composait la principale force de l’empire romain, étaient des barbares à dompter. Rappellerai-je que soixante-dix gladiateurs commencèrent la guerre des esclaves, et que cette poignée d’hommes, croissant en nombre et en fureur, en vint à triompher des généraux du peuple romain ? Comment citer toutes les villes qu’ils ont ruinées, toutes les contrées qu’ils ont dévastées ? À peine les historiens suffisent-ils à décrire toutes ces calamités. Et cette guerre ne fut pas la seule faite par les esclaves ; ils avaient auparavant ravagé la Macédoine, la Sicile et toute la côte. Enfin, qui pourrait raconter toutes les atrocités de ces pirates, qui, après avoir commencé par des brigandages, finirent par soutenir contre Rome des guerres redoutables ?
\subsection[{Chapitre XXVII}]{Chapitre XXVII}

\begin{argument}\noindent De la guerre civile entre Marius et Sylla.
\end{argument}

\noindent Marius, encore tout sanglant du massacre de ses concitoyens, ayant été vaincu à son tour et obligé de s’enfuir, Rome commençait un peu à respirer, quand Cinna et lui y rentrèrent plus puissants que jamais. « Ce fut alors », pour me servir des expressions de Cicéron, « que l’on vit, par le massacre des plus illustres citoyens, s’éteindre les flambeaux de la république. Sylla vengea depuis une victoire si cruelle ; mais à combien de citoyens il en coûta la vie, et que de pertes sensibles pour l’État ! » En effet, la vengeance de Sylla fut plus funeste à Rome que n’eût été l’impunité, et comme dit Lucain :\par
 {\itshape « Le remède passa toute mesure, et l’on porta la main sur des parties malades où il ne fallait pas toucher. Les coupables périrent, mais quand il ne pouvait survivre que des coupables. Alors la haine se donna carrière, et la vengeance, libre du joug des lois, précipita ses fureurs. »} \par
Dans cette lutte de Marius et de Sylla, outre ceux qui furent tués sur le champ de bataille, tous les quartiers de la ville, les places, les marchés, les théâtres, les temples même étaient remplis de cadavres, à ce point qu’on n’aurait pu dire si c’était avant ou après la victoire qu’il était tombé plus de victimes.\par
De retour de son exil, Marius eut à peine rétabli sa domination, qu’on vit, sans parler d’innombrables assassinats qui se commirent de tous côtés, la tête du consul Octavius exposée sur la tribune aux harangues, César et Fimbria tués dans leurs maisons, les deux Crassus, le père et le fils, égorgés sous les yeux l’un de l’autre, Bébius et Numitorius traînés par les rues et mis en pièces, Catulus forcé de recourir au poison pour se sauver des mains de ses ennemis ; Mérula, flamine de Jupiter, s’ouvrant les veines et faisant au dieu une libation de son propre sang ; enfin on massacrait sous les yeux de Marias tous ceux à qui il ne donnait pas la main quand ils le saluaient.
\subsection[{Chapitre XXVIII}]{Chapitre XXVIII}

\begin{argument}\noindent Comment Sylla victorieux tira vengeance des cruautés de Marius.
\end{argument}

\noindent Sylla, qui vint tirer vengeance de ces cruautés au prix de tant de sang, mit fin à la guerre ; mais comme sa victoire n’avait pas détruit les inimitiés, elle rendit la paix encore plus meurtrière. À toutes les atrocités du premier Marius, son fils Marins le Jeune et Carbon en ajoutèrent de nouvelles. Instruits de l’approche de Sylla et désespérant de remporter la victoire, et même de sauver leurs têtes, ils remplirent Rome de massacres où leurs amis n’étaient pas plus épargnés que leurs adversaires. Ce ne fut pas assez pour eux de décimer la ville ; ils assiégèrent le sénat et tirèrent du palais, comme d’une prison, un grand nombre de sénateurs qu’ils firent égorger en leur présence. Le pontife Mucius Scévola fut tué au pied de l’autel de Vesta, où il s’était réfugié comme dans un asile inviolable, et il s’en fallut de peu qu’il n’éteignît de son sang le feu sacré entretenu par les vestales. Bientôt Sylla entra victorieux à Rome, après avoir fait égorger dans une ferme publique sept mille hommes désarmés et sans défense. Ce n’était plus la guerre qui tuait, c’était la paix ; on ne se battait plus contre ses ennemis, un mot suffisait pour les exterminer. Dans la ville, les partisans de Sylla massacrèrent qui bon leur sembla ; les morts ne se comptaient plus, jusqu’à ce qu’enfin on conseilla à Sylla de laisser vivre quelques citoyens, afin que les vainqueurs eussent à qui commander. Alors s’arrêta cette effroyable liberté du meurtre, et onaccueillit avec reconnaissance la table de proscription où étaient portés deux mille noms de sénateurs et de chevaliers. Ce nombre, si attristant qu’il pût être, avait au moins cela de consolant qu’il mettait fin au carnage universel, et on s’affligeait moins de la perte de tant de proscrits qu’on ne se réjouissait de ce que le reste des citoyens n’avait rien à craindre. Mais malgré cette cruelle sécurité on ne laissa pas de gémir des divers genres de supplices qu’une férocité ingénieuse faisait souffrir à quelques-unes des victimes dévouées et à la mort. Il y en eut un que l’on déchira à belles mains, et on vit des hommes plus cruels pour un homme vivant que les bêtes farouches ne le sont pour un cadavre. On arracha les yeux à un autre et on lui coupa tous les membres par morceaux, puis on le laissa vivre ou plutôt mourir lentement au milieu de tortures effroyables. On mit des villes célèbres à l’encan, comme on aurait fait d’une ferme ; il y en eut même une dont on condamna à mort tous les habitants, comme s’il se fût agi d’un seul criminel. Toutes ces horreurs se passèrent en pleine paix, non pour hâter une victoire, mais pour n’en pas perdre le fruit. Il y eut entre la paix et la guerre une lutte de cruauté, et ce fut la paix qui l’emporta ; car la guerre n’attaquait que des gens armés, au lieu que la paix immolait des hommes sans défense. La guerre laissait à l’homme attaqué la faculté de rendre blessure pour blessure ; la paix ne laissait au vaincu, à la place du droit de vivre, que la nécessité de mourir sans résistance.
\subsection[{Chapitre XXIX}]{Chapitre XXIX}

\begin{argument}\noindent Rome eut moins à souffrir des invasions des Gaulois et des Goths que des guerres civiles.
\end{argument}

\noindent Quel acte cruel des nations barbares et étrangères peut être comparé à ces victoires de citoyens sur des citoyens, et Rome a-t-elle jamais rien vu de plus funeste, de plus hideux, de plus déplorable ? Y a-t-il à mettre en balance l’ancienne irruption des Gaulois, ou l’invasion récente des Goths, avec ces atrocités inouïes exercées par Marius, par Sylla, par tant d’autres chefs renommés, sur des hommes qui formaient avec eux les membres d’un même corps ? Il est vrai que les Gaulois égorgèrent tout ce qu’ils trouvèrent de sénateurs dans Rome, mais au moins permirent-ils à ceux qui s’étaient sauvés dans le Capitole, et qu’ils pouvaient faire périr par un long siège, de racheter leur vie à prix d’argent. Quant aux Goths, ils épargnèrent un si grand nombre de sénateurs, qu’on ne saurait affirmer s’ils en tuèrent en effet quelques-uns. Mais Sylla, du vivant même de Marius, entra dans le Capitole, qu’avaient respecté les Gaulois, et ce fut de là qu’il dicta en vainqueur ses arrêts de mort et de confiscation, qu’il fit autoriser par un sénatus-consulte. Et quand Marius, qui avait pris la fuite, rentra dans Rome en l’absence de Sylla, plus féroce et plus sanguinaire que jamais, y eut-il rien de sacré qui échappât à sa fureur, puisqu’il n’épargna pas même Mucius Scévola, citoyen, sénateur et pontife, qui embrassait l’autel où on croyait les destins de Rome attachés ? Enfin, cette dernière proscription de Sylla, pour ne point parler d’une infinité d’autres massacres, ne fit-elle point périr plus de sénateurs que les Goths n’en ont pu même dépouiller ?
\subsection[{Chapitre XXX}]{Chapitre XXX}

\begin{argument}\noindent De l’enchaînement des guerres nombreuses et cruelles qui précédèrent l’avènement de Jésus-Christ.
\end{argument}

\noindent Quelle est donc l’effronterie des païens, quelle audace à eux, quelle déraison, ou plutôt quelle démence, de ne pas imputer leurs anciennes calamités à leurs dieux et d’imputer les nouvelles à Jésus-Christ ! Ces guerres civiles, plus cruelles, de l’aveu de leurs propres historiens, que les guerres étrangères, et qui n’ont pas seulement agité, mais détruit la république, sont arrivées longtemps avant Jésus-Christ, et par un enchaînement de crimes, se rattachent de Marius et Sylla à Sertorius et Catilina, le premier proscrit et l’autre formé par Sylla. Vint ensuite la guerre de Lépide et de Catulus, dont l’un voulait abroger ce qu’avait fait Sylla et l’autre le maintenir ; puis la lutte de Pompée et de César, celui-là partisan de Sylla qu’il égala ou surpassa même en puissance ; celui-ci, qui ne put souffrir la grandeur de son rival et la voulut dépasser encore après l’avoir vaincu ; puis enfin, nous arrivons à ce grand César, qui fut depuis appelé Auguste, et sous l’empire duquel naquit le Christ. Or, Auguste, lui aussi, prit part à plusieurs guerres civiles où périrent beaucoup d’illustres personnages entre autres cet homme d’État si éloquent, Cicéron. Quant à Jules César, après avoir vaincu Pompée, et usé avec tant de modération de sa victoire, qu’il pardonna à ses adversaires et leur rendit leurs dignités, il fut poignardé dans le sénat par quelques patriciens, prétendus vengeurs de la liberté romaine, sous prétexte qu’il aspirait à la royauté. Après sa mort, un homme d’un caractère bien différent et tout perdu de vice, Marc-Antoine, affecta la même puissance, mais Cicéron lui résista vigoureusement, toujours au nom de ce fantôme de liberté. On vit alors s’élever cet autre César, fils adoptif de Jules, qui depuis, comme je l’ai dit, fat nommé Auguste. Cicéron le soutenait contre Antoine, espérant qu’il renverserait cet ennemi de la république et rendrait ensuite la liberté aux Romains. Chimère d’un esprit aveuglé et imprévoyant peu après, ce jeune homme, dont il avait caressé l’ambition, livra sa tête à Antoine comme un gage de réconciliation, et confisqua à son profit cette liberté de la république pour laquelle Cicéron avait fait tant de beaux discours.
\subsection[{Chapitre XXXI}]{Chapitre XXXI}

\begin{argument}\noindent Il y a de l’impudence aux Gentils à imputer les malheurs présents au christianisme et à l’interdiction du culte des dieux, puisqu’il est avéré qu’à l’époque ou florissait ce culte, ils ont eu à subir les plus horribles calamités.
\end{argument}

\noindent Qu’ils accusent donc leurs dieux de tant de maux, ces mêmes hommes qui se montrent si peu reconnaissants envers le Christ ! Certes, quand ces maux sont arrivés, la flamme des sacrifices brûlait sur l’autel des dieux ; l’encens de l’Arabie s’y mêlait au parfum des fleurs nouvelles ; les prêtres étaient entourés d’honneurs, les temples étincelaient de magnificence ; partout des victimes, des jeux, des transports prophétiques, et dans le même temps le sang des citoyens coulait partout, versé par des citoyens jusqu’aux pieds des autels. Cicéron n’essaya pas de chercher un asile dans un temple, parce qu’avant lui\par
Mucius Scévola n’y avait pas évité la mort, au lieu qu’aujourd’hui ceux qui s’emportent le plus violemment contre le christianisme ont dû la vie à des lieux consacrés au Christ, soit qu’ils aient couru s’y réfugier, soit que les barbares eux-mêmes les y aient conduits pour les sauver. Et maintenant j’ose affirmer, certain de n’être contredit par aucun esprit impartial, que si le genre humain avait reçu le christianisme avant les guerres puniques, et si les mêmes malheurs qui ont désolé l’Europe et l’Afrique avaient suivi l’établissement du culte nouveau, il n’est pas un seul de nos adversaires qui ne les lui eût imputés. Que ne diraient-ils point, surtout si la religion chrétienne eût précédé l’invasion gauloise, ou le débordement du Tibre, ou l’embrasement de Rome, ou, ce qui surpasse tous ces maux, la fureur des guerres civiles ? et tant d’autres calamités si étranges qu’on les a mises au rang des prodiges, à qui les imputeraient-ils, sinon aux chrétiens, si elles étaient arrivées au temps du christianisme ? Je ne parle point d’une foule d’autres événements qui ont causé plus de surprise que de dommage ; et en effet que des bœufs parlent, que des enfants articulent quelques mots dans le ventre de leurs mères, que l’on voie des serpents voler, des femmes devenir hommes et des poules se changer en coqs, tous ces prodiges, vrais ou faux, qui se lisent, non dans leurs poètes, mais dans leurs historiens, étonnent plus les hommes qu’ils ne leur font de mal. Mais quand il pleut de la terre, ou de la craie, ou même des pierres, je parle sans métaphore, voilà des accidents qui peuvent causer de grands dégâts.\par
Nous lisons aussi que la lave enflammée du mont Etna se répandit jusque sur le rivage de la mer, au point de briser les rochers et de fondre la poix des navires, phénomène désastreux, à coup sûr, quoique singulièrement incroyable. Une éruption toute semblable jeta, dit-on, sur la Sicile entière une telle quantité de cendres que les maisons de Catane en furent écrasées et ensevelies, ce qui toucha les Romains de pitié et les décida à faire remise aux Siciliens du tribut de cette année. Enfin, on rapporte encore que l’Afrique, déjàréduite en ce temps-là en province romaine, fut couverte d’une prodigieuse quantité de sauterelles qui, après avoir dévoré les feuilles et les fruits des arbres, vinrent se jeter dans la mer comme une épaisse et effroyable nuée ; rejetées mortes par les flots, elles infectèrent tellement l’air que, dans le seul royaume de Massinissa, la peste fit mourir quatre-vingt mille hommes, et, sur les côtes, beaucoup plus encore. À Utique, il ne resta que des soldats de trente mille qui composaient la garnison. Est-il une seule de ces calamités que les insensés qui nous attaquent, et à qui nous sommes forcés de répondre, n’imputassent au christianisme, si elles étaient arrivées du temps des chrétiens ? Et cependant ils ne les imputent point à leurs dieux, et, pour éviter des maux de beaucoup moindres que ceux du passé, ils appellent le retour de ce même culte qui n’a pas su protéger leurs ancêtres.
\section[{Livre quatrième. À qui est due la grandeur des Romains}]{Livre quatrième. \\
À qui est due la grandeur des Romains}\renewcommand{\leftmark}{Livre quatrième. \\
À qui est due la grandeur des Romains}

\subsection[{Chapitre premier}]{Chapitre premier}

\begin{argument}\noindent Récapitulation des livres précédents.
\end{argument}

\noindent En commençant cet ouvrage de la Cité de Dieu, il m’a paru à propos de répondre d’abord à ses ennemis, lesquels, épris des biens de la terre et passionnés pour des objets qui passent, attribuent à la religion chrétienne, la seule salutaire et véritable, tout ce qui traverse la jouissance de leurs plaisirs, bien que les maux dont la main de Dieu les frappe soient bien plutôt un avertissement de sa miséricorde qu’un châtiment de sa justice. Et comme il y a parmi eux une foule ignorante qui se laisse animer contre nous par l’autorité des savants et se persuade que les malheurs de notre temps sont sans exemple dans les siècles passés (illusion grossière dont les habiles ne sont pas dupes, mais qu’ils entretiennent soigneusement pour alimenter les murmures du vulgaire), j’ai dû, en conséquence, faire voir par les historiens mêmes des Gentils que les choses se sont passées tout autrement. Il a fallu aussi montrer que ces faux dieux qu’ils adoraient autrefois publiquement et qu’ils adorent encore aujourd’hui en secret, ne sont que des esprits immondes, des démons artificieux et pervers au point de se complaire dans des crimes qui, véritables ou supposés, n’en sont toujours pas moins leurs crimes, puisqu’ils en ont exigé la représentation dans leurs fêtes, afin que les hommes naturellement faibles ne pussent se défendre d’imiter ces scandales, les voyant autorisés par l’exemple des dieux. Nos preuves à cet égard ne reposent pas sur de simples conjectures, mais en partie sur ce qui s’est passé de notre temps, ayant vu nous-mêmes célébrer ces jeux, et en partie sur les livres de nos adversaires, qui ont transmis les crimes des dieux à lapostérité, non pour leur faire injure, mais dans l’intention de les honorer. Ainsi Varron, ce personnage si docte et dont l’autorité est si grande parmi les païens, traitant des choses humaines et des choses divines qu’il sépare en deux classes distinctes et distribue selon l’ordre de leur importance, Varron met les jeux scéniques au rang des choses divines, tandis qu’on ne devrait seulement pas les placer au rang des choses humaines dans une société qui ne serait composée que d’honnêtes gens. Et ce n’est pas de son autorité privée que Varron fait cette classification ; mais, étant Romain, il s’est conformé aux préjugés de son éducation et à l’usage. Maintenant, comme à la fin du livre premier, j’ai annoncé en quelques mots les questions que j’avais à résoudre, il suffit de se souvenir de ce que j’ai dit dans le second livre et dans le troisième pour savoir ce qu’il me reste à traiter.
\subsection[{Chapitre II}]{Chapitre II}

\begin{argument}\noindent Récapitulation du second et du troisième livre.
\end{argument}

\noindent J’avais donc promis de réfuter ceux qui imputent à notre religion les calamités de l’empire romain, en rappelant tous les malheurs qui ont affligé Rome et les provinces soumises à sa domination avant l’interdiction des sacrifices du paganisme, malheurs qu’ils ne manqueraient pas de nous attribuer, si notre religion eût, dès ce temps-là, éclairé le monde et aboli leur culte sacrilège. C’est ce que je crois avoir suffisamment développé au second livre et au troisième. Dans l’un j’ai considéré les maux de l’âme, les seuls maux véritables, ou du moins les plus grands de tous, et dans l’autre j’ai parlé de ces maux extérieurs et corporels, communs aux bons et aux méchants, qui sont les seuls que ces derniers appréhendent, tandis qu’ils acceptent, je ne dis pas avec indifférence, mais avec plaisir, les autres maux qui les rendent méchants. Et cependant combien peu ai-je parlé de Rome et de son empire, à ne prendre que ce qui s’est passé jusqu’au temps d’Auguste ! Que serait-ce si j’avais voulu rapporter et accumuler non seulement les dévastations, les carnages de la guerre et tous les maux que se font les hommes, mais encore ceux qui proviennent de la discorde des éléments, comme tous ces bouleversements naturels qu’Apulée indique en passant dans son livre Du monde, pour montrer que toutes les choses terrestres sont sujettes à une infinité de changements et de révolutions. Il dit en propres termes que les villes ont été englouties par d’effroyables tremblements de terre, que des déluges ont noyé des régions entières, que des continents ont été changés en îles par l’envahissement des eaux, et les mers en continent par leur retraite, que des tourbillons de vent ont renversé des villes, que le feu du ciel a consumé en Orient certaines contrées et que d’autres pays en Occident ont été ravagés par des inondations. Ainsi on a vu quelquefois le volcan de l’Etna rompre ses barrières et vomir dans la plaine des torrents de feu. Si j’avais voulu recueillir tous ces désastres et tant d’autres dont l’histoire fait foi, quand serais-je arrivé au temps où le nom du Christ est venu arrêter les pernicieuses superstitions de l’idolâtrie ? J’avais encore promis de montrer pourquoi le vrai Dieu, arbitre souverain de tous les empires, a daigné favoriser celui des Romains, et de prouver du même coup que les faux dieux, loin de contribuer en rien à la prospérité de Rome, y ont nui au contraire par leurs artifices et leurs mensonges. C’est ce dont j’ai maintenant à parler, et surtout de la grandeur de l’empire romain ; car pour ce qui est de la pernicieuse influence des démons sur les mœurs, je l’ai déjà fait ressortir très amplement dans le second livre. Je n’ai pas manqué non plus, chaque fois que j’en ai trouvé l’occasion dans le cours de ces trois premiers livres, de signaler toutes les consolations dont les méchants comme les bons, au milieu des maux de la guerre, ont été redevables au nom de Jésus-Christ, selon l’ordre de cette providence « qui fait lever son soleil et tomber sa pluie sur les justes et sur les injustes » ?
\subsection[{Chapitre III}]{Chapitre III}

\begin{argument}\noindent Si un État qui ne s’accroît que par la guerre doit être estimé sage et heureux.
\end{argument}

\noindent Voyons donc maintenant sur quel fondement les païens osent attribuer l’étendue et la durée de l’empire romain à ces dieux qu’ils prétendent avoir pieusement honorés par des scènes infâmes jouées par d’infâmes comédiens. Mais avant d’aller plus loin, je voudrais bien savoir s’ils ont le droit de se glorifier de la grandeur et de l’étendue de leur empire, avant d’avoir prouvé que ceux qui l’ont possédé ont été véritablement heureux. Nous les voyons en effet toujours tourmentés de guerres civiles ou étrangères, toujours parmi le sang et le carnage, toujours en proie aux noires pensées de la crainte ou aux sanglantes cupidités de l’ambition, de sorte que s’ils ont eu quelque joie, on peut la comparer au verre, dont tout l’éclat ne sert qu’à faire plus appréhender sa fragilité. Pour en mieux juger, ne nous laissons point surprendre à ces termes vains et pompeux de peuples, de royaumes, de provinces ; mais puisque chaque homme, considéré individuellement, est l’élément composant d’un État, si grand qu’il soit, tout comme chaque lettre est l’élément composant d’un discours, représentons-nous deux hommes dont l’un soit pauvre, ou plutôt dans une condition médiocre, et l’autre extrêmement riche, mais sans cesse agité de craintes, rongé de soucis, tourmenté de convoitises, jamais en repos, toujours dans les querelles et les dissensions, accroissant toutefois prodigieusement ses richesses au sein de tant de misères, mais augmentant du même coup ses soins et ses inquiétudes ; que d’autre part l’homme d’une condition médiocre se contente de son petit bien, qu’il soit chéri de ses parents, de ses voisins, de ses amis, qu’il jouisse d’une agréable tranquillité d’esprit, qu’il soit pieux, bienveillant, sain de corps, sobre d’habitudes, chaste de mœurs et calme dans sa conscience, je ne sais s’il y a un esprit assez fou pour hésiter à qui des deux il doit donner la préférence. Or, il est certain que la même règle qui nous sert à juger du bonheur de ces deux hommes, doit nous servir pour celui de deux familles, de deux peuples, de deux empires, et que si nous voulons mettre de côté nos préjugés et faire une juste application de cette règle, nous démêlerons aisément ce qui est la chimère du bonheur et ce qui en est la réalité. C’est pourquoi, quand la religion du vrai Dieu est établie sur la terre, quand fleurit avec le culte légitime la pureté des mœurs, alors il est avantageux que les bons règnent au loin et maintiennent longtemps leur empire, non pas tant pour leur avantage que dans l’intérêt de ceux à qui ils commandent. Quant à eux, leur piété et leur innocence, qui sont les grands dons de Dieu, suffisent pour les rendre véritablement heureux dans cette vie et dans l’autre. Mais il en va tout autrement des méchants. La puissance, loin de leur être avantageuse, leur est extrêmement nuisible, parce qu’elle ne leur sert qu’à faire plus de mal. Quant à ceux qui la subissent, ce qui leur est avant tout préjudiciable, ce n’est pas la tyrannie d’autrui, mais leur propre corruption ; car tout ce que les gens de bien souffrent de l’injuste domination de leurs maîtres n’est pas la peine de leurs fautes, mais l’épreuve de leur vertu. C’est pourquoi l’homme de bien dans les fers est libre, tandis que le méchant est esclave jusque sur le trône ; et il n’est pas esclave d’un seul homme, mais il a autant de maîtres que de vices. L’Écriture veut parler de ces maîtres, quand elle dit : « Chacun est esclave de celui qui l’a vaincu. »
\subsection[{Chapitre IV}]{Chapitre IV}\phantomsection
\label{\_chapitre4}

\begin{argument}\noindent Les empires, sans la justice, ne sont que des ramas de brigands.
\end{argument}

\noindent En effet, que sont les empires sans la justice, sinon de grandes réunions de brigands ? Aussi bien, une réunion de brigands est-elle autre chose qu’un petit empire, puisqu’elle forme une espèce de société gouvernée par un chef, liée par un contrat, et où le partage du butin se fait suivant certaines règles convenues ? Que cette troupe malfaisante vienne à augmenter en se recrutant d’hommes perdus, qu’elle s’empare de places pour y fixer sa domination, qu’elle prenne des villes, qu’elle subjugue des peuples, la voilà qui reçoit le nom de royaume, non parce qu’elle a dépouillé sa cupidité, mais parce qu’elle a su accroître son impunité. C’est ce qu’un pirate, tombé au pouvoir d’Alexandre le Grand, sut fort bien lui dire avec beaucoup de raison et d’esprit. Le roi lui ayant demandé pourquoi il troublait ainsi la mer, il lui repartit fièrement : « Du même droit que tu troubles la terre. Mais comme je n’ai qu’un petit navire, on m’appelle pirate, et parce que tu as une grande flotte, on t’appelle conquérant. »
\subsection[{Chapitre V}]{Chapitre V}\phantomsection
\label{\_chapitre5}

\begin{argument}\noindent La puissance des gladiateurs fugitifs fut presque égale à celle des rois.
\end{argument}

\noindent En conséquence, je ne veux point examiner quelle espèce de gens ramassa Romulus pour composer sa ville ; car aussitôt que le droit de cité dont il les gratifia les eut mis à couvert des supplices qu’ils méritaient et dont la crainte pouvait les porter à des crimes nouveaux et plus grands encore, ils devinrent plus doux et plus humains. Je veux seulement rappeler ici un événement qui causa de graves difficultés à l’empire romain et le mit à deux doigts de sa perte, dans un temps où il était déjà très puissant et redoutable à tous les autres peuples. Ce fut quand un petit nombre de gladiateurs de la Campanie, désertant les jeux de l’amphithéâtre, levèrent une armée considérable sous la conduite de trois chefs et ravagèrent cruellement toute l’Italie. Qu’on nous dise par le secours de quelle divinité, d’un si obscur et si misérable brigandage ils parvinrent à une puissance capable de tenir en échec toutes les forces de l’empire ! Conclura-t-on de la courte durée de leurs victoires que les dieux ne les ont point assistés ? Comme si la vie de l’homme, quelle qu’elle soit, était jamais de longue durée ! À ce compte, les dieux n’aideraient personne à s’emparer du pouvoir, personne n’en jouissant que peu de temps, et on ne devrait point tenir pour un bienfait ce qui dans chaque homme et successivement dans tous les hommes s’évanouit comme une vapeur. Qu’importe à ceux qui ont servi les dieux sous Romulus et qui sont morts depuis longues années, qu’après eux l’empire se soit élevé au comble de la grandeur, lorsqu’ils sont réduits pour leur propre compte à défendre leur cause dans les enfers ? Qu’elle soit bonne ou mauvaise, cela ne fait rien à la question ; mais enfin, tous tant qu’ilssont, après avoir vécu sous cet empire pendant une longue suite de siècles, ils ont promptement achevé leur vie et ont passé comme un éclair ; après quoi ils ont disparu, chargés du poids de leurs actions. Que si au contraire il faut attribuer à la faveur des dieux tous les biens, si courte qu’en soit la durée, les gladiateurs dont je parle ne leur sont pas médiocrement redevables, puisque nous les voyons briser leurs fers, s’enfuir, assembler une puissante armée, et, sous la conduite et le gouvernement de leurs chefs, faire trembler l’empire romain, battre ses armées, prendre ses villes, s’emparer de tout, jouir de tout, contenter tous leurs caprices, vivre en un mot comme des princes et des rois, jusqu’au jour où ils ont été vaincus et domptés, ce qui ne s’est pas fait aisément. Mais passons à des exemples d’un ordre plus relevé.
\subsection[{Chapitre VI}]{Chapitre VI}

\begin{argument}\noindent De l’ambition du roi Ninus qui, le premier, déclara la guerre à ses voisins afin d’étendre son empire.
\end{argument}

\noindent Justin, qui a écrit en latin l’histoire de la Grèce, ou plutôt l’histoire des peuples étrangers, et abrégé Trogue-Pompée, commence ainsi son ouvrage : « Dans le principe, les peuples étaient gouvernés par des rois qui étaient redevables de cette dignité suprême, non à la faveur populaire, mais à leur vertu consacrée par l’estime des gens de bien. Il n’y avait point alors d’autres lois que la volonté du prince. Les rois songeaient plutôt à conserver leurs États qu’à les accroître, et chacun d’eux se contenait dans les bornes de son empire. Ninus fut le premier qui, poussé par l’ambition, s’écarta de cette ancienne coutume. Il porta la guerre chez ses voisins, et comme il avait affaire à des peuples encore neufs dans le métier des armes, il assujettit tout jusqu’aux frontières de la Lybie. » Et un peu après : « Ninus affermit ses grandes conquêtes par une longue possession. Après avoir vaincu ses voisins et accru ses forces par celles des peuples soumis, il fit servir ses premières victoires à en remporter de nouvelles et soumit tout l’Orient. » Quelque opinion qu’on ait sur la véracité de Justin ou de Trogue-Pompée, caril y a des historiens plus exacts qui les ont convaincus plus d’une fois d’infidélité, toujours est-il qu’on tombe d’accord que Ninus étendit beaucoup l’empire des Assyriens. Et quant à la durée de cet empire, elle excède celle de l’empire romain, puisque les chronologistes comptent douze cent quarante ans depuis la première année du règne de Ninus jusqu’au temps de la domination des Mèdes, Or, faire la guerre à ses voisins, attaquer des peuples de qui on n’a reçu aucune offense et seulement pour satisfaire son ambition, qu’est-ce autre chose que du brigandage en grand ?
\subsection[{Chapitre VII}]{Chapitre VII}

\begin{argument}\noindent S’il faut attribuer à l’assistance ou à l’abandon des dieux la prospérité ou la décadence des empires.
\end{argument}

\noindent Si l’empire d’Assyrie a eu cette grandeur et cette durée sans l’assistance des dieux, pourquoi donc attribuer aux dieux de Rome la grandeur et la durée de l’empire romain ? Quelle que soit la cause qui a fait prospérer les deux empires, elle est la même dans les deux cas. D’ailleurs si l’on prétend que l’empire d’Assyrie a prospéré par l’assistance des dieux, je demanderai : de quels dieux ? car les peuples subjugués par Ninus n’adoraient point d’autres dieux que les siens. Dira-t-on que les Assyriens avaient des dieux particuliers, plus habiles ouvriers dans l’art de bâtir et de conserver des empires ; je demanderai alors si ces dieux étaient morts quand l’empire d’Assyrie s’est écroulé ? Ou bien serait-ce que faute d’avoir été payés de leur salaire, ou sur la promesse d’une plus forte récompense, ils ont mieux aimé passer aux Mèdes, pour se tourner ensuite du côté des Perses, en faveur de Cyrus qui les appelait et leur faisait espérer une condition plus avantageuse ? En effet, ce dernier peuple, depuis la domination, vaste en étendue, mais courte en durée, d’Alexandre le Grand, a toujours conservé son ancien État, et il occupe aujourd’hui dans l’Orient une vaste étendue de pays. Or, s’il en est ainsi, ou bien les dieux sont coupables d’infidélité, puisqu’ils abandonnent leurs amis pourpasser du côté de leurs ennemis, et font ce que Camille, qui n’était qu’un homme, ne voulut pas faire, quand, après avoir vaincu les ennemis les plus redoutables de Rome, il éprouva l’ingratitude de sa patrie, et qu’au lieu d’en conserver du ressentiment, il sauva une seconde fois ses concitoyens en les délivrant des mains des Gaulois ; ou bien ces dieux ne sont pas aussi puissants qu’il conviendrait à leur divinité, puisqu’ils peuvent être vaincus par la prudence ou par la force ; ou enfin, s’il n’est pas vrai qu’ils soient vaincus par des hommes, mais par d’autres dieux, il y a donc entre ces esprits célestes des inimitiés et des luttes, suivant que chacun se range de tel ou tel parti, et alors pourquoi un État adorerait-il ses dieux propres de préférence à d’autres dieux que ceux-ci peuvent appeler comme auxiliaires ? Quoi qu’il en soit au surplus de ce passage, de cette fuite, de cette migration ou de cette défection des dieux, il est certain qu’on ne connaissait point encore Jésus-Christ quand ces monarchies ont été détruites ou transformées. Car lorsque, après une durée de douze cents ans et plus, l’empire des Assyriens s’est écroulé, si déjà la religion chrétienne eût annoncé le royaume éternel et fait interdire le culte sacrilège des faux dieux, les Assyriens n’auraient pas manqué de dire que beur empire ne succombait, après avoir duré si longtemps, que pour avoir abandonné la religion des ancêtres et embrassé celle de Jésus-Christ. Que la vanité manifeste de ces plaintes soit comme un miroir où nos adversaires pourront reconnaître l’injustice des leurs, et qu’ils rougissent de les produire, s’il leur reste encore quelque pudeur. Mais je me trompe : l’empire romain n’est pas détruit, comme l’a été celui d’Assyrie ; il n’est qu’éprouvé. Bien avant le christianisme, il a connu ces dures épreuves et il s’en est relevé. Ne désespérons pas aujourd’hui qu’il se relève encore ; car en cela qui sait la volonté de Dieu ?
\subsection[{Chapitre VIII}]{Chapitre VIII}

\begin{argument}\noindent Les Romains ne sauraient dire quels sont parmi leurs dieux ceux à qui ils croient devoir l’accroissement et la conservation de leur empire, chaque dieu en particulier étant capable tout au plus de veiller à sa fonction particulière.
\end{argument}

\noindent Mais cherchons, je vous prie, parmi cette multitude de dieux qu’adoraient les Romains, quel est celui ou quels sont ceux à qui ils se croient particulièrement redevables de la grandeur et de la conservation de leur empire ? Je ne pense pas qu’ils osent attribuer quelque part dans un si grand et si glorieux ouvrage à la déesse de Cloacina, ou à Volupia, qui tire son nom de la volupté, ou à Libentina, qui prend le sien du libertinage, ou à Vaticanus, qui préside aux vagissements des enfants, ou à Cunina, qui veille sur leur berceau. Je ne puis ici rappeler en quelques lignes tous ces noms de dieux et de déesses qui peuvent à peine tenir dans de gros volumes, où l’on attache chaque divinité à son objet particulier, suivant la fonction qui lui est propre. Par exemple, on n’a pas jugé à propos de confier à un seul dieu le soin des campagnes ; on a donné la plaine à Rusina, le sommet des montagnes à Jugatinus, la colline à Collatina, la vallée à Valbonia. On n’a même pas trouvé une divinité assez vigilante pour lui donner exclusivement la direction des moissons : on a recommandé à Séia les semences, pendant qu’elles sont encore en terre ; à Segetia, les blés quand ils sont levés ; à Tutilina, la tutelle des récoltes et des grains, quand ils sont recueillis dans les greniers. Évidemment Segetia n’a pas été jugée suffisante pour soigner les moissons depuis leur naissance jusqu’à leur maturité. Mais comme si ce n’était pas encore assez de cette foule de divinités à ces idolâtres insatiables dont l’âme corrompue dédaignait les chastes embrassements de son dieu pour se prostituer à une troupe infâme de démons, ils ont fait présider Proserpine aux germes des blés, le dieu Nodatus aux nœuds du tuyau, la déesse Volutina à l’enveloppe de l’épi ; vient ensuite Patelana, quand l’épi s’ouvre ; Hostilina, quand la barbe et l’épi sont de niveau ; Flora, quand il est en fleur ; Lacturnus, quand il est en lait ; Matuta,quand il mûrit ; Runcina, quand on le coupe. Je ne dis pas tout, car je me lasse de nommer ce qu’ils n’ont pas honte d’adorer ; mais le peu que j’en ai dit suffit pour montrer qu’il est déraisonnable d’attribuer l’origine, les progrès et la conservation de l’empire romain à des divinités tellement appliquées à leur office particulier qu’aucune tâche générale ne pouvait leur être confiée. Comment Segetia se fût-elle mêlée du gouvernement de l’empire, elle à qui il n’était pas permis d’avoir soin à la fois des arbres et des moissons ? comment Cunina eût-elle pensé à la guerre, lorsque sa charge ne s’étendait pas au-delà du berceau des enfants ? que pouvait-on attendre de Nodatus dans les combats, puisque son pouvoir, borné aux nœuds du tuyau, ne s’élevait pas jusqu’à la barbe de l’épi ? On se contente d’un portier pour garder l’entrée de sa maison, et ce portier suffit parfaitement, c’est un homme ; nos idolâtres y ont mis trois dieux : Forculus, à la porte ; Cardea, aux gonds ; Limentinus, au seuil ; en sorte que Forculus ne pouvait garder à la fois le seuil et les gonds.
\subsection[{Chapitre IX}]{Chapitre IX}

\begin{argument}\noindent Si l’on doit attribuer la grandeur et la durée de l’empire romain à Jupiter, que ses adorateurs regardent comme le premier des dieux.
\end{argument}

\noindent Mais laissons là, pour quelque temps du moins, la foule des petits dieux et cherchons quel a été le rôle de ces grandes divinités par qui Rome est devenue la dominatrice des nations. Voilà sans doute une œuvre digne de Jupiter, de ce dieu qui passe pour le roi de tous les dieux et de toutes les déesses, ainsi que le marquent et le sceptre dont il est armé, et ce Capitole construit en son honneur au sommet d’une haute colline.\par
 {\itshape « Tout est plein de Jupiter »} \par
s’écrie Virgile, et ce mot, quoique d’un poète, est cité comme exactement vrai. Suivant Varron, c’est Jupiter qu’adorent en réalité ceux qui ne veulent adorer qu’un dieu sans image auquel ils donnent un autre nom. Si cela est, d’où vient qu’on l’a respecté assez peu à Rome et ailleurs pour le représenter par une statue ? Superstition blâmée expressément par Varron, qui, tout entraîné qu’il pût être par le torrent de la coutume et par l’autorité de Rome, n’a pas laissé de dire et d’écrire qu’en élevant des statues aux dieux, on avait banni la crainte pour introduire l’erreur.
\subsection[{Chapitre X}]{Chapitre X}

\begin{argument}\noindent Des systèmes qui attachent des dieux différents aux différentes parties de l’univers.
\end{argument}

\noindent Pourquoi avoir marié Jupiter avec Junon qu’on nous donne pour être à la fois « et sa sœur et sa femme » ? C’est, dit-on, que Jupiter occupe l’éther, Junon, l’air, et que ces deux éléments, l’un supérieur, l’autre inférieur, sont étroitement unis. Mais alors, si Junon remplit la moitié du monde, elle ôte de sa place à ce dieu dont le poète a dit :\par
 {\itshape « Tout est plein de Jupiter. »} \par
Dira-t-on que les deux divinités remplissent l’une et l’autre les deux éléments et qu’elles sont ensemble chacun d’eux ? Je demanderai pourquoi l’on assigne particulièrement l’éther à Jupiter et l’air à Junon ? D’ailleurs, s’il suffit de ces deux divinités pour tout remplir, à quoi sert d’avoir donné la mer à Neptune et la terre à Pluton ? Et qui plus est, de peur de laisser ces dieux sans femmes, on a marié Neptune avec Salacie et Pluton avec Proserpine. C’est, dit-on, que Proserpine occupe la région inférieure de la terre, comme Salacie la région inférieure de la mer, et Junon la région inférieure du ciel, qui est l’air. Voilà comment les païens essaient de coudre leurs fables ; mais ils n’y parviennent pas. Car si les choses étaient comme ils le disent, leurs anciens sages admettraient trois éléments et non pas quatre, afin d’en accorder le nombre avec celui des couples divins. Or, ils distinguent positivement l’éther d’avec l’air. Quant à l’eau, supposé que l’eau supérieure diffère en quelque façon de l’eau inférieure, en haut ou en bas, c’est toujours de l’eau. De même pour la terre ; la différence du lieu peut bien changer ses qualités, mais non sa nature. Maintenant, avec ces trois ou ces quatre éléments, voilà lemonde complet : où donc sera Minerve ? quelle partie du monde aura-t-elle à remplir, quel lieu à habiter ? Car on s’est avisé de la mettre au Capitole avec Jupiter et Junon, bien qu’elle ne soit pas le fruit de leur mariage. Si on dit qu’elle habite la plus haute région de l’air et que c’est pour cela que les poètes la font naître du cerveau de Jupiter, je demande pourquoi on ne l’a pas mise à la tête des dieux, puisqu’elle est située au-dessus de Jupiter. Serait-ce qu’il n’eût pas été juste de mettre la fille au-dessus du père ? mais alors pourquoi n’a-t-on pas gardé la même justice entre Jupiter et Saturne ? C’est, dira-t-on, que Saturne a été vaincu par Jupiter. Ces deux dieux se sont donc battus ! Point du tout, s’écrie-t-on ; ce sont là des bavardages de la fable. Eh bien ! soit ; ne croyons pas à la fable et ayons meilleure opinion des dieux. Puis donc que l’on n’a pas mis Saturne au-dessus de Jupiter, que ne plaçait-on le père et le fils sur le même rang ? C’est, dit-on, que Saturne est l’image du temps. À ce compte, ceux qui adorent Saturne adorent le temps, et voilà Jupiter, le roi des dieux, qui est issu du temps. Aussi bien, quelle injure fait-on à Jupiter et à Junon de dire qu’ils sont issus du temps, s’il est vrai que Jupiter soit le ciel et Junon la terre, le ciel et la terre ayant été créés dans le temps ? C’est la doctrine qu’on trouve dans les livres de leurs savants et de leurs sages ; et Virgile s’inspire, non des fictions de la poésie, mais des systèmes des philosophes, quand il dit :\par
{\itshape « Alors le Père tout-puissant, l’Éther, descend au sein de son épouse et la réjouit par des pluies fécondes »}, \par
c’est-à-dire qu’il descend au sein de Tellus ou de la Terre ; car encore ici, on veut voir des différences et soutenir qu’autre chose est la Terre, autre chose Tellus, autre chose enfin Tellumo. Chacune de ces trois divinités a son nom, ses fonctions, son culte et ses autels. On donne encore à la terre le nom de mère des dieux, en sorte qu’il n’y a pas tant à se récrier contre les poètes, puisque voilà les livres sacrés qui font de Junon, non seulement la sœur et la femme, mais aussi la mère de Jupiter. On veut encore que la terre soit Cérès ou Vesta, quoique le plus souvent Vesta ne soit que le feu, la divinité des foyers, sans lesquels une cité ne peut exister. Et c’est pour cela que l’on consacre des vierges au service de Vesta, le feu ayant cette analogie avec les vierges, que, comme elles, il n’enfante rien. Mais tous ces vains fantômes devaient s’évanouir devant celui qui a voulu naître d’une vierge. Et qui pourrait souffrir, en effet, qu’après avoir attribué au feu une dignité si grande et une sorte de chasteté, ils ne rougissent point d’identifier quelquefois Vesta avec Vénus, afin sans doute que la virginité, si révérée dans les vestales, ne soit plus qu’un vain nom ? Si Vesta n’est autre que Vénus, comment des vierges la serviraient-elle en s’abstenant des œuvres de Vénus ? Y aurait-il par hasard deux Vénus, l’une vierge et l’autre épouse ? ou plutôt trois, la Vénus des vierges ou Vesta, la Vénus des femmes, et la Vénus des courtisanes, à qui les Phéniciens offraient le prix de la prostitution de leurs filles avant que de les marier ? Laquelle de ces trois Vénus est la femme de Vulcain ? Ce n’est pas la vierge, puisqu’elle a un mari. Loin de moi la pensée que ce soit la courtisane ! ce serait faire trop d’injure au fils de Junon, à l’émule de Minerve. C’est donc la Vénus des épouses ; mais alors que les épouses prennent garde d’imiter leur patronne dans ce qu’elle a fait avec Mars. Vous en revenez encore aux fables, me dira-t-on ; mais, en vérité, où est la justice à nos adversaires de s’emporter contre nous, quand nous parlons ainsi de leurs dieux, et de ne pas s’emporter contre eux-mêmes, quand ils assistent avec tant de plaisir au spectacle des crimes de ces dieux, et, chose incroyable si le fait n’était pas avéré, quand ils veulent faire tourner à l’honneur de la divinité ces représentations scandaleuses ?
\subsection[{Chapitre XI}]{Chapitre XI}

\begin{argument}\noindent De cette opinion des savants du paganisme que tous les dieux ne sont qu’un seul et même dieu, savoir : Jupiter.
\end{argument}

\noindent Qu’ils apportent donc autant de raisonsphysiques et autant de raisonnements qu’il leur plaira pour établir tantôt que Jupiter est l’âme du monde, laquelle pénètre et meut toute cette masse immense composée de quatre éléments ou d’un plus grand nombre ; tantôt qu’il donne une part de sa puissance à sa sœur et à ses frères ; tantôt qu’il est l’éther et qu’il embrasse Junon, qui est l’air répandu au-dessous de lui ; tantôt qu’avec l’air il est tout le ciel, et que, par ses pluies et ses semences, il féconde la terre, qui se trouve être à la fois sa femme et sa mère, car cela n’a rien de déshonnête entre dieux ; tantôt enfin, pour n’avoir pas à voyager dans toute la nature, qu’il est le dieu unique, celui dont a voulu parler, au sentiment de plusieurs, le grand poète qui a dit :\par
 {\itshape « Dieu circule à travers toutes les terres, toutes les mers, toutes les profondeurs des cieux. »} \par
Qu’ainsi, dans l’éther, il soit Jupiter, dans l’air, Junon ; dans la région supérieure de la mer, Neptune, et Salacie dans la région inférieure ; Pluton au haut de la terre, et au bas, Proserpine ; dans les foyers domestiques, Vesta ; dans les forges, Vulcain ; parmi les astres, le Soleil, la Lune et les Etoiles ; parmi les devins, Apollon ; dans le commerce, Mercure ; en tout ce qui commence, Janus, et Terminus en tout ce qui finit ; dans le temps, Saturne ; dans la guerre, Mars et Bellone ; dans les fruits de la vigne, Liber ; dans les moissons, Cérès ; dans les forêts, Diane ; dans les arts, Minerve ; enfin, qu’il soit encore cette foule de petits dieux, pour ainsi dire plébéiens : qu’il préside, sous le nom de Liber, à la vertu génératrice des hommes, et sous le nom de Libera à celle des femmes ; qu’il soit Diespiter qui conduit les accouchements à terme ; Mona, qui veille au flux menstruel ; Lucina, qu’on invoque au moment de la délivrance ; que sous le nom d’Opis il assiste les nouveau-nés et les recueille sur le sein de la terre ; qu’il leur ouvre la bouche à leurs premiers vagissements et soit alors le dieu Vaticanus ; qu’il devienne Levana pour les soulever de terre, et Cunina pour les soigner dans leur berceau ; qu’il réside en ces déesses qui prophétisent les destinées, et qu’on appelle Carmentes ; qu’il préside, sous le nom de Fortune, aux événements fortuits ; qu’il soit Rumina, quand il présente aux enfants la mamelle, par la raison que le vieux langage nomme la mamelle {\itshape ruma} ; qu’il soit Potina pour leur donner à boire, et Educa pour leur donner à manger ; qu’il doive à la peur enfantine le nom de Paventin ; à l’espérance qui vient celui de Venilia ; à la volupté celui de Volupia ; à l’action celui d’Agenoria ; aux stimulants qui poussent l’action jusqu’à l’excès, celui de Stimula ; qu’on l’appelle Strenia, parce qu’il excite le courage ; Numeria, comme enseignant à nombrer ; Camena, comme apprenant à chanter ; qu’il soit le dieu Consus, pour les conseils qu’il donne, et la déesse Sentia pour les sentiments qu’il inspire ; qu’il veille, sous le nom de Juventa, au passage de l’enfance à la jeunesse ; qu’il soit encore la Fortune Barbue, qui donne de la barbe aux adultes, et qu’on aurait dû, pour leur faire honneur, appeler du nom mâle de Fortunius, plutôt que d’un nom femelle, à moins qu’on n’eût préféré, selon l’analogie qui a tiré le dieu Nodatus des nœuds de la tige, donner à la Fortune le nom de Barbatus, puisqu’elle a les barbes dans son domaine ; que ce soit encore le même dieu qu’on appelle Jugatinus, quand il joint les époux ; Virginiensis, quand il détache du sein de la jeune mariée la ceinture virginale ; qu’il soit même, s’il n’en a point de honte, le dieu Mutunus ou Tutunus, que les Grecs appellent Priape ; en un mot, qu’il soit tout ce que j’ai dit et tout ce que je n’ai pas dit, car je n’ai pas eu dessein de tout dire ; que tous ces dieux et toutes ces déesses forment un seul et même Jupiter, ou que toutes ces divinités soient ses parties, comme le pensent quelques-uns, ou ses vertus, selon l’opinion qui fait de lui l’âme du monde ; admettons enfin celle de ces alternatives qu’on voudra, sans examiner en ce moment ce qu’il en est, je demande ce que perdraient les païens à faire un calcul plus court et plus sage, et à n’adorer qu’un seul Dieu ? Que mépriserait-on de lui, en effet, en l’adorant lui-même ? Si l’on a eu à craindre que quelques parties de sa divinité omises ou négligées ne vinssent à s’en irriter, il n’est donc pas vrai qu’il soit, comme on le prétend, la vie universelle embrassant dans son unité tous les dieux comme ses vertus, ses membres ou ses parties ; et il faut croire alors que chaque partie a sa vie propre, séparée de la vie des autres parties, puisque l’une d’elles peut s’irriter, s’apaiser, s’émouvoir sans l’autre. Dira-t-on que toutes ses parties ensemble, c’est-à-dire tout Jupiter s’offenserait, si chaque partie n’était point particulièrement adorée ? Ce serait dire une absurdité ; car aucune partie ne serait négligée, du moment qu’on servirait celui qui les comprend toutes. D’ailleurs, sans entrer ici dans des détails infinis, quand les païens soutiennent que tous les astres sont des parties de Jupiter, qu’ils ont la vie et des âmes raisonnables, et qu’à ce titre ils sont évidemment des dieux, ils ne s’aperçoivent pas qu’à ce compte il y a une infinité de dieux qu’ils n’adorent pas et à qui ils n’élèvent ni temples, ni autels, puisqu’il y a très peu d’astres qui aient un culte et des sacrifices particuliers. Si donc les dieux s’offensent quand ils ne sont pas singulièrement adorés, comment les païens ne craignaient-ils pas, pour quelques dieux qu’ils se rendent propices, d’avoir contre eux tout le reste du ciel ? Que s’ils pensent adorer toutes les étoiles en adorant Jupiter qui les embrasse toutes, ils pourraient donc aussi résumer dans le culte de Jupiter celui de tous les dieux. Ce serait le moyen de les contenter tous ; au lieu que le culte rendu à quelques-uns doit mécontenter le nombre beaucoup plus grand de ceux qu’on néglige, surtout quand ils se voient préférer un Priape étalant sa nudité obscène, eux qui resplendissent de lumière dans les hauteurs du ciel.
\subsection[{Chapitre XII}]{Chapitre XII}

\begin{argument}\noindent Du système qui fait de Dieu l’âme du monde et du monde le corps de Dieu.
\end{argument}

\noindent Que dirai-je maintenant de cette doctrine d’un Dieu partout répandu ? ne doit-elle pas soulever tout homme intelligent ou plutôt tout homme quel qu’il soit ? Certes il n’est pas besoin d’une grande sagacité, à quiconque sait se dégager de l’esprit de contention, pour reconnaître que si Dieu est l’âme du monde et le monde le corps de cette âme, si ce Dieu réside en quelque façon au sein de la nature, contenant toutes choses en soi, de telle sorte que l’âme universelle qui vivifie la masse tout entière soit la substance commune d’où naissent chacune à son tour les âmes de tous les vivants, il suit de là qu’il n’y a aucun être qui ne soit une partie de Dieu. Or, qui ne voit que les conséquences de ce système sont impies et irréligieuses au suprême degré, puisqu’il s’ensuit qu’en marchant sur un corps, je marche sur une partie de Dieu, et qu’en tuant un animal, c’est une partie de Dieu que je tue ? Mais je ne veux pas dire tout ce que peut ici suggérer la pensée, sans que le langage puisse décemment l’exprimer.
\subsection[{Chapitre XIII}]{Chapitre XIII}

\begin{argument}\noindent Du système qui n’admet comme parties de Dieu que les seuls animaux raisonnables.
\end{argument}

\noindent Dira-t-on qu’il n’y a que les animaux raisonnables, comme les hommes, par exemple, qui soient des parties de Dieu ? Mais si le monde tout entier est Dieu, je ne vois pas de quel droit on retrancherait aux bêtes leur portion de divinité. Au surplus, à quoi bon insister ? ne parlons que de l’animal raisonnable, de l’homme. Quoi de plus tristement absurde que de croire qu’en donnant le fouet à un enfant, on le donne à une partie de Dieu ? Que dire de ces parties de Dieu qui deviennent injustes, impudiques, impies, damnables enfin, si ce n’est que pour supporter de pareilles conséquences, il faut avoir perdu le sens ? Je demanderai enfin pourquoi Dieu s’irrite contre ceux qui ne l’adorent pas, puisque c’est s’irriter contre des parties de soi-même. Il ne reste donc qu’une chose à dire, c’est que chacun des dieux a sa vie propre, qu’il vit pour soi, sans faire partie d’un autre que soi, et qu’il faut adorer, sinon tous les dieux, car ils sont tellement nombreux que cela est impossible, du moins tous ceux que l’on peut connaître et servir. Ainsi, comme Jupiter est le roi des dieux, j’imagine que c’est à lui qu’on attribue la fondation et l’accroissement de l’empire romain. Car s’il n’était pas l’auteur d’un si grand ouvrage, à quel autre dieu en pourrait-on faire honneur, chacun ayant son emploi distinct qui l’occupe assez et ne lui laisse pas le temps d’entreprendre sur la charge des autres ? Il n’y a donc sans contredit que le roi des dieux qui ait pu travailler à l’accroissement et à la grandeur du roi des peuples.
\subsection[{Chapitre XIV}]{Chapitre XIV}

\begin{argument}\noindent On a tort de croire que c’est Jupiter qui veille à la prospérité des empires, attendu que la Victoire, si elle est une déesse, comme le veulent les païens, a pu seule suffire à cet emploi.
\end{argument}

\noindent Je demanderai ici tout d’abord pourquoi on n’a pas fait de l’empire un dieu. On n’en peut donner aucune raison, puisqu’on a fait de la victoire une déesse. Qu’est-il même besoin dans cette affaire de recourir à Jupiter, si la victoire a ses faveurs et ses préférences, et si elle va toujours trouver ceux qu’elle veut rendre vainqueurs ? Avec la protection de cette déesse, quand même Jupiter resterait les bras croisés ou s’occuperait d’autre chose, de quelles nations, de quels royaumes ne viendrait-on pas à bout ? On dira que les gens de bien sont arrêtés par la crainte d’entreprendre des guerres injustes qui n’ont d’autre objet que de s’agrandir aux dépens de voisins pacifiques et inoffensifs. Voilà de beaux sentiments ; si ce sont ceux de mes adversaires, je m’en réjouis et je m’en félicite.
\subsection[{Chapitre XV}]{Chapitre XV}

\begin{argument}\noindent S’il convient à un peuple vertueux de souhaiter de s’agrandir.
\end{argument}

\noindent Mais il y a dès lors une nouvelle question qui s’élève : c’est de savoir s’il convient à un peuple vertueux de se réjouir de l’agrandissement de son empire. La cause, en effet, ne saurait en être que dans l’injustice de ses voisins qui en l’attaquant sans raison lui ont donné occasion de s’agrandir justement par la guerre. Supposez, en effet, qu’entre tous les peuples voisins régnassent la justice et la paix, tout État serait de peu d’étendue, et au sein de cette médiocrité et de ce repos universels les divers États seraient dans le monde ce que sont les diverses familles dans la cité. Ainsi la guerre et les conquêtes, qui sont un bonheur pour les méchants, sont pour les bons une nécessité. Toutefois, comme le mal serait plus grand si les auteurs d’une agression injuste réussissaient à subjuguer ceux qui ont eu à la subir, on a raison de regarder la victoire des bons comme une chose heureuse ; mais cela n’empêche pas que le bonheur ne soit plus grand de vivre en paix avec un bon voisin que d’être obligé d’en subjuguer un mauvais, Car il est d’un méchant de souhaiter un sujet de haine ou de crainte pour avoir un sujet de victoire. Si donc ce n’est que par des guerres justes et légitimes que les Romains sont parvenus à posséder un si vaste empire, je leur propose une nouvelle déesse à adorer : c’est l’Injustice des nations étrangères, qui a si fort contribué à leur grandeur par le soin qu’elle a pris de leur susciter d’injustes ennemis, à qui ils pouvaient faire justement et avantageusement la guerre. Et pourquoi l’injustice ne serait-elle pas une déesse, et une déesse étrangère, puisque la Crainte, la Pâleur et la Fièvre sont au rang des divinités romaines ? C’est donc à ces deux déesses, l’Injustice étrangère et la Victoire, qu’il convient d’attribuer la grandeur des Romains, l’une pour leur avoir donné des sujets de guerres, l’autre pour les avoir heureusement terminées sans que Jupiter ait eu la peine de s’en mêler. Quelle part en effet pourrait-on lui attribuer, du moment où les faveurs qui seraient réputées venir de lui sont elles-mêmes prises pour des divinités, et sont honorées et invoquées comme telles ? Il y aurait part s’il s’appelait Empire, comme l’autre s’appelle Victoire. Or, si l’on dit que l’empire est un présent de Jupiter, pourquoi la victoire n’en serait-elle pas un aussi ? Et certes elle en serait un en effet, si au lieu d’adorer une pierre au Capitole, on reconnaissait et on adorait le Roi des rois et le Seigneur des seigneurs.
\subsection[{Chapitre XVI}]{Chapitre XVI}

\begin{argument}\noindent Pourquoi les Romains, qui attachaient une divinité à tous les objets extérieurs et à toutes les passions de l’âme, avaient placé hors de la ville le temple du Repos.
\end{argument}

\noindent Je suis fort surpris que les Romains, qui affectaient une divinité à chaque objet et presque à chaque mouvement de l’âme, et qui avaient bâti des temples dans la ville à la déesse Agenoria, qui nous fait agir, à la déesse Stimula, qui nous stimule aux actions excessives, à la déesse Murcia, qui, tout au contraire, au lieu de nous exciter, nous rend, dit Pomponius, mous et languissants, à la déesse Strenia, qui nous donne de la résolution ; je m’étonne, dis-je, qu’ils n’aient pas voulu admettre le Repos aux honneurs publics de Rome et l’aient laissé hors de la porte Colline. Était-ce un signe de leur esprit ennemi du repos, ou plutôt n’était-ce pas une preuve que les adorateurs obstinés de cette troupe de divinités ou plutôt de démons ne peuvent jouir de ce repos auquel le vrai Médecin nous convie, quand il dit : « Apprenez de moi à être doux et humbles de cœur, et vous trouverez dans vos âmes le repos. »
\subsection[{Chapitre XVII}]{Chapitre XVII}

\begin{argument}\noindent Si, en supposant Jupiter tout-puissant, la Victoire doit être tenue pour déesse.
\end{argument}

\noindent Dira-t-on que c’est Jupiter qui envoie la Victoire, et que cette déesse, étant obligée d’obéir au roi des dieux, va trouver ceux qu’il lui désigne et se range de leur côté ? Cela aurait un sens raisonnable si, au lieu de Jupiter, roi tout imaginaire, il s’agissait du véritable Roi des siècles, lequel envoie son ange (et non la Victoire, qui n’est pas un être réel) pour distribuer à qui il lui plaît le triomphe ou le revers selon les conseils quelquefois mystérieux, jamais injustes, de sa Providence. Mais si l’on voit dans la Victoire une déesse, pourquoi le Triomphe ne serait-il pas un dieu ; et lue n’en fait-on le mari de la Victoire, ou son frère, ou son fils ? En général, les idées que les païens se sont formées des dieux sont telles que si je les trouvais dans les poètes et si je voulais les discuter sérieusement, mes adversaires ne manqueraient pas de me dire que ce sont là des fictions poétiques dont il faut rire au lieu de les prendre au pied de la lettre ; et cependant ils ne riaient pas d’eux-mêmes, quand ils allaient, non pas lire dans les poètes, mais consacrer dans les temples ces traditions insensées. C’est donc à Jupiter qu’ils devaient demander toutes choses, c’est à lui seul qu’il fallait s’adresser ; car, supposez que la Victoire soit une déesse, mais une déesse soumise à un roi, de quelque côté qu’il l’eût envoyée, on ne peut admettre qu’elle eût osé lui désobéir.
\subsection[{Chapitre XVIII}]{Chapitre XVIII}

\begin{argument}\noindent Si les païens ont eu quelque raison de faire deux déesses de la Félicité et de la Fortune.
\end{argument}

\noindent N’a-t-on pas fait aussi une déesse de la Félicité ? ne lui a-t-on pas construit un temple, dressé un autel, offert des sacrifices ? Il fallait au moins s’en tenir à elle ; car où elle se trouve, quel bien peut manquer ? Mais non, la Fortune a obtenu comme elle le rang et les honneurs divins. Y a-t-il donc quelque différence entre la Fortune et la Félicité ? On dira que la fortune peut être mauvaise, tandis que la félicité, si elle était mauvaise, ne serait plus la félicité. Mais tous les dieux, de quelque sexe qu’ils soient, si toutefois ils ont un sexe, ne doivent-ils pas être réputés également bons ? C’était du moins le sentiment de Platon et des autres philosophes, aussi bien que des plus excellents législateurs. Comment donc se fait-il que la Fortune soit tantôt bonne et tantôt mauvaise ? Serait-ce par hasard que, lorsqu’elle devient mauvaise, elle cesse d’être déesse, et se change tout d’un coup en un pernicieux démon ? Combien y a-t-il donc de Fortunes ? Si vous considérez un certain nombre d’hommes fortunés, voilà l’ouvrage de la bonne fortune, et puisqu’il existe en même temps plusieurs hommes infortunés, c’est évidemment le fait de la mauvaise fortune ; or, comment une seule et même fortune serait-elle à la fois bonne et mauvaise, bonne pour ceux-ci, mauvaise pour ceux-là ? La question est de savoir si celle qui est déesse est toujours bonne. Si vous dites oui, elle se confond avec la Félicité. Pourquoi alors lui donner deux noms différents ? Mais passons sur cela, car il n’est pas fort extraordinaire qu’une même chose porte deux noms. Je me borne à demander pourquoi deux temples, deux cultes, deux autels ? Cela vient, disent-ils, de ce que la Félicité est la déesse qui se donne à ceux qui l’ont méritée, tandis que la Fortune arrive aux bons et aux méchants d’une manière {\itshape fortuite}, et c’est de là même qu’elle tire son nom. Mais comment la Fortune est-elle bonne, si elle se donne aux bons et aux méchants sans discernement ; et pourquoi la servir, si elle s’offre à tous, se jetant comme une aveugle sur le premier venu, et souvent même abandonnant ceux qui la servent pour s’attacher à ceux qui la méprisent ? Que si ceux qui l’adorent se flattent, par leurs hommages, de fixer son attention et ses faveurs, elle a donc égard aux mérites et n’arrive pas fortuitement. Mais alors que devient la définition de la Fortune, et comment peut-on dire qu’elle se nomme ainsi parce qu’elle arrive fortuitement ? De deux choses l’une : ou il est inutile de la servir, si elle est vraiment la Fortune ; ou si elle sait discerner ceux qui l’adorent, elle n’est plus la Fortune. Est-il vrai aussi que Jupiter l’envoie où il lui plaît ? Si cela est, qu’on ne serve donc que Jupiter, la Fortune étant incapable de résister à ses ordres et devant aller où il l’envoie ; ou du moins qu’elle n’ait pour adorateurs que les méchants et ceux qui ne veulent rien faire pour mériter et obtenir les dons de la Félicité.
\subsection[{Chapitre XIX}]{Chapitre XIX}

\begin{argument}\noindent De la Fortune féminine.
\end{argument}

\noindent Les païens ont tant de respect pour cette prétendue déesse Fortune, qu’ils ont très soigneusement conservé une tradition suivant laquelle la statue, érigée en son honneur par les matrones romaines sous le nom de {\itshape Fortune féminine}, aurait parlé et dit plusieurs fois que cet hommage lui était agréable. Le fait serait-il vrai, on ne devrait pas être fort surpris, car il est facile aux démons de tromper les hommes. Mais ce qui aurait dû ouvrir les yeux aux païens, c’est que la déesse qui a parlé est celle qui se donne au hasard, et non celle qui a égard aux mérites. La Fortune a parlé, dit-on, mais la Félicité est restée muette ; pourquoi cela, je vous prie, sinon pour que les hommes se missent peu en peine de bien vivre, assurés qu’ils étaient de la protection de la déesse aux aveugles faveurs ? Et en vérité, si la Fortune a parlé, mieux eût valu que ce fût la Fortune virile que la Fortune féminine, afin de ne pas laisser croire que ce grand miracle n’est en réalité qu’un bavardage de matrones.
\subsection[{Chapitre XX}]{Chapitre XX}

\begin{argument}\noindent De la Vertu et de la Foi, que les païens ont honorées comme des déesses par des temples et des autels, oubliant qu’il y a beaucoup d’autres vertus qui ont le même droit à être tenues pour des divinités.
\end{argument}

\noindent Ils ont fait une déesse de la Vertu, et certes, s’il existait une telle divinité, je conviens qu’elle serait préférable à beaucoup d’autres ; mais comme la vertu est un don de Dieu, et non une déesse, ne la demandons qu’à Celui qui seul peut la donner, et toute la tourbe des faux dieux s’évanouira. Pourquoi aussi ont-ils fait de la Foi une déesse, et lui ont-ils consacré un temple et un autel ? L’autel de la Foi est dans le cœur de quiconque est assez éclairé pour la posséder. D’où savent-ils d’ailleurs ce que c’est que la Foi, dont le meilleur et le principal ouvrage est de faire croire au vrai Dieu ? Et puis le culte de la Vertu ne suffisait-il pas ? La Foi n’est-elle pas où est la Vertu ? Eux-mêmes n’ont-ils pas divisé la Vertu en quatre espèces : la prudence, la justice, la force et la tempérance ? Or, la foi fait partie de la justice, surtout parmi nous qui savons que « le juste vit de la foi ». Mais je m’étonne que des gens si disposés à multiplier les dieux, et qui faisaient une déesse de la Foi, aient cruellement offensé plusieurs déesses en négligeant de diviniser toutes les autres vertus. La Tempérance, par exemple, n’a-t-elle pas mérité d’être une déesse, ayant procuré tant de gloire à quelques-uns des plus illustres Romains ? Pourquoi la Force n’a-t-elle pas des autels, elle qui assura la main de Mucius Scévola sur le brasier ardent, elle qui précipita Curtius dans un gouffre pour le bien de la patrie, elle enfin qui inspira aux deux Décius de dévouer leur vie au salut de l’armée, si toutefois il est vrai que ces Romains eussent la force véritable, ce que nous n’avons pas à examiner présentement. Qui empêche aussi que la Sagesse et la Prudence ne figurent au rang des déesses ? Dira-t-on qu’en honorant la Vertu en général, on honore toutes ces vertus ? À ce compte, on pourrait donc aussi n’adorer qu’un seul Dieu, si on croit que tous les dieux ne sont que des parties du Dieu suprême. Enfin la Vertu comprend aussi la Foi et la Chasteté, qui ont été jugées dignes d’avoir leurs autels propres dans des temples séparés.
\subsection[{Chapitre XXI}]{Chapitre XXI}

\begin{argument}\noindent Les païens, n’ayant pas la connaissance des dons de Dieu, auraient dû se borner au culte de la Vertu et de la Félicité.
\end{argument}

\noindent Disons-le nettement : toutes ces déesses ne sont pas filles de la vérité, mais de la vanité. Dans le fait, les vertus sont des dons du vrai Dieu, et non pas des déesses. D’ailleurs, quand on possède la Vertu et la Félicité, qu’y a-t-il à souhaiter de plus ? et quel objet pourrait suffire à qui ne suffisent pas la Vertu, qui embrasse tout ce qu’on doit faire, et la Félicité, qui renferme tout ce qu’on peut désirer ? Si les Romains adoraient Jupiter pour en obtenir ces deux grands biens (car le maintien d’un empire et son accroissement, supposé que ce soient des biens, sont compris dans la Félicité), comment n’ont-ils pas vu que la Félicité, aussi bien que la Vertu, est un don de Dieu, et non pas une déesse ? Ou si on voulait y voir des divinités, pourquoi ne pas s’en contenter, sans recourir à un si grand nombre d’autres dieux ? Car enfin rassemblez par la pensée toutes les attributions qu’il leur a plu de partager entre tous les dieux et toutes les déesses, je demande s’il est possible de découvrir un bien quelconque qu’une divinité puisse donner à qui posséderait la Vertu et la Félicité. Quelle science aurait-il à demander à Mercure et à Minerve, du moment que la Vertu contient en soi toutes les sciences, suivant la définition des anciens, qui entendaient par Vertu l’art de bien vivre, et faisaient venir le mot latin {\itshape ars} du mot grec {\itshape àreté} qui signifie vertu ? Si la Vertu suppose de l’esprit, qu’était-il besoin du père Catius, divinité chargée de rendre les hommes fins et avisés, la Félicité pouvant aussi d’ailleurs leur procurer cet avantage car naître spirituel est une chose heureuse ; et c’est pourquoi ceux qui n’étaient pas encore nés, ne pouvant servir la Félicité pour en obtenir de l’esprit, le culte que lui rendaient leurs parents devait suppléer à ce défaut. Quelle nécessité pour les femmes en couche d’invoquer Lucine, quand, avec l’assistance de la Félicité, elles pouvaient non seulement accoucher heureusement, mais encore mettre au monde des enfants bien partagés ? était-i besoin de recommander à la déesse Opis l’enfant qui naît, au dieu Vaticanus l’enfant qui vagit, à la déesse Cunina l’enfant au berceau, à la déesse Rumina l’enfant qui tète, au dieu Statilinus les gens qui sont debout, à la déesse Adéona ceux qui nous abordent, à la déesse Abéona ceux qui s’en vont ? pourquoi fallait-il s’adresser à la déesse Mens pour être intelligent, au dieu Volumnus et à la déesse Volumna pour posséder le bon vouloir, aux dieux des noces pour se bien marier, aux dieux des champs et surtout à la déesse Fructesea pour avoir une bonne récolte, à Mars et à Bellone pour réussir à la guerre, à la déesse Victoire pour être victorieux, au dieu Honos pour avoir des honneurs, à la déesse Pecunia pour devenir riche, enfin au dieu Asculanus et à son fils Argentinus pour avoir force cuivre et force argent ? Au fait, la monnaie d’argent a été précédée par la monnaie de cuivre ; et ce qui m’étonne, c’est qu’Argentinus n’ait pas à son tour engendré Aurinus, puisque la monnaie d’or est venue après. Si ce dieu eût existé, il est à croire qu’ils l’auraient préféré à son père Argentinus et à son grand-père Asculanus, comme ils ont préféré Jupiter à Saturne. Encore une fois, qu’était-il nécessaire, pour obtenir les biens de l’âme ou ceux du corps, ou les biens extérieurs, d’adorer et d’invoquer cette foule de dieux que je n’ai pas tous nommés, et que les païens eux-mêmes n’ont pu diviser et multiplier à l’égal de leurs besoins, alors que la déesse Félicité pouvait si aisément les résumer tous ? Et non seulement elle seule suffisait pour obtenir tous les biens, mais aussi pour éviter tous les maux ; car à quoi bon invoquer la déesse Fessonia contre la fatigue, la déesse Pellonia pour expulser l’ennemi, Apollon ou Esculape contre les maladies, ou ces deux médecins ensemble, quand le cas était grave ? à quoi bon enfin le dieu Spiniensis pour arracher les épines des champs, et la déesse Rubigo pour écarter la nielle ? La seule Félicité, par sa présence et sa protection, pouvait détourner ou dissiper tous ces maux. Enfin, puisque nous traitons ici de la Vertu et de la Félicité, si la Félicité est la récompense de la Vertu, ce n’est donc pas une déesse, mais un don de Dieu ; ou si c’est une déesse, pourquoi ne dit-on pas que c’est elle aussi qui donne la vertu, puisque être vertueux est une grande félicité ?
\subsection[{Chapitre XXII}]{Chapitre XXII}

\begin{argument}\noindent De la science qui apprend à servir les dieux, science que Varron se glorifie d’avoir apportée aux Romains.
\end{argument}

\noindent Quel est donc ce grand service que Varron se vante d’avoir rendu à ses concitoyens, en leur enseignant non seulement quels dieux ils doivent honorer, mais encore quelle est la fonction propre de chaque divinité ? Comme il ne sert de rien, dit-il, de connaître un médecin de nom et de visage, si l’on ne sait pas qu’il est médecin ; de même il est inutile de savoir qu’Esculape est un dieu, si l’on ignore qu’il guérit les maladies, et à quelle fin on peut avoir à l’implorer. Varron insiste encore sur cette pensée à l’aide d’une nouvelle comparaison : « On ne peut vivre agréablement », dit-il, « et même on ne peut pas vivre du tout, si l’on ignore ce que c’est qu’un forgeron, un boulanger, un couvreur, en un mot tout artisan à qui on peut avoir à demander un ustensile, ou encore si l’on ne sait où s’adresser pour un guide, pour un aide, pour un maître ; de même la connaissance des dieux n’est utile qu’à condition de savoir quelle est pour chaque divinité la faculté, la puissance, la fonction qui lui sont propres. » Et il ajoute : « Par ce moyen nous pouvons apprendre quel dieu il faut appeler et invoquer dans chaque cas particulier, et nous n’irons pas faire comme les baladins, qui demandent de l’eau à Bacchus et aux Nymphes du vin. » Oui certes, Varron a raison : voilà une science très utile, et il n’y a personne qui ne lui rendît grâce, si sa théologie était conforme à la vérité, c’est-à-dire s’il apprenait aux hommes à adorer le Dieu unique et véritable, source de tous les biens.
\subsection[{Chapitre XXIII}]{Chapitre XXIII}

\begin{argument}\noindent Les Romains sont restés longtemps sans adorer la Félicité, bien qu’ils adorassent un très grand nombre de divinités, et que celle-ci dut leur tenir lieu de toutes les autres.
\end{argument}

\noindent Mais revenons à la question, et supposons que les livres et le culte des païens soient fondés sur la Vérité, et que la Félicité soit une déesse ; pourquoi ne l’ont-ils pas exclusivement adorée, elle qui pouvait tout donner et rendre l’homme parfaitement heureux ? Car enfin on ne peut désirer autre chose que le bonheur. Pourquoi ont-ils attendu si tard, après tant de chefs illustres, et jusqu’à Lucullus, pour leur élever des autels ? pourquoi Romulus, qui voulait fonder une cité heureuse, n’a-t-il pas consacré un temple à cette divinité, de préférence à toutes les autres qu’il pouvait se dispenser d’invoquer, puisque rien ne lui aurait manqué avec elle ? En effet, sans son assistance il n’aurait pas été roi, ni placé ensuite au rang des dieux. Pourquoi donc a-t-il donné pour dieux aux Romains Janus, Jupiter, Mars, Picus, Faunus, Tibérinus, Hercule ? Quelle nécessité que Titus Tatius y ait ajouté Saturne, Ops, le Soleil, la Lune, Vulcain, la Lumière, et je ne sais combien d’autres, jusqu’à la déesse Cloacine, en même temps qu’il oubliait la Félicité ? D’où vient que Numa a également négligé cette divinité, lui qui a introduit tant de dieux et tant de déesses ? Serait-ce qu’il n’a pu la découvrir dans la foule ? Certes, si le roi Hostilius l’eût connue et adorée, il n’eût pas élevé des autels à la Peur et à la Pâleur. En présence de la Félicité, la Peur et la Pâleur eussent disparu, je ne dis pas apaisées, mais mises en fuite.\par
Au surplus, comment se fait-il que l’empire romain eût déjà pris de vastes accroissements, avant que personne adorât encore la Félicité ? Serait-ce pour cela qu’il était plus vaste qu’heureux ? Car comment la félicité véritable se fût-elle trouvée où la véritable piété n’était pas ? Or, la piété, c’est le culte sincère du vrai Dieu, et non l’adoration de divinités fausses qui sont autant de démons. Mais depuis même que la Félicité eut été reçue au nombre des dieux, cela n’empêcha pas les guerres civiles d’éclater. Serait-ce par hasard qu’elle fut justement indignée d’avoir reçu si tardivement des honneurs qui devenaient une sorte d’injure, étant partagés avec Priapa et Cloacine, avec la Peur, la Pâleur et la Fièvre, et tant d’autres idoles moins faites pour être adorées que pour perdre leurs adorateurs ?\par
Si l’on voulait après tout associer une si grande déesse à une troupe si méprisable, quene lui rendait-on tout au moins des honneurs plus distingués ? Est-ce une chose supportable que la Félicité n’ait été admise ni parmi les dieux Consentes, qui composent, dit-on, le conseil de Jupiter, ni parmi les dieux qu’on appelle « choisis » ? qu’on ne lui ait pas élevé quelque temple qui se fît remarquer par la hauteur de sa situation et par la magnificence de son architecture ? Pourquoi même n’aurait-on pas fait plus pour elle que pour Jupiter ? car si Jupiter occupe le trône, c’est la Félicité qui le lui a donné. Je suppose, il est vrai, qu’en possédant le trône il a possédé la félicité ; mais la félicité vaut encore mieux qu’un trône : car vous trouverez sans peine un homme à qui la royauté fasse peur ; vous n’en trouverez pas qui refuse la félicité. Que l’on demande aux dieux eux-mêmes, par les augures ou autrement, s’ils voudraient céder leur place à la Félicité, au cas où leurs temples ne laisseraient pas assez d’espace pour lui élever un édifice digne d’elle ; je ne doute point que Jupiter en personne ne lui abandonnât sans résistance les hauteurs du Capitole. Car nul ne peut résister à la félicité, à moins qu’il ne désire être malheureux, ce qui est impossible. Assurément donc, Jupiter n’en userait pas comme firent à son égard les dieux, Mars et Terme et la déesse Juventas, qui refusèrent nettement de lui céder la place, bien qu’il soit leur ancien et leur roi. On lit, en effet, dans les historiens romains, que Tarquin, lorsqu’il voulut bâtir le Capitole en l’honneur de Jupiter, voyant la place la plus convenable occupée par plusieurs autres dieux, et n’osant en disposer sans leur agrément, mais persuadé en même temps que ces dieux ne feraient pas difficulté de se déplacer pour un dieu de cette importance et qui était leur roi, s’enquit par les augures de leurs dispositions ; tous consentirent à se retirer, excepté ceux que j’ai déjà dits : Mars, Terme et Juventas ; de sorte que ces trois divinités furent admises dans le Capitole, mais sous des représentations si obscures qu’à peine les plus doctes savaient les y découvrir. Je dis donc que Jupiter n’eût pas agi de cette façon, ni traité la Félicité comme il fut traité lui-même par Mars, Terme et Juventas ; maisassurément ces divinités mêmes, qui résistèrent à Jupiter, n’eussent pas résisté à la Félicité, qui leur a donné Jupiter pour roi ; ou si elles lui eussent résisté, c’eût été moins par mépris que par le désir de garder une place obscure dans le temple de la Félicité, plutôt que de briller sans elle dans des sanctuaires particuliers.\par
Supposons donc la Félicité établie dans un lieu vaste et éminent ; tous les citoyens sauraient alors où doivent s’adresser leurs vœux légitimes. Secondés par l’inspiration de la nature, ils abandonneraient cette multitude inutile de divinités, de sorte que le temple de la Félicité serait désormais le seul fréquenté par tous ceux qui veulent être heureux, c’est-à-dire par tout le monde, et qu’on ne demanderait plus la félicité qu’à la Félicité elle-même, au lieu de la demander à tous les dieux. Et en effet que demande-t-on autre chose à quelque dieu que ce soit, sinon la félicité ou ce qu’on croit pouvoir y contribuer ? Si donc il dépend de la Félicité de se donner à qui bon lui semble, ce dont on ne peut douter qu’en doutant qu’elle soit déesse, n’est-ce pas une folie de demander la félicité à toute autre divinité, quand on peut l’obtenir d’elle-même ? Ainsi donc il est prouvé qu’on devait lui donner une place éminente et la mettre au-dessus de tous les dieux. Si j’en crois une tradition consignée dans les livres des païens, les anciens Romains avaient en plus grand honneur je ne sais quel dieu Summanus, à qui ils attribuaient les foudres de la nuit, que Jupiter lui-même, qui ne présidait qu’aux foudres du jour ; mais depuis qu’on eut élevé à Jupiter un temple superbe et un lieu éminent, la beauté et la magnificence de l’édifice attirèrent tellement la foule, qu’à peine aujourd’hui se trouverait-il un homme, je ne dis pas qui ait entendu parler du dieu Summanus, car il y a longtemps qu’on n’en parle plus, mais qui se souvienne même d’avoir jamais lu son nom. Concluons que la Félicité n’étant pas une déesse, mais un don de Dieu, il ne reste qu’à se tourner vers Celui qui seul peut la donner, et à laisser là cette multitude de faux dieux adorée par une multitude d’hommes insensés, qui travestissent en dieux les dons de Dieu et offensent par l’obstinationd’une volonté superbe le dispensateur de ces dons. Il ne peut manquer en effet d’être malheureux celui qui sert la Félicité comme une déesse et abandonne Dieu, principe de la félicité, semblable à un homme qui lécherait du pain en peinture, au lieu de s’adresser à qui possède du pain véritable.
\subsection[{Chapitre XXIV}]{Chapitre XXIV}

\begin{argument}\noindent Quelles raisons font valoir les païens pour se justifier d’adorer les dons divins comme des dieux.
\end{argument}

\noindent Voyons maintenant les raisons des païens : Peut-on croire, disent-ils, que nos ancêtres eussent assez peu de sens pour ignorer que la Félicité et la Vertu sont des dons divins et non des dieux ? mais comme ils savaient aussi que nul ne peut posséder ces dons à moins de les tenir de quelque dieu, faute de connaître les noms des dieux qui président aux divers objets qu’on peut désirer, ils les appelaient du nom de ces objets mêmes, tantôt avec un léger changement, comme de {\itshape bellum}, guerre, ils ont fait Bellone ; de {\itshape cunae}, berceau, Cunina ; de {\itshape seges}, moisson, Segetia ; de {\itshape pomum}, fruit, Pomone ; de {\itshape boves}, bœufs, Bubona ; et tantôt sans aucun changement, comme quand ils ont nommé Pecunia la déesse qui donne l’argent, sans penser toutefois que l’argent fût une divinité ; et de même, Vertu la déesse qui donne la vertu ; Honos, le dieu qui donne l’honneur ; Concordia, la déesse qui donne la concorde, et Victoria, celle qui donne la victoire. Ainsi, disent-ils, quand on croit que la Félicité est une déesse, on n’entend pas la félicité qu’on obtient, mais le principe divin qui la donne.
\subsection[{Chapitre XXV}]{Chapitre XXV}

\begin{argument}\noindent On ne doit adorer qu’un Dieu, qui est l’unique dispensateur de la félicité, comme le sentent ceux-là mêmes qui ignorent son nom.
\end{argument}

\noindent Acceptons cette explication ; ce sera peut-être un moyen de persuader plus aisément ceux d’entre les païens qui n’ont pas le cœur tout à fait endurci. Si l’humaine faiblesse n’a pas laissé de reconnaître qu’un dieu seul peutlui donner la félicité ; si le sentiment de cette vérité animait en effet les adorateurs de cette multitude de divinités, à la tête desquelles ils plaçaient Jupiter ; si enfin, dans l’ignorance où ils étaient du principe qui dispense la félicité, ils se sont accordés à lui donner le nom de l’objet même de leurs désirs, je dis qu’ils ont assez montré par là que Jupiter était incapable, à leurs propres yeux, de procurer la félicité véritable, mais qu’il fallait l’attendre de cet autre principe qu’ils croyaient devoir honorer sous le nom même de félicité. Je conclus qu’en somme ils croyaient que la félicité est un don de quelque dieu qu’ils ne connaissaient pas. Qu’on le cherche donc ce dieu, qu’on l’adore, et cela suffit. Qu’on bannisse la troupe tumultueuse des démons, et que le vrai Dieu suffise à qui suffit la félicité. S’il se rencontre un homme, en effet, qui ne se contente pas d’obtenir la félicité en partage, je veux bien que celui-là ne se contente pas d’adorer le dispensateur de la félicité ; mais quiconque ne demande autre chose que d’être heureux (et en vérité peut-on porter plus loin ses désirs ?) doit servir le Dieu à qui seul il appartient de donner le bonheur. Ce Dieu n’est pas celui qu’ils nomment Jupiter ; car s’ils reconnaissaient Jupiter pour le principe de la félicité, ils ne chercheraient pas, sous le nom de Félicité, un autre dieu ou une autre déesse qui pût le leur assurer. Ils ne mêleraient pas d’ailleurs au culte du roi des dieux les plus sanglants outrages, et n’adoreraient pas en lui l’époux adultère, le ravisseur et l’amant impudique d’un bel enfant.
\subsection[{Chapitre XXVI}]{Chapitre XXVI}

\begin{argument}\noindent Des jeux scéniques institués par les païens sur l’ordre de leurs dieux.
\end{argument}

\noindent Ce sont là, nous dit Cicéron, des fictions poétiques : « Homère, ajoute-t-il, transportait chez les dieux les faiblesses des hommes ; j’aimerais mieux qu’il eût transporté chez les hommes les perfections des dieux ». Juste réflexion d’un grave esprit, qui n’a pu voir sans déplaisir un poète prêter des crimes à la divinité. Pourquoi donc les plus doctes entre les païens mettent-ils au rang des choses divines les jeux scéniques où ces crimes sont débités, chantés, joués et célébrés pour faire honneur aux dieux ? C’est ici que Cicéron aurait dû se récrier, non contre les fictions des poètes, mais contre les institutions des ancêtres ! Mais ceux-ci, à leur tour, n’auraient-ils pas eu raison de répliquer : De quoi nous accusez-vous ? Ce sont les dieux eux-mêmes qui ont voulu que ces jeux fussent établis parmi les institutions de leur culte, qui les ont demandés avec instance et avec menaces, qui nous ont sévèrement punis d’y avoir négligé le moindre détail, et ne se sont apaisés qu’après avoir vu réparer cette négligence. Et, en effet, voici ce que l’on rapporte comme un de leurs beaux faits : Un paysan nommé Titus Latinius, reçut en songe l’ordre d’aller dire au sénat de recommencer les jeux, parce que, le premier jour où on les avait célébrés, un criminel avait été conduit au supplice en présence du peuple, triste incident qui avait déplu aux dieux et troublé pour eux le plaisir du spectacle. Latinius, le lendemain, à son réveil, n’ayant pas osé obéir, le même commandement lui fut fait la nuit suivante, mais d’une façon plus sévère ; car, comme il n’obéit pas pour la seconde fois, il perdit son fils. La troisième nuit, il lui fut dit que s’il n’était pas docile, un châtiment plus terrible lui était réservé. Sa timidité le retint encore, et il tomba dans une horrible et dangereuse maladie. Ses amis lui conseillèrent alors d’avertir les magistrats, et il se décida à se faire porter en litière au sénat, où il n’eut pas plutôt raconté le songe en question qu’il se trouva parfaitement guéri et put s’en retourner à pied. Le sénat, stupéfait d’un si grand miracle, ordonna une nouvelle célébration des jeux, où l’on ferait quatre fois plus de dépenses. Quel homme de bon sens ne reconnaîtra que ces malheureux païens, asservis à la domination des démons, dont on ne peut être délivré que par la grâce de Notre-Seigneur Jésus-Christ, étaient forcés de donner à leurs dieux immondes des spectacles dont l’impureté étau manifeste ? On y représentait en effet, par l’ordre du sénat, contraint lui-même d’obéir aux dieux, ces mêmes crimes qui se lisent dans les poètes. D’infâmes histrions y figuraient un Jupiter adultère et ravisseur, et ce spectacle était un honneur pour le dieu et un moyen de propitiation pour les hommes. Ces crimes étaient-ils une fiction ? Jupiter aurai dû s’en indigner. Étaient-ils réels et Jupiter s’y complaisait-il ? il est clair alors qu’en l’adorant on adorait les démons. Et maintenant, comment croire que ce soit Jupiter qui ait fondé l’empire romain, qui l’ait agrandi, qui l’ait conservé, lui plus vil, à coup sûr, que le dernier des Romains révoltés de ces infamies ? Aurait-il donné le bonheur, celui qui recevait de si malheureux hommages et qui, si on les lui refusait, se livrait à un courroux plus malheureux encore ?
\subsection[{Chapitre XXVII}]{Chapitre XXVII}

\begin{argument}\noindent Des trois espèces de dieux distingués par le pontife Scévola.
\end{argument}

\noindent Certains auteurs rapportent que le savant pontife Scévola distinguait les dieux en trois espèces, l’une introduite par les poètes, l’autre par les philosophes, et la troisième par les politiques. Or, disait-il, les dieux de la première espèce ne sont qu’un pur badinage d’imagination, où l’on attribue à la divinité ce qui est indigne d’elle ; et quant aux dieux de la seconde espèce, ils ne conviennent pas aux États, soit parce qu’il est inutile de les connaître, soit parce que cela peut être préjudiciable aux peuples. — Pour moi, je n’ai rien à dire des dieux inutiles ; cela n’est pas de grande conséquence, puisqu’en bonne jurisprudence, ce qui est superflu n’est pas nuisible ; mais je demanderai quels sont les dieux dont la connaissance peut être préjudiciable aux peuples ? Selon le docte pontife, ce sont Hercule, Esculape, Castor et Pollux, lesquels ne sont pas véritablement des dieux, car les savants déclarent qu’ils étaient hommes et qu’ils ont payé à la nature le tribut de l’humanité. Qu’est-ce à dire, sinon que les dieux adorés par le peuple ne sont que de fausses images, le vrai Dieu n’ayant ni âge, ni sexe, ni corps ? Et c’est cela que Scévola veut laisser ignorer aux peuples, justement parce que c’est la vérité. Il croit donc qu’il est avantageux aux États d’être trompés en matière de religion, d’accord en ce point avec Varron, qui s’en explique très nettement dans son livre des choses divines. Voilà une sublime religion, et bien capable de sauver le faible qui implore d’elle son salut ! Au lieu de lui présenter la vérité qui doit le sauver, elle estime qu’il faut le tromper pour son bien.\par
Quant aux dieux des poètes, nous apprenons à la même source que Scévola les rejette, comme ayant été défigurés à tel point qu’ils ne méritent pas même d’être comparés à des hommes de quelque probité. L’un est représenté comme un voleur, l’autre comme un adultère ; on ne leur prête que des actions et des paroles déshonnêtes ou ridicules : trois déesses se disputent le prix de la beauté, et les deux rivales de Vénus ruinent Troie pour se venger de leur défaite ; Jupiter se change en cygne ou en taureau pour jouir d’une femme ; on voit une déesse qui se marie avec un homme, et Saturne qui dévore ses enfants ; en un mot, il n’y a pas d’action monstrueuse et de vice imaginable qui ne soit imputé aux dieux, bien qu’il n’y ait rien de plus étranger que tout cela à la nature divine. Ô grand pontife Scévola ! abolis ces jeux, si tu en as le pouvoir ; défends au peuple un culte où l’on se plaît à admirer des crimes, pour avoir ensuite à les imiter. Si le peuple te répond que les pontifes eux-mêmes sont les instituteurs de ces jeux, demande au moins aux dieux qui leur ont ordonné de les établir, qu’ils cessent de les exiger ; car enfin ces jeux sont mauvais, tu en conviens, ils sont indignes de la majesté divine ; et dès lors l’injure est d’autant plus grande qu’elle doit rester impunie. Mais les dieux ne t’écoutent pas ; ou plutôt ce ne sont pas des dieux, mais des démons ; ils enseignent le mal, ils se complaisent dans la turpitude ; loin de tenir à injure ces honteuses fictions ; ils se courrouceraient, au contraire, si on ne les étalait pas publiquement. Tu invoquerais en vain Jupiter contre ces jeux, sous prétexte que c’est à lui que l’on prête le plus de crimes ; car vous avez beau l’appeler le chef et le maître de l’univers, vous lui faites vous-même la plus cruelle injure, en le confondant avec tous ces autres dieux dont vous dites qu’il est le roi.
\subsection[{Chapitre XXVIII}]{Chapitre XXVIII}

\begin{argument}\noindent Si le culte des dieux a été utile aux Romains pour établir et accroître leur empire.
\end{argument}

\noindent Ces dieux que l’on apaise, ou plutôt que l’on accuse par de semblables honneurs, et qui seraient moins coupables de se plaire au spectacle de crimes réels que de forfaits supposés, n’ont donc pu en aucune façon agrandir ni conserver l’empire romain. S’ils avaient eu un tel pouvoir, ils en auraient usé de préférence en faveur des Grecs, qui leur ont rendu, en cette partie du culte, de beaucoup plus grands honneurs, eux qui ont consenti à s’exposer eux-mêmes aux mordantes satires dont les poètes déchiraient les dieux, et leur ont permis de diffamer tous les citoyens à leur gré ; eux enfin qui, loin de tenir les comédiens pour infâmes, les ont jugés dignes des premières fonctions de l’État. Mais tout comme les Romains ont pu avoir de la monnaie d’or sans adorer le dieu Aurinus ; ainsi ils n’eussent pas laissé d’avoir de la monnaie d’argent et de cuivre, alors même qu’ils n’eussent pas adoré Argentinus et Æsculanus. De même, sans pousser plus avant la comparaison, il leur était absolument impossible de parvenir à l’empire sans la volonté de Dieu, tandis que, s’ils eussent ignoré ou méprisé cette foule de fausses divinités, ne connaissant que le seul vrai Dieu et l’adorant avec une foi sincère et de bonnes mœurs, leur empire sur la terre, plus grand ou plus petit, eût été meilleur, et n’eussent-ils pas régné sur la terre, ils seraient certainement parvenus au royaume éternel.
\subsection[{Chapitre XXIX}]{Chapitre XXIX}

\begin{argument}\noindent De la fausseté du présage sur lequel les Romains fondaient la puissance et la stabilité de leur empire.
\end{argument}

\noindent Que dire de ce beau présage qu’ils ont cru voir dans la persistance des dieux Mars et Terme et de la déesse Juventas, à ne pas céder la place au roi des dieux ? Cela signifiait, selon eux, que le peuple de Mars, c’est-à-dire le peuple romain, ne quitterait jamais un terrain une fois occupé ; que, grâce au dieu Terme, nul ne déplacerait les limites qui terminent l’empire ; enfin que la déesse Juventas rendrait la jeunesse romaine invincible. Mais alors, comment pouvaient-ils à la fois reconnaître en Jupiter le roi des dieux et le protecteur de l’empire, et accepter ce présage au nom des divinités qui faisaient gloire de lui résister ? Au surplus, que les dieux aient résisté en effet à Jupiter, ou non, peu importe ; car, supposé que les païens disent vrai, ils n’accorderont certainement pas que les dieux, qui n’ont point voulu céder à Jupiter, aient cédé à Jésus-Christ. Or, il est certain que Jésus-Christ a pu les chasser, non seulement de leurs temples, mais du cœur des croyants, et cela sans que les bornes de l’empire romain aient été changées. Ce n’est pas tout : avant l’Incarnation de Jésus-Christ, avant que les païens n’eussent écrit les livres que nous citons, mais après l’époque assignée à ce prétendu présage, c’est-à-dire après le règne de Tarquin, les armées romaines, plusieurs fois réduites à prendre la fuite, n’ont-elles pas convaincu la science des augures de fausseté ? En dépit de la déesse Juventas, du dieu Mars et du dieu Terme, le peuple de Mars a été vaincu dans Rome même, lors de l’invasion des Gaulois, et les bornes qui terminaient l’empire ont été resserrées, au temps d’Annibal, par la défection d’un grand nombre de cités. Ainsi se sont évanouies les belles promesses de ce grand présage, et il n’est resté que la seule rébellion, non pas de trois divinités, mais de trois démons contre Jupiter. Car on ne prétendra pas apparemment que ce soit la même chose de ne pas quitter la place qu’on occupait et de s’y réintégrer. Ajoutez même à cela que l’empereur Adrien changea depuis, en Orient, les limites de l’empire romain, par la cession qu’il fit au roi de Perse de trois belles provinces, l’Arménie, la Mésopotamie et la Syrie ; en sorte qu’on dirait que le dieu Terme, gardien prétendu des limites de l’empire, dont la résistance à Jupiter avait donné lieu à une si flatteuse prophétie, a plus appréhendé d’offenser Adrien que le roi des dieux. Je conviens que les provinces un instant cédées furent dans la suite réunies à l’empire, mais depuis, et presque de notre temps, le dieu Terme a encore été contraint de reculer, lorsque l’empereur Julien, si adonné aux oracles des faux dieux, mit le feu témérairement à sa flotte chargée de vivres ; le défaut de subsistances, et peu après la blessure et la mort de l’empereur lui-même, réduisirent l’armée à une telle extrémité, que pas un soldat n’eût échappé, si par un traité de paix on n’eût remis les bornes de l’empire où elles sont aujourd’hui ; traité moins onéreux sans doute que celui de l’empereur Adrien, mais dont les conditions n’étaient pas, tant s’en faut, avantageuses. C’était donc un vain présage que la résistance du dieu Terme, puisque après avoir tenu bon contre Jupiter, il céda depuis à la volonté d’Adrien, à la témérité de Julien et à la détresse de Jovien, son successeur. Les plus sages et les plus clairvoyants parmi les Romains savaient tout cela ; mais ils étaient trop faibles pour lutter contre des superstitions enracinées par l’habitude, outre qu’eux-mêmes croyaient que la nature avait droit à un culte, qui n’appartient en vérité qu’au maître et au roi de la nature : « Adorateurs de la créature », comme dit l’Apôtre, « plutôt que du Créateur, qui est béni dans « tous les siècles ». Il était donc nécessaire que la grâce du vrai Dieu envoyât sur la terre des hommes vraiment saints et pieux, capables de donner leur vie pour établir la religion vraie, et pour chasser les religions fausses du milieu des vivants.
\subsection[{Chapitre XXX}]{Chapitre XXX}

\begin{argument}\noindent Ce que pensaient, de leur propre aveu, les païens eux-mêmes touchant les dieux du paganisme.
\end{argument}

\noindent Cicéron, tout augure qu’il était, se moque des augures et gourmande ceux qui livrent la conduite de leur vie à des corbeaux et à des corneilles. On dira qu’un philosophe de l’Académie, pour qui tout est incertain, ne peut faire autorité en ces matières. Mais dans son traité {\itshape De la nature des dieux}, Cicéron introduit au second livre Q. Lucilius Balbus, qui, après avoir assigné aux superstitions une origine naturelle et philosophique, ne laisse pas de s’élever contre l’institution des idoles et contre les opinions fabuleuses. « Voyez-vous, dit-il, comment on est parti de bonnes et utiles découvertes physiques, pour en venir à des dieux imaginaires et faits à plaisir ? Telle est la source d’une infinité de fausses opinions, d’erreurs pernicieuses et de superstitions ridicules. On sait les différentes figures de ces dieux, leur âge, leurs babillements, leurs ornements, leurs généalogies, leurs mariages, leurs alliances, tout cela fait à l’image de l’humaine fragilité. On les dépeint avec nos passions, amoureux, chagrins, colères ; on leur attribue même des guerres et des combats, non seulement lorsque, partagés entre deux armées ennemies, comme dans Homère, les uns sont pour celle-ci, et les autres pour celle-là ; mais encore quand ils combattent pour leur propre compte contre les Titans ou les Géants. Certes, il y a bien de la folie et à débiter et à croire des fictions si vaines et si mal fondées. » Voilà les aveux des défenseurs du paganisme. Il est vrai qu’après avoir traité toutes ces croyances de superstition, Balbus en veut distinguer la religion véritable, qui est pour lui, à ce qu’il paraît, dans la doctrine des stoïciens. « Ce ne sont pas seulement les philosophes, dit-il, mais nos ancêtres mêmes qui ont séparé la religion de la superstition. En effet, ceux qui passaient toute la journée en prières et en sacrifices pour obtenir que leurs enfants leur survécussent, furent appelés superstitieux. » Qui ne voit ici que Cicéron, craignant de heurter le préjugé public, fait tous ses efforts pour louer la religion des ancêtres, et pour la séparer de la superstition, mais sans pouvoir y parvenir ? En effet, si les anciens Romains appelaient superstitieux ceux qui passaient les jours en prières et en sacrifices, ceux-là ne l’étaient-ils pas également, qui avaient imaginé ces statues dont se moque Cicéron, ces dieux d’âge et d’habillements divers, leurs généalogies, leurs mariages et leurs alliances ? Blâmer ces usages comme superstitieux, c’est accuser de superstition les anciens qui les ont établis ; l’accusation retombe même ici sur l’accusateur qui, en dépit de la liberté d’esprit ou il essaie d’atteindre en paroles, était obligé de respecter en fait les objets de ses risées, et qui fut reste aussi muet devant le peuple qu’il est disert et abondant en ses écrits Pour nous, chrétiens, rendons grâces, non pas au ciel et à la terre, comme le veut ce philosophe, mais au Seigneur, notre Dieu, qui a fait le ciel et la terre, de ce que par la profonde humilité de Jésus-Christ, par la prédication des Apôtres, par la foi des martyrs, qui sont morts pour la vérité, mais qui vivent avec la vérité, il a détruit dans les cœurs religieux, et aussi dans les temples, ces superstitions que Balbus ne condamne qu’en balbutiant.
\subsection[{Chapitre XXXI}]{Chapitre XXXI}

\begin{argument}\noindent Varron a rejeté les superstitions populaires et reconnu qu’il ne faut adorer qu’un seul Dieu, sans être parvenu toutefois à la connaissance du Dieu véritable.
\end{argument}

\noindent Varron, que nous avons vu au reste, et non sans regret, se soumettre à un préjugé qu’il n’approuvait pas, et placer les jeux scéniques au rang des choses divines, ce même Varron ne confesse-t-il point dans plusieurs passages, où il recommande d’honorer les dieux, que le culte de Rome n’est point un culte de son choix, et que, s’il avait à fonder une nouvelle république, il se guiderait, pour la consécration des dieux et des noms des dieux, sur les lois de la nature ? Mais étant né chez un peuple déjà vieux, il est obligé, dit-il, de s’en tenir aux traditions de l’antiquité ; et son but, en recueillant les noms et les surnoms des dieux, c’est de porter le peuple à la religion, bien loin de la lui rendre méprisable. Par où ce pénétrant esprit nous fait assez comprendre que dans son livre sur la religion il ne dit pas tout, et qu’il a pris soin de taire, non seulement ce qu’il trouvait déraisonnable, mais ce qui aurait pu le paraître au peuple. On pourrait prendre ceci pour une conjecture, si Varron lui-même, parlant ailleurs des religions, ne disait nettement qu’il y a des vérités que le peuple ne doit pas savoir, et des impostures qu’il est bon de lui inculquer comme des vérités. C’est pour cela, dit-il, que les Grecs ont caché leurs mystères et leurs initiations dans le secret des sanctuaires. Varron nous livre ici toute la politique de ces législateurs réputés sages, qui ont jadis gouverné les cités et les peuples ; et cependant rien n’est plus fait que cette conduite artificieuse pour être agréable aux démons, à ces esprits de malice qui tiennent également en leur puissance et ceux qui trompent et ceux qui sont trompés, sans qu’il y ait un autre moyen d’échapper à leur joug que la grâce de Dieu par Jésus-Christ Notre-Seigneur.\par
Ce même auteur, dont la pénétration égale la science, dit encore que ceux-là seuls lui semblent avoir compris la nature de Dieu, qui ont reconnu en lui l’âme qui gouverne le monde par le mouvement et l’intelligence. On peut conclure de là que, sans posséder encore la vérité, car le vrai Dieu n’est pas une âme, mais le Créateur de l’âme, Varron toutefois, s’il eût pu secouer le joug de la coutume, eût reconnu et proclamé qu’on ne doit adorer qu’un seul Dieu qui gouverne le monde par le mouvement et l’intelligence ; de sorte que toute la question entre lui et nous serait de lui prouver que Dieu n’est point une âme, mais le Créateur de l’âme. Il ajoute que les anciens Romains, pendant plus de cent soixante-dix ans, ont adoré les dieux sans en faire aucune image. « Et si cet usage », dit-il, « s’était maintenu, le culte qu’on leur rend en serait plus pur et plus saint. » Il allègue même, entre autres preuves, à l’appui de son sentiment, l’exemple du peuple juif, et conclut sans hésiter que ceux qui ont donné les premiers au peuple les images des dieux, ont détruit la crainte et augmenté l’erreur, persuadé avec raison que le mépris des dieux devait être la suite nécessaire de l’impuissance de leurs simulacres. En ne disant pas qu’ils ont fait naître l’erreur, mais qu’ils l’ont augmentée, il veut faire entendre qu’on était déjà dans l’erreur à l’égard des dieux, avant même qu’il y eût des idoles. Ainsi, quand il soutient que ceux-là seuls ont connu la nature de Dieu, qui ont vu en lui l’âme du monde, et que la religion en serait plus pure, s’il n’y avait point d’idoles, qui ne voit combien il a approché de la vérité ? S’il avait eu quelque pouvoir contre une erreur enracinée depuis tant de siècles, je ne doute point qu’il n’eût recommandé d’adorer ce Dieu unique par qui il croyait le monde gouverné, et dont il voulait le culte pur de toute image ; peut-être même, se trouvant si près de la vérité, et considérant la nature changeante de l’âme, eût-il été amené à reconnaître que le vrai Dieu, Créateur de l’âme elle-même, est un principe essentiellement immuable, S’il en est ainsi, on peut croire que dans les conseils de la Providence toutes les railleries de ces savants hommes contre la pluralité des dieux étaient moins destinées à ouvrir les yeux au peuple qu’à rendre témoignage à la vérité. Si donc nous citons leurs ouvrages, c’est pour y trouver une arme contre ceux qui s’obstinent à ne pas reconnaître combien est grande et tyrannique la domination des démons, dont nous sommes délivrés par le sacrifice unique du sang précieux versé pour notre salut, etpar le don du Saint-Esprit descendu sur nous.
\subsection[{Chapitre XXXII}]{Chapitre XXXII}

\begin{argument}\noindent Dans quel intérêt les chefs d’État ont maintenu parmi les peuples de fausses religions.
\end{argument}

\noindent Varron dit encore, au sujet de la génération des dieux, que les peuples s’en sont plutôt rapportés aux portes qu’aux philosophes, et que c’est pour cela que les anciens Romains ont admis des dieux mâles et femelles, des dieux qui naissent et qui se marient. Pour moi, je crois que l’origine de ces croyances est dans l’intérêt qu’ont eu les chefs d’État à tromper le peuple en matière de religion ; en cela imitateurs fidèles des démons qu’ils adoraient, et qui n’ont pas de plus grande passion que de tromper les hommes. De même, en effet, que les démons ne peuvent posséder que ceux qu’ils abusent, ainsi ces faux sages, semblables aux démons, ont répandu parmi les hommes, sous le nom de religion, des croyances dont la fausseté leur était connue, afin de resserrer les liens de la société civile et de soumettre plus aisément les peuples à leur puissance. Or, comment des hommes faibles et ignorants auraient-ils pu résister à la double imposture des chefs d’État et des démons conjurés ?
\subsection[{Chapitre XXXIII}]{Chapitre XXXIII}

\begin{argument}\noindent La durée des empires et des rois ne dépend que des conseils et de la puissance de Dieu.
\end{argument}

\noindent Ce Dieu donc, auteur et dispensateur de la félicité, parce qu’il est le seul vrai Dieu, est aussi le seul qui distribue les royaumes de la terre aux bons et aux méchants. Il les donne, non pas d’une manière fortuite, car il est Dieu et non la Fortune, mais selon l’ordre des choses et des temps qu’il connaît et que nous ignorons. Ce n’est pas qu’il soit assujetti en esclave à cet ordre ; loin de là, il le règle en maître et le dispose en arbitre souverain. Aux bons seuls il donne la félicité : car, qu’on soit roi ou sujet, il n’importe, on peut également la posséder comme ne la posséder pas ; mais nul n’en jouira pleinement que dans cette vie supérieure où il n’y aura ni maîtres ni sujets. Or, si Dieu donne les royaumes de la terre aux bons et aux méchants, c’est de peur que ceux de ses serviteurs dont l’âme est encore jeune et peu éprouvée, ne désirent de tels objets comme des récompenses de la vertu et des biens d’un grand prix. Voilà tout le secret de l’Ancien Testament qui cachait le Nouveau sous ses figures. On y promettait les biens de la terre, mais les âmes spirituelles comprenaient déjà, quoique sans le proclamer hautement, que ces biens temporels figuraient ceux de l’éternité, et elles n’ignoraient pas en quels dons de Dieu consiste la félicité véritable.
\subsection[{Chapitre XXXIV}]{Chapitre XXXIV}

\begin{argument}\noindent Le royaume des Juifs fut institué par le vrai Dieu et par lui maintenu, tant qu’ils persévérèrent dans la vraie religion.
\end{argument}

\noindent Au surplus, pour montrer que c’est de lui, et non de cette multitude de faux dieux adorés par les Romains, que dépendent les biens de la terre, les seuls où aspirent ceux qui n’en peuvent concevoir de meilleurs, Dieu voulut que son peuple se multipliât prodigieusement en Égypte, d’où il le tira ensuite par des moyens miraculeux. Cependant les femmes juives n’invoquaient point la déesse Lucine, quand Dieu sauva leurs enfants des mains des Égyptiens qui les voulaient exterminer tous. Ces enfants furent allaités sans la déesse Rumina, et mis au berceau sans la déesse Cunina. Ils n’eurent pas besoin d’Educa et de Potina pour boire et pour manger. Leur premier âge fut soigné sans le secours des dieux enfantins ; ils se marièrent sans les dieux conjugaux, et s’unirent à leurs femmes sans avoir adoré Priape. Bien qu’ils n’eussent pas invoqué Neptune, la mer s’ouvrit devant eux, et elle ramena ses flots sur les Égyptiens. Ils ne s’avisèrent point d’adorer une déesse Mannia, quand ils reçurent la marine du ciel, ni d’invoquer les Nymphes quand, du rocher frappé par Moïse, jaillit une source pour les désaltérer. Ils firent la guerre sans les folles cérémonies de Mars et de Bellone ; et s’ils ne furent pas, j’en conviens, victorieux sans la Victoire, ils virent en elle, non une déesse, mais un don de leur Dieu. Enfin ils ont eu des moissons sans Segetia, des bœufs sans Bubona, du miel sans Mellona, et des fruits sans Pomone ; et, en un mot, tout ce que les Romains imploraient de cette légion de divinités, les Juifs l’ont obtenu, et d’une façon beaucoup plus heureuse, de l’unique et véritable Dieu. S’ils ne l’avaient point offensé en s’abandonnant à une curiosité impie, qui, pareille à la séduction des arts magiques, les entraîna vers les dieux étrangers et vers les idoles, et finit par leur faire verser le sang de Jésus-Christ, nul doute qu’ils n’eussent maintenu leur empire, sinon plus vaste, au moins plus heureux que celui des Romains. Et maintenant les voilà dispersés à travers les nations, par un effet de la providence du seul vrai Dieu, qui a voulu que nous pussions prouver par leurs livres que la destruction des idoles, des autels, des bois sacrés et des temples, l’abolition des sacrifices ; en un mot que tous ces événements, dont nous sommes aujourd’hui témoins, ont été depuis longtemps prédits ; car si on ne les lisait que dans le Nouveau Testament, on s’imaginerait peut-être que nous les avons controuvés. Mais réservons ce qui suit pour un autre livre, celui-ci étant déjà assez long.
\section[{Livre cinquième. Anciennes mœurs des Romains}]{Livre cinquième. \\
Anciennes mœurs des Romains}\renewcommand{\leftmark}{Livre cinquième. \\
Anciennes mœurs des Romains}

\subsection[{Préface}]{Préface}
\noindent Puisqu’il est constant que tous nos désirs possibles ont pour terme la félicité, laquelle n’est point une déesse, mais un don de Dieu, et qu’ainsi les hommes ne doivent point adorer d’autre Dieu que celui qui peut les rendre heureux (car si la félicité était une déesse, elle seule devrait être adorée), voyons maintenant pourquoi Dieu, qui a dans ses mains, avec tout le reste, cette sorte de biens que peuvent posséder les hommes mêmes qui ne sont pas bons, ni par conséquent heureux, a voulu donner à l’empire romain tant de grandeur et de durée : avantage que leurs innombrables divinités étaient incapables de leur assurer, ainsi que nous l’avons déjà fait voir amplement, et que nous le montrerons à l’occasion.
\subsection[{Chapitre premier}]{Chapitre premier}

\begin{argument}\noindent La destinée de l’empire romain et celle de tous les autres empires ne dépendent ni de causes fortuites, ni de la position des astres.
\end{argument}

\noindent La cause de la grandeur de l’empire romain n’est ni fortuite, ni fatale, à prendre ces mots dans le sens de ceux qui appellent fortuit ce qui arrive sans cause ou ce dont les causes ne se rattachent à aucun ordre raisonnable, et fatal, ce qui arrive sans la volonté de Dieu ou des hommes, en vertu d’une nécessité inhérente à l’ordre des choses. Il est hors de doute, en effet, que c’est la providence de Dieu qui établit les royaumes de la terre ; et si quelqu’un vient soutenir qu’ils dépendent du destin, en appelant destin la volonté de Dieu ou sa puissance, qu’il garde son sentiment, mais qu’il corrige son langage. Car pourquoi ne pas dire tout d’abord ce qu’il dira ensuite quand on lui demandera ce qu’il entend par destin ? Le destin, en effet, dans le langage ordinaire, désigne l’influence de la position des astres sur les événements, comme il arrive, dit-on, à la naissance d’une personne ou au moment qu’elle est conçue. Or, les uns veulent que cette influence ne dépende pas de la volonté de Dieu, les autres qu’elle en dépende.\par
Mais, à dire vrai, le sentiment qui affranchit nos actions de la volonté de Dieu, et fait dépendre des astres nos biens et nos maux, doit être rejeté, non seulement de quiconque professe la religion véritable, mais de ceux-là mêmes qui en ont une fausse, quelle qu’elle soit. Car où tend cette opinion, si ce n’est à supprimer tout culte et toute prière ? Mais ce n’est pas à ceux qui la soutiennent que nous nous adressons présentement ; nos adversaires sont les païens qui, pour la défense de leurs dieux, font la guerre à la religion chrétienne. Quant à ceux qui font dépendre de la volonté de Dieu la position des étoiles, s’ils croient qu’elles tiennent de lui, par une sorte de délégation de son autorité, le pouvoir de décider à leur gré de la destinée et du bonheur des hommes, ils font une grande injure au ciel de s’imaginer que dans cette cour brillante, dans ce sénat radieux, on ordonne des crimes tellement énormes qu’un État qui en ordonnerait de semblables, verrait le genre humain tout entier se liguer pour le détruire. D’ailleurs, si les astres déterminent nécessairement les actions des hommes, que reste-t-il à la décision de Celui qui est le maître des astres et des hommes ? Dira-t-on que les étoiles ne tiennent pas de Dieu le pouvoir de disposer à leur gré des choses humaines, mais qu’elles se bornent à exécuter ses ordres ? Nous demanderons comment il est possible d’imputer à la volonté de Dieu ce qui serait indigne de celle des étoiles. Il ne reste donc plus qu’à soutenir, comme ont fait quelques hommes d’un raresavoir, que les étoiles ne font pas les événements, mais qu’elles les annoncent, qu’elles sont des signes et non des causes. Je réponds que les astrologues n’en parlent pas de la sorte. Ils ne disent pas, par exemple : Dans telle position Mars annonce un assassin ; ils disent Mars fait un assassin. Je veux toutefois qu’ils ne s’expliquent pas exactement, et qu’il faille les renvoyer aux philosophes pour apprendre d’eux à s’énoncer comme il faut, et à dire que les étoiles annoncent ce qu’ils disent qu’elles font ; d’où vient qu’ils n’ont jamais pu rendre compte de la diversité qui se rencontre dans la vie de deux enfants jumeaux, dans leurs actions, dans leur destinée, dans leurs professions, dans leurs talents, dans leurs emplois, en un mot dans toute la suite de leur existence et dans leur mort même ; diversité quelquefois si grande, que des étrangers leur sont plus semblables qu’ils ne le sont l’un à l’autre, quoiqu’ils n’aient été séparés dans leur naissance que par un très petit espace de temps, et que leur mère les ait conçus dans le même moment ?
\subsection[{Chapitre II}]{Chapitre II}

\begin{argument}\noindent Ressemblance et diversité des maladies de deux jumeaux.
\end{argument}

\noindent L’illustre médecin Hippocrate a écrit, au rapport de Cicéron, que deux frères étant tombés malades ensemble, la ressemblance des accidents de leur mal, qui s’aggravait et se calmait en même temps, lui fit juger qu’ils étaient jumeaux. De son côté, le stoïcien Posidonius, grand partisan de l’astrologie expliquait le fait en disant que les deux frères étaient nés et avaient été conçus sous la même constellation. Ainsi, ce que le médecin faisait dépendre de la conformité des tempéraments, le philosophe astrologue l’attribuait à celle des influences célestes. Mais la conjecture du médecin est de beaucoup la plus acceptable et la plus plausible ; car on comprend fort bien que ces deux enfants, au moment de la conception, aient reçu de la disposition physique de leurs parents une impression analogue, et qu’ayant pris leurs premiers accroissements au ventre de la même mère, ils soient nés avec la même complexion. Ajoutez à cela que, nourris dansla même maison, des mêmes aliments, respirant le même air, buvant la même eau, faisant les mêmes exercices, toutes choses qui, selon les médecins, influent beaucoup sur la santé, soit en bien, soit en mal, ce genre de vie commun a dû rendre leur tempérament si semblable, que les mêmes causes les faisaient tomber malades en même temps. Mais vouloir expliquer cette conformité physique par la position qu’occupaient les astres au moment de leur conception ou de leur naissance, quand il a pu naître sous ces mêmes astres, semblablement disposés, un si grand nombre d’êtres si prodigieusement différents d’espèces, de dispositions et de destinées, c’est à mon avis le comble de l’impertinence. Je connais des jumeaux qui non seulement diffèrent dans la conduite et les vicissitudes de leur carrière, mais dont les maladies ne se ressemblent nullement. Il me semble qu’Hippocrate rendrait aisément raison de cette diversité en l’attribuant à la différence des aliments et des exercices, lesquels dépendent de la volonté et non du tempérament ; mais quant à Posidonius ou à tout autre partisan de l’influence fatale des astres, je ne vois pas ce qu’il aurait à dire ici, à moins qu’il ne voulût abuser de la crédulité des personnes peu versées dans ces matières. On essaie de se tirer d’affaire en arguant du petit intervalle qui sépare toujours la naissance de deux jumeaux, d’où provient, dit-on, la différence de leurs horoscopes ; mais ou bien cet intervalle n’est pas assez considérable pour motiver la diversité qui se rencontre dans la conduite des jumeaux, dans leurs actions, leurs mœurs et les accidents de leur vie, où il l’est trop pour s’accorder avec la bassesse ou la noblesse de condition commune aux deux enfants, puisqu’on veut que la condition de chacun dépende de l’heure où il est né. Or, si l’un naît immédiatement après l’autre, de manière à ce qu’ils aient le même horoscope, je demande pour eux une parfaite conformité en toutes choses, laquelle ne peut jamais se rencontrer dans les jumeaux les plus semblables ; et si le second met un si long temps à venir après le premier, que cela change l’horoscope, je demande ce qui ne peut non plias se rencontrer en deux jumeaux, la diversité de père et de mère.
\subsection[{Chapitre III}]{Chapitre III}

\begin{argument}\noindent De l’argument de la roue du potier, allégué par le mathématicien Nigidius dans la question des jumeaux.
\end{argument}

\noindent On aurait donc vainement recours au fameux argument de la roue du potier, que Nigidius imagina, dit-on, pour sortir de cette difficulté, et qui lui valut le surnom de Figulus. Il imprima à une roue de potier le mouvement le plus rapide possible, et pendant qu’elle tournait, il la marqua d’encre à deux reprises, mais si rapprochées, qu’on aurait pu croire qu’il ne l’avait touchée qu’une fois ; or, quand on eut arrêté la roue, on y trouva deux marques, séparées l’une de l’autre par un intervalle assez grand. C’est ainsi, disait-il, qu’avec la rotation de la sphère céleste, encore que deux jumeaux se suivent d’aussi près que les deux coups dont j’ai touché la roue, cela fait dans le ciel une grande distance, d’où résulte la diversité qui se rencontre dans les mœurs des deux enfants et dans les accidents de leur destinée. À mon avis, cet argument est plus fragile encore que les vases façonnés avec la roue du potier. Car si cet énorme intervalle qui se trouve dans le ciel entre la naissance de deux jumeaux, est cause qu’il vient un héritage à celui-ci et non à celui-là, sans que leur horoscope pût faire deviner cette différence, comment ose-t-on prédire à d’autres personnes dont on prend l’horoscope, et qui ne sont point jumelles, qu’il leur arrivera de semblables bonheurs dont la cause est impénétrable, et cela avec la prétention de faire tout dépendre du moment précis de la naissance. Diront-ils que dans l’horoscope de ceux qui ne sont point jumeaux, ils fondent leurs prédictions sur de plus grands intervalles de temps, au lieu que la courte distance qui se rencontre entre la naissance de deux jumeaux ne peut produire dans leur destinée que de petites différences, sur lesquelles on n’a pas coutume de consulter les astrologues, telles que s’asseoir, se promener, se mettre à table, manger ceci ou cela ? mais ce n’est pas là résoudre la difficulté, puisque la différence que nous signalons entre les jumeaux comprendleurs mœurs, leurs inclinations et les vicissitudes de leur destinée.
\subsection[{Chapitre IV}]{Chapitre IV}

\begin{argument}\noindent Des deux jumeaux Ésaü et Jacob, fort différents de caractère et de conduite.
\end{argument}

\noindent Du temps de nos premiers pères naquirent deux jumeaux (pour ne parler que des plias célèbres), qui se suivirent de si près en venant au monde, que le premier tenait l’autre par le pied. Cependant leur vie et leurs mœurs furent si différentes, leurs actions si contraires, l’affection de leurs parents si dissemblable, que le petit intervalle qui sépara leur naissance suffit pour les rendre ennemis. Qu’est-ce à dire ? S’agit-il de savoir pourquoi l’un se promenait quand l’autre était assis, pourquoi celui-ci dormait ou gardait le silence quand celui-là veillait ou parlait ? nullement ; car de si petites différences tiennent à ces courts intervalles de temps que ne sauraient mesurer ceux qui signalent la position des astres au moment de la naissance, pour consulter ensuite les astrologues. Mais point du tout : l’un des jumeaux de la Bible a été longtemps serviteur à gages, l’autre n’a pas été serviteur ; l’un était aimé de sa mère, l’autre ne l’était pas ; l’un perdit son droit d’aînesse, si important chez les Juifs, et l’autre l’acquit. Parlerai-je de leurs femmes, de leurs enfants, de leurs biens ? Quelle diversité à cet égard entre les deux frères ? Si tout cela est une suite du petit intervalle qui sépare la naissance des deux jumeaux et ne peut être attribué aux constellations, je demande encore comment on ose, sur la foi des constellations, prédire à d’autres leur destinée ? Aime-t-on mieux dire que les destinées ne dépendent pas de ces intervalles imperceptibles, mais bien d’espaces de temps plus grands qui peuvent être observés ? À quoi sert alors ici la roue du potier, sinon à faire tourner des cœurs d’argile et à cacher le néant de la science astrologique ?
\subsection[{Chapitre V}]{Chapitre V}

\begin{argument}\noindent Preuves de la vanité de l’astrologie.
\end{argument}

\noindent Ces deux frères, dont la maladie augmentait ou diminuait en même temps, et qu’à ce signe le coup d’œil médical d’Hippocrate reconnut jumeaux, ne suffisent-ils pas àconfondre ceux qui veulent imputer aux astres une conformité qui s’explique par celle du tempérament ? Car, d’où vient qu’ils étaient malades en même temps, au lieu de l’être l’un après l’autre, suivant l’ordre de leur naissance, qui n’avait pu être simultanée ? Ou si le moment différent de leur naissance n’a pu faire qu’ils fussent malades en des moments différents, de quel droit vient-on soutenir que cette première différence en a produit une foule d’autres dans leurs destinées ? Quoi ! ils ont pu voyager en des temps différents, se marier, avoir des enfants, toujours en des temps différents, et cela, dit-on, parce qu’ils étaient nés en des temps différents ; et ils n’ont pu être malades en des temps différents ! Si la différence dans l’heure de la naissance a influé sur l’horoscope et causé les mille diversités de leurs destinées, pourquoi l’identité dans le moment de la conception s’est-elle fait sentir par la conformité de leurs maladies ? Dira-t-on que les destins de la santé sont attachés au moment de la conception, et ceux du reste de la vie au moment de la naissance ? mais alors les astrologues ne devraient rien prédire touchant la santé d’après les constellations de la naissance, puisqu’on leur laisse forcément ignorer le moment de la conception. D’un autre côté, si on prétend prédire les maladies sans consulter l’horoscope de la conception, sous prétexte qu’elles sont indiquées par le moment de la naissance, comment aurait-on pu annoncer à un de nos jumeaux, d’après l’heure où il était né, à quelle époque il serait malade, puisque l’intervalle qui a séparé la naissance des deux frères ne les a pas empêchés de tomber malades en même temps. Je demande en outre à ceux qui soutiennent que le temps qui s’écoule entre la naissance de deux jumeaux est assez considérable pour changer les constellations et l’horoscope, et tous ces ascendants mystérieux qui ont tant d’influence sur les destinées, je demande, dis-je, comment cela est possible, puisque les deux jumeaux ont été nécessairement conçus au même instant. De plus, si les destinées de deux jumeaux peuvent être différentes quant au moment de la naissance, bien qu’ils aient été conçus au même instant, pourquoi les destinées de deux enfants nés en même temps ne seraient-elles pas différentes pour la vie et pour la mort ? En effet, si le même moment où ils ont été conçus n’a pas empêché que l’un ne vînt avant l’autre, je ne vois pas par quelle raison le même moment où ils sont nés s’opposerait à ce que celui-ci mourût avant celui-là ; et si une conception simultanée a eu pour eux des effets si différents dans le ventre de leurs mères, pourquoi une naissance simultanée ne serait-elle pas suivie dans le cours de la vie d’accidents non moins divers, de manière à confondre également toutes les rêveries d’un art chimérique ? Quoi ! deux enfants conçus au même moment, sous la même constellation, peuvent avoir, même à l’heure de la naissance, une destinée différente ; et deux enfants, nés dans le même instant et sous les mêmes signes, de deux différentes mères, ne pourront pas avoir deux destinées différentes qui fassent varier les accidents de leur vie et de leur mort, à moins qu’on ne s’avise de prétendre que les enfants, bien que déjà conçus, ne peuvent avoir une destinée qu’à leur naissance ? Mais pourquoi dire alors que, si l’on pouvait savoir le moment précis de la conception, les astrologues feraient des prophéties encore plus surprenantes, ce qui a donné lieu à cette anecdote, que plusieurs aiment à répéter, d’un certain sage qui sut choisir son heure pour avoir de sa femme un enfant merveilleux. Cette opinion était aussi celle de Posidonius, grand astrologue et philosophe, puisqu’il expliquait la maladie simultanée de nos jumeaux par la simultanéité de leur naissance et de leur conception. Remarquez qu’il ajoutait conception, afin qu’on ne lui objectât pas que les deux jumeaux n’étaient pas nés au même instant précis ; il lui suffisait qu’ils eussent été conçus en même temps pour attribuer leur commune maladie, non à la ressemblance de leur tempérament, mais à l’influence des astres. Mais si le moment de la conception a tant de force pour régler les destinées et les rendre semblables, la naissance ne devrait pas les diversifier ; ou, si l’on dit que les destinées des jumeaux sont différentes à cause qu’ils naissent en des temps différents, que ne dit-on qu’elles sont déjà changées par cela seul qu’ils naissent en des temps différents ? Se peut-il que la volonté des vivants ne change point les destins de la naissance, lorsque l’ordre même de la naissance change ceux de la conception ?
\subsection[{Chapitre VI}]{Chapitre VI}

\begin{argument}\noindent Des jumeaux de sexe différent.
\end{argument}

\noindent Il arrive même souvent dans la conception des jumeaux, laquelle a certainement lieu au même moment et sous la même constellation, que l’un est mâle et l’autre femelle. Je connais deux jumeaux de sexe différent qui sont encore vivants et dans la fleur de l’âge. Bien qu’ils se ressemblent extérieurement autant que le comporte la différence des sexes, ils mènent toutefois un genre de vie très opposé, et cela, bien entendu, abstraction faite des occupations qui sont propres au sexe de chacun : l’un est comte, militaire, et voyage presque toujours à l’étranger ; l’autre ne quitte jamais son pays, pas même sa maison de campagne. Mais voici ce qui paraîtra incroyable si l’on croit à l’influence des astres ; et ce qui n’a rien de surprenant si l’on considère le libre arbitre de l’homme et la grâce divine : le frère est marié, tandis que la sœur est vierge consacrée à Dieu ; l’un a beaucoup d’enfants, et l’autre n’en veut point avoir. On dira, je le sais, que la force de l’horoscope est grande. Pour moi, je pense en avoir assez prouvé la vanité ; et, après tout, les astrologues tombent d’accord qu’il n’a de pouvoir que pour la naissance. Donc il est inutile pour la conception, laquelle s’opère indubitablement par une seule action, puisque tel est l’ordre inviolable de la nature qu’une femme qui vient de concevoir cesse d’être propre à la conception ; d’où il résulte que deux jumeaux sont de toute nécessité conçus au même instant précis, Dira-t-on qu’étant nés sous un horoscope différent, ils ont été changés au moment de leur naissance, l’un en mâle et l’autre en femelle ? Peut-être ne serait-il pas tout à fait absurde de soutenir que les influences des astres soient pour quelque chose dans la forme des corps ainsi, l’approche ou l’éloignement du soleil produit la variété des saisons, et suivant que la lune est à son croissant ou à son décours, on voit certaines choses augmenter ou diminuer, comme les hérissons de mer, les huîtres et les marées ; mais vouloir soumettre aux mêmes influences les volontés des hommes, c’est nous donner lieu de chercher des raisons pour en affranchir
\subsection[{Chapitre VII}]{Chapitre VII}

\begin{argument}\noindent Du choix des jours, soit pour se marier, soit pour semer ou planter.
\end{argument}

\noindent Comment s’imaginer qu’en choisissant tel ou tel jour pour commencer telle ou telle entreprise, on puisse se faire de nouveaux destins ? Cet homme, disent-ils, n’était pas né pour avoir un fils excellent, mais plutôt pour en avoir un méprisable ; mais il a eu l’art, voulant devenir père, de choisir son heure. Il s’est donc fait un destin qu’il n’avait pas, et par là une fatalité a commencé pour lui, qui n’existait pas au moment de sa naissance. Étrange folie ! on choisit un jour pour se marier, et c’est, j’imagine, pour ne pas tomber, faute de choix, sur un mauvais jour, ers d’autres termes, pour ne pas faire un mariage malheureux ; mais, s’il en est ainsi, à quoi servent les destins attachés à notre naissance ? Un homme peut-il, par le choix de tel ou tel jour, changer sa destinée, et ce que sa volonté détermine ne saurait-il être changé par une puissance étrangère ? D’ailleurs, s’il n’y a sous le ciel que les hommes qui soient soumis aux influences des astres, pourquoi choisir certains jours pour planter, pour semer, d’autres jours pour dompter les animaux, pour les accoupler, et pour toutes les opérations semblables ? Si l’on dit que ce choix a de l’importance, parce que tous les corps animés ou inanimés sont assujettis à l’action des astres, il suffira de faire observer combien d’êtres naissent ou commencent en même temps, dont la destinée est tellement différente que cela suffit pour faire rire un enfant, même aux dépens de l’astrologie. Où est en effet l’homme assez dépourvu de sens pour croire que chaque arbre, chaque plante, chaque bête, serpent, oiseau, vermisseau, ait pour naître son moment fatal ? Cependant, pour éprouver la science des astrologues, on a coutume de leur apporter l’horoscope des animaux et de donner la palme à ceux qui s’écrient en le regardant : Ce n’est pas un homme qui est né, c’est une bête. Ils vont jusqu’à désigner hardiment à quelle espèce elle appartient, si c’est une bête à laine ou une bête de trait, si elle est propre au labourage ou à la garde de la maison. On les consulte même sur la destinée des chiens, et l’os écoute leurs réponses avec de grands applaudissements. Les hommes seraient-ils donc assez sots pour s’imaginer que la naissance d’un homme arrête si bien le développement de tous les autres germes, qu’une mouche ne puisse naître sous la même constellation que lui ? car, si on admet la production d’une mouche, il faudra remonter par une gradation nécessaire à la naissance d’un chameau ou d’un éléphant. Ils ne veulent pas remarquez qu’au jour choisi par eux pour ensemencer un champ, il y a une infinité de grains qui tombent sur terre ensemble, germent ensemble, lèvent, croissent, mûrissent en même temps, et que cependant, de tous ces épis de même âge et presque de même germe, les uns sont brûlés par la nielle, les autres mangés par les oiseaux, les autres arrachés par les passants. Dira-t-on que ces épis, dont la destinée est si différente, sont sous l’influence de différentes constellations, ou, si on ne peut le dire, conviendra-t-on de la vanité du choix des jours et de l’impuissance des constellations sur les êtres inanimés, ce qui réduit leur empire à l’espèce humaine, c’est-à-dire aux seuls êtres de ce monde à qui Dieu ait donné une volonté libre ? Tout bien considéré, il y a quelque raison de croire que si les astrologues étonnent quelquefois par la vérité de leurs réponses, c’est qu’ils sont secrètement inspirés par les démons, dont le soin le plus assidu est de propager dans les esprits ces fausses et dangereuses opinions sur l’influence fatale des astres ; de sorte que ces prétendus devins n’ont été en rien guidés dans leurs prédictions par l’inspection de l’horoscope, et que toute leur science des astres se trouve réduite à rien.
\subsection[{Chapitre VIII}]{Chapitre VIII}

\begin{argument}\noindent De ceux qui appellent destin l’enchaînement des causes conçu comme dépendant de la volonté de Dieu.
\end{argument}

\noindent Quant à ceux qui appellent destin, non la disposition des astres au moment de la conception ou de la naissance, mais la suite et l’enchaînement des causes qui produisent tout ce qui arrive dans l’univers, je ne m’arrêterai pas à les chicaner sur un mot, puisqu’au fond ils attribuent cet enchaînement de causes à la volonté et à la puissance souveraine d’un principe souverain qui est Dieu même, dont il est bon et vrai de croire qu’il sait d’avance et ordonne tout, étant le principe de toutes les puissances sans l’être de toutes les volontés. C’est donc cette volonté de Dieu, dont la puissance irrésistible éclate partout, qu’ils appellent destin, comme le prouvent ces vers dont Annaeus Sénèque est l’auteur, si je ne me trompe :\par
{\itshape « Conduis-moi, père suprême, dominateur du vaste univers, conduis-moi partout où tu voudras, je l’obéis sans différer ; me voilà. Fais que je te résiste, et il faudra encore que je t’accompagne en gémissant ; il faudra que je subisse, en devenant coupable, le sort que j’aurais pu accepter avec une résignation vertueuse. Les destins conduisent qui les suit et entraînent qui leur résiste.} »\par
Il est clair que le poète appelle destin au dernier vers, ce qu’il a nommé plus haut la volonté du père suprême, qu’il se déclare prêt à suivre librement, afin de n’en pas être entraîné : « Car les destins conduisent qui les suit, et entraînent qui leur résiste. » C’est ce qu’expriment aussi deux vers homériques traduits par Cicéron :\par
 {\itshape « Les volontés des hommes sont ce que les fait Jupiter, le père tout-puissant, qui fait briller sa lumière autour de l’univers. »} \par
Je ne voudrais pas donner une grande autorité à ce qui ne serait qu’une pensée de poète ; mais, comme Cicéron nous apprend que les stoïciens avaient coutume de citer ces vers d’Homère en témoignage de la puissance du destin, il ne s’agit pas tant ici de la pensée d’un poète que de celle d’une école de philosophes, qui nous font voir très clairement ce qu’ils entendent par destin, puisqu’ils appellent Jupiter ce dieu suprême dont ils font dépendre l’enchaînement des causes.
\subsection[{Chapitre IX}]{Chapitre IX}

\begin{argument}\noindent De la prescience de Dieu et de la libre volonté de l’homme, contre le sentiment de Cicéron.
\end{argument}

\noindent Cicéron s’attache à réfuter le système stoïcien, et il ne croit pas en venir à bout, s’il ne supprime d’abord la divination ; mais en la supprimant il va jusqu’à nier toute science des choses à venir. Il soutient de toutes ses forces que cette science ne se rencontre ni en Dieu, ni dans l’homme, et que toute prédiction est chose nulle. Par là, il nie la prescience de Dieu et s’inscrit en faux contre toutes les prophéties, fussent-elles plus claires que le jour, sans autre appui que de vains raisonnements et certains oracles faciles à réfuter et qu’il ne réfute même pas. Tant qu’il n’a affaire qu’aux prophéties des astrologues, qui se détruisent elles-mêmes, son éloquence triomphe ; mais cela n’empêche pas que la thèse de l’influence fatale des astres ne soit au fond plus supportable que la sienne, qui supprime toute connaissance de l’avenir. Car, admettre un Dieu et lui refuser la prescience, c’est l’extravagance la plus manifeste. Cicéron l’a fort bien senti, mais il semble qu’il ait voulu justifier cette parole de l’Écriture : « L’insensé a dit dans son cœur : Il n’y a point de Dieu. » Au reste, il ne parle pas en son nom ; et ne voulant pas se donner l’odieux d’une opinion fâcheuse, il charge Cotta, dans le livre {\itshape De la nature des dieux}, de discuter contre les stoïciens et de soutenir que la divinité n’existe pas. Quant à ses propres opinions, il les met dans la bouche de Balbus, défenseur des stoïciens. Mais au livre {\itshape De la divination}, Cicéron n’hésite pas à se porter en personne l’adversaire de la prescience. Il est clair que son grand et unique objet, c’est d’écarter le destin et de sauver le libre arbitre, étant persuadé que si l’on admet la science des choses à venir, c’est une conséquence inévitable qu’on ne puisse nier le destin. Pour nous, laissons les philosophes s’égarer dans le dédale de ces combats et de ces disputes, et, convaincus qu’il existe un Dieu souverain et unique, croyons également qu’il possède une volonté, une puissance et une prescience souveraines. Ne craignons pas que les actes que nous produisons volontairement ne soient pas des actes volontaires ; car ces actes, Dieu les a prévus, et sa prescience est infaillible. C’est cette crainte qui a porté Cicéron à combattre la prescience, et c’est elle aussi qui a fait dire aux stoïciens que tout n’arrive pas nécessairement dans l’univers, bien que tout y soit soumis au destin.\par
Qu’est-ce donc que Cicéron appréhendait si fort dans la prescience, pour la combattre avec une si déplorable ardeur ? C’est, sans doute, que si tous les événements à venir sont prévus, ils ne peuvent manquer de s’accomplir dans le même ordre où ils ont été prévus ; or, s’ils s’accomplissent dans cet ordre, il y a donc un ordre des événements déterminé dans la prescience divine ; et si l’ordre des événements est déterminé, l’ordre des causes l’est aussi, puisqu’il n’y a point d’événement possible qui ne soit précédé par quelque cause efficiente. Or, si l’ordre des causes, par qui arrive tout ce qui arrive, est déterminé, tout ce qui arrive, dit Cicéron, est l’ouvrage du destin. « Ce point accordé, ajoute-t-il, toute l’économie de la vie humaine est renversée ; c’est en vain qu’on fait des lois, en vain qu’on a recours aux reproches, aux louanges, au blâme, aux exhortations ; il n’y a point de justice à récompenser les bons ni à punir les méchants. » C’est donc pour prévenir des conséquences si monstrueuses, si absurdes, si funestes à l’humanité, qu’il rejette la prescience et réduit les esprits religieux à faire un choix entre ces deux alternatives qu’il déclare incompatibles : ou notre volonté a quelque pouvoir, ou il y a une prescience. Démontrez-vous une de ces deux choses ? par là même, suivant Cicéron, vous détruisez l’autre, et vous ne pouvez affirmer le libre arbitre sans nier la prescience. C’est pour cela que ce grand esprit, en vrai sage, qui connaît à fond les besoins de la vie humaine, se décide pour le libre arbitre ; mais, afin de l’établir, il nietoute science des choses futures ; et voilà comme en voulant faire l’homme libre il le fait sacrilège. Mais un cœur religieux repousse cette alternative ; il accepte l’un et l’autre principe, les confesse également vrais, et leur donne pour base commune la foi qui vient de la piété. Comment cela ? dira Cicéron ; car, la prescience étant admise, il en résulte une suite de conséquences étroitement enchaînées qui aboutissent à conclure que notre volonté ne peut rien ; et si on admet que notre volonté puisse quelque chose, il faut, en remontant la chaîne, aboutir à nier la prescience. Et, en effet, si la volonté est libre, le destin ne fait pas tout ; si le destin ne fait pas tout, l’ordre de toutes les causes n’est point déterminé ; si l’ordre de toutes les causes n’est point déterminé, l’ordre de tous les événements n’est point déterminé non plus dans la prescience divine, puisque tout événement suppose avant lui une cause efficiente ; si l’ordre des événements n’est point déterminé pour la prescience divine, il n’est pas vrai que toutes choses arrivent comme Dieu a prévu qu’elles arriveraient ; et si toutes choses n’arrivent pas comme Dieu a prévu qu’elles arriveraient, il n’y a pas, conclut Cicéron, de prescience en Dieu.\par
Contre ces témérités sacrilèges du raisonnement, nous affirmons deux choses : la première, c’est que Dieu connaît tous les événements avant qu’ils ne s’accomplissent ; la seconde, c’est que nous faisons par notre volonté tout ce que nous sentons et savons ne faire que parce que nous le voulons. Nous sommes si loin de dire avec les stoïciens : le destin fait tout, que nous croyons qu’il ne fait rien, puisque nous démontrons que le destin, en entendant par là, suivant l’usage, la disposition des astres au moment de la naissance ou de la conception, est un mot creux qui désigne une chose vaine, Quant à l’ordre des causes, où la volonté de Dieu a la plus grande puissance, nous ne la nions pas, mais nous ne lui donnons pas le nom de destin, à moins qu’on ne fasse venir le {\itshape fatum} de {\itshape fari}, parler ; car nous ne pouvons contester qu’il ne soit écrit dans les livres saints : « Dieu a parlé une fois, et j’ai entendu ces deux choses : la puissance est à Dieu, et la miséricorde est aussi à vous, ô mon Dieu, qui rendrez à chacun selon ses œuvres. » Or, quand le Psalmiste dit : Dieu a parlé une fois, il faut entendre une parole immobile, immuable, comme la connaissance que Dieu a de tout ce qui doit arriver et de tout ce qu’il doit faire. Nous pourrions donc entendre ainsi le fatum, si on ne le prenait d’ordinaire en un autre sens, que nous ne voulons pas laisser s’insinuer dans les cœurs. Mais la vraie question est de savoir si, du moment qu’il y a pour Dieu un ordre déterminé de toutes les causes, il faut refuser tout libre arbitre à la volonté. Nous le nions ; et en effet, nos volontés étant les causes de nos actions, font elles-mêmes partie de cet ordre des causes qui est certain pour Dieu et embrassé par sa prescience. Par conséquent, celui qui a vu d’avance toutes les causes des événements, n’a pu ignorer parmi ces causes les volontés humaines, puisqu’il y a vu d’avance les causes de nos actions.\par
L’aveu même de Cicéron, que rien n’arrive qui ne suppose avant soi une cause efficiente, suffit ici pour le réfuter. Il ne lui sert de rien d’ajouter que toute cause n’est pas fatale, qu’il y en a de fortuites, de naturelles, de volontaires ; c’est assez qu’il reconnaisse que rien n’arrive qui ne suppose avant soi une cause efficiente. Car, qu’il y ait des causes fortuites, d’où vient même le nom de fortune, nous ne le nions pas ; nous disons seulement que ce sont des causes cachées, et nous les attribuons à la volonté du vrai Dieu ou à celle de quelque esprit. De même pour les causes naturelles, que nous ne séparons pas de la volonté du créateur de la nature. Restent les causes volontaires, qui se rapportent soit à Dieu, soit aux anges, soit aux hommes, soit aux bêtes, si toutefois on peut appeler volontés ces mouvements d’animaux privés de raison, qui les portent à désirer ou à fuir ce qui convient ou ne convient pas à leur nature. Quand je parle des volontés des anges, je réunis par la pensée les bons anges ou anges de Dieu avec les mauvais anges ou anges du diable, et ainsi des hommes, bons ou méchants. Il suit de là qu’il n’y a point d’autres causes efficientes de tout ce qui arrive que les causes volontaires, c’est-à-dire procédant de cette nature qui est l’esprit de vie. Car l’air ou le vent s’appelle aussi en latin esprit ; mais comme c’est un corps, ce n’est point l’esprit de vie. Le véritable esprit de vie, qui vivifie toutes choses et qui est lecréateur de tout corps et de tout esprit créé, c’est Dieu, l’esprit incréé. Dans sa volonté réside la toute-puissance, par laquelle il aide les bonnes volontés des esprits créés, juge les mauvaises, les ordonne toutes, accorde la puissance à celles-ci et la refuse à celles-là. Car, comme il est le créateur de toutes les natures, il est le dispensateur de toutes les puissances, mais non pas de toutes les volontés, les mauvaises volontés ne venant pas de lui, puisqu’elles sont contre la nature qui vient de lui. Pour ce qui est des corps, ils sont soumis aux volontés, les uns aux nôtres, c’est-à-dire aux volontés de tous les animaux mortels, et plutôt des hommes que des bêtes ; les autres à celles des anges ; mais tous sont soumis principalement à la volonté de Dieu, à qui même sont soumises toutes les volontés en tant qu’elles n’ont de puissance que par lui. Ainsi donc, la cause qui fait les choses et qui n’est point faite, c’est Dieu. Les autres causes font et sont faites : tels sont tous les esprits créés et surtout les raisonnables. Quant aux causes corporelles, qui sont plutôt faites qu’elles ne font, on ne doit pas les compter au nombre des causes efficientes, parce qu’elles ne peuvent que ce que font par elles les volontés des esprits. Comment donc l’ordre des causes, déterminé dans la prescience divine, pourrait-il faire que rien ne dépendît de notre volonté, alors que nos volontés tiennent une place si considérable dans l’ordre des causes ? Que Cicéron dispute tant qu’il voudra contre les stoïciens, qui disent que cet ordre des causes est fatal, ou plutôt qui identifient l’ordre des causes avec ce qu’ils appellent destin ; pour nous, cette opinion nous fait horreur, surtout à cause du mot, que l’usage a détourné de son vrai sens. Mais quand Cicéron vient nier que l’ordre des causes soit déterminé et parfaitement connu de la prescience divine, nous détestons sa doctrine plus encore que ne faisaient les stoïciens ; car, ou il faut qu’il nie expressément Dieu, comme il a essayé de le faire, sous le nom d’un autre personnage, dans son traité {\itshape De la nature des dieux} ; ou si en confessant l’existence de Dieu il lui refuse la prescience, cela revient encore à dire avec l’insensé dont parle l’Écriture : Il n’y a point de Dieu. En effet, celui qui ne connaît point l’avenir n’est point Dieu. En résumé, nos volontés ont le degré de puissance que Dieu leur assigne par sa volonté et sa prescience ; d’où il résulte qu’elles peuvent très certainement tout ce qu’elles peuvent, et qu’elles feront effectivement ce qu’elles feront, parce que leur puissance et leur action ont été prévues par celui dont la prescience est infaillible. C’est pourquoi, si je voulais me servir du mot destin, je dirais que le destin de la créature est la volonté du Créateur, qui tient la créature en son pouvoir, plutôt que de dire avec les stoïciens que le destin (qui dans leur langage est l’ordre des causes) est incompatible avec le libre arbitre.
\subsection[{Chapitre X}]{Chapitre X}

\begin{argument}\noindent S’il y a quelque nécessité qui domine les volontés des hommes.
\end{argument}

\noindent Cessons donc d’appréhender cette nécessité tant redoutée des stoïciens, et qui leur a fait distinguer deux sortes de causes : les unes qu’ils soumettent à la nécessité, les autres qu’ils en affranchissent, et parmi lesquelles ils placent la volonté humaine, étant persuadés qu’elle cesse d’être libre du moment qu’on la soumet à la nécessité. Et en effet, si on appelle nécessité pour l’homme ce qui n’est pas en sa puissance, ce qui se fait en dépit de sa volonté, comme par exemple la nécessité de mourir, il est évident que nos volontés, qui font que notre conduite est bonne ou mauvaise, ne sont pas soumises à une telle nécessité. Car nous faisons beaucoup de choses que nous ne ferions certainement pas si nous ne voulions pas les faire. Telle est la propre essence du vouloir : si nous voulons, il est ; si nous ne voulons pas, il n’est pas, puisque enfin on ne voudrait pas, si on ne voulait pas. Mais il y a une autre manière d’entendre la nécessité, comme quand on dit qu’il est nécessaire que telle chose soit ou arrive de telle façon ; prise en ce sens, je ne vois dans la nécessité rien de redoutable, rien qui supprime le libre arbitre de la volonté. On ne soumet pas en effet à la nécessité la vie et la prescience divines, en disant qu’il est nécessaire que Dieu vive toujours et prévoie toutes choses, pas plus qu’on ne diminue la puissance divine en disant que Dieu ne peut ni mourir, ni être trompé. Ne pouvoir pas mourir est si peu une impuissance, que si Dieu pouvait mourir, il ne serait pas la puissance infinie. On a donc raison de l’appeler le Tout-Puissant, quoiqu’il ne puisse ni mourir, ni être trompé ; car sa toute-puissance consiste à faire ce qu’il veut et à ne pas souffrir ce qu’il ne veut pas ; double condition sans laquelle il ne serait plus le Tout-Puissant. D’où l’on voit enfin que ce qui fait que Dieu ne peut pas certaines choses, c’est sa toute-puissance même. Pareillement donc, dire qu’il est nécessaire que lorsque nous voulons, nous voulions par notre libre arbitre, c’est dire une chose incontestable ; mais il ne s’ensuit pas que notre libre arbitre soit soumis à une nécessité qui lui ôte sa liberté. Nos volontés restent nôtres, et c’est bien elles qui font ce que nous voulons faire, ou, en d’autres termes, ce qui ne se ferait pas si nous ne le voulions faire. Et quand j’ai quelque chose à souffrir du fait de mes semblables et contre ma volonté propre, il y a encore ici une manifestation de la volonté, non sans doute de ma volonté propre, mais de celle d’autrui, et avant tout de la volonté et de la puissance de Dieu. Car, dans le cas même où la volonté de mes semblables serait une volonté sans puissance, cela viendrait évidemment de ce qu’elle serait empêchée par une volonté supérieure ; elle supposerait donc une autre volonté, tout en restant elle-même une volonté distincte, impuissante à faire ce qu’elle veut. C’est pourquoi, tout ce que l’homme souffre contre sa volonté, il ne doit l’attribuer, ni à la volonté des hommes, ni à celle des anges ou de quelque autre esprit créé, mais à la volonté de Dieu, qui donne le pouvoir aux volontés.\par
On aurait donc tort de conclure que rien ne dépend de notre volonté, sous prétexte que Dieu a prévu ce qui devait en dépendre. Car ce serait dire que Dieu a prévu là où il n’y avait rien à prévoir. Si en effet celui qui a prévu ce qui devait dépendre un jour de notre volonté, a véritablement prévu quelque chose, il faut conclure que ce quelque chose, objet de sa prescience, dépend en effet de notre volonté. C’est pourquoi nous ne sommes nullement réduits à cette alternative, ou de nier le libre arbitre pour sauver la prescience de Dieu, ou de nier la prescience de Dieu, pensée sacrilège ! pour sauver le libre arbitre ; mais nous embrassons ces deux principes, et nous les confessons l’un et l’autre avec la même foi et la même sincérité : la prescience, pour bien croire ; le libre arbitre, pour bien vivre. Impossible d’ailleurs de bien vivre, si on ne croit pas de Dieu ce qu’il est bien d’en croire. Gardons-nous donc soigneusement, sous prétexte de vouloir être libres, de nier la prescience de Dieu, puisque c’est Dieu seul dont la grâce nous donne ou nous donnera la liberté. Ainsi, ce n’est pas en vain qu’il y a des lois, ni qu’on a recours aux réprimandes, aux exhortations, à la louange et au blâme ; car Dieu a prévu toutes ces choses, et elles ont tout l’effet qu’il a prévu qu’elles auraient ; et de même les prières servent pour obtenir de lui les biens qu’il a prévu qu’il accorderait à ceux qui prient ; et enfin il y a de la justice à récompenser les bons et à châtier les méchants. Un homme ne pèche pas parce que Dieu a prévu qu’il pécherait ; tout au contraire, il est hors de doute que quand il pèche, c’est lui-même qui pèche, celui dont la prescience est infaillible ayant prévu que son péché, loin d’être l’effet du destin ou de la fortune, n’aurait d’autre cause que sa propre volonté. Et sans doute, s’il ne veut pas pécher, il ne pèche pas ; mais alors Dieu a prévu qu’il ne voudrait pas pécher.
\subsection[{Chapitre XI}]{Chapitre XI}

\begin{argument}\noindent La providence de Dieu est universelle et embrasse tout sous ses lois.
\end{argument}

\noindent Considérez maintenant ce Dieu souverain et véritable qui, avec son Verbe et son Esprit saint, ne forme qu’un seul Dieu en trois personnes, ce Dieu unique et tout-puissant, auteur et créateur de toutes les âmes et de tous les corps, source de la félicité pour quiconque met son bonheur, non dans les choses vaines, mais dans les vrais biens, qui a fait de l’homme un animal raisonnable, composé de corps et d’âme, et après son péché, ne l’a laissé-ni sans châtiment, ni sans miséricorde ; qui a donné aux bons et aux méchants l’être comme aux pierres, la vie végétative comme aux plantes, la vie sensitive comme aux animaux, la vie intellectuelle comme aux anges ; ce Dieu, principe de toute règle, de toute beauté, de tout ordre ; qui donne à tout le nombre, le poids et la mesure ; de qui dérive toute production naturelle, quels qu’en soient le genre et le prix : les semences des formes, les formes des semences, le mouvement des semences et des formes ; ce Dieu qui a créé la chair avec sa beauté, sa vigueur, sa fécondité, la disposition de ses organes et la concorde salutaire de ses éléments ; qui a donné à l’âme animale la mémoire, les sens et l’appétit, et à l’âme raisonnable la pensée, l’intelligence et la volonté ; ce Dieu qui n’a laissé aucune de ses œuvres, je ne dis pas le ciel et la terre, je ne dis pas les anges et les hommes, mais les organes du plus petit et du plus vil des animaux, la plume d’un oiseau, la moindre fleur des champs, une feuille d’arbre, sans y établir la convenance des parties, l’harmonie et la paix ; je demande s’il est croyable que ce Dieu ait souffert que les empires de la terre, leurs dominations et leurs servitudes, restassent étrangers aux lois de sa providence ?
\subsection[{Chapitre XII}]{Chapitre XII}

\begin{argument}\noindent Par quelles vertus les anciens Romains ont mérité que le vrai Dieu accrût leur empire, bien qu’ils ne l’adorassent pas.
\end{argument}

\noindent Voyons maintenant en faveur de quelles vertus le vrai Dieu, qui tient en ses mains tous les royaumes de la terre, a daigné favoriser l’accroissement de l’empire romain. C’est pour en venir là que nous avons montré, dans le livre précédent, que les dieux que Rome honorait par des jeux ridicules n’ont en rien contribué à sa grandeur ; nous avons montré ensuite, au commencement du présent livre, que le destin est un mot vide de sens, de peur que certains esprits, désabusés de la croyance aux faux dieux, n’attribuassent la conservation et la grandeur de l’empire romain à je ne sais quel destin plutôt qu’à la volonté toute-puissante du Dieu souverain.\par
Les anciens Romains adoraient, il est vrai, les faux dieux, et offraient des victimes aux démons, à l’exemple de tous les autres peuples de l’univers, le peuple hébreu excepté ; mais leurs historiens leur rendent ce témoignage qu’ils étaient « avides de renommée et prodigues d’argent, contents d’une fortune honnête et insatiables de gloire ». C’est la gloire qu’ils aimaient ; pour elle ils voulaient vivre, pour elle ils surent mourir. Cette passion étouffait dans leurs cœurs toutes les autres. Convaincus qu’il était honteux pour leur patrie d’être esclave, et glorieux pour elle de commander, ils la voulurent libre d’abord pour la faire ensuite souveraine. C’est pourquoi, ne pouvant souffrir l’autorité des rois, ils créèrent deux chefs annuels qu’ilsappelèrent consuls. Qui dit roi ou seigneur, parle d’un maître qui règne et domine ; un consul, au contraire, est une sorte de conseiller. Les Romains pensèrent donc que la royauté a un faste également éloigné de la simplicité d’un pouvoir qui exécute la loi, et de la douceur d’un magistrat qui conseille ; ils ne virent en elle qu’une orgueilleuse domination. Ils chassèrent donc les Tarquins, établirent des consuls, et dès lors, comme le rapporte à l’honneur des Romains l’historien déjà cité, « sous ce régime nouveau de liberté, la république, enflammée par un amour passionné de la gloire, s’accrut avec une rapidité incroyable ». C’est donc à cette ardeur de renommée et de gloire qu’il faut attribuer toutes les merveilles de l’ancienne Rome, qui sont, au jugement des hommes, ce qui peut se voir de plus glorieux et de plus digne d’admiration.\par
Salluste trouve aussi à louer quelques personnages de son siècle, notamment Marcus Caton et Caïus César, dont il dit que la république, depuis longtemps stérile, n’avait jamais produit deux hommes d’un mérite aussi éminent, quoique de mœurs bien différentes. Or, entre autres éloges qu’il adresse à César, il lui fait honneur d’avoir désiré un grand commandement, une armée et une guerre nouvelle où il pût montrer ce qu’il était. Ainsi, c’était le vœu des plus grands hommes que Bellone, armée de son fouet sanglant, excitât de malheureuses nations à prendre les armes, afin d’avoir une occasion de faire briller leurs talents. Et voilà les effets de cette ardeur avide pour les louanges et de ce grand amour de la gloire ! Concluons que les grandes choses faites par les Romains eurent trois mobiles : d’abord l’amour de la liberté, puis le désir de la domination et la passion des louanges. C’est de quoi rend témoignage le plus illustre de leurs poètes, quand il dit :\par
 {\itshape « Porsenna entourait Rome d’une armée immense, voulant lui imposer le retour des Tarquins bannis ; mais les fils d’Énée se précipitaient vers la mort pour défendre la liberté. »} \par
Telle était alors leur unique ambition : mourir vaillamment ou vivre libres. Mais quand ils eurent la liberté, l’amour de la gloire s’empara tellement de leurs âmes, que la liberté n’était rien pour eux si elle n’était accompagnée de la domination. Aussi accueillaient-ils avec la plus grande faveur ces prophéties flatteuses que Virgile mit depuis dans la bouche de Jupiter :\par
 {\itshape « Junon même, l’implacable Junon, qui fatigue aujourd’hui de sa haine jalouse la mer, la terre et le ciel, prendra des sentiments plus doux et protègera, de concert avec moi, la nation qui porte ta toge, devenue la maîtresse des autres nations, Telle est ma volonté ; un jour viendra où la maison d’Assaracus imposera son joug à la Thessalie et à l’illustre Mycènes, et dominera sur les Grecs vaincus. »} \par
On remarquera que Virgile fait prédire à Jupiter des événements accomplis de son temps et dont lui-même était témoin ; mais j’ai cité ses vers pour montrer que les Romains, après la liberté, ont tellement estimé la domination, qu’ils en ont fait le sujet de leurs plus hautes louanges. C’est encore ainsi que le même poète préfère à tous les arts des nations étrangères l’art propre aux Romains, celui de régner et de gouverner, de vaincre et de soumettre les peuples :\par
 {\itshape « D’autres, dit-il, animeront l’airain d’un ciseau plus délicat, je le crois sans peine ; ils sauront tirer du marbre des figures pleines de vie. Leur parole sera plus éloquente ; leur compas décrira les mouvements célestes et marquera le lever des étoiles. Toi, Romain, souviens-toi de soumettre les peuples à ton empire. Tes arts, les voici : être l’arbitre de la paix, pardonner aux vaincus et dompter les superbes. »} \par
Les Romains, en effet, excellaient d’autant mieux dans ces arts qu’ils étaient moins adonnés aux voluptés qui énervent l’âme et le corps, et à ces richesses fatales aux bonnes mœurs qu’on ravit à des citoyens pauvres pour les prodiguer à d’infâmes histrions. Et comme cette corruption débordait de toutes parts au temps où Salluste écrivait et où chantait Virgile, on ne marchait plus vers la gloire par des voies honnêtes, mais par la fraude et l’artifice. Salluste nous le déclare expressément : « Ce fut d’abord l’ambition, dit-il, plutôt que la cupidité, qui remua les cœurs. Or, le premier de ces vices touche de plus près que l’autre à la vertu. En effet, l’homme de bien et le lâche désirent également la gloire, les honneurs, le pouvoir ; seulement l’homme de bien y marche par la bonne voie ; l’autre, à qui manquent les moyens honnêtes, prétend y arriver par la fraude et le mensonge. » Quels sont ces moyens honnêtes de parvenir à la gloire, aux dignités, au pouvoir ? évidemment ils résident dans lavertu, seule voie où veuillent marcher les gens de bien. Voilà les sentiments qui étaient naturellement gravés dans le cœur des Romains, et je n’en veux pour preuve que ces temples qu’ils avaient élevés, l’un près de l’autre, à la Vertu et à l’Honneur, s’imaginant que ces dons de Dieu étaient des dieux. Rapprocher ces deux divinités de la sorte, c’était assez dire qu’à leurs yeux l’honneur était la véritable fin de la vertu ; c’est à l’honneur, en effet, que tendaient les hommes de bien, et toute la différence entre eux et les méchants, c’est que ceux-ci prétendaient arriver à leurs fins par des moyens déshonnêtes, par le mensonge et les tromperies.\par
Salluste a donné à Caton un plus bel éloge, quand il a dit de lui : « Moins il courait à la gloire, et plus elle venait à lui. » Qu’est-ce en effet que la gloire, dont les anciens Romains étaient si fortement épris, sinon la bonne opinion des hommes ? Or, au-dessus de la gloire il y a la vertu, qui ne se contente pas du bon témoignage des hommes, mais qui veut avant tout celui de la conscience. C’est pourquoi l’Apôtre a dit : « Notre gloire, à nous, c’est le témoignage de notre conscience. » Et ailleurs : « Que chacun examine ses propres œuvres, et alors il trouvera sa gloire en lui-même et non dans les autres. » Ce n’est donc pas à la vertu à courir après la gloire, les honneurs, le pouvoir, tous ces biens, en un mot, que les Romains ambitionnaient et que les gens de bien recherchaient par des moyens honnêtes ; c’est à ces biens, au contraire, à venir vers la vertu ; car la vertu véritable est celle qui se propose le bien pour objet, et ne met rien au-dessus. Ainsi, Caton eut tort de demander des honneurs à la république ; c’était à la république à les lui conférer, à cause de sa vertu, sans qu’il les eût sollicités.\par
Et toutefois, de ces deux grands contemporains, Caton et César, Caton est incontestablement celui dont la vertu approche le plus de la vérité. Voyez, en effet, ce qu’était alors la république et ce qu’elle avait été autrefois, au jugement de Caton lui-même : « Gardez-vous de croire, dit-il, que ce soit par les armes que nos ancêtres ont élevé la république, alors si petite, à un si haut point de grandeur. S’il en était ainsi, elle serait aujourd’hui plus florissante encore, puisque,citoyens, alliés, armes, chevaux, nous avons tout en plus grande abondance que nos pères. Mais il est d’autres moyens qui firent leur grandeur, et que nous n’avons plus : au dedans, l’activité ; au dehors, une administration juste ; dans les délibérations, une âme libre, affranchie des vices et des passions. Au lieu de ces vertus, nous avons le luxe et l’avarice ; l’État est pauvre, et les particuliers sont opulents ; nous vantons la richesse, nous chérissons l’oisiveté ; entre les bons et les méchants, nulle différence, et toutes les récompenses de la vertu sont le prix de l’intrigue. Pourquoi s’en étonner, puisque chacun de vous ne pense qu’à soi ; esclave, chez soi, de la volupté, et au dehors, de l’argent et de la faveur ? Et voilà pourquoi on se jette sur la république comme sur une proie sans défense. »\par
Quand on entend Caton ou Salluste parler de la sorte, on est tenté de croire que tous les anciens Romains, ou du moins la plupart, étaient semblables au portrait qu’ils en tracent avec tant d’admiration ; mais il n’en est rien ; autrement il faudrait récuser le témoignage du même Salluste dans un autre endroit de son ouvrage, que j’ai déjà eu occasion de citer : « Dès la naissance de Rome, dit-il, les injustices des grands amenèrent la séparation du peuple et du sénat, et une suite de dissensions intérieures ; on ne vit fleurir l’équité et la modération qu’à l’époque de l’expulsion des rois, et tant qu’on eut à redouter les Tarquins et la guerre contre l’Étrurie ; mais le danger passé, les patriciens traitèrent les gens du peuple comme des esclaves, accablant celui-ci de coups, chassant celui-là de son champ, gouvernant en maîtres et en rois… Les luttes et les animosités ne prirent fin qu’à la seconde guerre punique, parce qu’alors la terreur s’empara de nouveau des âmes, et, détournant ailleurs leurs pensées et leurs soucis, calma et soumit ces esprits inquiets. » Mais à cette époque même, les grandes choses qui s’accomplissaient étaient l’ouvrage d’un petit nombre d’hommes, vertueux à leur manière, et dont la sagesse, au milieu de ces désordres par eux tolérés, mais adoucis, faisait fleurir la république. C’est ce qu’atteste le mêmehistorien, quand il dit que, voulait comprendre comment le peuple romain avait accompli de si grandes choses, soit en paix, soit en guerre, sur terre et sur mer, souvent avec une poignée d’hommes contre des armées redoutables et des rois très puissants, il avait remarqué qu’il ne fallait attribuer ces magnifiques résultats qu’à la vertu d’un petit nombre de citoyens, laquelle avait donné la victoire à la pauvreté sur la richesse, et aux petites armées sur les grandes. « Mais depuis que Rome, ajoute Salluste, eut été corrompue par le luxe et l’oisiveté, ce fut le tour de la république de soutenir par sa grandeur les vices de ses généraux et de ses magistrats. » Ainsi donc, lorsque Caton célébrait les anciens Romains qui allaient à la gloire, aux honneurs, au pouvoir, par la bonne voie, c’est-à-dire par la vertu, c’est à un bien petit nombre d’hommes que s’adressaient ses éloges ; ils étaient bien rares ceux qui, par leur vie laborieuse et modeste, enrichissaient le trésor public tout en restant pauvres. Et c’est pourquoi la corruption des mœurs amena une situation toute contraire : l’État pauvre et les particuliers opulents.
\subsection[{Chapitre XIII}]{Chapitre XIII}

\begin{argument}\noindent L’amour de la gloire, qui est un vice, passe pour une vertu, parce qu’il surmonte des vices plus grands.
\end{argument}

\noindent Après que les royaumes d’Orient eurent brillé sur la terre pendant une longue suite d’années, Dieu voulut que l’empire d’Occident, qui était le dernier dans l’ordre des temps, devînt le premier de tous par sa grandeur et son étendue ; et comme il avait dessein de se servir de cet empire pour châtier un grand nombre de nations, il le confia à des hommes passionnés pour la louange et l’honneur, qui mettaient leur gloire dans celle de la patrie, et étaient toujours prêts à se sacrifier pour son salut, triomphant ainsi de leur cupidité et de tous leurs autres vices par ce vice unique : l’amour de la gloire. Car, il ne faut pas se le dissimuler, l’amour de la gloire est un vice. Horace en est convenu, quand il a dit :\par
 {\itshape « L’amour de la gloire enfle-t-il votre cœur ? il y a un remède pour ce mal : c’est de lire un bon livre avec candeur et par trois fois. »} \par
Écoutez encore ce poète s’élevant dans un de ses chants lyriques contre la passion de dominer :\par
 {\itshape « Dompte ton âme ambitieuse, et tu feras ainsi un plus grand empire que si, réunissant à la Libye la lointaine Gadès, tu soumettais à ton joug les deux Carthages ! »} \par
Et cependant, quand, on n’a pas reçu du Saint-Esprit la grâce de surmonter les passions honteuses par la foi, la piété et l’amour de la beauté intelligible, mieux vaut encore les vaincre par un désir de gloire purement humain que de s’y abandonner ; car si ce désir ne rend pas l’homme saint, il l’empêche de devenir infâme. C’est pourquoi Cicéron, dans son ouvrage {\itshape De la République}, où il traite de l’éducation du chef de l’État, dit qu’il faut le nourrir de gloire, et s’autorise, pour le prouver, des souvenirs de ses ancêtres, à qui l’amour de la gloire inspira tant d’actions illustres et merveilleuses. Il est donc avéré que les Romains, loin de résister à ce vice, croyaient devoir l’exciter et le développer dans l’intérêt de la république. Aussi bien Cicéron, jusque dans ses livres de philosophie, ne dissimule pas combien ce poison de la gloire lui est doux. Ses aveux sont plus clairs que le jour ; car, tout en célébrant ces hautes études où l’on se propose pour but le vrai bien, et non la vaine gloire, il ne laisse pas d’établir cette maxime générale : « L’honneur est l’aliment des arts ; c’est par amour de la gloire que nous embrassons avec ardeur les études, et toute science discréditée dans l’opinion languit et s’éteint. »
\subsection[{Chapitre XIV}]{Chapitre XIV}

\begin{argument}\noindent Il faut étouffer l’amour de la gloire temporelle, la gloire des justes étant toute en Dieu.
\end{argument}

\noindent Il vaut donc mieux, n’en doutons point, résister à cette passion que s’y abandonner ; car on est d’autant plus semblable à Dieu qu’on est plus pur de cette impureté. Je conviens qu’en cette vie il n’est pas possible de la déraciner entièrement du cœur de l’homme, les plus vertueux ne cessant jamais d’en être tentés ; mais efforçons-nous au moins de la surmonter par l’amour de la justice, et si l’on voit languir et s’éteindre, parce qu’elles sont discréditées dans l’opinion, des choses bonnes et solides en elles-mêmes, que l’amour de la gloire humaine en rougisse et qu’il cède à l’amour de la vérité. Une preuve que ce vice est ennemi de la vraie foi, quand il vient à l’emporter dans notre cœur sur la crainte ou sur l’amour de Dieu, c’est que Notre-Seigneur dit dans l’Évangile : « Comment pouvez-vous avoir la foi, vous qui attendez la gloire les uns des autres, et ne recherchez point la gloire qui vient de Dieu seul ? » L’Évangéliste dit encore de certaines personnes qui croyaient en Jésus-Christ, mais qui appréhendaient de confesser publiquement leur foi : « Ils ont plus aimé la gloire des hommes que celle de Dieu. » Telle ne fut pas la conduite des bienheureux Apôtres ; car ils prêchaient le christianisme en des lieux où non seulement il était en discrédit et ne pouvait, par conséquent, selon le mot de Cicéron, rencontrer qu’une sympathie languissante, mais où il était un objet de haine ; ils se souvinrent donc de cette parole du bon Maître, du Médecin des âmes : « Si quelqu’un me renonce devant les hommes, je le renoncerai devant mon Père qui est dans les cieux, et devant les anges de Dieu. » En vain les malédictions et les opprobres s’élevèrent de toutes parts ; les persécutions les plus terribles, les supplices les plus cruels ne purent les détourner de prêcher la doctrine du salut à la face de l’orgueil humain frémissant. Et quand par leurs actions, leurs paroles et toute leur vie vraiment divine, par leur victoire sur des cœurs endurcis, où ils faisaient pénétrer la justice et la paix, ils eurent acquis dans l’Église du Christ une immense gloire, loin de s’y reposer comme dans la fin de leur vertu, ils la rapportèrent à Dieu, dont la grâce les avait rendus forts et victorieux. C’est à ce foyer qu’ils allumaient l’amour de leurs disciples, les tournant sans cesse vers le seul être capable de les rendre dignes de marcher un jour sur leur trace, et d’aimer le bien sans souci de la vaine gloire, suivant cet enseignement du Maître : « Prenez garde de faire le bien devant les hommes pour être regardés ; autrement vous ne recevrez point de récompense de votre Père qui est dans les cieux. »\par
D’un autre côté de peur que ses disciples n’entendissent mal sa pensée, et que leur vertu perdît de ses fruits en se dérobant aux regards, il leur explique à quelle fin ils doivent laisservoir leurs œuvres : « Que vos actions, dit-il, brillent devant les hommes, afin qu’en les voyant ils glorifient votre Père qui est dans les cieux. » Comme s’il disait : Faites le bien, non pour que les hommes vous voient, non pour qu’ils s’attachent à vous, puisque par vous-mêmes vous n’êtes rien, mais pour qu’ils glorifient votre Père qui est dans les cieux, et que, s’attachant à lui, ils deviennent ce que vous êtes. Voilà le précepte dont se sont inspirés tous ces martyrs qui ont surpassé les Scévola, les Curtius et les Décius, non moins par leur nombre que par leur vertu ; vertu vraiment solide, puisqu’elle était fondée sur la vraie piété, et qui consistait, non à se donner la mort, mais à savoir la souffrir. Quant à ces Romains, enfants d’une cité terrestre, comme ils ne se proposaient d’autre fin de leur dévouement pour elle que sa conservation et sa grandeur, non dans le ciel, mais sur la ferre, non dans la vie éternelle, mais sur ce théâtre mobile du monde, où les morts sont remplacés par les mourants, qu’aimaient-ils, après tout, sinon la gloire qui devait les faire vivre, même après leur mort, dans le souvenir de leurs admirateurs ?
\subsection[{Chapitre XV}]{Chapitre XV}

\begin{argument}\noindent De la récompense temporelle que Dieu a donnée aux vertus des Romains.
\end{argument}

\noindent Si donc Dieu, qui ne leur réservait pas une place dans sa cité céleste à côté de ses saints anges, parce qu’il ne les donne qu’à la piété véritable, à celle qui rend à Dieu seul, pour parler comme les Grecs, un culte de {\itshape latrie}, si Dieu, dis-je, ne leur eût pas donné la gloire passagère d’un empire florissant, les vertus qu’ils ont déployées afin de parvenir à cette gloire seraient restées sans récompense ; car c’est en parlant de ceux qui font un peu de bien pour être estimés des hommes, que le Seigneur a dit : « Je vous dis en vérité qu’ils ont reçu leur récompense. » Ainsi il est vrai que les Romains ont immolé leurs intérêts particuliers à l’intérêt commun, c’est-à-dire à la chose publique, qu’ils ont surmonté la cupidité, préférant accroître le trésor de l’État que leur propre trésor, qu’ils ont porté dans les conseils de la patrie une âme libre, soumise aux lois, affranchie du joug des vices et des passions ; et toutes ces vertus étaient pour eux le droit chemin pour aller à l’honneur, au pouvoir, à la gloire. Or, ils ont été honorés parmi presque toutes les nations ; ils ont imposé leur pouvoir à un très grand nombre, et dans tout l’univers, les poètes et les historiens ont célébré leur gloire ; ils n’ont donc pas sujet de se plaindre de la justice du vrai Dieu : {\itshape ils ont reçu leur récompense}.
\subsection[{Chapitre XVI}]{Chapitre XVI}

\begin{argument}\noindent De la récompense des citoyens de la Cité éternelle, à qui peut être utile l’exemple des vertus des Romains.
\end{argument}

\noindent Mais il n’en est pas de même de la récompense de ceux qui souffrent ici-bas pour la Cité de Dieu, objet de haine à ceux qui aiment le monde. Cette Cité est éternelle ; personne n’y prend naissance, parce que personne n’y meurt ; là règne la véritable et parfaite félicité, qui n’est point une déesse, mais un don de Dieu. C’est de là que nous avons reçu le gage de la foi, nous qui passons le temps de notre pèlerinage à soupirer pour la beauté de ce divin séjour. Là, le soleil ne se lève point sur les bons et sur les méchants, mais le Soleil de justice n’y éclaire que les bons. Là, on ne sera point en peine d’enrichir le trésor public aux dépens de sa fortune privée, parce qu’il n’y a qu’un trésor de vérité commun à tous. Aussi ce n’a pas été seulement pour récompenser les Romains de leurs vertus que leur empire a été porté à un si haut point de grandeur et de gloire, mais aussi pour servir d’exemple aux citoyens de cette Cité éternelle et leur faire comprendre combien ils doivent aimer la céleste patrie en vue de la vie éternelle, puisqu’une patrie terrestre a été, pour une gloire tout humaine, tant aimée de ses enfants.
\subsection[{Chapitre XVII}]{Chapitre XVII}

\begin{argument}\noindent Les victoires des Romains ne leur ont pas fait une condition meilleure que celle des vaincus.
\end{argument}

\noindent Pour ce qui est de cette vie mortelle qui dure si peu, qu’importe à l’homme qui doit mourir d’avoir tel ou tel souverain, pourvu qu’on n’exige de lui rien de contraire à la justice et à l’honneur ? Les Romains ont-ils porté dommage aux peuples conquis autrement que par les guerres cruelles et si sanglantes qui ont précédé la conquête ? Certes, si leur domination eût été acceptée sans combat, le succès eût été meilleur, mais il eût manqué aux Romains la gloire du triomphe. Aussi bien ne vivaient-ils pas eux-mêmes sous les lois qu’ils imposaient aux autres ? Si donc cette conformité de régime s’était établie d’un commun accord, sans l’entremise de Mars et de Bellone, personne n’étant le vainqueur où il n’y a pas de combat, n’est-il pas clair que la condition des Romains et celle des autres peuples eût été absolument la même, surtout si Rome eût fait d’abord ce que l’humanité lui conseilla plus tard, je veux dire si elle eût donné le droit de cité à tous les peuples de l’empire, et étendu ainsi à tous un avantage qui n’était accordé auparavant qu’à un petit nombre, n’y mettant d’ailleurs d’autre condition que de contribuer à la subsistance de ceux qui n’auraient pas de terres ; et, au surplus, mieux valait infiniment payer ce tribut alimentaire entre les mains de magistrats intègres, que de subir les extorsions dont on accable les vaincus.\par
J’ai beau faire, je ne puis voir en quoi les bonnes mœurs, la sûreté des citoyens et leurs dignités même étaient intéressées à ce que tel peuple fût vainqueur et tel autre vaincu : il n’y avait là pour les Romains d’autre avantage que le vain éclat d’une gloire tout humaine, et voilà pourquoi cette gloire a été donnée comme récompense à ceux qui en étaient passionnément épris, et qui, pour l’obtenir, ont livré tant de furieux combats. Car enfin leurs terres ne paient-elles pas aussi tribut ? leur est-il permis d’acquérir des connaissances que les autres ne puissent acquérir comme eux ? n’y a-t-il pas plusieurs sénateurs dans les provinces qui ne connaissent pas Rome seulement de vue ? Ôtez le faste extérieur, que sont les hommes, sinon des hommes ? Quand même la perversité permettrait que les plus gens de bien fussent les plus considérés, devrait-on faire un si grand état de l’honneur humain, qui n’est en définitive qu’une légère fumée ? Mais profitons même en ceci des bienfaits du Seigneur notre Dieu : considérons combien de plaisirs ont méprisés, combien de souffrances ont supportées, combien de passions ont étouffées, en vue de la gloire humaine, ceux qui ont mérité de la recevoir comme récompense de telles vertus, et que ce spectacle serve à nous humilier. Puisque cette Cité, où il nous est promis que nous régnerons un jour, est autant au-dessus de la cité d’ici-bas que le ciel est au-dessus de la terre, la joie de la vie éternelle au-dessus des joies passagères, la solide gloire au-dessus des vaines louanges, la société des anges au-dessus de celle des mortels, la lumière enfin du Créateur des astres au-dessus de l’éclat de la lune et du soleil, comment les citoyens futurs d’une s-i noble patrie, pour avoir fait un peu de bien ou supporté un peu de mal à son service, croiraient-ils avoir beaucoup travaillé à se rendre dignes d’y habiter un jour, quand nous voyons que les Romains ont tant fait et tant souffert pour une patrie terrestre dont ils étaient déjà membres et possesseurs ? Et pour achever cette comparaison des deux cités, cet asile où Romulus réunit par la promesse de l’impunité tant de criminels, devenus les fondateurs de Rome, n’est-il point la figure de la rémission des péchés, qui réunit en un corps tous les citoyens de la céleste patrie ?
\subsection[{Chapitre XVIII}]{Chapitre XVIII}

\begin{argument}\noindent Les chrétiens n’ont pas à se glorifier de ce qu’ils font pour l’amour de la patrie céleste, quand les Romains ont fait de si grandes choses pour une patrie terrestre et pour une gloire tout humaine.
\end{argument}

\noindent Qu’y a-t-il donc de si grand à mépriser tous les charmes les plus séduisants de la vie présente pour cette patrie éternelle et céleste, quand pour une patrie terrestre et temporelle Brutus a pu se résoudre à faire mourir ses enfants, sacrifice que la divine patrie n’exige pas ? Il est sans doute bien plus difficile d’immoler ses enfants que de faire ce qu’elle exige, je veux dire de donner aux pauvres ou d’abandonner pour la foi ou pour la justice des biens qu’on n’amasse et qu’on ne conserve que pour ses enfants. Car ce ne sont pas les richesses de la terre qui nous rendent heureux, nous et nos enfants, puisque nous pouvons les perdre durant notre vie ou les laisser après notre mort en des mains inconnues ou détestées ; mais Dieu, qui est la vraie richesse des âmes, est aussi le seul qui puisse leur donner le bonheur. Brutus a-t-il été heureux ? Non, et j’en atteste le poète même qui célèbre son sacrifice :\par
 {\itshape « Ce père, dit-il, enverra au supplice des fils séditieux au nom de la liberté sainte. Malheureux, quelque jugement que porte sur lui la postérité ! »} \par
Et il ajoute pour le consoler :\par
 {\itshape « Mais l’amour de la patrie est plus fort, et la tendresse paternelle cède à un immense désir de la gloire. »} \par
C’est cet amour de la patrie et ce désir de la gloire qui ont inspiré aux Romains tout ce qu’ils ont fait de merveilleux. Si donc, pour la liberté de quelques hommes qui mourront demain, et pour une gloire terrestre, un père a pu sacrifier ses propres enfants, est-ce beaucoup faire pour gagner la liberté véritable, qui nous affranchit du péché, de la mort et du démon, et pour contenter, non pas notre vanité, mais notre charité, par la délivrance de nos semblables, captifs, non de Tarquin, mais des démons et de leur roi, est-ce beaucoup faire, encore une fois, je ne dis pas de faire mourir nos enfants, mais de mettre au nombre de nos enfants les pauvres de Jésus-Christ ?\par
On rapporte que Torquatus, général romain, punit de mort son fils victorieux, que l’ardeur de la jeunesse avait emporté à combattre, malgré l’ordre du chef, un ennemi qui le provoquait. Torquatus jugea sans doute que l’exemple de son autorité méprisée pouvait causer plus de mal que ne ferait de bien la victoire obtenue sur l’ennemi ; mais si un père a pu s’imposer une si dure loi, de quoi ont à se glorifier ceux qui, pour obéir aux lois de la céleste patrie, méprisent les biens de la terre, moins chers à leur cœur que des enfants ? Si Camille, après avoir délivré sa patrie des redoutables attaques des Véïens, ne laissa pas, quoiqu’elle l’eût sacrifié à ses envieux, de la sauver encore en repoussant les Gaulois, faute de trouver une autre patrie où il pût vivre avec gloire, pourquoi celui-là se vanterait-il, qui, ayant reçu dans l’Église la plus cruelle injure de la part de charnels ennemis, loin de se jeter parmi les hérétiques ou de former une hérésie nouvelle, aurait défendu de tout son pouvoir la pureté de la doctrine de l’Église contre les efforts de l’hérésie, pourquoi se vanterait-il, puisqu’il n’y apas d’autre Église où l’on puisse, je ne dis pas jouir de la gloire des hommes, mais acquérir la vie éternelle ? Si Mucius Scévola, trompé dans son dessein de tuer Porsenna qui assiégeait étroitement Rome, étendit la main sur un brasier ardent en présence de ce prince, l’assurant qu’il y avait encore plusieurs jeunes Romains aussi hardis que lui qui avaient juré sa mort, en sorte que Porsenna, frappé de son courage et effrayé d’une conjuration si terrible, conclut sans retard la paix avec les Romains, qui croira avoir mérité le royaume des cieux, quand, pour l’obtenir, il aura abandonné sa main, je dis plus, tout son corps aux flammes des persécuteurs ? Si Curtius se précipita tout armé avec son cheval dans un abîme, pour obéir à l’oracle qui avait commandé aux Romains d’y jeter ce qu’ils avaient de meilleur (les Romains, qui excellaient surtout par leurs guerriers et par leurs armes, ne croyaient rien avoir de meilleur qu’un guerrier armé), qui s’imaginera avoir fait quelque chose de grand en vue de la Cité céleste, pour avoir souffert, sans la prévenir, une semblable mort, quand surtout il a reçu de son Seigneur, du Roi de sa véritable patrie, cet oracle bien plus certain : « Ne craignez point ceux qui tuent le corps, mais qui ne peuvent tuer l’âme. » Si les Décius, se consacrant à la mort par de certaines paroles, ont versé leur sang pour apaiser les dieux irrités et sauver l’armée romaine, que les saints martyrs ne croient pas que pour avoir, eux aussi, répandu leur sang, ils aient rien fait qui soit digne du séjour de la véritable et éternelle félicité, alors même que soutenus par la charité de la foi et par la foi de la charité, ils auraient aimé non seulement leurs frères pour qui coulait leur sang, mais leurs ennemis mêmes qui le faisaient couler. Si Marcus Pulvillus, dédiant un temple à Jupiter, à Junon et à Minerve, se montra insensible à la fausse nouvelle de la mort de son fils, que ses ennemis lui portèrent pour qu’il quittât la cérémonie et en laissât à son collègue tout l’honneur ; si même il commanda que le corps de son fils fût jeté sans sépulture, faisant céder la douleur paternelleà l’amour de la gloire, osera-t-on prétendre avoir fait quelque chose de considérable pour la prédication de l’Évangile, qui délivre les hommes de mille erreurs pour les ramener vers la patrie véritable, par cela seul qu’on se sera conformé à cette parole du Seigneur, disant à un de ses disciples préoccupé d’ensevelir son père : « Suis-moi, et laisse les morts ensevelir leurs morts. » Si Régulus, pour ne pas manquer de parole à de cruels ennemis, retourna parmi eux, ne pouvant plus, disait-il, vivre à Rome avec honneur, après avoir été esclave des Africains ; s’il expia par les plus horribles supplices le conseil qu’il avait donné au sénat de repousser les offres de Carthage, quels tourments le chrétien ne doit-il pas mépriser pour garder sa foi envers cette patrie dont l’heureuse possession est le prix de cette foi même ? Et rendra-t-il au Seigneur tout ce qu’il lui doit en retour des biens qu’il en a reçus, s’il souffre, pour garder sa foi envers son bienfaiteur, ce que Régulus souffrit pour garder la sienne envers des ennemis impitoyables ? Comment osera-t-il s’enorgueillir d’avoir embrassé la Pauvreté afin de marcher d’un pas plus libre dans la voie qui mène à la patrie dont Dieufait toute la richesse, quand il peut savoir que L. Valérius, mort consul, était si pauvre que le peuple dut contribuer aux frais de ses funérailles ; que Quintus Cincinnatus, dont la fortune se bornait à quatre arpents de terre qu’il cultivait lui-même, fut tiré de la charrue pour être fait dictateur, et qu’après avoir vaincu les ennemis et s’être couvert d’une gloire immortelle, il resta pauvre comme auparavant ? Ou qui croira avoir fait preuve d’une grande vertu en ne se laissant pas entraîner par l’attrait des biens de ce monde loin de la patrie bienheureuse, lorsqu’il voit Fabricius rejeter toutes les offres de Pyrrhus, roi d’Épire, même le quart de son royaume, pour ne pas quitter Rome et y rester pauvre et simple citoyen ? En effet, au temps où la république était opulente, où florissait vraiment la chose publique, la chose du peuple, la chose de tous, les particuliers étaient sipauvres, qu’un personnage, qui avait été deux fois consul, fut chassé du sénat par le censeur, parce qu’il avait dans sa maison dix marcs de vaisselle d’argent. Or, si telle était la pauvreté de ces hommes dont les victoires enrichissaient le trésor public, les chrétiens qui mettent leurs biens en commun pour une fin tout autrement excellente, c’est-à-dire pour se conformer à ce qui est écrit dans les Actes des Apôtres : « Qu’il soit distribué à chacun selon ses besoins, et que nul ne possède rien en propre, mais que tout soit commun entre tous les fidèles » ; les chrétiens, dis-je, doivent comprendre qu’ils n’ont aucun sujet de se glorifier de ce qu’ils font pour être admis dans la compagnie des anges, quand ces idolâtres en ont fait presque autant pour conserver la gloire du nom romain.\par
Il est assez clair que tous ces traits de grandeur et beaucoup d’autres, qui se rencontrent dans les annales de Rome, ne seraient point parvenus à un tel renom, si l’empire romain n’avait pris de prodigieux accroissements ; d’où l’on voit que cette domination si étendue, si persistante, illustrée par les vertus de si grands hommes, a eu deux principaux effets : elle a été pour les Romains amoureux de la gloire, la récompense où ils aspiraient, et puis elle nous offre, dans le spectacle de leurs grandes actions, un exemple qui nous avertit de notre devoir, afin que si nous ne pratiquons pas pour la glorieuse Cité de Dieu les vertus véritables dont les Romains n’embrassaient que l’image en travaillant à la gloire d’une cité de la terre, nous en ayons de la confusion, et que, si nous les pratiquons, nous n’en ayons pas de vanité. Car nous apprenons de l’Apôtre « que les souffrances de cette vie n’ont point de proportion avec la gloire future qui sera manifestée en nous ». Quant à la gloire humaine et temporelle, la vertu des Romains y était proportionnée. Aussi, quand le Nouveau Testament, déchirant le voile de l’Ancien, est venu nous apprendre que le Dieu unique et véritable veut être adoré, non point en vue des biens terrestres et temporels que la Providence accorde également aux bons et aux méchants, mais en vue de la vie éternelle et des biensimpérissables de la Cité d’en haut, nous avons vu les Juifs justement livrés à l’empire romain pour servir de trophée à sa gloire : c’est que Dieu a voulu que ceux qui avaient recherché et conquis par leurs vertus, quoique purement humaines, la gloire des hommes, soumissent à leur joug une nation criminelle qui avait rejeté et mis à mort le Dispensateur de la véritable gloire, le Roi de l’éternelle Cité.
\subsection[{Chapitre XIX}]{Chapitre XIX}

\begin{argument}\noindent En quoi l’amour de la gloire diffère de l’amour de la domination.
\end{argument}

\noindent Il y a certainement de la différence entre l’amour de la gloire et l’amour de la domination ; car bien que l’amour immodéré de la gloire conduise à la passion de dominer, ceux qui aiment ce qu’il y a de plus solide dans les louanges des hommes n’ont garde de déplaire aux bons esprits. Parmi les vertus, en effet, il en est plusieurs dont beaucoup d’hommes sont bons juges, quoiqu’elles soient pratiquées par un petit nombre, et c’est par là que marchent à la gloire et à la domination ceux dont Salluste dit qu’ils suivent la bonne voie. Au contraire, quiconque désire la domination sans avoir cet amour de la gloire qui fait qu’on craint de déplaire aux bons esprits, aucun moyen ne lui répugne, pas même les crimes les plus scandaleux, pour contenter sa passion. Tout au moins celui qui aime la gloire, s’il ne prend pas la bonne voie, se sert de ruses et d’artifices pour paraître ce qu’il n’est pas. Aussi est-ce à un homme vertueux une grande vertu de mépriser la gloire, puisque Dieu seul en est le témoin et que les hommes n’en savent rien. Et, en effet, quoi qu’on fasse devant les hommes pour leur persuader qu’on méprise la gloire, on ne peut guère les empêcher de soupçonner que ce mépris ne cache le désir d’une gloire plus grande. Mais celui qui méprise en réalité les louanges des hommes, méprise aussi leurs soupçons téméraires, sans aller toutefois, s’il est vraiment homme de bien, jusqu’à mépriser leur salut ; car la vertu véritable, qui vient du Saint-Esprit, porte le véritable juste à aimer même ses ennemis, à les aimer jusqu’au point de les voir avec joie devenir, en se corrigeant, ses compagnons de félicité, non dans la patrie d’ici-bas, mais dans celle d’en haut. Et quant à ceux qui le louent, bien qu’il soit insensible à leurs louanges, il ne l’est pas à leur affection ; aussi, ne voulant pas être au-dessous de leur estime, de crainte d’être au-dessous de leur affection, il s’efforce de tourner leurs louanges vers l’Être souverain de qui nous tenons tout ce qui mérite en nous d’être loué. Quant à celui qui, sans être sensible à la gloire, désire ardemment la domination, il est plus cruel et plus brutal que les bêtes. Il s’est rencontré chez les Romains quelques hommes de cette espèce, indifférents à l’estime et toutefois très avides de dominer. Parmi ceux dont l’histoire fait mention, l’empereur Néron mérite incontestablement le premier rang. Il était si amolli par la débauche qu’on n’aurait redouté de lui rien de viril, et si cruel qu’on n’aurait rien soupçonné en lui d’efféminé, si on ne l’eût connu. Et pourtant la puissance souveraine n’est donnée à de tels hommes que par la providence de Dieu, quand il juge que les peuples méritent de tels maîtres. Sa parole est claire sur ce point ; c’est la sagesse même qui parle ainsi : « C’est moi qui fais régner les rois et dominer les tyrans. » Et afin qu’on n’entende pas ici {\itshape tyran} dans le sens de roi puissant, selon l’ancienne acception du mot, adoptée par Virgile dans ce vers :\par
{\itshape « Ce sera pour moi un gage de paix d’avoir touché la droite du tyran des Troyens »}, \par
il est dit clairement de Dieu en un autre endroit : « C’est lui qui fait régner les princes fourbes, à cause des péchés du peuple. » Ainsi, bien que j’aie assez établi, selon mes forces, pourquoi le seul Dieu véritable et juste a aidé les Romains à fonder un si grand empire, en récompense de ce que le monde appelle leurs vertus, il se peut toutefois qu’il y ait une raison plus cachée de leur prospérité ; car Dieu sait ce que méritent les peuples et nous l’ignorons. Mais il n’importe, pourvu qu’il demeure constant pour tout homme pieux qu’il n’y a pas de véritable vertu sans une véritable piété, c’est-à-dire sans le vrai culte du vrai Dieu, et que c’est une vertu fausse que celle qui a pour fin la gloire humaine ; bien toutefois que ceux qui ne sont pas citoyens de la Cité éternelle, nommée dans l’Écriture la Cité de Dieu, le soient plus utiles à la cité du monde par cette vertu, quoique fausse, que s’ils n’avaient aucune vertu. Que s’il vient à se trouver des hommes vraiment pieux qui joignent à la vertu la science de gouverner les peuples, rien ne peut arriver de plus heureux aux hommes que de recevoir de Dieu de tels souverains. Aussi bien ces princes d’élite, si grands que soient leurs mérites, ne les attribuent qu’à la grâce de Dieu, qui les a accordés à leur foi et à leurs prières, et ils savent reconnaître combien ils sont éloignés de la perfection des saints anges, à qui ils désirent ardemment d’être associés. Quant à cette vertu, séparée de la vraie piété, et qui a pour fin la gloire des hommes, quelques louanges qu’on lui donne, elle ne mérite seulement pas d’être comparée aux faibles commencements des fidèles qui mettent leur espérance dans la grâce et la miséricorde du vrai Dieu.
\subsection[{Chapitre XX}]{Chapitre XX}

\begin{argument}\noindent Il n’est guère moins honteux d’asservir les vertus à la gloire humaine qu’à la volupté.
\end{argument}

\noindent Des philosophes qui font consister le souverain bien dans la vertu ont coutume, pour faire honte à ceux qui, tout en estimant la vertu, la subordonnent néanmoins à la volupté comme à sa fin, de représenter celle-ci comme une reine délicate assise sur un trône et servie par les vertus qui observent tous ses mouvements et exécutent ses ordres. Elle commande à la Prudence de veiller au repos et à la sûreté de son empire ; à la Justice de répandre des bienfaits pour lui faire des amis utiles, et de ne nuire à personne pour éviter des révoltes ennemies de sa sécurité. Si elle vient à éprouver dans son corps quelque douleur, pas toutefois assez violente pour l’obliger à se délivrer de la vie, elle ordonne à la Force de tenir sa souveraine recueillie au fond de son âme, afin que le souvenir des plaisirs passés adoucisse l’amertume de la douleur présente ; enfin elle recommande à la Tempérance de ne pas abuser de la table, de peur que la santé, qui est un des éléments les plus essentiels du bonheur, n’en soit gravement altérée. Voilà donc les Vertus, avec touteleur gloire et toute leur dignité, servant la Volupté comme une femmelette impérieuse et impudente. Rien de plus scandaleux que ce tableau, disent nos philosophes, rien de plus laid, rien enfin dont la vue soit moins supportable aux gens de bien, et ils disent vrai mais, à mon tour, j’estime impossible de faire un tableau décent où les vertus soient au service de la gloire humaine. Je veux que cette gloire ne soit pas une femme délicate et énervée ; elle est tout au moins bouffie de vanité, et lui asservir la solidité et la simplicité des vertus, vouloir que la Prudence n’ait rien à prévoir, la Justice rien à ordonner, la Force rien à soutenir, la Tempérance rien à modérer qui ne se rapporte à la gloire et n’ait la louange des hommes pour objet, ce serait une indignité manifeste. Et qu’ils ne se croient pas exempts de cette ignominie, ceux qui, en méprisant la gloire et le jugement des hommes, se plaisent à eux-mêmes et s’applaudissent de leur sagesse ; car leur vertu, si elle mérite ce nom, est encore asservie en quelque façon à la louange humaine, puisque se plaire à soi-même, c’est plaire à un homme. Mais quiconque croit et espère en Dieu d’un cœur vraiment pieux et plein d’amour, s’applique beaucoup plus à considérer en soi-même ce qui lui déplaît que ce qui peut lui plaire, moins encore à lui qu’à la vérité ; et ce qui peut lui plaire, il l’attribue à la miséricorde de celui dont il redoute le déplaisir, lui rendant grâces pour les plaies guéries, et lui offrant des prières pour les plaies à guérir.
\subsection[{Chapitre XXI}]{Chapitre XXI}

\begin{argument}\noindent C’est le vrai Dieu, source de toute-puissance et providence souveraine de l’univers, qui a donné l’empire aux Romains.
\end{argument}

\noindent N’attribuons donc la puissance de disposer des royaumes qu’au vrai Dieu, qui ne donne qu’aux bons le royaume du ciel, mais qui donne les royaumes de la terre aux bons et aux méchants, selon qu’il lui plaît, lui à qui rien d’injuste ne peut plaire. Nous avons indiqué quelques-unes des raisons qui dirigent sa conduite, dans la mesure où il a daigné nous les découvrir ; mais nous reconnaissons qu’il est au-dessus de nos forces de pénétrer dans les secrets de la conscience des hommes, et de peser les mérites qui règlent la distribution des grandeurs temporelles. Ainsi ce seul vrai Dieu, dont les conseils et l’assistance ne manquent jamais à l’espèce humaine, a donné l’empire aux Romains, adorateurs de plusieurs dieux, quand il l’a voulu et aussi grand qu’il l’a voulu, comme il l’avait donné aux Assyriens et même aux Perses, qui, selon le témoignage de leurs propres livres, n’adoraient que deux dieux, l’un bon et l’autre mauvais, pour ne point parler ici des Hébreux qui, tant que leur empire a duré, n’ont reconnu qu’un seul Dieu. Celui donc qui a accordé aux Perses les moissons et les autres biens de la terre, sans qu’ils adorassent la déesse Segetia, ni tant d’autres divinités que les Romains imaginaient pour chaque objet particulier, et même pour les usages différents du même objet, celui-là leur a donné l’empire sans l’assistance de ces dieux à qui Rome s’est cru redevable de sa grandeur. C’est encore lui qui a élevé au pouvoir suprême Marius et César, Auguste et Néron, Titus, les délices du genre humain, et Domitien, le plus cruel des tyrans. C’est lui enfin qui a porté au trône impérial et le chrétien Constantin, et ce Julien l’Apostat dont le bon naturel fut corrompu par l’ambition et par une curiosité détestable et sacrilège. Adonné à de vains oracles, il osa, dans sa confiance imprudente, faire brûler les vaisseaux qui portaient les vivres nécessaires à son armée ; puis s’engageant avec une ardeur téméraire dans la plus audacieuse entreprise, il fut tué misérablement, laissant ses soldats à la merci de la faim et de l’ennemi retraite désastreuse où pas un soldat n’eût échappé si, malgré le présage du dieu Terme, dont j’ai parlé dans le livre précédent, on n’eût déplacé les limites de l’empire romain ; car ce Dieu, qui n’avait pas voulu céder à Jupiter, fut obligé de céder à la nécessité. Concluons que c’est le Dieu unique et véritable qui gouverne et régit tous ces événements au gré de sa volonté ; et s’il tient ses motifs cachés, qui oserait les supposer injustes ?
\subsection[{Chapitre XXII}]{Chapitre XXII}

\begin{argument}\noindent La durée et l’issue des guerres dépendent de la volonté de Dieu.
\end{argument}

\noindent De même qu’il dépend de Dieu d’affliger ou de consoler les hommes, selon les conseils de sa justice et de sa miséricorde, c’est lui aussi qui règle les temps des guerres, qui les abrégé ou les prolonge à son gré. La guerre des pirates et la troisième guerre punique furent terminées, celle-là par Pompée, et celle-ci par Scipion, avec une incroyable célérité. Il en fut de même de la guerre des gladiateurs fugitifs, où plusieurs généraux et deux consuls essuyèrent des défaites, où l’Italie tout entière fut horriblement ravagée, mais qui ne laissa pas de s’achever en trois ans. Ce ne fut pas encore une très longue guerre que celle des Picentins, Marses, Péligniens et autres peuples italiens qui, après avoir longtemps vécu sous la domination romaine avec toutes les marques de la fidélité et du dévouement, relevèrent la tête et entreprirent de recouvrer leur indépendance, quoique Rome eût déjà étendu son empire sur un grand nombre de nations étrangères et renversé Carthage. Les Romains furent souvent battus dans cette guerre, et deux consuls y périrent avec plusieurs sénateurs ; toutefois le mal fut bientôt guéri, et tout fut terminé au bout de cinq ans. Au contraire, la seconde guerre punique fut continuée pendant dix-huit années avec des revers terribles pour les Romains, qui perdirent en deux batailles plus de soixante-dix mille soldats, ce qui faillit ruiner la république. La première guerre contre Carthage avait duré vingt-trois ans, et il fallut quarante ans pour en finir avec Mithridate. Et afin qu’on ne s’imagine pas que les Romains terminaient leurs guerres plus vite en ces temps de jeunesse où leur vertu a été tant célébrée, il me suffira de rappeler que la guerre des Samnites se prolongea près de cinquante ans, et que les Romains y furent si maltraités qu’ils passèrent même sous le joug. Or, comme ils n’aimaient pas la gloire pour la justice, mais la justice pour la gloire, ils rompirent bientôt le traité qu’ils avaient conclu. Je rapporte tous ces faits parce que, soit ignorance, soit dissimulation, plusieurs vont attaquant notre religion avec une extrême insolence ; et quand ils voient de nos jours quelque guerre se prolonger, ils s’écrient que si l’on servait les dieux comme autrefois, cette vertu romaine, autrefois si prompte, avec l’assistance de Mars et de Bellone, à terminer les guerres, les terminerait de même aujourd’hui. Qu’ils songent donc à ces longues guerres des anciens Romains, qui eurent pour eux des suites si désastreuses et des chances si variées, et qu’ils considèrent que le monde est sujet à ces agitations comme la mer aux tempêtes, afin que, tombant d’accord de la vérité, ils cessent de tromper les ignorants et de se perdre eux-mêmes par les discours que leur langue insensée profère contre Dieu.
\subsection[{Chapitre XXIII}]{Chapitre XXIII}

\begin{argument}\noindent De la guerre contre Radagaise, roi des Goths, qui fut vaincu dans une seule action avec toute son armée.
\end{argument}

\noindent Cette marque éclatante que Dieu a donnée récemment de sa miséricorde à l’empire romain, ils n’ont garde de la rappeler avec la reconnaissance qui lui est due ; loin de là, ils font de leur mieux pour en éteindre à jamais le souvenir. Aussi bien, si de notre côté nous gardions le silence, nous serions complices de leur ingratitude. Rappelons donc que Radagaise, roi des Goths, s’étant avancé vers Rome avec une armée redoutable, avait déjà pris position dans les faubourgs, quand il fut attaqué par les Romains avec tant de bonheur qu’ils tuèrent plus de cent mille hommes sans perdre un des leurs et sans même avoir un blessé, s’emparèrent de sa personne et lui firent subir, ainsi qu’à ses fils, le supplice qu’il méritait. Si ce prince, renommé par son impiété, fût entré dans Rome avec cette multitude de soldats non moins impies que lui, qui eût-il épargné ? quel tombeau des martyrs eût-il respecté ? à qui eût-il fait grâce par la crainte de Dieu ? qui n’eût-il point tué ou déshonoré ? Et comme nos adversaires se seraient élevés contre nous en faveur de leurs dieux ! N’auraient-ils pas crié que si Radagaise était vainqueur, c’est qu’il avait pris soin de se rendre les dieux favorables au moyen de ces sacrifices de chaque jour que la religion chrétienne interdit aux Romains ? En effet, comme il s’avançait vers les lieux où il a été terrassé par la puissance divine, le bruit de son approche s’était partout répandu, et, si j’en crois ce qu’on disait à Carthage, les païenspensaient, disaient et allaient répétant en tout lieu que, le roi des Goths ayant pour lui les dieux auxquels il immolait chaque jour des victimes, il était impossible qu’il fût vaincu par ceux qui ne voulaient offrir aux dieux de Rome, ni permettre qu’on leur offrît aucun sacrifice. Et maintenant ces malheureux ne rendent point grâces à la bonté infinie de Dieu qui, ayant résolu de punir les crimes des hommes par l’irruption d’un barbare, a tellement tempéré sa colère qu’il a voulu que Radagaise fût vaincu d’une manière miraculeuse. Il y avait lieu de craindre en effet qu’une victoire des Goths ne fût attribuée aux démons que servait Radagaise, et la conscience des faibles pouvait en être troublée ; plus tard, Dieu a permis que Rome fût prise par Alaric, et encore est-il arrivé que les barbares, contre la vieille coutume de la guerre, ont épargné, par respect pour le christianisme, tous les Romains réfugiés dans les lieux saints, et se sont montrés ennemis si acharnés des démons et de tout ce culte où Radagaise mettait sa confiance, qu’ils semblaient avoir déclaré aux idoles une guerre plus terrible qu’aux hommes. Ainsi ce Maître et cet Arbitre souverain de l’univers a usé de miséricorde en châtiant les Romains, et fait voir par cette miraculeuse défaite des idolâtres que leurs sacrifices ne sont pas nécessaires au salut des empires, afin que les hommes sages et modérés ne quittent point la véritable religion par crainte des maux qui affligent maintenant le monde, mais s’y tiennent fermement attachés dans l’attente de la vie éternelle.
\subsection[{Chapitre XXIV}]{Chapitre XXIV}

\begin{argument}\noindent En quoi consiste le bonheur des princes chrétiens, et combien ce bonheur est véritable.
\end{argument}

\noindent Si nous appelons heureux quelques empereurs chrétiens, ce n’est pas pour avoir régné longtemps, pour être morts paisiblement en laissant leur couronne à leurs enfants, ni pour avoir vaincu leurs ennemis du dehors ou réprimé ceux du dedans. Ces biens ou ces consolations d’une misérable vie ont été aussi le partage de plusieurs princes qui adoraient les démons, et qui n’appartenaient pas au royaume de Dieu, et il en a été ainsi par un conseil particulier de la Providence, afin que ceux qui croiraient en elle ne désirassent pas ces biens temporels comme l’objet suprême de la félicité. Nous appelons les princes heureux quand ils font régner la justice, quand, au milieu des louanges qu’on leur prodigue ou des respects qu’on leur rend, ils ne s’enorgueillissent pas, mais se souviennent qu’ils sont hommes ; quand ils soumettent leur puissance à la puissance souveraine de Dieu ou la font servir à la propagation du vrai culte, craignant Dieu, l’aimant, l’adorant et préférant à leur royaume celui où ils ne craignent pas d’avoir des égaux ; quand ils sont lents à punir et prompts à pardonner, ne punissant que dans l’intérêt de l’État et non dans celui de leur vengeance, ne pardonnant qu’avec l’espoir que les coupables se corrigeront, et non pour assurer l’impunité aux crimes, tempérant leur sévérité par des actes de clémence et par des bienfaits, quand des actes de rigueur sont nécessaires ; d’autant plus retenus dans leurs plaisirs qu’ils sont plus libres de s’y abandonner à leur gré ; aimant mieux commander à leurs passions qu’à tous les peuples de la terre ; faisant tout cela, non pour la vaine gloire, mais pour la félicité éternelle, et offrant enfin au vrai Dieu pour leurs péchés le sacrifice de l’humilité, de la miséricorde et de la prière. Voilà les princes chrétiens que nous appelons heureux, heureux par l’espérance dès ce monde, heureux en réalité quand ce que nous espérons sera accompli.
\subsection[{Chapitre XXV}]{Chapitre XXV}

\begin{argument}\noindent Des prospérités que Dieu a répandues sur l’empereur chrétien Constantin.
\end{argument}

\noindent Le bon Dieu, voulant empêcher ceux qui l’adorent en vue de la vie éternelle de se persuader qu’il est impossible d’obtenir les royaumes et les grandeurs de la terre sans la faveur toute-puissante des démons, a voulu favoriser avec éclat l’empereur Constantin, qui, loin d’avoir recours aux fausses divinités, n’adorait que la véritable, et le combler de plus de biens qu’un autre n’en eût seulement osé souhaiter. Il a même permis que ce prince fondât une ville, compagne de l’empire, fille de Rome, mais où il n’y a pas un seul temple de faux dieux ni une seule idole. Son règne a été long ; il a soutenu, seul, le poids immense de tout l’empire, victorieux dans toutes ses guerres et fortuné dans sa lutte contre les tyrans. Il est mort dans son lit, chargé d’années, et a laissé l’empire à ses enfants. Et maintenant, afin que les empereurs n’adoptassent pas le christianisme par la seule ambition de posséder la félicité de Constantin, au lieu de l’embrasser comme on le doit pour obtenir la vie éternelle, Dieu a voulu que le règne de Jovien fût plus court encore que celui de Julien, et il a même permis que Gratien tombât sous le fer d’un usurpateur : plus heureux néanmoins dans sa disgrâce que le grand Pompée, qui adorait les dieux de Rome, puisque Pompée ne put être vengé par Caton, qu’il avait laissé pour ainsi dire comme son héritier dans la guerre civile. Gratien, au contraire, par une de ces consolations de la Providence dont les âmes pieuses n’ont pas besoin, Gratien fut vengé par Théodose, qu’il avait associé à l’empire, de préférence à son propre frère, se montrant ainsi plus jaloux de former une association fidèle que de garder une autorité plus étendue.
\subsection[{Chapitre XXVI}]{Chapitre XXVI}

\begin{argument}\noindent De la foi et de la piété de l’empereur Théodose.
\end{argument}

\noindent Aussi Théodose ne se borna pas à être fidèle à Gratien vivant, mais après sa mort il prit sous sa protection son frère Valentinien, que Maxime, meurtrier de Gratien, avait chassé du trône ; et avec la magnanimité d’un empereur vraiment chrétien, il entoura ce jeune prince d’une affection paternelle, alors qu’il lui eût été très facile de s’en défaire, s’il eût eu plus d’ambition que de justice. Loin de là, il l’accueillit comme empereur et lui prodigua les consolations. Cependant, Maxime étant devenu redoutable par le succès de ses premières entreprises, Théodose, au milieu des inquiétudes que lui causait son ennemi, ne se laissa pas entraîner vers des curiosités sacrilèges ; il s’adressa à Jean, solitaire d’Égypte, que la renommée lui signalait comme rempli de l’esprit de prophétie, et reçut de lui l’assurance de sa prochaine victoire. Il ne tarda pas, en effet, à vaincre le tyran Maxime, et aussitôt il rétablit le jeune Valentinien sur le trône. Ce prince étant mort peu après, par trahison ou autrement, et Eugène ayant été proclamé, sans aucun droit, son successeur, Théodose marcha contre lui, plein de foi en une prophétie nouvelle aussi favorable que la première, et défit l’armée puissante du tyran, moins par l’effort de ses légions que par la puissance de ses prières. Des soldats présents à la bataille m’ont rapporté qu’ils se sentaient enlever des mains les traits qu’ils dirigeaient contre l’ennemi ; il s’éleva, en effet, un vent si impétueux du côté de Théodose, que non seulement tout ce qui était lancé par ses troupes était jeté avec violence contre les rangs opposés, mais que les flèches de l’ennemi retombaient sur lui-même. C’est à quoi fait allusion le poète Claudien, tout ennemi qu’il est de la religion chrétienne, dans ces vers où il loue Théodose :\par
{\itshape « Ô prince trop aimé de Dieu ! Éole arme en ta faveur ses légions impétueuses ; la nature combat pour toi, et les vents conjurés accourent à l’appel de tes clairons} »\par
Au retour de cette expédition, où l’événement avait répondu à sa confiance et à ses prophétiques prévisions, Théodose fit abattre certaines statues de Jupiter, qu’on avait élevées dans les Alpes, en y attachant contre lui je ne sais quels sortilèges, et comme ses coureurs, avec cette familiarité que permet la joie de la victoire, lui disaient en riant que les foudres d’or dont ces statues étaient armées ne leur faisaient pas peur, et qu’ils seraient bien aise d’en être foudroyés, il leur en fit présent de bonne grâce. Ses ennemis morts sur le champ de bataille, moins par ses ordres que par l’emportement du combat, laissaient des fils qui se réfugièrent dans une église, quoiqu’ils ne fussent pas chrétiens ; il saisit cette occasion de leur faire embrasser le christianisme, montra pour eux une charité vraiment chrétienne, et loin de confisquer leurs biens, les leur conserva en y ajoutant des honneurs. Il ne permit à personne, après la victoire, d’exercer des vengeances particulières. Sa conduite dans la guerre civile ne ressembla nullement à celle de Cinna, de Marins, de Sylla et de tant d’autres, qui sans cesse recommençaient ce qui était fini ; lui, au contraire, déplora la lutte quand elle pritnaissance, et ne voulut en abuser contre personne quand elle prit fin. Au milieu de tant de soucis, il fit dès le commencement de son règne des lois très justes et très saintes en faveur de l’Église, que l’empereur Valens, partisan des Ariens, avait violemment persécutée ; c’était à ses yeux un plus grand honneur d’être un des membres de cette Église que d’être le maître de l’univers. Il fit abattre partout les idoles, persuadé que les biens mêmes de la terre dépendent de Dieu et non des démons. Mais qu’y a-t-il de plus admirable que son humilité, quand, après avoir promis, à la prière des évêques, de pardonner à la ville de Thessalonique, et s’être laissé entraîner à sévir contre elle par les instances bruyantes de quelques-uns de ses courtisans, rencontrant tout à coup devant lui la courageuse censure de l’Église, il fit une telle pénitence de sa faute que le peuple, intercédant pour lui avec larmes, fut plus affligé de voir la majesté de l’empereur humiliée qu’il n’avait été effrayé de sa colère. Ce sont ces bonnes œuvres et d’autres semblables, trop longues à énumérer, que Théodose a emportées avec lui quand, abandonnant ces grandeurs humaines qui ne sont que vapeur et fumée, il est allé chercher la récompense que Dieu n’a promise qu’aux hommes vraiment pieux. Quant aux biens de cette vie, honneurs ou richesses, Dieu les donne également aux bons et aux méchants, comme il leur donne le monde, la lumière, l’air, l’eau, la terre et ses fruits, l’âme, le corps, les sens, la raison et la vie ; et dans ces biens il faut comprendre aussi les empires, si grands qu’ils soient, que Dieu dispense selon les temps dans les conseils de sa providence.\par
Il s’agit maintenant de répondre à ceux qui, étant convaincus par les preuves les plus claires que la multitude des faux dieux ne sert de rien pour obtenir les biens temporels, seuls objets que désirent les hommes de peu de sens, se réduisent à prétendre qu’il faut les adorer, non en vue des avantages de la vie présente, mais dans l’intérêt de la vie future. Quant aux païens obstinés qui persistent à les servir pour les biens de ce monde, et se plaignent de ce qu’on ne leur permet pas de s’abandonner à ces vaines et ridicules superstitions, je crois leur avoir assez répondu dans ces cinq livres. Au moment où je publiais les trois premiers, et quand ils étaient déjà entre les mains de tout le monde, j’appris qu’on y préparait une réponse, et depuis j’ai été informé qu’elle était prête, mais qu’on attendait l’occasion de pouvoir la faire paraître sans danger. Sur quoi je dirai à mes contradicteurs de ne pas souhaiter une chose qui ne saurait leur être avantageuse. On se flatte aisément d’avoir répondu, quand on n’a pas su se taire. Et quelle source de paroles plus fertile que la vanité ! mais de ce qu’elle peut toujours crier plus fort que la vérité, il ne s’ensuit pas qu’elle soit la plus forte. Qu’ils y pensent donc sérieusement ; et si, jugeant la chose sans esprit de parti, ils reconnaissent par hasard qu’il est plus aisé d’attaquer nos principes par un bavardage impertinent et des plaisanteries dignes de la comédie ou de la satire, que par de solides raisons, qu’ils s’abstiennent de publier des sottises et préfèrent les remontrances des personnes éclairées aux éloges des esprits frivoles ; que s’ils attendent l’occasion favorable, non pour dire vrai avec toute liberté, mais pour médire avec toute licence, à Dieu ne plaise qu’ils soient heureux à la manière de cet homme dont Cicéron dit si bien : « Malheureux, à qui il est permis de mal faire. » Si donc il y a quelqu’un de nos adversaires qui s’estime heureux d’avoir la liberté de médire, nous pouvons l’assurer qu’il sera plus heureux d’en être privé, d’autant mieux que rien ne l’empêche, dès à présent, de venir discuter avec nous tant qu’il voudra, non pour satisfaire une vanité stérile, mais pour s’éclairer ; et il ne dépendra pas de nous qu’il ne reçoive, dans cette controverse amicale, une réponse digne, grave et sincère.
\section[{Livre sixième. Les dieux païens}]{Livre sixième. \\
Les dieux païens}\renewcommand{\leftmark}{Livre sixième. \\
Les dieux païens}

\subsection[{Préface}]{Préface}
\noindent Je crois avoir assez réfuté, dans les cinq livres précédents, ceux qui pensent qu’on doit honorer d’un culte de {\itshape latrie}, lequel n’est dû qu’au seul vrai Dieu, toutes ces fausses divinités, convaincues par la religion chrétienne d’être de vains simulacres, des esprits immondes ou des démons, en un mot, des créatures et non le Créateur. Je n’ignore pas toutefois que ces cinq livres et mille autres ne puissent suffire à satisfaire les esprits opiniâtres. La vanité ne se fait-elle pas un point d’honneur de résister à toutes les forces de la vérité ? et cependant le vice hideux de l’obstination tourne contre les malheureux mêmes qui en sont subjugués. C’est une maladie incurable, non par la faute du médecin, mais par celle du malade. Quant à ceux qui pèsent ce qu’ils ont lu et le méditent sans opiniâtreté, ou du moins sans trop d’attachement à leurs vieilles erreurs, ils jugeront, j’espère, que nous avons plus que suffisamment résolu la question proposée, et que le seul reproche qu’on nous puisse adresser est celui d’une surabondance excessive. Je crois aussi qu’ils se convaincront aisément que cette haine, qu’on excite contre la religion chrétienne à l’occasion des calamités et des bouleversements du monde, passion aveugle ressentie par des ignorants, mais que des hommes très savants, possédés par une rage impie, ont soin de fomenter contre le témoignage de leur conscience, toute cette haine est l’ouvrage de la légèreté et du dépit, et n’a aucun motif raisonnable.
\subsection[{Chapitre premier}]{Chapitre premier}

\begin{argument}\noindent De ceux qui prétendent adorer les dieux, non en vue de la vie présente, mais en vue de la vie éternelle.
\end{argument}

\noindent Ayant donc à répondre maintenant, selon l’ordre que je me suis prescrit, à ceux qui soutiennent qu’il faut servir les dieux dans l’intérêt de la vie à venir et non pour les biens d’ici-bas, je veux entrer en matière par cet oracle véridique du saint Psalmiste : « Heureux celui qui a mis son espérance dans le Seigneur et n’a point arrêté ses regards aux choses vaines et aux trompeuses folies. » Toutefois, au milieu des vanités et des folies du paganisme, ce qu’il y a de plus supportable, c’est la doctrine des philosophes qui ont méprisé les superstitions vulgaires, tandis que la foule se prosternait aux pieds des idoles et, tout en leur attribuant mille indignités, les appelait dieux immortels et leur offrait un culte et des sacrifices. C’est avec ces esprits d’élite qui, sans proclamer hautement leur pensée, l’ont au moins murmurée à demi-voix dans leurs écoles, c’est avec de tels hommes qu’il peut convenir de discuter cette question : faut-il adorer, en vue de la vie future, un seul Dieu, auteur de toutes les créatures spirituelles et corporelles, ou bien cette multitude de dieux qui n’ont été reconnus par les plus excellents et les plus illustres de ces philosophes qu’à titre de divinités secondaires créées par le Dieu suprême et placées de sa propre main dans les régions supérieures de l’univers ?\par
Quant à ces dieux bien différents sur lesquels je me suis expliqué au quatrième livre, et dont l’emploi est restreint aux plus minces objets, qui pourrait être reçu à soutenir qu’ils soient capables de donner la vie éternelle ? En effet, ces hommes si habiles et si ingénieux, qui croient que le monde leur est fort obligé de lui avoir appris ce qu’il faut demander à chaque dieu, de peur que, par une de ces méprises ridicules dont on se divertit à la comédie, on ne soit exposé à demander de l’eau à Bacchus ou du vin aux nymphes, voudraient-ils que celui qui s’adresse aux nymphes pour avoir du vin, sur cette réponse : Nous n’avons que de l’eau à donner, adressez-vous à Bacchus, — s’avisât de répliquer : Si vous n’avez pas de vin, donnez-moi la vie éternelle ? — Se peut-il concevoir rien de plus absurde ? et en supposant que les nymphes, au lieu de chercher, en leur qualité de démons, à tromper le malheureux suppliant, eussent envie de rire (car ce sont de grandes rieuses), ne pourraient-elles pas lui répondre : « Tu crois, pauvre homme, que nous disposons de la vie, nous qui ne disposons même pas de la vigne ! » C’est donc le comble de la folie d’attendre la vie éternelle de ces dieux, dont les fonctions sont tellement partagées, pour les objets mêmes de cette vie misérable, et dont la puissance est si restreinte et si limitée qu’on ne saurait demander à l’un ce qui dépend de la fonction de l’autre, sans se charger d’un ridicule digne de la comédie. On rit quand des auteurs donnent sciemment dans ces méprises, mais il y a bien plus sujet de rire, quand des superstitieux y tombent par ignorance. Voilà pourquoi de savants hommes ont écrit des traités où ils déterminent pertinemment à quel dieu ou à quelle déesse il convient de s’adresser pour chaque objet qu’on peut avoir à solliciter : dans quel cas, par exemple, il faut avoir recours à Bacchus, dans quel autre cas aux nymphes ou à Vulcain, et ainsi de tous les autres dont j’ai fait mention au quatrième livre, ou que j’ai cru devoir passer sous silence. Or, si c’est une erreur de demander du vin à Cérès, du pain à Bacchus, de l’eau à Vulcain et du feu aux nymphes, n’est-ce pas une extravagance de demander à aucun de ces dieux la vie éternelle ?\par
Et en effet, si nous avons établi, en traitant aux livres précédents des royaumes de la terre, que les plus grandes divinités du paganisme ne peuvent pas même disposer des grandeurs d’ici-bas, je demande s’il ne faut pas pousser l’impiété jusqu’à la folie pour croire que cette foule de petits dieux seront capables de disposer à leur gré de la vie éternelle, supérieure, sans aucun doute et sans aucune comparaison, à toutes les grandeurs périssables ? Car, qu’on ne s’imagine pas que leur impuissance à disposer des prospérités de la terre tient à ce que de tels objets sont au-dessous de leur majesté et indignes de leurs soins, non ; si peu de prix qu’on doive attacher aux choses de ce monde, c’est l’indignité de ces dieux qui les a fait paraître incapables d’en être les dispensateurs. Or, si aucun d’eux, comme je l’ai prouvé, ne peut, petit ou grand, donner à un mortel des royaumes mortels comme lui, à combien plus forte raison ne saurait-il donner à ce mortel l’immortalité ?\par
Il y a plus, et puisque nous avons maintenant affaire à ceux qui adorent les dieux, non pour la vie présente, mais pour la vie future, ils doivent tomber d’accord qu’il ne faut pas du moins les adorer en vue de ces objets particuliers qu’une vaine superstition assigne à chacun d’eux comme son domaine propre ; car ce système d’attributions particulières n’a aucun fondement raisonnable, et je crois l’avoir assez réfuté. Ainsi, alors même que les adorateurs de Juventas jouiraient d’une jeunesse plus florissante, et que les contempteurs de cette déesse mourraient ou se flétriraient avant le temps ; alors même que la Fortune barbue couvrirait d’un duvet agréable les joues de ses pieux serviteurs et refuserait cet ornement à tout autre ou ne lui donnerait qu’une barbe sans agrément, nous aurions toujours raison de dire que le pouvoir de ces divinités est enfermé dans les limites de leurs attributions, et par conséquent qu’on ne doit demander la vie éternelle ni à Juventas, qui ne peut même pas donner de la barbe, ni à la Fortune barbue, incapable aussi de donner cet âge où la barbe vient au menton. Si donc il n’est pas nécessaire de servir ces déesses pour obtenir les avantages dont on leur attribue la disposition (car combien ont adoré Juventas qui ont eu une jeunesse peu vigoureuse, tandis que d’autres, qui ne l’adorent pas, jouissent de la plus grande vigueur ? et combien aussi invoquent la Fortune barbue sans avoir de barbe, ou l’ont si laide qu’ils prêtent à rire à ceux qui l’ont belle sans l’avoir demandée ?), comment croire que le culte de ces dieux, inutile pour obtenir des biens passagers, où ils président uniquement, soit réellement utile pour obtenir la vie éternelle ? Ceux-là mêmes ne l’ont pas osé dire, qui, pour les faire adorer du vulgaire ignorant, ont distribué à chacun son emploi, de peur sans doute, vu leur grand nombre, qu’il n’y en eût quelqu’un d’oisif.
\subsection[{Chapitre II}]{Chapitre II}

\begin{argument}\noindent Sentiment de Varron touchant les dieux du paganisme, qu’il nous apprend à si bien connaître, qu’il leur eût mieux marqué son respect en n’en disant absolument rien.
\end{argument}

\noindent Où trouver, sur cette matière, des recherches plus curieuses, des découvertes plus savantes, des études plus approfondies que dans Marcus Varron, en un mot, un traité mieux divisé, plus soigneusement écrit et plus complet ? Malgré l’infériorité de son style, qui manque un peu d’agrément, il a tant de sens et de solidité, qu’en tout ce qui regarde les sciences profanes, que les païens nomment libérales, il satisfait ceux qui sont avides de choses, autant que Cicéron charme ceux qui sont avides de beau langage. J’en appelle à Cicéron lui-même, qui, dans ses {\itshape Académiques} nous apprend qu’il a discuté la question qui fait le sujet de son ouvrage, avec Varron, « l’homme, dit-il, le plus pénétrant du monde et sans aucun doute le plus savant ». Remarquez qu’il ne dit pas le plus éloquent ou le plus disert, parce qu’à cet égard l’infériorité de Varron est grande, mais il dit le plus pénétrant, et ce n’est pas tout : car il ajoute, dans un livre destiné à prouver qu’il faut douter de tout : et sans aucun doute le plus savant, comme si le savoir de Varron était la seule vérité dont il n’y eût pas à douter, et qui pût faire oublier à l’auteur, au moment de discuter le doute académique, qu’il était lui-même académicien.\par
Dans l’endroit du premier livre où il vante les ouvrages de Varron, il s’adresse ainsi à cet écrivain : « Nous étions errants et comme étrangers dans notre propre pays ; tes livres ont été pour nous comme des hôtes qui nous ont ramenés à la maison et nous ontappris à reconnaître notre nom et notre demeure. Par toi nous avons connu l’âge de notre patrie ; par toi, l’ordre et la suite des temps ; par toi, les lois du culte et les attributions des pontifes ; par toi, la discipline privée et publique ; par toi, la situation des lieux et des empires ; par toi, les noms, les espèces et les fonctions des dieux ; en un mot, les causes de toutes les choses divines et humaines. » Si donc ce personnage si excellent et si rare, dont Térentianus a dit, dans un vers élégant et précis, qu’il était savant de tout point ; si ce grand auteur, qui a tant lu qu’on s’étonne qu’il ait eu le temps d’écrire, et qui a plus écrit que personne ait peut-être jamais lu ; si cet habile et savant homme avait entrepris de combattre et de ruiner les institutions dont il traite comme de choses divines, s’il avait voulu soutenir qu’il se trouvait en tout cela plus de superstition que de religion, je ne sais, en vérité, s’il aurait relevé plus qu’il n’a fait de choses ridicules, odieuses et détestables. Mais comme il adorait ces mêmes dieux, comme il croyait à la nécessité de les adorer, jusque-là qu’il avoue dans son livre la crainte qu’il a de les voir périr, moins par une invasion étrangère que par la négligence de ses concitoyens, et déclare expressément n’avoir d’autre but que de les sauver de l’oubli en les mettant sous la sauvegarde de la mémoire des gens de bien (précaution plus utile, en effet, que le dévouement de Métellus pour arracher la statue de Vesta à l’incendie, ou que celui d’Énée pour dérober ses dieux pénates à la ruine de Troie), comme une laisse pas toutefois de conserver à la postérité des traditions contraires à la piété, et à ce titre également réprouvées par les savants et par les ignorants, que pouvons-nous penser, sinon que cet écrivain, d’ailleurs si habile et si pénétrant, mais que le Saint-Esprit n’avait pas rendu à la liberté, succombait sous le poids de la coutume et des lois de son pays, et toutefois, sous prétexte de rendre la religion plus respectable, ne voulait pas faire ce qu’il y trouvait à blâmer ?
\subsection[{Chapitre III}]{Chapitre III}

\begin{argument}\noindent Plan des {\itshape Antiquités} de Varron.
\end{argument}

\noindent Les {\itshape Antiquités} de Varron forment quarante et un livres : vingt-cinq sur les choses humaines et seize sur les choses divines. Le Traité des choses humaines est divisé en quatre parties, suivant que l’on considère les personnes, les temps, les lieux et les actions. Sur chacun de ces objets il y a six livres ; en tout vingt-quatre, plus un premier livre, qui est une introduction générale. Varron suit le même ordre pour les choses divines : considérant tour à tour les personnes qui sacrifient aux dieux, les temps, les lieux où elles sacrifient et les sacrifices eux-mêmes, il maintient exactement cette distinction subtile et emploie trois livres pour chacun de ces quatre objets ; ce qui fait en tout douze livres. Mais comme il fallait dire aussi à qui sont offerts les sacrifices, car c’est là le point le plus intéressant, il aborde cette matière dans les trois derniers livres, où il parle des dieux. Ajoutez ces trois livres aux douze précédents, et joignez-y encore un livre d’introduction sur les choses divines considérées en général, voilà les seize livres dont j’ai parlé. Dans ce qui regarde les choses divines, sur les trois livres qui traitent des personnes, le premier parle des pontifes ; le second, des augures ; le troisième, des quindécemvirs. Aux trois suivants, qui concernent les lieux, Varron traite premièrement des autels privés ; secondement, des temples ; troisièmement, des lieux sacrés. Viennent ensuite les trois livres sur les temps, c’est-à-dire sur les jours de fêtes publiques, où il parle d’abord des jours fériés, puis des jeux scéniques. Enfin, les trois livres qui concernent les sacrifices traitent successivement des consécrations, des sacrifices domestiques et des sacrifices publics. Tout cela forme une espèce de pompe religieuse où les dieux marchent les derniers à la suite du cortège ; car il reste encore trois livres pour terminer l’ouvrage : l’un sur les dieux certains, l’autre sur les dieux incertains et le dernier sur les dieux principaux et choisis.
\subsection[{Chapitre IV}]{Chapitre IV}

\begin{argument}\noindent Il résulte des dissertations de Varron que les adorateurs des faux dieux regardaient les choses humaines comme plus anciennes que les choses divines.
\end{argument}

\noindent Il résulte déjà très clairement de ce que nous avons dit, une conséquence qui deviendra plus claire encore par ce qui nous reste à dire : c’est que pour tout homme qui n’est point opiniâtre jusqu’à devenir ennemi de soi-même, il y aurait de l’impudence à s’imaginer que toutes ces belles et savantes divisions de Varron aient quelque pouvoir pour faire espérer la vie éternelle. Qu’est-ce, en effet, que tout cela, sinon des institutions tout humaines ou des inventions des démons ? Et je ne parle pas des démons que les païens appellent bons démons ; je parle de ces esprits immondes et sans contredit malfaisants, qui répandent en secret dans l’esprit des impies des opinions pernicieuses, et quelquefois les confirment ouvertement par leurs prestiges, afin d’égarer les hommes de plus en plus, et de les empêcher de s’unir à la vérité éternelle et immuable. Varron lui-même l’a si bien senti qu’il a placé dans son livre les choses humaines avant les choses divines, donnant pour raison que ce sont les sociétés qui ont commencé à s’établir, et qu’elles ont ensuite établi les cultes. Or, la vraie religion n’est point une institution de quelque cité de la terre ; c’est elle qui forme la Cité céleste, et elle est inspirée par le vrai Dieu, arbitre de la vie éternelle, qui enseigne lui-même la vérité à ses adorateurs.\par
Varron avoue donc que s’il a placé les choses humaines avant les divines, c’est que celles-ci sont l’ouvrage des hommes, et voici comment il raisonne : « De même, dit-il, que le peintre existe avant son tableau et l’architecte avant son édifice, ainsi les sociétés existent avant les institutions sociales. » Il ajoute qu’il aurait parlé des dieux avant de parler des hommes, s’il avait voulu dans son livre embrasser {\itshape toute la nature divine} ; comme s’il ne traitait que d’une partie de la nature divine et non de cette nature tout entière ! et comme si même une partie de la nature divine ne devait pas être mise avant la nature humaine ! Mais puisque dans les trois livres qui terminent son ouvrage, il classe les dieux d’une façon si exacte en certains, incertains et choisis, ne semble-t-il pas avoir voulu ne rien omettre dans la nature divine ? Que vient-il donc nous dire, que s’il eût embrassé la nature divine tout entière, il eût parlé des dieux avant de parler des hommes ? car enfin, de trois choses l’une : ou il traite de toute la nature divine, ou bien il traite d’une partie, ou enfin ce dont il traite n’est rien de la nature divine. S’il traite de la nature divine tout entière, elle doit sans nul doute avoir sur la nature humaine la priorité ; s’il traite d’une partie de la nature divine, pourquoi la priorité ne lui serait-elle pas acquise également ? Est-ce que toute partie quelconque de la nature divine ne doit pas être mise au-dessus de la nature humaine ? En tout cas, si c’est trop faire pour une partie de la nature divine que de la préférer à la nature humaine tout entière, du moins fallait-il la préférer à ce qui n’est qu’une partie des choses humaines, je veux dire aux institutions des Romains ; car les livres de Varron regardent Rome et non pas toute l’humanité. Et cependant il croit bien faire d’ajourner les choses divines, sous prétexte que le peintre précède son tableau et l’architecte son édifice ; n’est-ce pas avouer nettement que ce qu’il appelle choses divines n’est à ses yeux, comme la peinture et l’architecture, que l’ouvrage des hommes ? Il ne reste donc plus que la troisième hypothèse, savoir, que l’objet de son traité n’est rien de divin, et voilà ce dont il ne serait pas convenu ouvertement, mais ce qu’il a peut-être voulu faire entendre aux esprits éclairés. En effet, il se sert d’une expression équivoque, qui veut dire, dans le sens ordinaire, que l’objet de son traité n’est pas toute la nature divine, mais qui peut signifier aussi que ce n’est rien de vraiment divin. Dans le fait, s’il avait traité de toute la nature divine, le véritable ordre était, il en convient lui-même, de la placer avant la nature humaine ; et comme il est clair d’ailleurs, sinon par le témoignage de Varron, du moins par l’évidence de la vérité, que dans le cas même où il n’aurait voulu traiter que d’une partie de la nature divine, elle devait encore avoir la priorité, il s’ensuit finalement que l’objet dont il traite n’a rien de véritablement divin. Dès lors, il ne faut pas dire que Varron a voulu préférer les choses humaines aux choses divines ; il faut dire qu’il n’a pas voulu préférer des choses fausses à des choses vraies. Car dans ce qu’il écrit touchant les choses humaines, il suit l’ordre des événements, au lieu qu’en traitant des choses divines, qu’a-t-il suivi, sinon des opinions vaines et fantastiques ? Et c’est ce qu’il a voulu finement insinuer, non seulement par l’ordre qu’il a suivi, mais encore par la raison qu’il en donne. Peut-être, s’il eût suivi cet ordre sans en dire la raison, nierait-on qu’il ait eu aucune intention semblable ; mais, parlant comme il fait, on ne peut lui supposer aucune autre pensée, et il a fait assez voir qu’il a voulu placer les hommes avant les institutions des hommes, et non pas la nature humaine avant la nature des dieux. Ainsi il a reconnu que l’objet de son traité des choses divines n’est pas la vérité qui a son fondement dans la nature, mais la fausseté qui a le sien dans l’erreur. C’est ce qu’il a déclaré ailleurs d’une façon plus formelle encore, comme je l’ai rappelé dans mon quatrième livre, quand il dit que s’il avait à fonder un État nouveau, il traiterait des dieux selon les principes de la nature ; mais que, vivant dans un État déjà vieux, il ne pouvait que suivre la coutume.
\subsection[{Chapitre V}]{Chapitre V}

\begin{argument}\noindent Des trois espèces de théologies distinguées par Varron, l’une mythique l’autre naturelle, et l’autre civile.
\end{argument}

\noindent Que signifie-cette division de la théologie ou science des dieux en trois espèces : l’une mythique, l’autre physique, et l’autre civile ? Le nom de théologie fabuleuse conviendrait assez à la première espèce, mais je veux bien l’appeler {\itshape mythique}, du grec {\itshape muthos}, qui signifie fable. Appelons aussi la seconde espèce indifféremment {\itshape physique} ou {\itshape naturelle}, puisque l’usage l’autorise et, quant à la troisième espèce, à la théologie {\itshape politique}, nommée par Varron {\itshape civile}, il n’y a pas de difficulté. Voici comment il s’explique à cet égard : « On appelle mythique la théologie des poètes, physique, celle des philosophes, et civile, celle des peuples. » — « Or », poursuit-il, « dans la première espèce de théologie, il se rencontre beaucoup de fictions contraires à la dignité et à la nature des dieux immortels, comme, par exemple, la naissance d’une divinité qui sort du cerveau d’une autre divinité, ou de sa cuisse, ou de quelques gouttes de son sang ; ou bien encore un dieu voleur, un dieu adultère, un dieu serviteur de l’homme. Et pour tout dire, on y attribue aux dieux tous les désordres où tombent les hommes et même les hommes les plus infâmes. » Ainsi, quand Varron le peut, quand il l’ose, quand il parle avec la certitude de l’impunité, il s’explique sans détour sur l’injure faite à la divinité par les fables mensongères ; car il ne s’agit pas ici de la théologie naturelle ou de la théologie civile, mais seulement de la théologie mythique, et c’est pourquoi il a cru pouvoir la censurer librement. Voyons maintenant son opinion sur la théologie naturelle : « La seconde espèce de théologie que j’ai distinguée, dit-il, a donné matière à un grand nombre de livres où les philosophes font des recherches suries dieux, sur leur nombre, le lieu de leur séjour, leur nature et leurs qualités : sont-ils éternels ou ont-ils commencé ? tirent-ils leur origine du feu, comme le croit Héraclite, ou des nombres, suivant le système de Pythagore, ou des atomes, ainsi qu’Épicure le soutient ? et autres questions semblables, qu’il est plus facile de discuter dans l’intérieur d’une école que dans le forum. » On voit que Varron ne trouve rien à redire dans cette théologie naturelle, propre aux philosophes ; il remarque seulement la diversité de leurs opinions, qui a fait naître tant de sectes opposées, et cependant il bannit la théologie naturelle du forum et la renferme dans les écoles, tandis qu’il n’interdit pas au peuple la première espèce de théologie, qui est toute pleine de mensonges et d’infamies. Ô chastes oreilles du peuple, et surtout du peuple romain ! elles ne peuvent entendre les discussions des philosophes sur les dieux immortels ; mais que les poètes chantent leurs fictions, que des histrions les jouent, que la nature des dieux soit altérée, que leur majesté soit avilie par des récits qui les font tomber au niveau des hommes les, plus infâmes, on supporte tout cela ; que dis-je ? on l’écoute avec joie ; et on s’imagine que ces scandales sont agréables aux dieux et contribuent à les rendre favorables !\par
On me dira peut-être : Sachons distinguer la théologie mythique ou fabuleuse et la théologie physique ou naturelle de la théologie civile, comme fait Varron lui-même, et cherchons ce qu’il pense de celle-ci. Je réponds qu’en effet il y a de bonnes raisons de mettre à part la théologie fabuleuse : c’est qu’elle est fausse, c’est qu’elle est infâme, c’est qu’elle est indigne ; mais séparer la théologie naturelle de la théologie civile, n’est-ce pas avouer que la théologie civile est fausse ? Si, en effet, la théologie civile est conforme à la nature, pourquoi écarter la théologie naturelle ? Si elle ne lui est pas conforme, à quel titre la reconnaître pour vraie ? Et voilà pourquoi Varron a fait passer les choses humaines avant les choses divines ; c’est qu’en traitant de celles-ci, il ne s’est pas conformé à la nature des dieux, mais aux institutions des hommes. Examinons toutefois cette théologie civile : « La troisième espèce de théologie, dit-il, est celle que les citoyens, et surtout les prêtres, doivent connaître et pratiquer. Elle consiste à savoir quels sont les dieux qu’il faut adorer publiquement, et à quelles cérémonies, à quels sacrifices chacun est, obligé. » Citons encore ce qu’ajoute Varron : « La première espèce de théologie convient au théâtre, la seconde au monde, la troisième à la cité. » Qui ne voit à laquelle des trois il donne la préférence ? Ce ne peut être qu’à la seconde, qui est celle des philosophes. Elle se rapporte en effet au monde, et, suivant les philosophes, il n’y a rien de plus excellent que le monde. Quant aux deux autres espèces de théologie, celle du théâtre et celle de la cité, on ne sait s’il les distingue ou s’il les confond. En effet, de ce qu’un ordre de choses appartient à la cité, il ne s’ensuit pas qu’il appartienne au monde, quoique la cité soit dans le monde, et il peut arriver que sur de fausses opinions on croie et on adore dans la cité des objets qui ne sont ni dans le monde, ni hors du monde. Je demande en outre où est le théâtre, sinon dans la cité ? et pourquoi on l’a établi, sinon à cause des jeux scéniques ? et à quoi se rapportent les jeux scéniques, sinon aux choses divines, qui ont tant exercé la sagacité de Varron ?
\subsection[{Chapitre VI}]{Chapitre VI}

\begin{argument}\noindent De la théologie mythique ou fabuleuse et de la théologie civile, contre Varron.
\end{argument}

\noindent Ô Marcus Varron ! tu es le plus pénétrant et sans aucun doute le plus savant des hommes, mais tu n’es qu’un homme, tu n’es pas Dieu, et même il t’a manqué d’être élevé par l’Esprit de Dieu à ce degré de lumière et de liberté qui rend capable de connaître et d’annoncer les choses divines ; tu vois clairement qu’il faut séparer ces grands objets d’avec les folies et les mensonges des hommes ; mais tu crains de heurter les fausses opinions du peuple et les superstitions autorisées par la coutume ; et cependant, quand tu examines de près ces vieilles croyances, tu reconnais à chaque page et tu laisses partout éclater combien elles te paraissent contraires à la nature des dieux, même de ces dieux imaginaires tels que se les figure, parmi les éléments du monde, la faiblesse de l’esprit humain. Que fait donc ici le génie de l’homme et même le génie le plus excellent ? À quoi te sert, Varron, toute cette science si variée et si profonde pour sortir de l’inévitable alternative où tu es placé ? tu voudrais adorer les dieux de la nature et tu es contraint d’adorer ceux de la cité ! Tu as rencontré, à la vérité, d’autres dieux, les dieux de la fable, sur lesquels tu décharges librement ta réprobation ; mais tous les coups que tu leur portes retombent sur les dieux de la politique. Tu dis, en effet, que les dieux fabuleux conviennent au théâtre, les dieux naturels au monde et les dieux civils à l’État ; or, le monde n’est-il pas une œuvre divine, tandis que le théâtre et l’État sont des œuvres humaines ; et les dieux dont on rit au théâtre ou à qui l’on consacre des jeux, sont-ils d’autres dieux que ceux qu’on adore dans les temples de l’État et à qui on offre des sacrifices ? Combien il eût été plus sincère et même plus habile de diviser les dieux en deux classes, les dieux naturels et les dieux d’institution humaine, en ajoutant, quant à ceux-ci, que si les poètes et les prêtres n’en parlent pas de la même manière, il y a ce point commun entre eux que ce qu’ils en disent est également faux et par conséquent également agréable aux démons, ennemis de la vérité !\par
Laissons donc un moment de côté la théologie physique ou naturelle, et dis-moi s’il te semble raisonnable de solliciter et d’attendre la vie éternelle de ces dieux de théâtre et de comédie ? Le vrai Dieu nous garde d’une si monstrueuse et si sacrilège pensée ! Quoi ! nous demanderions la vie éternelle à des dieux qui se plaisent au spectacle de leurs crimes, cl qu’on ne peut apaiser que par ces infamies !\par
Non, personne ne poussera le délire jusqu’à se jeter dans cet abîme d’impiété. La vie éternelle ne peut donc s’obtenir ni par la théologie fabuleuse ni par la théologie civile. L’une, en effet, imagine des fictions honteuses et l’autre les protège ; l’une sème, l’autre moissonne ; l’une souille les choses divines par les crimes qu’elle invente à plaisir, l’autre met au rang des choses divines les jeux où ces crimes sont représentés ; l’une célèbre en vers les fictions abominables des hommes, l’autre les consacre aux dieux mêmes par des fêtes solennelles ; l’une chante les infamies des dieux et l’autre s’y complaît ; l’une les dévoile ou les invente, l’autre les atteste pour vraies, ou, quoique fausses, y prend plaisir ; toutes deux impures, toutes deux détestables, la théologie effrontée du théâtre étale son impudicité, et la théologie élégante de la cité se pare de cet étalage. Encore une fois, ira-t-on demander la vie éternelle à une théologie qui souille cette courte et passagère vie ? ou, tout en avouant que la compagnie des méchants souille la vie temporelle par la contagion de leurs exemples, soutiendra-t-on que la société des démons, à qui l’on fait un culte de leurs propres crimes, n’a rien de contagieux ni de corrupteur ? Si ces crimes sont vrais, que de malice dans les démons ! s’ils sont faux, que de malice dans ceux qui les adorent !\par
Mais peut-être ceux qui ne sont point versés dans ces matières s’imagineront-ils que c’est seulement dans les poètes et sur le théâtre que la majesté divine est profanée par des fictions et des représentations abominables ou ridicules, et que les mystères où président, non des histrions, mais des prêtres, sont purs de ces turpitudes. Si cela était, on n’eût jamaispensé qu’il fallût faire des infamies du théâtre des cérémonies honorables aux dieux, et jamais les dieux n’eussent demandé de tels honneurs. Ce qui fait qu’on ne rougit point de les honorer ainsi sur la scène, c’est qu’on n’en rougit pas dans les temples. Aussi, quand Varron s’efforce de distinguer la théologie civile de la fabuleuse et de la naturelle, comme une troisième espèce, il donne pourtant assez à entendre qu’elle est plutôt mêlée de l’une et de l’autre que véritablement distincte de toutes deux. Il dit en effet que les fictions des poètes sont indignes de la croyance des peuples, et que les systèmes des philosophes sont au-dessus de leur portée. « Et cependant », ajoute-t-il, « malgré la divergence de la théologie des poètes et de celle des philosophes, on a beaucoup pris à l’une et à l’autre pour composer la théologie civile. C’est pourquoi, en traitant de celle-ci, nous indiquerons ce qu’elle a de commun avec la théologie des poètes, quoiqu’elle doive garder un lien plus intime avec la théologie des philosophes. » La théologie civile n’est donc pas sans rapport avec la théologie des poètes. Il dit ailleurs, j’en conviens, que dans les généalogies des dieux, les peuples ont consulté beaucoup plus les poètes que les philosophes ; mais c’est qu’il parle tantôt de ce qu’on doit faire, et tantôt de ce qu’on fait. Il ajoute que les philosophes ont écrit pour être utiles et les poètes pour être agréables. Par conséquent, ce que les poètes ont écrit, ce que les peuples ne doivent point imiter, ce sont les crimes des dieux, et cependant c’est à quoi les peuples et les dieux prennent plaisir ; car c’est pour faire plaisir et non pour être utiles que les poètes écrivent, de son propre aveu, ce qui ne les empêche pas d’écrire les fictions que les dieux réclament des peuples et que les peuples consacrent aux dieux.
\subsection[{Chapitre VII}]{Chapitre VII}

\begin{argument}\noindent Il y a ressemblance et accord entre la théologie mythique et la théologie civile.
\end{argument}

\noindent Il est donc vrai que la théologie mythique, cette théologie de théâtre, toute pleine de turpitudes et d’indignités, se ramène à la théologie civile, de sorte que celle des deux qu’on réprouve et qu’on rejette n’est qu’une partie de celle qu’on juge digne d’être cultivée et pratiquée. Et quand je dis une partie, je n’entends pas une partie jointe à l’ensemble par un lien artificiel et comme attachée de force ; j’entends une partie homogène unie à toutes les autres comme le membre d’un même corps. Voyez, en effet, les statues des dieux dans les temples ; que signifient leurs figures, leur âge, leur sexe, leurs ornements, sinon ce qu’en disent les poètes ? Si les poètes ont un Jupiter barbu et un Mercure sans barbe, les pontifes ne les ont-ils pas de même ? Priape a-t-il des formes plus obscènes chez les histrions que chez les prêtres, et n’est-il pas, dans les temples où on adore l’image de sa personne, ce qu’il est sur le théâtre où on rit du spectacle de ses mouvements ? Saturne n’est-il pas vieux et Apollon jeune sur les autels comme sur la scène ? Pourquoi Forculus, qui préside aux portes, et Limentinus, qui préside au seuil, sont-ils mâles, tandis que Cardéa, qui veille sur les gonds, est femelle ? N’est-ce pas dans les livres des choses divines qu’on lit tous ces détails que la gravité des poètes n’a pas jugés dignes de leurs chants ? N’y a-t-il que la Diane des théâtres qui soit armée, et celle des temples est-elle vêtue en simple jeune fille ? Apollon n’est-il joueur de lyre que sur la scène, et à Delphes ne l’est-il plus ? Mais tout cela est encore honnête en comparaison du reste, Car Jupiter lui-même, quelle idée s’en sont faite ceux qui ont placé sa nourrice au Capitole ? n’ont-ils pas de la sorte confirmé le sentiment d’Évhémère, qui a soutenu, eu historien exact et non en mythologue bavard, que tous les dieux ont été originairement des hommes ? Et de même ceux qui ont donné à Jupiter des dieux pour commensaux et pour parasites, n’ont-ils pas tourné le culte des dieux en bouffonnerie ? Supposez qu’un bouffon s’avise de dire que Jupiter a des parasites à sa table, on croira qu’il veut égayer le public. Eh bien ! c’est Varron qui dit cela, et Varron ne veut pas faire rire aux dépens des dieux, il veut les rendre respectables ; Varron ne parle pas des choses humaines, mais des choses divines, et ce dont il est question ce n’est pas le théâtre et ses jeux, c’est le Capitole et ses droits. Aussi bien la force de la vérité contraint Varron d’avouer que le peuple, ayant donné aux dieux la forme humaine, a été conduit à se persuader qu’ils étaient sensibles aux plaisirs de l’homme.\par
D’un autre côté, les esprits du mal ne manquaient pas à leur rôle et avaient soin de confirmer par leurs prestiges ces pernicieuses superstitions. C’est ainsi qu’un gardien du temple d’Hercule, étant un jour de loisir et désœuvré, se mit à jouer aux dés tout seul, d’une main pour Hercule et de l’autre pour lui, avec cette condition que s’il gagnait, il se donnerait un souper et une maîtresse aux dépens du temple, et que si la chance tournait du côté d’Hercule, il le régalerait du souper et de la maîtresse à ses dépens. Ce fut Hercule qui gagna, et le gardien, fidèle à sa promesse,lui offrit le souper convenu et la fameuse courtisane Larentina. Or, celle-ci, s’étant endormie dans le temple, se vit en songe entre les bras du dieu, qui lui dit que le premier jeune homme qu’elle rencontrerait en sortant lui payerait la dette d’Hercule. Et en effet elle rencontra un jeune homme fort riche nommé Tarutius qui, après avoir vécu fort longtemps avec-elle, mourut en lui laissant tous ses biens. Maîtresse d’une grande fortune, Larentina, pour ne pas être ingrate envers le ciel, institua le peuple romain son héritier ; puis elle disparut, et on trouva son testament, en faveur duquel on lui décerna les honneurs divins.\par
Si les poètes imaginaient de pareilles aventures et si les comédiens les représentaient, on ne manquerait pas de dire qu’elles appartiennent à la théologie mythique et n’ont rien à démêler avec la gravité de la théologie civile. Mais lorsqu’un auteur si célèbre rapporte ces infamies, non comme des fictions de poètes, mais comme la religion des peuples, non comme des bouffonneries de théâtre et de comédiens, mais comme les mystères sacrés du temple ; quand, en un mot, il les rapporte, non à la théologie fabuleuse, mais à la théologie civile, je dis alors que ce n’est pas sans raison que les histrions représentent sur la scène les turpitudes des dieux, mais que c’est sans raison que les prêtres veulent donner aux dieux dans leurs mystères une honnêteté qu’ils n’ont pas. Quels mystères, dira-t-on ? Je parle des mystères de Junon, qui se célèbrent dans son île chérie de Samos, où elle épousa Jupiter ; je parle des mystères de Cérès, cherchant Proserpine enlevée par Pluton ; je parle des mystères de Vénus, où l’on pleure la mort du bel Adonis, son amant, tué par un sanglier ; je parle enfin des mystères de la mère des dieux, où des eunuques, nommés Galles, déplorent dans leur propre infortune celle du charmant Atys, dont la déesse était éprise et qu’elle mutila par jalousie. En vérité, le théâtre a-t-il rien de plus obscène ? et s’il en est ainsi, de quel droit vient-on nous dire que les fictions des poètes conviennent à la scène, et qu’il faut les séparer de la théologie civilequi convient à l’État, comme on sépare ce qui est impur et honteux de ce qui est honnête et pur ? Il faudrait plutôt remercier les comédiens d’avoir épargné la pudeur publique en ne dévoilant pas sur le théâtre toutes les impuretés que cachent les temples. Que penser de bon des mystères qui s’accomplissent dans les ténèbres, quand les spectacles étalés au grand jour sont si détestables ? Au surplus, ce qui se pratique dans l’ombre par le ministère de ces hommes mous et mutilés, nos adversaires le savent mieux que nous ; mais ce qu’ils n’ont pu laisser dans l’ombre, c’est la honteuse corruption de leurs misérables eunuques. Qu’ils persuadent à qui voudra qu’on fait des œuvres saintes avec de tels instruments ; car enfin ils ont mis les eunuques au nombre des institutions qui se rapportent à la sainteté. Pour nous, nous ne savons pas quelles sont les œuvres des mystères, mais nous savons quels en sont les ouvriers ; nous savons aussi ce qui se fait sur la scène, où jamais pourtant eunuque n’a paru, même dans le chœur des courtisanes, bien que les comédiens soient réputés infâmes et que leur profession ne passe pas pour compatible avec l’honnêteté. Que faut-il donc penser de ces mystères où la religion choisit pour ministres des hommes que l’obscénité duthéâtre ne peut accueillir ?
\subsection[{Chapitre VIII}]{Chapitre VIII}

\begin{argument}\noindent Des interprétations empruntées à la science de la nature par les docteurs du paganisme, pour justifier la croyance aux faux dieux.
\end{argument}

\noindent Mais, dit-on, toutes ces fables ont un sens caché et des explications fondées sur la science de la nature, ou, pour prendre leur langage, des explications physiologiques. Comme s’il s’agissait ici de physiologie et non de théologie, de la nature et non de Dieu ! Et sans doute, le vrai Dieu est Dieu par nature et non par opinion, mais il ne s’ensuit pas que toute nature soit Dieu ; car l’homme, la bête, l’arbre, la pierre ont une nature, et Dieu n’est rien de tout cela. À ne parler en ce moment que des mystères de la mère des dieux, si le fond de ce système d’interprétation se réduit à prétendre que la mère des dieux est le symbolede la terre, qu’avons-nous besoin d’une plus longue discussion ? Est-il possible de donner plus ouvertement raison à ceux qui veulent que tous les dieux du paganisme aient été des hommes ? N’est-ce pas dire que les dieux sont fils de la terre, que la terre est la mère des dieux ? Or, dans la vraie théologie, la terre n’est pas la mère de Dieu, elle est son ouvrage. Mais qu’ils interprètent leurs mystères comme il leur plaira, ils auront beau vouloir les ramener à la nature des choses, il ne sera jamais dans la nature que des hommes servent des femmes ; et ce crime, cette maladie, cette honte sera toujours une chose contre nature. Cela est si vrai qu’on arrache avec peine par les tortures aux hommes les plus vicieux l’aveu d’une prostitution dont on fait profession dans les mystères. Et d’ailleurs, si on excuse ces turpitudes, plus détestables encore que celles du théâtre, sous prétexte qu’elles sont des symboles de la nature, pourquoi ne pas excuser également les fictions des poètes ? car on leur a appliqué le même système d’interprétation, et, pour ne parler que de la plus monstrueuse et la plus exécrable de ces fictions, celle de Saturne dévorant ses enfants, n’a-t-on pas soutenu que cela devait s’entendre du temps, qui dévore tout ce qu’il enfante, ou, selon Varron, des semences qui retombent sur la terre d’où elles sont sorties ? Et cependant on donne à cette théologie le nom de fabuleuse, et malgré les interprétations les plus belles du monde, on la condamne, on la réprouve, on la répudie, et on prétend la séparer, non seulement de la théologie physique, mais aussi de la théologie civile, de la théologie des cités et des peuples, sous prétexte que ses fictions sont indignes de la nature des dieux. Qu’est-ce à dire, sinon que les habiles et savants hommes qui ont écrit sur ces matières réprouvaient également du fond de leur âme la théologie fabuleuse et la théologie civile ? mais ils osaient dire leur pensée sur la première et n’osaient pas la dire sur l’autre. C’est pourquoi, après avoir livré à la critique la théologie fabuleuse, ils ont laissé voir que la théologie civile lui ressemble parfaitement ; de telle sorte qu’au lieu de préférer celle-ci à celle-là, on les rejetât toutes deux ; et ainsi, sans effrayer ceux qui craignaient de nuire à la théologie civile, on conduisait insensiblement les meilleurs esprits à substituer la théologie des philosophes à toutes les autres. En effet, la théologie civile et la théologie fabuleuse sont également fabuleuses et également civiles ; toutes deux fabuleuses, si l’on regarde avec attention les folies et les obscénités de l’une et de l’autre ; toutes deux civiles, si l’on considère que les jeux scéniques, qui sont du domaine de la théologie fabuleuse, font partie des fêtes des dieux et de la religion de l’État. Comment se fait-il donc qu’on vienne attribuer le pouvoir de donner la vie éternelle à ces dieux convaincus, par leurs statues et par leurs mystères, d’être semblables aux divinités ouvertement répudiées de la fable, et d’en avoir la figure, l’âge, le sexe, le vêtement, les mariages, les générations et les cérémonies : toutes choses qui prouvent que ces dieux ont été des hommes à qui l’on a consacré des fêtes et des mystères par l’instigation des démons, selon les accidents de leur vie et de leur mort, ou du moins que ces esprits immondes n’ont manqué aucune occasion d’insinuer dans les esprits leurs tromperies et leurs erreurs.
\subsection[{Chapitre IX}]{Chapitre IX}

\begin{argument}\noindent Des attributions particulières de chaque dieu.
\end{argument}

\noindent Que dire de ces attributions partagées entre les dieux d’une façon si minutieuse et si mesquine, et dont nous avons déjà tant parlé sans avoir épuisé la matière ? Tout cela n’est-il pas plus propre à exciter les bouffonneries d’un comédien qu’à donner une idée de la majesté divine ? Si quelqu’un s’avisait de donner deux nourrices à un enfant, l’une pour le faire manger et l’autre pour le faire boire, à l’exemple des théologiens qui ont employé deux déesses pour ce double office, Educa et Potina, ne le prendrait-on pas pour un fou qui joue chez lui une espèce de comédie ? On nous dit encore que le nom de Liber vient de ce que, dans l’union des sexes, ce dieu aide les mâles à se délivrer de leur semence, et que le nom de Libera, déesse qu’on identifie avec Vénus, a une origine analogue, parce qu’on croit que les femelles ont aussi une semence à répandre, et c’est pour cela que dans le temple on offre à Liber les parties sexuelles de l’homme et à Libera celle de la femme. Ils ajoutent qu’onassigne à Liber les femmes et le vin, parce que c’est Liber qui excite les désirs. De là les incroyables fureurs des bacchanales, et Varron lui-même avoue que les bacchantes ne peuvent faire ce qu’elles font sans avoir l’esprit troublé. Aussi le sénat, devenu plus sage, vit cette fête de mauvais œil et l’abolit. Peut-être en cette rencontre finit-on par reconnaître ce que peuvent les esprits immondes sur les mœurs des hommes, quand on les adore comme des dieux. Quoi qu’il en soit, il est certain que l’on n’oserait rien faire de pareil sur les théâtres. On y joue, il est vrai, mais on n’y est pas ivre de fureur, encore que ce soit une sorte de fureur de reconnaître pour des divinités des esprits qui se plaisent à de pareils jeux.\par
Mais de quel droit Varron prétend-il établir une différence entre les hommes religieux et les superstitieux, sous prétexte que ceux-ci redoutent les dieux comme des ennemis, au lieu que ceux-là les honorent comme des pères, persuadés que leur bonté est si grande qu’il leur en coûte moins de pardonner à un coupable que de punir un innocent ? Cette belle distinction n’empêche pas Varron de remarquer qu’on assigne trois dieux à la garde des accouchées, de peur que Sylvain ne vienne les tourmenter la nuit ; pour figurer ces trois dieux, trois hommes font la ronde autour du logis, frappent d’abord le seuil de la porte avec une cognée, le heurtent ensuite avec un pilon, puis enfin le nettoient avec un balai, ces trois emblèmes de l’agriculture ayant pour effet d’empêcher Sylvain d’entrer ; car c’est le fer qui taille et coupe les arbres, c’est le pilon qui tire du blé la farine, et c’est le balai qui sert à amonceler les grains ; et de là tirent leurs noms : la déesse Intercidona, de l’incision faite par la cognée ; Pilumnus, du pilon ; Deverra, du balai ; en tout trois divinités occupées à préserver les accouchées des violences de Sylvain. Ainsi la protection des divinités bienfaisantes ne peut prévaloir contre la brutalité d’un dieu malfaisant qu’à condition d’être trois contre un, et d’opposer à ce dieu âpre, sauvage et inculte comme les bois où il habite, les emblèmes de culture qui lui répugnent et le font fuir. Oh ! l’admirable innocence ! Oh ! la parfaite concorde des dieux !\par
En vérité sont-ce là les dieux qui protègent les villes ou les jouets ridicules dont le théâtre se divertit ? Que le dieu Jugatinus préside à l’union des sexes, je le veux bien ; mais il faut conduire l’épousée au toit conjugal, et voici le dieu Domiducus ; il faut l’y installer, voici le dieu Domitius ; et pour la retenir près de son mari, on appelle encore la déesse Manturna. N’est-ce point assez ? épargnez, de grâce, la pudeur humaine ! laissez faire le reste dans le secret, à l’ardeur de la chair et du sang. Pourquoi, quand les paranymphes eux-mêmes se retirent, remplir la chambre nuptiale d’une foule de divinités ? Est-ce pour que l’idée de leur présence rende les époux plus retenus ? non ; c’est pour aider une jeune fille, faible et tremblante, à faire le sacrifice de sa virginité. Voici en effet la déesse Virginiensis qui arrive avec le père Subigus, la mère Préma, la déesse Pertunda, Vénus et Priape. Qu’est-ce à dire ? s’il fallait absolument que les dieux vinssent en aide à la besogne du mari, un seul dieu ne suffisait-il pas, ou même une seule déesse ? n’était-ce pas assez de Vénus, puisque c’est elle dont la puissance est, dit-on, nécessaire pour qu’une femme cesse d’être vierge ? S’il reste aux hommes une pudeur que n’ont pas les dieux, les mariés, à la seule pensée de tous ces dieux et de toutes ces déesses qui viennent les aider à l’ouvrage, n’éprouveront-ils pas une confusion qui diminuera l’ardeur d’un des époux et accroîtra la résistance de l’autre ? D’ailleurs, si la déesse Virginiensis est là pour dénouer la ceinture de l’épousée, le dieu Subigus pour la mettre aux bras du mari, la déesse Préma pour la maîtriser et l’empêcher de se débattre, à quoi bon encore la déesse Pertunda ? Qu’elle rougisse, qu’elle sorte, qu’elle laisse quelque chose à faire au mari ; car il est inconvenant qu’un autre que lui s’acquitte de cet office. Aussi bien, si l’on souffre sa présence, c’est sans doute qu’elle est déesse ; car si elle était divinité mâle, si elle était le dieu Pertundus, le mari alors, pour sauver l’honneur de sa femme, aurait plus de sujet d’appeler au secours contre lui, que les accouchées contre Sylvain. Mais que dire d’une autre divinité, cette fois trop mâle, de Priape, qui reçoit la nouvelle épousée sur ses genoux obscènes et monstrueux, suivant la très décente et très pieuse coutume des matrones ? Nos adversaires ont beau jeu après cela d’épuiser les subtilités pour distinguer la théologie civile de la théologie fabuleuse, la cité du théâtre, les temples de la scène, les mystères sacerdotaux des fictions poétiques, comme on distinguerait l’honnêteté de la turpitude, la vérité du mensonge, la gravité du badinage, le sérieux du bouffon, ce qu’on doit rechercher de ce qu’on doit fuir. Nous devinons leur pensée ; ils ne doutent pas au fond de l’âme que la théologie du théâtre et de la fable ne dépende de la théologie civile, et que les fictions des poètes ne soient un miroir fidèle de la théologie civile vient se réfléchir ? Que font-ils donc ? n’osant condamner l’original, ils se donnent carrière à réprouver son image, afin que les lecteurs intelligents détestent à la fois le portrait et l’original. Les dieux, au surplus, trouvent le miroir si fidèle qu’ils se plaisent à s’y regarder, et qui voudra bien les connaître devra étudier à la fois la théologie civile où sont les originaux, et la théologie fabuleuse où sont les copies. C’est pour cela que les dieux ont forcé leurs adorateurs, sous de terribles menaces, à leur dédier les infamies de la théologie fabuleuse, à les solenniser en leur honneur et à les mettre au rang des choses divines ; par où ils ont laissé voir clairement qu’ils ne sont que des esprits impurs, et qu’en faisant d’une théologie livrée au mépris une dépendance et un membre de la théologie respectée, ils ont voulu rendre les pontifes complices des trompeuses fictions des poètes. De savoir maintenant si la théologie païenne comprend encore une troisième partie, c’est une autre question ; il me suffit, je pense, d’avoir montré, en suivant la division de Varron, que la théologie du théâtre et la théologie de la cité sont une seule et même théologie, et puisqu’elles sont toutes deux également honteuses, également absurdes, également pleines d’erreurs et d’indignités, il s’ensuit que toutes les personnes pieuses doivent se garder d’attendre de celle-ci ou de celle-là la vie éternelle.\par
Enfin, Varron lui-même, dans son dénombrement des dieux, part du moment où l’homme est conçu : il met en tête Janus, et, parcourant la longue suite des divinités qui prennent soin de l’homme jusqu’à la plus extrême vieillesse, il termine cette série par la déesse Naenia, c’est-à-dire par l’hymne qu’on chante aux funérailles des vieillards. Il énumère ensuite d’autres divinités dont l’emploi ne se rapporte pas directement à l’homme, mais aux choses dont il fait usage, comme le vivre, le vêtement et les autres objets nécessaires à la vie ; or, dans la revue scrupuleuse où il marque la fonction propre de chaque dieu et l’objet particulier pour lequel il faut s’adresser à lui, nous ne voyons aucune divinité qui soit indiquée ou nommée comme celle à qui l’on doit demander la vie éternelle, l’unique objet pour lequel nous sommes chrétiens. Il faudrait donc avoir l’esprit singulièrement dépourvu de clairvoyance pour ne pas comprendre que, quand Varron développe et met au grand jour avec tant de soin la théologie civile, quand il fait voir sa ressemblance avec la théologie fabuleuse, et donne enfin assez clairement à entendre que cette théologie, si méprisable et si décriée, est une partie de la théologie civile, son dessein est d’insinuer aux esprits éclairés qu’il faut les rejeter toutes deux et s’en tenir à la théologie naturelle, à la théologie des philosophes, dont nous parlerons ailleurs plus amplement au lieu convenable et avec l’assistance de Dieu.
\subsection[{Chapitre X}]{Chapitre X}

\begin{argument}\noindent De la liberté d’esprit de Sénèque, qui s’est élevé avec plus de force contre la théologie civile que Varron contre la théologie fabuleuse.
\end{argument}

\noindent Mais si Varron n’a pas osé répudier ouvertement la théologie civile, quelque peu différente qu’elle soit de la théologie scénique, cette liberté d’esprit n’a pas manqué à Sénèque, qui florissait au temps des Apôtres, comme l’attestent certains documents. Timide dans sa conduite, ce philosophe ne l’a pas été dans ses écrits. En effet, dans le livre qu’il a publié contre les superstitions, il critique la théologie civile avec plus de force et d’étendue que Varron n’avait fait de la théologie fabuleuse. Parlant des statues des dieux : « On fait servir, dit-il, une matière vile et insensible à représenter la majesté inviolable des dieux immortels ; on nous les montre sous la figure d’hommes, de bêtes, de poissons ; on ose même leur donner des corps à double sexe, et ces objets, qui seraient des monstres s’ils étaient animés, on les appelle des dieux ! » Il en vient ensuite à la théologie naturelle, et après avoir rapporté les opinions de quelques philosophes, il se fait l’objection que voici : « Quelqu’un dira : me fera-t-on croire que le ciel et la terre sont des dieux, qu’il y a des dieux au-dessus de la lune et d’autres au-dessous ? Et comment écouter patiemment Platon et Straton le Péripatéticien, l’un qui fait Dieu sans corps, l’autre qui le fait sans âme ? » À quoi Sénèque répond : « Trouvez-vous mieux votre compte dans les institutions de Titius Tatius ou de Romulus ou de Tullus Hostilius ? Titus Tatius a élevé des autels à la déesse Cloacina et Romulus aux dieux Picus et Tibérinus ; Hostilius a divinisé la Peur et la Pâleur, qui ne sont autre chose que de violentes passions de l’homme, celle-là un mouvement de l’âme interdite, celle-ci un mouvement du corps, pas même une maladie, une simple altération du visage. » Aimez-vous mieux, demande Sénèque, croire à de telles divinités, et leur donnerez-vous une place dans le ciel ? Mais il faut voir avec quelle liberté il parle de ces mystères aussi cruels que scandaleux : « L’un, dit-il, se retranche les organes de la virilité ; l’autre se fait aux bras des incisions. Comment craindre la colère d’une divinité quand on se la rend propice par de telles infamies ? Si les dieux veulent un culte de cette espèce, ils n’en méritent aucun. Quel délire, quelle aveugle fureur de s’imaginer qu’on fléchira les dieux par des actes qui répugneraient à la cruauté des hommes ! Les tyrans, dont la férocité traditionnelle a servi de sujet aux tragédies, ont fait déchirer les mamelles de leurs victimes ; ils ne les ont pas obligées de se déchirer de leurs propres mains. On a mutilé des malheureux pour les faire servir aux voluptés des rois ; mais il n’a jamais été commandé à un esclave de se mutiler lui-même. Ces insensés, au-contraire, se déchirent le corps au milieu des temples, et leur prière aux dieux, ce sont des blessures et du sang. Examinez à loisir ce qu’ils font et ce qu’ils souffrent, vous verrez des actes si indignes de personnes d’honneur, d’hommes libres, d’esprits sains, que vous croiriez avoir affaire à une folie furieuse, si les fous n’étaient pas en si grand nombre. Leur multitude est la seule caution de leur bon sens. »\par
Sénèque rappelle ensuite avec le même courage ce qui se passe en plein Capitole, et, en vérité, de pareilles choses, si elles ne sont pas une folie, ne peuvent être qu’une dérision. En effet, dans les mystères d’Égypte, on pleure Osiris perdu, puis on se réjouit de l’avoir retrouvé et sans avoir, après tout, rien retrouvé ni perdu, on fait paraître la même joie et lamême douleur que si tout cela était le plus vrai du monde : « Toutefois, dit Sénèque, cette fureur a une durée limitée ; on peut être fou une fois l’an ; mais montez au Capitole, vous rougirez des extravagances qui s’y commettent et de l’audace avec laquelle la folie s’étale en public. L’un montre à Jupiter les dieux qui viennent le saluer, l’autre lui annonce l’heure qu’il est ; celui-ci fait l’office d’huissier, celui-là joue le rôle de parfumeur et agite ses bras comme s’il répandait des essences. Junon et Minerve ont leurs dévotes, qui, sans se tenir près de leurs statues et même sans venir dans leurs temples, ne laissent pas de remuer les doigts à leur intention, en imitant les mouvements des coiffeuses. Il y en a qui tiennent le miroir ; d’autres prient les dieux de s’intéresser à leurs procès et d’assister aux plaidoiries ; tel autre leur présente un placet ou leur explique son affaire. Un ancien comédien en chef, vieillard décrépit, jouait chaque jour ses rôles au Capitole, comme si un acteur abandonné des hommes était encore assez bon pour les dieux. Enfin, il se trouve là toute une troupe d’artisans de toute espèce qui travaille pour les dieux immortels. » Un peu après, Sénèque ajoute encore : « Toutefois, si ces sortes de gens rendent à la divinité des services inutiles, du moins ne lui en rendent-ils pas de honteux. Mais il y a des femmes qui viennent s’asseoir au Capitole, persuadées que Jupiter est amoureux d’elles, et Junon elle-même, fort colérique déesse, à ce qu’assurent les poètes, Junon ne leur fait pas peur. »\par
Varron ne s’est pas expliqué avec cette liberté ; il n’a eu de courage que pour réprouver la théologie fabuleuse, laissant à Sénèque l’honneur de battre en brèche la théologiecivile. À vrai dire pourtant, les temples où se font ces turpitudes sont plus détestables encoreque les théâtres, où on se contente de les figurer. C’est pourquoi Sénèque veut que le sage, en matière de théologie civile, se contente de cette adhésion tout extérieure qui n’engage pas les sentiments du cœur. Voici ses propres paroles : « Le sage observera toutes ces pratiques comme ordonnées par les lois et non comme agréables aux dieux. » Et quelques lignes plus bas : « Que dirai-je des alliances que nous formons entre les dieux, où la bienséance même n’est pas observée, puisqu’on y marie le frère avec la sœur ? Nous donnons Bellone à Mars, Vénus à Vulcain, Salacie à Neptune. Nous laissons d’autres divinités dans le célibat, faute sans doute d’un parti sortable ; et cependant les veuves ne manquent pas, comme Populonia, Fulgora, Rumina, qui ne doivent pas, j’en conviens, trouver aisément des maris. Il faudra donc se résigner à adorer cette ignoble troupe de divinités, qu’une longue superstition n’a cessé de grossir ; mais nous n’oublierons pas que si nous leur rendons un culte, c’est pour obéir à la coutume plutôt qu’à la vérité. Sénèque avoue donc que ni les lois ni la coutume n’avaient rien institué dans la théologie civile qui fût agréable aux dieux ou conforme à la vérité ; mais, bien que la philosophie eût presque affranchi son âme, il ne laissait pas d’honorer ce qu’il censurait, de faire ce qu’il désapprouvait, d’adorer ce qu’il avait en mépris, et cela parce qu’il était membre du sénat romain. La philosophie lui avait appris à ne pas être superstitieux devant la nature, mais les lois et la coutume le tenaient asservi devant la société ; il ne montait pas sur le théâtre, mais il imitait les comédiens dans les temples : d’autant plus coupable qu’il prenait le peuple pour dupe, tandis qu’un comédien divertit les spectateurs et ne les trompe pas.
\subsection[{Chapitre XI}]{Chapitre XI}\phantomsection
\label{\_chapitre11}

\begin{argument}\noindent Sentiment de Sénèque sur les Juifs.
\end{argument}

\noindent Entre autres superstitions de la théologie civile, ce philosophe condamne les cérémonies des Juifs et surtout leur sabbat, qui lui parait une pratique inutile, attendu que rester le septième jour sans rien faire, c’est perdre la septième partie de la vie, outre le dommage qui peut en résulter dans les nécessités urgentes. Il n’a osé parler toutefois, ni en bien ni en mal, des chrétiens, déjà grands ennemis des Juifs, soit qu’il eût peur d’avoir à les louer contre la coutume de sa patrie, soit aussi peut-être qu’il ne voulût pas les blâmer contre sa propre inclination. Voici comme il s’exprime touchant les Juifs : « Les coutumes de cette nation détestable se sont propagées avec tant de force qu’elles sont reçues parmi toutes les nations ; les vaincus ont fait la loi aux vainqueurs. » Sénèque s’étonnait, parce qu’il ignorait les voies secrètes de la Providence. Recueillons encore son sentiment sur les institutions religieuses des Hébreux : « Il en est parmi eux, dit-il, qui connaissent la raison de leurs rites sacrés mais la plus grande partie du peuple agit sans savoir ce qu’elle fait. » Mais il est inutile que j’insiste davantage sur ce point, ayant déjà expliqué dans mes livres contre les Manichéens, et me proposant d’expliquer encore en son lieu dans le présent ouvrage, comment ces rites sacrés ont été donnés aux Juifs par l’autorité divine, et comment, au jour marqué, la même autorité les a retirés à ce peuple de Dieu qui avait reçu en dépôt la révélation du mystère de la vie éternelle.
\subsection[{Chapitre XII}]{Chapitre XII}

\begin{argument}\noindent Il résulte évidemment de l’impuissance des dieux des Gentils en ce qui touche la vie temporelle, qu’ils sont incapables de donner la vie éternelle.
\end{argument}

\noindent Si ce que j’ai dit dans le présent livre ne suffit pas pour prouver que l’on ne doit demander la vie éternelle à aucune des trois théologies appelées par les Grecs mythique, physique et politique, et par les Latins, fabuleuse, naturelle et civile, si on attend encore quelque chose, soit de la théologie fabuleuse, hautement réprouvée par les païens eux-mêmes, soit de la théologie civile, toute semblable à la fabuleuse et plus détestable encore, je prie qu’on ajoute aux considérations précédentes toutes celles que j’ai développées plus haut, singulièrement dans le quatrième livre où j’ai prouvé que Dieu seul peut donner la félicité. Supposez, en effet, que la félicité fût une déesse, pourquoi les hommes adoreraient-ils une autre qu’elle en vue de la vie éternelle ?\par
Mais comme elle est un don de Dieu, et non pas une déesse, quel autre devons-nous invoquer que le Dieu dispensateur de la félicité, nous qui soupirons après la vie éternelle où réside la félicité véritable et parfaite ? Or, il me semble qu’après ce qui a été dit, personne ne peut plus douter de l’impuissance où sont ces dieux honorés par de si grandes infamies, et plus infâmes encore que le culte exigé par eux, de donner à personne la félicité que nous cherchons. Or, qui ne peut donner la félicité, comment donnerait-il la vie éternelle, qui n’est qu’une félicité sans fin ? Vivre dans les peines éternelles avec ces esprits impurs, ce n’est pas vivre, c’est mourir éternellement. Car quelle mort plus cruelle que cette mort où on ne meurt pas ? Mais comme il est de la nature de l’âme, ayant été faite immortelle, tic conserver toujours quelque vie, la mort suprême pour elle, c’est d’être séparée de la vie de Dieu dans un supplice éternel. D’où il suit que celui-là seul donne la vie éternelle, c’est-à-dire la vie toujours heureuse, qui donne le véritable bonheur. Concluons que, les dieux de la théologie civile étant convaincus de ne pouvoir nous rendre heureux, il ne faut les adorer ni pour les biens temporels, comme nous l’avons fait voir dans nos cinq premiers livres, ni à plus forte raison pour les biens éternels, comme nous venons de le montrer dans celui-ci. Au surplus, comme la coutume jette dans les âmes de profondes racines, si quelqu’un n’est pas satisfait de ce que j’ai dit précédemment contre la théologie civile, je le prie de lire attentivement le livre que je vais y ajouter, avec l’aide de Dieu.
\section[{Livre septième. Les dieux choisis}]{Livre septième. \\
Les dieux choisis}\renewcommand{\leftmark}{Livre septième. \\
Les dieux choisis}

\subsection[{Préface}]{Préface}
\noindent Si je m’efforce de délivrer les âmes des fausses doctrines qu’une longue et funeste erreur y a profondément enracinées, coopérant ainsi de tout mon pouvoir, avec le secours d’en haut, à la grâce de celui qui peut tout faire, parce qu’il est le vrai Dieu, j’espère que ceux de mes lecteurs, dont l’esprit plus prompt et plus perçant a jugé les six précédents livres suffisants pour cet objet, voudront bien écouter avec patience ce qui me reste à dire encore, et, en considération des personnes moins éclairées, ne pas regarder comme superflu ce qui pour eux n’est pas nécessaire. Il ne s’agit point ici d’une question de médiocre importance : il faut persuader aux hommes que ce n’est point pour les biens de cette vie mortelle, fragile et légère comme une vapeur, que le vrai Dieu veut être servi, bien qu’il ne laisse pas de nous donner tout ce qui est ici-bas nécessaire à notre faiblesse, mais pour la vie bienheureuse de l’éternité.
\subsection[{Chapitre premier}]{Chapitre premier}

\begin{argument}\noindent Si le caractère de la divinité, lequel n’est point dans la théologie civile, se rencontre dans les dieux choisis.
\end{argument}

\noindent Que le caractère de la divinité ou (pour mieux rendre le mot grec {\itshape Theotes}) de la {\itshape déité} ne se trouve pas dans la théologie civile exposée en seize livres par Varron, en d’autres termes, que les institutions religieuses du paganisme ne servent de rien pour conduire à la vérité éternelle, c’est ce dont quelques-uns n’auront peut-être pas été entièrement convaincus par ce qui précède ; mais j’ai lieu de croire qu’après avoir lu ce qui va suivre ils n’auront plus aucun éclaircissement à désirer. Les personnes que j’ai en vue ont pu en effet, s’imaginer qu’on doit au moins servir pour la vie bienheureuse, c’est-à-dire pour la vie éternelle, ces dieux choisis que Varron a réservés pour son dernier livre et dont j’ai encore très peu parlé. Or, je me garderai de leur opposer ce mot plus mordant que vrai de Tertullien : « Si on choisit les dieux comme on fait les oignons, tout ce qu’on ne prend pas est de rebut. » Non, je ne dirai pas cela, car il peut arriver que même dans une élite on fasse encore un choix pour quelque fin plus excellente et plus relevée, comme à la guerre on s’adresse pour un coup de main aux jeunes soldats et parmi eux aux plus braves. De même, dans l’Église, quand on fait choix de certains hommes pour être pasteurs, ce n’est pas à dire que le reste des fidèles soit réprouvé, puisqu’il n’en est pas un qui n’ait droit au nom d’élu. C’est ainsi encore qu’en construisant un édifice on choisit les grosses pierres pour les angles, sans pour cela rejeter les autres, qui trouvent également leur emploi ; et enfin, quand on réserve certaines grappes de raisin pour les manger, on n’en garde pas moins les autres pour en faire du vin. Il est inutile de pousser plus loin les exemples. Je dis donc qu’il ne s’ensuit pas, de ce que dans la multitude des dieux païens on en a distingué quelques-uns, qu’il y ait à blâmer ni l’auteur qui rapporte ce choix, ni ceux qui l’ont fait, ni les divinités préférées : il s’agit seulement d’examiner quelles sont ces divinités et pourquoi elles ont été l’objet d’une préférence.
\subsection[{Chapitre II}]{Chapitre II}

\begin{argument}\noindent Quels sont les dieux choisis et si on les regarde comme affranchis des fonctions des petites divinités.
\end{argument}

\noindent Voici les dieux choisis que Varron a compris en un seul livre : Janus, Jupiter, Saturne, Génius, Mercure, Apollon, Mars, Vulcain, Neptune, le Soleil, Orcus, Liber, la Terre, Cérès, Junon, la Lune, Diane, Minerve, Vénus et Vesta ; vingt en tout, douze mâles et huit femelles. Je demande pourquoi ces divinités sont appelées choisies : est-ce parce qu’elles ont des fonctions d’un ordre supérieur dans l’univers ou parce qu’elles ont été plus connues des hommes et ont reçu de plus grands honneurs ? Si c’est la grandeur de leurs emplois qui les distingue, on ne devrait pas les trouver mêlées dans cette populace d’autres divinités chargées des soins les plus bas et les plus minutieux. Par où commencent, en effet, les petites fonctions réparties entre tous ces petits dieux ? à la conception d’un enfant. Or, Janus intervient ici pour ouvrir une issue à la semence. La matière de cette semence regarde Saturne. Il faut aussi Liber pour aider l’homme à s’en délivrer et Libera, qu’ils identifient avec Vénus, pour rendre à la femme le même service. Tous ces dieux sont au nombre des dieux choisis ; mais voici Mena, qui préside aux mois des femmes, déesse assez peu connue, quoique fille de Jupiter. Et cependant Varron, dans le livre des dieux choisis, confère cet emploi à Junon, qui n’est pas seulement une divinité d’élite, mais la reine des divinités ; toute reine qu’elle soit, elle n’en préside pas moins aux mois des femmes, conjointement avec Mena, sa belle-fille. Je trouve encore ici deux autres dieux des plus obscurs, Vitumnus et Sentinus, dont l’un donne la vie, et l’autre le sentiment au nouveau-né. Aussi bien, si peu considérables qu’ils soient, ils font beaucoup plus que toutes ces autres divinités patriciennes et choisies ; car sans la vie et le sentiment, qu’est-ce, je vous prie, que ce fardeau qu’une femme porte dans son sein, sinon un misérable mélange très peu différent de la poussière et du limon ?
\subsection[{Chapitre III}]{Chapitre III}

\begin{argument}\noindent On ne peut assigner aucun motif raisonnable du choix qu’on a fait de certains dieux d’élite, plusieurs des divinités inférieures ayant des fonctions plus relevées que les leurs.
\end{argument}

\noindent D’où vient donc que tant de dieux choisis se sont abaissés à de si petits emplois, au point même de jouer un rôle moins considérable que des divinités obscures, telles que Vitumnus et Sentinus ? Voilà Janus, dieu choisi, qui introduit la semence et lui ouvre pour ainsi dire la porte ; voilà Saturne, autre dieu choisi, qui fournit la semence même ; voilà Liber, encore un dieu choisi, qui aide l’homme à s’en délivrer, et Libera, qu’on appelle aussi Cérès ou Vénus, qui rend à la femme le même service ; enfin, voilà la déesse choisie Junon, qui procure le sang aux femmes pour l’accroissement de leur fruit, et elle ne fait pas seule cette besogne, étant assistée de Mena, fille de Jupiter ; or, en même temps, c’est un Vitumnus, un Sentinus, dieux obscurs et sans gloire, qui donnent la vie et le sentiment : fonctions éminentes, qui surpassent autant celles des autres dieux que la vie et le sentiment sont surpassés eux-mêmes par l’intelligence et la raison. Car autant les êtres intelligents et raisonnables l’emportent sur ceux qui sont réduits, comme les bêtes, à vivre et à sentir, autant les êtres vivants et sensibles l’emportent sur la matière insensible et sans vie. Il était donc plus juste de mettre au rang des dieux choisis Vitumnus et Sentinus, auteurs de la vie et du sentiment, que Janus, Saturne, Liber et Libera, introducteurs, pourvoyeurs ou promoteurs d’une vile semence qui n’est rien tant qu’elle n’a pas reçu le sentiment et la vie. N’est-il pas étrange que ces fonctions d’élite soient retranchées aux dieux d’élite pour être conférées à des dieux très inférieurs en dignité et à peine connus ? On répondra peut-être que Janus préside à tout commencement et qu’à ce titre on est fondé à lui attribuer la conception de l’enfant ; que Saturne préside à toute semence et qu’en cette qualité il a droit à ce que la semence de l’homme ne soit pas retranchée de ses attributions ; que Liber et Libera président à l’émission de toute semence, et que par conséquent celle qui sert à propager l’espèce humaine tombe sous leur juridiction ; que Junon, enfin, préside à toute purgation, à toute délivrance, et que dès lors elle ne peut rester étrangère aux purgations et à la délivrance des femmes ; soit, mais alors que répondra-t-on sur Vitumnus et Sentinus, quand je demanderai si ces dieux président, oui ou non, à tout ce qui a vie et sentiment ? Dira-t-on qu’ils y président ? c’est leur donner une importance infinie ; car, tandis que tout ce qui naît d’une semence naît dans la terre ou sur la terre, vivre et sentir, suivant les païens, sont des privilèges qui s’étendent jusqu’aux astres mêmes dont ils ont fait autant de dieux. Dira-t-on, au contraire, que le pouvoir de Vitumnus et de Sentinus se termine aux êtres qui vivent dans la chair et qui sentent par des organes ? mais alors pourquoi le dieu qui donne la vie et le sentiment à toutes choses ne les donne-t-il pas aussi à la chair ? pourquoi toute génération n’est-elle pas comprise dans son domaine ? et qu’est-il besoin de Vitumnus et de Sentinus ? Que si le dieu de la vie universelle a confié à ces petits dieux, comme à des serviteurs, les soins de la chair, comme choses basses et secondaires, d’où vient que tous ces dieux choisis sont si mal pourvus de domestiques, qu’ils n’ont pu se décharger aussi sur eux de mille détails infimes, et qu’en dépit de toute leur dignité, ils ont été obligés de vaquer aux mêmes fonctions que les divinités du dernier ordre ? Ainsi Junon, déesse choisie, reine des dieux, sœur et femme de Jupiter, partage, sous le nom d’Iterduca, le soin de conduire les enfants avec deux déesses de la plus basse qualité, Abéona et Adéona. On lui adjoint encore la déesse Mens, chargée de donner bon esprit aux enfants, et qui néanmoins n’a pas été mise au rang des divinités choisies, quoiqu’un bon esprit soit assurément le plus beau présent qu’on puisse faire à l’homme. Chose singulière ! l’honneur qu’on refuse à Mens, on l’accorde à Junon Iterduca et Domiduca, comme s’il servait de quelque chose de ne pas s’égarer en chemin et de revenir chez soi, quand on n’a pas l’esprit comme il faut. Certes, la déesse qui le rend bien fait méritait d’être préférée à Minerve, à qui on a donné, parmi tant de menues fonctions, celle de présider à la mémoire des enfants. Qui peut douter qu’il ne vaille beaucoup mieux avoir un bon esprit que de posséder la meilleure mémoire ? Nul ne saurait être méchant avec un bon esprit, au lieu qu’il y a de très méchantes personnes qui ont une mémoire admirable, et elles sont d’autant plus méchantes qu’elles peuvent moins oublier leurs méchantes pensées. Cependant Minerve est du nombre des dieux choisis, tandis que Mens est perdue dans la foule des petits dieux. Que n’aurais-je pas à dire de la Vertu et de la Félicité, si je n’en avais déjà beaucoup parlé au quatrième livre ? On en a fait des déesses, et néanmoins on n’a pas voulu les mettre au rang des divinités d’élite, bien qu’on y mît Mars et Orcus, dontl’un est chargé de faire des morts et l’autre de les recevoir. Puis donc que nous voyons les dieux d’élite confondus dans ces fonctions mesquines avec les dieux inférieurs, comme des membres du sénat avec la populace, et que même quelques-uns de ces petits dieux ont des offices plus importants et plus nobles que les dieux qu’on appelle choisis, il s’ensuit que ceux-ci n’ont pas mérité leur rang par la grandeur de leurs emplois dans le gouvernement du monde, mais qu’ils ont eu seulement la bonne fortune d’être plus connus des peuples. C’est ce qui fait dire à Varron lui-même qu’il est arrivé à certains dieux et à certaines déesses du premier ordre de tomber dans l’obscurité, comme cela se voit parmi les hommes. Mais alors, si on a bien fait de ne pas placer la Félicité parmi les dieux choisis, parce que c’est le hasard et non le mérite qui a donné à ces dieux leur rang, au moins fallait-il placer avec eux, et même au-dessus d’eux, la Fortune, qui passe pour dispenser au hasard ses faveurs. Évidemment elle avait droit à la première place parmi les dieux choisis ; c’est envers eux, en effet, qu’elle a montré ce dont elle est capable, tous ces dieux ne devant leur grandeur ni à l’éminence de leur vertu, ni à une juste félicité, {\itshape mais à la puissance aveugle et téméraire de la Fortune}, comme parlent ceux qui les adorent. N’est-ce pas aux dieux que fait allusion l’éloquent Salluste, quand il dit : « La Fortune gouverne le monde ; c’est elle qui met tout en lumière et qui obscurcit tout, plutôt par caprice que par raison. » Je défie les païens, en effet, d’assigner la raison qui fait que Vénus est en lumière, tandis que la Vertu, déesse comme elle et d’un tout autre mérite, est dans l’obscurité. Dira-t-on que l’éclat de Vénus vient de la masse de ses adorateurs, beaucoup plus nombreux, en effet, que ceux de la Vertu ? mais alors pourquoi Minerve est-elle si renommée, et la déesse Pecunia si inconnue ? car assurément la science est beaucoup moins recherchée par les hommes que l’argent, et entre ceux qui cultivent les sciences et les arts, il en est bien peu qui ne s’y proposent la récompense et le gain. Or, ce qui importe avant tout, c’est la fin qu’on poursuit en faisant une chose, plutôt que la chose même qu’on fait, Si donc l’élection des dieux a dépendu de la populace ignorante, pourquoi la déesse Pecunia n’a-t-elle pas été préférée à Minerve, la plupart des hommes ne travaillant qu’en vue de l’argent ? et si, au contraire, c’est un petit nombre de sages qui a fait le choix, pourquoi la Vertu n’a-t-elle pas été préférée à Vénus, quand la raison lui donne une préférence si marquée ? La Fortune tout au moins, qui domine le monde, au sentiment de ceux qui croient à son immense pouvoir, la Fortune, qui met au grand jour ou obscurcit toute chose plutôt par caprice que par raison, s’il est vrai qu’elle ait eu assez de puissance sur les dieux eux-mêmes pour les rendre à son gré célèbres ou obscurs, la Fortune, dis-je, devrait occuper parmi les dieux choisis la première place. Pourquoi ne l’a-t-elle pas obtenue ? serait-ce qu’elle a eu la fortune contraire ? Voilà la fortune contraire à elle-même ; la voilà qui sait tout faire pour élever les autres et ne sait rien faire pour soi.
\subsection[{Chapitre IV}]{Chapitre IV}

\begin{argument}\noindent On a mieux traité les dieux inférieurs, qui ne sont souillés d’aucune infamie, que les dieux choisis, chargés de mille turpitudes.
\end{argument}

\noindent Je concevrais qu’un esprit amoureux de l’éclat et de la gloire félicitât les dieux choisis de leur grandeur et les regardât comme heureux, s’il pouvait ignorer que cette grandeur même leur est plus honteuse qu’honorable. En effet, la foule des petites divinités est protégée contre l’opprobre par son obscurité bien qu’il soit difficile de ne pas rire quand on voit cette troupe de dieux occupés aux différents emplois que leur a départis la fantaisie humaine : semblables à l’armée des petits fermiers d’impôts, ou encore à ces nombreux ouvriers qui, dans la rue des Orfèvres, travaillent à un seul vase, où chacun met un peu du sien, quand il suffirait d’un habile homme pour l’achever ; mais on a jugé que le meilleur emploi de cette multitude d’ouvriers, c’était de leur diviser le travail, afin que chacun fît sa part de l’œuvre avec promptitude et facilité, au lieu d’acquérir par un long et pénible labeur le talent d’accomplir l’œuvre tout entière. Quoi qu’il en soit, il en est fort peu parmi ces petits dieux dont la réputation ait souffert quelque atteinte, au lieu, qu’on aurait de la peine à citer un seul des grands dieux qui ne soit déshonoré par quelque infamie. Les grands dieux sont descendus aux basses fonctions des petits ; mais les petits dieux ne se sont pas élevés aux crimes sublimes des grands. Pour Janus, il est vrai, je ne vois pas qu’on dise rien de lui qui souille son honneur, et peut-être a-t-il mené une meilleure vie que les autres. Il fit bon accueil à Saturne fugitif et partagea avec lui son royaume, d’où prirent naissance les deux villes de Janiculum et de Saturnia ; mais les païens, empressés de mettre à tout prix du scandale dans le culte de leurs dieux, ont déshonoré l’image de celui-ci, faute de pouvoir déshonorer sa vie ; ils l’ont représenté avec un corps double et monstrueux, ayant deux et même quatre visages. Serait-ce par hasard qu’il a fallu donner du front en abondance à ce dieu vertueux, les autres dieux n’en ayant pas assez pour rougir de leur turpitude ?
\subsection[{Chapitre V}]{Chapitre V}

\begin{argument}\noindent De la doctrine secrète des païens et de leur explication de la théologie par la physique.
\end{argument}

\noindent Mais écoutons les explications physiques dont ils se servent pour couvrir des apparences d’une doctrine profonde la turpitude de leurs misérables superstitions. Varron prétend que les statues des dieux, leurs attributs et leurs ornements ont été institués par les anciens, afin que les esprits initiés au sens mystérieux de ces symboles pussent, en les voyant, s’élever à la contemplation de l’âme du monde et de ses parties, c’est-à-dire à la connaissance des dieux véritables. Si on a représenté la divinité sous une figure humaine, c’est, selon lui, parce que l’esprit qui anime le corps de l’homme est semblable à l’esprit divin. Supposez, dit-il, qu’on se serve de différents vases pour distinguer les dieux, un œnophore placé dans le temple de Bacchus servira à désigner le vin ; le contenant sera le signe du contenu ; c’est ainsi qu’une statue de forme humaine est le symbole de l’âme raisonnable dont le corps humain est comme le vase et qui par son essence est semblable à l’âme des dieux. Voilà les mystères de doctrine où Varron avait pénétré et qu’il a voulu révéler au monde. Mais, je vous le demande, ô habile homme ! n’auriez-vous pas égaré dans ces profondeurs le sens judicieux qui vous faisait dire tout à l’heure que les premiers instituteurs du culte des idoles ont ôté aux peuples la crainte pour la remplacer par la superstition, et que les anciens qui n’avaient point d’idoles adoraient les dieux d’un culte plus pur ? C’est l’autorité de ces vieux Romains qui vous a donné la hardiesse de parler de la sorte à leurs descendants, et peut-être si l’antiquité eût adoré des idoles, eussiez-vous enseveli dans un silence discret cet hommage à la vérité, et célébré d’une voix plus pompeuse encore et plus complaisante les mystères de sagesse cachés sous une vaine et pernicieuse idolâtrie. Et cependant tous ces mystères n’ont pu élever votre âme, malgré les trésors de science et de lumière que nous aimons à y reconnaître et qui redoublent nos regrets, jusqu’à la connaissance de son Dieu, de ce Dieu qui est son principe créateur et non sa substance, dont elle n’est point une partie, mais une production, qui n’est pas l’âme de toutes choses, mais l’auteur de toutes les âmes et la source unique de la béatitude pour celles qui se montrent touchées de ses dons. — Au surplus, que signifient au fond et que valent les mystères du paganisme ? c’est ce que nous aurons tout à l’heure à examiner de près. Constatons, dès ce moment, cet aveu de Varron, que l’âme du monde et ses parties sont les dieux véritables ; d’où il suit que toute sa théologie, même la naturelle qu’il tient en si haute estime, ne s’est pas élevée au-dessus de l’idée de l’âme raisonnable. Il s’étend du reste fort peu sur cette théologie naturelle dans le livre où il en parle, et nous verrons si, avec ses explications physiologiques, il parvient à y ramener cette partie de la théologie civile qui regarde les dieux choisis. S’il le fait, toute la théologie sera théologie naturelle ; et alors quel besoin d’en séparer si soigneusement la théologie civile ? Veut-il que cette séparation soit légitime ? en ce cas, la théologie naturelle, qui lui plaît si fort, n’étant déjà pas la théologie vraie, puisqu’elle s’arrête à l’âme et ne s’élève pas jusqu’au vrai Dieu, créateur de l’âme, à combien plus forte raison la théologie civile sera-t-elle méprisable ou fausse, puisqu’elle s’attache presque uniquement à la nature corporelle, comme on pourra le voir par quelques-unes des savantes et subtiles explications que j’aurai à citer dans la suite.
\subsection[{Chapitre VI}]{Chapitre VI}

\begin{argument}\noindent De cette opinion de Varron que Dieu est l’âme du monde et qu’il comprend en soi une multitude d’âmes particulières dont l’essence est divine.
\end{argument}

\noindent Varron dit encore, dans son introduction à la théologie naturelle, qu’il croit que Dieu est l’âme du monde ou du {\itshape cosmos}, comme parlent les Grecs, et que ce monde est Dieu ; mais de même qu’un homme sage, quoique formé d’une âme et d’un corps, est appelé sage à cause de son âme, ainsi le monde est appelé Dieu à cause de l’âme qui le gouverne, bien qu’il soit également composé d’une âme et d’un corps. Il semble ici que Varron reconnaisse en quelque façon l’unité de Dieu ; mais pour faire en même temps la part du polythéisme, il ajoute que le monde est divisé en deux parties, le ciel et la terre, le ciel en deux autres, l’éther et l’air, la terre, de même, en eau et en continent ; que l’éther occupe la région la plus haute, l’air la seconde, l’eau la troisième, la terre enfin la plus basse région ; que ces quatre éléments sont l’emplis d’âmes, le feu et l’air d’âmes immortelles, l’eau et la terre d’âmes mortelles ; que dans l’espace qui s’étend depuis la limite circulaire du ciel jusqu’au cercle de la lune habitent les âmes éthérées, qui sont les astres et les étoiles, dieux célestes, visibles aux sens en même temps qu’intelligibles à la raison ; qu’entre la sphère lunaire et la partie de l’air où se forment les nuées et les vents habitent les âmes aériennes, que l’esprit conçoit sans que les yeux les puissent voir, c’est-à-dire les héros, les lares, les génies ; voilà l’abrégé que nous offre Varron de sa théologie naturelle qui est aussi celle d’un grand nombre de philosophes. Nous aurons à l’examiner à fond, quand ce qui nous reste à dire sur la théologie civile relativement aux dieux choisis aura été conduit à bonne fin, avec la grâce de Dieu.
\subsection[{Chapitre VII}]{Chapitre VII}

\begin{argument}\noindent Était-il raisonnable de faire deux divinités de Janus et de Terme ?
\end{argument}

\noindent Je demande d’abord ce que c’est que Janus, qu’on place à la tête de ces dieux choisis ? on me dit : c’est le monde. Voilà une réponse courte et claire assurément ; mais pourquoi n’attribue-t-on à Janus que le commencement des choses, tandis qu’on en réserve la fin à un autre dieu nommé Terme ? car c’est pour cela, dit-on, qu’en dehors des dix mois qui s’écoulent de mars à décembre, on a consacré deux mois à ces divinités, janvier à Janus et février à Terme ; d’où vient aussi que les Terminales se célèbrent en février et qu’il s’y fait une cérémonie expiatrice appelée {\itshape Februum}, laquelle a donné au mois son nom. Quoi donc ! est-ce à dire que le commencement des choses appartienne à Janus et que la fin ne lui appartienne pas, étant réservée à un autre dieu ? Mais n’est-il pas reconnu des païens que tout ce qui prend commencement en ce monde y prend également fin ? Voilà une dérision étrange de ne donner à ce dieu qu’une demi-puissance dans la réalité, tandis qu’on donne à sa statue un double visage ! Ne serait-ce pas une explication plus heureuse de cet emblème, de dire que Janus et Terme sont un seul et même dieu dont une face répond au commencement des choses et l’autre à leur fin ? car on ne peut agir sans considérer ces deux points. Quiconque, en effet, perd de vue le commencement de son action, ne saurait en prévoir la fin, et il faut que l’intention qui regarde l’avenir se lie à la mémoire qui regarde le passé. Autrement, après avoir oublié par où on a commencé, on ne sait plus par où finir. Dira-t-on que si la vie bienheureuse commence dans le monde, elle s’achève ailleurs, et que c’est pour cela que Janus, qui est le monde, n’a de pouvoir que sur les commencements ? mais à ce compte on aurait dû mettre le dieu Terme au-dessus de Janus, au lieu de l’écarter du nombre des divinités choisies ; et même dès cette vie, où l’on partage le commencement et la fin des choses entre Jan us et Terme, Terme aurait dû être plus honoré que Janus. C’est en effet quand on touche au terme d’une entreprise qu’on éprouve le plus de joie. Les commencements sont pleins d’inquiétude, et l’âme n’est tranquille qu’en voyant la fin de son action ; c’est à la fin qu’elle tend ; c’est la fin qu’elle désire, qu’elle espère, qu’elle appelle de ses vœux, et il n’y a de triomphe pour elle que dans le complet achèvement.
\subsection[{Chapitre VIII}]{Chapitre VIII}

\begin{argument}\noindent Pourquoi les adorateurs de Janus lui ont donné tantôt deux visages et tantôt quatre.
\end{argument}

\noindent Mais voyons un peu comment on explique cette statue à double face. On dit que Janus a deux visages, l’un devant, l’autre derrière, parce que notre bouche ouverte a quelque ressemblance avec la forme du monde, ce qui fait que les Grecs ont appelé le palais de la bouche {\itshape ouranos} (ciel), comme aussi quelques poètes latins ont donné au ciel le nom de palais. Ce n’est pas tout : notre bouche ouverte a deux issues, l’une extérieure du côté des dents ; l’autre intérieure vers le gosier. Et voilà ce qu’on a fait du monde avec un mot grec ou poétique qui signifie palais ! Mais quel rapport y a-t-il entre tout cela et l’âme et la vie éternelle ? Qu’on adore ce dieu seulement pour la salive qui entre ou sort sous le ciel du palais, je le veux bien ; mais quoi de plus absurde à des gens incapables de trouver dans le monde deux portes opposées l’une à l’autre et servant à y introduire les choses du dehors et à en rejeter celles du dedans, que de vouloir, de notre bouche et de notre gosier auxquels le monde ne ressemble en rien, figurer le monde sous les traits de Janus, à cause du palais seul auquel Janus ne ressemble pas davantage ? D’autre part, quand on lui donne quatre faces en le nommant double Janus, on veut y voir un emblème des quatre parties du monde ; comme si le monde regardait quelque chose hors de soi ainsi que Janus regarde par ses quatre visages ! Et puis, si Janus est le monde et si le monde a quatre parties, il s’ensuit que le Janus à deux faces est une fausse image, ou si elle est vraie en ce sens que l’Orient et l’Occident embrassent le monde entier, l’emblème ne laisse pas d’être faux à un autre point de vue ; car en considérant les deux autres parties du monde, le Septentrion et le Midi, nous ne disons pas que le monde est double, comme on appelle double le Janus à quatre visages. Toujours est-il que si on a trouvé dans la bouche de l’homme une analogie avec le Janus à double visage, on ne saurait trouver dans le monde rien qui ressemble aux quatre portes figurées par les quatre visages de Janus ; à moins que Neptune n’arrive au secours des interprètes, tenant à la main un poisson qui, outre la bouche et le gosier, nous présente à droite et à gauche la double ouverture de ses ouïes. Et cependant, avec toutes ces portes, il n’en est pas une seule par laquelle l’âme puisse échapper aux vaines superstitions, à moins qu’elle n’écoute la vérité, qui a dit : « Je suis la porte. »
\subsection[{Chapitre IX}]{Chapitre IX}

\begin{argument}\noindent De la puissance de Jupiter, et de ce dieu comparé à Janus.
\end{argument}

\noindent Je voudrais encore savoir quel est ce Jovis qu’ils nomment aussi Jupiter. C’est, disent-ils, le dieu de qui dépendent les causes de tout ce qui se fait dans le monde. Voilà une fonction admirable et dont Virgile exprime fort bien la grandeur dans ce vers célèbre\par
 {\itshape « Heureux qui a pu connaître les causes des choses ! »} \par
Mais d’où vient qu’on place Jupiter après Janus ? Que le docte et pénétrant Varron nous réponde là-dessus : « C’est, dit-il, que Janus gouverne le commencement des choses, et Jupiter leur accomplissement. Il est donc juste que Jupiter soit estimé le roi des dieux ; car si l’accomplissement a la seconde place dans l’ordre du temps, il a la première dans l’ordre de l’importance. » Cela serait vrai s’il s’agissait ici de distinguer dans les choses l’origine et le terme de leur développement. Ainsi, partir est l’origine d’une action, arriver en est le terme ; l’étude est une action qui commence et qui-se termine à la science ; or partout, en général, le commencement n’est le premier qu’en date et la perfection est dans la fin. C’est un procès déjà vidé entre Janus et Terme mais les causes dont on donne le gouvernement à Jupiter sont des principes efficients et non des effets ; et il est impossible, même dans l’ordre du temps, que les effets et les commencements des effets soient avant les causes ; car ce qui fait une chose est toujours antérieur à la chose qui est faite. Qu’importe donc que les commencements soient gouvernés par Janus ? ils n’en sont pas pour celaantérieurs aux causes efficientes gouvernées par Jupiter ; car de même que rien n’arrive, rien aussi ne commence qui ne soit précédé d’une cause. Si donc c’est ce dieu, arbitre de toutes les causes et de tout ce qui existe et arrive dans la nature, que l’on salue du nom de Jupiter et que l’on adore par tant d’opprobres et d’infamies, je, dis qu’il y a là une impiété plus grande qu’à ne reconnaître aucun dieu, Ne serait-il pas, en effet, préférable d’appeler Jupiter quelque objet digne de ces adorations honteuses, quelque fantôme, par exemple, comme celui qu’on présenta, dit-on, à Saturne à la place de son enfant, plutôt que de se figurer un dieu tout à la fois tonnant et adultère, maître du monde et asservi à l’impudicité, disposant de toutes les causes des actions naturelles et ne sachant pas donner des causes légitimes à ses propres actions ?\par
Je demanderai ensuite, en supposant que Janus soit le monde, quel sera le rôle de Jupiter parmi les dieux ? Varron n’a-t-il pas déclaré que les vrais dieux sont l’âme du monde et ses parties ? par conséquent tout ce qui n’est pas cela n’est pas vraiment dieu. Dira-t-on que Jupiter est l’âme du monde et que Janus en est le corps, c’est-à-dire qu’il est le monde visible ? Mais à ce compte Janus n’est pas vraiment dieu, puisqu’il est accordé par nos adversaires que la divinité consiste, non dans le corps du monde, mais dans l’âme du monde et dans ses parties ; et c’est ce qui a fait dire nettement à Varron que Dieu, pour lui, n’est autre chose que l’âme du monde, et que si le monde lui-même est appelé Dieu, c’est au même sens où un homme est appelé sage à cause de son âme, bien qu’il soit composé d’une âme et d’un corps ; ainsi le monde, quoique formé d’une âme et d’un corps, doit à son âme seule d’être appelé dieu. D’où il suit que le corps du monde, pris isolément, n’est pas dieu ; il n’y a de divin que l’âme toute seule, ou la réunion de l’âme et du corps, de telle façon pourtant que dans cette réunion même, la divinité vienne de l’âme et non pas du corps. Si donc Janus est le monde, et si Janus est dieu, comment Jupiter sera-t-il dieu, à moins d’être une partie de Janus ? Or, on a coutume, au contraire, d’attribuer l’univers entier à Jupiter, d’où vient ce mot du poète :\par
 {\itshape « … Tout est plein de Jupiter. »} \par
Si donc on veut que Jupiter soit dieu, bien plus qu’il soit le roi des dieux, il faut nécessairement qu’il soit le monde, afin de pouvoir régner sur les autres dieux, c’est-à-dire sur ses propres parties. Voilà sans doute en quel sens Varron, dans cet autre ouvrage qu’il a composé sur le culte des dieux, rapporte les deux vers suivants de Valérius Soranus :\par
 {\itshape « Jupiter tout-puissant, père et mère des rois, des choses et des dieux, dieu unique, embrassant tous les dieux. »} \par
Varron explique en son traité que le mâle est ici le principe qui répand la semence, et la femelle celui qui la reçoit ; or, Jupiter étant le monde, toute semence vient de lui et rentre en lui : « C’est pourquoi, ajoute Varron, Soranus appelle Jupiter père et mère, et fait de lui tout ensemble l’unité et le tout ; car « le monde est un et cet un comprend tout. »
\subsection[{Chapitre X}]{Chapitre X}

\begin{argument}\noindent S’il était raisonnable de distinguer Janus de Jupiter.
\end{argument}

\noindent Si donc Janus est le monde, et si Jupiter l’est aussi, pourquoi, n’y ayant qu’un seul monde, Janus et Jupiter sont-ils deux dieux ? pourquoi ont-ils chacun son temple et ses autels, ses sacrifices et ses statues ? Dira-t-on qu’autre chose est la vertu des commencements, autre chose celle des causes, et que c’est pour cela qu’on a nommé l’une Janus et l’autre Jupiter ? Je demanderai à mon tour si parce qu’un homme est revêtu d’un double pouvoir ou parce qu’il exerce une double profession, on est autorisé à voir en lui deux magistrats ou deux artisans ? Pourquoi donc d’un seul Dieu, qui gouverne les commencements et les causes, ferait-on deux dieux distincts, sous prétexte que les commencements et les causes sont deux choses distinctes ? À ce compte, il faudrait dire aussi que Jupiter est à lui seul autant de dieux qu’on lui a donné de noms différents à cause de ses attributions différentes, puisque les objets qui sont l’origine de ces noms sont différents. Je vais en citer quelques exemples.
\subsection[{Chapitre XI}]{Chapitre XI}

\begin{argument}\noindent Des divers surnoms de Jupiter, lesquels ne se rapportent pas à plusieurs dieux, mais a un seul.
\end{argument}

\noindent Jupiter a été appelé Victor, Invictus, Opitulus, Inpulsor, Stator, Centipeda, Supinalis, Tigillus, Almus, Ruminus, et autres surnoms qu’il serait trop long d’énumérer ; tous ces titres sont fondés sur la diversité des puissances d’un même dieu, et non sur la diversité de plusieurs dieux. On a nommé Jupiter Victor, parce qu’il est toujours vainqueur ; Invictus, parce qu’il est invincible ; Opitulus, parce qu’il est secourable aux faibles ; Propulsor et Stator, Centipeda et Supinalis, parce qu’il donne et arrête le mouvement, parce qu’il soutient et renverse tout ; Tigillus, parce qu’il est l’appui du monde ; Almus, parce qu’il nourrit les êtres ; Ruminus, parce qu’il allaite les animaux. De toutes ces fonctions, il est assez clair que les unes sont grandes, les autres mesquines, et cependant on les attribue au même dieu. De plus, n’y a-t-il pas plus de rapport entre les causes et les commencements des choses, qu’entre soutenir le monde et donner la mamelle aux animaux ? Et cependant on a voulu, pour les commencements et les causes, admettre deux dieux, Janus et Jupiter, en dépit de l’unité du monde, au lieu que pour deux fonctions bien différentes en importance et en dignité on s’est contenté du seul Jupiter, en l’appelant tour à tour Tigillus et Ruminus. Je pourrais ajouter qu’il eût été plus à propos de faire donner la mamelle aux animaux par Junon que par Jupiter, du moment surtout qu’il y avait là une autre déesse, Rumina, toute prête à l’aider dans cet office ; mais on me répondrait que Junon elle-même n’est autre que Jupiter, comme cela résulte des vers de Valérius. Soranus déjà cités :\par
 {\itshape « Jupiter tout-puissant, père et mère des rois, des choses et des dieux. »} \par
Mais alors pourquoi l’appeler Ruminus, du moment, qu’à y regarder de près, il est aussi la déesse Rumina ? Si, en effet, c’est une chose indigne de la majesté des dieux, comme nous l’avons montré plus haut, que pour un même épi de blé, un dieu soit chargé des nœuds du tuyau et un autre de l’enveloppe des grains, combien n’est-il pas plus indigne encore qu’une fonction aussi misérable que l’allaitement des animaux soit partagée entre deux dieux, dont l’un est Jupiter même, le roi de tous les dieux, et qu’il la remplisse, non pas avec sa femme Junon, mais avec je ne sais quelle absurde Rumina ? à moins qu’il ne soit tout ensemble Ruminus et Rumina, Ruminus pour les mâles et Rumina pour les femelles. Dirai-je qu’ils n’ont pas voulu donner à Jupiter un nom féminin ? mais il est appelé père et mère dans les vers qu’on vient de lire, et d’ailleurs je rencontre sur la liste de ses noms celui d’une de ces petites déesses que nous avons mentionnées au quatrième livre, la déesse Pecunia. Sur quoi je demande pour quel motif on n’a pas admis Pecunius avec Pecunia, comme on a fait Ruminus avec Rumina ; car enfin, mâles et femelles, tous les hommes regardent à l’argent.
\subsection[{Chapitre XII}]{Chapitre XII}

\begin{argument}\noindent Jupiter est aussi appelé Pecunia.
\end{argument}

\noindent Mais quoi ! ne faut-il pas admirer la raison ingénieuse qu’on donne de ce surnom ? Jupiter, dit-on, s’appelle Pecunia, parce que tout est à lui. Ô la belle raison d’un nom divin ! et n’est-ce pas plutôt avilir et insulter celui à qui tout appartient que de le nommer Pecunia ? car au prix de ce qu’enferment le ciel et la terre, que vaut la richesse des hommes ? C’est l’avarice qui seule a donné ce nom à Jupiter, pour fournir à ceux qui aiment l’argent le prétexte d’aimer une divinité, et non pas quelque déesse obscure, mais le roi même des dieux. Il n’en serait pas de même si on l’appelait Richesse. Car autre chose est la richesse, autre chose est l’argent. Nous appelons riches ceux qui sont sages, justes, gens de bien quoique n’ayant pas d’argent ou en ayant peu ; car ils sont effectivement riches en vertus qui leur enseignent à se contenter de ce qu’ils ont, alors même qu’ils sont privés des commodités de la vie ; nous disons au contraire que les avares sont pauvres, parce que, si grands que soient leurs trésors, comme ils en désirent toujours davantage, ils sont toujours dans l’indigence. Nous disons encore fort bien que le vrai Dieu est riche, non certes en argent, mais en toute-puissance. Je sais que les hommes pécunieux sont aussi appelés riches, mais ils sont pauvres au dedans, s’ils sont cupides. Je sais aussi qu’un homme sans argent est réputé pauvre, mais il est riche au dedans, s’il est sage. Quel cas peut donc faire un homme sage d’une théologie qui donne au roi des dieux le nom d’une chose qu’aucun sage n’a jamais désirée ? n’eût-il pas été plus simple, sans la radicale impuissance du paganisme à rien enseigner d’utile à la vie éternelle, de donner au souverain Maître du monde le nom de Sagesse plutôt que celui de Pecunia ? car c’est l’amour de la sagesse qui purifie le cœur des souillures de l’avarice, c’est-à-dire de l’amour de l’argent.
\subsection[{Chapitre XIII}]{Chapitre XIII}

\begin{argument}\noindent Saturne et Génius ne sont autres que Jupiter.
\end{argument}

\noindent Mais à quoi bon parler davantage de ce Jupiter, à qui peut-être il convient de rapporter toutes les autres divinités ? Et dès lors la pluralité des dieux ne subsiste plus, du moment que Jupiter les comprend tous, soit qu’on les regarde comme ses parties ou ses puissances, soit qu’on donne à l’âme du monde partout répandue le nom de plusieurs dieux à cause des différentes parties de l’univers ou des différentes opérations de la nature. Qu’est-ce, en effet, que Saturne ? « C’est, dit Varron, un des principaux dieux, dont le pouvoir s’étend sur toutes les semences. » Or, n’a-t-il pas expliqué tout à l’heure les vers de Valérius Soranus en soutenant que Jupiter est le monde, qu’il répand hors de soi toutes les semences et les absorbe toutes en soi ? Jupiter ne diffère donc pas du dieu dont le pouvoir s’étend sur toutes les semences. Qu’est-ce maintenant que Génius ? « Un dieu, dit Varron, qui a autorité et pouvoir sur toute génération. » Mais le dieu qui a ce pouvoir, qu’est-il autre chose que le monde, invoqué par Valérius sous le nom de « Jupiter père et mère de toutes choses » ? Et quand Varron soutient ailleurs que Génius est l’âme raisonnable de chaque homme, assurant d’autre part que c’est l’âme raisonnable du monde qui est Dieu, ne donne-t-il pas à entendre que l’âme du monde est une sorte de Génie universel ? C’est donc ce Génie que l’on nomme Jupiter ; car si vous entendez que tout Génie soit un dieu et que l’âme de chaque homme soit un Génie, il en résultera que l’âme de chaque homme sera un dieu, conséquence tellement absurde que les païens eux-mêmes sont obligés de la-rejeter ; d’où il suit qu’il ne leur reste plus qu’à nommer proprement et par excellence Génius le dieu, qui est, suivant eux, l’âme du monde, c’est-à-dire Jupiter.
\subsection[{Chapitre XIV}]{Chapitre XIV}

\begin{argument}\noindent Des fonctions de Mercure et de Mars.
\end{argument}

\noindent Quant à Mercure et à Mars, ne sachant comment les rapporter à aucune partie du monde ni à aucune opération divine sur les éléments, ils se sont contentés de les faire présider à quelques autres actions humaines et de leur donner puissance sur la parole et sur la guerre. Or, si le pouvoir de Mercure s’étend aussi sur la parole des dieux, il s’ensuit que le roi même des dieux lui est soumis, puisque Jupiter ne peut prendre la parole qu’avec le consentement de Mercure, ce qui est absurde. Dira-t-on qu’il n’est maître que du discours des hommes ? mais il est incroyable que Jupiter, qui a pu s’abaisser jusqu’à allaiter non seulement les enfants, mais encore les bêtes, d’où lui est venu le nom de Ruminus, n’ait pas voulu prendre soin de la parole, laquelle élève l’homme au-dessus des bêtes ? Donc Mercure n’est autre que Jupiter. Que si l’on veut identifier Mercure avec la parole (comme font ceux qui dérivent Mercure de {\itshape medius currens}, parce que la parole court au milieu des hommes ; et c’est pourquoi, selon eux, Mercure s’appelle en grec {\itshape Hermes}, parce que la parole ou l’interprétation de la pensée se dit {\itshape hermeneia}, d’où vient encore que Mercure préside au commerce, où la parole sert de médiatrice entre les vendeurs et les acheteurs ; et si ce dieu a des ailes à la tête et aux pieds, c’est que la parole est un son qui s’envole ; et enfin le nom de messager qu’on lui donne vient de ce que la parole est la messagère de nos pensées), tout cela posé, que s’ensuit-il, sinon que Mercure, n’étant autre que le langage, n’est pas vraiment un dieu ? Et voilà comment il arrive que les païens, en se faisant des dieux qui ne sont pas même des démons, et en adressant leurs supplications à des esprits immondes, sont sous l’empire, non des dieux, mais des démons. Même conclusion pour ce qui regarde Mars : dans l’impossibilité de lui assigner aucun élément, aucune partie du monde où il pût contribuer à quelque action de la nature, ils en ont fait le dieu de la guerre, laquelle est le triste ouvrage des hommes. D’où il résulte que si la déesse Félicité donnait aux hommes la paix perpétuelle, le dieu Mars n’aurait rien à faire. Veut-on dire que la guerre même fait la réalité de Mars comme la parole fait celle de Mercure ? plût au ciel alors que la guerre ne fût pas plus réelle qu’une telle divinité !
\subsection[{Chapitre XV}]{Chapitre XV}

\begin{argument}\noindent De quelques étoiles que les païens ont désignées par les noms de leurs dieux.
\end{argument}

\noindent On dira, peut-être que ces dieux ne sont autre chose que les étoiles auxquelles les païens ont donné leurs noms ; et, en effet, il y a une étoile qu’on appelle Mercure et une autre qu’on appelle Mars ; mais il y en a une aussi qu’on appelle Jupiter, et cependant les païens soutiennent que Jupiter est le monde. Ce n’est pas tout, il y en a une qu’on appelle Saturne, et cependant Saturne est déjà pourvu d’une fonction considérable, celle de présider à toutes les semences ; il y en a une enfin, et la plus éclatante de toutes, qu’on appelle Vénus, et cependant on veut que Vénus soit aussi la lune, bien qu’au surplus les païens ne tombent pas plus d’accord au sujet de cet astre que ne firent Vénus et Junon au sujet de la pomme d’or. Les uns, en effet, donnent l’étoile du matin à Vénus, les autres à Junon ; mais, ici comme toujours, c’est Vénus qui l’emporte, et presque toutes les voix sont en sa faveur. Or, qui ne rirait d’entendre appeler Jupiter le roi des dieux, quand on voit son étoile si pâle à côté de celle de Vénus ? L’étoile de ce dieu souverain ne devrait-elle pas être d’autant plus brillante qu’il est lui-même plus puissant ? On répond qu’elle paraît moins lumineuse parce qu’elle est plus haute et plus éloignée de la terre ; mais si elle est plus haute parce qu’elle appartient à un plus grand dieu, pourquoi l’étoile de Saturne est-elle placée plus haut que Jupiter ? Est-ce donc que le mensonge de la fable, qui a fait roi Jupiter, n’a pu monter jusqu’aux astres, et que Saturne a obtenu dans le ciel ce qu’il n’a pu obtenir ni dans son royaume ni dans le Capitole ? Et puis, pourquoi Janus n’a-t-il pas son étoile ? Est-ce parce qu’il est le monde et qu’à ce titre il embrasse toutes les étoiles ? mais Jupiter est le monde aussi, et cependant il y a une étoile qui porte son nom. Janus se serait-il arrangé de son mieux, et, au lieu d’une étoile qu’il devait avoir dans le ciel, se serait-il contenté d’avoir plusieurs visages sur la terre ? Enfin, si c’est seulement à cause de leurs étoiles qu’on regarde Mercure et Mars comme des parties du monde, afin d’en pouvoir faire des dieux, le langage et la guerre n’étant point des parties du monde, mais des actes de l’humanité, pourquoi n’a-t-on pas dressé des temples et des autels au Bélier, au Taureau, au Cancer, au Scorpion et autres signes célestes, lesquels ne sont pas composés d’une seule étoile, mais de plusieurs, et sont placés au plus haut des cieux avec des mouvements si justes et si réglés ? Pourquoi ne pas les mettre, sinon au rang des dieux choisis, au moins parmi les dieux de l’ordre plébéien.
\subsection[{Chapitre XVI}]{Chapitre XVI}

\begin{argument}\noindent D’Apollon, de Diane et des autres dieux choisis.
\end{argument}

\noindent Ils veulent qu’Apollon soit devin et médecin ; et cependant, pour lui donner une place dans l’univers, ils disent qu’il est aussi le soleil, et que sa sœur Diane est la lune et tout ensemble la déesse des chemins. De là vient qu’ils la font vierge, les chemins étant stériles ; et s’ils donnent des flèches au frère et à la sœur, c’est comme symbole des rayons qu’ils lancent du ciel sur la terre. Vulcain est le feu, Neptune l’eau, Dis ou Orcus l’élément inférieur et terrestre. Liber et Cérès président aux semences : le premier à celle des mâles, la seconde à celle des femelles, ou encore l’un à ce qu’elles ont de liquide, et l’autre à ce qu’elles ont de sec. Et ils rapportent tout cela au monde, c’est-à-dire à Jupiter, qui est appelé père et mère, comme répandant hors de soi toutes les semences et les recevanttoutes en soi. Ils veulent encore que la grande mère des dieux soit Cérès, laquelle n’est autre chose que la terre, et qu’elle soit aussi Junon. C’est pourquoi on la fait présider aux causes secondes, quoique Jupiter, en tant qu’il est le monde entier, soit appelé, comme nous l’avons vu, père et mère des dieux. Pour Minerve, dont ils ont fait la déesse des arts, ne trouvant pas une étoile où la placer, ils ont dit qu’elle était l’éther, ou encore la lune. Vesta passe aussi pour la plus grande des déesses, en tant qu’elle est la terre, ce qui n’a pas empêché de lui départir ce feu léger mis au service de l’homme, et qui n’est pas le feu violent dont l’intendance est à Vulcain. Ainsi tous les dieux choisis ne sont que le monde ; les uns le monde entier, les autres, quelques-unes de ses parties : le monde entier, comme Jupiter ; ses parties, comme Génius, la grande Mère, le Soleil et la Lune, ou plutôt Apollon et Diane ; tantôt un seul dieu en plusieurs choses, tantôt une seule chose en plusieurs dieux : un dieu en plusieurs choses, comme Jupiter, par exemple, qui est le monde entier et qui est aussi le ciel et une étoile. De même, Junon est la déesse des causes secondes, et elle est encore l’air et la terre, et elle serait en outre une étoile, si elle l’eût emporté sur Vénus. Minerve, elle aussi, est la plus haute région de l’air, ce qui ne l’empêche pas d’être en même temps la lune, qui est pourtant située dans la région la plus basse. Voici enfin qu’une seule et même chose est plusieurs dieux : le monde est Jupiter, et il est aussi Janus ; la terre est Junon, et elle est aussi la grande Mère et Cérès.
\subsection[{Chapitre XVII}]{Chapitre XVII}

\begin{argument}\noindent Varron lui-même a donné comme douteuses ses opinions touchant les dieux.
\end{argument}

\noindent On peut juger, par ce qui précède, de tout le reste de la théologie des païens : ils embrouillent toutes choses en essayant de les débrouiller et courent à l’aventure, selon que les pousse ou les ramène le flux ou le reflux de l’erreur ; c’est au point que Varron a mieux aimé douter de tout que de rien affirmer sans réserve. Après avoir achevé le premier de ses trois derniers livres, celui où il traite des dieux certains, voici ce qu’il dit sur les dieux incertains au commencement du second livre : « Si j’émets dans ce livre des opinions douteuses touchant les dieux, on ne doit point le trouver mauvais. Libre à tout autre, s’il croit la chose possible et nécessaire, de trancher ces questions avec assurance ; pour moi, on m’amènerait plus aisément à révoquer en doute ce que j’ai dit dans le premier livre, qu’à donner pour certain tout ce que je dirai dans celui-ci. » C’est ainsi que Varron a rendu également incertain, et ce qu’il avance des dieux incertains, et ce qu’il affirme des dieux certains. Bien plus, dans le troisième livre, qui traite des dieux choisis, passant de quelques vues préliminaires sur la théologie naturelle aux folies et aux mensonges de la théologie civile, où, loin d’être conduit par la vérité des choses, il est pressé par l’autorité de la coutume : « Je vais parler, dit-il, des dieux publics du peuple romain, de ces dieux à qui on a élevé des temples et des statues ; mais, pour me servir des expressions de Xénophane de Colophon je dirai plutôt ce que je pense que ce que j’affirme ; car l’homme a sur de tels objets des opinions, Dieu a la science. » Ce n’est donc qu’en tremblant qu’il promet de parler de ces choses, qui ne sont point à ses yeux l’objet d’une claire compréhension et d’une ferme croyance, mais d’une opinion incertaine, étant l’ouvrage de la main des hommes. Il savait bien, dans le fait, qu’il y a au monde un ciel et une terre ; que le ciel est orné d’astres étincelants, que la terre est riche en semences, et ainsi du reste ; il croyait également que toute nature est conduite et gouvernée par une force invisible et supérieure qui est l’âme de ce grand corps ; mais que Janus soit le monde, que Saturne, père de Jupiter, devienne son sujet, et autres choses semblables, c’est ce que Varron ne pouvait pas aussi positivement affirmer.
\subsection[{Chapitre XVIII}]{Chapitre XVIII}

\begin{argument}\noindent Quelle est la cause la plus vraisemblable de la propagation des erreurs du paganisme.
\end{argument}

\noindent Ce qu’on peut dire de plus vraisemblable sur ce sujet, c’est que les dieux du paganisme ont été des hommes à qui leurs flatteurs ont offert des fêtes et des sacrifices selon leurs mœurs, leurs actions et les accidents de leur vie, et que ce culte sacrilège s’est glissé peu à peu dans l’âme des hommes, semblable à celle des démons et amoureuse de frivolités, pour être bientôt propagé par les ingénieux mensonges des poètes et par les séductions des malins esprits. En effet, qu’un fils impie, poussé par l’ambition ou par la crainte d’un père impie, ait chassé son père de son royaume, cela est plus aisé à croire que de s’imaginer Saturne vaincu par son fils Jupiter, sous prétexte que la cause des êtres est antérieure à leur semence ; car si cette explication était bonne, jamais Saturne n’eût existé avant Jupiter, puisque la cause précède toujours la semence et n’en est jamais engendrée. Mais quoi ! dès que nos adversaires s’efforcent de relever de vaines fables et des actions purement humaines par des explications tirées de la nature, les plus habiles se trouvent réduits à de telles extrémités, que nous sommes forcés de les plaindre.
\subsection[{Chapitre XIX}]{Chapitre XIX}

\begin{argument}\noindent Des explications qu’on donne du culte de Saturne.
\end{argument}

\noindent « Quand on raconte (c’est Varron qui parle) que Saturne avait coutume de dévorer ses enfants, cela veut dire que les semences rentrent au même lieu où elles ont pris naissance. Quant à la motte de terre substituée à Jupiter, elle signifie qu’avant l’invention du labourage, les hommes recouvraient les blés de terre avec leurs mains. » À ce compte, il fallait dire que Saturne était la terre, et non pas la semence, puisqu’en effet la terre dévore en quelque sorte ce qu’elle a engendré, quand les semences sorties de son sein y rentrent de nouveau. Et cette motte de terre, que Saturne prit pour Jupiter, quel rapport a-t-elle avec l’usage de jeter de la terre sur les grains de blé ? Est-ce que la semence, ainsi recouverte de terre, en était moins dévorée pour cela ? Il semblerait, à entendre cette explication, que celui qui jetait de la terre emportait le grain, comme on emporta, dit-on, Jupiter, tandis qu’au contraire, en jetant de la terre sur le grain, cela ne servait qu’à le faire dévorer plus vite. D’ailleurs, de cette façon, Jupiter est la semence, et non, comme Varron le disait tout à l’heure, la cause de la semence. Aussi bien, que peuvent dire de raisonnable des gens qui veulent expliquer des folies ?\par
« Saturne a une faux, poursuit Varron, comme symbole de l’agriculture. » Mais l’agriculture n’existait pas sous le règne de Saturne, puisqu’on fait remonter ce règne aux temps primitifs, ce qui signifie, suivant Varron, que les hommes de cette époque vivaient de ce que la terre produisait sans culture. Serait-ce qu’après avoir perdu son sceptre, Saturne aurait pris une faux, afin de devenir sous le règne de son fils un laborieux mercenaire, après avoir été aux anciens jours un prince oisif ? Varron ajoute que dans certains pays, à Carthage par exemple, on immolait des enfants à Saturne, et que les Gaulois lui sacrifiaient même des hommes faits, parce que, de toutes les semences, celle de l’homme est la plus excellente. Mais qu’est-il besoin d’insister sur une folie si cruelle ? Il nous suffit de remarquer et de tenir pour certain que toutes ces explications ne se rapportent point au vrai Dieu, à cette nature vivante, immuable, incorporelle, à qui l’on doit demander la vie éternellement heureuse, mais qu’elles se terminent à des objets temporels, corruptibles, sujets au changement et à la mort. « Quand on dit que Saturne a mutilé le Ciel, son père, cela signifie, dit encore Varron, que la semence divine n’appartient pas au Ciel, mais à Saturne, et cela parce que rien au Ciel, autant qu’on en peut juger, ne provient d’une semence. » Mais si Saturne est fils du Ciel, il est fils de Jupiter ; car on reconnaît d’un commun accord que le Ciel est Jupiter. Et voilà comme ce qui ne vient pas de la vérité se ruine de soi-même, sans que personne y mette la main. Varron dit aussi que Saturne est appelé Cronos, mot grec qui signifie le Temps, parce que sans le temps les semences ne sauraient devenir fécondes ; et il y a encore sur Saturne une foule de récits que les théologiens ramènent tous à l’idée de semence. Il semble tout au moins que Saturne, avec une puissance si étendue, aurait dû suffire à lui tout seul pour ce qui regarde la semence ; pourquoi donc lui adjoindre d’autres divinités, comme Liber et Libera, c’est-à-dire Cérès ? pourquoi entrer, comme fait Varron, dans mille détails sur les attributions de ces divinités relativement à la semence, comme s’il n’avait pas déjà été question de Saturne ?
\subsection[{Chapitre XX}]{Chapitre XX}

\begin{argument}\noindent Des mystères de Cérès Éleusine.
\end{argument}

\noindent Entre les mystères de Cérès, les plus fameux sont ceux qui se célébraient à Éleusis, ville de l’Attique. Tout ce que Varron en dit ne regarde que l’invention du blé attribuée à Cérès, et l’enlèvement de sa fille Proserpine par Pluton. Il voit dans ce dernier récit le symbole de la fécondité des femmes : « La terre, dit-il, ayant été stérile pendant quelque temps, cela fit dire que Pluton avait enlevé et retenu aux enfers la fille de Cérès, c’est-à-dire la fécondité même, appelée Proserpine, de {\itshape proserpere} (pousser, lever). Et comme après cette calamité qui avait causé un deuil public on vit la fécondité revenir, on dit que Pluton avait rendu Proserpine, et on institua des fêtes solennelles en l’honneur de Cérès. » Varron ajoute que les mystères d’Éleusis renferment plusieurs autres traditions, qui toutes se rapportent à l’invention du blé.
\subsection[{Chapitre XXI}]{Chapitre XXI}

\begin{argument}\noindent De l’infamie des mystères de Liber ou Bacchus.
\end{argument}

\noindent Quant aux mystères du dieu Liber, qui préside aux semences liquides, c’est-à-dire non seulement à la liqueur des fruits, parmi lesquels le vin tient le premier rang, mais aussi aux semences des animaux, j’hésite à prolonger mon discours par le récit de ces turpitudes ; il le faut néanmoins pour confondre l’orgueilleuse stupidité de nos adversaires. Entre autres rites que je suis forcé d’omettre, parce qu’il y en a trop, Varron rapporte qu’en certains lieux de l’Italie, aux fêtes de Liber, la licence était poussée au point d’adorer, en l’honneur de ce dieu, les parties viriles de l’homme, non dans le secret pour épargner la pudeur, mais en public pour étaler l’impudicité. On plaçait en triomphe ce membre honteux sur un char que l’on conduisait dans la ville, après l’avoir d’abord promené à travers la campagne. À Lavinium, on consacrait à Liber un mois entier, pendant lequel chacun se donnait carrière en discours scandaleux, jusqu’au moment où le membre obscène, après avoir traversé la place publique, était mis en repos dans le lieu destiné à le recevoir. Là il fallait que la mère de famille la plus honnête allât couronner et déshonnête objet devant tous les spectateurs. C’est ainsi qu’on rendait le dieu Liber favorable aux semences, et qu’on détournait de la terre tout sortilège en obligeant une matrone à faire en public ce qui ne serait pas permis sur le théâtre à une courtisane, si les matrones étaient présentes. On voit maintenant pourquoi Saturne n’a pas été jugé suffisant pour ce qui regarde les semences ; c’est afin que l’âme corrompue eût occasion de multiplier les dieux, et qu’abandonnée du Dieu véritable en punition de son impureté, de jour en jour plus impure et plus misérablement prostituée à une multitude de divinités fausses, elle couvrît ces sacrilèges du nom de mystères sacrés et s’abandonnât aux embrassements et aux turpitudes de cette foule obscène de démons.
\subsection[{Chapitre XXII}]{Chapitre XXII}

\begin{argument}\noindent De Neptune, de Salacie et de Vénilie.
\end{argument}

\noindent Neptune avait pour femme Salacie, qui figure, dit-on, la région inférieure des eaux de la mer : à quoi bon lui donner encore Vénilie ? Je ne vois là que le goût dépravé de l’âme corrompue qui veut se prostituer à un plus grand nombre de démons. Mais écoutons les interprétations de cette belle théologie et les raisons secrètes qui vont la mettre à couvert de notre censure : « Vénilie, dit Varron, est l’eau qui vient battre le rivage, Salacie l’eau qui rentre dans la pleine mer ({\itshape salum}). » Pourquoi faire ici deux déesses, puisque l’eau qui vient et l’eau qui s’en va ne sont qu’une seule et même eau ? En vérité, cette fureur de multiplier les dieux ressemble elle-même à l’agitation tumultueuse des flots. Car bien que l’eau du flux et celle du reflux ne soient pas deux eaux différentes, toutefois, sous le vain prétexte de ces deux mouvements, l’âme « qui s’en va et qui ne revient plus » se plonge plus avant dans la fange en invoquant deux démons. Je t’en prie, Varron, et je vous en conjure aussi, vous tous qui avez lu les écrits de tant de savants hommes, et vous vantez d’y avoir appris de grandes choses, de grâce expliquez-moi ce point, je ne dis pas en partant de cette nature éternelle et immuable qui est Dieu seul, mais du moins selon la doctrine de l’âme du monde et de ses parties qui sont pour vous des dieux véritables. Que vous ayez fait le dieu Neptune de cette partie de l’âme du monde qui pénètre la mer, c’est une erreur supportable ; mais l’eau qui vient battre contre le rivage et qui retourne dans la pleine mer, voyez-vous là deux parties du monde ou deux parties de l’âme du monde, et y a-t-il quelqu’un parmi vous d’assez extravagant pour le supposer ? Pourquoi donc vous en a-t-on fait deux déesses, sinon parce que vos ancêtres, ces hommes pleins de sagesse, ont pris soin, non pas que vous fussiez conduits par plusieurs dieux, mais possédés par plusieurs démons amis de ces vanités et de ces mensonges ? Je demande en outre de quel droit cette explication théologique exile Salacie de cette partie inférieure de la mer où elle vivait soumise à son mari ; car, identifier Salacie avec le reflux, c’est la faire monter à la surface de la mer. Serait-ce qu’elle a chassé son mari de la partie supérieure pour le punir d’avoir fait sa concubine de Vénilie ?
\subsection[{Chapitre XXIII}]{Chapitre XXIII}

\begin{argument}\noindent De la terre, que Varron regarde comme une déesse, parce qu’à son avis l’âme du monde, qui est Dieu, pénètre jusqu’à cette partie inférieure de son corps et lui communique une force divine.
\end{argument}

\noindent Il n’y a qu’une seule terre, peuplée, il est vrai, d’êtres animés, mais qui n’est après tout qu’un grand corps parmi les éléments et la plus basse partie du monde. Pourquoi veut-on en faire une déesse ? est-ce à cause de sa fécondité ? mais alors les hommes seraient des dieux, à plus forte raison, puisque leurs soins lui donnent un surcroît de fécondité en la cultivant et non pas en l’adorant. On répond qu’une partie de l’âme du monde, en pénétrant la terre, l’associe à la divinité. Comme si l’âme humaine, dont l’existence ne fait pas question, ne se manifestait pas d’une manière plus sensible ! et cependant les hommes ne passent point pour des dieux. Ce qu’il y a de plus déplorable, c’est qu’ils sont assez aveugles pour adorer des êtres qui ne sont pas des dieux et qui ne les valent pas.\par
Dans ce même livre des dieux choisis, Varron distingue dans tout l’ensemble de la nature trois degrés d’âmes au premier degré, l’âme, bien que pénétrant les parties d’un corps vivant, ne possède pas le sentiment, mais seulement la force qui fait vivre, celle, par exemple, qui s’insinue dans nos os, dans nos ongles et dans nos cheveux. C’est ainsi que nous voyons les plantes se nourrir, croître et vivre à leur manière, sans avoir le sentiment. Au second degré l’âme est sensible, et cette force nouvelle se répand dans les yeux, dans les oreilles, dans le nez, dans la bouche et dans les organes du toucher. Le troisième degré, le plus élevé de l’âme, c’est l’âme raisonnable où brille l’intelligence, et qui, entre tous les êtres mortels, ne se trouve que dans l’homme. Cette partie de l’âme du monde est Dieu ; dans l’homme elle s’appelle Génie. Varron dit encore que les pierres et la terre, où le sentiment ne pénètre pas, sont comme les os et les ongles de Dieu ; que le soleil, la lune et les étoiles sont ses organes et ses sens ; que l’éther est son âme, et que l’influence de ce divin principe, pénétrant les astres, les transforme en dieux ; de là, gagnant la terre, en fait la déesse Tellus, et atteignant enfin lamer et l’Océan, constitue la divinité de Neptune.\par
Que Varron veuille bien quitter un instant cette théologie naturelle où, après mille détours et mille circuits, il est venu se reposer ; qu’il revienne à la théologie civile. Je l’y veux retenir encore ; il me reste quelques mots à lui adresser. Je pourrais lui dire en passant que si la terre et les pierres sont pareilles à nos os et à nos ongles, elles sont pareillement destituées d’intelligence comme de sentiment, à moins qu’il ne se trouve un esprit assez extravagant pour prétendre que nos os et nos ongles ont de l’intelligence, parce qu’ils sont des parties de l’homme intelligent ; d’où il suit qu’il y a autant de folie à regarder la terre et les pierres comme des dieux, qu’à vouloir que les os et les ongles des hommes soient des hommes. Mais ce sont là des questions que nous aurons peut-être à discuter avec des philosophes ; je n’ai affaire encore qu’à un politique. Car, bien que Varronsemble, en cette rencontre, avoir voulu relever un peu la tête et respirer l’air plus libre de la théologie naturelle, il est très supposable que le sujet de ce livre, qui roule sur les dieux choisis, l’aura ramené au point de vue de la théologie politique, et qu’il n’aura pas voulu laisser croire que les anciens Romains et d’autres peuples aient rendu un vain culte à Tellus et à Neptune. Je lui demande donc pourquoi, n’y ayant qu’une seule et même terre, cette partie de l’âme du monde qui la pénètre n’en fait pas une seule divinité sous le nom de Tellus ? Et si la terre est une divinité unique, que devient alors Orcus ou Dis, frère de Jupiter et de Neptune ? Que devient sa femme Proserpine qui, selon une autre opinion rapportée dans les mêmes livres, n’est pas la fécondité de la terre, mais sa plus basse partie ? Si l’on prétend que l’âme du monde, en pénétrant la partie supérieure de la terre, fait le dieu Dis, et Proserpine en pénétrant sa partie inférieure, que devient alors la déesse Tellus ? Elle est tellement divisée entre ces deux parties et ces deux divinités, qu’on ne sait plus ce qu’elle est, ni où elle est, à moins qu’on ne s’avise de prétendre que Pluton et Proserpine ne sont ensemble que la déesse Tellus, et qu’il n’y a pas là trois dieux, mais un seul, ou deux tout au plus. Et cependant on s’obstine à en compter trois, on les adore tous trois ; ils ont tous trois leurs temples, leurs autels, leurs statues, leurs sacrifices, leurs prêtres, c’est-à-dire autant de sacrilèges, autant de démons à qui se livre l’âme prostituée. Qu’on me dise encore quelle est la partie de la terre que pénètre l’âme du monde pour faire le dieu Tellumon ? — Ce n’est pas cela, dira Varron ; la même terre a deux vertus : l’une, masculine, pour produire les semences ; l’autre, féminine, pour les recevoir et les nourrir ; de celle-ci lui vient le nom de Tellus, de celle-là le nom de Tellumon. Mais alors pourquoi, selon Varron lui-même, les pontifes ajoutaient-ils à ces deux divinités Altor et Rusor ? Supposons Tellus et Tellumon expliqués ; pourquoi Altor ? C’est, dit Varron, que la terre nourrit tout ce qui naît. Et Rusor ? C’est que tout retourne à la terre.
\subsection[{Chapitre XXIV}]{Chapitre XXIV}

\begin{argument}\noindent Sur l’explication qu’on donne des divers noms de la terre, lesquels désignent, il est vrai, différentes vertus ; mais n’autorisent pas l’existence de différentes divinités.
\end{argument}

\noindent La terre ayant les quatre vertus qu’on vient de dire, je conçois qu’on lui ait donné quatre noms, mais non pas qu’on en ait fait quatre divinités. Jupiter est un, malgré tous ses surnoms ; Junon est une avec tous les siens ; dans la diversité des désignations se maintient l’unité du principe, et plusieurs noms ne font pas plusieurs dieux. De même qu’on voit des courtisanes prendre en dégoût la foule de leurs amants, il arrive aussi sans doute qu’uneâme, après s’être abandonnée aux esprits impurs, vient à rougir de cette multitude de démons dont elle recherchait les impures caresses. Car Varron lui-même, comme s’il avait honte d’une si grande foule de divinités, veut que Tellus ne soit qu’une seule déesse : « On l’appelle aussi, dit-il, la grande Mère. Le tambour qu’elle porte figure le globe terrestre ; les tours qui couronnent sa tête sont l’image des villes ; les sièges dont elle est environnée signifient que dans le mouvement universel elle reste immobile. Si elle a des Galles pour serviteurs, c’est que pour avoir des semences il faut cultiver la terre, qui renferme tout dans son sein. En s’agitant autour d’elle, ces prêtres enseignent aux laboureurs qu’ils ne doivent pas demeurer oisifs, ayant toujours quelque chose à faire. Le son des cymbales marque le bruit que font les instruments du labourage, et ces instruments sont d’airain, parce qu’on seservait d’airain avant la découverte du fer. Enfin, dit Varron, on place auprès de la déesse un lion libre et apprivoisé pour faire entendre qu’il n’y a point de terre si sauvage et si stérile qu’on ne la puisse dompter et cultiver. » Il ajoute que les divers noms et surnoms donnés à Tellus l’ont fait prendre pour plusieurs dieux. « On croit, dit-il, que Tellus est la déesse Ops, parce que la terre s’améliore par le travail, qu’elle est la grande Mère, parce qu’elle est féconde, Proserpine, parce que les blés sortent de son sein, Vesta, parce que l’herbe est son vêtement, et c’estainsi qu’on rapporte, non sans raison, plusieurs divinités à celle-ci. » — Soit Tellus, je le veux bien, n’est qu’une déesse, elle qui, dans le fond, n’est rien de tout cela ; mais pourquoi supposer cette multitude de divinités ? Que ce soient les noms divers d’une seule, à la bonne heure, mais que des noms ne soient pas des déesses. Cependant, l’autorité d’une erreur ancienne est si grande sur l’esprit de Varron, qu’après ce qu’il vient de dire, il tremble encore et ajoute : « Cette opinion n’est pas contraire à celle de nos ancêtres, qui voyaient là plusieurs divinités. » Comment cela ? y a-t-il rien de plus différent que de donner plusieurs noms à une seule déesse et de reconnaître autant de déesses que de noms ? « Mais il se peut, dit-il, qu’une chose soit à la fois une et multiple. » J’accorderai bien, en effet, qu’il y a plusieurs choses dans un seul homme ; mais s’ensuit-il que cet homme soit plusieurs hommes ? Donc, de ce qu’il y a plusieurs choses en une déesse, il ne s’ensuit pas qu’elle soit plusieurs déesses. Qu’ils en usent, au surplus, comme il leur plaira : qu’ils les divisent, qu’ils les réunissent, qu’ils les multiplient, qu’ils les mêlent et les confondent, cela les regarde.\par
Voilà les beaux mystères de Tellus et de la grande Mère, où il est clair que tout se rapporte à des semences périssables et à l’art de l’agriculture ; et tandis que ces tambours, ces tours, ces Galles, ces folles convulsions, ces cymbales retentissantes et ces lions symboliques viennent aboutir à cela, je cherche où est la promesse de la vie éternelle. Comment soutenir d’ailleurs que les eunuques mis au service de cette déesse font connaître la nécessité de cultiver la terre pour la rendre féconde, tandis que leur condition même les condamne à la stérilité ? Acquièrent-ils, en s’attachant au culte de cette déesse, la semence qu’ils n’ont pas, ou plutôt ne perdent-ils pas celle qu’ils ont ? Ce n’est point là vraiment expliquer des mystères, c’est découvrir des turpitudes ; mais voici une chose qu’on oublie de remarquer, c’est à quel degré est montée la malignité des démons, d’avoir promis si peu aux hommes et toutefois d’en avoir obtenu contre eux-mêmes des sacrifices si cruels. Si l’on n’eût pas fait de la terre une déesse, l’homme eût dirigé ses mains uniquement contre elle pour en tirer de la semence, et non contre soi pour s’en priver en son honneur ; il eût rendu la terre féconde et ne se serait pas rendu stérile. Que dans les fêtes de Bacchus une chaste matrone couronne les parties honteuses de l’homme, devant une foule où se trouve peut-être son mari qui sue et rougit de honte, s’il y a parmi les hommes un reste de pudeur ; que l’on oblige, aux fêtes nuptiales, la nouvelle épouse de s’asseoir sur un Priape, tout cela n’est rien en comparaison de ces mystères cruellement honteux et honteusement cruels, où l’artifice des démons trompe et mutile l’un et l’autre sexe sans détruire aucun des deux. Là on craint pour les champs les sortilèges, ici on ne craint pas pour les membres la mutilation ; là on blesse la pudeur de la nouvelle mariée, mais on ne lui ôte ni la fécondité, ni même la virginité ; ici on mutile un homme de telle façon qu’il ne devient point femme et cesse d’être homme.
\subsection[{Chapitre XXV}]{Chapitre XXV}

\begin{argument}\noindent Quelle explication la science des sages de la Grèce a imaginée de la mutilation d’Atys.
\end{argument}

\noindent Varron ne dit rien d’Atys et ne cherche pas à expliquer pourquoi les Galles se mutilent en mémoire de l’amour que lui porta Cybèle. Mais les savants et les sages de la Grèce n’ont eu garde de laisser sans explication une tradition si belle et si sainte. Porphyre, le célèbre philosophe, y voit un symbole du printemps qui est la plus brillante saison de l’année ; Atys représente les fleurs, et, s’il est mutilé, c’est que la fleur tombe avant le fruit. À ce compte le vrai symbole des fleurs n’est pas cet homme ou ce semblant d’homme qu’on appelle Atys, ce sont ses parties viriles qui tombèrent, en effet, par la mutilation ; ou plutôt elles ne tombèrent pas ; elles furent, non pas cueillies, mais déchirées en lambeaux, citant s’en faut que la chute de cette fleur ait fait place à aucun fruit qu’elle fût suivie de stérilité. Que signifie donc cet Atys mutilé, ce reste d’homme ? à quoi le rapporter et quel sens lui découvrir ? Certes, les efforts impuissants où l’on se consume pour expliquer ce prétendu mystère font bien voir qu’il faut s’en tenir à ce que la renommée en publie et à ce qu’on en a écrit, je veux dire que cet Atys est un homme qu’on a mutilé. Aussi Varron garde-t-il ici le silence ; et comme un si savant homme n’a pu ignorer ce genre d’explication, il faut en conclure qu’il ne la goûtait nullement.
\subsection[{Chapitre XXVI}]{Chapitre XXVI}

\begin{argument}\noindent Infamies des mystères de la grande Mère.
\end{argument}

\noindent Un mot maintenant sur ces hommes énervés que l’on consacre à la grande Mère par une mutilation également injurieuse à la pudeur des deux sexes ; hier encore on les voyait dans les rues et sur les places de Carthage, les cheveux parfumés, le visage couvert de fard, imitant de leur corps amolli la démarche des femmes, demander aux passants de quoi soutenir leur infâme existence. Cette fois encore Varron a trouvé bon de ne rien dire, et je ne me souviens d’aucun auteur qui se soit expliqué sur ce sujet. Ici l’exégèse fait défaut, la raison rougit, la parole expire. La grande Mère a surpassé tous ses enfants, non par la grandeur de la puissance, mais par celle du crime. C’est une monstruosité qui éclipse le monstrueux Janus lui-même ; car Janus n’est hideux que dans ses statues, elle est hideuse et cruelle dans ses mystères ; Janus n’a qu’en effigie des membres superflus, elle fait perdre en réalité des membres nécessaires. Son infamie est si grande, qu’elle surpasse toutes les débauches de Jupiter. Séducteur de tant de femmes, il n’a déshonoré le ciel que du seul Ganymède ; mais elle, avec son cortège de mutilés scandaleux, a tout ensemble souillé la terre et outragé le ciel. Je ne trouve rien à lui comparer que Saturne, qui, dit-on, mutila son père. Encore, dans les mystères de ce dieu, les hommes périssent par la main d’autrui ; ils ne se mutilent point de leur propre main. Les poètes, il est vrai, imputent à Saturne d’avoir dévoré ses enfants, et la théologie physique interprète cette tradition comme il lui plaît ; mais l’histoire porte simplement qu’il les tua ; et si à Carthage on lut sacrifiait des enfants, c’est un usage que les Romains ont répudié. La mère des dieux, au contraire, a introduit ses eunuques dans les temples des Romains, et cette cruelle coutume s’est conservée, comme si on pouvait accroître la virilité de l’âme en retranchant la virilité du corps. Au prix d’un tel usage, que sont les larcins de Mercure, les débauches de Vénus, les adultères des autres dieux, et toutes ces turpitudes dont nous trouverions la preuve dans les livres, si chaque jour on ne prenait soin de les chanter et de les danser sur le théâtre ? Qu’est-ce que tout cela au prix d’une abomination qui, par sa grandeur même, ne pouvait convenir qu’à la grande Mère, d’autant plus qu’on a soin de rejeter les autres scandales sur l’imagination des poètes ! Et, en effet, que les poètes aient, beaucoup inventé, j’en tombe d’accord ; seulement je demande si le plaisir que procurent aux dieux ces fictions est aussi une invention des poètes ? Qu’on impute donc, j’y consens, à leur audace ou à leur impudence l’éclat scandaleux que la poésie et la scène donnent aux aventures des dieux ; mais quand j’en vois faire, par l’ordre des dieux, une partie de leur culte et de leurs honneurs, n’est-ce pas le crime des dieux mêmes, ou plutôt un aveu fait par les démons et un piège tendu aux misérables ? En tout cas, ces consécrations d’eunuques à la Mère des dieux ne sont point une fiction, et les poètes en ont eu tellement horreur qu’ils se sont abstenus de les décrire. Qui donc voudrait se consacrer à de telles divinités, afin de vivre heureusement dans l’autre monde, quand il est impossible, en s’y consacrant, de vivre honnêtement dans celui-ci ? — « Vous oubliez, me dira Varron, que tout ce culte n’a rapport qu’au monde. » — J’ai bien peur que ce soit plutôt à l’immonde. D’ailleurs, il est clair que tout ce qui est dans le monde peut aisément y être rapporté ; mais ce que nous cherchons, nous, n’est pas dans le monde : c’est une âme affermie par la vraie religion, qui n’adore pas le monde comme un dieu, mais qui le glorifie comme l’œuvre de Dieu et pour la gloire de Dieu même, afin de se dégager de toute souillure mondaine et de parvenir pure et sans tache à Dieu, Créateur du monde.
\subsection[{Chapitre XXVII}]{Chapitre XXVII}

\begin{argument}\noindent Sur les explications physiques données par certains philosophes qui ne connaissent ni le vrai Dieu ni le culte qui lui est dû.
\end{argument}

\noindent Nous voyons à la vérité que ces dieux choisis ont plus de réputation que les autres ; mais elle n’a servi, loin de mettre leur mérite en lumière, qu’à faire mieux éclater leur indignité, ce qui porte à croire de plus en plus que ces dieux ont été des hommes, suivant le témoignage des poètes et même des historiens. Virgile n’a-t-il pas dit :\par
 {\itshape « Saturne, le premier, descendit des hauteurs éthérées de l’Olympe, exilé de son royaume et poursuivi par les armes de Jupiter. »} \par
Or, ces vers et les suivants ne font que reproduire le récit développé tout au long par Évhémère et traduit par Ennius : mais comme les écrivains grecs et latins, qui avant nous ont combattu les erreurs du paganisme, ont suffisamment discuté ce point, il n’est pas nécessaire d’y insister.\par
Quant aux raisons physiques proposées par des hommes aussi doctes que subtils pour transformer en choses divines ces choses purement humaines, plus je les considère, moins j’y vois rien qui ne se rapporte à des œuvres terrestres et périssables, à une nature corporelle qui, même conçue comme invisible, ne saurait être le vrai Dieu. Du moins, si ce culte symbolique avait un caractère de religion, tout en regrettant son impuissance complète à faire connaître le vrai Dieu, il serait consolant de penser qu’il n’y a là du moins ni commandements impurs, ni honteuses pratiques. Mais, d’abord, c’est déjà un crime d’adorer le corps ou l’âme à la place du vrai Dieu, qui seul peut donner à l’âme où il habite la félicité ; combien donc est-il plus criminel encore de leur offrir un culte qui ne contribue ni au salut, ni même à l’honneur de celui qui le rend ? Que des temples, des prêches, des sacrifices, que tous ces tributs, qui ne sont dus qu’au vrai Dieu, soient consacrés à quelque élément du monde ou à quelque esprit créé, ne fût-il d’ailleurs ni impur ni méchant, c’est un mal, sans aucun doute ; non que le mal se trouve dans les objets employés à ce culte, mais parce qu’ils ne doivent servir qu’à honorer celui à qui ce culte est dû. Que si l’on prétend adorer le Vrai Dieu, c’est-à-dire le Créateur de toute âme et de tout corps, par des statues ridicules ou monstrueuses, par des couronnes déposées sur des organes honteux, par des prix décernés à l’impudicité, par des incisions et des mutilations cruelles, par la consécration d’hommes énervés, par desspectacles impurs et scandaleux, c’est encore un grand mal, non qu’on ne doive adorer celui qu’on adore ainsi, mais parce que ce n’est pas ainsi qu’on le doit adorer. Mais d’adorer une créature quelle qu’elle soit, même la plus pure, soit âme, soit corps, soit âme et corps tout ensemble, et de l’adorer par ce culte infâme et détestable, c’est pécher doublement contre Dieu, en ce qu’on adore, au lieu de lui, ce qui n’est pas lui, et en ce qu’on lui offre un culte qui ne doit être offert ni à lui, ni à ce qui n’est pas lui. Pour le culte des païens, il est aisé de voir combien il est honteux et abominable ; mais on ne s’expliquerait pas suffisamment l’origine et l’objet de ce culte, si les propres historiens du paganisme ne nous apprenaient que ce sont les dieux eux-mêmes qui, sous de terribles menaces, ont imposé ce culte à leurs adorateurs. Concluons donc sans hésiter, que toute cette théologie civile se réduit à attirer les esprits de malice et d’impureté sous de stupides simulacres pour s’emparer du cœur insensé des hommes.
\subsection[{Chapitre XXVIII}]{Chapitre XXVIII}

\begin{argument}\noindent La théologie de Varron partout en contradiction avec elle-même.
\end{argument}

\noindent Que sert au savant et ingénieux Varron de se consumer en subtilités pour rattacher tous les dieux païens au ciel et à la terre ? Vains efforts ! ces dieux lui échappent des mains ; ils s’écoulent, glissent et tombent. Voici en quels termes il commence son exposition des divinités femelles ou déesses : « Ainsi que je l’ai dit en parlant des dieux au premier livre, les dieux ont deux principes, savoir : le ciel et la terre, ce qui fait qu’on les a divisés en dieux célestes et dieux terrestres. Dans les livres précédents j’ai commencé par le ciel, c’est-à-dire par Janus, qui est le ciel pour les uns et le monde pour les autres ; dans celui-ci je commencerai par la déesse Tellus. » Ainsi parle Varron, et je crois sentir ici l’embarras qu’éprouve ce grand génie. Il est soutenu par quelques analogies assez vraisemblables, quand il fait du ciel le principe actif, de la terre le principe passif, et qu’il rapporte en conséquence la puissance masculine à celui-là et la féminine à celle-ci ; mais il ne prend pas garde que le vrai principe de toute action et de toute passion, de tout phénomène terrestre ou céleste, c’est le Créateur de la terre et du ciel. Varron ne paraît pas moins aveuglé au livre précédent, où il prétend donner l’explication des fameux mystères de Samothrace, et s’engage avec une sorte de solennité pieuse à révéler à ses concitoyens des choses inconnues. À l’entendre, il s’est assuré par un grand nombre d’indices que, parmi les statues des dieux, l’une est le symbole du ciel, l’autre celui de la terre ; une autre est l’emblème de ces exemplaires des choses que Platon appelle {\itshape idées}. Dans Jupiter il voit le ciel, la terre dans Junon et les idées dans Minerve ; le ciel est le principe actif des choses ; la terre, le principe passif, et les idées en sont les types. Je ne rappellerai pas ici l’importance supérieure que Platon attribue aux idées (à ce point que, suivant lui, le ciel, loin d’avoir rien produit sans idées, a été lui-même produit sur le modèle des idées) ; je remarquerai seulement que Varron, dans son livre des dieux choisis, perd de vue cette doctrine des trois divinités auxquelles il avait réduit tout le reste. En effet, il rapporte au ciel les dieux et à la terre les déesses, parmi lesquelles il range Minerve, placée tout à l’heure au-dessus du ciel. Remarquez encore que Neptune, divinité mâle, a pour demeure la mer, laquelle fait partie de la terre plutôt que du ciel. Enfin, Dis, le Pluton des Grecs, frère de Jupiter et de Neptune, habite la partie supérieure du ciel, laissant la partie inférieure à son épouse Proserpine ; or, que devient ici la distribution faite plus haut qui assignait le ciel aux dieux et la terre aux déesses ? où est la solidité de ces théories, où en est la conséquence, la précision, l’enchaînement ? La suite des déesses commence par Tellus, la grande Mère, autour de laquelle s’agite bruyamment cette foule insensée d’hommes sans sexe et sans force qui se mutilent en son honneur ; la tête des dieux c’est Janus, comme Tellus est la tête des déesses. Mais quoi ! la superstition multiplie la tête du dieu, et la fureur trouble celle de la déesse. Que de vains efforts pour rattacher tout cela au monde ! et à quoi bon, puisque l’âme pieuse n’adorera jamais le monde à la place du vrai Dieu ? L’impuissance des théologiens est donc manifeste, et il ne leur reste plus qu’à rapporter ces fables à des hommes morts et à d’impurs démons ; à ce prix toute difficulté disparaîtra.
\subsection[{Chapitre XXIX}]{Chapitre XXIX}

\begin{argument}\noindent Il faut rapporter à un seul vrai Dieu tout ce que les philosophes ont rapporté au monde et à ses parties.
\end{argument}

\noindent Et en effet, tout ce que la théologie physique rapporte au monde, combien il serait plus aisé, sans crainte d’une opinion sacrilège, de le rapporter au vrai Dieu, Créateur du monde, principe de toutes les âmes et de tous les corps ! C’est ce qui résulte de ce simple énoncé de notre croyance : Nous adorons Dieu, et non pas le ciel et la terre, ces deux parties dont se compose le monde ; nous n’adorons ni l’âme ni les âmes répandues dans tous les corps vivants, mais le Créateur du ciel, de la terre et de tous les êtres, l’Auteur de toutes les âmes, végétatives, sensibles ou raisonnables.
\subsection[{Chapitre XXX}]{Chapitre XXX}

\begin{argument}\noindent Une religion éclairée distingue les créatures du Créateur, afin de ne pas adorer, à la place du Créateur, autant de dieux qu’il y a de créatures.
\end{argument}

\noindent Pour commencer à parcourir les œuvres de ce seul vrai Dieu, lesquelles ont donné lieu aux païens de se forger une multitude de fausses divinités dont ils s’efforcent vainement d’interpréter en un sens honnête les mystères infâmes et abominables, je dis que nous adorons ce Dieu qui a marqué à toutes les natures, dont il est le Créateur, le commencement et la fin de leur existence et de leur mouvement ; qui renferme en soi toutes les causes, les connaît et les dispose à son gré ; qui donne à chaque semence sa vertu ; qui a doué d’une âme raisonnable tels animaux qu’il lui a plu ; qui leur a départi la faculté et l’usage de la parole ; qui communique à qui bon lui semble l’esprit de prophétie, prédisant l’avenir par la bouche de ses serviteurs privilégiés, et par leurs mains guérissant les malades ; qui est l’arbitre de la guerre et qui en règle le commencement, le progrès et la fin, quand il a trouvé bon de châtier ainsi les hommes ; qui a produit le feu élémentaire et en gouverne l’extrême violence et la prodigieuse activité suivant les besoins de la nature ; qui est le principe et le modérateur des eaux universelles ; qui a fait le soleil le plus brillant des corps lumineux, et lui a donné une force et un mouvement convenables ; qui étend sa domination et sa puissance jusqu’aux enfers ; qui a communiqué aux semences et, aux aliments, tant liquides que solides, les propriétés qui leur conviennent ; qui a posé le fondement de la terre et qui lui donne sa fécondité ; qui en distribue les fruits d’une main libérale aux hommes et aux animaux ; qui connaît et gouverne les causes secondes aussi bien que les causes premières ; qui a imprimé à la lune son mouvement ; qui, sur la terre et dans le ciel, ouvre des routes au passage des corps ; qui a doté l’esprit humain, son ouvrage, des sciences et des arts pour le soulagement de la vie ; qui a établi l’union du mâle et de la femelle pour la propagation des espèces ; qui enfin a fait présent du feu terrestre aux sociétés humaines pour en tirer à leur usage lumière et chaleur. Voilà les œuvres divines que le docte et ingénieux Varron s’est efforcé de distribuer entre ses dieux, par je ne sais quelles explications physiques, tantôt empruntées à autrui, et tantôt imaginées par lui-même. Mais Dieu seul est la cause véritable et universelle ; Dieu, dis-je, en tant qu’il est tout entier partout, sans être enfermé dans aucun lieu ni retenu par aucun obstacle, indivisible, immuable, emplissant le ciel et la terre, non de sa nature, mais de sa puissance. Si en effet il gouverne tout ce qu’il a créé, c’est de telle façon qu’il laisse à chaque créature son action et son mouvement propres ; aucune ne peut être sans lui, mais aucune n’est lui. Il agit souvent par le ministère des anges, mais il fait seul la félicité des anges. De même, bien qu’il envoie quelquefois des anges aux hommes, ce n’est point par les anges, c’est par lui-même qu’il rend les hommes heureux. Tel est le Dieu unique et véritable de qui nous espérons la vie éternelle.
\subsection[{Chapitre XXXI}]{Chapitre XXXI}

\begin{argument}\noindent Quels bienfaits particuliers Dieu ajoute en faveur des sectateurs de la vérité à ceux qu’il accorde à tous les hommes.
\end{argument}

\noindent Outre les biens qu’il dispense aux bons etaux méchants dans ce gouvernement général de la nature dont nous venons de dire quelques mots, nous avons encore une preuve du grand amour qu’il porte aux bons en particulier. Certes, en nous donnant l’être, la vie, le privilège de contempler le ciel et la terre, enfin cette intelligence et cette raison qui nous élèvent jusqu’au Créateur de tant de merveilles, il nous a mis dans l’impuissance de trouver des remerciements dignes de ses bienfaits ; mais si nous venons à considérer que dans l’état où nous sommes tombés, c’est-à-dire accablés sous le poids de nos péchés et devenus aveugles par la privation de la vraie lumière et l’amour de l’iniquité, loin de nous avoir abandonnés à nous-mêmes, il a daigné nous envoyer son Verbe, son Fils unique, pour nous apprendre par son incarnation et par sa passion combien l’homme est précieux à Dieu, pour nous purifier de tous nos péchés par ce sacrifice unique, répandre son amour dans nos cœurs par la grâce de son Saint-Esprit, et nous faire arriver, malgré tous les obstacles, au repos éternel et à l’ineffable douceur de la vision bienheureuse, quels cœurs et quelles paroles peuvent suffire aux actions de grâces qui lui sont dues ?
\subsection[{Chapitre XXXII}]{Chapitre XXXII}

\begin{argument}\noindent Le mystère de l’Incarnation n’a manqué à aucun des siècles passés, et par des signes divers il a toujours été annoncé aux hommes.
\end{argument}

\noindent Dès l’origine du genre humain, les anges ont annoncé à des hommes choisis ce mystère de la vie éternelle par des figures et des signes appropriés aux temps. Plus tard, les Hébreux ont été réunis en corps de nation pour figurer ce même mystère, et c’est parmi eux que toutes les choses accomplies depuis l’avènement du Christ jusqu’à nos jours, et toutes celles qui doivent s’accomplir dans la suite des siècles, ont été prédites par des hommes dont les uns comprenaient et les autres ne comprenaient pas ce qu’ils prédisaient. Puis la nation hébraïque a été dispersée parmi les nations, afin de servir de témoin aux Écritures qui annonçaient le salut éternel en Jésus-Christ. Car non seulement toutes les prophéties transmises par la parole, aussi bien que les préceptes de morale et de piété contenus dans les saintes Lettres, mais encore les rites sacrés, les prêtres, le tabernacle, le temple, les autels, les sacrifices, les cérémonies, les fêtes, et généralement tout ce qui appartient au culte qui est dû à Dieu et que les Grecs nomment proprement culte de {\itshape latrie}, tout cela était autant de figures et de prophéties de ce que nous croyons s’être accompli dans le présent, et de ce que nous espérons devoir s’accomplir dans l’avenir par rapport à la vie éternelle dont les fidèles jouiront en Jésus-Christ.
\subsection[{Chapitre XXXIII}]{Chapitre XXXIII}

\begin{argument}\noindent La fourberie des démons, toujours prêts à se réjouir des erreurs des hommes, n’a pu être dévoilée que par la religion chrétienne.
\end{argument}

\noindent La religion chrétienne, la seule véritable, est aussi la seule qui ait pu convaincre les divinités des Gentils de n’être que d’impurs démons, dont le but est de se faire passer pour dieux sous le nom de quelques hommes morts ou de quelques autres créatures, afin d’obtenir des honneurs divins qui flattent leur orgueil et où se mêlent de coupables et abominables impuretés. Ces esprits immondes envient à l’homme son retour salutaire vers Dieu ; mais l’homme s’affranchit de leur domination cruelle et impie, quand il croit en Celui qui lui a enseigné à se relever par l’exemple d’une humilité égale à l’orgueil qui fit tomber les démons. C’est parmi ces esprits de malice qu’il faut placer non seulement tous les dieux dont j’ai déjà beaucoup parlé, et tant d’autres semblables qu’on voit adorés des autres peuples, mais particulièrement ceux dont il est question dans ce livre, je veux dire cette élite et comme ce sénat de dieux qui durent leur rang non à l’éclat de leurs vertus, mais à l’énormité de leurs crimes. En vain Varron s’efforce de justifier les mystères de ces dieux par des explications physiques ; il veut couvrir d’un voile d’honnêteté des choses honteuses et il n’y parvient pas la raison en est simple, c’est que les causes des mystères du paganisme ne sont pas celles qu’il croit ou plutôt qu’il veut faire croire. Si les causes qu’il assigne étaient les véritables, s’il était possible, en effet, d’expliquer les mystères par des raisons naturelles, cette interprétation aurait au moins l’avantage de diminuer le scandale de certaines pratiques qui paraissent obscènes ou absurdes, tant qu’on en ignore le sens. Et c’est justement ce que Varron a essayé de faire pour certainesfictions du théâtre ou certains mystères du temple : or, bien qu’il ait moins réussi à justifier le théâtre par le temple qu’à condamner le temple par le théâtre, il n’a toutefois rien négligé pour affaiblir par de prétendues explications physiques la répugnance qu’inspirent tant de choses abominables.
\subsection[{Chapitre XXXIV}]{Chapitre XXXIV}

\begin{argument}\noindent Des livres de Numa Pompilius, que le sénat fit bruler pour ne point divulguer les causes des institutions religieuses.
\end{argument}

\noindent Et cependant, au témoignage de Varron lui-même, on ne put souffrir les livres de Numa, où sont expliqués les principes de ses institutions religieuses, et on les jugea indignes non seulement d’être lus par les personnes de piété, mais encore d’être conservés par écrit dans le secret des ténèbres. C’est ici le moment de rapporter ce que j’ai promis au troisième livre de placer en son lieu. Voici donc ce qu’on lit dans le traité de Varron sur le culte des dieux : « Un certain Térentius », dit ce savant homme, « possédait une terre au pied du Janicule. Or, il arriva un jour que son bouvier, faisant passer la charrue près du tombeau de Numa Pompilius, déterra les livres où ce roi avait consigné les raisons de ses institutions religieuses. Térentius s’empressa de les porter au préteur, qui, en ayant lu le commencement, jugea la chose assez importante pour en donner avis au sénat. Les principaux de cette assemblée eurent à peine pris connaissance de quelques-unes des raisons par où chaque institution était expliquée, qu’il fut décidé que, sans toucher aux règlements de Numa, il était de l’intérêt de la religion que ses livres fussent brûlés par le préteur. » Chacun en pensera ce qu’il voudra, et il sera même permis à quelque habile défenseur d’une si étrange impiété de dire ici tout ce que l’amour insensé de la dispute lui pourra suggérer ; pour nous, qu’il nous suffise de faire observer que les explications données sur le culte par son propre fondateur, devaient rester inconnues au peuple, au sénat, aux prêtres eux-mêmes, ce qui fait bien voir qu’une curiosité illicite avait initié Numa Pompilius aux secrets des démons ; il les mit donc par écrit pour son usage et afin de s’en souvenir ; mais il n’osa jamais, tout roi qu’il était et n’ayant personne à craindre, ni les communiquer à qui que ce soit, de peur de découvrir aux hommes des mystères d’abominations, ni les effacer ou les détruire, de peur d’irriter ses dieux, et c’est ce qui le porta à les enfouir dans un lieu qu’il crut sûr, ne prévoyant pas que la charrue dût jamais approcher de son tombeau. Quant au sénat, bien qu’il eût pour maxime de respecter la religion des ancêtres, et qu’il fût obligé par là de ne pas toucher aux institutions de Numa, il jugea toutefois ces livres si pernicieux qu’il ne voulut point qu’on les remît en terre, de peur d’irriter la curiosité, et ordonna de livrer aux flammes ce scandaleux monument. Estimant nécessaire le maintien des institutions établies, il pensa qu’il valait mieux laisser les hommes dans l’erreur en leur en dérobant les causes, que de troubler l’État en les leur découvrant.
\subsection[{Chapitre XXXV}]{Chapitre XXXV}

\begin{argument}\noindent De l’hydromancie dont les démons se servaient pour tromper Numa en lui montrant dans l’eau leurs images.
\end{argument}

\noindent Comme aucun prophète de Dieu, ni aucun ange ne fut envoyé à Numa, il eut recours à l’hydromancie pour voir dans l’eau les images des dieux ou plutôt les prestiges des démons, et apprendre d’eux les institutions qu’il devait fonder. Varron dit que ce genre de divination a son origine chez les Perses, et que le roi Numa, et après lui le philosophe Pythagore, en ont fait usage. Il ajoute qu’on interroge aussi les enfers en répandant du sang, ce que les Grecs appellent {\itshape nécromancie} ; mais hydromancie et nécromancie ont ce point commun qu’on se sert des morts pour connaître l’avenir. Comment y réussit-on ? cela regarde les experts en ces matières ; pour moi, je ne veux pas soutenir que ces sortes de divinations fussent interdites par les lois chez tous les peuples et sous des peines rigoureuses, même avant l’avènement du Christ ; je ne dis pas cela, car peut-être étaient-elles permises ; je dis seulement que c’est par des pratiques de ce genre que Numa connut les mystères qu’il institua et dont il dissimula les causes, tant il avait peur lui-même de ce qu’il avait appris. Que vient donc faire ici Varron avec ses explications tirées de la physique ? Si les livres de Numa n’en eussent renfermé que de cette espèce, on ne les eût pas brûlés, ou bien on eût brûlé également les livres de Varron, lesquels sont dédiés au souverain pontife César. La vérité est que le mariage prétendu de Numa Pompilius avec la nymphe Égérie vient de ce qu’il puisait de l’eau pour ses opérations d’hydromancie, ainsi que Varron lui-même le rapporte. Et voilà comme le mensonge fait une fable d’un fait réel. C’est donc par l’hydromancie que ce roi trop curieux fut initié, soit aux mystères qu’il consigna dans les livres des pontifes, soit aux causes de ces mystères dont il se réserva à lui le secret et qu’il fit pour ainsi dire mourir avec lui, en prenant soin de les ensevelir dans son tombeau. Il faut assurément, ou que ces livres continssent des choses assez abominables pour révolter ceux-là mêmes qui avaient déjà reçu des démons bien des rites honteux, ou qu’ils fissent connaître que toutes ces divinités prétendues n’étaient que des hommes morts dont le temps avait consacré le culte chez la plupart des peuples, à la grande joie des démons qui se faisaient adorer sous le nom de ces morts transformés en dieux. Qu’est-il arrivé ? c’est que, par une secrète providence de Dieu, Numa s’étant fait l’ami des démons, grâce à l’hydromancie, ils lui ont tout révélé, sans toutefois l’avertir de brûler en mourant ses livres plutôt que de les enfouir. Ils n’ont pu même empêcher qu’ils n’aient été découverts par un laboureur, et que Varron n’ait fait passer jusqu’à nous cette aventure. Après tout, ils ne peuvent que ce que Dieu leur permet, et Dieu, par un conseil aussi profond qu’équitable, ne leur donne pouvoir que sur ceux qui méritent d’être tentés par leurs prestiges ou trompés par leurs illusions. Ce qui montre, au surplus, à quel point ces livres étaient dangereux et contraires au culte du Dieu véritable, c’est que le sénat passa par-dessus la crainte qui avait arrêté Numa et les fit brûler. Que ceux donc qui n’aspirent point, même en ce monde, à une vie pieuse, demandent la vie éternelle à de tels mystères ! mais pour ceux qui ne veulent point avoir de société avec les démons, qu’ils sachent bien que toutes ces superstitions n’ont rien qui leur puisse être redoutable, et qu’ils embrassent la religion vraie par qui les démons sont dévoilés et vaincus.
\section[{Livre huitième. Théologie naturelle}]{Livre huitième. \\
Théologie naturelle}\renewcommand{\leftmark}{Livre huitième. \\
Théologie naturelle}

\subsection[{Chapitre premier}]{Chapitre premier}

\begin{argument}\noindent De la théologie naturelle et des philosophes qui ont soutenu sur ce point la meilleure doctrine.
\end{argument}

\noindent Nous arrivons à une question qui réclame plus que les précédentes toute l’application de notre esprit. Il s’agit de la théologie naturelle, et nous n’avons point affaire ici à des adversaires ordinaires ; car la théologie qu’on appelle de ce nom n’a rien à démêler, ni avec la théologie fabuleuse des théâtres, ni avec la théologie civile, l’une qui célèbre les crimes des dieux, l’autre qui dévoile les désirs encore plus criminels de ces dieux ou plutôt de ces démons pleins de malice. Nos adversaires actuels, ce sont les philosophes, c’est-à-dire ceux qui font profession d’aimer la sagesse. Or, si la sagesse est Dieu même, Créateur de toutes choses, comme l’attestent la sainte Écriture et la vérité, le vrai philosophe est celui qui aime Dieu. Toutefois, comme il faut bien distinguer entre le nom et la chose, car quiconque s’appelle philosophe n’est pas amoureux pour cela de la véritable sagesse, je choisirai, parmi ceux dont j’ai pu connaître la doctrine par leurs écrits, les plus dignes d’être discutés. Je n’ai pas entrepris, en effet, de réfuter ici toutes les vaines opinions de tous les philosophes, mais seulement les systèmes qui ont trait à la théologie, c’est-à-dire à la science de la Divinité ; et encore, parmi ces systèmes, je ne m’attacherai qu’à ceux des philosophes qui, reconnaissant l’existence de Dieu et sa providence, n’estiment pas néanmoins que le culte d’un Dieu unique et immuable suffise pour obtenir une vie heureuse après la mort, et croient qu’il faut en servir plusieurs, qui tous cependant ont été créés par un seul. Ces philosophes sont déjà très supérieurs à Varron et plus près que lui de la vérité, celui-ci n’ayant pu étendre la théologie naturelle au-delà du monde ou de l’âme du monde, tandis que, suivant les autres, il y a au-dessus de toute âme un Dieu qui a créé non seulement le monde visible, appelé ordinairement le ciel et la terre, mais encore toutes les âmes, et qui rend heureuses les âmes raisonnables et intellectuelles, telles que l’âme humaine, en les faisant participer de sa lumière immuable et incorporelle. Personne n’ignore, si peu qu’il ait ouï parler de ces questions, que les philosophes dont je parle sont les Platoniciens, ainsi appelés de leur maître Platon. Je vais donc parler de Platon ; mais avant de toucher rapidement les points essentiels du sujet, je dirai un mot de ses devanciers.
\subsection[{Chapitre II}]{Chapitre II}

\begin{argument}\noindent Des deux écoles philosophiques, l’école italique et l’école ionienne, et de leurs chefs.
\end{argument}

\noindent Si l’on consulte les monuments de la langue grecque, qui passe pour la plus belle de toutes les langues des Gentils, on trouve deux écoles de philosophie, l’une appelée italique, de cette partie de l’Italie connue sous le nom de grande Grèce, l’autre ionique, du pays qu’on appelle encore aujourd’hui la Grèce. Le chef de l’école italique fut Pythagore de Samos, de qui vient, dit-on, le nom même de philosophie. Avant lui on appelait sages ceux qui paraissaient pratiquer un genre de vie supérieur à celui du vulgaire ; mais Pythagore, interrogé sur sa profession, répondit qu’il était philosophe, c’est-à-dire ami de la sagesse, estimant que faire profession d’être sage, c’était une arrogance extrême. Thalès de Muet fut le chef de la secte ionique. On le compte parmi les sept sages, tandis que les six autres ne se distinguèrent que par leur manière de vivre et par quelques préceptes de morale, Thalès s’illustra par l’étude de la nature des choses, et, afin de propager ses recherches, il les écrivit. Ce qui le fit surtout admirer, c’est qu’ayant saisi les lois de l’astronomie, il put prédire les éclipses du soleil et aussi celles de la lune. Il crut néanmoins que l’eau était le principe de toutes choses, des éléments du monde, du monde lui-même et de tout ce qui s’y produit, sans qu’aucune intelligence divine ne préside à ce grand ouvrage, qui paraît si admirable à quiconque observe l’univers. Après Thalès vint Anaximandre, son disciple, qui se forma une autre idée de la nature des choses. Au lieu de faire venir toutes choses d’un seul principe, tel que l’humide de Thalès, il pensa que chaque chose naît de principes propres. Et ces principes, il en admet une quantité infinie, d’où résultent des mondes innombrables et tout ce qui se produit en chacun d’eux ; ces mondes se dissolvent et renaissent pour se maintenir pendant une certaine durée, et il n’est pas non plus nécessaire qu’aucune intelligence divine prenne part à ce travail des choses. Anaximandre eut pour disciple et successeur Anaximène, qui ramena toutes les causes des êtres à un seul principe, l’air. Il ne contestait ni ne dissimulait l’existence des dieux ; mais, loin de croire qu’ils ont créé l’air, c’est de l’air qu’il les faisait naître. Telle ne fut point la doctrine d’Anaxagore, disciple d’Anaximène ; il comprit que le principe de tous ces objets qui frappent nos yeux est dans un esprit divin. Il pensa qu’il existe une matière infinie, composée de particules homogènes, et que de là sortent tous les genres d’êtres, avec la diversité de leurs modes et de leurs espèces, mais tout cela par l’action de l’esprit divin. Un autre disciple d’Anaximène,Diogène, admit aussi que l’air est la matière où se forment toutes choses, l’air lui-même étant animé par une raison divine, sans laquelle rien n’en pourrait sortir. Anaxagore eut pour successeur son disciple Archélaüs, lequel soutint, à son exemple, que les éléments constitutifs de l’univers sont des particules homogènes d’où proviennent tous les êtres particuliers par l’action d’une intelligence partout présente, qui, unissant et séparant les corps éternels, je veux dire ces particules, est le principe de tous les phénomènes naturels. On assure qu’Archélaüs eut pour disciple Socrate, qui fut le maître de Platon, et c’est pourquoi je suis rapidement remonté jusqu’à ces antiques origines.
\subsection[{Chapitre III}]{Chapitre III}

\begin{argument}\noindent De la philosophie de Socrate.
\end{argument}

\noindent Socrate est le premier qui ait ramené toute la philosophie à la réforme et à la discipline des mœurs car avant lui les philosophes s’appliquaient par-dessus tout à la physique, c’est-à-dire à l’étude des phénomènes de la nature. Est-ce le dégoût de ces recherches obscures et incertaines qui le conduisit à tourner son esprit vers une étude plus accessible, plus assurée, et qui est même nécessaire au bonheur de la vie, ce grand objet de tous les efforts et de toutes les veilles des philosophes ? Ou bien, comme le supposent des interprètes encore plus favorables, Socrate voulait-il arracher les âmes aux passions impures de la terre, en les excitant à s’élever aux choses divines ? c’est une question qu’il me semble impossible d’éclaircir complétement. Il voyait les philosophes tout occupés de découvrir les causes premières, et, persuadé qu’elles dépendent de la volonté d’un Dieu supérieur et unique, il pensa que les âmes purifiées peuvent seules les saisir ; c’est pourquoi il voulait que le premier soin du philosophe fût de purifier son âme par de bonnes mœurs, afin que l’esprit, affranchi des passions qui le courbent vers la terre, s’élevât par sa vigueur native vers les choses éternelles, et pût contempler avec la pure intelligence cette lumière spirituelle et immuable où les causes de toutes les natures créées ontun être stable et vivant. Il est constant qu’il poursuivit et châtia, avec une verve de dialectique merveilleuse et une politesse pleine de sel, la sottise de ces ignorants qui prétendent savoir quelque chose ; confessant, quant à lui, son ignorance, ou dissimulant sa science, même sur ces questions morales où il paraissait avoir appliqué toute la force de son esprit. De là ces inimitiés et ces accusations calomnieuses qui le firent condamner à mort. Mais cette même Athènes, qui l’avait publiquement déclaré criminel, le réhabilita depuis par un deuil public, et l’indignation du peuple alla si loin contre ses accusateurs, que l’un d’eux fut mis en pièces par la multitude, et l’autre obligé de se résoudre à un exil volontaire et perpétuel, pour éviter le même traitement. Également admirable par sa vie et par sa mort, Socrate laissa un grand nombre de sectateurs qui, s’appliquant à l’envi aux questions de morale, disputèrent sur le souverain bien, sans lequel l’homme ne peut être homme. Et comme l’opinion de Socrate ne se montrait pas très clairement au milieu de ces discussions contradictoires, où il agite, soutient et renverse tous les systèmes, chaque disciple y prit ce qui lui convenait et résolut à sa façon la question, de la fin suprême, par où ils entendent ce qu’il faut posséder pour être heureux. Ainsi se formèrent, parmi les socratiques, plusieurs systèmes sur le souverain bien, avec une opposition si incroyable entre ces disciples d’un même maître, que les uns mirent le souverain bien dans la volupté, comme Aristippe, les autres dans la vertu, comme Antisthène, et d’autres dans d’autres fins, qu’il serait trop long de rapporter.
\subsection[{Chapitre IV}]{Chapitre IV}

\begin{argument}\noindent De Platon, principal disciple de Socrate, et de sa division de la philosophie en trois parties.
\end{argument}

\noindent Mais entre tous les disciples de Socrate, celui qui à bon droit effaça tous les autres par l’éclat de la gloire la plus pure, ce fut Platon. Né athénien, d’une famille honorable, son merveilleux génie le mit de bonne heure au premier rang. Estimant toutefois que la doctrine de Socrate et ses propres recherches nesuffisaient pas pour porter la philosophie à sa perfection, il voyagea longtemps et dans les pays les plus divers, partout où la renommée lui promettait quelque science à recueillir. C’est ainsi qu’il apprit en Égypte toutes les grandes choses qu’on y enseignait ; il se dirigea ensuite vers les contrées de l’Italie où les pythagoriciens étaient en honneur, et là, dans le commerce des maîtres les plus éminents, il s’appropria aisément toute la philosophie de l’école italique. Et comme il avait pour Socrate un attachement singulier, il le mit en scène dans presque tous ses dialogues, unissant ce qu’il avait appris d’autres philosophes, et même ce qu’il avait trouvé par les plus puissants efforts de sa propre intelligence, aux grâces de la conversation de Socrate et à ses entretiens familiers sur la morale, Or, si l’étude de la sagesse consiste dans l’action et dans la spéculation, ce qui fait qu’on peut appeler l’une de ses parties, active et l’autre spéculative, la partie active se rapportant à la conduite de la vie, c’est-à-dire aux mœurs, et la partie spéculative à la recherche des causes naturelles et de la vérité en soi, on peut dire que l’homme qui avait excellé dans la partie active, c’était Socrate, et que celui qui s’était appliqué de préférence à la partie contemplative avec toutes les forces de son génie, c’était Pythagore. Platon réunit ces deux parties, et s’acquit ainsi la gloire d’avoir porté la philosophie à sa perfection. Il la divisa en trois branches la morale, qui regarde principalement l’action ; la physique, dont l’objet est la spéculation ; la logique enfin, qui distingue le vrai d’avec le faux ; or, bien que cette dernière science soit également nécessaire pour la spéculation et pour l’action, c’est à la spéculation toutefois qu’il appartient plus spécialement d’étudier la nature du vrai, par où l’on voit que la division de la philosophie en trois parties s’accorde avec la distinction de la science spéculative et de la science pratique, De savoir maintenant quels ont été les sentiments de Platon surchacun de ces trois objets, c’est-à-dire où il a mis la fin de toutes les actions, la cause de tous les êtres et la lumière de toutes les intelligences, ce serait une question longue à discuter et qu’il ne serait pas convenable de trancher légèrement. Comme il affecte constamment de suivre la méthode de Socrate, interlocuteur ordinaire de ses dialogues, lequel avait coutume, comme on sait, de cacher sa science ou ses opinions, il n’est pas aisé de découvrir ce que Platon lui-même pensait sur un grand nombre de points. Il nous faudra pourtant citer quelques passages de ses écrits, où, exposant tour à tour sa propre pensée et celle des autres, tantôt il se montre favorable à la religion véritable, à celle qui a notre foi et dont nous avons pris la défense, et tantôt il y paraît contraire, comme quand il s’agit, par exemple, de l’unité divine et de la pluralité des dieux, par rapport à la vie véritablement heureuse qui doit commencer après la mort. Au surplus, ceux qui passent pour avoir le plus fidèlement suivi ce philosophe, si supérieur à tous les autres parmi les Gentils, et qui sont le mieux entrés dans le fond de sa pensée véritable, paraissent avoir de Dieu une si juste idée, que c’est en lui qu’ils placent la cause de toute existence, la raison de toute pensée et la fin de toute vie : trois principes dont le premier appartient à la physique, le second à la logique, et le troisième à la morale ; et véritablement, si l’homme a été créé pour atteindre, à l’aide de ce qu’il y a de plus excellent en lui, ce qui surpasse tout en excellence, c’est-à-dire un seul vrai Dieu souverainement bon, sans lequel aucune nature n’a d’existence, aucune science de certitude, aucune action d’utilité, où faut-il donc avant tout le chercher, sinon où tous les êtres ont un fondement assuré, où toutes les vérités deviennent certaines, et où se rectifient toutes nos affections ?
\subsection[{Chapitre V}]{Chapitre V}

\begin{argument}\noindent Il faut discuter de préférence avec les Platoniciens en matière de théologie, leurs opinions étant meilleures que celles de tous les autres philosophes.
\end{argument}

\noindent Si Platon a défini le sage celui qui imite le vrai Dieu, le connaît, l’aime et trouve la béatitude dans sa participation avec lui, à quoi bon discuter contre les philosophes ? il est clair qu’il n’en est aucun qui soit plus près de nous que Platon. Qu’elle cède donc aux Platoniciens cette théologie fabuleuse qui repaît les âmes des impies des crimes de leurs dieux ! qu’elle leur cède aussi cette théologie civile où les démons impurs, se donnant pour des dieux afin de mieux séduire les peuples asservis aux voluptés de la terre, ont voulu consacrer l’erreur, faire de la représentation de leurs crimes une cérémonie du culte, et trouver ainsi pour eux-mêmes, dans les spectateurs de ces jeux, le plus agréable des spectacles : théologie impure où ce que les temples peuvent avoir d’honnête est corrompu par son mélange avec les infamies du théâtre, et où ce que le théâtre a d’infâme est justifié par les abominations des temples ! Qu’elles cèdent encore à ces philosophes les explications de Varron qui a voulu rattacher le paganisme à la terre et au ciel, aux semences et aux opérations de la nature ; car, d’abord, les mystères du culte païen n’ont pas le sens qu’il veut leur donner, et par conséquent la vérité lui échappe en dépit de tous ses efforts ; de plus, alors même qu’il aurait raison, l’âme raisonnable ne devrait pas adorer comme son Dieu ce qui est au-dessous d’elle dans l’ordre de la nature, ni préférer à soi, comme des divinités, des êtres auxquels le vrai Dieu l’a préférée. Il faut en dire autant de ces écrits que Numa consacra en effet aux mystères sacrés, mais qu’il prit soin d’ensevelir avec lui, et qui, exhumés par la charrue d’un laboureur, furent livrés aux flammes par le sénat ; et pour traiter plus favorablement Numa, mettons au même rang cette lettre où Alexandre de Macédoine, confiant à sa mère les secrets que lui avaient dévoilés un certain Léon, grand-prêtre égyptien, lui faisait voir non seulement que Picus, Faunus, Énée, Romulus, ou encore Hercule, Esculape, Liber, fils de Sémélé, les Tyndarides et autres mortels divinisés, mais encore les grands dieux, ceux dont Cicéron a l’air de parler dans les {\itshape Tusculanes} sans les nommer, Jupiter, Junon, Saturne, Vulcain, Vesta et plusieurs autres dont Varron a fait les symboles des éléments et des parties du monde, ont été des hommes, et rien de plus ; or, ce prêtre égyptien craignant, lui aussi, que ces mystères ne vinssent à être divulgués, pria Alexandre de recommander à sa mère de jeter sa lettre au feu. Que cette théologie donc, civile et fabuleuse, cède aux philosophes platoniciens qui ont reconnu le vrai Dieu comme auteur de la nature, comme source de la vérité, comme dispensateur de la béatitude ! et je ne parle pas seulement de la théologie païenne, mais que sont auprès de ces grands adorateurs d’un si grand Dieu tous les philosophes dont l’intelligence asservie au corps n’a donné à la nature que des principes corporels, comme Thalès qui attribue tout à l’eau, Anaximène à l’air, les stoïciens au feu, Épicure aux atomes, c’est-à-dire à de très petits corpuscules invisibles et impalpables, et tant d’autres qu’il est inutile d’énumérer, qui ont cru que des corps, simples ou composés, inanimés ou vivants, mais après tout des corps, étaient la cause et le principe des choses. Quelques-uns, en effet, ont pensé que des choses vivantes pouvaient provenir de choses sans vie : c’est le sentiment des Épicuriens ; d’autres ont admis que choses vivantes et choses sans vie proviennent d’un vivant ; mais ce sont toujours des corps qui proviennent d’un corps ; car pour les stoïciens, c’est le feu, c’est-à-dire un corps un des quatre éléments qui constituent l’univers visible, qui est vivant, intelligent, auteur du monde et de tous les êtres, en un mot, qui est Dieu. Voilà donc les plus hautes pensées où aient pu s’élever ces philosophes et tous ceux qui ont cherché la vérité d’un cœur assiégé par les chimères des sens. Et cependant ils avaient en eux, d’une certaine manière, des objets que leurs sens ne pouvaient saisir ; ils se représentaient au dedans d’eux-mêmes les choses qu’ils avaient vues au dehors, alors même qu’ils ne les voyaient plus par les yeux, mais seulement par la pensée. Or, ce qu’on voit de la sorte n’est plus un corps, mais son image, et ce qui perçoit dans l’âme cette image n’est ni un corps ni une image ; enfin, le principe qui juge cette image comme étant belle ou laide, est sans doute supérieur à l’objet de son jugement. Ce principe, c’est l’intelligence de l’homme, c’est l’âme raisonnable ; et certes il n’a rien de corporel, puisque déjà l’image qu’il perçoit et qu’il juge n’est pas un corps. L’âme n’est donc ni terre, ni eau, ni air, ni feu, ni en général aucun de ces quatre corps nommés éléments qui forment le monde matériel. Et comment Dieu, Créateur de l’âme, serait-il un corps ? Qu’ils cèdent donc, je le répète, aux Platoniciens, tous ces philosophes, et je n’en excepte pas ceux qui, à la vérité, rougissent de dire que Dieu est un corps, mais qui le font de même nature que nos âmes. Se peut-il qu’ils n’aient point vu dans l’âme humaine cette étrange mutabilité, qu’on ne peut attribuer à Dieu sans crime ? Mais, disent-ils, c’est le corps qui rend l’âme changeante, car de soi elle est immuable. Que ne disent-ils aussi que ce sont les corps extérieurs qui blessent la chair et qu’elle est invulnérable de soi ? La vérité est que rien ne peut altérer l’immuable ; d’où il suit que ce qui peut être altéré par un corps n’est pas véritablement immuable.
\subsection[{Chapitre VI}]{Chapitre VI}

\begin{argument}\noindent Sentiments des Platoniciens touchant la physique.
\end{argument}

\noindent Ces philosophes, si justement supérieurs aux autres en gloire et en renommée, ont compris que nul corps n’est Dieu, et c’est pourquoi ils ont cherché Dieu au-dessus de tous les corps. Ils ont également compris que tout ce qui est muable n’est pas le Dieu suprême, et c’est pourquoi ils ont cherché le Dieu suprême au-dessus de toute âme et de tout esprit sujet au changement. Ils ont compris enfin qu’en tout être muable, la forme qui le fait ce qu’il est, quels que soient sa nature et ses modes, ne peut venir que de Celui qui est en vérité, parce qu’il est immuablement. Si donc vous considérez tour à tour le corps du monde entier avec ses figures, ses qualités, ses mouvements réguliers et ses éléments qui embrassent dans leur harmonie le ciel, la terre et tous les êtres corporels, puis l’âme en général, tant celle qui maintient les parties du corps et le nourrit, comme dans les arbres, que celles qui donnent en outre le sentiment, comme dans les animaux, et celle qui ajoute au sentiment la pensée, comme dans les hommes, et celle enfin qui n’a pas besoin de la faculté nutritive et se borne à maintenir, sentir et penser, comme chez les anges, rien de tout cela, corps ou âme, ne peut tenir l’être que de Celui qui est ; car, en lui, être n’est pas une chose, et vivre, une autre, comme s’il pouvait être sans être vivant ; et de même, la vie en lui n’est pas une chose et la pensée une autre, comme s’il pouvait vivre et vivre sans penser, et enfin la pensée en lui n’est pas une chose et le bonheur une autre, comme s’il pouvait penser et ne pas être heureux ; mais, pour lui, vivre, penser, être heureux, c’est simplement être. Or, ayant compris cette immutabilité et cette simplicité parfaites, les Platoniciens ont vu que toutes choses tiennent l’être de Dieu, et que Dieu ne le tient d’aucun. Tout ce qui est, en effet, est corps ou âme, et il vaut mieux être âme que corps ; de plus, la forme du corps est sensible, celle de l’âme est intelligible ; d’où ils ont conclu que la forme intelligible est supérieure à la forme sensible. Il faut entendre par sensible ce qui peut être saisi par la vue et le tact corporel, par intelligible ce qui peut être atteint par le regard de l’âme. La beauté corporelle, en effet, soit qu’elle consiste dans l’état extérieur d’un corps, dans sa figure, par exemple, soit dans son mouvement, comme cela se rencontre en musique, a pour véritable juge l’esprit. Or, cela serait impossible s’il n’y avait point dans l’esprit une forme supérieure, indépendante de la grandeur, de la masse, du bruit des sons, de l’espace et du temps. Admettez maintenant que cette forme ne soit pas muable, comment tel homme jugerait-il mieux que tel autre des choses sensibles, le plus vif d’esprit mieux que le plus lent, le savant mieux que l’ignorant, l’homme exercé mieux que l’inculte, la même personne une fois cultivée mieux qu’avant de l’être ? Or, ce qui est susceptible de plus et de moins est muable ; d’où ces savants et pénétrants philosophes, qui avaient fort approfondi ces matières, ont conclu avec raison que la forme première ne pouvait se rencontrer dans des êtres convaincus de mutabilité. Voyant donc que le corps et l’âme ont des formes plus ou moins belles et excellentes, et que, s’ils n’avaient point de forme, ils n’auraient point d’être, ils ont compris qu’il y a un être où se trouve la forme première et immuable, laquelle à ce titre n’est comparable avec aucune autre ; par suite, que là est le principe des choses, qui n’est fait par rien et par qui tout est fait. Et c’est ainsi que ce qui est connu de Dieu, Dieu lui-même l’a manifesté à ces philosophes, depuis que les profondeurs invisibles de son essence, sa vertu créatrice et sa divinité éternelle, sont devenues visibles par ses ouvrages. J’en aidit assez sur cette partie de la philosophie qu’ils appellent physique, c’est-à-dire relative à la nature.
\subsection[{Chapitre VII}]{Chapitre VII}

\begin{argument}\noindent Combien les Platoniciens sont supérieurs dans la logique au reste des philosophes.
\end{argument}

\noindent Quant à la logique ou philosophie rationnelle, loin de moi la pensée de comparer aux Platoniciens ceux qui placent le critérium de la vérité dans les sens, et mesurent toutes nos connaissances avec cette règle inexacte et trompeuse ! tels sont les Épicuriens et plusieurs autres philosophes, parmi lesquels il faut comprendre les Stoïciens, qui ont fait venir des sens les principes de cette dialectique où ils exercent avec tant d’ardeur la souplesse de leur esprit. C’est à cette source qu’ils ramènent leurs concepts généraux, {\itshape ennoiai}, qui servent de base aux définitions ; c’est de là, en un mot, qu’ils tirent la suite et le développement de toute leur méthode d’apprendre et d’enseigner. J’admire, en vérité, comment ils peuvent soutenir en même temps leur principe que les sages seuls sont beaux, et je leur demanderais volontiers quel est le sens qui leur a fait apercevoir cette beauté, et avec quels yeux ils ont vu la forme et la splendeur de la sagesse. C’est ici que nos philosophes de prédilection ont parfaitement distingué ce que l’esprit conçoit de ce qu’atteignent les sens, ne retranchant rien à ceux-ci de leur domaine légitime, n’y ajoutant rien et déclarant nettement que cette lumière de nos intelligences qui nous fait comprendre toutes choses, c’est Dieu même qui a tout créé.
\subsection[{Chapitre VIII}]{Chapitre VIII}

\begin{argument}\noindent En matière de philosophie morale les Platoniciens ont encore le premier rang.
\end{argument}

\noindent Reste la morale ou, pour parler comme les Grecs, l’éthique, où l’on cherche le souverain bien, c’est-à-dire l’objet auquel nous rapportons toutes nos actions, celui que nous désirons pour lui-même et non en vue de quelque autre chose, de sorte qu’en le possédant il ne nous manque plus rien pour être heureux. C’est encore ce qu’on nomme la fin, parce que nous voulons tout le reste en vue de notre bien, et ne voulons pas le bien pour autre chose que lui. Or, le bien qui produit la béatitude, les uns l’ont fait venir du corps, les autres de l’esprit, d’autres de tous deux ensemble. Les philosophes, en effet, voyant que l’homme est composé de corps et d’esprit, ont pensé que l’un ou l’autre ou tous deux ensemble pouvaient constituer son bien, je veux dire ce bien final, source du bonheur, dernier terme de toutes les actions, et qui ne laisse rien à désirer au-delà de soi. C’est pourquoi ceux qui ont ajouté une troisième espèce de biens qu’on appelle extérieurs, comme l’honneur, la gloire, les richesses, et autres semblables, ne les ont point regardés comme faisant partie du bien final, mais comme de ces choses qu’on désire en vue d’une autre fin, qui sont bonnes pour les bons et mauvaises pour les méchants. Mais, quoi qu’il en soit, ceux qui ont fait dépendre le bien de l’homme, soit du corps, soit de l’esprit, soit de tous deux, n’ont pas cru qu’il fallût le chercher ailleurs que dans l’homme même. Les premiers le font dépendre de la partie la moins noble de l’homme, les seconds, de la partie la plus noble, les autres, de l’homme tout entier ; mais dans tous les cas, c’est de l’homme que le bien dépend. Au surplus, ces trois points de vue n’ont pas donné lieu à trois systèmes seulement, mais à un beaucoup plus grand nombre, parce que chacun s’est formé une opinion différente sur le bien du corps sur le bien de l’esprit, sur le bien de l’un et l’autre réunis. Que tous cèdent donc à ces philosophes qui ont fait consister le bonheur de l’homme, non à jouir du corps ou de l’esprit, mais à jouir de Dieu, et non pas à en jouir comme l’esprit jouit du corps ou de soi-même, ou comme un ami jouit d’un ami, mais comme l’œil jouit de la lumière. Il faudrait insister peut-être pour montrer la justesse de cette comparaison ; mais j’aime mieux le faire ailleurs, s’il plaît à Dieu, et selon la mesure de mes forces. Présentement il me suffit de rappeler que le souverain bien pour Platon, c’est de vivre selon la vertu, ce qui n’est possible qu’à celui qui connaît Dieu et qui l’imite ; et voilà l’unique source du bonheur. Aussi n’hésite-t-il point à dire que philosopher, c’est aimer Dieu, dont la nature est incorporelle ; d’où il suit que l’ami de la sagesse, c’est-à-dire le philosophe, ne devient heureux que lors qu’il commence de jouir de Dieu. En effet, bien que l’on ne soit pas nécessairement heureux pour jouir de ce qu’on aime, car plusieurs sont malheureux d’aimer ce qui ne doit pas être aimé, et plus malheureux encore d’en jouir, personne toutefois n’est heureux qu’autant qu’il jouit de ce qu’il aime. Ainsi donc, ceux-là mêmes qui aiment ce qui ne doit pas être aimé, ne se croient pas heureux par l’amour, mais par la jouissance. Qui donc serait assez malheureux pour ne pas réputer heureux celui qui aime le souverain bien et jouit de ce qu’il aime ! Or, Platon déclare que le vrai et souverain bien, c’est Dieu, et voilà pourquoi il veut que le vrai philosophe soit celui qui aime Dieu, car le philosophe tend à la félicité, et celui qui aime Dieu est heureux en jouissant de Dieu.
\subsection[{Chapitre IX}]{Chapitre IX}

\begin{argument}\noindent De la philosophie qui a le plus approché de la vérité chrétienne.
\end{argument}

\noindent Ainsi donc tous les philosophes, quels qu’ils soient, qui ont eu ces sentiments touchant le Dieu suprême et véritable, et qui ont reconnu en lui l’auteur de toutes les choses créées, la lumière de toutes les connaissances et la fin de toutes les actions, c’est-à-dire le principe de la nature, la vérité de la doctrine et la félicité de la vie, ces philosophes qu’on appellera platoniciens ou d’un autre nom, soit qu’on n’attribue de tels sentiments qu’aux chefs de l’école Ionique, à Platon par exemple et à ceux qui l’ont bien entendu, soit qu’on en fasse également honneur à l’école italique, à cause de Pythagore, des Pythagoriciens, et peut-être aussi de quelques autres philosophes de la même famille, soit enfin qu’on veuille les étendre aux sages et aux philosophes des autres nations, Libyens atlantiques, Égyptiens, Indiens, Perses, Chaldéens, Scythes, Gaulois, Espagnols et à d’autres encore, ces philosophes, dis-je, nous les préférons à tous les autres et nous confessons qu’ils ont approché de plus près de notre croyance.
\subsection[{Chapitre X}]{Chapitre X}

\begin{argument}\noindent La foi d’un bon chrétien est fort au-dessus de toute la science des philosophes.
\end{argument}

\noindent Un chrétien qui s’est uniquement appliqué à la lecture des saints livres, ignore peut-être le nom des Platoniciens ; il ne sait pas qu’il y a eu parmi les Grecs deux écoles de philosophie, l’ionienne et l’Italique ; mais il n’est pas tellement sourd au bruit des choses humaines, qu’il n’ait appris que les philosophes font profession d’aimer la sagesse ou même de la posséder. Il se défie pourtant de cette philosophie qui s’enchaîne aux éléments dumonde au lieu de s’appuyer sur Dieu, Créateur du monde, averti par ce précepte de l’Apôtre qu’il écoute d’une oreille fidèle : « Prenez garde de vous laisser abuser par la philosophie et par de vains raisonnements sur les éléments du monde. » Mais, afin de ne pas appliquer ces paroles à tous les philosophes, le chrétien écoute ce que l’Apôtre dit de quelques-uns : « Ce qui peut être connu de Dieu, ils l’ont connu clairement, Dieu-même le leur ayant fait connaître ; car depuis la création du monde les profondeurs invisibles de son essence sont devenues saisissables et visibles par ses ouvrages ; et sa vertu et sa divinité sont éternelles. » Et de même, quand l’Apôtre parle aux Athéniens, après avoir dit de Dieu cette grande parole qu’il est donné à peu de comprendre « C’est en lui que nous avons la vie, le mouvement et l’être » ; il poursuit et ajoute : « Comme l’ont même dit quelques-uns de vos sages. » Ici encore le chrétien sait se garder des erreurs où ces grands philosophes sont tombés ; car, au même endroit où il est écrit que Dieu leur a rendu saisissables et visibles par ses ouvrages ses invisibles profondeurs, il est dit aussi qu’ils n’ont pas rendu à Dieu le culte légitime, parce qu’ils ont transporté à d’autres objets les honneurs qui ne sont dus qu’à lui « Ils ont connu Dieu, dit l’Apôtre, et ils ne l’ont pas glorifié et adoré comme Dieu ; mais ils se sont perdus dans leurs chimériques pensées, et leur cœur insensé s’est rempli de ténèbres. En se disant sages ils sont devenus fous, et ils ont prostitué la gloire du Dieu incorruptible à l’image de l’homme corruptible, à des figures d’oiseaux, dequadrupèdes et de serpents. » L’Apôtre veut désigner ici les Romains, les Grecs et les Égyptiens, qui se sont fait gloire de leur sagesse ; mais nous aurons affaire à eux dans la suite de cet ouvrage. Bornons-nous à dire encore une fois que notre préférence est acquise à ces philosophes qui confessent avec nous un Dieu unique, Créateur de l’univers, non seulement incorporel et à ce titre au-dessus de tous les corps, mais incorruptible et comme tel au-dessus de toutes les âmes ; en un mot, notre principe, notre lumière et notre bien.\par
Que si un chrétien, étranger aux lettres profanes, ne se sert pas en discutant de termes qu’il n’a point appris, et n’appelle pas {\itshape naturelle} avec les Latins et {\itshape physique} avec les Grecs cette partie de la philosophie qui regarde la nature, rationnelle ou logique celle qui traite de la connaissance de la vérité, morale enfin ou éthique celle où il est question des mœurs, des biens à poursuivre et des maux à éviter, est-ce à dire qu’il ignore que nous tenons du vrai Dieu, unique et parfait, la nature qui nous fait être à son image, la science qui le révèle à nous et nous révèle à nous-mêmes, la grâce enfin qui nous unit à lui pour nous rendre heureux ? Voilà donc pourquoi nous préférons les Platoniciens au reste des philosophes : c’est que ceux-ci ont vainement consumé leur esprit et leurs efforts pour découvrir les causes des êtres, la règle de la vérité et celle de la vie, au lieu que les Platoniciens, ayant connu Dieu, ont trouvé par là même où est la cause de tous les êtres, la lumière où l’on voit la vérité, la source où l’on s’abreuve du bonheur. Platoniciens ou philosophes d’une autre nation, s’il en est qui aient eu aussi de Dieu une telle idée, je dis qu’ils pensent comme nous. Pourquoi maintenant, dans la discussion qui va s’ouvrir, n’ai-je voulu avoir affaire qu’aux disciples de Platon ? c’est que leurs écrits sont plus connus. En effet, les Grecs, dont la langue est la première parmi les Gentils, ont partout répandu la doctrine platonicienne, et les Latins, frappés de son excellence ou séduits par la renommée, l’ont étudiée de préférence à toute autre, et cri la traduisant dans notre langue ont encore ajouté à son éclat et à sa popularité.
\subsection[{Chapitre XI}]{Chapitre XI}

\begin{argument}\noindent Comment Platon a pu autant approcher de la doctrine chrétienne.
\end{argument}

\noindent Parmi ceux qui nous sont unis dans la grâce de Jésus-Christ, quelques-uns s’étonnent d’entendre attribuer à Platon ces idées sur la Divinité, qu’ils trouvent singulièrement conformes à la véritable religion. Aussi cette ressemblance a-t-elle fait croire à plus d’un chrétien que Platon, lors de son voyage en Égypte, avait entendu le prophète Jérémie ou lu les livres des Prophètes. J’ai moi-même admis cette opinion dans quelques-uns de mes ouvrages ; mais une étude approfondie de la chronologie démontre que la naissance de Platon est postérieure d’environ cent ans à l’époque où prophétisa Jérémie ; et Platon ayant vécu quatre-vingt-un ans, entre le moment de sa mort et celui de la traduction des Écritures demandée par Ptolémée, roi d’Égypte, à soixante-dix Juifs versés dans la langue grecque, il s’est écoulé environ soixante années. Platon, par conséquent, n’a pu, pendant son voyage, ni voir Jérémie, mort depuis si longtemps, ni lire en cette langue grecque, où il excellait, une version des Écritures qui n’était pas encore faite ; à moins que, poussé par sa passion de savoir, il n’ait connu les livres hébreux comme il avait fait les livres égyptiens, à l’aide d’un interprète, non sans doute en se les faisant traduire, ce qui n’appartient qu’à un roi puissant comme Ptolémée par les bienfaits et par la crainte, mais en mettant à profit la conversation de quelques Juifs pour comprendre autant que possible la doctrine contenue dans l’Ancien Testament. Ce qui favorise cette conjecture, c’est le début de la Genèse : « Au commencement Dieu fit le ciel et la terre. Et la terre était une masse confuse et informe, et lesténèbres couvraient la surface de l’abîme, et l’esprit de Dieu était porté sur les eaux. » Or, Platon, dans le {\itshape Timée}, où il décrit la formation du monde, dit que Dieu a commencé son ouvrage en unissant la terre avec le feu ; et comme il est manifeste que le feu tient ici la place du ciel, cette opinion a quelque analogie avec la parole de l’Écriture : « Au commencement Dieu fit le ciel et la terre. » — Platon ajoute que l’eau et l’air furent les deux moyens de jonction qui servirent à unir les deux extrêmes, la terre et le feu ; on a vu là une interprétation de ce passage de l’Écriture : « Et l’esprit de Dieu était porté sur les eaux. »\par
Platon ne prenant pas garde au sens du mot esprit de Dieu dans l’Écriture, où l’air est souvent appelé esprit, semble avoir cru qu’il est question dans ce passage des quatre éléments. Quant à cette doctrine de Platon, que le philosophe est celui qui aime Dieu, les saintes Écritures ne respirent pas autre chose. Mais ce qui me fait surtout pencher de ce côté, ce qui me déciderait presque à affirmer que Platon n’a pas été étranger aux livres saints, c’est la réponse faite à Moïse, quand il demande à l’ange le nom de celui qui lui ordonne de délivrer le peuple hébreu captif en Égypte : « Je suis Celui qui suis », dit la Bible, « et vous direz aux enfants d’Israël : “Celui qui est m’a envoyé vers vous.” » Par où il faut entendre que les choses créées et changeantes sont comme si elles n’étaient pas, au prix de Celui qui est véritablement, parce qu’il est immuable. Or, voilà ce que Platon a soutenu avec force, et ce qu’il s’est attaché soigneusement à inculquer à ses disciples. Je ne sais si on trouverait cette pensée dans aucun monument antérieur à Platon, excepté le livre où il est écrit : « Je suis Celui qui suis ; et vous leur direz : Celui qui est m’envoie vers vous. »
\subsection[{Chapitre XII}]{Chapitre XII}

\begin{argument}\noindent Les Platoniciens, tout en ayant une juste idée du Dieu unique et véritable, n’en ont pas moins jugé nécessaire le culte de plusieurs divinités.
\end{argument}

\noindent Mais ne déterminons pas de quelle façon Platon a connu ces vérités, soit qu’il les ait puisées dans les livres de ceux qui l’ont précédé, soit que, comme dit l’Apôtre, « les sages a aient connu avec évidence ce qui peut être connu de Dieu, Dieu lui-même le leur ayant rendu manifeste. Car depuis la création du monde les perfections invisibles de Dieu, sa vertu et sa divinité éternelles, sont devenues saisissables et visibles par ses ouvrages ». Quoi qu’il en soit, je crois avoir assez prouvé que je n’ai pas choisi sans raison les Platoniciens, pour débattre avec eux cette question de théologie naturelle : s’il faut servir un seul Dieu on en servir plusieurs pour la félicité de l’autre vie. Je les ai choisis en effet, parce que l’excellence de leur doctrine sur un seul Dieu, Créateur du ciel et de la terre, leur a donné parmi les philosophes le rang le plus illustre et le plus glorieux ; or, cette supériorité a été depuis si bien reconnue que vainement Aristote, disciple de Platon, homme d’un esprit éminent, inférieur sans doute à Platon par l’éloquence, mais de beaucoup supérieur à tant d’autres, fonda la secte péripatéticienne, ainsi nommée de l’habitude qu’avait Aristote d’enseigner en se promenant ; vainement il attira, du vivant même de son maître, vers cette école dissidente un grand nombre de disciples séduits par l’éclat de sa renommée ; vainement aussi, après la mort de Platon, Speusippe, son neveu, et Xénocrate, son disciple bien-aimé, le remplacèrent à l’Académie et eurent eux-mêmes des successeurs qui prirent le nom d’Académiciens ; tout cela n’a pas empêché les meilleurs philosophes de notre temps qui ont voulu suivre Platon, de se faire appeler non pas Péripatéticiens ni Académiciens, mais Platoniciens. Les plus célèbres entre les Grecs sont Plotin, Jamblique et Porphyre ; joignez à ces Platoniciens, illustres l’africain Apulée, également versé dans les deux langues, la grecque et la latine. Or, maintenant il est de fait que tous ces philosophes et les autres de la même école, et Platon lui-même, ont cru qu’il fallait adorer plusieurs dieux.
\subsection[{Chapitre XIII}]{Chapitre XIII}

\begin{argument}\noindent De l’opinion de Platon touchant les dieux, qu’il définit des êtres essentiellement bons et amis de la vertu.
\end{argument}

\noindent Bien qu’il y ait entre les Platoniciens et nous plusieurs autres dissentiments de grande conséquence, la discussion que j’ai soulevée n’est pas médiocrement grave, et c’est pourquoi je leur pose cette question : quels dieux faut-il adorer ? les bons ou les méchants ? ou les uns et les autres ? Nous avons sur ce point le sentiment de Platon ; car il dit que tous les dieux sont bons et qu’il n’y a pas de dieux méchants ; d’où il suit que c’est aux bons qu’il faut rendre hommage, puisque, s’ils n’étaient pas bons, ils ne seraient pas dieux. Mais s’il en est ainsi (et comment penser autrement des dieux ?), que devient cette opinion qu’il faut apaiser les dieux méchants par des sacrifices, de peur qu’ils ne nous nuisent, et invoquer les bons afin qu’ils nous aident ? En effet, il n’y a pas de dieux méchants, et c’est aux bons seulement que doit être rendu le culte qu’ils appellent légitime. Je demande alors ce qu’il faut penser de ces dieux qui aiment les jeux scéniques au point de vouloir qu’on les mêle aux choses divines et aux cérémonies célébrées en leur honneur ? La puissance de ces dieux prouve leur existence, et leur goût pour les jeux impurs atteste leur méchanceté. On sait assez ce que pense Platon des représentations théâtrales, puisqu’il chasse les poètes de l’État, pour avoir composé des fictions indignes de la majesté et de la bonté divines. Que faut-il donc penser de ces dieux qui sont ici en lutte avec Platon ? lui ne souffrant pas que les dieux soient déshonorés par des crimes imaginaires, ceux-ci ordonnant de représenter ces crimes en leur honneur. Enfin, quand ils prescrivirent des jeux scéniques, ils firent éclater leur malice en même temps que leur impureté, soit en privant Latinius de son fils, soit en le frappant lui-même pour leur avoir désobéi, et ne lui rendant la santé qu’après qu’il eut exécuté leur commandement. Et cependant, si méchants qu’ils soient, Platon n’estime pas qu’on doive les craindre, et il demeure ferme dans son sentiment, qu’il faut bannir d’un État bien réglé toutes ces folies sacrilèges des prêtres, qui n’ont de charme pour les dieux impurs que par leur impureté même. Or, ce même Platon, comme je l’ai remarqué au second livre du présent ouvrage, est mis par Labéon au nombre des demi-dieux ; ce qui n’empêche pas Labéon de penser qu’il faut apaiser les dieux méchants par des sacrifices sanglants et des cérémonies analogues à leur caractère, et honorer les bons par des jeux et des solennités riantes. D’où vient donc que le demi-dieu Platon persiste si fortement à priver, non pas des demi-dieux, mais des dieux, des dieux bons par conséquent, de ces divertissements qu’il répute infâmes ? Au surplus, ces dieux ont eux-mêmes pris soin de réfuter Labéon, puisqu’ils ont montré à l’égard de Latinius, non seulement leur humeur lascive et folâtre, mais leur impitoyable cruauté. Que les Platoniciens nous expliquent cela, eux qui soutiennent avec leur maître que tous les dieux sont bons, chastes, amis de la vertu et des hommes sages, et qu’il y a de l’impiété à en juger autrement ? Nous l’expliquons, disent-ils. Écoutons-les donc avec attention.
\subsection[{Chapitre XIV}]{Chapitre XIV}\phantomsection
\label{\_chapitre14}

\begin{argument}\noindent Des trois espèces d’âmes raisonnables admises par les Platoniciens, celles des dieux dans le ciel, celles des démons dans l’air et celles des hommes sur la terre.
\end{argument}

\noindent Il y a suivant eux trois espèces d’animaux doués d’une âme raisonnable, savoir : les dieux, les hommes et les démons. Les dieux occupent la région la plus élevée, les hommes la plus basse, les démons la moyenne ; car la région des dieux, c’est le ciel, celle des hommes la terre, celle des démons l’air. À cette différence dans la dignité, de leur séjour répond la diversité de leur nature. Les dieux sont plus excellents que les hommes et que les démons ; les hommes le sont moins que les démons et que les dieux. Ainsi donc, let démons étant au milieu, de même qu’il faut les estimer moins que les dieux, puisqu’ils habitent plus bas, il faut les estimer plus queles hommes, puisqu’ils habitent plus haut. Et en effet, s’ils partagent avec les dieux le privilège d’avoir un corps immortel, ils ont, comme les hommes, une âme sujette aux passions. Pourquoi donc s’étonner, disent les Platoniciens, que les démons se plaisent aux obscénités du théâtre et aux fictions des poètes, puisqu’ils ont des passions comme les hommes, au lieu d’en être exempts par leur nature comme les dieux ? D’où on peut conclurequ’en réprouvant et en interdisant les fictions des poètes, ce n’est point aux dieux, qui sont d’une nature excellente, que Platon a voulu ôter le plaisir des spectacles, mais aux démons.\par
Voilà ce qu’on trouve dans Apulée de Madaure, qui a composé sur ce sujet un livre intitulé {\itshape Du dieu de Socrate} ; il y discute et y explique à quel ordre de divinités appartenait cet esprit familier, cet ami bienveillant qui avertissait Socrate, dit-on, de se désister de toutes les actions qui ne devaient pas tourner à son avantage. Après avoir examiné avec soin l’opinion de Platon touchant les âmes sublimes des dieux, les âmes inférieures des hommes et les âmes mitoyennes des démons, il déclare nettement et prouve fort au long que cet esprit familier n’était point un dieu, mais un démon. Or, s’il en est ainsi, comment Platon a-t-il été assez hardi pour ôter, sinon aux dieux, purs de toute humaine contagion, du moins aux démons, le plaisir des spectacles en bannissant les poètes de l’État ? n’est-il pas clair qu’il a voulu par là enseigner aux hommes, tout engagés qu’ils sont dans les misères d’un corps mortel, à mépriser les commandements honteux des démons et à fuir ces impuretés pour se tourner vers la lumière sans tache de la vertu ? Point de milieu : ou Platon s’est montré honnête en réprimant et en proscrivant les jeux du théâtre, ou les démons, en les demandant et les prescrivant, se sont montrés corrompus. Il faut donc dire qu’Apulée se trompe et que Socrate n’a pas eu un démon pour ami, ou bien que Platon se contredit en traitant les démons avec respect, après avoir banni leurs jeux favoris de tout État bien réglé, ou bien enfin qu’il n’y a pas à féliciter Socrate de l’amitié de son démon ; et en effet, Apulée lui-même en a été si honteux qu’il a intitulé son livre : {\itshape Du dieu de Socrate}, tandis que pour rester fidèle à sa distinction si soigneusement et si longuement établie entre les dieux et les démons, il aurait dû l’intituler, non Du dieu, mais Du démon de Socrate. Il a mieux aimé placer cette distinction dans le corps de l’ouvrage que sur le titre. C’est ainsi que, depuis le moment où la saine doctrine a brillé parmi les hommes, le nom des démons est devenu presque universellement odieux, au point même qu’avant d’avoir lu le plaidoyer d’Apulée en faveur des démons, quiconque aurait rencontré un titre comme celui-ci : {\itshape Du démon de Socrate}, n’aurait pu croire que l’auteur fût dans son bon sens. Aussi bien, qu’est-ce qu’Apulée a trouvé à louer dans les démons, si ce n’est la subtilité et la vigueur de leur corps et la hauteur de leur séjour ? Quand il vient à parler de leurs mœurs en général, loin d’en dire du bien, il en dit beaucoup de mal ; de sorte qu’après avoir lu son livre, on ne s’étonne plus que les démons aient voulu placer les turpitudes du théâtre parmi les choses divines, qu’ils prennent plaisir aux spectacles des crimes des dieux, voulant eux-mêmes passer pour des dieux ; enfin que les obscénités dont on amuse le public et les atrocités dont on l’épouvante, soient en parfaite harmonie avec leurs passions.
\subsection[{Chapitre XV}]{Chapitre XV}

\begin{argument}\noindent Les démons ne sont vraiment supérieurs aux hommes, ni par leur corps aérien, ni par la région plus élevée où ils font leur séjour.
\end{argument}

\noindent À Dieu ne plaise donc qu’une âme vraiment pieuse se croie inférieure aux démons parce qu’ils ont un corps plus parfait ! À ce compte, il faudrait qu’elle mît au-dessus de soi un grand nombre de bêtes qui nous surpassent par la subtilité de leurs sens, l’aisance et la rapidité de leurs mouvements et la longévité de leur corps robuste ! Quel homme a la vue perçante des aigles et des vautours, l’odorat subtil des chiens, l’agilité des lièvres, des cerfs, de tous les oiseaux, la force du lion et de l’éléphant ? Vivons-nous aussi longtemps que les serpents, qui passent même pour rajeunir et quitter la vieillesse avec la tunique dont ils se dépouillent ? Mais, de même que la raison et l’intelligence nous élèvent au-dessus de tous ces animaux, la pureté et l’honnêteté de notre vie doivent nous mettre au-dessus des démons. Il a plu à la divine Providence de donner à des êtres qui nous sont très inférieurs certains avantages corporels, pour nous apprendre à cultiver, de préférence au corps, cette partie de nous-mêmes qui fait notre supériorité, et à compter pour rien au prix de la vertu la perfection corporelle des démons. Et d’ailleurs, ne sommes-nous pas destinés, nous aussi, à l’immortalité du corps, non pour subir, comme les démons, une éternité de peines, mais pour recevoir la récompense d’une vie pure ?\par
Quant à l’élévation de leur séjour, s’imaginer que les démons valent mieux que nous parce qu’ils habitent l’air et nous la terre, cela est parfaitement ridicule. Car à ce titre nous serions au-dessous de tous les oiseaux. Mais, disent-ils, les oiseaux s’abattent sur la terre pour se reposer ou se repaître, ce que ne font pas les démons. Je leur demande alors s’ils veulent estimer les oiseaux supérieurs aux hommes, au même titre qu’ils préfèrent les démons aux oiseaux ? Que si cette opinion est extravagante, l’élément supérieur qu’habitent les démons ne leur donne donc aucun droit à nos hommages. De même, en effet, que les oiseaux, habitants de l’air, ne sont pas pour cela au-dessus de nous, habitants de la terre, mais nous sont soumis au contraire à cause de l’excellence de l’âme raisonnable qui est en nous, ainsi les démons, malgré leur corps aérien, ne doivent pas être estimés plus excellents que nous, sous prétexte que l’air est supérieur à la terre ; mais ils sont au contraire au-dessous des hommes, parce qu’il n’y a point de comparaison entre le désespoir où ils sont condamnés et l’espérance des justes. L’ordre même et la proportion que Platon établit dans les quatre éléments, lorsqu’il place entre le plus mobile de tous, le feu, et le plus immobile, la terre, les deux éléments de l’air et de l’eau comme termes moyens en sorte qu’autant l’air est au-dessus de l’eau et le feu au-dessus de l’air, autant l’eau est au-dessus de la terre, cet ordre, dis-je, nous apprend à ne point mesurer la valeur des êtres animés selon la hiérarchie des éléments. Apulée lui-même, aussi bien que les autres Platoniciens, appelle l’homme un animal terrestre ; et cependant cet animal est plus excellent que tous les animaux aquatiques, bienque Platon place l’eau au-dessus de la terre. Ainsi donc, quand il s’agit de la valeur des âmes, ne la mesurons pas selon l’ordre apparent des corps, et sachons qu’il peut se faire qu’une âme plus parfaite anime un corps plus grossier, et une âme moins parfaite un corps supérieur.
\subsection[{Chapitre XVI}]{Chapitre XVI}

\begin{argument}\noindent Sentiment du platonicien Apulée touchant les mœurs et les actions des démons.
\end{argument}

\noindent Le même platonicien, parlant des mœurs des démons, dit qu’ils sont agités des mêmes passions que les hommes, que les injures les irritent, que les hommages et les offrandes les apaisent, qu’ils aiment les honneurs, qu’ils prennent plaisir à la variété des rites sacrés, et que la moindre négligence à cet égard leur cause un sensible déplaisir. C’est d’eux que relèvent, à ce qu’il nous assure, les prédictions des augures, aruspices, devins, les présages des songes, à quoi il ajoute les miracles de la magie. Puis il les définit brièvement en ces termes : Les démons, quant au genre, sont des animaux ; ils sont, quant à l’âme, sujets aux passions ; quant à l’intelligence, raisonnables ; quant au corps, aériens ; quant au temps, éternels ; et il fait observer que les trois premières qualités se rencontrent également chez les hommes, que la quatrième est propre aux démons et que la cinquième leur est commune avec les dieux. Mais je remarque à mon tour qu’entre les trois premières qualités qu’ils partagent avec les hommes, il en est deux qui leur sont aussi communes avec les dieux. Les dieux, en effet, sont des animaux dans les idées d’Apulée qui, assignant à chaque espèce son élément, appelle les hommes animaux terrestres, les poissons et tout ce qui nage, animaux aquatiques, les démons, animaux aériens, et les dieux, animaux célestes. Par conséquent, si les démons sont des animaux, cela leur est commun, non seulement avec les hommes, mais aussi avec les dieux et avec les brutes ; raisonnables, cela leur est commun avec les dieux et avec les hommes ; éternels, avec les dieux seuls ; sujets aux passions, avec les seuls hommes ; aériens, voilà ce qui est propre aux seuls démons. Ce n’est donc pas un grand avantage pour eux d’appartenir au genre animal, puisque les brutes y sont avec eux ; avoir une âme raisonnable, ce n’est pas être au-dessus de nous, puisque nous sommes aussi doués de raison ; à quoi bon posséder une vie éternelle, si ce n’est point une vie heureuse ? car mieux vaut une félicité temporelle qu’une éternité misérable ; être sujets aux passions, c’est un triste privilège que nous possédons comme eux et qui est un effet de notre misère. Enfin, comment un corps aérien serait-il une qualité d’un grand prix, quand il est certain que toute âme, quelle que soit sa nature, est de soi supérieure à tout corps ; et dès lors, comment le culte divin, hommage de l’âme, serait-il dû à ce qui est au-dessous d’elle ? Que si, parmi les qualités qu’Apulée attribue aux démons, il comptait la vertu, la sagesse et la félicité, s’il disait que ces avantages leur sont communs avec les dieux et qu’ils les possèdent éternellement, je verrais là quelque chose de grand et de désirable ; et cependant on ne devrait pas encore les adorer comme on adore Dieu, mais plutôt adorer en Dieu la source de ces merveilleux dons. Tant il s’en faut qu’ils méritent les honneurs divins, ces animaux aériens qui n’ont la raison que pour pouvoir être misérables, les passions que pour l’être en effet, l’éternité que pour l’être éternellement !
\subsection[{Chapitre XVII}]{Chapitre XVII}\phantomsection
\label{\_chapitre17}

\begin{argument}\noindent S’il convient à l’homme d’adorer des esprits dont il lui est commandé de fuir les vices.
\end{argument}

\noindent Pour ne considérer maintenant dans les démons que ce qui leur est commun avec les hommes suivant Apulée, c’est-à-dire les passions, s’il est vrai que chacun des quatre éléments ait ses animaux, le feu et l’air les immortels, la terre et l’eau les mortels, je voudrais bien savoir pourquoi les âmes des démons sont sujettes aux troubles et aux orages des passions ; car le mot passion, comme le mot grec {\itshape Pathos} ; dont il dérive, marque un état de perturbation, un mouvement de l’âme contraire à la raison. Comment se fait-il donc que l’âme des démons éprouve ces passions dont les bêtes sont exemptes ? Si en effet il se trouve en elles quelques mouvements analogues, on n’y peut voir des perturbations contraires à la raison, les bêtes étant privées de raison. Dans les hommes, quand la passion trouble l’âme, c’est un effet de sa folie ou de sa misère ; car nous ne possédons point ici-bas cette béatitude et cette perfection de la sagesse qui nous sont promises à la fin des temps au sortir de ce corps périssable. Quant aux dieux, nos philosophes prétendent que s’ils sont à l’abri des passions, c’est qu’ils possèdent non seulement l’éternité, mais la béatitude ; et quoiqu’ils aient une âme comme le reste des animaux, cette âme est pure de toute tache et de toute altération. Eh bien ! s’il en va de la sorte, si les dieux ne sont point sujets aux passions en tant qu’animaux doués de béatitude et exempts de misère, si les bêtes en sont affranchies en qualité d’animaux incapables de misère comme de béatitude, il reste que les démons y soient accessibles au même titre que les hommes, à titre d’animaux misérables.\par
Quelle déraison, ou plutôt quelle folie de nous asservir aux démons par un culte, quand la véritable religion nous délivre des passions vicieuses qui nous rendent semblables à eux ! Car Apulée, qui les épargne beaucoup et les juge dignes des honneurs divins, Apulée lui-même est forcé de reconnaître qu’ils sont sujets à la colère ; et la vraie religion nous ordonne de ne point céder à la colère, mais d’y résister. Les démons se laissent séduire par des présents, et la vraie religion ne veut pas que l’intérêt décide de nos préférences. Les démons se complaisent aux honneurs, et la vraie religion nous défend d’y être sensibles. Les démons aiment ceux-ci, haïssent ceux-là, non par le choix sage et calme de la raison, mais par l’entraînement d’une âme passionnée ; et la vraie religion nous prescrit d’aimer même nos ennemis. Enfin tous ces mouvements du cœur, tous ces orages de l’esprit, tous ces troubles et toutes ces tempêtes de l’âme, dont Apulée convient que les démons sont agités, la vraie religion nous ordonne de nous en affranchir. N’est-ce donc pas une folie et un aveuglement déplorables que de s’humilier par l’adoration devant des êtres à qui on désire ne pas être semblable, et de prendre pour objet de sa religion des dieux qu’on ne veut pas imiter, quand toute la substance de la religion, c’est d’imiter ce qu’on adore ?
\subsection[{Chapitre XVIII}]{Chapitre XVIII}

\begin{argument}\noindent Ce qu’on doit penser d’une religion qui reconnaît les démons pour médiateurs nécessaires des hommes auprès des dieux.
\end{argument}

\noindent C’est donc en vain qu’Apulée et ses adhérents font aux démons l’honneur de les placer dans l’air, entre le ciel et la terre, pour transmettre aux dieux les prières des hommes et aux hommes les faveurs des dieux, sous prétexte qu’« aucun dieu ne communique avec l’homme », suivant le principe qu’ils attribuent à Platon. Chose singulière ! ils ont pensé qu’il n’était pas convenable aux dieux de se mêler aux hommes, mais qu’il était convenable aux démons d’être le lien entre les prières des hommes et les bienfaits des dieux ; de sorte que l’homme juste, étranger par cela même aux arts de la magie, sera obligé de prendre pour intercesseurs auprès des dieux ceux qui se plaisent à ces criminelles pratiques, alors que l’aversion qu’elles lui inspirent est justement ce qui le rend plus digne d’être exaucé par les dieux. Aussi bien ces mêmes démons aiment les turpitudes du théâtre, tandis que la pudeur les déteste ; ils se plaisent à tous les maléfices de la magie, tandis que l’innocence les a en mépris. Voilà donc l’innocence et la pudeur condamnées, pour obtenir quelque faveur des dieux, à prendre pour intercesseurs leurs propres ennemis. C’est en vain qu’Apulée chercherait à justifier les fictions des poètes et les infamies du théâtre ; nous avons à lui opposer l’autorité respectée de son maître Platon, si toutefois l’homme peut à ce point renoncer à la pudeur que non seulement il aime des choses honteuses, mais qu’il les juge agréables à la Divinité.
\subsection[{Chapitre XIX}]{Chapitre XIX}

\begin{argument}\noindent La magie est impie quand elle a pour base la protection des esprits malins.
\end{argument}

\noindent Pour confondre ces pratiques de la magie, dont quelques hommes sont assez malheureux et assez impies pour tirer vanité, je ne veux d’autre témoin que l’opinion publique. Si en effet les opérations magiques sont l’ouvrage de divinités dignes d’adoration, pourquoi sont-elles si rudement frappées par la sévérité des lois ? Sont-ce les chrétiens qui ont fait ces lois ? Admettez que les maléfices des magiciens ne soient pas pernicieux au genre humain, pourquoi ces vers d’un illustre poète ?\par
 {\itshape « J’en atteste les dieux et toi-même, chère sœur, et ta tête chérie c’est à regret que j’ai recours aux conjurations magiques. »} \par
Et pourquoi cet autre vers ?\par
{\itshape « Je l’ai vu transporter des moissons d’un champ dans un autre »}, \par
allusion à cette science pernicieuse et criminelle qui fournissait, disait-on, le moyen de transporter à son gré les fruits de la terre ? Et puis Cicéron ne remarqua-t-il pas qu’une loi des douze Tables, c’est-à-dire une des plus anciennes lois de Rome, punit sévèrement les magiciens ? Enfin, est-ce devant les magistrats chrétiens qu’Apulée fut accusé de magie + ? Certes, s’il eût pensé que ces pratiques fassent innocentes, pieuses et en harmonie avec les œuvres de la puissance divine, il devait non seulement les avouer, mais faire profession de s’en servir et protester contre les lois qui interdisent et condamnent un art digne d’admiration et de respect. De cette façon, ou il aurait persuadé ses juges, ou si, trop attachés à d’injustes lois, ils l’avaient condamné à mort, les démons n’auraient pas manqué de récompenser son courage. C’est ainsi que lorsqu’on imputait à crime à nos martyrs cette religion chrétienne où ils croyaient fermement trouver leur salut et une éternité de gloire, ils ne la reniaient pas pour éviter des peines temporelles, mais au contraire ils la confessaient, ils la professaient, ils la proclamaient ; et c’est en souffrant pour elle avec courage et fidélité, c’est en mourant avec une tranquillité pieuse, qu’ils firent rougir la loi de son injustice et en amenèrent la révocation. Telle n’a point été la conduite du philosophe platonicien. Nous avons encore le discours très étendu et très disert où il se défend contre l’action de magie ; et s’il s’efforce d’y paraître innocent, c’est en niant les actions qu’on ne peut faire innocemment. Or, tous ces prodiges de la magie, qu’il juge avec raison condamnables, ne s’accomplissent-ils point par la science et par les œuvres des démons ? Pourquoi donc veut-il qu’on les honore ? pourquoi dit-il que nos prières ne peuvent parvenir aux dieux que par l’entremise de ces mêmes démons dontnous devons fuir les œuvres, si nous voulons que nos prières parviennent jusqu’au vrai Dieu ? D’ailleurs, je demande quelle sorte de prières les démons présentent aux dieux bons :des prières magiques ou des prières permises ? les premières, ils n’en veulent pas ; les secondes, ils les veulent par d’autres médiateurs. De plus, si un pécheur pénitent vient à prier, se reconnaissant coupable d’avoir donné dans la magie, obtiendra-t-il son pardon par l’intercession de ceux qui l’ont poussé au crime ? ou bien les démons eux-mêmes, pour obtenir le pardon des pécheurs, feront-ils tous les premiers pénitence pour les avoir séduits ? C’est ce qui n’est jamais venu à l’esprit de personne ; car s’ils se repentaient de leurs crimes et en faisaient pénitence, ils n’auraient pas la hardiesse de revendiquer pour eux les honneurs divins ; une superbe si détestable ne peut s’accorder avec une humilité si digne de pardon.
\subsection[{Chapitre XX}]{Chapitre XX}

\begin{argument}\noindent S’il est croyable que des dieux bons préfèrent avoir commerce avec les démons qu’avec les hommes.
\end{argument}

\noindent Il y a, suivant eux, une raison pressante et impérieuse qui fait que les démons sont les médiateurs nécessaires entre les dieux et les hommes. Voyons cette raison, cette prétendue nécessité. C’est, disent-ils, qu’aucun dieu ne communique avec l’homme. Voilà une étrange idée de la sainteté divine ! elle empêche Dieu de communiquer avec l’homme suppliant, et le fait entrer en commerce avec le démon superbe ! Ainsi, Dieu ne communique pas avec l’homme pénitent, et il communique avec le démon séducteur ; il ne communique pas avec l’homme qui invoque la Divinité, et il communique avec le démon qui l’usurpe ; il ne communique pas avec l’homme implorant l’indulgence, et il communique avec le démon conseillant l’iniquité ; il ne communique pas avec l’homme qui, éclairé par les livres des philosophes, chasse les poètes d’un État bien réglé, et il communique avec le démon, qui exige du sénat et des pontifes qu’on représente sur la scène les folles imaginations des poètes ; il ne communique pas avec l’homme qui interdit d’imputer aux dieux des crimes fantastiques, et il communique avec le démon qui se complaît à voir ces crimes donnés en spectacle ; il ne communique pas avec l’homme qui punit par de justes lois les pratiques des magiciens, et il communique avec le démon qui enseigne et exerce la magie ; il ne communique pas avec l’homme qui fuit les œuvres des démons, et il communique avec le démon qui tend des pièges à la faiblesse de l’homme.
\subsection[{Chapitre XXI}]{Chapitre XXI}\phantomsection
\label{\_chapitre21}

\begin{argument}\noindent Si les dieux se servent des démons comme de messagers et d’interprètes, et s’ils sont trompés par eux, à leur insu ou de leur plein gré.
\end{argument}

\noindent Mais, disent-ils, ce qui vous paraît d’une absurdité et d’une indignité révoltantes est absolument nécessaire, les dieux de l’éther ne pouvant rien savoir de ce que font les habitants de la terre que par l’intermédiaire des démons de l’air ; car l’éther est loin de la terre, à une hauteur prodigieuse, au lieu que l’air est à la fois contigu à l’éther et à la terre. Ô l’admirable sagesse et le beau raisonnement ! Il faut, d’un côté, que les dieux dont la nature est essentiellement bonne, aient soin des choses humaines, de peur qu’on ne les juge indignes d’être honorés ; de l’autre côté, il faut que, par suite de la distance des éléments, ils ignorent ce qui se passe sur la terre, afin de rendre indispensable le ministère des démons et d’accréditer leur culte parmi les peuples, sous prétexte que c’est par leur entremise que les dieux peuvent être informés des choses d’en bas, et venir au secours des mortels. Si cela est, les dieux bons connaissent mieux les démons par la proximité de leurs corps que les hommes par la bonté de leurs âmes. Ô déplorable nécessité, ou plutôt ridicule et vaine erreur, imaginée pour couvrir le néant de vaines divinités ! En effet, s’il est possible aux dieux de voir notre esprit par leur propre esprit libre des obstacles du corps, ils n’ont pas besoin pour cela du ministère des démons ; si, au contraire, les dieux ne connaissent les esprits qu’en percevant, à l’aide de leurs propres corps éthérés, les signes corporels tels que le visage, la parole, les mouvements ; si c’est de la sorte qu’ils recueillent les messages des démons, rien n’empêche qu’ils ne soient abusés par leurs mensonges. Or, comme il est impossible que la Divinité soit trompée par les démons, il est impossible aussi que la Divinité ignore ce que font les hommes.\par
J’adresserais volontiers une question à ces philosophes : Les démons ont-ils fait connaître aux dieux l’arrêt prononcé par Platon contre les fictions sacrilèges des poètes, sans leur avouer le plaisir qu’ils prennent à ces fictions ? ou bien ont-ils gardé le silence sur ces deux choses ? ou bien les ont-ils révélées toutes deux, ainsi que leur libertinage, plus injurieux à la divinité que la religieuse sagesse de Platon ? ou bien, enfin, ont-ils caché aux dieux la condamnation dont Platon a frappé la licence calomnieuse du théâtre ? et, en même temps, ont-ils eu l’audace et l’impudeur de leur avouer le plaisir criminel qu’ils prennent à ce spectacle des dieux avilis ? Qu’on choisisse entre ces quatre suppositions : je n’en vois aucune où il ne faille penser beaucoup de mal des dieux bons. Si l’on admet la première, il faut accorder qu’il n’a pas été permis aux dieux bons de communiquer avec un bon philosophe qui les défendait contre l’outrage, et qu’ils ont communiqué avec les démons qui se réjouissaient de les voir outragés. Ce bon philosophe, en effet, était trop loin des dieux bons pour qu’il leur fût possible de le connaître autrement que par des démons méchants qui ne leur étaient pas déjà très bien connus malgré le voisinage. Si l’on veut que les démons aient caché aux dieux tout ensemble et le pieux arrêt de Platon et leurs plaisirs sacrilèges, à quoi sert aux dieux, pour la connaissance des choses humaines, l’entremise des démons, du moment qu’ils ne savent pas ce que font des hommes pieux, par respect pour la majesté divine, contre le libertinage des esprits méchants ? J’admets la troisième supposition, que les démons n’ont pas fait connaître seulement aux dieux le pieux sentiment de Platon, mais aussi le plaisir criminel qu’ils prennent à voir la Divinité avilie, je dis qu’un tel rapport adressé aux dieux est plutôt un insigne outrage. Et cependant on admet que les dieux, sachant tout cela, n’ont pas rompu commerce avec les démons, ennemis de leur dignité comme de la piété de Platon, mais qu’ils ont chargé ces indignes voisins de transmettre leurs dons au vertueux Platon, trop éloigné d’eux pour les recevoir de leur main. Ils sont donc tellement liés par la chaîne indissoluble des éléments, qu’ils peuvent communiquer avec leurs calomniateurs et ne le peuvent pas avec leurs défenseurs, connaissant les uns et les autres, mais ne pouvant pas changer le poids de la terre et de l’air. Reste la quatrième supposition, mais c’est la pire de toutes : car comment admettre que les démons aient révélé aux dieux, et les fictions calomnieuses de la poésie, et les folies sacrilèges du théâtre, et leur passion ardente pour les spectacles, et le plaisir singulier qu’ils y prennent, et qu’en même temps ils leur aient dissimulé que Platon, au nom d’une philosophie sévère, a banni ces jeux criminels d’un État bien réglé ? À ce compte les dieux seraient contraints d’apprendre par ces étranges messagers les dérèglements les plus coupables, ceux de ces messagers mêmes, et il ne leur serait pas permis de connaître les bons sentiments des philosophes ; singulier moyen d’information, qui leur apprend ce qu’on fait pour les outrager, et leur cache ce qu’on fait pour les honorer !
\subsection[{Chapitre XXII}]{Chapitre XXII}

\begin{argument}\noindent Il faut malgré Apulée rejeter le culte des démons.
\end{argument}

\noindent Ainsi donc, puisqu’il est impossible d’admettre aucune de ces quatre suppositions, il faut rejeter sans réserve cette doctrine d’Apulée et de ses adhérents, que les démons sont placés entre les hommes et les dieux, comme des interprètes et des messagers, pour transmettre au ciel les vœux de la terre et à la terre les bienfaits du ciel. Tout au contraire, ce sont des esprits possédés du besoin de nuire, étrangers à toute idée de justice, enflés d’orgueil, livides d’envie, artisans de ruses et d’illusions ; ils habitent l’air, en effet, mais comme une prison analogue à leur nature, où ils ont été condamnés à faire leur séjour après avoir été chassés des hauteurs du ciel pour leur transgression inexpiable ; et, bien que l’air soit situé au-dessus de la terre et des eaux, les démons ne sont pas pour cela moralement supérieurs aux hommes, qui ont sur eux un tout autre avantage que celui du corps, c’est de posséder une âme pieuse et d’avoir mis leur confiance dans l’appui du vrai Dieu. Je conviens que les démons dominent sur un grand nombre d’hommes indignes de participer à la religion véritable ; c’est aux yeux de ceux-là qu’ils se sont fait passer pour des dieux, grâce à leurs faux prestiges et à leurs fausses prédictions. Encore n’ont-ils pu réussir à tromper ceux de ces hommes qui ont considéré leurs vices de plus près, et alors ils ont pris le parti de se donner pour médiateurs entre les dieux et les hommes, et pour distributeurs des bienfaits du ciel. Ainsi s’est formée l’opinion de ceux qui, connaissant les démons pour des esprits méchants, et persuadés que les dieux sont bons par nature, ne croyaient pas à la divinité des démons et refusaient de leur rendre les honneurs divins, sans oser toutefois les en déclarer indignes, de crainte de heurter les peuples asservis à leur culte par une superstition invétérée.
\subsection[{Chapitre XXIII}]{Chapitre XXIII}

\begin{argument}\noindent Ce que pensait Hermès Trismégiste de l’idolâtrie, et comment il a pu savoir que les superstitions de l’Égypte seraient abolies.
\end{argument}

\noindent Hermès l’Égyptien, celui qu’on appelle Trismégiste, a eu d’autres idées sur les démons. Apulée, en effet, tout en leur refusant le titre de dieux, voit en eux les médiateurs nécessaires des hommes auprès des dieux, et dès lors le culte des démons et celui des dieux restent inséparables ; Hermès, au contraire, distingue deux sortes de dieux : les uns qui ont été formés par le Dieu suprême, les autres qui sont l’ouvrage des hommes. À s’en tenir là, on conçoit d’abord que ces dieux, ouvrages des hommes, ce sont les statues qu’on voit dans les temples ; point du tout ; suivant Hermès, les statues visibles et tangibles ne sont que le corps des dieux, et il les croit animées par de certains esprits qu’on a su y attirer et qui ont le pouvoir de nuire comme aussi celui de faire du bien à ceux qui leur rendent les hommages du culte et les honneurs divins. Unir ces esprits invisibles à une matière corporelle pour en faire des corps animés, des symboles vivants dédiés et soumis aux esprits qui les habitent, voilà ce qu’il appelle faire des dieux, et il soutient que les hommes possèdent ce grand et merveilleux pouvoir. Je rapporterai ici ses paroles, telles qu’elles sont traduites dans notre langue : « Puisque l’alliance et la société des hommes et des dieux font le sujet de notre entretien, considérez, Esculape, quelle est la puissance et la force de l’homme. De même que le Seigneur et Père, Dieu en un mot, a produit les dieux du ciel ; ainsi l’homme a formé les dieux qui font leur séjour dans les temples et habitent auprès de lui. » Et un peu après : « L’homme donc, se souvenant de sa nature et de son origine, persévère dans cette imitation de la Divinité, de sorte qu’à l’exemple de ce Père et Seigneur qui a fait des dieux éternels comme lui, l’homme s’est formé des dieux à sa ressemblance. » Ici Esculape, à qui Hermès s’adresse, lui ayant dit : « Tu veux parler des statues, Trismégiste », celui-ci répond : « Oui, c’est des statues que je parle, Esculape, quelque doute qui puisse t’arrêter, de ces statues vivantes toutes pénétrées d’esprit et de sentiment, qui font tant et de si grandes choses, de ces statues qui connaissent l’avenir et le prédisent par les sortilèges, les devins, les songes et de plusieurs autres manières, qui envoient aux hommes des maladies et qui les guérissent, qui répandent enfin dans les cœurs, suivant le mérite de chacun, la joie ou la tristesse. Ignores-tu, Esculape, que l’Égypte est l’image du ciel, ou, pour mieux parler, que le ciel, avec ses mouvements et ses lois, y est comme descendu ; enfin, s’il faut tout dire, que notre pays est le temple de l’univers ? Et cependant, puisqu’il est d’un homme sage de tout prévoir, voici une chose que vous ne devez pas ignorer : un temps viendra où il sera reconnu que les Égyptiens ont vainement gardé dans leur cœur pieux un culte fidèle à la Divinité, et toutes leurs cérémonies saintes tomberont dans l’oubli et le néant. »\par
Hermès s’étend fort longuement sur ce sujet, et il semble prédire le temps où la religion chrétienne devait détruire les vaines superstitions de l’idolâtrie par la puissance de sa vérité et de sa sainteté librement victorieuses, alors que la grâce du vrai Sauveur viendrait arracher l’homme au joug des dieux qui sont l’ouvrage de l’homme, pour le soumettre au Dieu dont l’homme est l’ouvrage. Mais, quand il fait cette prédiction, Hermès, tout en parlant en ami déclaré des prestiges des démons, ne prononce pas nettement le nom du christianisme ; il déplore au contraire, avec l’accent de la plus vive douleur, la ruine future de ces pratiques religieuses qui, suivant lui, entretenaient en Égypte la ressemblance de l’homme avec les dieux. Car il était de ceux dont l’Apôtre dit : « Ils ont connu Dieu sans le glorifier et l’adorer comme Dieu ; mais ils se sont perdus dans leurs chimériques pensées, et leur cœur insensé s’est rempli de ténèbres. En se disant sages ils sont devenus fous, et ils ont prostitué la gloire de l’incorruptible divinité à l’image de l’homme corruptible. »\par
On trouve en effet dans Hermès un grand nombre de pensées vraies sur le Dieu unique et véritable qui a créé l’univers ; et je ne sais par quel aveuglement de cœur il a pu vouloir que les hommes demeurassent toujours soumis à ces dieux qui sont, il en convient, leur propre ouvrage, et s’affliger de la ruine future de cette superstition. Comme s’il y avait pour l’homme une condition plus malheureuse que d’obéir en esclave à l’œuvre de ses mains ! Après tout, il lui est plus facile de cesser d’être homme en adorant les dieux qu’il a faits, qu’il ne l’est à ces idoles de devenir dieux par le culte qu’il leur rend ; que l’homme, en effet, {\itshape déchu de l’état glorieux où il a été mis, descende au rang des brutes}, c’est une chose plus facile que de voir l’ouvrage de l’homme devenir plus excellent que l’ouvrage de Dieu fait à son image, c’est-à-dire que l’homme même. Et il est juste par conséquent que l’homme tombe infiniment au-dessous de son Créateur, quand il met au-dessus de soi sa propre créature.\par
Voilà les illusions pernicieuses et les erreurs sacrilèges dont Hermès l’Égyptien prévoyait et déplorait l’abolition ; mais sa plainte était aussi impudente que sa science était téméraire. Car le Saint-Esprit ne lui révélait pas l’avenir comme il faisait aux saints Prophètes qui, certains de la chute future des idoles, s’écriaient avec joie : « Si l’homme se fait des dieux, ce ne seront point des dieux véritables. » Et ailleurs : « Le jour viendra, dit le Seigneur, où je chasserai les noms des idoles de la face de la terre, et la mémoire même en périra. » Et Isaïe, prophétisant de l’Égypte en particulier : « Les idoles de l’Égypte seront renversées devant le Seigneur, et le cœur des Égyptiens se sentira vaincu. » Parmi les inspirés du Saint-Esprit, il faut placer aussi ces personnages qui se réjouissaient des événements futurs dévoilés à leurs regards, comme Siméon et Anne qui connurent Jésus-Christ aussitôt après sa naissance ; ou comme Élisabeth, qui le connut en esprit dès sa conception ; ou comme saint Pierre qui s’écria, éclairé par une révélation du Père : « Vous êtes le Christ, Fils du Dieu vivant. » Quant à cet Égyptien, les esprits qui lui avaient révélé le temps de leur défaite, étaient ceux-là mêmes qui dirent en tremblant à Notre-Seigneur pendant sa vie mortelle : « Pourquoi êtes-vous venu nous perdre avant le temps ? » soit qu’ils fussent surpris de voir arriver sitôt ce qu’ils prévoyaient à la vérité, mais sans le croire si proche, soit qu’ils fissent consister leur perdition à être démasqués et méprisés. Et cela arrivait avant le temps, c’est-à-dire avant l’époque du jugement, où ils seront livrés à la damnation éternelle avec tous les hommes qui auront accepté leur société ; car ainsi l’enseigne la religion, celle qui ne trompe pas, qui n’est pas trompée, et qui ne ressemble pas à ce prétendu sage flottant à tout vent de doctrine, mêlant le faux avec le vrai, et se lamentant sur la ruine d’une religion convaincue d’erreur par son propre aveu.
\subsection[{Chapitre XXIV}]{Chapitre XXIV}

\begin{argument}\noindent Tout en déplorant la ruine future de la religion de ses pères, Hermès en confesse ouvertement la fausseté.
\end{argument}

\noindent Après un long discours Hermès reprend en ces termes ce qu’il avait dit des dieux formés par la main des hommes : « En voilà assez pour le moment sur ce sujet ; revenons à l’homme et à ce don divin de la raison qui lui mérite le nom d’animal raisonnable. On a beaucoup célébré les merveilles de la nature humaine ; mais, si étonnantes qu’elles paraissent, elles ne sont rien à côté de cette merveille incomparable, l’art d’inventer et de faire des dieux. Nos pères, en effet, tombés dans l’incrédulité et aveuglés par de grandes erreurs qui les détournaient de la religion et du culte, imaginèrent de former des dieux de leurs propres mains ; cet art une fois inventé, ils y joignirent une vertu mystérieuse empruntée à la nature universelle, et, dans l’impuissance où ils étaient de faire des âmes, ils évoquèrent celles des démons ou des anges, en les attachant à ces images sacrées et aux divins mystères, ils donnèrent leurs idoles le pouvoir de faire du bien ou du mal. » Je ne sais en vérité si les démons évoqués en personne voudraient faire des aveux aussi complets ; Hermès, en effet, dit en propres termes : « Nos pères, tombés dans l’incrédulité et aveuglés par de grandes erreurs qui les détournaient de la religion et du culte, imaginèrent de former des dieux de leurs propres mains. » Or, ne pourrait-il pas se contenter de dire : Nos pères ignoraient la vérité ? Mais non ; il prononce le mot d’{\itshape erreur}, et il dit même {\itshape de grandes erreurs}. Telle est donc l’origine de ce grand art de faire des dieux : c’est l’erreur, c’est l’incrédulité, c’est l’oubli de la religion et du culte. Et cependant notre sage égyptien déplore la ruine future de cet art, comme s’il s’agissait d’une religion divine. N’est-il pas évident, je le demande, qu’en confessant de la sorte l’erreur de ses pères, il cède à une force divine, comme en déplorant la défaite future des démons, il cède à une force diabolique ? Car enfin, si c’est par l’erreur, par l’incrédulité, par l’oubli de la religion et du culte qu’a été trouvé l’art de faire des dieux, il ne faut plus s’étonner que toutes les œuvres de cet art détestable, conçues en haine de la religion divine, soient détruites par cette religion, puisqu’il, appartient à la vérité de redresser l’erreur, à la foi de vaincre l’incrédulité, à l’amour qui ramène à Dieu de triompher de la haine qui en détourne.\par
Supposons que Trismégiste, en nous apprenant que ses pères-avaient inventé l’art de faire des dieux, n’eût rien dit des causes de cette invention, c’eût été à nous de comprendre, pour peu que nous fussions éclairés par la piété, que jamais l’homme n’eût imaginé rien de semblable s’il ne se fût détourné du vrai, s’il eût gardé à Dieu une foi digne de lui, s’il fût resté attaché au culte légitime et à la bonne religion. Et toutefois, si nous eussions, nous, attribué l’origine de l’idolâtrie à l’erreur, à l’incrédulité l’oubli de la vraie religion l’impudence des adversaires du christianisme serait jusqu’à un certain point supportable ; mais quand celui qui admire avec transport dans l’homme cette puissance de faire des dieux, et prévoit avec douleur le temps où les lois humaines elles-mêmes aboliront ces fausses divinités instituées par les hommes, quand ce même personnage vient confesser ouvertement les causes de cette idolâtrie savoir : l’erreur, l’incrédulité et l’oubli de la religion véritable, que devons-nous dire, ou plutôt que devons-nous faire, sinon rendre des actions de grâces immortelles au Seigneur notre Dieu, pour avoir renversé ce culte sacrilège par des causes toutes contraires à celles qui le firent établir ? Car, ce qui avait été établi par l’erreur a été renversé par la vérité ; ce-qui avait été établi par l’incrédulité a été renversé par la roi ; ce qui avait été établi par la haine du culte véritable a été rétabli par l’amour du seul vrai Dieu. Ce merveilleux changement ne s’est pas opéré seulement en Égypte, unique objet des lamentations que l’esprit des dénions inspire à Trismégiste ; il s’est étendu à toute la terre, qui chante au Seigneur un nouveau cantique, selon cette prédiction des Écritures vraiment saintes et vraiment prophétiques : « Chantez au Seigneur un cantique nouveau, chantez au Seigneur, peuples de toute la terre. » Aussi le titre de ce psaume porte-t-il : « Quand la maison s’édifiait après la captivité ». En effet la maison du Seigneur, cette Cité de Dieu qui est la sainte Église, s’édifie par toute la terre, après la captivité où les démons retenaient les vrais croyants, devenus maintenant les pierres vivantes de l’édifice. Car, bien que l’homme fût l’auteur de ses dieux, cela n’empêchait pas qu’il ne leur fût soumis par le culte qu’il leur rendait et qui le faisait entrer dans leur société, je parle de la société des démons, et non de celle de ces idoles sans vie. Que sont en-effet les idoles, sinon des êtres qui ont eu des yeux et ne voient pas », suivant la parole de l’Écriture, et qui, pour être des chefs-d’œuvre de l’art, n’en restent pas moins dépourvus de sentiment et de vie ? Mais les esprits immondes, liés à ces idoles par un art détestable, avaient misérablement asservi les âmes de leurs adorateurs en se les associant. C’est pourquoi l’Apôtre dit : « Nous savons qu’une idole n’est rien et c’est aux démons, et non à Dieu, que les Gentils offrent leurs victimes. Or, je ne veux pas que vous ayez aucune société avec les démons. » C’est donc après cette captivité qui asservissait les hommes aux démons, que la maison de Dieu s’édifie par toute la terre, et de là le titre du psaume où il est dit : « Chantez au Seigneur un cantique nouveau ; chantez au Seigneur, peuples de toute la terre ; chantez au Seigneur et bénissez son saint nom ; annoncez dans toute la suite des jours son assistance salutaire ; annoncez sa gloire parmi les nations et ses merveilles au milieu de tous les peuples ; car le Seigneur est grand et infiniment louable ; il est plus redoutable que tous les dieux, car tous les dieux des Gentils sont des démons, mais le Seigneur a fait les cieux. »\par
Ainsi, celui qui s’affligeait de prévoir un temps où le culte des idoles serait aboli, et où les démons cesseraient de dominer sur leurs adorateurs, souhaitait, sous l’inspiration de l’esprit du mal, que cette captivité durât toujours, au lieu que le Psalmiste célèbre le moment où elle finira et où une maison sera édifiée par toute la terre. Trismégiste prédisait donc en gémissant ce que le Prophète prédit avec allégresse ; et comme le Saint-Esprit qui anime les saints Prophètes est toujours victorieux, Trismégiste lui-même a été miraculeusement contraint d’avouer que les institutions dont la ruine lui causait tant de douleur, n’avaient pas été établies par des hommes sages, fidèles et religieux, mais par des ignorants, des incrédules et des impies. Il a beau appeler les idoles des dieux ; du moment qu’il avoue qu’elles sont l’ouvrage d’hommes auxquels nous ne devons pas nous rendre semblables, par là même il-confesse, malgré qu’il en ait, qu’elles ne doivent point être adorées par ceux qui ne ressemblent pas à ces hommes, c’est-à-dire qui sont sages, croyants et religieux. Il confesse, en outre, que ceux mêmes qui ont inventé l’idolâtrie ont consenti à reconnaître pour dieux des êtres qui ne sont point dieux, suivant cette parole du Prophète : « Si l’homme se fait des dieux, ce ne sont point des dieux véritables. » Lors donc que Trismégiste appelle dieux de tels êtres, reconnus par de tels adorateurs et formés par de tels ouvriers, lorsqu’il prétend que des démons, qu’un art ténébreux a attachés à de certains simulacres par le lien de leurs passions, sont des dieux de fabrique humaine, il ne va pas du moins jusqu’à cette opinion absurde du platonicien Apulée, que les démons sont des médiateurs entre les dieux que Dieu a faits, et les hommes qui sont également son ouvrage, et qu’ils transmettent aux dieux les prières des hommes, ainsi qu’aux hommes les faveurs des dieux. Car il serait par trop absurde que les dieux créés par l’homme eussent auprès des dieux que Dieu a faits, plus de pouvoir que n’en a l’homme, qui a aussi Dieu pour auteur. En effet, le démon qu’un homme a lié à une statue par un art impie, est devenu un dieu, mais pour cet homme seulement, et non pour tous les hommes. Quel est donc ce dieu qu’un homme ne saurait faire sans être aveugle, incrédule et impie ?\par
Enfin, si les démons qu’on adore dans les temples et qui sont liés par je ne sais quel art à leurs images visibles, ne sont point des médiateurs et des interprètes entre les dieux et les hommes, soit à cause de leurs mœurs détestables, soit parce que les hommes, même en cet état d’ignorance, d’incrédulité et d’impiété où ils ont imaginé de faire des dieux, sont d’une nature supérieure à ces démons enchaînas par leur art au corps des idoles, il s’ensuit finalement que ces prétendus dieux n’ont de pouvoir qu’à titre de démons, et que dès lors ils nuisent ouvertement aux hommes, ou que, s’ils semblent leur faire du bien, c’est pour leur nuire encore plus en les trompant. Remarquons toutefois qu’ils n’ont ce double pouvoir qu’autant que Dieu le permet par un conseil secret et profond de la Providence, et non pas en qualité de médiateurs et d’amis des dieux. Ils ne sauraient, en effet, être amis de ces dieux excellents que nous appelons Anges, Trônes, Dominations, Principautés, Puissances, toutes créatures raisonnables qui habitent le ciel, et dont ils sont aussi éloignés par la disposition de leur âme, que le vice l’est de la vertu et la malice de la bonté.
\subsection[{Chapitre XXV}]{Chapitre XXV}

\begin{argument}\noindent De ce qu’il peut y avoir de commun entre les saints anges et les hommes.
\end{argument}

\noindent Ce n’est donc point par la médiation des démons que nous devons aspirer à la bienveillance et aux bienfaits des dieux, ou plutôt des bons anges, mais par l’imitation de leur bonne volonté ; de la sorte, en effet, nous sommes avec eux, nous vivons avec eux et nous adorons avec eux le Dieu qu’ils adorent, bien que nous ne puissions le voir avec les yeux du corps. Aussi bien, la distance des lieux n’est pas tant ce qui nous sépare des anges, que l’égarement de notre volonté et la défaillance de notre misérable nature. Et si nous ne sommes point unis avec eux, la raison n’en est pas dans notre condition charnelle et terrestre, mais dans l’impureté de notre cœur, qui nous attache à la terre et à la chair. Mais, quand arrive pour nous la guérison, quand nous devenons semblables aux anges, alors la foi nous rapproche d’eux, pourvu que nous ne doutions pas que par leur assistance Celui qui les a rendus bienheureux fera aussi notre bonheur.
\subsection[{Chapitre XXVI}]{Chapitre XXVI}

\begin{argument}\noindent Toute la religion des païens se réduisait à adorer des hommes morts.
\end{argument}

\noindent Quand il déplore la ruine future de ce culte, qui pourtant, de son propre aveu, ne doit son existence qu’à des hommes pleins d’erreurs, d’incrédulité et d’irréligion, notre égyptien écrit ces mots dignes de remarque : « Alors cette terre, sanctifiée par les temples et les autels, sera remplie de sépulcres et de morts. » Comme si les hommes ne devaient pas toujours être sujets à mourir, alors même que l’idolâtrie n’eût pas succombé ! comme si on pouvait donner aux morts une autre place que la terre ! comme si le progrès du temps et des siècles, en multipliant le nombre des morts, ne devait pas accroître celui des tombeaux ! Mais le véritable sujet de sa douleur, c’est qu’il prévoyait sans doute que les monuments de nos martyrs devaient succéder à leurs temples et à leurs autels ; et peut-être, en lisant ceci, nos adversaires vont-ils se persuader, dans leur aversion pour les chrétiens et dans leur perversité, que nous adorons les morts dans les tombeaux comme les païens adoraient leurs dieux dans les temples. Car tel est l’aveuglement de ces impies, qu’ils se heurtent, pour ainsi dire, contre des mensonges, et ne veulent pas voir des choses qui leur crèvent les yeux. Ils ne considèrent pas que, de tous les dieux dont il est parlé dans les livres des païens, à peine s’en trouve-t-il qui n’aient été des hommes, ce qui ne les empêche pas de leur rendre les honneurs divins. Je ne veux pas m’appuyer ici du témoignage de Varron, qui assure que tous les morts étaient regardés comme des dieux mânes, et qui en donne pour preuve les sacrifices qu’on leur offrait, notamment les jeux funèbres, marque évidente, suivant lui, de leur caractère divin, puisque la coutume réservait cet honneur aux dieux ; mais pour citer Hermès lui-même, qui nous occupe présentement, dans le même livre où il déplore l’avenir en ces termes : « Cette terre, sanctifiée par les temples et les autels, sera remplie de sépulcres et de morts, il avoue que les dieux des Égyptiens n’étaient que des hommes morts. Il vient, en effet, de rappeler que ses ancêtres, aveuglés par l’erreur, l’incrédulité et l’oubli de la religion divine, trouvèrent le secret de faire des dieux, et, cet art une fois inventé, y joignirent une vertu mystérieuse empruntée à la nature universelle ; après quoi, dans l’impuissance où ils étaient de faire des âmes, ils évoquèrent celles des démons et des anges, et, les attachant à ces images sacrées et aux divins mystères, donnèrent ainsi à leurs idoles le pouvoir de faire du bien et du mal » ; puis, il poursuit, comme pour confirmer cette assertion par des exemples, et s’exprime ainsi : « Votre aïeul, Esculape, a été l’inventeur de la médecine, et on lui a consacré sur la montagne de Libye, près du rivage des Crocodiles, un temple où repose son humanité terrestre, c’est-à-dire son corps ; car ce qui reste de lui, ou plutôt l’homme tout entier, si l’homme est tout entier dans le sentiment de la vie, est remonté meilleur au ciel ; et maintenant il rend aux malades, par sa puissance divine, les mêmes services qu’il leur rendait autrefois par la science médicale. » Peut-on avouer plus clairement que l’on adorait comme un dieu un homme mort, au lieu même où était son tombeau ? Et, quant au retour d’Esculape au ciel, Trismégiste, en l’affirmant, trompe les autres et se trompe lui-même. « Mon aïeul Hermès », ajoute-t-il, « ne fait-il pas sa demeure dans une ville qui porte son nom, où il assiste et protège tous les hommes qui s’y rendent de toutes parts ? » On rapporte, en effet, que le grand Hermès, c’est-à-dire Mercure, que Trismégiste appelle son aïeul, a son tombeau dans Hermopolis. Voilà donc des dieux qui, de son propre aveu, ont été des hommes, Esculape et Mercure. Pour Esculape, les Grecs et les Latins en conviennent ; mais à l’égard de Mercure, plusieurs refusent d’y voir un mortel, ce qui n’empêche pas Trismégiste de l’appeler son aïeul. À ce compte le Mercure de Trismégiste ne serait pas le Mercure des Grecs, bien que portant le même nom. Pour moi, qu’il y en ait deux ou un seul, peu m’importe. Il me suffit d’un Esculape qui d’homme soit devenu dieu, suivant Trismégiste, son petit-fils, dont l’autorité est si grande parmi les païens.\par
Il poursuit, et nous apprend encore « qu’Isis, femme d’Osiris, fait autant de bien quand elle est propice, que de mal quand elle est irritée ». Puis il veut montrer que tous les dieux de fabrique humaine sont de la même nature qu’Isis, ce qui nous fait voir que les démons se faisaient passer pour des âmes de morts attachées aux statues des temples par cet art mystérieux dont Hermès nous a raconté l’origine. C’est dans ce sens qu’après avoir parlé du mal que fait Isis quand elle est irritée, il ajoute : « Les dieux de la terre et du monde sont sujets à s’irriter, ayant reçu des hommes qui les ont formés l’une et l’autre nature » ; ce qui signifie que ces dieux ont une âme et un corps : l’âme, c’est le démon ; le corps, c’est la statue. « Voilà pourquoi, dit-il, les Égyptiens les appellent de saints animaux ; voilà aussi pourquoi chaque ville honore l’âme de celui qui l’a sanctifiée de son vivant, obéit à ses lois, et porte son nom. » Que dire maintenant de ces plaintes lamentables de Trismégiste, s’écriant que la terre, sanctifiée par les temples et les autels, va se remplir de sépulcres et de morts ? Évidemment, l’esprit séducteur qui inspirait Hermès se sentait contraint d’avouer par sa bouche que déjà la terre d’Égypte était pleine en effet de sépulcres et de morts, puisque ces morts y étaient adorés comme des dieux. Et de là cette douleur des démons, qui prévoient les supplices qui les attendent sur les tombeaux des martyrs ; car c’est dans ces lieux vénérables qu’on les a vus plusieurs fois souffrir des tortures, confesser leur nom et sortir des corps des possédés.
\subsection[{Chapitre XXVII}]{Chapitre XXVII}

\begin{argument}\noindent De l’espèce d’honneurs que les chrétiens rendent aux martyrs.
\end{argument}

\noindent Et toutefois, nous n’avons en l’honneur des martyrs, ni temples, ni prêtres, ni cérémonies, parce qu’ils ne sont pas des dieux pour nous, et que leur Dieu est notre seul Dieu. Nous honorons, il est vrai, leurs tombeaux comme ceux de bons serviteurs de Dieu, qui ont combattu jusqu’à la mort pour le triomphe de la vérité et de la religion, pour la chute de l’erreur et du mensonge ; courage admirable que n’ont pas eu les sages qui avant eux avaient soupçonné la vérité ! Mais, qui d’entre les fidèles a jamais entendu un prêtre devant l’autel consacré à Dieu, sur les saintes reliques d’un martyr, dire dans les prières Pierre, Paul ou Cyprien, je vous offre ce sacrifice ? C’est à Dieu seul qu’est offert le sacrifice célébré en leur mémoire ; à Dieu, qui les a faits hommes et martyrs, et qui a daigné les associer à la gloire de ses saints anges. On ne veut donc par ces solennités que rendre grâce au vrai Dieu des victoires des martyrs, et exciter les fidèles à partager un jour, avec l’assistance du Seigneur, leurs palmes et leurs couronnes. Voilà le véritable objet de tous ces actes de piété qui se pratiquent aux tombeaux des saints martyrs : ce sont des honneurs rendus à des mémoires vénérables, et non des sacrifices offerts à des morts comme à des dieux. Ceux mêmes qui y portent des mets, coutume qui n’est d’ailleurs reçue qu’en fort peu d’endroits, et que les meilleurs chrétiens n’observent pas, les emportent après quelques prières, soit pour s’en nourrir, soit pour les distribuer aux pauvres, et les tiennent seulement pour sanctifiés par les mérites des martyrs, au nom du Seigneur des martyrs. Mais, pour voir là des sacrifices, il faudrait ne pas connaître l’unique sacrifice des chrétiens, celui-là même qui s’offre en effet sur ces tombeaux.\par
Ce n’est donc ni par des honneurs divins, ni par des crimes humains que nous rendons hommage à nos martyrs, comme font les païens à leurs dieux ; nous ne leur offrons pas des sacrifices, et nous ne travestissons pas leurs crimes en choses sacrées. Parlerai-je d’Isis, femme d’Osiris, déesse égyptienne, etde ses ancêtres qui sont tous inscrits au nombre des rois ? Un jour qu’elle leur offrait un sacrifice, elle trouva, dit-on, une moisson d’orge dont elle montra quelques épis au roi Osiris, son mari, et à Mercure, conseiller de ce prince ; et c’est pourquoi on a prétendu l’identifier avec Cérès. Si l’on veut savoir tout le mal qu’elle a fait, qu’on lise, non les poètes, mais les livres mystiques, ceux dont parla Alexandre à sa mère Olympias, quand il eut reçu les révélations du pontife Léon, et l’on verra à quels hommes et à quelles actions on a consacré le culte divin. À Dieu ne plaise qu’on ose comparer ces dieux, tout dieux qu’on les appelle, à nos saints martyrs, dont nous ne faisons pourtant pas des dieux ! Nous n’avons institué en leur honneur ni prêtres, ni sacrifices, parce que tout cela serait inconvenant, illicite, impie, étant offert à tout autre qu’à Dieu ; nous ne cherchons pas non plus à les divertir en leur attribuant des actions honteuses ou en leur consacrant des jeux infâmes, comme on fait à ces dieux dont on célèbre les crimes sur la scène, soit qu’ils les aient commis, en effet, quand ils étaient hommes, soit qu’on les invente à plaisir pour le divertissement de ces esprits pervers. Certes, ce n’est pas un dieu de cette espèce que Socrate aurait eu pour inspirateur, s’il avait été véritablement inspiré par un Dieu ; mais peut-être est-ce un conte imaginé après coup par des hommes qui ont voulu avoir pour complice dans l’art de faire des dieux un philosophe vertueux, fort innocent, à coup sûr, de pareilles œuvres. Pourquoi donc nous arrêter plus longtemps à démontrer qu’on ne doit point honorer les démons en vue du bonheur de la vie future ? Il suffit d’un sens médiocre pour n’avoir plus aucun doute à cet égard. Mais on dira peut-être que si tous les dieux sont bons, il y a parmi les démons les bons et les mauvais, et que c’est aux bons qu’il faut adresser un culte pour obtenir la vie éternelle et bienheureuse ; c’est ce que nous allons examiner au livre suivant.
\section[{Livre neuvième. Deux espèces de démons}]{Livre neuvième. \\
Deux espèces de démons}\renewcommand{\leftmark}{Livre neuvième. \\
Deux espèces de démons}

\subsection[{Chapitre premier}]{Chapitre premier}

\begin{argument}\noindent Du point où en est la discussion et de ce qui reste à examiner.
\end{argument}

\noindent Quelques-uns ont avancé qu’il y a de bons et de mauvais dieux : d’autres, qui se sont fait de ces êtres une meilleure idée, les ont placés à un si haut degré d’excellence et d’honneur, qu’ils n’ont pas osé croire à de mauvais dieux. Les premiers donnent aux démons le titre de dieux, et quelquefois, mais plus rarement, ils ont appelé les dieux du nom de démons. Ainsi ils avouent que Jupiter lui-même, dont ils font le roi et le premier de tous les dieux, a été appelé démon par Homère. Quant à ceux qui ne reconnaissent que des dieux bons et qui les regardent comme très supérieurs aux plus vertueux des hommes, ne pouvant nier les actions des démons, ni les regarder avec indifférence, ni les imputer à des dieux bons, ils sont forcés d’admettre une différence entre les démons et les dieux ; et lorsqu’ils trouvent la marque des affections déréglées dans les œuvres où se manifeste la puissance des esprits invisibles, ils les attribuent non pas aux dieux, mais aux démons. D’un autre côté, comme dans leur système aucun dieu n’entre en communication directe avec l’homme, il a fallu faire de ces mêmes démons les médiateurs entre les hommes et les dieux, chargés de porter les vœux et de rapporter les grâces. Telle est l’opinion des Platoniciens, que nous avons choisis pour contradicteurs, comme les plus illustres et les plus excellents entre les philosophes, quand nous avons discuté la question de savoir si le culte de plusieurs dieux est nécessaire pour obtenir la félicité de la vie future. Et c’est ainsi que nous avons été conduit à rechercher, dans le livre précédent, comment il est possible que les démons, qui se plaisent aux crimes réprouvés par les hommes sages et vertueux, à tous ces sacrilèges, à tous ces attentats que les poètes racontent, non seulement des hommes, mais aussi des dieux, enfin à ces manœuvres violentes et impies des arts magiques, soient regardés comme plus voisins et plus amis des dieux que les hommes, et capables à ce titre d’appeler les faveurs de la bonté divine sur les gens de bien. Or, c’est ce qui a été démontré absolument impossible.
\subsection[{Chapitre II}]{Chapitre II}

\begin{argument}\noindent Si parmi les démons, tous reconnus pour inférieurs aux dieux, il en est de bons dont l’assistance puisse conduire les hommes à la béatitude véritable.
\end{argument}

\noindent Le présent livre roulera donc, comme je l’ai annoncé à la fin du précédent, non pas sur la différence qui existe entre les dieux, que les Platoniciens disent être tous bons, ni sur celle qu’ils imaginent entre les dieux et les démons, ceux-là séparés des hommes, à leur avis, par un intervalle immense, ceux-ci placés entre les hommes et les dieux, mais sur la différence, s’il y en a une, qui est entre les démons. La plupart, en effet, ont coutume de dire qu’il y a de bons et de mauvais démons, et cette opinion, qu’elle soit professée par les Platoniciens ou par toute autre secte, mérite un sérieux examen ; car quelque esprit mal éclairé pourrait s’imaginer qu’il doit servir les bons démons, afin de se concilier la faveur des dieux, qu’il croit aussi tous bons, et de se réunir à eux après la mort, tandis que, enlacé dans les artifices de ces esprits malins et trompeurs, il s’éloignerait infiniment du vrai Dieu, avec qui seul, en qui seul et par qui seul l’âme de l’homme, c’est-à-dire l’âme raisonnable et intellectuelle, possède la félicité.
\subsection[{Chapitre III}]{Chapitre III}\phantomsection
\label{\_chapitre3}

\begin{argument}\noindent Des attributions des démons, suivant Apulée, qui, sans leur refuser la raison, ne leur accorde cependant aucune vertu.
\end{argument}

\noindent Quelle est donc la différence des bons et des mauvais démons ? Le platonicien Apulée, dans un traité général sur la matière, où il s’étend longuement sur leurs corps aériens, ne dit pas un mot des vertus dont ils ne manqueraient pas d’être doués, s’ils étaient bons. Il a donc gardé le silence sur ce qui peut les rendre heureux, mais il n’a pu taire ce qui prouve qu’ils sont misérables ; car il avoue que leur esprit, qui en fait des êtres raisonnables, non seulement n’est pas armé par la vertu contre les passions contraires à la raison, mais qu’il est agité en quelque façon par des émotions orageuses, comme il arrive aux âmes insensées. Voici à ce sujet ses propres paroles « C’est cette espèce de démons dont parlent les poètes, quand ils nous disent, sans trop s’éloigner de la vérité, que les dieux ont de l’amitié ou de la haine pour certains hommes, favorisant et élevant ceux-ci, abaissant et persécutant ceux-là. Aussi, compassion, colère, douleur, joie, toutes les passions de l’âme humaine, ces dieux les éprouvent, et leur cœur est agité comme celui des hommes par ces tempêtes et ces orages qui n’approchent jamais de la sérénité des dieux du ciel. » N’est-il pas clair, par ce tableau de l’âme des démons, agitée comme une mer orageuse, qu’il ne s’agit point de quelque partie inférieure de leur nature, mais de leur esprit même, qui en fait des êtres raisonnables ? À ce compte ils ne souffrent pas la comparaison avec les hommes sages qui, sans rester étrangers à ces troubles de l’âme, partage inévitable de notre faible condition, savent du moins y résister avec une force inébranlable, et ne rien approuver, ne rien faire qui s’écarte des lois de la sagesse et des sentiers de la justice. Les démons ressemblent bien plutôt, sinon par le corps, au moins par les mœurs, aux hommes insensés et injustes, et ils sont même plus méprisables, parce que, ayant vieilli dans le mal et devenus incorrigibles par le châtiment, leur esprit est, suivant l’image d’Apulée, une mer battue par la tempête, incapables qu’ils sont de s’appuyer, par aucune partie deleur âme, sur la vérité et sur la vertu, qui donnent la force de résister aux passions turbulentes et déréglées.
\subsection[{Chapitre IV}]{Chapitre IV}

\begin{argument}\noindent Sentiments des Péripatéticiens et des Stoïciens touchant les passions.
\end{argument}

\noindent Il y a deux opinions parmi les philosophes touchant ces mouvements de l’âme que les Grecs nomment {\itshape pathè}, et qui s’appellent, dans notre langue, chez Cicéron, par exemple, {\itshape perturbations}, ou chez d’autres écrivains, {\itshape affections}, ou encore, pour mieux rendre l’expression grecque, {\itshape passions}. Les uns disent qu’elles se rencontrent même dans l’âme du sage, mais modérées et soumises à la raison, qui leur impose des lois et les contient dans de justes bornes. Tel est le sentiment des Platoniciens ou des Aristotéliciens ; car Aristote, fondateur du péripatétisme, est un disciple de Platon. Les autres, comme les Stoïciens, soutiennent que l’âme du sage reste étrangère aux passions. Mais Cicéron, dans son traité {\itshape Des biens et des maux}, démontre que le combat des Stoïciens contre les Platoniciens et les Péripatéticiens se réduit à une querelle de mots. Les Stoïciens, en effet, refusent le nom de biens aux avantages corporels et extérieurs, parce qu’à leur avis le bien de l’homme est tout entier dans la vertu, qui est l’art de bien vivre et ne réside que dans l’âme. Or, les autres philosophes, en appelant biens les avantages corporels pour parler simplement et se conformer à l’usage, déclarent que ces biens n’ont qu’une valeur fort minime et ne sont pas considérables en comparaison de la vertu. D’où il suit que des deux côtés ces objets sont estimés au même prix, soit qu’on leur donne, soit qu’on leur refuse le nom de biens ; de sorte que la nouveauté du stoïcisme se réduit au plaisir de changer les mots. Pour moi, il me semble que, dans la controverse sur les passions du sage, c’est encore des mots qu’il s’agit plutôt que des choses, et que les Stoïciens ne diffèrent pas au fond des disciples de Platon et d’Aristote.\par
Entre autres preuves que je pourrais alléguer à l’appui de mon sentiment, je n’en apporteraiqu’une que je crois péremptoire. Aulu-Gelle, écrivain non moins recommandable par l’élégance de son style que par l’étendue et l’abondance de son érudition, rapporte dans ses {\itshape Nuits attiques} que, dans un voyage qu’il faisait sur mer avec un célèbre stoïcien, ils furent assaillis par une furieuse tempête qui menaçait d’engloutir leur vaisseau ; le philosophe en pâlit d’effroi. Ce mouvement fut remarqué des autres passagers qui, bien qu’aux portes de la mort, le considéraient attentivement pour voir si un philosophe aurait peur comme les autres. Aussitôt que la tempête fut passée et que l’on se fut un peu rassuré, un riche et voluptueux asiatique de la compagnie se mit à railler le stoïcien de ce qu’il avait changé de couleur, tandis qu’il était resté, lui, parfaitement impassible. Mais le philosophe lui répliqua ce que Aristippe, disciple de Socrate, avait dit à un autre en pareille rencontre : « Vous avez eu raison de ne pas vous inquiéter pour l’âme d’un vil débauché, mais moi je devais craindre pour l’âme d’Aristippe. » Cette réponse ayant dégoûté le riche voluptueux de revenir à la charge, Aulu-Gelle demanda au philosophe, non pour le railler, mais pour s’instruire, quelle avait été la cause de sa peur. Celui-ci, s’empressant de satisfaire un homme si jaloux d’acquérir des connaissances, tira de sa cassette un livre d’Épictète, où était exposée la doctrine de ce philosophe, en tout conforme aux principes de Zénon et de Chrysippe, chefs de l’école stoïcienne. Aulu-Gelle dit avoir lu dans ce livre que les Stoïciens admettent certaines perceptions de l’âme, qu’ils nomment fantaisies, et qui se produisent en nous indépendamment de la volonté. Quand ces images sensibles viennent d’objets terribles et formidables, il est impossible que l’âme du sage n’en soit pas remuée : elle ressent donc quelque impression de crainte quelque émotion de tristesse, ces passions prévenant en elle l’usage de la raison ; maiselle ne les approuve pas, elle n’y cède pas, elle ne convient pas qu’elle soit menacée d’un mal véritable. Tout cela, en effet, dépend de la volonté, et il y a cette différence entre l’âme du sage et celle des autres hommes, que celle-ci cède aux passions et y conforme le jugement de son esprit, tandis que l’âme du sage, tout en subissant les passions, garde en son esprit inébranlable un jugement stable et vrai, touchant les objets qu’il est raisonnable de fuir ou de rechercher. J’ai rapporté ceci de mon mieux, non sans doute avec plus d’élégance qu’Aulu-Gelle, qui dit l’avoir lu dans Épictète, mais avec plus de précision, ce me semble, et plus de clarté.\par
S’il en est ainsi, la différence entre les Stoïciens et les autres philosophes, touchant les passions, est nulle ou peu s’en faut, puisque tous s’accordent à dire qu’elles ne dominent pas sur l’esprit et la raison du sage ; et quand les Stoïciens soutiennent que le sage n’est point sujet aux passions, ils veulent dire seulement que sa sagesse n’en reçoit aucune atteinte, aucune souillure. Or, si elles se rencontrent en effet dans son âme, quoique sans dommage pour sa sagesse et sa sérénité, c’est à la suite de ces avantages et de ces inconvénients qu’ils se refusent à nommer des biens et des maux. Car enfin, si ce philosophe dont parle Aulu-Gelle n’avait tenu aucun compte de sa vie et des autres choses qu’il était menacé de perdre en faisant naufrage, le danger qu’il courait ne l’aurait point fait pâlir. Il pouvait en effet subir l’impression de la tempête et maintenir son esprit ferme dans cette pensée que la vie et le salut du corps, menacés par le naufrage, ne sont pas de ces biens dont la possession rend l’homme bon, comme fait celle de la justice. Quant à la distinction des noms qu’il faut leur donner, c’est une pure querelle de mots. Qu’importe enfin qu’on donne ou qu’on refuse le nom de biens aux avantages corporels ? La crainte d’en être privé effraie et fait pâlir le stoïcien tout autant que le péripatéticien ; s’ils ne les appellent pas du même nom, ils les estiment au même prix. Aussi bien tous deux assurent que si on leur lin posait un crime sans qu’ils pussent l’éviter autrement que par la perte de tels objets, ils aimeraient mieux renoncer à des avantages qui ne regardent que la santé et le bien-être du corps, que de se charger d’une action qui viole la justice. C’est ainsi qu’un esprit où restent gravés les principes de la sagesse a beau sentir le trouble des passions qui agitent les parties inférieures de l’Âme, il ne les laisse pas prévaloir contre la raison ; loin d’y céder, il les domine, et, sur cette résistance victorieuse il fonde le règne de la vertu. Tel Virgile a représenté son héros, quand il a dit d’Énée :\par
 {\itshape « Son esprit reste inébranlable, tandis que ses yeux versent inutilement des pleurs. »} 
\subsection[{Chapitre V}]{Chapitre V}

\begin{argument}\noindent Les passions qui assiègent les âmes chrétiennes, loin de les porter au vice, les exercent a la vertu.
\end{argument}

\noindent Il n’est pas nécessaire présentement d’exposer avec étendue ce qu’enseigne touchant les passions, la sainte Écriture, source de la science chrétienne. Qu’il nous suffise de dire en général qu’elle soumet l’âme à Dieu pour en être gouvernée et secourue, et les passions à la raison pour en être modérées, tenues en bride et tournées à un usage avoué par la vertu. Dans notre religion on ne se demande pas si une âme pieuse se met en colère, mais pourquoi elle s’y met ; si elle est triste, mais d’où vient sa tristesse ; si elle craint, mais ce qui fait l’objet de ses craintes. Aussi bien je doute qu’une personne douée de sens puisse trouver mauvais qu’on s’irrite contre un pécheur pour le corriger, qu’on s’attriste des souffrances d’un malheureux pour les soulager, qu’on s’effraie à la vue d’un homme en péril pour l’en arracher. C’est une maxime habituelle du stoïcien, je le sais, de condamner la pitié, mais combien n’eût-il pas été plus honorable au stoïcien d’Aulu-Gelle d’être ému de pitié pour un homme à tirer du danger que d’avoir peur du naufrage ! Et que Cicéron est mieux inspiré, plus humain, plus conforme aux sentiments des âmes pieuses, quand il dit dans son éloge de César : « Parmi vos vertus, la plus admirable et la plus touchante c’est la miséricorde ! » Mais qu’est-ce que la miséricorde, sinon la sympathie qui nous associe à la misère d’autrui et nous porte à la soulager ? Or, ce mouvement de l’âme sert la raison toutes les fois qu’il est d’accord avec la justice, soit qu’il nous dispose à secourir l’indigence, soit qu’il nous rende indulgents au repentir. C’est pourquoi Cicéron, si judicieux dans son éloquent langage, donne sans hésiter le nom de vertu à un sentiment que les Stoïciens ne rougissent pas de mettre au nombre des vices. Et remarquez que ces mêmes philosophes conviennent que les passions de cette espèce trouvent place dans l’âme du sage, où aucun vice ne peut pénétrer ; c’est ce qui résulte du livre d’Épictète, éminent stoïcien, qui d’ailleurs écrivait selon les principes des chefs de l’école, Zénon et Chrysippe. Il en faut conclure qu’au fond, ces passions qui ne peuvent rien dans l’âme du sage contre la raison et la vertu, ne sont pas pour les Stoïciens de véritables vices, et dès lors que leur doctrine, celle des Péripatéticiens et celle enfin des Platoniciens se confondent entièrement. Cicéron avait donc bien raison de dire que ce n’est pas d’aujourd’hui que les disputes de mots mettent à la torture la subtilité puérile des Grecs, plus amoureux de la dispute que de la vérité. Il y aurait pourtant ici une question sérieuse à traiter, c’est de savoir si ce n’est point un effet de la faiblesse inhérente à notre condition passagère de subir ces passions, alors même que nous pratiquons le bien. Ainsi les saints anges punissent sans colère ceux que la loi éternelle de Dieu leur ordonne de punir, comme ils assistent les misérables sans éprouver la compassion, et secourent ceux qu’ils aiment dans leurs périls sans ressentir la crainte ; et cependant, le langage ordinaire leur attribue ces passions humaines à cause d’une certaine ressemblance qui se rencontre entre nos actions et les leurs, malgré l’infirmité de notre nature, C’est ainsi que Dieu lui-même s’irrite, selon l’Écriture, bien qu’aucune passion ne puisse atteindre son essence immuable. Il faut entendre par cette expression biblique l’effet de la vengeance de Dieu et non l’agitation turbulente de la passion.
\subsection[{Chapitre VI}]{Chapitre VI}

\begin{argument}\noindent Des passions qui agitent les démons, de l’aveu d’Apulée qui leur attribue le privilège d’assister les hommes auprès des dieux.
\end{argument}

\noindent Laissons de côté, pour le moment, la question des saints anges, et examinons cette opinion platonicienne que les démons, qui tiennent le milieu entre les dieux et les hommes, sont livrés au mouvement tumultueux des passions. En effet, si leur esprit, tout en les subissant, restait libre et maître de soi, Apulée ne nous le peindrait pas agité comme le nôtre par le souffle des passions et semblable à une mer orageuse. Cet esprit donc, cette partie supérieure de leur âme qui en fait des êtres raisonnables, et qui soumettrait les passions turbulentes de la région inférieure aux lois de la vertu et de la sagesse, si les démons pouvaient être sages et vertueux, c’est cet esprit même qui, de l’aveu du philosophe platonicien, est agité par l’orage des passions. J’en conclus que l’esprit des démons est sujet à la convoitise, à la crainte, à la colère et à toutes les affections semblables. Où est donc cette partie d’eux-mêmes, libre, capable de sagesse, qui les rend agréables aux dieux et utiles aux hommes de bien ? Je vois des âmes livrées tout entières au joug des passions et qui ne font servir la partie raisonnable de leur être qu’à séduire et à tromper, d’autant plus ardentes à l’œuvre qu’elles sont animées d’un plus violent désir de faire du mal.
\subsection[{Chapitre VII}]{Chapitre VII}

\begin{argument}\noindent Les Platoniciens croient les dieux outragés par les fictions des poètes, qui les représentent combattus par des affections contraires, ce qui n’appartient qu’aux démons.
\end{argument}

\noindent On dira peut-être que les poètes, en nous peignant les dieux comme amis ou ennemis de certains hommes, ont voulu parler, non de tous les démons, mais seulement des mauvais, de ceux-là mêmes qu’Apulée croit agités par l’orage des passions. Mais comment admettre cette interprétation, quand Apulée, en attribuant les passions aux démons, ne fait entre eux aucune distinction et nous les représente en général comme tenant le milieu entre les dieux et les hommes à cause de leurs corps aériens ? Suivant ce philosophe, la fiction des poètes consiste à transformer les démons en dieux, et, grâce à l’impunité de la licence poétique, à les partager à leur gré entre les hommes, coin me protecteurs ou comme ennemis, tandis que les dieux sont infiniment au-dessus de ces faiblesses des démons, et par l’élévation de leur séjour et par la plénitudede leur félicité. Cette fiction se réduit donc à donner le nom de dieux à des êtres qui ne sont pas dieux, et Apulée ajoute qu’elle n’est pas très éloignée de la vérité, attendu que, au nom près, ces êtres sont représentés selon leur véritable nature, qui est celle des démons. Telle est, à son avis, cette Minerve d’Homère qui intervient au milieu des Grecs pour empêcher Achille d’outrager Agamemnon. Que Minerve ait apparu aux Grecs, voilà la fiction poétique, selon Apulée, pour qui Minerve est une déesse qui habite loin du commerce des mortels, dans la région éthérée, eu compagnie des dieux, qui sont tous des êtres heureux et bons, Mais qu’il y ait eu un démon favorable aux Grecs et ennemi des Troyens, qu’un autre démon, auquel le même poète a donné le nom d’un des dieux qui habitent paisiblement le ciel, comme Mars et Vénus, ait favorisé au contraire les Troyens en haine des Grecs ; enfin, qu’une lutte se soit engagée entre ces divers démons, animés de sentiments opposés, voilà ce qui, pour Apulée, n’est pas un récit très éloigné de la vérité. Les poètes, en effet, n’ont attribué ces passions qu’à des êtres qui sont en effet sujets aux mêmes passions que les hommes, aux mêmes tempêtes des émotions contraires, capables, par conséquent, d’éprouver de l’amour et de la haine, non selon la justice, mais à la manière du peuple qui, dans les chasses et les courses du cirque, se partage entre les adversaires au gré de ses aveugles préférences. Le grand souci du philosophe platonicien, c’est uniquement qu’au lieu de rapporter ces fictions aux démons, on ne prenne les poètes à la lettre en les attribuant aux dieux.
\subsection[{Chapitre VIII}]{Chapitre VIII}

\begin{argument}\noindent Comment Apulée définit les dieux, habitants du ciel ; les démons, habitants de l’air ; et les hommes, habitants de la terre.
\end{argument}

\noindent Si l’on reprend la définition des démons, il suffira d’un coup d’œil pour s’assurer qu’Apulée les caractérise tous indistinctement, quand il dit qu’ils sont, quant au genre, des animaux, quant à l’âme, sujets aux passions, quant à l’esprit, raisonnables, quant aux corps, aériens, quant au temps, éternels. Ces cinq qualités n’ont rien qui rapproche les démons des hommes vertueux et les sépare des méchants. Apulée, en effet, quand il passe des dieux habitants du ciel aux hommes habitants de la terre, pour en venir plus tard aux démons qui habitent la région mitoyenne entre ces deux extrémités, Apulée s’exprime ainsi : « Les hommes, ces êtres qui jouissent de la raison et possèdent la puissance de la parole, dont l’âme est immortelle et les membres moribonds, esprits légers et inquiets, corps grossiers et corruptibles, différents par les mœurs et semblables par les illusions, d’une audace obstinée, d’une espérance tenace, les hommes dont les travaux sont vains et la fortune changeante, espèce immortelle où chaque individu périt après avoir à son tour renouvelé les générations successives, dont la durée est courte, la sagesse tardive, la mort prompte, la vie plaintive, les hommes, dis-je, ont la terre pour séjour. » Parmi tant de caractères communs à la plupart des hommes, Apulée a-t-il oublié celui qui est propre à un petit nombre, la sagesse tardive ? S’il l’eût passé sous silence, cette description, si soigneusement tracée, n’eût pas été complète. De même, quand il veut taire ressortir l’excellence des dieux, il insiste sur cette béatitude qui leur est propre et où les hommes s’efforcent de parvenir par la sagesse. Certes, s’il avait voulu nous persuader qu’il y a de bons démons, il aurait placé dans la description de ces êtres quelque trait qui les rapprochât des dieux par la béatitude, ou des hommes par la sagesse. Point du tout, il n’indique aucun attribut qui fasse distinguer les bons d’avec les méchants. Si donc il n’a pas dévoilé librement leur malice, moins par crainte de les offenser que pour ne pas choquer leurs adorateurs devant qui il parlait, il n’en a pas moins indiqué aux esprits éclairés ce qu’il faut penser à cet égard. En effet, il affirme que tous les dieux sont bons et heureux, et, les affranchissant de ces passions turbulentes qui agitent les démons, il ne laisse entre ceux-ci et les dieux d’autre point commun qu’un corps éternel. Quand, au contraire, il parle de l’âme des démons, c’est aux hommes et non pas aux dieux qu’il les assimile par cet endroit ; et encore, quel est le trait de ressemblance ? ce n’est pas la sagesse, à laquelle les hommes peuvent participer ; ce sont les passions, ces tyrans des âmes faibles et mauvaises, que les hommes sages et bons parviennent à vaincre, mais dont ils aimeraient mieux encore n’avoir pas à triompher. Si, en effet, quand il dit que l’immortalité est commune aux démons et aux dieux, il avait voulu faire entendre celle des esprits et non celle des corps, il aurait associé les hommes à ce privilège, loin de les en exclure, puisqu’en sa qualité de platonicien il croit les hommes en possession d’une âme immortelle. N’a-t-il pas dit de l’homme, dans la description citée plus haut : Son âme est immortelle et ses membres moribonds ? Par conséquent, ce qui sépare les hommes des dieux, quant à l’éternité, c’est leur corps périssable ; ce qui en rapproche les démons, c’est seulement leur corps immortel.
\subsection[{Chapitre IX}]{Chapitre IX}

\begin{argument}\noindent Si l’intercession des démons peut concilier aux hommes la bienveillance des dieux.
\end{argument}

\noindent Voilà d’étranges médiateurs entre les dieux et les hommes, et de singuliers dispensateurs des faveurs célestes ! La partie la meilleure de l’animal, l’âme, c’est ce qu’il y a de vicieux en eux, comme dans l’homme ; et ce qu’ils ont de meilleur, ce qui est immortel en eux comme chez les dieux, c’est la pire partie de l’animal, le corps. L’animal, en effet, se compose de corps et d’âme, et l’âme est meilleure que le corps ; même faible et vicieuse, elle vaut mieux que le corps le plus vigoureux et le plus sain, parce que l’excellence de sa nature se maintient jusque dans ses vices, de même que l’or, souillé de fange, reste plus précieux que l’argent ou le plomb le plus pur. Or, il arrive que ces médiateurs, chargés d’unir la terre avec le ciel, n’ont de commun avec les dieux qu’un corps éternel, et sont par l’âme aussi vicieux que les hommes ; comme si cette religion, qui rattache les hommes aux dieux par l’entremise des démons, consistait, non dans l’esprit, mais dans le corps. Quel est donc le principe de malignité du plutôt de justice qui tient ces faux et perfides médiateurs comme suspendus la tête en bas, la partie inférieure de leur être, le corps, engagé avec les natures supérieures, la partie supérieure, l’âme, avec les inférieures, unis aux dieux du ciel par la partie qui obéit, malheureux comme les habitants de la terre par la partie qui commande ? car le corps est un esclave, et, comme dit Salluste : « À l’âme appartient le commandement et au corps l’obéissance. » À quoi il ajoute : « Celle-là nous est commune avec les dieux, et celui-ci avec les brutes. »
\section[{Livre dixième. Le culte de latrie}]{Livre dixième. \\
Le culte de latrie}\renewcommand{\leftmark}{Livre dixième. \\
Le culte de latrie}

\subsection[{Chapitre premier}]{Chapitre premier}

\begin{argument}\noindent Les Platoniciens tombant d’accord que Dieu seul est la source de la béatitude véritable, pour les anges comme pour les hommes, il reste à savoir si les anges, que ces philosophes croient qu’il faut honorer en vue de cette béatitude même, veulent qu’on leur fasse des sacrifices ou qu’on n’en offre qu’à Dieu seul.
\end{argument}

\noindent C’est un point certain pour quiconque use un peu de sa raison que tous les hommes veulent être heureux ; mais qui est heureux et d’où vient le bonheur ? voilà le problème où s’exerce la faiblesse humaine et qui a soulevé parmi les philosophes tant de grandes et vives controverses. Nous n’avons pas dessein de les ranimer ; ce serait un long travail, inutile à notre but. Il nous suffit qu’on se rappelle ce que nous avons dit au huitième livre, alors que nous étions en peine de faire un choix parmi les philosophes, pour débattre avec eux la question du bonheur de la vie future et savoir s’il est nécessaire pour y parvenir d’adorer plusieurs dieux ou s’il ne faut adorer que le seul vrai Dieu, créateur des dieux eux-mêmes.\par
On peut se souvenir, ou au besoin s’assurer par une seconde lecture, que nous avons choisi les Platoniciens, les plus justement célèbres parmi les philosophes, parce qu’ayant su comprendre que l’âme humaine, toute immortelle et raisonnable qu’elle est, ne peut arriver à la béatitude que par sa participation à la lumière de celui qui l’a faite et qui a fait le monde, ils en ont conclu que nul n’atteindra l’objet des désirs de tous les hommes, savoir le bonheur, qu’à condition d’être uni par un amour chaste et pur à cet être unique, parfait et immuable qui est Dieu. Mais comme ces mêmes philosophes, entraînés par les erreurs populaires, ou, suivant le mot de l’Apôtre, {\itshape perdus dans le néant de leurs spéculations}, ont cru qu’il fallait adorer plusieurs dieux, au point même que quelques-uns d’entre eux sont tombés dans l’erreur déjà longuement réfutée du culte des démons, il faut rechercher maintenant, avec l’aide de Dieu, quel est, touchant la religion et la piété, le sentiment des anges, c’est-à-dire de ces êtres immortels et bienheureux établis dans les sièges célestes, Dominations, Principautés, Puissances, que ces philosophes appellent dieux, et quelques-uns bons démons, ou, comme nous, anges ; en termes plus précis, il faut savoir si ces esprits célestes veulent que nous leur rendions les honneurs sacrés, que nous leur offrions des sacrifices, que nous leur consacrions nos biens et nos personnes, ou que tout cela soit réservé à Dieu seul, leur dieu et le nôtre.\par
Tel est, en effet, le culte qui est dû à la divinité ou plus expressément à la déité, et pour désigner ce culte en un seul mot, faute d’expression latine suffisamment appropriée, je me servirai d’un mot grec. Partout où les saintes Écritures portent {\itshape latreia}, nous traduisons par {\itshape service} ; mais ce service qui est dû aux hommes et dont parle l’Apôtre, quand il prescrit aux serviteurs d’être soumis à leurs maîtres, est désigné en grec par un autre terme. Le mot {\itshape latrei} au contraire, selon l’usage de ceux qui ont traduit en grec le texte hébreu de la Bible, exprime toujours, ou presque toujours, le service qui est dû à Dieu. C’est pourquoi il semble que le mot {\itshape culte} ne se rapporte pas d’une manière assez exclusive à Dieu, puisqu’on s’en sert pour désigner aussi les honneurs rendus à des hommes, soit pendant leur vie, soit après leur mort. De plus, il ne se rapporte pas seulement aux êtres auxquels nous nous soumettons par une humilité religieuse, mais aussi aux choses qui nous sont soumises ; car de ce mot dérivent a{\itshape griculteurs, colons} et autres. De même, les païens n’appellent leurs dieux {\itshape coelicoles} qu’à titre de colons du ciel, ce qui ne veut pas dire qu’on les assimile à cette espèce de colons qui sont attachés au sol natal pour le cultiver sous leurs maîtres ; le mot colon est pris ici au sens où l’a employé un des maîtres de la langue latine dans ce vers :\par
 {\itshape « Il était une antique cité habitée par des colons tyriens. »} \par
C’est dans le même sens qu’on appelle {\itshape colonies} les États fondés par ces essaims de peuples qui sortent d’un État plus grand. En somme, il est très vrai que le mot {\itshape culte}, pris dans un sens propre et précis, ne se rapporte qu’à Dieu seul ; mais comme on lui donne encore d’autres acceptions, il s’ensuit que le culte exclusivement dû à Dieu ne peut en notre langue s’exprimer d’un seul mot.\par
Le mot de {\itshape religion} semblerait désigner plus distinctement, non toute sorte de culte, mais le culte de Dieu, et c’est pour cela qu’on s’en est servi pour rendre le mot grec {\itshape treskeia}. Toutefois, comme l’usage de notre langue fait dire aux savants aussi bien qu’aux ignorants, qu’il faut garder la religion de la famille, la religion des affections et des relations sociales, il est clair qu’en appliquant ce mot au culte de la déité, on n’évite pas l’équivoque ; et dire que la religion n’est autre chose que le culte de Dieu, ce serait retrancher par une innovation téméraire l’acception reçue, qui comprend dans la religion le respect des liens du sang et de la société humaine. Il en est de même du mot piété, en grec {\itshape eusebeia}. Il désigne proprement le culte de Dieu ; et cependant on dit aussi la piété envers les parents, et le peuple s’en sert même pour marquer les œuvres de miséricorde, usage qui me paraît venir de ce que Dieu recommande particulièrement ces œuvres et les égale ou même les préfère aux sacrifices. De là vient qu’on donne à Dieu même le titre de pieux. Toutefois les Grecs ne se servent pas du mot {\itshape eusebein} dans ce sens, et c’est pourquoi, en certains passages de l’Écriture, afin de marquer plus fortement la distinction, ils ont préféré au mot {\itshape eusebeia}, qui désigne le culte en général, le mot {\itshape theosebeia} qui exprime exclusivement le culte de Dieu. Quant à nous, il nous est impossible de rendre par un seul mot l’une ou l’autre de ces deux idées. Nous disons donc que ce culte, que les Grecs appellent {\itshape latreia} et nous {\itshape service}, mais service exclusivement voué à Dieu, ce culte que les Grecs appellent aussi {\itshape treskeia}, et nous {\itshape religion}, mais religion qui nous attache à Dieu seul, ce culte enfin que les Grecs appellent d’un seul mot, {\itshape theosebeia}, et nous en trois mots, {\itshape culte de Dieu}, ce culte n’appartient qu’à Dieu seul, au vrai Dieu qui transforme en dieux ses serviteurs. Cela posé, il suit, de deux choses l’une : que si les esprits bienheureux et immortels qui habitent les demeures célestes ne nous aiment pas et ne veulent pas notre bonheur, nous ne devons pas les honorer, et si, au contraire, ils nous aiment et veulent notre bonheur, ils ne peuvent nous vouloir heureux que comme ils le sont eux-mêmes ; car comment notre béatitude aurait-elle une autre source que la leur ?
\subsection[{Chapitre II}]{Chapitre II}

\begin{argument}\noindent Sentiment de Plotin sur l’illumination d’en haut.
\end{argument}

\noindent Mais nous n’avons sur ce point aucun sujet de contestation avec les illustres philosophes de l’école platonicienne. Ils ont vu, ils ont écrit de mille manières dans leurs ouvrages, que le principe de notre félicité est aussi celui de la félicité des esprits célestes, savoir cette lumière intelligible, qui est Dieu pour ces esprits, qui est autre chose qu’eux, qui les illumine, les fait briller de ses rayons, et, par cette communication d’elle-même, les rend heureux et parfaits. Plotin, commentant Platon, dit nettement et à plusieurs reprises, que cette âme même dont ces philosophes font l’âme du monde, n’a pas un autre principe de félicité que la nôtre, et ce principe est une lumière supérieure à l’âme, par qui elle a été créée, qui l’illumine et la fait briller de la splendeur de l’intelligible. Pour faire comprendre ces choses de l’ordre spirituel, il emprunte une comparaison aux corps célestes. Dieu est le soleil, et l’âme, la lune ; car c’est du soleil, suivant eux, que la lune tire sa clarté. Ce grand platonicien pense donc que l’âme raisonnable, ou plutôt l’âme intellectuelle (car sous ce nom il comprend aussi les âmes des bienheureux immortels dont il n’hésite pas à reconnaître l’existence et qu’il place dans le ciel), cette âme, dis-je, n’a au-dessus de soi que Dieu, créateur du monde etde l’âme elle-même, qui est pour elle comme pour nous le principe de la béatitude et de lavérité. Or, cette doctrine est parfaitement d’accord avec l’Évangile, où il est dit : « Il y eut un homme envoyé de Dieu qui s’appelait Jean. Il vint comme témoin pour rendre témoignage à la lumière, afin que tous crussent par lui. Il n’était pas la lumière, mais il vint pour rendre témoignage à celui qui était la lumière. Celui-là était la vraie lumière qui illumine tout homme venant en ce monde. » Cette distinction montre assez que l’âme raisonnable et intellectuelle, telle qu’elle était dans saint Jean, ne peut pas être à soi-même sa lumière, et qu’elle ne brille qu’en participant à la lumière véritable. C’est ce que reconnaît le même saint Jean, quand il ajoute, rendant témoignage à la lumière : « Nous avons tous reçu de sa plénitude. »
\subsection[{Chapitre III}]{Chapitre III}

\begin{argument}\noindent Bien qu’ils aient connu le Créateur de l’univers, les Platoniciens se sont écartés du vrai culte de Dieu en rendant les honneurs divins aux bons et aux mauvais anges.
\end{argument}

\noindent Cela étant, si les Platoniciens et les autres philosophes qui acceptent ces mêmes principes, connaissant Dieu, le glorifiaient comme Dieu et lui rendaient grâces, s’ils ne se perdaient pas dans leurs vaines pensées, s’ils n’étaient point complices des erreurs populaires, soit qu’ils en aient eux-mêmes semé le germe, soit qu’ils n’osent en surmonter l’entraînement, ils confesseraient assurément que ni les esprits immuables et bienheureux, ni les hommes mortels et misérables ne peuvent être ou devenir heureux qu’en servant cet unique Dieu des dieux, qui est le nôtre et le leur.\par
C’est à lui que nous devons, pour parler comme les Grecs, rendre le culte de {\itshape latrie}, soit dans les actes extérieurs, soit au dedans de nous ; car nous sommes son temple, tous ensemble comme chacun en particulier et il daigne également prendre pour demeure et chaque fidèle et le corps de l’Église, sans être plus grand dans le tout que dans chaque partie, parce que sa nature est incapable de toute extension et de toute division. Quand notre cœur est élevé vers lui, il est son autel ; son Fils unique est le prêtre par qui nous le fléchissons ; nous lui immolons des victimes sanglantes, quand nous versons notre sang pour la vérité et pour lui ; l’amour qui nous embrase en sa présence d’une flamme sainte et pieuse lui est le plus agréable encens ; nous lui offrons les dons qu’il nous a faits, et nous nous offrons, nous nous rendons nous-mêmes à notre créateur ; nous rappelons le souvenir de ses bienfaits, par des fêtes solennelles, de peur que le temps n’amène l’ingratitude avec l’oubli ; enfin nous lui vouons sur l’autel de notre cœur, où rayonne le feu de la charité, une hostie d’humilité et de louange. C’est pour le voir, autant qu’il peut être vu, c’est pour être unis à lui que nous nous purifions de la souillure des péchés et des passions mauvaises, et que nous cherchons une consécration dans la vertu de son nom ; car il est la source de notre béatitude et la fin de tous nos désirs. Nous attachant donc à lui, ou plutôt nous y rattachant, au lieu de nous en détacher pour notre malheur, le méditant et le relisant sans cesse (d’où vient, dit-on, le mot religion), nous tendons vers lui par l’amour, afin de trouver en lui le repos et de posséder la béatitude en possédant la perfection. Ce souverain bien, en effet, dont la recherche a tant divisé les philosophes, n’est autre chose que l’union avec Dieu ; c’est en le saisissant, si on peut ainsi dire, par un embrassement spirituel, que l’âme devient féconde en véritables vertus. Aussi nous est-il ordonné d’aimer ce bien de tout notre cœur, de toute notre âme et de toute notre vertu. Vers lui doivent nous conduire ceux qui nous aiment ; vers lui nous devons conduire ceux que nous aimons. Et par là s’accomplissent ces deux commandements qui renferment la loi et les Prophètes : « Tu aimeras le Seigneur ton Dieu de tout ton cœur et de tout ton esprit » — « Tu aimeras ton prochain comme toi-même ». Pour apprendre à l’homme à s’aimer lui-même comme il convient, une fin lui a été proposée à laquelle il doit rapporter toutes ses actions pour être heureux ; car on ne s’aime que pour être heureux, et cette fin, c’est d’être uni à Dieu. Lors donc que l’on commande à celui qui sait déjà s’aimer comme il faut, d’aimer son prochain comme soi-même, que lui commande-t-on, sinon de se porter, autant qu’il est en son pouvoir, à aimer Dieu ? Voilà le vrai culte de Dieu, voilà la vraie religion, voilà la solide piété, voilà le service qui n’est dû qu’à Dieu. Quelque hautes, par conséquent, que soient l’excellence et les vertus des puissances angéliques, si elles nous aiment comme elles-mêmes, elles doivent souhaiter que nous soyons soumis, pour être heureux, à celui qui doit aussi avoir leur soumission pour faire leur bonheur, Si elles ne servent pas Dieu, elles sont malheureuses, étant privées de Dieu ; si elles servent Dieu, elles ne veulent pas qu’on les serve à la place de Dieu, et leur amour pour lui les fait au contraire acquiescer à cette sentence divine : « Celui qui sacrifiera à d’autres dieux qu’au Seigneur sera exterminé. »
\subsection[{Chapitre IV}]{Chapitre IV}

\begin{argument}\noindent Le sacrifice n’est dû qu’à Dieu seul.
\end{argument}

\noindent Sans parler en ce moment des autres devoirs religieux, il n’y a personne au monde qui osât dire que le sacrifice soit dû à un autre qu’à Dieu. Il est vrai qu’on a déféré à des hommes beaucoup d’honneurs qui n’appartiennent qu’à Dieu, soit par un excès d’humilité, soit par une pernicieuse flatterie ; mais, outre qu’on ne cessait pas de regarder comme des hommes ceux à qui on donnait ces témoignages d’honneur, de vénération, et, si l’on veut, d’adoration, qui jamais a pensé devoir offrir des sacrifices à un autre qu’à celui qu’il savait, ou croyait, ou voulait faire croire être Dieu ? Or, que le sacrifice soit une pratique très ancienne du culte de Dieu, c’est ce qui est assez prouvé par les sacrifices de Caïn et d’Abel, le premier rejeté de Dieu, le second regardé d’un œil favorable.
\subsection[{Chapitre V}]{Chapitre V}

\begin{argument}\noindent Des sacrifices que Dieu n’exige pas et qui ont été la figure de ceux qu’il exige effectivement.
\end{argument}

\noindent Qui serait assez insensé pour croire que Dieu ait besoin des choses qu’on lui offre en sacrifice ? L’Écriture sainte témoigne le contraire en plusieurs endroits, et il suffira de rapporter cette parole du Psaume : « J’ai dit au Seigneur : Vous êtes mon Dieu, car vous n’avez pas besoin de mes biens. » Ainsi, Dieu n’a besoin ni des animaux qu’on lui sacrifie, ni d’aucune chose terrestre et corruptible, ni même de la justice de l’homme, et tout le culte légitime qui lui est rendu n’est utile qu’à l’homme qui le lui rend. Car on ne dira pas qu’il revienne quelque chose à la fontaine de ce qu’on s’y désaltère, ou à la lumière de ce qu’on la voit. Que si les anciens patriarches ont immolé à Dieu des victimes, ainsi que nous en trouvons des exemples dans l’Écriture, mais sans les imiter, ce n’était qu’une figure de nos devoirs actuels envers Dieu, c’est-à-dire du devoir de nous unir à lui et de porter vers lui notre prochain. Le sacrifice est donc un sacrement, c’est-à-dire un signe sacré et visible de l’invisible sacrifice. C’est pour cela que l’âme pénitente dans le Prophète ou le Prophète lui-même, cherchant à fléchir Dieu pour ses péchés, lui dit : « Si vous aviez voulu un sacrifice, je vous l’aurais offert avec joie ; mais vous n’avez point les holocaustes pour agréables. Le vrai sacrifice est une âme brisée de tristesse ; vous ne dédaignez pas, ô mon Dieu ! un cœur contrit et humilié. » Remarquons qu’en disant que Dieu ne veut pas de sacrifices, le Prophète fait voir en même temps qu’il en est un exigé de Dieu. Il ne veut point le sacrifice d’une bête égorgée, mais celui d’un cœur contrit. Ainsi ce que Dieu ne veut pas, selon le Prophète, est ici la figure de ce que Dieu veut. Dieu ne veut pas les sacrifices, mais seulement au sens où les insensés s’imaginent qu’il les veut, c’est-à-dire pour y prendre plaisir et se satisfaire lui-même ; car s’il n’avait pas voulu que les sacrifice qu’il demande, comme, par exemple, celui d’un cœur contrit et humilié par le repentir, fussent signifiés par les sacrifices charnels qu’on a cru qu’il désirait pour lui-même, il n’en aurait pas prescrit l’offrande dans l’ancienne loi. Aussi devaient-ils être changés au temps convenable et déterminé, de peur qu’on ne les crût agréables à Dieu par eux-mêmes, et non comme figure de sacrifices plus dignes de lui. De là ces paroles d’unautre psaume : « Si j’ai faim, je ne vous le dirai pas ; car tout l’univers est à moi, avec tout ce qu’il enferme. Mangerai-je la chair des taureaux, ou boirai-je le sang des boucs ? » Comme si Dieu disait : Quand j’aurais besoin de ces choses, je ne vous les demanderais pas, car elles sont en ma puissance. Le Psalmiste, pour expliquer le sens de ces paroles, ajoute : « Immolez à Dieu un sacrifice de louanges, et offrez vos vœux au Très-Haut. Invoquez-moi au jour de la tribulation ; je vous délivrerai et je vous glorifierai. » — « Qu’offrirai-je », dit un autre prophète, « qu’offrirai-je au Seigneur qui soit digne de lui ? fléchirai-je le genou devant le Très-Haut ? lui offrirai-je pour holocaustes des veaux d’un an ? peut-il être apaisé par le sacrifice de mille béliers ou de mille boucs engraissés ? lui sacrifierai-je mon premier-né pour mon impiété et le fruit de mes entrailles pour le péché de mon âme ? Je t’apprendrai, ô homme ! ce que tu dois faire et ce que Dieu demande de toi : pratique la justice, aime la miséricorde, et sois toujours prêt à marcher devant le Seigneur ton Dieu. » Ces paroles font assez voir que Dieu ne demande pas les sacrifices charnels pour eux-mêmes, mais comme figure des sacrifices véritables. Il est dit aussi dans l’épître aux Hébreux : « N’oubliez pas d’exercer la charité et de faire part de votre bien aux pauvres ; car c’est par de tels sacrifices qu’on est agréable à Dieu. » Ainsi, quand il est écrit : « J’aime mieux la miséricorde que le sacrifice », il ne faut entendre autre chose sinon qu’un sacrifice est préféré à l’autre, attendu que ce qu’on appelle vulgairement sacrifice n’est que le signe du sacrifice véritable. Or, la miséricorde est le sacrifice véritable ; ce qui a fait dire à l’Apôtre : « C’est par de tels sacrifices qu’on se rend agréable à Dieu. » Donc toutes les prescriptions divines touchant les sacrifices du temple ou du tabernacle se rapportent à l’amour de Dieu et du prochain ; car, ainsi qu’il est écrit : « Ces deux commandements renferment la loi et les Prophètes. »
\subsection[{Chapitre VI}]{Chapitre VI}

\begin{argument}\noindent Du vrai et parfait sacrifice.
\end{argument}

\noindent Ainsi le vrai sacrifice, c’est toute œuvre accomplie pour s’unir à Dieu d’une sainte union, c’est-à-dire toute œuvre qui se rapporte à cette fin suprême et unique où est le bonheur. C’est pourquoi la miséricorde même envers le prochain n’est pas un sacrifice, si on ne l’exerce en vue de Dieu. Le sacrifice en effet, bien qu’offert par l’homme, est chose divine, comme l’indique le mot lui-même, qui signifie {\itshape action sacrée}. Aussi l’homme même consacré et voué à Dieu est un sacrifice, en tant qu’il meurt au monde pour vivre en Dieu ; car cette consécration fait partie de la miséricorde que chacun exerce envers soi-même, et c’est pour cela qu’il est écrit : « Aie pitié de son âme en te rendant agréable à Dieu. » Notre corps est pareillement un sacrifice, quand nous le mortifions par la tempérance, si nous agissons de la sorte pour plaire à Dieu, comme nous y sommes tenus, et que loin de prêter nos membres au péché pour lui servir d’instrument d’iniquité, nous les consacrions à Dieu pour en faire des instruments de justice. C’est à quoi l’Apôtre nous exhorte en nous disant : « Je vous conjure, mes frères, par la miséricorde de Dieu, de lui offrir vos corps comme une victime vivante, sainte et agréable à ses yeux, et de « lui rendre un culte raisonnable et spirituel ». Or, si le corps, dont l’âme se sert comme d’un serviteur et d’un instrument, est un sacrifice, quand l’âme rapporte à Dieu le service qu’elle en tire, à combien plus forte raison l’âme elle-même est-elle un sacrifice, quand elle s’offre à Dieu, afin qu’embrasée du feu de son amour, elle se dépouille de toute concupiscence du siècle et soit comme renouvelée par sa soumission à cet être immuable qui aime en elle les grâces qu’elle a reçues de sa souveraine beauté ? C’est ce que le même apôtre insinue en disant : « Ne vous conformez point au siècle présent ; mais transformez-vous par le renouvellement de l’esprit, afin que vous connaissiez ce que Dieu demande de vous, c’est-à-dire ce qui est bon, ce qui lui est agréable, ce qui est parfait. » Puis donc que les œuvres de miséricorde rapportées à Dieu sont de vrais sacrifices, que nous les pratiquions envers nous-mêmes ou envers le prochain, et qu’elles n’ont d’autre fin que de nous délivrer de toute misère et de nous rendre bienheureux. Ce qui ne peut se faire que par la possession de ce bien dont il est écrit : « M’attacher à Dieu c’est mon bien », il s’ensuit que toute la cité du Rédempteur, c’est-à-dire l’assemblée et la société des saints, est elle-même un sacrifice universel offert à Dieu par le suprême pontife, qui s’est offert pour nous dans si passion, afin que nous fussions le corps de ce chef divin selon cette forme d’esclave dont il s’est revêtu. C’est cette forme, en effet, qu’il a offerte à Dieu, et c’est en elle qu’il a été offert, parce que c’est selon elle qu’il est le médiateur, le prêtre et le sacrifice. Voilà pourquoi l’Apôtre, après nous avoir exhortés à faire de nos corps une victime vivante, sainte et agréable à Dieu, à lui rendre un culte raisonnable et spirituel, à ne pas nous conformer au siècle, mais à nous transformer par un renouvellement d’esprit, afin de connaître ce que Dieu demande de nous, ce qui est bon, ce qui lui est agréable, ce qui est parfait, c’est-à-dire le vrai sacrifice qui est celui de tout notre être, l’Apôtre, dis-je, ajoute ces paroles : « Il vous recommande à tous, selon le ministère qui m’a été donné par grâce, de ne pas aspirer à être plus sages qu’il ne faut, mais de l’être avec sobriété, selon la mesure de foi que Dieu a départie à chacun de vous. Car, comme dans un seul corps nous avons plusieurs membres, lesquels n’ont pas tous la même fonction ; ainsi, quoique nous soyons plusieurs, nous n’avons qu’un seul corps en Jésus-Christ et nous sommes membres les uns des autres, ayant des dons différents, selon la grâce qui nous a été donnée. » Tel est le sacrifice des chrétiens : être tous un seul corps en Jésus-Christ, et c’est ce mystère que l’Église célèbre assidûment dans le sacrement de l’autel, connu des fidèles, où elle apprend qu’elle est offerte elle-même dans l’oblation qu’elle fait à Dieu.
\subsection[{Chapitre VII}]{Chapitre VII}

\begin{argument}\noindent Les saints anges ont pour nous un amour si pur qu’ils veulent, non pas que nous les adorions, mais que nous adorions le seul vrai Dieu.
\end{argument}

\noindent Comme les esprits qui résident dans le ciel, où ils jouissent de la possession de leur créateur, forts de sa vérité, fermes de son éternité et saints par sa grâce, comme ces esprits justement immortels et bienheureux nous aiment d’un amour plein de miséricorde, et désirent que nous soyons délivrés de notre condition de mortalité et de misère pour devenir comme eux bienheureux et immortels, ils ne veulent pas que nos sacrifices s’adressent à eux, mais à celui dont ils savent qu’ils sont comme nous le sacrifice. Nous formons en effet avec eux une seule cité de Dieu, à qui le Psalmiste adresse ces mots : « On a dit des choses glorieuses de toi, ô cité de Dieu ! » et de cette cité une partie est avec nous errante, et l’autre avec eux secourable. C’est de cette partie supérieure, qui n’a point d’autre loi que la volonté de Dieu, qu’est descendue, par le ministère des anges, cette Écriture sainte où il est dit que celui qui sacrifiera à tout autre qu’au Seigneur sera exterminé. Et cette défense a été confirmée par tant de miracles, que l’on voit assez à qui ces esprits immortels et bienheureux, qui nous souhaitent le même bonheur dont ils jouissent eux-mêmes, veulent que nous offrions nos sacrifices.
\subsection[{Chapitre VIII}]{Chapitre VIII}

\begin{argument}\noindent Des miracles que Dieu a daigné opérer par le ministère des anges à l’appui de ses promesses, pour corroborer la foi des justes.
\end{argument}

\noindent Si je ne craignais de remonter trop haut, je rapporterais tous les anciens miracles qui furent accomplis pour attester la vérité de cette promesse faite à Abraham tant de milliers d’années avant son accomplissement, que toutes les nations seraient bénies dans sa race. En effet, qui n’admirerait qu’une femme stérile ait donné un fils à Abraham, lorsqu’elle avait passé l’âge de la fécondité ? que, dans le sacrifice de ce même Abraham, une flamme descendue du ciel ait couru au milieu des victimes divisées ? que les anges, à qui il donna l’hospitalité comme à des voyageurs, lui aient prédit l’embrasement de Sodome et la naissance d’un fils ? qu’au moment où Sodome allait être consumée par le feu du ciel, ces mêmes anges aient délivré miraculeusement de cette ruine Loth, son neveu ? que la femme de Loth, ayant eu la curiosité de regarder derrière elle pendant sa fuite, ait été transformée en statue de sel, pour nous apprendre qu’une fois rentrés dans la voie du salut, nous ne devons rien regretter de ce que nous laissons derrière nous ? Mais combien furent plus grands encore les miracles que Dieu accomplit par Moïse pour délivrer son peuple de la captivité, puisqu’il ne fut permis aux mages du Pharaon, c’est-à-dire du roi d’Égypte, de faire quelques prodiges que pour rendre la victoire de Moïse plus glorieuse ! Ils n’opéraient, en effet, que par les charmes et les enchantements de la magie, c’est-à-dire par l’entremise des démons ; aussi furent-ils aisément vaincus par Moïse, qui opérait au nom du Seigneur, créateur du ciel et de la terre, et avec l’assistance des bons anges ; de sorte que les mages se trouvant sans pouvoir à la troisième plaie, Moïse en porta le nombre jusqu’à dix (figures de grands mystères) qui fléchirent enfin le cœur du Pharaon et des Égyptiens et les décidèrent à rendre aux Hébreux la liberté. Ils s’en repentirent aussitôt, et, comme ils poursuivaient les fugitifs, la mer s’ouvrit pour les Hébreux qui la passèrent à pied sec, tandis que les Égyptiens furent tous submergés par le retour des eaux. Que dirai-je de ces autres miracles du désert où éclata la puissance divine ? de ces eaux dont on ne pouvait boire et qui perdirent leur amertume au contact du bois qu’on y jeta par l’ordre de Dieu ; de la manne tombant du ciel pour rassasier ce peuple affamé, avec cette circonstance que ce que l’on en ramassait par jour au-delà de la mesure prescrite se corrompait, excepté la veille du sabbat, où la double mesure résistait à la corruption, à cause qu’il n’était pas permis d’en recueillir le jour du sabbat ; du camp israélite couvert de cailles venues en troupe pour satisfaire ce peuple qui voulait manger de la chair et qui en mangea jusqu’au dégoût ; des ennemis qui s’opposaient au passage de la mer Rouge défaits et taillés en pièces à la prière de Moïse, qui, tenant ses bras étendus en forme de croix, sauva tous les Hébreux jusqu’au dernier ; de la terre entrouverte pour engloutir tout vivants des séditieux et des transfuges, et pour les faire servir d’exemple visible d’une peine invisible ; du rocher frappé de la verge et fournissant assez d’eau pour désaltérer une si grande multitude ; du serpent d’airain élevé sur un mât et dont l’aspect guérissait les blessures mortelles que les serpents avaient faites aux Hébreux en punition de leurs péchés, afin que la mort fût détruite par la figure de la mort crucifiée ? c’est ce serpent qui, après avoir été conservé longtemps en mémoire d’un événement si merveilleux, fut depuis brisé avec raison par le roi Ézéchias, parce que le peuple commençait à l’adorer comme une idole.
\subsection[{Chapitre IX}]{Chapitre IX}

\begin{argument}\noindent Des incertitudes du platonicien Porphyre touchant les arts illicites et démoniaques.
\end{argument}

\noindent Ces miracles et beaucoup d’autres qu’il serait trop long de rapporter, avaient pour objet de consolider le culte du vrai Dieu et d’interdire le polythéisme ; ils se faisaient par une foi simple, par une pieuse confiance en Dieu, et non par les charmes et les enchantements de cette curiosité criminelle, de cet art sacrilège qu’ils appellent tantôt magie, tantôt d’un nom plus odieux, {\itshape goétie}, ou d’un nom moins décrié, {\itshape théurgie} ; car on voudrait faire une différence entre deux sortes d’opérations, et parmi les partisans des arts illicites déclarés condamnables, ceux qui pratiquent la goétie et que le vulgaire appelle {\itshape magiciens}, tandis qu’au contraire ceux qui se bornent à la théurgie seraient dignes d’éloges ; mais la vérité est que les uns et les autres sont entraînés au culte trompeur des démons qu’ils adorent sous le nom d’anges.\par
Porphyre promet une certaine purification de l’âme à l’aide de la théurgie, mais il ne la promet qu’en hésitant et pour ainsi dire en rougissant, et d’ailleurs il nie formellement que le retour de l’âme à Dieu se puisse faire par ce chemin ; de sorte qu’on le voit flotter entre les coupables secrets d’une curiosité sacrilège et les maximes de la philosophie. Tantôt en effet il nous détourne de cet art impur comme dangereux dans la pratique et prohibé par les lois, tantôt entraîné par les adeptes, il accorde que la théurgie sert à purifier une partie de l’âme, non pas, il est vrai, cette partie intellectuelle qui perçoit la vérité des choses intelligibles et absolument éloignées des sens, mais du moins cette partie spirituelle qui saisit les images sensibles. Celle-ci, suivant Porphyre, à l’aide de certaines consécrations théurgiques nommées Télètes, devient propre au commerce des esprits et des anges et capable de la vision des dieux. Il convient toutefois que ces consécrations ne servent de rien pour purifier l’âme intellectuelle et la rendre apte à voir son Dieu et à contempler les existences véritables. On jugera par un tel aveu de ce que peut être cette vision théurgique où l’on ne voit rien de ce qui existe véritablement. Porphyre ajoute que l’âme, ou, pour me servir de son expression favorite, l’âme intellectuelle peut s’élever aux régions supérieures sans que la partie spirituelle ait été purifiée par aucune opération de la théurgie, et que la théurgie, en purifiant cette partie spirituelle, ne peut pas aller jusqu’à lui donner la durée immortelle de l’éternité. Enfin, tout en distinguant les anges qui habitent, suivant lui, l’éther ou l’empyrée, d’avec les démons, dont l’air est le séjour, et tout en nous conseillant de rechercher l’amitié de quelque démon, qui veuille bien après notre mort nous soulever un peu de terre (car c’est par une autre voie que nous parvenons, suivant lui, à la société des anges), Porphyre en définitive avoue assez clairement qu’il faut éviter le commerce des démons, quand il nous représente l’âme tourmentée des peines de l’autre vie et maudissant le culte des démons dont elle s’est laissé charmer. Il n’a pu même s’empêcher de reconnaître que cette théurgie, par lui vantée comme nous conciliant les anges et les dieux, traite avec des puissances qui envient à l’âme sa purification ou qui favorisent la passion de ceux qui la lui envient. Il rapporte à ce sujet les plaintes de je ne sais quel Chaldéen : « Un homme de bien, de Chaldée, dit-il, se plaint qu’après avoir pris beaucoup de peine à purifier une âme, il n’y a pas réussi, parce qu’un autre magicien, poussé par l’envie, a lié les puissances par ses conjurations et rendu leur bonne volonté inutile. » Ainsi, ajoute Porphyre, « les liens formés par celui-ci, l’autre n’a pu les rompre » ; d’où il conclut que la théurgie sert à faire du mal comme du bien chez les dieux et chez les hommes ; et, de plus, que les dieux ont aussi des passions et sont agités par ces mêmes troubles qui, suivant Apulée, sont communs aux hommes et aux démons, mais ne peuvent atteindre les dieux placés par Platon dans une région distincte et supérieure.
\subsection[{Chapitre X}]{Chapitre X}

\begin{argument}\noindent De la théurgie, qui permet d’opérer dans les âmes une purification trompeuse par l’invocation des démons.
\end{argument}

\noindent Voici donc qu’un philosophe platonicien, Porphyre, réputé plus savant encore qu’Apulée, nous dit que les dieux peuvent être assujettis aux passions et aux agitations des hommes Par je ne sais quelle science théurgique ; nous voyons en effet que des conjurations ont suffi pour les effrayer et pour les faire renoncer à la purification d’une âme, de sorte que celui qui commandait le mal a eu plus d’empire sur eux que celui qui leur commandait le bien et qui se servait pourtant du même art. Qui ne reconnaît là les démons et leur imposture, à moins d’être du nombre de leurs esclaves et entièrement destitué de la grâce du véritable libérateur ? Car si l’on avait affaire à des dieux bons, la purification bienveillante d’une âme triompherait sans doute de la jalousie d’un magicien malfaisant ; ou si les dieux jugeaient que la purification ne fût pas méritée, au moins ne devaient-ils pas s’épouvanter des conjurations d’un envieux, ni être arrêtés, comme le rapporte formellement Porphyre, par la crainte d’un dieu plus puissant, mais plutôt refuser ce qu’on leur demande par une libre décision. N’est-il pas étrange que ce bon Chaldéen, qui désirait purifier une âme par des consécrations théurgiques, n’ait pu trouver un dieu supérieur, qui, en imprimant aux dieux subalternes une terreur plus forte, les obligeât à faire le bien qu’on réclamait d’eux, ou, en les délivrant de toute crainte, leur permît de faire ce bien librement ? Et toutefois l’honnête théurge manqua de recettes magiques pour purifier d’abord de cette crainte fatale les dieux qu’il invoquait comme purificateurs. Je voudrais bien savoir comment il se fait qu’il y ait un dieu plus puissant pour imprimer la terreur aux dieux subalternes, et qu’il n’y en ait pas pour les en délivrer. Est-ce donc à dire qu’il est aisé de trouver un dieu quand il s’agit non d’exaucer la bienveillance, mais l’envie, non de rassurer les dieux inférieurs, pour qu’ils fassent du bien, mais de les effrayer, pour qu’ils n’en fassent pas ? Ô merveilleuse purification des âmes ! sublime théurgie, qui donne à l’immonde envie plus de force qu’à la pure bienfaisance ! ou plutôt détestable et dangereuse perfidie des malins esprits, dont il faut se détourner avec horreur, pour prêter l’oreille à une doctrine salutaire ! Car ces belles imagés des anges et des dieux, qui, suivant Porphyre, apparaissent à l’âme purifiée, que sont-elles autre chose, en supposant que ces rites impurs et sacrilèges aient en effet la vertu de les faire voir, que sont-elles, sinon ce que dit l’Apôtre, c’est à savoir : « Satan transformé en ange de lumière » ? C’est lui qui, pour engager les âmes dans les mystères trompeurs des faux dieux et pour les détourner du vrai culte et du vrai Dieu, seul purificateur et médecin des âmes, leur envoie ces fantômes décevants, véritable protée, {\itshape habile à revêtir toutes les formes}, tour à tour persécuteur acharné et persécuteur perfide, toujours malfaisant.
\subsection[{Chapitre XI}]{Chapitre XI}

\begin{argument}\noindent De la lettre de Porphyre a l’Égyptien Anébon, où il le prie de l’instruire touchant les diverses espèces de démons.
\end{argument}

\noindent Porphyre a été mieux inspiré dans sa lettre à l’égyptien Anébon, où, en ayant l’air de le consulter et de lui faire des questions, il démasque et renverse tout cet art sacrilège. 11 s’y déclare ouvertement contre tous les démons, qu’il tient pour des êtres dépourvus de sagesse, attirés vers la terre par l’odeur des sacrifices, et séjournant à cause de cela, non dans l’éther, mais dans l’air, au-dessous de la lune et dans le globe même de cet astre. Il n’ose pas cependant attribuer à tous les démons toutes les perfidies, malices et stupidités dont il est justement choqué. Il dit, comme les autres, qu’il y a quelques bons démons, tout en confessant que cette espèce d’êtres est généralement dépourvue de sagesse. Il s’étonne que les sacrifices aient l’étrange vertu non seulement d’incliner les dieux, mais de les contraindre à faire ce que veulent les hommes, et il n’est pas moins surpris qu’on mette au rang des dieux le soleil, la lune et les autres astres du ciel, qui sont des corps, puisqu’on fait consister la différence des dieux et des démons en ce point que les démons ont un corps et que les dieux n’en ont pas ; et en admettant que ces astres soient en effet des dieux, il ne peut comprendre que les uns soient bienfaisants, les autres malfaisants, ni qu’on les mette au rang des êtres incorporels, puisqu’ils ont un corps. Il demande encore avec l’accent du doute si ceux qui prédisent l’avenir et qui font des prodiges ont des âmes douées d’une puissance supérieure, ou si cette puissance leur est communiquée du dehors par de certains esprits, et il estime que cette dernière opinion est la plus plausible, parce que ces magiciens se servent de certaines pierres et de certaines herbes pour opérer des alligations, ouvrir des portes et autres effets miraculeux. C’est là, suivant Porphyre, ce qui fait croire à plusieurs qu’il existe des êtres d’un ordre supérieur, dont le propre est d’être attentifs aux vœux des hommes, esprits perfides, subtils, susceptibles de toutes les formes, tour à tour dieux, démons, âmes des morts. Ces êtres produisent tout ce qui arrive de bien ou de mal, du moins ce qui nous paraît tel ; car ils ne concourent jamais au bien véritable, et ils ne le connaissent même pas ; toujours occupés de nuire, même dans les amusements de leurs loisirs, habiles à inventer des calomnies et à susciter des obstacles contre les amis de la vertu, vains et téméraires, séduits par la flatterie et par l’odeur des sacrifices. Voilà le tableau que nous trace Porphyre de ces esprits trompeurs et malins qui pénètrent du dehors dans les âmes et abusent nos sens pendant le sommeil et pendant la veille. Ce n’est pas qu’il parle du ton d’un homme convaincu et en son propre nom ; mais en rapportant les opinions d’autrui, il n’émet ses doutes qu’avec une réserve extrême. Il était difficile en effet à ce grand philosophe, soit de connaître, soit d’attaquer résolument tout ce diabolique empire, que la dernière des bonnes femmes chrétiennes découvre sans hésiter et déteste librement ; ou peut-être craignait-il d’offenser Anébon, un des principaux ministres du culte, et les autres, admirateurs de toutes ces pratiques réputées divines et religieuses.\par
Il poursuit cependant, et toujours par forme de questions ; il dévoile certains faits qui, bien considérés, ne peuvent être attribués qu’à des puissances pleines de malice et de perfidie. Il demande pourquoi, après avoir invoqué les bons esprits, on commande aux mauvais d’anéantir les volontés injustes des hommes ; pourquoi les démons n’exaucent pas les prières d’un homme qui vient d’avoir commerce avec une femme, quand ils ne se font aucun scrupule de convier les débauchés à des plaisirs incestueux ; pourquoi ils ordonnent à leurs prêtres de s’abstenir de la chair des animaux, sous prétexte d’éviter la souillure des vapeurs corporelles, quand eux-mêmes se repaissent de la vapeur des sacrifices ; pourquoi il est défendu aux initiés de toucher un cadavre, quand la plupart de leurs mystères se célèbrent avec des cadavres ; pourquoi enfin un homme, sujet aux vices les plus honteux, peut faire des menaces, non seulement à un démon ou à l’âme de quelque trépassé, mais au soleil et à la lune, ou à tout autre des dieux célestes qu’il intimide par de fausses terreurs pour leur arracher la vérité ; car il les menace de briser les cieux et d’autres choses pareilles, impossibles à l’homme, afin que ces dieux, effrayés comme des enfants de ces vaines etridicules chimères, fassent ce qui leur est ordonné. Porphyre rapporte qu’un certain Chérémon, fort habile dans ces pratiques sacrées ou plutôt sacrilèges, et qui a écrit sur les mystères fameux de l’Égypte, ceux d’Isis et de son mari Osiris, attribue à ces mystères un grand pouvoir pour contraindre les dieux à exécuter les commandements humains, quand surtout le magicien les menace de divulguer les secrets de l’art et s’écrie d’une voix terrible que, s’ils n’obéissent pas, il va mettre en pièces les membres d’Osiris. Qu’un homme fasse aux dieux ces vaines et folles menaces, non pas à des dieux secondaires, mais aux dieux célestes, tout rayonnants de la lumière sidérale, et que ces menaces, loin d’être sans effet, forcent les dieux par la terreur et la violence à exécuter ce qui leur est prescrit, voilà ce dont Porphyre s’étonne avec raison, ou plutôt, sous le voile de la surprise et en ayant l’air de chercher la cause de phénomènes si étranges, il donne à entendre qu’ils sont l’ouvrage de ces esprits dont il vient de décrire indirectement la nature : esprits trompeurs, non par essence, comme il le croit, mais par corruption, qui feignent d’être des dieux ou des âmes de trépassés, mais qui ne feignent pas, comme il le dit, d’être des démons, car ils le sont véritablement. Quant à ces pratiques bizarres, à ces herbes, à ces animaux, à ces sons de voix, à ces figures, tantôt de pure fantaisie, tantôt tracées d’après le cours des astres, qui paraissent à Porphyre capables de susciter certaines puissances et de produire certains effets, tout cela est un jeu des démons, mystificateurs des faibles et qui font leur amusement et leurs délices des erreurs des hommes. De deux choses l’une : ou Porphyre est resté en effet dans le doute sur ce sujet, tout en rapportant des faits qui montrent invinciblement que tous ces prestiges sont l’œuvre, non des puissances qui nous aident à acquérir la vie bienheureuse, mais des démons séducteurs ; ou, s’il faut mieux penser d’un philosophe, Porphyre a jugé à propos de prendre ce détour avec un Égyptien attaché à ses erreurs et enflé de la grandeur de son art, dans l’espoir de le convaincre plus aisément de la vanité et du péril de cette science trompeuse, aimant mieux prendre le personnage d’un homme qui veut s’instruire et propose humblement des questions que de combattre ouvertement la superstition et d’affecter l’autorité superbe d’un docteur. Il finit sa lettre en priant Anébon de lui enseigner comment la science des Égyptiens peut conduire à la béatitude. Du reste, quant à ceux dont tout le commerce avec les dieux se réduit à obtenir leur secours pour un esclave fugitif à recouvrer, ou pour l’acquisition d’une terre, ou pour un mariage, il déclare sans hésiter qu’ils n’ont que la vaine apparence de la sagesse ; et alors même que les puissances évoquées pour une telle fin feraient des prédictions vraies touchant d’autres événements, du moment qu’elles n’ont rien de certain à dire aux hommes en ce qui regarde la béatitude véritable, Porphyre, loin de les reconnaître pour des dieux ou pour de bons démons, n’y voit autre chose que l’esprit séducteur ou une pure illusion.
\subsection[{Chapitre XII}]{Chapitre XII}

\begin{argument}\noindent Des miracles qu’opère le vrai Dieu par le ministère des saints anges.
\end{argument}

\noindent Toutefois, comme il se fait par le moyen de ces arts illicites un grand nombre de prodiges qui surpassent la mesure de toute puissance humaine, que faut-il raisonnablement penser, sinon que ces prédictions et opérations qui se font d’une manière miraculeuse et comme surnaturelle, et qui n’ont cependant pas pour objet de glorifier le seul être où réside, du propre aveu des Platoniciens, le vrai bien et la vraie béatitude, tout cela, dis-je, n’est que pièges des démons et illusions dangereuses dont une piété bien entendue doit nous préserver ? Au contraire, nous devons croire que les miracles et toutes les œuvres surnaturelles faites par les anges ou autrement, qui ont pour objet la gloire du seul vrai Dieu, source unique de la béatitude, s’opèrent en effet par l’entremise de ceux qui nous aiment selon la vérité et la piété, et que Dieu se sert pour cela de leur ministère. N’écoutons point ceux qui ne peuvent souffrir qu’un Dieu invisible fasse des miracles visibles, puisque, de leur propre aveu, c’est Dieu qui a fait le monde, c’est-à-dire une œuvre incontestablement visible. Et certes tout ce qui arrive de miraculeux dans l’univers est moins miraculeux que l’univers lui-même, qui embrasse le ciel, la terre et toutes les créatures. Comment cet univers a-t-il été fait ? c’est ce qui nous est aussi obscur et aussi incompréhensible que la nature de son auteur. Mais bien que le miracle permanent de l’univers visible ait perdu de son prix par l’habitude où nous sommes de le voir, il suffit d’y jeter un coup d’œil attentif pour reconnaître qu’il surpasse les phénomènes les plus extraordinaires et les plus rares. Il y a, en effet, un miracle pins grand que tous les miracles dont l’homme est l’instrument, et c’est l’homme même. Voilà pourquoi Dieu, qui a fait les choses visibles, le ciel et la terre, ne dédaigne pas de faire dans le ciel et sur la terre des miracles visibles, afin d’exciter l’âme encore attachée aux choses visibles à adorer son invisible créateur ; et quant au lieu et au temps où ces miracles s’accomplissent, cela dépend d’un conseil immuable de sa sagesse, où les temps à venir sont d’avance disposés et comme accomplis. Car il meut les choses temporelles sans être mû lui-même dans le temps ; il ne connaît pas ce qui doit se faire autrement que ce qui est fait ; il n’exauce pas qui l’invoque autrement qu’il ne voit qui le doit invoquer. Quand ses anges exaucent une prière, il l’exauce en eux comme en son vrai temple, qui n’est pas l’œuvre d’une main mortelle et où il habite comme il habite aussi dans l’âme des saints. Enfin, les volontés divines s’accomplissent dans le temps ; Dieu les forme et les conçoit dans l’éternité.
\subsection[{Chapitre XIII}]{Chapitre XIII}

\begin{argument}\noindent Invisible en soi, Dieu s’est rendu souvent visible, non tel qu’il est, mais tel que les hommes le pouvaient voir.
\end{argument}

\noindent On ne doit pas trouver étrange que Dieu, tout invisible que soit son essence, ait souvent apparu sous une forme visible aux patriarches. Car, comme le son de la voix, qui fait éclater au dehors la pensée conçue dans le silence de l’entendement, n’est pas la pensée même, ainsi la forme sous laquelle Dieu, invisible en soi, s’est montré visible, était autre chose que Dieu ; et cependant c’est bien lui qui apparaissait sous cette forme corporelle, comme c’est bien la pensée qui se fait entendre dans le son de la voix. Les patriarches eux-mêmes n’ignoraient pas qu’ils voyaient Dieu sous une forme corporelle qui n’était pas lui. Ainsi, bien que Dieu parlât à Moïse et que Moïse lui répondît, Moïse ne laissait pas de dire à Dieu « Si j’ai trouvé grâce devant vous, montrez-vous vous-même à moi, afin que je sois assuré de vous voir. » Et comme il fallait que la loi de Dieu fût publiée avec un appareil terrible, étant donnée, non à un homme ou à un petit nombre de sages, mais à une nation tout entière, à un peuple immense, Dieu fit de grandes choses par le ministère des anges sur le Sinaï, où la loi fut révélée à un seul en présence de la multitude qui contemplait avec effroi tant de signes surprenants. C’est qu’il n’en était pas du peuple d’Israël par rapport à Moïse comme des Lacédémoniens qui crurent à la parole de Lycurgue déclarant tenir ses lois de Jupiter ou d’Apollon ; la loi de Moïse ordonnait d’adorer un seul Dieu, et dès lors il était nécessaire que Dieu fît éclater sa majesté par des effets assez merveilleux pour montrer que Moïse n’était qu’une créature dont se servait le créateur.
\subsection[{Chapitre XIV}]{Chapitre XIV}

\begin{argument}\noindent Il ne faut adorer qu’un seul Dieu, non seulement en vue des biens éternels, mais en vue même des biens terrestres qui dépendent tous de sa providence.
\end{argument}

\noindent L’espèce humaine, représentée par le peuple de Dieu, peut être assimilée à un seul homme dont l’éducation se fait par degrés. La suite des temps a été pour ce peuple ce qu’est la suite des âges pour l’individu, et il s’est peu à peu élevé des choses temporelles aux choses éternelles, et du visible à l’invisible ; et toutefois, alors même qu’on lui promettait des biens visibles pour récompense, on ne cessait pas de lui commander d’adorer un seul Dieu, afin de montrer à l’homme que, pour ces biens eux-mêmes, il ne doit point s’adresser à un autre qu’à son maître et créateur. Quiconque, en effet, ne conviendra pas qu’un seul Dieu tout-puissant est le maître absolu de tous les biens que les anges ou les hommes peuvent faire aux hommes, est véritablement insensé. Plotin, philosophe platonicien, a discuté la question de la providence ; et il lui suffit de la beauté des fleurs et des feuilles pour prouver cette providence dont la beauté est intelligible et ineffable, qui descend des hauteurs de la majesté divine jusqu’aux choses de la terre les plus viles et les plus basses, puisque, en effet, ces créatures si frêles et qui passent si vite n’auraient point leur beauté et leurs harmonieuses proportions, si elles n’étaient formées par un être toujours subsistant qui enveloppe tout dans sa forme intelligible et immuable. C’est ce qu’enseigne Notre-Seigneur Jésus-Christ quand il dit : « Regardez les lis des champs ; ils ne travaillent, ni ne filent ; or, je vous dis que Salomon même, dans toute sa gloire, n’était point vêtu comme l’un d’eux. Que si Dieu prend soin de vêtir de la sorte l’herbe des champs, qui est aujourd’hui et qui demain sera jetée au four, que ne fera-t-il pas pour vous, hommes de peu de foi ? » Il était donc convenable d’accoutumer l’homme encore faible et attaché aux objets terrestres à n’attendre que de Dieu seul les biens nécessaires à cette vie mortelle, si méprisables qu’ils soient d’ailleurs au prix des biens de l’autre vie, afin que, dans le désir même de ces biens imparfaits, il ne s’écartât pas du culte de celui qu’on ne possède qu’en les méprisant.
\subsection[{Chapitre XV}]{Chapitre XV}

\begin{argument}\noindent Du ministère des saints anges, instruments de la Providence divine.
\end{argument}

\noindent Il a donc plu à la divine Providence, comme je l’ai déjà dit et comme on le peut voir dans les Actes des Apôtres, d’ordonner le cours des temps de telle sorte que la loi qui commandait le culte d’un seul Dieu fût publiée par le ministère des anges. Or, Dieu voulut dans cette occasion se manifester d’une manière visible, non en sa propre substance, toujours invisible aux yeux du corps, mais par de certains signes qui font des choses créées la marque sensible de la présence du Créateur. Il se servit du langage humain, successif et divisible, pour transmettre aux hommes cette voix spirituelle, intelligible et éternelle qui ne commence, ni ne cesse deparler, et qu’entendent dans sa pureté, non par l’oreille, mais par l’intelligence, les ministres de sa volonté, ces esprits bienheureux admis à jouir pour jamais de sa vérité immuable et toujours prêts à exécuter sans retard et sans effort dans l’ordre des choses visibles les ordres qu’elle leur communique d’une manière ineffable. La loi divine a donc été donnée selon la dispensation des temps ; elle ne promettait d’abord, je le répète, que des biens terrestres, qui étaient à la vérité la figure des biens éternels ; mais si un grand nombre de Juifs célébraient ces promesses par des solennités visibles, peu les comprenaient. Toutefois, et les paroles et les cérémonies de la loi prêchaient hautement le culte d’un seul Dieu, non pas d’un de ces dieux choisis dans la foule des divinités païennes, mais de celui qui a fait et le ciel et la terre, et tout esprit et toute âme, et tout ce qui n’est pas lui ; car il est le créateur et tout le reste est créature ; et rien n’existe et ne se conserve que par celui qui a tout fait.
\subsection[{Chapitre XVI}]{Chapitre XVI}

\begin{argument}\noindent Si nous devons, pour arriver à la vie bienheureuse, croire plutôt ceux d’entre les anges qui veulent qu’on les adore que ceux qui veulent qu’on n’adore que Dieu.
\end{argument}

\noindent À quels anges devons-nous ajouter foi pour obtenir la vie éternelle et bienheureuse ? À ceux qui demandent aux hommes un culte religieux et des honneurs divins, ou à ceux qui disent que ce culte n’est dû qu’au Dieu créateur, et qui nous commandent d’adorer en vérité celui dont la vision fait leur béatitude et en qui ils nous promettent que nous trouverons un jour la nôtre ? Cette vision de Dieu est en effet la vision d’une beauté si parfaite et si digne d’amour, que Plotin n’hésite pas à déclarer que sans elle, fût-on d’ailleurs comblé de tous les autres biens, on est nécessairement malheureux. Lors donc que les divers anges font des miracles, les uns, pour nous inviter à rendre à Dieu seul le culte de latrie, les autres pour se le faire rendre à eux-mêmes, mais avec cette différence que les premiers nous défendent d’adorer des anges, au lieu que les seconds ne nous défendent pas d’adorer Dieu, je demande quels sont ceux à qui l’on doit ajouter foi ? Que les Platoniciens répondent à cette question ; que tous les autres philosophes y répondent ; qu’ils y répondent aussi ces théurges, ou plutôt ces périurges, car ils ne méritent pas un nom plus flatteur ; en un mot, que tous les hommes répondent, s’il leur reste une étincelle de raison, et qu’ils nous disent si nous devons adorer ces anges ou ces dieux qui veulent qu’on les adore de préférence au Dieu que les autres nous commandent d’adorer, à l’exclusion d’eux-mêmes et des autres anges. Quand ni les uns ni les autres ne feraient de miracles, cette seule considération que les uns ordonnent qu’on leur sacrifie, tandis que les autres le défendent et exigent qu’on ne sacrifie qu’au vrai Dieu, suffirait pour faire discerner à une âme pieuse de quel côté est le faste et l’orgueil, de quel côté la véritable religion. Je dis plus : alors même que ceux qui demandent à être adorés seraient les seuls à faire des miracles et que les autres dédaigneraient ce moyen, l’autorité de ces derniers devrait être préférable aux yeux de quiconque se détermine par la raison plutôt que par les sens. Mais puisque Dieu, pour consacrer la vérité, a permis que ces esprits immortels aient opéré, en vue de sa gloire et non de la leur, des miracles d’une grandeur et d’une certitude supérieures, afin, sans doute, de mettre ainsi les âmes faibles en garde contre les prestiges des démons orgueilleux, ne serait-ce pas le comble de la déraison que de fermer les yeux à la vérité, quand elle éclate avec plus de force que le mensonge ?\par
Pour toucher un mot, en effet, des miracles attribués par les historiens aux dieux des Gentils, en quoi je n’entends point parler des accidents monstrueux qui se produisent de loin en loin par des causes cachées, comprises dans les plans de la Providence, tels, par exemple, que la naissance d’animaux difformes, ou quelque changement inusité sur la face du ciel et de la terre, capable de surprendre ou même de nuire, je n’entends point, dis-je, parler de ce genre d’événements dont les démons fallacieux prétendent que leur culte préserve le monde, mais d’autres événements qui paraissent en effet devoir être attribués à leur action et à leur puissance, comme ce que l’on rapporte des images des dieux pénates, rapportées de Troie par Énée et qui passèrent d’elles-mêmes d’un lieu à un autre ; de Tarquin, qui coupa un caillou avec un rasoir ; du serpent d’Épidaure, qui accompagna Esculape dans son voyage à Rome ; de cette femme qui, pour prouver sa chasteté, tira seule avec sa ceinture le vaisseau qui portait la statue de la mère des dieux, tandis qu’un grand nombre d’hommes et d’animaux n’avaient pu seulement l’ébranler ; de cette vestale qui témoigna aussi son innocence en puisant de l’eau du Tibre dans un crible ; voilà bien des miracles, mais aucun n’est comparable, ni en grandeur, ni en puissance, à ceux que l’Écriture nous montre accomplis pour le peuple de Dieu. Combien moins peut-on leur comparer ceux que punissent et prohibent les lois des peuples païens eux-mêmes, je veux parler de ces œuvres de magie et de théurgie qui ne sont pour la plupart que de vaines apparences et de trompeuses illusions, comme, par exemple, quand il s’agit de faire descendre la lune, afin, dit le poète Lucain, qu’elle répande de plus près son écume sur les herbes, Et s’il est quelques-uns de ces prodiges qui semblent égaler ceux qu’accomplissent les serviteurs de Dieu, la diversité de leurs fins, qui sert à les distinguer les uns des autres, fait assez voir que les nôtres sont incomparablement plus excellents. En effet, les uns ont pour objet d’établir le culte de fausses divinités que leur vain orgueil rend d’autant plus indignes de nos sacrifices qu’elles les souhaitent avec plus d’ardeur ; les autres ne tendent qu’à la gloire d’un Dieu qui témoigne dans ses Écritures qu’il n’a aucun besoin de tels sacrifices, comme il l’a montré plus tard en les refusant pour l’avenir. En résumé, s’il y a des anges qui demandent le sacrifice pour eux-mêmes, il faut leur préférer ceux qui ne le réclament que pour le Dieu qu’ils servent et qui a créé l’univers ; ces derniers, en effet, font bien voir de quel sincère amour ils nous aiment, puisqu’au lieu de nous soumettre à leur propre empire, ils ne cherchent qu’à nous faire parvenir vers l’être dont la contemplation leur promet à eux-mêmes une félicité inébranlable. En second lieu, s’il y a des anges qui, sans vouloir qu’on leur sacrifie, ordonnent qu’on sacrifie à plusieurs dieux dont ils sont les anges, il faut encore leur préférer ceux qui sont les anges d’un seul Dieu et qui nous défendent de sacrifier à tout autre qu’à lui, tandis que les autres n’interdisent pas de sacrifier à ce Dieu-là. Enfin, si ceux qui veulent qu’on leur sacrifie ne sont ni de bons anges, ni les anges de bonnes divinités, mais de mauvais démons, comme le prouvent leurs impostures et leur orgueil, à quelle protection plus puissante avoir recours contre eux qu’à celle du Dieu unique et véritable que servent les anges, ces bons anges qui ne demandent pas nos sacrifices pour eux, mais pour celui dont nous devons nous-mêmes être le sacrifice ?
\subsection[{Chapitre XVII}]{Chapitre XVII}

\begin{argument}\noindent De l’arche du Testament et des miracles que Dieu opéra pour fortifier l’autorité de sa loi et de ses promesses.
\end{argument}

\noindent C’est pour cela que la loi de Dieu, donnée au peuple juif par le ministère des anges, et qui ordonnait d’adorer le seul Dieu des dieux, à l’exclusion de tous les autres, était déposée dans l’arche dite du Témoignage. Ce nom indique assez que Dieu, à qui s’adressait tout ce culte extérieur, n’est point contenu et enfermé dans un certain lieu, et que si ses réponses et divers signes sensibles sortaient en effet de cette arche, ils n’étaient que le témoignage visible de ses volontés. La loi elle-même était gravée sur des tables de pierre et renfermée dans l’arche, comme je viens de le dire. Au temps que le peuple errait dans le désert, les prêtres la portaient avec respect avec le tabernacle, dit aussi du Témoignage, et le signe ordinaire qui l’accompagnait était une colonne de nuée durant le jour et une colonne de feu durant la nuit. Quand cette nuée marchait, les Hébreux levaient leur camp, et ils campaient, quand elle s’arrêtait. Outre ce miracle et les voix qui se faisaient entendre de l’arche, il y en eut encore d’autres qui rendirent témoignage à la loi ; car, lorsque lepeuple entra dans la terre de promission, le Jourdain s’ouvrit pour donner passage à l’arche aussi bien qu’à toute l’armée. Cette même arche ayant été portée sept fois autour de la première ville ennemie qu’on rencontra (laquelle adorait plusieurs dieux à l’instar des Gentils), les murailles tombèrent d’elles-mêmes sans être ébranlées ni par la sape ni par le bélier. Depuis, à une époque où les Israélites étaient déjà établis dans la terre promise, il arriva que l’arche fut prise en punition de leurs péchés, et que ceux qui s’en étaient emparés l’enfermèrent avec honneur dans le temple du plus considérable de leurs dieux. Or, le lendemain, à l’ouverture du temple, ils trouvèrent la statue du dieu renversée par terre et honteusement fracassée. Divers prodiges et la plaie honteuse dont ils furent frappés les engagèrent dans la suite à restituer l’arche de Dieu. Mais comment fut-elle rendue ? ils la mirent sur un chariot, auquel ils attelèrent des vaches dont ils eurent soin de retenir les petits, puis ils laissèrent aller ces animaux à leur gré, pour voir s’il se produirait quelque chose de divin. Or, les vaches, sans guide, sans conducteur, malgré les cris de leurs petits affamés, marchèrent droit en Judée et rendirent aux Hébreux l’arche mystérieuse, Ce sont là de petites choses au regard de Dieu ; mais elles sont grandes par l’instruction et la terreur salutaire qu’elles doivent donner aux hommes. Si certains philosophes, et à leur tête les Platoniciens, ont montré plus de sagesse et mérité plus de gloire que tous les autres, pour avoir enseigné que la Providence divine descend jusqu’aux derniers êtres de la nature, et fait éclater sa splendeur dans l’herbe des champs aussi bien que dans les corps des animaux, comment ne pas se rendre aux témoignages miraculeux d’une religion qui ordonne de sacrifier à Dieu seul, à l’exclusion de toute créature du ciel, de la terre et des enfers ? Et quel est le Dieu de cette religion ? Celui qui peut seul faire notre bonheur par l’amour qu’il nous porte et par l’amour que nous lui rendons, celui qui, bornant le temps des sacrifices de l’ancienne loi dont il avait prédit la réforme par un meilleur pontife, a témoigné qu’il ne les désire pas pour eux-mêmes, et que s’il les avait ordonnés, c’était comme figure de sacrifices plus parfaits ; car enfin Dieu ne veut pas notreculte pour en tirer de la gloire, mais pour nous unir étroitement à lui, en nous enflammant d’un amour qui fait notre bonheur et non pas le sien.
\subsection[{Chapitre XVIII}]{Chapitre XVIII}

\begin{argument}\noindent Contre ceux qui nient qu’il faille s’en fier aux livres saints touchant les miracles accomplis pour l’instruction du peuple de Dieu.
\end{argument}

\noindent S’avisera-t-on de dire que ces miracles sont faux et supposés ? quiconque parle de la sorte et prétend qu’en fait de miracles il ne faut s’en fier à aucun historien, peut aussi bien prétendre qu’il n’y a point de dieux qui se mêlent des choses de ce monde. C’est par des miracles, en effet, que les dieux ont persuadé aux hommes de les adorer, comme l’atteste l’histoire des Gentils, et nous y voyons les dieux plus occupés de se faire admirer que de se rendre utiles. C’est pourquoi nous n’avons pas entrepris dans cet ouvrage de réfuter ceux qui nient toute existence divine ou qui croient la divinité indifférente aux événements du monde, mais ceux qui préfèrent leurs dieux au Dieu fondateur de l’éternelle et glorieuse Cité, ne sachant pas qu’il est pareillement le fondateur invisible et immuable de ce monde muable et visible, et le véritable dispensateur de cette félicité qui réside en lui-même et non pas en ses créatures. Voilà le sens de ce mot du très véridique prophète : « Être uni à Dieu, voilà mon bien. » Je reviens sur cette citation, parce qu’il s’agit ici de la fin de l’homme, de ce problème tant controversé entre les philosophes, de ce souverain bien où il faut rapporter tous nos devoirs. Le Psalmiste ne dit pas : Mon bien, c’est de posséder de grandes richesses, ou de porter la pourpre, le sceptre et le diadème ; ou encore, comme quelques philosophes n’ont point rougi de le dire : Mon bien, c’est de jouir des voluptés du corps ; ou même enfin, suivant l’opinion meilleure de philosophes meilleurs : Mon bien, c’est la vertu de mon âme ; non, le Psalmiste le déclare le vrai bien, c’est d’être uni à Dieu. Il avait appris cette vérité de celui-là même que les anges, par des miracles incontestables, lui avaient appris à adorer exclusivement. Aussi était-il lui-même le sacrifice de Dieu, puisqu’il était consumé du feu de son amour etdésirait ardemment de jouir de ses chastes et ineffables embrassements. Mais enfin, si ceux qui adorent plusieurs dieux (quelque sentiment qu’ils aient touchant leur nature) ne doutent point des miracles qu’on leur attribue, et s’en rapportent soit aux historiens, soit aux livres de la magie, soit enfin aux livres moins suspects de la théurgie, pourquoi refusent-ils de croire aux miracles attestés par nos Écritures, dont l’autorité doit être estimée d’autant plus grande que celui à qui seul elles commandent de sacrifier est plus grand ?
\subsection[{Chapitre XIX}]{Chapitre XIX}

\begin{argument}\noindent Quel est l’objet du sacrifice visible que la vraie religion ordonne d’offrir au seul Dieu invisible et véritable.
\end{argument}

\noindent Quant à ceux qui estiment que les sacrifices visibles doivent être offerts aux autres dieux, mais que les sacrifices invisibles, tels que les mouvements d’une âme pure et d’une bonne volonté, appartiennent, comme plus grands et plus excellents, au Dieu invisible, plus grand lui-même et plus excellent que tous les dieux, ils ignorent sans doute que les sacrifices visibles ne sont que les signes des autres, comme les mots ne sont que les signes des choses. Or, puisque dans la prière nous adressons nos paroles à celui-là même à qui nous offrons les pensées de nos cœurs, n’oublions pas, quand nous sacrifions, qu’il ne faut offrir le sacrifice visible qu’à celui dont nous devons être nous-mêmes le sacrifice invisible. C’est alors que les Anges et les Vertus supérieures, dont la bonté et la piété font la puissance, se réjouissent avec nous de ce culte que nous rendons à Dieu, et nous aident à le lui rendre. Mais si nous voulons les adorer, ces purs esprits sont si peu disposés à agréer notre culte qu’ils le rejettent positivement, quand ils viennent remplir quelque mission visible auprès des hommes. L’Écriture sainte en fournit des exemples. Nous y voyons, en effet, que quelques fidèles ayant cru devoir leur rendre les honneurs divins, soit par l’adoration, soit par le sacrifice, ils les en ont empêchés, avec ordre de les reporter au seul être à qui ils savent qu’ils sont dus. Les saints ont imité les anges : après la guérison miraculeuse que saint Paul et saint Barnabé opérèrent en Lycaonie, le peuple les prit pour des dieux et voulut leur sacrifier ; mais leur humble piété s’y opposa, et ils annoncèrent aux Lycaoniens le Dieu en qui ils devaient croire. Les esprits trompeurs eux-mêmes n’exigent ces honneurs que parce qu’ils savent qu’ils n’appartiennent qu’au vrai Dieu. Ce qu’ils aiment, ce n’est pas, comme le rapporte Porphyre, et comme quelques-uns le croient, les odeurs corporelles, mais les honneurs divins. Dans le fait, ils ont assez de ces sortes d’odeurs qui leur viennent de tout côté, et, s’ils en voulaient davantage, il ne tiendrait qu’à eux de s’en donner ; mais ces mauvais esprits, qui affectent la divinité, ne se contentent pas de la fumée des corps, ils demandent les hommages du cœur, afin d’exercer leur domination sur ceux qu’ils abusent, et de leur fermer la voie qui mène au vrai Dieu, en les empêchant par ces sacrifices impies de devenir eux-mêmes un sacrifice agréable à Dieu.
\subsection[{Chapitre XX}]{Chapitre XX}

\begin{argument}\noindent Du véritable et suprême sacrifice effectué par le Christ lui-même, médiateur entre dieu et les hommes.
\end{argument}

\noindent De là vient que ce vrai médiateur entre Dieu et les hommes, médiateur en tant qu’il a pris la forme d’esclave, Jésus-Christ homme, bien qu’il reçoive le sacrifice, à titre de Dieu consubstantiel au Père, a mieux aimé être lui-même le sacrifice, à titre d’esclave, que de le recevoir, et cela, pour ne donner occasion à personne de croire qu’il soit permis de sacrifier à une créature, quelle qu’elle soit. Il est donc à la fois le prêtre et la victime, et voilà le sens du sacrifice que l’Église lui offre chaque jour ; car l’Église, comme corps dont il est le chef, s’offre elle-même par lui. Les anciens sacrifices des saints n’étaient aussi que des signes divers et multipliés de ce sacrifice véritable, de même que plusieurs mots servent quelquefois à exprimer une seule chose en l’inculquant plus fortement et sans ennui. Devant ce suprême et vrai sacrifice, tous les faux sacrifices ont disparu.
\subsection[{Chapitre XXI}]{Chapitre XXI}

\begin{argument}\noindent Du degré de puissance accordé aux démons pour procurer, par des épreuves patiemment subies, la gloire des saints, lesquels n’ont pas vaincu les démons en leur faisant des sacrifices, mais en restant fidèles à Dieu.
\end{argument}

\noindent Toutefois les démons ont reçu le pouvoir, en des temps réglés et limités par la Providence, d’exercer leur fureur contre la Cité de Dieu à l’aide de ceux qu’ils ont séduits, et non seulement de recevoir les sacrifices qu’on leur offre mais aussi d’en exiger par de violentes persécutions. Or, tant s’en faut que cette tyrannie soit préjudiciable à l’Église, qu’elle lui procure, au contraire, de grands avantages ; elle sert, en effet, à compléter le nombre des saints, qui tiennent un rang d’autant plus honorable dans la Cité de Dieu qu’ils combattent plus généreusement et jusqu’à la mort contre les puissances de l’impiété. Si le langage de l’Église le permettait, nous les appellerions à bon droit nos héros. On fait venir ce nom de celui de Junon, qui, en grec, est appelé Héra, d’où vient que, suivant les fables de la Grèce, je ne sais plus lequel de ses fils porte le nom d’Héros. Le sens mystique de ces noms est, dit-on, que Junon représente l’air, dans lequel on place, en compagnie des démons, les héros, c’est-à-dire les âmes des morts illustres. C’est dans un sens tout contraire qu’on pourrait, je le répète, si le langage ecclésiastique le permettait, appeler nos martyrs des héros ; non certes qu’ils aient aucun commerce dans l’air avec les démons, mais parce qu’ils ont vaincu les démons, c’est-à-dire les puissances de l’air et Junon elle-même, quelle qu’elle soit, cette Junon que les poètes nous représentent, non sans raison, comme ennemie de la vertu et jalouse de la gloire des grands hommes qui aspirent au ciel. Virgile met ceux-ci au-dessus d’elle quand il lui fait dire :\par
 {\itshape « Énée est mon vainqueur… »} \par
mais il lui cède ensuite et faiblit misérablement quand il introduit Hélénus donnant à Énée ce prétendu conseil de piété :\par
 {\itshape « Rends hommage de bon cœur à Junon et triomphe par tes offrandes suppliantes du courroux de cette redoutable divinité. »} \par
Porphyre est du même avis, tout en ne parlant, il est vrai, qu’au nom d’autrui, quand il dit que le bon génie n’assiste point celui qui l’invoque, à moins que le mauvais génie n’ait été préalablement apaisé ; d’où il suivrait que les mauvaises divinités sont plus puissantes que les bonnes ; car les mauvaises peuvent mettre obstacle à l’action des bonnes, et celles-ci ne peuvent rien sans la permission de celles-là, tandis qu’au contraire les mauvaises divinités peuvent nuire, sans que les autres soient capables de les en empêcher. Il en est tout autrement dans la véritable religion ; et ce n’est pas ainsi que nos martyrs triomphent de Junon, c’est-à-dire des puissances de l’air envieuses de la vertu des saints. Nos héros, si l’usage permettait de les appeler ainsi, n’emploient pour vaincre Héra que des vertus divines et non des offrandes suppliantes. Et certes, Scipion a mieux mérité le Surnom d’Africain en domptant l’Afrique par sa valeur que s’il eût apaisé ses ennemis par des présents et des supplications.
\subsection[{Chapitre XXII}]{Chapitre XXII}

\begin{argument}\noindent Où est la source du pouvoir des saints contre les démons et de la vraie purification du cœur.
\end{argument}

\noindent Les hommes véritablement pieux chassent ces puissances aériennes par des exorcismes, loin de rien faire pour les apaiser, et ils surmontent toutes les tentations de l’ennemi, non en les priant, mais en priant Dieu contre lui. Aussi, les démons ne triomphent-ils que des âmes entrées dans leur commerce par le péché. On triomphe d’eux, au contraire, au nom de celui qui s’est fait homme, et homme sans péché, pour opérer en lui-même, comme pontife et comme victime, la rémission des péchés, c’est-à-dire au nom du médiateur Jésus-Christ homme, par qui les hommes, purifiés du péché, sont réconciliés avec Dieu. Le péché seul, en effet, sépare les hommes d’avec Dieu, et s’ils peuvent en être purifiés en cette vie, ce n’est point par la vertu, mais bien par la miséricorde divine ; ce n’est point par leur puissance propre, mais par l’indulgencede Dieu, puisque la faible et misérable vertu qu’on appelle la vertu humaine n’est elle-même qu’un don de sa bonté. Nous serions trop disposés à nous enorgueillir dans notre condition charnelle, si, avant de la dépouiller, nous ne vivions pas sous le pardon. C’est pourquoi la vertu du Médiateur nous a fait cette grâce que, souillés par la chair du péché, nous trouvons notre purification dans un Dieu fait chair ; grâce merveilleuse, où éclate la miséricorde de Dieu, et qui, après nous avoir conduits durant cette vie dans le chemin de la foi, nous prépare, après la mort, par la contemplation de la vérité immuable, la plénitude de la perfection.
\subsection[{Chapitre XXIII}]{Chapitre XXIII}

\begin{argument}\noindent Des principes de la purification de l’âme selon les Platoniciens.
\end{argument}

\noindent Des oracles divins, dit Porphyre, ont répondu que les sacrifices les plus parfaits à la lune et au soleil sont incapables de purifier, et il a voulu montrer par là qu’il en est de même des sacrifices offerts à tous les autres dieux. Quels sacrifices, en effet, auraient une vertu purifiante, si ceux de la lune et du soleil, divinités du premier ordre, ne l’ont pas ? Porphyre, d’ailleurs, ajoute que le même oracle a déclaré que les Principes peuvent purifier ; par où l’on voit assez que ce philosophe a craint que sur la première réponse, qui refuse aux sacrifices parfaits du soleil et de la lune la vertu purifiante, on ne s’avisât de l’attribuer aux sacrifices de quelqu’un des petits dieux. Mais qu’entend Porphyre par ses Principes ? dans la bouche d’un philosophe platonicien, nous savons ce que cela signifie il veut désigner Dieu le Père d’abord, puis Dieu le Fils, qu’il appelle la Pensée ou l’Intelligence du Père ; quant au Saint-Esprit, il n’en dit rien, ou ce qu’il en dit n’est pas clair ; car je n’entends pas quel est cet autre Principe qui tient le milieu, suivant lui, entre les deux autres. Est-il du sentiment de Plotin, qui, traitant des trois hypostases principalesdonne à l’âme le troisième rang ? mais alors il ne dirait pas que la troisième hypostase tient le milieu entre les deux autres, c’est-à-dire entre le Père et le Fils. En effet, Plotin place l’âme au-dessous de la seconde hypostase, qui est la pensée du Père, tandis que Porphyre, en faisant de l’âme une substance mitoyenne, ne la place pas au-dessous des deux autres, mais entre les deux. Porphyre, sans doute, a parlé comme il a pu, ou comme il a voulu car nous disons, nous, que le Saint-Esprit n’est pas seulement l’esprit du Père, ou l’esprit du Fils, mais l’esprit du Père et du Fils. Aussi bien, les philosophes sont libres dans leurs expressions, et, en parlant des plus hautes matières, ils ne craignent pas d’offenser les oreilles pieuses, Mais nous ; nous sommes obligés de soumettre nos paroles à une règle précise, de crainte que la licence dans les mots n’engendre l’impiété dans les choses.
\subsection[{Chapitre XXIV}]{Chapitre XXIV}

\begin{argument}\noindent Du Principe unique et véritable qui seul purifie et renouvelle la nature humaine.
\end{argument}

\noindent Lors donc que nous parlons de Dieu, nous n’affirmons point deux ou trois principes, pas plus que nous n’avons le droit d’affirmer deux ou trois dieux ; et toutefois, en affirmant tour à tour le Père, le Fils et le Saint-Esprit, nous disons de chacun qu’il est Dieu. Car nous ne tombons pas dans l’hérésie des Sabelliens, qui soutiennent que le Père est identique au Fils, et que le Saint-Esprit est identique au Fils et au Père ; nous disons, nous, que le Père est le Père du Fils, que le Fils est le Fils du Père, et que le Saint-Esprit est l’Esprit du Père et du Fils, sans être ni le Père, ni le Fils. Il est donc vrai de dire que le Principe seul purifie l’homme, et non les Principes, comme l’ont soutenu les Platoniciens. Mais Porphyre, soumis à ces puissances envieuses dont il rougissait sans oser les combattre, ouvertement, n’a pas voulu reconnaître que le Seigneur Jésus-Christ est le principe qui nous purifie par son incarnation. Il l’a sans doute méprisé dans la chair qu’il a revêtue pour accomplir le sacrifice destiné à nous purifier ; grand mystère que n’a point compris Porphyre, par un effet de cet orgueil que le bon, le vrai Médiateur a vaincu par son humilité, prenant la nature mortelle pour se montrer à des êtres mortels, tandis que les faux et méchants médiateurs, fiers de n’être pas sujets à la mort, se sont exaltés dans leur orgueil, et par le prestige de leur immortalité ont fait espérer à des êtres mortels un secours trompeur. Ce bon et véritable Médiateur a donc montré que le mal consiste dans le péché, et non dans la substance ou la nature de la chair, puisqu’il a pris la chair avec l’âme de l’homme sans prendre le péché, puisqu’il a vécu dans cette chair, et qu’après l’avoir quittée par la mort, il l’a reprise transfigurée dans sa résurrection. Il a montré aussi que la mort même, peine du péché, qu’il a subie pour nous sans avoir péché, ne doit pas être évitée par le péché, mais plutôt supportée à l’occasion pour la justice car s’il a eu la puissance de racheter nos péchés par sa mort, c’est qu’il est mort lui-même et n’est pas mort par son péché. Mais Porphyre n’a point connu le Christ comme Principe ; car autrement il l’eût connu comme purificateur. Le Principe, en effet, dans le Christ, ce n’est pas la chair ou l’âme humaine, mais bien le Verbe par qui tout a été fait. D’où il suit que la chair du Christ ne purifie pointpar elle-même, mais par le Verbe qui a pris cette chair, quand « le Verbe s’est fait chair et a habité parmi nous ». C’est pourquoi, quand Jésus parlait dans un sens mystique de la manducation de sa chair, plusieurs qui l’écoutaient sans le comprendre s’étant retirés en s’écriant : « Ces paroles sont dures ; est-il possible de les écouter ? » il dit à ceux qui restèrent auprès de lui : « C’est l’esprit qui vivifie ; la chair ne sert de rien. » Il faut conclure que c’est le Principe qui, en prenant une chair et une âme, purifie l’âme et la chair des fidèles, et voilà le sens de la réponse de Jésus aux Juifs qui lui demandaient qui il était : « Je suis le Principe. » Nous-mêmes, faibles que nous sommes, charnels et pécheurs, nous ne pourrions, enveloppés dans les ténèbres de l’ignorance, comprendre cette parole, si le Christ ne nous avait doublement purifiés et par ce que nous étions et par ce que nous n’étions pas ; car nous étions hommes, et nous n’étions pas justes, et dans l’Incarnation il y a l’homme, mais juste et sans péché. Voilà le Médiateur qui nous a tendu la main pour nous relever, quand nousétions tombés et gisants par terre ; voilà la semence organisée par le ministère des anges, promulgateurs de la loi qui contenait tout ensemble le commandement d’obéir à un seul Dieu et la promesse du médiateur à venir.
\subsection[{Chapitre XXV}]{Chapitre XXV}

\begin{argument}\noindent Tous les saints qui ont vécu sous la loi écrite et dans les temps antérieurs ont été justifiés par la foi en Jésus-Christ.
\end{argument}

\noindent C’est par leur foi en ce mystère, accompagnée de la bonne vie, que les justes des anciens jours ont pu être purifiés, soit avant la loi de Moïse (car en ce temps Dieu et les anges leur servaient de guides), soit même sous cette loi, bien qu’elle ne renfermât que des promesses temporelles, simple figure de promesses plus hautes, ce qui a fait donner à la loi de Moïse le nom d’Ancien Testament. Il y avait alors, en effet, des Prophètes dont la voix, comme celle des anges, publiait la céleste promesse, et de ce nombre était celui dont j’ai cité plus haut cette divine sentence touchant le souverain bien de l’homme : « Être uni à Dieu, voilà mon bien. » Le psaume d’où elle est tirée distingue assez clairement les deux Testaments, l’ancien et le nouveau ; car le prophète dit que la vue de ces impies qui nagent dans l’abondance des biens temporels a fait chanceler ses pas, comme si le culte fidèle qu’il avait rendu à Dieu eût été chose vaine, en présence de la félicité des contempteurs de la loi. Il ajoute qu’il s’est longtemps consumé à comprendre ce mystère, jusqu’au jour où, entré dans le sanctuaire de Dieu, il a vu la fin de cette trompeuse félicité. Il a compris alors que ces hommes, par cela même qu’ils se sont élevés, ont été abaissés, qu’ils ont péri à cause de leurs iniquités, et que ce comble de félicité temporelle a été comme le songe d’un homme qui s’éveille et tout à coup se trouve privé des joies dont le berçait un songe trompeur. Et comme dans cette cité de la terre, ils étaient pleins du sentiment de leur grandeur, le Psalmiste parle ainsi : « Seigneur, vous anéantirez leur image dans votre Cité. » Il montre toutefois combien il lui a été avantageux de n’attendre les biens mêmes de la terre que du seul vrai Dieu, quand il dit : « Je suis devenu semblable, devant vous, à une bête brute, et je demeure toujours avec vous. » Par ces mots, {\itshape semblable à une bête brute}, le Prophète s’accuse de n’avoir pas eu l’intelligence de la parole divine, comme s’il disait : Je ne devais vous demander que les choses qui ne pouvaient m’être communes avec les impies, et non celles dont je les ai vus jouir avec abondance, alors que le spectacle de leur félicité était un scandale à mes faibles yeux. Toutefois le Prophète ajoute qu’il n’a pas cessé d’être avec le Seigneur, parce qu’en désirant les biens temporels il ne les a pas demandés à d’autres que lui. Il poursuit en ces termes « Vous m’avez soutenu par la main droite, me conduisant selon votre volonté, et me faisant marcher dans la gloire » ; marquant par ces mots, {\itshape la main droite}, que tous les biens possédés par les impies, et dont la vue l’avait ébranlé, sont choses de la gauche de Dieu. Puis il s’écrie : « Qu’y a-t-il au ciel et sur la terre que je désire, si ce n’est vous ? » ; il se condamne lui-même ; il se reproche, ayant au ciel un si grand bien, mais dont il n’a eu l’intelligence que plus tard, d’avoir demandé à Dieu des biens passagers, fragiles, et pour ainsi dire une félicité de boue. « Mon cœur et ma chair, dit-il, sont tombés en défaillance, ô Dieu de mon cœur ! » Heureuse défaillance, qui fait quitter les choses de la terre pour celles du ciel ! ce qui lui fait dire ailleurs : « Mon âme, enflammée de désir, tombe en défaillance dans la maison du Seigneur. » Et dans un autre endroit : « Mon âme est tombée en défaillance dans l’attente de votre salut. » Néanmoins, après avoir dit plus haut : {\itshape Mon cœur et ma chair sont tombés en défaillance}, il n’a pas ajouté : {\itshape Dieu de mon cœur et de ma chair}, mais seulement : {\itshape Dieu de mon cœur}, parce que c’est le cœur qui purifie la chair. C’est pourquoi Notre-Seigneur a dit : « Purifiez d’abord le dedans, et le dehors sera pur. » Le Prophète continue et déclare que Dieu même est son partage, et non les biens qu’il a créés : « Dieu de mon cœur, dit-il, Dieu de mon partage pour toujours » ; voulant dire par là que, parmi tant d’objets où s’attachent les préférences des hommes, il trouve Dieu seul digne de la sienne. « Car », poursuit-il, « voilà que ceux qui s’éloignent de vous périssent, et vous avez condamné à jamais toute âme adultère. » Entendez toute âme qui se prostitue à plusieurs dieux. Ici, en effet, se place ce mot qui nous a conduit à citer tout le reste : « Être uni à Dieu, voilà mon bien » ; c’est-à-dire, mon bien est de ne point m’éloigner de Dieu, de ne point me prostituer à plusieurs divinités. Or, en quel temps s’accomplira cette union parfaite avec Dieu ? alors seulement que tout ce qui doit être affranchi en nous sera affranchi. Jusqu’à ce moment, qu’y a-t-il à faire ? ce qu’ajoute le Psalmiste : « Mettre son espérance en Dieu. » Or, comme l’Apôtre nous l’enseigne : « Lorsqu’on voit ce qu’on a espéré, ce n’est plus espérance. Car, qui espère ce qu’il voit déjà ? Mais si nous espérons ce que nous ne voyons pas, nous l’attendons d’un cœur patient. » Soyons donc fermes dans cette espérance, suivons le conseil du Psalmiste et devenons, nous aussi, selon notre faible pouvoir, les anges de Dieu, c’est-à-dire ses messagers, annonçant sa volonté et glorifiant sa gloire et sa grâce : « Afin de chanter vos louanges, ô mon Dieu, devant les portes de la fille de Sion. » Sion, c’est la glorieuse Cité de Dieu, celle qui ne connaît et n’adore qu’un seul Dieu, celle qu’ont annoncée les saints anges qui nous invitent à devenir leurs concitoyens, ils ne veulent pas que nous les adorions comme nos dieux, mais que nous adorions avec eux leur Dieu et le nôtre. Ils ne veulent pas que nous leur offrions des sacrifices, mais que nous soyons comme eux un sacrifice agréable à Dieu. Ainsi donc, quiconque y réfléchira sans coupable obstination, ne doutera pas que tous ces esprits immortels et bienheureux, qui, loin de nous porter envie (car ils ne seraient pas heureux, s’ils étaient envieux), nous aiment au contraire et veulent que nous partagions leur bonheur, ne nous soient plus favorables, si nous adorons avec eux un seul Dieu, Père, Fils et Saint-Esprit, que si nous leur offrions à eux-mêmes notre adoration et nos sacrifices.
\subsection[{Chapitre XXVI}]{Chapitre XXVI}

\begin{argument}\noindent Des contradictions de Porphyre flottant incertain entre la confession du vrai Dieu et le culte des démons.
\end{argument}

\noindent J’ignore comment cela se fait, mais il me semble que Porphyre rougit pour ses amis lesthéurges. Car enfin tout ce que je viens dire, il le savait, mais il n’était pas libre de le maintenir résolument contre le culte de plusieurs dieux. Il dit, en effet, qu’il y a des anges qui descendent ici-bas pour initier les théurges à la science divine, et que d’autres y viennent annoncer la volonté du Père et révéler ses profondeurs. Je demande s’il est croyable que ces anges, dont la fonction est d’annoncer la volonté du Père, veuillent nous forcer à reconnaître un autre Dieu que celui dont ils annoncent la volonté. Aussi Porphyre lui-même nous conseille-t-il excellemment de les imiter plutôt que de les invoquer. Nous ne devons donc pas craindre d’offenser ces esprits bienheureux et immortels, entièrement soumis à un seul Dieu, en ne leur sacrifiant pas ; car ils savent que le sacrifice n’est dû qu’au seul vrai Dieu dont la possession fait leur bonheur, et dès lors ils n’ont garde de le demander pour eux, ni en figure, ni en réalité. Cette usurpation insolente n’appartient qu’aux démons superbes et malheureux, et rien n’en est plus éloigné que la piété des bons anges unis à Dieu sans partage et heureux par cette union. Loin de s’arroger le droit de nous dominer, ils nous aident dans leur bienveillance sincère à posséder le vrai bien et à partager en paix leur propre félicité.\par
Pourquoi donc craindre encore, ô philosophe ! d’élever une voix libre contre des puissances ennemies des vertus véritables et des dons du véritable Dieu ? Déjà tu as su distinguer les anges qui annoncent la volonté de Dieu d’avec ceux qu’appelle je ne sais par quel art l’évocation du théurge. Pourquoi élever ainsi ces esprits impurs à l’insigne honneur de révéler des choses divines ? Et comment seraient-ils les interprètes des choses divines, ceux qui n’annoncent pas la volonté du Père ? Ne sont-ce pas ces mêmes esprits qu’un envieux magicien a enchaînés par ses conjurations pour les empêcher de purifier une âme, sans qu’il fût possible, c’est toi qui le dis, à un théurge vertueux de rompre ces chaînes et de replacer cette âme sous sa puissance ? Quoi ! tu doutes encore que ce ne soient de mauvais démons ! Mais non, tu feins sans doute de l’ignorer ; tu ne veux pas déplaire aux théurges vers lesquels t’a enchaîné une curiosité décevante et qui t’ont transmis comme un don précieux cette sciencepernicieuse et insensée. Oses-tu bien élever au-dessus de l’air et jusqu’aux régions sidérales ces puissances ou plutôt ces pestes moins dignes du nom de souveraines que de celui d’esclaves, et ne vois-tu pas qu’en faire les divinités du ciel, c’est infliger au ciel un opprobre !
\subsection[{Chapitre XXVII}]{Chapitre XXVII}

\begin{argument}\noindent Porphyre s’engage dans l’erreur plus avant qu’Apulée et tombe dans l’impiété.
\end{argument}

\noindent Combien l’erreur d’Apulée, platonicien comme toi, est moins choquante et plus supportable ! Il n’attribue les agitations de l’âme humaine et la maladie des passions qu’aux démons qui habitent au-dessous du globe de la lune, et encore hésite-t-il dans cet aveu qu’il fait touchant des êtres qu’il honore ; quant aux dieux supérieurs, à ceux qui habitent l’espace éthéré, soit visibles, comme le soleil, la lune et les autres astres que nous contemplons au ciel, soit invisibles, comme Apulée en suppose, il s’efforce de les purifier de la souillure des passions. Ce n’est donc pas à l’école de Platon, mais à celle de tes maîtres chaldéens que tu as appris à élever les vices des hommes jusque dans les régions de l’empyrée et sur les hauteurs sublimes du firmament, afin que les théurges aient un moyen d’obtenir des dieux la révélation des choses divines. Et cependant, ces choses divines, tu te mets au-dessus d’elles par ta vie intellectuelle, ne jugeant pas qu’en ta qualité de philosophe les purifications théurgiques te soient nécessaires. Elles le sont aux autres, dis-tu, et afin sans doute de récompenser tes maîtres, tu renvoies aux théurges tous ceux qui ne sont pas philosophes, non pas, il est vrai, pour être purifiés dans la partie intellectuelle de l’âme, car la théurgie, tu l’avoues, ne porte pas jusque-là, mais pour l’être au moins dans la partie spirituelle. Or, comme le nombre des âmes peu capables de philosophie est sans comparaison le plus grand, tes écoles secrètes et illicites seront plus fréquentées que celles de Platon. Ils t’ont sans doute promis, ces démons impurs, qui veulent passer pour des dieux célestes et dont tu t’es fait le messager et lehéraut, ils t’ont promis que les âmes purifiées par la théurgie, sans retourner au Père, à la vérité, habiteraient au-dessus de l’air parmi les dieux célestes. Mais tu ne feras pas accepter ces extravagances à ce nombre immense de fidèles que le Christ est venu délivrer de la domination des démons. C’est en lui qu’ils trouvent la vraie purification infiniment miséricordieuse, celle qui embrasse l’âme, l’esprit et le corps. Car, pour guérir tout l’homme de la peste du péché, le Christ a revêtu sans péché l’homme tout entier. Plût à Dieu que tu l’eusses connu, ce Christ, lui donnant ton âme à guérir plutôt que de te confier en ta vertu, infirme et fragile comme toute chose humaine et en ta pernicieuse curiosité. Celui-là ne t’aurait pas trompé, puisque vos oracles, par toi-même cités, le déclarent saint et immortel. C’est de lui, en effet, que parle le plus illustre des poètes, dans ces vers qui n’ont qu’une vérité prophétique, étant tracés pour un autre personnage, mais qui s’appliquent très bien au Sauveur :\par
 {\itshape « Par toi, s’il reste quelque trace de notre crime, elle s’évanouira, laissant le monde affranchi de sa perpétuelle crainte. »} \par
Par où le poète veut dire qu’à cause de l’infirmité humaine, les plus grands progrès dans la justice laissent subsister, sinon les crimes, au moins de certaines traces que le Sauveur seul peut effacer. Car c’est au Sauveur seul que se rapportent ces vers, et Virgile nous fait assez entendre qu’il ne parle pas en son propre nom par ces mots du début de la même églogue :\par
 {\itshape « Voici qu’est arrivé le dernier âge prédit par la sibylle de Cumes. »} \par
C’est dire ouvertement qu’il va parler d’après la sibylle. Mais les théurges, ou plutôt les démons, qui prennent la figure des dieux, souillent bien plutôt l’âme par leurs vains fantômes qu’ils ne la purifient. Eh ! comment la purifieraient-ils, puisqu’ils sont l’impureté même ! Sans cela, il ne serait pas possible à un magicien envieux de les enchaîner par ses incantations et de les contraindre, soit par crainte, soit par envie, à refuser à une âme souillée le bienfait imaginaire de la purification. Mais il me suffit de ce double aveu queles opérations théurgiques ne peuvent rien sur l’âme intellectuelle, c’est-à-dire sur notre entendement, et que, si elles purifient la partie spirituelle et inférieure de l’âme, elles sont incapables de lui donner l’immortalité et l’éternité. Le Christ, au contraire, promet la vie éternelle, et c’est pourquoi le monde entier court à lui, en dépit de vos colères et en dépit aussi de vos étonnements et de vos stupeurs. À quoi te sert, Porphyre, d’avoir été forcé de convenir que la théurgie est une source d’illusions où le plus grand nombre puise une science aveugle et folle, et que l’erreur la plus certaine, c’est de recourir par des sacrifices aux anges et aux puissances ? Cet aveu à peine fait, comme si tu craignais d’avoir perdu ton temps avec les théurges, tu leur renvoies la masse du genre humain, pour qu’ils aient à purifier dans leur âme spirituelle ceux qui ne savent pas vivre selon leur âme intellectuelle !
\subsection[{Chapitre XXVIII}]{Chapitre XXVIII}

\begin{argument}\noindent Quels conseils ont aveuglé Porphyre et l’ont empêché de connaître la vraie sagesse, qui est Jésus-Christ.
\end{argument}

\noindent Ainsi tu jettes les hommes dans une erreur manifeste, et un si grand mal ne te fait pas rougir, et tu fais profession d’aimer la vertu et la sagesse ! Si tu les avais véritablement aimées, tu aurais connu le Christ, qui est la vertu et la sagesse de Dieu, et l’orgueil d’une science vaine ne t’aurait pas poussé à te révolter contre son humilité salutaire. Tu avoues cependant que l’âme spirituelle elle-même peut être purifiée par la seule vertu de la continence, sans le secours de ces arts théurgiques et de ces télètes où tu as consommé vainement tes études. Tu vas jusqu’à dire quelquefois que les télètes ne sauraient élever l’âme après la mort, de sorte qu’à ce compte la théurgie ne servirait de rien au-delà de cette vie, même pour la partie spirituelle de l’âme ; et cet aveu ne t’empêche pas de revenir en mille façons sur ces pratiques mystérieuses, sans que je puisse te supposer un autre but que de paraître habile en théurgie, de plaire aux esprits déjà séduits par ces arts illicites, et d’en inspirer aux autres la curiosité.\par
Je te sais gré du moins d’avoir déclaré que la théurgie est un art redoutable, soit à cause des lois qui l’interdisent, soit par la nature même de ses pratiques. Et plût à Dieu que cet avertissement fût entendu de ses malheureux partisans et les fit tomber ou s’arrêter devant l’abîme ! Tu dis à la vérité qu’il n’y a point de télètes qui guérissent de l’ignorance et de tous les vices qu’elle amène avec soi, et que cette guérison ne peut s’accomplir que par le {\itshape Patrikon Noun}, c’est-à-dire par l’intelligence du Père, laquelle a conscience de sa volonté ; mais tu ne veux pas croire que le Christ soit cette Intelligence du Père, et tu le méprises à cause du corps qu’il a pris d’une femme et de l’opprobre de la croix ; car ta haute sagesse, dédaignant et rejetant les choses viles, n’aime à s’attacher qu’aux objets les plus relevés. Mais lui, il est venu pour accomplir ce qu’avaient dit de lui les véridiques Prophètes : « Je détruirai la sagesse des sages, et j’anéantirai la prudence des prudents. » Il ne détruit pas en effet, il n’anéantit pas la sagesse qu’il a donnée aux hommes, mais celle qu’ils s’arrogent et qui ne vient pas de lui. Aussi l’Apôtre, après avoir rapporté ce témoignage des Prophètes, ajoute : « Où sont les sages ? où sont les docteurs de la loi ? où sont les esprits curieux des choses du siècle ? Dieu n’a-t-il pas convaincu de folie la sagesse de ce monde ? Car le monde avec sa sagesse n’ayant point reconnu Dieu dans la sagesse de Dieu, il a plu à Dieu de sauver les croyants par la folie de la prédication. Les Juifs demandent des miracles, et les Gentils cherchent la sagesse, et nous, nous prêchons Jésus-Christ crucifié, qui est un scandale pour les Juifs et une folie pour les Gentils, mais qui pour tous les appelés, Juifs ou Gentils, est la vertu et la sagesse de Dieu ; car ce qui paraît folie en Dieu est plus sage que les hommes, et ce qui paraît faible en Dieu est plus puissant que les hommes. » C’est cette folie et cette faiblesse apparentes que méprisent ceux qui se croient forts et sages par leur propre vertu ; mais c’est aussi cette grâce qui guérit les faibles et tous ceux qui, au lieu de s’enivrer d’orgueil dans leur fausse béatitude, confessent leur trop réelle misère d’un cœur plein d’humilité.
\subsection[{Chapitre XXIX}]{Chapitre XXIX}

\begin{argument}\noindent De l’Incarnation de Notre-Seigneur Jésus-Christ repoussée par l’orgueil impie des Platoniciens.
\end{argument}

\noindent Tu reconnais hautement le Père, ainsi que son Fils que tu appelles l’intelligence du Père, et enfin un troisième principe, qui tient le milieu entre les deux autres et où il semble que tu reconnaisses le Saint-Esprit. Voilà, pour dire comme vous, les trois dieux. Si peu exact que soit ce langage, vous apercevez pourtant, comme à travers l’ombre d’un voile, le but où il faut aspirer ; mais le chemin du salut, mais le Verbe immuable fait chair, qui seul peut nous élever jusqu’à ces objets de notre foi où notre intelligence n’atteint qu’à peine, voilà ce que vous ne voulez pas reconnaître. Vous entrevoyez, quoique de loin et d’un œil offusqué par les nuages, la patrie où il faut se fixer ; mais vous ne marchez pas dans la voie qui y conduit. Vous confessez pourtant la grâce, quand vous reconnaissez qu’il a été donné à un petit nombre de parvenir à Dieu par la force de l’intelligence. Tu ne dis pas en effet : {\itshape Il a plu à un petit nombre}, ou bien : {\itshape Un petit nombre a voulu}, mais : {\itshape Il a été donné à un petit nombre}, et en parlant ainsi, tu reconnais expressément l’insuffisance de l’homme et la grâce de Dieu. Tu parles encore de la grâce en termes plus clairs dans ce passage où, commentant Platon, tu affirmes avec lui qu’il est impossible à l’homme de parvenir en cette vie à la perfection de la sagesse, mais que la Providence et la grâce de Dieu peuvent après cette vie achever ce qui manque dans les hommes qui auront vécu selon la raison. Oh ! si tu avais connu la grâce de Dieu par Jésus-Christ Notre-Seigneur, et ce mystère même de l’incarnation où le Verbe a pris l’âme et le corps de l’homme, tu aurais pu y voir le plus haut exemple de la grâce. Mais que dis-je ? et pourquoi parler en vain à un homme qui n’est plus ? mes discours, je le sais, sont perdus pour toi ; mais ils ne le seront pas, j’espère, pour tes admirateurs, pour ces hommes qu’enflamme l’amour de la sagesse ou la curiosité et qui t’aiment ; c’est à eux que je m’adresse en parlant à toi, et peut-être ne sera-ce pas en vain !\par
La grâce de Dieu pouvait-elle se signaler d’une manière plus gratuite qu’en inspirant au Fils unique de Dieu de se revêtir de la nature humaine sans cesser d’être immuable en soi, et de donner aux hommes un gage de son amour dans un homme-Dieu, médiateur entre Dieu et les hommes, entre l’immortel et les mortels, entre l’être immuable et les êtres changeants, entre les justes et les impies, entre les bienheureux et les misérables ? Et comme il a mis en nous le désir naturel du bonheur et de l’immortalité, demeurant lui-même heureux alors qu’il devient mortel pour nous donner ce que nous aimons, il nous a appris par ses souffrances à mépriser ce que nous craignons.\par
Mais pour acquiescer à cette vérité, il vous fallait de l’humilité, et c’est une vertu qu’il est difficile de persuader aux têtes orgueilleuses. Au fond qu’y a-t-il de si incroyable, pour vous surtout, préparés par toute votre doctrine à une telle foi, qu’y a-t-il de si incroyable dans notre dogme de l’incarnation ? Vous avez une idée tellement haute de l’âme intellectuelle, qui est humaine après tout, que vous la croyez consubstantielle à l’intelligence du Père, laquelle est, de votre propre aveu, le Fils de Dieu. Qu’y a-t-il donc à vos yeux de si incroyable à ce que ce Fils de Dieu se soit uni d’une façon ineffable et singulière à une âme intellectuelle pour en sauver une multitude ? Le corps est uni à l’âme, et cette union fait l’homme total et complet ; voilà ce que nous apprend le spectacle de notre propre nature ; et certes, si nous n’étions pas habitués à une pareille union, elle nous paraîtrait plus incroyable qu’aucune autre ; donc l’union de l’homme avec Dieu, de l’être changeant avec l’être immuable, si mystérieuse qu’elle soit, s’opérant entre deux termes spirituels, ou, comme vous dites, incorporels, est plus aisée à croire que l’union d’un esprit incorporel avec un corps. Est-ce la merveille d’un fils ru d’une vierge qui vous choque ? Mais qu’un homme miraculeux naisse d’une manière miraculeuse, il n’y a là rien de choquant, et c’est bien plutôt le sujet d’une pieuse émotion. Serait-ce la résurrection, serait-ce Jésus-Christ quittant son corps pour le reprendre transfiguré et l’emporter incorruptible et immortel dans les régions célestes, serait-ce là le point délicat ? Votre maître Porphyre, en effet, dans ses livres que j’ai déjà souvent cités : {\itshape Du retour de l’âme}, prescrit fortement à l’âme humaine de fuir toute espèce de corps pour être heureuse en Dieu. Mais au lieu de suivre ici Porphyre, vous devriez bien plutôt le redresser, puisque son sentiment est contraire à tant d’opinions merveilleuses que vous admettez avec lui touchant l’âme du monde visible qui anime tout ce vaste univers. Vous dites en effet, sur la foi de Platon, que le monde est un animal très heureux, et vous voulez même qu’il soit éternel ; or, si toute âme, pour être heureuse, doit fuir absolument tout corps, comment se fait-il que, d’une part, l’âme du monde ne doive jamais être délivrée de son corps, et que, de l’autre, elle ne cesse jamais d’être bienheureuse ? Vous reconnaissez de même avec tout le monde que le soleil et les autres astres sont des corps, et vous ajoutez, au nom d’une science, à ce que vous croyez, plus profonde, que ces astres sont des animaux très heureux et éternels. D’où vient, je vous prie, que, lorsqu’on vous prêche la foi chrétienne, vous oubliez ou faites semblant d’oublier ce que vous enseignez tous les jours ? d’où vient que vous refusez d’être chrétiens, sous prétexte de rester fidèles à vos opinions, quand c’est vous-mêmes qui les démentez ? d’où vient cela, sinon de ce que le Christ est venu dans l’humilité et de ce que vous êtes superbes ? On demande de quelle nature seront les corps des saints après la résurrection, et voilà certes une question délicate à débattre entre les chrétiens les plus versés dans les Écritures ; mais ce qui ne fait l’objet d’aucun doute, c’est que les corps des saints seront éternels et semblables au modèle que le Christ en a donné dans sa résurrection glorieuse. Or, quels qu’ils soient, du moment qu’ils seront incorruptibles et immortels, et n’empêcheront point l’âme d’être unie à Dieu par la contemplation, comment pouvez-vous soutenir, vous qui donnez des corps éternels à des êtres éternellement heureux, que l’âme ne peut être heureuse qu’à condition d’être séparée du corps ? Pourquoi vous tourmenter ainsi à chercher un motif raisonnable ou plutôt un prétexte spécieux de fuir la religion chrétienne, si ce n’est, je le répète, que le Christ est humble et que vous êtes orgueilleux ? Avez-vous honte par hasard de vousrétracter ? C’est encore un vice des orgueilleux. Ils rougissent, ces savants hommes, ces disciples de Platon, de devenir disciples de ce Jésus-Christ qui a mis dans la bouche d’un simple pêcheur pénétré de son esprit cette parole : « Au commencement était le Verbe, et le Verbe était en Dieu, et le Verbe était Dieu. Il était au commencement en Dieu. Toutes choses ont été faites par lui, et rien de ce qui a été fait n’a été fait sans lui. Ce qui a été fait était vie en lui, et la vie était la lumière des hommes, et la lumière luit dans les ténèbres, et les ténèbres ne l’ont point comprise. » Voilà ce début de l’Évangile de saint Jean, qu’un philosophe platonicien aurait voulu voir écrit en lettres d’or dans toutes les églises au lieu le plus apparent, comme aimait à nous le raconter le saint vieillard Simplicien, qui a été depuis évêque de Milan. Mais les superbes ont dédaigné de prendre ce Dieu pour maître, parce qu’{\itshape il s’est fait chair et a habité parmi nous} ; de sorte que c’est peu d’être malade pour ces misérables, il faut encore qu’ils se glorifient de leur maladie et qu’ils rougissent du médecin qui seul pourrait les guérir. Ils travaillent pour s’élever et n’aboutissent qu’à se préparer une chute plus terrible.
\subsection[{Chapitre XXX}]{Chapitre XXX}

\begin{argument}\noindent Sur combien de points Porphyre a réfuté et corrigé la doctrine de Platon.
\end{argument}

\noindent Si l’on croit qu’après Platon il n’y a rien à changer en philosophie, d’où vient que sa doctrine a été modifiée par Porphyre en plusieurs points qui ne sont pas de peu de conséquence ? Par exemple, Platon a écrit, cela est certain, que les âmes des hommes reviennent après la mort sur la terre, et jusque dans le corps des bêtes. Cette opinion a été adoptée par Plotin, le maître de Porphyre. Eh bien I Porphyre l’a condamnée, et non sans raison. Il a cru avec Platon que les âmes humaines retournent dans de nouveaux corps, mais dans des corps humains, de peur, sans doute, qu’il n’arrivât à une mère devenue mule de servir de monture à son enfant. Porphyre oublie parmalheur que dans son système une mère devenue jeune fille est exposée à rendre son fils incestueux. Combien est-il plus honnête de croire ce qu’ont enseigné les saints anges, les Prophètes inspirés du Saint-Esprit et les Apôtres envoyés par toute la terre : que les âmes, au lieu de retourner tant de fois dans des corps différents, ne reviennent qu’une seule fois et dans leur propre corps ? Il est vrai cependant que Porphyre a très fortement corrigé l’opinion de Platon, en admettant seulement la transmigration des âmes humaines dans des corps humains, et en refusant nettement de les emprisonner dans des corps de bêtes. Il dit encore que Dieu amis l’âme dans le monde pour que, voyant les maux dont la matière est le principe, elle retournât au Père et fût affranchie pour jamais d’une semblable contagion. Encore qu’il y ait quelque chose à reprendre dans cette opinion (car l’âme a été mise dans le corps pour faire le bien, et elle ne connaîtrait point le mal, si elle ne le faisait pas), Porphyre a néanmoins amendé sur un point considérable la doctrine des autres Platoniciens, quand il a reconnu que l’âme purifiée de tout mal et réunie au Père serait éternellement à l’abri des maux d’ici-bas. Par là, il a renversé ce dogme éminemment platonicien, que les vivants naissent toujours des morts, comme les morts des vivants ; par là il a convaincu de fausseté cette tradition, empruntée, à ce qu’il semble, par Virgile au platonisme, que les âmes devenues pures sont envoyées aux Champs-Élysées (symbole des joies des bienheureux), après avoir bu dans les eaux du Léthé l’oubli du passé :\par
 {\itshape « Afin, dit le poète, que dégagées de tout souvenir elles consentent à revoir la voûte céleste et à recommencer dans des corps une vie nouvelle. »} \par
Porphyre a justement répudié cette doctrine ; car il est vraiment absurde que les âmes désirent quitter une vie où elles ne pourraient être bienheureuses qu’avec la certitude d’y persévérer toujours, et cela pour retourner en ce monde et rentrer dans des corps corruptibles, comme si leur suprême purification ne faisait que rendre nécessaire une nouvelle souillure. Dire que la purification efface réellement de leur mémoire tous les maux passés, et ajouter que cet oubli les porteà désirer de nouvelles épreuves, c’est dire que la félicité suprême est cause de l’infélicité, la perfection de la sagesse cause de la folie, et la pureté la plus haute cause de l’impureté. De plus, ce bonheur de l’âme pendant son séjour dans l’autre monde ne sera pas fondé sur la vérité, si elle ne peut le posséder qu’en étant trompée. Or, elle ne peut avoir le bonheur qu’avec la sécurité, et elle ne peut avoir la sécurité qu’en se croyant heureuse pour toujours, sécurité fausse, puisqu’elle redeviendra bientôt misérable. Comment donc sera-t-elle heureuse dans la vérité, si la cause de sa joie est une fausseté ? Voilà ce qui n’a pas échappé à Porphyre, et c’est pourquoi il a soutenu que l’âme purifiée retourne au Père, pour y être affranchie à jamais de la contagion du mal. D’où il faut conclure que cette doctrine de quelques Platoniciens sur la révolution nécessaire qui emporte les âmes hors du monde et les y ramène est une erreur. Au surplus, alors même que la transmigration serait vraie, à quoi servirait de le savoir ? Les Platoniciens chercheraient-ils à prendre avantage sur nous de ce que nous ne saurions pas en cette vie ce qu’ils ignoreraient eux-mêmes dans une vie meilleure, où, malgré toute leur pureté et toute leur sagesse, ils ne seraient bienheureux qu’en étant trompés ? Mais quoi de plus absurde et de plus insensé ! Il est donc hors de doute que le sentiment de Porphyre est préférable à cette théorie d’un cercle dans la destinée des âmes, alternative éternelle de misère et de félicité. Voilà donc un platonicien qui se sépare de Platon pour penser mieux que lui, qui a vu ce que Platon ne voyait pas, et qui n’a pas hésité à corriger un si grand maître, préférant à Platon la vérité.
\subsection[{Chapitre XXXI}]{Chapitre XXXI}

\begin{argument}\noindent Contre les Platoniciens qui font l’âme coéternelle à Dieu.
\end{argument}

\noindent Pourquoi ne pas s’en rapporter plutôt à la Divinité sur ces problèmes qui passent la portée de l’esprit humain ? pourquoi ne pas croire à son témoignage, quand elle nous dit que l’âme elle-même n’est point coéternelle à Dieu, mais qu’elle a été créée et tirée du néant ? La seule raison invoquée par les Platoniciens à l’appui de l’éternité de l’âme, c’est que si elle n’avait pas toujours existé, elle ne pourrait pas durer toujours, Or, il se trouve que Platon, dans l’ouvrage où il décrit le monde et les dieux secondaires qui sont l’ouvrage de Dieu, affirme en termes exprès que leur être a eu un commencement et qu’il n’aura pourtant pas de fin, parce que la volonté toute-puissante du Créateur les fait subsister pour l’éternité. Pour expliquer cette doctrine, les Platoniciens ont imaginé de dire qu’il ne s’agit pas d’un commencement de temps, mais d’un commencement de cause. « Il en est, disent-ils, comme d’un pied qui serait de toute éternité posé sur la poussière ; l’empreinte existerait toujours au-dessous, et cependant elle est faite par le pied, de sorte que le pied n’existe pas avant l’empreinte, bien qu’il la produise. C’est ainsi, à les entendre, que le monde et les dieux créés dans le monde ont toujours été, leur créateur étant toujours, et cependant ils sont faits par lui. » Je demanderai à ceux qui soutiennent que l’âme a toujours été, si elle a toujours été misérable ? Car s’il est quelque chose en elle qui ait commencé d’exister dans le temps et qui ne s’y rencontrât pas de toute éternité, pourquoi elle-même n’aurait-elle pas commencé d’exister dans le temps ? D’ailleurs, la béatitude dont elle jouit, de leur propre aveu, sans mesure et sans fin après les maux de cette vie, a évidemment commencé dans le temps, et toutefois elle durera toujours. Que devient donc cette argumentation destinée à établir que rien ne peut durer sans fin que ce qui existe sans commencement ? La voilà qui tombe en poussière, en se heurtant contre cette félicité qui a un commencement et qui n’aura pas de fin. Que l’infirmité humaine cède donc à l’autorité divine ! Croyons-en sur la religion ces esprits bienheureux et immortels qui ne demandent pas qu’on leur rende les honneurs faits pour Dieu seul, leur maître et le nôtre, et qui n’ordonnent d’offrir le sacrifice, comme je l’ai déjà dit et ne puis trop le redire, qu’à celui dont nous devons être avec eux le sacrifice ; immolation salutaire offerte à Dieu par ce même prêtre qui, en revêtant la nature humaine selon laquelle il a voulu être prêtre, s’est offert lui-même en sacrifice pour nous.
\subsection[{Chapitre XXXII}]{Chapitre XXXII}

\begin{argument}\noindent La voie universelle de la délivrance de l’âme nous est ouverte par la seule grâce du Christ.
\end{argument}

\noindent Voilà cette religion qui nous ouvre la voie universelle de la délivrance de l’âme, voie unique, voie vraiment royale, par où on arrive à un royaume qui n’est pas chancelant comme ceux d’ici-bas, mais qui est appuyé sur le fondement inébranlable de l’éternité. Et quand Porphyre, vers la fin de son premier livre {\itshape Du retour de l’âme}, assure que la voie universelle de la délivrance de l’âme n’a encore été indiquée, à sa connaissance, par aucune secte, qu’il ne la trouve ni dans la philosophie la plus vraie, ni dans la doctrine et les règles morales des Indiens, ni dans les systèmes des Chaldéens, en un mot dans aucune tradition historique, cela revient à avouer que cette voie existe, mais qu’il n’a pu encore la découvrir. Ainsi, toute cette science si laborieusement acquise, tout ce qu’il savait ou paraissait savoir sur la délivrance de l’âme, ne le satisfaisait nullement. Il sentait qu’en si haute matière il lui manquait une grande autorité devant laquelle il fallût se courber. Quand donc il déclare que, même dans la philosophie la plus vraie, il ne trouve pas la voie universelle de la délivrance de l’âme, il montre assez l’une de ces deux choses ou que la philosophie dont il faisait profession n’était pas la plus vraie, ou qu’elle ne fournissait pas cette voie. Et, dans ce dernier cas, comment pouvait-elle être vraie, puisqu’il n’y a pas d’autre voie universelle de l’âme que celle par laquelle toutes les âmes sont délivrées et sans laquelle par conséquent aucune âme n’est délivrée ? Quand il ajoute que cette vote ne se rencontre « ni dans la doctrine et les règles morales des Indiens, ni dans les systèmes des Chaldéens, ni ailleurs », il montre, par le témoignage le plus éclatant, qu’il a étudié sans en être satisfait les doctrines de l’Inde et de la Chaldée, et qu’il a notamment emprunté aux Chaldéens ces oracles divins qu’il ne cesse de mentionner, Quelle est donc cette voie universelle de la délivrance de l’âme dont parle Porphyre, et qui, selon lui, ne se trouve nulle part, pas même parmi ces nations qui ont dû leur célébrité dans la science des choses divines à leur culte assidu et curieux des bons et des mauvais anges ? quelle est cette voie universelle, sinon celle qui n’est point particulière à une nation, mais qui a été divinement ouverte à tous les peuples du monde ? Et remarquez que ce grand esprit n’en conteste pas l’existence, étant convaincu que la Providence n’a pu laisser les hommes privés de ce secours. Il se borne à dire que la voie universelle de la délivrance de l’âme n’est point encore arrivée à sa connaissance, et le fait n’a rien de surprenant ; car Porphyre vivait dans un temps où Dieu permettait que la voie tant cherchée, qui n’est autre que la religion chrétienne, fût envahie par les idolâtres et par les princes de la terre ; épreuve nécessaire, qui devait accomplir et consacrer le nombre des martyrs, c’est-à-dire des témoins de la vérité, destinés à faire éclater par leur constance l’obligation où sont les chrétiens de souffrir toutes sortes de maux pour la défense de la vraie religion. Porphyre était témoin de ce spectacle et ne pouvait croire qu’une religion, qui lui semblait condamnée à périr, fût la voie universelle de la délivrance de l’âme ; ces persécutions dont la vue effrayante le détournait du christianisme, il ne comprenait pas qu’elles servaient à son triomphe et qu’il allait en sortir plus fort et plus glorieux.\par
Voilà donc la voie universelle de la délivrance de l’âme ouverte à tous les peuples de l’univers par la miséricorde divine, et comme les desseins de Dieu sont au-dessus de la portée humaine, en quelque lieu que cette voie soit aujourd’hui connue ou doive l’être un jour, nul n’a droit de dire : Pourquoi sitôt ? pourquoi si tard ? Porphyre lui-même en a senti la raison, quand, après avoir dit que ce don de Dieu n’avait pas encore été reçu et n’était pas jusque-là venu à sa connaissance, il se garde d’en conclure qu’il n’existe pas. Voilà, je le répète, la voie universelle de la délivrance de tous les croyants, qui fut ainsi annoncée par le ciel au fidèle Abraham : « Toutes les nations seront bénies en votre semence. » Abraham était chaldéen, à la vérité ; mais afin qu’il pût recevoir l’effet de ces promesses et qu’il sortît de lui une race disposée par les anges dans la main d’un médiateur en quidevait se trouver cette voie universelle de la délivrance de l’âme, il lui fut ordonné d’abandonner son pays, ses parents et la maison de son père. Alors Abraham, délivré des superstitions des Chaldéens, adora le seul vrai Dieu et ajouta foi à ses promesses. La voilà cette voie universelle dont le Prophète a dit : « Que Dieu ait pitié de nous et qu’il nous bénisse ; qu’il fasse luire sur nous-la lumière de son visage, et qu’il nous soit miséricordieux, afin que nous connaissions votre voie sur la terre et le salut que vous envoyez à toutes les nations. » Voilà pourquoi le Sauveur, qui prit chair si longtemps après de la semence d’Abraham, a dit de soi-même : « Je suis la voie, la vérité et la vie. » C’est encore cette voie universelle dont un autre prophète a parlé en ces termes, tant de siècles auparavant : « Aux derniers temps, la montagne de la maison du Seigneur paraîtra sur le sommet des montagnes et sera élevée par-dessus toutes les collines. Tous les peuples y viendront, et les nations y accourront et diront : Venez, montons sur la montagne du Seigneur et dans la maison du Dieu de Jacob ; il nous enseignera sa voie et nous marcherons dans ses sentiers ; car la loi sortira de Sion, et la parole du Seigneur, de Jérusalem. » Cette voie donc n’est pas pour un seul peuple, mais pour toutes les nations ; et la loi et la parole du Seigneur ne sont pas demeurées dans Sion et dans Jérusalem ; mais elles en sont sorties pour se répandre par tout l’univers. Le Médiateur même, après sa résurrection, dit par cette raison à ses disciples, que sa mort avait troublés : « Il fallait que tout ce qui est écrit de moi, dans la loi, dans les prophètes et dans les psaumes, fût accompli. Alors il leur ouvrit l’esprit pour entendre les Écritures, et il leur dit : “Il fallait que le Christ souffrît et qu’il ressuscitât d’entre les morts le troisième jour, et que l’on prêchât en son nom la pénitence et la rémission des péchés parmi toutes les nations, à commencer par Jérusalem.” » La voilà donc cette voie universelle de la délivrance de l’âme, que les saints anges et les saints prophètes ont d’abord figurée partout où ils ont pu, dans le petit nombre de personnes en qui ils ont honoré la grâce de Dieu, et surtout dans les Hébreux, dont la républiqueétait comme consacrée pour la prédication de la Cité de Dieu chez toutes les nations de la terre : ils l’ont figurée par le tabernacle, par le temple, par le sacerdoce et par les sacrifices ; ils l’ont prédite par des prophéties, quelquefois claires et plus souvent obscures et mystérieuses ; mais quand le Médiateur lui-même, revêtu de chair, et ses bienheureux Apôtres ont manifesté la grâce du Nouveau Testament, ils ont fait connaître plus clairement cette voie qui avait été cachée dans les ombres des siècles précédents, quoiqu’il ait toujours plu à Dieu de la faire entrevoir en tous temps, comme je l’ai montré plus haut, par des signes miraculeux de sa puissance. Les anges ne sont pas seulement apparus comme autrefois, mais, à la seule voix des serviteurs de Dieu agissant d’un cœur simple, les esprits immondes ont été chassés du corps des possédés, les estropiés et les malades guéris ; les bêtes farouches de la terre et des cieux, les oiseaux du ciel, les arbres, les éléments, les astres ont obéi à leurs ordres ; l’enfer a cédé à leur pouvoir et les morts sont ressuscités. Et je ne parle point des miracles particuliers au Sauveur, tels surtout que sa naissance, où s’accomplit le mystère de la virginité de sa mère, et sa résurrection, type de notre résurrection à venir. Je dis donc que cette voie conduit à la purification de l’homme tout entier, et, de mortel qu’il était, le dispose en toutes ses parties à devenir immortel. Car afin que l’homme ne cherchât point divers modes de purification, l’un pour la partie que Porphyre appelle intellectuelle, l’autre pour la partie spirituelle, un autre enfin pour le corps, le Sauveur et purificateur véritable et tout-puissant a revêtu l’homme tout entier. Hors de cette voie, qui jamais n’a fait défaut aux hommes, soit au temps des promesses, soit au temps de l’accomplissement, nul n’a été délivré, nul n’est délivré, nul ne sera délivré.\par
Porphyre nous dit que la voie universelle de la délivrance de l’âme n’est point encore venue à sa connaissance par aucune tradition historique ; mais peut-on trouver une histoire à la fois plus illustre et plus fidèle que celle du Sauveur, laquelle a conquis une si grande autorité par toute la terre, et où les choses passées sont racontées de manière à prédire les choses futures, dont un grand nombre déjà accompli nous garantit l’accomplissement des autres ? Ni Porphyre ni les autres Platoniciens ne peuvent être reçus à mépriser ces prophéties, comme ne concernant que des choses passagères et relatives à cette vie mortelle. Ils ont raison, sans nul doute, pour des prédictions d’une autre sorte celles qui s’obtiennent par la divination et par d’autres arts. Que ces prédictions et ceux qui les cultivent ne méritent pas grande estime, j’y consens volontiers ; car elles se font soit par la prénotion des causes inférieures, comme dans la médecine, où l’on peut prévoir divers accidents de la maladie à l’aide des signes qui la précèdent, soit parce que les démons prédisent ce qu’ils ont résolu de faire, et se servent pour l’exécuter des passions déréglées des méchants, de manière à persuader que les événements d’ici-bas sont entre leurs mains. Les saints qui ont marché dans la voie universelle de la délivrance de l’âme ne se sont point souciés de faire de telles prédictions, comme si elles avaient une grande importance ; et ce n’est pas qu’ils aient ignoré les événements de cet ordre, puisqu’ils en ont souvent prédit à l’appui de vérités plus hautes, supérieures aux sens et aux vérifications de l’expérience ; mais il avait d’autres événements véritablement grands et divins qu’ils annonçaient selon les lumières qu’il plaisait à Dieu de leur départir. En effet, l’incarnation de Jésus-Christ et toutes les merveilles qui ont éclaté en lui, ou qui ont été accomplies en son nom, telles que la pénitence des hommes plongés en toutes sortes de crimes, la conversion des volontés à Dieu, la rémission des péchés, la grâce justifiante, la foi des âmes pieuses et cette multitude d’hommes qui croient au vrai Dieu par toute la terre, la destruction du culte des idoles et des démons, les tentations qui éprouvent les fidèles, les lumières qui éclairent et purifient ceux qui font des progrès dans la vertu, la délivrance de tous les maux, le jour du jugement, la résurrection des morts, la damnation éternelle des impies et le royaume immortel de cette glorieuse Cité de Dieu destinée à jouir éternellement de la contemplation bienheureuse, tout cela a été prédit et promis dans les Écritures de cette voie sainte, et nous voyons accomplies un si grand nombre de ces promesses que nous avons une pieuse confiance dans l’accomplissement de toutes les autres. Quant à ceux qui ne croient pas et par suite ne comprennent pas que cette voie est la voie droite pour parvenir à la contemplation et à l’union bienheureuses, selon la parole et le témoignage véridiques des saintes Écritures, ils peuvent bien combattre la religion, mais ils ne l’abattront jamais.\par
C’est pourquoi dans ces dix livres, inférieurs sans doute à l’attente de plusieurs, mais où j’ai répondu peut-être au vœu de quelques-uns, dans la mesure où le vrai Dieu et Seigneur a daigné me prêter son aide, j’ai combattu les objections des impies qui préfèrent leurs dieux au fondateur de la Cité sainte. De ces dix livres, les cinq premiers sont contre ceux qui croient qu’on doit adorer les dieux en vue des biens de cette vie, les cinq derniers contre ceux qui veulent conserver le culte des dieux en vue des biens de la vie à venir. Il me reste à traiter, comme je l’ai promis dans le premier livre, des deux cités qui sont ici-bas mêlées et confondues. Je vais donc, si Dieu me continue son appui, parler de leur naissance, de leur progrès et de leur fin.
\section[{Livre onzième. Origine des deux Cités}]{Livre onzième. \\
Origine des deux Cités}\renewcommand{\leftmark}{Livre onzième. \\
Origine des deux Cités}

\subsection[{Chapitre premier}]{Chapitre premier}

\begin{argument}\noindent Objet de cette partie de notre ouvrage où nous commençons d’exposer l’origine et la fin des deux Cités.
\end{argument}

\noindent Nous appelons Cité de Dieu celle à qui rend témoignage cette Écriture dont l’autorité divine s’est assujettie toutes sortes d’esprits, non par le caprice des volontés humaines, mais par la disposition souveraine de la providence de Dieu. « On a dit de toi des choses glorieuses, Ô Cité de Dieu ! » Et dans un autre psaume : « Le Seigneur est grand et digne des plus hautes louanges dans la Cité de notre Dieu et sur sa montagne sainte, d’où il accroît les allégresses de toute la terre. » Et un peu après : « Ce que nous avions entendu, nous l’avons vu dans la Cité du Seigneur des armées, dans la Cité de notre Dieu ; Dieu l’a fondée pour l’éternité. » Et encore dans un autre psaume : « Un torrent de joie inonde la Cité de Dieu ; le Très-Haut a sanctifié son tabernacle ; Dieu est au milieu d’elle, elle ne sera point ébranlée. » Ces témoignages, et d’autres semblables qu’il serait trop long de rapporter, nous apprennent qu’il existe une Cité de Dieu dont nous désirons être citoyens par l’amour que son fondateur nous a inspiré. Les citoyens de la Cité de la terre préfèrent leurs divinités à ce fondateur de la Cité sainte, faute de savoir qu’il est le Dieu des dieux, non des faux dieux, c’est-à-dire des dieux impies et superbes, qui, privés de la lumière immuable et commune à tous, et réduits à une puissance stérile, s’attachent avec fureur à leurs misérables privilèges pour obtenir des honneurs divins de ceux qu’ils ont trompés et assujettis, mais des dieux saints et pieux qui aiment mieux rester soumis à un seul que de se soumettre aux autres et adorer Dieu que d’être adorés en sa place. J’ai répondu aux ennemis de cette sainte Cité dans les livresprécédents, selon les forces que m’a données le Seigneur ; je dois maintenant, avec son secours, exposer, ainsi que je l’ai promis, la naissance, le progrès et la fin des deux Cités, de celle de la terre et de celle du ciel, toujours mêlées ici-bas. Voyons d’abord comment elles ont préexisté dans la diversité des anges.
\subsection[{Chapitre II}]{Chapitre II}

\begin{argument}\noindent Personne ne peut arriver à la connaissance de Dieu que par Jésus-Christ homme, médiateur entre Dieu et les hommes.
\end{argument}

\noindent C’est chose difficile et fort rare, après avoir considéré toutes les créatures corporelles et incorporelles, et reconnu leur instabilité, de s’élever au-dessus d’elles pour contempler la substance immuable de Dieu et apprendre de lui-même que nul autre que lui n’a créé tous les êtres qui diffèrent de lui. Car pour cela Dieu ne parle pas à l’homme par le moyen de quelque créature corporelle, comme une voix qui se fait entendre aux oreilles en frappant l’air interposé entre celui qui parle et celui qui écoute, ni par quelque image spirituelle, telle que celles qui se présentent à nous dans nos songes et qui ont beaucoup de ressemblance avec les corps, mais il parle par la vérité même, dont l’esprit seul peut entendre ce langage. Il s’adresse à ce que l’homme a de plus excellent et en quoi il ne reconnaît que Dieu qui lui soit supérieur. L’homme, en effet, ainsi que l’enseigne la saine raison, ou à défaut d’elle, la foi, ayant été créé à l’image de Dieu, il est hors de doute qu’il approche d’autant plus de Dieu qu’il s’élève davantage au-dessus des bêtes par cette partie de lui-même supérieure à celles qui sont communes à la bête et à l’homme. Mais comme ce même esprit, naturellement doué de raison et d’intelligence, se trouve incapable, au milieu des vices invétérés qui l’offusquent, non seulement de jouir de cette lumière immuable, mais même d’en soutenir l’éclat, jusqu’à ce que sa lente et successive guérison le renouvelle et le rende capable d’une si grande félicite, ii fallait qu’au préalable il fût pénétré et purifié par la foi. Et afin que par elle il marchât d’un pas plus ferme vers la vérité, la Vérité même, c’est-à-dire Dieu, Fils de Dieu, fait homme sans cesser d’être Dieu, a fondé et établi cette foi qui ouvre à l’homme la voie du Dieu de l’homme par l’homme-Dieu ; car c’est Jésus-Christ homme qui est médiateur entre Dieu et les hommes, et c’est comme homme qu’il est notre médiateur aussi bien que notre voie. En effet, quand il y a une voie entre celui qui marche et le lieu où il veut aller, il peut espérer d’aboutir ; mais quand il n’y en a point ou quand il l’ignore, à quoi lui sert de savoir où il faut aller ? Or, pour que l’homme ait une voie assurée vers le salut, il faut que le même principe soit Dieu et homme tout ensemble ; on va à lui comme Dieu, et comme homme, on va par lui.
\subsection[{Chapitre III}]{Chapitre III}

\begin{argument}\noindent De l’autorité de l’Écriture canonique, ou visage de l’esprit divin.
\end{argument}

\noindent Ce Dieu, après avoir parlé autant qu’il l’a jugé à propos, d’abord par les Prophètes, ensuite par lui-même et en dernier lieu par les Apôtres, a fondé en outre l’Écriture, dite canonique, laquelle a une autorité si haute et s’impose à notre foi pour toutes les choses qu’il ne nous est pas bon d’ignorer et que nous sommes incapables de savoir par nous-mêmes. Aussi bien, s’il nous est donné de connaître directement les objets qui tombent sous nos sens, il n’en est pas de même pour ceux qui sont placés au-delà de leur portée, et alors il nous faut bien recourir à d’autres moyens d’information et nous en rapporter aux témoins. Eh bien ! ce que nous faisons pour les objets des semis, nous devons aussi le faire pour les objets de l’intelligence ou du sens intellectuel. Et par conséquent, nous ne saurions nous empêcher d’ajouter foi, pour les choses invisibles qui ne tombent point sous les sens extérieurs, aux saints qui les ont vues ou aux anges qui les voient sans cesse dans la lumière immuable et incorporelle.
\subsection[{Chapitre IV}]{Chapitre IV}

\begin{argument}\noindent Le monde n’a pas été créé de toute éternité, sans qu’on puisse dire qu’en le créant Dieu ait fait succéder une volonté nouvelle à une autre volonté antérieure.
\end{argument}

\noindent Le monde est le plus grand de tous les êtres visibles, comme le plus grand de tous les invisibles est Dieu ; mais nous voyons le monde et nous croyons que Dieu est. Or, que Dieu ait créé le monde, nous n’en pouvons croire personne plus sûrement que Dieu même, qui dit dans les Écritures saintes par la bouche du Prophète : « Dans le principe, Dieu créa le ciel et la terre. » Il est incontestable que le Prophète n’assistait pas à cette création mais la sagesse de Dieu, par qui toutes choses ont été faites, était présente ; et c’est elle qui pénètre les âmes des saints, les fait amis et prophètes de Dieu, et leur raconte ses œuvres intérieurement et sans bruit. Ils conversent aussi avec les anges de Dieu, qui voient toujours la face du Père et qui annoncent sa volonté à ceux qui leur sont désignés. Du nombre de ces prophètes était celui qui a écrit : « Dans le principe, Dieu créa le ciel et la terre », et nous devons d’autant plus l’en croire que le même Esprit qui lui a révélé cela lui a fait prédire aussi, tant de siècles à l’avance, que nous y ajouterions foi.\par
Mais pourquoi a-t-il plu au Dieu éternel de faire alors le ciel et la terre que jusqu’alors il n’avait pas faits ? Si ceux qui élèvent cette objection veulent prétendre que le monde est éternel et sans commencement, et qu’ainsi Dieu ne l’a point créé, ils s’abusent étrangement et tombent dans une erreur mortelle. Sans parler des témoignages des Prophètes, le monde même proclame en silence, par ses révolutions si régulières et par la beauté de toutes les choses visibles, qu’il a été créé, et qu’il n’a pu l’être que par un Dieu dont la grandeur et la beauté sont invisibles et ineffables. Quant à ceux qui, tout en avouant qu’il est l’ouvrage de Dieu, ne veulent pas lui reconnaître un commencement de durée, mais un simple commencement de création, ce qui se terminerait à dire d’unefaçon presque inintelligible que le monde a toujours été fait, ils semblent, il est vrai, mettre par là Dieu à couvert d’une témérité fortuite, et empêcher qu’on ne croie qu’il ne lui soit venu tout d’un coup quelque chose en l’esprit qu’il n’avait pas auparavant, c’est-à-dire une volonté nouvelle de créer le monde, à lui qui est incapable de tout changement ; mais je ne vois pas comment cette opinion peut subsister à d’autres égards et surtout à l’égard de l’âme. Soutiendront-ils qu’elle est coéternelle à Dieu ? mais comment expliquer alors d’où lui est survenue une nouvelle misère qu’elle n’avait point eue pendant toute l’éternité ? En effet, s’ils disent qu’elle a toujours été dans une vicissitude de félicité et de misère, il faut nécessairement qu’ils disent qu’elle sera toujours dans cet état ; d’où s’ensuivra cette absurdité qu’elle est heureuse sans l’être, puisqu’elle prévoit sa misère et sa difformité à venir. Et si elle ne la prévoit pas, si elle croit devoir être toujours heureuse, elle n’est donc heureuse que parce qu’elle se trompe, ce que l’on ne peut avancer sans extravagance. S’ils disent que dans l’infinité des siècles passés elle a parcouru une continuelle alternative de félicité et de misère, mais qu’immédiatement après sa délivrance elle ne sera plus sujette à cette vicissitude, il faut donc toujours qu’ils tombent d’accord qu’elle n’a jamais été vraiment heureuse, qu’elle commencera à l’être dans la suite, et qu’ainsi il lui surviendra quelque chose de nouveau et une chose extrêmement importante qui ne lui était jamais arrivée dans toute l’éternité. Nier que la cause de cette nouveauté n’ait toujours été dans les desseins éternels de Dieu, c’est nier que Dieu soit l’auteur de sa béatitude : sentiment qui serait d’une horrible impiété. S’ils prétendent d’un autre côté que Dieu a voulu, par un nouveau dessein, que l’âme soit désormais éternellement bienheureuse, comment le défendront-ils de cette mutabilité dont ils avouent eux-mêmes qu’il est exempt ? Enfin, s’ils confessent qu’elle a été créée dans le temps, mais qu’elle subsistera éternellement, comme les nombres qui ont un commencement et point de fin, et qu’ainsi, après avoir éprouvé la misère, ellen’y retombera plus, lorsqu’elle sera une fois délivrée, ils avoueront sans doute aussi que cela se fait sans qu’il arrive aucun changement dans les desseins immuables de Dieu. Qu’ils croient donc de même que le monde a pu être créé dans le temps, sans que Dieu en le créant ait changé de dessein et de volonté.
\subsection[{Chapitre V}]{Chapitre V}

\begin{argument}\noindent Il ne faut pas plus se figurer des temps infinis avant le monde que des lieux infinis au-delà du monde.
\end{argument}

\noindent D’ailleurs, que ceux qui, admettant avec nous un Dieu créateur, ne laissent pas de nous faire des difficultés sur le moment où a commencé la création, voient comment ils nous satisferont eux-mêmes touchant le lieu où le monde a été créé. De même qu’ils veulent que nous leur disions pourquoi il a été créé à un certain moment plutôt qu’auparavant, nous pouvons leur demander pourquoi il a été créé où il est plutôt qu’autre part. En effet, s’ils s’imaginent avant le monde des espaces infinis de temps, où il ne leur semble pas possible que Dieu soit demeuré sans rien faire, qu’ils s’imaginent donc aussi hors du monde des espaces infinis de lieux ; et si quelqu’un juge impossible que le Tout-Puissant soit resté oisif au milieu de tous ces espaces sans bornes, ne sera-t-il pas obligé d’imaginer, comme Épicure, une infinité de mondes, avec cette seule différence qu’Épicure veut qu’ils soient formés et détruits par le concours fortuit des atomes, au lieu que ceux-ci diront, selon leurs principes, que tous ces mondes sont l’ouvrage de Dieu et ne peuvent être détruits. Car il ne faut pas oublier que nous discutons ici avec des philosophes persuadés comme nous que Dieu est incorporel et qu’il a créé tout ce qui n’est pas lui. Quant aux autres, ils ne méritent pas d’avoir part à une discussion religieuse, et si les adversaires que nous avons choisis ont surpassé tous les autres en gloire et en autorité, c’est uniquement pour avoir approché de plus près de la vérité, quoiqu’ils en soient encore fort éloignés. Diront-ils donc que la substance divine, qu’ils ne limitent à aucun lieu, mais qu’ils reconnaissent être tout entière partout (sentiment bien digne de la divinité), est absente de ces grands espaces qui sont hors du monde, et n’occupe que le petit espace où le monde est placé ? Je ne pense pas qu’ils soutiennent une opinion aussi absurde. Puis donc qu’ils disent qu’il n’y a qu’un seul monde, grand à la vérité, mais fini néanmoins et compris dans un certain espace, et que c’est Dieu qui l’a créé, qu’ils se fassent à eux-mêmes touchant les temps infinis qui ont précédé le monde, quand ils demandent pourquoi Dieu y est demeuré sans rien faire, la réponse qu’ils font aux autres touchant les lieux infinis qui sont hors du monde, quand on leur demande pourquoi Dieu n’y fait rien. De même, en effet, qu’il ne s’ensuit pas, de ce que Dieu a choisi pour créer le monde un lieu que rien ne rendait plus digne de ce choix que tant d’autres espaces en nombre infinis, que cela soit arrivé par hasard, quoique nous n’en puissions pénétrer la raison, de même on ne peut pas dire qu’il soit arrivé quelque chose de fortuit en Dieu, parce qu’il a fixé à la création un temps plutôt qu’un autre. Que s’ils disent que c’est une rêverie de s’imaginer qu’il y ait hors du monde des lieux infinis, n’y ayant point d’autre lieu que le monde, nous disons de même que c’est une chimère de s’imaginer qu’il y ait eu avant le monde des temps infinis où Dieu soit demeuré sans rien faire, puisqu’il n’y a point de temps avant le monde.
\subsection[{Chapitre VI}]{Chapitre VI}

\begin{argument}\noindent Le monde et le temps ont été créés ensemble.
\end{argument}

\noindent Si la véritable différence du temps et de l’éternité consiste en ce que le temps n’est pas sans quelque changement et qu’il n’y a point de changement dans l’éternité, qui ne voit qu’il n’y aurait point de temps, s’il n’y avait quelque créature dont les mouvements successifs, qui ne peuvent exister simultanément, fissent des intervalles plus longs ou plus courts, ce qui constitue le temps ? Et dès lors je ne conçois pas comment on peut dire que Dieu, être éternel et immuable, qui est le créateur et l’ordonnateur des temps, a créé le monde après de longs espaces de temps, àmoins qu’on ne veuille dire aussi qu’avant le monde il y avait déjà quelque créature dont les mouvements mesuraient le temps. Mais puisque l’Écriture sainte, dont l’autorité est incontestable, nous assure que : « Au commencement Dieu créa le ciel et la terre », ce qui fait bien voir qu’il n’avait rien créé auparavant, il est indubitable que le monde n’a pas été créé dans le temps, mais avec le temps : car ce qui se fait dans le temps se fait après et avant quelque temps, après le temps passé et avant le temps à venir. Or, avant le monde, il ne pouvait y avoir aucun temps passé, puisqu’il n’y avait point de créature dont les mouvements pussent mesurer le temps. Le monde a donc été créé avec le temps, puisque le mouvement a été créé avec le monde, comme cela est visible par l’ordre même des six ou sept premiers jours, pour lesquels le soir et le matin sont marqués, jusqu’à ce que l’œuvre des six jours fût accomplie et que le septième jour fût marqué par le grand mystère du repos de Dieu. Maintenant quels sont ces jours ? c’est ce qui nous est très difficile ou même impossible d’entendre ; combien plus de l’expliquer !
\subsection[{Chapitre VII}]{Chapitre VII}

\begin{argument}\noindent De la nature de ces premiers jours qui ont eu un soir et un matin avant la création du soleil.
\end{argument}

\noindent Nos jours ordinaires n’ont leur soir que par le coucher du soleil et leur matin que par son lever. Or, ces trois premiers jours se sont écoulés sans soleil, puisque cet astre ne fut créé que le quatrième jour. L’Écriture nous dit bien que Dieu créa d’abord la lumière, et la sépara des ténèbres, qu’il appela la lumière {\itshape jour}, et les ténèbres {\itshape nuit} ; mais quelle était cette lumière et par quel mouvement périodique se faisait le soir et le matin, voilà ce qui échappe à nos sens et ce que nous devons pourtant croire sans hésiter, malgré l’impossibilité de le comprendre. En effet, ou bien il s’agit d’une lumière corporelle, soit qu’elle réside loin de nos regards, dans les parties supérieures du monde, soit qu’elle ait servi plus tard à allumer le soleil ; ou bien ce mot de lumière signifie la sainte Cité composée des anges et des esprits bienheureux dont l’Apôtre parle ainsi : « La Jérusalem d’en haut, notre mère éternelle dans les cieux. » Il dit, en effet, ailleurs : « Vous êtes tous enfants de lumière et enfants du jour ; nous ne sommes point les fils de la nuit ni des ténèbres. » Peut-être aussi pourrait-on dire, en quelque façon, que ce jour a son soir et son matin, dans ce sens que la science des créatures est comme un soir en comparaison de celle du Créateur, mais qu’elle devient un jour et un matin, lorsqu’on la rapporte à sa gloire et à son amour, et, pareillement, qu’elle ne penche point vers la nuit, quand on n’abandonne point le Créateur pour s’attacher à la créature. Remarquez enfin que l’Écriture, comptant par ordre ces premiers jours, ne se sert jamais du mot de nuit ; car elle ne dit nulle part : Il y eut nuit, mais : « Du soir et du matin se fit un jour » ; et ainsi du second et du suivant. Aussi bien, la connaissance des choses créées, quand on les regarde en elles-mêmes, a moins d’éclat que si on les contemple dans la sagesse de Dieu comme dans l’art qui les a produites, de sorte qu’on peut l’appeler plus convenablement un soir qu’une nuit ; et néanmoins, comme je l’ai dit, si on la rapporte à la gloire et à l’amour du Créateur, elle devient en quelque façon un matin. Ainsi envisagée, la connaissance des choses créées constitue le premier jour en tant qu’elle se connaît elle-même ; en tant qu’elle a pour objet le firmament, qui a été placé entre les eaux inférieures et supérieures et a été appelé le ciel, c’est le second jour ; appliquée à la terre, à la mer et à toutes les plantes qui tiennent à la terre par leurs racines, c’est le troisième jour ; aux deux grands astres et aux étoiles, c’est le quatrième jour ; à tous les animaux engendrés des eaux, soit qu’ils nagent, soit qu’ils volent, c’est le cinquième jour ; enfin, le sixième jour est constitué par la connaissance de tous les animaux terrestres et de l’homme même.
\subsection[{Chapitre VIII}]{Chapitre VIII}

\begin{argument}\noindent Ce qu’il faut entendre par le repos de Dieu après l’œuvre des six jours.
\end{argument}

\noindent Quand l’Écriture dit que Dieu se reposa le septième jour et le sanctifia, il ne faut pas entendre cela d’une manière puérile, comme si Dieu s’était lassé à force de travail ; {\itshape Dieu a parlé et l’univers a été fait}, et cette parole n’est pas sensible et passagère, mais intelligible et éternelle. Le repos de Dieu, c’est le repos de ceux qui se reposent en lui, comme la joie d’une maison, c’est la joie de ceux qui se réjouissent dans la maison, bien que ce ne soit pas la maison même qui cause leur joie. Combien donc sera-t-il plus raisonnable d’appeler cette maison joyeuse, si par sa beauté elle inspire de la joie à ceux qui l’habitent ? En sorte qu’on l’appelle joyeuse, non seulement par cette façon de parler qui substitue le contenant au contenu (comme quand on dit que les théâtres applaudissent, que les prés mugissent, parce que les hommes applaudissent sur les théâtres et que les bœufs mugissent dans les prés), mais encore par cette figure qui exprime l’effet par la cause, comme quand on dit qu’une lettre est joyeuse, pour marquer la joie qu’elle donne à ceux qui la lisent. Ainsi, lorsque le prophète dit que Dieu s’est reposé, il marque fort bien le repos de ceux qui se reposent en Dieu et dont Dieu même fait le repos ; et cette parole regarde aussi les hommes pour qui les saintes Écritures ont été composées ; elle leur promet un repos éternel à la suite des bonnes œuvres que Dieu opère en eux et par eux, s’ils s’approchent d’abord de lui par la foi. C’est ce qui a été pareillement figuré par le repos du sabbat que la loi prescrivait à l’ancien peuple de Dieu, et dont je me propose de parler ailleurs plus au long.
\subsection[{Chapitre IX}]{Chapitre IX}

\begin{argument}\noindent Ce que l’on doit penser de la création des anges, d’après les témoignages de l’Écriture sainte.
\end{argument}

\noindent Puisque j’ai entrepris d’exposer la naissance de la sainte Cité en commençant par les saints anges, qui en sont la partie la plus considérable, élite glorieuse qui n’a jamais connu les épreuves du pèlerinage d’ici-bas, je vais avec l’aide de Dieu expliquer, autant qu’il me paraîtra convenable, les témoignages divins qui se rapportent à cet objet. Lorsque l’Écriture parle de la création du monde, elle n’énonce pas positivement si les anges ont été créés, ni quand ils l’ont été ; mais à moins qu’ils n’aient été passés sous silence, ils sont indiqués, soit par le {\itshape ciel}, quand il est dit « Dans le principe, Dieu créa le ciel et la terre » ; soit par la lumière dont je viens de parler. Ce qui me persuade qu’ils n’ont pas été omis dans le divin livre, c’est qu’il est écrit d’une part que Dieu se reposa le septième jour de tous les ouvrages qu’il avait faits, et que, d’autre part, la Genèse commence ainsi : « Dans le principe, Dieu créa le ciel et la terre », ce qui semble dire que Dieu n’avait rien fait auparavant. Puis donc qu’il a commencé par le ciel et la terre, et que la terre, ajoute l’Écriture, était d’abord invisible et désordonnée, la {\itshape lumière} n’étant pas encore faite et les ténèbres couvrant la face de l’abîme, c’est-à-dire le mélange confus des éléments, puisque enfin toutes choses ont été successivement ordonnées par une opération qui a duré six jours, comment les anges auraient-ils été omis, eux qui font une partie si considérable de ces ouvrages dont Dieu se reposa le septième jour ? Et cependant il faut convenir que, sans avoir été omis, ils ne sont pas marqués d’une manière claire dans ce passage ; aussi l’Écriture s’en explique-t-elle ailleurs en termes de la plus grande clarté. Dans le cantique des trois jeunes hommes dans la fournaise qui commence ainsi : « Ouvrages du Seigneur, bénissez tous le Seigneur », les anges sont nommés immédiatement après, dans le dénombrement de ces ouvrages. Et dans les Psaumes : « Louez le Seigneur dans les cieux ; louez-le du haut des lieux sublimes. Louez-le, vous tous qui êtes ses anges ; louez-le, vous qui êtes ses Vertus ! Soleil et Lune, louez le Seigneur ; étoiles et lumière, louez-le toutes ensemble. Cieux des cieux, louez le Seigneur, et que toutes les eaux qui sont au-dessus des cieux louent son saint nom ; car il a dit, et toutes choses ont été faites : il a commandé, et elles ont été créées ». Les anges sont donc évidemment un des ouvrages de Dieu. Le texte divin le déclare, quand après avoir énuméré toutes les choses célestes, il est dit de l’ensemble : Dieu a parlé, et tout a été fait. Osera-t-on prétendre maintenant que la création des anges est postérieure à l’œuvre des six jours ? Cette folle hypothèse est confondue par l’Écriture, où Dieu dit : « Quand les astres ont été créés, tous mes anges m’ont béni à haute voix. » Les anges étaient donc déjà, quand furent faits les astres. Les astres, il est vrai, n’ont été créés que le quatrième jour : en conclurons-nous que les anges ont été créés le troisième ? nullement ; car l’emploi de jour est connu : les eaux furent séparées la terre ; ces deux éléments reçurent les espèces d’animaux qui leur conviennent, et la terre produisit tout ce qui lient à elle par des racines. Remonterons-nous au second jour ? pas davantage ; car en ce jour le firmament fut créé entre les eaux supérieures et inférieures ; il reçut le nom de ciel, et ce fut dans son enceinte que les astres furent créés le quatrième jour. Si donc les anges doivent être comptés parmi les ouvrages des six jours, ils sont certainement cette lumière qui est appelée jour et dont l’Écriture marque l’unité en ne l’appelant pas le premier jour ({\itshape dies primus}), mais un jour ({\itshape dies unus}). Car le second jour, le troisième et les suivants ne sont pas d’autres jours, mais ce jour unique, qui a été ainsi répété pour accomplir le nombre six ou le nombre sept, dont l’un figure la connaissance des œuvres de Dieu, et l’autre celle de son repos. En effet, quand Dieu a dit : Que la lumière soit et la lumière fut, s’il est raisonnable d’entendre par là la création des anges, ils ont été certainement créés participants de la lumière éternelle, qui est la sagesse immuable de Dieu, par qui toutes choses ont été faites, et que nous appelons son Fils unique ; et s’ils ont été éclairés de cette lumière qui les avait créés, ç’a été pour devenir eux-mêmes lumière et être appelés {\itshape jour} par la participation de cette lumière et de ce jour immuable qui est le Verbe de Dieu, par qui eux et toutes choses ont été créés. La vraie lumière qui éclaire tout homme venant en ce monde éclaire pareillement tout ange pur, afin qu’il soit lumière, non en soi, mais en Dieu ; aussi tout ange qui s’éloigne de Dieu devient-il impur, comme sont tous ceux qu’on nomine esprits immondes, lorsqu’ils ne sont plus lumière dans le Seigneur, mais ténèbres en eux-mêmes, parce qu’ils sont privés de la participation de la lumière éternelle. En effet, le mal n’est point une substance, mais on a appelé mal la privation du bien.
\subsection[{Chapitre X}]{Chapitre X}

\begin{argument}\noindent De l’immuable et indivisible Trinité, ou le Père, le Fils et le Saint-Esprit ne font qu’un seul Dieu, en qui la qualité et la substance s’identifient.
\end{argument}

\noindent Il existe un bien, seul simple, seul immuable, qui est Dieu. Par ce bien, tous les autres biens ont été créés ; mais ils ne sont point simples, et partant ils sont muables. Quand je dis, en effet, qu’ils ont été créés, j’entends qu’ils ont été faits et non pas engendrés, attendu que ce qui est engendré du bien simple est simple comme lui, est la même chose que lui. Tel est le rapport de Dieu le Père avec Dieu le Fils, qui tous deux ensemble, avec le Saint-Esprit, ne font qu’un seul Dieu ; et cet Esprit du Père et du Fils est appelé le Saint-Esprit dans l’Écriture, par appropriation particulière de ce nom. Or, il est autre que le Père et le Fils, parce qu’il n’est ni le Père ni le Fils ; je dis autre, et non autre chose, parce qu’il est, luiaussi, le bien simple, immuable et éternel. Cette Trinité n’est qu’un seul Dieu, qui n’en est pas moins simple pour être une Trinité ; car nous ne faisons pas consister la simplicité du bien en ce qu’il serait dans le Père seulement, ou seulement dans le Fils, ou enfin dans le seul Saint-Esprit et nous ne disons pas non plus, comme les Sabelliens, que cette Trinité n’est qu’un nom, qui n’implique aucune subsistance des personnes ; mais nous disons que ce bien est simple, parce qu’il est ce qu’il a, sauf la seule réserve de ce qui appartient à chaque personne de la Trinité relativement aux autres. En effet, le Père a un Fils et n’est pourtant pas Fils, le Fils a un Père sans être Père lui-même. Le bien est donc ce qu’il a, dans tout ce qui le constitue en soi-même, sans rapport à un autre que soi. Ainsi, comme il est vivant en soi-même et sans relation, il est la vie même qu’il a.\par
La nature de la Trinité est donc appelée une nature simple, par cette raison qu’elle n’a rien qu’elle puisse perdre et qu’elle n’est autre chose que ce qu’elle a. Un vase n’est pas l’eau qu’il contient, ni un corps la couleur qui le colore, ni l’air la lumière ou la chaleur qui l’échauffe ou l’éclaire, ni l’âme la sagesse qui la rend sage. Ces êtres ne sont donc pas simples, puisqu’ils peuvent être privés de ce qu’ils ont, et recevoir d’autres qualités ou habitudes. Il est vrai qu’un corps incorruptible, tel que celui qui est promis aux saints dans la résurrection, ne peut perdre cette qualité ; mais cette qualité n’est pas sa substance même. L’incorruptibilité réside tout entière dans chaque partie du corps, sans être plus grande ou plus petite dans l’une que dans l’autre, une partie n’étant pas plus incorruptible que l’autre, au lieu que le corps même est plus grand dans son tout que dans une de ses parties. Le corps n’est pas partout tout entier, taudis que l’incorruptibilité est tout entière partout ; elle est dans le doigt, par exemple, comme dans le reste de la main, malgré la différence qu’il y a entre l’étendue de toute la main et celle d’un seul doigt. Ainsi, quoique l’incorruptibilité soit inséparable d’un corps incorruptible, elle n’est pas néanmoinsla substance même du corps, et par conséquent le corps n’est pas ce qu’il a. Il en est de même de l’âme. Encore qu’elle doive être un jour éternellement sage, elle ne le sera que par la participation de la sagesse immuable, qui n’est pas elle. En effet, quand même l’air ne perdrait jamais la lumière qui est répandue dans toutes ses parties, il ne s’ensuivrait pas pour cela qu’il fût la lumière même ; et ici je n’entends pas dire que l’âme soit un air subtil, ainsi que l’ont cru quelques philosophes, qui n’ont pas pu s’élever à l’idée d’une nature incorporelle. Mais ces choses, dans leur extrême différence, ne laissent pas d’avoir assez de rapport pour qu’il soit permis de dire que l’âme incorporelle est éclairée de la lumière incorporelle de la sagesse de Dieu, qui est parfaitement simple, de la même manière l’air corporel est éclairé par la lumière corporelle, et que, comme l’air s’obscurcit quand la lumière vient à se retirer (car ce qu’on appelle ténèbres n’est autre chose que l’air privé de lumière), l’âme s’obscurcit pareillement, lorsqu’elle est privée de la lumière de la sagesse.\par
Si donc on appelle simple la nature divine, c’est qu’en elle la qualité n’est autre chose que la substance, en sorte que sa divinité, sa béatitude et sa sagesse ne sont point différentes d’elle-même. L’Écriture, il est vrai, appelle {\itshape multiple} l’esprit de sagesse, mais c’est à cause de la multiplicité des choses qu’il renferme en soi, lesquelles néanmoins ne sont que lui-même, et lui seul est toutes ces choses. Il n’y a pas, en effet, plusieurs sagesses, mais une seule, en qui se trouvent ces trésors immenses et infinis où sont les raisons invisibles et immuables de toutes les choses muables et visibles qu’elle a créées ; car Dieu n’a rien fait sans connaissance, ce qui ne pourrait se dire avec justice du moindre artisan. Or, s’il a fait tout avec connaissance, il est hors de doute qu’il n’a fait que ce qu’il avait premièrement connu : d’où l’on peut tirer cette conclusion merveilleuse, mais véritable, que nous ne connaîtrions point ce monde, s’il n’était, au lieu qu’il ne pourrait être, si Dieu ne le connaissait.
\subsection[{Chapitre XI}]{Chapitre XI}

\begin{argument}\noindent Si les anges prévaricateurs ont participé à la béatitude dont les anges fidèles ont joui sans interruption depuis qu’ils ont été créés ?
\end{argument}

\noindent Il suit de là qu’en aucun temps ni d’aucune manière les anges n’ont commencé par être des esprits de ténèbres ; dès qu’ils ont été, ils ont été lumière, n’ayant pas été créés pour être ou pour vivre d’une manière quelconque, mais pour vivre sages et heureux. Quelques-uns, il est vrai, s’étant éloignés de la lumière, n’ont point possédé la vie parfaite, la vie sage et heureuse, qui est essentiellement une vie éternelle accompagnée d’une confiance parfaite en sa propre éternité ; mais ils ont encore la vie raisonnable, tout en l’ayant pleine de folie, et ils ne sauraient la perdre, quand ils le voudraient. Au surplus, qui pourrait déterminer à quel degré ils ont participé à la sagesse avant leur chute, et comment croire qu’ils y aient participé autant que les anges fidèles qui trouvent la perfection de leur bonheur dans la certitude de sa durée ? S’il en était de la sorte, les mauvais anges seraient demeurés, eux aussi, éternellement heureux, étant également assurés de leur bonheur. Mais si longue qu’on suppose une vie, elle ne peut être appelée éternelle, si elle doit avoir une fin. Par conséquent, bien que l’éternité ne suppose pas nécessairement la félicité (témoin le feu d’enfer qui, selon l’Écriture, sera éternel), si une vie ne peut être pleinement et véritablement heureuse qu’elle ne soit éternelle, la vie de ces mauvais anges n’était pas bienheureuse, puisqu’elle devait cesser de l’être, soit qu’ils l’aient su, soit qu’ils l’aient ignoré. Dans l’un ou l’autre cas, la crainte ou l’erreur s’opposait à leur parfaite félicité. Et si l’on suppose que, sans être ignorants ou trompés, ils étaient seulement dans le doute sur l’avenir, cela même était incompatible avec la béatitude parfaite que nous attribuons aux bons anges. Quand nous parlons de béatitude, en effet, nous ne restreignons pas tellement l’étendue de ce mol qu’il ne puisse convenir qu’à Dieu seul ; et toutefois Dieu seul est heureux en ce sens qu’il ne peut y avoir de béatitude plus grande que la sienne, et celle des anges, appropriée à leur nature, qu’est-elle en comparaison ?
\subsection[{Chapitre XII}]{Chapitre XII}

\begin{argument}\noindent Comparaison de la félicité des justes sur la terre et de celle de nos premiers parents avant le péché.
\end{argument}

\noindent Nous ne bornons même pas la béatitude aux bons anges. Et qui oserait nier que nos premiers parents, avant la chute, n’aient été heureux dans le paradis terrestre, tout en étant incertains de la durée de leur béatitude, qui aurait été éternelle, s’ils n’eussent point péchés ? Aujourd’hui même, nous n’hésitons point à appeler heureux les bons chrétiens qui, pleins de l’espérance de l’immortalité future, vivent exempts de crimes et de remords, et obtiennent aisément de la miséricorde de Dieu le pardon des fautes attachées à l’humaine fragilité. Et cependant, quelque assurés qu’ils soient du prix de leur persévérance, ils ne le sont pas de leur persévérance même. Qui peut, en effet, se promettre de persévérer jusqu’à la fin, à moins que d’en être assuré par quelque révélation de celui qui, par un juste et mystérieux conseil, ne découvre pas l’avenir à tous, mais qui ne trompe jamais personne ? Pour ce qui regarde la satisfaction présente, le premier homme était donc plus heureux dans le paradis que quelque homme de bien que ce soit en cette vie mortelle ; mais quant à l’espérance du bien avenir, quiconque est assuré de jouir un jour de Dieu en la compagnie des anges, est plus heureux, quoiqu’il souffre, que ne l’était le premier homme, incertain de sa chute ; dans toute la félicité du paradis.
\subsection[{Chapitre XIII}]{Chapitre XIII}

\begin{argument}\noindent Tous les anges ont été créés dans un même état de félicité, de telle sorte que ceux qui devaient déchoir ignoraient leur chute future, et que les bons n’ont eu la prescience de leur persévérance qu’après la chute des mauvais.
\end{argument}

\noindent Dès lors, il est aisé de voir que l’union de deux choses constitue la béatitude, objet légitime des désirs de tout être intelligent : premièrement, jouir sans trouble du bien immuable, qui est Dieu même ; secondement, être pleinement assuré d’en jouir toujours. La foi nous apprend que les anges de lumière possèdent cette béatitude, et la raison nous fait conclure que les anges prévaricateurs ne la possédaient pas, même avant leur chute. Cependant on ne peut leur refuser quelque félicité, je veux dire une félicité sans prescience, s’ils ont vécu quelque temps avant leur péché. Semble-t-il trop dur de penser que, parmi les anges, les uns ont été créés dais l’ignorance de leur persévérance future ou de leur chute, tandis que les autres ont su de science certaine l’éternité de leur béatitude, et veut-on que tous aient été créés dans une égale félicité, y étant demeurés jusqu’au moment mi quelques-uns ont quitté volontairement la source de leur bonheur ? mais il est certes beaucoup plus dur de croire que les bons anges soient encore, à cette heure, incertains de leur béatitude, et qu’ils ignorent sur eux-mêmes ce que nous avons pu, nous, en apprendre par le témoignage des saintes Écritures. Car quel chrétien catholique ne sait qu’il ne se fera plus de démons d’aucun des bons anges, comme il ne se fera point de bons anges d’aucun des démons ? En effet, la Vérité promet dans l’Évangile aux fidèles chrétiens, qu’ils seront semblables aux anges de Dieu, et elle dit en même temps qu’ils jouiront de la vie éternelle. Or, si nous devons être un jour certains de ne jamais déchoir de la félicité immortelle, supposez que les anges ne le fussent pas, nous ne serions plus leurs égaux, nous serions leurs supérieurs. Mais la Vérité ne trompe jamais, et puisque nous devons être leurs égaux, il s’ensuit qu’ils sont certains de l’éternité de leur bonheur. Et comme d’ailleurs les autres anges n’en pouvaient pas être certains, il faut conclure ou que la félicité n’était pas pareille, ou que, si elle l’était, les bons n’ont été assurés de leur bonheur qu’après la chute des autres. Mais, dira-t-on peut-être, est-ce que cette parole de Notre-Seigneur dans l’Évangile touchant le diable « Qu’il était homicide dès le commencement et qu’il n’est point demeuré dans la vérité », ne doit pas s’entendre du commencement de la création ? et à ce compte, le diable n’aurait jamais été heureux avec les saints anges, parce que, dès le moment de sa création, il aurait refusé de se soumettre à son Créateur, et c’est aussi dans ce sens qu’il faudrait entendre le mot de l’apôtre saint Jean : « Le diable pèche dès le commencement », c’est-à-dire que, dès l’instant de sa création, il aurait rejeté la justice, qu’on ne peut conserver, si l’on ne soumet sa volonté à celle de Dieu. En tout cas, ce sentiment est bien éloigné de l’hérésie des Manichéens et autres fléaux de la vérité, qui prétendent que le diable possède en propre une nature mauvaise qu’il a reçue d’un principe contraire à Dieu : esprits extravagants, qui ne prennent pas garde que dans cet Évangile dont ils admettent l’autorité aussi bien que nous, Notre-Seigneur ne dit pas : Le diable a été étranger à la vérité, mais : {\itshape Il n’est point demeuré dans la vérité}, ce qui veut dire qu’il est déchu, et certes, s’il y était demeuré, il en participerait encore et serait bienheureux avec les saints anges.
\subsection[{Chapitre XIV}]{Chapitre XIV}

\begin{argument}\noindent Explication de cette parole de l’Évangile : « Le diable n’est point demeuré dans la vérité, parce que la vérité n’est point en lui. »
\end{argument}

\noindent Notre-Seigneur semble avoir voulu répondre à cette question : Pourquoi le diable n’est-il point demeuré dans la vérité ? quand il ajoute : « Car la vérité n’est point en lui. » Or, elle serait en lui, s’il fût demeuré en elle. Cette parole est donc assez extraordinaire, puisqu’elle paraît dire que si le diable n’est point demeuré dans la vérité, c’est que la vérité n’est point en lui ; tandis qu’aucontraire, ce qui fait que la vérité n’est point en lui, c’est qu’il n’est point demeuré dans la vérité. Cette même façon de parler se retrouve aussi dans un psaume : « J’ai crié, mon Dieu », dit le Prophète, « parce que vous m’avez exaucé », au lieu qu’il semble qu’il devait dire : Vous m’avez exaucé, mon Dieu, parce que j’ai crié. Mais il faut entendre que le Prophète, après avoir dit : « J’ai crié », prouve la réalité de son invocation par l’effet qu’elle a obtenu : la preuve que j’ai crié, c’est que vous m’avez exaucé.
\subsection[{Chapitre XV}]{Chapitre XV}\phantomsection
\label{\_chapitre15}

\begin{argument}\noindent Comment il faut entendre cette parole : « Le diable pèche dès le commencement. »
\end{argument}

\noindent Quant à cette parole de saint Jean : « Le diable pèche dès le commencement », les hérétiques ne comprennent pas que si le péché est naturel, il cesse d’être. Mais que peuvent-ils répondre à ce témoignage d’Isaïe qui, désignant le diable sous la figure du prince de Babylone, s’écrie : « Comment est tombé Lucifer, qui se levait brillant au matin ? » et ce passage d’Ézéchiel : « Tu as joui des délices du paradis, orné de toutes sortes de pierres précieuses » ? Le diable a donc été quelque temps sans péché ; et c’est ce que le prophète lui dit un peu après en termes plus formels : « Tu as marché pur de souillure en tes jours. » Que si l’on ne peut donner un sens plus naturel à ces paroles, il faut donc entendre par celle-ci : « Il n’est point demeuré dans la vérité », que le diable a été dans la vérité, mais qu’il n’y est pas demeuré ; et quant à cette autre, « que le diable pèche dès le commencement », il ne faut pas entendre qu’il a péché dès le commencement de sa création, mais dès celui de son orgueil. De même, quand nous lisons dans Job, à propos du diable : « Il est le commencement de l’ouvrage de Dieu, qui l’a fait pour le livrer aux railleries de ses anges » ; et ce passage analogue du psaume : « Ce dragon que vous avez formé pour servir de jouet » ; nous ne devons pas croire que le diable ait été créé primitivement pour êtremoqué des anges, mais bien que leurs railleries sont la peine de son péché. Il est donc l’ouvrage du Seigneur ; car il n’y a pas de nature si vile et si infime qu’on voudra, même parmi les plus petits insectes, qui ne soit l’ouvrage de celui d’où vient toute mesure, toute beauté, tout ordre, c’est-à-dire ce qui fait l’être et l’intelligibilité de toute chose. À plus forte raison est-il le principe de la créature angélique, qui surpasse par son excellence tous les autres ouvrages de Dieu.
\subsection[{Chapitre XVI}]{Chapitre XVI}

\begin{argument}\noindent Des degrés et des différences qui sont entre les créatures selon qu’on envisage leur utilité relative ou l’ordre absolu de la raison.
\end{argument}

\noindent Parmi les êtres que Dieu a créés, on préfère ceux qui ont la vie à ceux qui ne l’ont pas, ceux qui ont la puissance de la génération ou seulement l’appétit à ceux qui en sont privés. Parmi les vivants, on préfère ceux qui ont du sentiment, comme les animaux, aux plantes, qui sont insensibles ; et entre les êtres doués de sentiment, les êtres intelligents, comme les hommes, à ceux qui sont dépourvus d’intelligence, comme les bêtes ; et entre les êtres intelligents, les immortels, comme les anges, aux mortels, comme les hommes. Cet ordre de préférence est celui de la nature. Il en est un autre qui dépend de l’estime que chacun fait des choses, selon l’utilité qu’il en tire ; par où il arrive que nous préférons quelquefois certains objets insensibles à des êtres doués de sentiment, et cela à tel point que, s’il ne dépendait que de nous, nous retrancherions ceux-ci de la nature, soit par ignorance du rang qu’ils y tiennent, soit par amour pour notre avantage personnel que nous mettons au-dessus de tout. Qui n’aimerait mieux, par exemple, avoir chez soi du pain que des souris, et des écus que des puces ? Et il n’y a pas lieu de s’en étonner, quand on voit tes hommes, dont la nature est si noble, acheter souvent plus cher un cheval ou une pierre précieuse qu’un esclave ou une servante. Ainsi les jugements de la-raison sont bien différents de ceux de la nécessité ou de la volupté : la raison juge des choses en elles-mêmes et selon la vérité, au lieu que la nécessité n’en juge que selon les besoins, et lavolupté selon les plaisirs. Mais la volonté et l’amour sont d’un tel prix dans les êtres raisonnables que, malgré la supériorité des anges sur les hommes selon l’ordre de la nature, l’ordre de la justice veut que les hommes bons soient mis au-dessus des mauvais anges.
\subsection[{Chapitre XVII}]{Chapitre XVII}

\begin{argument}\noindent La malice n’est pas dans la nature, mais contre la nature, et elle a pour principe, non le Créateur, mais la volonté.
\end{argument}

\noindent C’est donc de la nature du diable et non de sa malice qu’il est question dans ce passage : « Il est le commencement de l’ouvrage de Dieu » ; car la malice, qui est un vice, ne peut se rencontrer que dans une nature auparavant non viciée, et tout vice est tellement contre la nature qu’il en est par essence la corruption. Ainsi, s’éloigner de Dieu ne serait pas un vice, s’il n’était naturel d’être avec Dieu. C’est pourquoi la mauvaise volonté même est une grande preuve de la bonté de la nature. Mais comme Dieu est le créateur parfaitement bon des natures, il est le régulateur parfaitement juste des mauvaises volontés, et il se fait bien servir d’elles, quand elles se servent mal de la bonté naturelle de ses dons. C’est ainsi qu’il a voulu que le diable, qui était bon par sa nature et qui est devenu mauvais par sa volonté, servît de jouet à ses anges, ce qui veut dire que les tentations dont le diable se sert pour nuire aux saints tournent à leur profit. En créant Satan, Dieu n’ignorait pas sa malignité future, et comme il savait d’une manière certaine le bien qu’il devait tirer de ce mal, il a dit par l’organe du Psalmiste : « Ce dragon que vous avez formé pour servir de jouet à vos anges », cela signifie que tout en le créant bon, sa providence disposait déjà les moyens de se servir utilement de lui, quand il serait devenu mauvais.
\subsection[{Chapitre XVIII}]{Chapitre XVIII}

\begin{argument}\noindent De la beauté de l’univers qui, par l’art de la Providence, tire une splendeur nouvelle de l’opposition des contraires.
\end{argument}

\noindent En effet, Dieu n’aurait pas créé un seul ange, que dis-je ? un seul homme dont il aurait prévu la corruption, s’il n’avait su en même temps comment il ferait tourner cemal à l’avantage des justes et relèverait la beauté de l’univers par l’opposition des contraires, comme on embellit un poème par les antithèses. C’est, en effet, une des plus brillantes parures du discours que l’antithèse, et si ce mot n’est pas encore passé dans la langue latine, la figure elle-même, je veux dire l’opposition ou le contraste, n’en fait pas moins l’ornement de cette langue ou plutôt de toutes les langues du monde. Saint Paul s’en est servi dans ce bel endroit de la seconde épître aux Corinthiens : « Nous agissons en toutes choses comme de fidèles serviteurs de Dieu,… par les armes de justice pour combattre à droite et à gauche, parmi la gloire et l’infamie, parmi les calomnies et les louanges, semblables à des séducteurs et sincères, à des inconnus et connus de tous, toujours près de subir la mort et toujours vivants, sans cesse frappés, mais non exterminés, tristes et toujours dans la joie, pauvres et enrichissant nos frères, n’ayant rien et possédant tout. » Comme l’opposition de ces contraires fait ici la beauté du langage, de même la beauté du monde résulte d’une opposition, mais l’éloquence n’est plus seulement dans les mots, elle est dans les choses. C’est ce qui est clairement exprimé dans ce passage de l’Ecclésiastique : « Le bien est contraire au mal, et la mort à la vie ainsi le pécheur à l’homme pieux ; regarde toutes les œuvres du Très-Haut : elles vont ainsi deux à deux, et l’une contraire à l’autre. »
\subsection[{Chapitre XIX}]{Chapitre XIX}

\begin{argument}\noindent Ce qu’il faut entendre par ces paroles de l’Écriture : « Dieu sépara la lumière des ténèbres. »
\end{argument}

\noindent L’obscurité même de l’Écriture a cet avantage, que l’on peut d’un passage tirer divers sens, tous conformes à la vérité, tous confirmés par le témoignage de choses manifestes ou par d’autres passages très clairs, de sorte que, dans le cours d’un long travail, si on ne parvient pas à découvrir le véritable sens du texte, on a du moins l’occasion de proclamer d’autres vérités. C’est pourquoi je crois pouvoir proposer d’entendre par la création de la première lumière la création des anges, et de voir la distinction des bons et des mauvaisdans ces paroles : « Dieu sépara la lumière des ténèbres, et nomma la lumière jour et les ténèbres nuit. » En effet, celui-là seul a pu les séparer qui a pu prévoir leur chute et connaître qu’ils demeureraient obstinés dans leur présomptueux aveuglement. Quant au jour proprement dit et à la nuit, Dieu les sépara par ces deux grands astres qui frappent nos sens : « Que les astres, dit-il, soient faits dans le firmament du ciel pour luire sur la terre et séparer le jour de la nuit. » Et un peu après : « Dieu fit deux grands astres, l’un plus grand pour présider au jour, et l’autre moindre pour présider à la nuit avec les étoiles ; Dieu les mit dans le firmament du ciel pour luire sur la terre, et présider au jour et à la nuit, et séparer la lumière des ténèbres. » Mais cette lumière, qui est la sainte société des anges, toute éclatante des splendeurs de la vérité intelligible, et ces ténèbres qui lui sont contraires, c’est-à-dire ces esprits corrompus, ces mauvais anges éloignés par leur faute de la lumière de la justice, je répète que celui-là seul pouvait opérer leur séparation, à qui le mal à venir (mal de la volonté, non de la nature) n’a pu être, avant de se produire, douteux ou caché.
\subsection[{Chapitre XX}]{Chapitre XX}

\begin{argument}\noindent Explication de ce passage : « Et Dieu vit que la lumière était bonne. »
\end{argument}

\noindent Il importe de remarquer aussi qu’après cette parole : « Que la lumière soit faite, et la lumière fut faite », l’Écriture ajoute aussitôt : « Et Dieu vit que la lumière était bonne. » Or, elle n’ajoute pas cela après que Dieu eût séparé la lumière des ténèbres et appelé la lumière jour et les ténèbres nuit. Pourquoi ? c’est que Dieu aurait paru donner également son approbation à ces ténèbres et à cette lumière. Quant aux ténèbres matérielles, incapables par conséquent de faillir, qui, à l’aide des astres, sont séparées de cette lumière sensible qui éclaire nos yeux, l’Écriture ne rapporte le témoignage de l’approbation de Dieu qu’après la séparation accomplie : « Et Dieu plaça ces astres dans le firmament du ciel pour luire sur la terre, présider au jour et à la nuit, et séparer la lumière des ténèbres. Et Dieu vit que cela étaitbon. » L’un et l’autre lui plut, parce que l’un et l’autre est sans péché. Mais lorsque Dieu eut dit : « Que la lumière soit faite, et la lumière fut faite : et Dieu vit que la lumière était bonne » ; l’Écriture ajoute aussitôt : « Et Dieu sépara la lumière des ténèbres, et appela la lumière jour et les ténèbres nuit. » Elle n’ajoute pas : Et Dieu vit que cela était bon, de peur que l’un et l’autre ne fut nommé bon, tandis que l’un des deux était mauvais, non par nature, mais par son propre vice. C’est pourquoi, en cet endroit, la seule lumière plut au Créateur, et quant aux ténèbres, c’est-à-dire aux mauvais anges, tout en les faisant servir à l’ordre de ses desseins, il ne devait pas les approuver.
\subsection[{Chapitre XXI}]{Chapitre XXI}

\begin{argument}\noindent De la science éternelle et immuable de Dieu et de sa volonté, par qui toutes ses œuvres lui ont toujours plu, avant d’être créées, telles qu’il les a créées en effet.
\end{argument}

\noindent En quel sens entendre ces paroles qui sont répétées après chaque création nouvelle : « Dieu vit que cela était bon », sinon comme une approbation que Dieu donne à son ouvrage fait selon les règles d’un art qui n’est autre que sa sagesse ? En effet, Dieu n’apprit pas que son ouvrage était bon, après l’avoir fait, puisqu’il ne l’aurait pas fait s’il ne l’avait connu bon avant de le faire. Lors donc qu’il dit : Cela était bon, il ne l’apprend pas, il l’enseigne. Platon est allé plus loin, quand il dit que Dieu fut transporté de joie après avoir achevé le monde. Certes, Platon était trop sage pour croire que la nouveauté de la création eût ajouté à la félicité divine ; mais il a voulu faire entendre que l’ouvrage qui avait plu à Dieu avant que de le faire, lui avait plu aussi lorsqu’il fut fait. Ce n’est pas que la science de Dieu éprouve aucune variation et qu’il connaisse de plusieurs façons diverses ce qui est, ce qui a été et ce qui sera. La connaissance qu’il a du présent, du passé et de l’avenir n’a rien de commun avec la nôtre. Prévoir, voir, revoir, pour lui c’est tout un. Il ne passe pas comme nous d’une chose àune autre en changeant de pensée, mais il contemple toutes choses d’un regard immuable. Ce qui est actuellement, ce qui n’est pas encore, ce qui n’est plus, sa présence stable et éternelle embrasse tout. Et il ne voit pas autrement des yeux, autrement de l’esprit, parce qu’il n’est pas composé de corps et d’âme ; il ne voit pas aujourd’hui autrement qu’il ne faisait hier ou qu’il ne fera demain, parce que sa connaissance ne change pas, comme la nôtre, selon les différences du temps. C’est de lui qu’il est dit : « Qu’il ne reçoit de changement ni d’ombre par aucune révolution. » Car il ne passe point d’une pensée à une autre, lui dont le regard incorporel embrasse tous les objets comme simultanés. Il connaît le temps d’une connaissance indépendante, du temps, comme il meut les choses temporelles sans subir aucun mouvement temporel. Il a donc vu que ce qu’il avait fait était bon là même où il avait vu qu’il était bon de le faire, et, en regardant son ouvrage accompli, il n’a pas doublé ou accru sa connaissance, comme si elle eût été moindre auparavant, lui dont l’ouvrage n’aurait pas toute sa perfection, si l’accomplissement de sa volonté pouvait ajouter quelque chose à la perfection de sa connaissance. C’est pourquoi, s’il n’eût été question que de nous apprendre quel est l’auteur de la lumière, il aurait suffi de dire : Dieu fit la lumière ; ou si l’Écriture eût voulu nous faire savoir en outre par quel moyen il l’a faite, c’eût été assez de ces paroles : « Dieu dit : Que la lumière soit faite, et la lumière fut faite », car nous aurions su de la sorte que non seulement Dieu a fait la lumière, mais qu’il l’a faite par sa parole. Mais comme il était important de nous apprendre trois choses touchant la créature qui l’a faite, par quel moyen, et pourquoi elle a été faite, l’Écriture a marqué tout cela en disant : « Dieu dit : Que la lumière soit faite, et la lumière fut faite, et Dieu vit que la lumière était bonne. » Ainsi, c’est Dieu qui a fait toutes choses ; c’est par sa parole qu’il les a faites, et il les a faites parce qu’elles sont bonnes. Il n’y a point de plus excellent ouvrier que Dieu, ni d’art plus efficace que sa parole, ni de meilleure raison de la création que celle-ci : une œuvre bonne a été produitepar un bon ouvrier. Platon apporte aussi cette même raison de la création du monde, et dit qu’il était juste qu’une œuvre bonne fût produite par un Dieu bon ; soit qu’il ait lu cela dans nos livres, soit qu’il l’ait appris de ceux qui l’y avaient lu, soit que la force de son génie l’ait élevé de la connaissance des ouvrages visibles de Dieu à celle de ses grandeurs invisibles, soit enfin qu’il ait été instruit par ceux qui étaient parvenus à ces hautes vérité.
\subsection[{Chapitre XXII}]{Chapitre XXII}

\begin{argument}\noindent De ceux qui trouvent plusieurs choses à reprendre dans cet univers, ouvrage excellent d’un excellent créateur, et qui croient à l’existence d’une mauvaise nature.
\end{argument}

\noindent Cependant quelques hérétiques n’ont pas su reconnaître cette raison suprême de la création, savoir, la bonté de Dieu, raison si juste et si convenable qu’il suffit de la considérer avec attention et de la méditer avec piété pour mettre fin à toutes les difficultés qu’on peut élever sur l’origine des choses. Mais on ne veut considérer que les misères de notre corps, devenu mortel et fragile en punition du péché, et exposé ici-bas à une foule d’accidents contraires, comme le feu, le froid, les bêtes farouches et autres choses semblables. On ne remarque pas combien ces choses sont excellentes dans leur essence, et dans la place qu’elles occupent avec quel art admirable elles sont ordonnées, à quel point elles contribuent chacune en particulier à la beauté de l’univers, et quels avantages elles nous apportent quand nous savons en bien user, en sorte que les poisons mêmes deviennent des remèdes, étant employés à propos, et qu’au contraire les choses qui nous flattent le plus, comme la lumière, le boire et le manger, sont nuisibles par l’abus que l’on en fait. La divine Providence nous avertit par là de ne pas blâmer témérairement ses ouvrages, mais d’en rechercher soigneusement l’utilité, et, lorsque notre intelligence se trouve en défaut, de croire que ces choses sont cachées comme l’étaient plusieurs autres que nous avons eu peine à découvrir. Si Dieu permet qu’elles soient cachées, c’est pour exercer notre humilité ou pour abaisser notre orgueil. En effet, il n’y a aucune nature mauvaise, et le mal n’est qu’une privation du bien ; mais depuis les choses de la terre jusqu’à celles du ciel, depuis les visibles jusqu’aux invisibles, il en est qui sont meilleures les unes que les autres, et leur existence à toutes tient essentiellement à leur inégalité. Or, Dieu n’est pas moins grand dans les petites choses que dans les grandes ; car il ne faut pas mesurer les petites par leur grandeur naturelle, qui est presque nulle, mais par la sagesse de leur auteur. C’est ainsi qu’en rasant un sourcil à un homme on ôterait fort peu de son corps, mais on ôterait beaucoup de sa beauté, parce que la beauté du corps ne consiste pas dans la grandeur de ses membres, mais dans leur proportion. Au reste, il ne faut pas trop s’étonner de ce que ceux qui croient à l’existence d’une nature mauvaise, engendrée d’un mauvais principe, ne veulent pas reconnaître la bonté de Dieu comme la raison de la création du monde, puisqu’ils s’imaginent au contraire que Dieu n’a créé cette machine de l’univers que dans la dernière nécessité, et pour se défendre du mal qui se révoltait contre lui ; qu’ainsi il a mêlé sa nature qui est bonne avec celle du mal, afin de le réprimer et de le vaincre ; qu’il a bien de la peine à la purifier et à la délivrer, parce que le mal l’a étrangement corrompue, et qu’il ne la purifie pas même tout entière, si bien que cette partie non purifiée servira de prison et de chaîne à son ennemi vaincu. Les Manichéens ne donneraient pas dans de telles extravagances, s’ils étaient convaincus de ces deux vérités : l’une, que la nature de Dieu est immuable, incorruptible, inaltérable ; l’autre, que l’âme qui a pu déchoir par sa volonté et ainsi être corrompue par le péché et privée de la lumière de la vérité immuable, l’âme, dis-je, n’est pas une partie de Dieu ni de même nature que la sienne, mais une créature infiniment éloignée de la perfection de son Créateur.
\subsection[{Chapitre XXIII}]{Chapitre XXIII}

\begin{argument}\noindent De l’erreur reprochée à la doctrine d’Origène.
\end{argument}

\noindent Mais voici qui est beaucoup plus surprenant : c’est que des esprits persuadés comme nous qu’il n’y a qu’un seul principe de toutes choses, et que toute nature qui n’est pas Dieu ne peut avoir d’autre créateur que Dieu, ne veuillent pas admettre d’un cœur simple et bon cette explication si simple et si bonne de la création, savoir qu’un Dieu bon a fait de bonnes choses, lesquelles, étant autres que Dieu, sont inférieures à Dieu, sans pouvoir provenir toutefois d’un autre principe qu’un Dieu bon. Ils prétendent que les âmes, dont ils ne font pas à la vérité les parties de Dieu, mais ses créatures, ont péché en s’éloignant de leur Créateur ; qu’elles ont mérité par la suite d’être enfermées, depuis le ciel jusqu’à la terre, dans divers corps, comme dans une prison, suivant la diversité de leurs fautes ; que c’est là le monde, et qu’ainsi la cause de sa création n’a pas été de faire de bonnes choses mais d’en réprimer de mauvaises. Tel est le sentiment d’Origène, qu’il a consigné dans son livre {\itshape Des principes}. Je ne saurais assez m’étonner qu’un si docte personnage et si versé dans les lettres sacrées n’ait pas vu combien cette opinion est contraire à l’Écriture sainte, qui, après avoir mentionné chaque ouvrage de Dieu, ajoute : « Et Dieu vit que cela était bon » ; et qui, après les avoir dénombrés tous, s’exprime ainsi : « Et Dieu vit toutes les choses qu’il avait faites, et elles étaient très bonnes », pour montrer qu’il n’y a point eu d’autre raison de créer le monde, sinon la nécessité que des choses parfaitement bonnes fussent créées par un Dieu tout bon, de sorte que si personne n’eût péché, le monde ne serait rempli et orné que de bonnes natures. Mais, de ce que le péché a été commis, il ne s’ensuit pas que tout soit plein de souillures, puisque dans le ciel le nombre des créatures angéliques qui gardent l’ordre de leur nature est le plus grand. D’ailleurs, la mauvaise volonté, pour s’être écartée de cet ordre, ne s’est pas soustraite aux lois de la justice de Dieu, qui dispose bien de toutes choses. De même qu’un tableau plaît avec ses ombres, quand elles sont bien distribuées, ainsi l’univers est beau, même avec les pécheurs, quoique ceux-ci, pris en eux-mêmes, soient laids et difformes.\par
Origène devait en outre considérer que si le monde avait été créé afin que les âmes, en punition de leurs péchés, fussent enfermées dans des corps comme dans une prison, en sorte que celles qui, sont moins coupables eussent des corps plus légers, et les autres de plus pesants, il faudrait que les démons, qui sont les plus perverses de toutes les créatures, eussent des corps terrestres plutôt que les hommes. Cependant, pour qu’il soit manifeste que ce n’est point par là qu’on doit juger du mérite des âmes, les démons ont des corps aériens, et l’homme, méchant, il est vrai, mais d’une malice beaucoup moins profonde, que dis-je ? l’homme, avant son péché, a reçu un corps de terre. Qu’y a-t-il, au reste, de plus impertinent que de dire que, s’il n’y a qu’un soleil dans le monde, cela ne vient pas de la sagesse admirable de Dieu qui l’a voulu ainsi et pour la beauté et pour l’utilité de l’univers, mais parce qu’il est arrivé qu’une âme a commis un péché qui méritait qu’on l’enfermât dans un tel corps ? De sorte que s’il fût arrivé, non pas qu’une âme, mais que deux, dix ou cent eussent commis le même péché, il y aurait cent soleils dans le monde. Voilà une étrange chute des âmes, et ceux qui imaginent ces belles choses, sans trop savoir ce qu’ils disent, font assez voir que leurs propres âmes ont fait de lourdes chutes sur le chemin de la vérité. Maintenant, pour revenir à la triple question posée plus haut : Qui a fait le monde ? par quel moyen ? pour quelle fin ? et la triple réponse : Dieu, par son Verbe, pour le bien, on peut se demander s’il n’y a pas dans les mystiques profondeurs de ces vérités une manifestation de la Trinité divine, Père, Fils et Saint-Esprit, ou bien s’il y a quelque inconvénient à interpréter ainsi l’Écriture sainte ? C’est une question qui demanderait un long discours, et rien ne nous oblige à tout expliquer dans un seul livre.
\subsection[{Chapitre XXIV}]{Chapitre XXIV}\phantomsection
\label{\_chapitre24}

\begin{argument}\noindent De la Trinité divine, qui a répandu en toutes ses œuvres des traces de sa présence.
\end{argument}

\noindent Nous croyons, nous maintenons, nous enseignons comme un dogme de notre foi, que le Père a engendré le Verbe (c’est-à-dire la sagesse, par qui toutes choses ont été faites), Fils unique du Père, un comme lui, éternel comme lui, et souverainement bon comme lui ; que le Saint-Esprit est ensemble l’esprit du Père et du Fils, consubstantiel et coéternel à tous deux ; et que tout cela est Trinité, à cause de la propriété des personnes, et un seul Dieu, à cause de la divinité inséparable, comme un seul tout-puissant, à cause de la toute-puissance inséparable ; de telle sorte que chaque personne est Dieu et tout-puissant, et que toutes les trois ensemble ne sont point trois dieux, ni trois tout-puissants, mais un seul Dieu tout-puissant ; tant l’unité de ces trois personnes divines est inséparable Or, le Saint-Esprit du Père, qui est bon, et du Fils, qui est bon aussi, peut-il avec raison s’appeler la bonté des deux, parce qu’il est commun aux deux ? Je n’ai pas la témérité de l’assurer. Je dirais plutôt qu’il est la sainteté des deux, en ne prenant pas ce mot pour une qualité, mais pour une substance et pour la troisième personne de la Trinité. Ce qui me déterminerait à hasarder cette réponse, c’est qu’encore que le Père soit esprit et soit saint, et le Fils de même, la troisième personne divine ne laisse pas toutefois de s’appeler proprement l’Esprit-Saint, comme la sainteté substantielle et consubstantielle de tous deux. Cependant, si la bonté divine n’est autre-chose que la sainteté divine, ce n’est plus une témérité de l’orgueil, mais un exercice légitime (le la raison, de chercher sous le voile d’une expression mystérieuse le dogme de la Trinité manifestée dans ces trois conditions, dont on peut s’enquérir en chaque créature : qui l’a faite, par quel moyen a-t-elle été faite et pour quelle fin ? Car c’est le Père du Verbe qui a dit : « Que cela soit fait » ; ce qui a été fait à sa parole, l’a sans doute été par le Verbe ; et lorsque l’Écriture ajoute : « Dieu vit que cela était bon », ces paroles nous montrent assez que ce n’a point été par nécessité, ni par indigence, mais par bonté, que Dieu a fait ce qu’il a fait, c’est-à-dire parce que cela est bon. Et c’est pourquoi la créature n’a été appelée bonne qu’après sa création, afin de marquer qu’elle est conforme â cette bonté, qui est la raison finale de son existence. Or,si par cette bonté on peut fort bien entendre le Saint-Esprit, voilà la Trinité tout entière manifestée dans tous ses ouvrages. C’est en elle que la Cité sainte, la Cité d’en haut et des saints anges trouve son origine, sa forme et sa félicité. Si l’on demande quel est l’auteur de son être, c’est Dieu qui l’a créée ; pourquoi elle est sage, c’est que Dieu l’éclaire ; d’où vient qu’elle est heureuse, c’est qu’elle jouit de Dieu. Ainsi Dieu est le principe de son être, de sa lumière et de sa joie ; elle est, elle voit, elle aime ; elle est dans l’éternité de Dieu, elle brille dans sa vérité, elle jouit dans sa bonté.
\subsection[{Chapitre XXV}]{Chapitre XXV}

\begin{argument}\noindent De la division de la philosophie en trois parties.
\end{argument}

\noindent Tel est aussi, autant qu’on en peut juger, le principe de cette division de la philosophie en trois parties, établie ou, pour mieux dire, reconnue par les sages ; car si la philosophie se partage en physique, logique et éthique, ou, pour employer des mots également usités, en science naturelle, science rationnelle et science morale, ce ne sont pas les philosophes qui ont fait ces distinctions, ils n’ont eu qu’à les découvrir. Par où je n’entends pas dire qu’ils aient pensé à Dieu et à la Trinité, quoique Platon, à qui on rapporte l’honneur de la découverte, ait reconnu Dieu comme l’unique auteur de toute la nature, le dispensateur de l’intelligence et l’inspirateur de cet amour qui est la source d’une bonne et heureuse vie ; je remarque seulement que les philosophes, tout en ayant des opinions différentes sur la nature des choses, sur la voie qui mène à la vérité et sur le bien final auquel nous devons rapporter toutes nos actions, s’accordent tous à reconnaître cette division générale, et nul d’entre eux, de quelque secte qu’il soit, ne révoque en doute que la nature n’ait une cause, la science une méthode et la vie une loi. De même chez tout artisan, trois choses concourent à la production de ses ouvrages, la nature, l’art et l’usage. La nature se fait reconnaître par le génie, l’art par l’instruction et l’usage par le fruit. Je sais bienqu’à proprement parler, le fruit concerne la jouissance et l’usage l’utilité, et qu’il y a cette différence entre jouir d’une chose et s’en servir, qu’en jouir, c’est l’aimer pour elle-même, et s’en servir, c’est l’aimer pour une autre fin, d’où vient que nous ne devons qu’user des choses passagères, afin de mériter de jouir des éternelles, et ne pas faire comme ces misérables qui veulent jouir de l’argent et se servir de Dieu, n’employant pas l’argent pour Dieu, mais adorant Dieu pour l’argent. Toutefois, à prendre ces mots dans l’acception la plus ordinaire, nous usons des fruits de la terre, quoique nous ne fassions que nous en servir. C’est donc en ce sens que j’emploie le nom d’usage en parlant des trois choses propres à l’artisan, savoir la nature, l’art ou la science, et l’usage. Les philosophes ont tiré de là leur division de la science qui sert à acquérir la vie bienheureuse, en naturelle, à cause de la nature, rationnelle à cause de la science, et morale à cause de l’usage. Si nous étions les auteurs de notre nature, nous serions aussi les auteurs de notre science et nous n’aurions que faire des leçons d’autrui ; il suffirait pareillement, pour être heureux, de rapporter notre amour à nous-mêmes et de jouir de nous ; mais puisque Dieu est l’auteur de notre nature, il faut, si nous voulons connaître le vrai et posséder le bien, qu’il soit notre maître de vérité et notre source de béatitude.
\subsection[{Chapitre XXVI}]{Chapitre XXVI}

\begin{argument}\noindent L’image de la Trinité est en quelque sorte empreinte dans l’homme, avant même qu’il ne soit devenu bienheureux.
\end{argument}

\noindent Nous trouvons en nous une image de Dieu, c’est-à-dire de cette souveraine Trinité, et, bien que la copie ne soit pas égale au modèle, ou, pour mieux dire, qu’elle en soit infiniment éloignée, puisqu’elle ne lui est ni coéternelle ni consubstantielle, et qu’elle a même besoin d’être réformée pour lui ressembler en quelque sorte, il n’est rien néanmoins, entre tous les ouvrages de Dieu, qui approche de plus près de sa nature. En effet, nous sommes, nous connaissons que nous sommes, et nous aimons notre être et la connaissance que nous en avons. Aucune illusion n’est possible sur ces trois objets ; car nous n’avons pas besoinpour les connaître de l’intermédiaire d’un sens corporel, ainsi qu’il arrive des objets qui sont hors de nous, comme la couleur qui n’est pas saisie sans la vue, le son sans l’ouïe, les senteurs sans l’odorat, les saveurs sans le goût, le dur et le mou sans le toucher, toutes choses sensibles dont nous avons aussi dans l’esprit et dans la mémoire des images très ressemblantes et cependant incorporelles, lesquelles suffisent pour exciter nos désirs ; mais je suis très certain, sans fantôme et sans illusion de l’imaginative, que j’existe pour moi-même, que je connais et que j’aime mon être. Et je ne redoute point ici les arguments des Académiciens ; je ne crains pas qu’ils me disent : Mais si vous vous trompez ? Si je me trompe, je suis ; car celui qui n’est pas ne peut être trompé, et de cela même que je suis trompé, il résulte que je suis. Comment donc me puis-je tromper, en croyant que je suis, du moment qu’il est certain que je suis, si je suis trompé ? Ainsi, puisque je serais toujours, moi qui serais trompé, quand il serait vrai que je me tromperais, il est indubitable que je ne puis me tromper, lorsque je crois que je suis. Il suit de là que, quand je connais que je connais, je ne me trompe pas non plus ; car je connais que j’ai cette connaissance de la même manière que je connais que je suis. Lorsque j’aime ces deux choses, j’y en ajoute une troisième qui est mon amour, dont je ne suis pas moins assuré que des deux autres. Je ne me trompe pas, lorsque je pense aimer, ne pouvant pas me tromper touchant les choses que j’aime : car alors même que ce que j’aime serait faux, il serait toujours vrai que j’aime une chose fausse. Et comment serait-on fondé à me blâmer d’aimer une chose fausse, s’il était faux que je l’aimasse ? Mais l’objet de mon amour étant certain et véritable, qui peut douter de la certitude et de la vérité de mon amour ? Aussi bien, vouloir ne pas être, c’est aussi impossible que vouloir ne pas être heureux ; car comment être heureux, si l’on n’est pas ?
\subsection[{Chapitre XXVII}]{Chapitre XXVII}\phantomsection
\label{\_chapitre27}

\begin{argument}\noindent De l’être et de la science, et de l’amour de l’un et de l’autre.
\end{argument}

\noindent Être, c’est naturellement une chose si douce que les misérables mêmes ne veulent pas mourir, et quand ils se sentent misérables, ce n’est pas de leur être, mais de leur misère qu’ils souhaitent l’anéantissement. Voici des hommes qui se croient au comble du malheur, et qui sont en effet très malheureux, je ne dis pas au jugement des sages qui les estiment tels à cause de leur folies mais dans l’opinion de ceux qui se trouvent heureux et qui font consister le malheur des autres dans l’indigence et la pauvreté ; donnez à ces hommes le choix ou de demeurer toujours dans cet état de misère sans mourir, ou d’être anéantis, vous les verrez bondir de joie et s’arrêter au premier parti. J’en atteste leur propre sentiment. Pourquoi craignent-ils de mourir et aiment-ils mieux vivre misérablement que de voir finir leur misère par la mort, sinon parce que la nature abhorre le néant ? Aussi, lorsqu’ils sont près de mourir, ils regardent comme une grande faveur tout ce qu’on fait pour leur conserver la vie, c’est-à-dire pour prolonger leur misère. Par où ils montrent bien avec quelle allégresse ils recevraient l’immortalité, alors même qu’ils seraient certains d’être toujours malheureux. Mais quoi ! les animaux mêmes privés de raison, à qui ces pensées sont inconnues, tous depuis les immenses reptiles jusqu’aux plus petits vermisseaux, ne témoignent-ils pas, par tous les mouvements dont ils sont capables, qu’ils veulent être et qu’ils fuient le néant ? Les arbres et les plantes, quoique privés de sentiment, ne jettent-ils pas des racines en terre à proportion qu’ils s’élèvent dans l’air, afin d’assurer leur nourriture et de conserver leur être ? Enfin, les corps bruts, tout privés qu’ils sont et de sentiment et même de vie, tantôt s’élancent vers les régions d’en haut, tantôt descendent vers celles d’en bas, tantôt enfin se balancent dans une région intermédiaire, pour se maintenir dans leur être et dans les conditions de leur nature.\par
Pour ce qui est maintenant de l’amour que nous avons pour connaître et de la crainte qui nous est naturelle d’être trompés, j’en donnerai pour preuve qu’il n’est personne qui n’aime mieux l’affliction avec un esprit sain que la joie avec la démence. L’homme est le seul de tous les êtres mortels qui soit capable d’un sentiment si grand et si noble. Plusieurs animaux ont les yeux meilleurs que nous pour voir la lumière d’ici-bas ; mais ils ne peuvent atteindre à cette lumière spirituelle qui éclaire notre âme et nous fait juger sainement de toutes choses ; car nous n’en saurions juger qu’à proportion qu’elle nous éclaire. Remarquons toutefois que s’il n’y a point de science dans les bêtes, elles en ont du moins quelque reflet, au lieu que, pour le reste des êtres corporels, on ne les appelle pas sensibles parce qu’ils sentent, mais parce qu’on les sent, encore que les plantes, par la faculté de se nourrir et d’engendrer, se rapprochent quelque peu des créatures douées de sentiment. En définitive, toutes ces choses corporelles ont leurs causes secrètes dans la nature, et quant à leurs formes, qui servent à l’embellissement de ce monde visible, elles font paraître ces objets à nos sens, afin que s’ils ne peuvent connaître, ils soient du moins connus. Mais, quoique nos sens corporels en soient frappés, ce ne sont pas eux toutefois qui en jugent. Nous avons un sens intérieur beaucoup plus excellent, qui connaît ce (lui est juste et ce qui ne l’est pas, l’un par une idée intelligible, et l’autre par la privation de cette idée. Ce sens n’a besoin pour s’exercer ni de pupille, ni d’oreille, ni de narines, ni de palais, ni d’aucun toucher corporel. Par lui, je suis certain que je suis, que je connais que je suis, et que j’aime mon être et ma connaissance.
\subsection[{Chapitre XXVIII}]{Chapitre XXVIII}

\begin{argument}\noindent Si nous devons aimer l’amour même par lequel nous aimons notre être et notre connaissance, pour mieux ressembler à la Trinité.
\end{argument}

\noindent Mais c’en est assez sur notre être, notre connaissance, et l’amour que nous avons pour l’un et pour l’autre, aussi bien que sur la ressemblance qui se trouve à cet égard entre l’homme et les créatures inférieures. Quant à savoir si nous aimons l’amour même que nous avons pour notre être et notre connaissance, c’est ce dont je n’ai encore rien dit. Mais il est aisé de montrer que nous l’aimons en effet, puisqu’en ceux que nous aimons d’un amour plus pur et plus parfait, nous aimons cet amour-là encore plus que nous ne les aimons eux-mêmes. Car on n’appelle pas homme de bien celui qui sait ce qui est bon, mais celui qui l’aime. Comment donc n’aimerions-nous pas en nous l’amour même qui nous fait aimer tout ce que nous aimons de bon ? En effet, il y a un autre amour par lequel on aime ce qu’il ne faut pas aimer, et celui qui aime cet amour par lequel on aime ce qu’on doit aimer, hait cet autre amour-là. Le même homme peut les réunir tous les deux, et cette réunion luit est profitable lorsque l’amour qui fait que nous vivons bien augmente, et que l’autre diminue, jusqu’à ce qu’il soit entièrement détruit et que tout ce qu’il y a de vie en nous soit purifié. Si nous étions brutes, nous aimerions la vie de la chair et des sens, et ce bien suffirait pour nous rendre contents, sans que nous eussions la peine d’en chercher d’autres. Si nous étions arbres, quoique nous ne puissions rien aimer de ce qui flatte les sens, toutefois nous semblerions comme désirer tout ce qui pourrait nous rendre plus fertiles. De même encore, si nous étions pierres, flots, vent ou flamme, ou quelque autre chose semblable, nous serions privés à la vérité de vie et de sentiment, mais nous ne laisserions pas d’éprouver comme un certain désir de conserver le lieu et l’ordre où la nature nous aurait mis. Le poids des corps est comme leur amour, qu’il les fasse tendre en haut ou en bas ; et c’est ainsi que le corps, partout où il va, est entraîné par son poids comme l’esprit par son amour. Puis donc que nous sommes hommes, faits à l’image de notre Créateur, dont l’éternité est véritable, la vérité éternelle, et la charité éternelle et véritable, et qui est lui-même l’aimable, l’éternelle et la véritable Trinité, sans confusion ni division, parcourons tous ses ouvrages d’un regard pour ainsi dire immobile, et recueillons des traces plus ou moins profondes de sa divinité dans les choses qui sont au-dessous de nous et qui ne seraient en aucune façon, ni n’auraient aucune beauté, ni ne demanderaient et ne garderaient aucun ordre, si elles n’avaient été créées par celui qui possède un être souverain, une sagesse souveraine et une souveraine bonté. Quant à nous, après avoir contemplé son image en nous-mêmes, levons-nous et rentrons dans notre cœur, à l’exemplede l’enfant prodigue de l’Évangile ou pour retourner vers celui de qui nous nous étions éloignés par nos péchés. Là, notre être ne sera point sujet à la mort, ni notre connaissance à l’erreur, ni notre amour au dérèglement.\par
Et maintenant, bien que nous soyons assurés que ces trois choses sont en nous et que nous n’ayons pas besoin de nous en rapporter à d’autres, parce que nous les sentons et que nous en avons une évidence intérieure, toutefois, comme nous ne pouvons savoir par nous-mêmes combien de temps elles dureront, si elles ne finiront jamais et où elles doivent aller, selon le bon et le mauvais usage que nous en aurons fait, il y a lieu de chercher à cet égard (et nous en avons déjà trouvé) d’autres témoignages dont l’autorité ne souffre aucun doute, comme je le prouverai en son lieu. Ne fermons donc pas le présent livre sans achever ce que nous avions commencé d’expliquer touchant cette Cité de Dieu, qui n’est point sujette au pèlerinage de la vie mortelle, mais qui est toujours immortelle dans les cieux : parlons des saints anges demeurés pour jamais fidèles à Dieu et que Dieu sépara des anges prévaricateurs, devenus ténèbres pour s’être éloignés de la lumière éternelle.
\subsection[{Chapitre XXIX}]{Chapitre XXIX}

\begin{argument}\noindent De la science des anges qui ont connu la Trinité dans l’essence même de Dieu et les causes des œuvres divines dans l’art du divin ouvrier.
\end{argument}

\noindent Ces saints anges n’apprennent pas à connaître Dieu par des paroles sensibles, mais par la présence même de la parole immuable de la vérité, c’est-à-dire par le Verbe, Fils unique de Dieu, et ils connaissent le Verbe, et son Père, et leur Esprit, et cette Trinité inséparable où trois personnes distinctes ne font qu’une seule et même substance, de sorte qu’il n’y a pas trois dieux, mais un seul, ils connaissent cela plus clairement que nous ne nous connaissons nous-mêmes. C’est encore ainsi qu’ils connaissent les créatures, non en elles-mêmes, mais dans la sagesse de Dieu comme dans l’art qui les a produites ; par conséquent, ils se connaissent mieux en Dieu qu’en eux-mêmes, quoiqu’ils seconnaissent aussi en eux-mêmes. Mais comme ils ont été créés, ils sont autre chose que celui qui les a créés ; ainsi ils se connaissent en lui comme dans la lumière du jour, et en eux-mêmes comme dans celle du soir, ainsi que nous l’avons dit ci-dessus. Or, il y a une grande différence entre connaître une chose dans la raison qui est la cause de son être, ou la connaître en elle-même ; comme on connaît autrement les figures de mathématiques en les contemplant par l’esprit qu’en les voyant tracées sur le sable, ou comme la justice est autrement représentée dans la vérité immuable que dans l’âme du juste. Il en est ainsi de tous les objets de la connaissance : du firmament, que Dieu a étendu entre les eaux supérieures et les inférieures, et qu’il a nommé ciel, de la mer et de la terre, des herbes et des arbres, du soleil, de la lune et des étoiles, des animaux sortis des eaux, oiseaux, poissons et monstres marins, des animaux terrestres, tant quadrupèdes que reptiles, de l’homme même, qui surpasse en excellence toutes les créatures de la terre et de tout le reste. Toutes ces merveilles de la création sont autrement connues des anges dans le Verbe de Dieu, où elles ont leurs causes et leurs raisons éternellement subsistantes et selon lesquelles elles ont été faites qu’elles ne peuvent être connues en elles-mêmes. Ici, connaissance obscure qui n’atteint que les ouvrages de l’art ; là, connaissance claire qui atteint l’art lui-même ; et cependant ces ouvrages où s’arrête le regard de l’homme, quand on les rapporte à la louange et à la gloire du Créateur, il semble que, dans l’esprit qui les contemple, brille la lumière du matin.
\subsection[{Chapitre XXX}]{Chapitre XXX}

\begin{argument}\noindent De la perfection du nombre sénaire, qui, le premier de tous les nombres, se compose de ses parties.
\end{argument}

\noindent Or, l’Écriture dit que la création fut achevée en six jours, non que Dieu ait eu besoin de ce temps, comme s’il n’eût pu créer tous les êtres à la fois et leur faire ensuite marquer le cours du temps par des mouvements convenables ; mais le nombre sénaire exprime ici la perfection de l’ouvrage divin. Il est parmi tous les nombres le premier qui se compose de ses parties, je veux dire du sixième, du tiers et de la moitié de lui-même ; en effet, le sixième de six est un, le tiers est deux et la moitié est trois, or, un, deux et trois font six. Les parties dont je parle ici sont celles dont on peut préciser le rapport exact avec le nombre entier, comme la moitié, le tiers, le quart ou telle autre fraction semblable. Quatre, par exemple, n’est point partie aliquote de neuf, comme un, qui en est le neuvième, ou trois, qui en est le tiers ; d’un autre côté, le neuvième de neuf qui est un, et le tiers de neuf qui est trois, ajoutés ensemble, ne font pas neuf. Quatre est encore partie de dix, mais non partie aliquote, comme un qui en est le dixième. Deux en est le cinquième, cinq la moitié ; ajoutez maintenant ces trois parties, un, deux et cinq, vous formez non le total dix, mais le total huit. Au contraire, les parties additionnées du nombre douze le surpassent ; car, prenez le douzième de douze qui est un, le sixième qui est deux, le tiers qui est trois, le quart qui est quatre, et la moitié qui est six, vous obtenez, en ajoutant tout cela, non pas douze, mais seize. J’ai cru devoir toucher en passant cette question, afin de montrer la perfection du nombre sénaire, qui est, je le répète, le premier de tous qui se compose de la somme de ses parties. C’est dans ce nombre parfait que Dieu acheva ses ouvrages. On aurait donc tort de mépriser les explications qu’on peut tirer des nombres, et ceux qui y regardent de près reconnaissent combien elles sont considérables en plusieurs endroits de l’Écriture. Ce n’est pas en vain qu’elle a donné à Dieu cette louange : « Vous avez ordonné toutes choses avec poids, nombre et mesure. »
\subsection[{Chapitre XXXI}]{Chapitre XXXI}

\begin{argument}\noindent Du septième jour, qui est celui où Dieu se repose après l’accomplissement de ses ouvrages.
\end{argument}

\noindent Quant au septième jour, c’est-à-dire aumême jour répété sept fois, nombre qui est également parfait, quoique pour une autre raison, il marque le repos de Dieu, et il est le premier que Dieu ait sanctifié. Ainsi, Dieu n’a pas voulu sanctifier ce jour par ses ouvrages, mais par son repos, qui n’a point de soir, car il n’y a plus dès lors de créature, qui, étant connue dans le Verbe de Dieu autrement qu’en elle-même, constitue la distinction du jour en matin et en soir. Il y aurait beaucoup de choses à dire touchant la perfection du nombre sept ; mais ce livre est déjà long, et je crains que l’on ne m’accuse de vouloir faire un vain étalage de ma faible science. Je dois donc imposer une règle à mes discours, de peur que, parlant du nombre avec excès, il ne semble que je manque moi-même à la loi du nombre et de la mesure. Qu’il me suffise d’avertir ici que trois est le premier nombre impair, et quatre le premier pair, et que ces deux nombres pris ensemble font celui de sept. On l’emploie souvent par cette raison, pour marquer indéfiniment tous les nombres, comme quand il est dit : « Sept fois le juste tombera, et il se relèvera », c’est-à-dire, il ne périra point, quel que soit le nombre de ses chutes. Par où il ne faut pas entendre des péchés, mais des afflictions qui conduisent à l’humilité. Le Psalmiste dit aussi : « Je vous louerai sept fois le jour » ; ce qui est exprimé ailleurs ainsi : « Les louanges seront toujours en ma bouche. » Il y a beaucoup d’autres endroits semblables dans l’Écriture, où le nombre sept marque une généralité indéfinie. Il est encore souvent employé pour signifier le Saint-Esprit, dont Notre-Seigneur dit : « Il vous enseignera toute vérité. » En ce nombre est le repos de Dieu, je veux dire le repos qu’on goûte en Dieu ; car le repos se trouve dans le tout, c’est à savoir dans le plein accomplissement, et le travail dans la partie. Aussi la vie présente est-elle le temps du travail, parce que nous n’avons que des connaissances partielles ; mais lorsque ce qui est parfait sera arrivé, ce qui n’est que partiellement s’évanouira. De là vient encore que nous avons ici-bas de là peiné à découvrir le sens de l’Écriture ; mais il en est tout autrement des saints anges, dont la sociétéglorieuse fait l’objet de nos désirs dans ce laborieux pèlerinage : comme ils jouissent d’un état permanent et immuable, ils ont une facilité pour comprendre égale à la félicité de leur repos. C’est sans peine qu’ils nous aident, et leurs mouvements spirituels, libres et purs, ne leur coûtent aucun effort.
\subsection[{Chapitre XXXII}]{Chapitre XXXII}

\begin{argument}\noindent De ceux qui croient que la création des anges a précédé celle du monde.
\end{argument}

\noindent Quelqu’un prétendra-t-il que ces paroles de la Genèse : « Que la lumière soit faite, et la lumière fut faite », ne doivent point s’entendre de la création des anges, mais d’une lumière corporelle, quelle qu’elle soit ; et que les anges ont été créés, non seulement avant le firmament, mais aussi avant toute autre créature ? alléguera-t-il, à l’appui de cette opinion, que le premier verset de la Genèse ne signifie pas que le ciel et la terre furent les premières choses que Dieu créa, puisqu’il avait déjà créé les anges, mais que toutes choses furent créées dans sa sagesse, c’est-à-dire dans son Verbe, que l’Écriture nomme ici {\itshape Principe}, nom qu’il prend lui-même dans l’Évangile, lorsqu’il répond aux Juifs qui lui demandaient qui il était. Je ne combattrai point cette interprétation, à cause de la vive satisfaction que j’éprouve à voir la Trinité marquée dès le commencement du saint livre de la Genèse. On y lit, en effet : « Dans le principe, Dieu créa le ciel et la terre », ce qui peut signifier que le Père a créé le monde dans son Fils, suivant ce témoignage du psaume : « Que vos œuvres, Seigneur, sont magnifiques ! Vous avez fait toutes choses dans votre sagesse. » Aussi bien l’Écriture ne tarde pas à faire mention du Saint-Esprit. Après avoir décrit la terre, telle que Dieu l’a créée primitivement, c’est-à-dire cette masse ou matière que Dieu avait préparée sous le nom du ciel et de la terre pour la structure de l’univers, après avoir dit : « Or, la terre était invisible et informe, et les ténèbres étaient répandues sur l’abîme » ; elle ajoute aussitôt, comme pour compléter le nombre des personnes de la Trinité : « Et l’Esprit de Dieu était porté sur les eaux. » Chacun, au reste, est libre d’entendre comme il le voudra ces paroles si obscures et si profondes qu’on en peut faire sortir beaucoup d’opinions différentes toutes conformes à la foi, pourvu cependant qu’il soit bien entendu que les saints anges, sans être coéternels à Dieu, sont certains de leur véritable et éternelle félicité. C’est à la société bienheureuse de ces anges qu’appartiennent les petits enfants dont parle le Seigneur, quand il dit : « Ils seront les égaux des anges du ciel. » Il nous apprend encore de quelle félicité les anges jouissent au ciel, par ces paroles : « Prenez garde de ne mépriser aucun de ces petits ; car je vous déclare que leurs anges voient sans cesse la face de mon Père, qui est dans les cieux. »
\subsection[{Chapitre XXXIII}]{Chapitre XXXIII}

\begin{argument}\noindent On peut entendre par la lumière et les ténèbres les deux sociétés contraires des bons et des mauvais anges.
\end{argument}

\noindent Que certains anges aient péché et qu’ils aient été précipités dans la plus basse partie du monde, où ils sont comme en prison jusqu’à la condamnation suprême, c’est ce que l’apôtre saint Pierre montre clairement lorsqu’il dit que Dieu n’a point épargné les anges prévaricateurs, mais qu’il les a précipités dans les prisons obscures de l’enfer, en attendant qu’il les punisse au jour du jugement. Qui doutera dès lors que Dieu, soit dans sa prescience, soit dans le fait, n’ait séparé les mauvais anges d’avec les bons ? et qui niera que ces derniers ne soient fort bien appelés lumière, alors que l’Apôtre nous donne ce nom, à nous qui ne vivons encore que par la foi et qui espérons, il est vrai, devenir les égaux des anges, mais ne le sommes pas encore ? « Autrefois, dit-il, vous étiez ténèbres, mais maintenant vous êtes lumière en Notre-Seigneur. » À l’égard des mauvais anges, quiconque sait qu’ils sont au-dessous des hommes infidèles, reconnaîtra que l’Écriture les a pu nommer très justement ténèbres. Ainsi, quand on devrait prendre lumière et ténèbres au sens littéral dans ces passages de la Genèse : « Dieu dit : Que la lumière soit faite, et la lumière fut faite. » — « Dieu sépara la lumière des ténèbres », on ne saurait toutefois nous blâmer de reconnaître ici les deux sociétés des anges : l’une qui jouit de Dieu, et l’autre qui est enflée d’orgueil ; l’une à qui l’on dit : « Vous tous qui êtes ses anges, adorez-le » ; et l’autre qui ose dire par la bouche de son prince : « Je vous donnerai tout cela, si vous voulez vous prosterner devant moi et m’adorer » ; l’une embrasée du saint amour de Dieu, et l’autre consumée de l’amour impur de sa propre grandeur ; l’une habitant dans les cieux des cieux, et l’autre précipitée de ce bienheureux séjour et reléguée dans les plus basses régions de l’air, suivant ce qui est écrit que « Dieu résiste aux superbes et donne sa grâce aux humbles » ; l’une tranquille et doucement animée d’une piété lumineuse, l’autre turbulente et agitée d’aveugles convoitises ; l’une qui secourt avec bonté et punit avec justice, selon le bon plaisir de Dieu, et l’autre à qui son orgueil inspire une passion furieuse de nuire et de dominer ; l’une ministre de la bonté de Dieu pour faire du bien autant qu’il lui plaît, et l’autre liée par la puissance de Dieu pour ne pas nuire autant qu’elle voudrait ; la première enfin se riant de la seconde et de ses vains efforts pour entraver son glorieux progrès à travers les persécutions, et celle-ci consumée d’envie quand elle voit sa rivale recueillir partout des pèlerins. Et maintenant que, d’après d’autres passages de l’Écriture qui nous représentent plus clairement ces deux sociétés contraires, l’une bonne par sa nature et par sa volonté, et l’autre mauvaise par sa volonté, quelque bonne par sa nature, nous avons cru les voir marquées dans ce premier chapitre de la Genèse sous les noms de lumière et de ténèbres, si nous supposons que telle n’ait pas été la pensée de l’écrivain sacré, il n’en résulte pas que nous ayons perdu le temps en paroles inutiles ; car enfin, bien que le texte reste obscur, la règle de la foi n’a pas été atteinte et elle est assez claire aux fidèles par d’autres endroits. Si en effet le livre de la Genèse ne fait mention que des ouvrages corporels de Dieu, ces ouvrages-mêmes ne laissent pas d’avoir quelque rapport avec les spirituels, suivant cette parole de saint Paul : « Vous êtes tous enfants de lumière et enfants du jour ; nous ne sommes pas enfants de la nuit ni des ténèbres. » Et si, au contraire, l’écrivain sacré a eu les pensées que nous lui supposons, alors le commentaire auquel nous nous sommes livré en tire une nouvelle force, et il faut conclure que cet homme de Dieu, tout pénétré d’une sagesse divine, ou plutôt que l’esprit de Dieu qui parlait en lui n’a pas oublié les anges dans l’énumération des ouvrages de Dieu, soit que par ces mots : « Dans le principe, Dieu créa le ciel et la terre », on entende que Dieu créa les anges dès le principe, c’est-à-dire dès le commencement, soit, ce qui me paraît plus raisonnable, qu’on entende qu’il les créa dans le Verbe de Dieu, son Fils unique, en qui il a créé toutes choses. De même, par {\itshape le ciel et la terre}, on peut entendre toutes les créatures, tant spirituelles que corporelles, explication la plus vraisemblable, ou ces deux grandes parties du monde corporel qui contiennent tout le reste des êtres, et que Moïse mentionne d’abord en général, pour en faire ensuite une description détaillée selon le nombre mystique des six jours.
\subsection[{Chapitre XXXIV}]{Chapitre XXXIV}

\begin{argument}\noindent De ceux qui croient que par les eaux que sépara le firmament il faut entendre les anges, et de quelques autres qui pensent que les eaux n’ont point été créées.
\end{argument}

\noindent Quelques-uns ont cru que les eaux, dans la Genèse, désignent la légion des anges, et que c’est ce qu’on doit entendre par ces paroles : « Que le firmament soit fait entre l’eau et l’eau » ; en sorte que les eaux supérieures seraient les bons anges, et que par les eaux inférieures il faudrait entendre, soit les eaux visibles, soit les mauvais anges, soit toutes les nations de la terre. À ce compte, la Genèse ne nous dirait pas quand les anges ont été créés, mais quand ils ont été séparés. Mais croira-t-on qu’il se soit trouvé des espritsassez frivoles et assez impies pour nier que Dieu ait créé les eaux, sous prétexte qu’il n’est écrit nulle part : Dieu dit : Que les eaux soient faites ? Par la même raison, ils pourraient en dire autant de la terre, puisqu’on ne lit nulle part : Dieu dit : Que la terre soit faite. Mais, objectent ces téméraires, il est écrit : « Dans le principe, Dieu créa le ciel et la terre. » Que conclure de là ? que l’eau est ici sous-entendue, et qu’elle est comprise avec la terre sous un même nom. Car « la mer est à lui », dit le Psalmiste, « et c’est lui qui l’a faite ; et ses mains ont formé la terre ». Pour revenir à ceux qui veulent que, par les eaux qui sont au-dessus des cieux, on entende les anges, ils n’adoptent cette opinion qu’à cause de la nature à la fois pesante et liquide de cet élément, qu’ils ne croient pas pouvoir demeurer ainsi suspendu. Mais cela prouve simplement que s’ils pouvaient faire un homme, ils ne mettraient pas dans sa tête le flegme ou la pituite, laquelle joue le rôle de l’eau dans les quatre éléments dont notre corps est composé. Cependant, la tête n’en reste pas moins le siège de la pituite, et cela est fort bien ordonné. Quant au raisonnement de ces esprits hasardeux, il est tellement absurde que si nous ignorions ce qui en est et qu’il fût écrit de même dans le livre de la Genèse que Dieu a mis un liquide froid et par conséquent pesant dans la plus haute partie du corps de l’homme, ces peseurs d’éléments ne le croiraient pas et diraient que c’est une expression allégorique. Mais si nous voulions examiner en particulier tout ce qui est contenu dans ce récit divin de la création du monde, l’entreprise demanderait trop de temps et nous mènerait trop loin. Comme il nous semble avoir assez parlé de ces deux sociétés contraires des anges, où se trouvent quelques commencements des deux cités dont nous avons dessein de traiter dans la suite, il est à propos de terminer ici ce livre.
\section[{Livre douzième. L’ange et l’homme}]{Livre douzième. \\
L’ange et l’homme}\renewcommand{\leftmark}{Livre douzième. \\
L’ange et l’homme}

\subsection[{Chapitre premier}]{Chapitre premier}

\begin{argument}\noindent La nature des anges, bons et mauvais, est une.
\end{argument}

\noindent Avant de parler de la création de l’homme, avant de montrer les deux cités se formant parmi les êtres raisonnables et mortels, comme on les a vues, dans le livre précédent, se former parmi les anges, il me reste encore quelques mots à dire pour faire comprendre que la société des anges avec les hommes n’a rien d’impossible, de sorte qu’il n’y a pas quatre cités, quatre sociétés, deux pour les anges et autant pour les hommes, mais deux cités en tout, l’une pour les bons, l’autre pour les méchants, anges ou hommes, peu importe.\par
Que les inclinations contraires des bons et des mauvais anges proviennent, non de la différence de leur nature et de leur principe, puisqu’ils sont les uns et les autres l’œuvre de Dieu, auteur et créateur excellent de toutes les substances, mais de la diversité de leurs désirs et de leur volonté, c’est ce qu’il n’est pas permis de révoquer en doute. Tandis que les uns, attachés au bien qui leur est commun à tous, lequel n’est autre que Dieu même, se maintiennent dans sa vérité, dans son éternité, dans sa charité, les autres, trop charmés de leur propre puissance, comme s’ils étaient à eux-mêmes leur propre bien, de la hauteur du bien suprême et universel, source unique de la béatitude, sont tombés dans leur bien particulier, et, remplaçant par une élévation fastueuse la gloire éminente de l’éternité, par une vanité pleine d’astuce la solide vérité, par l’esprit de faction qui divise, la charité qui unit, ils sont devenus superbes, fallacieux, rongés d’envie. Quelle est donc la cause de la béatitude des premiers ? leur union avec Dieu ; et celle, au contraire, de la misère des autres ? leur séparation de Dieu. Si donc il faut répondre à ceux qui demandent pourquoi les uns sont heureux : c’est qu’ils sont unis à Dieu, et à ceux qui veulent savoir pourquoi les autres sont malheureux : c’est qu’ils sont séparés de Dieu, il s’ensuit qu’il n’y a pour la créature raisonnable ou intelligente d’autre bien ni d’autre source de béatitude que Dieu seul. Ainsi donc, quoique toute créature ne puisse être heureuse (car une bête, une pierre, du bois et autres objets semblables sont incapables de félicité), celle qui le peut, ne le peut point par elle-même, étant créée de rien, mais par celui qui l’a créée. Le même objet, dont la possession la rend heureuse, par son absence la fait misérable ; au lieu que l’être qui est heureux, non par un autre, mais par soi, ne peut être malheureux, parce qu’il ne peut être absent de soi.\par
Nous disons donc qu’il n’y a de bien entièrement immuable que Dieu seul dans son unité, sa vérité et sa béatitude, et quant à ses créatures, qu’elles sont bonnes parce qu’elles viennent de lui, mais muables, parce qu’elles ont été tirées, non de sa substance, mais du néant. Si donc aucune d’elles ne peut jamais être souverainement bonne, puisque Dieu est infiniment au-dessus, elles sont pourtant très bonnes, quoique muables, ces créatures choisies qui peuvent trouver la béatitude dans leur union avec le bien immuable, lequel est si essentiellement leur bien, que sans lui elles ne sauraient être que misérables. Et il ne faut pas conclure de là que le reste des créatures répandues dans cet immense univers, ne pouvant pas être misérables, en soient meilleures pour cela ; car on ne dit pas que les autres membres de notre corps soient plus nobles que les yeux, sous prétexte qu’ils ne peuvent devenir aveugles ; mais tout comme la nature sensible est meilleure, lors même qu’elle souffre, que la pierre qui ne peut souffrir en aucune façon, ainsi la nature raisonnable l’emporte, quoique misérable, sur celle qui est privée de raison ou de sentiment et qui est à cause de cela incapable de misère. S’il en va de la sorte, puisque cette créature a un tel degré d’excellence que sa mutabilité ne l’empêche pas de trouver la béatitude dans son union avec le souverain bien, et puisqu’elle ne peut ni combler son indigence qu’en étant souverainement heureuse, ni être heureuse que par Dieu, il faut conclure que, pour elle, ne pas s’unir à Dieu, c’est un vice. Or, tout vice nuit à la nature et par conséquent lui est contraire. Dès lors la créature qui ne s’unit pas à Dieu diffère de celle qui s’unit à lui non par nature, mais par vice. Et ce vice même marque la grandeur et la dignité de sa nature, le vice étant blâmable et odieux par cela même qu’il déshonore la nature. Lorsqu’on dit que la cécité est le vice des yeux, on témoigne que la vue leur est naturelle, et lorsqu’on dit que la surdité est le vice des oreilles, on affirme que l’ouïe appartient à leur nature ; de même donc, lorsqu’on dit que le vice de la créature angélique est de ne pas être unie à Dieu, on déclare qu’il est de sa nature de lui être unie. Quelle gloire plus haute que d’être uni à Dieu de telle sorte qu’on vive pour lui, qu’on n’ait de sagesse et de joie que par lui, et qu’on possède un si grand bien sans que la mort, l’erreur et la souffrance puissent nous le ravir ! comment élever sa pensée à ce comble de béatitude, et qui trouvera des paroles pour l’exprimer dignement ? Ainsi, tout vice étant nuisible à la nature, le vice même des mauvais anges, qui les tient séparés de Dieu, fait éclater l’excellence de leur nature, à qui rien ne peut nuire que de ne pas s’attacher à Dieu.
\subsection[{Chapitre II}]{Chapitre II}

\begin{argument}\noindent Aucune essence n’est contraire à dieu, tout ce qui n’est pas différant absolument de celui qui est souverainement et toujours.
\end{argument}

\noindent J’ai dit tout cela de peur qu’on ne se persuade, quand je parle des anges prévaricateurs, qu’ils ont pu avoir une autre nature que celle des bons anges, la tenant d’un autre principe et n’ayant point Dieu pour auteur. Or, il sera d’autant plus aisé de se défendre de cette erreur impie que l’on comprendra mieux ce que Dieu dit par la bouche d’un ange, quand il envoya Moïse vers les enfants d’Israël : « Je suis celui qui suis. » Dieu, en effet, étant l’essence souveraine, c’est-à-dire étant souverainement et par conséquent étant immuable, quand il a créé les choses de rien, il leur a donné l’être, à la vérité, mais non l’être suprême qui est le sien ; il leur a donné l’être, dis-je, aux unes plus, aux autres moins, et c’est ainsi qu’il a établi des degrés dans les natures des essences. De même que du mot {\itshape sapere} s’est formé {\itshape sapientia}, ainsi du mot esse on a tiré {\itshape essentia}, mot nouveau en latin, dont les anciens auteurs ne se sont pas servis, mais qui est entré dans l’usage pour que nous eussions un terme correspondant à l’{\itshape ousia} des Grecs. Il suit de là qu’aucune nature n’est contraire à cette nature souveraine qui a fait être tout ce qui est, aucune, dis-je, excepté celle qui n’est pas. Car le non-être est le contraire de l’être. Et, par conséquent, il n’y a point d’essence qui soit contraire à Dieu, c’est-à-dire à l’essence suprême, principe de toutes les essences, quelles qu’elles soient.
\subsection[{Chapitre III}]{Chapitre III}

\begin{argument}\noindent Les ennemis de Dieu ne le sont point par leur nature, mais par leur volonté.
\end{argument}

\noindent L’Écriture appelle ennemis de Dieu ceux qui s’opposent à son empire, non par leur nature, mais par leurs vices ; or, ce n’est point à Dieu qu’ils nuisent, mais à eux-mêmes. Car ils sont ses ennemis par la volonté de lui résister, non par le pouvoir d’y réussir. Dieu, en effet, est immuable et par conséquent inaccessible à toute dégradation. Ainsi donc le vice qui fait qu’on résiste à Dieu est un mal, non pour Dieu, mais pour ceux qu’on appelle ses ennemis. Et pourquoi cela, sinon parce que ce vice corrompt en eux un bien, savoir le bien de leur nature ? Ce n’est donc pas la nature, mais le vice qui est contraire à Dieu. Ce qui est mal, en effet, est contraire au bien. Or, qui niera que Dieu ne soit le souverain bien ? Le vice est donc contraire à Dieu, comme le mal au bien. Cette nature, que le vice a corrompue, est aussi un bien sans doute, et, par conséquent, le vice est absolument contraire à ce bien ; mais voici la différence :s’il est contraire à Dieu, c’est seulement comme mal, tandis qu’il est contraire doublement à la nature corrompue, comme mal et comme chose nuisible. Le mal, en effet, ne peut nuireà Dieu ; il n’atteint que les natures muables et corruptibles, dont la bonté est encore attestée par leurs vices mêmes ; car si elles n’étaient pas bonnes, leurs vices ne pourraient leur être nuisibles. Comment leur nuisent-ils, en effet ? n’est-ce pas en leur ôtant leur intégrité, leur beauté, leur santé, leur vertu, en un mot tous ces biens de la nature que le vice a coutume de détruire ou de diminuer ? Supposez qu’elles ne renfermassent aucun bien, alors le vice, ne leur ôtant rien, ne leur nuirait pas, et partant, il ne serait plus un vice ; car il est de l’essence du vice d’être nuisible. D’où il suit que le vice, bien qu’il ne puisse nuire au bien immuable, ne peut nuire cependant qu’à ce qui renferme quelque bien, le vice ne pouvant être qu’où il nuit. Dans ce sens, on peut dire encore qu’il est également impossible au vice d’être dans le souverain bien et d’être ailleurs que dans un bien. Il n’y a donc que le bien qui puisse être seul quelque part ; le mal, en soi, n’existe pas. En effet, ces natures mêmes qui ont été corrompues par le vice d’une mauvaise volonté elles sont mauvaises, à la vérité, en tant que corrompues, mais, en tant que natures, elles sont bonnes. Et quand une de ces natures corrompues est punie, outre ce qu’elle renferme de bien, en tant que nature, il y a encore en elle cela de bien qu’elle n’est pas impunie. La punition est juste, en effet, et tout ce qui est juste est un bien. Nul ne porte la peine des vices naturels, mais seulement des volontaires, car le vice même, qui par le progrès de l’habitude est devenu comme naturel, a son principe dans la volonté. Il est entendu que nous ne parlons en ce moment que des vices de cette créature raisonnable où brille la lumière intelligible qui fait discerner le juste et l’injuste.
\subsection[{Chapitre IV}]{Chapitre IV}

\begin{argument}\noindent Les natures privées de raison et de vie, considérées dans leur genre et à leur place, n’altèrent point la beauté de l’univers.
\end{argument}

\noindent Condamner les défauts des bêtes, des arbres et des autres choses muables et mortelles, privées d’intelligence, de sentiment ou de vie, sous prétexte que ces défauts les rendent sujettes à se dissoudre et à se corrompre, c’estune absurdité ridicule. Ces créatures, en effet, ont reçu leur manière d’être de la volonté du Créateur, afin d’accomplir par leurs vicissitudes et leur succession cette beauté inférieure de l’univers qui est assortie, dans son genre, à tout le reste. Il ne convenait pas que les choses de la terre fussent égales aux choses du ciel, et la supériorité de celles-ci n’était pas une raison de priver l’univers de celles-là. Lors donc que nous voyons certaines choses périr pour faire place à d’autres qui naissent, les plus faibles succomber sous les plus fortes, et les vaincues servir en se transformant aux qualités de celles qui triomphent, tout cela en son lieu et à son heure, c’est l’ordre des choses qui passent. Et si la beauté de cet ordre ne nous plaît pas, c’est que liés par notre condition mortelle à une partie de l’univers changeant, nous ne pouvons en sentir l’ensemble où ces fragments qui nous blessent trouvent leur place, leur convenance et leur harmonie. C’est pourquoi dans les choses où nous ne pouvons saisir aussi distinctement la providence du Créateur, il nous est prescrit de la conserver par la foi, de peur que la vaine témérité de notre orgueil ne nous emporte à blâmer par quelque endroit l’œuvre d’un si grand ouvrier. Aussi bien, si l’on considère d’un regard attentif les défauts des choses corruptibles, je ne parle pas de ceux qui sont l’effet de notre volonté ou la punition de nos fautes, on reconnaîtra qu’ils prouvent l’excellence de ces créatures, dont il n’est pas une qui n’ait Dieu pour principe et pour auteur ; car c’est justement ce qui nous plaît dans leur nature que nous ne pouvons voir se corrompre et disparaître sans déplaisir, à moins que leur nature elle-même ne nous déplaise, comme il arrive souvent quand il s’agit de choses qui nous sont nuisibles et que nous considérons, non plus en elles-mêmes, mais par rapport à notre utilité, par exemple, ces animaux que Dieu envoya aux Égyptiens en abondance pour châtier leur orgueil. Mais à ce compte on pourrait aussi blâmer le soleil ; car il arrive que certains malfaiteurs ou mauvais débiteurs sont condamnés par les juges à être exposés au soleil. C’est donc la nature considérée en soi et non par rapport à nos convenances qui fait la gloire de son Créateur. Ainsi la nature du feu éternel est très certainement bonne, bien qu’elle doive servir au supplicedes damnés. Qu’y a-t-il en effet de plus beau que le feu, comme principe de flamme, de vie et de lumière ? quoi de plus utile, comme propre à échauffer, à cuire, à purifier ? Et cependant, il n’est rien de plus fâcheux que ce même feu, quand il nous brûle. Ainsi donc, nuisible en de certains cas, il devient, quand on en fait un usage convenable, d’une utilité singulière ; et qui pourrait trouver des paroles pour dire tous les services qu’il rend à l’univers ? Il ne faut donc point écouter ceux qui louent la lumière du feu et blâment son ardeur ; car ils en jugent, non d’après sa nature, mais selon leur commodité, étant bien aises de voir clair et ne l’étant pas de brûler. Ils ne considèrent pas que cette lumière qui leur plaît blesse les yeux malades, et que cette ardeur qui leur déplaît donne la vie et la santé à certains animaux.
\subsection[{Chapitre V}]{Chapitre V}

\begin{argument}\noindent Toute nature de toute espèce et de tout mode honore le Créateur.
\end{argument}

\noindent Ainsi toutes les natures, dès là qu’elles sont, ont leur mode, leur espèce, leur harmonie intérieure, et partant sont bonnes. Et comme elles sont placées au rang qui leur convient selon l’ordre de leur nature, elles s’y maintiennent. Celles qui n’ont pas reçu un être permanent sont changées en mieux ou en pis, selon le besoin et le mouvement des natures supérieures où les absorbe la loi du Créateur, allant ainsi vers la fin qui leur est assignée dans le gouvernement général de l’univers, de telle sorte toutefois que le dernier degré de dissolution des natures muables et mortelles n’aille pas jusqu’à réduire l’être au néant et à empêcher ce qui n’est plus de servir de germe à ce qui va naître. S’il en est ainsi, Dieu, qui est souverainement, et qui, pour cette raison, a fait toutes les essences, lesquelles ne peuvent être souverainement, puisqu’elles ne peuvent ni lui être égales, ayant été faites de rien, ni exister d’aucune façon s’il ne leur donne l’existence, Dieu, dis-je, ne doit être blâmé pour les défauts d’aucune des natures créées, et toutes, au contraire, doivent servir à l’honorer.
\subsection[{Chapitre VI}]{Chapitre VI}

\begin{argument}\noindent De la cause de la félicité des bons anges et de la misère des mauvais.
\end{argument}

\noindent Ainsi la véritable cause de la béatitude des bons anges, c’est qu’ils s’attachent à celui qui est souverainement, et la véritable cause de la misère des mauvais anges, c’est qu’ils se sont détournés de cet Être souverain pour se tourner vers eux-mêmes. Ce vice n’est-il pas ce qu’on appelle orgueil ? Or, « l’orgueil est le commencement de tout péché ». Ils n’ont pas voulu rapporter à Dieu leur grandeur ; et lorsqu’il ne tenait qu’à eux d’agrandir leur être, en s’attachant à celui qui est souverainement, ils ont préféré ce qui a moins d’être, en se préférant à lui. Voilà la première défaillance et le premier vice de cette nature qui n’avait pas été créée pour posséder la perfection de l’être, et qui néanmoins pouvait être heureuse par la jouissance de l’Être souverain, tandis que sa désertion, sans la précipiter, il est vrai, dans le néant, l’a rendue moindre qu’elle n’était, et par conséquent misérable. Demandera-t-on la cause efficiente de cette mauvaise volonté ? il n’y en a point. Rien ne fait la volonté mauvaise, puisque c’est elle qui fait ce qui est mauvais. La mauvaise volonté est donc la cause d’une mauvaise action ; mais rien n’est la cause de cette mauvaise volonté. En effet, si quelque chose en est la cause, cette chose a quelque volonté, ou elle n’en a point, et si elle a une volonté, elle l’a bonne ou mauvaise. Bonne, cela est impossible, car alors la bonne volonté serait cause du péché, ce qu’on ne peut avancer sans une absurdité monstrueuse. Mauvaise, je demande qui l’a faite ; en d’autres termes, je demande la cause de la première volonté mauvaise, car cela ne peut pas aller à l’infini ; en effet, une mauvaise volonté, née d’une autre mauvaise volonté, n’est pas quelque chose de premier, et il n’y a de première volonté mauvaise que celle qui n’est causée par aucune autre. Si on répond que cette première volonté mauvaise n’a pas de cause et qu’ainsi elle a toujours été, je demande si elle a été dans quelque nature. Si elle n’a été en aucune nature, elle n’a point été en effet, et si elle a été en quelque nature, elle la corrompait, elle lui était nuisible, elle la privait du bien ; par conséquent la mauvaise volonté ne pouvait être dans une mauvaise nature ; elle ne pouvait être que dans une nature bonne, et en même temps muable, qui pût être corrompue par le vice. Car si le vice ne l’eût pas corrompue, c’est qu’il n’y aurait pas eu de vice, et dès lors il n’y auraitpas eu non plus de mauvaise volonté. Si donc le vice l’a corrompue, ce n’a été qu’en ôtant ou diminuant le bien qui était en elle. Il n’est donc pas possible qu’il y ait eu éternellement une mauvaise volonté dans une chose où il y avait auparavant un bien naturel que cette mauvaise volonté a altéré en le corrompant. Si donc cette mauvaise volonté n’a pas été éternelle, je demande qui l’a faite. Tout ce qu’il reste à supposer, c’est que cette volonté ait été rendue mauvaise par une chose en qui il n’y avait point de volonté. Or, je demande si cette chose est supérieure, ou inférieure, ou égale. Supérieure, elle est meilleure. Comment, dès lors, n’a-t-elle aucune volonté ? comment n’en a-t-elle pas une bonne ? De même, si elle est égale, puisque tant que deux choses ont une bonne volonté, l’une n’en produit point de mauvaise dans l’autre. Il reste que le principe de la mauvaise volonté de la nature angélique, qui a péché la première, soit une chose inférieure à cette nature et privée elle-même de volonté. Mais cette chose, quelque inférieure qu’elle soit, quand ce ne serait que de la terre, le dernier et le plus bas des éléments, ne laisse pas, en sa qualité de nature et de substance, d’être bonne et d’avoir sa mesure et sa beauté dans son genre et dans son ordre. Comment donc une bonne chose peut-elle produire une mauvaise volonté ? comment, je le répète, un bien peut-il être cause d’un mal ? Lorsque la volonté quitte ce qui est au-dessus d’elle pour se tourner vers ce qui lui est inférieur, elle devient mauvaise, non parce que la chose vers laquelle elle se tourne est mauvaise, mais parce que c’est un mal que de s’y tourner. Ainsi ce n’est pas une chose inférieure qui a fait la volonté mauvaise, mais c’est la volonté même qui s’est rendue mauvaise en se portant irrégulièrement sur une chose inférieure. Que deux personnes également disposées de corps et d’esprit voient un beau corps, que l’une le regarde avec des yeux lascifs, tandis que l’autre conserve un cœur chaste, d’où vient que l’une a cette mauvaise volonté, et que l’autre ne l’a pas ? Quelle est la cause de ce désordre ? ce n’est pas la beauté du corps, puisque toutes deux l’ont vue également et que toutes deux n’en ont pas été également touchées ; ce n’est point non plus la différente disposition du corps ou de l’esprit de ces deux personnes, puisque nous les supposons également disposées. Dirons-nous que l’une a été tentée par une secrète suggestion du malin esprit ? comme si ce n’était pas par sa volonté qu’elle a consenti à cette suggestion ! C’est donc ce consentement de sa volonté dont nous recherchons la cause. Pour ôter toute difficulté, supposons que toutes deux soient tentées de même, que l’une cède à la tentation et que l’autre y résiste, que peut-on dire autre chose, sinon que l’une a voulu demeurer chaste et que l’autre ne l’a pas voulu ? Et comment cela s’est-il fait, sinon par leur propre volonté, attendu que nous supposons la même disposition de corps et d’esprit en l’une et en l’autre ? Toutes deux ont vu la même beauté, toutes deux ont été également tentées ; qui a donc produit cette mauvaise volonté en l’une des deux ? Certainement, si nous y regardons de près, nous trouverons que rien n’a pu la produire. Dirons-nous qu’elle-même l’a produite ? mais qu’était-elle elle-même avant cette mauvaise volonté, si ce n’est une bonne nature, dont Dieu, qui est le bien immuable, est l’auteur ? Comment, étant bonne avant cette mauvaise volonté, a-t-elle pu faire cette volonté mauvaise ? Est-ce en tant que nature, ou en tant que nature tirée du néant ? Qu’on y prenne garde, on verra que c’est à ce dernier titre. Car si la nature était cause de la mauvaise volonté, ne serions-nous pas obligés de dire que le mal ne vient que du bien, et que c’est le bien qui est cause du mal ? Or, comment se peut-il faire qu’une nature bonne, quoique muable, fasse quelque chose de mal, c’est-à-dire produise une mauvaise volonté, avant que d’avoir cette mauvaise volonté ?
\subsection[{Chapitre VII}]{Chapitre VII}

\begin{argument}\noindent Il ne faut point chercher de cause efficiente de la mauvaise volonté.
\end{argument}

\noindent Que personne ne cherche donc une cause efficiente de la mauvaise volonté. Cette cause n’est point positive, efficiente, mais négative, déficiente, parce que la volonté mauvaise n’est point une action, mais un défaut d’action. Déchoir de ce qui est souverainement vers ce qui a moins d’être, c’est commencer à avoir une mauvaise volonté. Or, il ne faut pas chercher une cause efficiente à cette défaillance, pas plus qu’il ne faut chercher à voir la nuit ou à entendre le silence. Ces deux choses nous sont connues pourtant, et ne nous sont connues qu’à l’aide des yeux et des oreilles ; mais ce n’est point par leurs espèces, c’est par la privation de ces espèces. Ainsi, que personne ne me demande ce que je sais ne pas savoir, si ce n’est pour apprendre de moi qu’on ne le saurait savoir. Les choses qui ne se connaissent que par leur privation ne se connaissent, pour ainsi dire, qu’en ne les connaissant pas. En effet, lorsque la vue se promène sur les objets sensibles, elle ne voit les ténèbres que quand elle commence à rien voir. Les oreilles de même n’entendent le silence que lorsqu’elles n’entendent rien. Il en est ainsi des choses spirituelles. Nous les concevons par notre entendement ; mais, lorsqu’elles viennent à manquer, nous ne les concevons qu’en ne les concevant pas, car : « {\itshape Qui peut comprendre les péchés} ? »
\subsection[{Chapitre VIII}]{Chapitre VIII}

\begin{argument}\noindent De l’amour déréglé par lequel la volonté se détache du bien immuable pour un bien muable.
\end{argument}

\noindent Ce que je sais, c’est que la nature de Dieu n’est point sujette à défaillance, et que les natures qui ont été tirées du néant y sont sujettes ; et toutefois, plus ces natures ont d’être et font de bien, plus leurs actions sont réelles et ont des causes positives et efficientes ; au contraire, quand elles défaillent et par suite font du mal, leurs actions sont vaines et n’ont que des causes négatives. Je sais encore que la mauvaise volonté n’est en celui en qui elle est que parce qu’il le veut, et qu’ainsi on punit justement une défaillance qui est entièrement volontaire. Cette défaillance ne consiste pas en ce que la volonté se porte vers une mauvaise chose, puisqu’elle ne peut se porter que vers une nature, et que toutes les natures sont bonnes, mais parce qu’elle s’y porte mal, c’est-à-dire contre l’ordre même des natures, en quittant ce qui est souverainement pour tendre vers ce qui a moins d’être. L’avarice, par exemple, n’est pas un vice inhérent à l’or, mais à celui qui aime l’or avec excès, en abandonnant pour cemétal la justice qui doit lui être infiniment préférée. De même l’impureté n’est pas le vice des corps qui ont de la beauté, mais celui de l’âme qui aime les voluptés corporelles d’un amour déréglé, en négligeant la tempérance qui nous unit à des choses bien plus belles, parce qu’elles sont spirituelles et incorruptibles. La vaine gloire aussi n’est pas le vice des louanges humaines, mais celui de l’âme qui méprise le témoignage de sa conscience et ne se soucie que d’être louée des hommes. Enfin l’orgueil n’est pas le vice de celui qui donne la puissance, ou la puissance elle-même, mais celui de l’âme qui a une passion désordonnée pour sa propre puissance, au mépris d’une puissance plus juste. Ainsi, quiconque aime mal un bien de quelque nature qu’il soit, ne laisse pas, tout en le possédant, d’être mauvais et misérable dans le bien même, parce qu’il est privé d’un bien plus grand,
\subsection[{Chapitre IX}]{Chapitre IX}

\begin{argument}\noindent Si Dieu est l’auteur de la bonne volonté des anges aussi bien que de leur nature.
\end{argument}

\noindent Il n’y a donc point de cause efficiente, ou, s’il est permis de le dire, de cause essentielle de la mauvaise volonté, puisque c’est d’elle-même que prend naissance le mal qui corrompt le bien de la nature ; or, rien ne rend la mauvaise volonté telle, sinon la défaillance qui fait qu’elle quitte Dieu, laquelle n’a point de cause positive. Quant à la bonne volonté, si nous disons qu’elle n’a point aussi de cause efficiente, prenons garde qu’il ne s’ensuive que la bonne volonté des bons anges n’a pas été créée, mais qu’elle est coéternelle à Dieu ; ce qui serait une absurdité manifeste. Puisque les bons anges eux-mêmes ont été créés, comment leur bonne volonté ne l’aurait-elle point été également ? Mais si elle a été créée, l’a-t-elle été avec eux, ou ont-ils été quelque temps sans elle ? Si l’on répond qu’elle a été créée avec eux, il n’y a point de doute qu’elle n’ait été créée par celui qui les a créés eux-mêmes ; et ainsi, dès le premier instant de leur création, ils se sont attachés à leur Créateur par l’amour même avec lequel ils ont été créés, et ils se sont séparés de la compagnie des autres anges, parce qu’ils sont toujours demeurés dans la même volonté, au lieu que les autres s’en sont départis en abandonnant volontairement le Souverain bien. Si l’on suppose au contraire que les bons anges aient été quelque temps sans la bonne volonté, et qu’ils l’aient produite en eux-mêmes sans le secours de Dieu, ils sont donc devenus par eux-mêmes meilleurs qu’ils n’avaient été créés. Dieu nous garde de cette pensée ! Qu’étaient-ils sans la bonne volonté, que des êtres mauvais ? Ou s’ils n’étaient pas mauvais par la raison qu’ils n’avaient pas une mauvaise volonté (car ils ne s’étaient point départis de la bonne qu’ils n’avaient pas encore), au moins n’étaient-ils pas aussi bons que lorsqu’ils ont commencé à avoir une bonne volonté. Ou s’il est vrai de dire qu’ils n’ont pas su se rendre eux-mêmes meilleurs que Dieu ne les avait faits puisque nul ne peut rien faire de meilleur que ce que Dieu fait, il faut conclure que cette bonne volonté est l’ouvrage du Créateur. Lorsque cette bonne volonté a fait qu’ils ne se sont pas tournés vers eux-mêmes qui avaient moins d’être, mais vers le souverain Être, afin d’être en quelque façon davantage en s’attachant à lui et de participer à sa sagesse et à sa félicité souveraines, qu’est-ce que cela nous apprend sinon que la volonté, quelque bonne qu’elle fût, serait toujours demeurée pauvre et n’aurait eu que des désirs imparfaits, si celui qui a créé la nature capable de le posséder ne remplissait lui-même cette capacité, en se donnant à elle, après lui en avoir inspiré un violent désir ?\par
Admettez que les bons anges eussent produit en eux-mêmes cette bonne volonté, on pourrait fort bien demander s’ils l’ont ou non produite par quelque autre volonté. Ils n’y seraient assurément point parvenus sans volonté ; mais cette volonté était nécessairement bonne ou mauvaise. Si elle était mauvaise, comment une mauvaise volonté en a-t-elle pu produire une bonne ? et si elle était bonne, ils avaient donc déjà une bonne volonté. Qui l’avait faite, sinon celui qui les a créés avec une bonne volonté, c’est-à-dire avec cet amour chaste qui les unit à lui, les comblant à la fois des dons de la nature et de ceux de la grâce ? Ainsi il faut croire que les bons anges n’ont jamais été sans la bonne volonté, c’est-à-dire sans l’amour de Dieu. Pour les autres qui, après avoir été créés bons, sont devenus méchants par leur mauvaise volonté, laquelle ne s’est corrompue que lorsque la nature, par sa propre défaillance, s’est séparée d’elle-même du souverain bien, en sorte que la cause du mal n’est pas le bien, mais l’abandon du bien, il faut dire qu’ils ont reçu un moindre amour que ceux qui y ont persévéré, ou, si les bons et les mauvais anges ont été créés également bons, on doit croire que, tandis que ceux-ci sont tombés par leur mauvaise volonté, ceux-là ont reçu un plus grand secours pour arriver à ce comble de bonheur d’où ils ont été assurés de ne point déchoir, comme nous l’avons déjà montré au livre précédent. Avouons donc à la juste louange du Créateur, que ce n’est pas seulement des gens de bien, mais des saints anges, que l’on peut dire que l’amour de Dieu est répandu en eux par le Saint-Esprit qui leur a été donné, et que c’est autant leur bien que celui des hommes d’être étroitement unis à Dieu. Ceux qui ont part à ce bien forment entre eux et avec celui à qui ils sont unis une sainte société, et ne composent ensemble qu’une même Cité de Dieu, qu’un même temple et qu’un même sacrifice. Il est temps maintenant, après avoir dit l’origine des anges, de parler de ces membres de la Cité sainte, dont les uns voyagent encore sur cette terre composée d’hommes mortels qui doivent être unis aux anges immortels, et les autres se reposent dans les demeures destinées aux bonnes âmes ; il faut raconter l’origine de cette partie de la Cité de Dieu, car tout le genre humain prend son commencement d’un seul homme que Dieu a créé le premier, selon le témoignage de l’Écriture sainte, qui s’est acquis avec raison une merveilleuse autorité dans toute la terre et parmi toutes les nations, ayant prédit, entre mille autres choses qui se sont vérifiées, la foi que lui accorderaient toutes ces nations.
\subsection[{Chapitre X}]{Chapitre X}

\begin{argument}\noindent De la fausseté de l’histoire qui compte dans le passé plusieurs milliers d’années.
\end{argument}

\noindent Laissons là les conjectures de ceux qui déraisonnent sur l’origine du genre humain. Les uns croient que les hommes ont toujours existé aussi bien que le monde, ce qui a fait dire à Apulée : « Chaque homme est mortel, pris en particulier, mais les hommes, pris ensemble, sont immortels. » Lorsqu’on leur demande comment cette opinion peut s’accorder avec le récit de leurs historiens sur les premiers inventeurs des arts ou sur ceux qui onthabité les premiers certains pays, ils répondent que d’âge en âge il arrive des déluges et des embrasements qui dépeuplent une partie de la terre et amènent la ruine des arts, de sorte que le petit nombre des hommes survivants paraît les inventer, quand il ne fait que les renouveler, mais qu’au reste un homme ne saurait venir que d’un autre homme. Parler ainsi, c’est dire, non ce qu’on sait, mais ce qu’on croit. Ils sont encore induits en erreur par certaines histoires fabuleuses qui font mention de plusieurs milliers d’années, au lieu que, selon l’Écriture sainte, il n’y a pas encore six mille ans accomplis depuis la création de l’homme. Pour montrer en peu de mots que l’on ne doit point s’arrêter à ces sortes d’histoires, je remarquerai que cette fameuse lettre écrite par Alexandre le Grand à sa mère, si l’on en croit le rapport d’un certain prêtre égyptien tiré des archives sacrées de son pays, cette lettre parle aussi des monarchies dont les historiens grecs font mention. Or, elle fait durer la monarchie des Assyriens depuis Bélus plus de cinq mille ans, au lieu que, selon l’histoire grecque, elle n’en a duré qu’environ treize cents. Cette lettre donne encore plus de huit mille ans à l’empire des Perses et des Macédoniens, tandis que les Grecs ne font durer ces deux monarchies qu’un peu plus de sept cents ans, celle des Macédoniens quatre cent quatre-vingt-cinq ans jusqu’à la mort d’Alexandre, et celle des Perses deux cent trente-trois ans. Mais c’est que les années étaient alors bien plus courtes chez les Égyptiens et n’avaient que quatre mois, de sorte qu’il en fallait trois pour faire une des nôtres ; encore cela ne suffirait-il pas pour faire concorder la chronologie des Égyptiens avec l’histoire grecque. Il faut dès lors croire plutôt cette dernière, attendu qu’elle n’excède point le nombre desannées qui sont marquées dans la sainte Écriture. Du moment que l’on remarque un si grand mécompte pour le temps dans cette lettre si célèbre d’Alexandre, combien doit-on moins ajouter foi à ces histoires inconnues et fabuleuses dont on veut opposer l’autorité à celle de ces livres fameux et divins qui ont prédit que toute la terre croirait un jour ce qu’ils contiennent, comme elle le croit en effet présentement, et qui, par l’accomplissement de leurs prophéties sur l’avenir, font assez voir que leurs récits sur le passé sont très véritables.
\subsection[{Chapitre XI}]{Chapitre XI}

\begin{argument}\noindent De ceux qui, sans admettre l’éternité du monde actuel, supposent, soit des mondes innombrables, soit un seul monde qui meurt et renaît au bout d’une certaine révolution de siècles.
\end{argument}

\noindent D’autres, ne croyant pas ce monde éternel, admettent soit des mondes innombrables, soit un seul monde qui meurt et qui naît une infinité de fois par de certaines révolutions de siècles ; mais alors il faut qu’ils avouent cette conséquence, qu’il a existé des hommes avant qu’il y en eût d’autres pour les engendrer. Ils ne sauraient prétendre en effet que lorsque le monde entier périt, il y reste un petit nombre d’hommes pour réparer le genre humain, comme il arrive, à ce qu’ils disent, dans les déluges et les incendies qui ne désolent qu’une partie de la terre ; mais comme ils estiment que le monde même renaît de sa propre matière, ils sont obligés de soutenir que le genre humain sort d’abord du sein des éléments et se multiplie ensuite comme les autres animaux par la voie de la génération.
\subsection[{Chapitre XII}]{Chapitre XII}

\begin{argument}\noindent Ce qu’il faut, répondre à ceux qui demandent pourquoi l’homme n’a pas été créé plus tôt.
\end{argument}

\noindent À l’égard de ceux qui demandent pourquoi l’homme n’a point été créé pendant les temps infinis qui ont précédé sa création, et pour quelle raison Dieu a attendu si tard que, selon l’Écriture, le genre humain ne compte pasencore six mille ans d’existence, je leur ferai la même réponse qu’à ces philosophes qui élèvent la même difficulté touchant la création du monde, et ne veulent pas croire qu’il n’a pas toujours été, bien que cette vérité ait été incontestablement reconnue par leur maître Platon ; mais ils prétendent qu’il a dit cela contre son propre sentiment. S’ils ne sont choqués que de la brièveté du temps qui s’est écoulé depuis la création de l’homme, qu’ils considèrent que tout ce qui finit est court, et que tous les siècles ne sont rien en comparaison de l’éternité. Ainsi, quand il y aurait, je ne dis pas six mille ans, mais six cents fois cent mille ans et plus que Dieu a fait l’homme, on pourrait toujours demander pourquoi il ne l’a pas fait plus tôt. À considérer cette éternité de repos où Dieu est demeuré sans créer l’homme, on trouvera qu’elle a plus de disproportion avec quelque nombre d’années imaginable qu’une goutte d’eau n’en a avec l’Océan, parce qu’au moins l’Océan et une goutte d’eau ont cela de commun qu’ils sont tous deux finis. Ainsi, ce que nous demandons après cinq mille ans et un peu plus, nos descendants pourraient le demander de même après six cents fois cent mille ans, si les hommes allaient jusque-là, et qu’ils fussent aussi faibles et aussi ignorants que nous. Ceux qui ont été avant nous vers les premiers temps de la création de l’homme pouvaient faire la même question. Enfin, le premier homme lui-même pouvait demander aussi pourquoi il n’avait pas été créé auparavant, sans que cette difficulté en fût moindre ou plus grande, en quelque temps qu’il eût pu être créé.
\subsection[{Chapitre XIII}]{Chapitre XIII}

\begin{argument}\noindent De la révolution régulière des siècles qui, suivant quelques philosophes, remet toutes choses dans le même ordre et le même état.
\end{argument}

\noindent Quelques philosophes, pour se tirer de cette difficulté, ont inventé je ne sais quelles révolutions de siècles qui reproduisent et ramènent incessamment les mêmes êtres, soit quel’on conçoive ces révolutions comme s’accomplissant au sein d’un monde qui subsiste identique sous ces transformations successives, soit que le monde lui-même périsse pour renaître dans une alternative éternelle. Rien n’est excepté de cette vicissitude, pas même l’âme immortelle ; quand elle est parvenue à la sagesse, ils la font toujours passer d’une fausse béatitude à une misère trop véritable. Comment, en effet, peut-elle être heureuse, si elle n’est jamais assurée de son bonheur, soit qu’elle ignore, soit qu’elle redoute la misère qui l’attend ; que si l’on dit qu’elle passe de la misère au bonheur pour ne plus le perdre absolument, il faut convenir alors qu’il arrive dans le temps quelque chose de nouveau qui ne finit point par le temps. Pourquoi ne pas dire la même chose du monde et de l’homme qui a été créé dans le monde, sans avoir recours à ces révolutions chimériques ?\par
En vain quelques-uns s’efforcent de les appuyer par ce passage de Salomon au livre de l’Ecclésiaste : « Qu’est-ce qui a été ? ce qui sera. Que s’est-il fait ? ce qui doit se faire encore. Il n’y a rien de nouveau sous le soleil, et personne ne peut dire : Cela est nouveau ; car cela même est déjà arrivé dans les siècles précédents. » Ce passage ne doit s’entendre que des choses dont il a été question auparavant, comme de la suite des générations, du cours du soleil, de la chute des torrents, ou au moins de tout ce qui naît et qui meurt dans le monde. En effet, il y a eu des hommes avant nous, comme il y en a avec nous, comme il y en aura après nous, et ainsi des plantes et des animaux. Les monstres mêmes, bien qu’ils diffèrent entre eux, et qu’il y en ait qui n’ont paru qu’une fois, sont semblables en cela qu’ils sont tous des monstres, et par conséquent il n’est pas nouveau qu’un monstre naisse sous le soleil. D’autres, expliquant autrement les paroles de Salomon, entendent que tout est déjà arrivé dans la prédestination de Dieu, et qu’ainsi il n’y a rien de nouveau sous le soleil. Quoi qu’il en soit, à Dieu ne plaise que nous trouvions dans l’Écriture ces révolutions imaginaires par lesquelles on veut que toutes les choses du monde soient incessamment recommencées,comme si, par exemple, un philosophe nommé Platon, ayant enseigné autrefois la philosophie dans une école d’Athènes, appelée l’Académie, il fallait croire que le même Platon aurait enseigné longtemps auparavant la même philosophie, dans la même ville, dans la même école, et devant les mêmes auditeurs, à des époques infiniment reculées, et qu’il devrait encore l’enseigner de même après une révolution de plusieurs siècles. Loin de nous une telle extravagance ! Car Jésus-Christ, qui est mort une fois pour nos péchés, ne meurt plus, et la mort n’aura plus d’empire sur lui et nous, après la résurrection, nous serons toujours avec le Seigneur, à qui nous disons maintenant comme le Psalmiste : « Vous nous conserverez toujours, Seigneur, depuis ce siècle jusqu’en l’éternité. » Il me semble encore que ce qui suit dans le même psaume : « Les impies vont tournant dans un cercle », ne convient pas mal à ces philosophes, non qu’ils soient destinés à passer par ces cercles qu’ils imaginent, mais parce qu’ils tournent dans un labyrinthe d’erreurs.
\subsection[{Chapitre XIV}]{Chapitre XIV}

\begin{argument}\noindent De la création du genre humain, laquelle a été opérée dans le temps, sans qu’il y ait eu en Dieu une décision nouvelle, ni un changement de volonté.
\end{argument}

\noindent Est-il surprenant qu’égarés en ces mille détours, ils ne puissent trouver ni entrée, ni issue ? Ils ignorent et l’origine du genre humain et le terme de sa destinée terrestre, parce qu’ils ne sauraient pénétrer la profondeur des conseils de Dieu, ni concevoir comment il a pu, lui éternel et sans commencement, donner un commencement au temps, et comment il a fait naître dans le temps un homme que nul homme n’avait précédé, non par une soudaine et nouvelle résolution, mais par un dessein éternel et immuable. Qui pourra sonder cet abîme et pénétrer ce mystère impénétrable ? Qui pourra comprendre que Dieu, sans changer de volonté, ait créé dans le temps l’homme temporel, et d’un premier homme fait sortir le genre humain ? Aussi le Psalmiste, après avoir dit : « Vous nous conserverez toujours, Seigneur, depuis ce siècle jusqu’en l’éternité », a-t-il rejeté ensuite l’opinion folle et impie de ceux quine veulent pas que la délivrance et la félicité de l’âme soient éternelles, en ajoutant : « Les impies vont, tournant dans un cercle », comme si on lui eût adressé ces paroles : Quelle est donc votre croyance, votre sentiment, votre pensée ? Faut-il croire que Dieu ait conçu tout d’un coup le dessein de créer l’homme, après être resté une éternité sans le créer, lui à qui rien ne peut survenir de nouveau, lui qui n’admet en son être rien de muable ? — Le Psalmiste répond, en s’adressant ainsi à Dieu : « Vous avez multiplié les enfants des hommes selon la profondeur de vos conseils » ; comme s’il disait : Que les hommes en pensent ce qu’il leur plaira, vous avez multiplié les enfants des hommes selon vos conseils, dont la profondeur est impénétrable. Et en effet, c’est un profond mystère que Dieu ait toujours été et qu’il ait voulu créer l’homme dans le temps, sans changer de dessein ni de volonté.
\subsection[{Chapitre XV}]{Chapitre XV}

\begin{argument}\noindent S’il faut croire que Dieu ayant toujours été souverain et seigneur comme il a toujours été Dieu, n’a jamais manqué de créatures pour adorer sa souveraineté, et en quel sens on peut dire que la créature a toujours été sans être coéternelle au Créateur.
\end{argument}

\noindent Pour moi, de même que je n’oserais pas dire que le Seigneur Dieu n’ait pas toujours été Seigneur, je dois dire aussi sans balancer que l’homme n’a point été avant le temps et qu’il a été créé dans le temps. Mais lorsque je considère de quoi Dieu a pu être Seigneur, s’il n’y a pas toujours eu des créatures, je tremble de rien assurer, parce que je sais qui je suis et me souviens qu’il est écrit : « Quel homme connaît les desseins de Dieu et peut sonder ses conseils ? Car les pensées des hommes sont timides et leur prévoyance incertaine, parce que le corps corruptible appesantit l’âme, et que cette demeure de terre et de boue accable l’esprit qui pense beaucoup. » Et peut-être, par cela même que je pense beaucoup de choses sur ce sujet, y en a-t-il une de vraie à laquelle je ne pense pas et que je ne puis trouver. Si je dis qu’il y a toujours eu des créatures, afin que Dieu ait toujours été Seigneur, en faisant cette réserve qu’elles ont toujours existé l’une après l’autre de siècle en siècle, de crainte d’admettre qu’il y ait quelque créature coéternelle à Dieu (sentiment contraire à la foi et à la saine raison), il faut prendre garde qu’il n’y ait de l’absurdité à soutenir ainsi d’une part qu’il y a toujours eu des créatures mortelles, et d’admettre d’une autre part que les créatures immortelles ont commencé d’exister à un certain moment, je veux dire au moment de la création des anges, si toutefois il est admis que les anges soient désignés par cette lumière primitive dont il est parlé au commencement de la Genèse, ou plutôt par ce {\itshape ciel} dont il est dit : « Dans le principe, Dieu créa le ciel et la terre. » Il suit de là qu’avant d’être créés, les anges n’existaient pas, à moins qu’on ne suppose que ces êtres immortels ont toujours existé, ce qui semble les faire coéternels à Dieu. Si en effet je dis qu’ils n’ont pas été créés dans le temps, mais qu’ils ont été avant tous les temps, et qu’ainsi Dieu, qui est leur Seigneur, a toujours possédé cette qualité, l’on demandera comment ceux qui ont été créés ont pu être toujours. On pourrait peut-être répondre : Pourquoi n’auraient-ils pas été toujours, s’il est vrai qu’ils ont été en tout temps ? Or il est si vrai qu’ils ont été en tout temps qu’ils ont même été faits avant tous les temps, pourvu néanmoins que les temps aient commencé avec les sphères célestes et que les anges aient été faits avant elles. Que si le temps, au lieu de commencer avec les sphères célestes, a été antérieurement, non pas à la vérité dans la suite des heures, des jours, des mois et des années, ces mesures des intervalles du temps n’ayant évidemment commencé qu’avec les mouvements des astres (d’où vient que Dieu a dit en les créant : « Qu’ils servent à marquer les temps, les jours et les années »), si donc le temps a été avant les sphères célestes, en ce sens qu’il y avait avant elles quelque chose de muable dont les modifications ne pouvaient pas exister simultanément et se succédaient l’une à l’autre, si on admet, dis-je, qu’il y ait eu quelque chose de semblable dans les anges avant la formation des sphères célestes et qu’ils aient été sujets à ces mouvements dès le premier instant de leur création, on peut dire qu’ils ont été en tout temps, puisque le temps a été fait avec eux. Or, qui prétendrait que ce qui a été en tout temps n’a pas toujours été ?\par
Mais si je réponds ainsi, on me répliquera : Comment les anges ne sont-ils point coéternels à Dieu, puisqu’ils ont toujours été aussi bien que lui ? comment même peut-on dire qu’il les ait créés, s’ils ont toujours été ? Que répondre à cela ? Alléguerons-nous qu’ils ont toujours été parce qu’ils ont été en tout temps, ayant été faits avec le temps ou le temps avec eux, et ajouterons-nous que néanmoins ils ont été créés ? Aussi bien, on ne saurait nier que le temps lui-même n’ait été créé ; et cependant personne ne doute que le temps n’ait été en tout temps, puisque, s’il en était autrement, il faudrait croire qu’il y a eu un temps où il n’y avait point de temps ; mais il n’est personne d’assez extravagant pour avancer pareille chose. Nous pouvons fort bien dire : Il y avait un temps où Rome n’était point ; il y avait un temps où Jérusalem n’était point ; il y avait un temps où Abraham n’était point ; il y avait un temps où l’homme n’était point ; et enfin, si le monde n’a point été fait au commencement du temps, mais après quelque temps, nous pouvons dire aussi : Il y avait un temps où le monde n’était point. Mais dire : Il y avait un temps où il n’y avait point de temps, c’est comme si l’on disait : Il y avait un homme quand il n’y avait aucun homme, ou : Le monde était quand il n’y avait pas de monde, ce qui est absurde. Si on ne parlait pas d’un seul et même objet, alors sans doute on pourrait dire : Il y avait un certain homme alors que tel autre homme n’était pas, et pareillement : En tel temps, en tel siècle, tel autre temps, tel autre siècle n’était pas ; mais dire Il y a eu un temps où il n’y avait pas de temps, c’est, je le répète, ce que l’homme le plus fou du monde n’oserait faire. Si donc il est vrai que le temps a été créé, tout en ayant toujours été, parce que le temps a nécessairement été de tout temps, on doit aussi reconnaître qu’il ne s’ensuit pas de ce que les anges ont toujours été, qu’ils n’aient point été créés. Car si l’on dit qu’ils ont toujours été, c’est qu’ils ont été en tout temps ; et s’ils ont été en tout temps, c’est que le temps n’a pu être sans eux. En effet, il n’y peut avoir de temps où il n’y a point de créature dont les mouvements successifs forment le temps ; et conséquemment, encore qu’ils aient toujours été, ils ne laissent pas d’avoir été créés et ne sont point pour cela coéternels à Dieu. Dieu a toujours été par une éternité immuable, au lieu que les anges n’ont toujours été que parce que le temps n’a pu être sans eux. Or, comme le temps passe par sa mobilité naturelle, il ne peut égaler une éternité immuable. C’est pourquoi, bien que l’immortalité des anges ne s’écoule pas dans le temps, bien qu’elle ne soit ni passée comme si elle n’était plus, ni future comme si elle n’était pas encore, néanmoins leurs mouvements qui composent le temps vont du futur au passé, et partant, ne sont point coéternels à Dieu, qui n’admet ni passé ni futur dans son immuable essence.\par
De cette manière, si Dieu a toujours été Seigneur, il a toujours eu des créatures qui lui ont été assujetties et qui n’ont pas été engendrées de sa substance, mais qu’il a tirées du néant, et qui, par conséquent, ne lui sont pas coéternelles. Il était avant elles, quoiqu’il n’ait jamais été sans elles, parce qu’il ne les a pas précédées par un intervalle de temps, mais par une éternité fixe. Si je fais cette réponse à ceux qui demandent comment le Créateur a toujours été Seigneur sans avoir toujours eu des créatures pour lui être assujetties, ou comment elles ont été créées, et surtout comment elles ne sont pas coéternelles à Dieu, si elles ont toujours été, je crains qu’on ne m’accuse d’affirmer ce que je ne sais pas, plutôt que d’enseigner ce que je sais. Je reviens donc à ce que notre Créateur a mis à la portée de notre esprit, et, quant aux connaissances qu’il a bien voulu accorder en cette vie à de plus habiles, ou qu’il réserve dans l’autre aux parfaits, j’avoue qu’elles sont au-dessus de mes facultés. J’ai cru par cette raison qu’il valait mieux en de telles matières ne rien assurer, afin que ceux qui liront ceci apprennent à s’abstenir des questions dangereuses, et qu’ils ne se croient pas capables de tout, mais plutôt qu’ils suivent ce précepte salutaire de l’Apôtre : « Je vous avertis tous, par la grâce qui m’a été donnée, de ne pas chercher plus de science qu’il n’en faut avoir ; soyez savants avec sobriété et selon la mesure de la foi que Dieu vous a départie. »\par
Quand on ne donne à un enfant qu’autant denourriture qu’il en peut porter, il devient capable, à mesure qu’il croît, d’en recevoir davantage ; mais quand on lui en donne trop, au lieu de croître, il meurt.
\subsection[{Chapitre XVI}]{Chapitre XVI}

\begin{argument}\noindent Comment on doit entendre que Dieu a promis à l’homme la vie éternelle avant les temps éternels.
\end{argument}

\noindent Quels sont ces siècles écoulés avant la création du genre humain ? j’avoue que je l’ignore, mais je suis certain du moins que rien de créé n’est coéternel au Créateur. L’Apôtre parle même des temps éternels, non de ceux qui sont à venir, mais, ce qui est plus étonnant, de ceux qui sont passés. Voici comment il s’exprime : « Nous sommes appelés à l’espérance de la vie éternelle, que Dieu, qui ne ment pas, a promise avant les temps éternels, et il a manifesté son Verbe aux temps convenables. » C’est dire clairement qu’il y a eu dans le passé des temps éternels, lesquels pourtant ne sont pas coéternels à Dieu. Or, avant ces temps éternels, Dieu non seulement était, mais il avait promis la vie éternelle qu’il a manifestée depuis aux temps convenables, et cette vie éternelle n’est autre chose que son Verbe. Maintenant, en quel sens faut-il entendre cette promesse faite avant les temps éternels à des hommes qui n’étaient pas encore ? c’est sans doute que ce qui devait arriver en son temps était déjà arrêté dans l’éternité de Dieu et dans son Verbe qui lui est coéternel.
\subsection[{Chapitre XVII}]{Chapitre XVII}

\begin{argument}\noindent De ce que la foi nous ordonne de croire touchant la volonté immuable de Dieu, contre les philosophes qui veulent que Dieu recommence éternellement ses ouvrages et reproduise les mêmes êtres dans un cercle qui revient toujours.
\end{argument}

\noindent Une autre chose dont je ne doute nullement, c’est qu’il n’y avait jamais eu d’homme avant la création du premier homme, et que ce n’est pas le même homme, ni un autre semblable, qui a été reproduit je ne sais combienbien de fois après je ne sais combien de révolutions. Les philosophes ont beau faire ; je ne me laisse point ébranler par leurs objections, pas même par la plus subtile de toutes, qui consiste à dire que nulle science ne peut embrasser des objets infinis ; d’où l’on tire cette conclusion que Dieu ne peut avoir en lui-même que des raisons finies pour toutes les choses finies qu’il a faites. Voici la suite du raisonnement : il ne faut pas croire, disent-ils, que la bonté de Dieu ait jamais été oisive ; car il s’ensuivrait qu’avant la création il a eu une éternité de repos, et qu’il a commencé d’agir dans le temps, comme s’il se fût repenti de sa première oisiveté, il est donc nécessaire que les mêmes choses reviennent toujours et passent pour revenir, soit que le monde reste identique dans son fond à travers la vicissitude de ses formes, ayant existé toujours, éternel et créé tout ensemble, soit qu’il périsse et renaisse incessamment ; autrement, il faudrait penser que Dieu s’est repenti à un certain jour de son éternelle oisiveté et que ses conseils ont changé. Il faut donc choisir l’une des deux alternatives ; car si l’on veut que Dieu ait toujours fait des choses temporelles, mais l’une après l’autre, de manière à ce qu’il en soit venu enfin à faire l’homme qu’il n’avait point fait auparavant, il s’ensuit que Dieu n’a pas agi avec science (car nulle science ne peut saisir cette suite indéfinie de créatures successives), mais qu’il a agi au hasard, à l’aventure, et pour ainsi dire au jour la journée. Il en est tout autrement, quand on conçoit la création comme un cercle qui revient toujours sur lui-même ; car alors, soit qu’on rapporte cette série circulaire de phénomènes à un monde permanent dans sa substance, soit qu’on suppose le monde périssant et renaissant tour à tour, on évite dans les deux cas d’attribuer à Dieu ou un lâche repos ou une téméraire imprévoyance. Sortez-vous de ce système, vous tombez nécessairement dans une succession indéfinie de créatures que nulle science, nulle prescience ne peuvent embrasser.\par
Je réponds qu’alors même que nous manquerions de raisons pour réfuter ces vaines subtilités dont les impies se servent pour nous détourner du droit chemin et nous engager dans leur labyrinthe, la foi seule devrait suffire pour nous les faire mépriser ; mais nous avons plus d’un moyen de briser le cercle de ces révolutions chimériques. Ce qui trompe nos adversaires, c’est qu’ils mesurent à leur esprit muable et borné l’esprit de Dieu qui est immuable et sans bornes, et qui connaît toutes choses par une seule pensée. Il leur arrive ce que dit l’Apôtre : « Que, pour ne se comparer qu’à eux-mêmes, ils n’entendent pas. » Comme ils agissent en vertu d’un nouveau dessein, chaque fois qu’ils font quelque chose de nouveau, parce que leur esprit est muable, ils veulent qu’il en soit ainsi à l’égard de Dieu ; de sorte qu’ils se mettent en sa place et ne le comparent pas à lui, mais à eux. Pour nous, il ne nous est pas permis de croire que Dieu soit autrement affecté lorsqu’il n’agit pas que lorsqu’il agit, puisqu’on ne doit pas dire même qu’il soit jamais affecté, en ce sens qu’il se produirait quelque chose en lui qui n’y était pas auparavant. En effet, être affecté, c’est être passif, et tout ce qui pâtit est muable. On ne doit donc pas supposer dans le repos de Dieu, oisiveté, paresse, langueur, pas plus que dans son action, peine, application, effort ; il sait agir en se reposant et se reposer en agissant. Il peut faire un nouvel ouvrage par un dessein éternel, et quand il se met à l’œuvre, ce n’est point par repentir d’être resté au repos. Quand on dit qu’il était au repos avant, et qu’après il a agi (toutes choses, il est vrai, que l’homme ne peut comprendre), cet avant et cet après ne doivent s’appliquer qu’aux choses créées, lesquelles n’étaient pas avant et ont commencé d’être après. Mais en Dieu une seconde volonté n’est pas venue changer la première ; sa même volonté éternelle et immuable a fait que les créatures n’ont pas été plus tôt et ont commencé d’être plus tard ; et peut-être a-t-il agi ainsi afin d’enseigner à ceux qui sont capables d’entendre de telles leçons qu’il n’a aucun besoin de ses créatures et qu’il les a faites par une bonté purement gratuite, ayant été une éternité sans elles et n’en ayant pas été moins heureux.
\subsection[{Chapitre XVIII}]{Chapitre XVIII}

\begin{argument}\noindent Contre ceux qui disent que dieu même ne saurait comprendre des choses infinies.
\end{argument}

\noindent Quant à ce qu’ils disent, que Dieu même ne saurait comprendre des choses infinies, il ne leur reste plus qu’à soutenir, pour mettre le comble à leur impiété, qu’il ne connaît pas tous les nombres ; car très certainement les nombres sont infinis, puisque à quelque nombre qu’on s’arrête, il est toujours possible d’y ajouter une unité, outre que tout nombre, si grand qu’il soit, si prodigieuse que soit la multitude dont il est l’expression rationnelle et scientifique, on peut toujours le doubler et même le multiplier à volonté. De plus, chaque nombre a ses propriétés, de sorte qu’il n’y a pas deux nombres identiques. Ils sont donc dissemblables entre eux et divers, finis en particulier, et infinis en général. Est-ce donc cette infinité qui échappe à la connaissance de Dieu, et faut-il dire qu’il connaît une certaine quantité de nombres et qu’il ignore le reste ? personne n’oserait soutenir une telle absurdité. Affecteront-ils de mépriser les nombres et oseront-ils les retrancher de la science de Dieu, alors que Platon, qui a tant d’autorité parmi eux, introduit Dieu créant le monde par les nombres ; et ne lisons-nous pas dans l’Écriture : « Vous avez fait toutes choses avec « poids, nombre et mesure ? » Écoutez aussi le prophète : « Il forme les siècles par nombre. » — Et l’Évangile : « Tous les cheveux de votre tête sont comptés. » Après tant de témoignages, comment pourrions-nous douter que tout nombre ne soit connu à celui « dont l’intelligence », comme dit le psaume, surpasse « toute mesure et tout nombre » ? Ainsi, bien que les nombres soient infinis et sans nombre, l’infinité du nombre ne saurait être incompréhensible à celui dont l’intelligence est au-dessus du nombre. Et, par conséquent, s’il faut que tout ce qui est compris soit fini dans l’intelligence qui le comprend, nous devons croire que l’infinité même est finie en Dieud’une certaine manière ineffable, puisqu’elle ne lui est pas incompréhensible. Dès lors, puisque l’infinité des nombres n’est pas infinie dans l’intelligence de Dieu, que sommes-nous, pauvres humains, pour assigner des limites à sa connaissance, et dire que, si les mêmes révolutions ne ramenaient périodiquement les mêmes êtres, Dieu ne pourrait avoir ni la prescience de ce qu’il doit faire, ni la science de ce qu’il a fait ! lui dont la science, simple dans sa multiplicité, uniforme dans sa variété, comprend tous les incompréhensibles d’une compréhension si incompréhensible que, voulût-il produire des choses nouvelles et différentes, il ne pourrait ni les produire sans ordre et sans prévoyance, ni les prévoir au jour la journée, parce qu’il les renferme toutes nécessairement dans sa prescience éternelle.
\subsection[{Chapitre XIX}]{Chapitre XIX}\phantomsection
\label{\_chapitre19}

\begin{argument}\noindent Sur les siècles des siècles.
\end{argument}

\noindent Je n’aurai pas la témérité de décider si, par les {\itshape siècles des siècles}, l’Écriture entend cette suite de siècles qui se succèdent les uns aux autres dans une succession continue et une diversité régulière, l’immortalité bienheureuse des âmes délivrées à jamais de la misère planant seule au-dessus de ces vicissitudes, ou bien si elle veut signifier par là les siècles qui demeurent immuables dans la sagesse de Dieu et sont comme les causes efficientes de ces autres siècles que le temps entraîne dans son cours. Peut-être {\itshape le siècle} ne veut-il rien dire autre chose que {\itshape les siècles}, et {\itshape le siècle du siècle} a-t-il même sens que {\itshape les siècles des siècles}, comme {\itshape le ciel du ciel} et {\itshape les cieux des cieux} ne sont qu’une même chose dans le langage de l’Écriture. En effet, Dieu a nommé {\itshape ciel} le firmament au-dessus duquel sont les eaux, et cependant le Psalmiste dit : « Que les eaux qui sont au-dessus des cieux louent le nom du Seigneur. » Il est donc très difficile de savoir, entre les deux sens des {\itshape siècles des siècles}, quel est le meilleur, ou s’il n’y en a pas un troisième qui soit le véritable ; mais cela importe peu à la question présentement agitée, dans le cas même où nous pourrions donner sur ce point quelque explication satisfaisante, comme dans celui où une sage réserve nous conseillerait de ne rien affirmer en si obscure matière. Il ne s’agit ici que de l’opinion de ceux qui veulent que toutes choses reviennent après certains intervalles de temps. Or, le sentiment, quel qu’il soit, que l’on peut avoir touchant les siècles des siècles, est absolument étranger à ces révolutions, puisque, soit que l’on entende par les siècles des siècles ceux qui s’écoulent ici-bas par une suite et un enchaînement continus sans aucun retour des mêmes phénomènes et sans que les âmes des bienheureux retombent jamais dans la misère d’où elles sont sorties, soit qu’on les considère comme ces causes éternelles qui règlent les mouvements de toutes les choses passagères et sujettes au temps, il s’ensuit également que ces retours périodiques qui ramènent les mêmes choses sont tout à fait imaginaires et complétement réfutés par la vie éternelle des bienheureux.
\subsection[{Chapitre XX}]{Chapitre XX}

\begin{argument}\noindent De l’impiété de ceux qui prétendent que les âmes, après avoir participé à la vraie et suprême béatitude, retourneront sur terre dans un cercle éternel de misère et de félicité.
\end{argument}

\noindent Quelle oreille pieuse pourrait entendre dire, sans en être offensée, qu’au sortir d’une vie sujette à tant de misères (si toutefois on peut appeler vie ce qui est véritablement une mort, à ce point que l’amour de cette mort même nous fait redouter la mort qui nous délivre), après tant de misères, dis-je, et tant d’épreuves traversées, enfin, après une vie terminée par les expiations de la vraie religion et de la vraie sagesse, alors que nous serons devenus heureux au sein de Dieu par la contemplation de sa lumière incorporelle et le partage de son immortalité, il nous faudra quitter un jour une gloire si pure, et tomber du faîte de cette éternité, de cette vérité, de cette félicité, dans l’abîme de la mortalité infernale, traverser-de nouveau un état où nous perdrons Dieu, où nous haïrons la vérité, où nous chercherons la félicité à travers toutes sortes de crimes ; et pourquoi ces révolutions se reproduisant ainsi sans fin d’époque en époque et ramenant une fausse félicité et une misère réelle ? c’est, dit-on, pour que Dieu ne reste pas sans rien faire, pour qu’il puisse connaître ses ouvrages, ce dont il serait incapable s’il n’en faisait pastoujours de nouveaux. Qui peut supporter de semblables folies ? qui peut les croire ? Fussent-elles vraies, n’y aurait-il pas plus de prudence à les taire, et même, pour exprimer tant bien que mal ma pensée, plus de science à les ignorer ? Si, en effet, notre bonheur dans l’autre vie tient à ce que nous ignorerons l’avenir, pourquoi accroître ici-bas notre misère par cette connaissance ? et si, au contraire, il nous est impossible d’ignorer l’avenir dans le séjour bienheureux, ignorons-le du moins ici-bas, afin que l’attente du souverain bien nous rende plus heureux que la possession de combien ne le pourra faire.\par
Diront-ils que nul ne peut arriver à la félicité de l’autre monde qu’à condition d’avoir été initié ici-bas à la connaissance de ces prétendues révolutions ? mais alors comment osent-ils en même temps avouer que plus on aime Dieu et plus aisément on arrive à cette félicité, eux qui enseignent des choses si capables de ralentir l’amour ? Quel homme n’aimerait moins vivement un Dieu qu’il sait qu’il doit quitter un jour, après l’avoir possédé autant qu’il en était capable, un Dieu dont il doit même devenir l’ennemi en haine de sa vérité et de sa sagesse ? Il serait impossible de bien aimer un ami ordinaire, si l’on prévoyait que l’on deviendrait son ennemi. Mais à Dieu ne plaise qu’il y ait un mot de vrai dans cette doctrine d’une véritable misère qui ne finira jamais et ne sera interrompue de temps en temps que par une fausse félicité ! Est-il rien de plus faux en effet que cette béatitude où nous ignorerons notre misère à venir, au milieu d’une si grande lumière de vérité dont nous serons éclairés ? est-il rien de plus trompeur que cette félicité sur laquelle nous ne pouvons jamais compter, même lorsqu’elle sera à son comble ? De deux choses l’une : ou nous ne devons pas prévoir là-haut la misère qui nous attend, et alors notre misère ici-bas est moins aveugle, puisque nous connaissons la béatitude où nous devons arriver ; ou nous devons connaître au ciel notre retour futur sur la terre, et alors nous sommes plus heureux quand nous sommes ici-bas misérables avec l’espérance d’un sort plus heureux, que lorsque nous sommes bienheureux là-haut avec la crainte de cesser de l’être. Ainsi, nous avons plus de sujet de souhaiter notre malheur que notre bonheur ; de sorte que, comme nous souffrons ici des maux présents et que là nous en craindrons de futurs, il est plus vrai de dire que nous sommes toujours misérables que de croire que nous soyons quelquefois heureux.\par
Mais la piété et la vérité nous crient que ces révolutions sont imaginaires ; la religion nous promet une félicité dont nous serons assurés et qui ne sera traversée d’aucune misère ; suivons donc le droit chemin, qui est Jésus-Christ, et, sous la conduite de ce Sauveur, détournons-nous des routes égarées de ces impies. Si Porphyre, quoique platonicien, n’a point voulu admettre dans les âmes ces vicissitudes perpétuelles de félicité et de misère, soit qu’il ait été frappé de l’extravagance de cette opinion, soit qu’il en ait été détourné par la connaissance qu’il avait du Christianisme, et si, comme je l’ai rapporté au dixième livre, il a mieux aimé penser que l’âme a été envoyée en ce monde pour y connaître le mal, afin de n’y plus être sujette, lorsqu’après en avoir été affranchie elle sera retournée au Père, à combien plus forte raison les fidèles doivent-ils fuir et détester un sentiment si faux et si contraire à la vraie religion ! Or, après avoir une fois brisé ce cercle chimérique de révolutions, rien ne nous oblige plus à croire que le genre humain n’a point de commencement, sous le prétexte, désormais vaincu, que rien ne saurait se produire dans les êtres qui leur soit entièrement nouveau. Si en effet l’on avoue que l’âme est délivrée sans retour par la mort de toutes ses misères, il lui survient donc quelque événement qui lui est nouveau, et certes un événement très considérable, puisque c’est une félicité éternelle. Or, s’il peut survenir quelque chose de nouveau à une nature immortelle, pourquoi n’en sera-t-il pas de même pour les natures mortelles ? Diront-ils que ce n’est pas une chose nouvelle à l’âme d’être bienheureuse, parce qu’elle l’était avant de s’unir au corps ? Au moins est-il nouveau pour elle d’être délivrée de sa misère, et la misère même lui a été nouvelle, puisqu’elle ne l’avait jamais soufferte auparavant. Je leur demanderai encore si cette nouveauté n’entre point dans l’ordre de la Providence et si elle arrive par hasard ; mais alors que deviennent toutes ces révolutions mesurées et régulières où rien n’arrive de nouveau, toutes chosesdevant se reproduire sans cesse ? Que si cette nouveauté est dans l’ordre de la Providence, soit que l’âme ait été envoyée dans le corps, soit qu’elle y soit tombée par elle-même, il peut donc arriver quelque chose de nouveau et qui néanmoins ne soit pas contraire à l’ordre de l’univers. Enfin, puisqu’il faut reconnaître que l’âme a pu se faire par son imprévoyance une nouvelle misère, laquelle n’a pu échapper à la Providence divine, qui a fait entrer dans ses desseins le châtiment de l’âme et sa délivrance future, gardons-nous de la témérité de refuser à Dieu le pouvoir de faire des choses nouvelles, alors surtout qu’elles ne sont pas nouvelles par rapport à lui, mais seulement par rapport au monde, ayant été prévues de toute éternité. Prendra-t-on ce détour de soutenir qu’à la vérité les âmes délivrées une fois de leur misère n’y retourneront plus, mais qu’en cela il n’arrive rien de nouveau, parce qu’il y a toujours eu et qu’il y aura toujours des âmes délivrées ? Il faut alors convenir qu’il se fait de nouvelles âmes à qui cette misère est nouvelle, et nouvelle cette délivrance. Et si l’on veut que les âmes dont se font tous les jours de nouveaux hommes (mais qui n’en animeront plus d’autres, pourvu qu’elles aient bien vécu) soient anciennes et aient toujours été, c’est admettre aussi qu’elles sont infinies ; car quelque nombre d’âmes que l’on suppose, elles n’auraient pas pu suffire pour faire perpétuellement de nouveaux hommes pendant un espace de temps infini. Or, je ne vois pas comment nos philosophes expliqueront un nombre infini d’âmes, puisque dans leur système Dieu serait incapable de les connaître, par l’impossibilité où il est de comprendre des choses infinies. Et maintenant que nous avons confondu la chimère de ces révolutions de béatitude et de misère, concluons qu’il n’est rien de plus conforme à la piété que de croire que Dieu peut, quand bon lui semble, faire de nouvelles choses, son ineffable prescience mettant sa volonté à couvert de tout changement. Quant à savoir si le nombre des âmes à jamais affranchies de leurs misères peut s’augmenter à l’infini, je le laisse à décider à ceux qui sont si subtils à déterminer jusqu’où doivent aller toutes choses. Pour nous, quoi qu’il en soit, nous trouvons toujours notre compte. Dans le cas del’affirmative, pourquoi nier que Dieu ait pu créer ce qu’il n’avait pas créé auparavant, puisque le nombre des âmes affranchies, qui auparavant n’était pas, non seulement est fait une fois, mais ne cesse jamais de se faire ? Dans l’autre cas, s’il ne faut pas que les âmes passent un certain nombre, ce nombre, quel qu’il soit, n’a jamais été auparavant, et il n’est pas possible que ce nombre croisse et arrive au terme de sa grandeur sans quelque commencement ; or, ce commencement n’avait jamais été non plus, et c’est pour qu’il fût que le premier homme a été créé.
\subsection[{Chapitre XXI}]{Chapitre XXI}

\begin{argument}\noindent De la formation du premier homme et du genre humain renfermé en lui.
\end{argument}

\noindent Et maintenant que j’ai résolu, dans la mesure de mes forces, ce difficile problème d’un Dieu éternel qui crée des choses nouvelles sans qu’il y ait de nouveauté dans son vouloir, il devient aisé de comprendre que Dieu a beaucoup mieux fait de ne créer d’abord qu’un seul homme, d’où le genre humain tout entier devait sortir, que d’en créer plusieurs. À l’égard des autres animaux, soit sauvages et solitaires, comme les aigles, les milans, les lions, les loups, soit privés ou vivant en troupes, tels que les pigeons, les étourneaux, les cerfs, les daims et tant d’autres, il ne les a pas fait sortir d’un seul, mais il en a créé plusieurs à la fois ; l’homme, au contraire, appelé à tenir le milieu entre les anges et les bêtes, demandait d’autres desseins. Si cette créature restait soumise à Dieu comme à son Seigneur véritable, elle était destinée à passer sans mourir dans la compagnie des anges pour y jouir d’un bonheur éternel ; au lieu que si elle offensait le Seigneur son Dieu par un orgueil et une désobéissance volontaires, elle devait être sujette à la mort, ravalée au niveau des bêtes, esclave de ses passions et destinée après la vie à des supplices éternels. Dieu donc, ayant de telles vues, a jugé à propos de ne créer qu’un seul homme, non certes pour le priver du bienfait de la société, mais pour lui faire aimer davantage l’union et la concorde, en unissant les hommes non seulement par laressemblance de la nature, mais aussi par les liens de la parenté ; et cela est si vrai qu’il ne voulut pas même créer la femme comme il avait créé l’homme, mais il la tira de l’homme, afin que tout le genre humain sortît d’un seul.
\subsection[{Chapitre XXII}]{Chapitre XXII}

\begin{argument}\noindent En même temps qu’il a prévu le péché du premier homme, Dieu a prévu aussi le grand nombre d’hommes pieux que sa grâce devait sauver.
\end{argument}

\noindent Cependant Dieu n’ignorait pas que l’homme devait pécher, et que, devenu mortel, il engendrerait des hommes qui se porteraient à de si grands excès que les bêtes privées de raison et qui ont été créées plusieurs à la fois vivraient plus sûrement et plus tranquillement entre elles que les hommes, qui devraient être d’autant plus unis, qu’ils viennent tous d’un seul ; car jamais les lions ni les dragons ne se sont fait la guerre comme les hommes. Mais Dieu prévoyait aussi que la multitude des fidèles serait appelée par sa grâce au bienfait de l’adoption, et qu’après la rémission de leurs péchés opérée par le Saint-Esprit, il les associerait aux anges pour jouir avec eux d’un repos éternel, après les avoir affranchis de la mort, leur dernière ennemie ; il savait combien ce serait chose préférable à cette multitude de fidèles de considérer qu’il a fait descendre tous les hommes d’un seul pour témoigner aux hommes combien l’union lui est agréable.
\subsection[{Chapitre XXIII}]{Chapitre XXIII}

\begin{argument}\noindent De la nature de l’âme humaine créée à l’image de dieu.
\end{argument}

\noindent Dieu a fait l’homme à son image ; car il lui a donné une âme douée de raison et d’intelligence qui l’élève au-dessus de toutes les bêtes de la terre, de l’air et des eaux. Après avoir formé le corps d’Adam avec de la poussière et donné une âme à ce corps, soit que cette âme fût déjà créée par avance, soit que Dieu l’ait lait naître en soufflant sur la face d’Adam, et que ce souffle divin soit l’âme humaine elle-même, il voulut donner au premier homme une femme pour l’assister dansla génération, et la forma par une puissance toute divine d’un os qu’il avait tiré de la poitrine d’Adam. Ceci au surplus ne veut pas dire être conçu grossièrement, comme si Dieu s’était servi de mains pour son œuvre, à l’exemple des artisans que nous voyons chaque jour exécuter leurs travaux matériels. La main de Dieu, c’est sa puissance, ouvrière invisible des choses visibles. Mais tout cela passe pour des fables dans l’esprit de ceux qui mesurent sur ce que leurs yeux ont l’habitude de voir la puissance et la sagesse d’un Dieu qui n’a pas besoin de semences pour produire tout et les semences elles-mêmes ; comme si les choses mêmes qui tombent sous le regard des hommes, telles que la conception et la naissance, ne leur sembleraient pas, s’ils n’en avaient l’expérience, plus incroyables encore que l’acte divin de la création ; mais la plupart aiment mieux attribuer ces effets aux causes naturelles qu’à la vertu-de Dieu.
\subsection[{Chapitre XXIV}]{Chapitre XXIV}

\begin{argument}\noindent Les anges ne sauraient créer la moindre chose.
\end{argument}

\noindent Mais nous n’avons rien à démêler ici avec ceux qui ne croient pas que Dieu ait fait le monde ou qu’il en prenne soin. Quant aux philosophes qui, sur la foi de leur Platon, pensent que la création des animaux mortels, et notamment de l’homme, n’est pas l’ouvrage du Dieu suprême auteur du monde, mais celui d’autres dieux inférieurs qui sont aussi son ouvrage, et dont l’homme est comme le parent, si nous sommes parvenu à leur persuader que c’est une superstition de sacrifier à ces dieux, ils renonceront aisément à voir en eux les créateurs du genre humain. C’estun sacrilège de croire ou de dire qu’un autre que Dieu soit le créateur d’un être quelconque, fût-il mortel et le plus chétif qui se puisse concevoir. Et pour ce qui est des anges, que l’école de Platon aime mieux appeler des dieux, il est très vrai qu’ils concourent au développement des êtres de l’univers, selon l’ordre ou la permission qu’ils en ont reçue ; mais ils ne sont pas plus les créateurs des animaux que les laboureurs ne le sont des blés ou des arbres.
\subsection[{Chapitre XXV}]{Chapitre XXV}

\begin{argument}\noindent Dieu seul est le créateur de toutes choses.
\end{argument}

\noindent Il y a pour les êtres deux espèces de forme : la forme extérieure, celle que le potier et l’artisan peuvent donner à un corps et que les peintres et les statuaires savent imiter ; il y a ensuite la forme intérieure, qui non seulement constitue les diverses natures corporelles, mais qui fait la vie des êtres animés, parce qu’elle renferme les causes efficientes et les emprunte à la source mystérieuse et incréée de l’intelligence et de la vie. Accordons à tout ouvrier la forme extérieure, mais pour cette forme intérieure où est le principe de la vie et du mouvement, elle n’a d’autre auteur que cet ouvrier unique qui n’a eu besoin d’aucun être ni d’aucun ange pour faire les anges et les êtres. La même vertu divine, et pour ainsi dire effective, qui a donné la forme ronde à la terre et au soleil, la donne à l’œil de l’homme et à une pomme, et ainsi de toutes les autres figures naturelles ; elles n’ont point d’autre principe que la puissance secrète de celui qui a dit : « Je remplis le ciel et la terre », et dont la sagesse atteint d’un bout du monde à l’autre sans aucun obstacle, et gouverne toutes choses avec douceur. J’ignore donc quel service les anges, créés les premiers, ont rendu au Créateur dans la formation des autres choses ; et comme je n’oserais leur attribuer un pouvoir que peut-être ils n’ont pas, je ne dois pas non plus leur dénier celui qu’ils ont. Toutefois, et quelle que soit la mesure de leur concours, je ne laisse pas d’attribuer la création tout entière à Dieu, en quoi je ne crains pas de leur déplaire,puisque c’est à Dieu aussi qu’ils rapportent avec action de grâces la formation de leur propre être. Nous ne disons pas que les laboureurs soient créateurs de quelque fruit que ce soit, car il est écrit : « Celui qui plante n’est rien, non plus que celui qui arrose, mais Dieu seul donne l’accroissement » ; bien plus, nous ne disons pas que la terre soit créatrice, bien qu’elle paraisse la mère féconde de tous les êtres qui tiennent à elle par leurs racines et dont elle aide les germes à éclore ; car il est également écrit : « Dieu donne à chaque plante le corps qu’il lui plaît, et à chaque semence le corps qui lui est propre. » De même, nous ne devons pas dire que la création d’un animal appartienne à sa mère, mais plutôt à celui qui a dit à l’un de ses serviteurs : « Je te connaissais avant que de te former dans le ventre de ta mère. » Je sais que l’imagination de la mère peut faire quelque impression sur son fruit, comme on peut l’inférer des agneaux bigarrés qu’eut Jacob en mettant des baguettes de diverses couleurs sous les yeux de ses brebis pleines mais cela n’empêche pas que la mère ne crée pas plus son fruit qu’elle ne s’est créée elle-même. Quelques causes donc que l’on suppose dans les générations corporelles ou séminales, entremise des anges ou des hommes, croisement des mâles et des femelles, et quelque pouvoir que les désirs et les imaginations des mères aient sur leurs fruits encore tendres et délicats, toujours faudra-t-il reconnaître que Dieu est le seul auteur de toutes les natures. C’est sa vertu invisible qui, présente en tout sans aucune souillure, donne l’être à tout ce qui est, de quelque manière qu’il soit, sans qu’aucune chose puisse être telle ou telle, ni absolument être sans lui. Si dans l’ordre des formes extérieures que la main de l’homme peut donner aux corps, nous ne disons pas que Rome et Alexandrie ont été bâties par les maçons et les architectes, mais bien par les rois dont l’ordre les a fait construire, et qu’ainsi l’une a eu Romulus et l’autre Alexandre pour fondateur, à combien plus forte raison devons-nous dire que Dieu seul est le créateur de toutes les natures, puisqu’il ne fait rien que de la matière qu’il a faite, qu’il n’a pour ouvriers que ceux mêmes qu’il a créés, et que s’il retirait sa puissance créatrice des choses qu’il a créées, elles retomberaient dans leurpremier néant. Je dis premier à l’égard de l’éternité, et non du temps ; car y a-t-il quelque autre créateur des temps que celui qui a fait les choses dont les mouvements mesurent les temps ?
\subsection[{Chapitre XXVI}]{Chapitre XXVI}

\begin{argument}\noindent Sur cette opinion des Platoniciens, que Dieu, après avoir créé les anges, leur a donné le soin de faire le corps humain.
\end{argument}

\noindent Voilà sans doute pourquoi Platon n’attribue aux dieux inférieurs, créés par le Dieu suprême, la création des animaux qu’avec cette réserve que la partie corporelle et mortelle de l’animal est seule leur ouvrage, la partie immortelle leur étant fournie par le souverain créateur. Ainsi donc, s’ils sont les créateurs des corps, ils ne le sont point des âmes. Mais alors, puisque Porphyre est convaincu que, pour purifier son âme, il faut fuir tout commerce avec les corps, puisqu’il fait d’ailleurs profession de penser avec Platon, son maître,et les autres Platoniciens, que ceux qui ont mal vécu ici-bas retournent, en punition de leurs fautes, dans des corps mortels, corps de brutes, selon Platon, corps humains, selon Porphyre, il s’ensuit que ces dieux, qu’on veut nous faire adorer comme les auteurs de notre être, ne sont que les auteurs de nos chaînes et les geôliers de notre prison. Que les Platoniciens cessent donc de nous menacer du corps comme d’un supplice, ou qu’ils ne proposent point à notre adoration des dieux dont ils nous exhortent à fuir et à rejeter l’ouvrage. Mais au fond, ces deux opinions sont aussi fausses l’une que l’autre : il est faux que les âmes retournent dans les corps en punition d’avoir mal vécu, et il est faux qu’il y ait un autre créateur de tout ce qui a vie au ciel et sur terre que celui qui a créé la terre et le ciel. En effet, si nous n’avons un corps qu’en punition de nos crimes, pourquoi Platon dit-il qu’il était nécessaire qu’il y eût des animaux de toute sorte ; mortels et immortels, pour que le monde fût l’ouvrage le plus beau et le plus parfait ? Et dès lors, puisque la création de l’homme, même à titre d’être corporel, est un bienfait divin, comment serait-ce un châtiment de reprendre de nouveau un corps ? Enfin, si Dieu renferme dans son intelligence éternelle les types de tous les animaux, comme Platon le répète si souvent, pourquoi ne les aurait-il pas créés tous de ses propres mains ? pourquoi lui aurait-il répugné d’être l’auteur de tant d’ouvrages qui réclament tout l’art de son intelligence infinie et infiniment louable ?
\subsection[{Chapitre XXVII}]{Chapitre XXVII}

\begin{argument}\noindent Toute la plénitude du genre humain était renfermée dans le premier homme, et Dieu y voyait d’avance toute la suite des élus et toute celle des réprouvés.
\end{argument}

\noindent C’est à juste titre que la véritable religion reconnaît et proclame Dieu comme le créateur de tout l’univers et de tous les animaux, c’est-à-dire des âmes aussi bien que des corps. Parmi les animaux terrestres, l’homme tient le premier rang, comme ayant été fait à l’image de Dieu ; et s’il a été créé un (sans être créé seul), c’est, je crois, par la raison que j’ai donnée ou par quelque autre encore meilleure. Il n’est point sur terre, en effet, d’animal plus sociable de sa nature, quoiqu’il n’y en ait point que le vice rende plus farouche. La nature, pour empêcher ou pour guérir le mal de la discorde, n’a pas de plus puissantmoyen que de faire souvenir les hommes qu’ils viennent tous d’un seul et même père. De même, la femme n’a été tirée de la poitrine de l’homme que pour nous rappeler combien doit être étroite l’union du mari et de la femme. Si les ouvrages de Dieu paraissent extraordinaires, c’est parce qu’ils sont les premiers ; et ceux qui n’y croient pas ne doivent non plus croire à aucun prodige ; car ce qui arrive selon le cours ordinaire de la nature n’est plus un prodige. Mais est-il possible que rien ait été fait en vain, si cachées qu’en soient les causes, sous le gouvernement de la divine Providence ? « Venez, s’écrie le Psalmiste, voyez les ouvrages du Seigneur, et les prodiges qu’il a faits sur la terre. » Je ne veux point du reste insister ici sur cet objet, et je me réserve d’expliquer ailleurs pourquoi la femme a été tirée du côté de l’homme et de quelle vérité ce premier prodige est la figure.\par
Terminons donc ce livre et disons, sinon encore au nom de l’évidence, au nom du moins de la prescience de Dieu, que deux sociétés, comme deux grandes cités, ont pris naissance dans le premier homme. En effet, de cet homme devaient sortir d’autres hommes, dont les uns, par un secret mais juste jugement de Dieu, seront compagnons des mauvais anges dans leurs supplices, et les autres des bons dans leur gloire, et, puisqu’il est écrit que « toutes les voies du Seigneur sont miséricorde et vérité », sa grâce ne peut être injuste, ni sa justice cruelle.
\section[{Livre treizième. De la mort}]{Livre treizième. \\
De la mort}\renewcommand{\leftmark}{Livre treizième. \\
De la mort}

\subsection[{Chapitre premier}]{Chapitre premier}

\begin{argument}\noindent De la chute du premier homme et de la mort qui en a été la suite.
\end{argument}

\noindent Sorti de ces épineuses questions de l’origine des choses temporelles et de la naissance du genre humain, l’ordre que nous nous sommes prescrit demande que nous parlions maintenant de la chute du premier homme, ou plutôt des premiers hommes, et de la mort qui l’a suivie. Dieu, en effet, n’avait pas placé les hommes dans la même condition que les anges, c’est-à-dire de telle sorte qu’ils ne pussent pas mourir, même en devenant pécheurs ; il les avait créés pour passer sans mourir à la félicité éternelle des anges, s’ils fussent demeurés dans l’obéissance, ou pour tomber dans la peine très juste de la mort, s’ils venaient à désobéir.
\subsection[{Chapitre II}]{Chapitre II}

\begin{argument}\noindent De la mort de l’âme et de celle du corps.
\end{argument}

\noindent Mais il me semble qu’il est à propos d’approfondir un peu davantage la nature de la mort. L’âme humaine, quoique immortelle, a néanmoins en quelque façon une mort qui lui est propre. En effet, on ne l’appelle immortelle que parce qu’elle ne cesse jamais de vivre et de sentir, au lieu que le corps est mortel, parce qu’il peut être entièrement privé de vie et qu’il ne vit point par lui-même. La mort de l’âme arrive donc quand Dieu l’abandonne, comme celle du corps quand l’âme le quitte. Et quand l’âme abandonnée de Dieu abandonne le corps, c’est alors la mort de l’homme tout entier, Dieu n’étant plus la vie de l’âme, ni l’âme la vie du corps. Or, cette mort de l’homme tout entier est suivie d’une autre que la sainte Écriture nomme la seconde mort, et c’est celle dont veut parler le Sauveur lorsqu’il dit : « Craignez celui qui peut faire périr et le corps et l’âme dans la géhenne de feu. » Comme cette menace ne peut avoir son effet qu’au temps où l’âme sera tellement unie au corps qu’ils feront un tout indissoluble, on peut trouver étrange que l’Écriture dise que le corps périt, puisque l’âme ne le quitte point et qu’il reste sensible pour être éternellement tourmenté. Qu’on dise que l’âme meurt dans ce dernier et éternel supplice dont nous parlerons plus amplement ailleurs, cela s’entend fort bien, puisqu’elle ne vit plus de Dieu ; mais comment le dire du corps, lorsqu’il est vivant ? Et il faut bien qu’il le soit pour sentir les tourments qu’il souffrira après la résurrection. Serait-ce que la vie, quelle qu’elle soit, étant un bien, et la douleur un mal, on peut dire qu’un corps ne vit plus, lorsque l’âme ne l’anime que pour le faire souffrir ? L’âme vit donc de Dieu, quand elle vit bien ; car elle ne peut bien vivre qu’en tant que Dieu opère en elle ce qui est bien ; et quant au corps, il est vivant, lorsque l’âme l’anime, qu’elle vive de Dieu ou non. Car les méchants ne vivent pas de la vie de l’âme, mais de celle du corps, que l’âme lui communique ; et encore que celle-ci soit morte, c’est-à-dire abandonnée de Dieu, elle conserve une espèce de vie qui lui est propre et qu’elle ne perd jamais, d’où vient qu’on la nomme immortelle. Mais en la dernière condamnation, bien que l’homme ne laisse pas de sentir, toutefois, comme ce sentiment ne sera pas agréable, mais douloureux, ce n’est pas sans raison que l’Écriture l’appelle plutôt une mort qu’une vie. Elle l’appelle la seconde mort, parce qu’elle arrivera après cette première mort qui sépare l’âme, soit de Dieu, soit du corps. On peut donc dire de la première mort du corps, qu’elle est bonne pour les bons et mauvaise pour les méchants, et de la seconde, que, comme elle n’est pas pour les bons, elle ne peut être bonne pour personne.
\subsection[{Chapitre III}]{Chapitre III}

\begin{argument}\noindent Si la mort qui a suivi le péché des premiers hommes et s’est étendue à toute leur race est pour les justes eux-mêmes une peine du péché.
\end{argument}

\noindent Ici se présente une question qu’il ne faut pas éluder : cette mort, qui consiste dans la séparation du corps et de l’âme, est-elle un bien pour les bons ? et, s’il en est ainsi, comment y voir une peine du péché ? car enfin, sans le péché, les hommes ne l’auraient point subie. Comment donc serait-elle bonne pour les bons, n’ayant pu arriver qu’à des méchants ? Et d’un autre côté, si elle ne pouvait arriver qu’à des méchants, les bons n’y devraient point être sujets. Pourquoi une peine où il n’y a rien à punir ? Si l’on veut sortir de cette difficulté, il faut avouer que les premiers hommes avaient été créés pour ne subir aucun genre de mort, s’ils ne péchaient point, mais qu’ayant péché, ils ont été condamnés à une mort qui s’est étendue à toute leur race. Mortels, ils ne pouvaient engendrer que des mortels, et leur crime a tellement corrompu la nature que la mort, qui n’était pour eux qu’une punition, est devenue une condition naturelle pour leurs enfants. En effet, un homme ne naît pas d’un autre homme de la même manière que le premier homme est né de la poussière. La poussière n’a été pour former l’homme primitif que le principe matériel, au lieu que le père est pour le fils le principe générateur. Aussi bien, la chair est d’une autre nature que la terre, quoiqu’elle en ait été tirée ; mais un fils n’est point d’une autre nature que son père. Tout le genre humain était donc renfermé par la femme dans le couple primitif au moment où il reçut de Dieu l’arrêt de sa condamnation. Devenu pécheur et mortel, l’homme a engendré un homme mortel et pécheur comme lui avec cette différence que le premier homme ne fut pas réduit à cette stupidité ni à cette faiblesse de corps et d’esprit que nous voyons dans les enfants ; car Dieu a voulu que leur entrée dans la vie fût semblable à celle des bêtes « L’homme, dit le Prophète, quand il était en honneur, n’a pas su comprendre ; il est tombé dans la condition des bêtes brutes etleur est devenu semblable. » Il y a plus : les hommes, en venant au monde, ont encore moins d’usage de leurs membres et moins de sentiment que les bêtes ; comme si l’énergie humaine, pareille à la flèche qui sort de l’arc tendu, s’élançait au-dessus du reste des animaux avec d’autant plus de force que, plus longtemps ramenée sur soi, elle a plus contenu son essor. Le premier homme n’est donc pas tombé par l’effet de son crime dans cet état de faiblesse où naissent les enfants ; mais la nature humaine a été tellement viciée et changée en lui qu’il a senti dans ses membres la révolte de la concupiscence, et qu’étant devenu sujet à la mort, il a engendré des hommes semblables à lui, c’est-à-dire sujets à la mort et au péché. Quand les enfants sont délivrés de ces liens du péché par la grâce du Médiateur, ils souffrent seulement cette mort qui sépare l’âme du corps, et ils sont affranchis de cette seconde mort où l’âme doit endurer des supplices éternels.
\subsection[{Chapitre IV}]{Chapitre IV}

\begin{argument}\noindent Pourquoi ceux qui sont absous du péché par le baptême sont encore sujets à la mort, qui est la peine du péché.
\end{argument}

\noindent On dira : si la mort est la peine du péché, pourquoi ceux dont le péché est effacé par le baptême sont-ils également sujets à la mort ? c’est une question que nous avons déjà discutée et résolue dans notre ouvrage {\itshape Du baptême des enfants}, où nous avons dit que la séparation de l’âme et du corps est une épreuve à laquelle l’âme reste encore soumise, quoique libre du lien du péché, parce que, si le corps devenait immortel aussitôt après le baptême, la foi en serait affaiblie. Or, la foi n’est vraiment la foi que quand on attend dans l’espérance ce qu’ors ne voit pas encore dans la réalité, c’est elle qui, dans les temps passés du moins, élevait les âmes au-dessus de la crainte de la mort : témoins ces saints martyrs en qui la foi n’aurait pu remporter tant d’illustres victoires sur la mort, s‘ils avaient été immortels. D’ailleurs, qui n’accourrait au baptême avec les petits enfants, si le baptême délivrait de la mort ? Tant s’en faut donc que la foi fût éprouvée par la promesse des récompenses invisibles, qu’il n’y aurait pas de foi, puisqu’elle chercherait et recevrait à l’heure même sa récompense ; tandis que, dans la nouvelle loi, par une grâce du Sauveur bien plus grande et bien plus admirable, la peine du péché est devenue un sujet de mérite. Autrefois il était dit à l’homme : Vous mourrez, si vous péchez ; aujourd’hui il est dit aux martyrs : Mourez, pour ne pécher point. Dieu disait aux premiers hommes : « Si vous désobéissez, vous mourrez » ; il nous dit présentement : « Si vous fuyez la mort vous désobéirez. » Ce qu’il fallait craindre autrefois, afin de ne pécher point, est ce qu’il faut maintenant souffrir, de crainte de pécher. Et de la sorte, par la miséricorde ineffable de Dieu, la peine du crime devient l’instrument de la vertu ; ce qui faisait le supplice du pécheur fait le mérite du juste, et la mort qui a été la peine du péché est désormais l’accomplissement de la justice. Mais il n’en est ainsi que pour les martyrs à qui leurs persécuteurs donnent le choix ou de renoncer à la foi, ou de souffrir la mort ; car les justes aiment mieux souffrir, en croyant, ce que les premiers prévaricateurs ont souffert pour n’avoir pas cru. Si ceux-ci n’avaient point péché, ils ne seraient pas morts ; et les martyrs pèchent, s’ils ne meurent. Les uns sont donc morts parce qu’ils ont péché ; les autres ne pèchent point parce qu’ils meurent. La faute des premiers a amené la peine, et la peine des seconds prévient la faute : non que la mort, qui était un mal, soit devenue un bien, mais Dieu a fait à la foi une telle grâce que la mort, qui est le contraire de la vie, devient l’instrument de la vie même.
\subsection[{Chapitre V}]{Chapitre V}

\begin{argument}\noindent Comme les méchants usent mal de la loi qui est bonne, ainsi les bons usent bien de la mort qui est mauvaise.
\end{argument}

\noindent L’Apôtre, voulant faire éclater toute la puissance malfaisante du péché en l’absence de la grâce, n’a pas craint d’appeler force du péché la loi même qui le défend. « Le péché, dit-il, est l’aiguillon de la mort, et la loi est la force du péché. » Parole parfaitement vraie ; car la défense du mal en augmente le désir, si l’on n’aime tellement la vertu que le plaisir qu’on y trouve surmonte la passion de mal faire. Or, la grâce de Dieu peut seule nous donner l’amour et le goût de la vertu. Mais de peur que l’expression {\itshape force du péché} ne donnât à croire que la loi est mauvaise, l’Apôtre dit, dans un autre endroit, sur le même sujet : « Assurément la loi est sainte et le commandement est saint, juste et bon. Quoi donc ? Ce qui est bon est-il devenu une mort pour moi ? Non, mais le péché, pour faire paraître sa malice, s’est servi d’un bien pour me donner la mort, de sorte que le pécheur et le péché ont passé toute mesure à cause du commandement même. » Saint Paul dit que toute mesure a été passée, parce que la prévarication augmente par le progrès de la concupiscence et le mépris de la loi. Pourquoi citons-nous ce texte ? Pour faire voir que tout comme la loi n’est pas un mal, quand elle accroît la convoitise de ceux qui pèchent, ainsi la mort n’est point un bien, quand elle augmente la gloire de ceux qui meurent, bien que celle-là soit violée pour l’iniquité et fasse des prévaricateurs, et que celle-ci soit embrassée pour la vérité et fasse des martyrs. Ainsi donc la loi est bonne, parce qu’elle est une défense du péché, et la mort est mauvaise, parce qu’elle est la peine du péché. Mais de même que les méchants usent mal, non seulement des maux, mais aussi des biens, de même les bons font également bon usage et des biens et des maux, et voilà pourquoi les méchants usent mal de la loi, qui est un bien, et les bons usent bien de la mort, qui est un mal.
\subsection[{Chapitre VI}]{Chapitre VI}

\begin{argument}\noindent Du mal de la mort qui rompt la société de l’âme et du corps.
\end{argument}

\noindent La mort n’est donc un bien pour personne, puisque la séparation du corps et de l’âme est un déchirement violent qui révolte la nature et fait gémir la sensibilité, jusqu’au moment où, avec le mutuel embrassement de la chair et de l’âme cesse toute conscience de la douleur. Quelquefois un seul coup reçu par lecorps ou bien l’élan de l’âme interrompent l’agonie et empêchent de sentir les angoisses de la dernière heure. Mais quoi qu’il en soit de cette crise où la sensibilité s’éteint dans une sensation de douleur, quand on souffre la mort avec la patience d’un vrai chrétien, tout en restant une peine, elle devient un mérite. Peine de tous ceux qui naissent d’Adam, elle est un mérite pour ceux qui renaissent de Jésus-Christ, étant endurée pour la foi et pour la justice ; et elle peut même en certains cas racheter entièrement du péché, elle qui est le prix du péché.
\subsection[{Chapitre VII}]{Chapitre VII}

\begin{argument}\noindent De la mort que souffrent pour Jésus-Christ ceux qui n’ont point reçu le baptême.
\end{argument}

\noindent Tous ceux, en effet, qui meurent pour la confession de Jésus-Christ obtiennent, sans avoir reçu le baptême, le pardon de leurs péchés, comme s’ils avaient été baptisés. Il est écrit, à la vérité, que « personne n’entrera dans le royaume des cieux, qu’il ne renaisse de l’eau et du Saint-Esprit ». Mais l’exception à cette règle est contenue dans ces paroles non moins formelles : « Quiconque me confessera devant les hommes, je le confesserai aussi devant mon Père qui est dans les cieux. » Et ailleurs : « Qui perdra sa vie pour moi, la trouvera. » Voilà pourquoi il est écrit : « Précieuse est devant le Seigneur la mort de ses saints. » Quoi de plus précieux en effet qu’une mort qui efface les péchés et qui accroît les mérites ? Car il n’y a pas à établir de parité entre ceux qui, ne pouvant différer leur mort, sont baptisés et sortent de cette vie après que tous leurs péchés leur ont été remis, et ceux qui, pouvant s’empêcher de mourir ne l’ont pas fait, parce qu’ils ont mieux aimé perdre la vie en confessant Jésus-Christ, que d’être baptisés après l’avoir renié. Et cependant, alors même qu’ils l’auraient renié par crainte de la mort, ce crime leur eût aussi été remis au baptême, puisque les meurtriers de Jésus-Christ, quand ils ont été baptisés, ont aussi obtenumiséricorde. Mais combien a dû être puissante la grâce de cet Esprit qui souffle où il veut, pour avoir inspiré aux martyrs la force de ne pas renier Jésus-Christ dans un si grand péril de leur vie, avec une si grande espérance de pardon ? La mort des saints est donc précieuse, puisque le mérite de celle de Jésus-Christ leur a été si libéralement appliqué, qu’ils n’ont point hésité à lui sacrifier leur vie pour jouir de lui, de sorte que l’antique peine du péché est devenue en eux une source nouvelle et plus abondante de justice. Toutefois ne concluons pas de là que la mort soit un bien en soi ; si elle a été cause d’un si grand bien, ce n’est point par sa propre vertu, mais par le secours de la grâce. Elle était autrefois un objet de crainte, afin que le péché ne fût pas commis ; elle doit être aujourd’hui acceptée avec joie, afin que le péché soit évité, ou s’il a été commis, afin qu’il soit effacé par le martyre, et que la palme de la justice appartienne au chrétien victorieux.
\subsection[{Chapitre VIII}]{Chapitre VIII}

\begin{argument}\noindent Les saints, en subissant la première mort pour la vérité, se sont affranchis de la seconde.
\end{argument}

\noindent À considérer la chose de plus près, on trouvera que ceux mêmes qui meurent pour la vérité ne le font que pour se garantir de la mort, et qu’ils n’en souffrent une partie que pour l’éviter tout entière. En effet, s’ils endurent la séparation de l’âme et du corps, c’est de peur que Dieu ne se sépare de l’âme, et qu’ainsi la première mort ne soit suivie de la seconde qui ne finira jamais. Ainsi, encore une fois, la mort n’est bonne à personne, mais on la souffre pour conserver ou pour acquérir quelque bien. Et quant à ce qui arrive après la mort, on peut dire â ce point de vue que la mort est mauvaise pour les méchants et bonne pour les bons, puisque les âmes des bons séparées du corps sont dans le repos, et que celles des méchants sont dans les tortures jusqu’à ce que les corps des uns revivent pour la vie éternelle, et ceux des autres pour la mort éternelle, qui est la seconde mort.
\subsection[{Chapitre IX}]{Chapitre IX}

\begin{argument}\noindent Quel est l’instant précis de la mort ou de l’extinction du sentiment de la vie, et s’il le faut fixer au moment où l’on meurt, ou à celui ou on est mort.
\end{argument}

\noindent Le moment où les âmes séparées du corps sont heureuses ou malheureuses est-il le moment même de la mort ou celui qui la suit ? Dans ce dernier cas, ce ne serait pas la mort, puisqu’elle est déjà passée, mais la vie ultérieure, la vie propre à l’âme, qu’on devrait appeler bonne ou mauvaise. La mort, en effet, est mauvaise quand elle est présente, c’est-à-dire au moment même de la mort, parce que dans ce moment le mourant ressent de grandes douleurs, lesquelles sont un mal (dont les bons savent d’ailleurs bien user) ; mais comment, lorsque la mort est passée, peut-elle être bonne ou mauvaise, puisqu’elle a cessé d’être ? Il y a plus : si nous y prenons garde, nous verrons que les douleurs mêmes des mourants ne sont pas la mort. Ils vivent tant qu’ils ont du sentiment, et ainsi ils ne sont pas encore dans la mort, qui ôte tout sentiment, mais dans les approches de la mort, qui seules sont douloureuses. Comment donc appelons-nous mourants ceux qui ne sont pas encore morts et qui agonisent, nul n’étant mourant qu’à condition de vivre encore ? Ils sont donc tout ensemble vivants et mourants, c’est-à-dire qu’ils s’approchent de la mort en s’éloignant de la vie ; mais après tout, ils sont encore en vie, parce que l’âme est encore unie au corps. Que si, lorsqu’elle en sera sortie, on ne peut pas dire qu’ils soient dans la mort, mais après la mort, quand sont-ils donc dans la mort ? D’une part, nul ne peut être mourant, si nul ne peut être ensemble mourant et vivant, puisque évidemment, tant que l’âme est dans le corps, on ne peut nier qu’on ne soit vivant ; et d’autre part, si on dit que celui-là est mourant qui tend vers la mort, je ne sais plus quand on est vivant.
\subsection[{Chapitre X}]{Chapitre X}

\begin{argument}\noindent La vie des mortels est plutôt une mort qu’une vie.
\end{argument}

\noindent En effet, dès que nous avons commencé d’être dans ce corps mortel, nous n’avons cessé de tendre vers la mort, et nous ne faisons autre chose pendant toute cette vie (si toutefois il faut donner un tel nom à notre existence passagère). Y a-t-il personne qui ne soit plus proche de la mort dans un an qu’à cette heure, et demain qu’aujourd’hui, et aujourd’hui qu’hier ? Tout le temps que l’on vit est autant de retranché sur celui que l’on doit vivre, et ce qui reste diminue tous les jours, de sorte que tout le temps de cette vie n’est autre chose qu’une course vers la mort, dans laquelle il n’est permis à personne de se reposer ou de marcher plus lentement ; tous y courent d’une égale vitesse. En effet, celui dont la vie est plus courte ne passe pas plus vite un jour que celui dont la vie est plus longue ; mais l’un a moins de chemin à faire que l’autre. Si donc nous commençons à mourir, c’est-à-dire à être dans la mort, du moment que nous commençons à avancer vers la mort, il faut dire que nous commençons à mourir dès que nous commençons à vivre. De cette manière, l’homme n’est jamais dans la vie, s’il est vrai qu’il ne puisse être ensemble dans la vie et dans la mort ; ou plutôt ne faut-il point dire qu’il est tout ensemble dans la vie et dans la mort ? dans la vie, parce qu’elle ne lui est pas tout à fait ôtée, dans la mort, parce qu’il meurt à tout moment ? Si en effet il n’est point dans la vie, que lui est-il donc retranché ? et s’il n’est pas dans la mort, qu’est-ce que ce retranchement même ? Quand toute vie a été retranchée au corps, ces mots {\itshape après la mort} n’auraient pas de sens, si la mort n’était déjà, lorsque se faisait le retranchement ; car dès qu’il est fait, on n’est plus mourant, on est mort. On était donc dans la mort au moment où était retranchée la vie.
\subsection[{Chapitre XI}]{Chapitre XI}

\begin{argument}\noindent Si l’on peut dire qu’un homme est en même temps mort et vivant.
\end{argument}

\noindent Mais s’il est absurde de dire qu’un homme soit dans la mort avant qu’il soit arrivé à la mort, ou qui soit ensemble vivant et mourant, par la même raison qu’il ne peut être ensemble veillant et dormant, je demande quand il sera mourant. Avant que la mort ne vienne, il n’est pas mourant, mais vivant ; et, lorsqu’elle sera venue, il ne sera pas mourant, mais mort. Or, l’une de ces deux choses est avant la mort, et l’autre après ; quandsera-t-il donc dans la mort pour pouvoir dire qu’il est mourant ? Comme il y a trois moments distincts : avant la mort, dans la mort et après la mort, il faut aussi qu’il y ait trois états qui y répondent, c’est-à-dire être vivant, être mourant, être mort. Il est donc très difficile de déterminer quand un homme est mourant, c’est-à-dire dans la mort, en sorte qu’il ne soit ni vivant ni mort ; car tant que l’âme est dans le corps, surtout si le sentiment n’est pas éteint, il est certain que l’homme vit ; et dès lors il ne faut pas dire qu’il est dans la mort, mais avant la mort ; et lorsque l’âme a quitté le corps et qu’elle lui a ôté tout sentiment, l’homme est après la mort, et l’on dit qu’il est mort. Je ne vois pas comment il peut être mourant, c’est-à-dire dans la mort, puisque s’il vit encore, il est avant la mort, et que, s’il a cessé de vivre, il est après la mort. De même, dans le cours des temps, on cherche le présent, et on ne le trouve point, parce que le passage du futur au passé n’a aucune étendue appréciable. Ne faut-il point conclure de là qu’il n’y a point de mort du corps ? car s’il y en a une, quand est-elle, puisqu’elle n’est en personne et que personne n’est en elle ? En effet, si l’on vit, elle n’est pas encore, et si l’on a cessé de vivre, elle n’est plus. D’un autre côté, s’il n’y a point de mort, pourquoi dit-on avant ou après la mort ? Ah ! plût à Dieu que nous eussions assez bien vécu dans le paradis pour qu’en effet il n’y en eût point ! au lieu que dans notre condition présente, non seulement il y en a une, mais elle est même si fâcheuse qu’il est aussi impossible de l’expliquer que de la fuir.\par
Conformons-nous donc à l’usage, comme c’est notre devoir, et disons de la mort, avant qu’elle n’arrive, ce qu’en dit l’Écriture : « Ne louez personne avant sa mort. » Disons aussi, lorsqu’elle est arrivée : Telle ou telle chose s’est faite après la mort de celui-ci ou de celui-là. Disons encore, autant que possible, du temps présent : Telle personne en mourant a fait son testament, et elle a laissé en mourant telle et telle chose à tels et tels, quoiqu’elle n’ait pu rien faire de cela si elle n’était vivante, et qu’elle l’ait plutôt fait avant la mort que dans la mort. Parlons aussi commeparle l’Écriture, qui déclare positivement que les morts mêmes sont dans la mort. Elle dit en effet : « Il n’est personne dans la mort qui se souvienne de vous. » Aussi bien, jusqu’à ce qu’ils ressuscitent, on dit fort bien qu’ils sont dans la mort, comme on dit qu’une personne est dans le sommeil jusqu’à ce qu’elle se réveille. Et cependant, quoique nous appelions dormants ceux qui sont dans le sommeil, nous ne pouvons pas appeler de même mourants ceux qui sont déjà morts ; car la séparation de leur âme et de leur corps étant accomplie, on ne peut pas dire qu’ils continuent de mourir. Et voilà toujours cette difficulté qui revient d’exprimer une chose qui paraît inexprimable : à savoir comment on peut dire d’un mourant qu’il vif, ou d’un mort qu’après la mort il est dans la mort, surtout quand le mot mourant n’est pas pris dans le sens de dormant, c’est-à-dire qui est dans le sommeil, ou de languissant, c’est-à-dire qui est dans la langueur, et qu’on appelle mort, et non pas mourant, celui qui est dans la mort et attend la résurrection. Je crois, et cette opinion n’a rien de téméraire ni d’invraisemblable, à ce qu’il me semble, que si le verbe {\itshape mori} (mourir) ne peut se décliner comme les autres verbes, c’est la suite, non d’une institution humaine, mais d’un décret divin. En effet, le verbe {\itshape oriri} (se lever), entre autres, fait au passé {\itshape ortus est}, tandis que {\itshape mori} fait {\itshape mortuus} et redouble l’{\itshape u}. Ainsi on dit {\itshape mortuus} comme {\itshape fatuus, arduus, conspicuus}, et autres mots qui sont des adjectifs ne se déclinant pas selon les temps, et non des participes. Or, {\itshape mortuus} est pris comme participe passé, comme si ce qu’on ne peut décliner devait se décliner. Il est donc arrivé, par une raison assez juste, que, de même que la mort ne peut se décliner, le mot qui l’exprime est aussi indéclinable. Mais au moins pouvons-nous décliner la seconde mort, avec la grâce de notre Rédempteur ; celle-là est la pire de toutes ; elle n’a pas lieu par la séparation de l’âme et du corps, mais plutôt par l’union de l’une et l’autre pour souffrir ensemble une peine éternelle. C’est là que les hommes seront toujours dans la mort et toujours mourants, parce que cette mort sera immortelle.
\subsection[{Chapitre XII}]{Chapitre XII}

\begin{argument}\noindent De quelle mort dieu entendait parler, quand il menaça de la mort les premiers hommes, s’ils contrevenaient à son commandement.
\end{argument}

\noindent Quand on demande de quelle mort Dieu menaça les premiers hommes en cas de désobéissance, si c’était de celle de l’âme ou de celle du corps, ou de toutes les deux ensemble, ou de celle qu’on nomme la seconde mort, il faut répondre : de toutes. De la même manière que toute la terre est composée de plusieurs terres, et toute l’Église de plusieurs Églises ; ainsi toute la mort est composée de toutes les morts. La première mort, en effet, comprend deux parties, la mort de l’âme et celle du corps, alors que l’âme, séparée de Dieu et du corps, est soumise à une expiation temporaire ; et la seconde mort a lieu quand l’âme, séparée de Dieu et réunie au corps, souffre des peines éternelles. Lors donc que Dieu dit au premier homme qu’il avait mis dans le paradis terrestre, en lui parlant du fruit défendu : « Du jour que vous en mangerez, vous mourrez » ; cette menace ne comprenait pas seulement la première partie de cette première mort, qui sépare l’âme de Dieu, ni seulement la seconde partie, qui sépare l’âme du corps, ni seulement toute cette première mort qui consiste dans le châtiment temporaire de l’âme séparée de Dieu et du corps, mais toutes les morts, jusqu’à la dernière, qui est la seconde mort, et après laquelle il n’y en a point.
\subsection[{Chapitre XIII}]{Chapitre XIII}

\begin{argument}\noindent Quel fut le premier châtiment de la désobéissance de nos premiers parents.
\end{argument}

\noindent Abandonnés de la grâce de Dieu aussitôt qu’ils eurent désobéi, ils rougirent de leur nudité. C’est pour cela qu’ils se couvrirent de feuilles de figuier, les premières sans doute qui se présentèrent à eux dans le trouble où ils étaient, et en cachèrent leurs parties honteuses, dont ils n’avaient pas honte auparavant. Ils sentirent donc un nouveau mouvement dans leur chair devenue indocile en représailles de leur propre indocilité. Comme l’âme s’était complu dans un mauvais usage de sa liberté et avait dédaigné de se soumettre à Dieu, le corps refusa de s’assujettir à elle ; et de même qu’elle avait abandonné volontairement son Seigneur, elle ne put désormais disposer à sa volonté de son esclave, ni conserver son empire sur son corps, comme elle eût fait si elle fût demeurée soumise à son Dieu. Ce fut alors que la chair commença à convoiter contre l’esprit, et nous naissons avec ce combat, traînant depuis la première faute un germe de mort, et portant la discorde trop souvent victorieuse dans nos membres rebelles et dans notre nature corrompue.
\subsection[{Chapitre XIV}]{Chapitre XIV}

\begin{argument}\noindent L’homme créé innocent ne s’est perdu que par le mauvais usage de son libre arbitre.
\end{argument}

\noindent Dieu, en effet, auteur des natures et non des vices, a créé l’homme pur ; mais l’homme corrompu par sa volonté propre et justement condamné, a engendré des enfants corrompus et condamnés comme lui. Nous étions véritablement tous en lui, alors que nous étions tous cet homme qui tomba dans le péché par la femme tirée de lui avant le péché. Nous n’avions pas encore reçu à la vérité notre essence individuelle, mais le germe d’où nous devions sortir était déjà, et comme il était corrompu par le péché, chargé des liens de la mort et frappé d’une juste condamnation, l’homme ne pouvait pas, naissant de l’homme, naître d’une autre condition que lui. Toute cette suite de misères auxquelles nous sommes sujets ne vient donc que du mauvais usage du libre arbitre, et elle nous conduit jusqu’à la seconde mort qui ne doit jamais finir, si la grâce de Dieu ne nous en préserve.
\subsection[{Chapitre XV}]{Chapitre XV}

\begin{argument}\noindent En devenant pécheur, Adam a plutôt abandonné Dieu que Dieu ne l’a abandonné, et cet abandon de Dieu a été la première mort de l’âme.
\end{argument}

\noindent On remarquera peut-être que dans cette parole : « Vous mourrez de mort », mort est mis au singulier et non au pluriel ; mais alors même que sur ce fondement on réduirait la menace divine à cette seule mort qui a lieu quand l’âme est abandonnée de Dieu (par où il ne faut pas entendre que ce soit Dieu qui abandonne l’âme le premier ; car la volonté de l’âme prévient Dieu pour le mal, commela volonté de Dieu prévient l’âme pour le bien, soit pour la créer quand elle n’est pas encore, soif pour la recréer après qu’elle a failli, alors, dis-je, qu’on n’entendrait que cette seule mort, et que ces paroles de Dieu : « Du jour que vous en mangerez, vous mourrez de mort », seraient prises comme s’il disait : Du jour que vous m’abandonnerez par désobéissance, je vous abandonnerai par justice ; il n’en est pas moins certain que cette mort comprenait en soi toutes les autres, qui en étaient une suite inévitable. Déjà ce mouvement de rébellion qui s’éleva dans la chair contre l’âme devenue rebelle et qui obligea nos premiers parents à couvrir leur nudité, leur fit sentir l’effet de cette mort qui arrive quand Dieu abandonne l’âme. Elle est marquée expressément dans ces paroles que Dieu adresse au premier homme qui se cachait tout éperdu : « Adam, où es-tu ? » Car il ne le cherchait pas comme s’il eût ignoré où il était, mais il lui faisait sentir que l’homme ne sait plus où il est quand Dieu n’est plus avec lui plus tard, lorsque l’âme de nos premiers parents abandonna leurs corps épuisés de vieillesse, ils éprouvèrent cette autre mort, nouveau châtiment du péché de l’homme, qui avait fait dire à Dieu : « Vous êtes terre, et vous retournerez en terre » ; afin que ces deux morts accomplissent ensemble la première qui est celle de l’homme entier, et qui est à la fin suivie de la seconde, si la grâce de Dieu ne nous en délivre. En effet, le corps qui est de terre ne retournerait point en terre, si l’âme qui est sa vie ne le quittait ; et c’est pour cela que les chrétiens, sincèrement attachés à la foi catholique, croient fermement que la mort même du corps ne vient point de la nature, mais qu’elle est une peine du péché et un effet de cette parole que Dieu, châtiant le péché, dit au premier homme en qui nous étions tous alors : « Tu es terre, et tu retourneras en terre. »
\subsection[{Chapitre XVII}]{Chapitre XVII}

\begin{argument}\noindent Contre les Platoniciens, qui ne veulent pas que la séparation du corps et de l’âme soit une peine du péché.
\end{argument}

\noindent Les philosophes contre qui nous avons entrepris de défendre la Cité de Dieu, c’est-à-direson Église, pensent être bien sages quand ils se moquent de nous au sujet de la séparation de l’âme et du corps, que nous considérons comme un des châtiments de l’âme ; car à leurs yeux l’âme n’atteint la parfaite béatitude que lorsque entièrement dépouillée du corps, elle retourne à Dieu dans sa simplicité, dans son indépendance et comme dans sa nudité primitive. Ici peut-être, si je ne trouvais dans leurs propres livres de quoi les réfuter, je serais obligé d’entrer dans une longue discussion pour montrer que le corps n’est à charge à l’âme que parce qu’il est corruptible. De là ce mot de l’Écriture, déjà rappelé au livre précédent : « Le corps corruptible appesantit l’âme. » L’Écriture dit corruptible, pour faire voir que ce n’est pas le corps en soi qui appesantit l’âme, mais le corps dans l’état où il est tombé par le péché ; et elle ne le dirait pas que nous devrions l’entendre ainsi. Mais quand Platon déclare en termes formels que les dieux inférieurs créés par le Dieu souverain ont des corps immortels, quand il introduit ce même Dieu promettant à ses ministres comme une grande faveur qu’ils demeureront éternellement unis à leur corps, sans qu’aucune mort les en sépare, comment se fait-il que nos adversaires, dans leur zèle contre la foi chrétienne, feignent de ne pas savoir ce qu’ils savent, et s’exposent à parler contre leurs propres sentiments, pour le plaisir de nous contredire ? Voici, en effet (d’après Cicéron, qui les traduit), les propres paroles que Platon prête au Dieu souverain s’adressant aux dieux créés : « Dieux, fils de dieux, considérez de quels ouvrages je suis l’auteur et le père. Ils sont indissolubles, parce que je le veux ; car tout ce qui est composé peut se dissoudre ; mais il est d’un méchant de vouloir séparer ce que la raison a uni. Ainsi, ayant commencé d’être, vous ne sauriez être immortels, ni absolument indissolubles ; mais vous ne serez jamais dissous et vous ne connaîtrez aucune sorte de mort, parce que la mort ne peut rien contre ma volonté, laquelle est un lien plus fort et plus puissant que ceux dont vous fûtes, unisau moment de votre naissance. » Voilà donc les dieux qui, tout mortels qu’ils sont comme composés de corps et d’âme, ne laissent pas, suivant Platon, d’être immortels par la volonté de Dieu qui les a faits. Si donc c’est une peine pour l’âme d’être unie à un corps, quel qu’il soit, d’où vient que Dieu cherche en quelque sorte à rassurer les dieux contre la mort, c’est-à-dire contre la séparation de l’âme et du corps, et leur promet qu’ils seront immortels, non par leur nature, composée et non simple, mais par sa volonté ?\par
De savoir maintenant si ce sentiment de Platon touchant les astres est véritable, c’est une autre question. Nous ne tombons pas d’accord que ces globes de lumière qui nous éclairent le jour et la nuit aient des âmes intelligentes et bienheureuses qui les animent, ainsi que Platon l’affirme également de l’univers, comme d’un grand et vaste animal qui contient tous les autres ; mais, je le répète, c’est une autre question que je n’ai pas entrepris d’examiner ici. J’ai cru seulement devoir dire ce peu de mots contre ceux qui sont si fiers de s’appeler platoniciens : orgueilleux porteurs de manteaux, d’autant plus superbes qu’ils sont moins nombreux et qui rougiraient d’avoir à partager le nom de chrétien avec la multitude. Ce sont eux qui, cherchant un point faible dans notre doctrine, s’attaquent à l’éternité des corps, comme s’il y avait de la contradiction à vouloir que l’âme soit bienheureuse et qu’elle soit éternellement unie à un corps ; ils oublient que Platon, leur maître, considère comme une grâce que le Dieu souverain accorde aux dieux créés le privilège de ne point mourir, c’est-à-dire de n’être jamais séparés de leur corps.
\subsection[{Chapitre XVII}]{Chapitre XVII}

\begin{argument}\noindent Contre ceux qui ne veillent pas que des corps terrestres puissent devenir incorruptibles et éternels.
\end{argument}

\noindent Ces mêmes philosophes soutiennent encore que des corps terrestres ne peuvent êtreéternels, bien qu’ils ne balancent point à déclarer que toute la terre, qui est un membre de leur dieu, non du Dieu souverain, mais pourtant d’un grand dieu, c’est-à-dire du monde, est éternelle. Puis donc que le Dieu souverain leur a fait un autre dieu, savoir le monde, supérieur à tous les autres dieux créés, et puisqu’ils croient que ce dieu est un animal doué d’une âme raisonnable ou intellectuelle, qui a pour membres les quatre éléments, dont ils veulent que la liaison soit éternelle et indissoluble, de crainte qu’un si grand dieu ne vienne à périr, pourquoi la ferre, qui est comme le nombril dans le corps de ce grand animal, serait-elle éternelle et les corps des autres animaux terrestres ne le seraient-ils pas, si Dieu le veut ? Il faut, disent-ils, que la terre soit rendue à la terre, et comme c’est de là que les corps des animaux terrestres ont été tirés, ils doivent y retourner et mourir. Mais si quelqu’un disait la même chose du feu, soutenant qu’il faut lui rendre tous les corps qui en ont été tirés pour en former les animaux célestes, que deviendrait l’immortalité promise par le Dieu souverain à tous ces dieux ? Dira-t-on que cette dissolution ne se fait pas pour eux, parce que Dieu, dont la volonté, comme dit Platon, surmonte tout obstacle, ne le veut pas ? Qui empêche donc que Dieu ne le veuille pas non plus pour les corps terrestres, puisqu’il peut faire que ce qui a commencé existe sans fin, que ce qui est formé de parties demeure indissoluble, que ce qui est tiré des éléments n’y retourne pas ? Pourquoi ne ferait-il pas que les corps terrestres fussent impérissables ? Est-ce que Dieu n’est puissant qu’autant que le veulent les Platoniciens, au lieu de l’être autant que le croient les chrétiens ? Vous verrez que les philosophes ont connu le pouvoir et les desseins de Dieu, et que les Prophètes n’ont pu les connaître, c’est-à-dire que les hommes inspirés de l’Esprit de Dieu ont ignoré sa volonté, et que ceux-là l’ont découverte qui ne se sont appuyés que sur d’humaines conjectures !\par
Ils devaient au moins prendre garde de ne pas tomber dans cette contradiction manifeste, de soutenir d’un côté que l’âme ne saurait être heureuse, si elle ne fuit toute sorte de corps, et de dire de l’autre que les âmes des dieux sont bienheureuses quoique éternellement unies à des corps, celle même de Jupiter qui pour eux est le monde, étant liée à tom les éléments qui composent cette sphère immense de la terre aux cieux. Platon veut que cette âme s’étende, selon des lois musicales, depuis le centre de la terre jusqu’aux extrémités du ciel, et que le monde soit un grand et heureux animal dont l’âme parfaitement sage ne doit jamais être séparée de son corps, sans toutefois que cette masse composée de tant d’éléments divers puisse la retarder, ni l’appesantir. Voilà les libertés que les philosophes laissent prendre à leur imagination, et en même temps ils ne veulent pas croire que des corps terrestres puissent devenir immortels par la puissance de la volonté de Dieu, et que les âmes y puissent vivre éternellement bienheureuses sans en être appesanties, comme font cependant leurs dieux dans des corps de feu, et Jupiter même, le roi des dieux, dans la masse de tous ces éléments ? S’il faut qu’une âme, pour être heureuse, fuie toutes sortes de corps, que leurs dieux abandonnent donc les globes célestes ; que Jupiter quitte le ciel et la terre ; ou s’il ne peut s’en séparer, qu’il soit réputé misérable. Mais nos philosophes reculent devant cette alternative :ils n’osent point dire que leurs dieux quittent leur corps, de peur de paraître adorer des divinités mortelles ; et ils ne veulent pas les priver de la félicité, de crainte d’avouer que des dieux sont misérables. Concluons qu’il n’est pas nécessaire pour être heureux de fuir toutes sortes de corps, mais seulement ceux qui sont corruptibles, pesants, incommodes et moribonds, non tels que la bonté de Dieu les donna aux premiers hommes, mais tels qu’ils sont devenus en punition du péché.
\subsection[{Chapitre XVIII}]{Chapitre XVIII}

\begin{argument}\noindent Des corps terrestres que les philosophes prétendent ne pouvoir convenir aux êtres célestes par cette raison que tout ce qui est terrestre est appelé vers la terre par la force naturelle de la pesanteur.
\end{argument}

\noindent Mais il est nécessaire, disent-ils, que le poids naturel des corps terrestres les fixe sur la terre ou les y appelle, et ainsi ils ne peuvent être dans le ciel. Il est vrai que les premiers hommes étaient sur la terre, dans cette région fertile et délicieuse qu’on a nommée le paradis ; mais que nos adversaires considèrent d’un œil plus attentif la nature de la pesanteur ; cela est important pour résoudre plusieurs questions, notamment celle du corps avec lequel Jésus-Christ est monté au ciel, et celle aussi des corps qu’auront les saints au moment de la résurrection. Je dis donc que si les hommes parviennent par leur adresse à faire soutenir sur l’eau certains vases composés des métaux les plus lourds, il est infiniment plus simple et plus croyable que Dieu, par des ressorts qui nous sont inconnus, puisse empêcher les corps pesants de tomber sur la terre, lui qui, selon Platon, fait, quand il le veut, que les choses qui ont un commencement n’aient point de fin, et que celles qui sont composées de plusieurs parties ne soient point dissoutes ? or, l’union des esprits avec les corps est mille fois plus merveilleuse que celle des corps les uns avec les autres. N’est-ce pas aussi une chose aisée à comprendre que des esprits parfaitement heureux meuvent leurs corps sans peine où il leur plaît, corps terrestres à la vérité, mais incorruptibles ? Les anges n’ont-ils pas le pouvoir d’enlever sans difficulté les animaux terrestres d’où bon leur semble, et de les placer où il leur convient ? Pourquoi donc ne croirions-nous pas que les âmes des bienheureux pourront porter ou arrêter leurs corps à leur gré ? Le poids des corps est d’ordinaire en raison de leur masse, et plus il y a de matière, plus la pesanteur est grande ; cependant l’âme porte plus légèrement son corps quand il est sain et robuste que quand il est maigre et malade, bien qu’il reste plus lourd à porter pour autrui dans son embonpoint que dans sa langueur ; d’où il faut conclure que, dans les corps même mortels et corruptibles, l’équilibre et l’harmonie des parties font plus que la masse et le poids. Qui peut d’ailleurs expliquer l’extrême différence qu’il y a entre ce que nous appelons santé et l’immortalité future ? Ainsi donc, que les philosophes ne croient pas avec l’argument du poids des corps avoir raison de notre foi ! Je pourrais leur demander pourquoi ils ne croient pas qu’un corps terrestre puisse être dans le ciel, alors que toute la terre est suspendue dans le vide ; mais ils me répondraient peut-être que tous les corps pesants tendent vers le centre du monde. Je dis donc seulement que si les moindres dieux, à qui Platon adonné la commission de créer l’homme avec les autres animaux terrestres, ont pu, comme il l’avance, ôter au feu la vertu de brûler, sans lui ôter celle de luire et d’éclairer par les yeux, douterons-nous que le Dieu souverain, à qui ce philosophe donne le pouvoir d’empêcher que les choses qui ont un commencement n’aient une fin, et que celles qui sont composées de parties aussi différentes que le corps et l’esprit ne se dissolvent, soit capable d’ôter la corruption et la pesanteur à la chair, qu’il saura bien rendre immortelle sans détruire sa nature ni la configuration de ses membres ? Mais nous parlerons plus amplement, s’il plaît à Dieu, sur la fin de cet ouvrage, de la résurrection des morts et de leurs corps immortels.
\subsection[{Chapitre XIX}]{Chapitre XIX}

\begin{argument}\noindent Contre le système de ceux qui prétendent que les premiers hommes seraient morts, quand même ils n’auraient point péché.
\end{argument}

\noindent Je reprends maintenant ce que j’ai dit plus haut du corps des premiers hommes, et j’affirme que la mort, par où j’entends cette mort dont l’idée est familière à tous et qui consiste dans la séparation du corps et de l’âme, ne leur serait point arrivée, s’ils n’eussent péché. Car bien qu’il ne soit pas permis de douter que les âmes des justes après la mort ne vivent en repos, c’est pourtant une chose manifeste qu’il leur serait plus avantageux de vivre avec leurs corps sains et vigoureux, et cela est si vrai que ceux qui regardent comme une condition de parfait bonheur de n’avoir point de corps condamnent eux-mêmes cette doctrine par leurs propres sentiments. Qui d’entre eux, en effet, oserait placer les hommes les plus sages au-dessus des dieux immortels ? et cependant le Dieu souverain, chez Platon, promet à ces dieux, comme une faveur signalée, qu’ils ne mourront point, c’est-à-dire que leur âme sera toujours unie à leur corps. Or, ce même Platon croit que les hommes qui ont bien vécu en ce monde auront pour récompense de quitter leur corps pour être reçus dans le sein des dieux (qui pourtant ne quittent jamais le leur). C’est de là que plus tard :\par
{\itshape « Ces âmes reviennent aux régions terrestres, libres de leur souvenir et désirant entrer dans des corps nouveaux »} ; \par
comme parle Virgile d’après Platon ; car Platon estime, d’une part, que les âmes des hommes ne peuvent pas être toujours dans leur corps et qu’elles en sont nécessairement séparées par la mort, et, d’autre part, qu’elles ne peuvent pas demeurer toujours sans corps, mais qu’elles les quittent et les reprennent par de continuelles révolutions. Ainsi il y a cette différence, selon lui, entre les sages et le reste des hommes, que les premiers sont portés dans le ciel après leur mort pour y reposer quelque temps, chacun dans son astre, d’où, ensuite, oubliant leurs misères passées, et entraînées par l’impérieux désir d’avoir un corps, ils retournent aux travaux et aux souffrances de cette vie, au lieu que ceux qui ont mal vécu rentrent aussitôt dans des corps d’hommes ou de bêtes suivant leurs démérites. Platon a donc assujetti à cette dure condition de vivre sans cesse les âmes mêmes des gens de bien : sentiment si étrange que Porphyre, comme nous l’avons dit aux livres précédents, Porphyre en a eu honte et a pris le parti non seulement d’exclure les âmes des hommes du corps des bêtes, mais d’assigner aux âmes des gens de bien, une fois délivrées du corps, une demeure éternelle au sein du Père. De cette façon, pour n’en pas dire moins que Jésus-Christ, qui promet une vie éternelle aux saints, il établit dans une éternelle félicité les âmes purifiées de leurs souillures, sans les faire retourner désormais à leurs anciennes misères, et, pour contredire Jésus-Christ, il nie la résurrection des corps et assure que les âmes vivront éternellement d’une vie incorporelle. Et cependant il ne leur défend point d’adorer les dieux, qui ont des corps, ce qui fait voir qu’il n’a pas cru ces âmes d’élite, toutes dégagées du corps qu’elles soient, plus excellentes que les dieux. Pourquoi donc trouver absurde ce que notre religion enseigne, savoir : que les premiers hommes n’auraient point été séparés de leur corps par la mort s’ils n’eussent péché, et que les bienheureux reprendront dans la résurrection les mêmes corps qu’ils ont eus en cette vie, mais tels néanmoins qu’ils ne leur causeront plus aucune peine et ne seront d’aucun obstacle à leur pleine félicité.
\subsection[{Chapitre XX}]{Chapitre XX}

\begin{argument}\noindent Les corps des bienheureux ressuscités seront plus parfaits que n’étaient ceux des premiers hommes dans le paradis terrestre.
\end{argument}

\noindent Ainsi la mort paraît légère aux âmes des fidèles trépassés, parce que leur chair repose en espérance, quelque outrage qu’elle ait paru recevoir après avoir perdu la vie. Car n’en déplaise à Platon, si les âmes soupirent après un corps, ce n’est pas parce qu’elles ont perdu la mémoire, mais plutôt parce qu’elles se souviennent de ce que leur a promis celui qui ne trompe personne et qui nous a garanti jusqu’au moindre de nos cheveux. Elles souhaitent donc avec ardeur et attendent avec patience la résurrection de leurs corps, où elles ont beaucoup souffert, mais où elles ne doivent plus souffrir. Aussi bien, puisqu’elles ne haïssaient pas leur chair lorsqu’elle entrait en révolte contre leur faiblesse et qu’il fallait la retenir sous l’empire de l’esprit, combien leur est-elle plus précieuse, au moment de devenir spirituelle ? Car de même qu’on appelle charnel l’esprit esclave de la chair, on peut bien aussi appeler spirituelle la chair soumise à l’esprit, non qu’elle doive être convertie en esprit, comme le croientquelques-uns sur la foi de cette parole de l’Apôtre : « Corps animal, quand il est mis en terre, notre corps ressuscitera spirituel » ; mais parce qu’elle sera parfaitement soumise à l’esprit, qui en pourra disposer à son gré sans éprouver jamais aucune résistance. En effet, après la résurrection, le corps n’aura pas seulement toute la perfection dont il est capable ici-bas dans la meilleure santé, mais il sera même beaucoup plus parfait que celui des premiers hommes avant le péché. Bien qu’ils ne dussent point mourir, s’ils ne péchaient point, ils ne laissaient pas toutefois de se servir d’aliments, leurs corps n’étant pas encore spirituels. Il est vrai aussi qu’ils ne vieillissaient point, par une grâce merveilleuse que Dieu avait attachée en leur faveur à l’arbre de vie, planté au milieu du paradis avec l’arbre défendu ; mais cela ne les empêchait pas de se nourrir du fruit de tous les autres arbres du paradis, à l’exception d’un seul toutefois, qui leur avait été défendu, non comme une chose mauvaise, mais pour glorifier cette chose excellente qui est la pure et simple obéissance, une des plus grandes vertus que puisse exercer la créature raisonnable à l’égard de son créateur. Ils se nourrissaient donc des autres fruits pour se garantir de la faim et de la soif, et ils mangeaient du fruit de l’arbre de vie pour arrêter les progrès de la mort et de la vieillesse, tellement qu’il semble que le fruit de la vie était dans le paradis terrestre ce qu’est dans le paradis spirituel la sagesse de Dieu, dont il est écrit : « C’est un arbre de vie pour ceux qui l’embrassent. »
\subsection[{Chapitre XXI}]{Chapitre XXI}

\begin{argument}\noindent On peut donner un sens spirituel à ce que l’écriture dit du paradis, pourvu que l’on conserve la vérité de récit historique.
\end{argument}

\noindent De là vient que quelques-uns expliquent allégoriquement tout ce paradis où la sainte Écriture rapporte que furent mis nos premiers parents ; ce qui est dit des arbres et des fruits, ils l’entendent des vertus et des mœurs, soutenant que toutes ces expressions ont un sens exclusivement symbolique. Mais quoi ? faut-il nier la réalité du paradis terrestre parce qu’il peut figurer un paradis spirituel ? c’est comme si l’on voulait dire qu’il n’y a point eu deux femmes, dont l’une s’appelait Agar et l’autre Sara, d’où sont sortis deux enfants d’Abraham, l’un de la servante et l’autre de la femme libre, parce que l’Apôtre dit qu’il découvre ici la figure des deux Testaments ; ou encore qu’il ne sortit point d’eau de la pierre que Moïse frappa de sa baguette, parce que cette pierre peut figurer Jésus-Christ, suivant cette parole du même Apôtre « Or, la pierre était Jésus-Christ. » Rien n’empêche donc d’entendre par le paradis terrestre la vie des bienheureux, par les quatre fleuves, les quatre vertus cardinales, c’est-à-dire la prudence, la force, la tempérance et la justice, par les arbres toutes les sciences utiles, par les fruits des arbres les bonnes mœurs, par l’arbre de vie, la sagesse qui est la mère de tous les biens, et par l’arbre de la science du bien et du mal, l’expérience du commandement violé. Car la peine du péché est bonne puisqu’elle est juste, mais elle n’est pas bonne pour l’homme qui la subit. Et tout cela peut encore se mieux entendre de l’Église, à titre de prophétie, en disant que le paradis est l’Église même, à laquelle on donne ce nom dans le Cantique des Cantiques ; les quatre fleuves du paradis, les quatre évangiles ; les arbres fruitiers, les saints ; leurs fruits, leurs bonnes œuvres ; l’arbre de vie, le Saint des saints, Jésus-Christ ; l’arbre de la science du bien et du mal, le libre arbitre. L’homme en effet qui a méprisé la volonté de Dieu ne saurait faire de soi qu’un usage funeste ; ce qui lui fait connaître quelle différence il y a de se tenir attaché au bien commun de tous, ou de se complaire en son propre bien ; car celui qui s’aime est abandonné à lui-même, afin que comblé de craintes et de misères, il s’écrie avec le Psalmiste, si toutefois il sent ses maux : « Mon âme, s’étant tournée vers elle-même, est tombée dans la confusion », et qu’il ajoute après avoir reconnu sa faiblesse : « Seigneur, je ne mettrai plus ma force qu’en vous. » Ces explications allégoriques du paradis et autres semblables sont très bonnes, pourvu que l’on croie en même temps à la très fidèle exactitude du récit historique.
\subsection[{Chapitre XXII}]{Chapitre XXII}

\begin{argument}\noindent Les corps des saints seront spirituels après la résurrection, mais d’une telle façon pourtant que la chair ne sera pas convertie en esprit.
\end{argument}

\noindent Les corps des saints après la résurrection n’auront plus besoin d’aucun arbre pour les empêcher de mourir de vieillesse ou de maladie, ni d’autres aliments corporels pour les garantir de la faim ou de la soif, parce qu’ils seront revêtus d’une immortalité glorieuse, en sorte que si les élus mangent, ce sera parce qu’ils le voudront, et non par nécessité. C’est ainsi que nous voyons que les anges ont quelquefois mangé avec les hommes, non qu’ils en eussent besoin, mais par complaisance et-pour se proportionner à eux. Et il ne faut pas croire que les anges n’aient mangé qu’en apparence, quand les hommes les ont reçus chez eux sans les connaître et persuadés qu’ils mangeaient comme nous par besoin ; car ces mots de l’ange à Tobie : « Vous m’avez vu manger, mais vous ne l’avez vu qu’avec vos yeux », signifient : Vous croyez que je mangeais comme vous par besoin. — Que si toutefois il est permis d’entendre ce passage autrement et d’adopter une autre opinion peut-être plus vraisemblable, au moins la foi nous oblige-t-elle de croire que Jésus-Christ, après la résurrection, a réellement mangé avec ses disciples, bien qu’il eût déjà une chair spirituelle. Ce n’est donc que le besoin, et non le pouvoir de boire et manger, qui sera ôté aux corps spirituels, et ils ne seront pas spirituels, parce qu’ils cesseront d’être corps, mais parce qu’ils seront animés d’un esprit vivifiant.
\subsection[{Chapitre XXIII}]{Chapitre XXIII}

\begin{argument}\noindent Ce qu’il faut entendre par le corps animal et par le corps spirituel, et ce que c’est que mourir en Adam et être vivifié en Jésus-Christ.
\end{argument}

\noindent De même que nous appelons corps animauxceux qui ont une âme vivante, ainsi on nomme corps spirituels ceux qui ont un esprit vivifiant. Dieu nous garde toutefois de croire que ces corps glorieux deviennent des esprits ! ils gardent la nature du corps, sans en avoir la pesanteur ni la corruption. L’homme alors ne sera pas terrestre, mais céleste, non que le corps qui a été tiré de la terre cesse d’être, mais parce que Dieu le rendra capable de demeurer dans le ciel, en ne changeant pas sa nature, mais ses qualités. Or, le premier homme, qui était terrestre et formé de la terre, a été créé avec une âme vivante et non avec un esprit vivifiant, qui lui était réservé comme prix de son obéissance. C’est pourquoi il avait besoin de boire et de manger pour se garantir de la faim et de la soif, et il n’était pas immortel par sa nature, mais seulement par le moyen de l’arbre de vie qui le défendait de la vieillesse et de la mort ; il ne faut donc point douter que son corps ne fût animal et non spirituel, et cependant, il ne serait point mort, s’il n’eût encouru par son péché l’effet des menaces divines, condamné dès ce moment à disputer au temps et à la vieillesse, à l’aide des aliments dont la bonté de Dieu lui a continué le secours, une vie que son obéissance aurait pu prolonger à jamais.\par
Alors donc que nous entendrions aussi de cette mort sensible qui sépare l’âme d’avec le corps ce que Dieu dit aux premiers hommes : « Du jour que vous mangerez de ce fruit, vous mourrez », on ne devrait point trouver étrange que cette séparation de l’âme et du corps ne se fût pas faite dès le jour même qu’ils mangèrent du fruit défendu, Dès ce jour, en effet, leur nature fut corrompue, et, par une séparation très juste de l’arbre de vie, ils tombèrent dans la nécessité de mourir, avec laquelle nous naissons tous. Aussi, l’Apôtre ne dit pas que le corps mourra, « mais qu’il est mort à cause du péché, et que l’esprit est vivant à cause de la justice ». Et il ajoute : « Si l’Esprit de celui qui a ressuscité Jésus-Christ habite en vous, celui qui a ressuscité Jésus-Christ donnera aussi la vie à vos corps mortels, parce que son Esprit habitera en vous. » Ainsi donc le corps, qui n’a maintenant qu’une âme vivante, recevra alors un esprit vivifiant ; mais, quoiqu’il ait une âme vivante, l’Apôtre ne laisse pas de dire qu’il est mort, parce qu’il est soumis à la nécessité de mourir, au lieuque dans le paradis terrestre, quoiqu’il eût une âme vivante sans avoir encore un esprit vivifiant, on ne pouvait pas dire qu’il fût mort, parce qu’il n’avait point péché et qu’il n’était pas encore sujet à la mort. Or, Dieu ayant marqué la mort de l’âme (qui a lieu lorsqu’il la quitte), en disant : « Adam, où es-tu ? » et celle du corps (qui arrive quand l’âme l’abandonne), en disant encore : « Vous êtes terre, et vous retournerez en terre », il faut croire qu’il n’a rien dit de la seconde mort, parce qu’il a voulu qu’elle fût cachée dans l’Ancien Testament, la réservant pour le Nouveau, où elle est ouvertement déclarée, afin de faire voir que cette première mort, qui est commune à tous, vient du premier péché, qui d’un seul homme s’est communiqué à tous. Quant à la seconde mort, elle n’est pas commune à tous, « à cause de ceux que Dieu a connus et prédestinés de toute éternité », comme dit l’Apôtre, « pour être conformes à l’image de son Fils, afin « qu’il fût l’aîné de plusieurs frères » ; ceux-là, en effet, la grâce du Médiateur les en a délivrés.\par
Voici comment l’Apôtre témoigne que le premier homme a été créé dans un corps animal. Voulant distinguer notre corps, qui est maintenant animal, de ce même corps qui sera spirituel dans la résurrection, il dit : « Le corps est semé plein de corruption, et il ressuscitera incorruptible ; il est semé avec ignominie, et il ressuscitera glorieux ; il est semé dans la faiblesse, et il ressuscitera dans la vigueur ; il est semé corps animal, et il ressuscitera corps spirituel. » Et pour montrer ce que c’est qu’un corps animal : « Il est écrit », ajoute-t-il, « que le premier homme a été créé avec une âme vivante. » L’Apôtre veut donc qu’on entende par ces paroles de l’Écriture : « Le premier homme a été créé avec une âme vivante », qu’il a été créé avec un corps animal ; et il montre ce qu’il faut entendre par un corps spirituel, quand il ajoute : « Mais le second Adam a été rempli d’un esprit vivifiant » ; par où il marque Jésus-Christ, qui est ressuscité d’une telle manière qu’il ne peut plus mourir. Il poursuit et dit : « Mais ce n’est pas le corps spirituel qui a été formé le premier, c’est le corps animal, et ensuite le spirituel » ; par où il montre encore plus clairement qu’il a entendu le corps animal dans ces paroles : « Le premier homme a été créé avec une âmevivante », et le spirituel, quand il a dit : « Le second Adam a été rempli d’un esprit vivifiant. »\par
Le corps animal est le premier, tel que l’a eu le premier Adam (qui toutefois ne serait point mort s’il n’eût péché), tel que nous l’avons depuis que la nature corrompue par le péché nous a soumis à la nécessité de mourir, tel que Jésus-Christ même a voulu l’avoir d’abord ; mais après vient le spirituel, tel qu’il est déjà dans Jésus-Christ comme dans notre chef et tel qu’il sera dans ses membres lors de la dernière résurrection des morts.\par
L’Apôtre signale ensuite une notable différence entre ces deux hommes, lorsqu’il dit : « Le premier homme est terrestre et formé de la terre, et le second est céleste et descendu du ciel, Comme le premier homme a été terrestre, ses enfants aussi sont terrestres ; et comme le second homme est céleste, ses enfants aussi sont célestes. De même donc que nous portons l’image de l’homme terrestre, portons aussi l’image de l’homme céleste. » Ce que dit ici l’Apôtre commence maintenant en nous par le sacrement de la régénération, ainsi qu’il le témoigne ailleurs par ces paroles : « Tous, tant que vous êtes, qui avez été baptisés en Jésus-Christ, vous vous êtes revêtus de Jésus-Christ » ; mais la chose ne s’accomplira entièrement que lorsque ce qu’il y a d’animal en nous par la naissance sera devenu spirituel par la résurrection ; car, pour me servir encore des paroles de saint Paul : « Nous sommes sauvés par l’espérance. » Or, nous portons l’image de l’homme terrestre à cause de la désobéissance et de la mort qui sont passées en nous par la génération, et nous portons celle de l’homme céleste à cause du pardon et de la vie que nous recevons dans la régénération par le médiateur entre Dieu et les hommes, Jésus-Christ homme, qui est cet homme céleste dont veut parler saint Paul, parce qu’il est venu du ciel pour se revêtir d’un corps mortel et le revêtir lui-même d’immortalité. S’il appelle aussi les enfants du Christ célestes, c’est qu’ils deviennent ses membres par sa grâce pour faire ensemble un même Christ. Il déclare encore ceci plusexpressément dans la même épître, quand il dit : « La mort est venue par un homme, et la résurrection doit aussi venir par un homme ; car comme tous meurent en Adam, ainsi tous revivent en Jésus-Christ », c’est-à-dire dans un corps spirituel qui sera animé d’un esprit vivifiant. Ce n’est pas toutefois que tous ceux qui meurent en Adam doivent devenir membres de Jésus-Christ, puisqu’il y en aura beaucoup plus qui seront punis pour toute l’éternité de la seconde mort ; mais l’Apôtre se sert du terme général de tous, pour montrer que comme personne ne meurt qu’en Adam dans ce corps animal, personne ne ressuscitera qu’en Jésus-Christ avec un corps spirituel. Il ne faut donc pas s’imaginer que nous devions avoir à la résurrection un corps semblable à celui du premier homme avant le péché : alors même, le sien n’était pas spirituel, mais animal ; et ceux qui sont dans un autre sentiment ne se rendent pas assez attentifs à ces paroles du grand docteur : « Comme il y a, dit-il, un corps animal, il y a aussi un corps spirituel, ainsi qu’il est écrit Adam, le premier homme, a été créé avec une âme vivante. » Peut-on dire qu’il soit ici question de l’âme d’Adam après le péché ? évidemment non ; car il s’agit du premier état où l’homme a été créé, et l’Apôtre rapporte ce passage de la Genèse pour montrer justement ce que c’est que le corps animal.
\subsection[{Chapitre XXIV}]{Chapitre XXIV}

\begin{argument}\noindent Comment il faut entendre ce souffle de Dieu dont parle l’Écriture et qui donne à l’homme une âme vivante, et cet autre souffle que Jésus-Christ exhale en disant : recevez l’Esprit-Saint.
\end{argument}

\noindent Quelques-uns se sont persuadé avec assez peu de raison que le passage de la Genèse où on lit : « Dieu souffla contre la face d’Adam un esprit de vie, et l’homme fut créé âme vivante », ne doit pas s’entendre de Dieu donnant au premier homme une âme, mais de Dieu ne faisant que vivifier par le Saint-Esprit celle que l’homme avait déjà. Ce qui les porte à interpréter ainsi l’Écriture, c’estque Notre-Seigneur Jésus-Christ, après la résurrection, souffla sur ses disciples et leur dit : « Recevez le Saint-Esprit » ; d’où ils concluent que, puisque la même chose se passa dans la création de l’homme, le même effet s’ensuivit aussi : comme si l’Évangéliste, après avoir parlé du souffle de Jésus sur ses disciples, avait ajouté, ainsi que fait Moïse, qu’ils devinrent âmes vivantes. Mais quand il l’aurait ajouté, cela ne signifierait autre chose, sinon que l’Esprit de Dieu est en quelque façon la vie de l’âme, et que sans lui elle est morte, quoique l’homme reste vivant. Mais rien de semblable n’eut lieu au moment de la création de l’homme, ainsi que le prouvent ces paroles de la Genèse : « Dieu créa ({\itshape formavit}) l’homme poussière de la terre » ; ce que certains interprètes croient rendre plus clair en traduisant : « Dieu composa ({\itshape finxit}) l’homme du limon de la terre », parce qu’il est écrit aux versets précédents : « Or, une fontaine s’élevait de la terre et en arrosait toute la surface » ; ce qui engendrait, suivant eux, ce limon dont l’homme fut formé ; et c’est immédiatement après que l’Écriture ajoute : « Dieu créa l’homme poussière de la terre », comme le portent les exemplaires grecs sur lesquels l’Écriture a été traduite en latin. Au surplus, que l’on rende par {\itshape formavit} ou par {\itshape finxit} le mot grec {\itshape eplasen}, peu importe à la question ; {\itshape finxit} est le mot propre, et c’est la crainte de l’équivoque qui a décidé ceux qui ont préféré {\itshape formavit}, l’usage donnant à l’expression {\itshape finxit} le sens de fiction mensongère. C’est donc cet homme ainsi fait de la poussière de la terre ou du limon, c’est-à-dire d’une poussière trempée d’eau, dont saint Paul dit qu’il devint un corps animal, lorsqu’il reçut l’âme. « Et l’homme devint âme vivante », entendez que cette poussière ainsi pétrie devint une âme vivante.\par
Mais, disent-ils, il avait déjà une âme ; autrement on ne l’appellerait pas homme, l’homme n’étant pas le corps seul ou l’âme seule, mais le composé des deux. Il est vrai que l’âme, non plus que le corps, n’est pas l’homme entier ; mais l’âme en est la plus noble partie. Quand elles sont unies ensemble, elles prennent le nom d’homme, qu’elles ne quittent pas néanmoins après leur séparation. Ne disons-nous pas tous les jours : Cet homme est mort, et maintenant il est dans la paix ou dans les supplices, bien que cela ne se puisse dire que de l’âme seule ; ou : Cet homme a été enterré en tel ou tel lieu, quoique cela ne se puisse entendre que du corps seul ? Diront-ils que ce n’est pas la façon de parler de l’Écriture ? Mais elle ne fait point difficulté d’appeler homme l’une ou l’autre de ces deux parties, lors même qu’elles sont unies, et de dire que l’âme est l’homme intérieur et le corps l’homme extérieur, comme si c’étaient deux hommes, bien qu’en effet ce n’en soit qu’un. Aussi bien il faut entendre dans quel sens l’Écriture dit que l’homme est fait à l’image de Dieu, et dans quel sens elle l’appelle terre et dit qu’il retournera en terre. La première parole s’entend de l’âme raisonnable, telle que Dieu la créa par son souffle dans l’homme, c’est-à-dire dans le corps de l’homme ; et la seconde s’entend du corps, tel que Dieu le forma de la poussière, et à qui l’âme fut donnée pour en faire un corps animal, c’est-à-dire un homme ayant une âme vivante.\par
C’est pourquoi, quand Notre-Seigneur souffla sur ses disciples en disant : « Recevez le Saint-Esprit », il voulait nous apprendre que le Saint-Esprit n’est pas seulement l’Esprit du Père, mais encore l’Esprit du Fils unique, attendu que le même Esprit est l’Esprit du Père et du Fils, formant avec tous deux la Trinité, Père, Fils et Saint-Esprit, qui n’est pas créature, mais créateur. En effet, ce souffle corporel qui sortit de la bouche de Jésus-Christ n’était point la substance ou la nature du Saint-Esprit, mais plutôt, je le répète, un signe pour nous faire entendre que le Saint-Esprit est commun au Père et au Fils ; car ils n’en ont pas chacun un, et il n’y en a qu’un pour deux. Or, ce Saint-Esprit est toujours dans l’Écriture appelé en grec {\itshape pneuma}, ainsi que Notre-Seigneur l’appelle ici, lorsque l’exprimant par le souffle de sa bouche, il le donne à ses disciples ; et je ne me souviens point qu’il y soit appelé autrement : au lieu que dans le passage de la Genèse, où il est dit que « Dieu forma l’homme de la poussière de la terre, et qu’il souffla contre sa face un esprit de vie », le grec ne porte pas {\itshape pneuma}, mais {\itshape pnoè}, terme dont l’Écriture se sert plus souvent pour désigner la créature que le Créateur ; d’où vient que quelques interprètes, pour en marquer la différence, ont mieux aimé le rendre par le mot souffle, que par celui d’esprit. Il se trouve employé de la sorte dans Isaïe, où Dieu dit : « J’ai fait tout souffle », c’est-à-dire toute âme. Les interprètes donc expliquent quelquefois, il est vrai, ce dernier mot par souffle, ou par esprit, ou par inspiration ou aspiration, ou même par âme ; mais jamais ils ne traduisent l’autre que par esprit, soit celui de l’homme dont l’Apôtre dit : « Quel est celui des hommes qui connaît ce qui est en l’homme, si ce n’est l’esprit même de l’homme qui est en lui ? » soit celui de la bête, comme quand Salomon dit : « Qui sait si l’esprit de l’homme monte en haut dans le ciel, et si l’esprit de la bête descend en bas dans la terre ? » soit même cet esprit corporel qu’on nomme aussi vent, comme dans le Psalmiste : « Le feu, la grêle, la neige, la glace, l’esprit de tempête » ; soit enfin l’esprit créateur, tel que celui dont Notre-Seigneur dit dans l’Évangile, en l’exprimant par son souffle : « Recevez le Saint-Esprit », et ailleurs : « Allez, baptisez toutes les nations « au nom du Père, du Fils et du Saint-Esprit », paroles qui déclarent clairement et excellemment la très sainte Trinité ; et encore : « Dieu est esprit », et en beaucoup d’autres endroits de l’Écriture. Dans tous ces passages, le grec ne porte point le mot équivalent à souffle, mais bien celui qui ne peut se rendre que par esprit. Ainsi, alors même que dans un endroit de la Genèse où il est dit que « Dieu souffla contre la face de l’homme un esprit de vie », il y aurait dans le grec {\itshape pneuma} et non {\itshape pnoè}, il ne s’ensuivrait pas pour cela que nous fussions obligés d’entendre l’Esprit créateur, puisque, comme nous avons dit, l’Écriture ne se sert pas seulement du premier de ces mots pour le Créateur, mais aussi pour la créature,\par
Mais, répliquent-ils, elle ne dirait pas esprit {\itshape de vie}, si elle ne voulait marquer le Saint-Esprit, ni âme {\itshape vivante}, si elle n’entendait la vie de l’âme qui lui est communiquée par le don de l’Esprit de Dieu, puisque, l’âme vivant d’une vie qui lui est propre, il n’était pas besoin d’ajouter vivante, si l’Écriture n’eût voulu signifier cette vie qui lui est donnée par le Saint-Esprit. Qu’est-ce à dire ? et raisonner ainsi, n’est-ce pas s’attacher avec ardeur à ses propres pensées au lieu de se rendre attentif au sens de l’Écriture ? Sans aller bien loin, qu’y avait-il de plus aisé que de lire ce qui est écrit un peu auparavant au même livre de la Genèse : « Que la terre produise des âmes vivantes », quand tous les animaux de la terre furent créés ? Et quelques lignes après, mais toujours au même livre : « Tout ce qui a esprit de vie et tout homme habitant la terre péri », pour dire que tout ce qui vivait sur la terre périt par le déluge ? Puis donc que nous trouvons une âme vivante et un esprit de vie, même dans les bêtes, selon la façon de parler de l’Écriture, et qu’au lieu même où elle dit : « Toutes les choses qui ont un esprit de vie », le grec ne porte pas {\itshape pneuma}, mais {\itshape pnoè}, que ne disons-nous aussi : Où est la nécessité de dire vivante, l’âme ne pouvant être, si elle ne vit, et d’ajouter de vie, après avoir dit esprit ? Cela nous fait donc voir que lorsque l’Écriture use de ces mêmes termes en parlant de l’homme, elle ne s’est point éloignée de son langage ordinaire ; mais elle a voulu que l’on entendît par là le principe du sentiment dans les animaux ou les corps animés. Et dans la formation de l’homme, n’oublions pas encore que l’Écriture reste fidèle à son langage habituel, quand elle nous enseigne qu’en recevant l’âme raisonnable, non pas émanée de la terre ou des eaux, comme l’âme des créatures charnelles, mais créée par le souffle de Dieu, l’homme n’en est pas moins destiné à vivre dans un corps animal, où réside une âme vivante, comme ces animaux dont l’Écriture a dit : « Que la terre produise toute âme vivante » ; et quand elle dit également qu’ils ont l’esprit de vie, le grec portant toujours {\itshape pnoè} et non {\itshape pneuma}, ce n’est assurément pas le Saint-Esprit, mais bien l’âme vivante qui est désignée par cette expression.\par
Le souffle de Dieu, disent-ils encore, est sorti de sa bouche ; de sorte que si nous croyons que c’est l’âme, il s’ensuivra que nous serons obligés aussi d’avouer qu’elle est consubstantielle et égale à cette Sagesse qui a dit : « Je suis sortie de la bouche du Très-Haut. » Mais la Sagesse ne dit pas qu’elle est le souffle de Dieu, mais qu’elle est sortie de sa bouche. Or, de même que nous pouvons former un souffle, non de notre âme, qui nous fait hommes, mais de l’air qui nous entoure et que nous respirons, ainsi Dieu, qui est tout-puissant, a pu très bien aussi en former un, non de sa nature, ni d’aucune chose créée, mais du néant, et le mettre dans le corps de l’homme. D’ailleurs, afin que ces habiles personnes qui se mêlent de parler de l’Écriture et n’en étudient pas le langage, apprennent qu’elle ne fait pas sortir de la bouche de Dieu seulement ce qui est de même nature que lui, qu’elles écoutent ce que Dieu y dit : « Tu es tiède, tu n’es ni froid ni chaud ; c’est pourquoi je vais te vomir de ma bouche. »\par
Il ne faut donc plus résister aux paroles expresses de l’Apôtre, lorsque distinguant lecorps animal du corps spirituel, c’est-à-dire celui que nous avons maintenant de celui que nous aurons un jour, il dit : « Le corps est semé animal, et il ressuscitera spirituel. Comme il y a un corps animal, il y a aussi un corps spirituel, ainsi qu’il est écrit : Adam, le premier homme, a été créé avec une âme vivante, et le second Adam a été rempli d’un esprit vivifiant. Mais ce n’est pas le corps spirituel qui a été formé le premier, c’est le corps animal, et ensuite le spirituel. Le premier homme est le terrestre formé de la terre, et le second homme est le céleste descendu du ciel. Comme le premier homme a été terrestre, ses enfants sont aussi terrestres ; et comme le second homme est céleste, ses enfants sont aussi célestes. De la même manière donc que nous avons porté l’image de l’hommeterrestre, portons aussi l’image de l’homme céleste. » Ainsi le corps animal, dans lequel l’Apôtre dit que fut créé le premier homme, n’était pas composé de telle façon qu’il ne pût mourir, mais de telle façon qu’il ne fût point mort si l’homme n’eût péché. Le corps qui sera spirituel, parce que l’Esprit le vivifiera, ne pourra mourir, non plus que l’âme, qui, bien qu’elle meure en quelque façon en se séparant de Dieu, conserve néanmoins toujours une vie qui lui est propre. Il en est de même des mauvais anges qui, pour être séparés de Dieu, ne laissent pas de vivre et de sentir, parce qu’ils ont été créés immortels, tellement que la seconde mort même où ils seront précipités après le dernier jugement ne leur ôtera pas la vie, puisqu’elle leur fera souffrir de cruelles douleurs. Mais les hommes qui appartiennent à la grâce et qui seront associés aux saints anges dans la béatitude seront revêtus de corps spirituels, de manière à ce qu’ils ne pécheront ni ne mourront plus.\par
Reste une question qui doit être discutée et, avec l’aide de Dieu, résolue, c’est de savoir comment les premiers hommes auraient pu engendrer des enfants s’ils n’eussent point péché, puisque nous disons que les mouvements de la concupiscence sont des suites du péché. Mais il faut finir ce livre, et d’ailleurs la question demande à être traitée avec quelque étendue ; il vaut donc mieux la remettre au livre suivant.
\section[{Livre quatorzième. Le péché originel}]{Livre quatorzième. \\
Le péché originel}\renewcommand{\leftmark}{Livre quatorzième. \\
Le péché originel}

\subsection[{Chapitre premier}]{Chapitre premier}

\begin{argument}\noindent La désobéissance du premier homme entraînerait tous ses enfants dans l’abîme éternel de la seconde mort, si la grâce de Dieu n’en sauvait plusieurs.
\end{argument}

\noindent Nous avons déjà dit aux livres précédent que Dieu, voulant unir étroitement les hommes non seulement par la communauté de nature mais aussi par les nœuds de la parenté, les a fait tous sortir d’un seul, et que l’espèce humaine n’eût point été sujette à la mort, si Adam et Ève (celle-ci tirée du premier homme, tiré lui-même du néant) n’eussent mérité ce châtiment par leur désobéissance, qui a corrompu toute la nature humaine et transmis leur péché à leurs descendants, aussi bien que la nécessité de mourir. Or, l’empire de la mort s’est dès lors tellement établi parmi les hommes, qu’ils seraient tous précipités dans la seconde mort qui n’aura point de fin, si une grâce de Dieu toute gratuite n’en sauvait quelques-uns. De là vient que tant de nations qui sont dans le monde, si différentes de mœurs, de coutumes et de langage, ne forment toutes ensemble que deux sociétés d’hommes, que nous pouvons justement appeler cités, selon le langage de l’Écriture. L’une se compose de ceux qui veulent vivre selon la chair, et l’autre de ceux qui veulent vivre selon l’esprit ; et quand les uns et les autres ont obtenu ce qu’ils désirent, ils sont en paix chacun dans son genre.
\subsection[{Chapitre II}]{Chapitre II}

\begin{argument}\noindent Ce qu’il faut entendre par vivre selon la chair.
\end{argument}

\noindent Et d’abord, qu’est-ce que vivre selon la chair, qu’est-ce que vivre selon l’esprit ? Celui qui ne serait pas fort versé dans le langage de l’Écriture pourrait s’imaginer que les Épicuriens et les autres philosophes sensualistes, et tous ceux qui, sans faire profession de philosophie, ne connaissent et n’aiment que les plaisirs des sens, sont les seuls qui vivent selon la chair, parce qu’ils mettent le souverain bien de l’homme dans la volupté du corps, tandis que les Stoïciens, qui le mettent dans l’âme, vivent selon l’esprit ; mais il n’en est point ainsi, et, dans le sens de l’Écriture, les uns et les autres vivent selon la chair. En effet, elle n’appelle pas seulement chair le corps de tout animal mortel et terrestre, comme quand elle dit : « Toute chair n’est pas la même chair ; car autre est la chair de l’homme, autre celle des bêtes, autre celle des oiseaux, autre celle des poissons » ; elle donne encore à ce mot beaucoup d’autres acceptions ; elle lui fait entre autres signifier l’homme même, en prenant la partie pour le tout, comme dans ce passage de l’Apôtre « Nulle chair ne sera justifiée par les œuvres « de la loi » ; où par nulle chair on doit entendre nul homme, ainsi que saint Paul le déclare lui-même dans son épître aux Galates : « {\itshape Nul homme ne sera justifié par la loi} », et peu après : « Sachant que nul homme ne sera justifié par les œuvres de la loi. » C’est en ce sens que doivent se prendre ces paroles de saint Jean : « Le Verbe s’est fait chair », c’est-à-dire {\itshape homme}. Quelques-uns, pour avoir mal entendu ceci, ont pensé que Jésus-Christ n’avait point d’âme humaine. De même, en effet, que l’on entend la partie pour le tout dans ces paroles de Marie-Madeleine : « Ils ont enlevé mon Seigneur et je ne sais où ils l’ont mis » ; par où elle n’entend parler que de son corps, qu’elle croyait enlevé du tombeau, de même on entend quelquefois le tout pour la partie, comme dans les expressions que nous venons de rapporter.\par
Puis donc que l’Écriture prend ce mot de chair en plusieurs façons qu’il serait trop long de déduire, si nous voulons savoir ce que c’est que vivre selon la chair, considérons attentivement cet endroit de saint Paul aux Galates, où il dit : « Les œuvres de la chair sont aisées à connaître, comme l’adultère, la fornication, l’impureté, l’impudicité, l’idolâtrie, les empoisonnements, les inimitiés, les contentions, les jalousies, les animosités, les dissensions, les hérésies, les envies, l’ivrognerie, les débauches, et autres semblables dont je vous ai dit et vous dis encore que ceux qui commettent ces crimes ne posséderont point le royaume de Dieu. » Parmi les œuvres de la chair que l’Apôtre dit qu’il est aisé de connaître et qu’il condamne, nous ne trouvons pas seulement celles qui concernent la volupté du corps, comme la fornication, l’impureté, l’impudicité, l’ivrognerie, la gourmandise, mais encore celles qui ne regardent que l’esprit. En effet, qui ne demeurera d’accord que l’idolâtrie, les empoisonnements, les inimitiés, les contentions, les jalousies, les animosités, les dissensions, les hérésies et les envies, sont plutôt des vices de l’âme que du corps ? Il se peut faire qu’on s’abstienne des plaisirs du corps pour se livrer à l’idolâtrie ou pour former quelque hérésie, et cependant un homme de la sorte est convaincu par l’autorité de l’Apôtre de ne pas vivre selon l’esprit, et, dans son abstinence même des voluptés de la chair, il est certain qu’il pratique les œuvres damnables de la chair. Les inimitiés ne sont-elles pas dans l’esprit ? Qui s’aviserait de dire à son ennemi : Vous avez une mauvaise chair contre moi, pour dire une mauvaise volonté ? Enfin, il est clair que les animosités se rapportent à l’âme, comme les ardeurs charnelles à la chair. Pourquoi donc le Docteur des Gentils appelle-t-il tout cela œuvres de la chair, si ce n’est en usant de cette façon de parler qui fait qu’on exprime le tout par la partie, c’est-à-dire par la chair l’homme tout entier ?
\subsection[{Chapitre III}]{Chapitre III}

\begin{argument}\noindent La chair n’est pas cause de tous les péchés.
\end{argument}

\noindent Prétendre que la chair est cause de tous les vices, et que l’âme ne fait le mal que parcequ’elle est sujette aux affections de la chair, ce n’est pas faire l’attention qu’il faut à toutela nature de l’homme. Il est vrai que « le corps corruptible appesantit l’âme » ; d’oùvient que l’Apôtre, parlant de ce corps corruptible, dont il avait dit un peu auparavant : « Quoique notre homme extérieur se corrompe », ajoute : « Nous savons que si cette maison de terre vient à se dissoudre, Dieu doit nous donner dans le ciel une autre maison qui ne sera point faite de la main des hommes. C’est ce qui nous fait soupirer après le moment de nous revêtir de la gloire de cette maison céleste, si toutefois nous sommes trouvés vêtus, et non pas nus. Car, pendant que nous sommes dans cette demeure mortelle, nous gémissons sous le faix ; et néanmoins nous ne désirons pas être dépouillés, mais revêtus par-dessus, en sorte que ce qu’il y a de mortel en nous soit absorbé par la vie. » Nous sommes donc tirés en bas par ce corps corruptible comme par un poids ; mais parce que nous savons que cela vient de la corruption du corps et non de sa nature et de sa substance, nous ne voulons pas en être dépouillés, mais être revêtus d’immortalité. Car ce corps demeurera toujours ; mais comme il ne sera pas corruptible, il ne nous appesantira point. Il reste donc vraiqu’ici-bas « le corps corruptible appesantit l’âme, et que cette demeure de terre abat l’esprit qui pense beaucoup », et, en même temps, c’est une erreur de croire que tous lesdérèglements de l’âme viennent du corps. Vainement Virgile exprime-t-il en ces beauxvers la doctrine platonicienne :\par
 {\itshape « Filles du ciel, les âmes sont animées d’une flamme divine, tant qu’une enveloppe corporelle ne vient pas engourdir leur activité sous le poids de terrestres organes et de membres moribonds. »} \par
Vainement rattache-t-il au corps ces quatre passions bien connues de l’âme : le désir et la crainte, la joie et la tristesse, où il voit la source de tous les vices : Notre foi nous enseigne toute autre chose. Elle nous dit que la corruption du corps qui appesantit l’âme n’est pas la cause, mais là peinedu premier péché ; de sorte qu’il ne faut pas attribuer tous les désordres à la chair, encore qu’elle excite en nous certains désirs déréglés ; car ce serait justifier le diable, qui n’a point de chair. On ne peut assurément pas dire qu’il soit fornicateur, ni ivrogne, ni sujet aux autres péchés de la chair ; et cependant il ne laisse pas d’être extrêmement superbe et envieux ; il l’est au point que c’est pour cela que, selon l’apôtre saint Pierre, il a été précipité dans les prisons obscures de l’air et destiné à des supplices éternels. Or, ces vices qui ont établi leur empire chez le diable, saint Paul les attribue à la chair, bien qu’il soit certain que le diable n’a point de chair. Il dit que les inimitiés, les contentions, les jalousies, les animosités et les envies sont les œuvres de la chair, aussi bien que l’orgueil, qui est la source de tous ces vices, et celui qui domine particulièrement dans le diable. En effet, qui est plus ennemi des saints que lui ? qui a plus d’animosité contre eux ? qui est plus jaloux de leur gloire ? tous ces vices étant en lui sans la chair, comment entendre que ce sont les œuvres de la chair, sinon parce que ce sont les œuvres de l’homme, identifié par saint Paul avec la chair ? Ce n’est pas, en effet, pour avoir une chair (car le diable n’en a point), mais pour avoir voulu vivre selon lui-même, c’est-à-dire selon l’homme, que l’homme est devenu semblable au diable. Le diable a voulu vivre aussi selon lui-même, quand il n’est pas demeuré dans la vérité ; en sorte que quand il mentait, cela ne venait pas de Dieu, mais de lui-même, de lui qui n’est pas seulement menteur, mais aussi le père du mensonge ; de lui qui a menti le premier, et qui n’est l’auteur du péché que parce qu’il est l’auteur du mensonge.
\subsection[{Chapitre IV}]{Chapitre IV}

\begin{argument}\noindent Ce que c’est que vivre selon l’homme et que vivre selon Dieu.
\end{argument}

\noindent Lors donc que l’homme vit selon l’homme, et non selon Dieu, il est semblable au diable, parce que l’ange même ne devait pas vivre selon l’ange, mais selon Dieu, pour demeurer dans la vérité et pour parler le langage de la vérité qui vient de Dieu, et non celui du mensongesonge qu’il tire de son propre fond. Si le même Apôtre dit dans un autre endroit : « La vérité a éclaté davantage par mon mensonge » ; n’est-ce pas déclarer que le mensonge est de l’homme, et la vérité de Dieu ? Ainsi, quand l’homme vit selon la vérité, il ne vit pas selon lui-même, mais selon Dieu ; car c’est Dieu qui a dit : « Je suis la vérité. » Quand il vit selon lui-même, il vit selon le mensonge, non qu’il soit lui-même mensonge, ayant pour auteur et pour créateur un Dieu qui n’est point auteur ni créateur du mensonge, mais parce que l’homme n’a pas été créé innocent pour vivre selon lui-même, mais pour vivre selon celui qui l’a créé, c’est-à-dire pour faire plutôt la volonté de Dieu que la sienne. Or, ne pas vivre de la façon pour laquelle il a été créé, voilà le mensonge. Car il veut certainement être heureux, même en ne vivant pas comme il faut pour l’être, et quoi de plus mensonger que cette volonté ? Aussi peut-on fort bien dire que tout péché est un mensonge. Nous ne péchons en effet que par la même volonté qui nous porte à désirer d’être heureux, ou à craindre d’être malheureux. Il y a donc mensonge, quand ce que nous faisons pour devenir heureux ne seul qu’à nous rendre malheureux. Et d’où vient cela, sinon de ce que l’homme ne saurait trouver son bonheur qu’en Dieu, qu’il abandonne en péchant, et non en soi-même ?\par
Nous avons dit que tous les hommes sont partagés en deux cités différentes et contraires, parce que les uns vivent selon la chair, et les autres selon l’esprit ; on peut aussi exprimer la même idée en disant que les uns vivent selon l’homme, et les autres selon Dieu. Saint Paul use même de cette expression dans son épître aux Corinthiens, quand il dit : « Puisqu’il y a encore des rivalités et des jalousies parmi vous, n’est-il pas visible que vous êtes charnels et que vous marchez encore selon l’homme ? » C’est donc la même chose de marcher selon l’homme et d’être charnel, en prenant la chair, c’est-à-dire une partie de l’homme pour l’homme tout entier. Il avait appelé un peu auparavant animaux ceux qu’il nomme ici charnels : « Qui des hommes, dit-il, connaît ce qui est en l’homme, si ce n’est l’esprit même de l’homme qui est en lui ? Ainsi personne ne connaît ce qui est en Dieu que l’esprit de Dieu. Or, nous n’avons pas reçu l’esprit du monde, mais l’esprit de Dieu, pour connaître les dons que Dieu nous a faits ; et nous les annonçons, non dans le docte langage de la sagesse humaine, mais comme des hommes instruits par l’esprit de Dieu et qui parlent spirituellement des choses spirituelles. Pour l’homme animal, il ne conçoit point ce qui est l’esprit de Dieu ; car cela passe à son sens pour une folie. » Il s’adresse à ces sortes d’hommes qui sont encore animaux, lorsqu’ildit un peu après : « Aussi, mes frères, n’ai-je pu vous parler comme à des personnes spirituelles, mais comme à des hommes qui sont encore charnels » ; ce que l’on doit encore entendre de la même manière, c’est-à-dire la partie pour le tout. L’homme tout entier peut être désigné par l’esprit ou par la chair, qui sont les deux parties qui le composent ; et dès lors l’homme animal et l’homme charnel ne sont point deux choses différentes, mais une même chose, c’est-à-dire l’homme vivant selon l’homme. Et c’est ainsi qu’on ne doit entendre que l’homme, soit en ce passage : « Nulle chair ne sera justifiée par les œuvres de la loi » ; soit en celui-ci : « Soixante et quinze âmes descendirent en Égypte avec Jacob. » Toute chair veut dire tout homme, et soixante-quinze âmes est pour soixante-quinze hommes. L’Apôtre dit : « Je ne vous parlerai pas le docte langage de la sagesse humaine » ; il aurait pu dire : {\itshape de la sagesse charnelle}. Il dit aussi : « Vous marchez selon l’homme » ; dans le même sens où ilaurait dit : {\itshape selon la chair}. Mais cela paraît plus clairement dans ces paroles : « Lorsque l’un dit : Je suis à Paul, et l’autre : Je suis à Apollon, n’êtes-vous pas encore des hommes ? ». Il appelle hommes ceux qu’il avait auparavant appelés charnels et animaux. Vous êtes des hommes, dit-il, c’est-à-dire vous vivez selon l’homme, et non pas selon Dieu ; car si vous viviez selon Dieu, vous seriez des dieux.
\subsection[{Chapitre V}]{Chapitre V}

\begin{argument}\noindent L’opinion des Platoniciens touchant la nature de l’âme et celle du corps est plus supportable que celle des Manichéens ; toutefois nous la rejetons en ce point qu’elle fait venir du corps tous les désirs déréglés.
\end{argument}

\noindent Il ne faut donc pas, lorsque nous péchons, accuser la chair en elle-même, et faire retomber ce reproche sur le Créateur, puisque la chair est bonne en son genre ; ce qui n’est pas bon, c’est d’abandonner le Créateur pour vivre selon un bien créé, soit qu’on veuille vivre selon la chair, ou selon l’âme, ou selon l’homme tout entier, qui est composé des deux ensemble. Celui qui glorifie l’âme comme le souverain bien et qui condamne la chair comme un mal, aime l’une et fuit l’autre charnellement, parce que sa haine, aussi bien que son amour, ne sont pas fondés sur la vérité, mais sur une fausse imagination. Les Platoniciens, je l’avoue, ne tombent pas dans l’extravagance des Manichéens et ne détestent pas avec eux les corps terrestres comme une nature mauvaise, puisqu’ils font venir tous les éléments dont ce monde visible est composé et toutes leurs qualités de Dieu comme créateur. Mais ils croient que le corps mortel fait de telles impressions sur l’âme, qu’il engendre en elle la crainte, le désir, la joie et la tristesse, quatre {\itshape perturbations}, pour parler avec Cicéron, ou, si l’on veut se rapprocher du grec, quatre {\itshape passions}, qui sont la source de la corruption des mœurs. Or, si cela est, d’où vient qu’Énée, dans Virgile, entendant dire à son père que les âmes retourneront dans les corps après les avoir quittés, est surpris et s’écrie :\par
 {\itshape « Ô mon père, faut-il croire que les âmes, après être montées au ciel, quittent ces sublimes régions pour revenir dans des corps grossiers ? Infortunés ! d’où leur vient ce funeste amour de la lumière ? »} \par
Je demande à mon tour si, dans cette pureté tant vantée où s’élèvent ces âmes, le funeste amour de la lumière peut leur venir de ces organes terrestres et de ces membres moribonds ? Le poète n’assure-t-il pas qu’elles ont été délivrées de toute contagion charnelle alors qu’elles veulent retourner dans des corps ? Il résulte de là que cette révolutionéternelle des âmes, fût-elle aussi vraie qu’elle est fausse, on ne pourrait pas dire que tous leurs désirs déréglés leur viennent du corps, puisque, selon les Platoniciens et leur illustre interprète, le funeste amour de la lumière ne vient pas du corps, mais de l’âme, qui en est saisie au moment même où elle est libre de tout corps et purifiée de toutes les souillures de la chair. Aussi conviennent-ils que ce n’est pas seulement le corps qui excite dans l’âme des craintes, des désirs, des joies et des tristesses, mais qu’elle peut être agitée par elle-même de tous ces mouvements.
\subsection[{Chapitre VI}]{Chapitre VI}

\begin{argument}\noindent Les mouvements de l’âme sont bons ou mauvais, selon que la volonté est bonne ou mauvaise.
\end{argument}

\noindent Ce qui importe, c’est de savoir quelle est la volonté de l’homme. Si elle est déréglée, ces mouvements seront déréglés, et si elle est droite, ils seront innocents et même louables. Car c’est la volonté qui est en tous ces mouvements, ou plutôt tous ces mouvements ne sont que des volontés. En effet, qu’est-ce que le désir et la joie, sinon une volonté qui consent à ce qui nous plaît ? et qu’est-ce que la crainte et la tristesse, sinon une volonté qui se détourne de ce qui nous déplaît ? Or, quand nous consentons à ce qui nous plaît en le souhaitant, ce mouvement s’appelle désir, et quand c’est en jouissant, il s’appelle joie. De même, quand nous nous détournons de l’objet qui nous déplaît avant qu’il nous arrive, cette volonté s’appelle crainte, et après qu’il est arrivé, tristesse. En un mot, la volonté de l’homme, selon les différents objets qui l’attirent ou qui la blessent, qu’elle désire ou qu’elle fuit, se change et se transforme en ces différentes affections. C’est pourquoi il faut que l’homme qui ne vit pas selon l’homme, mais selon Dieu, aime le bien, et alors il haïra nécessairement le mal ; or, comme personne n’est mauvais par nature, mais par vice, celui qui vit selon Dieu doit avoir pour les méchants une haine parfaite, en sorte qu’il ne haïsse pas l’homme à cause du vice, et qu’il n’aime pas le vice à cause de l’homme, mais qu’il haïsse le vice et aime l’homme. Le vice guéri, tout ce qu’il doit aimer restera, et il ne restera rien de ce qu’il doit haïr.
\subsection[{Chapitre VII}]{Chapitre VII}

\begin{argument}\noindent Les mots amour et dilection se prennent indifféremment en bonne et en mauvaise part dans les saintes Lettres.
\end{argument}

\noindent On dit de celui qui ale ferme propos d’aimer Dieu et d’aimer son prochain comme lui-même, non pas selon l’homme, mais selon Dieu, qu’il a une bonne volonté. Cette bonne volonté s’appelle ordinairement charité dans l’Écriture sainte, qui la nomme aussi quelquefois amour. En effet, l’Apôtre veut que celui dont on fait choix pour gouverner le peuple aime le bien ; et nous lisons aussi dans l’Évangile que Notre-Seigneur ayant dit à Pierre : « Me chéris-tu plus que ne font ceux-ci ? » Pierre répondit : « Seigneur, vous savez que je vous aime. » Et le Seigneur lui ayant demandé de nouveau, non pas s’il l’aimait, mais s’il le chérissait, Pierre lui répondit encore : « Seigneur, vous savez que je vous aime. » Enfin, le Seigneur lui ayant demandé une troisième fois s’il le chérissait, l’Évangéliste ajoute : « Pierre fut contristé de ce que le Seigneur lui avait dit trois fois : M’aimes-tu ? » Et cependant le Seigneur ne lui avait fait la question en ces termes qu’une seule fois, s’étant servi les deux autres fois du mot chérir. D’où je conclus que le Seigneur n’attachait pas au mot chérir ({\itshape diligere}) un autre sens qu’au mot aimer ({\itshape amare}). Aussi bien Pierre répond sans avoir égard à cette différence d’expressions : « Seigneur, vous savez tout ; vous savez donc bien que je vous aime. »\par
J’ai cru devoir m’arrêter sur ces deux mots, parce que plusieurs imaginent une différence entre dilection et charité ou amour. À leur avis, la dilection se prend en bonne part et l’amour en mauvaise part. Mais il est certain que les auteurs profanes n’ont jamais fait cette distinction, et je laisse aux philosophes le soin de résoudre le problème. Je remarquerai seulement que, dans leurs livres, ils ne manquent pas de relever l’amour qui a pour objet le bien et Dieu même. Quant à l’Écriture sainte, dont l’autorité surpasse infiniment celle de tous les monuments humains, nulle part elle n’insinue la moindre différence entre l’amour et la dilection ou charité. J’ai déjà prouvé que l’amour y est pris en bonne part ; et si l’on s’imagine que l’amour y est pris, à la vérité, en bonne et en mauvaise part, mais que la dilection s’y prend en bonne part exclusivement, il suffit, pour se convaincre du contraire, de se souvenir de ce passage du Psalmiste : « Celui qui chérit ({\itshape diligit}) l’iniquité hait son âme », et cet autre de l’apôtre saint Jean : « Celui qui chérit le monde ({\itshape si quis dilexerit}), la dilection du Père n’est pas en lui. » Voilà, dans un même passage, le mot {\itshape diligere} pris tour à tour en mauvaise et en bonne part. Et qu’on ne me demande pas si l’amour, que j’ai montré entendu en un sens favorable, peut aussi être pris dans le sens opposé ; car il est écrit « Les hommes deviendront amoureux d’eux-mêmes, amoureux de l’argent. »\par
La volonté droite est donc le bon amour, et la volonté déréglée est le mauvais, et les différents mouvements de cet amour font toutes les passions. S’il se porte vers quelque objet, c’est le désir ; s’il en jouit, c’est la joie ; s’il s’en détourne, c’est la crainte ; s’il le sent malgré lui, c’est la tristesse. Or, ces passions sont bonnes ou mauvaises, selon que l’amour est bon ou mauvais. Prouvons ceci par l’Écriture. L’Apôtre « désire de sortir de cette vie et d’être avec Jésus-Christ ». Écoutez maintenant le Prophète : « Mon âme languit dans le désir dont elle brûle sans cesse pour votre loi. » Et encore : « La concupiscence de la sagesse mène au royaume de Dieu. » L’usage toutefois a voulu que le mot concupiscence, employé isolément, fût pris en mauvaise part. Mais la joie est prise en bonne part dans ce passage du Psalmiste : « Réjouissez-vous dans le Seigneur ; justes, tressaillez de joie. » Et ailleurs : « Vous avez versé la joie dans mon cœur. » Et encore : « Vous me remplirez de joie en me dévoilant votre face. » Maintenant, ce qui prouve que la crainte est bonne, c’est ce mot de l’Apôtre : « Opérez votre salut avec crainte et frayeur. » Et cet autre passage : « Gardez-vous de viser plus haut qu’il ne convient, et craignez. » Et encore : « Je crains que, comme le serpent séduisit Ève, vous nevous écartiez de cet amour chaste qui est en Jésus-Christ. » Enfin, quant à la tristesseque Cicéron appelle une maladie et que Virgile assimile à la douleur en disant : « Et de là leurs douleurs et leurs joies », peut-elle se prendre aussi en bonne part ? c’est unequestion plus délicate.
\subsection[{Chapitre VIII}]{Chapitre VIII}

\begin{argument}\noindent Des trois seuls mouvements que les Stoïciens consentent à admettre dans l’âme du sage, à l’exclusion de la douleur ou de la tristesse, qu’ils croient incompatibles avec la vertu.
\end{argument}

\noindent Les Stoïciens substituent dans l’âme du sage aux perturbations trois mouvements de l’âme que la langue grecque appelle {\itshape eupathies}, et Cicéron {\itshape constantiae} : ils remplacent le désir par la volonté, la joie par le contentement, et la crainte par la précaution ; quant à la souffrance ou à la douleur, que nous avons de préférence appelée tristesse afin d’éviter toute ambiguïté, ils prétendent que rien de semblable ne peut se rencontrer dans l’âme du sage. La volonté, disent-ils, se porte vers le bien, qui est ce que fait le sage ; le contentement est la suite du bien accompli, et le sage accomplit toujours le bien ; enfin la précaution évite le mal, et le sage le doit constamment éviter ; mais la tristesse naissant du mal qui survient, comme il ne peut survenir aucun mal au sage, rien dans l’âme du sage ne peut tenir la place de la tristesse. Ainsi, dans leur langage, volonté, entendement, précaution, voilà qui n’appartient qu’au sage, et le désir, la joie, la crainte et la tristesse, sont le partage de l’insensé. Les trois premières affections sont ce que Cicéron appelle {\itshape constantiae}, les quatre autres, sont ce que le même philosophe appelle perturbations, et le langage ordinaire passions, et cette distinction des affections du sage et de celles du vulgaire est marquée en grec par les mots d’{\itshape eupatheiai} et de {\itshape pathè}. J’ai voulu examiner si ces manières de parler des Stoïciens étaient conformes à l’Écriture, et j’ai trouvé que le Prophète dit « qu’il n’y a pas de contentement d’esprit pour les impies » ; le propre des méchants étant plutôt de se réjouir du mal que d’être contents, ce qui n’appartient qu’aux gens de bien. J’ai aussi trouvé dans l’Évangile : « Faites aux hommes tout ce que vous voulez qu’ils vous fassent » ; comme si l’on ne pouvait vouloir que le bien, le mal étant l’objet des désirs, mais non celui de la volonté. Il est vrai que quelques versions portent : « Tout le bien que vous voulez qu’ils vous fassent », par où on a coupé court à toute interprétation mauvaise, de crainte par exemple que dans le désordre d’une orgie, quelque débauché ne se crût autorisé à l’égard d’autrui à une action honteuse sous la seule condition de la subir à son tour ; mais cette version n’est pas conforme à l’original grec, et j’en conclus qu’en disant : Tout ce que vous voulez qu’ils vous fassent, l’Apôtre a entendu {\itshape tout le bien}, car il ne dit pas : Que vous désirez qu’ils vous fassent, mais : Que vous {\itshape voulez}.\par
Au surplus, bien que ces sortes d’expressions soient les plus propres, il ne faut pas pour cela s’y assujettir ; il suffit de les prendre en cette acception dans les endroits de l’Écriture où elles n’en peuvent avoir d’autre, tels que ceux que je viens d’alléguer. Ne dit-on pas en effet que les impies sont transportés de joie, bien que le Seigneur ait dit : « Il n’y a pas de contentement pour les impies. » D’où vient cela, sinon de ce que contentement veut dire autre chose que joie, quand il est employé proprement et dans un sens étroit ? De même, il est clair que le précepte de l’Évangile, ainsi exprimé « Faites aux autres ce que vous désirez qu’ils vous fassent », n’impliquerait pas la défense de désirer des choses déshonnêtes, au lieu qu’exprimé de la sorte : « Faites aux autres ce que vous voulez qu’ils vous fassent », il est salutaire et vrai. Encore une fois, d’où vient cela, sinon de ce que la volonté, prise en un sens étroit, ne peut s’entendre qu’en bonne part ? Et cependant, il est certain que cette manière de parler ne serait point passée en usage : « Ne veuillez point mentir » ; s’il n’y avait aussi une mauvaise volonté, profondément distincte de celle que les anges ont recommandée par ces paroles : « Paix sur la terre aux hommes de bonne volonté. » Ce serait inutilement que l’Évangile ajouterait {\itshape bonne}, s’il n’y en avait aussi une mauvaise. D’ailleurs, quelle si grande louange l’Apôtre aurait-il donnée à la charité, lorsqu’il a dit « qu’elle ne prend point son contentement dans le mal », si la malignité ne l’y prenait ? Nous voyons aussi que les auteurs profanes se servent indifféremment de ces termes : « Je désire, Pères conscrits », dit le grand orateur Cicéron, « ne point sortir des voies de la douceur. » Il prend ici le désir en bonne part. Dans Térence, au contraire, le désir est pris en mauvaise part. Il introduit un jeune libertin qui, brûlant d’assouvir sa convoitise, s’écrie :\par
 {\itshape « Je ne veux rien que Philuména. »} \par
La preuve que cette volonté n’est qu’une ardeur brutale, c’est la réponse du vieux serviteur :\par
 {\itshape « Ah ! qu’il vaudrait mieux prendre soin d’éloigner cet amour de votre cœur que d’irriter inutilement votre passion par de pareils discours. »} \par
Quant au contentement, que les auteurs païens l’aient aussi employé en mauvaise part, Virgile seul suffit pour le prouver, dans ce vers si plein et si précis où il embrasse les quatre passions de l’âme :\par
 {\itshape « Et de là leurs craintes et leurs désirs, leurs douleurs et leurs contentements. »} \par
Le même poète dit encore :\par
 {\itshape « Les mauvais contentements de l’esprit. »} \par
C’est donc un trait commun des bons et des méchants de vouloir, de se tenir en garde et d’être contents, ou pour m’exprimer d’une autre sorte : Les bons et les méchants désirent, craignent et se réjouissent également, mais les uns bien, les autres mal, selon que leur volonté est bonne ou mauvaise. La tristesse même, à laquelle les Stoïciens n’ont pu rien substituer dans l’âme de leur sage, se prend aussi quelquefois en bonne part, surtout dans nos auteurs. L’Apôtre loue les Corinthiens de s’être attristés selon Dieu. Quelqu’un dira peut-être que cette tristesse dont saint Paul les félicite venait du repentir de leurs fautes ; car c’est en ces termes qu’il s’exprime : « Quoique ma lettre vous ait attristéspour un peu de temps, je ne laisse pas maintenant de me réjouir, non de ce que vous avez été tristes, mais de ce que votre tristesse vous a portés à faire pénitence. Votre tristesse a été selon Dieu, et ainsi vous n’avez pas sujet de vous plaindre de nous ; car la tristesse qui est selon Dieu produit un repentir salutaire dont on ne se repent point, au lieu que la tristesse du monde cause la mort. Et voyez déjà combien cette tristesse selon Dieu a excité votre vigilance. » À ce compte, les Stoïciens peuvent répondre que la tristesse est, à la vérité, utile pour se repentir, mais qu’elle ne peut pas tomber en l’âme du sage, parce qu’il est incapable de pécher pour se repentir ensuite et que nul autre mal ne peut l’attrister. On rapporte qu’Alcibiade, qui se croyait heureux, pleura, quand Socrate lui eut prouvé qu’il était misérable, parce qu’il était fou. La folie donc fut cause en lui de cette tristesse salutaire qui fait que l’homme s’afflige d’être autre qu’il ne devrait ; or, ce n’est pas au fou que les Stoïciens interdisent la tristesse, mais au sage.
\subsection[{Chapitre IX}]{Chapitre IX}

\begin{argument}\noindent Du bon usage que les gens de bien font des passions.
\end{argument}

\noindent Voilà ce que les Stoïciens peuvent dire ; mais nous avons déjà répondu là-dessus à ces philosophes au neuvième livre de cet ouvrage, Où nous avons montré que ce n’est qu’une question de nom-et qu’ils sont plus amoureux de la dispute que de la vérité. Parmi nous, selon la divine Écriture et la saine doctrine, les citoyens de la sainte Cité de Dieu qui vivent selon Dieu dans le pèlerinage de cette vie, craignent, désirent, s’affligent et se réjouissent ; et comme leur amour est pur, toutes ces passions sont en eux innocentes. Ils craignent les supplices éternels et désirent l’immortalité bienheureuse. Ils s’affligent, parce qu’ils soupirent encore intérieurement dans l’attente de l’adoption divine, qui aura lieu lorsqu’ils seront délivrés de leurs corps. Ils se réjouissent en espérance, parce que cette parole s’accomplira, qui annonce que « la mort sera absorbée dans la victoire ». Bien plus, ils craignent de fléchir ; ils désirent de persévérer ; ils s’affligent de leurs péchés ; ils se réjouissent de leurs bonnes œuvres. Ils craignent de pécher, parce qu’ils entendent que « la charité se refroidira en plusieurs, quand ils verront le vice triompher ». Ils désirent de persévérer, parce qu’il est écrit « qu’il n’y aura de sauvé que celui qui persévérera jusqu’à la fin ». Ils s’affligent de leurs péchés, parce qu’il est dit : « Si nous nous prétendons exempts de tout péché, nous nous abusons nous-mêmes, et la vérité n’est point en nous ». Ils se réjouissent de leurs bonnes œuvres, parce que saint Paul leur dit : « Dieu aime celui qui donne avec joie. » D’ailleurs, selon qu’ils sont faibles ou forts, ils craignent ou désirent d’être tentés, et s’affligent ou se réjouissent de leurs tentations. Ils craignent d’être tentés, à cause de cette parole : « Si quelqu’un tombe par surprise en quelque péché, vous autres qui êtes spirituels, ayez soin de l’en reprendre avec douceur, dans la crainte d’être tentés comme lui ». Ils désirent d’être tentés, parce qu’ils entendent un homme fort de la Cité de Dieu, qui dit : « Éprouvez-moi, Seigneur, et me tentez, brûlez mes reins et mon cœur. » Ils s’effrayent dans les tentations, parce qu’ils voient saint Pierre pleurer. Ils se réjouissent dans les tentations, parce qu’ils entendent cette parole de saint Jacques : « N’ayez jamais plus de joie, mes frères, que lorsque vous êtes attaqués de plusieurs tentations ? »\par
Or, ils n’e sont pas seulement touchés de ces mouvements pour eux-mêmes, mais aussi pour ceux dont ils désirent la délivrance et craignent la perte, et dont la perte ou la délivrance les afflige ou les réjouit. Pour ne parler maintenant que de ce grand homme qui se glorifie de ses infirmités, de ce docteur des nations qui a plus travaillé que tous les autres Apôtres et qui a instruit ceux de son temps et toute la postérité par ses admirables Épîtres, du bienheureux saint Paul, de ce brave athlète de Jésus-Christ, formé par lui, oint par lui, crucifié avec lui, glorieux en lui, combattant vaillamment sur le théâtre de ce monde à la vue des anges et des hommes, et s’avançant à grands pas dans la carrière pour remporter le prix de la lutte, qui ne serait ravi de le contempler des yeux de la foi, se réjouissant avec ceux qui se réjouissent, pleurant avec ceux qui pleurent, ayant à soutenir des combats au dehors et des frayeurs au dedans, souhaitant de mourir et d’être avec Jésus-Christ, désirant de voir les Romains, pour, amasser du fruit parmi eux, comme il avait fait parmi les autres nations, ayant pour les Corinthiens une sainte jalousie qui lui fait appréhender qu’ils ne se laissent séduire et qu’ils ne s’écartent de l’amour chaste qu’ils avaient pour Jésus-Christ, touché pour les Juifs d’une tristesse profonde et d’une douleur continuelle qui le pénètre jusqu’au cœur, de ce qu’ignorant la justice dont Dieu est auteur, et voulant établir leur propre justice, ils n’étaient point soumis à Dieu, saisi enfin d’une profonde tristesse au point d’éclater en gémissements et en plaintes au sujet de quelques-uns qui, après être tombés dans de grands désordres, n’en faisaient point pénitence ?\par
Si l’on doit appeler vices ces mouvements qui naissent de l’amour de la vertu et de la charité, il ne reste plus que d’appeler vertus les affections qui sont réellement des vices. Mais puisque ces mouvements suivent la droite raison, étant dirigés où il faut, qui oserait alors les appeler des maladies de l’âme ou des passions vicieuses ? Aussi Notre-Seigneur, qui a daigné vivre ici-bas revêtu de la forme d’esclave, mais sans aucun péché, a fait usage des affections, lorsqu’il a cru le devoir faire. Comme il avait véritablement un corps et une âme, il avait aussi de véritables passions. Lors donc qu’il fut touché d’une tristesse mêlée d’indignation, en voyant l’endurcissement des Juifs, et que, dans une-autre occasion, il dit : « Je me réjouis pour l’amour de vous de ce que je n’étais pas là, afin que vous croyiez » ; quand, avant de ressusciter Lazare, il pleura, quand il désira ardemment de manger la pâque avec ses disciples, quand enfin son âme fut triste jusqu’à la mort aux approches de sa passion nous ne devons point douter que toutes ces choses ne se soient effectivement passées en lui. Il s’est revêtu de ces passions quand il lui a plu pour l’accomplissement de ses desseins, comme il s’est fait homme quand il a voulu.\par
Mais quelque bon usage qu’on puisse faire des passions, il n’en faut pas moins reconnaître que nous ne les éprouverons point dans l’autre vie, et qu’en celle-ci elles nous emportent souvent plus loin que nous ne voudrions ; ce qui fait que nous pleurons même quelquefois malgré nous, dans une effusion d’ailleurs innocente et toute de charité. C’est en nous une suite de notre condition faible et mortelle ; mais il n’en était pas ainsi de Notre-Seigneur Jésus-Christ, qui était maître de toutes ces faiblesses. Tant que nous sommes dans ce corps fragile, ce serait un défaut d’être exempt de toute passion ; car l’Apôtre blâme et déteste certaines personnes qu’il accuse d’être sans amitié. Le Psalmiste de même condamne ceux dont il dit : « J’ai attendu quelqu’un qui prendrait part à mon affliction, et personne n’est venu. » En effet, n’avoir aucun sentiment de douleur, tandis que nous sommes dans ce lieu de misère, c’est, comme le disait un écrivain profane, un état que nous ne saurions acheter qu’au prix d’une merveilleuse stupidité. Voilà pourquoi ce que les Grecs appellent apathie, mot qui ne pourrait se traduire que par impassibilité, c’est-à-dire cet état de l’âme dans lequel elle n’est sujette à aucune passion qui la trouble et qui soit contraire à la raison, est assurément une bonne chose et très souhaitable, mais qui n’est pas de cette vie. Écoutez, en effet, non pas un homme vulgaire, mais un des plus saints et des plus parfaits, qui a dit : « Si nous nous prétendons exempts de tout péché, nous nous abusons nous-mêmes, et la vérité n’est point en nous. » Cette apathie n’existera donc en vérité que quand l’homme sera affranchi de tout péché. Il suffit maintenant de vivre sans crime, et quiconque croit vivre sans péché éloigne de lui moins le péché que le pardon. Si donc l’apathie consiste à n’être touché de rien, qui ne voit que cette insensibilité est pire que tous les vices ? On peut fort bien dire, il est vrai, que la parfaite béatitude dont nous espérons jouir en l’autre vie sera exempte de crainte et de tristesse ; mais qui peut soutenir avec quelque ombre de raison que l’amour et la joie en seront bannis ? Si par cette apathie on entend un état entièrement exempt de crainte et de douleur, il faut fuir cet état en cette vie, si nous voulons bienvivre, c’est-à-dire vivre selon Dieu ; mais pour l’autre, où l’on nous promet une félicité éternelle, la crainte n’y entrera pas.\par
Cette crainte, en effet, dont saint Jean dit : « La crainte ne se trouve point avec la charité ; car la charité parfaite bannit la crainte, parce que la crainte est pénible » ; cette crainte, dis-je, n’est pas du genre de celle qui faisait redouter à saint Paul que les Corinthiens ne se laissassent surprendre aux artifices du serpent, attendu que la charité est susceptible de cette crainte, ou, pour mieux dire, il n’y a que la charité qui en soit capable ; mais elle est du genre de celle dont parle ce même Apôtre quand il dit : « Vous n’avez point reçu l’esprit de servitude pour vivre encore dans la crainte. » Quant à cette crainte chaste « qui demeure dans le siècle du siècle », si elle demeure dans le siècle à venir (et comment entendre autrement le siècle du siècle ?), ce ne sera pas une crainte qui nous donne appréhension du mal, mais une crainte qui nous affermira dans un bien que nous ne pourrons perdre. Lorsque l’amour du bien acquis est immuable, on est en quelque sorte assuré contre l’appréhension de tout mal. En effet, cette crainte chaste dont parle le Prophète signifie cette volonté par laquelle nous répugnerons nécessairement au péché, en sorte que nous éviterons le péché avec cette tranquillité qui accompagne un amour parfait, et non avec les inquiétudes qui sont maintenant des suites de notre infirmité. Que si toute sorte de crainte est incompatible avec cet état heureux où nous serons entièrement assurés de notre bonheur, il faut entendre cette parole de l’Écriture : « La crainte chaste du Seigneur qui demeure dans le siècle du siècle », au même sens que celle-ci : « La patience des pauvres ne périra jamais » ; non que la patience doive être réellement éternelle, puisqu’elle n’est nécessaire qu’où il y a des maux à souffrir, mais le bien qu’on acquiert par la patience sera éternel, au même sens peut-être où l’Écriture dit que la crainte chaste demeurera dans le siècle du siècle, parce que la récompense en sera éternelle.\par
Ainsi, puisqu’il faut mener une bonne vie pour arriver à la vie bienheureuse, concluons que toutes les affections sont bonnes en ceuxqui vivent bien, et mauvaises dans les autres. Mais dans cette vie bienheureuse et éternelle, l’amour et la joie ne seront pas seulement bons, mais assurés, et il n’y aura ni crainte ni douleur. Par là, on voit déjà en quelque façon quels doivent être dans ce pèlerinage les membres de la Cité de Dieu qui vivent selon l’esprit et non selon la chair, c’est-à-dire selon Dieu et non selon l’homme, et quels ils seront un jour dans cette immortalité à laquelle ils aspirent. Mais pour ceux de l’autre Cité, c’est-à-dire pour la société des impies qui ne vivent pas selon Dieu, mais selon l’homme, et qui embrassent la doctrine des hommes et des démons dans le culte d’une fausse divinité et dans le mépris de la véritable, ils sont tourmentés de ces passions comme d’autant de maladies, et si quelques-uns semblent les modérer, on les voit enflés d’un orgueil impie, d’autant plus monstrueux qu’ils en ont moins le sentiment. En se haussant jusqu’à cet excès de vanité de n’être touchés d’aucune passion, non pas même de celle de la gloire, ils ont plutôt perdu toute humanité qu’ils n’ont acquis une tranquillité véritable. Une âme n’est pas droite pour être inflexible, et l’insensibilité n’est pas la santé.
\subsection[{Chapitre X}]{Chapitre X}

\begin{argument}\noindent Si les premiers hommes avant le péché étaient exempts de toute passion.
\end{argument}

\noindent On a raison de demander si nos premiers parents, avant le péché, étaient sujets dans le corps animal à ces passions dont ils seront un jour affranchis dans le corps spirituel. En effet, s’ils les avaient, comment étaient-ils bienheureux ? La béatitude peut-elle s’allier avec la crainte ou la douleur ? Mais, d’un autre côté, que pouvaient-ils craindre ou souffrir au milieu de tant de biens, dans cet état où ils n’avaient à redouter ni la mort ni les maladies, où leurs justes désirs étaient pleinement comblés et où rien ne les troublait dans la jouissance d’une si parfaite félicité ? l’amour mutuel de ces époux, aussi bien que celui qu’ils portaient à Dieu, était libre de toute traverse, et de cet amour naissait une joie admirable, parce qu’ils possédaient toujours ce qu’ils aimaient. Ils évitaient le péché sans peine et sans inquiétude, et ils n’avaient point d’autre mal à craindre. Dirons-nous qu’ils désiraient de manger du fruit défendu, mais qu’ils craignaient de mourir, et qu’ainsi ils étaient agités de crainte et de désirs ? Dieu nous garde d’avoir cette pensée ! car la nature humaine était encore alors exempte de péché. Or, n’est-ce pas déjà un péché de désirer ce qui est défendu par la loi de Dieu, et de s’en abstenir par la crainte de la peine et non par l’amour de la justice ? Loin de nous donc l’idée qu’ils fussent coupables dès lors à l’égard du fruit détendu de cette sorte de péché dont Notre-Seigneur dit à l’égard d’une femme : « Quiconque regarde une femme pour la convoiter, a déjà commis l’adultère dans son cœur. » Tous les hommes seraient maintenant aussi heureux que nos premiers parents et vivraient sans être troublés dans leur âme par aucune passion, ni affligés dans leur corps par aucune incommodité, si le péché n’eût point été commis par Adam et Ève, qui ont légué leur corruption à leurs descendants, et cette félicité aurait duré jusqu’à ce que le nombre des prédestinés eût été accompli, en vertu de cette bénédiction de Dieu : « Croissez et multipliez » ; après quoi ils seraient passés sans mourir dans cette félicité dont nous espérons jouir après la mort et qui doit nous égaler aux anges.
\subsection[{Chapitre XI}]{Chapitre XI}

\begin{argument}\noindent De la chute du premier homme, en qui la nature a été créée bonne et ne peut être réparée que par son auteur.
\end{argument}

\noindent Dieu, qui prévoit tout, n’ayant pu ignorer que l’homme pécherait, il convient que nous considérions la sainte Cité selon l’ordre de la prescience de Dieu, et non selon les conjectures de notre raison imparfaite à qui échappent les plans divins. L’homme n’a pu troubler par son péché les desseins éternels de Dieu et l’obliger à changer de résolution, qui que Dieu avait prévu à quel point l’homme qu’il a créé bon devait devenir méchant et quel bien il devait tirer de sa malice. En effet, quoique l’on dise que Dieu change ses conseils (d’où vient que, par une expression figurée, on lit dans l’Écriture qu’il s’est repenti), cela ne doit s’entendre que par rapport à ce que l’homme attendait ou à l’ordre des causes naturelles, et non par rapport à la prescience de Dieu. Dieu, comme parle l’Écriture, a créé l’homme droit, et par conséquent avec unebonne volonté ; autrement il n’aurait pas été droit. La bonne volonté est donc l’ouvrage de Dieu, puisque l’homme l’a reçue dès l’instant de sa création. Quant à la première mauvaise volonté, elle a précédé dans l’homme toutes les mauvaises œuvres ; elle a plutôt été en lui une défaillance et un abandon de l’ouvrage de Dieu, pour se porter vers ses propres ouvrages, qu’aucune œuvre positive. Si ces ouvrages de la volonté ont été mauvais, c’est qu’ils n’ont pas eu Dieu pour fin, mais la volonté elle-même ; en sorte que c’est cette volonté ou l’homme en tant qu’ayant une mauvaise volonté, qui a été comme le mauvais arbre qui a produit ces mauvais fruits. Or, bien que la mauvaise volonté, loin d’être selon la nature, lui soit contraire, parce qu’elle est un vice, il n’en est pas moins vrai que, comme tout vice, elle ne peut être que dans une nature, mais dans une nature que le Créateur a tirée du néant, et non dans celle qu’il a engendrée de lui-même, telle qu’est le Verbe, par qui toutes choses ont été faites. Dieu a formé l’homme de la poussière de la terre, mais la terre elle-même a été créée de rien, aussi bien que l’âme de l’homme. Or, le mal est tellement surmonté par le bien, qu’encore que Dieu permette qu’il y en ait, afin de faire voir comment sa justice en peut bien user, ce bien néanmoins peut être sans le mal, comme en Dieu, qui est le souverain bien, et dans toutes les créatures célestes et invisibles qui font leur demeure au-dessus de cet air ténébreux, au lieu que le mal ne saurait subsister sans le bien, parce que les natures en qui il est sont bonnes comme natures. Aussi l’on ôte le mal, non en ôtant quelque nature étrangère, ou quelqu’une de ses parties, mais en guérissant celle qui était corrompue. Le libre arbitre est donc vraiment libre quand il n’est point esclave du péché. Dieu l’avait donné tel à l’homme ; et maintenant qu’il l’a perdu par sa faute, il n’y a que celui qui le lui avait donné qui puisse le lui rendre. C’est pourquoi la Vérité dit : « Si le Fils vous met en liberté, c’est alors que vous serez vraiment libres » ; ce qui revient à ceci : Si le Fils vous sauve, c’est alors que vous serez vraiment sauvés. En effet, le Christ n’est notre libérateur que par cela même qu’il est notre sauveur.\par
L’homme vivait donc selon Dieu dans le paradis à la fois corporel et spirituel. Car il n’y avait pas un paradis corporel pour les biens du corps, sans un paradis spirituel pour ceux de l’esprit ; et, d’un autre côté, un paradis spirituel, source de jouissances intérieures, ne pouvait être sans un paradis corporel, source de jouissances extérieures. Il y avait donc, pour ce double objet, un double paradis. Mais cet ange superbe et envieux (dont j’ai raconté la chute aux livres précédents, aussi bien que celle des autres anges devenus ses compagnons), ce prince des démons qui s’éloigne de son Créateur pour se tourner vers lui-même, et s’érige en tyran plutôt que de rester sujet, ayant été jaloux du bonheur de l’homme, choisit le serpent, animal fin et rusé, comme l’instrument le plus propre à l’exécution de son dessein, et s’en servit pour parler à la femme, c’est-à-dire à la partie la plus faible du premier couple humain, afin d’arriver au tout par degrés, parce qu’il ne croyait pas l’homme aussi crédule, ni capable de se laisser abuser, si ce n’est par complaisance pour l’erreur d’un autre. De même qu’Aaron ne se porta pas à fabriquer une idole aux Hébreux de son propre mouvement, mais parce qu’il y fut forcé par leurs instances, de même encore qu’il n’est pas croyable que Salomon ait cru qu’il fallait adorer des simulacres, mais qu’il fut entraîné à ce culte sacrilège par les caresses de ses concubines, ainsi n’y a-t-il pas d’apparence que le premier homme ait violé la loi de Dieu pour avoir été trompé par sa femme, mais pour n’avoir pu résister à l’amour qu’il lui portait. Si l’Apôtre a dit : « Adam n’a point été séduit, mais bien la femme » ; ce n’est que parce que la femme ajouta foi aux paroles du serpent et que l’homme ne voulut pas se séparer d’elle, même quand il s’agissait de mal faire. Il n’en est pas toutefois moins coupable, attendu qu’il n’a péché qu’avec connaissance. Aussi saint Paul ne dit pas : Il n’a point péché, mais : Il n’a point été séduit. L’Apôtre témoigne bien au contraire qu’Adam a péché, quand il dit : « Le péché est entré dans le monde par un seul homme » ; et peu après, encore plus clairement : « À la ressemblance de la prévarication d’Adam. » Il entend donc que ceux-là sont séduits qui ne croientpas mal faire ; or, Adam savait fort bien qu’il faisait mal ; autrement, comment serait-il vrai qu’il n’a pas été séduit ? Mais n’ayant pas encore fait l’épreuve de la sévérité de la justice de Dieu, il a pu se tromper en jugeant sa faute vénielle. Ainsi il n’a pas été séduit, puisqu’il n’a pas cru ce que crut sa femme, mais il s’est trompé en se persuadant que Dieu se contenterait de cette excuse qu’il lui allégua ensuite : « La femme que vous m’avez donnée pour compagne m’a présenté du fruit et j’en ai mangé. » Qu’est-il besoin d’en dire davantage ? Il est vrai qu’ils n’ont pas tous deux été crédules, mais ils ont été tous deux pécheurs et sont tombés tous deux dans les filets du diable.
\subsection[{Chapitre XII}]{Chapitre XII}

\begin{argument}\noindent Grandeur du péché du premier homme.
\end{argument}

\noindent Si quelqu’un s’étonne que la nature humaine ne soit pas changée par les autres péchés, comme elle l’a été par celui qui est la cause originelle de cette grande corruption à laquelle elle est sujette, de la mort et de tant d’autres misères dont l’homme était exempt dans le paradis terrestre, je répondrai qu’on ne doit pas juger de la grandeur de ce péché par sa matière (car le fruit défendu n’avait rien de mauvais en soi), mais par la gravité de la désobéissance. En effet, Dieu, dans le commandement qu’il fit à l’homme, ne considérait que son obéissance, vertu qui est la mère et la gardienne de toutes les autres, puisque la créature raisonnable a été ainsi faite que rien ne lui est plus utile que d’être soumise à son Créateur, ni rien de plus pernicieux que de faire sa propre volonté. Et puis, ce commandement était si court à retenir et si facile à observer au milieu d’une si grande abondance d’autres fruits dont l’homme était libre de se nourrir ! Il a été d’autant plus coupable de le violer qu’il lui était plus aisé d’être docile, à une époque surtout où le désir ne combattait pas encore sa volonté innocente, ce qui n’est arrivé depuis qu’en punition de son péché.
\subsection[{Chapitre XIII}]{Chapitre XIII}

\begin{argument}\noindent Le péché d’Adam a été précédé d’une mauvaise volonté.
\end{argument}

\noindent Mais nos premiers parents étaient déjàcorrompus au dedans avant que de tomber au dehors dans cette désobéissance ; car une mauvaise action est toujours précédée d’une mauvaise volonté. Or, qui a pu donner commencement à cette mauvaise volonté, sinon l’orgueil, puisque, selon l’Écriture, tout péché commence par là ? Et qu’est-ce que l’orgueil, sinon le désir d’une fausse grandeur ? Grandeur bien fausse, en effet, que d’abandonner celui à qui l’âme doit être attachée comme à son principe pour devenir en quelque sorte son principe à soi-même ! C’est ce qui arrive à quiconque se plaît trop en sa propre beauté, en quittant cette beauté souveraine et immuable qui devait faire l’unique objet de ses complaisances. Ce mouvement de l’âme qui se détache de son Dieu est volontaire, puisque si la volonté des premiers hommes fût demeurée stable dans l’amour de ce souverain bien qui l’éclairait de sa lumière et l’échauffait de son ardeur, elle ne s’en serait pas détournée pour se plaire en elle-même, c’est-à-dire pour tomber dans la froideur et dans les ténèbres, et la femme n’aurait pas cru le serpent, ni l’homme préféré la volonté de sa femme au commandement de Dieu, sous le prétexte illusoire de ne commettre qu’un péché véniel. Ils étaient donc méchants avant que de transgresser le commandement. Ce mauvais fruit ne pouvait venir que d’un mauvais arbre, et cet arbre ne pouvait devenir mauvais que par un principe contraire à la nature, c’est-à-dire par le vice de la mauvaise volonté. Or, la nature ne pourrait être corrompue par le vice, si elle n’avait été tirée du néant ; en tant qu’elle est comme nature, elle témoigne qu’elle a Dieu pour auteur ; en tant qu’elle se détache de Dieu, elle témoigne qu’elle est faite de rien. L’homme néanmoins, en se détachant de Dieu, n’est pas retombé dans le néant, mais il s’est tourné vers lui-même, et a commencé dès lors à avoir moins d’être que lorsqu’il était attaché à l’Être souverain. Être dans soi-même, ou, en d’autres termes, s’y complaire après avoir abandonné Dieu, ce n’est pas encore être un néant, mais c’est approcher du néant. De là vient que l’Écriture sainte appelle superbes ceux qui se plaisent où eux-mêmes. Il est bon d’avoir le cœur élevé en haut, non pas cependant vers soi-même, ce qui tient de l’orgueil, mais vers Dieu, ce qui est l’effet d’une obéissance dontles humbles sont seuls capables. Il y a donc quelque chose dans l’humilité qui élève le cœur en haut et quelque chose dans l’orgueil qui le porte en bas. On a quelque peine à entendre d’abord que ce qui s’abaisse tende en haut, et que ce qui s’élève aille en bas ; mais c’est que notre humilité envers Dieu nous unit à celui qui ne voit rien de plus élevé que lui, et par conséquent nous élève, tandis que l’orgueil qui refuse de s’assujettir à lui se détache et tombe. Alors s’accomplit cette parole du Prophète : « Vous les avez abattus lorsqu’ils s’élevaient. » Il ne dit pas : Lorsqu’ils s’étaient élevés, comme si leur chute avait suivi leur élévation, mais : Ils ont été abattus, dit-il, lorsqu’ils s’élevaient, parce que s’élever de la sorte, c’est tomber. Aussi est-ce, d’une part, l’humilité, si fort recommandée en ce monde à la Cité de Dieu et si bien pratiquée par Jésus-Christ, son roi, et, de l’autre, l’orgueil, apanage de l’ennemi de cette Cité sainte, selon le témoignage de l’Écriture, qui mettent cette grande différence entre les deux Cités dont nous parlons, composées, l’une de l’assemblée des bons, et l’autre de celle des méchants, chacune avec les anges de son parti, que l’amour-propre et l’amour de Dieu ont distingués dès le commencement.\par
Le diable n’aurait donc pas pris l’homme dans ses pièges, si l’homme ne s’était plu auparavant en lui-même. Il se laissa charmer par cette parole : « Vous serez comme des dieux » ; mais ils l’auraient bien mieux été en se tenant unis par l’obéissance à leur véritable et souverain principe qu’en voulant par l’orgueil devenir eux-mêmes leur principe. En effet, les dieux créés ne sont pas dieux par leur propre vertu, mais par leur union avec le véritable Dieu. Quand l’homme désire d’être plus qu’il ne doit, il devient moins qu’il n’était, et, en croyant se suffire à lui-même, il perd celui qui lui pourrait suffire réellement. Ce désordre qui fait que l’homme, pour se trop plaire en lui-même, comme s’il était lui-même lumière, se sépare de cette lumière qui le rendrait lumière, lui aussi, s’il savait se plaire en elle, ce désordre, dis-je, était déjà dans le cœur de l’homme avant qu’il passât à l’action qui lui avait été défendue. Car la Vérité a dit : « Le cœur s’élève avant la chute et s’humilie avant la gloire » ; c’est-à-dire que la chute qui sefait dans le cœur précède celle qui arrive au dehors, la seule qu’on veuille reconnaître. Car qui s’imaginerait que l’élévation fût une chute ? Et cependant, celui-là est déjà tombé qui s’est séparé du Très-Haut. Qui ne voit au contraire qu’il y a chute, quand il y a violation manifeste et certaine du commandement ? J’ose dire qu’il est utile aux superbes de tomber en quelque péché évident et manifeste, afin que ceux qui étaient déjà tombés par la complaisance qu’ils avaient en eux commencent à se déplaire à eux-mêmes. Les larmes et le déplaisir de saint Pierre lui furent plus salutaires que la fausse complaisance de sa présomption. C’est ce que le Psalmiste dit aussi quelque part : « Couvrez-les de honte, Seigneur, et ils chercheront votre nom » ; en d’autres termes : « Ceux qui s’étaient plu dans la recherche de leur gloire se plairont à rechercher la vôtre. »
\subsection[{Chapitre XIV}]{Chapitre XIV}

\begin{argument}\noindent L’orgueil de la transgression dans le péché originel a été pire que la transgression elle-même.
\end{argument}

\noindent Mais l’orgueil le plus condamnable est de vouloir excuser les péchés manifestes, comme fit Ève, quand elle dit : « Le serpent m’a trompée, et j’ai mangé du fruit de l’arbre » ; et Adam, quand il répondit : « La femme que vous m’avez donnée m’a donné du fruit de l’arbre, et j’en ai mangé. » On ne voit point qu’ils demandent pardon de leur crime, ni qu’ils en implorent le remède. Quoiqu’ils ne le désavouent pas, à l’exemple de Caïn, leur orgueil, néanmoins, tâche de le rejeter sur un autre, la femme sur le serpent, et l’homme sur la femme. Mais quand le péché est manifeste, c’est s’accuser que de s’excuser. En effet, l’avaient-ils moins commis pour avoir agi, la femme sur les conseils du serpent, et l’homme sur les instances de la femme ? comme s’il y avait quelqu’un à qui l’on dût plutôt croire ou céder qu’à Dieu !
\subsection[{Chapitre XV}]{Chapitre XV}

\begin{argument}\noindent La peine du premier péché est très juste.
\end{argument}

\noindent Lors donc que l’homme eût méprisé le commandement de Dieu, de ce Dieu quil’avait créé, fait à son image, établi sur les autres animaux, placé dans le paradis, comblé de tous les biens, et qui, loin de le charger d’un grand nombre de préceptes fâcheux, ne lui en avait donné qu’un très facile, pour lui recommander l’obéissance et le faire souvenir qu’il était son Seigneur et que la véritable liberté consiste à servir Dieu, ce fut avec justice que l’homme tomba dans la damnation, et dans une damnation telle que son esprit devint charnel, lui dont le corps même devait devenir spirituel, s’il n’eût point péché ; et comme il s’était plu en lui-même par son orgueil, la justice de Dieu l’abandonna à lui-même, non pour vivre dans l’indépendance qu’il affectait, mais pour être esclave de celui à qui il s’était joint en péchant, pour souffrir malgré lui la mort du corps, comme il s’était volontairement procuré celle de l’âme, et pour être même condamné à la mort éternelle (si Dieu ne l’en délivrait par sa grâce), en puni-lion d’avoir abandonné la vie éternelle. Quiconque estime cette condamnation ou trop grande ou trop injuste ne sait certainement pas peser la malice d’un péché qui était si facile à éviter. De même que l’obéissance d’Abraham a été d’autant plus grande que le commandement que Dieu lui avait fait était plus difficile, ainsi la désobéissance du premier homme a été d’autant plus criminelle qu’il n’y avait aucune difficulté à faire ce qui lui avait été commandé ; et comme l’obéissance du second Adam est d’autant plus louable qu’il a été obéissant jusqu’à la mort, la désobéissance du premier est d’autant plus détestable qu’il a été désobéissant jusqu’à la mort. Ce que le Créateur commandait étant si peu considérable et la peine de la désobéissance si grande, qui peut mesurer la faute d’avoir manqué à faire une chose si aisée et de n’avoir point redouté un si grand supplice ?\par
Enfin, pour le dire en un mot, quelle a été la peine de la désobéissance, sinon la désobéissance même ? En quoi consiste au fond la misère de l’homme, si ce n’est dans une révolte de soi contre soi, en sorte que, comme il n’a pas voulu ce qu’il pouvait, il veut maintenant ce qu’il ne peut ? En effet, bien que dans le paradis il ne fût pas tout-puissant, il ne voulait que ce qu’il pouvait, et ainsi ilpouvait tout ce qu’il voulait ; mais maintenant, comme dit l’Écriture, l’homme n’est que vanité. Qui pourrait compter combien il veut de choses qu’il ne peut, tandis que sa volonté est contraire à elle-même et que sa chair ne lui veut pas obéir ? Ne voyons-nous pas qu’il se trouble souvent malgré lui, qu’il souffre malgré lui, qu’il vieillit malgré lui, qu’il meurt malgré lui ? Combien endurons-nous de choses que nous n’endurerions pas, si notre nature obéissait en tout à notre volonté ? Mais, dit-on, c’est que notre chair est sujette à certaines infirmités qui l’empêchent de nous obéir. Qu’importe la raison pour laquelle notre chair, qui nous était soumise, nous cause de la peine en refusant de nous obéir, puisqu’il est toujours certain que c’est un effet de la juste vengeance de Dieu, à qui nous n’avons pas voulu nous-mêmes être soumis, ce qui du reste n’a pu lui causer aucune peine ? Car il n’a pas besoin de notre service comme nous avons besoin de celui de notre corps, et ainsi notre péché n’a fait tort qu’à nous. Pour les douleurs qu’on nomme corporelles, c’est l’âme qui les souffre dans le corps et par son moyen. Et que peut souffrir ou désirer par elle-même une chair sans âme ? Quand on dit que la chair souffre ou désire, l’on entend par là ou l’homme entier, comme nous l’avons montré ci-dessus, ou quelque partie de l’âme que la chair affecte d’impressions fâcheuses ou agréables qui produisent en elle un sentiment de douleur onde volupté. Ainsi la douleur du corps n’est autre chose qu’un chagrin de l’âme à cause du corps et la répulsion qu’elle oppose à ce qui se fait dans le corps, comme la douleur de l’âme qu’on nomme tristesse est la répulsion qu’elle oppose aux choses qui arrivent contre son gré. Mais la tristesse est ordinairement précédée de la crainte, qui est aussi dans l’âme et non dans la chair, au lieu que la douleur de la chair n’est précédée d’aucune crainte de la chair qui se sente dans la chair avant la douleur. Pour la volupté, elle est précédée dans la chair même d’un certain aiguillon, comme la faim, la soif et ce libertinage des parties de la génération que l’on nomme convoitise aussi bien que toutes les autres passions. Les anciens ont défini la colère même une convoitise de la vengeance, quoique parfois un homme sefâche contre des objets qui ne sont pas capables de ressentir sa vengeance, comme quand il rompt en colère une plume qui ne vaut rien. Mais bien que ce désir de vengeance soit plus déraisonnable que les autres, il ne laisse pas d’être une convoitise et d’être même fondé sur quelque ombre de cette justice qui veut que ceux qui font le mal souffrent à leur tour. Il y a donc une convoitise de vengeance qu’on appelle colère ; il y a une convoitise d’amasser qu’on nomme avarice ; il y a une convoitise de vaincre qu’on appelle opiniâtreté ; et il y a une convoitise de se glorifier qu’on appelle vanité. Il y en a encore bien d’autres, soit qu’elles aient un nom, soit qu’elles n’en aient point ; car quel nom donner à la convoitise de dominer, qui néanmoins est si forte dans l’âme des tyrans, comme les guerres civiles le font assez voir ?
\subsection[{Chapitre XVI}]{Chapitre XVI}

\begin{argument}\noindent Du danger du mal de la convoitise, à n’entendre ce mot que des mouvements impurs du corps.
\end{argument}

\noindent Bien qu’il y ait plusieurs espèces de convoitises, ce mot, quand on ne le détermine pas, ne fait guère penser à autre chose qu’à ce désir particulier qui excite les parties honteuses de la chair. Or, cette passion est si forte qu’elle ne s’empare pas seulement du corps tout entier, au dehors et au dedans, mais qu’elle émeut tout l’homme en unissant et mêlant ensemble l’ardeur de l’âme et l’appétit charnel, de sorte qu’au moment où cette volupté, la plus grande de toutes entre celles du corps, arrive à son comble, l’âme enivrée en perd la raison et s’endort dans l’oubli d’elle-même. Quel est l’ami de la sagesse et des joies innocentes qui, engagé dans le mariage, mais sachant, comme dit l’Apôtre, « conserver le vase de son corps saint et pur, au lieu de s’abandonner à la maladie des désirs déréglés, à l’exemple des païens qui ne connaissent point Dieu », quel est le chrétien, dis-je, qui ne voudrait, s’il était possible, engendrer des enfants sans cette sorte de volupté, de telle façon que les membres destinés à la génération fussent soumis, comme les autres, à l’empire de la volonté plutôt qu’emportés par le torrent impétueux de la convoitise ? Aussi bien, ceux mêmes qui recherchent avec ardeur cettevolupté, soit dans l’union légitime du mariage, soit dans les commerces honteux de l’impureté, ne ressentent pas à leur gré l’émotion charnelle. Tantôt ces mouvements les importunent malgré eux et tantôt ils les abandonnent dans le transport même de la passion ; l’âme est tout en feu et le corps reste glacé. Ainsi, chose étrange ! ce n’est pas seulement aux désirs légitimes du mariage, mais encore aux désirs déréglés de la concupiscence, que la concupiscence elle-même refuse d’obéir. Elle, qui d’ordinaire résiste de tout son pouvoir à l’esprit qui fait effort pour l’arrêter, d’autres fois, elle se divise contre soi et se trahit soi-même en remuant l’âme sans émouvoir le corps.
\subsection[{Chapitre XVII}]{Chapitre XVII}

\begin{argument}\noindent Comment Adam et Ève connurent qu’ils étaient nus.
\end{argument}

\noindent C’est avec raison que nous avons honte de cette convoitise, et les membres qui sont, pour ainsi dire, de son ressort et indépendants de la volonté, sont justement appelés honteux. Il n’en était pas ainsi avant le péché. « Ils étaient nus, dit l’Écriture, et ils n’en avaient point honte. » Ce n’est pas que leur nudité leur fût inconnue, mais c’est qu’elle n’était pas encore honteuse ; car alors la concupiscence ne faisait pas mouvoir ces membres contre le consentement de la volonté, et la désobéissance de la chair ne témoignait pas encore contre la désobéissance de l’esprit. En effet, ils n’avaient pas été créés aveugles, comme le vulgaire ignorant se l’imagine, puisque Adam vit les animaux auxquels il donna des noms, et qu’il est dit d’Ève : « Elle vit que le fruit défendu était bon à manger et agréable à la vue. » Leurs yeux étaient donc ouverts, mais ils ne l’étaient pas sur leur nudité, c’est-à-dire qu’ils ne prenaient pas garde à ce que la grâce couvrait en eux, alors que leurs membres ne savaient ce que c’était que désobéir à la volonté. Mais quand ils eurent perdu cette grâce, Dieu, vengeant leur désobéissance par une autre, un mouvement déshonnête se fit sentir tout à coup dans leur corps, qui leur apprit leur nudité et les couvrit de confusion.\par
De là vient qu’après qu’ils eurent violé le commandement de Dieu, l’Écriture dit : « Leurs yeux furent ouverts, et, connaissant qu’ils étaient nus, ils entrelacèrent des feuilles de figuier et s’en firent une ceinture. » Leurs yeux, dit-elle, furent ouverts, non pour voir, car ils voyaient auparavant, mais pour connaître le bien qu’ils avaient perdu et le mal qu’ils venaient d’encourir. C’est pour cela que l’arbre même dont le fruit leur était défendu et qui leur devait donner cette funeste connaissance s’appelait l’arbre de la science du bien et du mal. Ainsi, l’expérience de la maladie fait mieux sentir le prix de la santé. Ils connurent donc qu’ils étaient nus, c’est-à-dire dépouillés de cette grâce qui les empêchait d’avoir honte de leur nudité, parce que la loi du péché ne résistait pas encore à leur esprit ; ils connurent ce qu’ils eussent plus heureusement ignoré, si, fidèles et obéissants à Dieu, ils n’eussent pas commis un péché qui leur fît connaître les fruits de l’infidélité et de la désobéissance. Confus de la révolte de leur chair comme d’un témoignage honteux de leur rébellion, ils entrelacèrent des feuilles de figuier et s’en firent une ceinture, dit la Genèse. (Ici, quelques traductions portent {\itshape succinctoria} au lieu de {\itshape campestria}, mot latin qui désigne le vêtement court des lutteurs dans le champ de Mars, {\itshape in campo}, d’où {\itshape campestria} et {\itshape campestrati}.) La honte leur fit donc couvrir, par pudeur, ce qui n’obéissait plus à la volonté déchue. De là vient qu’il est naturel à tous les peuples de couvrir ces parties honteuses, à ce point qu’il y a des nations barbares qui ne les découvrent pas même dans le bain ; et parmi les épaisses et solitaires forêts de l’Inde, les gymnosophistes, ainsi nommés parce qu’ils philosophent nus, font exception pour ces parties et prennent soin de les cacher.
\subsection[{Chapitre XVIII}]{Chapitre XVIII}

\begin{argument}\noindent De la honte qui accompagne, même dans le mariage, la génération des enfants.
\end{argument}

\noindent Quand la convoitise veut se satisfaire, je ne parle pas seulement de ces liaisons coupables qui cherchent l’obscurité pour échapper à la justice des hommes, mais de ces commerces impurs que la loi humaine tolère, elle ne laisse pas de fuir le jour et les regards ; ce qui prouve que, même dans les lieux de débauche il a été plus aisé à l’impudicité de s’affranchir du joug des lois qu’à l’impudence de fermer tout asile à la pudeur. Les débauchés appellent eux-mêmes leurs actions déshonnêtes ; et, quoiqu’ils les aiment, ils rougissent de les publier. Que dirai-je de l’union légitime du mariage, dont pourtant l’objet exprès, suivant la loi civile, est la procréation des enfants ? Ne cherche-t-elle pas aussi le secret, et, avant la consommation, ne chasse-t-elle pas tous ceux qui avaient été présents jusque-là, serviteurs, amis et même les paranymphes ? Un grand maître de l’éloquence romaine dit que toutes les bonnes actions veulent paraître au grand jour, c’est-à-dire être connues ; et celle-ci, quelle que soit sa bonté, ne veut l’être qu’en ayant honte de se montrer Chacun sait, par exemple, ce qui se passe entre les époux en vue de la génération des enfants, et pour quelle autre fin célèbre-t-on te mariage avec tant de solennité ? et néanmoins, quand les époux veulent s’unir, ils ne souffrent pas que leurs enfants, s’ils en ont déjà, soient témoins d’une action à laquelle ils doivent la vie. D’où vient cela, sinon de ce que cette action, bien qu’honnête et permise, se ressent toujours de la honte qui accompagne la peine du péché ?
\subsection[{Chapitre XIX}]{Chapitre XIX}

\begin{argument}\noindent Il est nécessaire d’opposer à l’activité de la colère et de la convoitise le frein de la sagesse.
\end{argument}

\noindent Voilà pour quel motif les philosophes qui ont le plus approché de la vérité sont demeurés d’accord que la colère et la concupiscence sont des passions vicieuses de l’âme, en ce qu’elles se portent en tumulte et avec désordre aux choses même que la sagesse ne défend point ; elles ont donc besoin d’être conduites et modérées par la raison qui, selon eux, a son siège dans la plus haute partie de l’âme, d’où, comme d’un lieu éminent, elle gouverne ces deux autres parties inférieures, afin que des commandements de l’une et de l’obéissance des autres naisse dans l’homme une justice accomplie. Mais ces deux parties qu’ils tiennent pour vicieuses, même dans l’homme sage et tempérant, en sorte qu’il faut que la raison les retienne et les arrête pour ne leur permettre de se porter qu’à de bonnes actions, comme la colère à châtier justement, la concupiscence à engendrer des enfants, ces parties, dis-je, n’étaient point vicieuses dans le paradis avant le péché. Elles n’avaient point alors de mouvements qui ne fussent parfaitement soumis à la droite raison, et si elles en ont aujourd’hui qui lui sont contraires et que les gens de bien tâchent de réprimer, ce n’est point là l’état naturel d’une âme saine, mais celui d’une âme rendue malade par le péché. Comment se fait-il maintenant que nous n’ayons pas honte des mouvements de la colère et des autres passions comme nous faisons de ceux de la concupiscence, et que nous ne nous cachions pas pour leur donner un libre cours ? c’est que les membres du corps que nous employons pour les exécuter ne se meuvent pas au gré de ces passions, mais par le commandement de la volonté. Lorsque, dans la colère, nous frappons ou injurions quelqu’un, c’est bien certainement la volonté qui meut notre langue ou notre main, comme elle les meut aussi lorsque nous ne sommes pas en colère ; mais pour les parties du corps qui servent à la génération, la concupiscence se les est tellement assujetties qu’elles n’ont de mouvement que ce qu’elle leur en donne : voilà ce dont nous avons honte, voilà ce qu’on ne peut regarder sans rougir ; aussi un homme souffre-t-il plus aisément une multitude de témoins, quand il se fâche injustement, qu’il n’en souffrirait un seul dans des embrassements légitimes
\subsection[{Chapitre XX}]{Chapitre XX}

\begin{argument}\noindent Contre l’infamie des cyniques.
\end{argument}

\noindent C’est à quoi les philosophes cyniques n’ont pas pris garde, lorsqu’ils ont voulu établir leur immonde et impudente opinion, bien digne du nom de la secte, savoir que l’union des époux étant chose légitime, il ne faut pas avoir honte de l’accomplir au grand jour, dans la rue ou sur la place publique. Cependant la pudeur naturelle a cette fois prévalu sur l’erreur. Car bien qu’on rapporte que Diogène osa mettre son système en pratique, dansl’espoir sans doute de rendre sa secte d’autant plus célèbre qu’il laisserait dans la mémoire des hommes un plus éclatant témoignage de son effronterie, cet exemple n’a pas été imité depuis par les cyniques ; la pudeur a eu plus de pouvoir pour leur inspirer le respect de leurs semblables que l’erreur pour leur faire imiter l’obscénité des chiens. J’imagine donc que Diogène et ses imitateurs ont plutôt fait le simulacre de cette action, devant un public qui ne savait pas ce qui se passait sous leur-manteau, qu’ils n’ont pu l’accomplir effectivement ; et ainsi des philosophes n’ont pas rougi de paraître faire des choses où la concupiscence même aurait eu honte de les assister. Chaque jour encore nous voyons de ces philosophes cyniques : ce sont ces hommes qui ne se contentent pas de porter le manteau et qui y joignent une massue or, si quelqu’un d’eux était assez effronté pour risquer l’aventure dont il s’agit, je ne doute point qu’on ne le lapidât, ou du moins qu’on ne lui crachât à la figure. L’homme donc a naturellement honte de cette concupiscence, et avec raison, puisqu’elle atteste son indocilité, et il fallait que les marques en parussent surtout dans les parties qui servent à la génération de la nature humaine, cette nature ayant été tellement corrompue par le premier péché que tout homme en garde la souillure, à moins que la grâce de Dieu n’expie en lui le crime commis par tous et vengé sur tous, quand tous étaient en un seul.
\subsection[{Chapitre XXI}]{Chapitre XXI}

\begin{argument}\noindent La prévarication des premiers hommes n’a pas détruit la sainteté du commandement qui leur fut donné de croître et de multiplier.
\end{argument}

\noindent Loin de nous la pensée que nos premiers parents aient ressenti dans le paradis cette concupiscence dont ils rougirent ensuite en couvrant leur nudité, et qu’ils en eussent besoin pour accomplir le précepte de Dieu : « Croissez et multipliez, et remplissez la terre. » Cette concupiscence est née depuis le péché ; c’est depuis le péché que notre nature, déchue de l’empire qu’elle avait sur son corps, mais non déshéritée de toute pudeur,sentit ce désordre, l’aperçut, en eut honte et le couvrit.\par
Quant à cette bénédiction qu’ils reçurent pour croître, multiplier et remplir la terre, quoiqu’elle soit demeurée depuis le péché, elle leur fut donnée auparavant, afin de montrer que la génération des enfants est l’honneur du mariage et non la peine du péché. Mais maintenant les hommes qui ne savent pas quelle était la félicité du paradis, s’imaginent qu’on n’y aurait pu engendrer des enfants que par le moyen de cette concupiscence dont nous voyons que le mariage même, tout honorable qu’il est, ne laisse pas de rougir. En effet, les uns rejettent avec un mépris insolent cette partie de l’Écriture sainte où il est dit que les premiers hommes, après avoir péché, eurent honte de leur nudité et se couvrirent ; les autres, il est vrai, la reçoivent respectueusement, mais ils ne veulent pas qu’on entende ces paroles : « Croissez et multipliez », de la fécondité du mariage, parce qu’on lit dans les Psaumes une parole toute semblable et qui ne concerne point le corps, mais l’âme : « Vous multiplierez, dit le Prophète, la vertu dans mon âme » ; et quant à ce qui suit dans la Genèse : « Remplissez la terre et dominez sur elle » ; par la terre, ils entendent le corps que l’âme remplit par sa présence et sur qui elle domine quand la vertu est multipliée en elle. Mais ils assurent que les enfants n’eussent point été engendrés dans le paradis autrement qu’ils le sont à cette heure, et même que, sans le péché, on n’y en eût point engendré du tout, ce qui est réellement arrivé ; car Adam n’a connu sa femme et n’en a eu des enfants qu’après être sorti du paradis.
\subsection[{Chapitre XXII}]{Chapitre XXII}

\begin{argument}\noindent De l’union conjugale instituée originairement par dieu, qui l’a bénie.
\end{argument}

\noindent Pour nous, nous ne doutons point que croître, multiplier et remplir la terre en vertu de la bénédiction de Dieu, ce ne soit un don du mariage que Dieu a établi dès le commencement avant le péché, en créant un homme et une femme, c’est-à-dire deux sexes différents. Cet ouvrage de Dieu fut immédiatement suivi de sa bénédiction ; ce qui résulte évidemment de l’Écriture, qui, après ces paroles : « Il les créa mâle et femelle », ajoute aussitôt : « Et Dieu les bénit, disant : Croissez et multipliez, et remplissez la terre et dominez sur elle. » Malgré la possibilité de donner un sens spirituel à tout cela, on ne peut pas dire pourtant que ces mots mâle et femelle puissent s’entendre de deux choses qui se trouvent en un même homme, sous prétexte qu’en lui autre chose est ce qui gouverne, et autre chose ce qui est gouverné ; mais il paraît clairement que deux hommes de différent sexe furent créés, afin que, par la génération des enfants, ils crussent, multipliassent et remplissent la terre. On ne saurait, sans une extrême absurdité, combattre une chose aussi manifeste. Ce ne fut ni à propos de l’esprit qui commande et du corps qui obéit, ni de la raison qui gouverne et de la convoitise qui est gouvernée, ni de la vertu active qui est soumise à la contemplative, ni de l’entendement, qui est de l’âme, et des sens qui sont du corps, mais à propos du lien conjugal qui unit ensemble les deux sexes, que Notre-Seigneur, interrogé s’il était permis de quitter sa femme (car Moïse avait permis le divorce aux Juifs à cause de la dureté de leur cœur), répondit : « N’avez-vous point lu que celui qui les créa dès le commencement les créa mâle et femelle, et qu’il est dit : C’est pour cela que l’homme quittera son père et sa mère pour s’unir à sa femme, et ils ne seront tous deux qu’une même chair ? Ainsi ils ne sont « plus deux, mais une seule chair. Que l’homme donc ne sépare pas ce que Dieu a joint ». Il est dès lors certain que les deux sexes ont été créés d’abord en différentes personnes, telles que nous les voyons maintenant, et l’Évangile les appelle une seule chair, soit à cause de l’union du mariage, soit à cause de l’origine de la femme, qui a été formée du côté de l’homme ; c’est en effet de cette origine que l’Apôtre prend sujet d’exhorter les maris à aimer leurs femmes.
\subsection[{Chapitre XXIII}]{Chapitre XXIII}

\begin{argument}\noindent Comment on eût engendré des enfants dans le paradis sans aucun mouvement de concupiscence.
\end{argument}

\noindent Quiconque soutient qu’ils n’eussent point eu d’enfants, s’ils n’eussent point péché, ne dit autre chose sinon que le péché de l’homme était nécessaire pour accomplir le nombre des saints. Or, si cela ne se peut avancer sans absurdité, ne vaut-il pas mieux croire que le nombre des saints nécessaire à l’accomplissement de cette bienheureuse Cité serait aussi grand, quand personne n’aurait péché, qu’il l’est maintenant que la grâce de Dieu le recueille de la multitude des pécheurs, tandis que les enfants de ce siècle engendrent et sont engendrés ?\par
Ainsi, sans le péché, ces mariages, dignes de la félicité du paradis, eussent été exempts de toute concupiscence honteuse et féconds en aimables fruits. Comment cela eût-il pu se faire ? Nous n’avons point d’exemple pour le montrer ; et toutefois il n’y a rien d’incroyable à ce que la partie sexuelle eût obéi à la volonté, puisque tant d’autres parties du corps lui sont soumises. Si nous remuons les pieds et les mains et tous les autres membres du corps avec une facilité qui étonne, surtout chez les artisans en qui une heureuse industrie vient au secours de notre faible et lente nature, pourquoi, sans le secours de la concupiscence, fille du péché, n’eussions-nous pas trouvé dans les organes de la génération la même docilité ? En parlant de la différence des gouvernements dans son ouvrage de la République, Cicéron ne dit-il pas que l’on commande aux membres du corps comme à des enfants, à cause de leur promptitude à obéir, mais que les parties vicieuses de l’âme sont comme des esclaves qu’il faut gourmander pour en venir à bout ? Cependant, selon l’ordre naturel, l’esprit est plus excellent que le corps ; ce qui n’empêche pas que l’esprit ne commande plus aisément au corps qu’à soi-même. Mais cette concupiscence dont je parle est d’autant plus honteuse que l’esprit n’y est absolument maître ni de soi-même, ni de son corps, et que c’est plutôt la concupiscence que la volonté qui le meut. Sans cela, nous n’aurions point sujet de rougir de ces sortes de mouvements ; au lieu qu’il nous semble honteux de voir ce corps, qui naturellement devait être soumis à l’esprit, lui résister. Certes, la résistance que souffre l’esprit dans les autres passions est moins honteuse, puisqu’elle vient de lui-même, et qu’il est tout ensemble le vainqueur et le vaincu ; et toutefois, il n’en est pas moins contraire à l’ordre que les parties de l’âme qui devraient être dociles à la raison lui fassent la loi. Quant aux victoires que l’esprit remporte sur soi-même en soumettant ses affections brutales et déréglées, elles lui sont glorieuses, pourvu qu’il soit lui-même soumis à Dieu. Mais enfin il est toujours vrai de dire qu’il y a moins de honte pour lui à être son propre vainqueur, de quelque manière que ce soit, que d’être vaincu par son propre corps, lequel, outre l’infériorité de sa nature, n’a de vie que ce que l’esprit lui en communique.\par
La chasteté est sauve toutefois, tant que la volonté retient les autres membres sans lesquels ceux que la concupiscence excite en dépit de nous ne peuvent accomplir leur action. C’est cette résistance, c’est ce combat entre la concupiscence et la volonté qui n’auraient point eu lieu dans le paradis sans le péché ; tous les membres du corps y eussent été entièrement soumis à l’esprit. Ainsi le champ de la génération eût été ensemencé par les organes destinés à cette fin, de même que la terre reçoit les semences que la main y répand ; et tandis qu’à cette heure la pudeur m’empêche de parler plus ouvertement de ces matières, et m’oblige de ménager les oreilles chastes, nous aurions pu en discourir librement dans le paradis, sans craindre de donner de mauvaises pensées ; il n’y aurait point même eu de paroles déshonnêtes, et tout ce que nous aurions dit de ces parties aurait été aussi honnête que ce que nous disons des autres membres du corps. Si donc quelqu’un lit ceci avec des sentiments peu chastes, qu’il accuse la corruption de l’homme, et non sa nature ; qu’il condamne l’impureté de son cœur, et non les paroles dont la nécessité nous oblige de nous servir et que les lecteurs chastes nous pardonneront aisément, jusqu’à ce que nous ayons terrassé l’infidélité sur le terrain où elle nous a conduit. Celui qui n’est pointscandalisé d’entendre saint Paul parler de l’impudicité monstrueuse de ces femmes « qui changeaient l’usage qui est selon la nature en un autre qui est contre la nature », lira tout ceci sans scandale, alors surtout que sans parler, comme fait saint Paul, de cette abominable infamie, mais nous bornant à expliquer selon notre pouvoir ce qui se passe dans la génération des enfants, nous évitons, à son exemple, toutes les paroles déshonnêtes.
\subsection[{Chapitre XXIV}]{Chapitre XXIV}

\begin{argument}\noindent Si les hommes fussent demeurés innocents dans le paradis, l’acte de la génération serait soumis à la volonté comme toutes nos autres actions.
\end{argument}

\noindent L’homme aurait semé et la femme aurait recueilli, quand il eût fallu et autant qu’il eût été nécessaire, les organes n’étant pas mus par la concupiscence, mais par la volonté. Nous ne remuons pas seulement à notre gré les membres où il y a des os et des jointures, comme les pieds, les mains et les doigts, mais aussi ceux où il n’y a que des chairs et des nerfs, et nous les étendons, les plions, les accourcissons comme il nous plaît, ainsi que cela se voit dans la bouche et dans le visage. Les poumons enfin, c’est-à-dire les plus mous de tous les viscères, plus mous même que la moelle des os, et pour cette raison enfermés dans la poitrine qui leur sert de rempart, ne se meuvent-ils pas à notre volonté comme des soufflets d’orgue, quand nous respirons ou quand nous parlons ? Je ne rappellerai pas ici ces animaux qui donnent un tel mouvement à leur peau, lorsqu’il en est besoin, qu’ils ne chassent pas seulement les mouches en remuant l’endroit où elles sont sans remuer les autres, mais qu’ils font même tomber les flèches dont on les a percés. Les hommes, il est vrai, n’ont pas cette sorte demouvement, mais niera-t-on que Dieu eût pu le leur donner ? Ne pouvait-il donc point pareillement faire que ce qui se meut maintenant dans son corps par la concupiscence n’eût été mû que par le commandement de la volonté ?\par
Ne voyons-nous pas certains hommes qui font de leur corps tout ce qu’ils veulent ? Il y en a qui remuent les oreilles, ou toutes deux ensemble, ou chacune séparément, comme bon leur semble ; on en rencontre d’autres qui, sans mouvoir la tête, font tomber tous leurs cheveux sur le front, puis les redressent et les renversent de l’autre côté ; d’autres qui, en pressant un peu leur estomac, d’une infinité de choses qu’ils ont avalées, en tirent comme d’un sac celles qu’il leur plaît ; quelques-uns contrefont si bien le chant des oiseaux ou la voix des bêtes et des hommes, qu’on ne saurait s’en apercevoir si on ne les voyait ; il s’en trouve même qui font sortir par en bas, sans aucune ordure, tant de vents harmonieux qu’on dirait qu’ils chantent. J’ai vu, pour mon compte, un homme qui suait à volonté. Tout le monde sait qu’il y en a qui pleurent quand ils veulent et autant qu’ils veulent. Mais voici un fait bien plus incroyable, qui s’est passé depuis peu et dont la plupart de nos frères ont été témoins. Il y avait un prêtre de l’église de Calame, nommé Restitutus, qui, chaque fois qu’on l’en priait (et cela arrivait souvent), pouvait, au bruit de certaines voix plaintives, perdre les sens et rester étendu par terre comme mort, ne se sentant ni pincer, ni piquer, ni même brûler. Or, ce qui prouve que son corps ne demeurait ainsi immobile que parce qu’il était privé de tout sentiment, c’est qu’il n’avait plus du tout de respiration non plus qu’un mort. Il disait néanmoins que quand on parlait fort haut, il entendait comme des voix qui venaient de loin. Puis donc que, dans la condition présente, il est des hommes à qui leur corps obéit en des choses si extraordinaires, pourquoi ne croirions-nous pas qu’avant le péché et la corruption de la nature, il eût pu nous obéir pour ce qui regarde la génération ? L’homme a été abandonné à soi, parce qu’il a abandonné Dieu par une vaine complaisance en soi, et il n’a pu trouver en soi l’obéissance qu’il n’avait pas voulu rendre à Dieu. De là vient qu’il est manifestement misérable en ce qu’il ne vit pas comme il l’entend. Il est vrai que s’il vivait à son gré, il se croirait bienheureux ; mais il ne le serait pas même de la sorte, à moins qu’il ne vécût comme il faut.
\subsection[{Chapitre XXV}]{Chapitre XXV}

\begin{argument}\noindent On ne saurait être vraiment heureux en cette vie.
\end{argument}

\noindent À y regarder de près, l’homme heureux seul vit selon sa volonté, et nul n’est heureux s’il n’est juste ; mais le juste même ne vit pas comme il veut, avant d’être parvenu à un état où il ne puisse plus ni mourir, ni être trompé, ni souffrir de mal, et tout cela avec la certitude d’y demeurer toujours. Tel est l’état que la nature désire ; et elle ne saurait être pleinement et parfaitement heureuse qu’elle n’ait obtenu l’objet de ses vœux. Or, quel est l’homme qui puisse dès à présent vivre comme il veut, lorsqu’il n’est pas seulement en son pouvoir de vivre ? Il veut vivre, et il est contraint de mourir. Comment donc vivra-t-il comme il l’entend, cet être qui ne vit pas autant qu’il le souhaite ? Que s’il veut mourir, comment peut-il vivre comme il veut, lorsqu’il ne veut pas vivre ? Et même, de ce qu’il veut mourir, il ne s’ensuit pas qu’il ne soit bien aise de vivre ; mais il veut mourir pour vivre après la mort. Il ne vit donc pas encore comme il veut, mais il vivra selon son désir, quand il sera arrivé en mourant où il désire arriver. À la bonne heure ! qu’il vive comme il veut, puisqu’il a gagné sur lui de ne vouloir que ce qui se peut, suivant le précepte de Térence :\par
 {\itshape « Ne pouvant faire ce que tu veux, tâche de vouloir ce qui se peut. »} \par
Mais est-ce bien le bonheur que de souffrir son mal en patience ? Si l’on n’aime réellement la vie bienheureuse, on ne la possède point. Or, pour l’aimer comme il faut, il est nécessaire de l’aimer par-dessus tout, puisque c’est pour elle que l’on doit aimer tout ce que l’on aime. Mais si on l’aime autant qu’elle mérite d’être aimée (car celui-là n’est pas heureux qui n’aime pas la vie bienheureuse autant qu’elle le mérite), il ne se peut faire que celui qui l’aime ainsi, ne désire qu’elle soit éternelle : sa béatitude tient donc essentiellement à son éternité.
\subsection[{Chapitre XXVI}]{Chapitre XXVI}

\begin{argument}\noindent Les hommes auraient rempli sans rougir, dans le paradis, l’office de la génération.
\end{argument}

\noindent L’homme vivait donc dans le paradis commeil voulait, puisqu’il ne voulait que ce qui était conforme au commandement divin ; il vivait jouissant de Dieu, et bon par sa bonté ; il vivait sans aucune indigence, et pouvait vivre éternellement. S’il avait faim, les aliments ne lui manquaient pas, ni, s’il avait soif, les breuvages, et l’arbre de vie le défendait contre la vieillesse. Aucune corruption dans sa chair qui pût lui causer la moindre douleur. Point de maladies à craindre au dedans, point d’accidents au dehors. Son corps jouissait d’une pleine santé, et son âme d’une tranquillité absolue. Tout comme le froid et le chaud étaient inconnus dans le paradis, ainsi son heureux habitant était à l’abri des vicissitudes de la crainte et du désir. Ni tristesse, ni fausses joies ; toute sa joie venait de Dieu, qu’il aimait d’une ardente charité, et cette charité prenait sa source dans un cœur pur, une bonne conscience et une foi sincère. La société conjugale y était accompagnée d’un amour honnête. Le corps et l’esprit vivaient dans un parfait accord, et l’obéissance au commandement de Dieu était facile ; car il n’y avait à redouter aucune surprise, soit de la fatigue, soit du sommeil. Dieu nous garde de croire qu’avec une telle facilité en toutes choses et une si grande félicité, l’homme eût été incapable d’engendrer sans le secours de la concupiscence. Les parties destinées à la génération auraient été mues, comme les autres membres, par le seul commandement de la volonté. Il aurait pressé sa femme dans ses bras avec une entière tranquillité de corps et d’esprit, sans ressentir en sa chair aucun aiguillon de volupté, et sans que la virginité de sa femme en souffrît aucune atteinte. Si l’on objecte que nous ne pouvons invoquer ici le témoignage de l’expérience, je réponds que ce n’est pas une raison d’être incrédule ; car il suffit de savoir que c’est la volonté et non une ardeur turbulente qui aurait présidé à la génération. Et d’ailleurs, pourquoi la semence conjugale eût-elle nécessairement fait tort à l’intégrité de la femme, quand nous savons que l’écoulement des mois n’en fait aucun à l’intégrité de la jeune fille ? Injection, émission, les deux opérations sont inverses, mais la route est lamême. La génération se serait donc accomplie avec la même facilité que l’accouchement ; car la femme aurait enfanté sans douleur, et l’enfant serait sorti du sein maternel sans aucun effort, comme un fruit qui tombe lorsqu’il est mûr. Nous parlons de choses qui sont maintenant honteuses, et quoique nous tâchions de les concevoir telles qu’elles auraient pu être, alors qu’elles étaient honnêtes, il vaut mieux néanmoins céder à la pudeur qui nous retient, que de nous laisser aller au mouvement de notre faible éloquence. L’observation nous faisant ici défaut, tout comme à nos premiers parents (car le péché et l’exil, juste châtiment du péché, les empêchèrent de s’unir saintement), il nous est difficile de concevoir cette union calme et libre sans le cortège des mouvements déréglés qui la troublent présentement ; et de là celle retenue qu’on observe à parler de ces matières, quoique l’on ne manque pas de bons raisonnements pour les éclaircir. Mais le Dieu tout-puissant et souverainement bon, créateur de toutes les natures, qui aide et récompense les bonnes volontés, abandonne et condamne lesmauvaises, et les ordonne toutes, ce Dieu n’a pas manqué de moyens pour tirer de la masse corrompue du genre humain un certain nombre de prédestinés, comme autant de pierres vivantes qu’il veut faire entrer dans la structure de sa cité, ne les discernant point par leurs mérites, puisqu’ils étaient tous également corrompus, mais par sa grâce, et leur montrant, non seulement par eux-mêmes qu’il délivre, mais aussi par ceux qu’il ne délivre pas, combien ils lui sont redevables. On ne peut en effet imputer sa délivrance qu’à la bonté gratuite de son libérateur, quand on se voit délivré de la compagnie de ceux avec qui l’on méritait d’être châtié. Pourquoi donc Dieu n’aurait-il pas créé ceux qu’il prévoyait devoir pécher, puisqu’il était assez puissant pour les punir ou pour leur faire grâce, et que, sous un maître si sage, les désordres mêmes des méchants contribuent à l’ordre de l’univers ?
\subsection[{Chapitre XXVII}]{Chapitre XXVII}

\begin{argument}\noindent Des hommes et des anges prévaricateurs, dont le péché ne trouble pas l’ordre de la divine Providence.
\end{argument}

\noindent Les anges et les hommes pécheurs ne font rien dès lors qui puisse troubler l’économie des grands ouvrages de Dieu, dans lesquels sa volonté se trouve toujours accomplie. Comme il dispense à chaque chose ce qui lui appartient avec une sagesse égale à sa puissance, il ne sait pas seulement bien user des bons, mais encore des méchants. Ainsi, usant bien du mauvais ange, dont la volonté s’était tellement endurcie qu’il n’en pouvait plus avoir de bonne, pourquoi n’aurait-il pas permis qu’il tentât le premier homme, qui avait été créé droit, c’est-à-dire avec une bonne volonté ? En effet, il avait été créé de telle sorte qu’il pouvait vaincre le diable en s’appuyant sur Dieu, et qu’il en devait être vaincu en abandonnant son créateur et son protecteur pour se complaire vainement en soi-même. Si sa volonté, aidée de la grâce, fût demeurée droite, elle aurait été en lui une source de mérite, comme elle devint une source de péché, parce qu’il abandonna Dieu, Quoiqu’il ne pût au fond mettre sa confiance dans ce secours du ciel sans ce secours même, il était néanmoins en son pouvoir de ne pas s’en servir. De même que nous ne saurions vivre ici-bas sans prendre des aliments, et que nous pouvons néanmoins n’en pas prendre, comme font ceux qui se laissent mourir de faim, ainsi, même dans le paradis, l’homme ne pouvait vivre sans le secours de Dieu, et toutefois il pouvait mal vivre par lui-même, mais en perdant sa béatitude et tombant dans la peine très juste qui devait suivre son péché. Qui s’opposait donc à ce que Dieu, lors même qu’il prévoyait la chute de l’homme, permît que le diable le tentât et le vainquît, puisqu’il prévoyait aussi que sa postérité, assistée de sa grâce, remporterait sur le diable une victoire bien plus glorieuse ? De cette sorte, rien de ce qui devait arriver n’a été caché à Dieu ; sa prescience n’a contraint personne à pécher, et il a fait voir à l’homme et à l’ange, par leur propre expérience, l’intervalle qui sépare la présomption de la créature de la protection du créateur. Qui oserait dire que Dieu n’ait pu empêcher la chute de l’homme et de l’ange ? Mais il a mieux aimé la laisser en leur pouvoir, afin de montrer de quel mall’orgueil est capable, et ce que peut sa grâce victorieuse.
\subsection[{Chapitre XXVIII}]{Chapitre XXVIII}

\begin{argument}\noindent Différence des deux Cités.
\end{argument}

\noindent Deux amours ont donc bâti deux cités : l’amour de soi-même jusqu’au mépris de Dieu, celle de la terre, et l’amour de Dieu jusqu’au mépris de soi-même, celle du ciel. L’une se glorifie en soi, et l’autre dans le Seigneur ; l’une brigue la gloire des hommes, et l’autre ne veut pour toute gloire que le témoignage de sa conscience ; l’une marche la tête levée, toute bouffie d’orgueil, et l’autre dit à Dieu : « Vous êtes ma gloire, et c’est vous qui me faites marcher la tête levée » ; en l’une, les princes sont dominés par la passion de dominer sur leurs sujets, et en l’autre, les princes et les sujets s’assistent mutuellement, ceux-là par leur bon gouvernement, et ceux-ci par leur obéissance ; l’une aime sa propre force en la personne de ses souverains, et l’autre dit à Dieu : « Seigneur, qui êtes ma vertu, je vous aimerai. » Aussi les sages de l’une, vivant selon l’homme, n’ont cherché que les biens du corps ou de l’âme, ou de tous les deux ensemble ; et si quelques-uns ont connu Dieu, ils ne lui ont point rendu l’homme et l’hommage qui lui sont dus, mais ils se sont perdus dans la vanité de leurs pensées et sont tombés dans l’erreur et l’aveuglement. En se disant sages, c’est-à-dire en se glorifiant de leur sagesse, ils sont devenus fous et ont rendu l’honneur qui n’appartient qu’au Dieu incorruptible à l’image de l’homme corruptible et à des figures d’oiseaux, de quadrupèdes et de serpents ; car, ou bien ils ont porté les peuples à adorer les idoles, ou bien ils les ont suivis, aimant mieux rendre le culte souverain à la créature qu’au Créateur, qui est béni dans tous les siècles. Dans l’autre cité, au contraire, il n’y a de sagesse que la piété, qui fonde le culte légitime du vrai Dieu et attend pour récompense dans la société des saints, c’est-à-dire des hommes et des anges, l’accomplissement de cette parole : « Dieu tout en tous. »
\section[{Livre quinzième. Avant le déluge}]{Livre quinzième. \\
Avant le déluge}\renewcommand{\leftmark}{Livre quinzième. \\
Avant le déluge}

\subsection[{Chapitre premier}]{Chapitre premier}

\begin{argument}\noindent De la séparation des hommes en deux sociétés, à partir des enfants d’Adam.
\end{argument}

\noindent On a beaucoup écrit sur le paradis terrestre, sur la félicité dont on y jouissait, sur la vie qu’y menaient les premiers hommes, sur leur crime et leur punition. Et nous aussi, nous en avons parlé dans les livres précédents, selon ce que nous en avons lu ou pu comprendre dans l’Écriture ; mais un examen détaillé de tous ces points ferait naître une infinité de questions qui demanderaient à être traitées avec plus d’étendue, et qui passeraient de beaucoup les bornes de cet ouvrage et de notre loisir. Où en trouver assez, si nous prétendions répondre à toutes les difficultés que nous pourraient faire des esprits oisifs et pointilleux, toujours plus prêts à former des objections que capables d’en comprendre les solutions ? J’estime toutefois avoir déjà éclairci les grandes et difficiles questions du commencement et de la fin du monde, de la création de l’âme et de celle de tout le genre humain, qui a été distingué en deux ordres, l’un composé de ceux qui vivent selon l’homme, et l’autre de ceux qui vivent selon Dieu. Nous donnons encore à ces deux ordres le nom mystique de Cités, par où il faut entendre deux sociétés d’hommes, dont l’une est prédestinée à vivre éternellement avec Dieu, et l’autre à souffrir un supplice éternel avec le diable. Telle est leur fin, dont nous traiterons dans la suite. Maintenant, puisque nous avons assez parlé de leur naissance, soit dans les anges, soit dans les deux premiers hommes, il est bon, ce me semble, que nous en considérions le cours et le progrès, depuis le moment où les deux premiers hommes commencèrent à engendrer jusqu’à la fin des générations humaines. C’est de tout cet espace de temps, où il se fait une révolution continuelle de personnes qui meurent, et d’autres qui naissent et qui prennent leur place, que se compose la durée des deux cités.\par
Caïn, qui appartient à la cité des hommes, naquit le premier des deux auteurs du genre humain ; vint ensuite Abel, qui appartient à la cité de Dieu. De même que nous expérimentons dans chaque homme en particulier la vérité de cette parole de l’Apôtre, que ce n’est pas ce qui est spirituel qui est formé le premier, mais ce qui est animal, d’où vient que nous naissons d’abord méchants et charnels, comme sortant d’une racine corrompue, et ne devenons bons et spirituels qu’en renaissant de Jésus-Christ, ainsi en est-il de tout le genre humain. Lorsque les deux cités commencèrent à prendre leur cours dans l’étendue des siècles, l’homme de la cité de la terre fut celui qui naquit le premier, et, après lui, le membre de la cité de Dieu, prédestiné par la grâce, élu par la grâce, étranger ici-bas par la grâce, et par la grâce citoyen du ciel. Par lui-même, en effet, il sortit de la même masse qui avait été toute condamnée dans son origine ; mais Dieu, comme un potier de terre (car c’est la comparaison dont se sert saint Paul, à dessein, et non pas au hasard), fit d’une même masse un vase d’honneur et un vase d’ignominie. Or, le vase d’ignominie a été fait le premier, puis le vase d’honneur, parce que dans chaque homme, comme je viens de le dire, précède ce qui est mauvais, ce par où il faut nécessairement commencer, mais où il n’est pas nécessaire de demeurer ; et après vient ce qui est bon, où nous parvenons par notre progrès dans la vertu, et où nous devons demeurer. Il est vrai dès lors que tous ceux qui sont méchants ne deviendront pas bons ; mais il l’est aussi qu’aucun ne sera bon qui n’ait été originairement méchant. L’Écriture dit donc de Caïn qu’il bâtit une ville ; mais Abel,qui était étranger ici-bas, n’en bâtit point. Car la cité des saints est là-haut, quoiqu’elle enfante ici-bas des citoyens en qui elle est étrangère à ce monde, jusqu’à ce que le temps de son règne arrive et qu’elle rassemble tous ses citoyens au jour de la résurrection des corps, quand ils obtiendront le royaume qui leur est promis et où ils régneront éternellement avec le Roi des siècles, leur souverain.
\subsection[{Chapitre II}]{Chapitre II}

\begin{argument}\noindent Des fils de la terre et des fils de promission.
\end{argument}

\noindent Il a existé sur la terre, à la vérité, une ombre et une image prophétique de cette cité, pour en être le signe obscur plutôt que la représentation expresse, et cette image a été appelée elle-même la cité sainte, comme le symbole et non comme la réalité de ce qui doit s’accomplir un jour. C’est de cette image inférieure et subordonnée dans son contrasteavec la cité libre qu’elle marquait, que l’Apôtre parle ainsi aux Galates : « Dites-moi, je vous prie, vous qui voulez être sous la loi, n’avez-vous point ouï ce que dit la loi ? Car il est écrit qu’Abraham a eu deux fils, l’un de la servante et l’autre de la femme libre. Mais celui qui naquit de la servante naquit selon la chair, et celui qui naquit de la femme libre naquit en vertu de la promesse de Dieu. Or, tout ceci est une allégorie. Ces deux femmes sont les deux alliances, dont la première, qui a été établie sur le mont Sina et qui n’engendre que des esclaves, est figurée par Agar. Agar est en figure la même chose que Sina, montagne d’Arabie, et Sina représente la Jérusalem terrestre qui est esclave avec ses enfants, au lieu que la Jérusalem d’en haut est vraiment libre, et c’est elle qui est notre mère ; car il est écrit : Réjouissez-vous, stériles qui n’enfantez point ; poussez des cris de joie, vous qui ne concevez point ; car celle qui était délaissée a plus d’enfants que celle qui a un mari. Nous sommes donc, mes frères, les enfants de la promesse, ainsi qu’Isaac. Et comme alors celui qui était né selon la chair persécutait celui qui était né selon l’esprit, il en est encore de même aujourd’hui. Mais que dit l’Écriture ? Chassez la servante et son fils ; car le fils de la servante ne sera point héritier avec le fils de la femme libre. Or, mes frères, nous ne sommes point les enfants de la servante, mais de la femme libre ; et c’est Jésus-Christ qui nous a acquis cette liberté. » Cette explication de l’Apôtre nous apprend comment nous devons entendre les deux Testaments. Une partie de la cité de la terre est devenue une image de la cité du ciel. Elle n’a pas été établie pour elle-même, mais pour être le symbole d’une autre ; et ainsi la cité de la terre, image de la cité du ciel, a en elle-même une image qui la représentait. En effet, Agar, servante de Sarra, et son fils étaient en quelque façon une image de cette image, une figure de cette figure ; et comme, à l’arrivée de la lumière, les ombres devaient s’évanouir, Sarra, qui était la femme libre et signifiait la cité libre, laquelle figurait elle-même la Jérusalem terrestre, dit : « Chassez la servante et son fils ; car le fils de la servante ne sera point héritier avec mon fils Isaac », ou, comme dit l’Apôtre : « Avec le fils de la femme libre ». Nous trouvons donc deux choses dans la cité de la terre, d’abord la figure d’elle-même, et puis celle de la cité du ciel qu’elle représentait. Or, la nature corrompue par le péché enfante les citoyens de la cité de la terre, et la grâce, qui délivre la nature du péché, enfante les citoyens de la cité du ciel ; d’où vient que ceux-là sont appelés des vases de colère, et ceux-ci des vases de miséricorde. C’est encore ce qui a été figuré dans les deux fils d’Abraham, attendu que l’un d’eux, savoir Ismaël, est né selon la chair, de la servante Agar, et l’autre, Isaac, est né de la femme libre, en exécution de la promesse de Dieu. L’un et l’autre à la vérité sont enfants d’Abraham, mais l’un engendré selon le cours ordinaire des choses, qui marquait la nature, et l’autre donné en vertu de la promesse, qui signifiait la grâce. En l’un paraît l’ordre des choses humaines, et dans l’autre éclate un bienfait particulier de Dieu.
\subsection[{Chapitre III}]{Chapitre III}

\begin{argument}\noindent De la stérilité de Sarra que Dieu féconda par sa grâce.
\end{argument}

\noindent Sarra était réellement stérile ; et, comme elle désespérait d’avoir des enfants, elle résolut d’en avoir au moins de sa servante qu’elle donna à son mari pour habiter avec elle. De cette sorte, elle exigea de lui le devoir conjugal, usant de son droit en la personne d’une autre. Ismaël naquit comme les autreshommes de l’union des deux sexes, suivant la loi ordinaire de la nature : c’est pour cela que l’Écriture dit qu’il naquit selon la chair, non que les enfants nés de cette manière ne soient des dons et des ouvrages de Dieu, de ce Dieu dont la sagesse atteint sans aucun obstacle d’une extrémité à l’autre et qui dispose toutes choses avec douceur, mais parce que, pour marquer un don de la grâce de Dieu entièrement gratuit et nullement dû aux hommes, il fallait qu’un enfant naquît contre le cours ordinaire de la nature. En effet, la nature a coutume de refuser des enfants à des personnes aussi âgées que l’étaient Abraham et Sarra quand ils eurent Isaac, outre que Sarra était même naturellement stérile. Or, cette impuissance de la nature à produire des enfants dans cette disposition, est un symbole de la nature humaine, corrompue par le péché et justement condamnée, et désormais déchue de toute véritable félicité. Ainsi Isaac, né en vertu de la promesse de Dieu, figure très bien les enfants de la grâce, les citoyens de la cité libre, les cohéritiers de l’éternelle paix, où ne règne pas l’amour de la volonté propre, mais une charité humble et soumise, unie dans la jouissance commune du bien immuable, et qui de plusieurs cœurs n’en fait qu’un.
\subsection[{Chapitre IV}]{Chapitre IV}

\begin{argument}\noindent De la paix et de la guerre dans la cité terrestre.
\end{argument}

\noindent Mais la cité de la terre, qui ne sera pas éternelle (car elle ne sera plus cité, quand elle sera condamnée au dernier supplice), trouvera-ici-bas son bien, dont la possession lui procure toute la joie que peuvent donner de semblables choses. Comme ce bien n’est pas tel qu’il ne cause quelques traverses à ceux qui l’aiment, il en résulte que cette cité est souvent divisée contre elle-même, que ses citoyens se font la guerre, donnent des batailles et remportent des victoires sanglantes. Là chaque parti veut demeurer le maître, tandis qu’il est lui-même esclave de ses vices. Si, lorsqu’il est vainqueur, il s’enfle de ce succès, sa victoire lui devient mortelle ; si, au contraire, pensant à la condition et aux disgrâces communes, il se modère par la considération des accidents de la fortune, cette victoire lui est plus avantageuse ; mais lamort lui en ôte enfin le fruit ; car il ne peut pas toujours dominer sur ceux qu’il s’est assujettis. On ne peut pas nier toutefois que les choses dont cette cité fait l’objet de ses désirs ne soient des biens, puisque elle-même, en son genre, est aussi un bien, et de tous les biens de la terre le plus excellent. Or, pour jouir de ces biens terrestres, elle désire une certaine paix, et ce n’est que pour cela qu’elle fait la guerre. Lorsqu’elle demeure victorieuse et qu’il n’y a plus personne qui lui résiste, elle a la paix que n’avaient pas les partis contraires qui se battaient pour posséder des choses qu’ils ne pouvaient posséder ensemble. C’est cette paix qui est le but de toutes les guerres et qu’obtient celui qui remporte la victoire. Or, quand ceux qui combattaient pour la cause la plus juste demeurent vainqueurs, qui doute qu’on ne doive se réjouir de leur victoire et de la paix qui la suit ? Ces choses sont bonnes, et viennent sans doute de Dieu ; mais si l’on se passionne tellement pour ces moindres biens, qu’on les croie uniques ou qu’on les aime plus que ces autres biens beaucoup plus excellents qui appartiennent à la céleste cité, où il y aura une victoire suivie d’une paix éternelle et souveraine, la misère alors est inévitable et tout se corrompt de plus en plus.
\subsection[{Chapitre V}]{Chapitre V}

\begin{argument}\noindent Du premier fondateur de la cité de la terre, qui tua son frère ; en quoi il fut imité depuis par le fondateur de Rome.
\end{argument}

\noindent C’est ainsi que le premier fondateur de la cité de la terre fut fratricide. Transporté de jalousie, il tua son frère, qui était citoyen de la cité éternelle et étranger ici-bas. Il n’y a donc rien d’étonnant que ce crime primordial et, comme diraient les Grecs, ce type du crime, ait été imité si longtemps après, lors de la fondation de cette ville qui devait être la maîtresse de tant de peuples et la capitale de la cité de la terre. Ainsi que l’a dit un de leurs poètes :\par
 {\itshape « Les premiers murs de Rome furent teints du sang d’un frère tué par son frère. »} \par
En effet, l’histoire rapporte que Romulus tua son frère Rémus, et il n’y a d’autre différence entre ce crime et celui de Caïn, sinonqu’ici les frères étaient tous deux citoyens de la cité de la terre, et que tous deux prétendaient être les fondateurs de la république romaine. Or, tous deux ne pouvaient avoir autant de gloire qu’un seul ; car une puissance partagée est toujours moindre. Afin donc qu’un seul la possédât tout entière, il se défit de son compétiteur et accrut par son crime un empire qui autrement aurait été moins grand, mais plus juste. Caïn et Abel n’étaient pas touchés d’une pareille ambition, et ce n’était pas pour régner seul que l’un des deux tua l’autre. Abel ne se souciait pas, en effet, de dominer sur la ville que son frère bâtissait ; en sorte qu’il ne fut tué que par cette malignité diabolique qui fait que les méchants portent envie aux gens de bien, sans autre raison sinon que les uns sont bons et les autres méchants. La bonté ne se diminue pas pour être possédée par plusieurs ; au contraire, elle devient d’autant plus grande, que ceux qui la possèdent sont plus unis ; pour tout dire en un mot, le moyen de la perdre est de la posséder tout seul, et l’on ne la possède jamais plus entière que quand on est bien aise de la posséder avec plusieurs. Or, ce qui arriva entre Rémus et Romulus montre comment la cité de la terre se divise contre elle-même ; et ce qui survint entre Caïn et Abel fait voir la division qui existe entre les deux cités, celle de Dieu et celle des hommes. Les méchants combattent donc les uns contre les autres, et les méchants combattent aussi contre les bons ; mais les bons, s’ils sont parfaits, ne peuvent avoir aucun différend entre eux. Ils en peuvent avoir, quand ils n’ont pas encore atteint cette perfection ; comme un homme peut n’être pas d’accord avec soi-même, puisque dans le même homme la chair convoite souvent contre l’esprit et l’esprit contre la chair. Les inclinations spirituelles de l’un peuvent dès lors combattre les inclinations charnelles de l’autre, et réciproquement, de même que les bons et les méchants se font la guerre les uns aux autres ; ou encore, les inclinations charnelles de deux hommes de bien, mais qui ne sont pas encore parfaits, peuvent se combattre l’une l’autre, comme font entre eux les méchants, jusqu’à ce que la grâce victorieuse de Jésus-Christ les ait entièrement guéris de ces faiblesses.
\subsection[{Chapitre VI}]{Chapitre VI}

\begin{argument}\noindent Des langueurs auxquelles sont sujets, en punition du péché, les citoyens mêmes de la Cité de Dieu, et dont ils sont enfin délivrés par la grâce.
\end{argument}

\noindent Cette langueur, c’est-à-dire cette désobéissance dont nous avons parlé au quatorzième livre, est la peine de la désobéissance du premier homme, et ainsi elle ne vient pas de la nature, mais du vice de la volonté ; c’est pourquoi il est dit aux bons, qui s’avancentdans la vertu et qui vivent de la foi dans ce pèlerinage : « Portez les fardeaux les uns des autres, et vous accomplirez la loi de Jésus-Christ » ; et dans un autre endroit : « Reprenez ceux qui sont turbulents, consolez les affligés, supportez les faibles, et soyez débonnaires à tout le monde. Prenez garde de ne point rendre le mal pour le mal » ; et encore : « Si quelqu’un est tombé par surprise en quelque péché, vous qui êtes spirituels, reprenez-le avec douceur, songeant que vous pouvez être tentés de même » ; et ailleurs : « Que le soleil ne se couche point sur votre colère » ; et dans l’Évangile : « Lorsque votre frère vous a offensé, reprenez-le en particulier entre vous et lui. » L’Apôtre dit aussi, à l’occasion des péchés oùl’on craint le scandale : « Reprenez devant tout le monde ceux qui ont commis quelque crime, afin de donner de la crainte aux autres. » L’Écriture recommande vivementpour cette raison le pardon des injures, afin d’entretenir la paix, sans laquelle personnene pourra voir Dieu. De là ce terrible jugement contre ce serviteur que l’on condamneà payer les dix mille talents qui lui avaient été remis, parce qu’il n’en avait pas vouluremettre cent à un autre serviteur comme lui. Après cette parabole, Notre-Seigneur Jésus-Christ ajouta : « Ainsi vous traitera votre Père qui est dans les cieux, si chacun de vous ne pardonne à son frère du fond du cœur. » Voilà comme sont guéris les citoyens de la cité de Dieu, qui sont voyageurs ici-bas et qui soupirent après le repos de la céleste patrie. Mais c’est le Saint-Esprit qui opère au dedans et qui donne la vertu aux remèdes qu’on emploie au dehors. Quand Dieu lui-même se servirait des créatures qui lui sont soumises, pour nous parler en songes ou de toute autre manière, cela serait inutile, si en même temps il ne nous touchait l’âme d’une grâce intérieure. Or, il en use de la sorte lorsque, par un jugement très secret, mais très juste, il sépare des vases de colère les vases de miséricorde. Si, en effet, à l’aide du secours qu’il nous prête par des voies cachées et admirables, le péché qui habite dans nos membres, ou plutôt la peine du péché, ne règne point dans notre corps mortel, si, domptant ses désirs déréglés, nous ne lui abandonnons point nos membres pour accomplir l’iniquité, notre esprit acquiert dès ce moment un empire sur nos passions qui les rend plus modérées, jusqu’à ce que, parfaitement guéri et revêtu d’immortalité, il jouisse dans le ciel d’une paix souveraine.
\subsection[{Chapitre VII}]{Chapitre VII}

\begin{argument}\noindent La parole de Dieu ne détourna point Caïn de tuer son frère.
\end{argument}

\noindent Mais de quoi servit à Caïn d’être averti de tout cela par Dieu même, quand Dieu s’adressa à lui en lui parlant sous la forme dont il avait coutume de se servir pour parler aux premiers hommes ? En accomplit-il moins le fratricide qu’il méditait ? Comme Dieu avait discerné les sacrifices des deux frères, agréant ceux de l’un parce qu’il était homme de bien, et rejetant ceux de l’autre à cause de sa méchanceté, Caïn, qui s’en aperçut sans doute par quelque signe visible, en ressentit un vif déplaisir et en fut tout abattu. Voici comment l’Écriture s’exprime à ce sujet : « Dieu dit à Caïn : Pourquoi êtes-vous triste et abattu ? Quand vous faites une offrande qui est bonne, mais dont le partage n’est pas bon, ne péchez-vous pas ? Tenez-vous en repos. Car il se tournera vers vous, et vous lui commanderez ». Dans cet avertissement que Dieu donne à Caïn, il n’est pas aisé de bien entendre ces mots : « Quand vous faites une offrande qui est bonne, mais dont le partage n’est pas bon, ne péchez-vous pas ? » C’est ce qui a donné lieu aux commentateurs d’en tirer divers sens. La vérité est que l’on offre bien le sacrifice, lorsqu’on l’offre au Dieu véritable à qui seul il est dû, mais on ne partage pas bien, lorsqu’on ne discerne pas comme il faut ou les lieux, ou les temps, ou les choses offertes, ou celui qui les offre, ou ceux à qui l’on fait part de l’offrande pour en manger. Ainsi, partage serait synonyme de discernement, soit quand on n’offre pas où il faut, ou ce qu’il y faut offrir, soit lorsqu’on offre dans un temps ce qu’il faudrait offrir dans un autre, ou qu’on offre ce qui ne doit être offert en aucun lieu ni en aucun temps, soit qu’on retienne pour soi le meilleur du sacrifice au lieu de l’offrir à Dieu, soit enfin qu’on en fasse part à un profane ou à quelque autre qu’il n’est pas permis d’y associer. Il est difficile de décider en laquelle de ces choses Caïn déplut à Dieu ; toutefois, comme l’Apôtre saint Jean dit, à propos de ces deux frères : « N’imitez pas Caïn qui était possédé du malin esprit, et qui tua son frère. Et pourquoi le tua-t-il ? parce que ses propres œuvres ne valaient rien, et que celles de son frère étaient bonnes » ; nous en pouvons conclure que les offrandes de Caïn n’attirèrent point les regards de Dieu, parce qu’il ne partageait pas bien et se réservait pour lui-même une partie de ce qu’il offrait à Dieu. C’est ce que font tous ceux qui n’accomplissent pas la volonté de Dieu, mais la leur, c’est-à-dire qui, n’ayant pas le cœur pur, offrent des présents à Dieu pour le corrompre, afin qu’il ne les aide pas à guérir leurs passions, mais à les satisfaire. Tel est proprement le caractère de la cité du monde, de servir Dieu ou les dieux pour remporter par leur secours des victoires sur ses ennemis et jouir d’une paix humaine, dans le désir non de faire du bien, mais de s’agrandir. Les bons se servent du monde pour jouir de Dieu, et les méchants au contraire veulent se servir de Dieu pour jouir du monde ; encore, je parle de ceux qui croient qu’il y a un Dieu et qu’il prend soin des choses d’ici-bas, car il en est même qui ne le croient pas. Lors donc que Caïn connut que Dieu n’avait point regardé son sacrifice et qu’il avait regardé celui de son frère, il devait imiter Abel et non pas lui porter envie ; mais la tristesse et l’abattement qu’il en ressentit constituent principalement le péché que Dieu reprit en lui, savoir de s’attrister de la bonté d’autrui, et surtout de celle de son frère. Ce fut le sujet de la réprimandequ’il lui adressa, quand il lui dit : « Pourquoi êtes-vous triste et abattu ? » Dieu voyait bien au fond qu’il portait envie à son frère, et c’est de quoi il le reprenait. En effet, comme les hommes ne voient pas le cœur, ils pourraient se demander si cette tristesse ne venait pas de ce qu’il était fâché d’avoir déplu à Dieu par sa mauvaise conduite, plutôt que du déplaisir de ce que Dieu avait regardé favorablement le sacrifice de son frère. Mais du moment que Dieu lui déclare pour quelle raison il n’avait pas voulu recevoir son offrande, et qu’il devait moins imputer ce refus à son frère qu’à lui-même, il fait voir que Caïn était rongé d’une secrète jalousie.\par
Comme Dieu ne voulait pas, après tout, l’abandonner sans lui donner quelque avis salutaire : « Tenez-vous en repos, lui dit-il ; car il se tournera vers vous, et vous lui commanderez. » Est-ce de son frère qu’il parle ? Non vraiment, mais bien de son péché, car il avait dit auparavant : « Ne péchez-vous pas ? » puis il ajoute : « Tenez-vous en repos ; car il se tournera vers vous, et vous lui commanderez. » On peut entendre par là que l’homme ne doit s’en prendre qu’à lui-même de ce qu’il pèche, et que le véritable moyen d’obtenir le pardon de son péché et l’empire sur ses passions, c’est de se reconnaître coupable ; autrement, celui qui prétend excuser le péché ne fera que le renforcer et lui donner plus de pouvoir sur lui. Le péché peut se prendre aussi en cet endroit pour la concupiscence de la chair, dont l’Apôtre dit : « La chair convoite contre l’esprit » car il met aussi l’envie au nombre de ses convoitises, et c’est elle qui anima Caïn contre son frère. D’après cela, ces paroles : « Il se tournera vers vous, et vous lui commanderez », signifieraient que la concupiscence nous sera soumise et que nous en deviendrons les maîtres. Lorsque, en effet, cette partie charnelle de l’âme que l’Apôtre appelle péché dans ce passage où il dit : « Ce n’est pas moi qui fais le mal, mais c’est le péché qui habite en moi », cette partie dont les philosophes avouent qu’elle est vicieuse et ne doit pas commander, mais obéir à l’esprit ; lors, dis-je, que cette partie charnelle est émue, si l’on pratique ce que prescrit l’Apôtre : « N’abandonnez point vos membres au péché pour lui servir d’instruments à mal faire », elle se tourne vers l’esprit et sesoumet à l’empire de la raison. C’est l’avertissement que Dieu donne à celui qui était transporté d’envie contre son frère, et qui voulait ôter du monde celui qu’il devait plutôt imiter « Tenez-vous en repos », lui dit-il, c’est-à-dire : Ne commettez pas le crime que vous méditez ; que le péché ne règne point en votre corps mortel, et n’accomplissez point ses désirs déréglés ; n’abandonnez point vos membres au péché pour lui servir d’instruments à mal faire ; car il se tournera vers vous, pourvu que, au lieu de le seconder, vous tâchiez de le réprimer, et vous aurez empire sur lui, parce que, lorsqu’on ne lui permet pas d’agir au dehors, il s’accoutume à ne se plus soulever au dedans contre la raison. On voit au même livre de la Genèse qu’il en est à peu près de même pour la femme, quand, après le péché, le diable reçut l’arrêt de sa condamnation dans le serpent, et Adam et Ève dans leur propre personne. Après que Dieu eut dit à Ève : « Je multiplierai les sujets de vos peines et de vos gémissements, et vous enfanterez avec douleur », il ajoute : « Et vous vous tournerez vers votre mari, et il aura empire sur vous. » Ce qui est dit ensuite à Caïn du péché ou de la concupiscence de la chair, est dit ici de la femme pécheresse, pour montrer que le mari doit gouverner sa femme comme l’esprit gouverne la chair. C’est ce qui fait dire à l’Apôtre : « Celui qui aime sa femme s’aime soi-même ; car jamais personne ne hait sa propre chair. » Il faut donc guérir ces maux comme étant véritablement en nous, au lieu de les condamner comme s’ils ne nous appartenaient pas. Mais Caïn, qui était déjà corrompu, ne tint aucun compte de l’avertissement de Dieu, et, l’envie se rendant maîtresse de son cœur, il égorgea perfidement son frère. Voilà ce qu’était le fondateur de la cité de la terre. Quant à considérer Caïn comme figurant aussi les Juifs qui ont fait mourir Jésus-Christ, ce grand Pasteur des âmes, représenté par Abel, pasteur de brebis, je n’en veux rien faire ici, et je me souviens d’en avoir touché quelque chose contre Fauste le Manichéen.
\subsection[{Chapitre VIII}]{Chapitre VIII}

\begin{argument}\noindent Quelle raison porta Caïn à bâtir une ville dès le commencement du monde.
\end{argument}

\noindent J’aime mieux maintenant défendre la vérité de l’Écriture contre ceux qui prétendent qu’il n’est pas croyable qu’un seul homme ait bâti une ville, parce qu’il semble qu’il n’y avait encore alors que quatre hommes sur la terre, ou même trois depuis le meurtre d’Abel, savoir : Adam, Caïn et son fils Énoch, qui donna son nom à cette ville. Ceux qui raisonnent de la sorte ne considèrent pas que l’auteur de l’Histoire sainte n’était pas obligé de mentionner tous les hommes qui pouvaient exister alors, mais seulement ceux qui servaient à son sujet. Le dessein de l’écrivain, qui servait en cela d’organe au Saint-Esprit, était de descendre jusqu’à Abraham par la suite de certaines générations, et de venir des enfants d’Abraham au peuple de Dieu, qui, séparé de tous les autres peuples de la terre, devait annoncer en figure tout ce qui regardait la cité dont le règne sera éternel, et Jésus-Christ son roi et son fondateur, sans néanmoins oublier l’autre société d’hommes que nous appelons la cité de la terre, et d’en dire autant qu’il fallait pour rehausser par cette opposition l’éclat de la cité de Dieu. En effet, lorsque l’Écriture sainte rapporte le nombre des années de la vie de ces premiers hommes, et conclut toujours ainsi de chacun d’eux : « Et il engendra des fils et des filles, et un tel vécut tant de temps, et puis il mourut » ; dira-t-on, sous prétexte qu’elle ne nomine pas ces fils et ces filles, que, pendant un si grand nombre d’années qu’on vivait alors, il n’ait pu naître assez d’hommes pour bâtir même plusieurs villes ? Mais il était de l’ordre de la providence de Dieu, par l’inspiration duquel ces choses ont été écrites, de distinguer d’abord ces deux sociétés : d’une part les générations des hommes, c’est-à-dire de ceux qui vivaient selon l’homme, et de l’autre, les générations des enfants de Dieu, en allant jusqu’au déluge où tous les hommes furent noyés, excepté Noé et sa femme, avec leurs trois fils et leurs trois brus, huit personnes qui méritèrent seules d’échapper dans l’arche à cette ruine universelle.\par
Lors donc qu’il est écrit : « Caïn connut sa femme, et elle enfanta Énoch, et il bâtit uneville du nom de son fils Énoch », il ne s’ensuit pas qu’Énoch ait été son premier fils. L’Écriture dit la même chose d’Adam, lorsqu’il engendra Seth : « Adam, dit-elle, connut Ève sa femme, et elle conçut et enfanta un fils qu’elle nomma Seth » ; et cependant, Adam avait déjà engendré Caïn et Abel. Il ne s’ensuit pas non plus, de ce qu’Énoch donne son nom à la ville bâtie par Caïn, qu’il ait été son premier-né. Il se pouvait qu’il l’aimât plus que ses autres enfants. En effet, Juda, qui donna son nom à la Judée et aux Juifs, n’était pas l’aîné des enfants de Jacob. Mais quand Énoch serait le fils aîné de Caïn, il n’en faudrait pas conclure qu’il ait donné son nom à cette ville dès qu’il fut né ; car un seul homme ne pouvait pas faire une ville, qui n’est autre chose qu’une multitude d’hommes unis ensemble par quelque lien de société. Il faut croire plutôt que, la famille de Caïn s’étant si fort accrue qu’elle formait un peuple, il bâtit une ville et l’appela du nom de son aîné. Dans le fait, la vie de ces premiers hommes était si longue, quo celui qui a le moins vécu avant le déluge, selon le témoignage de l’Écriture, a vécu sept cent cinquante-trois ans. Plusieurs même ont passé neuf cents ans, quoique aucun n’ait été jusqu’à mille. Qui peut donc douter que, pendant la vie d’un seul homme, le genre humain n’ait pu tellement se multiplier qu’il ait été suffisant pour peupler plusieurs villes ? Cela se peut facilement conjecturer, puisque le peuple hébreu, sorti du seul Abraham, s’accrut de telle façon, en l’espace d’un peu plus de quatre cents ans, qu’à la sortie d’Égypte l’Écriture compte jusqu’à six cent mille hommes capables de porter les armes, pour ne rien dire des Iduméens qui sortirent d’Ésaü, petit-fils d’Abraham, ni de plusieurs autres nations issues du même Abraham, mais non pas par sa femme Sarra.
\subsection[{Chapitre IX}]{Chapitre IX}

\begin{argument}\noindent Les hommes vivaient plus longtemps et étaient plus grands avant le déluge que depuis.
\end{argument}

\noindent Il n’est donc point d’esprit judicieux quidoute que Caïn n’ait pu bâtir une ville, même fort grande, dans un temps où la vie des hommes était si longue, à moins qu’on ne veuille encore discuter là-dessus et prétendre qu’il n’est pas vrai qu’ils aient vécu aussi longtemps que l’Écriture le rapporte. Une chose encore que les incrédules se refusent à croire, c’est que les hommes fussent alors beaucoup plus grands qu’ils ne sont aujourd’hui. Cependant le plus célèbre de leurs poètes, Virgile, à propos d’une grosse pierre qui servait de borne à un champ et qu’un homme très robuste des temps anciens leva dans le combat et lança en courant contre son ennemi, s’exprime ainsi :\par
 {\itshape « À peine douze hommes de nos jours, choisis parmi les plus forts, l’auraient-ils pu porter. »} \par
Par où il veut montrer que la terre produisait alors des hommes bien plus grands qu’à présent. Combien donc l’étaient-ils encore davantage dans les premiers âges du monde avant le déluge ? Mais les sépulcres, découverts par la suite des années ou par des débordements de fleuves et autres accidents, où l’on a trouvé des ossements d’une grandeur incroyable, doivent convaincre les plus opiniâtres. J’ai vu moi-même, sur le rivage d’Utique, et plusieurs l’ont vue avec moi, une dent mâchelière d’homme, si grosse qu’on en eût pu faire cent des nôtres : elle avait appartenu, je crois, à quelque géant ; car si les hommes d’alors étaient généralement plus grands que nous, ils l’étaient moins que les géants. Aussi bien, dans tous les temps et même au nôtre, des phénomènes de ce genre n’ont pas cessé de se produire. Pline, ce savant homme, assure que plus le temps avance dans sa marche, plus les corps diminuent ; et il ajoute que c’est une chose dont Homère se plaint souvent. Mais, comme j’ai déjà dit, les os que l’on découvre quelquefois dans de vieux monuments peuvent justifier la grandeur descorps des premiers hommes, tandis que l’on ne saurait prouver de même la durée de leur vie, parce que personne ne vit plus aussi longtemps. Cependant cela ne doit pas empêcher d’ajouter foi à l’Histoire sainte, puisqu’il y aurait d’autant plus d’imprudence à ne pas croire ce qu’elle nous raconte du passé, que nous voyons de nos yeux l’accomplissement de ce qu’elle a prédit de l’avenir. Le même Pline dit toutefois qu’il existe encore une nation où l’on vit deux cents ans. Si donc quelques pays qui nous sont inconnus conservent encore des restes de cette longue vie dont nous n’avons pas d’expérience, pourquoi ne croirions-nous pas aussi qu’il y a eu des temps où l’on vivait autant que l’Écriture le témoigne ? S’il est croyable que ce qui n’est point ici soit ailleurs, pourquoi serait-il incroyable que ce qui n’est pas maintenant ait été autrefois ?
\subsection[{Chapitre X}]{Chapitre X}

\begin{argument}\noindent De la diversité qui se rencontre entre les livres hébreux et les Septante quant au nombre des années des premiers hommes.
\end{argument}

\noindent Ainsi, bien qu’il semble qu’il y ait quelque diversité, quant au nombre des années, entre les livres hébreux et les nôtres, sans que je sache d’où elle provient, elle n’est pas telle néanmoins qu’ils ne s’accordent touchant la longue vie des hommes de ce temps-là. Nos livres portent qu’Adam engendra Seth à l’âge de deux cent trente ans, et ceux des Hébreux à l’âge de cent trente ; mais aussi, selon les leurs, il vécut huit cents ans depuis, au lieu que, selon les nôtres, il n’en vécut que sept cents ; et ainsi ils conviennent dans la somme totale. Il en est de même des autres générations ; les cent années que les Hébreux comptent de moins que nous avant qu’un père ait engendré un tel qu’ils nomment, ils les reprennent ensuite, en sorte que cela revient au même. Dans la sixième génération, il n’y a aucune diversité. Pour la septième, il y a la même que dans les cinq premières, et elle s’accorde aussi de même. La huitième n’estpas plus difficile à accorder. Il est vrai que, suivant les Hébreux, Énoch, lorsqu’il engendra Mathusalem, avait vingt ans de plus que nous ne lui en donnons ; mais aussi lui en donnent-ils vingt de moins lorsqu’il l’eut engendré. Ce n’est que dans la neuvième génération, c’est-à-dire dans les années de Lamech, fils de Mathusalem et père de Noé, qu’il se rencontre quelque différence dans la somme totale ; encore n’est-elle pas considérable, puisqu’elle se borne à vingt-quatre années d’existence que les Hébreux donnent de plus que nous à Lamech ils lui attribuent six ans de moins que nous avant qu’il engendrât Noé, et trente de plus que nous après qu’il l’eût engendré ; de sorte que, rabattant ces six ans, restent vingt-quatre.
\subsection[{Chapitre XI}]{Chapitre XI}

\begin{argument}\noindent Il faut, d’après l’âge de Mathusalem, qu’il ait encore vécu quatorze ans après le déluge.
\end{argument}

\noindent Cette diversité entre les livres hébreux et les nôtres a fait mettre en question si Mathusalem a vécu quatorze ans après le déluge, tandis que l’Écriture ne parle que de huit personnes qui turent sauvées par le moyen de l’arche, entre lesquelles elle ne compte point Mathusalem. Selon les Septante, Mathusalem avait soixante-sept ans lorsqu’il engendra Lamech, et Lamech cent quatre-vingt-huit ans avant d’engendrer Noé, ce qui fait ensemble trois cent cinquante-cinq ans ; ajoutez-y les six cents ans de Noé avant le déluge, cela fait neuf cent cinquante-cinq ans depuis la naissance de Mathusalem jusqu’au déluge. Or, Mathusalem vécut en tout neuf cent soixante et neuf ans, cent soixante et sept avant que d’engendrer Lamech, et huit cent deux ans depuis par conséquent, il vécut quatorze ans après le déluge, qui n’arriva que la neuf cent cinquante-cinquième année de la vie de Mathusalem. De là vient que quelques-uns aiment mieux dire qu’il vécut quelque temps avec son père Énoch, que Dieu avait ravi hors du monde, que de demeurer d’accord qu’il y ait faute dans la version des Septante, à qui l’Église donne tant d’autorité ; et en conséquence ils prétendent que l’erreur est plutôt du côté des exemplaires hébreux. Ils allèguent,à l’appui de leur sentiment, qu’il n’est pas croyable que les Septante, qui se sont rencontrés mot pour mot dans leur version, aient pu se tromper ou voulu mentir sur un point qui n’était pour eux d’aucun intérêt, et qu’il est bien plus probable que les Juifs, jaloux de ce que la loi et les Prophètes sont venus à nous par le moyen de cette version, ont altéré leurs exemplaires afin de diminuer l’autorité des nôtres. Chacun peut croire là-dessus ce qui lui plaira ; toujours est-il certain que Mathusalem ne vécut point après le déluge, mais qu’il mourut la même année, si la chronologie des Hébreux est véritable. Pour les Septante, j’en dirai ce que j’en pense, lorsque je parlerai du temps auquel ils ont écrit. Il suffit, en ce qui touche la difficulté présente, que, selon les uns et les autres, les hommes d’alors aient vécu assez longtemps pour qu’il en soit né durant la vie de Caïn un nombre capable de constituer une ville.
\subsection[{Chapitre XII}]{Chapitre XII}

\begin{argument}\noindent De l’opinion de ceux qui croient que les années des anciens n’étaient pas aussi longues que les nôtres.
\end{argument}

\noindent Il ne faut point écouter ceux qui prétendent que l’on comptait alors les années autrement qu’à cette heure, et qu’elles étaient si courtes qu’il en fallait dix pour en faire une des nôtres. C’est pour cette raison, disent-ils, que, quand l’Écriture dit de quelqu’un qu’il vécut neuf cents ans, on doit entendre quatre-vingt-dix ans ; car dix de leurs années en font une des nôtres, et dix des nôtres en font cent des leurs. Ainsi, à leur compte, Adam n’avait que vingt-trois ans quand il engendra Seth, et Seth vingt ans et six mois quand il engendra Énos. Selon cette opinion, les anciens divisaient une de nos années en dix parties, chacune valant pour eux une année et étant composée d’un sénaire carré, parce que Dieu acheva ses ouvrages en six jours et se reposa le septièmes. Or, le sénaire carré, ou six fois six, est de trente-six, qui, multipliés par dix, font trois cent soixante jours, c’est-à-dire douze mois lunaires. Quant aux cinq jours qui restaient pour accomplir l’année solaire, et aux six heures qui sont cause que tous les quatre ans nous avons une année bissextile, les anciens suppléaient de temps en temps quelques jours afin de compléter le nombre des années, et les Romains appelaient ces jours intercalaires. De même Énos, fils de Seth, n’avait que dix-neuf ans quand il engendra Caïnan ; ce qui revient aux quatre-vingt-dix ans que lui donne l’Écriture. Aussi, poursuivent-ils, nous ne voyons point, selon les Septante, qu’aucun homme ait engendré avant le déluge qu’il n’eût au moins cent soixante ans, c’est-à-dire seize ans, en comptant dix années pour une, parce que c’est l’âge destiné par la nature pour avoir des enfants. À l’appui de leur opinion, ils ajoutent que la plupart des historiens rapportent que l’année des Égyptiens était de quatre mois, celle des Acarnaniens de six, et celle des Laviniens de treize. Pline le naturaliste, à propos de quelques personnes que certaines histoires témoignent avoir vécu jusqu’à huit cents ans, pense que cette assertion tient à l’ignorance de ces temps-là ; attendu, dit-il, que certains peuples ne faisaient leur année que d’un été et d’un hiver, et que les autres comptaient les quatre saisons de l’année pour quatre ans, comme les Arcadiens dont les années n’étaient que de trois mois. Il ajoute même que les Égyptiens, dont nous avons dit que les années n’étaient composées que de quatre mois, les réglaient quelquefois sur le cours de la lune, tellement que chez eux on vivait jusqu’à mille ans.\par
Telles sont les raisons sur lesquelles se fondent des critiques dont le dessein n’est pas d’ébranler l’autorité de l’Écriture, mais plutôt de l’affermir en empêchant que ce qu’elle rapporte de la longue vie des premiers hommes ne paraisse incroyable. Il est aisé de montrer évidemment que tout cela est très faux ; mais, avant que de le faire, je suis bien aise de me servir d’une autre preuve pour réfuter cette opinion. Selon les Hébreux, Adam n’avait que cent trente ans lorsqu’il engendra son troisième fils. Or, si ces cent trente ans ne reviennent qu’à treize des nôtres, il est certain qu’il n’en avait que onze ou peu davantage quand il eut le premier. Or, qui peut engendrer à cet âge-là selon la loi ordinaire de la nature ? Mais, sans parler de lui, qui peut-être fut capable d’engendrer dès qu’il fut créé, car il n’est pas croyable qu’il ait été créé aussi petit que nos enfants lorsqu’ils viennent au monde, son fils, d’après les mêmes Hébreux, n’avait que cent cinq ans quand il engendra Énos, et par conséquent il n’avait pas encore onze ans, selon nos adversaires. Que dirai-je de son fils Caïnan qui, suivant le texte hébreu, n’avait que soixante et dix ans quand il engendra Malaléhel ? Comment engendrer à sept ans, si soixante et dix ans d’alors n’en font réellement que sept de nos jours ?
\subsection[{Chapitre XIII}]{Chapitre XIII}

\begin{argument}\noindent Si, dans la supputation des années, il faut plutôt s’arrêter aux textes hébreux qu’à la traduction des Septante.
\end{argument}

\noindent Je prévois bien ce que l’on me répliquera : que c’est une imposture des Juifs qui ont falsifié leurs exemplaires, comme nous l’avons dit plus haut, et qu’il n’est pas présumable que les Septante, ces hommes d’une renommée si légitime, aient pu en imposer. Cependant, si je demande lequel des deux est le plus croyable, ou que les Juifs, qui sont répandus en tant d’endroits différents, aient conspiré ensemble pour écrire cette fausseté, et qu’ils se soient privés eux-mêmes de la vérité pour ôter l’autorité aux autres, ou que les Septante, qui étaient aussi Juifs, assemblés en un même lieu par Ptolémée, roi d’Égypte, pour traduire l’Écriture, aient envié la vérité aux Gentils et concerté ensemble cette imposture, qui ne devine la réponse que l’on fera à ma question ? Mais à Dieu ne plaise qu’un homme sage s’imagine que les Juifs, quelque méchants et artificieux qu’on les suppose, aient pu glisser cette fausseté dans un si grand nombre d’exemplaires dispersés en tant de lieux, ou que les Septante, qui ont acquis une si haute réputation, se soient accordés entre eux pour ravir la vérité aux Gentils. Il est donc plus simple de dire que, quand on commença à transcrire ces livres de la bibliothèque de Ptolémée, cette erreur se glissa d’abord dans un exemplaire par la faute du copiste et passa de la sorte dans tous les autres. Cette réponse est assez plausible pour ce qui regarde la vie de Mathusalem et pour les vingt-quatre années qui se rencontrent de plus dans les exemplaires hébreux. À l’égard des cent années qui sont d’abord en plus dans les Septante, et ensuite en moins pour faire cadrer la somme totale avec le nombre des années du texte hébreu, et cela dans les cinq premières générations et dans la septième, c’est une erreur trop uniforme pour l’imputer au hasard.\par
Il est plus présumable que celui qui a opéré ce changement, voulant persuader que les premiers hommes n’avaient vécu tant d’années que parce qu’elles étaient extrêmement courtes et qu’il en fallait dix pour en faire une des nôtres, a ajouté cent ans d’abord aux cinq premières générations et à la septième, parce qu’eu suivant l’hébreu, les hommes eussent encore été trop jeunes pour avoir des enfants, et les a retranchés ensuite pour retrouver le compte juste des années. Ce qui porte encore plus à croire qu’il en a usé de la sorte dans ces générations, c’est qu’il n’a pas fait la même chose dans la sixième, parce qu’il n’en était pas besoin, et que Jared, selon les textes hébreux, avait cent soixante et deux ans lorsqu’il engendra Énoch, c’est-à-dire seize ans et près de deux mois, âge auquel on peut avoir des enfants.\par
Mais, d’un autre côté, on pourrait demander pourquoi, dans la huitième génération, tandis que l’hébreu donne cent quatre-vingt-deux ans à Mathusalem avant qu’il engendrât Lamech, la version des Septante lui en retranche vingt, au lieu qu’ordinairement elle en donne cent de plus que l’hébreu aux patriarches, avant que de les faire engendrer, On pourrait penser peut-être que cela est arrivé par hasard, si, après avoir ôté vingt années à Mathusalem, il ne les lui redonnait ensuite, afin de trouver le compte des années de sa vie. Ne serait-ce point une manière adroite de couvrir les additions précédentes de cent années, par le retranchement d’un petit nombre d’autres qui n’était pas d’importance, puisque, malgré cela, Mathusalem aurait toujours eu cent soixante-deux ans, c’est-à-dire plus de seize ans, avant que d’engendrer Lamech ? Quoi qu’il en soit, je ne doute point que, lorsque les exemplaires grecs ou hébreux ne s’accordent pas, il ne faille plutôt suivre l’hébreu, comme l’original, que les Septante, qui ne sont qu’une version, attendu surtout que quelques exemplaires grecs, un latin et un syriaque s’accordent en ce point, que Mathusalem mourut six ans avant le déluge.
\subsection[{Chapitre XIV}]{Chapitre XIV}

\begin{argument}\noindent Les années étaient autrefois aussi longues qu’à présent.
\end{argument}

\noindent Je vais maintenant prouver jusqu’à l’évidence que durant le premier âge du monde les années n’étaient pas tellement courtes qu’il en fallût dix pour en faire une des nôtres, mais qu’elles égalaient en durée celles d’aujourd’hui que règle le cours du soleil. Voici en effet ce que porte l’Écriture : « Le déluge arriva sur la terre l’an 600 de la vie de Noé, au second mois, le vingt-septième jour du mois. » Comment s’exprimerait-elle de la sorte si les années des anciens n’avaient que trente-six jours ? Dans ce cas, ou ces années n’auraient point eu de mois, ou les mois n’auraient été que de trois jours, pour qu’il s’en trouvât douze dans l’année. N’est-il pas visible que leurs mois étaient comme les nôtres, puisque, autrement, l’Écriture sainte ne dirait pas que le déluge arriva le vingt-septième jour du second mois ? Elle dit encore un peu après, à la fin du déluge : « L’arche s’arrêta sur les montagnes d’Ararat le septième mois, le vingt-septième jour du mois. Cependant les eaux diminuaient jusqu’à l’onzième mois ; or, le premier jour de ce mois, on vit paraître les sommets des montagnes. » Que si leurs mois étaient semblables aux nôtres, il faut étendre cette similitude à leurs années. Ces mois de trois jours n’en pouvaient pas avoir vingt-sept ; ou si la trentième partie de ces trois jours s’appelait alors un jour, un si effroyable déluge qui, selon l’Écriture, tomba durant quarante jours et quarante nuits, se serait donc fait en moins de quatre de nos jours. Qui pourrait souffrir une si palpable absurdité ? Loin, bien loin de nous cette erreur qui ruine la foi des Écritures sacrées, en voulant l’établir sur de fausses conjectures ! Il est certain que le jour était aussi long alors qu’à présent, c’est-à-dire de vingt-quatre heures, les mois égaux aux nôtres et réglés sur le cours de la lune, et les années composées de douze mois lunaires, en y ajoutant cinq jours et un quart, pour les ajuster aux années solaires, et par conséquent ces premiers hommes vécurent plus de neuf cents années, lesquelles étaient aussi longues que les cent soixante-quinze que vécut ensuite Abraham,que les cent quatre-vingts que vécut Isaac, que les cent quarante ou environ que vécut Jacob, que les cent vingt que vécut Moïse, et que les soixante-dix ou quatre-vingts que les hommes vivent aujourd’hui et dont il est dit : « Si les plus robustes vont jusqu’à quatre-vingts ans, ils en ont d’autant plus de mal. »\par
Quant à la différence qui se rencontre entre les exemplaires hébreux et les nôtres, elle ne concerne point du tout la longueur de la vie des premiers hommes, sur quoi les uns et les autres conviennent ; ajoutez à cela que, lorsqu’il y a diversité, il faut plutôt s’en tenir à la langue originale qu’à une version. Cependant, ce n’est pas sans raison que personne n’a encore osé corriger les Septante sur l’hébreu, en plusieurs endroits où ils semblaient différents. Cela prouve qu’on n’a pas cru que ce défaut de concordance fût une faute, et je ne le crois pas non plus ; mais, à la réserve des erreurs de copiste, lorsque le sens est conforme à la vérité, ou doit croire que les Septante ont changé le sens du texte, non en qualité d’interprètes qui se trompent, mais comme des prophètes inspirés par l’esprit de Dieu. De là vient que, lorsque les Apôtres allèguent quelques témoignages de l’Ancien Testament dans leurs écrits, ils ne se servent pas seulement de l’hébreu, mais de la version des Septante. Comme j’ai promis de traiter plus amplement cette matière au lieu convenable, où je pourrai le faire plus commodément, je reviens à mon sujet, et dis qu’il ne faut point douter que le premier des enfants du premier homme n’ait pu bâtir une cité à une époque où la vie des hommes était si longue : cité, au reste, bien différente de celle que nous appelons la Cité de Dieu, pour laquelle nous avons entrepris ce grand ouvrage.
\subsection[{Chapitre XV}]{Chapitre XV}

\begin{argument}\noindent S’il est présumable que les hommes du premier âge aient persévéré dans l’abstinence jusqu’à l’époque où l’on rapporte qu’ils ont eu des enfants.
\end{argument}

\noindent Est-il croyable, dira-t-on, qu’un homme, qui n’avait pas dessein de garder le célibat, se soit contenu cent ans et plus, ou, selon l’hébreu, quatre-vingts, soixante-dix ou soixante ans, et qu’il n’ait point eu d’enfantsauparavant ? Il y a deux réponses à cela. Ou l’âge d’avoir des enfants venait plus tard en ce temps-là, à proportion des années de la vie ; où, ce qui me paraît plus vraisemblable, l’Écriture n’a pas fait mention des aînés, mais seulement de ceux dont il fallait parler selon l’ordre des générations, pour parvenir à Noé et ensuite à Abraham, et pour marquer le progrès de la glorieuse Cité de Dieu, étrangère ici-bas et qui soupire après la céleste patrie. En effet, on ne saurait nier que Caïn ne soit le premier fils d’Adam, puisque Adam n’aurait pas dit, comme le lui fait dire l’Écriture : « J’ai acquis un homme par la grâce de Dieu », si cet homme n’avait été ajouté en naissant à nos deux premiers parents. Abel vint après, qui fut tué par son frère Caïn, en quoi il fut la première figure de la Cité de Dieu, exilée en ce monde et destinée à être en butte aux injustes persécutions des méchants, c’est-à-dire des hommes du siècle attachés aux biens passagers de la cité de la terre ; mais on ne voit pas à quel âge Adam les engendra l’un et l’autre. Ensuite sont rapportées les deux branches d’hommes, l’une sortie de Caïn, et l’autre de Seth, que Dieu donna à Adam à la place d’Abel. Ainsi ces deux ordres de générations, l’une de Seth et l’autre de Caïn, marquant distinctement les deux cités dont nous parions, l’Écriture sainte ne dit point quel âge avaient ceux de la race de Caïn quand ils eurent des enfants, parce que l’esprit de Dieu n’a jugé dignes de cet honneur que ceux qui représentaient la Cité du ciel. La Genèse, à la vérité, marque à quel âge Adam engendra Seth, mais il en avait engendré d’autres auparavant, savoir : Caïn et Abel ; qui sait même s’il n’avait engendré que ceux-là ? De ce qu’ils sont nommés seuls à cause des généalogies qu’il fallait établir, ce n’est pas à dire qu’Adam n’en ait point eu d’autres. Aussi bien, lorsque l’Écriture sainte dit en général qu’il engendra des fils et des filles qu’elle ne nomme pas, qui oserait sans témérité en déterminer le nombre ? Ce qu’Adam dit après la naissance de Seth : « Dieu m’a donné un autre fils au lieu d’Abel », il a pu fort, bien le dire par une inspiration divine, en tant que Seth devait imiter la vertu d’Abel, et non en tant qu’il fut né immédiatement après lui. De même, quand il est écrit : « Seth avait deux cent cinq ans », ou, selon l’hébreu, cent cinq, lorsqu’il engendra Énos, qui serait assez hardi pour assurer qu’Énos fût son premier-né ? Outre qu’il n’y a point d’apparence qu’il se soit contenu pendant tant d’années, n’ayant point dessein de garder la continence. L’Écriture dit aussi de lui : « Et il engendra des fils et des filles, et Seth vécut en tout neuf cent douze ans. » L’Écriture, qui ne se proposait, comme je l’ai déjà dit, que de descendre jusqu’à Noé par une suite de générations, n’a pas marqué celles qui étaient les premières, mais celles où cette suite était gardée.\par
J’appuierai ces considérations d’un exemple clair et indubitable. Saint Matthieu, faisant la généalogie temporelle de Notre-Seigneur, et commençant par Abraham pour venir d’abord à David : « Abraham, dit-il, engendra Isaac. » Que ne dit-il Ismaël, qui fut le fils aîné d’Abraham ? « Isaac, ajoute-t-il, engendra Jacob. » Pourquoi ne dit-il pas Ésaü, qui fut son aîné ? C’est sans doute qu’il ne pouvait pas arriver par eux à David. Poursuivons : « Jacob engendra Juda et ses frères. » Est-ce que Juda fut l’aîné des enfants de Jacob ? « Juda », dit-il encore, « engendra Pharès et Zaram. » Et cependant il avait déjà eu trois enfants avant ceux-là. Voilà l’unique et irrécusable solution qu’il faut apporter à ces difficultés de la Genèse, sans aller s’embarrasser dans cette question obscure et superflue, si les hommes avaient en ce temps-là des enfants plus tard qu’aujourd’hui.
\subsection[{Chapitre XVI}]{Chapitre XVI}

\begin{argument}\noindent Des mariages entre proches, permis autrefois à cause de la nécessité.
\end{argument}

\noindent Le besoin qu’avait le monde d’être peuplé, et le défaut d’autres hommes que ceux qui étaient sortis de nos premiers parents, rendirent indispensables entre frères et sœurs des mariages qui seraient maintenant des crimes énormes, à cause de la défense que la religion en a faite depuis. Cette défense est fondée sur une raison très juste, puisqu’il est nécessaire d’entretenir l’amitié et la société parmi les hommes ; or, ce but est mieux atteint par les alliances entre étrangers que par celles qui unissent les membres d’une même famille, lesquels sont déjà unis par les liens du sang. Père et beau-père sont desnoms qui désignent deux alliances. Lors donc que ces qualités sont partagées entre différentes personnes, l’amitié s’étend et se multiplie davantage. Adam était obligé de les réunir en lui seul, parce que ses fils ne pouvaient épouser que leurs sœurs ; Ève, de même, était à la fois la mère et la belle-mère de ses enfants, comme les femmes de ses fils étaient ensemble ses filles et ses brus. La nécessité, je le répète, excusait alors ces sortes de mariages.\par
Depuis que les hommes se sont multipliés, les choses ont bien changé sous ce rapport, même parmi les idolâtres. Ces alliances ont beau être permises en certains pays, une plus louable coutume a proscrit cette licence, et nous en avons autant d’horreur que si cela ne s’était jamais pratiqué. Véritablement la coutume fait une merveilleuse impression sur les esprits ; et, comme elle sert ici à arrêter les excès de la convoitise, on ne saurait la violer sans crime. S’il est injuste de remuer les bornes des terres pour envahir l’héritage d’autrui, combien l’est-il plus de renverser celles des bonnes mœurs par des unions illicites ? Nous avons éprouvé, même de notre temps, dans le mariage des cousins germains, combien il est rare que l’on suive la permission de la loi, lorsqu’elle est opposée à la coutume. Bien que ces mariages ne soient point défendus par la loi de Dieu, et que celles des hommes n’en eussent point encore parlé, toutefois on en avait horreur à cause de la proximité du degré, et parce qu’il semble que ce soit presque faire avec une sœur ce que l’on fait avec une cousine germaine. Aussi voyons-nous que les cousins et les cousines à ce degré s’appellent frères et sœurs. Il est vrai que les anciens patriarches ont eu grand soin de ne pas trop laisser éloigner la parenté et de la rapprocher en quelque sorte par le lien du mariage, de sorte qu’encore qu’ils n’épousassent pas leurs sœurs, ils épousaient toujours quelque personne de leur famille. Mais qui peut douter qu’il ne soit plus honnête de nos jours de défendre le mariage entre cousins germains, non seulement pour les raisons que nous avons alléguées, afin de multiplier les alliances et n’en pas mettre plusieurs en une seule personne, mais aussi parce qu’une certaine pudeur louable fait que nous avons naturellement honte de nous unir, même par mariage, aux personnes pour qui la parenté nous donne du respect.\par
Ainsi l’union de l’homme et de la femme est comme la pépinière des villes et des cités ; mais la cité de la terre se contente de la première naissance des hommes, au lieu que la Cité du ciel en demande une seconde pour effacer la corruption de la première. Or, l’Histoire sainte ne nous apprend pas si, avant le déluge, il y a eu quelque signe visible et corporel de cette régénération, comme fut depuis la circoncision. Elle rapporte toutefois que les premiers hommes ont fait des sacrifices à Dieu, comme cela se voit clairement par ceux de Caïn et d’Abel, et par celui de Noé au sortir de l’arche ; et nous avons dit à ce sujet, dans les livres précédents, que les démons qui veulent usurper la divinité et passer pour dieux n’exigent des hommes ces sortes d’honneurs que parce qu’ils savent bien qu’ils ne sont dus qu’au vrai Dieu.
\subsection[{Chapitre XVII}]{Chapitre XVII}

\begin{argument}\noindent Des deux chefs de l’une et l’autre Cité issus du même père.
\end{argument}

\noindent Comme Adam était le père de ces deux sortes d’hommes, tant de ceux qui appartiennent à la cité de la terre que de ceux qui composent la Cité du ciel, après la mort d’Abel, qui figurait un grand mystère, il y eut deux chefs de chaque cité, Caïn et Seth, dans la postérité de qui l’on voit paraître des marques plus évidentes de ces deux cités. En effet, Caïn engendra Énoch et bâtit une cité de son nom, laquelle n’était pas étrangère ici-bas, mais citoyenne du monde, et mettait son bonheur dans la possession paisible des biens temporels. Or, Caïn veut dire {\itshape Possession}, d’où vient que quand il fut né, son père ou sa mère dit : « J’ai acquis un homme par la grâce de Dieu » ; et Énoch signifie {\itshape Dédicace}, à cause que la cité de la terre est dédiée en ce monde même où elle est fondée, parce que dès ce monde elle atteint le but de ses désirs et de ses espérances. Seth, au contraire, veut dire {\itshape Résurrection}, et Énos, son fils, signifie {\itshape Homme}, non comme Adam qui, en hébreu, est un nom commun à l’homme et à la femme, suivant cette parole de l’Écriture : « Il les créa homme et femme, et les bénit et les nomma Adam » ; ce qui fait voir qu’Ève s’appelait aussi Adam, d’un nom commun aux deux sexes. Mais Énos signifie tellement un homme, que ceux qui sont versés dans la langue hébraïque assurent qu’il ne peut pas être dit d’une femme ; Énos est en effet le fils de la résurrection, où il n’y aura plus de mariage ; car il n’y aura point de génération dans l’endroit où la génération nous aura conduits. Je crois, pour cette raison, devoir remarquer ici que, dans la généalogie de Seth, il n’est fait nommément mention d’aucune femme, au lieu que, dans celle de Caïn, il est dit : « Mathusalem engendra Lamech, et Lamech épousa deux femmes, l’une appelée Ada, et l’autre Sella, et Ada enfanta Jobel. Celui-ci fut le père des bergers, le premier qui habita dans des cabanes. Son frère s’appelait Jubal, l’inventeur de la harpe et de la cithare. Sella eut à son tour Thobel, qui travaillait en fer et en cuivre. Sa sœur s’appelait Noéma. » Là finit la généalogie de Caïn, qui est toute comprise en huit générations en comptant Adam, sept jusqu’à Lamech, qui épousa deux femmes, et la huitième dans ses enfants, parmi lesquels l’Écriture fait mention d’une femme. Elle insinue par là qu’il y aura des générations charnelles et des mariages jusqu’à la fin dans la cité de la terre ; et de là vient aussi que les femmes de Lamech, le dernier de la lignée de Caïn, sont désignées par leurs noms, distinction qui n’est point faite pour d’autres que pour Ève avant le déluge. Or, comme Caïn, fondateur de la cité de la terre, et son fils Énoch, qui nomma cette cité, marquent par leurs noms, dont l’un signifie possession et l’autre dédicace, que cette même cité a un commencement et une fin, et qu’elle borne ses espérances à ce monde-ci, de même Seth, qui signifie résurrection, étant le père d’une postérité dont la généalogie est rapportée à part, il est bon de voir ce que l’Histoire sainte dit de son fils.
\subsection[{Chapitre XVIII}]{Chapitre XVIII}

\begin{argument}\noindent Figure de Jésus-Christ et de son église dans Adam, Seth et Énos.
\end{argument}

\noindent « Seth », dit la Genèse, « eut un fils, qu’il appela Énos ; celui-ci mit son espérance à invoquer le nom du Seigneur. » Voilà le témoignage que rend la vérité. L’homme donc, fils de la résurrection, vit en espérance tant que la Cité de Dieu, qui naît de la foi dans la résurrection de Jésus-Christ, est étrangère en ce monde. La mort et la résurrection du Sauveur sont figurées par ces deux hommes, par Abel, qui signifie deuil, et par Seth, son frère, qui veut dire résurrection. C’est par la foi en Jésus ressuscité qu’est engendrée ici-bas la Cité de Dieu, c’est-à-dire l’homme qui a mis son espérance à invoquer le nom du Seigneur. « Car nous sommes sauvés par l’espérance, dit l’Apôtre : or, quand on voit ce qu’on avait espéré voir, il n’y a plus d’espérance ; car qui espère voir ce qu’il voit déjà ? Que si nous espérons voir ce que nous ne voyons pas encore, c’est la patience qui nous le fait attendre. » En effet, qui ne jugerait qu’il y a ici quelque grand mystère ? Abel n’a-t-il pas mis son espérance à invoquer le nom du Seigneur, lui dont le sacrifice fut si agréable à Dieu, selon le témoignage de l’Écriture ? Seth n’a-t-il pas fait aussi la même chose, lui dont il est dit : « Dieu m’a donné un autre fils à la place d’Abel ? » Pourquoi donc attribuer particulièrement à Énos ce qui est commun à tous les gens de bien, sinon parce qu’il fallait que celui qui naquit le premier du père des prédestinés à la Cité de Dieu figurât l’assemblée des hommes qui ne vivent pas selon l’homme dans la possession d’une félicité passagère, mais dans l’espérance d’un bonheur éternel ? Il n’est pas dit : Celui-ci espéra dans le Seigneur ; ou : Celui-ci invoqua le nom du Seigneur » ; mais : « Celui-ci mit son espérance à invoquer le nom du Seigneur. » Que signifie : « Mit son espérance à invoquer » si ce n’est l’annonce prophétique de la naissance d’un peuple qui, selon l’élection de la grâce, invoquerait le nom de Dieu ? C’est ce qui a été dit par un autre prophète ; et l’Apôtre l’explique de ce peuple qui appartient à la grâce de Dieu : « Tous ceux qui invoqueront le nom du Seigneur seront sauvés. » Ces paroles de l’Écriture : « Ill’appela Énos, c’est-à-dire l’homme », et ensuite : « Celui-ci mit son espérance à invoquer le nom du Seigneur », montrent bien que l’homme ne doit pas placer son espérance en lui-même. Comme il est écrit ailleurs « Maudit est quiconque met son espérance en l’homme » ; personne par conséquent ne doit non plus la mettre en soi-même, afin de devenir citoyen de cette autre cité qui n’est pas dédiée sur la terre par le fils de Caïn, c’est-à-dire pendant le cours de ce monde périssable, mais dans l’immortalité de la béatitude éternelle.
\subsection[{Chapitre XIX}]{Chapitre XIX}

\begin{argument}\noindent Ce que figure le ravissement d’Énoch.
\end{argument}

\noindent Cette lignée, dont Seth est le père, a aussi un nom qui signifie dédicace dans la septième génération depuis Adam, en y comprenant Adam lui-même. En effet, Énoch, qui signifie dédicace, est né le septième depuis lui ; mais c’est cet Énoch, si agréable à Dieu, qui fut transporté hors du monde, et qui, dans l’ordre des générations, tient un rang remarquable, en ce qu’il désigne le jour consacré au repos. Il est aussi le sixième, à compter depuis Seth, c’est-à-dire depuis le père de ces générations qui sont séparées de la lignée de Caïn. Or, c’est le sixième jour que l’homme fut créé et que Dieu acheva tous ses ouvrages. Mais le ravissement d’Énoch marque le délai de notre dédicace ; il est vrai qu’elle est déjà faite en Jésus-Christ, notre chef, qui est ressuscité pour ne plus mourir et qui a été lui-même transporté ; mais il reste une autre dédicace, celle de toute la maison dont Jésus-Christ est le fondateur, et celle-là est différée jusqu’à la fin des siècles, où se fera la résurrection de tous ceux qui ne mourront plus. Il n’importe au fond qu’on l’appelle la maison de Dieu, ou son temple, ou sa cité ; car nous voyons Virgile donner à la cité dominatrice par excellence le nom de la maison d’Assaracus, désignant ainsi les Romains, qui tirent leur origine de ce prince par les Troyens. Il les appelle aussi la maison d’Énée, parce que les Troyens, qui bâtirent dans la suite la ville de Rome, arrivèrent en Italie sous la conduite d’Énée. Le poète a imité en cela les saintes Lettres qui nomment le peuple nombreux des Israélites la maison de Jacob.
\subsection[{Chapitre XX}]{Chapitre XX}

\begin{argument}\noindent Comment la postérité de Caïn est renfermée en huit générations, et pourquoi Noé appartient à la dixième depuis Adam.
\end{argument}

\noindent Quelqu’un dira : Si celui qui a écrit cette histoire avait l’intention, dans le dénombrement de ces générations, de nous conduire d’Adam par Seth jusqu’à Noé, sous qui arriva le déluge, et de Noé à Abraham, auquel l’évangéliste saint Matthieu commence les générations qui mènent à Jésus-Christ, roi éternel de la Cité de Dieu, quel était son dessein dans le dénombrement de celles de Caïn, et jusqu’où prétendait-il aller ? Je réponds : jusqu’au déluge, où toute la race des habitants de la cité de la terre fut engloutie, mais réparée par les enfants de Noé. Quant à cette société d’hommes qui vivent selon l’homme, elle subsistera jusqu’à la fin du siècle dont Notre-Seigneur a dit : « Les enfants de ce siècle engendrent et sont engendrés. » Mais, pour la Cité de Dieu qui est étrangère en ce siècle, la régénération la conduit à un siècle dont les enfants n’engendrent ni ne sont engendrés. Ici-bas donc, il est commun à l’une ou à l’autre cité d’engendrer et d’être engendré, quoique la Cité de Dieu ait dès ce monde plusieurs milliers de citoyens qui vivent dans la continence ; mais l’autre en a aussi quelques-uns qui les imitent en cela, bien qu’ils soient dans l’erreur sur tout le reste. À cette société appartiennent aussi ceux qui, s’écartant de la foi, ont formé diverses hérésies, et qui, par conséquent, vivent selon l’homme et non selon Dieu. Les gymnosophistes des Indes qui, dit-on, philosophent nus au milieu des forêts, sont de ses citoyens ; et néanmoins ils s’abstiennent du mariage. Aussi la continence n’est-elle un bien que quand on la garde pour l’amour du souverain bien qui est Dieu. On ne voit pas toutefois que personne l’ait pratiquée avant le déluge, puisque Énoch même, ravi du monde pour son innocence, engendra des fils et des filles, et entre autres Mathusalem qui continue l’ordre des générations choisies.\par
Pourquoi compte-t-on un si petit nombre d’individus dans les générations de Caïn, si elles vont jusqu’au déluge et si les hommes ence temps-là étaient en état d’avoir des enfants d’aussi bonne heure qu’aujourd’hui ? Si l’auteur de la Genèse n’avait pas eu en vue quelqu’un auquel il voulût arriver par une suite de générations, comme c’était son dessein à l’égard de celle de la postérité de Seth, qu’il voulait conduire jusqu’à Noé, pour reprendre ensuite l’ordre des générations jusqu’à Abraham, qu’était-il besoin de passer les premiers-nés pour arriver à Lamech, auquel finit cette généalogie, c’est-à-dire à la huitième génération depuis Adam, et à la septième depuis Caïn, comme si de là il eût voulu passer à quelque autre généalogie pour arriver ou au peuple d’Israël, en qui la Jérusalem terrestre même a servi de figure à la Cité céleste, ou à Jésus-Christ comme homme, qui est le Dieu suprême élevé au-dessus de toutes choses, béni dans tous les siècles, et le fondateur et le roi, de la Jérusalem du ciel ; qu’était-il besoin, dis-je, d’en user de la sorte, attendu que toute la postérité de Caïn fut exterminée par le déluge ? Cela pourrait faire croire que ce sont les premiers-nés qui sont nommés dans cette généalogie. Mais pourquoi y a-t-il si peu de personnes, si, comme nous l’avons dit, les hommes avaient des enfants en ce temps-là d’aussi bonne heure qu’ils en ont à présent ? Supposé qu’ils eussent tous trente ans quand ils commencèrent à en avoir, comme il y a huit générations en comptant Adam et les enfants de Lamech, huit fois trente font deux cent quarante ans. Or, est-il croyable qu’ils n’aient point eu d’enfants tout le reste du temps jusqu’au déluge ? Et, s’ils en ont eu, pourquoi l’Écriture n’en fait-elle point mention ? Depuis Adam jusqu’au déluge, il s’est écoulé deux mille deux cent soixante-deux ans, selon nos livres, et mille six cent cinquante-six, selon les Hébreux. Lors donc que nous nous arrêterions à ce dernier nombre comme au véritable, si de mille six cent cinquante-six ans on retranche deux cent quarante, restent mille quatre cents ans et quelque chose de plus. Or, peut-on s’imaginer que la postérité de Caïn soit demeurée pendant tout ce temps-là sans avoir des enfants ?\par
Mais il faut se rappeler ici ce que nousavons dit, lorsque nous demandions comment il se peut faire que ces premiers hommes, qui n’avaient aucun dessein de garder la continence, se soient pu contenir si longtemps. Nous avons en effet montré qu’il y a deux moyens de résoudre cette difficulté : ou et disant que, comme ils vivaient si longtemps ils n’étaient pas sitôt en âge d’engendrer, et que les enfants dont il est parlé dans ces généalogies ne sont pas les aînés, mais ceux qui servirent à perpétuer l’ordre des générations, jusqu’au déluge. Si donc dans celles de Caïn l’auteur de la Genèse n’a pas eu cette intention comme dans celles de Seth, il faudra avoir recours à l’autre solution, et dire qu’en ce temps-là les hommes n’étaient capables d’avoir des enfants qu’après cent ans. Il se peut faire néanmoins que cette généalogie de Caïn n’aille pas jusqu’au déluge, et que l’Écriture sainte, pour quelque raison que j’ignore, ne l’ait portée que jusqu’à Lamech et à ses enfants. Indépendamment de cette réponse que les hommes avaient des enfants plus tard en ce temps-là, il se peut que la cité bâtie par Caïn ait étendu au loin sa domination et ait eu plusieurs rois de père en fils, les uns après les autres, sans garder l’ordre de primogéniture. Caïn a pu être le premier de ces rois ; son fils Énoch, qui donna le nom au siège de cet empire, le second ; le troisième, Gaïdad, fils d’Énoch ; le quatrième, Manihel, fils de Gaïdad ; le cinquième, Mathusaël, fils de Manihel ; et le sixième, Lamech, fils de Mathusaël, qui est le septième depuis Adam par Caïn. Il n’était pas nécessaire que les aînés succédassent à leurs pères ; le sort, ou le mérite, ou l’affection du père appelait indifféremment un de ses fils à la couronne. Rien ne s’oppose à ce que le déluge soit arrivé sous le règne de Lamech et l’ait fait périr avec les autres. Aussi voyons-nous que l’Écriture ne désigne pas un seul fils de Lamech, comme dans les générations précédentes, mais plusieurs, parce qu’il était incertain quel devait être son successeur, si le déluge ne fût point survenu.\par
Mais de quelque façon que l’on compte les générations de Caïn, ou par les aînés, ou par les rois, il me semble que je ne dois pas passer sous silence que Lamech, étant le septième en ordre depuis Adam, l’Écriture, qui lui donne trois fils et une fille, parle d’autant de ses enfants qu’il en faut pour accomplir le nombre onze, qui signifie le péché. En effet, comme la loi est comprise en dix commandements, d’où vient le mot décalogue, il est hors de doute que le nombre onze, qui passe celui de dix, marque la transgression de la loi, et par conséquent le péché. C’est pour cela que Dieu commanda de faire onze voiles de poil de chèvre dans le tabernacle du témoignage, qui était comme le temple portatif de son peuple pendant son voyage, attendu que cette étoffe fait penser aux péchés, à cause des boucs qui doivent être mis à la gauche. Aussi, lorsque nous faisons pénitence, nous nous prosternons devant Dieu couverts d’un cilice, comme pour dire avec le Psalmiste : « Mon péché est toujours présent devant moi. » La postérité d’Adam par le fratricide Caïn finit donc au nombre de onze, qui signifie le péché ; et ce nombre est fermé par une femme, dont le sexe a donné commencement au péché par lequel nous avons tous été assujettis à la mort. Et ce péché a été suivi d’une volupté charnelle qui résiste à l’esprit ; d’où vient que le nom de cette fille de Lamech signifie volupté. Mais le nombre dix termine les générations descendues d’Adam par Seth jusqu’à Noé. Ajoutez à ce nombre les trois fils de Noé, dont deux seulement furent bénis, et l’autre fut réprouvé à cause de ses crimes, vous aurez douze : nombre illustre dans les Patriarches et dans les Apôtres, et composé des parties du nombre sept multipliées l’une par l’autre, puisque trois fois quatre et quatre fois trois font douze. Dans cet état de choses, il nous reste à voir comment ces deux lignées, qui, par des générations distinctes, marquent les deux cités, l’une des hommes de la terre, et l’autre des élus, se sont ensuite tellement mêlées ensemble que tout le genre humain, à la réserve de huit personnes, a mérité de périr par le déluge.
\subsection[{Chapitre XXI}]{Chapitre XXI}

\begin{argument}\noindent L’écriture ne parle qu’en passant de la cité de la terre, et seulement en vue de celle du ciel.
\end{argument}

\noindent Il faut considérer d’abord pourquoi, dans le dénombrement des générations de Caïn, après que l’Écriture a fait mention d’Énoch, qui donna son nom à la ville que son père bâtit, elle les continue tout de suite jusqu’audéluge, où finit entièrement toute cette branche, au lieu qu’après avoir parlé d’Énos, fils de Seth, elle interrompt le fil de cette généalogie, en disant : « Voici la généalogie des hommes. Lorsque Dieu créa l’homme, il le créa à son image. Il les créa homme et femme, les bénit, et les appela Adam. » Il me semble que cette interruption a eu pour objet de recommencer le dénombrement des temps par Adam ; ce que l’Écriture n’a pas voulu faire à l’égard de la cité de la terre, comme si Dieu en parlait en passant plutôt qu’il n’en tient compte. Mais d’où vient qu’après avoir déjà nommé le fils de Seth, cet homme qui mit sa confiance à invoquer le nom du Seigneur, elle y revient encore, sinon de ce qu’il fallait représenter ainsi ces deux cités, l’une descendant d’un homicide jusqu’à un homicide, car Lamech avoue à ses deux femmes qu’il a tué un homme, et l’autre, fondée par celui qui mit sa confiance à invoquer le nom de Dieu ? Voilà, en effet, quelle doit être l’unique occupation de la Cité de Dieu, étrangère en ce monde pendant le cours de cette vie mortelle, et ce qu’il a fallu lui recommander par un homme engendré de celui en qui revivait Abel assassiné. Cet homme marque l’unité de toute la Cité céleste, qui recevra, un jour son accomplissement, après avoir été représentée ici-bas par cette figure prophétique. D’où le fils de Caïn, c’est-à-dire le fils de possession, pouvait-il prendre son nom, si ce n’est des biens de la terre dans la cité de la terre à qui il a donné le sien ? Il est de ceux dont il est dit dans le psaume : « Ils ont donné leurs noms à leurs terres » ; aussi tombent-ils dans le malheur dont il est parlé en un autre psaume : « Seigneur, vous anéantirez leur image dans votre cité. » Pour le fils de Seth, c’est-à-dire le fils de la résurrection, qu’il mette sa confiance à invoquer le nom du Seigneur ; c’est lui qui figure cette société d’hommes qui dit : « Je serai comme un olivier fertile en la maison du Seigneur, parce que j’ai espéré en sa miséricorde. » Qu’il n’aspire point à la vaine gloire d’acquérir un nom célèbre sur la terre ; car « heureux celui qui met son espérance au nom du Seigneur, et qui ne tourne point ses regards vers les vanités et les folies du monde ». Après avoir proposéces deux cités, l’une établie dans la jouissance des biens du siècle, l’autre mettant son espérance en Dieu, mais toutes deux sorties d’Adam comme d’une même barrière pour fournir leur course et arriver chacune à sa fin, l’Écriture commence le dénombrement des temps, auquel elle ajoute d’autres générations en reprenant depuis Adam, de la postérité de qui, comme d’une masse justement réprouvée, Dieu a fait des vases de colère et d’ignominie, et des vases d’honneur et de miséricorde traitant les uns avec justice et les autres avec bonté, afin que la Cité céleste, étrangère ici-bas, apprenne, aux dépens des vases de colère, à ne pas se fier en son libre arbitre, mais à mettre sa confiance à invoquer le nom du Seigneur. La volonté a été créée bonne, mais muable, parce qu’elle a été tirée du néant : ainsi, elle peut se détourner du bien et du mal ; mais elle n’a besoin pour le niai que de son libre arbitre et ne saurait faire le bien sans le secours de la grâce.
\subsection[{Chapitre XXII}]{Chapitre XXII}

\begin{argument}\noindent Le mélange des enfants de Dieu avec les filles des hommes a causé le déluge qui a anéanti tout le genre humain, à l’exception de huit personnes.
\end{argument}

\noindent Comme les hommes, en possession de ce libre arbitre, croissaient et s’augmentaient, il se fit une espèce de mélange et de confusion des deux cités par un commerce d’iniquité ; et ce mal prit encore son origine de ha femme, quoique d’une autre manière qu’au commencement du monde. Dans le fait, les femmes de la cité de la terre ne portèrent pas les hommes au péché, après avoir été séduites elles-mêmes par l’artifice d’un autre ; mais les enfants de Dieu, c’est-à-dire les citoyens de la cité étrangère sur la terre, commencèrent à les aimer pour leur beauté, laquelle véritablement est un don de Dieu, mais qu’il accorde aussi aux méchants, de peur que les bons ne l’estiment un grand bien. Aussi les enfants de Dieu ayant abandonné le bien souverain qui est propre aux bons, se portèrent vers un moindre bien commun aux bons et aux méchants, et épris d’amour pour les filles des hommes, ils abandonnèrent, afin de les épouser, la piété qu’ils gardaient dans la sainte société. Il est vrai, comme je viens de le dire,que la beauté du corps est un don de Dieu ; mais comme c’est un bien misérable, charnel et périssable, on ne l’aime pas comme il faut quand on l’aime plus que Dieu, qui est un bien éternel, intérieur et immuable. Lorsqu’un avare aime plus son argent que la justice, ce n’est pas la faute de l’argent, mais celle de l’homme ; il en est de même de toutes les autres créatures : comme elles sont bonnes, elles peuvent être bien ou mal aimées. On les aime bien quand on garde l’ordre, on les aime mal quand on le pervertit. C’est ce que j’ai exprimé en ces quelques vers dans un éloge du Cierge :\par
 {\itshape « Toutes ces choses, Seigneur, sont à vous et sont bonnes, parce qu’elles viennent de vous, qui êtes souverainement bon. Il n’y a rien de nous en elles que le péché, qui fait que, renversant l’ordre, nous aimons, au lieu de vous, ce qui vient de vous. »} \par
Quant au Créateur, si on l’aime véritablement, c’est-à-dire si on l’aime lui-même sans aimer autre chose à la place de lui, on ne le saurait mal aimer. Nous devons même aimer avec ordre l’amour qui fait qu’on aime comme il convient tout ce qu’il faut aimer, si nous voulons être bons et vertueux. D’où je conclus que la meilleure et la plus courte définition de la vertu est celle-ci : l’ordre de l’amour. L’épouse de Jésus-Christ, qui est la Cité de Dieu, chante pour cette raison dans le Cantique des cantiques : « Ordonnez en moi la charité. » Pour avoir confondu l’ordre de cet amour, les enfants de Dieu méprisèrent Dieu et aimèrent les filles des hommes. Or, ces deux noms, enfants de Dieu, filles des hommes, distinguent assez l’une et l’autre cité. Bien que ceux-là fussent aussi enfants des hommes par nature, la grâce avait commencé à les rendre enfants de Dieu. En effet, l’Écriture sainte, dans l’endroit où elle parle de leur amour pour les filles des hommes, les appelle aussi anges de Dieu ; ce qui a fait croire à plusieurs que ce n’était pas des hommes, mais des anges.
\subsection[{Chapitre XXIII}]{Chapitre XXIII}

\begin{argument}\noindent Les enfants de Dieu qui, suivant l’Écriture, épousèrent les filles des hommes, dont naquirent les géants, étaient-ils des anges ?
\end{argument}

\noindent Nous avons touché, sans la résoudre, au troisième livre de cet ouvrage, la question de savoir si les anges, en tant qu’esprits, peuvent avoir commerce avec les femmes. Il est écrit en effet : « Il se sert d’esprits pour ses anges », c’est-à-dire que de ceux qui sont esprits par leur nature, il en a fait ses anges, ou, ce qui revient au même, ses messagers ; mais il n’est pas aisé de décider si le Prophète parle de leurs corps, lorsqu’il ajoute : « Et d’un feu ardent pour ses ministres » ; ou s’il veut faire entendre par là que ses ministres doivent être embrasés de charité comme d’un feu spirituel. Toutefois l’Écriture témoigne que les anges ont apparu aux hommes dans des corps tels que non seulement ils pouvaient être vus, mais touchés. Il y a plus : comme c’est un fait public et que plusieurs ont expérimenté ou appris de témoins non suspects que les Sylvains et les Faunes, appelés ordinairement incubes, ont souvent tourmenté les femmes et contenté leur passion avec elles, et comme beaucoup de gens d’honneur assurent que certains démons, à qui les Gaulois donnent le nom de Dusiens, tentent et exécutent journellement toutes ces impuretés, en sorte qu’il y aurait une sorte d’impudence à les nier, je n’oserais me déterminer là-dessus, ni dire s’il y a quelques esprits revêtus d’un corps aérien qui soient capables ou non (car l’air, simplement agité par un éventail, excite la sensibilité des organes) d’avoir eu un commerce sensible avec les femmes. Je ne pense pas néanmoins que les saints anges de Dieu aient pu alors tomber dans ces faiblesses, et que ce soit d’eux que parle saint Pierre, quand il dit : « Car Dieu n’a pas épargné les anges qui ont péché, mais il les a précipités dans les cachots obscurs de l’enfer, où il les réserve pour les peines du dernier jugement » ; je crois plutôt que cet apôtre parle ici de ceux qui, après s’être révoltés au commencement contre Dieu, tombèrent du ciel avec le diable, leur prince, dont la jalousie déçut le premier homme sous la forme d’un serpent. D’ailleurs, l’Écriture sainte appelle aussi quelquefois anges les hommes de bien, comme quand il dit de saint Jean : « Voilà que j’envoie mon ange devant vous, pour vous préparer le chemin. » Et le prophète Malachie est appelé ange par une grâce particulière.\par
Ce qui fait croire à quelques-uns que les anges, dont l’Écriture dit qu’ils épousèrent les filles des hommes, étaient de véritables anges, c’est qu’elle ajoute que de ces mariages sortirent des géants ; comme si dans tous les temps il n’y avait pas eu des hommes d’une stature extraordinaire ! Quelques années avant le sac de Rome par les Goths, n’y vit-on pas une femme d’une grandeur démesurée ? et ce qui est plus merveilleux, c’est que le père et la mère n’étaient pas d’une taille égale à celle que nous voyons aux hommes très grands. Il a donc fort bien pu y avoir des géants, même avant que les enfants de Dieu, que l’Écriture appelle aussi des anges, se fussent mêlés avec les filles des hommes, c’est-à-dire avec les filles de ceux qui vivaient selon l’homme, et que les enfants de Seth eussent épousé les filles de Caïn. Voici le texte même de l’Écriture : « Comme les hommes se furent multipliés sur la terre et qu’ils eurent engendré des filles, les anges de Dieu, voyant que les filles des hommes étaient bonnes, choisirent pour femmes celles qui leur plaisaient. Alors Dieu dit : Mon esprit ne demeurera plus dans ces hommes ; car ils ne sont que chair, et ils ne vivront plus que cent vingt ans. Or, en ce temps-là, il y avait des géants sur la terre. Et depuis, les enfants de Dieu ayant commerce avec les filles des hommes. Ils engendraient pour eux-mêmes, et ceux qu’ils engendraient étaient ces géants si renommés. » Ces paroles marquent assezqu’il y avait déjà des géants sur la terre, quand les enfants de Dieu épousèrent les filles des hommes et qu’ils les aimèrent parce qu’elles étaient bonnes, c’est-à-dire belles ; car c’est la coutume de l’Écriture d’appeler bon ce qui est beau. Quant à ce qu’elle ajoute, qu’ils engendraient pour eux-mêmes, cela montre qu’auparavant ils engendraient pour Dieu, ou, en d’autres termes, qu’ils n’engendraient pas par volupté, mais pour avoir des enfants, et qu’ils n’avaient pas pour but l’agrandissement fastueux de leur famille, mais le nombre des citoyens de la Cité de Dieu, à qui, comme des anges de Dieu, ils recommandaient de mettre leur espérance en lui et d’être semblables à ce fils de Seth, à cet enfant de résurrection qui mit sa confiance à invoquer le nom du Seigneur, afin de devenir tous ensemble avec leur postérité les héritiers des biens éternels.\par
Mais il ne faut pas s’imaginer qu’ils aient tellement été anges de Dieu, qu’ils n’aient point été hommes, puisque l’Écriture déclare nettement qu’ils l’ont été. Après avoir dit que les anges de Dieu, épris de la beauté des filles des hommes, choisirent pour femmes celles qui leur plaisaient le plus, elle ajoute aussitôt ci Alors le Seigneur dit : « Mon esprit ne demeurera plus dans ces hommes, car ils ci ne sont que chair. » L’esprit de Dieu les avait rendus anges de Dieu et enfants de Dieu ; mais, comme ils s’étaient portés vers les choses basses et terrestres, l’Écriture les appelle hommes, qui est un nom de nature, et non de grâce ; elle les appelle aussi chair, parce qu’ils avaient abandonné l’esprit, et mérité par là d’en être abandonnés. Entre les exemplaires des Septante, les uns les nomment anges et enfants de Dieu, et les autres ne leur donnent que cette dernière qualité ; et Aquila, que les Juifs préfèrent à tous les autres interprètes, n’a traduit ni anges de Dieu, ni enfants de Dieu, mais enfants des dieux. Or, toutes ces versions sont acceptables. Ils étaient enfants de Dieu et frères de leurs pères, qui avaient comme eux Dieu pour père ; et ils étaient enfants des dieux, parce qu’ils étaient nés de dieux avec qui ils étaient aussi des dieux, suivant cette parole du psaume : « Je l’ai dit, vous êtes des dieux, vous êtes tous des enfants du Très-Haut. » Aussi bien, on pense avec raison que les Septante ont été animés d’un esprit prophétique, et on ne doute point que ce qu’ils ont changé dans la version, ils ne l’aient fait par une inspiration du ciel, encore qu’ici l’on reconnaisse que le mot hébreu est équivoque, et qu’il peut aussi bien signifier enfants de Dieu comme enfants des dieux.\par
Laissons donc les fables de ces écritures qu’on nomine apocryphes, parce que l’origine en a été inconnue à nos pères, qui nous ont transmis les véritables par une succession très connue et très assurée. Bien qu’il se trouve quelque vérité dans ces écritures apocryphes, elles ne sont d’aucune autorité, à cause des diverses faussetés qu’elles contiennent. Nous ne pouvons nier qu’Énoch, qui est le septième depuis Adam, n’ait écrit quelque chose ; car l’apôtre saint Jude le témoigne dans son Épître canonique ; mais ce n’est pas sans raison que ces écrits mie se trouvent point dans le catalogue des Écritures, qui était conservé dans le temple des Juifs par le soin des prêtres, attendu que ces prétendus livres d’Énoch ont été jugés suspects, à cause de leur trop grande antiquité, et parce qu’on ne pouvait justifier que ce fussent les mêmes qu’Énoch avait écrits, dès lors qu’ils n’étaient pas produits par ceux à qui la garde de ces sortes de livres était confiée. De là vient que les écrits allégués sous son nom, qui portent que les géants n’ont pas eu des hommes pour pères, sont justement rejetés par les chrétiens sages, ainsi que beaucoup d’autres que les hérétiques produisent sous le nom d’autres anciens prophètes, ou même sous celui des Apôtres, et qui sont tous mis par l’Église au rang des livres apocryphes. Il est donc certain, selon les Écritures canoniques, soit juives, soit chrétiennes, qu’il y a eu avant le déluge beaucoup de géants citoyens de la cité de la terre, et que les enfants de Seth, qui étaient enfants de Dieu par la grâce, s’unirent à eux après s’être écartés de la voie de la justice. On ne doit pas s’étonner qu’il ait pu sortir aussi d’eux des géants. À coup sûr, ils n’étaient pas tous géants ; mais il y en avait plus alors que danstoute la suite des temps qui se sont écoulés depuis ; et il a plu au Créateur de les produire, pour apprendre aux sages à ne faire pas grand cas, non seulement de la beauté, mais même de la grandeur et de la force du corps, et à mettre plutôt leur bonheur en des biens spirituels et immortels, comme beaucoup plus durables et propres aux seuls gens de bien. C’est ce qu’un autre prophète déclare en ces termes : « Alors étaient ces géants si fameux, hommes d’une haute stature et qui étaient habiles à la guerre. Le Seigneur ne les a pas choisis et ne leur a pas donné la science véritable ; mais ils ont péri et se sont perdus par leur imprudence, parce qu’ils ne possédaient pas la sagesse. »
\subsection[{Chapitre XXIV}]{Chapitre XXIV}

\begin{argument}\noindent Comment il faut entendre ce que Dieu dit à ceux qui devaient périr par le déluge : « ils ne vivront plus que cent vingt ans. »
\end{argument}

\noindent Quand Dieu dit : « Ils ne vivront plus que cent vingt ans », il ne faut pas entendre que les hommes ne devaient pas passer cet âge après le déluge, puisque quelques-uns ont vécu depuis plus de cinq cents ans ; mais cela signifie que Dieu ne leur donnait plus que ce temps-là jusqu’au déluge. Noé avait alors quatre cent quatre-vingts ans ; ce que l’Écriture, selon sa coutume, appelle cinq cents ans pour faire le compte rond. Or, le déluge arriva l’an six cent de la vie de Noé, en sorte qu’il y avait encore, au moment de la menace divine, cent vingt ans à écouler jusqu’au déluge. On croit avec raison que, lorsqu’il arriva, il n’y avait plus sur la terre que des gens dignes d’être exterminés par ce fléau : car, bien que ce genre de mort n’eût pu nuire en aucune façon aux gens de bien, qui seraient toujours morts sans cela, toutefois il est vraisemblable que le déluge ne fit périr aucun des descendants de Seth. Voici quelle fut la cause du déluge, au rapport de l’Écriture sainte : « Comme Dieu, dit-elle, eût vu que les hommes devenaient de jour en jour plus méchants et que toutes leurs pensées étaient sans cesse tournées au mal, il se mit à penser et à réfléchir que c’était lui qui les avait créés, et il dit : J’exterminerai l’homme que j’ai créé, et depuis l’homme jusqu’à la bête, depuis les serpents jusqu’aux oiseaux ; car j’ai de la colère de les avoir créés. »
\subsection[{Chapitre XXV}]{Chapitre XXV}

\begin{argument}\noindent La colère de Dieu ne trouble point son immuable tranquillité.
\end{argument}

\noindent La colère de Dieu n’est pas en lui une passion qui le trouble, mais un jugement par lequel il punit le crime, de même que sa pensée et sa réflexion ne sont que la raison immuable qu’il a de changer les choses. Il ne se repent pas, comme l’homme, de ce qu’il a fait, parce que son conseil est aussi ferme que sa prescience certaine ; mais si l’Écriture ne se servait pas de ces expressions familières, elle ne se proportionnerait pas à la capacité de tous les hommes dont elle veut procurer le bien et l’avantage, en étonnant les superbes, en réveillant les paresseux, en exerçant les laborieux, en éclairant les savants. Quant à la mort qu’elle annonce à tous les animaux, et même à ceux de l’air, c’est une image qu’elle donne de la grandeur de cette calamité à venir, et non une menace qu’elle fait aux animaux dépourvus de raison, comme s’ils avaient aussi péché.
\subsection[{Chapitre XXVI}]{Chapitre XXVI}

\begin{argument}\noindent Tout ce qui est dit de l’arche de Noé dans la Genèse figure Jésus-Christ et l’Église.
\end{argument}

\noindent En ce qui regarde le commandement que Dieu fit à Noé, qui était, selon le témoignage de l’Écriture même, un homme parfait, non de cette perfection qui doit un jour égaler aux anges les citoyens de la Cité de Dieu, mais de celle dont ils sont capables en cette vie, en ce qui regarde, dis-je, le commandement que Dieu lui fit de construire une arche pour s’y sauver de la fureur du déluge, avec sa femme, ses enfants, ses brus et les animaux qu’il eut ordre d’y faire entrer, c’est sans doute la figure de la Cité de Dieu étrangère ici-bas, c’est-à-dire de l’Église, qui est sauvée par le bois où a été attaché le médiateur entre Dieu et les hommes, Jésus-Christ homme. Les mesures même de sa longueur, de sa hauteur et de sa largeur, sont un symbole du corps humain dont Jésus-Christ s’est vraiment revêtu, comme il avait été prédit. En effet, la longueur ducorps de l’homme, de la tête aux pieds, a six fois autant que sa largeur, d’un côté à l’autre, et dix fois autant que sa hauteur, c’est-à-dire que son épaisseur, prise du dos au ventre. C’est pourquoi l’arche avait trois cents coudées de long, cinquante de large et trente de haut. La porte qu’elle avait sur le côté est la plaie que la lance fit au côté de Jésus-Christ crucifié. C’est, en effet, par là qu’entrent ceux qui viennent à lui, parce que c’est de là que sont sortis les sacrements par qui les fidèles sont initiés. Dieu commande qu’on la construise de poutres cubiques, pour figurer la vie stable et égale des saints ; car dans quelque sens que vous tourniez un cube, il demeure ferme sur sa base. Les autres choses de même qui sont marquées dans la structure de l’arche sont des figures de ce qui se passe dans l’Église.\par
Il serait trop long d’expliquer tout cela en détail, outre que nous l’avons déjà fait dans nos livres contre Fauste le manichéen, qui prétend qu’il n’y a aucune prophétie de Jésus-Christ dans l’Ancien Testament. Il se peut bien faire qu’entre les explications qu’on en donnera, celles-ci soient meilleures que celles-là, et même que les nôtres ; mais il faut au moins qu’elles se rapportent toutes à cette Cité de Dieu qui voyage dans ce monde corrompu comme au milieu d’un déluge, à moins qu’on ne veuille s’écarter du sens de l’Écriture. Par exemple, j’ai dit, dans mes livres contre Fauste, au sujet de ces paroles : « Vous ferez en bas deux ou trois étages », que ces deux étages signifient l’Église, cette assemblée de toutes les nations, à cause des deux genres d’hommes qui la composent, les Juifs et les Gentils a, et que trois étages la figurent aussi, parce que toutes les nations sont sorties après le déluge des trois fils de Noé. Un autre, par ces trois étages, entendra peut-être ces trois vertus principales que recommande l’Apôtre, savoir : la foi, l’espérance et la charité. On peut aussi et mieux encore y voir l’image de ces trois abondantes moissons de l’Évangile, dont l’une rend trente pour un, l’autre soixante et l’autre cent, en sorte que la chasteté conjugale occupe le dernier étage, la continence des veuves le second, et celle des vierges le troisième et le plus haut ; et ainsi du reste, qu’on peutexpliquer de différentes manières, mais où l’on doit toujours prendre garde de ne s’éloigner en rien de la foi catholique.
\subsection[{Chapitre XXVII}]{Chapitre XXVII}

\begin{argument}\noindent On ne doit pas plus donner les mains à ceux qui ne voient que de l’histoire dans ce que la Genèse dit de l’arche de Noé et du déluge, et rejettent les allégories, qu’à ceux qui n’y voient que des allégories et rejettent l’histoire.
\end{argument}

\noindent On aurait tort de croire qu’aucune de ces choses ait été écrite en vain, ou qu’on n’y doive chercher que la vérité historique sans allégories, ou au contraire que ce ne soient que des allégories, ou enfin, quoi qu’on en pense, qu’elles ne contiennent aucune prophétie de l’Église. Quel homme de bon sens pourrait prétendre que des livres si religieusement conservés durant tant de milliers d’années aient été écrits à l’aventure, ou qu’il y faille seulement considérer la vérité de l’histoire ? Pour ne parler que d’un point, il n’y avait aucune nécessité de faire entrer dans l’arche deux animaux immondes de chaque espèce, et sept des autres ; on y en pouvait faire entrer et des uns et des autres en nombre égal, et Dieu, qui commandait de les garder ainsi pour en réparer l’espèce, était apparemment assez puissant pour les refaire de la même façon qu’il les avait faits.\par
Pour ceux qui soutiennent que ces choses ne sont pas arrivées en effet et que ce ne sont que des figures et des allégories, ce qui les porte à en juger ainsi, c’est surtout qu’ils ne croient pas que ce déluge ait pu être assez grand pour dépasser de quinze coudées la cime des plus hautes montagnes, par cette raison, disent-ils, que les nuées n’arrivent jamais au sommet de l’Olympe, et qu’il n’y a point là de cet air épais et grossier où s’engendrent les vents, les pluies et les nuages. Mais ils ne prennent pas garde qu’il y a de la terre, laquelle est le plus matériel de tous les éléments. N’est-ce point peut-être qu’ils prétendent aussi que le sommet de cette montagne n’est pas de terre ? Pourquoi ces peseurs d’éléments veulent-ils donc que la terre aitpu s’élever si haut et que l’eau ne l’ait pas pu de même, eux qui avouent que l’eau est plus légère que la terre ? Ils disent encore que l’arche ne pouvait pas être assez grande pour contenir tant d’animaux. Mais ils ne songent pas qu’il y avait trois étages, chacun de trois cents coudées de long, de cinquante de large et de trente de haut, ce qui fait en tout neuf cents coudées en longueur, cent cinquante en largeur et quatre-vingt-dix en hauteur. Si nous ajoutons à cela, suivant la remarque ingénieuse d’Origène, que Moïse, parfaitement versé, au rapport de l’Écriture, dans toutes les sciences des Égyptiens, qui s’adonnaient fort aux mathématiques, a pu prendre ces coudées pour des coudées, de géomètres, qui en valent six des nôtres, qui ne voit combien il pouvait tenir de choses dans un lieu si vaste ? Quant à la prétendue impossibilité de faire une arche si grande, elle ne mérite pas qu’on s’y arrête, attendu que tous les jours on bâtit des villes immenses, et qu’il ne faut pas oublier que Noé fut cent ans à construire son ouvrage. Ajoutez à cela que cette arche n’était faite que de planches droites, qu’il ne fut besoin d’aucun effort pour la mettre en mer, mais qu’elle fut insensiblement soulevée par les eaux du déluge, et enfin que Dieu même la conduisait et l’empêchait de naufrager.\par
Que répondre encore à ceux qui demandent si des souris et des lézards, ou même encore des sauterelles, des scarabées, des mouches et des puces entrèrent aussi dans l’arche en même nombre que les autres animaux ? ceux qui proposent cette question doivent savoir d’abord qu’il n’était point nécessaire qu’il y eût dans l’arche, non seulement aucun des animaux qui peuvent vivre dans l’eau, comme les poissons, mais même aucun de ceux qui vivent sur sa surface, comme une infinité d’oiseaux aquatiques. De plus, l’Écriture marque expressément que Noé y fit entrer un mâle et une femelle de chaque espèce, pour montrer que c’était pour en réparer la race, et qu’ainsi il n’était point besoin d’y mettre ceux qui naissent sans l’union des sexes ou qui proviennent de la corruption ; ou que si l’on y en mit, ce fut sans aucun nombre certain, comme ils sont ordinairement dans les maisons ; ou enfin, si l’on prétend que, pour figurer avec une exactitude parfaite le plus auguste des mystères, il fallait qu’il y eût un nombre limité de toutes les sortes d’animaux qui ne peuvent vivre naturellement dans l’eau, je réponds que la providence de Dieu pourvut à tout cela sans que les hommes eussent à s’en mêler. Noé ne prenait pas les animaux pour les mettre dans l’arche, mais ils y venaient d’eux-mêmes. Les paroles de l’Écriture le font assez entendre : « Ils viendront à vous » ; c’est-à-dire qu’ils n’y viendront pas par l’entremise des hommes, mais par la volonté de Dieu, qui leur en donnera l’instinct. Il ne faut pas s’imaginer néanmoins que les animaux qui n’ont point de sexe y soient entrés, car l’Écriture dit en termes formels qu’il devait y entrer un mâle et une femelle de chaque espèce. Il existe en effet certains animaux qui s’engendrent de corruption et qui ne laissent pas ensuite de s’accoupler, comme les mouches ; il en est d’autres en qui l’on ne remarque aucune différence de sexe, comme les abeilles. Pour les bêtes qui ont un sexe, mais qui n’engendrent point, comme les mules et les mulets, je ne sais si elles y eurent place, et peut-être n’y eût-il que celles dont elles procèdent, et ainsi des autres animaux hybrides. Si toutefois cela était nécessaire pour le mystère, elles y étaient, puisque dans cette espèce d’animaux il y a aussi mâle et femelle.\par
Quelques-uns demandent encore quelle sorte de nourriture pouvaient avoir là les animaux que l’on croit ne vivre que de chair, si Noéen fit entrer dans l’arche quelques autres pour les nourrir, outre ceux que Dieu lui avait commandés, ou, ce qui est plus vraisemblable, s’il y avait quelques aliments communs à tous ; car nous savons que plusieurs animaux qui se nourrissent de chair mangent aussi des fruits et particulièrement des figues et des châtaignes. Quelle merveille donc que Noé, ce sage et saint personnage, ait préparé dans l’arche une nourriture convenable à tous les animaux et qu’au surplus Dieu même avait pu lui indiquer ? D’ailleurs, que ne mange-t-on point, quand on a faim ? Et puis, Dieu n’était-il pas assez puissant pour leur rendre agréables et salutaires toutes sortes d’aliments, lui qui n’en aurait pas eu besoin pour les faire subsister, si cela n’eût été compris dans l’accomplissement figuré du mystère ? Au reste, que tant de choses spécifiées dans le plus grand détail soient des figures de l’Église, c’est ce qu’on ne saurait nier sans opiniâtreté. Les nations, tant pures qu’impures, ont déjà tellement rempli l’Église et sont si bien unies par les liens inviolables de son unité, jusqu’à l’accomplissement final, que ce fait seul, qui est si évident, suffit pour ne nous laisser aucun doute sur les autres choses qui ne sont pas aussi claires ; et par conséquent, il faut croire que c’est avec beaucoup de sagesse que ces événements ont été confiés à la tradition et à l’écriture, qu’ils sont arrivés en effet, qu’ils signifient quelque chose, et que ce qu’ils signifient concerne l’Église. Mais il est temps de finir ce livre, pour continuer dans le suivant l’histoire des deux cités depuis le déluge.
\section[{Livre seizième. De Noé à David}]{Livre seizième. \\
De Noé à David}\renewcommand{\leftmark}{Livre seizième. \\
De Noé à David}

\subsection[{Chapitre premier}]{Chapitre premier}

\begin{argument}\noindent Si, depuis Noé jusqu’à Abraham, il y a eu des hommes qui aient servi le vrai Dieu.
\end{argument}

\noindent Il est difficile de savoir par l’Écriture si, après le déluge, il resta quelques traces de la sainte cité, ou si elles furent entièrement effacées pendant quelque temps, en sorte qu’il n’y eût plus personne qui adorât le vrai Dieu. Depuis Noé, qui mérita avec sa famille d’être sauvé de la ruine générale de l’univers, jusqu’à Abraham, nous ne trouvons point que les livres canoniques parlent de la piété de qui que ce soit. On y rapporte seulement que Noé, pénétré d’un esprit prophétique et lisant dans l’avenir, bénit deux de ses enfants, Sem et Japhet ; c’est aussi à titre de prophète qu’il ne maudit pas son fils coupable, Cham, dans sa propre personne, mais dans celle de Chanaan. Voici ses paroles : « Maudit soit l’enfant Chanaan ! il sera l’esclave de ses frères. » Or, Chanaan était né de Cham, qui, au lieu de couvrir la nudité de son père endormi, l’avait mise au grand jour. De là vient encore que cette bénédiction de ses deux autres enfants, de l’aîné et du cadet : « Que le Seigneur Dieu bénisse Sem ! Chanaan sera son esclave. Que Dieu comble de joie Japhet, et qu’il habite dans les maisons de Sem ! » cette bénédiction, dis-je, et la vigne que Noé planta, et son ivresse, et sa nudité, et la suite de ce récit, tout cela est rempli de mystères et voilé de figures.
\subsection[{Chapitre II}]{Chapitre II}

\begin{argument}\noindent De ce qui a été figuré prophétiquement dans les enfants de Noé.
\end{argument}

\noindent Mais les événements ont assez découvert ce que ces mystères tenaient caché. Qui ne reconnaît, à considérer les choses avec un peude soin et quelque lumière, que les prophéties sont accomplies en Jésus-Christ ? Sem, de qui le Sauveur est né selon la chair, signifie Renommé. Or, qu’y a-t-il de plus renommé que Jésus-Christ dont le nom jette une odeur si agréable de toutes parts qu’il est comparé, dans le Cantique des cantiques, à un parfum épanché ? N’est-ce pas aussi dans les maisons de Jésus-Christ, c’est-à-dire dans ses églises, qu’habite cette multitude nombreuse de nations figurée par Japhet, qui signifie {\itshape Étendue} ? Pour Cham, qui signifie {\itshape Chaud}, Cham, dis-je, qui était le second fils de Noé, entre Sem et Japhet, comme se distinguant de l’un et de l’autre, et ne faisant partie ni des prémices d’Israël, ni de la plénitude des Gentils, que figure-t-il, sinon les hérétiques, hommes ardents et animés, non de l’esprit de sagesse, mais d’une impatience qui les transporte et leur fait troubler le repos des fidèles ? Cette ardeur aveugle tourne, du reste, au profit de ceux qui s’avancent dans la vertu, suivant cette parole de l’Apôtre : « Il faut qu’il y ait des hérésies, afin que l’on reconnaisse par là ceux qui sont solidement vertueux. » C’est pour cela qu’il est écrit ailleurs : « Un homme sage se servira utilement de celui qui ne l’est pas. » Tandis que la chaleur inquiète des hérétiques, agite plusieurs questions qui concernent la foi, leur contradiction nous oblige de les examiner avec plus de soin, afin de pouvoir mieux les défendre contre eux, en sorte que les difficultés qu’ils proposent servent à l’instruction des fidèles. On peut dire aussi que non seulement ceux qui sont publiquement séparés de l’Église, mais encore tous ceux qui, se glorifiant d’être chrétiens, vivent mal, sont représentés par le second fils de Noé ; car ils annoncent par leur foi la passion du Sauveur figurée par la nudité de ce patriarche, et en même temps ils la déshonorent par leurs actions. C’est d’eux qu’il est dit : « Vous les reconnaîtrez par leurs fruits. » De là vient que Cham fut maudit en son fils comme en son fruit, c’est-à-dire en son œuvre, et que Chanaan signifie leurs mouvements, c’est-à-dire leurs œuvres. Quant à Sem et Japhet, c’est-à-dire la circoncision et l’incirconcision (ou, pour les désigner autrement avec l’Apôtre, les Juifs et les Gentils, mais appelés et justifiés), ayant connu en quelque façon que j’ignore la nudité de leur père, laquelle figure la passion du Rédempteur, ils prirent leur manteau sur leurs épaules, et, marchant à reculons, en couvrirent Noé et ne voulurent point voir ce que le respect leur faisait cacher. Ainsi, nous honorons ce qui a été fait pour nous dans la passion de Jésus-Christ, et nous ne laissons pas toutefois d’avoir en horreur le crime des Juifs. Le manteau que prirent ces deux enfants de Noé pour couvrir la nudité de leur père, signifie le divin sacrement, et leurs épaules, la mémoire des choses passées, parce que l’Église célèbre la passion du Sauveur comme déjà arrivée, et ne la regarde pas comme une chose à venir, maintenant que Japhet demeure dans les maisons de Sem et que leur mauvais frère habite au milieu d’eux.\par
Mais ce mauvais frère est esclave de ses bons frères en son fils, c’est-à-dire en son œuvre, lorsque les gens de bien se servent des méchants ou pour l’exercice de leur patience, ou pour l’affermissement de leur vertu. En effet, l’Apôtre témoigne qu’il y en a qui ne prêchent pas Jésus-Christ avec une intention pure. « Mais pourvu, dit-il, que Jésus-Christ soit annoncé, par prétexte ou par un vrai zèle, il n’importe, je m’en réjouis et m’en réjouirai toujours. » C’est Jésus-Christ qui a planté la vigne, dont le Prophète dit : « La vigne du Seigneur des armées, c’est la maison d’Israël. » Et il a bu du vin de cette vigne, soit que par ce vin on entende le calice dont il dit aux enfants de Zébédée : « Pouvez-vous boire le calice que je dois boire ? » et encore : « Mon père, si cela se peut, que ce calice passe sans que je le boive ! » par où il marque sans contredit sa passion, soit que, comme le vin est le fruit de la vigne, on veuille entendre plutôt par là qu’il a pris de la vigne même, c’est-à-dire de la race des Israélites, sa chair et son sang, afin de pouvoir souffrir pour nous, et qu’il s’est enivré et qu’il a été nu, parce que c’est là qu’a paru sa faiblesse, dont l’Apôtre dit : « S’il a été crucifié, c’est un effet de sa faiblesse. » Mais ainsi que le déclare le même Apôtre : « Ce qui paraît faiblesse en Dieu est plus fort que toute la force des hommes, et sa folie apparente est plus sage que toute leur sagesse. » Quand l’Écriture, après avoir dit de Noé qu’il {\itshape demeura nu}, ajoute : {\itshape dans sa maison}, cela montre ingénieusement que c’étaient des hommes de même origine que Jésus-Christ, savoir des Juifs, qui devaient lui faire souffrir le supplice de la mort et de la croix. Les réprouvés annoncent cette passion de Jésus-Christ seulement de bouche et au dehors, parce qu’ils ne comprennent pas ce qu’ils annoncent ; mais les gens de bien portent gravé au dedans d’eux-mêmes un si grand mystère, et adorent dans leur cœur cette faiblesse et cette folie de Dieu, parce qu’elles surpassent tout ce qu’il y a de plus fort et de plus sage parmi les hommes. C’est ce qui est très bien figuré, d’un côté, par Cham, qui sortit pour publier la nudité de son père, et, de l’autre, par Sem et Japhet qui, touchés de respect, entrèrent pour la cacher, fidèle image de ceux qui honorent intérieurement ce mystère.\par
Nous sondons ces secrets de l’Écriture comme nous pouvons. D’autres le feront peut-être avec plus ou moins de succès ; mais, de quelque façon qu’on le fasse, il faut toujours tenir pour constant que ces choses n’ont pas été faites ni écrites sans mystère, et qu’il ne les faut rapporter qu’à Jésus-Christ et à son Église, qui est la Cité de Dieu annoncée dès le commencement du monde par des figures dont nous voyons tous les jours la réalité. L’Écriture donc, après avoir parlé de la bénédiction des deux enfants de Noé et de la malédiction du second, ne fait mention jusqu’à Abraham d’aucun serviteur du vrai Dieu. Ce n’est pas néanmoins, à mon avis, qu’il n’y en ait eu quelques-uns dans cet espace de temps, qui est de plus de mille ans, mais c’est qu’il aurait été trop long de les rapporter tous, et que cela serait plus de l’exactitude d’un historien que de la prévoyance d’un prophète. Aussi bien, le dessein de l’auteur des saintes Lettres, ou plutôt de l’esprit de Dieu, dont il était l’organe, n’est pas seulement de raconter le passé, mais d’annoncer l’avenir, en tant qu’il concerne la Cité de Dieu. Tout ce qui y est dit de ceux qui n’en sont pas les citoyens, n’est que pour lui servir d’instruction ou pour rehausser sa gloire. Il ne faut pas s’imaginer toutefois que tous les événements qui y sont rapportés aient une signification mystique ; mais ce qui ne signifie rien y est mis en vue de ce qui a une signification. Il n’y a que le soc qui fende la terre, mais pour cela les autres parties de la charrue sont nécessaires. Dans les instruments de musique, on ne touche que les cordes ; elles seules font le son, et néanmoins on y joint d’autres ressorts qui servent à nouer et à tendre ces cordes retentissantes. Ainsi, dans l’histoire prophétique, on marque quelques événements qui n’ont aucune portée figurative, afin d’y attacher, pour ainsi dire, ceux qui figurent quelque chose.
\subsection[{Chapitre III}]{Chapitre III}

\begin{argument}\noindent Généalogie des trois enfants de Noé.
\end{argument}

\noindent Il faut considérer maintenant la généalogie des enfants de Noé, et en dire ce qui sera nécessaire pour marquer le progrès de l’une et de l’autre cité. L’Écriture commence par Japhet, le plus jeune des fils de Noé, qui eut huit enfants, l’un desquels en eut trois, l’autre quatre, ce qui fait quinze en tout. Cham, le second fils de Noé, en eut quatre, plus cinq petits-fils, dont l’un lui donna deux arrière-petits-fils, ce qui fait onze. Après quoi l’Écriture revient à Cham et dit : « Chus (qui est l’aîné de Cham) engendra Nebroth, qui était un géant et un grand chasseur contre le Seigneur ; d’où est venu le proverbe : Grand chasseur contre le Seigneur comme Nebroth. Les principales villes de son royaume étaient Babylone, Orech, Archad et Chalanné, dans le territoire de Sennaar. De cette contrée sortit Assur, qui bâtit Ninive, Robooth, Halach et, entre Ninive et Halach, la grande ville de Dasem. » Or, ce Chus, père du géant Nebroth, est nommé lepremier entre les enfants de Cham, et l’Écriture avait déjà fait mention de cinq de ses fils et de deux de ses petits-fils. Il faut donc qu’il ait engendré ce géant après la naissance de ses petits-fils, ou, ce qui est plus probable, que l’Écriture l’ait cité à part, parce qu’il était très puissant ; car en même temps elle parle aussi de son royaume, qui prit naissance dans la fameuse Babylone et autres villes ou contrées déjà citées. Quant à ce qu’elle dit d’Assur, qu’il sortit de cette contrée de Sennaar, qui dépendait du royaume de Nebroth, et qu’il bâtit Ninive et les autres villes dont elle fait mention, cela n’arriva que longtemps après ; mais elle en parle ici en passant et par occasion, à cause de l’empire fameux des Assyriens que Ninus, fils de Bélus et fondateur de cette grande ville de Ninive, qui prit son nom, étendit merveilleusement. Pour Assur, d’où sont sortis les Assyriens, il n’était pas fils de Cham, mais de Sem, aîné de Noé ; d’où Il paraît que, dans la suite, des descendants de Sem possédèrent le royaume de Nebroth, et, s’étendant plus loin, fondèrent d’autres villes dont Ninive fut la première. De là, l’Écriture remonte à un autre fils de Cham, nommé Mesraïm, et à ses sept enfants, et elle en parle, non comme de particuliers, mais comme de nations, disant que de la sixième sortit celle des Philistins ; ce qui en fait huit. Ensuite elle retourne à Chanaan, en qui Cham fut maudit, et fait mention d’onze de ses fils et de certaines contrées qu’ils occupaient. Ainsi toute la postérité de Cham monte à trente et une personnes. Reste à parler des enfants de Sem, aîné de Noé ; car c’est lui qui termine cette généalogie. Mais il y a ici quelque obscurité dans la Genèse, où il n’est pas aisé de découvrir quel fut le premier fils de Sem. Voici ce qu’elle dit : « De Sem, père de tous les enfants d’Héber et frère aîné de Japhet, naquirent Éla, etc. » Par là, il semblerait qu’Héber fût fils immédiat de Sem, et cependant il n’est que le cinquième de ses descendants. Sem, entre autres fils, engendra Arphaxat, Arphaxat engendra Caïnan, Caïnan engendra Sala, et Sala engendra Héber. L’Écriture a voulu faire entendre par là que Sem est le père de tous ses descendants, tant fils que petits-fils et autres de sa race ; et ce n’estpas sans raison qu’elle parle d’Héber avant que de parler des fils de Sem, quoiqu’il ni soit, comme je viens de le dire, que le vingtième de sa race, à cause que c’est de lui que les Hébreux ont pris leur nom, bien que d’autres veuillent que ce soit d’Abraham, mais avec moins d’apparence. Ainsi l’Écriture nomme d’abord six enfants de Sem, l’un desquels en eut quatre ; puis elle fait mention d’un autre fils de Sem qui lui engendra un petit-fils, et celui-ci un arrière-petit-fils dont sortit Héber. Héber eut deux fils, dont l’un fut nommé Phalech, c’est-à-dire {\itshape Divisant}, à cause, dit l’Écriture, que de son temps la terre fut divisée ; l’autre eut douze fils ; de sorte que toute la postérité de Sem est de vingt personnes. De cette manière, tous les descendants des trois fils de Noé, c’est-à-dire quinze de Japhet, trente et un de Cham et vingt-sept de Sem, font soixante-treize. Après, l’Écriture ajoute : « Voilà les enfants de Sem selon leurs familles, leurs langues, leurs contrées et leurs nations. » Et parlant de tous ensemble : « Voilà les familles des enfants de Noé, selon leurs générations et leurs peuples : d’elles fut peuplée la terre après le déluge. » On voit par là que c’est de nations et non d’hommes en particulier que parle l’Écriture, lorsqu’elle fait mention de ces soixante-treize, ou plutôt soixante-douze personnes, comme nous le montrerons ci-après, et que c’est pour cela qu’elle en a omis plusieurs de la postérité de Noé, non qu’ils n’aient eu des enfants aussi bien que les autres, mais parce qu’ils n’ont pas fait souche comme eux et n’ont pas été pères d’un peuple.
\subsection[{Chapitre IV}]{Chapitre IV}

\begin{argument}\noindent De Babylone et de la confusion des langues.
\end{argument}

\noindent Mais, quoique l’Écriture rapporte que ces nations furent divisées chacune en leur langue, elle ne laisse pas ensuite de revenir au temps où elles n’avaient toutes qu’un seul langage, et de déclarer comment arriva la différence qui y survint. « Toute la terre, dit-elle, parlait une même langue, lorsque les hommes, s’éloignant de l’Orient, trouvèrent une plaine dans la contrée de Sennaar, où ils s’établirent. Alors ils se dirent l’un à l’autre : Venez, faisons des briques et les cuisons au feu. Ils prirent donc des briques au lieu de pierres, et du bitume au lieu de mortier, et dirent : Bâtissons-nous une ville et une tour dont le sommet s’élève jusqu’au ciel, et faisons parler de nous avant de nous séparer. Mais le Seigneur descendit pour voir la ville et la tour que les enfants des hommes bâtissaient, et il dit : Voilà un seul peuple et une même langue, et, maintenant qu’ils ont commencé ceci, ils ne s’arrêteront qu’après l’avoir achevé. Venez donc, descendons et confondons leur langue, en sorte qu’ils ne s’entendent plus l’un l’autre. Et le Seigneur les dispersa par toute la terre, et ils cessèrent de travailler à la ville et à la tour. De là vient que ce lieu fut appelé Confusion, parce que ce fut là que Dieu confondit le langage des hommes et qu’il les dispersa ensuite par tout le monde. » Cette ville, qui fut appelée Confusion, c’est Babylone, et l’histoire profane elle-même en célèbre la construction merveilleuse. En effet, Babylone signifie {\itshape Confusion}, et nous voyons par là que le géant Nebroth en fut le fondateur, comme l’Écriture l’avait indiqué auparavant en disant que Babylone était la capitale de son royaume, quoiqu’elle ne fût pas arrivée au point de grandeur où l’orgueil et l’impiété des hommes se flattaient de la porter. Ils prétendaient la faire extraordinairement haute et l’élever jusqu’au ciel, comme parlait l’Écriture, soit qu’ils n’eussent ce dessein que pour une des tours de la ville, soit qu’ils l’étendissent à toutes ; l’Écriture ne parle que d’une, mais c’est peut-être de la même manière qu’elle dit le soldat pour signifier toute une armée, ou la grenouille et la sauterelle pour exprimer cette multitude de grenouilles et de sauterelles qui furent deux des plaies qui affligèrent l’Égypte. Mais qu’espéraient entreprendre contre Dieu ces hommes téméraires et présomptueux avec cette masse de pierres, quand ils l’auraient élevée au-dessus de toutes les montagnes et de la plus haute région de l’air ? En quoi peut nuire à Dieu quelque élévation que ce soit de corps ou d’esprit ? Le sûr et véritable chemin pour monter au ciel est l’humilité. Elle élève le cœur en haut, mais au Seigneur, et non pas contre le Seigneur, comme l’Écriture le dit de ce géant, qui était {\itshape un chasseur contre le Seigneur}. C’est en effet ainsi qu’il faut traduire, et non : {\itshape devant le Seigneur}, comme ont fait quelques-uns, trompés par l’équivoque du mot grec, qui peut signifier l’un et l’autre. La vérité est qu’il est employé au dernier sens dans ce verset du psaume : « Pleurons devant le Seigneur qui nous a faits » ; et au premier dans le livre de Job, lorsqu’il est dit : « Vous vous êtes emportés de colère contre le Seigneur. » Et que veut dire un chasseur sinon un trompeur, un meurtrier et un assassin des animaux de la terre ? Il élevait donc une tour contre Dieu avec son peuple, ce qui signifie un orgueil impie, et Dieu punit avec justice leur mauvaise intention, quoiqu’elle n’ait pas réussi. Mais de quelle façon la punit-il ? Comme la langue est l’instrument de la domination, c’est en elle que l’orgueil a été puni, tellement que l’homme, qui n’avait pas voulu entendre les commandements de Dieu, n’a point été à son tour entendu des hommes, quand il a voulu leur commander. Ainsi fut dissipée cette conspiration, chacun se séparant de celui qu’il n’entendait pas pour se joindre à celui qu’il entendait ; et les peuples furent divisés selon les langues et dispersés dans toutes les contrées de la terre par la volonté de Dieu, qui se servit pour cela de moyens qui nous sont tout à fait cachés et incompréhensibles.
\subsection[{Chapitre V}]{Chapitre V}

\begin{argument}\noindent De la descente de Dieu pour confondre les langues.
\end{argument}

\noindent « Le Seigneur, dit l’Écriture, descendit pour voir la ville et la tour que bâtissaient les enfants des hommes », c’est-à-dire non les enfants de Dieu, mais cette société d’hommes qui vit selon l’homme, et que nous appelons la cité de la terre. Cette descente de Dieu ne doit pas s’entendre matériellement, comme s’il changeait de lieu, lui qui est tout entier partout ; mais on dit qu’il descend, lorsqu’il fait sur la terre quelque chose d’extraordinaire qui marque sa présence. De même, quand on dit qu’il voit quelque chose, ce n’est pas qu’il ne l’eût vue auparavant, lui qui ne peut rien ignorer, mais c’est qu’il l’a fait voir aux hommes. On ne voyait donc pas cette ville comme on la vit depuis, quand Dieu eut montré combien elle lui déplaisait. Toutefois on peut fort bien entendre que Dieu descenditsur cette ville, parce que ses anges, en qui il habitait, y descendirent, en sorte que ces paroles : « Dieu dit : Ils ne parlent tous qu’une même langue », et le reste, et ensuite : « Venez, descendons et confondons leur langage », ne seraient qu’une récapitulation pour expliquer ce que l’Écriture avait déjà dit, « que le Seigneur descendit ». En effet, s’il était déjà descendu, que voudrait dire ceci : « Venez, descendons et confondons leur langage », ce qui semble bien s’adresser aux anges et signifier que celui qui était dans les anges descendait par leur ministère ? Il faut encore remarquer à ce propos que le texte hébreu ne dit pas : Venez et confondez, mais : « Venez et confondons », pour faire voir que Dieu agit tellement par ses ministres, que ses ministres agissent avec lui, suivant cette parole de l’Apôtre : « Nous sommes les coopérateurs de Dieu. »
\subsection[{Chapitre VI}]{Chapitre VI}

\begin{argument}\noindent Comment il faut entendre que Dieu parle aux anges.
\end{argument}

\noindent On pourrait croire que les paroles de la Genèse : « Faisons l’homme », auraient été aussi adressées aux anges, si Dieu n’ajoutait : « À notre image ». Ce dernier trait est décisif et ne nous permet pas de croire que l’homme ait été fait à l’image des anges, ou que Dieu et les anges n’aient qu’une même image. Nous avons donc raison d’entendre ce pluriel : « Faisons », des personnes de la Trinité. Et néanmoins comme cette Trinité n’est qu’un Dieu, après que Dieu a dit : « Faisons », l’Écriture ajoute : « Et Dieu fit l’homme à l’image de Dieu. » Elle ne dit pas : Les dieux firent ; ou : À l’image des dieux. — Or, dans le passage discuté tout à l’heure, on pourrait également trouver une trace de la Trinité, comme si le Père, s’adressant au Fils et au Saint-Esprit, leur eût dit : « Venez, descendons et confondons leur langage » ; mais ce qui retient l’esprit, c’est qu’ici rien n’empêche d’appliquer le pluriel aux anges. Ces paroles, en effet, leur conviennent mieux, parce que c’est surtout à eux à s’approcher de Dieu par de saints mouvements, c’est-à-dire par de pieuses pensées, et à consulter les oracles de la vérité immuable qui leur sert de loi éternelle dans leur bienheureux séjour. Ils ne sont pas eux-mêmes la vérité ; mais participant à cette vérité créatrice de toutes choses, ils s’en approchent comme de la source de la vie, afin de recevoir d’elle ce qu’ils ne trouvent pas en eux. Ç’est pourquoi le mouvement qui lei porte vers elle est stable en quelque façon, parce qu’ils ne s’éloignent jamais d’elle. Or, Dieu ne parle pas aux anges comme nous nous parlons les uns aux autres, ou comme nous parlons à Dieu ou aux anges, ou comme les anges nous parlent, ou comme Dieu nous parle par les anges ; il leur parle d’une manière ineffable, et cette parole nous est transmise d’une manière qui nous est proportionnée. La parole de Dieu, supérieure à tous ses ouvrages, est la raison même, la raison immuable de ces ouvrages ; elle n’a pas un son fugitif, mais une vertu permanente dans l’éternité et agissante dans le temps. C’est de cette parole éternelle qu’il se sert pour parler aux anges ; et quand il lui plaît de nous parler de la sorte au fond du cœur, nous leur devenons semblables en quelque façon : pour l’ordinaire, il nous parle autrement. Afin clone de n’être pas toujours obligé dans cet ouvrage de rendre raison des paroles de Dieu, je dirai ici, une fois pour toutes, que la vérité immuable parle par elle-même à la créature raisonnable d’une manière qui ne se peut expliquer, soit qu’elle s’adresse à la créature par l’entremise de la créature, soit qu’elle frappe notre esprit par des images spirituelles, ou nos oreilles par des voix ou des sous.\par
Expliquons encore ces mots : « Et maintenant qu’ils ont commencé ceci, ils ne s’arrêteront qu’après l’avoir achevé. » Quand Dieu parle de la sorte, ce n’est pas une affirmation, c’est plutôt une interrogation menaçante comme celle-ci dans Virgile :\par
 {\itshape « On ne prendra pas les armes ! toute la ville ne se mettra pas à leur poursuite. »} \par
La parole de Dieu doit donc être entendue ainsi : Ils ne s’arrêteront donc pas avant que d’avoir achevé ! — Mais, pour revenir à la suite du récit de la Genèse, disons que des trois enfants de Noé sortirent soixante et treize ou plutôt soixante et douze nations d’un langage différent qui commencèrent à se répandre par toute la terre et ensuite à peupler lesîles. Mais les peuples se sont bien plus multipliés que les langues ; car nous savons que dans l’Afrique plusieurs nations barbares n’usent que d’un seul langage. À l’égard des îles, qui peut douter que, le nombre des hommes croissant, ils n’aient pu y passer à l’aide de vaisseaux ?
\subsection[{Chapitre VII}]{Chapitre VII}

\begin{argument}\noindent Comment, depuis le déluge, toutes sortes de bêtes ont pu peupler les îles les plus éloignées.
\end{argument}

\noindent On demande comment les bêtes qui ne naissent pas de la terre ainsi que les grenouilles, mais par accouplement, comme les loups et autres animaux, ont pu se trouver dans les îles après le déluge, à moins qu’elles ne soient provenues de celles qui avaient été sauvées dans l’arche. Pour les îles qui sont proches, on peut croire qu’elles y ont passé à la nage ; mais il y en a qui sont si éloignées du continent qu’il n’est pas probable qu’aucun de ces animaux ait pu y arriver de la sorte. On peut répondre à cela que les hommes les y ont transportées sur leurs vaisseaux pour les faire servir à la chasse, et enfin que Dieu même a fort bien pu les y transporter par le ministère des anges. Que si elles sont sorties de la terre, comme à la création du monde, quand Dieu dit : « Que la terre produise une âme vivante », cela fait voir clairement que des animaux de tout genre ont été mis dans l’arche, moins pour en réparer l’espèce que pour être une figure de l’Église qui devait être composée de toutes sortes de nations.
\subsection[{Chapitre VIII}]{Chapitre VIII}

\begin{argument}\noindent Si les races d’hommes monstrueux dont parle l’histoire viennent d’Adam ou des fils de Noé.
\end{argument}

\noindent On demande encore s’il est croyable qu’il soit sorti d’Adam ou de Noé certaines races d’hommes monstrueux dont l’histoire fait mention. On assure, en effet, que quelques-uns n’ont qu’un œil au milieu du front, que d’autres ont la pointe du pied tournée en dedans ; d’autres possèdent les deux sexes dont ils se servent alternativement, et ils ont la mamelle droite d’un homme et la gauche d’une femme ; il y en a qui n’ont point de bouche et ne vivent que de l’air qu’ils respirent par le nez ; d’autres n’ont qu’une coudée de haut, d’où vient que les Grecs les nomment Pygmées ; on dit encore qu’en certaines contrées il y a des femmes qui deviennent mères à cinq ans et qui n’en vivent que huit. D’autres affirment qu’il y a des peuples d’une merveilleuse vitesse qui n’ont qu’une jambe sur deux pieds et ne plient point le jarret ; on les appelle Sciopodes, parce que l’été ils se couchent sur le dos et se défendent du soleil avec la Plante de leurs pieds ; d’autres n’ont point de tête et ont les yeux aux épaules ; et ainsi d’une infinité d’autres monstres de la sorte, retracés en mosaïque sur le port de Carthage et qu’on prétend avoir été tirés d’une histoire fort curieuse. Que dirai-je des Cynocéphales, dont la tête de chien et les aboiements montrent que ce sont plutôt des bêtes que des hommes ? Mais nous ne sommes pas obligés de croire tout cela. Quoi qu’il en soit, quelque part et de quelque figure que naisse un homme, c’est-à-dire un animal raisonnable et mortel, il ne faut point douter qu’il ne tire son origine d’Adam, comme du père de tous les hommes.\par
La raison que l’on rend des enfantements monstrueux qui arrivent parmi nous peut servir pour des nations tout entières. Dieu, qui est le créateur de toutes choses, sait en quel temps et en quel lieu une chose doit être créée, parce qu’il sait quels sont entre les parties de l’univers les rapports d’analogie et de contraste qui contribuent à sa beauté. Mais nous qui ne le saurions voir tout entier, nous sommes quelquefois choqués de quelques-unes de ses parties, par cela seul que nous ignorons quelle proportion elles ont avec tout le reste. Nous connaissons des hommes qui ont plus de cinq doigts aux mains et aux pieds ; mais encore que la raison nous en soit inconnue, loin de nous l’idée que le Créateur se soit mépris ! Il en est de même des autres différences plus considérables : Celui dont personne ne peut justement blâmer les ouvrages, sait pour quelle raison il les a faits de lasorte. Il existe un homme à Hippone-Diarrhyte, qui a la plante des pieds en forme de croissant, avec deux doigts seulement aux extrémités, et les mains de même. S’il y avait quelque nation entière de la sorte, on l’ajouterait à cette histoire curieuse et surprenante. Dirons-nous donc que cet homme ne tire pas son origine d’Adam ? Les androgynes, qu’on appelle aussi hermaphrodites, sont rares, et néanmoins il en paraît de temps en temps en qui les deux sexes sont si bien distingués qu’il est difficile de décider duquel ils doivent prendre le nom, bien que l’usage ait prévalu en faveur du plus noble. Il naquit en Orient, il y a quelques années, un homme double de la ceinture en haut ; il avait deux têtes, deux estomacs et quatre mains, un seul ventre d’ailleurs et deux pieds, comme un homme d’ordinaire, et il vécut assez longtemps pour être vu de plusieurs personnes qui accoururent à la nouveauté de ce spectacle. Comme on ne peut pas nier que ces individus ne tirent leur origine d’Adam, il faut en dire autant des peuples entiers en qui la nature s’éloigne de son cours ordinaire, et qui néanmoins sont des créatures raisonnables, si, après tout, ce qu’on en rapporte n’est point fabuleux : car supposez que nous ignorassions que les singes, les cercopithèques et les sphinx sont des bêtes, ces historiens nous feraient peut-être croire que ce sont des nations d’hommes. Mais en admettant que ce qu’on lit des peuples en question soit véritable, qui sait si Dieu n’a point voulu les créer ainsi, afin que nous ne croyions pas que les monstres qui naissent parmi nous soient des défaillances de sa sagesse ? Les monstres dans chaque espèceseraient alors ce que sont les races monstrueuses dans le genre humain. Ainsi, pour conclure avec prudence et circonspection : ou ce que l’on raconte de ces nations est faux, ou ce ne sont pas des hommes, ou, si ce sont des hommes, ils viennent d’Adam.
\subsection[{Chapitre IX}]{Chapitre IX}

\begin{argument}\noindent S’il y a des antipodes.
\end{argument}

\noindent Quant à leur fabuleuse opinion qu’il y a des antipodes, c’est-à-dire des hommes dont les pieds sont opposés aux nôtres et qui habitent cette partie de la terre où le soleil se lève quand il se couche pour nous, il n’y a aucune raison d’y croire. Aussi ne l’avancent-ils sur le rapport d’aucun témoignage historique, mais sur des conjectures et des raisonnements, parce que, disent-ils, la terre étant ronde, est suspendue entre les deux côtés de la voûte céleste, la partie qui est sous nos pieds, placée dans les mêmes conditions de température, ne peut pas être sans habitants. Mais quand on montrerait que la terre est ronde, il ne s’ensuivrait pas que la partie qui nous est opposée ne fût point couverte d’eau. D’ailleurs, ne le serait-elle pas, quelle nécessité qu’elle fût habitée, puisque, d’un côté, l’Écriture ne peut mentir, et que, de l’autre, il y a trop d’absurdité à dire que les hommes aient traversé une si vaste étendue de mer pour aller peupler cette autre partie du monde. — Voyons donc si nous pourrons trouver la Cité de Dieu parmi ces hommes qui, selon la Genèse, furent divisés en soixante-douze nations et autant de langues. Il est évident qu’elle a persévéré dans les enfants de Noé, surtout dans l’aîné, qui est Sem, puisque la bénédiction de Japhet enfermeen quelque sorte celle de Sem, et qu’il doit habiter dans les demeures de ses frères.
\subsection[{Chapitre X}]{Chapitre X}

\begin{argument}\noindent Généalogie de Sem, dans la race de qui le progrès de la Cité de Dieu se dirige vers Abraham.
\end{argument}

\noindent Il faut donc prendre la suite des générations depuis Sem, afin de faire voir la Cité de Dieu à partir du déluge, comme la suite des générations de Seth l’a montrée auparavant. C’est pour cela que l’Écriture, après avoir montré la cité de la terre dans Babylone, c’est-à-dire dans la confusion, retourne au patriarche Sem, et commence par lui l’ordre des générations jusqu’à Abraham, marquant combien chacun a vécu, avant que d’engendrer celui qui continue cette généalogie, et combien il a vécu depuis. Mais il faut, en passant, que je m’acquitte de ma promesse, et que je rende raison de ce que dit l’Écriture, que l’un des enfants d’Héber fut nommé Phalech, parce que la terre fut divisée de son temps. Que doit-on entendre par cette division, si ce n’est la diversité des langues ?\par
L’Écriture, laissant de côté les autres enfants de Sem, qui ne contribuent en rien à la suite des générations, parle seulement de ceux qui la conduisent jusqu’à Abraham ; ce qu’elle avait déjà fait avant le déluge dans la généalogie de Seth. Voici comme elle commence celle de Sem : « Sem, fils de Noé, avait cent ans lorsqu’il engendra Arphaxat, la seconde année après le déluge ; et il vécut, encore depuis cinq cents ans, et engendra des fils et des filles. » Elle poursuit de même pour les autres avec le soin d’indiquer l’année où chacun a engendré celui qui sert à cette généalogie, et la durée totale de sa vie, et elle ajoute toujours qu’il a eu d’autres enfants, afin que nous n’allions pas demander sottement comment la postérité de Sem a pu peupler tant de régions et fonder ce puissant empire des Assyriens que Ninus étendit si loin.\par
Mais, pour ne pas nous arrêter plus qu’il ne convient, nous ne marquerons que l’âge auquel chacun des descendants de Sem a eu le fils qui continue la suite de cette généalogie, afin de supputer combien d’années se sont écoulées depuis le déluge jusqu’à Abraham.\par
Deux ans donc après le déluge, Sem, âgé de cent ans, engendra Arphaxat ; Arphaxat engendra Caïnan à l’âge de cent trente-cinq ans ; Caïnan avait cent trente ans quand il engendra Salé ; Salé en avait autant lorsqu’il engendra Héber ; Héber cent trente-quatre lorsqu’il engendra Ragau ; Ragau cent trente-deux quand il engendra Seruch ; Seruch cent trente quand il eut Nachor ; Nachor soixante-dix-neuf à la naissance de son fils Tharé ; et Tharé, à l’âge de soixante-dix ans, engendra Abram, que Dieu appela depuis Abraham. Ainsi, depuis le déluge jusqu’à Abraham, il y a mille soixante-douze ans, selon les Septante, car on dit qu’il y en a beaucoup moins, selon l’hébreu : ce dont on ne rend aucune raison bien claire.\par
Lors donc que nous cherchons la Cité de Dieu dans ces soixante-douze nations dont parle l’Écriture, nous ne saurions affirmer positivement si dès ce temps, où les hommes ne parlaient tous qu’un même langage, ils abandonnèrent le culte du vrai Dieu, de telle sorte que la vraie piété ne se soit conservée que dans les descendants de Sem par Arphaxat jusqu’à Abraham ; ou bien si la cité de la terre ne commença qu’à la construction de la tour de Babel ; ou plutôt si les deux cités subsistèrent, celle de Dieu dans les deux fils de Noé, qui furent bénis dans leurs personnes et dans leur race, et celle de la terre, dans le fils qui fut maudit ainsi que sa postérité. Peut-être est-il plus vraisemblable qu’avant la fondation de Babylone il y avait des idolâtres dans la postérité de Sem et de Japhet, et des adorateurs du vrai Dieu dans celle de Cham ; au moins devons-nous croire qu’il y a toujours eu sur la terre des hommes de l’une et de l’autre sorte. Dans les deux psaumes où il est dit : « Tous ont quitté le droit chemin et se sont corrompus ; il n’y en a pas un qui soit homme de bien, il n’y en a pas un seul », on lit ensuite : « Ces impies qui ne font que du mal et qui dévorent mon peuple comme ils feraient un morceau de pain, ne se reconnaîtront-ils jamais ? » Le peuple de Dieu était donc alors ; et ainsi ces paroles : « Il n’y en a pas un qui soit homme de bien, il n’y en a pas un seul », doivent s’entendre des enfants des hommes, et non de ceux de Dieu. Le Prophète avait ditauparavant : « Dieu a jeté les yeux du haut du ciel sur les enfants des hommes, pour voir s’il y en a quelqu’un qui le connaisse et qui le cherche » ; après quoi il ajoute : « Il n’y en a pas un qui soit homme de bien », pour montrer qu’il ne parle que des enfants des hommes, c’est-à-dire de ceux qui appartiennent à la cité qui vit selon l’homme, et non selon Dieu.
\subsection[{Chapitre XI}]{Chapitre XI}

\begin{argument}\noindent La langue hébraïque, qui était celle dont tous les hommes se servaient d’abord, se conserva dans la postérité d’Héber, après la confusion des langues.
\end{argument}

\noindent De même que l’existence d’une seule langue avant le déluge n’empêcha pas qu’il n’y eût des méchants et que tous les hommes n’encourussent la peine d’être exterminés par les eaux, à la réserve de la maison de Noé, ainsi, lorsque les nations furent punies par la diversité des langues, à cause de leur orgueil impie, et répandues par toute la terre, et que la cité des méchants fut appelée Confusion ou Babylone, la langue dont tous les hommes se servaient auparavant demeura dans la maison d’Héber. De là vient, comme je l’ai remarqué ci-dessus, que l’Écriture, dans le dénombrement des enfants de Sem, met Héber le premier, quoiqu’il ne soit que le cinquième de ses descendants. Comme cette langue demeura dans sa famille, tandis que les autres nations furent divisées suivant les temps, celle-là fut depuis appelée hébraïque. Il fallait bien en effet lui donner un nom pour la distinguer de toutes les autres qui avaient aussi chacune le sien, au lieu que, quand elle était seule, elle n’avait point de nom particulier.\par
On dira peut-être : Si la terre fut divisée en plusieurs langues du temps de Phalech, fils d’Héber, celle de ces langues qui était auparavant commune à tous les hommes devait plutôt prendre son nom de Phalech. Mais il faut répondre qu’Héber n’appela son fils Phalech, c’est-à-dire {\itshape Division}, que parce qu’il vint au monde lorsque la terre fut divisée par langues, et que c’est ce qu’entend l’Écriture, quand elle dit : « La terre fut divisée de son temps. » Si Héber n’eût encore été vivant lors de cette division, il n’eût pas donné son nom à la langue qui demeura dans sa famille. Ce qui nous porte à croire que cette langue est celle qui était d’abord commune à tous les hommes, c’est que le changement et la multiplication des langues ont été une peine du péché, et partant que le peuple de Dieu a dire être exempt de cette peine. Aussi n’est-ce pas sans raison que cette langue a été celle d’Abraham, et qu’il ne l’a pu transmettre à tous ses enfants, mais seulement à ceux qui, issus de Jacob, ont composé le peuple de Dieu, reçu son alliance, et mis au monde le Christ. Héber lui-même n’a pas fait passer cette langue à toute sa postérité, mais seulement à la branche d’Abraham. Ainsi, bien que l’Écriture ne marque pas précisément qu’il y eût des gens de bien, lorsque les méchants bâtissaient Babylone, cette obscurité n’est pas tant pour nous priver de la vérité que pour exercer notre attention. Lorsqu’on voit, d’un côté, qu’il existe d’abord une langue commune à tous les hommes, qu’il est fait mention d’Héber avant tous les autres enfants de Sem, encore qu’il n’ait été que le cinquième de ses descendants, et que la langue des patriarches, des prophètes et de l’Écriture même est appelée langue hébraïque, et lorsqu’on demande, de l’autre côté, où cette langue, qui était commune avant la division des langues, s’est pu conserver, comme il n’est point douteux d’ailleurs que ceux parmi lesquels elle s’est conservée n’aient été exempts de la peine du changement des langues, que se présente-t-il à l’esprit, sinon qu’elle est demeurée dans la famille de celui dont elle a pris le nom, et que ce n’est pas une petite preuve de la vertu de cette famille d’avoir été à couvert de cette punition générale ?\par
Mais il se présente encore une autre difficulté : comment Héber et Phalech son fils ont-ils pu chacun faire une nation ? Il est certain au fond que le peuple hébreu est descendu d’Héber par Abraham. Comment donc tous les enfants des trois fils de Nué, dont parle l’Écriture, ont-ils établi chacun unenation, si Héber et Phalech n’en ont fait qu’une ? Il est fort probable que Nebroth a fondé aussi sa nation, et que l’Écriture a fait mention à part de ce personnage, à cause de sa stature extraordinaire et de la vaste étendue de son empire ; de sorte que le nombre des soixante-douze langues ou nations demeure toujours. Quant à Phalech, elle n’en parle pas pour avoir donné naissance à une nation ; mais à cause de cet événement mémorable de la division des langues qui arriva de son temps. On ne doit point être surpris que Nebroth ait vécu jusqu’à la fondation de Babylone et à la confusion des langues ; car de ce qu’Héber est le sixième, depuis Noé, et Nebroth seulement le quatrième, il ne s’ensuit pas que Nebroth n’ait pas pu vivre jusqu’au temps d’Héber. Lorsqu’il y avait moins de générations, les hommes vivaient davantage, ou venaient au monde plus tard. Aussi faut-il entendre que, quand la terre fut divisée en plusieurs nations, non seulement les descendants de Noé, qui en étaient les pères et les fondateurs, étaient nés, mais qu’ils avaient déjà des familles nombreuses et capables de composer chacune une nation. C’est pourquoi il ne faut pas s’imaginer qu’ils soient nés dans le même ordre où l’Écriture les nomme ; autrement, comment les douze fils de Jectan, autre fils d’Héber et frère de Phalech, auraient-ils pu déjà faire des nations, si Jectan ne vint au monde qu’après Phalech, puisque la terre fut divisée à la naissance de Phalech ? Il est donc vrai que Phalech a été nommé le premier, mais Jectan n’a pas laissé que de venir au monde bien avant lui ; en sorte que les douze enfants de Jectan avaient déjà de si grandes familles qu’elles pouvaient être divisées chacune en leur langue. On aurait tort de trouver étrange que l’Écriture en ait usé de la sorte, puisque dans la généalogie des trois enfants de Noé, elle commence par Japhet, qui était le cadet. Or, les noms de ces peuples se trouvent encore aujourd’hui en partie les mêmes qu’ils étaient autrefois comme ceux des Assyriens et des Hébreux ; et en partie ils ont été changés par la suite des temps, tellement que les plus versés dans l’histoire en peuvent à peine découvrir l’origine. En effet, on dit que les Égyptiens viennent de Mesraïm, et les Éthiopiens de Chus, deux des fils de Cham, et cependant on ne voit aucun rapport entre leurs noms actuels et leur origine. À tout considérer, on trouvera que, parmi ces noms, il y en a plus de ceux qui ont été changés que de ceux qui sont demeurés jusqu’à nous.
\subsection[{Chapitre XII}]{Chapitre XII}

\begin{argument}\noindent Du progrès de la Cité de Dieu, à partir d’Abraham.
\end{argument}

\noindent Voyons maintenant le progrès de la Cité de Dieu, depuis le temps d’Abraham, où elle a commencé à paraître avec plus d’éclat et où les promesses que nous voyons aujourd’hui accomplies en Jésus-Christ sont plus claires et plus précises. Abraham, au rapport de l’Écriture, naquit dans la Chaldée, qui dépendait de l’empire des Assyriens. Or, la superstition et l’impiété régnaient déjà parmi ces peuples, comme parmi les autres nations. La seule maison de Tharé, père d’Abraham, conservait le culte du vrai Dieu et vraisemblablement aussi la langue hébraïque, quoique Jésus Navé témoigne qu’Abraham même était d’abord idolâtre. De même que la seule maison de Noé demeura pendant le déluge pour réparer le genre humain, ainsi, dans ce déluge de superstitions qui inondaient l’univers, la seule maison de Tharé fut comme l’asile de la Cité de Dieu ; et comme, après le dénombrement des généalogies jusqu’à Noé, l’Écriture dit : « Voici la généalogie de Noé »,de même, après le dénombrement des générations de Sem, fils de Noé, jusqu’à Abraham, elle dit : « Voici la généalogie de Tharé. Tharé engendra Abram, Nachor et Aran. Aran engendra Lot, et mourut du vivant de son père Tharé, au lieu de sa naissance, au pays des Chaldéens, Abram et Nachor se marièrent. La femme d’Abram s’appelait Sarra, et celle de Nachor, Melca, fille d’Aran. » Celui-ci eut aussi une autre fille nommée Jesca, que l’on croit être la même que Sarra, femme d’Abraham.
\subsection[{Chapitre XIII}]{Chapitre XIII}

\begin{argument}\noindent Pourquoi l’Écriture ne parle point de Nachor, quand son père Tharé passa de Chaldée en Mésopotamie.
\end{argument}

\noindent L’Écriture raconte ensuite comment Tharé avec tous les siens laissa la Chaldée, vint en Mésopotamie et demeura à Charra ; mais elle ne parle point de son fils Nachor, comme s’il ne l’avait pas emmené avec lui. Voici de quelle façon elle fait ce récit : « Tharé prit donc son fils Abram, Lot, fils de son fils Aran, et Sarra, sa belle-fille, femme de son fils Abram, et il les emmena de Chaldée en Chanaan, et il vint à Charra où il établit sa demeure. » Il n’est point ici question de Nachor ni de sa femme Melca. Lorsque plus tard Abraham envoya son serviteur chercher une femme à son fils Isaac, nous trouvons ceci : « Le serviteur prit dix chameaux du troupeau de son maître et beaucoup d’autres biens, et se dirigea vers la Mésopotamie, en la ville de Nachor. » Par ce témoignage et plusieurs autres de l’histoire sacrée, il paraît que Nachor sortit de la Chaldée, aussi bien que son frère Abraham, et vint habiter avec lui en Mésopotamie. Pourquoi l’Écriture ne parle-t-elle donc point de lui, lorsque Tharé passe avec sa famille en Mésopotamie, tandis qu’elle ne marque pas seulement qu’il y mena son fils Abraham, mais encore Sarra, sa belle-fille, et son petit-fils Lot ? pourquoi, si ce n’est peut-être qu’il avait quitté la religion de son père et de son frère pour embrasser la superstition des Chaldéens, qu’il abandonna depuis, ou parce qu’il se repentit de son erreur, ou parce qu’il devint suspect aux habitants du pays et fut obligé d’en sortir, afin d’éviter leur persécution. En effet, dans le livre de Judith, quand Holopherne, ennemi des Israélites, demande quelle est cette nation et s’il lui faut faire la guerre, voici ce que lui dit Achior, général des Ammonites : « Seigneur, si vous voulez avoir la bonté de m’entendre, je vous dirai ce qui en est de ce peuple qui demeure dans ces montagnes prochaines, et je ne vous dirai rien que de très vrai. Il tire son origine des Chaldéens ; et comme il abandonna la religion de ses pères pour adorer le Dieu du ciel, les Chaldéens le chassèrent, et il s’enfuit en Mésopotamie, où il demeura longtemps. Ensuite leur Dieu leur commanda d’en sortir, et de s’en aller en Chanaan, où ils s’établirent, etc. » On voit clairement par là que la maison de Tharé fut persécutée par les Chaldéens, à cause de la religion et du culte du vrai Dieu.
\subsection[{Chapitre XIV}]{Chapitre XIV}

\begin{argument}\noindent Des années de Tharé, qui mourut à Charra.
\end{argument}

\noindent Or, après la mort de Tharé, qui vécut, dit-on, deux cent cinq ans en Mésopotamie, l’Écriture commence à parler des promesses que Dieu fit à Abraham ; elle s’exprime ainsi : « Tout le temps de la vie, de Tharé à Charra fut de deux cent cinq ans, puis il mourut. » Il ne faut pas entendre ce passage comme si Tharé avait passé tout ce temps à Charra ; l’Écriture dit seulement qu’il y finit sa vie, qui fut en tout de deux cent cinq ans : on ignorerait autrement combien il a vécu, puisque l’on ne voit point quel âge il avait quand il vint dans cette ville ; et il serait absurde de s’imaginer que, dans une généalogie qui énonce si scrupuleusement le temps que chacun a vécu, il fût le seul oublié. Cette omission, il est vrai, a lieu pour quelques-uns ; mais c’est qu’ils n’entrent point dans l’ordre de ceux qui composent la série de générations depuis Adam jusqu’à Noé, et depuis Noé jusqu’à Abraham : il n’est aucun de ces derniers dont l’Écriture ne marque l’âge.
\subsection[{Chapitre XV}]{Chapitre XV}

\begin{argument}\noindent Du temps de promission où Abraham sortit de Charra, d’après l’ordre de Dieu.
\end{argument}

\noindent L’Écriture, après avoir parlé de la mort de Tharé, père d’Abraham, ajoute : « Et Dieu dit à Abram : Sortez de votre pays, de votre parenté et de la maison de votre père. » Il ne faut pas penser que cela soit arrivé dans l’ordre qu’elle rapporte ; cette opinion donnerait lieu à une difficulté insoluble.\par
En effet, à la suite de ce commandement de Dieu à Abraham, on lit dans la Genèse : « Abram sortit donc avec Lot pour obéir aux paroles de Dieu ; et Abram avait soixante-quinze ans lorsqu’il sortit de Charra. » Comment cela se peut-il, si la chose arriva après la mort de Tharé ? Tharé avait soixante-dix ans quand il engendra Abraham ; si l’on ajoute les soixante-quinze ans qu’avait Abraham lorsqu’il partit de Charra, on a cent quarante-cinq ans. Tharé avait donc cet âge à l’époque où son fils quitta cette ville de Mésopotamie. Ce dernier n’en sortit donc pas après la mort de son père, qui vécut deux cent cinq ans : il faut entendre dès lors quec’est ici une récapitulation assez ordinaire dans l’Écriture, qui, parlant auparavant des enfants de Noé, après avoir dit qu’ils furent divisés en plusieurs langues et nations, ajoute : « Toute la terre parlait un même langage. » Comment étaient-ils divisés en plusieurs langues, si toute la terre ne parlait qu’un même langage, sinon parce que la Genèse reprend ce qu’elle avait déjà touché ? Elle procède de même dans la circonstance qui nous occupe elle a parlé plus haut de la mort de Tharé, mais elle revient à la vocation d’Abraham, qui arriva du vivant de son père, et qu’elle avait omise pour ne point interrompre le fil de son discours. Ainsi, lorsque Abraham sortit de Charra, il avait soixante-quinze ans, et son père cent quarante-cinq. D’autres ont résolu autrement la question : selon eux, les soixante-quinze années de la vie d’Abraham doivent se compter du jour qu’il fut délivré du feu où il fut jeté par les Chaldéens pour ne vouloir pas adorer cet élément, et non du jour de sa naissance, comme n’ayant proprement commencé à naître qu’alors.\par
Mais saint Étienne dit, touchant la vocation d’Abraham, dans les Actes des Apôtres : « Le Dieu de gloire apparut à notre père Abraham lorsqu’il était en Mésopotamie, avant qu’il demeurât à Charra, et lui dit : Sortez de votre pays, et de votre parenté, et de la maison de votre père, et venez en la terre que je vous montrerai. » Ces paroles de saint Étienne font voir que Dieu ne parla pas à Abraham après la mort de son père, qui mourut à Charra, où Abraham demeura avec lui, mais avant qu’il habitât cette ville, bien qu’il fût déjà en Mésopotamie. Il en résulte toujours qu’il était alors sorti de la Chaldée ; et ainsi ce que saint Étienne ajoute : « Alors Abraham sortit du pays des Chaldéens et vint demeurer à Charra », ne montre pas ce qui arriva après que Dieu lui eut parlé (car il ne sortit pas de la Chaldée après cet avertissement du ciel, puisque saint Étienne dit qu’il le reçut dans la Mésopotamie), mais se rapporte à tout le temps qui se passa depuis qu’il en fut sorti et qu’il eut fixé son séjour à Charra. Ce qui suit le prouve encore : « Etaprès la mort de son père, dit le premier martyr, Dieu l’établit en cette terre que vos pères ont habitée et que vous habitez encore aujourd’hui. » Il ne dit pas qu’il sortit de Charra après la mort de son père, mais que Dieu l’établit dans la terre de Chanaan après que son père fut mort. Il faut dès lors entendre que Dieu parla à Abraham lorsqu’il était en Mésopotamie, avant de demeurer à Charra, où il vint dans la suite avec son père, conservant toujours en son cœur le commandement de Dieu, et qu’il en sortit la soixante-quinzième année de son âge et la cent quarante-cinquième de celui de son père. Saint Étienne place son établissement dans la terre de Chanaan, et non sa sortie de Charra, après la mort de son père, parce que son père était déjà mort, quand il acheta cette terre et commença à la posséder en propre. Ce que Dieu lui dit : « Sortez de votre pays, de votre parenté et de la maison de votre père », bien qu’il fût déjà sorti de la Chaldée et qu’il demeurât en Mésopotamie, ce n’était pas un ordre d’en sortir de corps, car il l’avait déjà fait, mais d’y renoncer sans retour. Il est assez vraisemblable qu’Abraham sortit de Charra avec sa femme Sarra, et Lot, son neveu, pour obéir à l’ordre de Dieu, après que Nachor eut suivi son père.
\subsection[{Chapitre XVI}]{Chapitre XVI}

\begin{argument}\noindent Des promesses que dieu fit à Abraham.
\end{argument}

\noindent Il faut parler maintenant des promesses que Dieu fit à Abraham et où apparaissent clairement les oracles de notre Dieu, c’est-à-dire du vrai Dieu, en faveur du peuple fidèle annoncé par les Prophètes. La première est conçue en ces termes : « Le Seigneur dit à Abraham : Sortez de votre pays, de votre parenté, et de la maison de votre père, et allez en la terre que je vous montrerai. Je vous établirai chef d’un grand peuple ; je vous bénirai, et rendrai votre nom illustre en vertu de cette bénédiction. Je bénirai ceux qui vous béniront, et maudirai ceux qui vous maudiront, et toutes les nations de la terre seront bénies en vous. » Il est à remarquer ici que deux choses sont promises à Abraham : l’une, que sa postérité possédera la terre de Chanaan, ce qui est exprimé par ces mots : « Allez en la terre que je vousmontrerai, et je vous établirai chef d’un grand peuple » ; et l’autre, beaucoup plus excellente et qu’on ne doit pas entendre d’une postérité charnelle, mais spirituelle, qui ne le rend pas seulement père du peuple d’Israël, mais de toutes les nations qui marchent sur les traces de sa foi. Or, celle-ci est renfermée dans ces paroles : « Toutes les nations de la terre seront bénies en vous. » Eusèbe pense que cette promesse fut faite à Abraham la soixante-quinzième année de son âge, comme s’il était sorti de Charra aussitôt qu’il l’eut reçue, et cette opinion a pour but de ne point contrarier la déclaration formelle de l’Écriture qui dit qu’Abraham avait soixante-quinze ans quand il sortit de Charra ; mais si la promesse en question fut faite cette année, Abraham demeurait donc déjà avec son père à Charra, attendu qu’il n’en eût pas pu sortir, s’il n’y eût été. Cela n’a rien de contraire à ce que dit saint Étienne : « Le Dieu de gloire apparut à notre père Abraham lorsqu’il était en Mésopotamie avant de demeurer à Charra » ; il s’agit seulement de rapporter à la même année et la promesse de Dieu à Abraham qui précède son départ pour Charra et son séjour en cette ville et sa sortie du même lieu. Nous devons l’entendre ainsi, non seulement parce qu’Eusèbe, dans sa {\itshape Chronique}, commence à compter depuis l’an de cette promesse et montre qu’il s’écoula quatre cent trente années jusqu’à la sortie d’Égypte, époque où la loi fut donnée, mais aussi parce que l’apôtre saint Paul suppute de la même manière.
\subsection[{Chapitre XVII}]{Chapitre XVII}

\begin{argument}\noindent Des trois monarchies qui florissaient du temps d’Abraham, et notamment de celle des Assyriens.
\end{argument}

\noindent En ce temps-là, il y avait trois puissants empires où florissait merveilleusement la cité de la terre, c’est-à-dire l’assemblée des hommes qui vivent selon l’homme sous la domination des anges prévaricateurs, savoir : ceux des Sicyoniens, des Égyptiens et des Assyriens. Celui-ci était le plus grand et le plus puissant de tous ; car Ninus, fils de Bélus, avait subjugué toute l’Asie, à la réserve des Indes. Parl’Asie, je n’entends pas parler de celle qui n’est maintenant qu’une province de la seconde partie de la terre (ou, selon d’autres, de la troisième), mais de cette troisième partie elle-même, le monde étant ordinairement partagé en trois grandes divisions, l’Asie, l’Europe et l’Afrique, qui ne forment pas au reste trois portions égales. L’Asie s’étend du midi par l’orient jusqu’au septentrion ; au lieu que l’Europe ne s’étend que du septentrion à l’occident, et l’Afrique de l’occident au midi, de sorte qu’il semble que l’Europe et l’Afrique n’occupent ensemble qu’une partie de la terre et que l’Asie toute seule occupe l’autre. Mais on a fait deux parties de l’Europe et de l’Afrique, à cause qu’elles sont séparées l’une de l’autre par la mer Méditerranée. En effet, si l’on divisait tout le monde en deux parties seulement, l’orient et l’occident, l’Asie tiendrait l’une, et l’Europe et l’Afrique l’autre. Ainsi, des trois monarchies qui existaient alors, celle des Sicyoniens n’était pas sous les Assyriens, parce qu’elle était en Europe : mais comment l’Égypte ne leur était-elle pas soumise, puisqu’ils étaient maîtres de toute l’Asie, aux Indes près ? C’est donc principalement dans l’Assyrie que florissait alors la cité de la terré, cité impie dont la capitale était Babylone, c’est-à-dire Confusion, nom qui lui convient parfaitement. Ninus en était roi et avait succédé à son père Bélus, qui avait tenu le sceptre soixante-cinq ans : lui-même régna cinquante-deux ans, et en avait déjà régné quarante-trois lorsqu’Abraham vint au monde, c’est-à-dire environ douze cents ans avant la fondation de Rome, qui fut comme la Babylone d’Occident.
\subsection[{Chapitre XVIII}]{Chapitre XVIII}

\begin{argument}\noindent De la seconde apparition de Dieu a Abraham, à qui il promet la terre de Chanaan pour lui et sa postérité.
\end{argument}

\noindent Abraham sortit donc de Charra la soixante-quinzième année de son âge, et la cent quarante-cinquième de celui de son père, et passa avec Lot, son neveu, et sa femme Sarra, dans la terre de Chanaan jusqu’à Sichem, où il reçut encore un avertissement du ciel, que l’Écriture rapporte ainsi : « Le Seigneur apparut à Abraham, et lui dit : Je donneraicette terre à votre postérité. » Il ne lui est rien dit ici de cette postérité qui devait le rendre père de toutes les nations, mais seulement de celle qui le rendait père du peuple hébreu : c’est en effet ce peuple qui a possédé la terre de Chanaan.
\subsection[{Chapitre XIX}]{Chapitre XIX}

\begin{argument}\noindent De la pudicité de Sarra, que Dieu protège en Égypte, où Abraham la faisait passer, non pour sa femme, mais pour sa sœur.
\end{argument}

\noindent Lorsque ensuite Abraham eut dressé un autel en cet endroit et invoqué Dieu, il alla demeurer au désert, d’où, pressé de la faim, il passa en Égypte. Là il dit que Sarra était sa sœur, ce qui était vrai parce qu’elle était sa cousine germaine, de même que Lot, qui le touchait au même degré, est aussi appelé son frère. Il dissimula donc qu’elle était sa femme, mais il ne le nia pas, remettant à Dieu le soin de son honneur, et se gardant comme homme des insultes des hommes. S’il n’eût pris en cette rencontre toutes les précautions possibles, il aurait plutôt tenté Dieu que témoigné sa confiance en lui. Nous avons dit beaucoup de choses à ce sujet en répondant aux calomnies de Fauste le manichéen. Aussi arriva-t-il ce qu’Abraham s’était promis de Dieu, puisque Pharaon, roi d’Égypte, qui avait choisi Sarra pour épouse, frappé de plusieurs plaies, la rendit à son mari. Loin de nous la pensée que sa chasteté ait reçu aucun outrage de ce prince, tout portant à croire qu’il en fut détourné par ces fléaux du ciel.
\subsection[{Chapitre XX}]{Chapitre XX}

\begin{argument}\noindent De la séparation d’Abraham et de Lot, qui eut lieu sans rompre leur union.
\end{argument}

\noindent Lorsque Abraham fut retourné d’Égypte dans le lieu d’où il était sorti, Lot, son neveu, se sépara de lui sans rompre la bonne intelligence qui était entre eux, et se retira vers Sodome. Les richesses que tous deux avaient acquises et les fréquents démêlés de leurs bergers les déterminèrent à prendre ce parti, afin d’empêcher que les querelles des serviteurs ne vinssent à jeter la désunion parmi les maîtres. Abraham, voulant prévenir ce malheur, dit à Lot : « Je vous prie, qu’il n’y ait point de différend entre vous et moi, ni entre vos bergers et les miens, puisque nous sommes frères. Toute cette contrée n’est-elle pas à nous ? Je suis donc d’avis que nous nous séparions. Si vous allez à gauche, j’irai à droite ; et si vous allez à droite, j’irai à gauche. » Il se peut que la coutume reçue dans les partages, où l’aîné fait les lots et le cadet choisit de là son origine.
\subsection[{Chapitre XXI}]{Chapitre XXI}

\begin{argument}\noindent De la troisième apparition de Dieu a Abraham, où il lui réitère la promesse de la terre de Chanaan pour lui et ses descendants à perpétuité.
\end{argument}

\noindent Après qu’Abraham et Lot se furent ainsi séparés et que l’un se fut fixé dans la terre de Chanaan et l’autre à Sodome, Dieu apparut à Abraham pour la troisième fois, et lui dit : « Regardez de tous côtés, autant que votre vue peut s’étendre vers les quatre points du monde ; je vous donnerai, à vous et à tous vos descendants jusqu’à la fin du siècle, toute cette terre que vous voyez, et je multiplierai votre postérité comme la poussière de la terre. Si quelqu’un peut compter les grains de poussière de la terre, il pourra aussi compter votre postérité. Levez-vous, et mesurez cette terre en long et en large, car je vous la donnerai. » On ne voit pas bien si, dans cette promesse, est comprise celle qui a rendu Abraham père de toutes les nations ; on peut néanmoins le conjecturer d’après ces paroles : « Je multiplierai votre postérité comme la poussière de la terre », expression figurée que les Grecs appellent hyperbole et qui a lieu quand ce qu’on dit d’une chose la surpasse de beaucoup. Qui ne sait combien la poussière de la terre surpasse le nombre des hommes, quel qu’il puisse être, depuis Adam jusqu’à la fin du siècle, et à plus forte raison la postérité d’Abraham, soit la charnelle, soit la spirituelle ? En effet, cette dernière postérité est peu de chose en comparaison de la multitude des méchants, et cependant, malgré sa petitesse, elle forme encore un nombre innombrable, d’où vient que l’Écriture la désigne par la poussière de la terre. Mais elle n’est innombrable qu’aux hommes, et non à Dieu, qui sait même le compte de tous les grains depoussière. Ainsi, comme l’hyperbole de l’Écriture est mieux remplie par les deux postérités d’Abraham, on peut croire que cette promesse s’applique à l’une et à l’autre. Si j’ai dit que cela n’est pas très clair, c’est que le seul peuple juif a tellement multiplié qu’il s’est presque répandu dans toutes les contrées du monde, de sorte qu’il suffit pour justifier l’hyperbole, outre qu’on ne peut pas nier que la terre dont il est question ne soit celle de Chanaan. Néanmoins, ces mots : « Je vous la donnerai, à vous et à vos descendants jusqu’à la fin du siècle », peuvent en faire douter, si, par cette expression, {\itshape jusqu’à la fin du siècle}, on entend {\itshape éternellement} ; mais si on les prend comme nous pour la fin de ce monde et le commencement de l’autre, il n’y a point de difficulté. Bien que les Juifs aient été chassés de Jérusalem, ils demeurent dans les autres villes de la terre de Chanaan et y demeureront jusqu’à la fin du monde ; ajoutez à cela que, quand cette terre est habitée par des chrétiens, c’est la postérité d’Abraham qui l’habite.
\subsection[{Chapitre XXII}]{Chapitre XXII}

\begin{argument}\noindent Abraham sauve Lot des mains des ennemis et est béni par Melchisédech.
\end{argument}

\noindent Abraham, après avoir reçu cette promesse, alla demeurer en un autre endroit de cette contrée, près du chêne de Mambré, qui était en Hébron. Ensuite, les ennemis ayant ravagé le pays de Sodome et vaincu les habitants en bataille rangée, Abraham, accompagné de trois cent dix-huit des siens, alla au secours de Lot, que les vainqueurs avaient fait prisonnier, et le délivra de leurs mains après les avoir défaits, sans vouloir rien prendre des dépouilles que le roi de Sodome lui offrait. C’est en cette occasion qu’il fut béni par Melchisédech, prêtre du Dieu souverain, dont il est beaucoup parlé dans l’Épître aux Hébreux, que plusieurs disent être de saint Paul, ce dont quelques-uns ne tombent pas d’accord. On vit là pour la première fois le sacrifice que les chrétiens offrent aujourd’hui à Dieu par toute la terre, pour accomplir cette parole du Prophète à Jésus-Christ, qui ne s’était pas encore incarné : « Vous êtes prêtrepour jamais selon l’ordre de Melchisédech. » Il ne dit pas selon l’ordre d’Aaron, lequel devait être aboli par la vérité dont ces ombres étaient la figure.
\subsection[{Chapitre XXIII}]{Chapitre XXIII}

\begin{argument}\noindent Dieu promet à Abraham que sa postérité sera aussi nombreuse que les étoiles, et la foi d’Abraham aux paroles de Dieu le justifie, quoique non circoncis.
\end{argument}

\noindent Dieu parla encore à Abraham dans une vision, et l’assura de sa protection et d’une ample récompense ; et comme Abraham se plaignit à lui qu’il était déjà vieux, qu’il mourrait sans postérité, et qu’Éliézer, l’un de ses esclaves, serait son héritier, Dieu lui promit qu’il aurait un fils, et que sa postérité serait aussi nombreuse que les étoiles du ciel ; par où il me semble que Dieu voulait spécialement désigner la postérité spirituelle d’Abraham. Que sont, en effet, les étoiles, pour le nombre, en comparaison de la poussière de la terre, à moins qu’on ne veuille dire qu’il y a ici cette ressemblance qu’on ne peut compter les étoiles et que l’on ne saurait même toutes les voir ? On en découvre à la vérité d’autant plus qu’on a de meilleurs yeux ; mais il résulte précisément de là qu’il en échappe toujours quelques-unes aux plus clairvoyants, sans parler de celles qui se lèvent et se couchent dans l’autre hémisphère. C’est donc une rêverie de s’imaginer qu’il y en a qui ont connu et mis par écrit le nombre des étoiles, comme on le dit d’Aratus et d’Euxode ; et l’Écriture sainte suffit pour réfuter cette opinion. Au reste, c’est dans ce chapitre de la Genèse que se trouve la parole que l’Apôtre rappelle pour relever la grâce de Dieu : « Abraham crut Dieu, et sa foi lui fut imputée à justice » ; et il prouve par là que les Juifs ne devaient point se glorifier de leur circoncision, ni empêcher que les incirconcis ne fussent admis à la foi de Jésus-Christ, puisque, quand la foi d’Abraham lui fut imputée à justice, il n’était pas encore circoncis.
\subsection[{Chapitre XXIV}]{Chapitre XXIV}

\begin{argument}\noindent Ce que signifie le sacrifice que Dieu commanda à Abraham de lui offrir, quand ce patriarche le pria de lui donner quelque signe de l’accomplissement de sa promesse.
\end{argument}

\noindent Dans cette même vision, Dieu lui dit encore : « Je suis le Dieu qui vous ai tiré du pays des Chaldéens, pour vous donner cette terre et vous en mettre en possession. » Sur quoi, Abraham lui ayant demandé comment il connaîtrait qu’il la devait posséder, Dieu lui répondit : « Prenez une génisse de trois ans, une chèvre et un bélier de même âge, avec une tourterelle et une colombe. » Abraham prit tous ces animaux ; et, après les avoir divisés en deux, mit ces moitiés vis-à-vis l’une de l’autre ; mais il ne divisa point les oiseaux. Alors, comme il est écrit, les oiseaux descendirent sur ces corps qui étaient divisés, et Abraham s’assit auprès d’eux. Sur le coucher du soleil il fut saisi d’une grande frayeur qui le couvrit de ténèbres épaisses, et il lui fut dit : « Sachez que votre postérité demeurera parmi des étrangers qui la persécuteront et la réduiront en servitude l’espace de quatre cents ans ; mais je ferai justice de leurs oppresseurs, et elle sortira de leurs mains, chargée de dépouilles. Pour vous, vous vous en irez en paix avec vos pères, comblé d’une heureuse vieillesse, et vos descendants ne reviendront ici qu’à la quatrième génération, car les Amorrhéens n’ont pas encore comblé la mesure de leurs crimes. » Comme le soleil fut couché, une flamme s’éleva tout à coup et l’on vit une fournaise fumante et des brandons de feu qui passèrent au milieu des animaux divisés. Ce jour-là, Dieu fit alliance avec Abraham et lui dit : « Je donnerai cette terre à vos enfants, depuis le fleuve d’Égypte jusqu’au grand fleuve d’Euphrate ; je leur donnerai les Cénéens, les Cénézéens, les Cedmonéens, les Céthéens, les Phéréséens, les Raphaïms, les Amorrhéens, les Chananéens, les Évéens, les Gergéséens et les Jébuséens. »\par
Voilà ce qui se passa dans cette vision ; mais l’expliquer en détail nous mènerait trop loin et passerait toutes les bornes de cet ouvrage. Il suffira de dire ici qu’Abraham ne perdit pas la foi dont l’Écriture le loue, pour avoir dit à Dieu : « Seigneur, comment connaîtrai-je que je dois posséder cette terre ? » Il ne dit pas : Comment se pourra-t-il faire que je la possède ? comme s’il doutait de la promesse de Dieu, mais : Comment connaîtrai-je que je dois la posséder ? afin d’avoir quelque signe qui lui fit connaître la manière dont cela devait se passer : de même que la Vierge Marie n’entra en aucune défiance de ce que l’ange lui annonçait, quand elle dit : « Comment cela se fera-t-il, car je ne connais point d’homme ? » Elle ne doutait point de la chose, mais elle s’informait de la manière. C’est pourquoi l’ange lui répondit : « Le Saint-Esprit surviendra en vous, et la vertu du Très-Haut vous couvrira de son ombre. » Ici, de même, Dieu donna à Abraham le signe d’animaux immolés, comme la figure de ce qui devait arriver et dont il ne doutait pas. Par la génisse était signifié le peuple juif soumis au joug de la loi ; par la chèvre, le même peuple pécheur, et par le bélier, le même encore régnant et dominant. Ces animaux ont trois ans, à cause des trois époques fort remarquables : depuis Adam jusqu’à Noé, depuis Noé jusqu’à Abraham, et depuis Abraham jusqu’à David, qui, le premier d’entre les Israélites, monta sur le trône par la volonté de Dieu après la réprobation de Saül, dernière époque durant laquelle ce peuple prit ses plus grands accroissements. Que cela figuré ce que je dis, ou toute autre chose, au moins ne douté-je point que les hommes spirituels ne soient désignés par la tourterelle et par la colombe ; d’où vient qu’il est dit qu’Abraham ne divisa point les oiseaux. En effet, les charnels sont divisés entre eux, mais non les spirituels, soit qu’ils se retirent du commerce des hommes, comme la tourterelle, soit qu’ils vivent avec eux, comme la colombe. Quoi qu’il en soit, l’un comme l’autre de ces deux oiseaux est simple et innocent ; et ils étaient un signe que, même dans ce peuple juif, à qui cette terre devait être donnée, il y aurait des enfants de promission et des héritiers du royaume et de la félicité éternelle. Pour les oiseaux qui descendirent sur ces corps divisés, ils figurent les malins esprits, habitants de l’air et toujours empressés de se repaître de la division des hommes charnels.\par
Abraham, venant s’asseoir auprès d’eux, signifie que, même au milieu de ces divisions des hommes charnels, il y aura toujours quelques vrais fidèles jusqu’à la fin du monde. Par la frayeur dont Abraham fut saisi vers le coucher du soleil, entendez que, vers la fin du monde, il s’élèvera une cruelle persécution contre les fidèles, selon cette parole de Notre-Seigneur dans l’Évangile : « La persécution sera si grande alors, qu’il n’y en a jamais eu de pareille. »\par
Quant à ces paroles de Dieu à Abraham : « Sachez que votre postérité demeurera parmi des étrangers qui la persécuteront et la tiendront captive l’espace de quatre cents ans », cela s’entend sans difficulté du peuple juif qui devait être captif en Égypte. Ce n’est pas néanmoins que sa captivité ait duré quatre cents ans, mais elle devait arriver dans cet espace de temps ; de même que l’Écriture dit de Tharé, père d’Abraham, que tout le temps de sa vie à Charra fut de deux cent cinq ans, non qu’il ait passé toute sa vie en ce lieu, mais parce qu’il y acheva le reste de ses jours. Au reste, l’Écriture dit quatre cents ans pour faire un compte rond, car il y en a un peu plus, soit qu’on les prenne du temps que cette promesse fut faite à Abraham, ou du temps de la naissance d’Isaac. Ainsi que nous l’avons déjà dit, depuis la soixante-quinzième année de la vie d’Abraham que la première promesse lui fut faite, jusqu’à la sortie d’Égypte, on compte quatre cent trente ans, dont l’Apôtre parle ainsi : « Ce que je veux dire, c’est que Dieu ayant contracté une alliance avec Abraham, la loi, qui n’a été donnée que quatre cents ans après, ne l’a pu rendre nulle, ni anéantir la promesse faite à ce patriarche. » L’Écriture a donc fort bien pu appeler ici quatre cents ans ces quatre cent trente ans ; outre que depuis la première promesse faite à Abraham jusqu’à celle-ci, cinq années s’étaient déjà écoulées, et vingt-cinq jusqu’à la naissance d’Isaac.\par
Ce qu’elle ajoute que le soleil étant déjà couché, une flamme s’éleva tout d’un coup, et que l’on vit une fournaise fumante et des brandons de feu qui passèrent au milieu des animaux divisés, cela signifie qu’à la fin du monde les charnels seront jugés par le feu. De même, en effet, que la persécution de la Cité de Dieu, qui sera la plus grande de toutes sous l’Antéchrist, est marquée par cette frayeur extraordinaire qui saisit Abraham sur le coucher du soleil, symbole de la fin du monde, ainsi ce feu, qui parut après que le soleil fut couché, marque le jour du jugement qui séparera les hommes charnels que le feu doit sauver, de ceux qui sont destinés à être damnés dans ce feu. Enfin, l’alliance de Dieu avec Abraham, signifie proprement la terre de Chanaan, où onze nations sont nommées depuis le fleuve d’Égypte jusqu’au grand fleuve d’Euphrate. Or, par le fleuve d’Égypte, il ne faut pas entendre le Nil, mais un petit fleuve qui la sépare de la Palestine et passe à Rhinocorure.
\subsection[{Chapitre XXV}]{Chapitre XXV}

\begin{argument}\noindent D’Agar, servante de Sarra, que Sarra donna pour concubine à son mari.
\end{argument}

\noindent Viennent ensuite les enfants d’Abraham, l’un de la servante Agar, et l’autre de Sarra, la femme libre, dont nous avons déjà parlé au livre précédent. En ce qui touche les rapports d’Abraham avec Agar, on ne doit point les lui imputer à crime, puisqu’il ne se servit de cette concubine que pour en avoir des enfants, et non pour contenter sa passion, et plutôt pour obéir à sa femme que dans l’intention de l’outrager. Elle-même crut en quelque façon se consoler de sa stérilité en s’appropriant la fécondité de sa servante, et en usant du droit qu’elle avait en cela sur son mari, selon cette parole de l’Apôtre : « Le mari n’est point maître de son corps, mais sa femme. » Il n’y a ici aucune intempérance, aucune débauche. La femme donne sa servante à son mari pour en avoir des enfants, le mari la reçoit avec la même intention ; ni l’un ni l’autre ne recherche le dérèglement de la volupté, ils ne songent tous deux qu’au fruit de la nature. Aussi, quand la servante devenue enceinte commença à s’enorgueillir et à mépriser sa maîtresse, comme Sarra, par une défiance de femme, imputait l’orgueil d’Agar à son mari, Abraham fit bien voir de nouveau qu’il n’était pas l’esclave, mais le maître de son amour, qu’il avait gardé, en la personne d’Agar, la foi qu’il devait à Sarra, qu’il n’avait connu la servante que pour obéir à l’épouse, qu’il avait reçu d’elle Agar, mais qu’il ne l’avait pas demandée, qu’il s’en était approché, mais qu’il ne s’y était pas attaché, qu’il avait engendré, mais qu’il n’avait point aimé. Il dit en effet à Sarra : « Votre servante est en votre pouvoir, faites-en ce qu’il vous plaira. » Homme admirable, qui use des femmes comme un homme en doit user, de la sienne avec tempérance, de sa servante avec docilité, et chastement de l’une et de l’autre !
\subsection[{Chapitre XXVI}]{Chapitre XXVI}

\begin{argument}\noindent Dieu promet à Abraham, déjà vieux, un fils de sa femme Sarra, qui était stérile ; il lui annonce qu’il sera le père des nations, et confirme sa promesse par la circoncision.
\end{argument}

\noindent Lorsque dans la suite Ismaël fut né d’Agar, Abraham pouvait croire que cette naissance accomplissait ce qui lui avait été promis dans le temps où, pour le faire renoncer au dessein qu’il avait d’adopter son serviteur, Dieu lui dit : « Celui-ci ne sera pas votre héritier, mais un autre qui sortira de vous. » De peur donc qu’il ne crût que cette promesse fût accomplie dans le fils de sa servante, « comme Abraham était déjà âgé de quatre-vingt-dix-neuf ans, Dieu lui apparut et lui dit : Je suis Dieu, travaillez à me plaire, et menez une vie sans reproche, et je ferai alliance avec vous, et je vous comblerai de tous les biens. Alors Abram se prosterna par terre, et Dieu ajouta : C’est moi, je ferai alliance avec vous, et vous serez le père d’une grande multitude de nations. Vous ne vous appellerez plus Abram, mais Abraham, parce que je vous ai fait le père de plusieurs nations. Je vous rendrai extrêmement puissant, et vous établirai sur un grand nombre de peuples et des rois sortiront de vous. Je ferai alliance avec vous, et après vous avec vos descendants ; et cette alliance sera éternelle, afin que je sois votre Dieu et celui de toute votre postérité. Je donnerai à vous et à vos descendants cette terre où vous êtes maintenant étranger, toute la terre de Chanaan, pour la posséder à jamais, et je serai leur Dieu. Dieu dit encore à Abraham : Pour vous, vous aurez soin de garder mon alliance, et votre postérité après vous. Or, voici l’alliance que je désire que vous et vos enfants observiez soigneusement. Tout mâle parmi vous sera circoncis ; cette circoncision se fera en la chair de votre prépuce, et sera la marque de l’alliance qui est entre vous et moi. Tous les enfants mâles qui naîtront de vous seront circoncis au bout de huit jours. Vous circoncirez aussi les esclaves, tant ceux qui naîtront chez vous que les autres que vous achèterez des étrangers. Et cette circoncision sera une marque de l’alliance éternelle que j’ai contractée avec vous. Tout mâle qui ne la recevra pas le huitième jour sera exterminé comme un infracteur de mon alliance. Dieu dit encore à Abraham : Votre femme ne s’appellera plus Sara, mais Sarra : je la bénirai et vous donnerai d’elle un fils que je bénirai aussi, et qui sera père de plusieurs nations, et des rois sortiront de lui. Là-dessus, Abraham se prosterna en terre, en souriant et disant en lui-même : J’aurai donc un fils à cent ans, et Sarra accouchera à quatre-vingt-dix ? Conservez seulement en vie, dit-il à Dieu, mon fils Ismaël ! Et Dieu lui dit : Oui, votre femme Sarra vous donnera un fils que vous nommerez Isaac. Je ferai une alliance éternelle avec lui, et je serai son Dieu et le Dieu de sa postérité. Pour Ismaël, j’ai exaucé votre prière ; je l’ai béni et je le rendrai extrêmement puissant. Il sera le père de douze nations, et je l’établirai chef d’un grand peuple. Mais je contracterai alliance avec Isaac, dont votre femme Sarra accouchera l’année qui va venir ».\par
On voit ici des promesses plus expresses de la vocation des Gentils en Isaac, en ce fils de promission, qui est un fruit de la grâce et non de la nature, puisqu’il est promis à une femme vieille et stérile. Bien que Dieu concoure aussi aux productions qui se font selon les lois ordinaires de la nature, toutefois, lorsque sa main puissante en répare les défaillances, sa grâce paraît avec beaucoup plus d’éclat. Et parce que cette vocation des Gentils ne devait pas tant arriver par la génération des enfants que par leur régénération, Dieu commanda la circoncision, lorsqu’il promit le fils de Sarra. S’il veut que tous soient circoncis,tant libres qu’esclaves, c’est afin de signifier que cette grâce est pour tout le monde. Que figure, en effet la circoncision, sinon la nature renouvelée et dépouillée de sa vieillesse ? Le huitième jour représente-t-il autre chose que Jésus-Christ, qui ressuscita à la fin de la semaine, c’est-à-dire après le jour du sabbat ? Les noms même du père et de la mère sont changés ; tout respire la nouveauté, et l’Ancien Testament fait pressentir le Nouveau. Qu’est-ce, en effet, que le Nouveau Testament, sinon la manifestation de l’Ancien, et qu’est-ce que celui-ci, sinon la figure de l’autre ? Le rire d’Abraham est un témoignage de joie et non de défiance. Ces mots qu’il dit en son cœur : « J’aurai donc un fils à cent ans, et Sarra accouchera à quatre-vingt-dix », ne sont pas non plus d’un homme qui doute, mais d’un homme qui admire. Quant à ces paroles de Dieu à Abraham : « Je donnerai à vous et à vos descendants cette terre où vous êtes maintenant étranger, toute cette terre de Chanaan, pour la posséder éternellement » ; si l’on demande comment cela s’est accompli ou doit s’accomplir, attendu que la possession d’une chose, quelque longue qu’elle soit, ne peut pas durer toujours ; il faut dire qu’éternel se prend en deux façons, ou pour une durée infinie, ou pour celle qui est bornée par la fin du monde.
\subsection[{Chapitre XXVII}]{Chapitre XXVII}

\begin{argument}\noindent De la réprobation portée contre tout enfant mâle qui n’avait point été circoncis le huitième jour, comme ayant violé l’alliance de Dieu.
\end{argument}

\noindent On peut encore demander comment il faut interpréter ceci : « Tout enfant mâle qui ne sera point circoncis le huitième jour sera exterminé comme infracteur de mon alliance. » Ce n’est point l’enfant qui est coupable, puisque ce n’est pas lui qui a violé l’alliance de Dieu, mais bien les parents qui n’ont pas eu soin de le circoncire. On doit répondre à cela que les enfants même ont violé l’alliance de Dieu, non pas en leur propre personne, mais en la personne de celui par qui tous les hommes ont péché. Aussi bien, il y a d’autres alliances que celles de l’Ancien et du Nouveau Testament, La première alliance que Dieu fit avec l’homme est celle-ci : « Du jour où vous mangerez de ce fruit, vous mourrez » ; ce qui a donné lieu à cette parole de l’Ecclésiastique : « Tout homme vieillira comme un vêtement. » Tel est l’arrêt porté dès l’origine du siècle : « Vous mourrez de mort. » En effet, comment cette parole du Prophète : « J’ai regardé tous les pécheurs du monde comme des prévaricateurs », pourrait-elle s’accorder avec cette autre de saint Paul : « Où il n’y a point de loi, il n’y a point de prévarication », si tous ceux qui pèchent n’étaient pas coupables de la violation de quelque loi ? C’est pourquoi, si les enfants mêmes, comme la foi nous l’enseigne, naissent pécheurs, non pas proprement, mais originellement, d’où résulte la nécessité du baptême pour remettre leurs péchés, il faut croire aussi qu’ils sont prévaricateurs à l’égard de cette loi qui a été donnée dans le paradis terrestre, en sorte qu’il est également vrai de dire qu’où il n’y a point de loi, il n’y a point de prévarication, et que tous les pécheurs du monde sont des prévaricateurs. Ainsi, comme la circoncision était le signe de la régénération, c’est avec justice que le péché originel, qui a violé la première alliance de Dieu, perdait ces enfants, si la régénération ne les sauvait. Il faut donc entendre ainsi ces paroles de l’Écriture : « Tout enfant mâle, etc. », comme si elle disait : Quiconque ne sera point régénéré périra, parce qu’il a violé mon alliance lorsqu’il a péché en Adam avec tous les autres hommes. Si elle avait dit : Parce qu’il a violé cette alliance que je contracte avec vous, on ne pourrait l’entendre que de la circoncision ; mais comme elle n’a point exprimé quelle alliance l’enfant a violée, il est permis de l’entendre de celle dont la violation peut se rapporter à lui par voie de solidarité. Si toutefois quelqu’un prétend que cela doit s’appliquer exclusivement à la circoncision, et que l’enfant qui n’a point été circoncis a violé en cela l’alliance, il faut qu’il cherche une manière raisonnable de dire qu’une personne a violé une alliance, quoique ce ne soit pas elle qui l’ait violée, mais d’autres qui l’ont violée en lui ; outre qu’il est injuste qu’un enfant, qui demeure incirconcis sans qu’il y ait de sa faute, soit réprouvé,à moins qu’on ne remonte à un péché d’origine.
\subsection[{Chapitre XXVIII}]{Chapitre XXVIII}

\begin{argument}\noindent Du changement de nom d’Abraham et de Sarra, lesquels n’étaient point en état, celle-ci à cause de sa stérilité, tous deux à cause de leur âge, d’avoir des enfants, quand ils eurent Isaac.
\end{argument}

\noindent Lors donc qu’Abraham eut reçu de Dieu cette promesse : « Je vous ai rendu père de peuples nombreux, et je veux accroître votre puissance et vous élever sur les nations ; et des rois sortiront de vous, et je vous donnerai de Sarra un fils que je bénirai, et il sera le père de plusieurs nations, et des rois sortiront de lui » ; magnifique promesse que nous voyons maintenant accomplie en Jésus-Christ, Abraham et sa femme changèrent de nom, et l’Écriture ne les appelle plus Abram ni Sara, mais Abraham et Sarra. Elle rend raison de ce changement de nom à l’égard d’Abraham : « Car, dit le Seigneur, je vous ai établi père de plusieurs nations. » C’est le sens du mot Abraham ; pour Abram, qui était son premier nom, il signifie illustre père. L’Écriture ne rend point raison du changement de nom de Sarra, mais les traducteurs hébreux disent que Sara signifie ma princesse, et Sarra, vertu ; d’où vient cette parole de l’épître aux Hébreux : « C’est aussi par la foi que Sarra reçut la vertu de concevoir. » Or, ils étaient tous deux fort âgés, ainsi que l’Écriture le témoigne, et Sarra, qui d’ailleurs était stérile, n’avait plus ses mois, de sorte que, n’eût-elle pas été stérile, elle eût été incapable de concevoir. Une femme, quoique âgée, si elle a encore ses mois, peut avoir des enfants, mais d’un jeune homme, et non d’un vieillard ; et de même un vieillard peut en avoir d’une jeune femme, comme Abraham, après la mort de sa femme, en eut de Céthura, parce qu’il rencontra en elle la fleur de la jeunesse. C’est pourquoi l’Apôtre regarde comme un grand miracle que le corps d’Abraham étant mort, il n’ait pas laissé d’engendrer. Entendez par là que son corps était impuissant pour toute femme arrivée à l’âge de Sarra. Car il n’était mort qu’à cet égard ; autrement c’eût été un cadavre. Il y a une autre solution de cette difficulté : on dit qu’Abraham eut des enfants de Céthura, parce que Dieu lui conserva,après la mort de Sarra, le don de fécondité qu’il avait accordé : mais l’explication que j’ai suivie me semble meilleure ; car s’il est vrai qu’à cette heure un vieillard de cent ans soit hors d’état d’engendrer, il n’en était pas de même alors que les hommes vivaient plus longtemps.
\subsection[{Chapitre XXIX}]{Chapitre XXIX}

\begin{argument}\noindent Des trois anges qui apparurent à Abraham au chêne de Mambré.
\end{argument}

\noindent Dieu apparut encore à Abraham au chêne de Mambré dans la personne de trois hommes, qui indubitablement étaient des anges, quoique plusieurs estiment que l’un d’eux était Jésus-Christ, qui était visible, à les en croire, avant que de s’être revêtu d’une chair. Je tombe d’accord que Dieu, qui est invisible, incorporel et immuable par sa nature, est assez puissant pour se rendre visible aux yeux des hommes, sans aucun changement en son essence, non par soi-même, mais par le ministère de quelqu’une de ses créatures ; mais s’ils prétendent que l’un de ces trois hommes était Jésus-Christ, parce qu’Abraham s’adressa à tous trois comme s’ils n’eussent été qu’un seul homme, ainsi que le rapporte l’Écriture : « Il aperçut trois hommes auprès de lui, et aussitôt il courut au-devant d’eux, et dit : Seigneur, si j’ai trouvé grâce auprès de vous… », cette présomption n’a rien de concluant ; car la même Écriture témoigne que deux de ces anges étaient déjà partis pour détruire Sodome, lorsqu’Abraham s’adressa au troisième et l’appela son Seigneur, le conjurant de ne vouloir pas confondre l’innocent avec le coupable et de pardonner à Sodome. En outre, lorsque Lot parle aux deux premiers anges, il le fait comme s’il ne parlait qu’à un seul. Après qu’il leur a dit : « Seigneur, venez, s’il vous plaît, dans la maison de votre serviteur », l’Écriture ajoute : « Les anges le prirent par la main, lui, sa femme et ses deux filles, parce que Dieu lui faisait grâce. Et aussitôt qu’ils l’eurent tiré hors de la ville, ils lui dirent : Sauvez-vous, ne regardez pointderrière vous, et ne demeurez point dans toute cette contrée ; sauvez-vous dans la montagne, de peur que vous ne soyez enveloppé dans cette ruine. » Et Lot leur dit : « Je vous prie, Seigneur, puisque votre serviteur a trouvé grâce auprès de vous, etc. » Ensuite le Seigneur lui répond aussi au singulier, par la bouche de ces deux anges en qui il était, et lui dit : « J’ai eu pitié de vous. » Il est bien plus croyable qu’Abraham et Lot reconnurent le Seigneur en la personne de ses anges, et que c’est pour cela qu’ils lui adressèrent la parole. Au surplus, ils prenaient ces anges pour des hommes ; ce qui fit qu’ils les reçurent comme tels et les traitèrent comme s’ils avaient besoin de nourriture ; mais d’un autre côté, il paraissait en eux quelque chose de si extraordinaire que ceux qui exerçaient ce devoir d’hospitalité à leur égard ne pouvaient douter que Dieu ne fût présent en eux, comme il a coutume de l’être dans ses prophètes. De là vient qu’ils les appelaient quelquefois Seigneurs au pluriel en les regardant comme les ministres de Dieu, et d’autrefois Seigneur au singulier, en considérant Dieu même qui était en eux. Or, l’Écriture témoigne que c’étaient des anges, et ne le témoigne pas seulement dans la Genèse, où cette histoire est rapportée, mais aussi dans l’épître aux Hébreux, où faisant l’éloge de l’hospitalité : « C’est, dit-elle, en pratiquant cette vertu que quelques-uns, sans le savoir, ont reçu chez eux des anges mêmes. » Ce fut donc par ces trois hommes que Dieu, réitérant à Abraham la promesse d’un fils nommé Isaac qu’il devait avoir de Sarra, lui dit : « Il sera chef d’un grand peuple, et toutes les nations de la terre seront bénies en lui. » Paroles qui contiennent une promesse pleine et courte du peuple d’Israël, selon la chair, et de toutes les nations, selon la foi.
\subsection[{Chapitre XXX}]{Chapitre XXX}

\begin{argument}\noindent Destruction de Sodome ; délivrance de Lot ; convoitise infructueuse d’Abimélech pour Sarra.
\end{argument}

\noindent Lot étant sorti de Sodome après cette promesse, une pluie de feu tomba du ciel et réduisit en cendre ces villes infâmes, où le débordement était si grand que l’amour contrenature y était aussi commun que les autres actions autorisées par les lois. Ce châtiment effroyable fut une image du jugement dernier. Pourquoi, en effet, ceux qui échappèrent de cette ruine reçurent-ils des anges l’ordre de ne point regarder derrière eux, sinon parce que, si nous voulons éviter la rigueur du jugement à venir, nous ne devons pas retourner par nos désirs aux habitudes du vieil homme dont nous nous sommes dépouillés par la grâce du baptême. Aussi la femme de Loi, ayant contrevenu à ce commandement, fut punie sur-le-champ, et son changement en statue de sel est un avertissement très sensible donné aux fidèles pour qu’ils aient à se garantir d’un semblable malheur. Dans la suite, Abraham, à Gérara, employa, pour préserver sa femme, le même moyen dont il s’était servi en Égypte ; en sorte qu’Abimélech, roi de ces pays, lui rendit Sarra sans l’avoir touchée. Et comme il blâmait Abraham de son stratagème, celui-ci, tout en avouant que la crainte l’avait obligé d’en user de la sorte, ajouta : « De plus, elle est vraiment ma sœur, car elle est fille de mon père, quoiqu’elle ne le soit pas de ma mère. » En effet, Sarra, du côté de son père, était sœur d’Abraham et une de ses plus proches parentes ; et elle était si belle que même à cet âge, elle pouvait inspirer de l’amour.
\subsection[{Chapitre XXXI}]{Chapitre XXXI}

\begin{argument}\noindent De la naissance d’Isaac, dont le nom exprime la joie éprouvée par ses parents.
\end{argument}

\noindent Après cela, un fils naquit à Abraham de sa femme Sarra, selon la promesse de Dieu, et il le nomma Isaac, nom qui signifie {\itshape rire}, car le père avait ri quand un fils lui fut promis, témoignant par là sa joie et son contentement, et la mère avait ri aussi quand la promesse lui fut réitérée par les trois anges, quoique ce rire fût mêlé de doute, comme l’auge le lui reprocha. Mais ce doute fut ensuite dissipé par l’ange. Voilà d’où Isaac prit son nom. Sarra montre bien que ce rire n’était pas un rire de moquerie, mais de joie, lorsqu’elle dit, à la naissance d’Isaac : « Dieu m’a fait rire, car quiconque saura ceci se réjouira avec moi. » Peu de temps après, la servantefut chassée de la maison avec son fils ; et l’Apôtre voit ici une figure des deux Testaments, où Sarra représente la Jérusalem céleste, c’est-à-dire la Cité de Dieu.
\subsection[{Chapitre XXXII}]{Chapitre XXXII}

\begin{argument}\noindent Obéissance et foi d’Abraham éprouvées par le sacrifice de son fils ; mort de Sarra.
\end{argument}

\noindent Cependant Dieu tenta Abraham en lui commandant de lui sacrifier son cher fils Isaac, afin d’éprouver son obéissance et de la faire connaître à toute la postérité. Car il ne faut pas répudier toute tentation, mais au contraire on doit se réjouir de celle qui sert d’épreuve à la vertu. En effet, l’homme, le plus souvent, ne se connaît pas lui-même sans ces sortes d’épreuves ; mais s’il reconnaît en elles la main puissante de Dieu qui l’assiste, c’est alors qu’il est véritablement pieux, et qu’au lieu de s’enfler d’une vaine gloire, il est solidement affermi dans la vertu par, la grâce. Abraham savait fort bien que Dieu ne se plaît point à des victimes humaines ; mais quand il commande, il est question d’obéir et non de raisonner. Abraham crut donc que Dieu était assez puissant pour ressusciter son fils, et on doit le louer de cette foi. En effet, quand il hésitait à chasser de sa maison sa servante et son fils, sur les vives sollicitations de Sarra, Dieu lui dit « C’est d’Isaac que sortira votre postérité. » Cependant il ajouta tout de suite : « Je ne laisserai pas d’établir sur une puissante nation le fils de cette servante, parce que c’est votre postérité. » Comment Dieu peut-il assurer que c’est d’Isaac que sortira la postérité d’Abraham, tandis qu’il semble en dire autant d’Ismaël ? L’Apôtre résout cette difficulté, quand, expliquant ces paroles : « C’est d’Isaac que sortira votre postérité », il dit : « Cela signifie que ceux qui sont enfants d’Abraham selon la chair ne sont pas pour cela enfants de Dieu ; mais qu’il n’y a de vrais enfants d’Abraham que a ceux qui sont enfants de la promesse. » Dès lors, pour que les enfants de la promesse soient la postérité d’Abraham, il faut qu’ils sortent d’Isaac, c’est-à-dire qu’ils soient réunisen Jésus-Christ par la grâce qui les appelle. Ce saint patriarche, fortifié par la foi de cette promesse, et persuadé qu’elle devait être accomplie par celui que Dieu lui commandait d’égorger, ne douta point que Dieu ne pût lui rendre celui qu’il lui avait donné contre son espérance. Ainsi l’entend et l’explique l’auteur de l’Épître aux Hébreux : « C’est par la foi, dit-il, qu’Abraham fit éclater son obéissance, lorsqu’il fut tenté au sujet d’Isaac ; car il offrit à Dieu son fils unique, malgré toutes les promesses qui lui avaient été faites, et quoique Dieu lui eût dit : C’est d’Isaac que sortira votre véritable postérité. Mais il pensait en lui-même que Dieu pourrait bien le ressusciter après sa mort. » Et l’Apôtre ajoute : « Voilà pourquoi Dieu l’a proposé en figure. » Or, quelle est cette figure, sinon celle de la victime sainte dont parle le même Apôtre, quand il dit : « Dieu n’a pas épargné son propre Fils, mais il l’a livré à la mort pour nous tous ? » Aussi Isaac porta lui-même le bois du sacrifice dont il devait être la victime, comme Notre-Seigneur porta sa croix. Enfin, puisque Dieu a empêché Abraham de mettre la main sur Isaac, qui n’était pas destiné à mourir, que veut dire ce bélier, dont le sang symbolique accomplit le sacrifice, et qui était retenu par les cornes aux épines du buisson ? Que représente-t-il, si ce n’est Jésus-Christ couronné d’épines par les Juifs avant que d’être immolé ?\par
Mais écoutons plutôt la voix de Dieu par la bouche de l’ange : « Abraham, dit l’Écriture, étendit la main pour prendre son glaive et égorger son fils. Mais l’ange du Seigneur lui cria du haut du ciel : Abraham ? À quoi il répondit : Que vous plaît-il ? — Ne mettez point la main Sur votre fils, lui dit l’ange, et ne lui faites point de mal ; car je connais maintenant que vous craignez votre Dieu, puisque vous n’avez pas épargné votre fils bien-aimé pour l’amour de moi. » « Je connais maintenant », dit Dieu, c’est-à-dire j’ai fait connaître ; car Dieu ne l’avait pas ignoré. Lorsque ensuite Abraham eut immolé le bélier au lieu de son fils Isaac, l’Écriture dit : « Il appela ce lieu {\itshape le Seigneur a vu}, et c’est pourquoi nous disons aujourd’hui : Le Seigneur est apparu sur la montagne. » De même que Dieu dit : {\itshape Je connais maintenant}, pour dire : {\itshape J’ai fait maintenant connaître}, ainsi Abrahamdit : {\itshape Le Seigneur a vu, pour dire : Le Seigneur est apparu} ou s’est fait voir. « Et l’ange appela du ciel Abraham pour la seconde fois, et lui dit : J’ai juré par moi-même, dit le Seigneur, et pour prix de ce que vous venez de faire, n’ayant point épargné votre fils bien-aimé pour l’amour de moi, je vous comblerai de bénédictions, et je vous donnerai une postérité aussi nombreuse que les étoiles du ciel et que le sable de la mer. Vos enfants se rendront maîtres des villes de leurs ennemis ; et toutes les nations de la terre seront bénies en votre postérité, parce que vous avez obéi à ma voix. » C’est ainsi que Dieu confirma par serment la promesse de la vocation des Gentils, après qu’Abraham lui eut offert en holocauste ce bélier, qui était la figure de Jésus-Christ. Dieu le lui avait souvent promis, mais il n’en avait jamais fait serment, et qu’est-ce que le serment du vrai Dieu, du Dieu qui est la vérité même, sinon une confirmation de sa promesse et un reproche qu’il adresse aux incrédules ?\par
Après cela, Sarra mourut âgée de cent vingt-sept ans, lorsque Abraham en avait cent trente-sept ; il était en effet plus vieux qu’elle de dix ans, comme il le déclara lui-même, quand Dieu lui promit qu’elle lui donnerait un fils : « J’aurai donc, dit-il, un fils à cent ans, et Sarra accouchera à quatre-vingt-dix ? » Abraham acheta un champ où il ensevelit sa femme. Ce fut alors, ainsi que le rapporte saint Étienne, qu’il fut établi dans cette contrée, parce qu’il commença à y posséder un héritage ; ce qui arriva après la mort de son père, qui eut lieu environ deux ans auparavant.
\subsection[{Chapitre XXXIII}]{Chapitre XXXIII}

\begin{argument}\noindent Isaac épouse Rébecca, petite-fille de Nachor.
\end{argument}

\noindent Ensuite Isaac, âgé de quarante ans, à l’époque où son père en avait cent quarante, trois ans après la mort de sa mère, épousa Rébecca, petite-fille de son oncle Nachor. Or, quand Abraham envoya son serviteur en Mésopotamie, il lui dit : « Mettez votre main sur ma cuisse, et me faites serment par le Seigneur et le Dieu du ciel et de la terre que vous ne choisirez pour femme à mon filsaucune des filles des Chananéens. » Qu’est-ce que cela signifie, sinon que le Seigneur elle Dieu du ciel et de la terre devait se revêtir d’une chair tirée des flancs de ce patriarche ? Sont-ce là de faibles marques de la vérité que nous voyons maintenant accomplie en Jésus-Christ ?
\subsection[{Chapitre XXXIV}]{Chapitre XXXIV}

\begin{argument}\noindent Ce qu’il faut entendre par le mariage d’Abraham avec Céthura, après la mort de Sarra.
\end{argument}

\noindent Que signifie le mariage d’Abraham avec Céthura après la mort de Sarra ? Nous sommes loin de penser qu’un si saint homme l’ait contracté par incontinence, surtout dans un âge si avancé. Avait-il encore besoin d’enfants, lui qui croyait fermement que Dieu lui en donnerait d’Isaac autant qu’il y a d’étoiles au ciel et de sable sur le rivage de la mer ? Mais si Agar et Ismaël, selon la doctrine de l’Apôtre, sont la figure des hommes charnels de l’Ancien Testament, pourquoi Céthura et ses enfants ne seraient-ils pas de même la figure des hommes charnels qui pensent appartenir au Nouveau ? Toutes deux sont appelées femmes et concubines d’Abraham, au lieu que Sarra n’est jamais appelée que sa femme. Quand Agar fut donnée à Abraham, l’Écriture dit : « Sarra, femme d’Abraham, prit sa servante Agar dix ans après qu’Abraham fut entré dans la terre de Chanaan, et la donna pour femme à son mari. » Quant à Céthura, qu’il épousa après la mort de Sarra, voici comment l’Écriture en parle : « Abraham épousa une autre femme nommée Céthura. » Vous voyez que l’Écriture les appelle toutes deux {\itshape femmes} ; mais ensuite elle les nomme toutes deux {\itshape concubines} : « Abraham, dit-elle, donna tout son bien à son fils Isaac ; et quant aux enfants de ses concubines, il leur fit quelques présents, et les éloigna de son vivant de son fils Isaac, en les envoyant vers les contrées d’Orient. » Les enfants des concubines, c’est-à-dire les Juifs et les hérétiques, reçoivent donc quelques présents, mais ne partagent point le royaume promis, parce qu’il n’y a point d’autre héritier qu’Isaac, et que ce ne sontpas les enfants de la chair qui sont fils de Dieu, mais les enfants de la promesse, Dieu dont se compose cette postérité de qui il a été dit : « Votre postérité sortira d’Isaac. » Je ne vois pas pourquoi l’Écriture appellerait Céthura concubine, s’il n’y avait quelque mystère là-dessous. Quoi qu’il en soit, on ne peut pas justement reprocher ce mariage à ce patriarche. Que savons-nous si Dieu ne l’a point permis ainsi afin de confondre, par l’exemple d’un si saint homme, l’erreur de certain hérétiques qui condamnent les seconde noces comme mauvaises ? Abraham mourut à l’âge de cent soixante et quinze ans ; son fils en avait soixante et quinze, étant venu au monde la centième année de la vie de son père.
\subsection[{Chapitre XXXV}]{Chapitre XXXV}

\begin{argument}\noindent Des deux jumeaux qui se battaient dans le ventre de Rébecca.
\end{argument}

\noindent Voyons maintenant le progrès de la Cité de Dieu dans les descendants d’Abraham Comme Isaac n’avait point encore d’enfants à l’âge de soixante ans, parce que sa femme était stérile, il en demanda à Dieu, qui l’exauçai mais dans le temps que sa femme était enceinte, les deux enfants qu’elle portait se battaient dans son sein. Les grandes douleurs qu’elle en ressentait lui firent consulter Dieu qui lui répondit : « Deux nations sont dans votre sein, et deux peuples sortiront de vos entrailles ; l’un surmontera l’autre, et l’aîné sera soumis au cadet. » L’apôtre saint Paul tire de là un grand argument en faveur de la grâce, en ce que, avant que ni l’un ni l’autre ne fussent nés et n’eussent fait ni bien ni mal, le plus jeune fut choisi sans aucun mérite antérieur, et l’aîné réprouvé. Il est certain que, par rapport au péché originel, ils étaient également coupables, et que ni l’un ni l’autre n’avaient commis aucun péché qui leur fût propre ; mais le dessein que je me suis proposé dans cet ouvrage ne me permet pas de m’étendre davantage sur ce point, outre que je l’ai fait amplement ailleurs. À l’égard de ces paroles : « L’aîné sera soumisau cadet », presque tous nos interprètes l’expliquent du peuple juif, qui doit être assujetti au peuple chrétien ; et dans le fait, bien qu’il semble que cela soit accompli dans les Iduméens issus de l’aîné (il avait deux noms, Ésaü et Édom), parce qu’ils ont été assujettis aux Israélites sortis du cadet néanmoins il est plus croyable que cette prophétie : « Un peuple surmontera l’autre, et l’aîné servira le cadet », regardait quelque chose de plus grand ; et quoi donc, sinon ce que nous voyons clairement s’accomplir dans les Juifs et dans les Chrétiens ?
\subsection[{Chapitre XXXVI}]{Chapitre XXXVI}

\begin{argument}\noindent Dieu bénit Isaac, en considération de son père Abraham.
\end{argument}

\noindent Isaac reçut aussi la même promesse que Dieu avait si souvent faite à son père, et l’Écriture en parle ainsi : « Il y eut une grande famine sur la terre, outre celle qui arriva du temps d’Abraham ; en sorte qu’Isaac se retira à Gérara, vers Abimélech, roi des Philistins. Là, le Seigneur lui apparut et lui dit : Ne descendez point en Égypte, mais demeurez dans la terre que je vous dirai ; demeurez-y comme étranger, et je serai avec vous et vous bénirai ; car je vous donnerai, ainsi qu’à votre postérité, toute cette contrée, et j’accomplirai le serment que j’ai fait à votre père Abraham. Je multiplierai votre postérité comme les étoiles du ciel, et lui donnerai cette terre-ci, et en elle seront bénies toutes les nations de la terre, parce qu’Abraham, votre père, a écouté ma voix et observé mes commandements. » Ce patriarche n’eut point d’antre femme que Rébecca, ni de concubine ; mais il se contenta pour enfants de ses deux jumeaux. Il appréhenda aussi pour la beauté de sa femme, parce qu’il habitait parmi des étrangers, et, suivant l’exemple de son père, il l’appela sa sœur, car elle était sa proche parente du côté de son père et de sa mère. Ces étrangers, ayant su qu’elle était sa femme, ne lui causèrent toutefois aucun déplaisir. Faut-il maintenant le préférer à son père pour n’avoir eu qu’une seule femme ? non, car la foi et l’obéissance d’Abraham étaient, tellement incomparables, que ce fut en sa considération que Dieu promit, au fils tout le bien qu’il lui devait faire.\par
« Toutes les nations de la terre, dit-il, seront bénies en votre postérité, parce que votre père Abraham a écouté ma voix et observé mes commandements » ; et dans une autre vision : « Je suis le Dieu de votre père Abraham, ne craignez point, car je suis avec vous et vous ai béni, et je multiplierai votre postérité à cause d’Abraham, votre père » ; paroles qui montrent bien qu’Abraham a été chaste dans les actions mêmes que certaines personnes, avides de chercher des exemples dans l’Écriture pour justifier leurs désordres, veulent qu’il ait faites par volupté. Cela nous apprend aussi à ne pas comparer les hommes ensemble par quelques actions particulières, mais par toute la suite de leur vie. Il peut fort bien arriver qu’un homme l’emporte sur un autre en quelque point, et qu’il lui soit beaucoup intérieur peur tout le reste. Ainsi, quoique la continence soit préférable au mariage, toutefois un chrétien marié vaut mieux qu’un païen continent, et même celui-ci est d’autant plus digne de blâme qu’il demeure infidèle en même temps qu’il est continent. Supposons deux hommes de bien : sans doute celui qui est plus fidèle et plus obéissant à Dieu vaut mieux, quoique marié, que celui qui est moins fidèle et moins soumis, encore qu’il garde le célibat ; mais toutes choses égales d’ailleurs, il est indubitable qu’on doit préférer l’homme continent à celui qui est marié.
\subsection[{Chapitre XXXVII}]{Chapitre XXXVII}

\begin{argument}\noindent Ce que figuraient par avance Ésaü et Jacob.
\end{argument}

\noindent Or, les deux fils d’Isaac, Ésaü et Jacob, croissaient également en âge, et l’aîné vaincu par son intempérance, céda volontairement au plus jeune son droit d’aînesse pour un plat de lentilles. Nous apprenons de là que ce n’est pas la qualité des viandes, mais la gourmandise qui est blâmable. Isaac devient vieux et perd la vue par suite de son grand âge. Il veut bénir son aîné, et, sans le savoir, il bénit son cadet à la, place de l’autre, qui était velu, et auquel le cadet s’était substitué en ayant soin de se couvrir les mains et le cou d’une peau de chèvre, symbole des péchés d’autrui. Afin qu’on ne s’imaginât pas que cet artifice de Jacob fût répréhensible et ne contînt aucun mystère, l’Écriture a eu soin auparavant de nous avertir « qu’Ésaü étaitun homme farouche et grand chasseur, et que Jacob était un homme simple et qui demeurait au logis ». Quelques interprètes, au lieu de {\itshape simple}, traduisent {\itshape sans ruse}. Mais qu’on entende {\itshape sans ruse} ou {\itshape simple}, ou encore {\itshape sans artifice}, en grec {\itshape aplastos} quelle peut être, en recevant cette bénédiction, la ruse de cet homme sans ruse, l’artifice de cet homme simple, la feinte de cet homme incapable de mentir, sinon un très profond mystère de vérité ? Cela ne paraît-il point dans la bénédiction même ? « L’odeur qui sort de mon fils, dit Isaac, est semblable à l’odeur d’un champ émaillé de fleurs que le Seigneur a béni. Que Dieu fasse tomber la rosée du ciel sur vos terres et les rende fécondes en blé et en vin ; que les nations vous obéissent, et que les princes vous adorent. Soyez le maître de votre frère, et que les enfants de votre père se prosternent devant vous. Celui qui vous bénira sera béni, et celui qui vous maudira sera maudit. » La bénédiction de Jacob, c’est la prédication du nom de Jésus-Christ par toutes les nations. Elle se fait, elle s’accomplit en ce moment même. Isaac est la figure de la loi et des prophètes. Cette loi, ces prophéties, par la bouche des Juifs, bénissent Jésus-Christ sans le connaître, n’étant pas connues elles-mêmes par les Juifs. Le monde, comme un champ, est parfumé du nom de ce Sauveur. La parole de Dieu est la pluie et la rosée du ciel qui rendent ce champ fécond. Sa fécondité est la vocation des Gentils. Le blé et le vin dont il abonde, c’est la multitude des fidèles que le blé et le vin unissent dans le sacrement de son corps et de son sang. Les nations lui obéissent, et les princes l’adorent. Il est le maître de son frère, parce que son peuple commande aux Juifs. Les enfants de son père l’adorent, c’est-à-dire les enfants d’Abraham selon la foi, parce qu’il est lui-même fils d’Abraham selon la chair. Celui qui le maudira sera maudit, et celui qui le bénira sera béni. Ce Christ, qui est notre sauveur, est béni, je le répète, par la bouche des Juifs, dépositaires de la loi et des prophètes, bien qu’ils ne les comprennent pas et qu’ils attendent un autre Sauveur. Lorsque l’aîné demande à son père la bénédiction qu’il lui avait promise, Isaac s’étonne ; et, après avoir vu qu’il avait béni l’un pour l’autre, il admire cet événement, et toutefois ne se plaint pasd’avoir été trompé : au contraire, éclairé sur ce grand mystère par une lumière intérieure, au lieu de se fâcher contre Jacob, il confirme la bénédiction qu’il lui a donnée. « Quel est, dit-il, celui qui m’a apporté de la venaison dont j’ai mangé avant que vous vinssiez ? Je l’ai béni et il demeurera béni. » Qui n’attendrait ici la malédiction d’un homme en colère, si tout cela ne se passait plutôt par une inspiration d’en haut que selon la conduite ordinaire des hommes ? Ô merveilles réellement arrivées, mais prophétiquement ; arrivées sur la terre, mais inspirées par le ciel ; arrivées par l’entremise des hommes, mais conduites par la providence de Dieu ! À examiner toutes ces choses en détail, elles sont si fécondes en mystères, qu’il faudrait des volumes entiers pour les expliquer ; mais les bornes que je me suis prescrites dans cet ouvrage m’obligent à passer à d’autres considérations.
\subsection[{Chapitre XXXVIII}]{Chapitre XXXVIII}

\begin{argument}\noindent Du voyage de Jacob en Mésopotamie pour s’y marier, de la vision qu’il eut en chemin, et des quatre femmes qu’il épousa, bien qu’il n’en demandât qu’une.
\end{argument}

\noindent Jacob est envoyé par ses parents en Mésopotamie pour s’y marier. Voici ce que son père lui dit à son départ : « Ne vous mariez pas parmi les Chananéens ; mais allez en Mésopotamie, chez Bathuel, père de votre mère, et épousez là quelqu’une des filles de Laban, frère de votre mère. Que mon Dieu vous bénisse, et vous rende puissant, afin que vous soyez père de, plusieurs peuples. Qu’il vous donne, et à votre postérité, la bénédiction de votre père Abraham, afin que vous possédiez la terre où vous êtes maintenant étranger et que Dieu a donnée à Abraham. » Ici paraît clairement la division des deux branches de la postérité d’Isaac, celle de Jacob et celle d’Ésaü. Lorsque Dieu dit à Abraham : « Votre postérité sortira d’Isaac », il entendait parler nécessairement de celle qui devait composer la Cité de Dieu, et cette postérité d’Abraham fut dès cet instant séparée de celle qui sortit de lui par les enfants d’Agar et de Céthura ; mais il était encore douteux si cette bénédiction d’Isaac était pour ses deux enfants ou seulement pour l’un d’eux. Or, le doute disparaît maintenant dans cettebénédiction prophétique qu’Isaac donne à Jacob, lorsqu’il lui dit : « Vous serez le père de plusieurs peuples ; que Dieu vous donne la bénédiction de votre père Abraham. »\par
Pendant que Jacob allait en Mésopotamie, il reçut en songe l’oracle du ciel que l’Écriture rapporte en ces termes : « Jacob, laissant le puits du serment, prit son chemin vers Charra, et, étant arrivé en un lieu où la nuit le surprit, il ramassa quelques pierres qu’il trouva là, et, après les avoir mises sous sa tête, il s’endormit. Comme il dormait, il lui sembla voir une échelle dont l’un des bouts posait sur terre et l’autre touchait au ciel, et les anges de Dieu montaient et descendaient par cette échelle. Dieu était appuyé dessus, et il lui dit : Je suis le Dieu d’Abraham, votre père, et le Dieu d’Isaac ; ne craignez point. Je vous donnerai à vous et à votre postérité la terre où vous dormez, et le nombre de vos enfants égalera la poussière de la terre. Ils s’étendront depuis l’orient jusqu’à l’occident depuis le midi jusqu’au septentrion, et toutes les nations de la terre seront bénies en vous et en votre postérité. Je suis avec vous et vous garderai partout où vous irez, et je vous ramènerai en ce pays-ci, parce que je ne vous abandonnerai point que je n’aie accompli tout ce que je vous ai dit. Alors Jacob se réveilla, et dit : Le Seigneur est ici et je ne le savais pas. Et étant saisi de crainte : Que ce lieu, dit-il, est terrible ! ce ne peut être que la maison de Dieu et la porte du ciel. Là-dessus il se leva, et prenant la pierre qu’il avait mise sous sa tête, il la dressa pour servir de monument, « et l’oignit d’huile par en haut, et nomma ce lieu la maison de Dieu. » Ceci contient une prophétie ; et il ne faut pas s’imaginer que Jacob versa de l’huile sur cette pierre à la façon des idolâtres, comme s’il en eût fait un Dieu, car il ne l’adora point, ni ne lui offrit point de sacrifice ; mais comme le nom de Christ vient d’un mot grec qui signifie onction, ceci sans doute figure quelque grand mystère. Notre Sauveur lui-même semble expliquer le sens symbolique de cette échelle dans l’Évangile, lorsqu’après avoir dit de Nathanaël : « Voilà un véritable Israéliteen qui il n’y a point de ruse », pensant à la vision qu’avait eue Israël, qui est le même que Jacob, il ajoute : « En vérité, en vérité, je vous dis que vous verrez le ciel ouvert, et les anges de Dieu monter et descendre sur le fils de l’homme. »\par
Jacob continua donc son chemin en Mésopotamie, pour y choisir une femme. Or, l’Écriture nous apprend pourquoi il en épousa quatre dont il eut douze fils et une fille, lui qui n’en avait épousé aucune par un désir illégitime. Il était venu pour prendre une seule épouse ; mais comme on lui en supposa une autre à la place de celle qui lui était promise, il ne la voulut pas quitter, de peur qu’elle ne demeurât déshonorée ; et comme en ce temps-là il était permis d’avoir plusieurs femmes pour accroître sa postérité, il prit encore la première à qui il avait déjà donné sa foi. Cependant, celle-ci étant stérile, elle lui donna sa servante pour en avoir des enfants ; ce que son aînée fit aussi, quoique elle-même en eût déjà. Jacob n’en demanda qu’une, et il n’en connut plusieurs que pour en avoir des enfants, et à la prière de ses femmes, qui usaient en cela du pouvoir que les lois du mariage leur donnaient sur lui.
\subsection[{Chapitre XXXIX}]{Chapitre XXXIX}

\begin{argument}\noindent Pourquoi Jacob fut appelé Israël.
\end{argument}

\noindent Or, Jacob eut douze fils et une fille de quatre femmes. Ensuite, il vint en Égypte, à cause de son fils Joseph qui y avait été mené et y était devenu puissant, après avoir été vendu par la jalousie de ses frères. Jacob, comme je viens de le dire, s’appelait aussi Israël, d’où le peuple descendu de lui a pris son nom, et ce nom lui fut donné par l’ange qui lutta contre lui à son retour de Mésopotamie et qui était la figure de Jésus-Christ. L’avantage qu’il voulut bien que Jacob remportât signifie le pouvoir que Jésus-Christ donna sur lui aux Juifs au temps de sa passion. Toutefois, il demanda la bénédiction de celui qu’il avait surmonté, et cette bénédiction fut l’imposition de ce nom même. Israël signifie {\itshape voyant Dieu}, ce qui marque la récompense de tous les saints à la fin du monde. L’ange le toucha à l’endroit le plus large de la caisse et le rendit boiteux. Ainsi le même Jacob fut béni et boiteux : bénien ceux du peuple juif qui ont cru en Jésus-Christ, et boiteux en ceux qui n’y ont pas cru, car l’endroit le plus large de la cuisse marque une postérité nombreuse. En effet, il y en a beaucoup plus parmi ses descendants en qui cette prophétie s’est accomplie : « Ils se sont égarés du droit chemin, et ont boité. »
\subsection[{Chapitre XL}]{Chapitre XL}

\begin{argument}\noindent Comment on doit entendre que Jacob entra, lui soixante-quinzième, en Égypte
\end{argument}

\noindent L’Écriture dit que soixante-quinze personnes entrèrent en Égypte avec Jacob, en l’y comprenant avec ses enfants ; et dans ce nombre elle ne fait mention que de deux femmes, l’une fille, et l’autre petite-fille de ce patriarche. Mais à considérer la chose exactement, elle ne veut point dire que la maison de Jacob fût si grande le jour ni l’année qu’il y entra, puisqu’elle compte parmi ceux qui y entrèrent des arrière-petits-fils de Joseph, qui ne pouvaient pas être encore au monde. Jacob avait alors cent trente ans, et son fils Joseph trente-neuf. Or, il est certain que Joseph n’avait que trente ans, ou un peu plus, quand il se maria. Comment donc aurait-il pu en l’espace de neuf ans avoir des arrière-petits-fils ? Quand Jacob entra en Égypte, Éphraïm et Manassé, enfants de Joseph, n’avaient pas encore neuf ans. Or, dans le dénombrement que l’Écriture fait de ceux qui y entrèrent avec lui, elle parle de Machir, fils de Manassé et petit-fils de Joseph, et de Galaad, fils de Machir, c’est-à-dire arrière-petit-fils de Joseph. Elle parle aussi de Utalaam, fils d’Éphraïm, et de Édem, fils de Utalaam, c’est-à-dire d’un autre petit-fils et arrière-petit-fils de ce patriarche. L’Écriture donc, par l’entrée de Jacob en Égypte, n’entend pas parler du jour ni de l’année qu’il y entra, mais de tout le temps que vécut Joseph qui fut cause de cette entrée. Voici comment elle parle de Joseph : « Joseph demeura en Égypte avec ses frères et toute la maison de son père, et il vécut cent dix ans, et il vit les enfants d’Éphraïm jusqu’à la troisième génération », c’est-à-dire Édem, son arrière-petit-fils du côté d’Éphraïm. C’est là, en effet, ce que l’Écriture appelle troisième génération. Puis elle ajoute : « Et les enfants de Machir, fils de Manassé, naquirent sur les genoux de Joseph », c’est-à-dire Galaad, son arrière-petit-fils du côté de Manassé, dont l’Écriture, suivant son usage, qui est aussi celui de la langue latine, parle comme s’il y en avait plusieurs, ainsi que de la fille unique de Jacob, qu’elle appelle {\itshape les filles de Jacob}. Il ne faut donc pas s’imaginer que ces enfants de Joseph fussent nés quand Jacob entra en Égypte, puisque l’Écriture, pour relever la félicité de Joseph, dit qu’il les vit naître avant que de mourir ; mais ce qui trompe ceux qui n’y regardent pas de si près, c’est que l’Écriture dit : « Voici les noms des enfants d’Israël qui entrèrent en Égypte avec Jacob, leur père. » Elle ne parle donc de la sorte que parce qu’elle compte aussi toute la famille de Joseph, et qu’elle prend cette entrée pour toute la vie de ce patriarche, parce que c’est lui qui en fut cause.
\subsection[{Chapitre XLI}]{Chapitre XLI}

\begin{argument}\noindent Bénédiction de Juda.
\end{argument}

\noindent Si donc, à cause du peuple chrétien, en qui la Cité de Dieu est étrangère ici-bas, nouscherchons Jésus-Christ selon la chair dans la postérité d’Abraham, laissant les enfants desconcubines, Isaac se présente à nous ; dans celle d’Isaac, laissant Ésaü ou Édom, se présente Jacob ou Israël ; dans celle d’Israël, les autres mis à part, se présente Juda, parce que Jésus-Christ est né de la tribu de Juda. Voyons pour cette raison la bénédiction prophétique que Jacob lui donna lorsque, près de mourir, il bénit tous ses enfants : « Juda, dit-il, vos frères vous loueront ; vous emmènerez vos ennemis captifs ; les enfants de votre père vous adoreront. Juda est un jeune lion ; vous vous êtes élevé, mon fils, comme un arbre qui pousse avec vigueur ; vous vous êtes couché pour dormir comme un lion et comme un lionceau : qui le réveillera ? Le sceptre ne sera point ôté de la maison de Juda, et les princes ne manqueront point jusqu’à ce que tout ce qui lui a été promis soit accompli. Il sera l’attente des nations, et il attachera son poulain et l’ânon de son ânesse au cep de la vigne. Il lavera sa robe dans le vin, et son vêtement dans le sang de la grappe de raisin. Ses yeux sont rouges de vin, et ses dents plus blanches que le lait. » J’ai expliqué tout ceci contre Fauste le manichéen, et j’estime en avoir dit assez pour montrer la vérité de cette prophétie. La mort de Jésus-Christ y est prédite par le {\itshape sommeil} ; et par le {\itshape lion}, le pouvoir qu’il avait de mourir ou de ne mourir pas. C’est ce pouvoir qu’il relève lui-même dans l’Évangile, quand il dit : « J’ai pouvoir de quitter mon âme, et j’ai pouvoir de la reprendre. Personne ne me la peut ôter ; mais c’est de moi-même que je la quitte et que je la reprends. » C’est ainsi que le lion a rugi et qu’il a accompli ce qu’il a dit. À cette même puissance encore se rapporte ce qui est dit de sa résurrection : « Qui le réveillera ? » c’est-à-dire que nul homme ne le peut que lui-même, qui a dit aussi de son corps : « Détruisez ce temple, et je le relèverai en trois jours. » Le genre de sa mort, c’est-à-dire son élévation sur la croix, est compris en cette seule parole : « Vous vous êtes élevé. » Et ce que Jacob ajoute ensuite : « Vous vous êtes couché pour dormir », l’Évangéliste l’explique lorsqu’il dit : « Et penchant la tête, il rendit l’esprit » ; si l’on n’aime mieux l’entendre de son tombeau, où il s’est reposé et a dormi, et d’où aucun homme ne l’a ressuscité, comme les prophètes ou lui-même en ont ressuscité quelques-uns, mais d’où il est sorti tout seul comme d’un doux sommeil. Pour sa robe qu’il lave dans le vin, c’est-à-dire qu’il purifie de tout péché dans son sang, qu’est-ce autre chose que l’Église ? Les baptisés savent quel est le sacrement de ce sang, d’où vient que l’Écriture ajoute : « Et son vêtement dans le sang de la grappe. Ses yeux sont rouges de vin. » Qu’est-ce que cela signifie, sinon les personnes spirituelles enivrées de ce divin breuvage dont le Psalmiste dit : « Que votre breuvage qui enivre est excellent ! » — « Ses dents sont plus blanches que le lait » ; c’est ce lait que les petits boivent chez l’Apôtre, c’est-à-dire les paroles qui nourrissent ceux qui ne sont pas encore capables d’une viande solide. C’est donc en lui que résidaient les promesses faites à Juda, avant l’accomplissement desquelles les princes, c’est-à-dire les rois d’Israël, n’ont point manqué dans cette race. Lui seulétait l’attente des nations, et ce que nous en voyons maintenant est plus clair que tout ce que nous en pouvons dire.
\subsection[{Chapitre XLII}]{Chapitre XLII}

\begin{argument}\noindent Bénédiction des deux fils de Joseph par Jacob.
\end{argument}

\noindent Or, comme les deux fils d’Isaac, Ésaü et Jacob, ont été la figuré de deux peuplés, des Juifs et des Chrétiens, quoique selon la chair les Juifs ne soient pas issus d’Ésaü, mais bien les Iduméens, pas plus que les Chrétiens ne le sont de Jacob, mais bien les Juifs, tout le sens de la figure se résume en ceci : « L’aîné sera soumis au cadet » ; il en est arrivé de même dans les deux fils de Joseph. L’aîné était la figure des Juifs, et le cadet celle des Chrétiens. Aussi Jacob, les bénissant, mit sa main droite sur le cadet qui était à sa gauche, et sa gauche sur l’aîné qui était à sa droite ; et comme Joseph, leur père, fâché de cette méprise, voulut le faire changer, et lui montra l’aîné : « Je le sais bien, mon fils, répondit-il, je le sais bien. Celui-ci sera père d’un peuple et deviendra très puissant ; mais son cadet sera plus grand que lui, et de lui sortiront plusieurs nations. » Voilà deux promesses clairement distinctes. « L’un, dit l’Écriture, sera père d’un peuple, et l’autre de plusieurs nations. » N’est-il pas de la dernière évidence que ces deux promesses embrassent le peuple juif et tous les autres peuples de la terre qui devaient également sortir d’Abraham, le premier selon la chair, et le reste selon la foi ?
\subsection[{Chapitre XLIII}]{Chapitre XLIII}

\begin{argument}\noindent Des temps de Moïse, de Jésus Navé, des Juges et des Rois jusqu’à David.
\end{argument}

\noindent Après la mort de Jacob et de Joseph, le peuple juif se multiplia prodigieusement pendant les cent quarante-quatre années qui restèrent jusqu’à la sortie d’Égypte, quoique les Égyptiens, effrayés de leur nombre, leur fissent subir des persécutions si cruelles que, même à la fin, ils tuèrent tous les enfants mâles qui venaient au monde. Alors Moïse, choisi de Dieu pour exécuter de grandeschoses, fut dérobé à la fureur de ces meurtriers et porté dans la maison royale, où il fut nourri et adopté par la fille de Pharaon, nom qui était commun à tous les rois d’Égypte. Là il devint assez puissant pour affranchir ce peuple de la captivité où il gémissait depuis si longtemps, ou, pour mieux dire, Dieu, conformément à la promesse qu’il avait faite à Abraham, se servit du ministère de Moïse pour délivrer les Hébreux. Obligé d’abord de s’enfuir en Madian pour avoir tué un Égyptien qui outrageait un Juif, revenu ensuite par un ordre exprès du ciel, il surmonta les mages de Pharaon par la puissance de l’esprit de Dieu. Après ces prodiges, comme les Égyptiens refusaient encore de laisser sortir le peuple de Dieu, il les frappa de ces dix plaies si fameuses : l’eau changée en sang, les grenouilles, les moucherons, les mouches canines, la mort des bestiaux, les ulcères, la grêle, les sauterelles, les ténèbres et la mort de leurs aînés. Enfin, les Égyptiens, vaincus par tant de misères, furent, pour dernier malheur, engloutis sous les flots, tandis qu’ils poursuivaient les Juifs, après leur avoir permis de s’en aller. La mer, qui s’était ouverte pour donner passage aux Hébreux, submergea leurs ennemis par le retour de ses ondes. Depuis, ce peuple passa quarante ans dans le désert sous la conduite de Moïse, et c’est là que fut fait le tabernacle du témoignage, dans lequel Dieu était adoré par des sacrifices, figures des choses à venir. La loi y fut aussi donnée sur la montagne au milieu des foudres, des tempêtes et de voix éclatantes qui attestaient la présence de la divinité. Ceci arriva aussitôt que le peuple fut sorti d’Égypte et entré dans le désert, cinquante jours après la pâque et l’immolation de l’agneau, qui était si véritablement la figure de Jésus-Christ immolé sur la croix et passant de ce monde à son père (car Pâque en hébreu signifie {\itshape passage}), que lorsque le Nouveau Testament fut établi par le sacrifice de Jésus-Christ, qui est notre Pâque, cinquante jours après, le Saint-Esprit, appelé dans l’Évangile le doigt de Dieu, descendit du ciel afin de nous faire souvenir de l’ancienne figure, parce que la loi, au rapport de l’Écriture, fut aussi écrite sur les tables par le doigt de Dieu.\par
Après la mort de Moïse, Jésus, fils de Navé,prit la conduite du peuple et le fit entrer dans la terre promise qu’il partagea. Ces deux grands et admirables conducteurs achevèrent heureusement de grandes guerres, où Dieu montra que les victoires signalées qu’il fit remporter aux Hébreux sur leurs ennemis étaient plutôt pour châtier les crimes de ceux-ci que pour récompenser le mérite des autres. À ces deux chefs succédèrent les Juges, le peuple étant déjà établi dans la terre promise, afin que la première promesse faite à Abraham touchant un seul peuple et la terre de Chanaan commençât à s’accomplir, en attendant que l’avènement de Jésus-Christ accomplît celle de toutes les nations et de toute la terre. C’est en effet la foi de l’Évangile qui en devait faire l’accomplissement, et non les pratiques légales ; et cette vérité est figurée d’avance, en ce que ce ne fut pas Moïse qui avait reçu pour te peuple la loi sur la montagne, mais Jésus, à qui Dieu même donna ce nom, qui fit entrer les Hébreux dans la terre promise. Sous les Juges, il y eut une vicissitude de prospérités et de malheurs, selon que la miséricorde de Dieu ou les péchés du peuple en décidaient.\par
De là on passa au gouvernement des Rois, dont le premier fut Saül, qui, ayant été réprouvé avec toute sa race et tué dans une bataille, eut pour successeur David. C’est de ce roi que Jésus-Christ est surtout appelé fils par l’Écriture. C’est par lui que commença en quelque sorte la jeunesse du peuple de Dieu, dont l’adolescence avait été depuis Abraham jusqu’à lui. L’évangéliste saint Matthieu n’a pas marqué sans intention mystérieuse, dans la généalogie de Jésus-Christ, quatorze générations depuis Abraham jusqu’à David. En effet, c’est depuis l’adolescence que l’homme commence à être capable d’engendrer ; d’où vient que saint Matthieu commence cette généalogie à Abraham, qui fut père de plusieurs nations, quand son nom fut changé. Avant Abraham donc, c’était en quelque sorte l’âge qui suivit l’enfance du peuple de Dieu, depuis Noé jusqu’à ce patriarche ; et ce fut pour cette raison qu’il commença en ce temps-là à parler la première langue, c’est-à-dire l’hébraïque. La vérité est que c’est au sortir de l’enfance (qui tire son nom de l’impossibilité où sont lesnouveau-nés de parler) que l’homme commence à user de la parole, et de même que ce premier âge est enseveli dans l’oubli, le premier âge du genre humain fut aboli par les eaux du déluge. Ainsi dans le progrès de la Cité de Dieu, comme le livre précédent contient le premier âge du monde, celui-ci contient le second et le troisième. En ce troisième âge fut imposé le joug de la loi, qui est figurée par la génisse, la chèvre et le bélier de trois ans ; on y vit paraître une multitude effroyable de crimes, qui jetèrent les fondements du royaume de la terre, où néanmoins vécurent toujours des hommes spirituels figurés par la tourterelle et par la colombe.
\section[{Livre dix-septième. De David à Jésus-Christ}]{Livre dix-septième. \\
De David à Jésus-Christ}\renewcommand{\leftmark}{Livre dix-septième. \\
De David à Jésus-Christ}

\subsection[{Chapitre premier}]{Chapitre premier}

\begin{argument}\noindent Du temps des Prophètes.
\end{argument}

\noindent Comment se sont accomplies et s’accomplissent encore les promesses de Dieu à Abraham à l’égard de sa double postérité, le peuple juif, selon la chair, et toutes les nations de la terre, selon la foi, c’est ce que le progrès de la Cité de Dieu, selon l’ordre des temps, va nous découvrir. Nous avons fini le livre précédent au règne de David ; voyons maintenant ce qui s’est passé depuis ce règne, dans la mesure où peut nous le permettre le dessein que nous nous sommes proposé en cet ouvrage. Tout le temps écoulé depuis que Samuel commença à prophétiser jusqu’à la captivité de Babylone et au rétablissement du temple, qui arriva soixante-dix ans après, ainsi que Jérémie l’avait prédit, tout ce temps, dis-je, est le temps des Prophètes. Bien que nous puissions avec raison appeler prophètes Noé et quelques autres patriarches qui l’ont précédé ou suivi jusqu’aux Rois, à cause de certaines choses qu’ils ont faites ou dites en esprit de prophétie touchant la Cité de Dieu, d’autant plus qu’il y en a quelques-uns parmi eux à qui l’Écriture sainte donne ce nom, comme Abraham et Moïse, toutefois, à proprement parler, le temps des Prophètes ne commence que depuis Samuel, qui, par le commandement de Dieu, sacra d’abord roi Saül, et ensuite David, après la réprobation de Saül. Mais nous n’en finirions pas de rapporter tout ce que ces Prophètes ont prédit de Jésus-Christ, tandis que la Cité de Dieu se continuait dans le cours des siècles. Si l’on voulait surtout considérer attentivement l’Écriture sainte, dans les choses même qu’elle semble ne rapporter qu’historiquement des Rois, on trouverait qu’elle n’est pas moins attentive, si elle ne l’est plus, à prédire l’avenir qu’à raconter le passé. Or, qui ne voit avec un peu de réflexion queltravail ce serait d’entreprendre cette sorte de recherche, et combien il faudrait de volumes pour s’en acquitter comme il faut ? En second lieu, les choses même qui ont indubitablement le caractère prophétique sont en si grand nombre touchant Jésus-Christ et le royaume des cieux, qui est la Cité de Dieu, que cette explication passerait de beaucoup les bornes de cet ouvrage. Je tâcherai donc, avec l’aide de Dieu, de m’y contenir de telle sorte, que, sans omettre le nécessaire, je ne dise rien de superflu.
\subsection[{Chapitre II}]{Chapitre II}

\begin{argument}\noindent Ce ne fut proprement que sous les Rois, que la promesse de Dieu touchant la terre de Chanaan fut accomplie.
\end{argument}

\noindent Nous avons dit au livre précédent que Dieu promit deux choses à Abraham : l’une, que sa postérité posséderait la terre de Chanaan, ce qui est signifié par ces paroles : « Allez en la terre que je vous montrerai, et je vous ferai Père d’un grand peuple » ; et l’autre, beaucoup plus excellente et qui regarde une postérité, non pas charnelle, mais spirituelle, qui le rend père, non du seul peuple juif, mais de tous les peuples qui marchent sur les traces de sa foi. Celle-ci est exprimée en ces termes : « En vous seront bénies toutes les nations de la terre. » Ces deux promesses lui ont été faites beaucoup d’autres fois, comme nous l’avons montré. La postérité charnelle d’Abraham, c’est-à-dire le peuple juif, était donc déjà établi dans la terre promise, et, maître des villes ennemies, il vivait sous la domination de ses rois. Ainsi, les promesses de Dieu commencèrent dès lors à être accomplies en grande partie, non seulement celles qu’il avait faites aux trois patriarches, Abraham, Isaac et Jacob, mais encore celles qu’il fit à Moïse, par qui le peuplehébreu fut délivré de la captivité d’Égypte et à qui toutes les choses passées furent révélées, lorsqu’il conduisait ce peuple dans le désert. Toutefois, ce ne fut ni sous Jésus fils de Navé, ce fameux capitaine qui fit entrer les Hébreux dans la terre promise, et qui la divisa, selon l’ordre de Dieu, entre les douze tribus, ni sous les Juges, que s’accomplit la promesse que Dieu avait faite de donner aux Israélites toute la terre de Chanaan, depuis le fleuve d’Égypte jusqu’au grand fleuve d’Euphrate. Elle ne le fut que sous David et sous son fils Salomon, dont le royaume et toute cette étendue. Ils subjuguèrent, en effet, tous ces peuples et en firent leurs tributaires. Ce fut donc sous ces princes que la postérité d’Abraham se trouva établie en la terre de Chanaan, de sorte qu’il ne manquait plus rien à l’entier accomplissement des promesses de Dieu à cet égard, sauf cet unique point que les Juifs la posséderaient jusqu’à la fin des siècles ; mais il fallait pour cela qu’ils demeurassent fidèles à leur Dieu. Or, comme Dieu savait qu’ils ne le seraient pas, il se servit des châtiments temporels dont il les affligea pour exercer le petit nombre des fidèles qui étaient parmi eux, afin qu’ils instruisissent à l’avenir les fidèles des autres nations en qui il voulait accomplir l’autre promesse par l’incarnation de Jésus-Christ et la publication du Nouveau Testament.
\subsection[{Chapitre III}]{Chapitre III}

\begin{argument}\noindent Les trois sortes de prophéties de l’Ancien Testament se rapportent tantôt à la Jérusalem terrestre, tantôt à la Jérusalem céleste, et tantôt à l’une et à l’autre.
\end{argument}

\noindent Ainsi toutes les prophéties, tant celles qui ont précédé l’époque des Rois que celles qui l’ont suivie, regardent en partie la postérité charnelle d’Abraham, et en partie cette autre postérité en qui sont bénis tous les peuples cohéritiers de Jésus-Christ par le Nouveau Testament, et appelés à posséder la vie éternelle et le royaume des cieux. Elles se rapportent moitié à la servante qui engendre des esclaves, c’est-à-dire à la Jérusalem terrestre, qui est esclave avec ses enfants, et moitié à la cité libre, qui est la vraie Jérusalem, étrangèreici-bas en quelques-uns de ses enfants et éternelle dans les cieux ; mais il y en a qui se rapportent à l’une et à l’autre, proprement à la servante et figurativement à la femme libre.\par
Il y a donc trois sortes de prophéties, les unes relatives à la Jérusalem terrestre, les autres à la céleste, et les autres à toutes les deux. Donnons-en des exemples. Le prophète Nathan fut envoyé à David pour lui reprocher son crime et lui en annoncer le châtiment. Qui doute que ces avertissements du ciel et autres semblables, qui concernaient l’intérêt de tous ou celui de quelques particuliers, n’appartinssent à la cité de la terre ? Mais lorsqu’on lit dans Jérémie : « Voici venir le temps, dit le Seigneur, que je ferai une nouvelle alliance qui ne sera pas semblable à celle que je fis avec leurs pères, lorsque je les pris par la main pour les tirer d’Égypte ; car ils ne l’ont pas gardée, et c’est pourquoi je les ai abandonnés, dit le Seigneur. Mais voici l’alliance que je veux faire avec la maison d’Israël : « Après ce temps, dit le Seigneur, je déposerai mes lois dans leur esprit ; je les écrirai dans leur cœur, et mes yeux les regarderont et je serai leur Dieu, et ils seront mon peuple. » Il est certain que c’est là une prophétie de cette Jérusalem céleste où Dieu même est la récompense des justes et où l’unique et souverain bien est de le posséder et d’être à lui. Mais lorsque l’Écriture appelle Jérusalem la Cité de Dieu et annonce que la maison de Dieu s’élèvera dans son enceinte, cela se rapporte à l’une et l’autre cité : à la Jérusalem terrestre, parce que cela a été accompli, selon la vérité de l’histoire, dans le fameux temple de Salomon, et à la céleste, parce que ce temple en était la figure. Ce genre de prophétie mixte, dans les livres historiques de l’Ancien Testament, est fort considérable ; il a exercé et exerce encore beaucoup de commentateurs de l’Écriture qui cherchent la figure de ce qui doit s’accomplir en la postérité spirituelle d’Abraham dans ce qui a été prédit et accompli pour sa postérité charnelle. Quelques-uns portent ce goût si loin qu’ils prétendent qu’il n’y a rien en ces livres de ce qui est arrivé après avoir été prédit, ou même sans l’avoir été, qui ne doive se rapporter allégoriquement à la Cité de Dieu et à ses enfants qui sont étrangers en cette vie. Si cela est, il n’y aura pins que deux sortes de prophéties dans tous les livres de l’Ancien Testament, les unes relatives à la Jérusalem céleste, et les autres aux deux Jérusalem, sans qu’aucune se rapporte seulement à la terrestre. Pour moi, comme il me semble que ceux-là se trompent fort qui excluent toute allégorie des livres historiques de l’Écriture, j’estime aussi que c’est beaucoup entreprendre que de vouloir en trouver partout. C’est pourquoi j’ai dit qu’il vaut mieux distinguer trois sortes de prophéties, sans blâmer toutefois ceux qui, conservant la vérité de l’histoire, cherchent à trouver partout quelque sens allégorique. Quant aux choses qui ne peuvent se rattacher ni à l’action des hommes ni à celle de Dieu, il est évident que l’Écriture n’en parle pas sans dessein, et il faut conséquemment tâcher de les rappeler à un sens spirituel.
\subsection[{Chapitre IV}]{Chapitre IV}

\begin{argument}\noindent Figure du changement de l’empire et du sacerdoce d’Israël, et prophéties d’Anne, mère de Samuel, laquelle figurait l’Église.
\end{argument}

\noindent La suite des temps amène la Cité de Dieu jusqu’à l’époque des Rois, alors que, Saül ayant été réprouvé, David monta sur le trône, et que ses descendants régnèrent longtemps après lui dans la Jérusalem terrestre. Ce changement, qui arriva en la personne de Saül et de David, figurait le remplacement de l’Ancien Testament par le Nouveau, où le sacerdoce et la royauté ont été changés par le prêtre et le roi nouveau et immortel, qui est Jésus-Christ. Le grand-prêtre Héli réprouvé et Samuel mis en sa place et exerçant ensemble les fonctions de prêtre et de juge, et d’autre part, David sacré roi au lieu de Saül, figuraient cette révolution spirituelle. La mère de Samuel, Anne, stérile d’abord, et qui depuis eut tant de joie de sa fécondité, semble ne prophétiser autre chose, quand, ravie de son bonheur, elle rend grâces à Dieu et lui consacre son fils avec la même piété qu’elle le lui avait voué. Voici comme elle s’exprime : « Mon cœur a été affermi dans sa confiance au Seigneur, et mon Dieu a relevé ma force et ma gloire. Ma bouche a été ouverte contre mes ennemis, et je me suis réjouie de votre salut. Car il n’est point de saint comme le Seigneur, il n’est point de juste comme notre Dieu, il n’est de saint que vous. Ne vous glorifiez point, et ne parlez point autrement ; qu’aucune parole fière et superbe ne sorte de votre bouche, puisque c’est Dieu qui est le maître des sciences, et qui forme et conduit ses desseins. Il a détendu l’arc des puissants, et les faibles ont été revêtus de force. Ceux qui ont du pain en abondance sont devenus languissants, et ceux qui étaient affamés se sont élevés au-dessus de la terre, parce que celle qui était stérile est devenue mère de sept enfants, et celle qui avait beaucoup d’enfants est demeurée sans vigueur. C’est Dieu qui donne la mort et qui redonne la vie ; c’est lui qui mène aux enfers et qui en ramène. Le Seigneur rend pauvre ou riche, abaisse ou élève ceux qu’il lui plaît. Il élève de terre le pauvre, et tire le misérable du fumier, afin de le faire asseoir avec les princes de son peuple et de lui donner pour héritage un trône de gloire. Il donne à qui fait un vœu de quoi le faire, et il a béni les années du juste, parce que l’homme n’est pas fort par sa propre force. Le Seigneur désarmera son adversaire, le Seigneur qui est saint. Que le sage ne se glorifie point de sa sagesse, ni le puissant de sa puissance, ni le riche de ses richesses ; mais que celui qui eut se glorifier se glorifie de connaître Dieu et de rendre justice au milieu de la terre. Le Seigneur est monté aux cieux et a tonné ; il jugera les extrémités de la terre, parce qu’il est juste. C’est lui qui donne la vertu à nos rois, et il exaltera la gloire et la puissance de son Christ. »\par
Croira-t-on que c’est là le discours d’une simple femme qui se réjouit de la naissance de son fils, et sera-t-on assez aveugle pour ne pas voir qu’il est beaucoup au-dessus de sa portée ? En un mot, quiconque fait attention à ce qui est déjà accompli de ces paroles, ne reconnaît-il pas clairement que le Saint-Esprit, par le ministère, de cette femme (dont le nom même, en hébreu, signifie {\itshape grâce}), a prédit la religion chrétienne, la Cité de Dieu, dont Jésus-Christ est le roi et le fondateur, et enfin la grâce même de Dieu, dont les superbes s’éloignent pour tomber par terre et dont les humbles sont remplis pour se relever ? Il ne resterait qu’à prétendre que cette femme n’a rien prédit, et que ce sont de simples actions de grâces qu’elle rend à Dieu pour lui avoirdonné un fils ; mais que signifie en ce cas ce qu’elle dit : « Il a détendu l’arc des puissants, et les faibles ont été revêtus de force. Ceux qui ont du pain en abondance sont devenus languissants, et ceux qui étaient affamés se sont élevés au-dessus de la terre, parce que celle qui était stérile est devenue mère de sept enfants, et celle qui avait beaucoup d’enfants n’a plus de vigueur » ? Est-ce qu’Anne a eu sept enfants ? Elle n’en avait qu’un quand elle disait cela, et n’en eut en tout que cinq, trois garçons et deux filles. Bien plus, comme iln’y avait point encore de rois parmi les Juifs, qui la porte à dire : « C’est lui qui donne la force à nos rois, et qui relèvera la gloire et la puissance de son Christ », si ce n’est pas là une prophétie ?\par
Que l’Église de Jésus-Christ, la cité du grand roi, pleine de grâces, féconde en enfants, répète donc ce qu’elle reconnaît avoir prophétisé d’elle il y a si longtemps par la bouche de cette pieuse mère ! qu’elle répète : « Mon cœur a été affermi dans sa confiance au Seigneur, et mon Dieu a relevé ma force et ma gloire. » Son cœur a été vraiment affermi sa puissance a été vraiment augmentée, parce qu’elle ne l’a pas mise en elle-même, mais dans le Seigneur son Dieu. « Ma bouche a été ouverte contre mes ennemis » ; et en effet, la parole de Dieu n’est point captive au milieu des chaînes et de la captivité. « Je me suis réjouie de votre salut. » Ce salut, c’est Jésus-Christ lui-même, que le vieillard Siméon, selon le témoignage de l’Évangile, embrasse tout petit, mais dont il reconnaît la grandeur, quand il s’écrie : « Seigneur, vous laisserez aller votre serviteur en paix, parce que mes yeux ont vu votre salut. » Que l’Église répète donc : « Je me suis réjouie de votre salut ; car il n’est point de saint comme le Seigneur, il n’est point de juste comme notre Dieu » ; Dieu, en effet, n’est pas seulement saint et juste, mais la source de la sainteté et de la justice. « Il n’est de saint que vous ; car personne n’est saint que par lui. Ne vous glorifiez point, et ne parlez point hautement ; qu’aucune parole fière et superbe ne sorte de votre bouche, puisque c’est Dieu qui est le maître des sciences, et personne ne sait ce qu’il sait. » Entendez que celui qui n’étant rien se croit quelque chose, se trompe soi-même ; car cecis’adresse aux ennemis de la Cité de Dieu, qui appartiennent à Babylone, à ceux qui présument trop de leurs forces et se glorifient en eux-mêmes au lieu de se glorifier en Dieu. De ce nombre sont aussi les Israélites charnels, citoyens de la Jérusalem terrestre, qui, comme dit l’Apôtre, « ne connaissant point la justice de Dieu », c’est-à-dire la justice que Dieu donne aux hommes, lui qui seul est juste et rend juste, « et voulant établir leur propre justice », c’est-à-dire prétendant qu’ils l’ont acquise par leurs propres forces sans la tenir de lui, « ne sont point soumis à la justice de Dieu », parce qu’ils sont superbes et qu’ils croient pouvoir plaire à Dieu par leur propre mérite, et non par la grâce de celui qui est le Dieu des sciences, et par conséquent l’arbitre des consciences, où il voit que toutes les pensées des hommes ne sont que vanité, à moins que lui-même ne les leur inspire, « Il forme et conduit ses desseins. » Quels desseins, sinon ceux qui vont à terrasser les superbes et à relever les humbles ? Ce sont ces desseins qu’il exécute lorsqu’il dit : « L’arc des puissants a été détendu, et les faibles ont été revêtus de force. » L’arc a été détendu, c’est-à-dire que Dieu a confondu ceux qui se croyaient assez forts par eux-mêmes pour accomplir les commandements de Dieu, sans avoir besoin de son secours. Et ceux-là « sont revêtus de force » qui crient à Dieu dans le fond de leur cœur : « Ayez pitié de moi, Seigneur, parce que je suis faible. » — « Ceux qui ont du pain en abondance sont devenus languissants, et ceux qui étaient affamés se sont élevés au-dessus de la terre. » Qui sont ceux qui ont du pain en abondance, sinon ceux même qui se croient puissants, c’est-à-dire les Juifs, à qui les oracles de la parole de Dieu ont été confiés ? Mais, parmi ce peuple, les enfants de la servante sont devenus languissants, parce que dans ces pains, c’est-à-dire dans la parole de Dieu, que la seule nation juive avait reçue alors, ils ne goûtent que ce qu’il y a de terrestre ; au lieu que les Gentils, à qui ces pains n’avaient pas été donnés, n’en ont pas eu plutôt mangé que la faim dont ils étaient pressés les a fait élever au-dessus de la terre pour y savourer tout ce qu’ils renferment de céleste et de spirituel. Et comme si l’on demandait la cause d’un événement si étrange : « C’est, dit-elle, que celle qui était stérile est devenue mère de sept enfants, et que celle qui avait beaucoup enfants est demeurée sans vigueur. » Paroles qui montrent bien que tout ceci n’est qu’une prophétie à ceux qui savent que la perfection de toute l’Église est marquée dans l’Écriture par le nombre sept. C’est pourquoi l’apôtre saint Jean écrit à sept Églises, c’est-à-dire à toute l’Église ; et Salomon dit, dans les Proverbes, que « la Sagesse s’est bâti une maison et l’a appuyée sur sept colonnes. » La Cité de Dieu était réellement stérile chez toutes les nations, avant la naissance de ces enfants qui l’ont rendue féconde. Nous voyons, au contraire, que la Jérusalem terrestre, qui avait un si grand nombre d’enfants, est devenue sans vigueur, parce que les enfants de la femme libre, qui étaient dans son sein, faisaient toute sa force, et qu’elle n’a plus que la lettre sans l’esprit.\par
« C’est Dieu qui donne la mort et qui redonne la vie. » Il a donné la mort à celle qui avait beaucoup d’enfants, et redonné la vie à celle qui était stérile et qui a engendré sept enfants. On peut l’entendre aussi, et mieux encore, en disant qu’il rend la vie à ceux même à qui il avait donné la mort, comme ces paroles qui suivent semblent le confirmer : « C’est lui qui mène aux enfers et qui en ramène. » Ceux à qui l’Apôtre dit : « Si vous êtes morts avec Jésus-Christ, cherchez les choses du ciel où Jésus-Christ est assis à la droite de Dieu » ; ceux-là, dis-je, sont tués par le Seigneur pour leur salut, et c’est pour eux que l’Apôtre ajoute : « Goûtez les choses du ciel, et non pas celles de la terre, afin qu’eux-mêmes soient ceux qui, pressés de la faim, se sont élevés au-dessus de la terre. » Car saint Paul dit encore : « Vous êtes morts » ; et voilà comment Dieu fait mourir ses fidèles pour leur salut : « Et votre vie, ajoute cet Apôtre, est cachée avec Jésus-Christ et Dieu. » Et voilà comment il leur redonne la vie. Mais sont-ce les mêmes qu’il mène aux enfers et qu’il en ramène ? Les deux choses sont indubitablement accomplies en celui qui est notre chef, avec qui l’Apôtre dit que notre vie est cachée en Dieu. Car « celui qui n’a pas épargné son propre fils, mais l’a livré à la mort pour tout le monde », l’a certainement fait mourir de cette façon ; etd’autre part, comme il l’a ressuscité, il lui a redonné la vie. Il l’a aussi mené aux enfers, et l’en a ramené, puisque c’est lui-même qui dit dans le Prophète : « Vous ne laisserez point mon âme dans les enfers. » C’est cette pauvreté du Sauveur qui nous a enrichis. En effet, « c’est le Seigneur qui rend pauvre ou riche. » La suite nous explique ce que cela signifie : « Il abaisse, est-il dit, et il élève. » Il abaisse les superbes et élève les humbles. Tout le discours de cette sainte femme, dont le nom signifie grâce, ne respire autre chose que ce qui est dit dans cet autre endroit de l’Écriture : « Dieu résiste aux superbes, et « donne sa grâce aux humbles. »\par
L’Évangéliste ajoute : « Il relève le pauvre. » Ces paroles ne peuvent s’entendre que de celui qui, étant riche, s’est rendu pauvre pour l’amour de nous, afin que sa pauvreté nous enrichît. » Dieu ne l’a relevé sitôt de terre qu’afin de garantir son corps de corruption. J’estime qu’on peut encore lui attribuer ce qui suit : « Et il tire l’indigent de son fumier. » En effet, ce fumier d’où il a été tiré s’entend fort bien des Juifs qui ont persécuté Jésus-Christ, au nombre desquels se range saint Paul lui-même, dans le temps où il persécutait l’Église. « Ce que je considérais alors comme un gain, dit-il, je l’ai regardé depuis comme une perte, à cause de Jésus-Christ, et non seulement comme une perte, mais comme du fumier, pour gagner Jésus-Christ. » Ce pauvre a donc été relevé de terre au-dessus de tous les riches, et ce misérable tiré du fumier au-dessus des plus opulents, afin de tenir rang parmi les puissants du peuple, à qui il dit : « Vous serez assis sur douze trônes », et à qui, selon l’expression de notre sainte prophétesse, « il donne pour héritage un trône de gloire. » Ces puissants avaient dit : « Vous voyez que nous avons tout quitté pour vous suivre. » Il fallait qu’ils fussent bien puissants pour avoir fait un tel vœu ; mais de qui avaient-ils reçu la force de le faire, sinon de celui dont il est dit ici : « Il donne de quoi vouer à celui qui fait un vœu ? » Autrement, ils seraient de ces puissants dont l’arc a été détendu. « Il donne, dit l’Écriture, à qui fait un vœu de quoi le faire », parce que personne ne pourrait rien vouer à Dieu comme il faut, s’il ne recevaitde lui ce qu’il lui voue. « Et il a béni les années du juste », afin, sans doute, qu’il vive sans fin avec celui à qui il est dit : « Vos années ne finiront point. » Là, les années demeurent fixes, au lieu qu’ici elles passent, ou plutôt elles périssent. Elles ne sont pas avant qu’elles viennent, et quand elles sont venues, elles ne sont plus, parce qu’elles viennent en s’écoulant. Des deux choses exprimées en ces paroles : « Il donne à qui fait un vœu de quoi le faire, et il a béni les années du juste », nous faisons l’une et nous recevons l’autre ; mais on ne reçoit celle-ci de sa bonté que lorsqu’on a fait la première par sa grâce, « attendu que l’homme n’est pas fort par sa propre force. » « Le Seigneur désarmera son adversaire », c’est-à-dire l’envieux qui veut empêcher un homme d’accomplir son vœu. Comme l’expression est équivoque, l’on pourrait entendre par {\itshape son adversaire} l’adversaire de Dieu. Véritablement, lorsque Dieu commence à nous posséder, notre adversaire devient le sien, et nous le surmontons, mais non pas par nos propres forces, car ce que l’homme a de forces ne vient pas de lui. « Le Seigneur donc désarmera son adversaire, le Seigneur qui est saint », afin que cet adversaire soit vaincu par les saints que le Seigneur, qui est le saint des saints, a faits saints.\par
Ainsi, « que le sage ne se glorifie point de sa sagesse, ni le puissant de sa puissance, ni le riche de ses richesses ; mais que celui qui veut se glorifier se glorifie de connaître Dieu et de faire justice au milieu de la terre ». Ce n’est pas peu connaître Dieu, que de savoir que la connaissance qu’on en a est un don de sa grâce. Aussi bien, « qu’avez-vous, dit l’Apôtre, que vous n’ayez point reçu ? Et si vous l’avez reçu, pourquoi vous glorifiez-vous, comme si l’on ne vous l’eût point donné ? » c’est-à-dire comme si vous le teniez de vous-même. Or, celui-là pratique la justice qui vit bien, et celui-là vit bien qui observe les commandements de Dieu, « qui ont pour fin la charité qui naît d’un cœur pur, d’une bonne conscience et d’une foi sincère. » Cette charité vient de Dieu, comme le témoigne l’apôtre saint Jean ; et par conséquent le pouvoir de pratiquer la justice vient aussi de lui. Mais qu’est-ce quececi veut dire : {\itshape au milieu de la terre} ? Est-ce que ceux qui habitent les extrémités de la terre ne doivent point pratiquer la justice ? J’estime que par ces mots : {\itshape au milieu de la terre}, l’Écriture veut dire : {\itshape tant que nous vivons dans ce corps}, afin que personne ne s’imagine qu’après cette vie il reste encore du temps pour accomplir la justice qu’on n’a pas pratiquée ici-bas, et pour éviter le jugement de Dieu. Chacun, dans cette vie, porte sa terre avec soi ; et la terre commune reçoit cette terre particulière à la mort de chaque homme, pour la lui rendre au jour de la résurrection. Il faut donc pratiquer la vertu et la justice au milieu de la terre, c’est-à-dire tandis que notre âme est enfermée dans ce corps de terre, afin que cela nous serve pour l’avenir, « lorsque chacun recevra la récompense du bien et du mal qu’il aura fait par le corps ». Par le corps, dit l’Apôtre, c’est-à-dire pendant le temps qu’il a vécu dans le corps ; car les pensées de blasphème auxquelles on consent ne sont produites par aucun membre du corps ; et cependant on ne laisse pas d’en être coupable. Nous pouvons fort bien entendre de la même sorte cette parole du psaume : « Dieu, qui est notre roi avant tous les siècles, a accompli l’œuvre de notre salut au milieu de la terre », attendu que le Seigneur Jésus est notre Dieu, et il est avant les siècles, parce que les siècles ont été faits par lui. Il a accompli l’œuvre de notre salut au milieu de la terre, lorsque le Verbe s’est fait chair et qu’il a habité dans un corps de terre.\par
« Le Seigneur est monté aux cieux, et il a tonné ; il jugera les extrémités de la terre, parce qu’il est juste. » Cette sainte femme observe dans ces paroles l’ordre de la profession de foi des fidèles. Notre-Seigneur Jésus. Christ est monté au ciel, et il viendra de là juger les vivants et les morts. En effet, comme dit l’Apôtre : « Qui est monté, si ce n’est celui qui est descendu jusqu’aux plus basses parties de la terre ? Celui qui est descendu est le même que celui qui est monté au-dessus de tous les cieux, afin de remplir toutes choses de la présence de sa majesté. » Il a donc tonné par ses nuées qu’il a remplies du Saint. Esprit, quand il est monté aux cieux. Et c’est de ces nuées qu’il parle dans le prophète Isaïe, quand il menace la Jérusalem esclave, c’està-dire la vigne ingrate, d’empêcher qu’elles ne versent la pluie sur elle. « Il jugera les extrémités de la terre », c’est-à-dire même {\itshape les extrémités de la terre}. Et ne jugera-t-il point aussi les autres parties de la terre, lui qui indubitablement doit juger tous les hommes ? Mais peut-être il vaut mieux entendre par les extrémités de la terre l’extrémité de la vie de l’homme. L’homme en effet ne sera pas jugé sur l’état où il aura été au commencement ou au milieu de sa vie, mais sur celui où il se trouvera vers le temps de sa mort ; d’où vient cette parole de l’Évangile, « qu’il n’y aura de sauvé que celui qui persévérera jusqu’à la fin ». Celui donc qui persévère jusqu’à la fin à pratiquer la justice au milieu de la terre ne sera pas condamné, quand Dieu jugera les extrémités de la terre. « C’est lui qui donne la force à nos rois », afin de ne les pas condamner dans son jugement. Il leur donne la force de gouverner leur corps en rois, et de vaincre le monde par la grâce de celui qui a répandu son sang pour eux. « Et il relèvera la gloire et la puissance de son Christ. » Comment le Christ relèvera-t-il la gloire et la puissance de son Christ ? car celui dont il est dit auparavant : « Le Seigneur est monté aux cieux et a tonné », est celui-là même dont il est, dit ici qu’il relèvera la gloire et la puissance de son Christ. Quel est donc le Christ de son Christ ? Est-ce qu’il relèvera la gloire et la puissance de chaque fidèle, comme notre sainte prophétesse le dit elle-même au commencement de ce cantique : « Mon Dieu a relevé ma force et ma gloire » ? Dans le fait, nous pouvons fort bien appeler des Christs tous ceux qui ont été oints du saint chrême, qui tous, néanmoins, avec leur chef, ne sont qu’un même Christ. Voilà la prophétie d’Anne, mère du grand et illustre Samuel ; en lui était figuré alors le changement de l’ancien sacerdoce, qui est accompli aujourd’hui ; car elle qui avait beaucoup d’enfants est devenue sans vigueur, afin que celle qui était stérile et qui est devenue mère de sept enfants eût un nouveau sacerdoce en Jésus-Christ.
\subsection[{Chapitre V}]{Chapitre V}

\begin{argument}\noindent Abolition du sacerdoce d’Aaron Niédite a Héli.
\end{argument}

\noindent L’homme de Dieu qui fut envoyé au grandprêtre Héli et que l’Écriture ne nomme pas, mais que son ministère doit faire indubitablementreconnaître pour prophète, parle de ceci plus clairement. Voici ce que porte le texte sacré : « Un homme de Dieu vint trouver Héli et lui dit : Voici ce que dit le Seigneur : Je me suis fait connaître à la maison de votre père, lorsqu’elle était captive de Pharaon en Égypte, et je l’ai choisie entre toutes les tribus d’Israël pour me faire des prêtres qui montassent à mon autel, qui m’offrissent de l’encens et qui portassent l’éphod ; et j’ai donné à la maison de votre père, pour se nourrir, tout ce que les enfants d’Israël m’offrent en sacrifice. Pourquoi donc avez-vous foulé aux pieds mon encens et mes sacrifices, et pourquoi avez-vous fait plus de cas de vos enfants que de moi, en souffrant qu’ils emportassent les prémices de tous les sacrifices d’Israël ? C’est pourquoi voici ce que dit le Seigneur et le Dieu d’Israël ; J’avais résolu que votre maison et la maison de votre père passeraient éternellement en ma présence. Mais je n’ai garde maintenant d’en user de la sorte. Car je glorifierai ceux qui me glorifient ; et ceux qui me méprisent deviendront méprisables. Voici venir le temps que j’exterminerai votre race et celle de votre père, de sorte qu’il n’en demeurera pas un seul qui exerce les fonctions de la prêtrise, dans ma maison. Je les bannirai tous de mon autel, afin que ceux qui resteront de votre maison sèchent en voyant ce changement. Ils périront tous par l’épée ; et la marque de cela, c’est que vos enfants Ophni et Phinées mourront tous deux en un même jour. Je me choisirai un prêtre fidèle, qui fera tout ce que mon cœur et mon âme désirent, et je lui construirai une maison durable qui passera éternellement en la présence de mon Christ. Quiconque restera de votre maison viendra l’adorer avec une petite pièce d’argent et lui dira ; Donnez-moi, je vous prie, quelque part en votre sacerdoce, afin que je mange du pain. »\par
On ne peut pas dire que cette prophétie, qui prédit si clairement le changement de l’ancien sacerdoce, ait été accomplie en la personne de Samuel. Quoiqu’il ne fût pas d’une autre tribu que celle que Dieu avait destinée pour servir à l’autel, il n’était pas pourtant dela famille d’Aaron, dont la postérité était désignée pour perpétuer le ; et par conséquent tout ceci était la figure du changement qui devait se faire par Jésus-Christ, et appartenait proprement à l’Ancien Testament, et figurativement au Nouveau ; je dis quant à l’événement de la chose, et non quant aux paroles. Il y eut encore depuis des prêtres de la famille d’Aaron, comme Sadoch et Abiathar, sous le règne de David, et plusieurs autres, longtemps avant l’époque où ce changement devait s’accomplir en la personne de Jésus-Christ. Mais à présent quel est celui qui contemple ces choses des yeux de la foi et qui n’avoue qu’elles sont accomplies ? Il ne reste en effet aux Juifs ni tabernacle, ni temple, ni autel, ni sacrifice, ni par conséquent aucun de ces prêtres qui, selon la loi de Dieu, devraient être de la famille d’Aaron, comme le rappelle ici le Prophète : « Voici ce que dit le Seigneur et le Dieu d’Israël : J’avais résolu que votre maison et la maison de votre père passeraient éternellement en ma présence ; mais je n’ai garde maintenant d’en user de la sorte. Car je glorifierai ceux qui me glorifient ; et ceux qui me méprisent deviendront méprisables. » Par la maison de votre père, il n’entend pas parler de celui dont Héli avait pris immédiatement naissance, mais d’Aaron, le premier grand prêtre dont tous les autres sont descendus. Ce qui précède le montre clairement : « Je me suis fait connaître, dit-il, à la maison de votre père, lorsqu’elle était captive de Pharaon en Égypte, et je l’ai choisie entre toutes les tribus d’Israël pour les fonctions du sacerdoce. » Qui était ce père d’Héli dont la famille, après la captivité d’Égypte, fut choisie pour le sacerdoce, sinon Aaron ? C’est donc de cette race que Dieu dit ici qu’il n’y aura plus de prêtre à l’avenir : et c’est ce que nous voyons maintenant accompli. Que notre foi y fasse attention, les choses sont présentes ; on les voit, on les touche, et elles sautent aux yeux, malgré qu’on en ait. « Voici, dit le Seigneur, venir le temps que j’exterminerai votre race et celle de votre père, en sorte qu’il n’en demeurera pas un seul qui exerce les fonctions de la prêtrise dans ma maison. » Je les bannirai tous de mon autel, afin que ceux qui resteront de votre maison sèchent « en voyant ce changement ». Ce temps préditest venu. Il n’y a plus de prêtre selon l’ordre d’Aaron ; et quiconque reste de cette famille, lorsqu’il considère le sacrifice des chrétiens établis par toute la terre et qu’il se voit dépouillé d’un si grand honneur, sèche de regret et d’envie.\par
Ce qui suit appartient proprement à la maison d’Héli : « Tous ceux qui resteront de votre maison périront par l’épée ; et la marque de cela, c’est que vos enfants Ophni et Phinées mourront tous deux en un seul jour. » Le même signe donc qui marquait le sacerdoce enlevé à sa maison marquait aussi qu’il devait être aboli dans la maison d’Aaron. La mort des enfants d’Héli ne figurait la mort d’aucun homme, mais celle du sacerdoce même dans la famille d’Aaron. Ce qui suit se rapporte au grand prêtre, dont Samuel devint la figure en succédant à Héli, et par conséquent on doit l’entendre de Jésus-Christ, le véritable grand prêtre du Nouveau Testament : « Et je me choisirai un prêtre fidèle, qui fera tout ce que mon cœur et mon âme désirent, et je lui construirai une maison durable. » Cette maison est la céleste et éternelle Jérusalem. « Et elle passera, dit-il, éternellement en la présence de mon Christ », c’est-à-dire elle paraîtra devant lui, comme il a dit auparavant de la maison d’Aaron : « J’avais résolu que votre maison et la maison de votre père passeraient éternellement en ma présence. » On peut encore entendre qu’elle passera de la mort à la vie pendant tout le temps de notre mortalité, jusqu’à la fin des siècles. Quand Dieu dit : « Qui fera tout ce que mon cœur et mon âme désirent », ne pensons pas que Dieu ait une âme, lui qui est le créateur de l’âme ; c’est ici une de ces expressions figurées de l’Écriture, comme quand elle donne à Dieu des mains, des pieds, et les autres membres du corps. Au surplus, de peur qu’on ne s’imagine que c’est selon le corps qu’elle dit que l’homme a été fait à l’image de Dieu, elle donne aussi à Dieu des ailes, organe dont l’homme est privé, et elle dit : « Seigneur, mettez-moi à l’ombre de vos ailes », afin que les hommes reconnaissent que tout cela n’est dit que par métaphore de cette nature ineffable.\par
« Et quiconque restera de votre maison viendra l’adorer. » Ceci ne doit pas s’entendre proprement de la maison d’Héli, maisde celle d’Aaron, qui a duré jusqu’à l’avènement de Jésus-Christ et dont il en reste encore aujourd’hui quelques débris. À l’égard de la maison d’Héli, Dieu avait déjà dit que tous ceux qui resteraient de cette maison périraient par l’épée. Comment donc ce qu’il dit ici peut-il être vrai : « Quiconque restera de votre maison viendra l’adorer », à moins qu’on ne l’entende de toute la famille sacerdotale d’Aaron ? Si donc il existe de ces restes prédestinés dont un autre prophète dit : « Les restes seront sauvés » ; et l’Apôtre : « Ainsi, en ce temps même, les restes ont été sauvés selon l’élection de la grâce » ; si, dis-je, il est quelqu’un qui reste de la maison d’Aaron, indubitablement il croira en Jésus-Christ, comme du temps des Apôtres plusieurs de cette nation crurent en lui ; et encore aujourd’hui, l’on en voit quelques-uns, quoique en petit nombre, qui embrassent la foi et en qui s’accomplit ce que cet homme de Dieu ajoute : « Il viendra l’adorer avec une petite pièce d’argent. » Qui viendra-t-il adorer, sinon ce souverain prêtre qui est Dieu aussi ? Car dans le sacerdoce établi selon l’ordre d’Aaron, on ne venait pas au temple ni à l’autel pour adorer le grand prêtre. Que veut dire cette petite pièce d’argent, si ce n’est cette parole abrégée de la foi dont l’Apôtre fait mention après le Prophète, quand il dit : « Le Seigneur fera une parole courte et abrégée sur la terre » ? Or, que l’argent se prenne pour la parole de Dieu, le Psalmiste en témoigne, lorsqu’il dit : « Les paroles du Seigneur sont pures, c’est de l’argent qui a passé par le feu. »\par
Que dit donc celui qui vient adorer le prêtre de Dieu et le prêtre-Dieu ? « Donnez-moi, je vous prie, quelque part en votre sacerdoce, afin que je mange du pain. » Ce qui signifie : Je ne prétends rien à la dignité de mes pères, puisqu’elle est abolie ; faites-moi seulement part de votre sacerdoce. « Car j’aime mieux être méprisable dans la maison du Seigneur » ; entendez : pourvu que je devienne un membre de votre sacerdoce, quel qu’il soit. Il appelle ici sacerdoce le peuple même dont est souverain prêtre le médiateur entre Dieu et les hommes, Jésus-Christ homme. C’est à ce peuple que l’apôtre saint Pierre dit : « Vous êtes le peuple saint et le sacerdoce royal. »\par
Il est vrai que quelques-uns, au lieu de {\itshape votre sacerdoce}, traduisent {\itshape votre sacrifice}, mais cela signifie toujours le même peuple chrétien. De là vient cette parole de l’Apôtre : « Nous ne sommes tous ensemble qu’un seul pain et qu’un seul corps en Jésus-Christ » ; et celle-ci encore : « Offrez vos corps à Dieu comme une hostie vivante. » Ainsi, quand cet homme de Dieu ajoute : « Pour manger du pain », il exprime heureusement le genre même du sacrifice dont le prêtre lui-même dit : « Le pain que je donnerai pour la vie du monde, c’est ma chair. » C’est là le sacrifice qui n’est pas selon l’ordre d’Aaron, mais selon l’ordre de Melchisédech. Que celui qui lit ceci l’entende. Cette confession est en même temps courte, humble et salutaire. « Donnez-moi quelque part en votre sacerdoce, afin que je mange du pain. » C’est là cette petite pièce d’argent, parce que la parole du Seigneur, qui habite dans le cœur de celui qui croit, est courte et abrégée. Comme il avait dit auparavant qu’il avait donné pour nourriture à la maison d’Aaron les victimes de l’Ancien Testament, il parle ici de manger du pain, parce que c’est le sacrifice des chrétiens dans le Nouveau.
\subsection[{Chapitre VI}]{Chapitre VI}

\begin{argument}\noindent De l’éternité promise au sacerdoce et au royaume des Juifs, afin que, les voyant détruits, on reconnut que cette promesse concernait un autre royaume et un autre sacerdoce dont ceux-là étaient la figure.
\end{argument}

\noindent Bien que ces choses paraissent maintenant aussi claires qu’elles étaient obscures lorsqu’elles furent prédites, toutefois il semble qu’on pourrait faire cette objection avec quelque sorte de vraisemblance : Quelle certitude avons-nous que toutes les prédictions des Prophètes s’accomplissent, puisque cet oracle du ciel : « Votre maison et la maison de votre père passeront éternellement en ma présence », n’a pu s’accomplir ? Car nous voyons bien que ce sacerdoce a été changé, sans que cette maison puisse jamais espérer d’y rentrer, attendu qu’il a été aboli, et que cette promesse est plutôt pour l’autre sacerdoce qui a succédé à celui-là. — Quiconque parle de la sorte ne comprend pas encore ou ne se souvient pas que le sacerdoce, mêmeselon l’ordre d’Aaron, était comme l’ombre du sacerdoce à venir et éternel, et qu’ainsi, quand l’éternité lui a été promise, cette promesse ne lui appartenait pas, mais à celui dont il était l’ombre et la figure. Pour que l’on ne s’imaginât pas que l’ombre même dût demeurer, le changement en a dû être aussi prédit.\par
De même, le royaume de Saül, qui fut réprouvé et rejeté, était l’ombre du royaume à venir qui doit subsister éternellement ; car il faut considérer comme un grand mystère cette huile dont il fût sacré et ce chrême qui lui donna le nom de Christ. Aussi David lui-même le respectait si fort en Saül, qu’il frémit de crainte et se frappa la poitrine, au moment où ce prince étant entré dans une caverne obscure pour un besoin, il lui coupa le bord de la robe, afin de lui faire voir qu’il l’avait épargné, quand il pouvait s’en défaire, et de dissiper ainsi ses soupçons et sa furieuse animosité. Il craignait donc de s’être rendu coupable de la profanation d’un grand mystère, seulement pour avoir touché de la sorte au vêtement de Saül. Voici comment l’Écriture en parle : « Et David se frappa la poitrine, parce qu’il avait coupé le pan de sa robe. » Ceux qui l’accompagnaient lui conseillaient de tuer Saül, puisque Dieu le livrait entre ses mains. « À Dieu ne plaise, dit-il, que je le fasse et que je mette la main sur lui ! car il est le Christ du Seigneur. » Ce n’était donc pas proprement la figure qu’il respectait, mais la chose figurée. Ainsi, quand Samuel dit à Saül : « parce que vous n’avez pas fait ce que je vous avais dit, ou plutôt ce que Dieu vous avait dit par moi, le trône d’Israël, que Dieu vous avait préparé pour durer éternellement, ne subsistera point pour vous ; mais le Seigneur cherchera un homme selon son cœur, qu’il établira prince sur son peuple, à cause que vous n’avez pas obéi à ses ordres » ; ces paroles, dis-je, ne doivent pas s’entendre, comme si Dieu, après avoir promis un royaume éternel à Saut, ne voulait plus tenir sa promesse, lorsqu’il eut péché ; car Dieu n’ignorait pas qu’il devait pécher, mais il avait préparé son royaume pour être la figure d’un royaume éternel. C’est pourquoi Samuel ajoute : « Votre royaume ne subsistera point pour vous. » Celui qu’il figurait asubsisté et subsistera toujours, mais non pas pour Saül ni pour ses descendants. « Et le Seigneur, dit-il, cherchera un homme » ; c’est David, ou plutôt c’est le Médiateur même du Nouveau Testament, qui était aussi figuré par le chrême dont David et sa postérité furent sacrés. Or, Dieu ne cherche pas un homme, comme s’il ignorait où il est ; mais il s’accommode au langage des hommes et nous cherche par cela même qu’il nous parle ainsi. Nous étions dès lors si bien connus, non seulement à Dieu le Père, mais à son Fils unique, qui est venu chercher ce qui était perdu, qu’il nous avait élus en lui avant la création du monde. Lors donc que l’Écriture dit {\itshape qu’il cherchera}, c’est comme si elle disait qu’il fera reconnaître aux autres pour son ami celui qu’il sait déjà lui appartenir.
\subsection[{Chapitre VII}]{Chapitre VII}

\begin{argument}\noindent De la division du royaume d’Israël prédite par Samuel à Saül, et de ce qu’elle figurait.
\end{argument}

\noindent Saül pécha de nouveau en désobéissant à Dieu, et Samuel lui porta de nouveau cette parole au nom du Seigneur : « Parce que vous avez rejeté le commandement de Dieu, Dieu vous a rejeté, et vous ne serez plus roi d’Israël. » Comme Saül, avouant son crime, priait Samuel de retourner avec lui pour en obtenir de Dieu le pardon : « Je ne retournerai point avec vous, dit-il, parce que vous n’avez point tenu compte du commandement de Dieu. Aussi le Seigneur ne tiendra point compte de vous, et vous ne serez plus roi d’Israël. » Là-dessus, Samuel lui tourna le dos et s’en alla ; mais Saül le retint par le bas de sa robe, qu’il déchira, Alors Samuel lui dit : « Le Seigneur a ôté aujourd’hui le royaume à Israël en vous l’ôtant, et il le donnera à un de vos proches, qui est bien au-dessus de vous, et Israël sera divisé en deux, sans que le Seigneur change ni se repente, car il ne ressemble pas à l’homme, qui est sujet au repentir, et qui fait des menaces et ne les exécute pas. » Celui à qui il est dit : « Le Seigneur vous rejettera, et vous ne serez plus roi d’Israël » ; et encore : « Le Seigneur a ôté aujourd’hui le royaume à Israël en vous l’ôtant » ; celui-là, dis-je, régna encorequarante ans depuis, car cela lui fut dit dès le commencement de son règne ; mais Dieu entendait par là qu’aucun de sa famille ne devait lui succéder, et il voulait attirer nos regards vers la postérité de David, d’où est sorti, selon la chair, le médiateur entre Dieu et les hommes, Jésus-Christ homme.\par
Or, le texte de l’Écriture ne porte pas, comme beaucoup de traductions latines : « Le Seigneur vous a ôté le royaume d’Israël », mais comme nous l’avons lu dans le grec : « Le Seigneur a ôté aujourd’hui le royaume à Israël en vous l’ôtant » ; par où l’Écriture veut montrer que Saül représentait le peuple d’Israël, qui était destiné à perdre le royaume, Notre-Seigneur Jésus-Christ devant régner spirituellement par le Nouveau Testament.\par
Ainsi, quand il dit : « Et il le donnera à un de vos proches », cela s’entend d’une parenté selon la chair. En effet, selon la chair, Jésus-Christ a pris naissance d’Israël, aussi bien que Saül. Ce qui suit : « Qui est bon au-dessus de vous », peut s’entendre, « qui est meilleur que vous », et quelques-uns l’ont traduit ainsi ; mais je préfère cet autre sens : « Il est bon ; qu’il soit donc au-dessus de vous » ; ce qui est bien conforme à cette autre parole prophétique : « Jusqu’à ce que j’aie mis tous vos ennemis sous vos pieds. » Au nombre des ennemis est Israël, à qui le Christ enlève la royauté comme à son persécuteur. Et toutefois, là aussi était un autre Israël, en qui ne se trouva aucune malice, véritable froment caché sous la paille. C’est de là que sont sortis les Apôtres et tant de martyrs dont saint Étienne a été le premier ; de là ont pris naissance toutes ces Églises dont parle l’apôtre saint Paul et qui louent Dieu de sa conversion.\par
Je ne doute point que par ces mots : « Et Israël sera divisé en deux », il faille distinguer Israël ennemi de Jésus-Christ et Israël fidèle à Jésus-Christ, Israël appartenant à la servante et Israël appartenant à la femme libre. Ces deux Israël étaient d’abord mêlés ensemble, comme Abraham était attaché à la Servante, jusqu’à ce que celle qui était stérile, ayant été rendue féconde par la grâce de Jésus-Christ, s’écriât : « Chassez la servante avec son fils. » Il est vrai qu’Israël fut partagé en deux à cause du péché de Salomon, sous le règne de son fils Roboam, et qu’ildemeura en cet état, chaque faction ayant ses rois à part, jusqu’à ce que toute la nation fût vaincue par les Chaldéens et menée captive à Babylone. Mais qu’est-ce que cela fait à Saül ? Si cette menace était nécessaire, ne devait-on l’adresser plutôt à David, dont Salomon était fils ? maintenant même, les Juifs ne sont pas divisés entre eux, mais dispersés par toute la terre dans la société d’une même erreur. Or, cette division, dont Dieu menace ici ce peuple et ce royaume dans la personne de Saül qui le représentait, doit être éternelle et immuable, selon ces paroles qui suivent : « Dieu ne changera ni ne se repentira point, car il ne ressemble pas à l’homme, qui est sujet au repentir, et qui fait des menaces et ne les exécute pas. » Lorsque L’Écriture dit que Dieu se repent, cela ne marque du changement que dans les choses, lesquelles sont connues de Dieu par une prescience immuable. Quand donc elle dit qu’il ne se repent point, il faut entendre qu’il ne change point.\par
Ainsi l’arrêt de cette division d’Israël est un arrêt perpétuel et irrévocable. Tous ceux qui, en tous les temps, passent de la synagogue des Juifs à l’Église de Jésus-Christ, ne faisant point partie de cette synagogue dans la prescience de Dieu. Ainsi, tous les Israélites qui, s’attachant à Jésus-Christ, persévèrent dans cette union, ne seront jamais avec ces Israélites qui s’opiniâtrent toute leur vie à être ses ennemis, et la division qui est ici prédite subsistera toujours. L’Ancien Testament donné sur la montagne de Sinaï, et qui n’engendra que des esclaves, n’a de prix qu’en ce qu’il rend hommage au Nouveau ; et tous les Juifs qui maintenant lisent Moïse ont un voile sur le cœur qui leur en dérobe l’intelligence. Mais lorsque quelqu’un d’eux passe à Jésus-Christ, ce voile est déchiré. En effet, ceux qui changent de la sorte changent aussi d’intention et de désirs, et n’aspirent plus à la félicité de la chair, mais à celle de l’esprit. C’est pourquoi, dans cette fameuse journée des Juifs contre les Philistins, où le ciel se déclara si ouvertement en faveur des premiers, à la prière de Samuel, ce prophète, prenant une pierre, la posa entre les deux Massephat, la nouvelle et l’ancienne, et l’appela Abennezer, c’est-à-dire {\itshape pierre de secours}, parce que, dit-il, c’est jusqu’ici que Dieu nous a secourus. Or, Massephat signifie{\itshape  intention}, et cette pierre de secours, c’est la médiation du Sauveur, par qui il faut passer de la vieille Massephat à la nouvelle, c’est-à-dire de l’intention qui regardait une fausse et charnelle habitude dans un royaume charnel, à celle qui s’en propose une véritable et spirituelle dans le royaume des cieux par le moyen du Nouveau Testament. Comme il n’est rien de meilleur que cette félicité, c’est jusque-là que Dieu nous porte secours.
\subsection[{Chapitre VIII}]{Chapitre VIII}

\begin{argument}\noindent Les promesses de Dieu à David touchant Salomon ne peuvent s’entendre que de Jésus-Christ.
\end{argument}

\noindent Il faut voir maintenant, autant que cela peut servir à notre dessein, les promesses que Dieu fit à David même, qui prit la place de Saül, changement qui était la figure du changement suprême auquel se rapporte toute l’Écriture sainte. Toutes choses prospérant à David, il résolut de bâtir une maison à Dieu, ce fameux temple qui fut l’ouvrage de son fils Salomon. Comme il était dans cette pensée, Dieu parla au prophète Nathan, et, après lui avoir déclaré que David ne lui bâtirait pas une maison, et qu’il s’en était bien passé jusqu’alors : « Vous direz, ajouta-t-il, à mon serviteur David : Voici ce que dit le Seigneur tout-puissant : Je vous ai tiré de votre bergerie pour vous établir le conducteur de mon peuple. Je vous ai assisté dans toutes vos entreprises, j’ai dissipé tous vos ennemis, et j’ai égalé votre gloire à celle des plus grands rois. Je veux assigner un lieu à mon peuple et l’y établir, afin qu’il y demeure séparé des autres nations et que rien ne trouble son repos à l’avenir. Les méchants ne l’opprimeront plus comme autrefois, lorsque je lui donnai des Juges pour le conduire. Je ferai que tous vos ennemis vous laisseront en paix, et vous me bâtirez une maison. Car lorsque vos jours seront accomplis et que vous serez endormi avec vos pères, je ferai sortir de votre race un roi dont j’affermi rai le trône. C’est lui qui me construira une maison, et je maintiendrai éternelle ment son empire. Je lui tiendrai lieu de père et l’aimerai comme mon fils. Que s’ilvient à m’offenser, je lui ferai sentir les effets de ma colère et le châtierai avec rigueur ; mais je ne retirerai point de lui ma miséricorde, comme j’ai fait à l’égard de ceux dont j’ai détourné ma face. Sa maison me sera fidèle et son royaume durera autant que les siècles. »\par
Quiconque s’imagine que cette promesse a été accomplie en Salomon, se trompe gravement, et son erreur vient de ce qu’il ne s’arrête qu’à ces paroles : « C’est lui qui me construira une maison. » En effet, Salomon a élevé un temple superbe ; mais il faut faire attention à ce qui suit : « Sa maison me sera fidèle et son royaume durera autant que les siècles. » Regardez maintenant le palais de Salomon, tout rempli de femmes étrangères et idolâtres qui le portent à adorer les faux dieux avec elles ; et prenez garde d’être assez téméraires pour penser que les promesses de Dieu ont été vaines, ou qu’il n’a pu prévoir que ce prince et sa maison tomberaient dans de tels égarements. Lors même que nous ne verrions point les paroles divines accomplies en la personne de Notre-Seigneur Jésus-Christ, qui est né de David selon la chair, nous ne devrions point douter qu’elles ne se rapportent à lui, à moins que de vouloir attendre vainement un nouveau messie, comme font les Juifs. Il est si vrai que par ce fils, qui est ici promis à David, les Juifs mêmes n’entendent point Salomon, que, par un merveilleux aveuglement, ils attendent encore un autre Christ que celui qui s’est fait reconnaître pour tel par des marques si claires et si évidentes. À la vérité, on voit aussi en Salomon quelque image des choses à venir, en ce qu’il a bâti le temple, qu’il a eu la paix avec tous ses voisins, comme le porte son nom (car Salomon signifie {\itshape pacifique}) et que les commencements de son règne ont été admirables ; mais il faut demeurer d’accord qu’il n’était pas Jésus-Christ lui-même et qu’il n’en était que la figure. De là vient que l’Écriture dit beaucoup de choses de lui, non seulement dans les livres historiques, mais dans le psaume soixante-onzième qui porte son nom, lesquelles ne sauraient du tout lui convenir, et conviennent fort bien à Jésus-Christ, pour montrer que l’un n’était que la figure, et l’autre la vérité. Pour n’en citer qu’un exemple, on ignore quelles étaient les bornes du royaume de Salomon, et cependant nous lisons dans ce psaume : « Il étendra son empire de l’une à l’autre mer, et depuis le fleuve jusqu’aux extrémités de la terre » ; paroles que nous voyons accomplies en la personne du Sauveur, qui a commencé son règne au fleuve où il fut baptisé par saint Jean et reconnu par les disciples, qui ne l’appelaient pas seulement Maître, mais Seigneur.\par
Pourquoi Salomon commença-t-il à régner du vivant de son père David, ce qui n’arriva à aucun autre des rois d’Israël ? pour nous apprendre que ce n’est pas de lui que Dieu parle ici, quand il dit à David : « Lorsque vos jours seront accomplis et que vous serez endormi avec vos pères, je ferai sortir de votre race un roi dont j’affermirai le trône. » Quelque intervalle de temps qu’il y ait entre Jésus-Christ et David, toujours est-il certain que le premier est venu depuis la mort du second et qu’il a bâti une maison à Dieu, non de bois et de pierre, mais d’hommes. C’est à cette maison, ou en d’autres termes, aux fidèles, que l’apôtre saint Paul dit : « Le temple de Dieu est saint, et c’est vous qui êtes ce temple. »
\subsection[{Chapitre IX}]{Chapitre IX}

\begin{argument}\noindent De la prophétie du psaume quatre-vingt-huitième, laquelle est semblable à celle de Nathan dans le second livre des Rois.
\end{argument}

\noindent C’est pour cela qu’au psaume quatre-vingt-huitième, qui a pour titre : {\itshape Instruction pour Aethan, israélite}, il est fait mention des promesses de Dieu à David, et l’on y voit quelque chose de semblable à ce que nous venons de rapporter du second livre des Rois. « J’ai juré, dit Dieu, j’ai juré à David, mon serviteur, que je ferais fleurir éternellement sa race. » Puis : « Vous avez parlé en vision à vos enfants, et vous avez dit : J’ai remis mon assistance dans un homme puissant, et j’ai élevé sur le trône celui que j’ai choisi parmi mon peuple. J’ai trouvé mon serviteur David, je l’ai oint de mon huile sainte. Car ma main lui donnera secours et mon bras le soutiendra. L’ennemi n’aura point avantage sur lui, et l’enfant d’iniquité ne lui pourra nuire. J’abattrai ses ennemis à ses pieds et mettrai en fuite ceux qui le haïssent. Ma vérité et ma miséricorde seront avec lui, et jedélivrerai sa gloire et sa puissance. J’étendrai sa main gauche sur la mer et sa droite sur les fleuves. Il m’invoquera et me dira : Vous êtes mon père, vous êtes mon Dieu et mon asile. Et je le ferai mon fils aîné et l’élèverai au-dessus de tous les rois de la terre. Je lui conserverai toujours ma faveur, et l’alliance que je ferai avec lui sera inviolable. J’établirai sa race pour jamais, et son trône durera autant que les cieux. » Tout cela, sous le nom de David, doit s’entendre de Jésus-Christ, à cause de la forme d’esclave qu’il a prise, comme médiateur, dans le sein de la Vierge. Quelques lignes ensuite, il est parlé des péchés de nos enfants presque dans les mêmes termes où, au livre des Rois, il est parlé de ceux de Salomon : « S’il vient, dit Dieu en ce livre, à s’abandonner à l’iniquité, je le châtierai par la verge des hommes ; je le livrerai aux atteintes des enfants des hommes ; cependant je ne retirerai pas de lui ma miséricorde. » Ces atteintes sont les marques du châtiment ; et de là cette parole : « Ne touchez pas mes christs. » Qu’est-ce à dire, sinon : Ne blessez pas ? Or, dans le psaume où il s’agit de David en apparence, le Seigneur tient à peu près le même langage : « Si ses enfants, dit-il, abandonnent ma loi et ne marchent dans ma crainte, s’ils profanent mes ordonnances et ne gardent pas mes commandements, je les châtierai, la verge à la main, et je leur enverrai mes fléaux ; mais je ne retirerai point de lui ma miséricorde. » Il ne dit pas : {\itshape Je ne retirerai pas d’eux}, quoiqu’il parle de ses enfants, mais {\itshape de lui}, ce qui pourtant, à le bien prendre, est la même chose. Aussi bien on ne peut trouver en Jésus-Christ même, qui est le chef de l’Église, aucun péché qui ait besoin d’indulgence ou de punition, mais bien dans son peuple, qui compose ses membres et son corps mystique. C’est pour cela qu’au livre des Rois il est parlé de son {\itshape iniquité}, au lieu qu’ici il est parlé de celle de ses enfants, pour nous faire entendre que ce qui est dit de son corps est dit en quelque sorte de lui-même.\par
Par la même raison, lorsque Saul persécutait son corps, c’est-à-dire ses fidèles, il lui cria du ciel : « Saul, Saul, pourquoi me persécutez-vous. » Le psaume ajoute : « Je n’enfreindrai point mon serment, ni neprofanerai mon alliance ; je ne démentirai point les paroles qui sortent de ma bouche ; j’ai une fois juré par ma sainteté, je ne tromperai point David ; sa race durera éternellement ; son trône demeurera à jamais devant moi comme le soleil et la lune, et comme l’arc-en-ciel, témoin fidèle de mon alliance. »
\subsection[{Chapitre X}]{Chapitre X}

\begin{argument}\noindent La raison de la différence qui se rencontre entre ce qui s’est passé dans le royaume de la Jérusalem terrestre et les promesses de Dieu, c’est de faire voir que ces promesses regardaient un autre royaume et un plus grand roi.
\end{argument}

\noindent Après des assurances si certaines d’une si grande promesse, de peur qu’on ne la crût accomplie en Salomon et qu’on ne l’y cherchât inutilement, le Psalmiste s’écrie : « Pour vous, Seigneur, vous les avez rejetés et anéantis. » Cela est arrivé à l’égard du royaume de Salomon en ses descendants jusqu’à la ruine de la Jérusalem terrestre, qui était le siège de son empire, et à la destruction du temple qu’il avait élevé. Mais, pour qu’on n’aille pas en conclure que Dieu a contrevenu à sa parole, David ajoute aussitôt : « Vous avez différé votre Christ. » Ce Christ n’est donc ni David, ni Salomon, puisqu’il est différé. Encore que tous les rois des Juifs fussent appelés christs à cause du chrême dont on les oignait à leur sacre, et que David lui-même donne ce nom à Saül, il n’y avait toutefois qu’un seul Christ véritable, dont tous ceux-là étaient la figure. Et ce Christ était différé pour longtemps, selon l’opinion de ceux qui croyaient que ce devait être David ou Salomon ; mais il devait venir en son temps, selon l’ordre de la providence de Dieu. Cependant le psaume nous apprend ensuite ce qui arriva durant ce délai dans la Jérusalem terrestre, où l’on espérait qu’il régnerait : « Vous avez, dit-il, rompu l’alliance que vous aviez faite avec votre serviteur ; vous avez profané son temple. Vous avez renversé tous ses boulevards, et ses citadelles n’ont pu le mettre en sûreté. Tous les passants l’ont pillé ; il est devenu l’opprobre de ses voisins. Vous avez protégé ceux qui l’opprimaient et donné des sujets de joie à ses ennemis. Vous avez émoussé la pointe de son épée et ne l’avez point aidé dans le combat. Vous avez obscurci l’éclat de sa gloire et brisé son trône. Vous avez abrégé le temps de son règne, et il est couvert de confusion. » Tous ces malheurs sont tombés sur la Jérusalem esclave, où même quelques enfants de la liberté ont régné, quoiqu’ils ne soupirassent qu’après la Jérusalem céleste dont ils étaient sortis et où ils espéraient régner un jour par le moyen du Christ véritable. Mais si l’on veut savoir comment tous ces maux lui sont arrivés, il faut l’apprendre de l’histoire.
\subsection[{Chapitre XI}]{Chapitre XI}

\begin{argument}\noindent De la substance du peuple de Dieu, laquelle se trouve en Jésus-Christ fait homme, seul capable de délivrer son âme de l’enfer.
\end{argument}

\noindent Le Prophète adresse ensuite une prière à Dieu ; mais sa prière même est une prophétie : « Jusques à quand, Seigneur, détournerez-vous jusqu’à la fin ? » il faut sous-entendre {\itshape votre face} ou {\itshape votre miséricorde}. Par {\itshape la fin}, sont exprimés les derniers temps où cette nation même croira en Jésus-Christ. Mais, avant cela, il faut que tous les malheurs que le Prophète a déplorés arrivent. C’est pourquoi il ajoute : « Votre colère s’allumera comme un feu. Souvenez-vous quelle est ma substance. » Par cette substance, l’on ne peut rien concevoir de mieux que Jésus-Christ même, qui a tiré de ce peuple sa substance et sa nature humaine. « Car ce n’est pas en vain, dit-il, que vous avez créé tous les enfants des hommes. » En effet, sans ce fils de l’homme, sans cette substance d’Israël par qui sont sauvés plusieurs enfants deshommes, ce serait en vain que les enfants des hommes auraient été créés, tandis que maintenant il est vrai que toute la nature humaine est tombée de la vérité dans la vanité par le péché du premier homme, d’où vient cette parole d’un autre psaume : « L’homme est devenu semblable à une chose vaine et chimérique ; ses jours s’évanouissent comme l’ombre » ; mais ce n’est pourtant pas en vain que Dieu a créé tous les enfants des hommes, puisqu’il en délivre plusieurs par le médiateur Jésus, et que les autres, qu’il a prévus ne devoir pas délivrer, il les a créés en vertu d’un dessein très beau et très juste, pour servir au bien des élus, et pour releverpar l’opposition des deux cités l’éclat et la gloire de la céleste. Le Psalmiste ajoute : « Quel est cet homme qui vivra et ne mourra point ; il délivrera son âme des mains de l’enfer. » Quel est-il, en effet, sinon cette substance d’Israël tirée de David, c’est-à-dire Jésus-Christ, dont l’Apôtre dit : « Une fois ressuscité des morts, il ne meurt plus, et la mort n’a plus d’empire sur lui. » Bien qu’il vive maintenant et qu’il ne soit plus sujet à la mort, il n’a pas laissé de mourir ; mais il a délivré son âme de l’enfer, où il était descendu pour rompre les liens du péché qui en retenaient quelques-uns captifs. Or, il l’a délivrée par cette puissance dont il dit dans l’Évangile : « J’ai le pouvoir de quitter mon âme et j’ai le pouvoir de la reprendre. »
\subsection[{Chapitre XII}]{Chapitre XII}

\begin{argument}\noindent Comment il faut entendre ces paroles du psaume quatre-vingt-huitième : « Où sont, Seigneur, les anciennes miséricordes, etc. »
\end{argument}

\noindent Examinons maintenant la fin de ce psaume, qui est ainsi conçu : « Seigneur, où sont les anciennes miséricordes que vous avez fait serment d’exercer envers David ? Souvenez-vous, Seigneur, de l’opprobre de vos serviteurs, et qu’il m’a fallu essuyer sans rien dire les reproches de tant de nations, ces reproches injurieux que vos ennemis m’ont faits du changement de votre Christ. » En méditant ces paroles, il est permis de demander si elles s’appliquent aux Israélites, qui désiraient que Dieu accomplît la promesse qu’il avait faite à David, ou bien à la personne des chrétiens qui sont Israélites selon l’esprit et non selon la chair. Il est certain, en effet, qu’elles ont été dites ou écrites du vivant d’Aethan, dont le nom est à la tête de ce psaume et sous le règne de David ; et par conséquent il n’y a point d’apparence que l’on pût dire alors : « Seigneur, où sont les anciennes miséricordes que vous avez fait serment d’exercer envers David ? » à moins que le Prophète ne se mît à la place de ceux qui devaient venir longtemps après et à l’égard de qui ces promesses faites à David étaient anciennes. On peut donc entendre que lorsque les Gentils persécutaient les chrétiens, ils leur reprochaient la passion de Jésus-Christ, quel’Écriture appelle un changement, parce qu’en mourant il est devenu immortel. On peut aussi entendre que le changement du Christ a été reproché aux Juifs, en ce qu’au lieu qu’ils l’attendaient comme leur sauveur, il est devenu le sauveur des Gentils. C’est ce que plusieurs peuples, qui ont cru en lui par le Nouveau Testament, leur reprochent encore aujourd’hui ; de sorte que c’est en leur personne qu’il est dit : « Souvenez-vous, Seigneur, de l’opprobre de vos serviteurs », parce que Dieu, ne les oubliant pas, mais ayant compassion de leur misère, doit les attirer un jour eux-mêmes à la grâce de l’Évangile. Mais il me semble que le premier sens est meilleur. En effet, il ne paraît pas à propos d’appeler serviteurs de Dieu les ennemis de Jésus-Christ à qui l’on reproche que le Christ les a abandonnés pour passer aux Gentils, et que cette qualité convient mieux à ceux qui, exposés à de rudes persécutions pour le nom de Jésus-Christ, se sont souvenus du royaume promis à la race de David, et touchés d’un ardent désir de le posséder, ont dit à Dieu : « Seigneur, où sont les anciennes miséricordes que vous avez fait serment d’exercer envers David ? Souvenez-vous, Seigneur, de l’opprobre de vos serviteurs, et qu’il m’a fallu essuyer sans rien dire les reproches de tant de nations, ces reproches injurieux que vos ennemis m’ont faits du changement de votre Christ », ce changement étant pris par eux pour un anéantissement. Que veut dire : Souvenez-vous, Seigneur, sinon ayez pitié de moi, et, pour les humiliations que j’ai souffertes avec tant de patience, donnez-moi la gloire que vous avez promise à David avec serment. Que si nous attribuons ces paroles aux Juifs, assurément ces serviteurs de Dieu, qui furent emmenés captifs à Babylone après la prise de la Jérusalem terrestre et avant la naissance de Jésus-Christ, ont pu les dire aussi, entendant par le {\itshape changement du Christ}, qu’ils ne devaient pas attendre de lui une félicité temporelle semblable à celle dont ils avaient joui quelques années auparavant sous le règne de Salomon, mais une félicité céleste et spirituelle ; et c’est le changement que les nations idolâtres reprochaient, sans s’en douter, au peuple de Dieu, lorsqu’elles l’insultaient dans sa captivité. C’est aussi ce qui se trouve ensuite dans le même psaume et qui en fait la conclusion : « Que la bénédiction du Seigneur demeure éternellement ; ainsi soit-il, ainsi soit-il » ; vœu très convenable à tout le peuple de Dieu qui appartient à la Jérusalem céleste, soit à l’égard de ceux qui étaient cachés dans l’Ancien Testament avant que le Nouveau ne fût découvert, soit pour ceux qui dans le Nouveau sont manifestement à Jésus-Christ. La bénédiction du Seigneur promise à la race de David n’est pas circonscrite dans un aussi petit espace de temps que le règne de Salomon, mais elle ne doit avoir d’autres bornes que l’éternité. La certitude de l’espérance que nous en avons est marquée par la répétition de ces mots : « Ainsi soit-il, ainsi soit-il. » C’est ce que David comprenait bien quand il dit, au second livre des Rois, qui nous a conduits à cette digression du Psaume : « Vous avez parlé pour longtemps en faveur de la maison de David » ; et un peu après : « Commencez donc maintenant, et bénissez pour jamais la maison de votre serviteur, etc. » parce qu’il était prêt d’engendrer un fils dont la race était destinée à donner naissance à Jésus-Christ, qui devait rendre éternelle sa maison et en même temps la maison de Dieu. Elle est la maison de David à raison de sa race, et la maison de Dieu à cause de son temple, mais d’un temple qui est fait d’hommes et non de pierres, et où le peuple doit demeurer éternellement avec son Dieu et en son Dieu, et Dieu avec son peuple et en son peuple, en sorte que Dieu remplisse son peuple et que le peuple soit plein de son Dieu, lorsque Dieu sera tout en tous, Dieu, notre récompense dans la paix et notre force dans le combat. Comme Nathan avait dit à David : « Le Seigneur vous avertit que vous lui bâtirez une maison » ; David dit ensuite à Dieu : « Seigneur tout-puissant, Dieu d’Israël, vous avez révélé à votre serviteur que vous lui bâtiriez une maison. » En effet, nous bâtissons cette maison en vivant bien, et Dieu la bâtit aussi en nous aidant à bien vivre ; car, « si le Seigneur ne bâtit lui-même une maison, en vain travaillent ceux qui la bâtissent. » Lorsque le temps de la dernière dédicace de cette maison sera venu, alors s’accomplira ce que Dieu dit ici par Nathan : « J’assignerai un lieu à mon peuple, et l’y établirai, afin qu’il y demeure séparé des autres nations et querien ne trouble son repos à l’avenir. Les méchants ne l’opprimeront plus comme autrefois, lorsque je lui donnai des Juges pour le conduire. »
\subsection[{Chapitre XIII}]{Chapitre XIII}

\begin{argument}\noindent La paix promise à David par Nathan n’est point celle du règne de Salomon.
\end{argument}

\noindent C’est une folie d’attendre ici-bas un si grand bien, ou de s’imaginer que ceci ait été accompli sous le règne de Salomon, à cause de la paix dont on y jouit. L’Écriture ne relève cette paix que parce qu’elle était la figure d’une autre ; et elle-même a eu soin de prévenir cette interprétation, lorsque, après avoir dit : « Les méchants ne l’opprimeront plus », elle ajoute aussitôt : « comme autrefois, lorsque je lui donnai des Juges pour le conduire ». Ce peuple, avant d’être gouverné par des rois, fut gouverné par des Juges, et les {\itshape méchants}, c’est-à-dire ses ennemis, l’opprimaient par moments ; mais, avec tout cela, on trouve sous les Juges de plus longues paix que celle du règne de Salomon, qui dura seulement quarante ans. Or, il y en eut une de quatre-vingts ans sous Aod. Loin donc, loin de nous l’idée que cette promesse regarde le règne de Salomon, et beaucoup moins celui d’un autre roi, puisque pas un d’eux n’a joui de la paix aussi longtemps que lui, et que cette nation n’a cessé d’appréhender le joug des rois, ses voisins. Et n’est-ce pas une suite nécessaire de l’inconstance des choses du monde qu’aucun peuple ne possède un empire si bien affermi qu’il n’ait pas à redouter l’invasion étrangère ? Ainsi, ce lieu d’une habitation si paisible et si assurée, qui est ici promis, est un lieu éternel, et qui est dû à des habitants éternels dans la Jérusalem libre où régnera véritablement le peuple d’Israël ; car Israël signifie {\itshape voyant Dieu}. Et nous, pénétrés du désir de mériter une si haute récompense, que la foi nous fasse vivre d’une vie sainte et innocente à travers ce douloureux pèlerinage !
\subsection[{Chapitre XIV}]{Chapitre XIV}

\begin{argument}\noindent Des psaumes de David.
\end{argument}

\noindent La Cité de Dieu poursuivant son cours dans le temps, David régna d’abord sur la Jérusalem terrestre, qui était une ombre et unefigure de la Jérusalem à venir. Ce prince était savant dans la musique, et il aimait l’harmonie, non pour le plaisir de l’oreille, mais avec une intention plus élevée, pour consacrer à son Dieu des cantiques remplis de grands mystères. L’assemblage et l’accord de plusieurs tons différents sont en effet une image fidèle de l’union qui enchaîne les différentes par-tics d’une cité bien ordonnée. On sait que toutes les prophéties de David sont contenues dans les cent cinquante psaumes que nous appelons le Psautier. Or, les uns veulent qu’entre ces psaumes, ceux-là seulement soient de lui qui portent son nom ; d’autres ne lui attribuent que ceux qui ont pour titre {\itshape de David}, et disent que ceux où on lit {\itshape à David} ont été faits par d’autres et appropriés à sa personne. Mais ce sentiment est réfuté par le Sauveur même dans l’Évangile, lorsqu’il dit que David lui-même a appelé le Christ son Seigneur dans le psaume cent neuf, en ces termes : « Le Seigneur a dit à mon Seigneur : Asseyez-vous à ma droite, jusqu’à ce que j’aie abattu vos ennemis sous vos pieds. » Or, ce psaume n’a pas pour titre de David, mais â David. Il lui semble donc que l’opinion la plus vraisemblable, c’est que tous les psaumes sont de David, et que, s’il en a intitulé quelques-uns d’autres noms que du sien, c’est que ces noms ont un sens figuratif, quant à ceux qu’il a laissés sans y mettre de nom, c’est par une inspiration de Dieu, dont le motif caché couvre sans doute de profonds mystères. Il ne faut point s’arrêter à ce que certains psaumes portent en tête les noms de quelques prophètes qui ne sont venus que longtemps depuis David, et qui semblent toutefois y parler ; car l’esprit prophétique qui inspirait ce prince a fort bien pu aussi lui révéler les noms de ces prophètes, et lui suggérer des chants qui leur étaient appropriés, comme nous voyons qu’un certain prophète a parlé de Josias et de ses actions plus de trois cents ans avant la naissance de ce roi.
\subsection[{Chapitre XV}]{Chapitre XV}

\begin{argument}\noindent S’il convient d’entrer ici dans l’explication des prophéties contenues dans les psaumes touchant Jésus-Christ et son Église.
\end{argument}

\noindent Je vois bien qu’on attend de moi que j’explique ici les prophéties de Jésus-Christet de son Église qui sont dans les psaumes ; mais ce qui me retient, quoique ayant déjà donné l’explication d’un de ces divins cantiques, c’est plutôt l’abondance que le défaut de la matière. Il serait trop long, en effet, d’expliquer ces prophéties ; et si je restreignais mon choix, j’aurais à craindre que les hommes versés en ces problèmes ne m’accusassent d’avoir omis les plus essentielles. D’ailleurs, un témoignage qu’on produit d’un psaume doit être confirmé par toute la suite du psaume, afin que, si tout ne sert pas à l’appuyer, rien au moins n’y soit contraire. En procédant de toute autre façon, on ferait des centons que l’on appliquerait à son sujet dans un sens tout différent de celui que les pièces ont à leur place naturelle. Pour montrer ce rapport de toutes les parties du psaume, avec le témoignage qu’on en voudrait faire sortir, il serait besoin de l’expliquer tout entier. Or, quel travail exigerait cette méthode, il est aisé de l’imaginer, pour peu qu’on sache ce que d’autres ont entrepris en ce genre et ce que nous avons nous-même essayé ailleurs. Que celui qui en aura la volonté et le loisir lise ces commentaires, et il y verra combien de grandes choses David a prophétisées de Jésus-Christ et de son Église, c’est-à-dire de la cité qu’il a fondée et de son roi.
\subsection[{Chapitre XVI}]{Chapitre XVI}

\begin{argument}\noindent Le psaume quarante-quatre est une prophétie, tantôt expressive et tantôt figurée, de Jésus-Christ et de son Église.
\end{argument}

\noindent Quelles que soient, en toutes choses, la propriété et la clarté des expressions prophétiques, il faut aussi qu’il y en ait de figurées, et ce sont celles-là qui donnent de l’exercice aux savants, quand ils veulent les expliquer à des esprits moins ouverts. Il en est toutefois qui désignent, à la première vue, le Sauveur et son Église, quoiqu’il y reste toujours quelque chose d’obscur qui demande à être expliqué à loisir ; par exemple, ce passage du psaume quarante-quatre : « Mon cœur me presse de dire de grandes choses ; je veux consacrer mes ouvrages à la gloire de mon Roi. Ma langue est comme la plume d’un écrivain qui écrit très vite. Vous êtes le plus beau des enfants des hommes ; les grâces sont répandues sur vos lèvres ; c’est pourquoi Dieu vous a comblé de ses bénédictions pour jamais. Très puissant, ceignez votre épée. Beau et gracieux comme vous l’êtes, vous ne sauriez manquer de réussir dans vos entreprises et de vous rendre maître des cœurs. La vérité, la douceur et la justice accompagnent vos pas, et vous signalerez votre puissance par des actions miraculeuses. Dieu tout-puissant, que vos flèches sont aigües ! vous en percerez le cœur de vos ennemis, et les peuples tomberont à vos pieds. Votre trône, mon Dieu, est un trône éternel, et le sceptre de votre empire est un sceptre de justice. Vous avez aimé la justice et haï l’iniquité ; aussi votre Dieu a rempli votre cœur de joie comme d’un heaume exquis, dont il vous a sacré avec plus d’abondance que tous vos compagnons. Vos vêtements sont imprégnés de myrrhe et d’aloès ; des essences de parfum s’exhalent de vos palais d’ivoire, et c’est ce qui vous a gagné le cœur des jeunes filles au jour de votre triomphe. » Quel est l’esprit assez grossier pour ne pas reconnaître dans ces paroles le Christ que nous prêchons et en qui nous croyons ? Qui ne le voit désigné par ce Dieu dont le trône est éternel, et que Dieu sacre en Dieu, c’est-à-dire d’un chrême spirituel et invisible ? Est-il un homme assez étranger à notre religion et assez sourd au bruit qu’elle fait de toutes parts pour ignorer que le Christ s’appelle ainsi de son sacre et de son onction ? Or, ce roi une fois reconnu, que signifient les autres traits de cette peinture symbolique, par exemple, qu’il est le plus beau des enfants des hommes, d’une beauté sans doute d’autant plus digne d’amour et d’admiration qu’elle est moins corporelle ? Que veut dire cette épée, et que sont ces flèches ? c’est à quiconque sert ce Dieu et règne par la vérité, la douceur et la justice, à examiner ces questions à loisir. Jetez ensuite les yeux sur son Église, sur cette compagne unie à un si grand époux par un mariage spirituel et par les liens d’un amour divin, elle, dont il est dit peu après : « La reine s’est assise à votre droite avec un habit rehaussé d’or et de broderie. Écoutez, ma fille, voyez et prêtez l’oreille ; oubliez votre pays et la maison de votre père ; car le roi a été pris d’amour pour votre beauté, et il est le Seigneur votre Dieu. Les habitants de Tyr l’adoreront avec des présents ; les plus riches du peuple vous feront la cour. Toute la gloire de la fille du roi vient du dedans, et elle est vêtue d’une robe à franges d’or, toute couverte de broderies. On amènera au roi les filles de sa suite ; on vous offrira celles qui approchent de plus près de sa personne. On les amènera avec joie et allégresse ; on les fera entrer dans le palais du roi. Il vous est né des enfants à la place de vos pères ; vous les établirez princes sur tout l’univers. Ils se souviendront de votre nom, Seigneur, dans la suite de tous les âges. C’est pourquoi tous les peuples vous loueront éternellement et dans tous les siècles. » Je ne pense pas que quelqu’unsoit assez fou pour s’imaginer que ceci doit s’entendre d’une simple femme, puisque cettefemme est l’épouse de celui à qui il est dit : « Votre trône, mon Dieu, est un trône éternel, et le sceptre de votre empire est un sceptre de justice. Vous avez aimé la justice et haï l’iniquité ; aussi votre Dieu a rempli votre cœur de joie comme d’un baume exquis, dont il vous a sacré avec plus d’abondance que tous vos compagnons. » C’est Jésus-Christ qui a été ainsi sacré d’une onction plus pleine que tout le reste des chrétiens ; et ceux-là sont les compagnons de sa gloire, dont l’union et la concorde par tout l’univers sont figurées par cette reine appelée dans un autre psaume la cité du {\itshape grand roi}. Voilà cette spirituelle Sion dont le nom signifie {\itshape contemplation}, parce qu’elle contemple les grands biens de l’autre vie et y tourne toutesses pensées ; voilà cette Jérusalem céleste dont nous avons dit tant de choses, et qui a pour ennemie la cité du diable, Babylone, c’est-à-dire {\itshape confusion}. C’est par la régénération que cette reine est délivrée de la domination de Babylone, et passe de la domination d’un très méchant prince sous celle d’un très bon roi. On lui dit pour cette raison : « Oubliez votre pays et la maison de votre père. » Les Israélites, qui ne sont tels que selon la chair et non par la foi, font partie de cette cité impie, et sont ennemis du {\itshape grand roi} et de la reine, son épouse. Car, puisqu’ils ont mis à mort celui qui était venu vers eux, le Christ a été plutôt le sauveur de ceux qu’il n’a pas vus, alors qu’il était sur la terre revêtu d’une chair mortelle. Aussi dit-on à notre roi dans un psaume : « Vous me délivrerez des révoltes de ce peuple, vous m’établirez chef desnations. Un peuple que je ne connaissais point m’a servi ; il m’a obéi aussitôt qu’il a entendu parler de moi. » Le peuple des Gentils que le Christ n’a pas connu lorsqu’il était au monde, et qui néanmoins croit en lui sur ce qu’il a appris, en sorte que c’est justement qu’il est écrit de lui : « Il m’a obéi aussitôt qu’il a entendu parler de moi » car « la loi vient de l’ouïe » ; ce peuple, dis-je, joint aux vrais Israélites selon la chair et selon la foi, compose la cité de Dieu, qui a aussi engendré le Christ selon la chair, quand elle n’était qu’en ces seuls Israélites. De là était la vierge Marie, dans le sein de laquelle le Christ a pris chair pour devenir homme. C’est de cette cité qu’un autre psaume dit : « On dira de Sion, notre mère : Un homme et un homme par excellence a été fait en elle, et c’est le Très-Haut lui-même qui l’a fondé. » Quel est ce Très-Haut, sinon Dieu ? Et par conséquent le Christ, qui est Dieu et qui l’était avant que de devenir homme dans cette cité par l’entremise de Marle, l’a fondée lui-même dans les patriarches et dans les Prophètes. Puis donc que le Sauveur a été prédit si longtemps auparavant à cette cité de Dieu, à cette reine, suivant cette parole que nous voyons maintenant accomplie : « Il vous est né des enfants à la place de vos pères, que vous établirez princes sur tout l’univers » quelque obscurité qu’il y ait ici dans les autres expressions figurées, et de quelque façon qu’on les explique, elles doivent s’accorder avec des choses qui soit si claires.
\subsection[{Chapitre XVII}]{Chapitre XVII}

\begin{argument}\noindent Du sacerdoce et de la passion de Jésus-Christ prédits aux cent neuvième et vingt-unième psaumes.
\end{argument}

\noindent C’est ainsi que dans cet autre psaume où le sacerdoce de Jésus-Christ est déclaré ouvertement, comme ici sa royauté, ces paroles pouvaient sembler obscures : « Le Seigneur a dit à mon Seigneur : Asseyez-vous à ma droite, jusqu’à ce que j’abatte vos ennemis sous vos pieds. » En effet, nous ne voyons pas Jésus-Christ assis à la droite de Dieu le père, nous le croyons ; ni ses ennemis abattus sous ses pieds, cela ne se verra qu’à la fin dumonde. Mais lorsque le Psalmiste chante : « Le Seigneur fera sortir de Sion le sceptre de votre empire, et vous régnerez souverainement au milieu de vos ennemis » ; cela est si clair qu’il faudrait être aussi impudent qu’impie pour le nier. Nos adversaires mêmes avouent que la loi de Jésus-Christ, que nous appelons l’Évangile, et que nous reconnaissons pour le sceptre de son empire, est sortie de Sion. Quant au règne qu’il exerce au milieu de ses ennemis, ceux mêmes sur qui il l’exerce le témoignent assez par leur rage et leur jalousie. On lit un peu après : « Le Seigneur a juré, et il ne s’en dédira point, que vous serez le prêtre éternel selon l’ordre de Melchisédech » ; or, puisqu’il n’y a plus maintenant nulle part de sacerdoce ni de sacrifice selon l’ordre d’Aaron, et qu’on offre partout sous le souverain pontife, Jésus-Christ, ce qu’offrit Melchisédech quand il bénit Abraham, qui peut ne pas voir de qui ceci est dit ? Il faut donc rapporter à ces choses claires et évidentes celles qui dans le même psaume sont un peu obscures et que nous avons déjà expliquées dans les sermons que nous en avons faits au peuple. Ainsi, ce que Jésus-Christ dit dans un autre psaume où il parle de sa propre passion : « Ils ont percé mes mains et mes pieds, et ont compté mes os ; ils m’ont considéré et regardé » ; cela, dis-je, est clair, et l’on voit bien qu’il parle de son corps étendu sur la croix, pieds et mains cloués, et servant en cet état de spectacle à ses ennemis ; d’autant plus qu’il ajoute : « Ils ont partagé entre eux mes vêtements et jeté ma robe au sort » : prophétie dont l’accomplissement se trouve marqué dans le récit de l’Évangile. Les traits tout aussi clairs qui sont dans ce psaume doivent servir de lumière aux autres ; car, entre les faits qui y sont évidemment prédits, il y en a qui s’accomplissent encore tous les jours à nos yeux, comme ce qui suit : « Toutes les parties de la terre se souviendront du Seigneur, et se convertiront à lui, et toutes les autres nations du monde lui rendront leurs adorations et leurs hommages, parce que l’empire appartient au Seigneur, et il dominera sur toutes les nations. »
\subsection[{Chapitre XVIII}]{Chapitre XVIII}

\begin{argument}\noindent De la mort et de la résurrection du Sauveur prédites dans les psaumes trois, quarante, quinze et soixante-sept.
\end{argument}

\noindent Les oracles des psaumes n’ont pas non plus gardé le silence sur la résurrection du Christ. Que signifient en effet ces paroles du troisième psaume : « Je suis endormi et j’ai sommeillé, et je me suis éveillé, parce que le Seigneur m’a pris » ? Y a-t-il quelqu’un d’assez peu sensé pour croire que le Prophète nous aurait voulu apprendre comme une chose considérable qu’il s’est éveillé après s’être endormi, si ce sommeil n’était la mort, et ce réveil la résurrection de Jésus-Christ, qu’il devait prédire de la sorte ? Le psaume quarante en parle encore plus clairement, lorsqu’en la personne du médiateur, le Prophète, selon sa coutume, raconte comme passées des choses qu’il prophétise pour l’avenir, parce que, dans la prescience de Dieu, les choses à venir sont en quelque sorte arrivées, à cause de la certitude de leur accomplissement. « Mes ennemis, dit-il, ont fait des imprécations contre moi : quand mourra-t-il, et quand sa mémoire sera-t-elle abolie ? S’il venait me voir, il me parlait avec déguisement, et se fortifiait dans sa malice ; et il n’était pas plutôt sorti qu’il s’attroupait avec les autres. Tous mes ennemis formaient des complots contre moi ; ils faisaient tous le dessein de me perdre. Ils ont pris contre moi des résolutions injustes ; mais celui qui dort ne se réveillera-t-il pas ? » C’est comme s’il disait : Celui qui meurt ne ressuscitera-t-il pas ? Ce qui précède montre-assez que ses ennemis avaient conspiré sa mort, et que toute cette trame avait été conduite par celui qui entrait et sortait pour le trahir. Or, à qui ne se présente ici le traître Judas, devenu, de disciple de Jésus, le plus cruel de ses ennemis ? Pour leur faire sentir qu’ils l’immoleraient en vain, puisqu’il devait ressusciter, il leur dit : « Celui qui dort ne se réveillera-t-il pas ? » ce qui revient à ceci : Que faites-vous, pauvres insensés ? ce qui est un crime pour vous n’est qu’un sommeil pour moi. Celui qui dort ne se réveillera-t-il pas ? — Et néanmoins, pour prouver qu’un crime si énorme ne demeurerait pas impuni, il ajoute : « Celui qui vivait avec moi dans une si grande union, en qui j’avais mis ma confiance, et qui mangeait de mon pain, m’a mis le pied sur la gorge. Mais vous, Seigneur, ayez pitié de moi, et me rendez la vie, et je me vengerai d’eux. » Ne voit-on pas cette vengeance, quand on considère les Juifs expulsés de leur pays après de sanglantes défaites depuis la mort et la passion de Jésus-Christ ? Après qu’il eut été mis à mort par eux, il est ressuscité, et les a châtiés de peines temporelles, en attendant celles qu’il leur réserve pour ne s’être pas convertis, lorsqu’il jugera les vivants et les morts. Le Sauveur même montrant le traître à ses Apôtres en lui présentant un morceau de pain, fit mention de ce verset du psaume, et dit qu’il devait s’accomplir en lui : « Celui qui mangeait de mon pain m’a mis le pied sur la gorge. » Quant à ce qu’il ajoute : « En qui j’avais mis ma confiance », cela ne convient pas au chef, mais au corps ; car le Sauveur connaissait bien celui dont il avait déjà dit : « L’un de vous est le diable » ; mais il a coutume d’attribuer à sa personne ce qui appartient à ses membres, parce que la tête et le corps ne font qu’un Christ, d’où viennent ces paroles de l’Évangile : « J’ai eu faim, et vous m’avez donné à manger » ; ce que lui-même explique ainsi : « Quand vous avez, dit-il, rendu ces services aux plus petits de ceux qui sont à moi, c’est à moi que vous les avez rendus. » S’il dit qu’il avait mis sa confiance en Judas, c’est que ses disciples avaient bien espéré de celui-ci, quand il fut mis au nombre des Apôtres.\par
Quant aux Juifs, ils ne croient pas que le Christ qu’ils attendent doive mourir. Aussi ne pensent-ils pas que celui que la loi et les Prophètes ont annoncé soit pour nous ; mais ils prétendent qu’il doit leur appartenir unique-nient, et qu’il sera exempt de la mort. Ils soutiennent donc, par une folie et un aveuglement merveilleux, que les paroles que nous venons de rapporter ne doivent pas s’entendre de la mort et de la résurrection, mais du sommeil et du réveil. Mais le psaume quinze leur crie : « C’est pour cela que mon cœur est plein de joie, que ma langue se répand en « des chants d’allégresse, et que vous ne laisserez point mon âme en enfer, et que vous ne « permettrez pas que votre saint souffre aucune corruption. » Quel autre parlerait avec autant de confiance de celui qui est ressuscité letroisième jour ? Peuvent-ils l’entendre de David ? Le psaume soixante-sept crie de son côté : « Notre Dieu est un Dieu qui sauve, et le Seigneur même sortira par la mort. » Que peut-on dire de plus clair ? Le Seigneur Jésus n’est-il pas un Dieu qui sauve, lui dont le nom même signifie Sauveur ? En effet, c’est la raison qui en fut rendue quand l’ange dit à la Vierge : « Vous enfanterez un fils que vous « nommerez Jésus, parce qu’il sauvera son peuple en le délivrant de ses péchés. » Comme il a versé son sang pour obtenir la rémission de ces péchés, il n’a pas dû autrement sortir de cette vie que par la mort. C’est pour cette raison que le Prophète, après avoir dit : « Notre Dieu est un Dieu qui sauve », ajoute aussitôt : « Et le Seigneur même sortira par la mort », pour montrer que c’était en mourant qu’il devait sauver. Or, il dit avec admiration : « Et le Seigneur même », comme s’il disait : Telle est la vie des hommes mortels que le Seigneur même n’en a pu sortir que par la mort.
\subsection[{Chapitre XIX}]{Chapitre XIX}

\begin{argument}\noindent Le psaume soixante-huit montre l’obstination des juifs dans leur infidélité.
\end{argument}

\noindent Certes, les Juifs ne résisteraient pas à des témoignages si clairs confirmés par l’événement, si la prophétie du psaume soixante-huit ne s’accomplissait en eux. Après que David a introduit Jésus-Christ, qui dit, en parlant de sa passion, ce que nous voyons accompli dans l’Évangile : « Ils m’ont donné du fiel à manger, et du vinaigre à boire quand j’ai eu soif » ; il ajoute : « Qu’en récompense leur table devienne un piège et une pierre d’achoppement ; que leurs yeux « soient obscurcis, afin qu’ils ne voient point, et chargez-les de fardeaux qui les fassent marcher tout courbés », et autres malheurs qu’il ne leur souhaite pas, mais qu’il leur prédit comme s’il les leur souhaitait. Quelle merveille donc qu’ils ne voient pas des choses si évidentes, puisque leurs yeux ne sont obscurcis qu’afin qu’ils ne les voient pas ? quelle merveille qu’ils ne comprennent pas les choses du ciel, eux qui sont toujours accablés de pesants fardeaux qui les courbent contre terre ? Ces métaphores prises du corps marquent réellement les vices de l’esprit. Maisc’est assez parler des psaumes, c’est-à-dire de la prophétie de David, et il faut mettre quelques bornes à ce discours. Que ceux qui savent toutes ces choses m’excusent et ne se plaignent pas de moi, si j’ai peut-être omis d’autres témoignages qu’ils estiment encore plus forts.
\subsection[{Chapitre XX}]{Chapitre XX}

\begin{argument}\noindent Du règne et des vertus de David, et des prophéties sur Jésus-Christ qui se trouvent dans les livres de Salomon.
\end{argument}

\noindent David régna donc dans la Jérusalem terrestre, lui qui était enfant de la céleste, et à qui l’Écriture rend un témoignage de gloire, parce qu’il effaça tellement ses crimes par les humiliations d’une sainte patience qu’il est sans doute du nombre de ces pécheurs dont il dit lui-même : « Heureux ceux dont les iniquités sont pardonnées et les péchés couverts ! » À David succéda son fils Salomon, qui, comme nous l’avons dit ci-dessus, fut couronné du vivant de son père. La fin de son règne ne répondit pas aux espérances que les commencements avaient fait concevoir ; car la prospérité, qui corrompt d’ordinaire les plus sages, l’emporta sur cette haute sagesse dont le bruit s’est répandu dans tous les siècles. On reconnaît que ce prince a aussi prophétisé dans ses trois livres, que l’Église reçoit au nombre des canoniques et qui sont les Proverbes, l’Ecclésiaste et le Cantique des cantiques. Pour les deux autres, intitulés la Sagesse et l’Ecclésiastique, on a coutume de les lui attribuer, à cause de quelque ressemblance de style ; mais les doctes tombent d’accord qu’ils ne sont pas de lui. Toutefois il y a longtemps qu’ils ont autorité dans l’Église, surtout dans celle d’Occident. La passion du Sauveur est clairement prédite dans celui qu’on appelle la Sagesse. Les infâmes meurtriers de Jésus-Christ y parlent de la sorte : « Opprimons le juste, il nous est incommode et il s’oppose sans cesse à nos desseins ; il nous reproche nos péchés et publie partout nos crimes ; il se vante de connaître Dieu et il se nomine insolemment son fils ; il contrôle jusqu’à nos pensées, et sa vue même nous est à charge ; car il mène une vie toute différente de celle des autres, et sa conduite est tout extraordinaire. Il nous regarde comme des bagatelles et fuit notre manière d’agir comme la peste ; il estime heureuse la mort des gens de bien et se glorifie d’avoir Dieu pour père. Voyons donc si ce qu’il dit est vrai, et éprouvons quelle sera sa fin. S’il est vraiment fils de Dieu, Dieu le protègera et le tirera des mains de ses ennemis. Faisons-lui souffrir toutes sortes d’affronts et de tourments pour voir jusqu’où vont sa modération et sa patience. Condamnons-le à une mort ignominieuse, car nous jugerons de ses paroles par ses actions. Voilà quelles ont été leurs pensées ; mais ils se sont trompés, parce que leur malice les a aveuglés. » Quant à l’Ecclésiastique, la foi des Gentils y est prédite ainsi : « Seigneur, qui êtes le maître de tous les hommes, ayez pitié de nous, et que tous les peuples vous craignent. Étendez votre main sur les nations étrangères, afin qu’elles reconnaissent votre personne et que vous soyez glorieux en elles comme vous l’êtes en nous, et qu’elles apprennent avec nous qu’il n’y a point d’autre Dieu que vous, Seigneur. » Cette prophétie conçue en forme de souhait, nous ta voyons accomplie par Jésus-Christ ; mais comme ces Écritures ne sont pas canoniques parmi les Juifs, elles ont moins de force contre les opiniâtres.\par
Pour les autres trois livres, qui, certainement, sont de Salomon, et que les Juifs reconnaissent pour canoniques, il serait trop long et très pénible de montrer comment tout ce qui s’y trouve se rapporte à Jésus-Christ et à son Église. Toutefois ce discours des impies dans les Proverbes : « Mettons le juste au tombeau et dévorons-le tout vivant ; abolissons-en la mémoire sur la face de la terre, emparons-nous de ce qu’il possède de plus précieux » ; ce discours, dis-je, n’est pas si obscur qu’on ne le puisse aisément entendre de Jésus-Christ et de l’Église, qui est son plus précieux héritage. Notre-Seigneur lui-même, dans la parabole des mauvais vignerons, leur fait tenir un discours semblable, quand, apercevant le fils du père de famille : « Voici, disent-ils, l’héritier ; allons, tuons-le, et nous serons maîtres de son héritage. » Tous ceux qui savent que Jésus-Christ est la Sagesse de Dieu n’entendent aussi que de lui et de son Église cet autre endroit des Proverbes que nous avons touché plus haut, lorsque nous parlions de la femme stérile qui aengendré sept enfants : « La Sagesse, dit Salomon, s’est bâti une maison, et l’a appuyée sur sep colonnes. Elle a immolé ses victimes, mêlé son vin dans une coupe et dressé sa table ; elle a envoyé ses serviteurs pour convier hautement à boire du vin de sa coupe, disant : « Que celui qui n’est pas sage vienne à moi ; et à ceux qui manquent de sens, elle a parlé ainsi : Venez, mangez de mes pains, et buvez le vin que je vous ai préparé. » Ces paroles nous font connaître clairement que la sagesse de Dieu, c’est-à-dire le Verbe coéternel au Père, s’est bâti une maison dans le sein d’une vierge en y prenant un corps, qu’il s’est uni l’Église comme les membres à la tête, qu’il a immolé les martyrs comme des victimes, qu’il a couvert une table de pain et de vin, où se voit même le sacerdoce selon l’ordre de Melchisédech, enfin, qu’il y a invité les fous et les insensés, parce que, comme dit l’Apôtre : « Dieu a choisi les faibles selon le monde pour confondre les puissants. » Néanmoins, c’est à ces faibles que la Sagesse a dit ensuite : « Quittez votre folie afin de vivre, et cherchez la sagesse, afin d’acquérir la vie. » Or, avoir place à sa table, c’est commencer d’avoir la vie. Que peuvent signifier de mieux ces autres paroles de l’Ecclésiaste : « L’homme n’a d’autre bien que ce qu’il boit et mange » ? qu’est-ce, dis-je, que ces paroles peuvent signifier, sinon la participation à cette table, où le souverain prêtre et médiateur du Nouveau Testament nous donne son corps et son sang selon l’ordre de Melchisédech, et ce sacrifice a succédé à tous les autres de l’Ancien Testament, qui n’étaient que des ombres et des figures de celui-ci ? Aussi reconnaissons-nous la voix de ce même médiateur dans la prophétie du psaume trente-neuf : « Vous n’avez point voulu de victime ni d’offrande, mais vous m’avez disposé un corps », parce que, pour tout sacrifice et oblation, son corps est offert et servi à ceux qui y participent. Que l’Ecclésiaste n’entende pas parler de viandes charnelles dans son invitation perpétuelle à boire et à manger, cette parole le prouve clairement : « Il vaut mieux aller dans une maison de deuil que dans celle où l’on fait bonne chère » ; et un peu après : « Les sages aiment à aller dans une maison de deuil, etles fous dans une maison de festins et de débauches. » Mais il vaut mieux rapporter ici de ce livre ce qui regarde les deux cités, celle du diable et celle de Jésus-Christ, et les rois de l’une et de l’autre : « Malheur à vous, terre, dont le roi est jeune et dont les princes mangent dès le matin ! Mais bénie soyez-vous, terre, dont le roi est fils des libres, et dont les princes mangent dans le temps convenable, sans impatience et sans confusion. » Ce jeune roi est le diable, que Salomon appelle ainsi à cause de sa folie, de son orgueil, de sa témérité, de son insolence, et des autres vices auxquels les jeunes gens sont sujets. Jésus-Christ, au contraire, est fils des libres, c’est-à-dire des saints patriarches appartenant à la cité libre dont il est issu selon la chair. Les princes de cette cité qui mangent dès le matin, c’est-à-dire avant le temps, désignent ceux qui se hâtent de goûter la fausse félicité de ce monde, sans vouloir attendre celle de l’autre, qui est la seule véritable, au lieu que les princes de la cité de Jésus-Christ attendent avec patience le temps d’une félicité qui ne trompe point. C’est ce qu’il veut dire par ces paroles, « sans impatience et sans confusion », parce qu’ils ne se repaissent point d’une vaine espérance, suivant cette parole de l’Apôtre : « L’espérance ne confond point », et cette autre du psaume : « Tous ceux qui vous attendent avec patience ne seront point confondus. » Quant au Cantique des cantiques, c’est une réjouissance spirituelle des saintes âmes aux noces du roi et de la reine de la Cité céleste, c’est-à-dire de Jésus-Christ et de l’Église mais cette joie est cachée sous le voile de l’allégorie, afin qu’on ait plus d’envie de la connaître et plus de plaisir à la découvrir, et d’y voir cet époux à qui on dit au même cantique : « Ceux qui sont justes nous aiment », et cette épouse à qui l’on dit aussi : « La charité fait vos délices. » Nous passons sous silence plusieurs autres choses pour ne pas excéder les bornes de ces, ouvrage.
\subsection[{Chapitre XXI}]{Chapitre XXI}

\begin{argument}\noindent Des rois de Juda et d’Israël après Salomon.
\end{argument}

\noindent Peu de paroles ou d’actions des autres rois qui viennent après Salomon, soit dans Juda,soit dans Israël, peuvent se rapporter à Jésus-Christ et à son Église. Je dis dans Juda ou dans Israël, parce que ce furent les noms que portèrent ces deux parties du peuple, depuis que Dieu l’eut divisé pour le crime de Salomon sous son fils Roboam qui lui succéda. Les dix tribus dont Jéroboam, esclave de Salomon, fut établi roi, et dont Samarie était la capitale, retinrent le nom d’Israël, qui était celui de tout le peuple. Les deux autres tribus, Juda et Benjamin, qui étaient demeurées à Roboam en considération de David dont Dieu ne voulait pas entièrement détruire le royaume, et qui avaient Jérusalem pour capitale, s’appelèrent le royaume de Juda, parce que Juda était la tribu d’où David était issu. La tribu de Benjamin, dont était sorti Saül, prédécesseur de David, faisait aussi partie du royaume de Juda, qui s’appelait ainsi pour se distinguer du royaume d’Israël qui comprenait dix tribus. Celle de Lévi, comme sacerdotale et consacrée au service de Dieu, ne faisait partie ni de l’un ni de l’autre royaume, et était comptée pour la treizième. Or, ce nombre impair des tribus Venait de ce que, des douze enfants de Jacob qui en avaient établi chacun une, Joseph en avait fondé deux, Éphraïm et Manassé. Toutefois, on peut dire que la tribu de Lévi appartenait plutôt au royaume de Juda, à cause du temple de Jérusalem où elle exerçait son ministère. Après ce partage du peuple, Roboam, fils de Salomon, fut le premier roi de Juda, et établit le siège de son empire à Jérusalem ; et Jéroboam, son serviteur, fut le premier roi d’Israël, et fixa sa résidence à Samarie. Comme Roboam voulait faire la guerre à Israël sous prétexte de rejoindre à son empire cette partie que la violence d’un usurpateur avait démembrée, Dieu l’en empêcha et lui fit dire par son prophète que lui-même avait conduit tout cela ; ce qui montra que ni Israël ni Jéroboam n’étaient coupables de cette division, mais qu’elle était arrivée par la seule volonté de Dieu, qui avait ainsi vengé le crime de Salomon. Lors donc que les deux partis eurent reconnu que c’était un coup du ciel, ils demeurèrent en paix ; d’autant plus que ce n’était qu’une division de royaume, et non pas de religion.
\subsection[{Chapitre XXII}]{Chapitre XXII}

\begin{argument}\noindent Idolâtrie de jéroboam.
\end{argument}

\noindent Mais Jéroboam, roi d’Israël, assez malheureux pour se défier de la bonté de Dieu, bien qu’il l’eût éprouvé fidèle et reçu de sa main la couronne qu’il lui avait promise, appréhenda que Roboam ne séduisît ses sujets, lorsqu’ils iraient au temple de Jérusalem ; où tout le peuple juif était obligé par la loi de se rendre tous les ans pour sacrifier, et que les siens ne se remissent sous l’obéissance de la lignée royale de David. Pour empêcher cela, il introduisit l’idolâtrie dans son royaume et fut cause que son peuple sacrifia aux idoles avec lui. Toutefois, Dieu ne laissa pas de reprendre par ses Prophètes, non seulement ce prince, mais ses successeurs héritiers de son impiété, et tout le peuple. Parmi ces prophètes s’élevèrent Élie et Élisée, qui firent beaucoup de miracles ; et comme Eue disait à Dieu : « Seigneur, ils ont égorgé vos Prophètes, ils ont renversé vos autels, je suis resté seul, et ils me cherchent pour me faire mourir » ; il lui fut répondu qu’il y avait encore sept mille hommes qui n’avaient point plié le genou devant Baal.
\subsection[{Chapitre XXIII}]{Chapitre XXIII}

\begin{argument}\noindent De la captivité de Babylone et du retour des juifs.
\end{argument}

\noindent Le royaume de Juda, dont Jérusalem était la capitale, ne manqua pas non plus de prophètes, qui parurent de temps en temps, selon qu’il plaisait à Dieu de les envoyer, ou pour annoncer ce qui était nécessaire, ou pour reprendre les crimes et recommander la justice. Là se trouvèrent aussi des rois, quoiqu’en moins grand nombre que dans Israël, qui commirent contre Dieu d’énormes péchés qui attirèrent le courroux du ciel sur eux et sur leur peuple qui les imitait ; mais en récompense il y en eut d’autres d’une vertu signalée : au lieu que tous les rois d’Israël ont été méchants, les uns plus, les autres moins. L’un et l’autre parti éprouvait donc diversement la bonne ou la mauvaise fortune, ainsi que la divine Providence t’ordonnait ou le permettait ; et ils étaient affligés non seulement de guerres étrangères, mais de discordes civiles, où l’on voyait éclater tantôt la justice et tantôtla miséricorde de Dieu, jusqu’à ce que sa colère, s’allumant de plus en plus, toute cette nation fût entièrement vaincue par les Chaldéens, et emmenée captive en Assyrie, d’abord le peuple d’Israël, et ensuite celui de Juda, après la ruine de Jérusalem et de son temple fameux. Ils demeurèrent dans cette captivité l’espace de soixante-dix années ; après, ils furent renvoyés dans leur pays, où ils rebâtirent le temple ; et bien que plusieurs d’entre eux demeurassent en des régions étrangères et reculées, ils ne furent plus depuis divisés en deux partis, mais ils n’eurent qu’un roi qui résidait à Jérusalem ; et tous les Juifs, quelque éloignés qu’ils fussent, se rendaient au temple à un certain temps de l’année. Mais ils ne manquèrent pas non plus alors d’ennemis qui leur firent la guerre ; et quand le Messie vint au monde, il les trouva déjà tributaires des Romains.
\subsection[{Chapitre XXIV}]{Chapitre XXIV}

\begin{argument}\noindent Des derniers prophètes des juifs.
\end{argument}

\noindent Tout le temps qui s’écoula depuis leur retour jusqu’à l’avènement du Sauveur, c’est-à-dire depuis Malachie, Aggée, Zacharie et Esdras, ils n’eurent point de prophètes parmi eux. Zacharie, père de saint Jean-Baptiste, et Élisabeth, sa femme, prophétisèrent au temps de la naissance du Messie avec Siméon et Anne. On peut y joindre saint Jean-Baptiste, qui fut le dernier des Prophètes, et qui montra Jésus-Christ, s’il ne le prédit ; ce qui a fait dire à Notre-Seigneur que « la loi et les Prophètes ont duré jusqu’à Jean. »\par
L’Évangile nous apprend aussi que la Vierge même prophétisa avec saint Jean ; mais les Juifs infidèles ne reçoivent point ces prophéties, quoique reçues par tous ceux d’entre eux qui ont embrassé notre religion. C’est véritablement à cette époque qu’Israël a été divisé en deux, de cette division immuable prédite par Samuel et Saut. Pour Malachie, Aggée, Zacharie et Esdras, tous les Juifs les mettent au nombre des livres canoniques ; et il ne sera pas hors de propos d’en rapporter quelques témoignages qui concernent Jésus-Christ et son Église. Mais cela se fera plus commodément au livre suivant, et il est temps de mettre un terme à celui-ci.
\section[{Livre dix-huitième. Histoire des deux Cités}]{Livre dix-huitième. \\
Histoire des deux Cités}\renewcommand{\leftmark}{Livre dix-huitième. \\
Histoire des deux Cités}

\subsection[{Chapitre premier}]{Chapitre premier}

\begin{argument}\noindent Récapitulation de ce qui a été traité dans les livres précédents.
\end{argument}

\noindent J’ai promis de parler de la naissance, du progrès et de la fin des deux cités, après avoir réfuté, dans les dix premiers livres de cet ouvrage, les ennemis de la Cité de Dieu, qui préfèrent leurs dieux à Jésus-Christ, et dont l’âme dévorée d’une pernicieuse envie a conçu contre les chrétiens la plus implacable inimitié. J’ai fait voir en quatre livres, depuis le onzième jusqu’au quatorzième, la naissance des deux cités. Le quinzième en a montré le progrès, depuis le premier homme jusqu’au déluge, et depuis le déluge jusqu’à Abraham. Mais depuis Abraham jusqu’aux rois des Juifs, période exposée dans le seizième livre, et depuis ces rois jusqu’à la naissance du Sauveur, où nous conduit le dix-septième, il semble que la seule Cité de Dieu se soit montrée dans notre récit, quoique celle du monde n’ait pas laissé de continuer son cours. J’ai procédé de la sorte, afin que le progrès de la Cité de Dieu parût plus distinctement, depuis que les promesses de l’avènement du Messie ont commencé à être plus claires ; et toutefois il est vrai de dire que, jusqu’à la publication du Nouveau Testament, cette cité ne s’est montrée qu’à travers des ombres. Il faut donc reprendre maintenant le cours de la cité du monde depuis Abraham, afin qu’on puisse comparer ensemble le développement des deux cités.
\subsection[{Chapitre II}]{Chapitre II}

\begin{argument}\noindent Quels ont été les rois de la cité de la terre pendant que se développait la suite des saints depuis Abraham.
\end{argument}

\noindent La société des hommes répandue par toute la terre, dans les lieux et les climats les plus différents, ne cherchant qu’à satisfaire ses besoinsou ses convoitises, et l’objet de ses désirs n’étant capable de suffire ni à tous, ni à personne, parce que ce n’est pas le bien véritable, il arrive d’ordinaire qu’elle se divise contre elle-même et que le plus faible est opprimé par le plus fort. Accablé par le vainqueur, le vaincu achète la paix aux dépens de l’empire, et même de la liberté, et c’est un rare et admirable spectacle que celui d’un peuple qui aime mieux périr que de se soumettre. En effet, la nature crie en quelque sorte à l’homme qu’il vaut mieux subir le joug du vainqueur que de s’exposer aux dernières fureurs de la guerre. Et c’est ainsi que dans la suite des temps, non sans un conseil de la providence de Dieu, qui règle le sort des batailles, quelques peuples ont été les maîtres des autres. Or, entre tous les empires que les divers intérêts de la cité de la terre ont établis, il en est deux singulièrement puissants, celui des Assyriens et celui des Romains, distincts l’un de l’autre par les lieux comme par les temps. Celui des Assyriens, situé en Orient, a fleuri le premier ; et celui des Romains, qui n’est venu qu’après, s’est étendu en Occident : la fin de l’un a été le commencement de l’autre. On peut dire que les autres royaumes n’ont été que des rejetons de ceux-là.\par
Ninus, second roi des Assyriens, qui avait succédé à son père Bélus, tenait l’empire, quand Abraham naquit en Chaldée. En ce temps-là florissait aussi le petit royaume des Sicyoniens, par lequel le docte Varron commence son histoire romaine. Des rois des Sicyoniens, il descend aux Athéniens, de ceux-ci aux Latins, et des Latins aux Romains. Mais, comme je l’ai dit, tous ces empires qui ont précédé la fondation de Borne étaient peu de chose en comparaison de celui des Assyriens ; et Salluste, tout en reconnaissant que les Athéniens ont été célèbres dans la Grèce, croit pourtant que la renommée a exagéré leur puissance. « Les faits d’armes d’Athènes, dit-il, ont été grands et glorieux, je n’en disconviens pas ; mais toutefois je les crois un peu au-dessous de ce qu’on en publie. L’éloquence des historiens a beaucoup contribué à leur éclat, et la vertu de ses héros a été rehaussée de toute la grandeur de ses beaux génies. » Ajoutez à cela qu’Athènes a été l’école des lettres et de la philosophie, ce qui n’a pas peu contribué à sa gloire. Mais à ne considérer que la puissance matérielle, il n’y avait point en ce temps-là d’empire plus fort ni plus étendu que celui d’Assyrie, En effet, on dit que Ninus subjugua toute l’Asie, c’est-à-dire la moitié du monde, et porta ses conquêtes jusques aux confins de la Libye. Les Indiens furent les seuls de tous les peuples d’Orient qui demeurèrent libres de sa domination ; encore, après sa mort, furent-ils soumis par sa femme Sémiramis. Ce fut donc alors, sous le règne de Ninus, qu’Abraham naquit chez les Chaldéens ; mais, comme l’histoire des Grecs nous est bien plus connue que celle des Assyriens, ayant passé jusqu’à nous par les Latins, et, après ceux-ci, par les Romains, qui en sont descendus, j’estime qu’il ne sera pas hors de propos de rappeler à l’occasion les rois des Assyriens, afin qu’on voie comment Babylone, ainsi que l’ancienne Home, s’avance dans le cours des siècles avec la Cité de Dieu, étrangère ici-bas. Quant aux faits qui doivent nous servir à mettre en parallèle les deux cités, il vaut mieux les emprunter aux Grecs et aux Latins, parmi lesquels je comprends Rome, comme une seconde Babylone.\par
Or, à la naissance d’Abraham, Ninus était le second roi des Assyriens, et Europs le second roi des Sicyoniens ; l’un avait succédé à Bélus, et l’autre à Aegialeus. Quand Dieu promit à Abraham une postérité nombreuse, après qu’il fut sorti de Babylone, les Assyriens en étaient à leur quatrième roi, et les Sicyoniens à leur cinquième : Alors le fils de Ninus régnait chez les Assyriens après sa mère Sémiramis, qu’il tua, dit-on, parce qu’elle voulait former avec lui une union incestueuse. Quelques-uns croient qu’elle fonda Babylone, peut-être parce qu’elle la rebâtit ; car nous avons montré au seizième livre quand et comment Babylone fut fondée. Pour ce fils de Sémiramis, les uns le nomment Ninus comme son père, les autres Ninyas. Telxion tenait alors le sceptre des Sicyoniens, et son règne fut si tranquille que ses sujets, après sa mort, firent de lui un dieu et lui décernèrent des jeux et des sacrifices.
\subsection[{Chapitre III}]{Chapitre III}

\begin{argument}\noindent Sous quels rois des Assyriens et des Sicyoniens naquit Isaac, Abraham étant alors âgé de cent ans, et à quelle époque de ces mêmes empires Isaac, âgé de soixante ans, eut de Rébecca deux fils, Ésaü et Jacob.
\end{argument}

\noindent Ce fut sous le règne de Telxion que naquit Isaac, selon la promesse que Dieu en avait faite à son père Abraham, qui l’eut à l’âge de cent ans de sa femme Sarra, à qui la stérilité et le grand âge avaient ôté l’espérance d’avoir des enfants : Arrius, cinquième roi des Assyriens, régnait alors. Isaac, âgé de soixante ans, eut de sa femme Rébecca deux enfants jumeaux, Ésaü et Jacob, Abraham étant encore vivant et âgé de cent soixante ans ; mais il mourut quinze ans après, sous le règne de l’ancien Xerxès, roi des Assyriens, surnommé Baléus, et de Thuriacus ou Thurimachus, roi des Sicyoniens, tous deux septièmes souverains de leurs peuples. Le royaume des Argiens prit naissance sous les petits-fils d’Abraham, et Inachus en fut le premier roi. Il ne faut pas oublier, qu’au rapport de Varron, les Sicyoniens avaient coutume de sacrifier sur le sépulcre de Thurimachus. Sous les règnes d’Armamitres et de Leucippus, huitièmes rois des Assyriens et des Sicyoniens, et sous celui d’Inachus, premier roi des Argiens, Dieu parla à Isaac et lui promit, comme il avait fait à son père, qu’il donnerait la terre de Chanaan à sapostérité, et qu’en elle toutes les nations seraient bénies. Il promit la même chose à son fils Jacob, appelé depuis Israël, sous le règne de Bélocus, neuvième roi des Assyriens, et de Phoronée, fils d’Inachus, deuxième roi des Argiens ; car Leucippus, huitième roi des Sicyoniens, vivait encore. Ce fut sous ce Phoronée, roi d’Argos, que la Grèce commença à devenir célèbre par ses lois et ses institutions. Phegous, cadet de Phoronée, fut honoré comme un dieu après sa mort, et on lui bâtit un temple sur son tombeau. J’estime qu’on lui déféra cet honneur, parce que, dans la partie du royaume que son père lui avait laissée, il avait élevé des chapelles aux dieux, et divisé les temps par mois et par années. Surpris de ces nouveautés, les hommes encore grossiers crurent qu’il était devenu dieu après sa mort, ou le voulurent croire. On dit qu’Io, fille d’Inachus, appelée depuis Isis, fut honorée en Égypte comme une grande déesse ; d’autres pourtant la font venir d’Éthiopie en Égypte, où elle gouverna avec tant de sagesse et de justice que les Égyptiens, qui lui devaient en outre l’invention des lettres et beaucoup d’autres choses utiles, la révérèrent comme une divinité, et défendirent, sous peine de la vie, de dire qu’elle avait été une simple mortelle.
\subsection[{Chapitre IV}]{Chapitre IV}

\begin{argument}\noindent Des temps de Jacob et de son fils Joseph.
\end{argument}

\noindent Pendant que Baléus, dixième roi des Assyriens, occupait le trône sous le règne de Mes-sapas, surnommé Céphisus, neuvième roi des Sicyoniens (si toutefois ce ne sont point là deux noms différents), et sous celui d’Apis, troisième roi des Argiens, Isaac mourut âgé de cent quatre-vingts ans, et laissa ses deux jumeaux qui en avaient cent vingt. Le plus jeune des deux, Jacob, qui appartenait à la Cité de Dieu, à l’exclusion de l’aîné, avait douze fils. Joseph, l’un d’eux, ayant été vendu par ses frères du vivant d’Isaac, leur aïeul, à des marchands qui trafiquaient en Égypte, fut tiré de la prison où l’avait fait mettre sa chasteté, courageusement défendue contre la passion d’une femme adultère, et présenté à l’âge de trente ans à Pharaon, roi d’Égypte. Ceprince le combla d’honneurs et de biens, parce qu’il lui avait expliqué ses songes et prédit les sept années d’abondance, qui devaient être suivies des sept autres années de stérilité. Cc fut à la seconde de ces années stériles que Jacob vint en Égypte avec toute sa famille, âgé de cent trente ans, comme il le dit lui-même au roi Pharaon. Joseph en avait alors trente-neuf, attendu que les sept années d’abondance et les deux de stérilité s’étaient écoulées, depuis qu’il avait commencé à être en faveur.
\subsection[{Chapitre V}]{Chapitre V}

\begin{argument}\noindent D’Apis, troisième roi des Argiens, dont les égyptiens firent leur dieu Sérapis.
\end{argument}

\noindent En ce temps, Apis, roi des Argiens, qui était venu par mer en Égypte et qui y était mort, devint ce fameux Sérapis, le plus grand de tous les dieux des Égyptiens. Pourquoi ne fut-il pas nommé Apis après sa mort, mais Sérapis ? Varron en rend une raison fort claire, qui est que les Grecs appelant un cercueil {\itshape soros}, et celui d’Apis ayant été honoré avant qu’on lui eût bâti un temple, on le nomma d’abord Sorosapis ou Sorapis, et puis, en changeant une lettre, comme cela arrive souvent, Sérapis. Il fut ordonné que quiconque l’appellerait homme serait puni du dernier supplice ; et Varron dit que c’était pour signifier cette défense que les statues d’Isis et de Sérapis avaient toutes un doigt sur les lèvres. Quant à ce bœuf que l’Égypte, par une merveilleuse superstition, nourrissait si délicatement en l’honneur du dieu, comme ils l’adoraient vivant et non pas dans le cercueil, ils l’appelèrent Apis et non Sérapis. À la mort de ce bœuf, on en mettait un autre à sa place, marqué pareillement de certaines taches blanches, où le peuple voyait une grande merveille et un don de la divinité ; mais, en vérité, il n’était pas difficile aux démons, qui prenaient plaisir à tromper ces peuples, de représenter à une vache pleine un taureau pareil à Apis, comme fit Jacob, qui obtint des chèvres et des brebis de la même couleur que les baguettes bigarrées qu’il mettait devant les yeux de leurs mères. Ce que les hommes font avec des couleurs véritables, lesdémons le peuvent faire très aisément par le moyen de couleurs fausses et fantastiques.
\subsection[{Chapitre VI}]{Chapitre VI}

\begin{argument}\noindent Sous quels rois argiens et assyriens Jacob mourut en Égypte.
\end{argument}

\noindent Apis, roi des Argiens et non des Égyptiens, mourut donc en Égypte, et son fils Argus lui succéda. C’est de lui que les Argiens prirent leur nom, car on ne les appelait pas ainsi auparavant. Sous son règne, Eratus gouvernant les Sicyoniens, et Baléus, qui vivait encore, les Assyriens, Jacob mourut en Égypte, âgé de cent quarante-sept ans, après avoir béni ses enfants et les enfants de son fils Joseph, et annoncé clairement le Messie, lorsque, bénissant Juda, il dit : « Il ne manquera ni prince de la race de Juda, ni chef de son sang, jusqu’au jour où ce qui lui a été promis sera accompli ; et il sera l’attente des nations. » Sous le règne d’Argus, la Grèce commença à cultiver son sol et à semer du blé. Argus, après sa mort, fut adoré comme un dieu, et on lui décerna des temples et des sacrifices : honneur suprême déjà rendu avant lui sous son propre règne à un particulier nommé Homogyrus, qui fut tué d’un coup de foudre, et qui le premier avait attelé des bœufs à la charrue,
\subsection[{Chapitre VII}]{Chapitre VII}

\begin{argument}\noindent Sous quels rois mourut Joseph en Égypte.
\end{argument}

\noindent Sous le règne de Mamitus, douzième roi des Assyriens, et de Plemnaeus, le onzième des Sicyoniens, temps où Argus était encore roi des Argiens, Joseph mourut en Égypte, âgé de cent dix ans. Après sa mort, le peuple de Dieu, qui s’accroissait d’une façon prodigieuse, demeura en Égypte l’espace de cent quarante-cinq ans, assez tranquillement d’abord, tant que vécurent ceux qui avaient vu Joseph ; mais depuis, le grand nombre des Hébreux étant devenu suspect aux Égyptiens, ils persécutèrent cruellement cette race et lui firent souffrir mille maux ; ce qui n’en diminua pas la fécondité. Pendant ce temps, nul changement de règne en Assyrie ni en Grèce.
\subsection[{Chapitre VIII}]{Chapitre VIII}

\begin{argument}\noindent Des rois sous lesquels naquit Moïse, et des dieux dont le culte commença à s’introduire en ce même temps.
\end{argument}

\noindent Ainsi, au temps de Saphrus, quatorzième roi des Assyriens, et d’Orthopolis, le douzième des Sicyoniens, lorsque les Argiens comptaient Criasus pour leur cinquième roi, naquit en Égypte ce Moïse qui délivra le peuple de Dieu de la captivité sous laquelle il gémissait et où Dieu le laissait languir pour lui faire désirer l’assistance de son Créateur. Quelques-uns croient que Prométhée vivait alors ; et comme il faisait profession de sagesse, on dit qu’il avait formé des hommes avec de l’argile. On ne sait pas néanmoins quels étaient les sages de son temps. Son frère Atlas fut, dit-on, un grand astrologue ; ce qui a donné lieu de dire qu’il portait le ciel sur ses épaules, quoiqu’il existe une haute montagne du nom d’Atlas, d’où ce conte a bien pu tirer son origine. En ce temps-là beaucoup de fables commencèrent à avoir cours dans la Grèce ; et sous le règne de Cécrops, roi des Athéniens, la superstition des Grecs mit plusieurs morts au rang des dieux : Mélantomice, femme de Criasus, et Phorbas, leur fils, sixième roi des Argiens, furent de ce nombre, aussi bien que Jasus et Sthénélas, Sthénéléus ou Sthénélus (car les historiens ne s’accordent pas sur son nom), l’un fils de Triopas, septième roi, et l’autre de Jasus, neuvième roi des Argiens. Alors vivait Mercure, petit-fils d’Atlas par Maïa, suivant le témoignage de presque tous les historiens. Il apprit aux hommes beaucoup d’arts utiles à la vie, ce qui fut cause qu’ils en firent un Dieu après sa mort. Vers le même temps, mais après lui, vint Hercule, que quelques-uns néanmoins mettent auparavant, en quoi je pense qu’ils se trompent. Mais quoi qu’il en soit de l’époque de ces deux personnages, les plus graves historiens tombent d’accord que tous deux furent des hommes qui reçurent les honneurs divins pour avoir trouvé quantité de choses propres au soulagement de la condition humaine. Pour Minerve, elle est bien plus ancienne qu’eux, puisqu’on la vit, dit-on, jeune fille du temps d’Ogygès auprès du lac Triton, d’où elle fut surnomméeTritonienne. On lui doit beaucoup d’inventions rares et utiles, et l’on inclina d’autant plus à la croire une déesse que son origine n’était pas connue. Car ce que l’on raconte, qu’elle sortit de la tête de Jupiter, est plutôt une fiction de poète qu’une vérité historique. Toutefois, les historiens ne sont pas d’accord sur l’époque où vivait Ogygès, qui a donné son nom à un grand déluge, non pas à celui qui submergea tout le genre humain, à l’exception du petit nombre sauvé dans l’arche, car l’histoire grecque ni l’histoire latine n’ont point connu celui-là, mais à un autre, plus grand que celui de Deucalion. Varron n’a rien trouvé de plus ancien dans l’histoire que le déluge d’Ogygès, et c’est à ce temps qu’il commence son livre des {\itshape Antiquités romaines}. Mais nos chronologistes, Eusèbe, et Jérôme après lui, qui sans doute ici s’appuient sur le témoignage d’historiens antérieurs, reculent le déluge d’Ogygès de plus de trois cents ans, jusque sous Phoronée, second roi des Argiens. Quoi qu’il en soit, Minerve était déjà adorée comme une déesse du temps de Cécrops, roi des Athéniens, sous le règne duquel Athènes fut fondée ou rebâtie.
\subsection[{Chapitre IX}]{Chapitre IX}

\begin{argument}\noindent Origine du nom de la ville d’Athènes, fondée ou rebâtie sous Cécrops.
\end{argument}

\noindent Voici, selon Varron, la raison pour laquelle cette ville fut nommée Athènes, qui est un nom tiré de celui de Minerve, que les Grecs appellent {\itshape Athena}. Un olivier étant tout à coup sorti de terre, en même temps qu’une source d’eau jaillissait en un autre endroit, ces prodiges étonnèrent le roi, qui députa vers Apollon de Delphes pour savoir ce que cela signifiait et ce qu’il fallait faire. L’oracle répondit que l’olivier signifiait Minerve, et l’eau Neptune, et que c’était aux habitants de voir à laquelle de ces deux divinités ils emprunteraient son nom pour le donner à leur ville. Là-dessus Cécrops assemble tous les citoyens, tant hommes que femmes, car les femmes parmi eux avaient leur voix alors dans les délibérations. Quand il eut pris les suffrages, ilse trouva que tous les hommes étaient pour Neptune, et toutes les femmes pour Minerve mais comme il y avait une femme de plus, Minerve l’emporta. Alors Neptune irrité ravagea de ses flots les terres des Athéniens ; et, en effet, il n’est pas difficile aux démons de répandre telle masse d’eaux qu’il leur plaît. Pour apaiser le dieu, les femmes, à ce que dit le même auteur, furent frappées de trois sortes de peines : la première, que désormais elles n’auraient plus voix dans les assemblées ; la seconde, qu’aucun de leurs enfants ne porterait leur nom ; et la troisième enfin, qu’on ne les appellerait point Athéniennes. Ainsi, cette cité, mère et nourrice des arts libéraux et de tant d’illustres philosophes, à qui la Grèce n’a jamais rien eu de comparable, fut appelée Athènes par un jeu des démons qui se moquèrent de sa crédulité, obligée de punir le vainqueur pour calmer le vaincu et redoutant plus les eaux de Neptune que les armes de Minerve. Cependant Minerve, qui était demeurée victorieuse, fut vaincue dans ces femmes ainsi châtiées, et elle n’eut pas seulement le pouvoir de faire porter son nom à celles qui lui avaient donné la victoire. On voit assez tout ce que je pourrais dire là-dessus, s’il ne valait mieux passer à d’autres objets.
\subsection[{Chapitre X}]{Chapitre X}

\begin{argument}\noindent Origine du nom de l’Aréopage selon Varron, et déluge de Deucalion sous Cécrops.
\end{argument}

\noindent Cependant Varron refuse d’ajouter foi aux fables qui sont au désavantage des dieux, de peur d’adopter quelque sentiment indigne de leur majesté. C’est pour cela qu’il ne veut pas que l’Aréopage, où l’apôtre saint Paul discuta avec les Athéniens et dont les juges sont appelés Aréopagites, ait été ainsi nommé de ce que Mars, que les Grecs appellent Arès, accusé d’homicide devant douze dieux qui le jugèrent au lieu où le célèbre tribunal est aujourd’hui placé, fut renvoyé absous, ayant eu six voix pour lui, et le partage alors étant toujours favorable à l’accusé. Il rejette donc cette opinion commune et tâche d’établir une autre origine qu’il va déterrer dans de vieilles histoires surannées, sous prétexte qu’il est injurieux aux divinités de leur attribuer des querelles ou des procès ; et il soutient que cette histoire de Mars n’est pas moinsfabuleuse que ce qu’on dit de ces trois déesses, Junon, Minerve et Vénus, qui disputèrent devant Pâris le prix de la beauté, et ainsi de tous les mensonges semblables qui se débitent sur la scène au détriment de la majesté des dieux. Mais ce même Varron, qui se montre si scrupuleux à cet égard, ayant à donner une raison historique et non fabuleuse du nom d’Athènes, nous raconte qu’il survint un si grand différend entre Neptune et Minerve au sujet de ce nom, qu’Apollon n’osa s’en rendre l’arbitre, mais en remit la décision au jugement des hommes, à l’exemple de Jupiter, qui renvoya les trois déesses à la décision de Pâris ; et Varron ajoute que Minerve l’emporta par le nombre des suffrages, mais qu’elle fut vaincue en la personne de celles qui l’avaient fait vaincre, et n’eut pas le pouvoir de leur faire porter son nom ! En ce temps-là, sous le règne de Cranaüs, successeur de Cécrops, selon Varron, ou, selon Eusèbe et Jérôme, sous celui de Cécrops même, arriva le déluge de Deucalion, appelé ainsi parce que le pays où Deucalion commandait fut principalement inondé ; mais ce déluge ne s’étendit point jusqu’en Égypte, ni jusqu’aux lieux circonvoisins.
\subsection[{Chapitre XI}]{Chapitre XI}

\begin{argument}\noindent Sous quels rois arrivèrent la sortie d’Égypte dirigée par Moïse et la mort de Jésus Navé, son successeur.
\end{argument}

\noindent Moïse tira d’Égypte le peuple de Dieu sur la fin du règne de Cécrops, roi d’Athènes, Ascatadès étant roi des Assyriens, Marathus des Sicyoniens, et Triopas des Argiens. Il donna ensuite aux Israélites la loi qu’il avait reçue de Dieu sur le mont Sinaï et qui s’appelle l’Ancien Testament, parce qu’il ne contient que des promesses temporelles, au lieu que Jésus-Christ promet le royaume des cieux dans le Nouveau. Il était nécessaire de garder cet ordre qui, selon l’Apôtre, s’observe en tout homme qui s’avance dans la vertu, et qui consiste en ce que la partie corporelle précède la spirituelle : « Le premier homme, dit-il avec raison, le premier homme est le terrestre formé de la terre, et le second « homme est le céleste descendu du ciel. »\par
Or, Moïse gouverna le peuple dans le désert l’espace de quarante années, et mourut âgéde cent vingt ans, après avoir aussi prophétisé le Messie par les figures des observations légales, par le tabernacle, le sacerdoce, les sacrifices et autres cérémonies mystérieuses. À Moïse succéda Jésus, fils de Navé, qui établit le peuple dans la terre promise, après avoir exterminé, par l’ordre de Dieu, les peuples qui habitaient ces contrées. Il mourut après vingt-sept années de commandement, sous les règnes d’Amnyntas, dix-huitième roi des Assyriens, de Corax, le seizième des Sicyoniens, de Danaüs, le dixième des Argiens, et d’Érichthon, le quatrième des Athéniens.
\subsection[{Chapitre XII}]{Chapitre XII}

\begin{argument}\noindent Du culte des faux dieux établi par les rois de la Grèce, depuis l’époque de la sortie d’Égypte jusqu’à la mort de Jésus Navé.
\end{argument}

\noindent Durant ce temps, c’est-à-dire depuis que le peuple juif fut sorti d’Égypte jusqu’à la mort de Jésus Navé, les rois de la Grèce instituèrent en l’honneur des faux dieux plusieurs solennités qui rappelaient le souvenir du déluge et de ces temps misérables où les hommes tour à tour gravissaient le sommet des montagnes et descendaient dans les plaines. Telle est l’explication que l’on donne de ces courses fameuses des prêtres Luperques, montant et descendant tour à tour la Voie sacrée. C’est en ce temps que Dionysius, qu’on nomme aussi Liber, se trouvant dans l’Attique, apprit, dit-on, à son hôte l’art de planter la vigne, et fut honoré comme un dieu après sa mort. Alors aussi des jeux de musique furent dédiés à Apollon de Delphes, suivant son ordre, pour l’apaiser, parce qu’on attribuait la stérilité de la Grèce à ce qu’on n’avait pas garanti son temple du feu, lorsque Danaüs fit irruption dans leur pays. Érichthon fut le premier qui institua en Attique des jeux en son honneur et en l’honneur de Minerve. Le prix en était une branche d’olivier, parce que Minerve avait enseigné la culture de cet arbre, comme Bacchus celle de la vigne. Xanthus, roi de Crète, que d’autres nomment autrement, enleva en ce temps-là Europe, dont il eut Rhadamanthe, Sarpédon et Minos, que l’on fait communément fils de Jupiter. Mais les adorateurs de ces dieux prennent ce que nous avons rapporté du roi de Crète pour historique, et ce qu’on dit de Jupiter et ce qu’on en représente sur les théâtres comme fabuleux, de sorte qu’il ne faudrait voir dans ces aventures que des fictions dont on se sert pour apaiser les dieux, qui se plaisent à la représentation de leurs faux crimes. C’était aussi alors qu’Hercule florissait à Tirynthe, mais un autre Hercule que celui dont nous avons parlé plus haut. Les plus savants dans l’histoire comptent en effet plusieurs Bacchus et plusieurs Hercules. Cet Hercule dont nous parlons, et à qui l’on attribue les douze fameux travaux, n’est pas celui qui tua Antée, mais celui qui se brûla lui-même sur le mont Œta, lorsque cette vertu, qui lui avait fait dompter tant de monstres, succomba sous l’effort d’une légère douleur. C’est vers ce temps que le roi, ou plutôt le tyran Busiris, immolait ses hôtes à ses dieux. Il était fils de Neptune, qui l’avait eu de Lybia, fille d’Epaphus ; mais je veux que ce soit une fable inventée pour apaiser les dieux, et que Neptune n’ait pas cette séduction à se reprocher. On dit qu’Érichthon, roi d’Athènes, était fils de Vulcain et de Minerve. Toutefois, comme on veut que Minerve soit vierge, on raconte que Vulcain, la voulant posséder en dépit d’elle, répandit sa semence sur la terre, d’où naquit un enfant qui, à cause de cela, fut nommé Érichthon. Il est vrai que les plus savants rejettent ce récit et expliquent autrement la naissance d’Érichthon. Ils disent que dans le temple de Vulcain et de Minerve (car il n’y en avait qu’un pour tous deux à Athènes), on trouva un enfant entouré d’un serpent, et que, ne sachant à qui il était, on l’attribua à Vulcain et à Minerve. Sur quoi je trouve que la fable rend mieux raison de la chose que l’histoire. Mais que nous importe ? l’histoire est pour l’instruction des hommes religieux, et la fable pour le plaisir des démons impurs, que toutefois ces hommes religieux adorent comme des divinités. Aussi, encore qu’ils ne veuillent pas tout avouer de leurs dieux, ils ne les justifient pas tout à fait, puisque c’est par leur ordre qu’ils célèbrent des jeux où on représente leurs crimes, et que ces dieux,disent-ils, s’apaisent par de telles infamies. Les crimes ont beau être faux, les dieux païens n’en sont guère moins coupables, puisque prendre plaisir à des crimes faux est un crime très véritable.
\subsection[{Chapitre XIII}]{Chapitre XIII}

\begin{argument}\noindent Des superstitions répandues parmi les Gentils à l’époque des Juges.
\end{argument}

\noindent Après la mort de Jésus Navé, le peuple de Dieu fut gouverné par des Juges, et éprouva tour à tour la bonne et la mauvaise fortune, selon qu’il était digne de grâces ou de châtiments. Il faut rapporter à cette époque l’invention d’un grand nombre de fables célèbres : Triptolème, porté sur des serpents ailés et distribuant du blé, par ordre de Cérès, dans les pays affligés de la famine ; le Minotaure et ce labyrinthe inextricable d’où il était impossible de sortir ; les Centaures, moitié hommes et moitié chevaux ; Cerbère, chien à trois têtes, qui gardait l’entrée des enfers ; Phryxus et Hellé, sa sœur, s’envolant sur un bélier ; la Gorgone, à la chevelure de serpents, qui changeait en pierres ceux qui la regardaient ; Bellérophon, porté sur un cheval ailé ; Amphion, qui attirait les arbres et les rochers au son de sa lyre ; Dédale et son fils, qui se firent des ailes pour traverser les airs ; Œdipe, qui résolut l’énigme de Sphinx, monstre à quatre pieds et à visage humain, et le força de se jeter dans son propre abîme ; Antée enfin, qu’Hercule étouffa en le soulevant de terre, parce que ce fils de la terre se relevait plus fort toutes les fois qu’il la touchait. Ces fables et autres semblables, jusqu’à la guerre de Troie, où Varron finit son second livre des {\itshape Antiquités romaines}, ont été inventées à l’occasion de quelques événements véritables, et ne sont point honteuses aux dieux. Mais quant à ceux qui ont imaginé que Jupiter enleva Ganymède (crime qui fut commis en effet par le roi Tantalus) et qu’il abusa de Danaé en se changeant en pluie d’or, par où l’on a voulu figurer la séduction d’une femme intéressée, il faut qu’ils aient eu bien mauvaise opinion des hommes pour les avoir crus capables d’ajouter foi à ces rêveries. Cependant ceux qui honorent le plus Jupiter sont les premiers à les soutenir ; et, bien loin de s’indigner contre des inventions pareilles, ils appréhenderaient la colère des dieux, si l’on ne les représentait sur le théâtre. En ce même temps, Latone accoucha d’Apollon, non de celui dont on consultait les oracles, mais d’un autre qui fut berger d’Admète du temps d’Hercule, et qui néanmoins a tellement passé pour un dieu que presque tout le monde le confond avec l’autre. Ce fut aussi alors que Bacchus fil la guerre aux Indiens, accompagné d’une troupe de femmes appelées Bacchantes, plus célèbres par leur fureur que par leur courage. Quelques-uns écrivent qu’il fut vaincu et fait prisonnier ; et d’autres, qu’il fut même tué dans le combat par Persée, sans oublier le lieu où il fut enseveli ; et toutefois les démons ont fait instituer des fêtes en son honneur, qu’on appelle Bacchanales, dont le sénat a eu tant de honte après plusieurs siècles, qu’il les a bannies de Rome. Persée et sa femme Andromède vivaient vers le même temps, et, après leur mort, ils furent si constamment réputés pour dieux qu’on ne rougit point d’appeler quelques étoiles de leur nom.
\subsection[{Chapitre XIV}]{Chapitre XIV}

\begin{argument}\noindent Des poètes théologiens.
\end{argument}

\noindent À la même époque, il y eut des poètes qu’on appelait aussi théologiens, parce qu’ils faisaient des vers en l’honneur des dieux ; mais quels dieux ? des dieux qui, tout grands hommes qu’ils pussent avoir été, n’en étaient pas moins des hommes, ou qui même n’étaient autre chose que les éléments du monde, ouvrage du seul vrai Dieu ; ou enfin, si c’étaient des anges, ils devaient ce haut rang moins à leurs mérites qu’à la volonté du Créateur. Que si, parmi tant de fables, ces poètes ont dit quelque chose du vrai Dieu, comme ils en adoraient d’autres avec lui, ils ne lui ont pas rendu le culte qui n’est dû qu’à lui seul ; outre qu’ils n’ont pu se défendre de déshonorer ces dieux mêmes par des contes ridicules, comme ont fait Orphée, Musée et Linus. Du moins, si ces théologiens ont adoré les dieux, ils n’ont pas été adorés comme des dieux, quoique la cité des impies fasse présider Orphée aux sacrifices infernaux. Ce fut le temps où Ino, femme du roi Athamas, se jeta dans la muer avec son fils Mélicerte, et où ils furenttous deux mis au rang des dieux, comme beaucoup d’autres hommes de ce temps-là, et entre autres Castor et Pollux. Les Grecs donnent à la mère de Mélicerte le nom de Leucothée, et les Latins celui de Matuta ; mais les uns et les autres la prennent pour une déesse.
\subsection[{Chapitre XV}]{Chapitre XV}

\begin{argument}\noindent Fin du royaume des Argiens et naissance de celui des Laurentins.
\end{argument}

\noindent Vers ce temps, le royaume des Argiens prit fin et fut transféré à Mycènes, dont Agamemnon fut roi, et celui des Laurentins commença à s’établir : ils eurent pour premier roi Picus, fils de Saturne. Debbora était alors juge des Hébreux. Cette femme fut élevée à cet honneur par un ordre exprès de Dieu, car elle était prophétesse ; mais comme ses prophéties sont obscures, il faudrait trop nous étendre pour faire voir le rapport qu’elles ont à Jésus-Christ. Les Laurentins régnaient donc déjà en Italie, et ce peuple est, après les Grecs, l’origine la plus certaine de Rome. Cependant la monarchie des Assyriens subsistait toujours, et ils comptaient Lamparès pour leur vingt-troisième roi, quand Picus fut le premier des Laurentins. C’est aux adorateurs de ces dieux à voir ce qu’ils veulent qu’ait été Saturne, père de ce Picus ; car ils disent que ce n’était pas un homme. D’autres ont écrit qu’il avait régné en Italie avant Picus, et Virgile l’a célébré dans ces vers bien connus :\par
 {\itshape « C’est lui qui rassembla ces hommes indociles errant sur les hautes montagnes ; il leur donna des lois et voulut que cette contrée s’appelât Latium, parce qu’il s’y était caché pour éviter la fureur de son fils. C’est sous son règne que l’on place l’âge d’or. »} \par
Mais qu’ils traitent ceci de fiction poétique, et qu’ils disent, s’ils veulent, que le Père de Picus s’appelait Stercé, et qu’il fut ainsi nommé à cause qu’étant fort bon laboureur, il apprit aux hommes à amender la terre avec du fumier a, d’où vient que quelques auteurs l’appellent Stercutius. Quoi qu’il en soit, ils en ont fait pour cette raison le Dieu de l’agriculture. Ils ont mis aussi Picus parmi lesdieux, en qualité d’excellent augure et de grand capitaine. Picus engendra Faunus, second roi des Laurentins, qu’ils ont aussi déifié. Avant la guerre de Troie, ces apothéoses étaient fréquentes.
\subsection[{Chapitre XVI}]{Chapitre XVI}

\begin{argument}\noindent De Diomède et de ses compagnons, changés en oiseaux après la ruine de Troie.
\end{argument}

\noindent Après la ruine de Troie, ce grand désastre illustré par les poètes et connu même des petits enfants, qui arriva sous le règne de Latinus, fils de Faunus (ce Latinus qui donna aux Laurentins leur nom nouveau de Latins qu’ils portèrent depuis ce moment), les Grecs victorieux regagnèrent leur pays et souffrirent pendant ce retour une infinité de maux. Ils en prirent sujet d’augmenter le nombre de leurs divinités. En effet, ils firent un dieu de Diomède ; ce qui ne les empêcha pas de raconter, non comme une fable, mais comme une vérité historique, que les dieux s’opposèrent au retour de ce personnage pour le châtier de ses crimes, et que ses compagnons furent changés en oiseaux, sans que Diomède, devenu dieu, leur pût rendre leur première forme, ni obtenir cette grâce de Jupiter pour sa bienvenue. Ils assurent même que Diomède a un temple dans l’île Diomédéa, non loin du mont Garganus en Apulie, et qu’autour du lieu sacré volent ces oiseaux, jadis compagnons du héros divinisé, qui remplissent leur bec d’eau et arrosent son temple pour lui faire honneur. Ils ajoutent que lorsque des Grecs viennent en cette île, non seulement les oiseaux ne s’effarouchent point, mais ils caressent les visiteurs, au lieu que, quand ils voient des étrangers, ils volent contre eux en furie, et souvent les tuent avec leur bec, qui est d’une longueur et d’une force extraordinaires.
\subsection[{Chapitre XVII}]{Chapitre XVII}

\begin{argument}\noindent Sentiment de Varron sur certaines métamorphoses.
\end{argument}

\noindent Varron, à l’appui de cette tradition, en rapporte d’autres qui ne sont pas moins incroyables : celle de Circé, par exemple, la fameuse magicienne, qui changea en bêtes lescompagnons d’Ulysse ; et encore, celle de ces Arcadiens, désignés par le sort pour passer à la nage un certain étang où ils se transformaient en loups, vivant ensuite dans les forêts avec les animaux de leur espèce. Varron ajoute que si ces loups s’abstenaient de chair humaine, ils repassaient l’étang au bout de neuf ans, et reprenaient leur première forme. Il parle en outre d’un certain Demaenetus qui, ayant goûté du sacrifice d’un petit enfant que les Arcadiens font à leur dieu Lycaeus, fut changé en loup ; dix ans après, il redevint homme et remporta le prix aux jeux olympiens. Le même auteur estime qu’en Arcadie on ne donne le nom de Lycaeus à Pan et à Jupiter qu’à cause de ces changements d’hommes en loups, attribués par le peuple à un miracle de la volonté divine ; car les Grecs appellent un loup {\itshape lycos}, d’où le nom de {\itshape Lycaeus} est dérivé. Enfin, selon Varron, c’est de là que les Luperques de Rome tirent leur origine.
\subsection[{Chapitre XVIII}]{Chapitre XVIII}

\begin{argument}\noindent Ce qu’il faut croire des métamorphoses.
\end{argument}

\noindent Ceux qui lisent ces pages attendent peut-être que je donne mon sentiment ; mais que pourrais-je dire, sinon qu’il faut fuir du milieu de Babylone, c’est-à-dire sortir de la cité du monde, qui est la société des anges et des hommes impies, et nous retirer vers le Dieu vivant, sur les pas de la foi rendue féconde par la charité ? Plus nous voyons que la puissance des démons est grande ici-bas, plus nous devons nous attacher au Médiateur, qui nous retire des choses basses pour nous élever aux objets sublimes. En effet, si nous disons qu’il ne faut point ajouter foi à ces sortes de phénomènes, il ne manquera pas, même aujourd’hui, de gens qui assureront en avoir appris ou expérimenté de semblables. Comme nous étions en Italie, on nous assura que certaines hôtelières de notre voisinage, initiées aux arts sacrilèges, se vantaient de donner aux passants d’un certain fromage qui les changeait sur-le-champ en bêtes de somme dont elles se servaient pour transporter leurs bagages, après quoi elles leur rendaient leur première forme. Pendant la métamorphose, ils conservaient toujours leur raison, comme Apulée le raconte de lui-même dans son récit ou son roman de l’{\itshape Âne d’or}.\par
Je tiens tout cela pour faux, ou du moins ce sont là des phénomènes si rares qu’on a raison de n’y pas ajouter foi. Ce qu’il faut croire fermement, c’est que Dieu, l’être tout-puissant, peut faire tout ce qu’il veut, soit pour répandre ses grâces, soit pour punir, et que les démons, qui sont des anges, mais corrompus, ne peuvent rien au-delà de ce que leur permet celui dont les jugements sont quelquefois secrets, jamais injustes. Quand donc ils opèrent de semblables phénomènes, ils ne créent pas de nouvelles natures, mais se bornent à changer celles que le vrai Dieu a créées et à les faire paraître autres qu’elles ne sont. Ainsi, non seulement je ne crois pas que les démons puissent changer l’âme d’un homme en celle d’une bête, mais, à mon avis, ils ne peuvent pas même produire dans leurs corps cette métamorphose. Ce qu’ils peuvent, c’est de frapper l’imagination, qui tout incorporelle qu’elle soit, est susceptible de mille représentations corporelles ; appelant d’ailleurs à leur aide l’assoupissement ou la léthargie, ils parviennent, je ne sais comment, à imprimer dans les âmes une forme toute fantastique, assez fortement pour qu’elle semble réelle à nos faibles yeux. Il peut même arriver que celui dont ils se jouent de la sorte se croie tel qu’il paraît, tout comme il lui semble en dormant qu’il est un cheval et qu’il porte quelque fardeau. Si ces fardeaux sont de vrais corps, ce sont les démons qui les portent, afin de surprendre les hommes par cette illusion et de leur faire croire que la bête qu’ils voient est aussi réelle que le fardeau dont elle est chargée. Un certain Praestantius racontait que son père, ayant par hasard mangé de ce singulier fromage dont nous parlions tout à l’heure ; demeura comme endormi sur son lit sans qu’on le pût éveiller ; quelques jours après, il revint à lui comme d’un profond sommeil, disant qu’il était devenu cheval et qu’il avait porté à l’armée de ces vivres qu’on appelle {\itshape retica} à cause des filets qui les enveloppent ; or, le fait s’était passé, dit-on, comme il le décrivait, bien qu’il prît tout cela pour un songe. Un autre rapportait qu’une nuit, avant de s’endormir, il avait vu venir à lui un philosophe platonicien de sa connaissance, qui lui avait expliqué certains sentiments de Platon qu’il avait refusé auparavant de lui éclaircir. Comme on demandait à cephilosophe pourquoi il avait accordé hors de chez lui ce que chez lui il avait refusé : « Je n’ai pas fait cela, dit-il, mais j’ai songé que je le faisais. » Et ainsi, l’un vit en veillant, par le moyen d’une image fantastique, ce que l’autre avait rêvé.\par
Ces faits nous ont été rapportés, non par des témoins quelconques, mais par des personnes dignes de foi. Si donc ce que l’on dit des Arcadiens et de ces compagnons d’Ulysse dont parle Virgile :\par
{\itshape « Transformés par les enchantements de Circé »} ; \par
si tout cela est vrai, j’estime que les choses se sont passées comme je viens de l’expliquer. Quant aux oiseaux de Diomède, comme on dit que la race en subsiste encore, je pense que les compagnons du héros grec ne furent pas métamorphosés en oiseaux, mais que ces oiseaux furent mis à leur place, comme la biche à celle d’Iphigénie. Il était facile aux démons, avec la permission de Dieu, d’opérer de semblables prestiges. Mais, comme Iphigénie fut trouvée vivante après le sacrifice, on jugea aisément que la biche avait été supposée en sa place ; tandis que les compagnons de Diomède n’ayant point été trouvés depuis, parce que les mauvais anges les exterminèrent par l’ordre de Dieu, on a cru qu’ils avaient été changés en ces oiseaux que les démons eurent l’art de leur substituer. Maintenant, que ces oiseaux arrosent d’eau le temple de Diomède, qu’ils caressent les Grecs et déchirent les étrangers, c’est un stratagème des mêmes démons, auxquels il importe de faire croire que Diomède est devenu dieu, afin de tromper les simples, et d’obtenir pour des hommes morts, qui n’ont pas même vécu en hommes, ces temples, ces autels, ces sacrifices, ces prêtres, tout ce culte enfin qui n’est dû qu’au Dieu de vie et de vérité.
\subsection[{Chapitre XIX}]{Chapitre XIX}

\begin{argument}\noindent Énée est venu en Italie au temps où Labdon était juge des Hébreux.
\end{argument}

\noindent Après la ruine de Troie, Énée aborda en Italie avec vingt navires qui portaient les restes des Troyens. Latinus était roi de cette contrée, comme Mnesthéus l’était des Athéniens, Polyphidès des Sicyoniens, Tantanès des Assyriens ; Labdon était juge des Hébreux.\par
Après la mort de Latinus, Énée régna trois ans en Italie, tous les rois dont nous venons de parler étant encore vivants, à la réserve de Polyphidès, roi des Sicyoniens, à qui Pélasgus avait succédé. Samson était juge des Hébreux à la place de Labdon, et comme il était extraordinairement fort, on le prit pour Hercule. Énée ayant disparu après sa mort, les Latins en firent un dieu. Les Sabins mirent aussi au rang des dieux Sancus ou Sanctus, leur premier roi. Environ vers le même temps, Codrus, roi des Athéniens, se fit tuer volontairement par les Péloponnésiens, et ce dévouement sauva son pays. Ceux du Péloponnèse avaient reçu de l’oracle cette réponse, qu’ils vaincraient les Athéniens s’ils ne tuaient point leur roi. Codrus les trompa en changeant d’habit et leur disant des injures pour les provoquer à le tuer ; c’est cette {\itshape querelle de Codrus} à laquelle Virgile fait quelque part allusion. Des Athéniens honorèrent ce roi comme un dieu. Sous le règne de Sylvius, quatrième roi des Latins et fils d’Énée (non de Créusa, de laquelle naquit Ascanius, troisième roi de ces peuples, mais de Lavinia, fille de Latinus, qui accoucha de Sylvius après la mort d’Énée), Onéus étant le vingt-neuvième roi des Assyriens, Mélanthus le seizième d’Athènes, et le grand prêtre Héli jugeant le peuple hébreu, la monarchie des Sicyoniens fut éteinte, après avoir duré l’espace de neuf cent cinquante-neuf ans.
\subsection[{Chapitre XX}]{Chapitre XX}

\begin{argument}\noindent Succession des rois des Juifs après le temps des Juges.
\end{argument}

\noindent Ce fut vers ce temps-là que le gouvernement des Juges étant fini parmi les Juifs, ils élurent pour leur premier roi Saül, sous lequel vivait le prophète Samuel. Les rois latins commencèrent alors à s’appeler Sylviens, de Sylvius fils d’Énée, comme depuis on appela Césars tous les empereurs romains qui succédèrent à Auguste. Après la mort de Saül, qui régna quarante ans, David fut le second roi des Juifs. Depuis la mort de Codrus, les Athéniens n’eurent plus de rois, et confièrent à des magistrats le soin de gouverner leur république. À David, dont le règne dura aussi quarante ans, succéda son fils Salomon, qui bâtit ce fameux temple de Jérusalem. De son temps, les Latins fondèrent Albe, qui donna son nom à leurs rois. Salomon laissa son royaume à son fils Roboam, sous qui la Judée fut divisée en deux royaumes.
\subsection[{Chapitre XXI}]{Chapitre XXI}

\begin{argument}\noindent Des rois du Latium, dont le premier et le douzième, c’est-à-dire Énée et Aventinus, furent mis au rang des dieux.
\end{argument}

\noindent Les Latins eurent après Énée onze rois qu’ils ne mirent point comme lui au nombre des dieux ; mais Aventinus, qui fut le douzième, ayant été tué dans un combat et enseveli sur le mont qui porte encore aujourd’hui son nom, eut rang parmi ces étranges divinités. Selon d’autres historiens, il ne serait pas mort dans la bataille, mais il n’aurait plus reparu depuis, et ce n’est pas de lui que le mont Aventin aurait pris son nom, mais des oiseaux qui venaient s’y reposer. Après Aventinus, les Latins ne firent plus d’autre dieu que Romulus, fondateur de Rome. Mais entre ces deux rois, il s’en trouve deux autres, dont le premier est, pour parler avec Virgile :\par
 {\itshape « Procas, la gloire de la nation troyenne ».} \par
Ce fut sous le règne de celui-ci, tandis que se faisait l’enfantement de Rome, que la grande monarchie des Assyriens termina sa longue carrière. Elle passa aux Mèdes après avoir duré plus de treize cents ans, en la faisant commencer à Bélus, père de Ninus. Amulius succéda à Procas. On dit que Rhéa ou Ilia, fille de son frère Numitor, et mère de Romulus, qu’il avait faite vestale, conçut deux jumeaux du dieu Mars ; la preuve qu’il donne de cette paternité divine imaginée pour la gloire ou l’excuse de la vestale, c’est que, les deux enfants ayant été exposés par ordre d’Amulius, une louve les allaita. Or, la louve est consacrée au dieu Mars, et on veut qu’elle ait reconnu les enfants de son maître ; mais il ne manque pas de gens pour soutenir que les deux jumeaux furent recueillis par une femme publique (on appelait cette sorte de femmes louves, {\itshape lupae} d’où est venu {\itshape lupanar}), laquelle les allaita et les mit ensuite entre les mains de Faustulus, l’un des bergers du roi, qui les fit soigner parsa femme Acca. Mais quand Dieu aurait permis que des bêtes farouches eussent nourri ces enfants qui devaient fonder un si grand empire, pour faire plus de honte à ce roi cruel qui les avait fait jeter dans la rivière, qu’y aurait-il en cela de si merveilleux ? Numitor, grand-père de Romulus, succéda à son frère Amulius, et Rome fut bâtie la première année de son règne. Ainsi il gouverna conjointement avec son petit-fils Romulus.
\subsection[{Chapitre XXII}]{Chapitre XXII}

\begin{argument}\noindent Fondation de Rome à l’époque où l’empire d’Assyrie prit fin et où Ézéchias était roi de Juda.
\end{argument}

\noindent Pour abréger le plus possible, je dirai que Rome fut bâtie comme une autre Babylone, ou comme la fille de la première, et qu’il a plu à Dieu de s’en servir pour dompter l’univers et réduire toutes les nations à l’unité de la même république et des mêmes lois. Il y avait alors des peuples puissants et aguerris, qui ne se soumettaient pas aisément, et ne pouvaient être vaincus sans qu’il en coûtât beaucoup de peine et de sang aux vainqueurs. En effet, lorsque les Assyriens conquirent presque toute l’Asie, les peuples n’étaient ni en si grand nombre ni si exercés aux armes, de sorte qu’ils en eurent bien meilleur marché. Depuis ce grand déluge, dont il ne se sauva que huit personnes, jusqu’à Ninus qui se rendit maître de toute l’Asie, il ne s’était écoulé qu’environ mille ans. Mais Rome ne vint pas si aisément à bout de l’Orient et de l’Occident et de tant de nations que nous voyons aujourd’hui soumises à son empire, parce qu’elle trouva de toutes parts des ennemis puissants et belliqueux. Lors donc qu’elle fut fondée, il y avait déjà sept cent dix-huit ans que les Juifs dominaient dans la terre promise, Jésus Navé ayant gouverné ce peuple vingt-sept ans, les Juges trois cent vingt-neuf ans, et les Rois trois cent soixante-deux. Achaz régnait alors en Juda, ou, selon d’autres, son successeur Ézéchias, prince excellent en vertu et en piété, qui vivait du temps de Romulus ; Osée tenait le sceptre d’Israël.
\subsection[{Chapitre XXIII}]{Chapitre XXIII}

\begin{argument}\noindent De la sibylle d’Érythra, bien connue entre toutes les autres sibylles pour avoir fait les prophéties les plus claires touchant Jésus-Christ.
\end{argument}

\noindent Plusieurs historiens estiment que ce fut en ce temps que parut la sibylle d’Érythra. On sait qu’il y a eu plusieurs sibylles, selon Varron. Celle-ci a fait sur Jésus-Christ des prédictions très claires que nous avons d’abord lues en vers d’une mauvaise latinité et se tenant à peine sur leurs pieds, ouvrage de je ne sais quel traducteur maladroit, ainsi que nous l’avons appris depuis. Car le proconsul Flaccianus, homme éminent par l’étendue de son savoir et la facilité de son éloquence, nous montra, un jour que nous nous entretenions ensemble de Jésus-Christ, l’exemplaire grec qui a servi à cette mauvaise traduction. Or, il nous fit en même temps remarquer un certain passage, où en réunissant les premières lettres de chaque vers, on forme ces mots : {\itshape Iesous Kreistos Theou Uios Soter}, c’est-à-dire {\itshape Jésus-Christ, fils de Dieu, Sauveur}. Or, voici le sens de ces vers, d’après une autre traduction latine, meilleure et plus régulière :\par
 {\itshape « Aux approches du jugement, la terre se couvrira d’une sueur glacée. Le roi immortel viendra du ciel et paraîtra revêtu d’une chair pour juger le monde, et alors les bons et les méchants verront le Dieu tout-puissant accompagné de ses saints. Il jugera les âmes aussi revêtues de leurs corps, et la terre n’aura plus ni beauté ni verdure. Les hommes effrayés laisseront à l’abandon leurs trésors et ce qu’ils avaient de plus précieux. Le feu brûlera la terre, la mer et le ciel, et ouvrira les portes de l’enfer. Les bienheureux jouiront d’une lumière pure et brillante, et les coupables seront la proie des flammes éternelles. Les crimes les plus cachés seront découverts et les consciences mises à nu. Alors il y aura des pleurs et des grincements de dents. Le soleil perdra sa lumière et les étoiles seront éteintes. La lune s’obscurcira, les cieux seront ébranlés sur leurs pôles, et les plus hautes montagnes abattues et égalées aux vallons. Plus rien dans les choses humaines de sublime ni de grand. Toute la machine de l’univers sera détruite, et le feu consumera l’eau des fleuves et des fontaines. Alors on entendra sonner la trompette, et tout retentira de cris et de plaintes. La terre s’ouvrira jusque dans ses abîmes ; les rois paraîtront tous devant le tribunal du souverain Juge, et les cieux verseront un fleuve de feu et de soufre. »} \par
Ce passage comprend en grec vingt-sept vers, nombre qui compose le cube de trois.\par
Ajoutez à cela que, si l’on joint ensemble les premières lettres de ces cinq mots grecs que nous avons dit signifier {\itshape Jésus-Christ, Fils de Dieu, Sauveur}, on trouvera{\itshape  Ichthus}, qui veut dire en grec {\itshape poisson}, nom mystique du Sauveur, parce que lui seul a pu demeurer vivant, c’est-à-dire exempt de péché, au milieu des abîmes de notre mortalité, semblables aux profondeurs de la mer.\par
D’ailleurs, que ce poème, dont je n’ai rapporté que quelques vers, soit de la sibylle d’Érythra ou de celle de Cumes, car on n’est pas d’accord là-dessus, toujours est-il certain qu’il ne contient rien qui favorise le culte des faux dieux ; au contraire, il parle en certains endroits si fortement contre eux et contre leurs adorateurs qu’il me semble qu’on peut mettre cette sibylle au nombre des membres de la Cité de Dieu. Lactance a aussi inséré dans ses œuvres quelques prédictions d’une sibylle (sans dire laquelle) touchant Jésus-Christ, et ces témoignages, qui se trouvent dispersés en divers endroits de son livre, m’ont paru bons à être ici réunis : « Il tombera, dit la sibylle, entre les mains des méchants, qui lui donneront des soufflets et lui cracheront au visage. Pour lui, il présentera sans résistance son dos innocent aux coups de fouet, et il se laissera souffleter sans rien dire, afin que personne ne connaisse quel Verbe il est, ni d’où il vient pour parler aux enfers et être couronné d’épines. Les barbares, pour toute hospitalité, lui ont donné du fiel à manger et du vinaigre à boire. Tu n’as pas reconnu ton Dieu, nation insensée ! ton Dieu qui se joue de la sagesse des hommes ; tu l’as couronné d’épines et nourri de fiel. Le voile du temple se rompra, et il y aura de grandes ténèbres en plein jour pendant trois heures. Il mourra et s’endormira durant trois jours. Et puis retournant à la lumière, il montrera aux élus les prémices de la résurrection. »\par
Voilà les textes sibyllins que Lactance rapporte en plusieurs lieux de ses ouvrages et que nous avons réunis. Quelques auteurs assurent que la sibylle d’Érythra ne vivait pas à l’époque de Romulus, mais pendant la guerre de Troie.
\subsection[{Chapitre XXIV}]{Chapitre XXIV}

\begin{argument}\noindent Les Sept Sages ont fleuri sous le règne de Romulus, dans le temps où les dix tribus d’Israël furent menées captives en Chaldée.
\end{argument}

\noindent Sous le règne de ce même Romulus vivait Thalès le Milésien, l’un des Sages qui succédèrent à ces poètes théologiens parmi lesquels Orphée tient le premier rang. Environ au même temps, les dix tribus d’Israël furent vaincues par les Chaldéens et emmenées captives, tandis que les deux autres restaient paisibles à Jérusalem. Romulus ayant disparu d’une façon mystérieuse, les Romains le mirent au rang des dieux, ce qui ne se pratiquait plus depuis longtemps, et ne se fit dans la suite à l’égard des Césars que par flatterie. Cicéron prend de là occasion de donner de grandes louanges à Romulus pour avoir mérité cet honneur, non à ces époques de grossièreté et d’ignorance où il était si aisé de tromper les hommes, mais dans un siècle civilisé, déjà plein de lumières, bien que l’ingénieuse et subtile loquacité des philosophes ne se fût pas encore répandue de toutes parts. Mais si les époques suivantes n’ont pas transformé les hommes morts en dieux, elles n’ont pas laissé d’adorer les anciennes divinités, et même d’augmenter la superstition en construisant des idoles, usage inconnu à l’antiquité. Les démons portèrent les peuples à représenter sur les théâtres les crimes supposés des dieux et à consacrer des jeux en leur honneur, pour renouveler ainsi ces vieilles fables, le monde étant trop civilisé pour en introduire de nouvelles. Numa succéda à Romulus ; et bien qu’il eût peuplé Rome d’une infinité de dieux, il n’eut pas le bonheur, après sa mort, d’être de ce nombre, peut-être parce qu’on crut que le ciel en était si plein qu’il n’y restait pas de place pour lui. On dit que la sibylle de Samos vivait de son temps, vers le commencement du règne de Manassès, roi des Juifs, qui fit mourir cruellement le prophète Isaïe.
\subsection[{Chapitre XXV}]{Chapitre XXV}

\begin{argument}\noindent Des philosophes qui se sont signalés sous le règne de Sédéchias, roi des Juifs, et de Tarquin l’Ancien, roi des Romains, au temps de la prise de Jérusalem et de la ruine du temple.
\end{argument}

\noindent Sous le règne de Sédéchias, roi des Juifs, et de Tarquin l’Ancien, roi des Romains, qui avait succédé à Ancus Martius, le peuple juif fut mené captif à Babylone, après la ruine de Jérusalem et du temple de Salomon. Ce malheur leur avait été prédit par les Prophètes, et particulièrement par Jérémie, qui même en avait marqué l’année. Pittacus, de Mitylène, l’un des sept sages, vivait en ce temps-là, et Eusèbe y joint les cinq autres, car Thalès a déjà été mentionné, savoir : Solon d’Athènes, Chilon de Lacédémone, Périandre de Corinthe, Cléobule de Lindos, et Bias de Priène. Ils furent nommés Sages, parce que leur genre de vie les élevait au-dessus du commun des hommes, et comme ayant tracé quelques préceptes courts et utiles pour les mœurs. Du reste, ils n’ont point laissé d’autres écrits à la postérité, si ce n’est quelques lois qu’on dit que Solon donna aux Athéniens. Thalès a aussi composé quelques livres de physique, qui contiennent sa doctrine. D’autres physiciens parurent encore en ce temps, comme Anaximandre, Anaximène et Xénophane. Pythagore florissait aussi alors, et c’est lui qui porta le premier le nom de philosophe.
\subsection[{Chapitre XXVI}]{Chapitre XXVI}

\begin{argument}\noindent Fin de la captivité de Babylone et du règne des rois de Rome.
\end{argument}

\noindent En ce temps-là, Cyrus, roi de Perse, qui commandait aussi aux Chaldéens et aux Assyriens, relâchant un peu de la chaîne des Juifs, en renvoya cinquante mille pour rebâtir le temple. Mais ils se bornèrent à en jeter les fondements et à dresser un autel, à cause des courses continuelles des ennemis, de sorte que l’ouvrage fut différé jusqu’au règne de Darius. Ce fut alors qu’arriva ce qui est rapporté dans le livre de Judith que les Juifs nereçoivent point parmi les livres canoniques. Or, sous le règne de Darius, roi des Perses, les soixante-dix années prédites par Jérémie étant accomplies, la liberté fut rendue aux Juifs, pendant que les Romains chassaient Tarquin le Superbe et s’affranchissaient de la domination de leurs rois. Jusque-là, les Juifs eurent toujours des prophètes ; mais à cause de leur grand nombre, il y en a peu dont les écrits soient reçus comme canoniques, tant par les Juifs que par nous. Sur la fin du livre précédent, j’ai promis d’en dire quelque chose, et il est temps de m’acquitter de ma promesse.
\subsection[{Chapitre XXVII}]{Chapitre XXVII}

\begin{argument}\noindent Des prophètes qui s’élevèrent parmi les Juifs au commencement de l’empire romain.
\end{argument}

\noindent Afin que nous puissions bien voir en quel temps ils vivaient, remontons un peu plus haut. Le livre d’Osée, qui est le premier des douze petits prophètes, porte en tête : « Voici ce que le Seigneur a dit à Osée du temps d’Ozias, de Joathan, d’Achaz et d’Ézéchias, rois de Judée. » Amos de même dit qu’il prophétisa sous Ozias ; il ajoute et sous Jéroboam, roi d’Israël, qui vivait vers ce temps-là. Isaïe, fils d’Amos, soit du prophète, soit d’un autre Amos, indique au commencement de son ouvrage les quatre rois dont parle Osée au début du sien, et déclare comme lui qu’il prophétisa sous leur règne. Michée marque aussi le temps de sa prophétie après Ozias, sous Joathan, Achaz et Ézéchias. Il faudrait joindre à ces prophètes Jonas et Joël, dont l’un prophétisa sous Ozias, et l’autre sous Joathan, au moins selon les chronologistes, car eux-mêmes n’en disent rien. Or, tout cet espace de temps va depuis Procas, roi des Latins, ou Aventinus, son prédécesseur, jusqu’à Romulus, roi des Romains ou même jusqu’au commencement du règne de son successeur Numa Pompilius ; car l’époque d’Ézéchias se prolonge jusque-là. Ce fut donc en cet espace de temps que jaillirent ces sources de prophéties, sur la fin de l’empire des Assyriens et au commencement de celui des Romains. Comme en effet c’est à la naissance de la monarchie des Assyriens que les promesses du Messie furent faites à Abraham, elles devaient être renouvelées à ces prophètesau commencement de la monarchie romaine, Babylone de l’Occident, sous le règne de laquelle elles devaient s’accomplir par l’avènement de Jésus-Christ. Ces dernières prophéties sont encore plus claires que les autres, comme ne devant pas seulement servir aux Juifs, mais aussi aux païens.
\subsection[{Chapitre XXVIII}]{Chapitre XXVIII}

\begin{argument}\noindent Vocation des Gentils prédite par Osée et par Amos.
\end{argument}

\noindent Il est vrai qu’Osée est quelquefois difficile à saisir dans sa profondeur ; mais il faut en rapporter ici quelque chose pour m’acquitter de ma promesse : « Et il arrivera, dit-il, qu’au même lieu où il est écrit : Vous n’êtes point mon peuple, ils seront aussi appelés les enfants du Dieu vivant. » Les Apôtres mêmes ont entendu cette prophétie de la vocation des Gentils. Et comme les Gentils sont aussi spirituellement les enfants d’Abraham, et qu’à ce titre on a raison de les appeler le peuple d’Israël, le Prophète ajoute : « Et les enfants de Juda et d’Israël seront rassemblés en un même corps et n’auront plus qu’un chef, et ils s’élèveront sur la terre. » Ce serait ôter sa force à cette prophétie que de vouloir l’expliquer davantage. Qu’on se souvienne seulement de la pierre angulaire et de ces deux murailles, l’une composée des Juifs, et l’autre des Gentils ; celle-là sous le nom de Juda, et celle-ci sous le nom d’Israël, s’appuyant toutes deux sur un même chef, et toutes deux s’élevant sur la terre. À l’égard de ces Israélites charnels, qui ne veulent pas croire en Jésus-Christ, le même prophète témoigne qu’ils croiront un jour en lui (entendez : non pas eux, mais leurs enfants), lorsqu’il dit : « Les enfants d’Israël demeureront longtemps sans roi, sans prince, sans sacrifice, sans autel, sans sacerdoce, sans prophétie. » Qui ne voit que c’est l’état où sont maintenant les Juifs ? Mais écoutons ce qu’il ajoute : « Et après cela, les enfants d’Israël reviendront et chercheront le Seigneur, leur Dieu, et leur roi David ; et ils s’étonneront de leur aveuglement et de la grâce de Dieu dans les derniers temps. » Il n’y a rien de plus clair que cette prophétie Où Jésus-Christ est marqué par David, parceque, comme dit l’Apôtre : « Il est né selon la chair de la race de David. » Ce même prophète a prédit la résurrection du Sauveur au troisième jour, mais d’une manière mystérieuse et prophétique, lorsqu’il a dit : « Il nous guérira après deux jours, et nous ressusciterons le troisième. » C’est dans le même sens que l’Apôtre nous dit : « Si vous êtes ressuscités avec Jésus-Christ, cherchez les choses du ciel. » Voici encore une prophétie d’Amos sur ce sujet : « Israël, dit-il, préparez-vous pour invoquer votre Dieu, car c’est moi qui fais gronderie tonnerre, qui forme les tourbillons, et qui annonce aux hommes leur Sauveur. » Et ailleurs : « En ce jour-là, dit-il, je relèverai le pavillon de Dieu qui est tombé, et je rétablirai tout ce qui est détruit ; je le remettrai au même état qu’il était le premier jour ; en sorte que tout le reste des hommes me chercheront, ainsi que toutes les nations qui deviendront mon peuple, dit le Seigneur qui fait ces merveilles. »
\subsection[{Chapitre XXIX}]{Chapitre XXIX}

\begin{argument}\noindent Prophéties d’Isaïe touchant Jésus-Christ et son Église.
\end{argument}

\noindent Isaïe n’est pas du nombre des douze petits prophètes, qu’on nomme ainsi parce qu’ils ont écrit peu de chose au prix de ceux qu’on appelle les grands prophètes. Parmi ceux-là est Isaïe, que je joins à Osée et à Amos, comme ayant vécu du même temps. Ce prophète donc, entre les instructions qu’il donne au peuple et les menaces qu’il lui fait de la part de Dieu, a prédit beaucoup plus de choses que tous les autres de Jésus-Christ et de son Église, c’est-à-dire du roi de gloire et de la cité qu’il a bâtie, tellement, qu’il y en a qui disent que c’est plutôt un évangéliste qu’un prophète. Mais, pour abréger, je n’en rapporterai ici qu’un seul endroit, celui où il dit en la personne de Dieu le père : « Mon fils sera rempli de science et de sagesse ; il sera comblé d’honneur et de gloire. Comme il sera un spectacle d’horreur à plusieurs qui le verront déshonoré et défiguré, il sera un sujet d’admiration à une infinité de peuples, et les rois, pleins d’étonnement, demeureront dans un profond silence, parce que ceux à qui iln’a point été annoncé le verront, et ceux qui n’ont point entendu parler de lui sauront qui il est. Seigneur, qui a cru à notre parole, et à qui le bras de Dieu a-t-il été révélé ? Nous bégaierons devant lui comme un enfant, et notre langue sera sèche comme une racine dans une terre sans eau. Il n’a ni gloire, ni beauté. Nous l’avons vu sans majesté et sans grâce, et le dernier des hommes était moins difforme que lui. C’est un homme en butte aux coups et accablé de faiblesse. Il a caché sa gloire ; c’est pourquoi il a été méprisé et déshonoré. Il porte nos péchés, et c’est pour nous qu’il souffre ; et nous avons cru que c’était pour ses crimes. Cependant c’est à cause de nos iniquités qu’il a été couvert de blessures, et ce sont nos péchés qui l’ont réduit en cet état de faiblesse. Il nous a procuré la paix par ses souffrances, et ses plaies ont été notre guérison. Nous étions tous comme des brebis égarées ; tous les hommes s’étaient écartés du droit chemin, et le Seigneur l’a livré pour nos péchés, et il n’a pas ouvert la bouche pour se plaindre. Il a été mené comme une brebis à la boucherie, et il est demeuré muet comme un agneau qu’on tond. Son abaissement lui a servi de degré pour monter à la gloire : qui pourra raconter sa génération ? Il sera enlevé du monde, et les péchés de mon peuple le conduiront au supplice. Sa sépulture coûtera la vie aux méchants, et les riches porteront la vengeance de sa mort, parce qu’il n’a fait aucun mal, qu’il n’y a en lui ni artifice, ni déguisement, et que le Seigneur veut le guérir de ses blessures. Si vous souffrez la mort pour vos péchés, vous verrez une longue postérité. Le Seigneur veut le délivrer de toute douleur, lui rendre le jour, remplir son esprit de lumière, justifier le juste qui s’est sacrifié pour plusieurs et qui s’est chargé de leurs péchés. Aussi acquerrai-t-il un domaine sur plusieurs, et il partagera les dépouilles des puissants, parce qu’il a été livré à la mort et mis au rang des scélérats, qu’il a porté les péchés de plusieurs et qu’il est mort pour leurs péchés. »\par
Voilà ce que dit ce prophète au sujet de Jésus-Christ.\par
Citons ce qu’il ajoute de l’Église : « Réjouissez-vous, stérile qui n’enfantez pas ; éclatez en cris de joie, vous qui ne concevez point ; car celle qui est abandonnée aura plus d’enfants que celle qui a un mari. Étendez le lieu de votre demeure et dressez vos pavillons. Ne ménagez point le terrain, prenez de grands alignements et enfoncez de bons pieux en terre. Étendez-vous à droite et à gauche, car cette postérité possédera les nations comme son héritage, et vous peuplerez les cités désertes. Vous êtes maintenant honteuse à cause des reproches qu’on vous fait ; mais ne craignez rien : cette honte sera ensevelie dans un éternel oubli, et vous ne vous souviendrez plus de l’opprobre de votre veuvage, parce que le Seigneur qui vous a créée s’appelle le Dieu des armées, et celui qui vous a délivrée est le Dieu d’Israël et de toute la terre. » Cette citation suffit, et bien qu’il se trouve certaines choses dans ces passages qui auraient besoin d’explication, il en est d’autres qui sontsi claires que nos ennemis mêmes les entendent ; malgré qu’ils en aient.
\subsection[{Chapitre XXX}]{Chapitre XXX}

\begin{argument}\noindent Prophéties de Michée, Jonas et Joël qui regardent Jésus-Christ.
\end{argument}

\noindent Le prophète Michée, parlant de Jésus-Christ sous la figure d’une haute montagne, dit ceci : « Dans les derniers temps, la montagne du Seigneur paraîtra élevée au-dessus des plus hautes montagnes, et les peuples s’y rendront en foule de toutes parts, et diront : Venez, montons sur la montagne du Seigneur, et allons en la maison du Dieu de Jacob, et il nous enseignera le chemin qui mène à lui, et nous marcherons dans ses sentiers. Car la loi sortira de Sion, et la parole du Seigneur, de Jérusalem. Il jugera plusieurs peuples, et s’assujettira des nations puissantes pour longtemps. » Le même prophète dit du lieu de la naissance du Sauveur : « Et toi, Bethléem, maison d’Ephrata, tu es trop petite pour être mise au rang de ces villes de Juda qui fournissent des milliers d’hommes, et cependant c’est de toi que sortira le prince d’Israël. Sa sortie est dès le commencement et de toute éternité. C’est pourquoi Dieu abandonnera les siens jusqu’au temps où celle qui est en travail d’enfant doit accoucher, et le reste de ses frères se rangeront avec les enfantsd’Israël. Il s’arrêtera, il contemplera et paîtra son troupeau par l’autorité et le pouvoir qu’il en a reçu du Seigneur ; et ils rendront leurs hommages au Seigneur, leur Dieu, qui sera glorifié jusqu’aux extrémités de la terre. »\par
Le prophète Jonas n’a pas tant annoncé le Sauveur par ses discours que par cette espèce de passion qu’il a subie. Car pourquoi a-t-il été englouti dans le ventre d’une baleine et rejeté le troisième jour, sinon pour signifier la résurrection de Jésus-Christ ?\par
Pour Joël, il faudrait s’engager dans un long discours pour expliquer toutes les prophéties qu’il a faites de Jésus-Christ et de l’Église. Toutefois j’en rapporterai un passageque les Apôtres mêmes alléguèrent, quand le Saint-Esprit descendit sur eux, selon la promesse de Jésus-Christ : « Après cela, dit-il, je répandrai mon esprit sur toute chair. Vos fils et vos filles prophétiseront, vos vieillards auront des songes, et vos jeunes gens des visions. En ce temps-là, je répandrai mon esprit sur mes serviteurs et sur mes servantes. »
\subsection[{Chapitre XXXI}]{Chapitre XXXI}

\begin{argument}\noindent Salut du monde par Jésus-Christ prédit par Abdias, Nahum et Habacuc.
\end{argument}

\noindent Trois des petits prophètes, Abdias, Nahum et Habacuc, ne disent rien du temps où ils ont prophétisé, et l’on n’en trouve rien non plus dans les chronologies d’Eusèbe et de Jérôme. Il est vrai qu’elles joignent Abdias à Michée ; mais je pense que c’est une faute de copiste ; car elles mettent Abdias sous Josaphat, et il est certain que Michée n’est venu que longtemps après. Pour les deux autres, nous ne les avons trouvés mentionnés dans aucune chronologie. Toutefois, comme ils sont reçus parmi les livres canoniques, il ne faut pas que nous les omettions. Abdias, le plus court de tous les Prophètes, parle contre le peuple d’Idumée, c’est-à-dire contre Ésaü, l’aîné des deux enfants d’Isaac, qui fut réprouvé. Que si par l’Idumée nous entendons toutes les nations, en prenant la partie pour le tout, comme cela est assez ordinaire dans le langage, nous pouvons fort bien appliquer à Jésus-Christ ce qu’il dit entre autres choses : « Le salut et la sainteté seront sur la montagne de Sion » ; et un peu après, sur la fin de cette prophétie : « Ceux qui ont été rachetés de la montagne de Sion s’élèveront pour défendre la montagne d’Ésaü et y faire régner le Seigneur. » Il est évident que ceci a été accompli, lorsque ceux qui ont été rachetés de la montagne de Sion, c’est-à-dire les fidèles de la Judée, et surtout les Apôtres, se sont élevés pour défendre la montagne d’Ésaü. Comment l’ont-ils défendue, si ce n’est par la prédication de l’Évangile, en sauvant ceux qui ont cru, et les tirant de la puissance des ténèbres pour les faire passer au royaume de Dieu ? c’est ce qui est ensuite exprimé par ces paroles : « Afin d’y faire régner le Seigneur ». En effet, la montagne de Sion signifie la Judée, où devait commencer le salut et paraître la sainteté, qui est Jésus-Christ ; et la montagne d’Ésaü est l’Idumée, figure de l’Église des Gentils, que ceux qui ont été rachetés de la montagne de Sion ont défendue, comme je viens de le dire, pour y faire régner le Seigneur. Cela était obscur avant de s’accomplir ; mais qui ne le comprend depuis l’événement ?\par
Pour le prophète Nahum, voici comme il parle, ou plutôt comme Dieu parle par lui : « Je briserai, dit-il, les idoles taillées et celles qui sont de fonte, et je les ensevelirai, parce que voici sur les montagnes les pieds légers de ceux qui portent et annoncent la paix. Juda, solennisez vos fêtes et offrez vos vœux ; car vos jours de fête ne vieilliront plus désormais. Tout est consommé, tout est accompli. Celui qui souffle contre votre face et qui délivre de l’affliction va monter. » Qui est monté des enfers et qui a soufflé l’Esprit-Saint contre la face de Juda, c’est-à-dire des Juifs ses disciples ? Je le demande à quiconque a lu l’Évangile. Ceux dont les fêtes se renouvellent, de telle sorte qu’elles ne peuvent plus vieillir, appartiennent au Nouveau Testament, Du reste, nous voyons les idoles des faux dieux détruites par l’Évangile et comme ensevelies dans l’oubli ; et nous reconnaissons cette prophétie encore accomplie en ce point. Quant à Habacuc, de quel autre avènement que celui du Sauveur peut-il parler, quand il dit : « Le Seigneur me répondit : Écrivez nettement cette vision sur le buis, afin que celui qui la lira l’entende. Car cette visions’accomplira en son temps, à la fin, et ce ne sera pas une promesse vaine. S’il tarde à venir, attendez-le en patience, car il va venir sans délai. »
\subsection[{Chapitre XXXII}]{Chapitre XXXII}

\begin{argument}\noindent Prophéties du cantique d’Habacuc.
\end{argument}

\noindent Et dans sa prière ou son cantique, à quel autre qu’au Sauveur dit-il : « Seigneur, j’ai entendu ce que vous m’avez fait entendre, et j’ai été saisi de frayeur ; j’ai contemplé vos ouvrages, et j’ai été épouvanté » ? Qu’est-ce que cela, sinon une surprise extraordinaire à la vue du salut des hommes que Dieu lui avait fait connaître : « Vous serez reconnu au milieu de deux animaux. » Que signifient ces deux animaux ? ce sont les deux Testaments, ou les deux larrons, ou encore Moïse et Élie, qui parlaient avec Jésus sur la montagne où il se transfigura. « Vous serez connu dans la suite des temps. » Cela est trop clair pour avoir besoin qu’on l’explique. « Lorsque mon âme sera troublée, au plus fort de votre colère, vous vous souviendrez de votre miséricorde. » Il dit ceci en la personne des Juifs, parce que, dans le temps qu’ils crucifiaient Jésus-Christ, transportés de fureur, Jésus, se souvenant de sa miséricorde, dit « Mon père, pardonnez-leur, car ils ne savent ce qu’ils font. » Dieu viendra de Théman, et le saint viendra de la montagne couverte d’une ombre épaisse. D’autres, au lieu de {\itshape Théman}, traduisent {\itshape du côté du midi} ; ce qui marque l’ardeur de la charité et l’éclat de la vérité. Pour la montagne couverte d’une ombre épaisse, on peut l’expliquer de différentes façons ; mais il me paraît mieux de l’entendre de la profondeur des Écritures qui contiennent les prophéties de Jésus-Christ. On y trouve en effet beaucoup de choses obscures et cachées qui exercent ceux qui les veulent pénétrer. Or, Jésus-Christ sort de ces ténèbres, quand celui qui le cherche sait l’y découvrir : « Il a fait éclater son pouvoir dans les cieux, et la terre est pleine de ses merveilles. » C’est ce que le Psalmiste dit quelque part : « Mon Dieu, montez au-dessus des cieux et faites éclater votre gloire par toute la terre. Sa splendeur sera aussi vive que la plus vive lumière » : c’est-à-dire que le bruitde son nom fera ouvrir les yeux aux fidèles. « Il tiendra des cornes en ses mains » ; c’est le trophée de la croix. « Il a mis sa force dans la charité » ; cela n’a pas besoin d’explication. « La parole marchera devant lui et le suivra » ; c’est-à-dire qu’il a été prophétisé avant qu’il ne vînt, et annoncé depuis qu’il s’en est allé. « Il s’est arrêté et la terre a été ébranlée » ; il s’est arrêté pour nous secourir, et la terre a été portée à croire. « Il a tourné les yeux sur les nations, et elles ont séché » ; entendez qu’il a eu pitié d’elles et qu’elles ont été touchées de repentir. « Les montagnes ont été mises en poudre par un grand effort » ; c’est-à-dire que l’orgueil des superbes a cédé à la force des miracles. « Les collines éternelles ont été abaissées » ; elles ont été humiliées pour un temps, afin d’être élevées pour l’éternité. « J’ai vu ces entrées éternelles et triomphantes, prix de ses travaux », c’est-à-dire : J’ai reconnu que les travaux de la charité recevront une récompense éternelle. « Les Éthiopiens et les Madianites seront remplis d’étonnement » ; les peuples surpris de tant de merveilles, ceux mêmes qui ne sont pas sous l’empire romain, seront sous celui de Jésus-Christ. « Vous mettrez-vous en colère, Seigneur, contre les fleuves, et déchargerez-vous votre fureur sur la mer ? » C’est qu’il ne vient pas maintenant pour juger le monde, mais pour le sauver. « Vous monterez sur vos chevaux, et vos courses produiront le salut » ; c’est-à-dire : Vos évangélistes vous portent, et vous les conduisez, et votre Évangile procure le salut à ceux qui croient en vous. « Vous banderez votre arc contre les sceptres, dit le Seigneur » ; entendez qu’il menacera de son jugement les rois mêmes de la terre. « La terre s’ouvrira pour recevoir les fleuves dans son sein. » Cela signifie que les cœurs des hommes, à qui il est dit : « Déchirez vos cœurs et non pas vos vêtements », s’ouvriront pour recevoir la parole des prédicateurs et confesser le nom de Jésus-Christ. « Les peuples vous verront et s’affligeront » ; c’est-à-dire qu’ils pleureront, afin d’être bienheureux. « En marchant, vous ferez rejaillir de l’eau de toutes parts » ; vous répandrez de tous côtés des torrents de doctrine en marchant avec vos prédicateurs. « Une voix est sortie du creux de l’abîme » ; c’est-à-dire quele cœur de l’homme, qui est un abîme, n’a pu retenir ce qu’il pensait de vous, et a publié votre gloire partout. « La profondeur de son imagination » ; c’est une explication de ce qui précède ; car cette {\itshape profondeur} est un abîme. Et quand il ajoute : {\itshape de son imagination}, il faut sous-entendre : {\itshape a fait retentir sa voix}, c’est-à-dire a publié ce qu’elle voyait. En effet, l’imagination, c’est une vision que le cœur n’a pu cacher ni retenir, mais qu’il a proclamée à la gloire de Dieu. « Le soleil s’est levé et la lune a gardé son rang » ; Jésus-Christ est monté au ciel, et l’Église a été ordonnée sous son roi. « Vous lancerez vos flèches en plein jour », parce que votre parole sera prêchée publiquement. « Et elles brilleront à la lueur de vos armes. » Il avait dit à ses disciples : « Dites en plein jour ce que je vous dis dans les ténèbres. » — « Vos menaces abaisseront la terre » ; c’est-à-dire, humilieront les hommes. « Et vous abattrez les nations dans votre fureur » ; parce que vous dompterez les superbes, et ferez tomber vos vengeances sur leur tête. « Vous êtes sorti dans l’intention de sauver votre peuple, pour sauver vos christs, et vous avez donné les méchants en proie à la mort » ; cela est clair. « Vous les avez chargés de chaînes » ; par ces chaînes, on peut aussi entendre les heureux liens de la sagesse. « Vous avez mis des entraves à leurs pieds et un carcan à leur cou. Vous les avez rompues avec étonnement » ; il faut sous-entendre {\itshape les chaînes}. De même qu’il a noué celles qui sont bonnes, il a brisé les mauvaises, d’où vient cette parole du psaume : « Vous avez rompu mes chaînes. » — « Avec étonnement » ; c’est-à-dire, avec l’admiration de tous ceux qui ont été témoins de cette merveille. « Les plus grands en seront touchés ; ils seront affamés comme un pauvre qui mange en cachette » ; c’est que quelques-uns des premiers parmi les Juifs, touchés des paroles et des miracles du Sauveur, le venaient trouver, et, pressés par la faim, mangeaient le pain de sa doctrine, mais en secret, parce qu’ils craignaient le peuple, comme le remarque l’Évangile. « Vous avez poussé vos chevaux dans la mer et troublé ses eaux » ; c’est-à-dire les peuples. Les uns ne se convertiraient pas par crainte, et les autres ne persécuteraient pas avec fureur, si tous n’étaient troublés. « J’ai contempléces choses, et mes entrailles ont été émues. La frayeur a pénétré jusque dans mes os, et tout mon être intérieur en a été troublé. » Faisant réflexion sur ce qu’il disait, il en a été lui-même épouvanté. Il prévoyait ce tumulte des peuples, suivi de grandes persécutions contre l’Église, et aussitôt, s’en reconnaissant membre : « Je me reposerai, dit-il, au temps de l’affliction », comme étant de ceux qui, selon la parole de l’Apôtre, se réjouissent en espérance et souffrent constamment l’affliction. « Afin d’aller trouver le peuple qui a été étranger ici-bas comme moi », en s’éloignant de ce peuple méchant qui lui était uni selon la chair, mais qui, n’étant point étranger en ce monde, ne cherchait point la céleste patrie. « Car le figuier ne portera point de fruit, ni la vigne de raisin. Les oliviers tromperont l’attente du laboureur, et la campagne ne produira rien. Les brebis mourront faute de pâturage, et il n’y aura plus de bœufs dans les étables. » Il voyait que cette nation, qui devait mettre à mort Jésus-Christ, perdrait les biens spirituels qu’il a prophétiquement figurés par les temporels ; et parce que la colère du ciel est tombée sur ce peuple, à cause qu’ignorant la justice de Dieu, il a voulu établir la sienne à la place, il ajoute aussitôt : « Mais moi je me réjouirai, Seigneur, je me réjouirai en mon Seigneur et mon Dieu. Le Seigneur mon Dieu est ma force, il affermira mes pas jusqu’à la fin. Il m’élèvera sur les hauteurs, afin que je triomphe par son cantique » ; c’est-à-dire par ce cantique dont le Psalmiste dit quelque chose de pareil en ces termes : « Il a affermi mes pieds sur la pierre, et il a conduit mes pas. Il m’a mis en la bouche un nouveau cantique, un hymne à la louange de notre Dieu. » Celui-là donc triomphe par le cantique du Seigneur, qui se plaît à entendre les louanges de Dieu, et non les siennes, « afin que celui qui se glorifie, ne se glorifie que dans le Seigneur. » Au reste, quelques exemplaires portent : « Je me réjouirai en Dieu mon Jésus » ; ce qui me paraît meilleur que « en Dieu mon Sauveur », parce que Jésus est un nom plein de douceur et de confiance.
\subsection[{Chapitre XXXIII}]{Chapitre XXXIII}

\begin{argument}\noindent Prophéties de Jérémie et de Sophonias touchant Jésus-Christ et la vocation des Gentils.
\end{argument}

\noindent Jérémie est du nombre des grands prophètes, aussi bien qu’Isaïe. Il prophétisa sous Josias, roi de Jérusalem, et du temps d’Ancus Martius, roi des Romains, la captivité des Juifs étant proche, et sa prophétie alla jusqu’au cinquième mois de cette captivité, comme il le dit lui-même. On lui joint Sophonias, l’un des petits prophètes, parce qu’il prophétisa aussi sous Josias, comme lui-même le témoigne ; mais il ne dit point combien-de temps. Jérémie prophétisa, non seulement du temps d’Ancus Martius, mais aussi du temps de Tarquin l’Ancien, cinquième roi de Rome, qui l’était déjà lorsque les Juifs furent emmenés en captivité. Jérémie dit donc de Jésus-Christ : « Le Seigneur, le Christ par qui nous respirons, a été pris pour nos péchés », marquant ainsi en peu de paroles et que Jésus-Christ est notre Seigneur, et qu’il a souffert pour nous, Et dans un autre endroit : « Celui-ci est mon Dieu, et nul autre n’est comparable à lui. Il est l’auteur de toute sagesse, et il l’a donnée à Jacob son serviteur, et à Israël son bien-aimé. Après cela il a été vu sur terre, et il a conversé parmi les hommes. » Quelques-uns n’attribuent pas ce témoignage à Jérémie, mais à Baruch, son scribe, quoique ordinairement on le donne au premier. Le même prophète parlant encore du Messie : « Voici venir le temps, dit le Seigneur, que je ferai sortir du tronc de David un germe glorieux. Il régnera et sera rempli de sagesse et fera justice sur la terre. Alors Juda sera sauvé, et Ismaël demeurera en sûreté, et ils l’appelleront le Seigneur notre justice. » Voici comme il parle de la vocation des Gentils, qui devait arriver et que nous voyons maintenant accomplie : « Seigneur, mon Dieu et mon refuge au temps de l’affliction, les nations viendront à vous des extrémités de la terre, et diront : Il est vrai que nos pères ont adoré de vaines statues qui ne sont bonnes à rien. » Et parce que les Juifs ne devaient pas le connaître et qu’il fallait qu’ils le fissent mourir, le même prophète en parle de la sorte : « Leur esprit est extrêmement pesant : c’est un homme ; qui le connaîtra ? » Voici enfin un dernier passage de Jérémie que j’ai rapporté au dix-septième livre touchant le Nouveau Testament, dont Jésus-Christ est le médiateur : « Voici venir le temps, dit le Seigneur, que je contracterai une nouvelle alliance avec la maison de Jacob, etc. »\par
De Sophonias, qui prophétisait du même temps que Jérémie, je veux citer au moins quelques témoignages sur Jésus-Christ. Voici donc comme il en parle : « Attendez que je ressuscite, dit le Seigneur, car j’ai résolu d’assembler les nations et les royaumes » ; et encore : « Le Seigneur leur sera redoutable ; il exterminera tous les dieux de la terre, et toutes les nations de la terre l’adoreront, chacune en son pays » ; et un peu après : « Je ferai que tous les peuples parleront comme ils doivent ; ils invoqueront tous le nom du Seigneur, et lui seront assujettis. Ils m’apporteront des victimes des bords du fleuve d’Éthiopie. Alors vous n’aurez plus de confusion pour toutes les impiétés que vous avez commises contre moi ; car j’effacerai toute la malice de vos offenses, et il ne vous arrivera plus de vous enorgueillir sur ma montagne sainte. Je rendrai votre peuple doux et modeste, et les restes d’Israël craindront le Seigneur. » C’est de ces restes que l’Apôtre a dit après un autre prophète : « Quand le nombre des enfants d’Israël égalerait le sable de la mer, il n’y aura que les restes qui seront sauvés » ; car les restes de cette nation ont cru au Messie.
\subsection[{Chapitre XXXIV}]{Chapitre XXXIV}

\begin{argument}\noindent Prédictions de Daniel et d’Ézéchiel sur le même sujet.
\end{argument}

\noindent Daniel et Ézéchiel, deux des grands prophètes, prophétisèrent pendant la captivité même de Babylone ; et le premier a été jusqu’à dire combien il s’écoulerait d’années avant l’avènement et la passion du Sauveur. Cette supputation serait longue, et d’ailleurs elle a déjà été faite par d’autres avant nous ; mais voici comme il parle de la puissance et de la gloire du Messie : « J’eus une vision en dormant, où je voyais le fils de l’homme, environné de nuées, s’avançant jusqu’àl’Ancien des jours. Comme on le lui eût présenté, il lui donna puissance, honneur et empire, avec ordre à tous les peuplés, à toutes les tribus et à toutes les langues de lui rendre leurs hommages. Son pouvoir est un pouvoir éternel qui ne finira jamais, et son empire sera toujours florissant. »\par
Ézéchiel, de même, figurant Jésus-Christ par David, parce que c’est à cause de David que Jésus-Christ a pris cette nature charnelle, cette forme d’esclave qu’il a revêtue en venant au monde, d’où vient que, tout en étant fils de Dieu, il est appelé esclave de Dieu, Ézéchiel, dis-je, en parle ainsi au nom de Dieu le Père : « Je susciterai un pasteur pour paître mes troupeaux, mon serviteur David ; et il les fera paître, et il sera leur pasteur. Pour moi, je serai leur Dieu, et mon serviteur David régnera au milieu d’eux. C’est le Seigneur qui l’a dit » ; et dans un autre endroit : « Ils n’auront plus qu’un roi et ne formeront plus deux peuples, ni deux royaumes séparés. Ils ne se souilleront plus d’idolâtrie et d’autres abominations ; et je les tirerai de tous : les lieux où ils m’ont offensé et les purifierai de leurs crimes. Ils seront mon peuple, et je serai leur Dieu, et mon serviteur David sera à tous leur roi et leur pasteur. »
\subsection[{Chapitre XXXV}]{Chapitre XXXV}

\begin{argument}\noindent Prédictions d’Aggée, de Zacharie et de Malachie touchant Jésus-Christ.
\end{argument}

\noindent Restent trois petits prophètes qui ont prophétisé sur la fin de la captivité de Babylone : Aggée, Zacharie et Malachie. Aggée prédit en peu de mots Jésus-Christ et l’Église en ces termes : « Voici ce que dit le Seigneur des armées : Encore un peu de temps, et j’ébranlerai le ciel et la terre, la mer et le continent, et je remuerai toutes les nations ; et celui qui est désiré de tous les peuples viendra. » Cette prophétie est déjà accomplie en partie et le reste s’accomplira à la fin du monde. Dieu ébranla le ciel, quand Jésus-Christ prit chair, par le témoignage que les astres et les anges rendirent à son incarnation. Il émut la terre par le grand miracle de l’enfantement d’une vierge ; il émut la mer et le continent, lorsque le Sauveur fut annoncédans les îles et par tout le monde. Ainsi nous voyons que toutes les nations sont remuées et portées à embrasser la foi. Ce qui suit : « Et à celui qui est désiré de tous les peuples viendra », doit s’entendre de son dernier avènement ; car avant que de souhaiter qu’il vînt, il fallait l’aimer et croire en lui.\par
Zacharie parle ainsi de Jésus-Christ et de l’Église : « Réjouissez-vous, dit-il, fille de Sion, bondissez de-joie, fille de Jérusalem, car voici venir votre roi pour vous justifier et pour vous sauver. Il est pauvre, et vient monté sur une ânesse et sur le poulain d’une ânesse ; mais son pouvoir s’étend d’une mer à l’autre, et depuis les fleuves jusqu’aux confins de la terre. » L’Évangile nous apprend, en effet, en quelle occasion Notre-Seigneur se servit de cette monture, et fait même mention de cette prophétie. Un peu après, parlant à Jésus-Christ même de la rémission des péchés qui devait se faire par son sang : « Et vous aussi, dit-il, vous avez tiré vos captifs de la citerne sans eau, par le sang de votre Testament. » On peut expliquer diversement, et toujours selon la foi, cette citerne sans eau ; mais, pour moi, je pense qu’on doit entendre la misère humaine, qui est comme une citerne sèche et stérile, où les eaux de la justice ne coulent jamais, et qui est pleine de la boue et de la fange du péché. C’est de cette citerne que le Psalmiste dit : « Il m’a tiré d’une malheureuse citerne et d’un abîme de boue. »\par
Malachie, annonçant l’Église que nous voyons fleurir par Jésus-Christ, dit clairement aux Juifs en la personne de Dieu : « Vous ne m’agréez point, et je ne veux point de vos présents. Car depuis le soleil levant jusqu’au couchant, mon nom est grand parmi les nations. On me fera des sacrifices partout, et l’on m’offrira une oblation pure, parce que mon nom est grand parmi les nations, dit le Seigneur. » Ce sacrifice est celui du sacerdoce de Jésus-Christ selon l’ordre de Melchisédech, que nous voyons offrir depuis le soleil levant jusqu’au couchant, tandis qu’on ne peut nier que le sacrifice des Juifs à qui Dieu dit : « Vous ne m’agréez point, et je ne veux point de vos présents », ne soit aboli. Pourquoi donc attendent-ils encore un autre Christ, puisque cette prophétie qu’ils voient accomplie n’a pu s’accomplir que par lui ? Un peu après, le même prophète, parlant encore en la personne de Dieu, dit du Sauveur : « J’ai fait avec lui une alliance de vie et de paix ; je lui ai donné ma crainte, et il m’a craint et respecté. La loi de la vérité était en sa bouche ; il marchera en paix avec moi, et il en retirera plusieurs de leur iniquité. Car les lèvres du grand-prêtre seront les dépositaires de la science ; et ils l’iront consulter sur la loi, parce que c’est l’ange du Seigneur tout-puissant. » Il ne faut pas s’étonner que Jésus-Christ soit appelé l’ange de Dieu ; de même qu’il est esclave à cause de la forme d’esclave en laquelle il est venu parmi les hommes, il est aussi ange à cause de l’Évangile qu’il leur a annoncé ; car Évangile en grec signifie {\itshape bonne nouvelle}, et ange, {\itshape messager}. Aussi le même prophète dit encore de lui : « Je m’en vais envoyer mon ange pour préparer la voie devant moi, et aussitôt viendra dans son temple le Seigneur que vous cherchez, et l’ange du Testament que vous demandez. Le voici qui vient, dit le Seigneur et le Dieu tout-puissant ; et qui pourra supporter l’éclat de sa gloire et soutenir ses regards ? » On trouve prédit en cet endroit le premier et le second avènement de Jésus-Christ ; son premier avènement, lorsqu’il dit : « Et aussitôt le Seigneur viendra dans son temple », c’est-à-dire dans sa chair, dont il est dit dans l’Évangile : « Détruisez ce temple, et je le rétablirai en trois jours » et le second en ces termes : « Le voici qui vient, dit le Seigneur tout-puissant, et qui pourra supporter l’éclat de sa gloire et soutenir ses regards ? » Ces paroles : « Le Seigneur que vous cherchez, et l’ange du Testament que vous demandez », signifient que les Juifs mêmes cherchent le Christ dans les Écritures et désirent l’y trouver. Mais plusieurs d’entre eux, aveuglés par leurs péchés, ne voient pas que celui qu’ils cherchent et qu’ils désirent est déjà venu. Par le Testament, il entend parler du Nouveau, qui contient des promesses éternelles, et non de l’Ancien, qui n’en a que de temporelles ; mais ces promesses temporelles ne laissent pas de troubler beaucoup de personnes faibles qui s’yattachent, et qui, voyant les méchants comblés de ces sortes de biens, ne servent Dieu quepour les obtenir. C’est pourquoi le même prophète, pour distinguer la béatitude éternelledu Nouveau Testament, qui ne sera donnée qu’aux bons, de la félicité temporelle de l’Ancien, qui pour l’ordinaire est commune aux bons et aux méchants, s’exprime ainsi : « Vous avez tenu des discours qui me sont injurieux, dit le Seigneur. Et vous dites : En quoi avons-nous mal parlé de vous ? Vous avez dit : C’est une folie de servir Dieu ; que nous revient-il d’avoir observé ses commandements, et de nous être humiliés en la présence du Seigneur tout-puissant ? N’avons-nous donc pas raison d’estimer heureux les méchants et les ennemis de Dieu, puisqu’ils triomphent dans la gloire et dans l’opulence ? Voilà ce que ceux qui craignaient Dieu ont murmuré tout bas ensemble. Et le Seigneur a vu tout cela et entendu leurs plaintes ; et il a écrit un livre en mémoire de ceux qui le craignent et qui le révèrent. » Ce livre signifie le Nouveau Testament. Mais écoutons ce qui suit : « Et ils seront mon héritage, dit le Seigneur tout-puissant, au jour que j’agirai ; et je les épargnerai comme un père épargne un fils obéissant. Alors vous parlerez un autre langage, et vous verrez la différence qu’il y a entre le juste et l’injuste, entre celui qui sert Dieu et celui qui ne le sert pas. Car voici venir le jour allumé comme une fournaise ardente, et il les consumera. Tous les étrangers et tous les pécheurs seront comme du chaume, et ce jour qui approche les brûlera tous, dit le Seigneur, sans qu’il reste d’eux ni branches, ni racines. Mais, pour vous qui craignez mon nom, le soleil de justice se lèvera pour vous, et vous trouverez une abondance de tous biens à l’ombre de mes ailes. Vous bondirez comme de jeunes taureaux échappés, et vous foulerez aux pieds les méchants, et ils deviendront cendre sous vos pas, au jour que j’agirai, dit le Seigneur tout-puissant. » Ce jour est le jour du jugement, dont nous parlerons plusamplement en son lieu, si Dieu nous en fait la grâce.
\subsection[{Chapitre XXXVI}]{Chapitre XXXVI}

\begin{argument}\noindent D’Esdras et des livres des Maccabées.
\end{argument}

\noindent Après ces trois prophètes, Aggée, Zacharie et Malachie, écrivit Esdras, lorsque le peuple fut délivré de la captivité de Babylone. Mais il passa plutôt pour historien que pour prophète, aussi bien que l’auteur du livre d’Esther où sont rapportées les actions glorieuses de cette femme illustre, qui arrivèrent vers ce temps-là. On peut dire néanmoins qu’Esdras a prophétisé Jésus-Christ dans cette dispute qui s’éleva entre quelques jeunes gens pour savoir quelle était la chose du monde la plus puissante. L’un ayant dit que c’était les rois, l’autre le vin, et le troisième les femmes, qui souvent commandent en rois, ce dernier finit par montrer que c’est la vérité qui l’emporte par-dessus tout. Or, l’Évangile nous apprend que Jésus-Christ est la vérité. Depuis le temps que le temple fut rétabli jusqu’à Aristobule, les Juifs ne furent plus gouvernés par des rois, mais par des princes. La supputation de ces temps ne se trouve pas dans les Écritures canoniques, mais ailleurs, comme dans les Maccabées, que les Juifs ont rejetés comme apocryphes. Mais l’Église est d’un autre sentiment, à cause des souffrances admirables de ces martyrs qui, avant l’incarnation de Jésus-Christ, ont combattu pour la loi de Dieu jusqu’au dernier soupir et enduré des maux étranges et inouïs.
\subsection[{Chapitre XXXVII}]{Chapitre XXXVII}

\begin{argument}\noindent Nos prophètes sont plus anciens que les philosophes.
\end{argument}

\noindent Du temps de nos prophètes, dont les écrits sont maintenant répandus dans le monde entier, il n’y avait point encore de philosophes parmi les Gentils. Du moins ils n’étaient point connus sous ce nom ; car c’est Pythagore qui l’a porté le premier, et il n’a commencé à fleurir que sur la fin de la captivité de Babylone. À plus forte raison les autres philosophes sont-ils postérieurs aux prophètes. En effet, Socrate lui-même, le maître de ceux qui étaient alors le plus en honneur et lepremier de tous pour la morale, ne vient qu’après Esdras dans l’ordre des temps ; peu après parut Platon, qui a surpassé de beaucoup tous les autres disciples de Socrate. Les sept sages mêmes, qui ne s’appelaient pas encore philosophes, et les physiciens qui succédèrent à Thalès dans la recherche des choses naturelles, Anaximandre, Anaximène, Anaxagore, et quelques autres qui ont fleuri avant Pythagore, ne sont pas antérieurs à tous nos prophètes. Thalès, le plus ancien des physiciens, ne parut que sous le règne de Romulus, lorsque les torrents de prophétie qui devaient inonder toute la terre sortirent des sources d’Israël. Il n’y a que les poètes théologiens, Orphée, Linus et Musée, qui soient plus anciens que nos prophètes ; encore n’ont-ils pas devancé Moïse, ce grand théologien, qui a annoncé le Dieu unique et véritable, et dont les écrits tiennent le premier rang parmi les livres canoniques. Ainsi, quant aux Grecs, dont la langue a donné tant d’éclat aux lettres humaines, ils n’ont pas sujet de se glorifier de leur sagesse comme plus ancienne que notre religion, en qui seule se trouve la sagesse véritable. Il est vrai que parmi les Barbares, comme en Égypte, il y avait quelques semences de doctrine avant Moïse ; autrement l’Écriture sainte ne dirait pas qu’il avait été instruit dans toutes les sciences des Égyptiens à la cour de Pharaon ; mais la science même des Égyptiens n’a pas précédé celle de tous nos prophètes, puisque Abraham a aussi cette qualité. Et quelle science pouvait-il y avoir en Égypte, avant qu’Isis, qu’ils adorèrent après sa mort comme une grande déesse, leur eût communiqué l’invention des lettres et des caractères ? Or, Isis était fille d’Inachus, qui régna le premier sur les Argiens, au temps des descendants d’Abraham.
\subsection[{Chapitre XXXVIII}]{Chapitre XXXVIII}

\begin{argument}\noindent Pourquoi l’Église rejette les écrits de quelques prophètes.
\end{argument}

\noindent Si nous remontons plus haut avant le déluge universel, nous trouverons le patriarche Noé, que je puis aussi justement appeler prophète, puisque l’arche même qu’il fit était une prophétie du christianisme. Que dirai-jed’Énoch, le septième des descendants d’Adam ? L’apôtre saint Jude ne dit-il pas dans son épître canonique qu’il a prophétisé ? Que si les écrits de ces personnages ne sont pas reçus coin me canoniques par les Juifs, non plus que par nous, cela ne vient que de leur trop grande antiquité qui les a rendus suspects. Je sais bien qu’on produit quelques ouvrages dont l’authenticité ne paraît pas douteuse à ceux qui croient vrai tout ce qui leur plaît ; mais l’Église ne les reçoit pas, non qu’elle rejette l’autorité de ces grands hommes qui ont été si agréables à Dieu, mais parce qu’elle ne croit pas que ces ouvrages soient de leur main. Il ne faut pas trouver étrange que des écrits si anciens soient suspects, puisque, dans l’histoire des rois de Juda et d’Israël, il est fait mention de plusieurs circonstances qu’on chercherait en vain dans nos Écritures canoniques et qui se trouvent en d’autres prophètes dont les noms-ne sont pas inconnus et dont cependant les ouvrages n’ont point été reçus au nombre des livres canoniques. J’avoue que j’en ignore la raison ; à moins de dire que ces prophètes ont pu écrire certaines choses comme hommes et sans l’inspiration du Saint-Esprit, et que c’est celles-là que l’Église ne reçoit pas dans son canon pour faire partie de la religion, bien qu’elles puissent être d’ailleurs utiles et véritables. Quant aux ouvrages qu’on attribue aux prophètes et qui contiennent quelque chose de contraire aux Écritures canoniques, cela seul suffit pour les convaincre de fausseté.
\subsection[{Chapitre XXXIX}]{Chapitre XXXIX}

\begin{argument}\noindent La langue hébraïque a toujours eu des caractères.
\end{argument}

\noindent Il ne faut donc pas s’imaginer, comme font quelques-uns, que la langue hébraïque seule ait été conservée par Héber, qui a donné son nom aux Hébreux, et qu’elle soit passée de lui à Abraham, tandis que les caractères-hébreux n’auraient commencé qu’à la loi qui fut donnée à Moïse. Il est bien plus croyable que cette langue a été conservée avec ses caractères dès les époques primitives. En effet, nous voyons Moïse établir certains hommes pour enseigner les lettres, avant que la loi n’eût été dénuée, et l’Écriture les appelle des introducteurs aux lettres, parce qu’ils les introduisaient dans l’esprit de leurs disciples, ou plutôt, parce qu’ils introduisaient leurs disciples jusqu’à elles. Aucune nation n’a donc droit de se vanter de sa science, comme étant plus ancienne que nos patriarches et nos prophètes, puisque l’Égypte même, qui a cou-turne de se glorifier de l’antiquité de ses lumières, ne peut prétendre à cet avantage. Personne n’oserait dire que les Égyptiens aient été bien savants avant l’invention des caractères, c’est-à-dire avant Isis. D’ailleurs, cette science dont on a fait tant de bruit et qu’ils appelaient sagesse, qu’était-elle autre chose que l’astronomie, et peut-être quelques autres sciences analogues, plus propres à exercer l’esprit qu’à rendre l’homme véritablement sage ? Et quant à la philosophie, qui se vante d’apprendre aux hommes le moyen de devenir heureux, elle n’a fleuri en ce pays que vers le temps de Mercure Trismégiste, longtemps, il est vrai, avant les sages au les philosophes de la Grèce, mais toutefois après Abraham, Isaac, Jacob, Joseph, et même après Moïse ; car Atlas, ce grand astrologue, frère de Prométhée et aïeul maternel du grand Mercure, de qui Mercure Trismégiste fut petit-fils, vivait encore lorsque Moïse naquit.
\subsection[{Chapitre XL}]{Chapitre XL}

\begin{argument}\noindent Folie et vanité des Égyptiens, qui font leur science ancienne de cent mille ans.
\end{argument}

\noindent C’est donc en vainque certains discoureurs, enflés d’une sotte présomption, disent qu’il y a plus de quatre cent mille ans que l’astrologie est connue en Égypte. Et de quel livre ont-ils tiré ce grand nombre d’années, eux qui n’ont appris à lire de leur Isis que depuis environ deux mille ans ? C’est du moins ce qu’assure Varron, dont l’autorité n’est pas peu considérable, et cela s’accorde assez bien avec l’Écriture sainte. Du moment donc que l’on compte à peine six mille ans depuis la création du premier homme, ceux qui avancent des opinions si contraires à une vérité reconnue ne méritent-ils pas plutôt des railleries que des réfutations ? Aussi bien, à qui nous en pouvons-nous mieux rapporter, pour les choses passées, qu’à celui qui a prédit deschoses à venir que nous voyons maintenant accomplies ? La diversité même qui se rencontre entre les historiens sur ce sujet ne nous donne-t-elle pas lieu d’en croire plutôt ceux qui ne sont pas contraires à notre Histoire sacrée ? Quand les citoyens de la cité du monde qui sont répandus par toute la terre voient des hommes très savants, à peu près d’une égale autorité, qui ne conviennent pas en des choses de fait fort éloignées de notre temps, ils ne savent à qui donner créance. Mais pour nous, qui sommes appuyés sur une autorité divine en ce qui concerne l’histoire de notre religion, nous ne doutons point que tout ce qui contredit la parole de Dieu ne soit très faux, quoi qu’il faille penser à d’autres égards de la valeur des histoires profanes, question qui nous met peu en peine, parce que, vraies ou fausses, elles ne servent de rien pour nous rendre meilleurs ni plus heureux.
\subsection[{Chapitre XLI}]{Chapitre XLI}

\begin{argument}\noindent Les écrivains canoniques sont autant d’accord entre eux que les philosophes le sont peu.
\end{argument}

\noindent Mais laissons les historiens pour demander aux philosophes, qui semblent n’avoir eu d’autre but dans leurs études que de trouver le moyen d’arriver à la félicité, pourquoi ils ont eu tant d’opinions différentes, sinon parce qu’ils ont procédé dans cette recherche comme des hommes et par des raisonnements humains ? Je veux que la vaine gloire ne les ait pas tous déterminés à se départir de l’opinion d’autrui, afin de faire éclater la supériorité de leur sagesse et de leur génie et d’avoir une doctrine en propre ; j’admets que quelques-uns, et même un grand nombre, n’aient été animés que de l’amour de la vérité ; que peut la misérable prudence des hommes pour parvenir à la béatitude, si elle n’est guidée par une autorité divine ? Voyez nos auteurs, à qui l’on attribue justement une autorité canonique : il n’y a pas entre eux la moindre différence de sentiment. C’est pourquoi il ne faut pas s’étonner qu’on les ait crus inspirés de Dieu, et que cette créance, au lieu de se renfermer entre un petit nombre de personnes disputant dans une école, se soit répandue parmi tant de peuples, dans les champs comme dans les villes, parmi les savants comme parmi les ignorants. Du reste, il ne fallait pas qu’il y eût beaucoup de prophètes, de peur que leur grand nombre n’avilît ce que la religion devait consacrer, et, d’un autre côté, ils devaient être en assez grand nombre pour que leur parfaite conformité fût un sujet d’admiration. Lisez cette multitude de philosophes dont nous avons les ouvrages ; je ne crois pas qu’on en puisse trouver deux qui soient d’accord en toutes choses ; mais je ne veux pas trop insister là-dessus, de peur de trop longs développements. Je demanderai cependant si jamais cette cité terrestre, abandonnée au culte des démons, a tellement embrassé les doctrines d’un chef d’école qu’elle ait condamné toutes les autres ? N’a-t-on pas vu en vogue dans la même ville d’Athènes, et les Épicuriens qui soutiennent que les dieux ne prennent aucun soin des choses d’ici-bas, et les Stoïciens qui veulent au contraire que le monde soit gouverné et maintenu par des divinités protectrices ? Aussi, je m’étonne qu’Anaxagore ait été condamné pour avoir dit que le soleil était une pierre enflammée et non pas un dieu, tandis qu’Épicure a vécu en tout honneur et toute sécurité dans la même ville, quoiqu’il ne niât pas seulement la divinité du soleil et des autres astres, mais qu’il soutînt qu’il n’y avait ni Jupiter ni aucune autre puissance dans le monde à qui les hommes dussent adresser leurs vœux. N’est-ce pas à Athènes qu’Aristippe mettait le souverain bien dans la volupté du corps, au lieu qu’Antisthène le plaçait dans la vigueur de l’âme, tous deux philosophes célèbres, tous deux disciples de Socrate, et qui pourtant faisaient consister la souveraine félicité en des principes si opposés ? De plus, le premier disait que le sage doit fuir le gouvernement de la république, et le second, qu’il y doit prétendre, et tous deux avaient des sectateurs. Chacun combattait avec sa troupe pour son opinion ; car on discutait au grand jour, sous le vaste et célèbre Portique, dans les gymnases, dans les jardins, dans les lieux publics, comme dans les demeures particulières. Les uns soutenaient qu’il n’y a qu’un monde, les autres qu’il y en a plusieurs ; les uns que le monde a commencé, les autres qu’il est sans commencement ; les uns qu’il doit finir, les autres qu’il durera toujours ; ceux-ci qu’il est gouverné par une providence, ceux-là qu’il n’a d’autre guide que la fortune et le hasard. Quelques-uns voulaient que l’âme de l’homme fût immortelle, d’autres la faisaient mortelle ; et de ceux qui étaient pour l’immortalité, les uns disaient que l’âme passe dans le corps des bêtes par certaines révolutions, les autres rejetaient ce sentiment ; parmi ceux au contraire qui la faisaient mortelle, les uns prétendaient qu’elle meurt avec le corps, les autres qu’elle vit après, plus ou moins de temps, mais qu’à la fin elle meurt. Celui-ci mettait le souverain bien dans le corps, celui-là dans l’esprit, un troisième dans tous les deux, tel autre y ajoutait les biens de la fortune. Quelques-uns disaient qu’il faut toujours croire le rapport des sens, les autres pas toujours, les autres jamais.\par
Quel peuple, quel sénat, quelle autorité publique de la cité de la terre s’est jamais mise en peine de décider entre tant d’opinions différentes, pour approuver les unes et condamner les autres ? Ne les a-t-elle pas reçues toutes indifféremment, quoiqu’il s’agisse en tout ceci, non pas de quelque morceau de terre ou de quelque somme d’argent, mais des choses les plus importantes, de celles qui décident du malheur ou de la félicité des hommes ? Car, bien qu’on enseignât dans les écoles des philosophes quelques vérités, l’erreur s’y débitait aussi en toute licence ; de sorte que ce n’est pas sans raison que cette cité se nomme Babylone, c’est-à-dire confusion. Et il importe peu au diable, qui en est le roi, que les hommes soient dans des erreurs contraires, puisque leur impiété les rend tous également ses esclaves.\par
Mais il en est tout autrement de ce peuple, de cette cité, de ces Israélites à qui la parole de Dieu a été confiée ; ils n’ont jamais confondu les faux prophètes avec les véritables, reconnaissant pour les auteurs des Écritures sacrées ceux qui étaient en tout parfaitement d’accord. Ceux-là étaient leurs philosophes, leurs sages, leurs théologiens, leurs prophètes, leurs docteurs. Quiconque a vécu selon leurs maximes n’a pas vécu selon l’homme, mais selon Dieu qui parlait en eux. S’ils défendent l’impiété, c’est Dieu qui la défend. S’ils commandent d’honorer son père et sa mère, c’est Dieu qui le commande. S’ils disent : « Vous ne serez point adultère, ni homicide, ni voleur », ce sont autant d’oracles du ciel. Toutes les vérités qu’un certain nombre de philosophes ont aperçues parmi tant d’erreurs, et qu’ils ont tâché de persuader avec tant de peine, comme par exemple, que c’est Dieu qui a créé le monde et qui le gouverne par sa providence, tout ce qu’ils ont écrit de la beauté de la vertu, de l’amour de la patrie, de l’amitié, des bonnes œuvres et de toutes les choses qui concernent les mœurs, ignorant au surplus et la fin où elles doivent tendre et le moyen d’y parvenir, tout cela, dis-je, a été prêché aux membres de la Cité du ciel par la bouche des prophètes, sans arguments et sans disputes, afin que tout homme initié à ces vérités ne les regardât pas comme des inventions de l’esprit humain, mais comme la parole de Dieu même.
\subsection[{Chapitre XLII}]{Chapitre XLII}

\begin{argument}\noindent Par quel conseil de la divine Providence l’Ancien Testament a été traduit de l’hébreu en grec pour être connu des Gentils.
\end{argument}

\noindent Un des Ptolémées, roi d’Égypte, souhaita de connaître nos saintes Écritures. Car après la mort d’Alexandre le Grand, qui avait subjugué toute l’Asie et presque toute la terre, et conquis même la Judée, ses capitaines ayant démembré son empire, l’Égypte commença à avoir des Ptolémées pour rois. Le premier de tous fut le fils de Lagus, qui emmena captifs en Égypte beaucoup de Juifs. Mais Ptolémée Philadelphe, son successeur, les renvoya tous en leur pays, avec des présents pour le temple, et pria le grand-prêtre Éléazar de lui donner l’Écriture sainte pour la placer dans sa fameuse bibliothèque. Éléazar la lui ayant envoyée, Ptolémée lui demanda des interprètes pour la traduire en grec ; de sorte qu’on lui donna septante et deux personnes, six de chaque tribu, qui entendaient parfaitement l’une et l’autre langue, c’est-à-dire le grec et l’hébreu. Mais la coutume a voulu qu’on appelât cette version la version des Septante. On dit qu’ils s’accordèrent tellement dans cette traduction que, l’ayant faite chacun à part, selon l’ordre de Ptolémée, qui voulait éprouver par là leur fidélité, ils se rencontrèrent en tout, tant pour le sens que pour l’arrangement des paroles, si bien qu’il semblait qu’il n’y eût qu’un seul traducteur. Et il ne faut pas trouver cela étrange, puisqu’en effet ils étaient tous inspirés d’un même Esprit, Dieu ayant voulu, par un si grand miracle, rendre l’autorité de ces Écritures vénérable aux Gentils qui devaient croire un jour, comme cela est en effet arrivé.
\subsection[{Chapitre XLIII}]{Chapitre XLIII}

\begin{argument}\noindent Prééminence de la version des Septante sur toutes les autres.
\end{argument}

\noindent Bien que d’autres aient traduit en grec l’Écriture sainte, comme Aquila, Symmaque, Théodotion, et un auteur inconnu, dont la traduction, à cause de cela, s’appelle la Cinquième, l’Église a reçu la version des Septante comme si elle était seule, en sorte que la plupart des Grecs chrétiens ne savent pas même s’il y en a d’autres. C’est sur cette version qu’a été faite celles dont les Églises latines se servent, quoique de notre temps le savant prêtre Jérôme, très versé dans les trois langues, l’ait traduite en latin sur l’hébreu, Les Juifs ont beau reconnaître qu’elle est très fidèle, et soutenir au contraire que les Septante se sont trompés en beaucoup de points, cela n’empêche pas les Églises de Jésus-Christ de préférer celle-ci, parce qu’en supposant même qu’elle n’eût pas été exécutée d’une manière miraculeuse, l’autoritéde tant de savants hommes qui l’auraient faite de concert entre eux serait toujours préférable à celle d’un particulier. Mais la façon si extraordinaire dont elle a été composée portant des marques visibles d’une assistance divine, quelque autre version qu’on en fasse sur l’hébreu, elle doit être conforme aux Septante, ou si elle en paraît différente sur certaines choses, il faut croire qu’en ces endroits il y a quelque grand mystère caché dans celle des Septante. Le même Esprit qui était dans les prophètes, lorsqu’ils composaient l’Écriture, animait les Septante, lorsqu’ils l’interprétaient. Ainsi, il a fort bien pu tantôt leur faire dire autre chose que ce qu’avaient dit les Prophètes ; car cette différence n’empêche pas l’unité de l’inspiration divine, tantôt leur faire dire autrement la même chose, de sorte que ceux qui savent bien entendre y trouvent toujours le même sens. Il a pu même passer ou ajouter quelque chose, pour montrer que tout cela s’est fait par une autorité divine, et que ces interprètes ont plutôt suivi l’Esprit intérieur qui les guidait, qu’ils ne se sont assujettis à la lettre qu’ils avaient sous les yeux. Quelques-uns ont cru qu’il fallait corriger la version grecque des Septante sur les exemplaires hébreux : toutefois, ils n’ont pas osé retrancher ce que les Septante avaient de plus que l’hébreu ; ils ont seulement ajouté ce qui était de moins dans les Septante, et l’ont marqué avec de certains signes, en forme d’étoiles qu’on nomme astérisques, au commencement des versets. Ils ont marqué de même avec de petits traits horizontaux, semblables aux signes des onces, ce qui n’est pas dans l’hébreu et se trouve dans les Septante, et l’on voit encore aujourd’hui beaucoup de ces exemplaires, tant grecs que latins, marqués de la sorte. Pour les choses qui ne sont ni omises ni ajoutées dans la version des Septante, mais qui sont seulement dites d’une autre façon que dans l’hébreu, soit qu’elles fassent un sens manifestement identique, soit que le sens diffère en apparence, quoique concordant en réalité, on ne les peut trouver qu’en conférant le grec avec l’hébreu. Si donc nous ne considérons les hommes qui ont travaillé à ces Écritures que comme les organes de l’Esprit de Dieu, nous dirons pour les choses qui sont dans l’hébreu et qui ne setrouvent pas dans les Septante, que le Saint-Esprit ne les a pas voulu dire par ces prophètes, mais par les autres ; et pour celles au contraire qui sont dans les Septante et qui ne sont pas dans l’hébreu, que le même Saint-Esprit a mieux aimé les dire par ces derniers prophètes que par les premiers, mais nous les regarderons tous comme des prophètes. C’est de cette sorte qu’il â dit-une chose par Isaïe, et une autre par Jérémie, ou la même chose autrement par celui-ci et par celui-là. Et quand enfin les mêmes choses se trouvent également dans l’hébreu et dans les Septante, c’est que le Saint-Esprit s’est voulu servir des uns et des autres pour les dire, car, comme il a assisté les premiers pour établir entre leurs prédictions une concordance parfaite, il a conduit la plume des seconds pour rendre leurs interprétations identiques,
\subsection[{Chapitre XLIV}]{Chapitre XLIV}

\begin{argument}\noindent Conformité de la version des Septante et de l’hébreu.
\end{argument}

\noindent Quelqu’un fera cette objection : Comment saurai-je ce que Jonas a dit en effet aux Ninivites et s’il leur a dit : « Encore trois jours », ou bien : « Encore quarante jours, et Ninive sera détruite » ? Il est clair en effet que ce prophète, envoyé pour menacer Ninive d’une ruine imminente, n’a pu assigner deux termes différents et qui s’excluent l’un l’autre. Si l’on me demande lequel des deux il a marqué, je crois que c’est plutôt quarante jours, comme le porte l’hébreu. Car les Septante, qui sont venus longtemps après, ont très bien pu attribuer à Jonas d’autres paroles, lesquelles toutefois se rapportent parfaitement au sujet et expriment, quoique en d’autres termes, un seul et même sens, et cela pour inviter le lecteur à s’élever au-dessus de l’histoire et à, chercher ce qu’elle signifie, sans mépriser d’ailleurs en rien ni l’autorité des Septante ni celle de l’hébreu, Les événements prédits par Jonas se sont effectivement accomplis dans Ninive, mais ils en figuraient d’autres qui ne convenaient pas à cette ville ; tout comme il est vrai que ce prophète fut effectivement trois jours dans le ventre de la baleine, et néanmoins il figurait un autre personnage qui devait demeurer dans l’enfer pendant ce temps, et celui-là est le Seigneurde tous les prophètes. C’est pourquoi, si par Ninive était figurée l’Église des Gentils, qui a été détruite en quelque façon par la pénitence, en ce qu’elle n’est plus ce qu’elle était, comme c’est Jésus-Christ qui a opéré en elle ce changement, c’est lui-même qui est signifié, soit par les trois jours, soit par les quarante ; par les quarante, parce qu’il demeura cet espace de temps avec ses disciples après sa résurrection, avant que de monter au ciel ; et par les trois jours, parce qu’il ressuscita le troisième jour. Ainsi il semble que les Septante aient voulu réveiller l’esprit du lecteur qui se serait arrêté au récit historique, pour le porter à approfondir la prophétie qu’il contient, et lui aient dit en quelque sorte Cherchez dans les quarante jours celui-là même en qui vous pourrez aussi trouver les trois jours ; et vous verrez que l’un des deux termes assignés s’est accompli dans son ascension, et l’autre dans sa résurrection. — Il a donc fort bien pu être désigné par l’un et par l’autre nombre dans le prophète Jonas d’une façon, dans la prophétie des Septante de l’autre, mais toujours par un seul et même Esprit. J’abrège, et ne veux pas rapporter beaucoup d’autres exemples où l’on croirait que les Septante se sont éloignés de la vérité hébraïque, quoique, bien entendu, on les y trouve parfaitement conformes. Aussi les Apôtres se sont-ils servis indifféremment de l’hébreu et de la version des Septante, en quoi j’ai cru devoir les imiter, parce que ce n’est qu’une même autorité divine. Mais poursuivons, selon nos forces, l’œuvre que nous avons à cœur d’accomplir.
\subsection[{Chapitre XLV}]{Chapitre XLV}

\begin{argument}\noindent Décadence des Juifs depuis la captivité de Babylone.
\end{argument}

\noindent Du moment que les Juifs cessèrent d’avoir des prophètes, ils devinrent pires qu’ils n’étaient, bien que ce fût le temps où, la captivité de Babylone ayant pris fin et le temple étant rétabli, ils se flattaient de devenir meilleurs. C’est ainsi que ce peuple charnel entendait cette prophétie d’Aggée : « La gloire de cette dernière maison sera plus grande que celle de la première. » Mais ce qui précède fait bien voir que le prophète parle ici du Nouveau Testament, lorsque,promettant clairement le Christ, il dit : « J’ébranlerai toutes ces nations, et celui que tous les peuples désirent viendra. » Les Septante, de leur autorité de prophètes, ont rendu ces paroles dans un autre sens qui convient mieux au corps qu’à la tête, c’est-à-dire à l’Église qu’à Jésus-Christ. « Ceux, disent-ils, que le Seigneur a élus parmi toutes les nations, viendront » ; suivant cette parole du Sauveur dans l’Évangile : « Il y en a beaucoup d’appelés, mais peu d’élus. » En effet, c’est de ces élus des nations, comme de pierres vivantes, que la maison de Dieu est bâtie par le Nouveau Testament, maison bien plus illustre que le temple construit par Salomon et rétabli après la captivité de Babylone. Les Juifs ne virent donc plus de prophètes depuis ce temps-là, et eurent même beaucoup à souffrir des rois étrangers st des Romains, afin qu’on ne crût pas que cette prophétie d’Aggée eût été accomplie par le rétablissement du temple.\par
Peu de temps après, ils furent assujettis à l’empire d’Alexandre ; et quoique ce prince n’ait pas ravagé leur pays, parce qu’ils n’osèrent lui résister, toutefois {\itshape la gloire de cette maison}, pour parler comme le prophète, n’était pas alors si grande que sous la libre domination de ses rois. Il est vrai qu’Alexandre immola des victimes dans le temple de Dieu, mais il le fit moins par une véritable piété que par une vaine superstition, croyant qu’il devait aussi adorer le Dieu des Juifs comme il adorait les autres dieux. Après la mort d’Alexandre, Ptolémée, fils de Lagus, emmena les Juifs captifs en Égypte, et ils ne retournèrent en Judée que sous Ptolémée-Philadelphe, son successeur, celui qui fit traduire l’Écriture par les Septante. Ensuite ils eurent sur les bras les guerres rapportées aux livres des Maccabées. Ils furent vaincus par Ptolémée Épiphane, roi d’Alexandrie, et contraints par les cruautés inouïes d’Antiochus, roi de Syrie, d’adorer les idoles ; leur temple fut souillé de toutes sortes d’abominations, jusqu’à ce qu’il fût purifié de toute cette idolâtrie par la valeur de Judas Maccabée, grand capitaine, qui défit les chefs de l’armée d’Antiochus.\par
Peu de temps après, un certain Alcimus usurpa la souveraine sacrificature, quoiqu’il ne fût pas de la lignée sacerdotale, ce qui était un attentat. Cinquante ans s’écoulent, pendant lesquels, malgré quelques succès heureux, les Juifs ne furent pas en paix ; Aristobule prend le diadème et se fait roi et grand prêtre tout ensemble. C’est le premier roi que les Juifs aient eu après la captivité de Babylone, tous les autres depuis ce temps-là n’ayant porté que la qualité de chefs ou de princes. Alexandre succéda à Aristobule dans le sacerdoce et la royauté, et l’on dit qu’il maltraita fort ses sujets. Sa femme Alexandra fut après lui reine des Juifs ; et depuis, leurs maux augmentèrent toujours. Comme ses deux fils Aristobule et Hircan se disputaient l’empire, ils attirèrent les forces romaines contre les Juifs, parce que Hircan leur demanda secours contre son frère. Rome alors avait déjà dompté l’Afrique et la Grèce, et porté ses armes victorieuses en beaucoup d’autres parties du monde, en sorte qu’elle était comme accablée du poids de sa propre grandeur. Elle avait été tourmentée de furieuses séditions, qui furent suivies de la révolte des alliés et ensuite de guerres civiles, et les forces de la république étaient tellement abattues qu’elle ne pouvait encore subsister longtemps. Pompée, l’un des plus grands capitaines de Rome, étant entré en Judée, prit la ville de Jérusalem, ouvrit le temple comme vainqueur, et entra dans le Saint des saints ; ce qui n’était permis qu’au grand prêtre. Après avoir confirmé le pontificat d’Hircan et établi Antipater gouverneur de la Judée, il emmena avec lui Aristobule prisonnier. Depuis ce temps, les Juifs devinrent tributaires des Romains ; ensuite Cassius pilla le temple, et quelques années après, les Juifs eurent même pour roi un étranger qui fut Hérode, sous le règne duquel naquit le Messie. Le temps prédit par le patriarche Jacob en ces termes : « Les princes ne manqueront point dans la race de Juda, jusqu’à ce que vienne celui à qui la promesse est faite ; et il sera l’attente des nations » ; ce temps, dis-je, était déjà accompli. Les Juifs ne manquèrent donc point de rois de leur nation jusqu’à cet Hérode ; et ainsi, le moment était venu où celui en qui reposent les promesses du Nouveau Testament et qui est l’attente des nations devait paraître dans le monde. Or, les nations ne pourraient pas attendre, comme elles font, cet événement suprême où tous les hommes seront jugés par Jésus-Christ dans l’éclat de sa puissance, si elles ne croyaient à cet autre avènement où il a daigné, dans l’humilité de sa patience, subir le jugement des hommes.
\subsection[{Chapitre XLVI}]{Chapitre XLVI}

\begin{argument}\noindent Naissance du Sauveur et dispersion des Juifs par toute la terre.
\end{argument}

\noindent Hérode régnait en Judée, et l’empereur Auguste avait donné la paix au monde, après que toute la constitution de la république eut été changée, quand le Messie, selon la parole du prophète cité tout à l’heure, naquit à Bethléem, ville de Juda : homme visible, né humainement d’une vierge comme homme, Dieu caché, divinement engendré de Dieu le Père. Un autre prophète l’avait prédit en ces termes : « Voici venir le temps qu’une vierge concevra ou enfantera un fils qui sera appelé Emmanuel, c’est-à-dire Dieu avec nous. » Il fit plusieurs miracles pour rendre sa divinité manifeste, et l’Évangile en rapporte quelques-uns qu’elle croit suffisants pour la prouver. Le premier est celui de sa naissance ; le dernier est celui de sa résurrection et de son ascension au ciel. Peu après, les Juifs, qui l’avaient fait mourir et qui n’avaient pas voulu croire en lui, parce qu’il fallait qu’il mourût et qu’il ressuscitât, ont été chassés de leur pays par les Romains et dispersés dans toute la terre. Et ainsi, par leurs propres Écritures, ils nous rendent ce témoignage, que nous n’avons pas inventé les prophéties qui parlent de Jésus-Christ. Plusieurs même d’entre eux les ayant considérées avant la passion, mais surtout après la résurrection, ont cru en lui, et c’est d’eux qu’il est dit : « Quand le nombre des enfants d’Israël égalerait le sable de la mer, les restes seront sauvés. » Les autres ont été aveuglés, suivant cette prédiction : « Qu’en récompense, leur table devienne pour eux un piège et une pierre d’achoppement ; que leurs yeux soient obscurcis, afin qu’ils ne voient point, et faites que leur dos soit toujours courbé. » Ainsi, par cela même qu’ils n’ajoutent point foi à nos Écritures, les leurs s’accomplissent en eux, encore qu’ils soient assez aveugles pour ne le pas voir. Quelqu’un dira peut-être que les chrétiens ont supposé les prophéties des sibylles touchant Jésus-Christ, ainsi que quelques autres qui ne sont pas d’origine juive ; mais, sans nous arrêter à celles-là, nous nous contentons de celles que nos ennemis nous fournissent malgré eux, et dont ils sont eux-mêmes les dépositaires ; d’autant mieux que nous y trouvons prédite cette dispersion même dont les Juifs nous fournissent le témoignage éclatant. Chaque jour, ils peuvent lire dans les psaumes cette prophétie : « C’est mon Dieu ; il me préviendra par sa miséricorde, Mon Dieu m’a dit en me parlant de mes ennemis : Ne les tuez pas, de peur qu’ils n’oublient votre loi ; mais dispersez-les par votre puissance. » Dieu donc a fait voir sa miséricorde à l’Église dans les Juifs ses ennemis, parce que, comme dit l’Apôtre : « Leur crime est le salut des Gentils. » Et il ne les a pas tués, c’est-à-dire qu’il n’a pas entièrement détruit le judaïsme, de peur qu’ayant oublié la loi de Dieu, ils ne nous pussent rendre le témoignage dont nous parlons. Aussi ne s’est-il pas contenté de dire : « Ne les tuez pas, de peur qu’ils n’oublient votre loi » ; mais il ajoute : « Dispersez-les. » Si avec ce témoignage des Écritures ils demeuraient dans leur pays, sans être dispersés partout, l’Église, qui est répandue dans le monde entier, ne les pourrait pas avoir de tous côtés pour témoins des prophéties qui regardent Jésus-Christ.
\subsection[{Chapitre XLVII}]{Chapitre XLVII}

\begin{argument}\noindent Si, avant l’incarnation de Jésus-Christ d’autres que les Juifs ont appartenu à la Jérusalem céleste.
\end{argument}

\noindent Si d’autres que des Juifs ont prophétisé le Messie, c’est pour nous un surcroît de preuves ; mais nous n’avons pas besoin de leur témoignage. En effet, nous ne l’alléguons que pour montrer qu’il y a eu probablement parmi les autres peuples des hommes à qui ce mystère a été révélé, et qui ont été poussés à le prédire, soit qu’ils aient participé à la même grâce que les prophètes hébreux, soit qu’ils aient été instruits par les démons, que nous savons avoir confessé Jésus-Christ présent, tandis que les Juifs ne le connaissaient pas. Aussi je ne crois pas que les Juifs mêmes osent soutenir que nul, hors de leur race, n’a servi le vrai Dieu depuis l’élection de Jacob et la réprobation d’Ésaü. À la vérité, il n’y a point eud’autre peuple que le peuple israélite qui ait été proprement appelé le peuple de Dieu ; mais ils ne peuvent nier qu’il n’y ait eu parmi les autres nations quelques hommes dignes d’être appelés de véritables Israélites, en tant que citoyens de la céleste patrie. S’ils le nient, il est aisé de les convaincre par l’exemple de Job, cet homme saint et admirable, qui n’était ni juif ni prophète, mais un étranger originaire d’Idumée, à qui l’Écriture néanmoins accorde ce glorieux témoignage que nul homme de son temps ne lui était comparable pour la piété. Bien que l’histoire ne dise pas en quel temps il vivait, nous conjecturons par son livre placé par les Juifs entre les canoniques, à cause de son excellence, qu’il est venu au monde environ trois générations après le patriarche Jacob. Or, je ne doute point que ce ne soit un effet de la providence de Dieu de nous avoir appris par l’exemple de Job qu’il a pu y avoir parmi les autres peuples des membres de la Jérusalem spirituelle. Mais il faut croire que cette grâce n’a été faite qu’à ceux à qui l’unique médiateur entre Dieu et les hommes, Jésus-Christ homme, a été révélé, et que son incarnation leur était prédite avant qu’elle arrivât, comme elle nous a été annoncée depuis qu’elle est arrivée, en sorte qu’une seule et même foi conduise par lui à Dieu tous ceux qui sont prédestinés pour être sa cité, sa maison et son temple. Quant aux autres prophéties de Jésus-Christ qu’on produit d’ailleurs, on peut penser que les chrétiens les ont inventées. C’est pourquoi il n’est rien de plus fort contre tous ceux qui voudraient révoquer en doute notre foi, ni de plus propre pour nous y affermir, si nous prenons les choses comme il faut, que les prophéties de Jésus-Christ tirées des livres des Juifs, qui, ayant été arrachés de leur pays et dispersés dans tout le monde pour servir de témoignage à la foi de l’Église, ont contribué à la faire partout fleurir.
\subsection[{Chapitre XLVIII}]{Chapitre XLVIII}

\begin{argument}\noindent La prophétie d’Aggée touchant la seconde maison de Dieu, qui doit être plus illustre que la première, ne doit pas s’entendre du temple de Jérusalem, mais de l’Église.
\end{argument}

\noindent Cette maison de Dieu, qui est l’Église, est bien plus auguste que la première, bâtie debois précieux et toute couverte d’or. La prophétie d’Aggée n’a donc pas été accomplie par le rétablissement de ce temple, puisque, depuis le temps où il fut rebâti, il fut moins fameux que du temps de Salomon, On peut dire même qu’il perdit beaucoup de sa gloire, d’abord par les prophéties qui vinrent à cesser, et ensuite par les diverses calamités qui affligèrent les Juifs jusqu’à leur entière désolation. Il en est tout autrement de cette nouvelle maison qui appartient au Nouveau Testament ; elle est d’autant plus illustre qu’elle est composée de pierres meilleures, de pierres vivantes, c’est-à-dire des fidèles renouvelés par le baptême. Mais elle a été figurée par le rétablissement du temple de Salomon, parce qu’en langage prophétique ce rétablissement signifie le Testament nouveau. Ainsi, lorsque Dieu a dit par le prophète dont nous parlons : « Je donnerai la paix en ce lieu », comme ce lieu désignait l’Église qui devait être bâtie par Jésus-Christ, on doit entendre : J’établirai la paix dans le lieu que celui-ci figure. En effet, toutes les choses figuratives semblent en quelque sorte tenir la place des choses figurées. C’est ainsi que l’Apôtre a dit : « La pierre était Jésus-Christ », parce que la pierre dont il parle en était la figure. La gloire de cette maison du Nouveau Testament est donc plus grande que celle de l’Ancien, et elle paraîtra telle quand on en fera la dédicace. C’est alors que « viendra celui que tous les peuples désirent », comme le porte le texte hébreu, parce que son premier avènement ne pouvait pas être désiré de tous les peuples, qui ne connaissaient pas celui qu’ils devaient désirer, et par conséquent ne croyaient point en lui. C’est aussi alors que, selon la version des Septante, dont le sens est pareillement prophétique, « les élus du Seigneur viendront de tous les endroits de l’univers ». À partir de cette époque, il ne viendra rien que ce qui a été élu et dont l’Apôtre dit : « Il nous a élus en lui avant la création du monde. » Le grand Architecte qui a dit : « Il y en a beaucoup d’appelés, mais peu d’élus », n’entendait pas que ceux qui, ayant été appelés au festin, avaient mérité qu’on les en chassât, dussent entrer dans l’édifice de cette maison dont la durée sera éternelle, mais seulement les élus. Or, maintenant que ceux qui doivent êtreséparés de l’aire à l’aide du van, remplissent l’Église, la gloire de cette maison ne paraît pas si grande qu’elle paraîtra, quand chacun sera toujours où il sera une fois.
\subsection[{Chapitre XLIX}]{Chapitre XLIX}

\begin{argument}\noindent Les élus et les réprouvés sont mêlés ensemble ici-bas.
\end{argument}

\noindent Dans ce siècle pervers, en ces tristes jours où l’Église, par des humiliations passagères, s’acquiert une grandeur immortelle pour l’avenir et est exercée par une infinité de craintes, de douleurs, de travaux et de tentations, sans avoir d’autre joie que l’espérance, si elle se réjouit comme il faut, beaucoup de réprouvés sont mêlés avec les élus, et les uns et les autres renfermés en quelque sorte dans ce filet de l’Évangile, nagent pêle-mêle à travers l’océan du monde, jusqu’à ce que tous arrivent au rivage, où les méchants seront séparés des bons, alors que Dieu habitera dans les bons comme dans son temple, pour y être tout en tous. Ainsi, nous voyons s’accomplir cette parole de celui qui disait dans le psaume : « J’ai publié et annoncé partout, et ils se sont multipliés sans nombre. » C’est ce qui arrive maintenant, depuis qu’il a publié et annoncé, d’abord par la bouche de Jean-Baptiste son précurseur et en second lieu par la sienne propre : « Faites pénitence, car le royaume des cieux est proche. » Le Seigneur donc fit choix de quelques disciples qu’il nomma apôtres, sans naissance, sans considération, sans lettres, afin d’être et de faire en eux tout ce qu’ils seraient et feraient de grand. Parmi eux se trouva un méchant ; mais le Sauveur, usant bien d’une mauvaise créature, se servit d’elle pour accomplir ce qui était ordonné touchant sa passion, et pour apprendre, par son exemple, à son Église à supporter les méchants. Ensuite, après avoir jeté les semences de l’Évangile, il souffrit, mourut et ressuscita, montrant par sa passion ce que nous devons endurer pour la vérité, et par sa résurrection ce que nous devons espérer pour l’éternité, sans parler du profond mystère de son sang répandu pour la rémission des péchés. Il conversa quarante jours sur la terre avec ses disciples, et monta au ciel devant leurs yeux ; et dix jours après, il leur envoya,suivant sa promesse, l’Esprit-Saint de son père, dont la venue sur les fidèles est marquée par ce signe suprême et nécessaire qu’ils parlaient toute sorte de langues, figure de l’unité de l’Église catholique, qui devait se répandre dans tout l’univers et parler les langues de tous les peuples.
\subsection[{Chapitre L}]{Chapitre L}

\begin{argument}\noindent De la prédication de l’Évangile, devenue plus éclatante et plus efficace par la passion de ceux qui l’annonçaient.
\end{argument}

\noindent Ensuite, selon cette prophétie : « La loi sortira de Sion, et la parole du Seigneur, de Jérusalem », et suivant la prédiction du Sauveur même, quand après sa résurrection il ouvrit l’esprit à ses disciples étonnés, pour leur faire entendre les Écritures, et leur dit : « Il fallait, selon ce qui est écrit, que le Christ souffrît, et qu’il ressuscitât le troisième jour, et qu’on prêchât en son nom la pénitence et la rémission des péchés dans toutes les nations, en commençant par Jérusalem » ; et encore, quand il répondit à ses disciples qui s’enquéraient de son dernier avènement : « Ce n’est pas à vous à savoir les temps ou les moments dont mon Père s’est réservé la disposition ; mais vous recevrez la vertu du Saint-Esprit qui viendra en vous, et vous me rendrez témoignage à Jérusalem, et dans toute la Judée et la Samarie, et jusqu’aux extrémités de la terre » ; suivant, dis-je, toutes ces paroles, l’Église se répandit d’abord à Jérusalem, et de là en Judée et en Samarie ; et l’Évangile fut ensuite porté aux Gentils parle ministère de ceux que Jésus-Christ avait lui-même allumés comme des flambeaux pouréclairer toute la terre, et embrasés du Saint-Esprit. Il leur avait dit : « Ne craignez point ceux qui tuent le corps, mais qui ne peuvent tuer l’âme » ; et le feu de la charité qui brûlait leur cœur étouffait en eux toute crainte. Il ne s’est pas seulement servi pour la prédication de l’Évangile de ceux qui l’avaient vu et entendu avant et après sa passion et sa résurrection ; mais il a suscité à ces premiers disciples des successeurs qui ont aussi porté sa parole dans tout le monde, parmi de sanglantes persécutions, Dieu se déclarant en leur faveur par plusieurs prodigeset par divers dons du Saint-Esprit, afin que les Gentils, convertis à celui qui a été crucifié pour les racheter, prissent en vénération, avec un amour digne de chrétiens, le sang des martyrs qu’ils avaient répandu avec une fureur digne des démons, et que les rois mêmes, dont les édits ravageaient l’Église, se soumissent humblement à ce nom que leur cruauté s’était efforcée d’exterminer, et tournassent leurs persécutions contre les faux dieux, pour l’amour desquels ils avaient auparavant persécuté les adorateurs du Dieu véritable.
\subsection[{Chapitre LI}]{Chapitre LI}

\begin{argument}\noindent Les hérétiques sont utiles à l’Église.
\end{argument}

\noindent Mais le diable, voyant qu’on abandonnait les temples des démons, et que le genre humain courait au nom du Sauveur et du Médiateur, suscita les hérétiques pour combattre la doctrine chrétienne sous le nom de chrétiens. Comme s’il pouvait y avoir dans la Cité de Dieu des personnes de sentiments contraires, à l’exemple de ces philosophes qui se contredisent l’un l’autre dans la cité de confusion ! Quand donc ceux qui dans l’Église de Jésus-Christ ont des opinions mauvaises et dangereuses, après en avoir été repris, y persistent opiniâtrement, et refusent de se rétracter de leurs dogmes pernicieux, ils deviennent hérétiques, et une fois sortis de l’Église, elle les regarde comme des ennemis qui servent à exercer sa vertu. Or, tout hérétiques qu’ils sont, ils ne laissent pas d’être utiles aux vrais catholiques qui sont les membres de Jésus-Christ, Dieu se servant bien des méchants mêmes, et toutes choses contribuant à l’avantage de ceux qui l’aiment. En effet, tous les ennemis de l’Église, quelque erreur qui les aveugle ou quelque passion qui les anime, lui procurent, en la persécutant corporellement, l’avantage d’exercer sa patience, ou, s’ils la combattent seulement par leurs mauvais sentiments, ils exercent au moins sa sagesse mais, de quelque façon que ce soit, ils lui donnent toujours sujet de pratiquer la bienveillance ou la générosité envers ses ennemis, soit qu’elle procède avec eux par des conférences paisibles, soit qu’elle les frappe de châtiments redoutables. C’est pourquoi le diable, qui est le prince de la cité des impies, a beau soulever ses esclaves contre la Cité de Dieu étrangère en ce monde, il ne lui saurait nuire. Dieu ne la laisse point sans consolation dans l’adversité, de peur qu’elle ne s’abatte, ni sans épreuve dans la prospérité, de crainte qu’elle ne s’exalte, et ce juste tempérament est marqué dans cette parole du psaume : « Vos consolations ont rempli mon âme de joie, à proportion des douleurs qui affligent mon cœur » ; ou encore dans ces mots de l’Apôtre : « Réjouissez-vous en espérance, et portez avec constance les afflictions. »\par
Le docteur des nations dit aussi que « tous ceux qui veulent vivre saintement en Jésus-Christ seront persécutés » ; il ne faut donc pas s’imaginer que cela puisse manquer en aucun temps ; car alors même que l’Église est à couvert de la violence des ennemis du dehors, ce qui n’est pas une petite consolation pour les faibles, il y en a toujours beaucoup au dedans qui affligent cruellement le cœur des gens de bien par leur mauvaise conduite, en ce qu’ils sont cause qu’on blasphème la religion chrétienne et catholique ; et cette injure qu’ils lui font est d’autant plus sensible aux âmes pieuses qu’elles l’aiment davantage et qu’elles voient qu’on l’en aime moins. Un autre sujet de douleur, c’est de penser que les hérétiques qui se disent aussi chrétiens et ont les mêmes sacrements que nous et les mêmes Écritures, jettent dans le doute plusieurs esprits disposés à embrasser le christianisme, et donnent lieu de calomnier notre religion, Ce sont ces dérèglements des hommes qui font souffrir une sorte de persécution à ceux qui veulent vivre saintement en Jésus-Christ, lors même que personne ne les tourmente en leur corps. Aussi le Psalmiste fait sentir que cette persécution est intérieure, quand il dit : « À proportion des douleurs qui affligent mon cœur ». Mais au surplus, comme on sait que les promesses de Dieu sont immuables, et que l’Apôtre dit : « Dieu connaît ceux qui sont à lui », de sorte que nul ne peut périr de ceux « qu’il a connus par sa prescience et prédestinés pour être conformes à l’image de son fils », le Psalmiste ajoute : « Vos consolations ont rempli mon âme de joie. » Or, cette douleur qui afflige le cœur des gens de bien à cause des mœurs des mauvais ou des faux chrétiens, est utile à ceux qui la ressentent, parce qu’elle naît de la charité, qui s’alarme pour ces misérables et pour tous ceux dont ils empêchent le salut. Les fidèles reçoivent aussi beaucoup de consolations, quand ils voient s’amender les méchants, et leur conversion leur donne autant de joie que leur perte leur causait de douleur. C’est ainsi qu’en ce siècle, pendant ces malheureux jours, non seulement depuis Jésus-Christ et les Apôtres, mais depuis Abel, le premier juste égorgé par son frère, jusqu’à la fin des siècles, l’Église voyage parmi les persécutions du monde et les consolations de Dieu.
\subsection[{Chapitre LII}]{Chapitre LII}

\begin{argument}\noindent S’il n’y aura point de persécution contre l’Église jusqu’à l’Antéchrist.
\end{argument}

\noindent C’est pourquoi je ne pense pas qu’on doive croire légèrement ce que quelques-uns avancent, que l’Église ne souffrira plus jusqu’à l’Antéchrist aucune autre persécution, après les dix qu’elle a souffertes, et que c’est lui qui suscitera la onzième. Ils placent la première sous Néron, la seconde sous Domitien, la troisième sous Trajan, la quatrième sous Antonin, la cinquième sous Sévère, la sixième sous Maximin, la septième sous Décius, la huitième sous Valérien, la neuvième sous Aurélien, et la dixième sous Dioclétien et Maximien. Ils disent que les dix plaies d’Égypte qui précédèrent la sortie du peuple de Dieu sont les figures de ces dix persécutions, et que la dernière, celle de l’Antéchrist, a été figurée par la onzième plaie d’Égypte, qui arriva lorsque les Égyptiens, poursuivant les Hébreux jusque dans la mer Rouge qu’ils passèrent à pied sec, furent engloutis par le retour de ses flots. Pour moi, je ne puis voir dans ces anciens événements une figure des persécutions de l’Église, quoique ceux qui sont de ce sentiment y trouvent des rapports fort ingénieux, mais qui ne sont fondés que sur des conjectures de l’esprit humain, fort sujet à prendre l’erreur pour la vérité.\par
Que diront-ils en effet de cette persécution où le Sauveur même fut crucifié ? à quel rang la mettront-ils ? S’ils prétendent qu’il ne faut compter que les persécutions qui ont atteint le corps de l’Église et non celle qui en a frappéet retranché la tête, que diront-ils de celle qui s’éleva à Jérusalem après que Jésus-Christ fut monté au ciel, et où saint Étienne fut lapidé, où saint Jacques, frère de saint Jean, eut la tête tranchée, où l’apôtre saint Pierre fut mis en prison et délivré par un ange, où les fidèles furent chassés de Jérusalem, où Saul, qui allait devenir l’apôtre Paul, ravagea l’Église et souffrit ensuite pour elle ce qu’il lui avait fait souffrir, parcourant la Judée et toutes les autres nations où son zèle lui faisait prêcher Jésus-Christ ? Pourquoi donc veulent-ils faire commencer à Néron les persécutions de l’Église, puisque ce n’est que par d’horribles souffrances, qu’il serait trop long de raconter ici, qu’elle est arrivée au règne de ce prince ? S’ils croient que l’on doit mettre au nombre des persécutions de l’Église toutes celles qui lui ont été suscitées par des rois, « érode était roi, et il lui en fit souffrir une des plus cruelles après l’ascension du Sauveur. D’ailleurs, que deviendra celle de Julien, qu’ils ne mettent pas entre les dix ? Dira-t-on qu’il n’a point persécuté l’Église, lui qui défendit aux chrétiens d’apprendre ou d’enseigner les lettres humaines, lui qui fit perdre à Valentinien, depuis empereur, la charge qu’il avait dans l’armée, pour avoir confessé la foi chrétienne, et je ne dis rien de ce qu’il avait commencé de faire à Antioche, quand il s’arrêta effrayé par la constance admirable d’un jeune homme qui chanta tout le jour des psaumes au milieu des plus cruels tourments, parmi les ongles de fer et les chevalets. Enfin le frère de ce Valentinien, l’arien Valens, n’a-t-il pas exercé de notre temps en Orient une sanglante persécution contre l’Église ? Comme notre religion est répandue dans tout le monde, elle peut être persécutée dans un lieu sans qu’elle le soit dans un autre ; est-ce à dire que cette persécution ne doive pas compter ? Il ne faudra donc pas mettre au nombre des persécutions celle que le roi des Goths dirigea dans son pays contre les catholiques, durant laquelle plusieurs souffrirent le martyre, ainsi que nous l’avons appris de quelques-uns de nos frères, qui se souvenaient de l’avoir vue, lorsqu’ils étaient encore enfants. Que dirai-jede celle qui vient de s’élever en Perse, et qui n’est pas encore bien apaisée ? N’a-t-elle pas été si forte qu’un certain nombre de chrétiens ont été contraints de se retirer dans les villes romaines ? Plus je réfléchis sur tout cela, plus il me semble qu’on ne doit pas déterminer le nombre des persécutions de l’Église. Mais aussi il n’y aurait pas moins de témérité à assurer qu’elle en doit souffrir d’autres avant celle de l’Antéchrist dont ne doute aucun chrétien. Laissons donc ce point indécis, le parti le plus sage et le plus sûr étant de ne rien assurer positivement.
\subsection[{Chapitre LIII}]{Chapitre LIII}

\begin{argument}\noindent On ne sait point quand la dernière persécution du monde arrivera.
\end{argument}

\noindent Pour cette dernière persécution de l’Antéchrist, le Sauveur lui-même la fera cesser par sa présence. Il est écrit « qu’il le tuera du souffle de sa bouche, et qu’il l’anéantira par l’éclat de sa présence ». On demande d’ordinaire, et fort mal à propos, quand cela arrivera. Mais s’il nous était utile de le savoir, qui nous l’aurait pu mieux apprendre que Jésus-Christ, notre Dieu et notre maître, le jour où ses disciples l’interrogèrent là-dessus ? Loin de s’en taire avec lui, ils lui firent cette question, quand il était encore ici-bas : « Seigneur, si vous paraissez en ce temps, quand rétablirez-vous le royaume d’Israël ? » Mais il leur répondit : « Ce n’est pas à vous à savoir les temps dont mon père s’est réservé la disposition. » Ils ne demandaient pas l’heure, ni le jour, ni l’année, mais le temps ; et toutefois Jésus-Christ leur fit cette réponse. C’est donc en vain que nous tâchons de déterminer les années qui restent jusqu’à la fin du monde, puisque nous apprenons de la Vérité même qu’il ne nous appartient pas de le savoir. Cependant, les uns en comptent quatre cents, d’autres cinq cents, et d’autres mille, depuis l’ascension du Sauveur jusqu’à son dernier avènement. Or, dire maintenant sur quoi chacun d’eux appuie son opinion, ce serait trop long et même inutile. Ils ne se fondent que sur des conjectures humaines, sans alléguer rien de certain des Écritures canoniques. Mais celui qui a dit : « Il ne vous appartient pasde savoir les temps dont mon père s’est réservé la disposition », a tranché court toutesces suppositions et nous commande de nous tenir en repos là-dessus.\par
Comme néanmoins cette parole est de l’Évangile, il n’est pas surprenant qu’elle n’ait pas empêché les idolâtres de feindre des réponses des démons touchant la durée de la religion chrétienne. Voyant que tant de cruelles persécutions n’avaient servi qu’à l’accroître au lieu de la détruire, ils ont inventé je ne sais quels vers grecs, qu’ils donnent pour une réponse de l’oracle, et où Jésus-Christ, à la vérité, est absous du crime de sacrilège, mais, en revanche, saint Pierre y est accusé de s’être servi de maléfices pour faire adorer le nom de Jésus-Christ pendant trois cent soixante-cinq ans, après quoi son culte sera aboli. Ô la belle imagination pour des gens qui se piquent de science ! Et qu’il est digne de ces grands esprits qui ne veulent point croire en Jésus-Christ, de croire de lui de semblables rêveries, et de dire que Pierre, son disciple, n’a pas appris de lui la magie, mais que néanmoins il a été magicien et qu’il a mieux aimé faire adorer le nom de son maître que le sien, s’exposant pour cela à une infinité de périls et à la mort même. Si Pierre magicien a fait que le monde aimât tant Jésus, qu’a fait Jésus innocent pour être tant aimé de Pierre ? Qu’ils se répondent à eux-mêmes là-dessus, et qu’ils comprennent, s’ils peuvent, que la même grâce de Dieu qui a fait aimer Jésus-Christ au monde pour la vie éternelle, l’a fait aimer à saint Pierre pour la même vie éternelle, jusqu’à souffrir la mort temporelle en son nom. Quels sont d’ailleurs ces dieux qui peuvent prédire tant de choses, et qui ne les sauraient empêcher, ces dieux obligés de céder aux enchantements d’un magicien et d’un scélérat qui a tué, dit-on, un enfant d’un an, l’a mis en pièces, et l’a enseveli avec des cérémonies sacrilèges, ces dieux enfin qui souffrent qu’une secte qui leur est contraire ait subsisté si longtemps, surmonté tant d’horribles persécutions, non pas en y résistant, mais en les subissant, et détruit leurs idoles, leurs temples, leurs sacrifices et leurs oracles ? Quel est enfin le dieu, leur dieu, à coup sûr, et non le nôtre, qu’un si grand crime a pu porter ou contraindre à souffrir tout cela ? Car ce n’est pas à un démon, mais à un dieu que s’adressent ces vers où Pierre est accusé d’avoir imposé la loi chrétienne par son art magique. Certes, ils méritent bien un tel dieu, ceux qui ne veulent pas reconnaître Jésus-Christ pour Dieu.
\subsection[{Chapitre LIV}]{Chapitre LIV}

\begin{argument}\noindent De ce mensonge des païens, que le christianisme ne devait durer que trois cent soixante-cinq ans.
\end{argument}

\noindent Voilà une partie de ce que j’alléguerais contre eux, si cette année faussement promise et sottement crue n’était pas encore écoulée. Mais puisqu’il y a déjà quelque temps que ces trois cent soixante-cinq ans depuis l’établissement du culte de Jésus-Christ par son incarnation et par la prédication des Apôtres sont accomplis, que faut-il davantage pour réfuter cette fausseté ? Qu’on ne les prenne pas, si l’on veut, à la naissance du Sauveur, parce qu’il n’avait pas encore alors de disciples, au moins ne peut-on nier que la religion chrétienne n’ait commencé à paraître quand il commença à en avoir, c’est-à-dire après qu’il eut été baptisé par saint Jean dans le fleuve du Jourdain. En effet, c’est ce que marquait cette prophétie : « Il étend ra sa domination d’une mer à l’autre, et depuis le fleuve jusqu’aux extrémités de la terre. » Mais comme la foi n’avait pas encore été annoncée à tous avant sa passion et sa résurrection, ainsi que l’apôtre saint Paul le dit aux Athéniens en ces termes : « Il avertit maintenant tous les hommes, en quelque lieu qu’ils soient, de faire pénitence, parce qu’il a arrêté un jour pour juger le monde selon la justice, par celui en qui il a voulu que tous crussent en le ressuscitant d’entre les morts » ; il vaut mieux, pour résoudre la question, commencer à ce moment l’ère chrétienne, surtout parce que ce fut alors que le Saint-Esprit fut donné dans cette ville où devait commencer la seconde loi, c’est-à-dire le Nouveau Testament. La première loi, qui est l’Ancien Testament, fut promulguée par Moïse au mont Sina ; mais pour celle-ci, qui devait être apportée par le Messie, voici ce qui en avait été prédit : « La loi sortira de Sion, et la parole du Seigneur, de Jérusalem » ; d’où vient que lui-même a dit qu’il fallait qu’on prêchât en son nom la pénitence à toutes les nations, mais en commençant par Jérusalem. C’est donc là que le culte de ce nom a commencé, et qu’on a, pour la première fois, cru en Jésus-Christ crucifié et ressuscité. C’est là que la foi fut d’abord si fervente que des milliers d’hommes, s’étant miraculeusement convertis, vendirent tous leurs biens et les distribuèrent aux pauvres pour embrasser la sainte pauvreté et être plus prêts à combattre jusqu’à la mort pour la défense de la vérité au milieu des Juifs frémissants et altérés de carnage. Si cela ne s’est point fait par magie, pourquoi font-ils difficulté de croire que la même vertu divine, qui a opéré une si grande merveille en ce lieu, ait pu l’étendre dans tout le monde ? Et si ce furent les maléfices de Pierre qui causèrent ce prodigieux changement dans Jérusalem, et firent qu’une si grande multitude d’hommes, qui avaient crucifié le Sauveur ou qui l’avaient insulté sur la croix, furent tout d’un coup portés à l’adorer, il faut voir, par l’année où cela est arrivé, quand les trois cent soixante-cinq ans ont été accomplis. Jésus-Christ est mort le huit des calendes d’avril, sous le consulat des deux Géminus. Il ressuscita le troisième jour, suivant le témoignage des Apôtres, qui en furent témoins oculaires. Quarante jours après il monta au ciel, et envoya le Saint--Esprit le dixième jour suivant. Ce fut alors que mille hommes crurent en lui sur la prédication des Apôtres. Ce fut donc-alors que commença le culte de son nom par la vertu du Saint-Esprit, selon notre foi et selon la vérité, ou, comme l’impiété le feint ou le pense follement, par les enchantements de Pierre. Peu de temps après, cinq mille hommes se convertirent à la guérison miraculeuse d’un boiteux de naissance, qui était si impotent qu’on le portait tous les jours au seuil du temple pour demander l’aumône, et qui se leva et marcha à la parole de Pierre et au nom de Jésus-Christ. Et c’est ainsi que l’Église s’augmenta de plus en plus et fit rapidement de nouvelles conquêtes. Il est donc aisé de calculer le jour même auquel a commencé l’année que nouscherchons. Ce fut quand le Saint-Esprit fut envoyé, c’est-à-dire aux ides de mai. Or, en comptant les consuls, l’on trouve que ces trois cent soixante-cinq ans ont été accomplis pendant ces mêmes ides, sous le consulat d’Honorius et d’Eutychianus. Cependant l’année d’après, sous le consulat de Manlius Théodore, alors que, selon l’oracle des démons ou la fiction des hommes, il ne devait plus y avoir de christianisme, nous voyons à Carthage, la ville la plus considérable et la plus célèbre d’Afrique, sans parler de ce qui se passe ailleurs, Gaudentius et Jovius, comtes de l’empereur Honorius, donner, le 14 des calendes d’avril, l’ordre d’abattre les temples des faux dieux et de briser leurs idoles. Depuis ce temps jusqu’à cette heure, c’est-à-dire pendant l’espace d’environ trente années, qui ne voit combien le culte du nom de Jésus-Christ s’est augmenté, depuis surtout que plusieurs de ceux qui étaient retenus par cette vaine prophétie se sont faits chrétiens, voyant cette annéechimérique écoulée. Nous donc qui sommes chrétiens et qui en portons le nom, nous ne croyons pas en Pierre, mais en celui en qui Pierre a cru, et nous n’avons pas été charmés par ses sortilèges, mais édifiés par ses prédications. Jésus-Christ, qui est le maître de Pierre, est aussi notre maître, et il nous enseigne la doctrine qui conduit à la vie éternelle. Mais il est temps de terminer ce livre, où nous avons suffisamment fait voir, ce me semble, le progrès des deux cités qui sont mêlées ici-bas depuis le commencement jusqu’à la fin. Celle de la terre s’est fait tels dieux qu’il lui a plu pour leur offrir des sacrifices ; celle du ciel, étrangère sur la terre, ne se fait point de dieux, mais est faite elle-même par le vrai Dieu pour être son véritable sacrifice. Toutes deux néanmoins omit part égale aux biens et aux maux de cette vie ; mais leur foi, leur espérance et leur charité sont différentes, jusqu’à ce que le dernier jugement les sépare et que chacune d’elles arrive à sa fin qui n’aura point de fin. C’est de cette fin de l’une et de l’autre qu’il nous reste à parler.
\section[{Livre dix-neuvième. Le souverain bien}]{Livre dix-neuvième. \\
Le souverain bien}\renewcommand{\leftmark}{Livre dix-neuvième. \\
Le souverain bien}

\subsection[{Chapitre premier}]{Chapitre premier}

\begin{argument}\noindent Il peut y avoir, selon Varron, deux cent quatre-vingt-huit systèmes philosophiques touchant le souverain bien.
\end{argument}

\noindent Puisqu’il me reste à traiter de la fin de chacune des deux cités, je dois d’abord rapporter en peu de mots les raisonnements où s’égarent les hommes pour aboutir à se faire une béatitude parmi les misères de cette vie ; je dois en même temps faire voir, non seulement par l’autorité divine, mais encore par la raison, combien il y a de différence entre les chimères des philosophes et l’espérance que Dieu nous donne ici-bas et qui doit être suivie de la véritable félicité. Les philosophes ont agité fort diversement la question de la fin des biens et des maux, et se sont donné beaucoup de peine pour trouver ce qui peut rendre l’homme heureux. Car la fin suprême, quant à notre bien, c’est l’objet pour lequel on doit rechercher tout le reste et qui ne doit être recherché que pour lui-même ; et quant à notre mal, c’est aussi l’objet pour lequel il faut éviter tout le reste et qui ne doit être évité que pour lui-même. Ainsi, par la fin du bien, nous n’entendons pas une fin où il s’épuise jusqu’à n’être plus, mais où il s’achève pour atteindre à sa plénitude, et pareillement par la fin du mal, nous ne voulons pas parler de ce qui détruit le mal, mais de ce qui le porte à son comble. Ces deux fins sont donc le souverain bien et le souverain mal, et c’est pour les trouver que se sont beaucoup tourmentés, comme je le disais, ceux qui, parmi les vanités du siècle, ont fait profession d’aimer la sagesse. Mais, quoiqu’ils aient erré en plus d’une façon, la lumière naturelle ne leur a pas permis des’éloigner tellement de la vérité qu’ils n’aient mis le souverain bien et le souverain mal, les uns dans l’âme, les autres dans le corps, et les autres dans tous les deux. De cette triple division, Varron, dans son livre {\itshape De la Philosophie}, tire une si grande diversité de sentiments, qu’en y ajoutant quelques légères différences, il compte jusqu’à deux cent quatre-vingt-huit sectes, sinon réelles, du moins possibles.\par
Voici comment il procède : « Il y a, dit-il, quatre choses que les hommes recherchent naturellement, sans avoir besoin de maître ni d’art, et qui sont par conséquent antérieures à la vertu (laquelle est très certainement un fruit de la science) : premièrement, la volupté, qui est un mouvement agréable des sens ; en second lieu, le repos, qui exclut tout ce qui pourrait incommoder le corps ; en troisième lieu, ces deux choses réunies, qu’Épicure a même confondues sous le nom de volupté ; enfin, les premiers biens de la nature, qui comprennent tout ce que nous venons de dire et d’autres choses encore, comme la santé et l’intégrité des organes, voilà pour le corps, et les dons variés de l’esprit, voilà pour l’âme. Or, ces quatre choses, volupté, repos, repos et volupté, premiers biens de la nature, sont en nous de telle sorte qu’il faut de trois choses l’une : ou rechercher la vertu pour elles, ou les rechercher pour la vertu, ou ne les rechercher que pour elles-mêmes ; et de là naissent douze sectes. À ce compte, en effet, chacune est triplée, comme je vais le faire voir pour une d’elles, après quoi il ne sera pas difficile de s’en assurer pour les autres. Que la volupté du corps soit soumise, préférée ou associée à la vertu, cela fait trois sectes. Or, elle est soumise à la vertu, quand on la prend pour instrument de la vertu. Ainsi, il est du devoir de la vertu de vivre pour la patrie et de lui engendrer des enfants, deux choses qui ne peuvent se faire sans volupté. Mais quand on préfère la volupté à la vertu, on ne recherche plus la volupté que pour elle-même ; et alors la vertu n’est plus qu’un moyen pour acquérir ou pour conserver la volupté, et cette vertu esclave ne mérite plus son nom. Ce système infâme a pourtant trouvé des défenseurs et des apologistes parmi les philosophes. Enfin, la volupté est associée à la vertu, quand on ne les recherche point l’une pour l’autre, mais chacune pour elle-même. Maintenant, de même que la volupté, tour à tour soumise, préférée ou associée à la vertu, a fait trois sectes, de même le repos, la volupté avec le repos, et les premiers biens de la nature, en font aussi un égal nombre, sui vaut qu’elles sont soumises, préférées ou associées à la vertu, et ainsi voilà douze sectes. Mais ce nombre devient double en y ajoutant une différence, qui est la vie sociale. En effet, quiconque embrasse quelqu’une de ces sectes, ou le fait seulement pour soi, ou le fait aussi pour un autre qu’il s’associe et à qui il doit souhaiter le même avantage. Il y aura donc douze sectes de philosophes qui ne professeront leur doctrine que pour eux-mêmes, et douze qui l’étendront à leurs semblables, dont le bien ne les touchera pas moins que leur bien propre. Or, ces vingt-quatre sectes se doublent encore et montent jusqu’à quarante-huit, en y ajoutant une différence prise des opinions de la nouvelle Académie. De ces vingt-quatre opinions, en effet, chacune peut être soutenue comme certaine, et c’est ainsi que les Stoïciens ont prétendu qu’il est certain que le souverain bien de l’homme ne consiste que dans la vertu, ou comme incertaine et seulement vraisemblable, comme ont fait les nouveaux Académiciens. Voilà donc vingt-quatre sectes de philosophes qui défendent leur opinion comme assurée, et vingt-quatre autres qui la soutiennent comme douteuse. Bien plus, comme chacune de ces quarante-huit sectes peut être embrassée, ou en suivant la manière de vivre des autres philosophes, ou en suivant celle des cyniques, cette différence les double encore et en fait quatre-vingt-seize. Ajoutez enfin à cela que, comme on peut embrasser chacune d’elles, ou en menant une vie tranquille, à l’exemple de ceux qui, par goût ou par nécessité, ont donné tous leurs moments à l’étude, ou bien une vie active, à la manière de ceux qui ont joint l’étude de la philosophie au gouvernement de l’État, ou une vie mêlée des deux autres, tels que ceux qui ont donné une partie de leur loisir à la contemplation et l’autre à l’action, ces différences peuvent tripler le nombre des sectes et en faire jusqu’à deux cent quatre-vingt-huit. »\par
Voilà ce que j’ai recueilli du livre de Varron le plus succinctement et le plus clairement qu’il m’a été possible, en m’attachant à sa pensée sans citer ses expressions. Or, de dire maintenant comment cet auteur, après avoir réfuté les autres sectes, en choisit une qu’il prétend être celle des anciens Académiciens, et comment il distingue cette école, suivant lui dogmatique, dont Platon est le chef et Polémon le quatrième et dernier représentant, d’avec celle des nouveaux Académiciens qui révoquent tout en doute, et qui commencent à Arcésilas, successeur de Polémon ; de rapporter, dis-je, tout cela en détail, aussi bien que les preuves qu’il allègue pour montrer que les anciens Académiciens ont été exempts d’erreur comme de doute, c’est ce qui serait infiniment long, et cependant il est nécessaire d’en dire un mot. Varron rejette donc dès l’abord toutes les différences qui ont si fort multiplié ces sectes, et il les rejette parce qu’elles ne se rapportent pas au souverain bien. Suivant lui, en effet, une secte philosophique n’existe et ne se distingue des autres, qu’à condition d’avoir une opinion propre sur le souverain bien. Car l’homme n’a d’autre objet en philosophant que d’être heureux ; or, ce qui rend heureux, c’est le souverain bien, et par conséquent toute secte qui n’a pas pour aller au souverain bien sa propre voie n’est pas vraiment une secte philosophique. Ainsi, quand on demande si le sage doit mener une vie civile et sociale et procurer à son ami tout le bien qu’il se procure à lui-même, ou s’il ne doit rechercher la béatitude que pour soi, il est question, non pas du souverain bien, mais de savoir s’il y faut associer quelque autre avec soi. De même, quand on demande s’il faut révoquer toutes choses en doute comme les nouveaux Académiciens, ou si l’on doit les tenir pour certaines avec les autres philosophes, on ne demande pas quel est le bien qu’on doit rechercher, mais s’il faut douter ou non de la vérité du bien que l’on recherche. La manière de vivre des cyniques, différente de celle des autres philosophes, ne concerne pas non plus la question du souverain bien ; mais, la supposant résolue, on demande seulement s’il faut vivre comme les cyniques. Or, il s’est trouvé des hommes qui, tout en plaçant le souverain bien en différents objets, les uns dans la vertu et les autres dans la volupté, n’ont pas laissé de mener le genre de vie qui a valu aux cyniques leur nom. Ainsi, ce qui fait la différence entre les cyniques et les autres philosophes est étranger à la question de la nature du souverain bien. Autrement, la même manière de vivre impliquerait la même fin poursuivie, et réciproquement, ce qui n’a pas lieu.
\subsection[{Chapitre II}]{Chapitre II}

\begin{argument}\noindent Comment Varron réduit toutes ces sectes à trois, parmi lesquelles il faut choisir la bonne.
\end{argument}

\noindent De même, lorsqu’on demande si l’on doit embrasser la vie active ou la vie contemplative, ou celle qui est mêlée des deux, il ne s’agit pas du souverain bien, mais du genre de vie le plus propre à l’acquérir ou à le conserver. Du moment, en effet, que l’homme est supposé parvenu au souverain bien, il est heureux ; au lieu que la paix de l’étude, ou l’agitation des affaires publiques, ou le mélange de cette agitation et de cette paix, ne donnent pas immédiatement le bonheur. Car plusieurs peuvent adopter l’un de ces trois genres de vie et se tromper sur la nature du souverain bien. Ce sont donc des questionsentièrement différentes que celle du souverain bien, qui constitue chaque secte de philosophes, et celles de la vie civile, de l’incertitude des Académiciens, du genre de vie et du vêtement des cyniques, enfin des trois sortes de vie, l’active, la contemplative et le mélange de l’une et de l’autre. C’est pourquoi Varron, rejetant ces quatre différences qui faisaient monter les sectes presque au nombre de deux cent quatre-vingt-huit, revient aux douze, où il s’agit uniquement de savoir quel est le souverain bien de l’homme, afin d’établir qu’une seule, parmi elles, contient la vérité, tout le reste étant dans l’erreur. Écartez en effet les trois genres de vie, les deux tiers du nombre total sont retranchés, et il reste quatre-vingt-seize sectes. Ôtez la différence qui se tire des cyniques, elles se réduisent à la moitié, à quarante-huit. Ôtez encore la différence relative à la nouvelle Académie, elles diminuent encore de moitié, et tombent à vingt-quatre. Ôtez enfin la différence de la vie solitaire ou sociale, il ne restera plus que douze sectes, nombre que cette différence doublait et portait à vingt-quatre. Quant à ces douze sectes, on ne peut leur contester leur qualité, puisqu’elles ne se proposent d’autre recherche que celle du souverain bien. Or, pour former ces douze sectes, il faut tripler quatre choses : la volupté, le repos, le repos et la volupté, et les premiers biens de la nature, attendu que chacune d’elles est soumise, préférée ou associée à la vertu, ce qui donne bien douze pour nombre total. Maintenant, de ces quatre choses, Varron en ôte trois, la volupté, le repos, le repos joint à la volupté, non qu’il les improuve, mais parce qu’elles sont comprises dans les premiers biens de la nature. De sorte qu’il n’y a plus que trois sectes à examiner ; car ici, comme en toute autre matière, il ne peut y en avoir plus d’une qui soit véritable, et ces trois sectes consistent en ce que l’on y recherche soit les premiers biens de la nature pour la vertu, soit la vertu pour les premiers biens de la nature, soit chacune de ces deux choses pour elle-même.
\subsection[{Chapitre III}]{Chapitre III}

\begin{argument}\noindent Quel est, entre les trois systèmes sur le souverain bien, celui qu’il faut préférer, selon Varron, qui se déclare disciple d’Antiochus et de l’ancienne Académie.
\end{argument}

\noindent Voici comment Varron procède : il considère que le souverain bien que cherche la philosophie n’est pas le bien de la plante, ni de la bête, ni de Dieu, mais de l’homme ; d’où il conclut qu’il faut savoir d’abord ce que c’est que l’homme. Or, il croit qu’il y a deux parties dans la nature humaine : le corps et l’âme, et ne doute point que l’âme ne soit beaucoup plus excellente que le corps. Mais de savoir si l’âme seule est l’homme, en sorte que le corps soit pour elle ce que le cheval est au cavalier, c’est ce qu’il prétend qu’on doit examiner : le cavalier, en effet, n’est pas tout ensemble l’homme et le cheval, mais l’homme seul, qui pourtant s’appelle cavalier, à cause de son rapport au cheval. D’un autre côté, le corps seul est-il l’homme, avec quelque rapport à l’âme, comme la coupe au breuvage ? car ce n’est pas le vase et le breuvage tout ensemble, mais le vase seul qu’on appelle coupe, à condition toutefois qu’il soit fait de manière à contenir le breuvage. Enfin, si l’homme n’est ni l’âme seule, ni le corps seul, est-il un composé des deux, comme un attelage de deux chevaux n’est aucun des deux en particulier, mais tous les deux ensemble ? Varron s’arrête à ce parti, ce qui l’amène à conclure que le souverain bien de l’homme consiste dans la réunion des biens de l’âme et de ceux du corps. Il croit donc que ces premiers biens de la nature sont désirables pour eux-mêmes, ainsi que la vertu, cet art de vivre qu’enseigne la science et qui est, parmi les biens de l’âme, le bien le plus excellent. Lors donc que la vertu a reçu de la nature ces premiers biens, qui sont antérieurs à toute science, elle les recherche pour soi, en même temps qu’elle se recherche soi-même, et elle en use comme elle use de soi, de manière à y trouver ses délices et sa joie, se servant de tous, mais plus ou moins, selon qu’ils sont plus ou moins grands, et sachant mépriser les moindres, quand cela est nécessaire pour acquérir ou pour conserver les autres. Or, de tous ces biens de l’âme et du corps il n’en est aucun que la vertu se préfère, parce qu’elle sait user comme il faut et de soi et de tout ce qui rend l’homme heureux ; au contraire, où elle n’est pas, les autres biens, en quelque abondance qu’ils se trouvent, ne sont pas pour le bien de celui qui les possède, parce qu’il en use niai. La vie de l’homme est donc heureuse, quand il jouit et de la vertu et, parmi les autres biens de l’âme et du corps, de tous ceux sans lesquels la vertu ne peut subsister. Elle est encore plus heureuse, quand il possède d’autres biens dont la vertu n’a pas absolument besoin ; enfin, elle est très heureuse, lorsqu’il ne lui manque aucun bien, soit de l’âme, soit du corps. La vie, en effet, n’est pas la même chose que la vertu, puisque toute sorte de vie n’est pas vertu, mais celle-là seulement qui est sage et réglée : et cependant une vie, quelle qu’elle soit, peut être sans la vertu, au lieu que la vertu ne peut être sans la vie. On peut en dire autant de la mémoire et de la raison : elles sont en l’homme avant la science, et la science ne saurait être sans elles, ni par conséquent la vertu, puisqu’elle est un fruit de la science. Quant aux avantages du corps, comme la vitesse, la beauté, la force, et autres semblables, bien que la vertu puisse être sans eux, comme eux sans elle, toutefois ce sont des biens ; et selon ces philosophes, la vertu les aime pour l’amour d’elle-même, et s’en sert ou en jouit avec bienséance.\par
Ils disent que cette vie bienheureuse est aussi une vie sociale, qui aime le bien de ses amis comme le sien propre et leur souhaite les mêmes avantages qu’à elle-même soit qu’ils vivent dans la même maison, comme une femme, des enfants, des domestiques, ou dans la même ville, comme des citoyens, ou dans le monde, ce qui comprend le ciel et la terre, comme les dieux dont ils font les amis du sage et que nous sommes accoutumés à appeler les anges. En outre, ils soutiennent que sur la question du souverain bien et du souverain mal, il n’y a lieu à aucun doute, par où ils prétendent se séparer des nouveaux Académiciens. Car peu leur importe, d’ailleurs, quelle sorte de vie on choisira pour atteindre le souverain bien, soit celle des cyniques, soit toute autre. Enfin, quant aux trois genres de vie dont nous avons parlé, la vie active, la vie contemplative et le mélange des deux, c’est celle-ci qui leur plaît davantage. Voilà donc la doctrine de l’ancienne Académie, telle que Varron la reçut d’Antiochus, qui fut aussi le maître de Cicéron, quoique celui-ci le rattache plutôt à l’école stoïcienne qu’à l’Académie ; mais cela nous importe peu, puisque nous cherchons moins à distinguer les diverses opinions des hommes qu’à découvrir la vérité sur le fond des choses.
\subsection[{Chapitre IV}]{Chapitre IV}

\begin{argument}\noindent Ce que pensent les chrétiens sur le souverain bien, contre les philosophes qui ont cru le trouver en eux-mêmes.
\end{argument}

\noindent Si l’on nous demande quel est le sentiment de la Cité de Dieu sur tous ces points, et d’abord touchant la fin des biens et des maux, elle-même répondra que la vie éternelle est le souverain bien et la mort éternelle le souverain mal, et qu’ainsi nous devons tâcher de bien vivre, afin d’acquérir l’une et d’éviter l’autre. Il est écrit : « Le juste vit de la foi. » En effet, en cette vie, nous ne voyons point encore notre bien, de sorte que nous le devons chercher par la foi, n’ayant pas en nous-mêmes le pouvoir de bien vivre, si celui qui nous a donné la foi dans son assistance ne nous aide à croire et à prier. Pour ceux qui ont cru que le souverain bien est en cette vie, qu’ils l’aient placé dans le corps ou dans l’âme, ou dans tous les deux ensemble, ou, pour résumer tous les systèmes, qu’ils l’aient fait consister dans la volupté, ou dans la vertu, ou dans l’une et l’autre ; dans le repos, ou dans la vertu, ou dans l’un et l’autre ; dans la volupté et le repos, ou dans la vertu, ou dans tout cela pris ensemble ; enfin dans les premiers biens de la nature, ou dans la vertu, ou dans ces objets réunis, c’est en tous cas une étrange vanité d’avoir placé leur béatitude ici-bas, et surtout de l’avoir fait dépendre d’eux-mêmes. La Vérité se rit de cet orgueil, quand elle dit par un prophète : « Le Seigneur sait que les pensées des hommes sont vaines », ou comme parle l’apôtre saint Paul : « Le Seigneur connaît les pensées des sages et il sait qu’elles sont vaines. »\par
Quel fleuve d’éloquence suffirait à déroulertoutes les misères de cette vie ? Cicéron l’a essayé comme il a pu dans la {\itshape Consolation sur la mort de sa fille} ; mais que ce qu’il a pu est peu de chose ! En effet, ces premiers biens de la nature, les peut-on posséder en cette vie qu’ils ne soient sujets à une infinité de révolutions ? Y a-t-il quelque douleur et quelque inquiétude (deux affections diamétralement opposées à la volupté et au repos) auxquelles le corps du sage ne soit exposé ? Le retranchement ou la débilité des membres est contraire à l’intégrité des parties du corps, la laideur à sa beauté, la maladie à sa santé, la lassitude à ses forces, la langueur ou la pesanteur à son agilité ; et cependant, quel est celui de ces maux dont le sage soit exempt ? L’équilibre du corps et ses mouvements, quand ils sont dans la juste mesure, comptent aussi parmi les premiers biens de la nature. Mais que sera-ce, si quelque indisposition fait trembler les membres ? que sera-ce, si l’épine du dos se courbe, de sorte qu’un homme soit obligé de marcher à quatre pattes comme une bête ? Cela ne détruira-t-il pas l’assiette ferme et droite du corps, la beauté et la mesure de ses mouvements ? Que dirai-je des premiers biens naturels de l’âme, le sens et l’entendement, dont l’un lui est donné pour apercevoir la vérité, et l’autre pour la comprendre ? Où en sera le premier, si un homme devient sourd et aveugle ; et le second, s’il devient fou ? Combien les frénétiques font-ils d’extravagances qui nous tirent les larmes des yeux, quand nous les considérons sérieusement ? Parlerai-je de ceux qui sont possédés du démon ? Où leur raison est-elle ensevelie, quand le malin esprit abuse de leur âme et de leur corps à son gré ? Et qui peut s’assurer que cet accident n’arrivera point au sage pendant sa vie ? Il y a plus : combien défectueuse est la connaissance de la vérité ici-bas, où, selon les paroles de la Sagesse, « ce corps mortel et corruptible appesantit l’âme, et cette demeure de terre et de boue émousse l’esprit qui pense beaucoup ». Cette activité instinctive (que les Grecs appellent {\itshape ormè}) également comptée au nombre des premiers biens de la nature, n’est-elle pas dans les furieuxla cause de ces mouvements et de ces actions qui nous font horreur ?\par
Enfin, la vertu, qui n’est pas au nombre des biens de la nature, puisqu’elle est un fruit tardif de la science, mais qui toutefois réclame le premier rang parmi les biens de l’homme, que fait-elle sur terre, sinon une guerre continuelle contre les vices, je ne parle pas des vices qui sont hors de nous, mais de ceux qui sont en nous, lesquels ne nous sont pas étrangers, mais nous appartiennent en propre ? Quelle guerre doit surtout soutenir cette vertu que les Grecs nomment {\itshape sophrosunè}, et nous tempérance, quand il faut réprimer les appétits désordonnés de la chair, de peur qu’ils ne fassent consentir l’esprit à des actions criminelles ? Et ne nous imaginons pas qu’il n’y ait point de vice en nous, lorsque « la chair, comme dit l’Apôtre, convoite contre l’esprit » ; puisqu’il existe une vertu directement contraire, celle que désigne ainsi le même Apôtre : « L’esprit convoite contre la chair » ; et il ajoute : « Ces principes sont contraires l’un à l’autre, et vous ne faites pas ce que vous voudriez. » Or, que voulons-nous faire, quand nous voulons que le souverain bien s’accomplisse en nous, sinon que la chair s’accorde avec l’esprit et qu’il n’y ait plus entre eux de divorce ? Mais, puisque nous ne le saurions faire en cette vie, quelque désir que nous en ayons, tâchons au moins, avec le secours de Dieu, de ne point consentir aux convoitises déréglées de la chair. Dieu nous garde donc de croire, déchirés que nous sommes par cette guerre intestine, que nous possédions déjà la béatitude qui doit être le fruit de notre victoire ! Et qui donc est parvenu à ce comble de sagesse qu’il n’ait plus à lutter contre ses passions ?\par
Que dirai-je de cette vertu qu’on appelle prudence ? Toute sa vigilance n’est-elle pas occupée à discerner le bien d’avec le mal, pour rechercher l’un et fuir l’autre ? Or, cela ne prouve-t-il pas que nous sommes dans le mal et que le mal est en nous ? Nous apprenons par elle que c’est un mal de consentir à nos mauvaises inclinations, et que c’est un bien d’y résister ; et cependant ce mal, à qui la prudence nous apprend à ne pas consentir etque la tempérance nous fait combattre, ni la tempérance, ni la prudence ne le font disparaître. Et la justice, dont l’emploi est de rendre à chacun ce qui lui est dû (par où se maintient en l’homme cet ordre équitable de la nature, que l’âme soit soumise à Dieu, le corps à l’âme, et ainsi l’âme et le corps à Dieu), ne fait-elle pas bien voir, par la peine qu’elle prend à s’acquitter de cette fonction, qu’elle n’est pas encore à la fin de son travail ? L’âme est en effet d’autant moins soumise à Dieu qu’elle pense moins à lui ; et la chair est d’autant moins soumise à l’esprit qu’elle a plus de désirs qui lui sont contraires. Ainsi, tant que nous sommes sujets à ces faiblesses et à ces langueurs, comment osons-nous dire que nous sommes déjà sauvés ? Et si nous ne sommes pas encore sauvés, de quel front pouvons-nous prétendre que nous sommes bienheureux ? Quant à la force, quelque sagesse qui l’accompagne, n’est-elle pas un témoin irréprochable des maux qui accablent les hommes et que la patience est contrainte de supporter ? En vérité, je m’étonne que les Stoïciens aient la hardiesse de nier que ce soient des maux, en même temps qu’ils prescrivent au sage, si ces maux arrivent à un point qu’il ne puisse ou ne doive pas les souffrir, de se donner la mort, de sortir de la vie. Cependant telle est la stupidité où l’orgueil fait tomber ces philosophes, qui veulent trouver en cette vie et en eux-mêmes le principe de leur félicité, qu’ils n’ont point de honte de dire que leur sage, celui dont ils tracent le fantastique idéal, est toujours heureux, devînt-il aveugle, sourd, muet, impotent, affligé des plus cruelles douleurs et de celles-là mêmes qui l’obligent à se donner la mort. Ô la vie heureuse, qui, pour cesser d’être, cherche le secours de la mort ! Si elle est heureuse, que n’y demeure-t-on ; et si on la fuit à cause des maux qui l’affligent comment est-elle bienheureuse ? Se peut-il faire qu’on n’appelle point mal ce qui triomphe du courage même, ce qui ne l’oblige pas seulement à se rendre, mais le porte encore à ce délire de regarder comme heureuse une vie que l’on doit fuir ? Qui est assez aveugle pourne pas voir que si on doit la fuir, c’est qu’elle n’est pas heureuse ? et s’ils avouent qu’on la doit fuir à cause des faiblesses qui l’accablent, que ne quittent-ils leur superbe, pour avouer aussi qu’elle est misérable ? N’est-ce pas plutôt par impatience que par courage que ce fameux Caton s’est donné la mort, et pour n’avoir pu souffrir César victorieux ? Où est la force de cet homme tant vanté ? Elle a cédé, elle a succombé, elle a été tellement surmontée qu’il a fui et abandonné une vie bienheureuse. Elle ne l’était plus, dites-vous ? Avouez donc qu’elle était malheureuse. Et dès lors, comment ce qui rend une vie malheureuse et détestable ne serait-il pas un mal ?\par
Aussi les Péripatéticiens et ces philosophes de la vieille Académie, dont Varron se porte le défenseur, ont-ils eu la sagacité de céder sur ce point ; mais leur erreur est encore étrange de soutenir que malgré tous les maux, le sage ne laisse pas d’être heureux. « Les tortures et les douleurs du corps sont des maux, dit Varron, et elles le sont d’autant plus qu’elles prennent plus d’accroissement ; et voilà pourquoi il faut s’en délivrer en sortant de la vie. » De quelle vie, je vous prie ? De celle, dit Varron, qui est accablée de tant de maux. Quoi donc ! est-ce de cette vie toujours heureuse au milieu même des maux qui doivent nous en faire sortir ? ou ne l’appelez-vous heureuse que parce qu’il vous est permis de vous en délivrer ? Que serait-ce donc si quelque secret jugement de Dieu vous retenait parmi ces maux sans permettre à la mort de vous en affranchir jamais ! Alors du moins seriez-vous obligés d’avouer qu’une vie de cette sorte est misérable. Ce n’est donc pas pour être promptement quittée qu’elle n’est pas misérable, à moins de vouloir appeler félicité une courte misère. Certes, il faut que des maux soient bien violents pour obliger un homme, et un homme sage, à cesser d’être homme pour s’en délivrer. Ils disent, en effet, et avec raison, que c’est le premier cri de la nature que l’homme s’aime soi-même, et partant qu’il ait une aversion instinctive pour la mort et cherche tout ce qui peut entretenir l’union du corps et de l’âme. Il faut donc que des maux soient bien violents pourétouffer ce sentiment de la nature et l’éteindre à ce point que nous désirions la mort et tournions nos propres mains contre nous-mêmes, si personne ne consent à nous la donner. Encore une fois, il faut que des maux soient bien violents pour rendre la force homicide, si néanmoins la force mérite encore son nom, alors qu’elle succombe sous le mal et non seulement ne peut conserver par la patience un homme dont elle avait pris le gouvernement et la protection, mais se voit réduite à le tuer. Oui, j’en conviens, le sage doit souffrir la mort avec patience, mais quand elle lui vient d’une main étrangère ; si donc, suivant eux, il est obligé de se la donner, il faut qu’ils avouent que les accidents qui l’y obligent ne sont pas seulement des maux, mais des maux insupportables. À coup sûr, une vie sujette à tant de misères n’eût jamais été appelée heureuse, si ceux qui lui donnent ce nom cédaient à la vérité comme ils cèdent à la douleur, au lieu de prétendre jouir du souverain bien dans un lieu où les vertus même, qui sont ce que l’homme a de meilleur ici-bas, sont des témoins d’autant plus fidèles de nos misères qu’elles travaillent davantage à nous en garantir. Si ce sont donc des vertus véritables, et il ne peut y en avoir de telles qu’en ceux qui ont une véritable piété, elles ne promettent à personne de le délivrer de toutes sortes de maux ; non, elles ne font pas cette promesse, parce qu’elles ne savent pas mentir ; tout ce qu’elles peuvent faire, c’est de nous assurer que si nous espérons dans le siècle à venir, cette vie humaine, nécessairement misérable à cause des innombrables épreuves du présent, deviendra un jour bienheureuse en gagnant du même coup le salut et la félicité. Mais comment posséderait-elle la félicité, quand elle ne possède pas encore le salut ? Aussi l’apôtre saint Paul, parlant, non de ces philosophes véritablement dépourvus de sagesse, de patience, de tempérance et de justice, mais de ceux qui ont une véritable piété et par conséquent des vertus véritables, dit : « Nous sommes sauvés en espérance. Or, la vue de l’objet espéré n’est plus de l’espérance. Car qui espère ce qu’il voit déjà ? Si donc nous espérons ce que nous ne voyons pas encore, c’est que nous l’attendons par la patience. » Il en est de notre bonheur comme de notre salut ; nous ne lepossédons qu’en espérance ; il n’est pas dans le présent, mais dans l’avenir, parce que nous sommes au milieu de maux qu’il faut supporter patiemment, jusqu’à ce que nous arrivions à la jouissance de ces biens ineffables qui ne seront traversés d’aucun déplaisir. Le salut de l’autre vie sera donc la béatitude finale, celle que nos philosophes refusent de croire, parce qu’ils ne la voient pas, substituant à sa place le fantôme d’une félicité terrestre fondée sur une trompeuse vertu, d’autant plus superbe qu’elle est plus fausse.
\subsection[{Chapitre V}]{Chapitre V}

\begin{argument}\noindent De la vie sociale et des maux qui la traversent, toute désirable qu’elle soit en elle-même.
\end{argument}

\noindent Nous sommes beaucoup plus d’accord avec les philosophes, quand ils veulent que la vie du sage soit une vie de société. Comment la Cité de Dieu (objet de cet ouvrage dont nous écrivons présentement le dix-neuvième livre) aurait-elle pris naissance, comment se serait-elle développée dans le cours des temps, et comment parviendrait-elle à sa fin, si la vie des saints n’était une vie sociale ? Mais dans notre misérable condition mortelle, qui dira tous les maux auxquels cette vie est sujette ? qui en pourra faire le compte ? Écoutez leurs poètes comiques : voici ce que dit un de leurs personnages avec l’approbation de tout l’auditoire :\par
 {\itshape « Je me suis marié, quelle misère ! j’ai eu des enfants, surcroît de soucis ! »} \par
Que dirai-je des peines de l’amour décrites par le même poète : « Injures, soupçons, inimitiés, la guerre aujourd’hui, demain la paix ! » Le monde n’est-il pas plein de ces désordres, qui troublent même les plus honnêtes liaisons ? Et que voyons-nous partout, sinon les injures, les soupçons, les inimitiés et la guerre ? Voilà des maux certains et sensibles ; mais la paix est un bien incertain, parce que chez ceux avec qui nous la voudrions entretenir, le fond des cœurs nous reste inconnu, elle connaîtrions-nous aujourd’hui, qui sait s’il ne sera pas changé demain ? En effet, où y a-t-il d’ordinaire et où devrait-il y avoir plus d’amitié que parmi les habitants du même foyer ? Et toutefois, comment y trouver une pleine sécurité, quand on voit tous les jours des parents qui se trahissent l’un l’autre, et dont la haine longtemps dissimulée devient d’autant plus amère que la paix de leur liaison semblait avoir plus de douceur ? C’est ce qui a fait dire à Cicéron cette parole qui va si droit au cœur qu’elle en tire un soupir involontaire : « Il n’y a point de trahisons plus dangereuses que celles qui se couvrent du masque de l’affection ou du nom de la parenté. Car il est aisé de se mettre en garde contre un ennemi déclaré ; mais le moyen de rompre une trame secrète, intérieure, domestique, qui vous enchaîne avant que vous ayez pu la reconnaître ou la prévoir ! » De là vient aussi ce mot de l’Écriture, qu’on ne peut entendre sans un déchirement de cœur : « Les ennemis de l’homme, ce sont les habitants de sa maison. » Et quand on aurait assez de force pour supporter patiemment une trahison, assez de vigilance pour en détourner l’effet, il ne se peut faire néanmoins qu’un homme de bien ne s’afflige beaucoup (le trouver en ses ennemis une telle perversité, soit qu’ils l’aient dès longtemps dissimulée sous une bonté trompeuse, ou que, de bons qu’ils étaient, ils soient tombés dans cet abîme de corruption. Si donc le foyer domestique n’est pas un asile assuré contre tant de maux, que sera-ce d’une cité ? Plus elle est grande, plus elle est remplie de discordes privées et de crimes, et, si elle échappe aux séditions sanglantes et aux guerres civiles, n’a-t-elle point toujours à les redouter ?
\subsection[{Chapitre VI}]{Chapitre VI}

\begin{argument}\noindent De l’erreur des jugements humains, quand la vérité est cachée.
\end{argument}

\noindent Que dirons-nous de ces jugements que les hommes prononcent sur les hommes, et qui sont nécessaires à l’ordre social dans les cités même les plus paisibles ? Triste et misérable justice, puisque ceux qui jugent ne peuvent lire dans la conscience de ceux qui sont jugés ; et de là cette nécessité déplorable de mettre à la question des témoins innocents, pour tirer d’eux la vérité dans une cause qui leur est étrangère. Que dirai-je de la torture qu’on fait subir à l’accusé pour son propre fait ? On veut savoir s’il est coupable et on commence par letorturer ; pour un crime incertain, on impose, et souvent à un innocent, une peine certaine, non que l’on sache que le patient a commis le crime, mais parce qu’on ignore s’il l’a commis en effet ? Ainsi, l’ignorance d’un juge est presque toujours la cause du malheur d’un innocent. Mais ce qui est plus odieux encore et ce qui demanderait une source de larmes, c’est que le juge, ordonnant la question de peur de faire mourir un innocent par ignorance, il arrive qu’il tue cet innocent par les moyens mêmes qu’il emploie pour ne point le faire mourir. Si, en effet, d’après la doctrine des philosophes dont nous venons de parler, le patient aime mieux sortir de la vie que de souffrir plus longtemps la question, il dira qu’il a commis le crime qu’il n’a pas commis. Le voilà condamné, mis à mort, et cependant le juge ignore s’il a frappé un coupable ou un innocent, la question ayant été inutile pour découvrir son innocence, et n’ayant même servi qu’à le faire passer pour coupable. Parmi ces ténèbres de la vie civile, un juge qui est sage montera-t-il ou non sur le tribunal ? il y montera sans doute ; car la société civile, qu’il ne croit pas pouvoir abandonner sans crime, lui en fait un devoir ; et il ne pense pas que ce soit un crime de torturer des témoins innocents pour le fait d’autrui, ou de contraindre souvent un accusé par la violence des tourments à se déclarer faussement coupable et à périr comme tel, ou, s’il échappe à la condamnation, à mourir, comme il arrive le plus souvent, dans la torture même ou par ses suites ! Il ne pense pas non plus que ce soit un crime qu’un accusateur, qui n’a dénoncé un coupable que pour le bien public et afin que le désordre ne demeure pas impuni, soit envoyé lui-même au supplice, faute de preuves, parce que l’accusé a corrompu les témoins et que la question ne lui arrache aucun aveu. Un juge ne croit pas mal faire en produisant un si grand nombre de maux, parce qu’il ne les produit pas à dessein, mais par une ignorance invincible et par une obligation indispensable de la société civile ; mais si on ne peut l’accuser de malice, c’est toujours une grande misère qu’uneobligation pareille, et si la nécessité l’exempte de crime, quand il condamne des innocents et sauve des coupables, osera-t-on l’appeler bienheureux ? Ah ! qu’il fera plus sagement de reconnaître et de haïr la misère où cette nécessité l’engage ; et s’il a quelque sentiment de piété, de crier à Dieu : « Délivrez-moi de mes nécessités ! »
\subsection[{Chapitre VII}]{Chapitre VII}

\begin{argument}\noindent De la diversité des langues qui rompt la société des hommes, et de la misère des guerres, même les plus justes.
\end{argument}

\noindent Après la cité, l’univers, troisième degré de la société civile ; car le premier, c’est la maison. Or, à mesure que le cercle s’agrandit, les périls s’accumulent. Et d’abord, la diversité des langues ne rend-elle pas l’homme en quelque façon étranger à l’homme ? Que deux personnes, ignorant chacune la langue de l’autre, viennent à se rencontrer, et que la nécessité les oblige à demeurer ensemble, deux animaux muets, même d’espèce différente, s’associeront plutôt que ces deux créatures humaines, et un homme aimera mieux être avec son chien qu’avec un étranger. Mais, dira-t-on, voici qu’une Cité faite pour l’empire, en imposant sa loi aux nations vaincues, leur a aussi donné sa langue, de sorte que les interprètes, loin de manquer, sont en grande abondance. Cela est vrai ; mais combien de guerres gigantesques, de carnage et de sang humain a-t-il fallu pour en venir là ? Et encore, ne sommes-nous pas au bout de nos maux. Sans parler des ennemis extérieurs qui n’ont jamais manqué à l’empire romain et qui chaque jour le menacent encore, la vaste étendue de son territoire n’a-t-elle pas produit ces guerres mille fois plus dangereuses, guerres civiles, guerres sociales, fléaux du genre humain, dont la crainte seule est un grand mal ? Que si j’entreprenais de peindre ces horribles calamités avec les couleurs qu’un tel sujet pourraitrecevoir, mais que mon insuffisance ne saurait lui donner, quand verrait-on la fin de ce discours ? Mais, dira-t-on, le sage n’entreprendra que des guerres justes. Eh ! n’est-ce pas cette nécessité même de prendre les armes pour la justice qui doit combler le sage d’affliction, si du moins il se souvient qu’il est homme ? Car enfin, il ne peut faire une guerre juste-que pour punir l’injustice de ses adversaires, et cette injustice des hommes, même sans le cortège de la guerre, voilà ce qu’un homme ne peut pas ne pas déplorer. Certes, quiconque considérera des maux si grands et si cruels tombera d’accord qu’il y a là une étrange misère. Et s’il se rencontre un homme pour subir ces calamités ou seulement pour les envisager sans douleur, il est d’autant plus misérable de se croire heureux, qu’il ne se croit tel que pour avoir perdu tout sentiment humain.
\subsection[{Chapitre VIII}]{Chapitre VIII}

\begin{argument}\noindent Il ne peut y avoir pleine sécurité, même dans l’amitié des honnêtes gens, à cause des dangers dont la vie humaine est toujours menacée.
\end{argument}

\noindent Certes, s’il est une consolation parmi les agitations et les peines de la société humaine, c’est la foi sincère et l’affection réciproque de bons et vrais amis. Mais outre qu’une sorte d’aveuglement, voisin de la démence et toutefois très fréquent en cette vie, nous fait prendre un ennemi pour un ami, ou un ami pour un ennemi, n’est-il pas vrai que plus nous avons d’amis excellents et sincères, plus nous appréhendons pour eux les accidents dont la condition humaine est remplie ? Nous ne craignons pas seulement qu’ils soient affligés par la faim, les guerres, les maladies, la captivité et tous les malheurs qu’elle entraîne à sa suite ; nous craignons bien plus encore, c’est qu’ils ne deviennent perfides et méchants. Et quand cela arrive, qui peut concevoir l’excès de notre douleur, à moins que de l’avoir éprouvé soi-même ? Nous aimerions mieux savoir nos amis morts ; et cependant, quoi de plus capable qu’une telle perte de nous causer un sensible déplaisir ? Car, comment se pourrait-il faire que nous ne fussions point affligés de la mort de ceux dont la vie nous était-si agréable ? Que celui qui proscrit cette douleur, proscrive aussi le charme des entretiens affectueux, qu’il interdise l’amitié elle-même, qu’il rompe les liens les plus doux de la société humaine, en un mot, qu’il rende l’homme stupide. Et si cela est impossible, comment ne serions-nous pas touchés de la mort de personnes si chères ? De là ces deuils intérieurs et ces blessures de l’âme qui ne se peuvent guérir que par la douceur des consolations ; car dire que ces blessures se referment d’autant plus vite que l’âme est plus grande et plus forte, cela ne prouve pas qu’il n’y ait point dans l’âme une plaie à guérir. Ainsi, bien que la mort des personnes les plus chères, de celles surtout qui font les liens de la vie, soit une épreuve toujours plus ou moins cruelle, nous aimerions mieux toutefois les voir mourir que déchoir de la foi ou de la vertu, ce qui est mourir de la mort de l’âme. La terre est donc pleine d’une immense quantité de maux, et c’est pourquoi il est écrit « Malheur au monde à cause des scandales ! » Et encore : « Comme l’injustice surabonde, la charité de plusieurs se refroidira. » Voilà comment nous en venons à nous féliciter de la mort de nos meilleurs amis ; notre cœur, abattu par la tristesse, se relève à cette pensée que la mort a délivré nos frères de tous les maux qui accablent les plus vertueux, souvent les corrompent et toujours les mettent en péril.
\subsection[{Chapitre IX}]{Chapitre IX}

\begin{argument}\noindent Nous ne pouvons être assurés en cette vie de l’amitié des saints anges, à cause de la fourberie des démons, qui ont su prendre dans leurs pièges les adorateurs des faux dieux.
\end{argument}

\noindent Quant aux saints anges, c’est-à-dire à la quatrième société qu’établissent les philosophes qui veulent que nous ayons les dieux pour amis, nous ne craignons pas pour eux ni qu’ils meurent, ni qu’ils deviennent méchants. Mais comme nous ne conversons pas avec eux aussi familièrement qu’avec les hommes, et comme aussi il arrive souvent, selon ce que nous apprend l’Écriture, que Satan se transforme un ange de lumière pour tenter ceux qui ont besoin d’être éprouvés de la sorte ou qui méritent d’être trompés, la miséricorde de Dieu nous est bien nécessaire pour nous empêcher de prendre pour amis les démons au lieu des saints anges. N’est-cepas encore là une des grandes misères de la vie que d’être sujets à cette méprise ? Il est certain que ces philosophes, qui ont cru avoir les dieux pour amis, sont tombés dans le piège, et cela paraît assez par les sacrifices impies qu’on offrait à ces prétendus dieux, et par les jeux infâmes qu’on représentait en leur honneur et à leur sollicitation.
\subsection[{Chapitre X}]{Chapitre X}

\begin{argument}\noindent Quelle récompense est préparée aux saints qui ont surmonté les tentations de cette vie.
\end{argument}

\noindent Les saints mêmes et les fidèles adorateurs du seul vrai Dieu ne sont pas à couvert de la fourberie des démons et de leurs tentations toujours renaissantes. Mais cette épreuve ne leur est pas inutile pour exciter leur vigilance et leur faire désirer avec plus d’ardeur le séjour où l’on jouit d’une paix et d’une félicité accomplies. C’est là, en effet, que le corps et l’âme recevront du Créateur universel des natures toutes les perfections dont la leur est capable, l’âme étant guérie par la sagesse et le corps renouvelé par la résurrection. C’est là que les vertus n’auront plus de vices à combattre, ni de maux à supporter, mais qu’elles posséderont, pour prix de leur victoire, une paix éternelle qu’aucune puissance ennemie ne viendra troubler. Voilà la béatitude finale, voilà le terme suprême et définitif de la perfection. Le monde nous appelle heureux quand nous jouissons de la paix, telle qu’elle peut être en ce monde, c’est-à-dire telle qu’une bonne vie la peut donner ; mais cette béatitude, au prix de celle dont nous parlons, est une véritable misère. Or, cette paix imparfaite, quand nous la possédons, quel est le devoir de la vertu, sinon de faire un bon usage des biens qu’elle nous procure ? Et, quand elle vient à nous manquer, la vertu peut encore bien user des maux mêmes de notre condition mortelle. La vraie vertu consiste donc à faire un bon usage des biens et des maux de cette vie, avec cette condition essentielle de rapporter tout ce qu’elle fait et de se rapporter elle-même à la fin dernière qui nous doit mettre en possession d’une parfaite et incomparable paix.
\subsection[{Chapitre XI}]{Chapitre XI}

\begin{argument}\noindent Du bonheur de la paix éternelle, fin suprême et véritable perfection des saints.
\end{argument}

\noindent Nous pouvons dire de la paix ce que nous avons dit de la vie éternelle, qu’elle est la fin de nos biens, d’autant mieux que le Prophète, parlant de la Cité de Dieu, sujet de ce laborieux ouvrage, s’exprime ainsi : « Jérusalem, louez le Seigneur ; Sion, louez votre Dieu ; car il a consolidé les verrous de vos portes ; il a béni vos enfants en vous, et c’est lui qui a établi la paix comme votre fin. » En effet, quand seront consolidés les verrous des portes de Sion, nul n’y entrera, ni n’en sortira plus ; et ainsi, par cette fin dont parle le psaume, il faut entendre cette paix finale que nous cherchons ici à définir. Le nom même de la Cité sainte, c’est-à-dire Jérusalem, est un nom mystérieux qui signifie {\itshape vision de paix}. Mais, comme on se sert aussi du nom de paix dans les choses de cette vie périssable, nous avons mieux aimé appeler vie éternelle la fin où la Cité de Dieu doit trouver son souverain bien. C’est de cette fin que l’Apôtre dit : « Et maintenant, affranchis du péché et devenus les esclaves de Dieu, vous avez pour fruit votre sanctification, et pour fin la vie éternelle. » D’un autre côté, ceux qui ne sont pas versés dans l’Écriture sainte, pouvant aussi entendre par la vie éternelle celle des méchants, soit parce que l’âme humaine est immortelle, ainsi que l’ont reconnu quelques philosophes, soit parce que les méchants ne pourraient pas subir les tourments éternels que la foi nous enseigne, s’ils ne vivaient éternellement, il vaut mieux appeler la fin dernière où la Cité de Dieu goûtera son souverain bien : la paix dans la vie éternelle, ou la vie éternelle dans la paix. Aussi bien qu’y a-t-il de meilleur que la paix, même dans les choses mortelles et passagères ? Quoi de plus agréable à entendre, de plus souhaitable à désirer, de plus précieux à conquérir ? Il ne sera donc pas, ce me semble, hors de propos d’en dire ici quelque chose à l’occasion de la paix souveraine et définitive. C’est un bien si doux que la paix, et si cher à tout le monde, que ce que j’en dirai ne sera désagréable à personne.
\subsection[{Chapitre XII}]{Chapitre XII}

\begin{argument}\noindent Que les agitations des hommes et la guerre elle-même tendent à la paix, terme nécessaire ou aspirent tous les êtres.
\end{argument}

\noindent Quiconque observera d’un œil attentif les affaires humaines et la nature des choses reconnaîtra que, s’il n’y a personne qui ne veuille éprouver de la joie, il n’y a non plus personne qui ne veuille goûter la paix. En effet, ceux mêmes qui font la guerre ne la font que pour vaincre, et par conséquent pour parvenir glorieusement à la paix. Qu’est-ce que la victoire ? c’est la soumission des rebelles, c’est-à-dire la paix. Les guerres sont donc toujours faites en vue de la paix, même par ceux qui prennent plaisir à exercer leur vertu guerrière dans les combats ; d’où il faut conclure que le véritable but de la guerre, c’est la paix, l’homme qui fait la guerre cherchant la paix, et nul ne faisant la paix pour avoir la guerre. Ceux mêmes qui rompent la paix à dessein n’agissent point ainsi par haine pour cette paix, mais pour en obtenir une meilleure. Leur volonté n’est pas qu’il n’y ait point de paix, mais qu’il y ait une paix selon leur volonté. Et s’ils viennent à se séparer des autres par une révolte, ils ne sauraient venir à bout de leurs desseins qu’à condition d’entretenir avec leurs complices une espèce de paix. De là vient que les voleurs mêmes conservent la paix entre eux, afin de la pouvoir troubler plus impunément chez les autres. Que s’il se trouve quelque malfaiteur si puissant et si ennemi de toute société qu’il ne s’unisse avec personne et qu’il exécute seul ses meurtres et ses brigandages, pour le moins conserve-t-il toujours quelque ombre de paix avec ceux qu’il ne peut tuer et à qui il veut cacher ce qu’il fait. Dans sa maison, il a soin de vivre en paix avec sa femme, avec ses enfants et avec ses domestiques, parce qu’il désire en être obéi. Rencontre-t-il une résistance, il s’emporte, il réprime, il châtie, et, s’il le faut, il a recours à la cruauté pour maintenir la paix dans sa maison, sachant bien qu’elle n’est possible qu’avec un chef à qui tous les membres de la société domestique soient assujettis. Si donc une ville ou tout un peuple voulait se soumettre à lui de la même façon qu’il désire que ceux de sa maison lui soient soumis, il ne se cacherait plus dans une caverne comme un brigand ; il monterait sur le trône comme un roi. Chacun souhaite donc d’avoir la paix avec ceux qu’il veut gouverner à son gré, et quand un homme fait la guerre à des hommes, c’est pour les rendre siens, en quelque sorte, et leur dicter ses conditions de paix.\par
Supposons un homme comme celui de la fable et des poètes, farouche et sauvage au point de n’avoir aucun commerce avec personne. Pour royaume, il n’avait qu’un antre désert et affreux ; et il était si méchant qu’on l’avait appelé Cacus, nom qui exprime la méchanceté. Près de lui, point de femme, pour échanger des paroles affectueuses ; point d’enfants dont il pût partager les jeux dans leur jeune âge et guider plus tard l’adolescence ; point d’amis enfin avec qui s’entretenir, car il n’avait pas même pour ami Vulcain, son père : plus heureux du moins que ce dieu, en ce qu’il n’engendra point à son tour un monstre semblable à lui-même. Loin de rien donner à personne, il enlevait aux autres tout ce qu’il pouvait ; et cependant, au fond de cette caverne, toujours trempée, comme dit le poète, de quelque massacre récent, que voulait-il ? posséder la paix, goûter un repos que nulle crainte et nulle violence ne pussent troubler. Il voulait enfin avoir la paix avec son corps, et ne goûtait de bonheur qu’autant qu’il jouissait de cette paix. Il commandait à ses membres, et ils lui obéissaient ; mais afin d’apaiser cette guerre intestine que lui faisait la faim, et d’empêcher qu’elle chassât son âme de son corps, il ravissait, tuait, dévorait, ne déployant cette cruauté barbare que pour maintenir la paix entre les deux parties dont il était composé ; de sorte que, s’il eût voulu entretenir avec les autres la paix qu’il tâchait de se procurer à lui-même dans sa caverne, on ne l’eût appelé ni méchant ni monstre. Que si l’étrange figure de son corps et les flammes qu’il vomissait par la bouche l’empêchaient d’avoir commerce avec les hommes, peut-être était-il féroce à ce point, beaucoup moins par le désir de faire du mal que par la nécessité de vivre. Mais disons plutôt qu’un tel homme n’a jamais existé que dans l’imagination des poètes, qui ne l’ont dépeint de la sorte qu’afin de relever à ses dépens la gloire d’Hercule. En effet, les animaux mêmes les plus sauvages s’accouplent et ont des petits qu’ils nourrissent et qu’ils élèvent ; et je ne parle pas ici des brebis, des cerfs, des colombes, des étourneaux, des abeilles, mais des lions, des renards, des vautours, des hiboux. Un tigre devient doux pour ses petits et les caresse ; un milan, quelque solitaire et carnassier qu’il soit, cherche une femelle, fait son nid, couve ses œufs, nourrit ses petits, et se maintient en paix dans sa maison avec sa compagne comme avec une sorte de mère de famille. Combien donc l’homme est-il porté plus encore par les lois de sa nature à entrer en société avec les autres hommes et à vivre en paix avec eux ! C’est au point que les méchants mêmes combattent pour maintenir la paix des personnes qui leur appartiennent, et voudraient, s’il était possible, que tous les hommes leur fussent soumis, afin que tout obéît à un seul et fût en paix avec lui, soit par crainte, soit par amour. C’est ainsi que l’orgueil, dans sa perversité, cherche à imiter Dieu. Il ne veut point avoir de compagnons sous lui, mais il veut être maître au lieu de lui. Il hait donc la juste paix de Dieu, et il aime la sienne, qui est injuste ; car il faut qu’il en aime une, quelle qu’elle soit, n’y ayant point de vice tellement contraire à la nature qu’il n’en laisse subsister quelques vestiges.\par
Celui donc qui sait préférer la droiture à la perversité, et ce qui est selon l’ordre à ce qui est contre l’ordre, reconnaît que la paix des méchants mérite à peine ce nom en comparaison de celle des gens de bien. Et cependant il faut de toute nécessité que ce qui est contre l’ordre entretienne la paix à quelques égards avec quelqu’une des parties dont il est composé ; autrement il cesserait d’être. Supposons un homme suspendu par les pieds, la tête en bas, voilà l’ordre et la situation de ses membres renversés, ce qui doit être naturellement au-dessus étant au-dessous. Ce désordre trouble donc la paix du corps, et c’est en cela qu’il est pénible. Toutefois, l’âme ne cesse pas d’être en paix avec son corps et de travailler à sa conservation, sans quoi il n’y aurait ni douleur, ni patient qui la ressentît. Que si l’âme, succombant sous les maux que le corps endure, vient à s’en séparer, tant que l’union des membres subsiste, il y a toujours quelque sorte de paix entre eux ; ce qui fait qu’on peut encore dire : Voilà un homme qui est pendu. Pourquoi le corps du patient tend-il vers la terre et se débat-il contre le lien qui l’enchaîne ? C’est qu’il veut jouir de la paix qui lui est propre. Son poids est comme la voix par laquelle il demande qu’on le mette en un lieu de repos, et, quoique privé d’âme et de sentiment, il ne s’éloigne pourtant pas de la paix convenable à sa nature, soit qu’il la possède, soit qu’il y tende. Si on l’embaume pour l’empêcher de se dissoudre, il y a encore une sorte de paix entre ses parties, qui les tient unies les unes aux autres, et qui fait que le corps tout entier demeure dans un était convenable, c’est-à-dire dans un état paisible. Si on ne l’embaume point, il s’établit un combat des vapeurs contraires qui sont en lui et qui blessent nos sens, ce qui produit la putréfaction, jusqu’à ce qu’il soit d’accord avec les éléments qui l’environnent, et qu’il retourne pièce à pièce dans chacun d’eux. Au milieu de ces transformations, dominent toujours les lois du souverain Créateur, qui maintient l’ordre et la paix de l’univers ; car, bien que plusieurs petits animaux soient engendrés du cadavre d’un animal plus grand, chacun d’eux, par la loi du même Créateur, a soin d’entretenir avec soi-même la paix nécessaire à sa conservation. Et quand le corps mort d’un animal serait dévoré par d’autres, il rencontrerait toujours ces mêmes lois partout répandues, qui savent unir chaque chose à celle qui lui est assortie, quelque désunion et quelque changement qu’elle ait pu souffrir.
\subsection[{Chapitre XIII}]{Chapitre XIII}

\begin{argument}\noindent La paix universelle, fondée sur les lois de la nature, ne peut être détruite par les plus violentes passions, le juge équitable et souverain faisant parvenir chacun a la condition qu’il a méritée.
\end{argument}

\noindent Ainsi la paix du corps réside dans le juste tempérament de ses parties, et celle de l’âme sensible dans le calme régulier de ses appétits satisfaits. La paix de, l’âme raisonnable, c’est en elle le parfait accord de la connaissance et de l’action ; et celle du corps et de l’âme, c’est la vie bien ordonnée et la santé de l’animal. La paix entre l’homme mortel et Dieu est une obéissance réglée par la foi et soumise à la loi éternelle ; celle des hommes entre eux, une concorde raisonnable. La paix d’une maison, c’est une juste correspondance entre ceux qui y commandent et ceux qui y obéissent. La paix d’une cité, c’est la même correspondance entre ses membres. La paix de la Cité céleste consiste dans une union très réglée et très parfaite pour jouir de Dieu, et du prochain en Dieu ; et celle de toutes choses, c’est un ordre tranquille. L’ordre est ce qui assigne aux choses différentes la place qui leur convient. Ainsi, bien que les malheureux, en tant que tels, ne soient point en paix, n’étant point dans cet ordre tranquille que rien ne trouble, toutefois, comme ils sont justement malheureux, ils ne peuvent pas être tout à fait hors de l’ordre. À la vérité, ils ne sont pas avec les bienheureux ; mais au moins c’est la loi de l’ordre qui les en sépare. Ils sont troublés et inquiétés, et toutefois ils ne laissent pas d’avoir quelque convenance avec leur état. Ils ont dès lors quelque ombre de tranquillité dans leur ordre ; ils ont donc aussi quelque paix. Mais ils sont malheureux, parce qu’encore qu’ils soient dans le lieu où ils doivent être, ils ne sont pas dans le lieu où ils n’auraient rien à souffrir : moins malheureux toutefois encore que s’ils n’avaient point de convenance avec le lieu où ils sont. Or, quand ils souffrent, la paix est troublée à cet égard ; mais elle subsiste dans leur nature, que la douleur ne peut consumer ni détruire, et à cet autre égard, ils sont en paix. De même qu’il y a quelque vie sans douleur, et qu’il ne peut y avoir de douleur sans quelque vie ; ainsi il y a quelque paix sans guerre, mais il ne peut y avoir de guerre sans quelque paix, puisque la guerre suppose toujours quelque nature qui l’entretienne, et qu’une nature ne saurait subsister sans quelque sorte de paix.\par
Ainsi il existe une Nature souveraine où il ne se trouve point de mal et où il ne peut même s’en trouver ; mais il ne saurait exister de nature où ne se trouve aucun bien. Voilà pourquoi la nature du diable même n’est pas mauvaise en tant que nature ; la seule malice la rend telle. C’est pour cela qu’il n’est pas demeuré dans la vérité ; mais il n’a pu se soustraire au jugement de la vérité. Il n’est pas demeuré dans un ordre tranquille ; mais il n’a pas toutefois évité la puissance du souverain ordonnateur. La bonté de Dieu, qui a fait sa nature, ne le met pas à couvert de la justice de Dieu, qui conserve l’ordre en le punissant, et Dieu ne punit pas en lui ce qu’il a créé, mais le mal que sa créature a commis. Dieu ne lui ôte pas tout ce qu’il a donné à sa nature, mais seulement quelque chose, lui laissant le reste, afin qu’il subsiste toujours pour souffrir de ce qu’il a perdu. La douleur même qu’il ressent est un témoignage du bien qu’on lui a ôté et de celui qu’on lui a laissé, puisque, s’il ne lui était encore demeuré quelque bien, il ne pourrait pas s’affliger de celui qu’il a perdu. Car le pécheur est encore pire, s’il se réjouit de la perte qu’il fait de l’équité ; mais le damné, s’il ne retire aucun bien de ses tourments, au moins s’afflige-t-il de la perte de son salut. Comme l’équité et le salut sont deux biens, et qu’il faut plutôt s’affliger que se réjouir de la perte d’un bien, à moins que cette perte ne soit compensée d’ailleurs, les méchants ont sans doute plus de raison de s’affliger de leurs supplices qu’ils n’en ont eu de se réjouir de leurs crimes. De même que se réjouir, lorsqu’on pèche, est une preuve que la volonté est mauvaise ; s’affliger, lorsqu’on souffre, est aussi une preuve que la nature est bonne. Aussi bien celui qui s’afflige d’avoir perdu la paix de sa nature ne s’afflige que par certains restes de paix qui font qu’il aime sa nature. Or, c’est très justement que dans le dernier supplice les méchants déplorent, au milieu de leurs tortures, la perte qu’ils ont faite des biens naturels, et qu’ils sentent que celui qui les leur ôte est ce Dieu très juste envers qui ils ont été ingrats. Dieu donc, qui a créé toutes les natures avec une sagesse admirable, qui les ordonne avec une souveraine justice et qui a placé l‘homme sur la terre pour en être le plus bel ornement, nous a donné certains biens convenables à cette vie, c’est-à-dire la paix temporelle, dans la mesure où on peut l’avoir ici-bas, tant avec soi-même qu’avec les autres, et toutes les choses nécessaires peur la conserver ou pour la recouvrer, comme la lumière, l’air, l’eau, et tout ce qui sert à nourrir, à couvrir, à guérir ou à parer le corps, mais sous cette condition très équitable, que ceux qui feront bon usage de ces biens en recevront de plus grands et de meilleurs, c’est-à-dire une paix immortelle accompagnée d’une gloire sans fin et de la-jouissance de Dieu et du prochain en Dieu, tandis que ceux qui en feront mauvais usage perdront même ces biens inférieurs et n’auront pas les autres.
\subsection[{Chapitre XIV}]{Chapitre XIV}

\begin{argument}\noindent De l’ordre à la fois divin et terrestre qui fait que les maîtres de la société humaine en sont aussi les serviteurs.
\end{argument}

\noindent Tout l’usage des choses temporelles se rapporte dans la cité de la terre à la paix terrestre, dans la cité de Dieu à la paix éternelle. C’est pour cela que, si nous étions des animaux sans raison, nous ne désirerions rien que le juste tempérament des parties du corps et la satisfaction de nos appétits ; et la paix du corps servirait à la paix de l’âme ; car celle-ci ne peut subsister sans l’autre, mais elles s’aident mutuellement pour le bien du tout. De même en effet que les animaux font voir qu’ils aiment la paix du corps en fuyant la douleur, et celle de l’âme, lorsqu’ils cherchent la volupté pour contenter leurs appétits, ils montrent aussi en fuyant la mort combien ils aiment la paix qui fait l’union du corps et de l’âme. Mais l’homme, doué d’une âme raisonnable, fait servir à la paix de cette âme tout ce qu’il a de commun avec les bêtes, afin de contempler et d’agir, c’est-à-dire afin d’entretenir une juste harmonie entre la connaissance et l’action, en quoi consiste la paix de l’âme raisonnable. Il doit, pour cette raison, souhaiter que nulle douleur ne le tourmente, que nul désir ne l’inquiète, et que la mort ne sépare point les deux parties qui le composent, afin de se livrer à la connaissance des choses utiles, et de régler sa vie et ses mœurs sur cette connaissance. Toutefois comme son esprit est faible, s’il veut que le désir même de connaître ne l’engage point dans quelque erreur, il a besoin de l’enseignement de Dieu pour connaître avec certitude et de son secours pour agir avec liberté. Or, tant qu’il habite dans ce corps mortel, il est en quelque sorte étranger à l’égard de Dieu, et marche par la foi, comme dit l’Apôtre, et non par la claire vision il faut donc qu’il rapporte et la paix du corps et celle de l’âme, et celle enfin des deux ensemble, à cette paix supérieure qui est entre l’homme mortel et Dieu immortel, afin que son obéissance soit réglée par la foi et soumise à la loi éternelle. Et puisque ce divin maître enseigne deux choses principales, d’abord l’amour de Dieu, et puis l’amour du prochain où est renfermé l’amour de soi-même (lequel ne peut jamais égarer celui qui aime Dieu),il s’ensuit que chacun doit porter son prochain à aimer Dieu, pour obéir au précepte qui lui commande de l’aimer comme il s’aime lui-même. Il doit donc rendre cet office de charité à sa femme, à ses enfants, à ses domestiques et à tous les hommes, autant que possible, comme il doit vouloir que les autres le lui rendent, s’il en est besoin ; et ainsi il aura la paix avec tous, autant que cela dépendra de lui : j’entends une paix humaine, c’est-à-dire cette concorde bien réglée, dont la première loi est de ne faire tort à personne, et la seconde de faire du bien à qui l’on peut. En conséquence, l’homme commencera par prendre soin des siens ; car la nature et la société lui donnent auprès de ceux-là un accès plus facile et des moyens de secours plus opportuns. C’est ce qui fait dire à l’Apôtre, que « quiconque n’a pas soin des siens, et particulièrement de ceux de sa maison, est apostat et pire qu’un infidèle. » Voilà aussi d’où naît la paix domestique, c’est-à-dire la bonne intelligence entre ceux qui commandent et ceux qui obéissent dans une maison. Ceux-là y commandent qui ont soin des autres, comme le mari commande à la femme, le père et la mère aux enfants, et les maîtres aux serviteurs ; et les autres obéissent, comme les femmes à leurs maris ; les enfants à leurs pères et à leurs mères, et les serviteurs à leurs maîtres. Mais dans la maison d’un homme de bien qui vit de la foi et qui est étranger ici-bas, ceux qui commandent servent ceux à qui ils semblent commander ; car ils commandent, non par un esprit de domination, mais par un esprit de charité ; ils ne veulent pas donner avec orgueil des ordres, mais avec bonté des secours.
\subsection[{Chapitre XV}]{Chapitre XV}

\begin{argument}\noindent La première cause de la servitude, c’est le péché, et l’homme, naturellement libre, devient, par sa mauvaise volonté, esclave de ses passions, alors même qu’il n’est pas dans l’esclavage d’autrui.
\end{argument}

\noindent Voilà ce que demande l’ordre naturel et voilà aussi la condition où Dieu a créé l’homme : « Qu’il domine, dit-il, sur les poissons de la mer, sur les oiseaux du ciel et sur tous les animaux de la terre. » Après avoir créé l’homme raisonnable et l’avoir fait à son image, il n’a pas voulu qu’il dominât sur leshommes, mais sur les bêtes. C’est pourquoi les premiers justes ont été plutôt bergers que rois, Dieu voulant nous apprendre par là l’ordre de la nature, qui a été renversé par le désordre du péché. Car c’est avec justice que le joug de la servitude a été imposé au pécheur. Aussi ne voyons-nous point que l’Écriture sainte parle d’esclaves avant que le patriarche Noé n’eût flétri le péché de son fils de ce titre honteux. Le péché seul a donc mérité ce nom, et non pas la nature. Si l’on en juge par l’étymologie latine, les esclaves étaient des prisonniers de guerre à qui les vainqueurs {\itshape conservaient} la vie, alors qu’ils pouvaient les tuer par le droit de guerre : or, cela même fait voir dans l’esclavage une peine du péché. Car on ne saurait faire une guerre juste que les ennemis n’en fassent une injuste ; et toute victoire, même celle que remportent les méchants, est un effet des justes jugements de Dieu, qui humilie par là les vaincus, soit qu’il veuille les amender, soit qu’il veuille les punir. Témoin ce grand serviteur de Dieu, Daniel, qui, dans la captivité, confesse ses péchés et ceux de son peuple, et y reconnaît avec une juste douleur l’unique raison de toutes leurs infortunes. La première cause de la servitude est donc le péché, qui assujettit un homme à un homme ; ce qui n’arrive que par le jugement de Dieu, qui n’est point capable d’injustice et qui sait imposer des peines différentes selon la différence des coupables. Notre-Seigneur dit : « Quiconque pèche est esclave du péché » ; et ainsi il y a beaucoup de mauvais maîtres qui ont des hommes pieux pour esclaves et qui n’en sont pas plus libres pour cela. Car il est écrit : « L’homme est adjugé comme esclave à celui qui l’a vaincu. » Et certes il vaut mieux être l’esclave d’un homme que d’une passion ; car est-il une passion, par exemple, qui exerce une domination plus cruelle sur le cœur deshommes que la passion de dominer ? Aussi bien, dans cet ordre de choses qui soumet quelques hommes à d’autres hommes, l’humilité est aussi avantageuse à l’esclave que l’orgueil est funeste au maître. Mais dans l’ordre naturel où Dieu a créé l’homme, nul n’est esclave de l’homme ni du péché ; l’esclavage est donc une peine, et elle a été imposée par cette loi qui commande de conserver l’ordre naturel et qui défend de le troubler, puisque, si l’on n’avait rien fait contre cette loi, l’esclavage n’aurait rien à punir. C’est pourquoi l’Apôtre avertit les esclaves d’être soumis à leurs maîtres, et de les servir de bon cœur et de bonne volonté, afin que, s’ils ne peuvent être affranchis de leur servitude, ils sachent y trouver la liberté, en ne servant point par crainte, mais par amour, jusqu’à ce que l’iniquité passe et que toute domination humaine soit anéantie, au jour où Dieu sera tout en tous.
\subsection[{Chapitre XVI}]{Chapitre XVI}

\begin{argument}\noindent De la juste damnation.
\end{argument}

\noindent Aussi nous voyons que les patriarches ne mettaient de différence entre leurs enfants et leurs esclaves que relativement aux biens temporels ; mais pour ce qui regardait le culte de Dieu, de qui nous attendons les biens éternels, ils veillaient avec une affection égale sur tous les membres de leur maison ; et cela est si conforme à l’ordre naturel, que le nom de père de famille en tire son origines, et s’est si bien établi dans le monde que les méchants eux-mêmes aiment à être appelés de ce nom. Mais ceux qui sont vrais pères de famille veillent avec une égale sollicitude à ce que tous les membres de leur maison, qui sont tous en quelque façon leurs enfants, servent et honorent Dieu, et désirent parvenir à cette maison céleste où il ne sera plus nécessaire de commander aux hommes, parce qu’ils n’auront plus de besoins auxquels il faille pourvoir ; et jusque-là, les bons maîtres portent avec plus de peine le poids du commandement que les serviteurs celui de l’esclavage. Or, si quelqu’un vient à troubler la paix domestique, il faut le châtier pour son utilité, autant que cela peut se faire justementafin de le ramener à la paix dont il s’était écarté. Comme ce n’est pas être bienfaisant que de venir en aide à une personne pour lui faire perdre un plus grand bien, ce n’est pas non plus être innocent que de la laisser tomber dans un plus grand mal sous prétexte de lui en épargner un petit. L’innocence demande non seulement qu’on ne nuise à personne, mais encore qu’on empêche son prochain de mal faire, ou qu’on le châtie quand il a mal fait, soit afin de le corriger lui-même, soit au moins pour retenir les autres par cet exemple. Du moment donc que la maison est le germe et l’élément de la cité, tout germe, tout commencement devant se rapporter à sa fin, et tout élément, toute partie à son tout, il est visible que la paix de la maison doit se rapporter à celle de la cité, c’est-à-dire l’accord du commandement et de l’obéissance parmi les membres de la même famille à ce même accord parmi les membres de la même cité. D’où il suit que le père de famille doit régler sur la loi de la cité la conduite de sa maison, afin qu’il y ait accord entre la partie et le tout.
\subsection[{Chapitre XVII}]{Chapitre XVII}

\begin{argument}\noindent D’où viennent la paix et la discorde entre la Cité du ciel et celle de la terre.
\end{argument}

\noindent Mais ceux qui ne vivent pas de la foi cherchent la paix de leur maison dans les biens et les commodités de cette vie, au lieu que ceux qui vivent de la foi attendent les biens éternels de l’autre vie qui leur ont été promis, et se servent des félicités temporelles comme des voyageurs et des étrangers, non pour y mettre leur cœur et se détourner de Dieu, mais pour y trouver quelque soulagement et se rendre en quelque façon plus supportable le poids de ce corps corruptible qui appesantit l’âme. Ainsi il est vrai que l’usage des choses nécessaires à la vie est commun aux uns et aux autres dans le gouvernement de leur maison ; mais la fin à laquelle ils rapportent cet usage est bien différente. Il en est de même de la cité de la terre, qui ne vit pas de la foi. Elle recherche la paix temporelle, et l’unique but qu’elle se propose dans la concorde qu’elle tâche d’établir parmi ses membres, c’est de jouir plus aisément du repos et des plaisirs. Mais la cité céleste, ou plutôt la partie de cettecité qui traverse cette vie mortelle et qui vit de la foi, ne se sert de cette paix que par nécessité, en attendant que tout ce qu’il y a de mortel en elle passe. C’est pourquoi, tandis qu’elle est comme captive dans la cité de la terre, où toutefois elle a déjà reçu la promesse de sa rédemption et le don spirituel comme un gage de cette promesse, elle ne fait point difficulté d’obéir aux lois qui servent à régler les choses nécessaires à la vie mortelle ; car cette vie étant commune aux deux cités, il est bon qu’il y ait entre elles, pour tout ce qui s’y rapporte, une concorde réciproque. Mais la cité de la terre ayant eu certains sages, dont la fausse sagesse est condamnée par l’Écriture, et qui, sur la fol de leurs conjectures ou des conseils trompeurs des démons, ont cru qu’il fallait se rendre favorable une multitude de dieux, comme ayant autorité chacun sur diverses choses, l’un sur le corps, l’autre sur l’âme, et dans le corps même, celui-ci sur la tête, celui-là sur le cou, et ainsi des autres membres, et dans l’âme aussi, l’un sur l’esprit, l’autre sur la science, ou sur la colère, ou sur l’amour, et enfin dans les choses qui servent à la vie, celui-ci sur les troupeaux, cet autre sur les blés ou sur les vigiles, et ainsi du reste ; comme, d’un autre côté, la Cité céleste ne reconnaissait qu’un seul Dieu, et croyait qu’à lui seul était dû le culte de latrie, elle n’a pu par ces raisons avoir une religion commune avec la cité de la terre, et elle s’est trouvée obligée de différer d’elle à cet égard ; de sorte qu’elle aurait couru le risque d’être toujours exposée à la haine et aux persécutions de ses ennemis, s’ils n’eussent enfin été effrayés du nombre de ceux qui embrassaient son parti et de la protection visible que leur-accordait le ciel. Voilà donc comment cette Cité céleste, en voyageant sur la terre, attire à elle des citoyens de toutes les nations, et ramasse de tous les endroits du monde une société voyageuse comme elle, sans se mettre en peine de la diversité des mœurs, du langage et des coutumes de ceux qui la composent, pourvu que cela ne les empêche point de servir le même Dieu. Elle use d’ailleurs, pendant son pèlerinage, de la paix temporelle et des choses qui sont nécessairement attachées ànotre mortelle condition ; elle désire et protège le bon accord des volontés, autant que la piété et la religion le peuvent permettre, et rapporte la paix terrestre à la céleste, qui est la paix véritable, celle que la créature raisonnable peut seule appeler de ce nom, et qui consiste dans une union très réglée et très parfaite pour jouir de Dieu et du prochain en Dieu. Là, notre vie ne sera plus mortelle, ni notre corps animal ; nous posséderons une vie immortelle et un corps spirituel qui ne souffrira d’aucune indigence et sera complétement soumis à la volonté. La cité céleste possède cette paix ici-bas par la foi ; et elle vit de cette foi lorsqu’elle rapporte à l’acquisition de la paix véritable tout ce qu’elle fait de bonnes œuvres en ce monde, soit à l’égard de Dieu, soit à l’égard du prochain ; car la vie de la cité est une vie sociale.
\subsection[{Chapitre XVIII}]{Chapitre XVIII}

\begin{argument}\noindent Combien la foi inébranlable du chrétien diffère des incertitudes de la nouvelle Académie.
\end{argument}

\noindent Rien de plus contraire à la Cité de Dieu que cette incertitude dont Varron fait le trait distinctif de la nouvelle Académie. Un tel doute aux yeux d’un chrétien, est une folie. Sur les choses qui sont saisies par l’esprit et la raison, il affirme avec certitude, bien que cette connaissance soit fort limitée, à cause du corps corruptible qui appesantit l’âme : car, comme lit l’Apôtre, « notre science ici-bas est toute partielle. » Il croit aussi au rapport des sens tans les choses qui se manifestent avec évidence, par cette raison que, si l’un se trompe quelquefois en les croyant, on se trompe bien davantage en ne les croyant jamais. Enfin, il ajoute foi aux Écritures saintes, anciennes et nouvelles, que nous appelons canoniques, et lui sont comme la source de la foi dont le juste vit et qui nous fait marcher avec assurance à travers ce lieu de pèlerinage. Cette foi demeurant certaine et inviolable, nous pouvons douter sans crainte de certaines choses qui ne nous sont connues ni par les sens ni par la raison, et sur lesquelles l’Écriture ne s’explique point, ou qui ne nous ont point été confirmées par des témoignages incontestables.
\subsection[{Chapitre XIX}]{Chapitre XIX}

\begin{argument}\noindent De la vie et des mœurs du peuple chrétien.
\end{argument}

\noindent Il importe peu à la Cité céleste que celui qui embrasse la foi qui conduit à Dieu adopte tel ou tel genre de vie, pourvu qu’il ne soit pas contraire à ses commandements. C’est pourquoi, quand les philosophes mêmes se font chrétiens, elle ne les oblige point de quitter leur manière de vivre, à moins qu’elle ne choque la religion, mais seulement à abandonner leurs fausses doctrines. Ainsi elle néglige cette autre différence que Varron a tirée de la manière de vivre des Cyniques, à condition toutefois qu’il ne soit rien fait contre la tempérance et l’honnêteté. Quant à ces trois genres de vie, l’actif, le contemplatif, et celui qui est mêlé des deux, quoique tout croyant sincère puisse choisir comme il lui plaira, sans rien perdre de son droit aux promesses éternelles, il importe toutefois de considérer ce que l’amour de la vérité nous fait embrasser et ce que le devoir de la charité nous fait subir. On ne doit point tellement s’adonner au repos de la contemplation qu’on ne songe aussi à être utile au prochain, ni s’abandonner à l’action, de telle sorte qu’on en oublie la contemplation. Dans le repos, on ne doit pas aimer l’oisiveté, mais s’occuper à la recherche du vrai, afin de profiter soi-même de cette connaissance et de ne la pas envier aux autres ; et, dans l’action, il ne faut pas aimer l’honneur ni la puissance, parce que tout cela n’est que vanité, mais le travail qui l’accompagne, lorsqu’il contribue au salut de ceux qui nous sont soumis. C’est ce qui a fait dire à l’Apôtre que : « Celui qui désire l’épiscopat désire une bonne œuvre. » L’épiscopat est en effet un nom de charge, et non pas de dignité ; comme l’indiqué l’étymologie. Il consiste à veiller sur ses subordonnés et à en avoir soin, de sorte que celui-là n’est pas évêque qui aime à gouverner, sans se soucier d’être utile à ceux qu’il gouverne. Tout le monde peut s’appliquer à la recherche de la vérité, en quoi consiste le repos louable de la vie contemplative ; mais, pour les fonctions de l’Église, quand on serait capable de les remplir, il est toujours honteux de les désirer. Il ne faut qu’aimer la vérité pour embrasser le saint repos de la contemplation ; mais ce doit êtrela charité et la nécessité qui nous engagent dans l’action, en sorte que, si personne ne nous impose ce fardeau, il faut vaquer à la recherche et à la contemplation de la vérité, et si on nous l’impose, il faut s’y soumettre par charité et par nécessité. Et alors même il ne faut pas abandonner tout à fait les douceurs de la contemplation, de peur que, privés de cet appui, nous ne succombions sous le fardeau du gouvernement.
\subsection[{Chapitre XX}]{Chapitre XX}

\begin{argument}\noindent Les membres de la Cité de dieu ne sont heureux ici-bas qu’en espérance.
\end{argument}

\noindent Puis donc que le souverain bien de la Cité de Dieu consiste dans la paix, non cette paix que traversent les mortels entre la naissance et la mort, mais celle où ils demeurent, devenus immortels et à l’abri de tout mal, qui peut nier que cette vie future ne soit très heureuse, et que celle que nous menons ici-bas, quelques biens temporels qui l’accompagnent, ne soit en comparaison très misérable ? Et cependant, quiconque s’y conduit de telle sorte qu’il en rapporte l’usage à celle qu’il aime avec ardeur et qu’il espère avec fermeté, on peut avec raison l’appeler heureux, même dès ce monde, plutôt, il est vrai, parce qu’il espère l’autre vie que parce qu’il possède celle-ci. La possession de ce qu’il y a de meilleur en cette vie, sans l’espérance de l’autre, est au fond une fausse béatitude et une grande misère. En effet, on n’y jouit pas des vrais biens de l’âme, puisque cette sagesse n’est pas véritable, qui, dans les-choses mêmes qu’elle discerne avec prudence, qu’elle accomplit avec force, qu’elle réprime avec tempérance et qu’elle ordonne avec justice, ne se propose pas la fin suprême où Dieu sera tout en tous par une éternité certaine et par une parfaite paix.
\subsection[{Chapitre XXI}]{Chapitre XXI}

\begin{argument}\noindent D’après les définitions admises dans la {\itshape République} de Cicéron, il n’y a jamais eu de république parmi les Romains.
\end{argument}

\noindent Il s’agit maintenant de m’acquitter en peu de mots de la promesse que j’ai faite au second livre de cet ouvrage, et de montrer que, selon les définitions dont Scipion se sert dans la {\itshape République} de Cicéron, il n’y a jamais eu de république parmi les Romains. Il définit en deux mots la république : la chose du peuple. Si cette définition est vraie, il n’y a jamais eu de république romaine ; car jamais le gouvernement de Rome n’a été la chose du peuple. Comment, en effet, Scipion a-t-il défini le peuple ? « C’est, dit-il, une société fondée sur des droits reconnus et sur la communauté des intérêts. » Or, il explique ensuite ce qu’il entend par ces droits, lorsqu’il dit qu’une république ne peut être gouvernée sans justice. Là donc où il n’y a point de justice, il n’y a point de droit. Comme on fait justement ce qu’on a droit de faire, il est impossible qu’on ne soit pas injuste quand on agit sans droit. En effet, il ne faut pas appeler droits les établissements injustes des hommes, puisqu’eux-mêmes ne nomment droit que ce qui vient de la source de la justice, et rejettent comme fausse cette maxime de quelques-uns, que le droit du plus fort consiste dans ce qui lui est utile. Ainsi, ou il n’y a point de vraie justice, il ne peut y avoir de société fondée sur des droits reconnus et sur la communauté des intérêts, et par conséquent il ne peut y avoir de peuple. S’il n’y a point de peuple, il n’y a point aussi de chose du peuple ; il ne reste, au lieu d’un peuple, qu’une multitude telle quelle qui ne mérite pas ce nom. Puis donc que la république est la chose du peuple, et qu’il n’y a point de peuple, s’il n’est associé pour se gouverner par le droit, comme d’ailleurs il n’y a point de droit où il n’y a point de justice, il s’ensuit nécessairement qu’où il n’y a point de justice, il n’y a point de république. Considérons maintenant la définition de la justice : c’est une vertu qui fait rendre à chacun ce qui lui appartient. Or, quelle est cette justice qui ôte l’homme à Dieu pour le soumettre à d’infâmes démons ? Est-ce là rendre à chacun ce qui lui appartient ? Un homme qui ôte un fonds de terre à celui qui l’a acheté, pour le donner à celui qui n’y a point de droit, est injuste ; et un homme qui se soustrait soi-même à Dieu, son souverain Seigneur et Créateur, pour servir les malins esprits, serait juste !\par
Dans cette même {\itshape République}, on soutient fortement le parti de la justice contre l’injustice ; et, comme en parlant d’abord pourl’injustice, on avait dit que sans elle une république ne pouvait ni croître ni s’établir, puisqu’il est injuste que des hommes soient assujettis à d’autres hommes, on répond, au nom de la justice, que cela est juste, parce que la servitude est avantageuse à ceux qui la subissent (quand les autres n’en abusent pas), en ce qu’elle leur ôte la puissance de mal faire. Pour appuyer cette raison, on ajoute que la nature même nous en fournit un bel exemple : « Car pourquoi, dit-on, Dieu commande-t-il à l’homme, l’âme au corps, et la raison aux passions ? » Cet exemple fait voir assez que la servitude est utile à quelques-uns, mais que servir Dieu est utile à tous. Or, quand l’âme est soumise à Dieu, c’est avec justice qu’elle commande au corps et que dans l’âme même la raison commande aux passions. Lors donc que l’homme ne sert pas Dieu, quelle justice peut-il y avoir dans l’homme, puisque le service qu’il lui rend donne seul le droit à l’âme de commander au corps, et à la raison de gouverner les passions ? Et s’il n’y a point de justice dans un homme étranger au culte de Dieu, certainement il n’y en aura point non plus dans une société composée de tels hommes. Partant il n’y aura point aussi de droit dont ils conviennent et qui leur donne le nom de peuple, et par conséquent point de république. Que dirai-je de l’utilité que Scipion fait encore entrer dans la définition de peuple ? Il est certain qu’à y regarder de près, rien n’est utile à des impies, comme le sont tous ceux qui, au lieu de servir Dieu, servent ces démons, qui sont eux-mêmes d’autant plus impies, qu’étant des esprits immondes, ils veulent qu’on leur sacrifie comme à des dieux. Mais, laissant cela à part, ce que nous avons dit touchant le droit suffit, à mon avis, pour faire voir que, selon cette définition, il ne peut y avoir de peuple, ni par conséquent de république où il n’y a pas de justice. Prétendre que les Romains n’ont pas servi dans leur république des esprits immondes, mais des dieux bons et saints, c’est ce qui ne se peut soutenir sans stupidité ou sans impudence, après tout ce que nous avons dit sur ce sujet ; mais, pour ne point me répéter, je dirai seulement ici qu’il est écrit dans la loi du vrai Dieu que celui qui sacrifiera à d’autres dieux qu’à lui seul sera exterminé. Il veut doncen général et d’une manière absolue qu’on ne sacrifie point aux dieux, bons ou mauvais.
\subsection[{Chapitre XXII}]{Chapitre XXII}

\begin{argument}\noindent Le Dieu des chrétiens est le vrai Dieu et le seul à qui l’on doive sacrifier.
\end{argument}

\noindent Mais, dira-t-on, quel est ce Dieu, ou comment prouve-t-on, que lui seul méritait le culte des Romains ? Il faut être bien aveugle pour demander encore quel est ce Dieu : c’est ce Dieu dont les Prophètes ont prédit tout ce que nous voyons s’accomplir sous nos yeux ; c’est celui qui dit à Abraham : « En ta race, toutes les nations seront bénies » : parole qui s’est vérifiée en Jésus-Christ, né de cette race selon la chair, comme le reconnaissent malgré eux ses ennemis mêmes ; c’est lui qui a inspiré par son Saint-Esprit toutes les prédictions que j’ai rapportées touchant l’Église que nous voyons répandue par toute la terre ; c’est lui que Varron, le plus docte des Romains, croit être Jupiter, quoiqu’il ne sache ce qu’il dit. Au moins cela fait-il voir qu’un homme si savant n’a pas jugé que ce Dieu ne fût point, ou qu’il fût méprisable, puisqu’il l’a cru le même que celui qu’il prenait pour le souverain de tous les dieux. Enfin, c’est celui que Porphyre, le plus savant des philosophes, bien qu’ardent ennemi des chrétiens, avoue être un grand Dieu, même selon les oracles de ceux qu’il croyait des dieux.
\subsection[{Chapitre XXIII}]{Chapitre XXIII}

\begin{argument}\noindent Des oracles que Porphyre rapporte touchant Jésus-Christ.
\end{argument}

\noindent Porphyre, dans son ouvrage intitulé : {\itshape La Philosophie des oracles} (je me sers des expressions telles qu’elles ont été traduites du grec en latin), Porphyre, dis-je, dans ce recueil de réponses prétendues divines sur des questions relatives à la philosophie, s’exprime ainsi : « Quelqu’un demandant à Apollon à quel Dieu il devait s’adresser pour retirer sa femme du christianisme, Apollon lui répondit : Il te serait peut-être plus aisé d’écrire sur l’eau, ou de voler dans l’air, que de guérir l’esprit blessé de ta femme. Laisse-la donc dans sa ridicule erreur chanter d’une voix factice et lugubre un Dieu mort, condamné par des juges équitables, et livré publiquement à un supplice sanglant et ignominieux. » Après ces vers d’Apollon que nous traduisons librement en prose latine, Porphyre continue de la sorte : « Cet oracle fait bien voir combien la secte chrétienne est corrompue, puisqu’il est dit que les Juifs savent mieux que les chrétiens honorer Dieu. » Car c’est ainsi que ce philosophe, poussé par sa haine contre Jésus-Christ à préférer les Juifs aux chrétiens, explique ces paroles de l’oracle d’Apollon, que Jésus-Christ a été mis à mort par des juges équitables ; comme s’ils l’avaient fait mourir justement ! Je laisse la responsabilité de cet oracle à l’interprète menteur d’Apollon ou à Porphyre lui-même, qui peut-être l’a inventé ; et nous aurons à voir plus tard comment ce philosophe s’accorde avec lui-même, ou accorde ensemble les oracles. Maintenant il nous dit que les Juifs, en véritables adorateurs de Dieu, ont condamné justement Jésus-Christ à une mort ignominieuse ; mais ce Dieu des Juifs auquel Porphyre rend témoignage, pourquoi ne pas l’écouter quand il nous dit : « Celui qui sacrifiera à d’autres qu’au seul vrai Dieu sera exterminé » ? Voici, au surplus, d’autres aveux de Porphyre plus manifestes encore. Écoutons-le glorifier la grandeur du roi des Juifs : « Apollon, dit-il, interrogé pour savoir ce qui vaut le mieux du Verbe, c’est-à-dire de la raison ou de la loi, a répondu en ces termes » (ici Porphyre cite des vers d’Apollon, parmi lesquels je choisis les suivants :)\par
 {\itshape « Dieu est le principe générateur, le roi suprême, devant qui le ciel, la terre, la mer et les mystérieux abîmes de l’enfer tremblent, et les dieux mêmes sont saisis d’épouvante ; c’est le Père que les saints hébreux honorent très pieusement. »} \par
Voilà un oracle d’Apollon qui, selon Porphyre, reconnaît que le Dieu des Juifs est si grand qu’il épouvante les dieux mêmes. Or, puisque ce Dieu a dit que celui qui sacrifie aux dieux sera exterminé, je m’étonne que Porphyre n’ait pas aussi éprouvé quelque épouvante, et, dans ses sacrifices aux dieux, n’ait pas craint d’être exterminé.\par
Ce philosophe dit aussi du bien de Jésus-Christ, comme s’il avait oublié les paroles outrageantes que je viens de rapporter, ou comme si les dieux n’avaient mal parlé du Sauveur que pendant qu’ils étaient endormis, et, le connaissant mieux à leur réveil, luieussent donné les louanges qu’il mérite. Il s’écrie comme s’il allait révéler une chosemerveilleuse et incroyable : « Quelques-uns seront sans doute surpris de ce que je vais dire : c’est que les dieux ont déclaré que le Christ était un homme très pieux, qu’il a été fait immortel, et qu’il leur a laissé un très bon souvenir. Quant aux chrétiens, ils les déclarent impurs, chargés de souillures, enfoncés dans l’erreur, et les accablent de mille autres blasphèmes. » Porphyre rapporte ces blasphèmes comme autant d’oracles des dieux ; puis il continue ainsi : « Hécate, consultée pour savoir si le Christ est un Dieu, a répondu : Quel est l’état d’une âme immortelle séparée du corps ? vous le savez ; et si elle s’est écartée de la sagesse, vous n’ignorez pas qu’elle est condamnée à errer toujours ; celle dont vous me parlez est l’âme d’un homme excellent en piété ; mais ceux qui l’honorent sont dans l’erreur. » — « Voilà donc, poursuit Porphyre, qui cherche à rattacher ses propres pensées à celles qu’il impute aux dieux, voilà l’oracle qui déclare le Christ un homme éminent en piété, et qui assure que son âme a reçu l’immortalité comme celle des autres justes, mais que c’est une erreur de l’adorer. » — « Et comme quelques-uns, ajoute-t-il, demandaient à Hécate : Pourquoi donc a-t-il été condamné ? La déesse répondit : Le corps est toujours exposé aux tourments, mais l’âme des justes a le ciel pour demeure. Celui dont vous me parlez a été une fatale occasion d’erreur pour toutes les âmes qui n’étaient pas appelées par les destins à recevoir les faveurs des dieux, ni à connaître Jupiter immortel. Aussi les dieux n’aiment point ces âmes fatalement déshéritées ; mais lui, c’est un juste, admis au ciel en la compagnie des justes. Gardez-vous donc de blasphémer contre lui, et prenez pitié de la folie des hommes ; car du Christ aux chrétiens, la pente est rapide. »\par
Qui est assez stupide pour ne pas voir, ouque ces oracles ont été supposés par cet homme artificieux, ennemi mortel des chrétiens, ou qu’ils ont été rendus par les démons avec une intention toute semblable, c’est-à-dire afin d’autoriser, par les louanges qu’ils donnent à Jésus-Christ, la réprobation qu’ils soulèvent contre les chrétiens, détournant ainsi les hommes de la voie du salut, où l’on n’entre que par le christianisme ? Comme ils sont infiniment rusés, peu leur importe qu’on ajoute foi à leurs éloges de Jésus-Christ, pourvu que l’on croie aussi leurs calomnies contre ses disciples, et ils souffrent qu’on loue Jésus-Christ, à condition de n’être pas chrétien, et par conséquent de n’être pas délivré par le Christ de leur domination. Ajoutez qu’ils le louent de telle sorte que quiconque croira en lui sur leur rapport ne sera jamais vraiment chrétien, mais photinien, et ne verra dans le Christ que l’homme et non Dieu ; ce qui l’empêchera d’être sauvé par sa médiation et de se dégager des filets de ces démons imposteurs. Pour nous, nous fermons également l’oreille à la censure d’Apollon et aux louanges d’Hécate. L’un veut que Jésus-Christ ait été justement condamné à mort par ses juges, et l’autre en parle comme d’un homme très pieux, mais toujours un homme. Or, ils n’ont l’un et l’autre qu’un même dessein, celui d’empêcher les hommes de se faire chrétiens, seul moyen pourtant d’être délivré de leur tyrannie. Au surplus, que ce philosophe ou plutôt ceux qui ajoutent foi à ces prétendus oracles accordent, s’ils peuvent, Apollon et Hécate, et placent l’éloge ou la condamnation dans la bouche de tous deux ; mais quand ils le pourraient faire, nous n’en aurions pas moins pour ces démons, soit qu’ils louent le Christ, soit qu’ils le blasphèment, la même répulsion. Et comment les païens, qui voient un dieu et une déesse se contredire sur Jésus-Christ, et Apollon blâmer ce qu’approuve Hécate, peuvent-ils, pour peu qu’ils soient raisonnables, ajouter foi aux calomnies de ces démons contre les chrétiens ?\par
Au reste, quand Porphyre ou Hécate disent que Jésus-Christ a été une fatale occasion d’erreur pour les chrétiens, je leur demanderai s’il l’a été volontairement ou malgré lui. Si c’est volontairement, comment est-il juste ? et si c’est malgré lui, comment est-ilbienheureux ? Mais écoutons Porphyre expliquant la cause de cette prétendue erreur : « Il y a, dit-il, en certain lieu, des esprits terrestres et imperceptibles soumis au pouvoir des mauvais démons. Les sages (les Hébreux), entre lesquels était ce Jésus, selon les oracles d’Apollon que je viens de rapporter, détournaient les personnes religieuses du culte de ces mauvais démons et de ces esprits inférieurs, et les portaient à adorer plutôt les dieux célestes et surtout Dieu le père. C’est aussi, ajoute-t-il, ce que les dieux mêmes commandent, et nous avons montré ci-dessus comment ils avertissent de reconnaître Dieu et veulent qu’on l’adore partout. Mais les ignorants et les impies, qui ne sont pas destinés à recevoir les faveurs des dieux, ni à connaître Jupiter immortel, ont rejeté toute sorte de dieux, pour embrasser le culte des mauvais démons. Il est vrai qu’ils feignent de servir Dieu, mais ils ne font rien de ce qu’il faut pour cela. Dieu, comme le père de toutes choses, n’a besoin de rien ; et nous attirons ses grâces sur nous, lorsque nous l’honorons par la justice, par la chasteté et par les autres vertus, et que notre vie est une continuelle prière par l’imitation de ses perfections et la recherche de sa vérité. Cette recherche, dit-il, nous purifie, et l’imitation nous rapproche de lui. » Ici, j’en conviens, Porphyre parle dignement de Dieu le père et de l’innocence des mœurs, laquelle constitue principalement le culte qu’on lui rend. Aussi bien les livres des prophètes hébreux sont pleins de ces sortes de préceptes, soit qu’ils reprennent le vice, soit qu’ils louent la vertu. Mais Porphyre, quand il parle des chrétiens, ou se trompe, ou les calomnie autant qu’il plaît aux démons qu’il prend pour des dieux : comme s’il était bien malaisé de se souvenir des infamies qui se commettent dans les temples ou sur les théâtres en l’honneur des dieux, et de considérer ce qui se dit dans nos églises ou ce qu’on y offre au vrai Dieu, pour juger de quel côté est l’édification ou la ruine des mœurs. Et quel autre que l’esprit malin lui a dit ou inspiré ce mensonge ridicule et palpable, que les chrétiens révèrent plutôt qu’ils ne les haïssent ces démons que les Hébreux défendent d’adorer ? Mais ce Dieu, que les sages des Hébreux ont adoré, défend aussi de sacrifier aux esprits célestes, aux anges et aux vertus que nous aimons et honorons dans le pèlerinage de cette vie mortelle, comme nos concitoyens déjà bienheureux. Dans la loi qu’il a donnée à son peuple, il a fait entendre comme un coup de tonnerre cette terrible menace : « Celui qui sacrifiera aux dieux sera exterminé » ; et de peur qu’on ne s’imaginât que cette défense ne regarde que les mauvais démons et ces esprits terrestres que Porphyre appelle esprits inférieurs, parce que l’Écriture sainte les appelle aussi les dieux des Gentils, comme dans ce passage du psaume : « Tous les dieux des Gentils sont des démons », de peur qu’on ne crût que la défense de sacrifier aux démons n’emporte pas celle de sacrifier aux esprits célestes, ou au moins à quelques-uns d’entre eux, l’Écriture ajoute ces mots : Si ce n’est au Seigneur seul, {\itshape nisi Domino soli}. Et quant à ceux qui, trompés par le mot {\itshape soli}, se figurent que Dieu est ici confondu avec le soleil, il suffit de jeter les yeux sur le texte grec pour dissiper leur erreur.\par
Ainsi, ce Dieu à qui un si excellent philosophe rend un si excellent témoignage, a donné à son peuple, au peuple hébreu, une loi écrite en langue hébraïque, et cette loi, qui est connue par toute la terre, porte expressément que celui qui sacrifiera aux dieux et à d’autres qu’au Seigneur sera exterminé. Qu’est-il besoin d’aller chercher d’autres passages dans cette loi ou dans les Prophètes pour montrer que le Dieu véritable et souverain ne veut point qu’on sacrifie à d’autres qu’à lui ? Voici un oracle court, mais terrible, sorti de la bouche de ce Dieu que les plus savants hommes du paganisme exaltent si fort : qu’on l’écoute, qu’on le craigne, qu’on y obéisse, de peur qu’on encoure la peine dont il menace : « Celui qui sacrifiera aux dieux et à d’autres qu’au Seigneur sera exterminé. » Ce n’est pas que Dieu ait besoin de rien qui soit à nous, mais c’est qu’il nous est avantageux d’être à lui. Il est écrit dans les saintes Lettres des Hébreux : « J’ai dit au Seigneur : Vous êtes mon Dieu, parce que vous n’avez pas besoin de mes biens. » Or, nous-mêmes, c’est-à-dire sa Cité, nous sommes le plus noble et le plus excellent sacrifice qui lui puisse être offert ; et tel est le mystère que nous célébrons dans nos oblations bien connues des fidèles, ainsi que nous l’avons dit aux livres précédents. Les oracles du ciel ont déclaré hautement, par la bouche des Prophètes hébreux, que les sacrifices d’animaux que les Juifs offraient comme des figures de l’avenir cesseraient, et que les nations, du levant au couchant, n’offriraient qu’un seul sacrifice ; ce que nous voyons maintenant accompli. Nous avons rapporté dans cet ouvrage quelques-uns de ces témoignages, autant que nous l’avons trouvé à propos. Concluons qu’où n’est point cette justice, qui fait qu’on n’obéit qu’au Dieu souverain et qu’on ne sacrifie qu’à lui seul, là certainement aussi n’est point une société fondée sur des droits reconnus et sur des intérêts communs ; et par conséquent il n’y a point là non plus de peuple, si la définition qu’on en a donnée est la véritable. Il n’y a donc point enfin de république, puisque la chose du peuple ne saurait être où le peuple n’est pas.
\subsection[{Chapitre XXIV}]{Chapitre XXIV}

\begin{argument}\noindent Suivant quelle définition l’empire romain, ainsi que les autres États, peuvent s’attribuer justement les noms de peuple et de république.
\end{argument}

\noindent Mais écartons cette définition du peuple, et supposons qu’on en choisisse une autre, par exemple celle-ci : Le peuple est une réunion d’êtres raisonnables qui s’unissent afin de jouir paisiblement ensemble de ce qu’ils aiment. Pour savoir ce qu’est chaque peuple, il faudra examiner ce qu’il aime. Toutefois, quelque chose qu’il aimes du moment qu’il y a une réunion, non de bêtes, mais de créatures raisonnables, unies par la communauté des mêmes intérêts, on peut fort bien la nommer un peuple, lequel sera d’autant meilleur que les intérêts qui le lient seront plus nobles et d’autant plus mauvais qu’ils le seront moins. Suivant cette définition, le peuple romain est un peuple, et son gouvernement est sans doute une république. Or, l’histoire nous apprend ce qu’a aimé ce peuple au temps de son origine et aux époques suivantes, et comment il a été entraîné à de cruelles séditions par la dépravation de ses mœurs, et de là conduit aux guerres civiles et sociales, où il a sapé dans sa base la concorde qui est en quelque sorte le salut du peuple. Je ne voudrais cependant pas dire qu’à ce moment l’empire romain ne fût plus un peuple, ni son gouvernement une république, tant qu’il est resté une réunion de personnes raisonnables liées ensemble par un intérêt commun. Et ce que j’accorde pour ce peuple, je l’accorde également pour les Athéniens, les Égyptiens, les Assyriens, et pour tout autre empire, grand ou petit ; car, en général, la cité des impies, rebelle aux ordres du vrai Dieu qui défend de sacrifier à d’autres qu’à lui, et partant incapable de faire prévaloir l’âme sur le corps et la raison sur les vices, ne connaît point la justice véritable.
\subsection[{Chapitre XXV}]{Chapitre XXV}

\begin{argument}\noindent Il n’y a point de vraies vertus où il n’y a point de vraie religion.
\end{argument}

\noindent Quelque heureux empire que l’âme semble avoir sur le corps, et la raison sur les passions, si l’âme et la raison ne sont elles-mêmes soumises à Dieu et ne lui rendent le culte commandé par lui, cet empire n’existe pas dans sa vérité. Comment une âme qui ignore le vrai Dieu et qui, au lieu de lui être assujettie, se prostitue à des démons infâmes, peut-elle être maîtresse de son corps et de ses mauvaises inclinations ? C’est pourquoi les vertus qu’elle pense avoir, si elle ne les rapporte à Dieu, sont plutôt des vices que des vertus. Car, bien que plusieurs s’imaginent qu’elles sont des vertus véritables, quand elles se rapportent à elles-mêmes et n’ont qu’elles-mêmes pour fin, je dis que même alors elles sont pleines d’enflure et de superbe, et ainsi elles ne sont pas des vertus, mais des vices. En effet, comme ce qui fait vivre le corps n’est pas un corps, mais quelque chose au-dessus du corps, de même ce qui rend l’homme bienheureux ne vient pas de l’homme, mais est au-dessus de l’homme ; et ce que je dis de l’homme est vrai de tous les esprits célestes.
\subsection[{Chapitre XXVI}]{Chapitre XXVI}

\begin{argument}\noindent Le peuple de Dieu, en son pèlerinage ici-bas, fait servir la paix du peuple séparé de Dieu aux intérêts de la piété.
\end{argument}

\noindent Ainsi, de même que l’âme est la vie du corps, Dieu est la vie bienheureuse del’homme, d’où vient cette parole des saintes Lettres des Hébreux : « Heureux le peuple qui a son Seigneur en son Dieu. » Malheureux donc le peuple qui ne reconnaît pas ce Dieu ! Il ne laisse pas pourtant de jouir d’une certaine paix qui n’a rien de blâmable en soi mais il n’en jouira pas à la fin, parce qu’il n’en use pas bien avant la fin. Or, nous chrétiens, c’est notre intérêt qu’il jouisse de la paix pendant cette vie ; car, tant que les deux cités sont mêlées ensemble, nous nous servons aussi de la paix de Babylone, tout en étant affranchis de son joug par la foi et ne faisant qu’y passer comme des voyageurs. C’est pour cela que l’Apôtre avertit l’Église de prier pour les rois et les puissants du siècle, « afin, dit-il, que nous menions une vie tranquille en toute piété et charité ». Lorsque Jérémie prédit à l’ancien peuple d’Israël sa captivité et lui recommande au nom de Dieu d’aller à Babylone sans murmurer, afin de donner au Seigneur cette preuve de sa patience, il l’avertit aussi de prier pour cette ville, « parce que, dit-il, vous trouverez votre paix dans la sienne » ; c’est-à-dire une paix temporelle, celle qui est commune aux bons et aux méchants.
\subsection[{Chapitre XXVII}]{Chapitre XXVII}

\begin{argument}\noindent La paix des serviteurs de Dieu ne saurait être parfaite en cette vie mortelle.
\end{argument}

\noindent Mais il y a une autre paix, qui est propre à la Cité sainte, et celle-là, nous en jouissons avec Dieu par la foi, et nous l’aurons un jour éternellement avec lui par la claire vision. Ici-bas, au contraire, la paix dont nous jouissons, publique ou particulière, est telle qu’elle sert plutôt à soulager notre misère qu’à procurer notre félicité. Notre justice même, quoique vraie en tant que nous la rapportons au vrai bien, est si défectueuse en cette vie qu’elle consiste plutôt dans la rémission des péchés que dans aucune vertu parfaite. Témoin la prière de toute la Cité de Dieu étrangère en ce monde, et qui crie à Dieu par la bouche de tous ses membres : « Pardonnez-nous nos offenses, comme nous pardonnons à ceux qui nous ont offensés. » Et cette prière ne sert de rien à ceux dont la foi sans œuvres est une foi morte, mais seulement àceux dont la foi opère par amour. Les justes mêmes ont besoin de cette prière ; car bien que leur âme soit soumise à Dieu, la raison ne commande jamais parfaitement aux vices en cette vie mortelle et dans ce corps corruptible qui appesantit l’âme ; car elle ne leur commande pas sans combat et sans résistance. C’est pourquoi, avec quelque vigilance que l’on combatte en ce lieu d’infirmité, et quelque victoire qu’on remporte sur ses ennemis, on donne quelque prise sur soi, sinon par les actions, du moins par les paroles ou par les pensées. Tant que l’on ne fait que commander aux vices, on ne jouit pas encore d’une pleine paix, parce que ce qui résiste n’est jamais dompté sans danger, et l’on ne triomphe pas en repos de ceux qui sont domptés, parce qu’il faut toujours veiller à ce qu’ils ne se relèvent pas. Parmi ces tentations dont l’Écriture dit avec tant de concision, que « la vie de l’homme sur la terre est une continuelle tentation », qui présumera n’avoir point besoin de dire à Dieu : Pardonnez-nous nos offenses, si ce n’est l’homme superbe, qui n’a pas la glandeur, mais l’enflure, et à qui celui qui donne sa grâce aux humbles résiste avec justice ? Ici donc la justice consiste, à l’égard de l’homme, à obéir à Dieu à l’égard du corps, à être soumis à l’âme, et à l’égard des vices, à les vaincre ou à leur résister par la raison, et à demander à Dieu sa grâce et le pardon de ses fautes, comme à le remercier des biens qu’on en a reçus. Mais dans cette paix finale, qui doit être le but de toute la justice que nous tâchons d’acquérir ici-bas, comme la nature sera guérie sans retour de toutes les mauvaises inclinations, et que nous ne sentirons aucune résistance ni en nous-mêmes, ni de la part des autres, il ne sera pas nécessaire que la raison commande aux passions qui ne seront plus, mais Dieu commandera à l’homme, et l’âme au corps, avec une facilité et une douceur qui répondra à un état si glorieux et si fortuné. Cet état sera éternel, et nous serons assurés de son éternité, et c’est en cela que consistera notre souverain bien.
\subsection[{Chapitre XXVIII}]{Chapitre XXVIII}

\begin{argument}\noindent De la fin des méchants.
\end{argument}

\noindent Mais, au contraire, tous ceux qui n’appartiennent pas à cette Cité de Dieu, leur misère sera éternelle ; c’est pourquoi l’Écriture l’appelle aussi la seconde mort, parce que ni l’âme, ni le corps ne vivront : l’âme, parce qu’elle sera séparée de Dieu, qui est la vie, et le corps, parce qu’il souffrira d’éternelles douleurs. Aussi cette seconde mort sera la plus cruelle, parce qu’elle ne pourra finir par la mort. Or, la guerre étant contraire à la paix, comme la misère l’est à la béatitude et la mort à la vie, on peut demander si à ta paix dont on jouira dans le souverain bien répond une guerre dans le souverain mal. Que celui qui fait cette demande prenne garde à ce qu’il y a de mauvais dans la guerre, et il trouvera que cela ne consiste que dans l’opposition et la contrariété des choses entre elles. Quelle guerre donc plus grande et plus cruelle peut-on s’imaginer que celle où la volonté est tellement contraire à la passion et la passion à la volonté, que leur inimitié ne cesse jamais par, la victoire de l’une ou de l’autre, et où la douleur combat tellement contre le corps qu’aucun des deux adversaires ne triomphe jamais ? Quand il arrive en ce monde un pareil combat, ou bien la douleur a le dessus, et la mort en ôte le sentiment, ou la nature est victorieuse, et la santé chasse ta douleur. Mais dans la vie à venir, la douleur demeurera pour tourmenter, et la nature subsistera pour sentir la douleur ; car ni l’une ni l’autre ne sera détruite, afin que le supplice dure toujours. Or, comme c’est par le Jugement dernier que les bons et les méchants aboutiront, les uns au souverain bien et les autres au souverain mal, nous allons traiter ce sujet dans le livre suivant, s’il plaît à Dieu.
\section[{Livre vingtième. Le jugement dernier}]{Livre vingtième. \\
Le jugement dernier}\renewcommand{\leftmark}{Livre vingtième. \\
Le jugement dernier}

\subsection[{Chapitre premier}]{Chapitre premier}

\begin{argument}\noindent On ne traitera proprement dans ce livre que du jugement dernier, bien que dieu juge en tout temps.
\end{argument}

\noindent Ayant dessein présentement, avec la grâce de Dieu, de parler du jour du dernier jugement et d’en établir la certitude contre les impies et les incrédules, nous devons d’abord poser comme fondement de notre édifice les témoignages de l’Écriture. Ceux qui n’y veulent point croire ne leur opposent que des raisonnements humains, pleins d’erreurs et de mensonges, tantôt soutenant que l’Écriture doit s’entendre dans un autre sens, et tantôt qu’elle n’a point l’autorité de la parole divine. Pour ceux qui l’entendent en son vrai sens et qui croient qu’elle renferme la parole de Dieu, je ne doute point qu’ils n’y donnent leur assentiment, soit qu’ils le déclarent au grand jour, soit qu’ils rougissent ou qu’ils craignent, sous de vains scrupules, d’avouer leur foi, soit même que, par une opiniâtreté qui tient de la folie, ils s’obstinent à nier la vérité de choses qu’ils savent être vraies, la fausseté de choses qu’ils savent être fausses. Ainsi, ce que l’Église tout entière du vrai Dieu confesse et professe, à savoir que Jésus-Christ doit venir du ciel pour juger les vivants et les morts, voilà ce que nous appelons le dernier jour du jugement de Dieu, c’est-à-dire le dernier temps. Car combien de jours durera le jugement suprême ? cela est incertain ; mais personne n’ignore, pour peu qu’il soit versé dans l’Écriture sainte, que sa coutume est d’employer le mot {\itshape jour} pour celui de {\itshape temps}. Quand donc nous parlons du jour du jugement, nous ajoutons {\itshape dernier} ou {\itshape suprême}, parce que Dieu juge sans cesse et qu’il a jugé dès le commencement du genre humain, quand il a chassé du paradis et séparé de l’arbre de la vie les premiers hommes coupables. Bien plus, on peut dire qu’il a jugé, quand il a refusé son pardon aux anges prévaricateurs, dont leprince, vaincu par l’envie, trompa les hommes, après s’être trompé lui-même. Ce n’est pas non plus sans un juste et profond jugement de Dieu que les démons et les hommes mènent une vie si misérable et sujette à tant d’erreurs et de peines, les uns dans l’air, et les autres sur la terre. Mais quand personne n’aurait péché, ce serait encore par un jugement équitable de Dieu que toutes les créatures raisonnables demeureraient éternellement unies à leur Seigneur. Et il ne se contente pas de porter sur tous les démons et sur tous les hommes un jugement général, en ordonnant qu’ils soient misérables à cause du péché du premier ange et du premier homme ; il juge encore en particulier les œuvres que chacun d’eux accomplit en vertu de son libre arbitre. En effet, les démons le prient de ne point les tourmenter, et c’est avec justice qu’il les épargne ou les punit, selon qu’ils l’ont mérité. Les hommes aussi sont punis de leurs fautes, le plus souvent d’une manière manifeste, et toujours du moins en secret, soit dans cette vie, soit après la mort, bien qu’aucun ne puisse faire le bien, s’il n’est aidé du ciel, ni faire le mal, si Dieu ne le permet par un jugement très juste. Car, ainsi que le dit l’Apôtre : « Il n’y a point d’injustice en Dieu » ; et ailleurs : « Les jugements de Dieu sont impénétrables, et ses voies incompréhensibles. » Mais nous ne parlerons dans ce livre ni des jugements que Dieu a rendus dès le principe, ni de ceux qu’il rend dans le présent, mais seulement du dernier jugement, alors que Jésus-Christ viendra du ciel juger les vivants et les morts. C’est bien là le jour suprême du jugement ; car alors il n’y aura plus lieu à de vaines plaintes sur le bonheur du méchant ou sur le malheur du juste. Alors, en effet, la félicité véritable et éternelle des seuls justes, et le malheur irrévocable et mérité des seuls méchants seront également manifestes.
\subsection[{Chapitre II}]{Chapitre II}

\begin{argument}\noindent Du spectacle des choses humaines, où l’on ne peut nier que les jugements de Dieu ne se fassent sentir, bien qu’ils se dérobent souvent à nos regards.
\end{argument}

\noindent Nous apprenons ici-bas à souffrir patiemment les maux, parce que les bons même les souffrent, et à ne pas attacher un grand prix aux biens, parce que les méchants même y ont part. Ainsi nous trouvons un enseignement salutaire jusque dans les choses où les raisons de la conduite de Dieu nous sont cachées. Nous ignorons en effet par quel jugement de Dieu cet homme de bien est pauvre, et ce méchant opulent ; pourquoi celui-ci vit dans la joie, lorsqu’il devrait être affligé en punition de ses crimes, tandis que celui-là qui devrait vivre dans la joie, à cause de sa conduite exemplaire, est toujours dans la peine. Nous ne savons pas pourquoi l’innocent n’obtient pas justice, pourquoi il est condamné, au contraire, et opprimé par un juge inique ou confondu par de faux témoignages, tandis que le coupable reste non seulement impuni, mais encore insulte à l’innocent par son triomphe ; pourquoi l’homme religieux est consumé par la langueur, tandis que l’impie est plein de santé. On voit des hommes jeunes et vigoureux vivre de rapines, et d’autres, incapables de nuire, même par un mot, être accablés de maladies et de douleurs. Ceux dont la vie pourrait être utile aux hommes sont emportés par une mort prématurée, et d’autres, qui ne méritaient pas de voir le jour, vivent plus longtemps que personne. Des infâmes, coupables de tous les crimes, parviennent au faîte des grandeurs, et l’homme sans reproche vit caché dans la plus humble obscurité !\par
Encore si ces contradictions étaient ordinaires dans la vie, où, comme dit le Psalmiste : « L’homme n’est que vanité et ses jours passent comme l’ombre » ; si les méchants possédaient seuls les biens temporels et terrestres, tandis que les bons souffriraient seuls tous les maux, on pourrait attribuer cette disposition à un juste jugement de Dieu, et même à un jugement bienveillant : on pourrait croire qu’il veut que les hommes qui n’obtiendront pas les biens éternels soient trompés ou consolés par les temporels, qui lesrendent heureux, et que ceux auxquels ne sont point réservées les peines éternelles, endurent quelques afflictions passagères en punition de fautes légères ou pour s’exercer à la vertu. Mais la plupart du temps, les méchants ont aussi leurs maux, et les bons leurs joies ; ce qui rend les jugements de Dieu plus impénétrables et ses voies plus incompréhensibles. Et cependant, bien que nous ignorions par quel jugement Dieu fait ou permet ces choses, lui qui est la vertu, la sagesse et la justice suprêmes, lui qui n’a ni faiblesse, ni témérité, ni injustice, il nous est avantageux en définitive d’apprendre à ne pas estimer beaucoup des biens et des maux communs aux bons et aux méchants, pour ne chercher que des biens qui n’appartiennent qu’aux bons et pour fuir des maux qui ne sont propres qu’aux méchants. Lorsque nous serons arrivés à ce jugement suprême de Dieu, dont le temps s’appelle proprement le jour du jugement, et quelquefois le jour du Seigneur, alors nous reconnaîtrons la justice des jugements de Dieu, non seulement de ceux qu’il rend maintenant, mais aussi des jugements qu’il a rendus dès le principe, et de ceux qu’il rendra jusqu’à ce moment. Alors on verra clairement la justice de Dieu, que la faiblesse de notre raison nous empêche de voir dans un grand nombre et presque dans le nombre entier de ses jugements, quoique d’ailleurs les âmes pieuses aient toute confiance en sa justice mystérieuse.
\subsection[{Chapitre III}]{Chapitre III}

\begin{argument}\noindent Du sentiment de Salomon, dans le livre de l’Ecclésiaste, sur les choses qui sont communes aux bons et aux méchants.
\end{argument}

\noindent Salomon, le plus sage roi d’Israël, qui régna à Jérusalem, commence ainsi l’Ecclésiaste,que les Juifs, comme nous, reconnaissent pour canonique : « Vanité des hommes de vanité, a dit l’Ecclésiaste, vanité des hommes de vanité, et tout est vanité ! Que revient-il à l’homme de tout ce travail qu’il accomplit sous le soleil ? » Puis, rattachant à cettepensée le tableau des misères humaines, il rappelle les erreurs et les tribulations de cettevie, et démontre qu’il n’y a rien de stable ni de solide ici-bas. Au milieu de cette vanité des choses de la terre, il déplore surtout que, la sagesse ayant autant d’avantage sur la folie que la lumière sur les ténèbres, et le sage étant aussi éclairé que le fou est aveugle, tous néanmoins aient un même sort dans ce monde i, par où il veut dire sans doute que les maux sont communs aux bons et aux méchants. Il ajoute que les bons souffrent comme s’ils étaient méchants, et que les méchants jouissent des biens comme s’ils étaient bons. Et il parle ainsi : « Il y a encore une vanité sur la terre : on y voit des justes à qui le mal arrive comme à des impies, et des impies qui sont traités comme des justes. J’appelle aussi cela une vanité. » Cet homme si sage consacre presque tout son livre à relever ces sortes de vanités, sans doute pour nous porter à désirer cette vie où il n’y a point de vanité sous le soleil, mais où brille la vérité sous celui qui a fait le soleil. Comment donc l’homme se laisserait-il séduire par ces vanités, sans un juste jugement de Dieu ? Et toutefois, tandis qu’il y est sujet, ce n’est pas une chose vaine que de savoir s’il résiste ou s’il obéit à la vérité, s’il est vraiment religieux ou s’il ne l’est pas ; cela importe beaucoup au contraire, non pour acquérir les biens de cette vie ou pour en éviter les maux, mais en vue du jugement dernier, où les biens seront donnés aux bons et les maux aux méchants pour l’éternité. Enfin le sage Salomon termine ainsi ce livre : « Craignez Dieu, et observez ses commandements, parce que là est tout l’homme. Car Dieu jugera toute œuvre, celle même du plus méprisable, bonne ou mauvaise. » Que dire de plus court, de plus vrai, de plus salutaire ? « Craignez Dieu, dit-il, et observez ses commandements ; car là est tout l’homme. » En effet, tout homme n’est que le gardien fidèle des commandements de Dieu ; celui qui n’est point cela n’est rien ; car il n’est point formé à l’image de la vérité, tant qu’il demeure semblable à la vanité. Salomon ajoute : « Car Dieu jugera toute œuvre, c’est-à-dire tout ce qui se fait en cette vie, celle même du plus méprisable », entendez : de celui qui paraît le plus méprisable et auquel les hommes ne font aucune attention ; mais Dieu voit chaque action de l’homme, il n’en méprise aucune, et quand il juge, rien n’est oublié.
\subsection[{Chapitre IV}]{Chapitre IV}

\begin{argument}\noindent Il convient, pour traiter du jugement dernier, de produire d’abord les passages du Nouveau Testament, puis ceux de l’Ancien.
\end{argument}

\noindent Les preuves du dernier jugement de Dieu que nous voulons tirer de l’Écriture sainte, nous les puiserons d’abord dans le Nouveau Testament, ensuite dans l’Ancien. Bien que l’Ancien soit le premier dans l’ordre des temps, le Nouveau néanmoins a plus d’autorité, parce que le premier n’a servi qu’à annoncer l’autre. Nous commencerons donc par les témoignages tirés du Nouveau Testament, et pour leur donner plus de poids, nous les confirmerons par ceux de l’Ancien. L’Ancien comprend la loi et les Prophètes ; le Nouveau, l’Évangile et les Épîtres des Apôtres. Or, l’Apôtre dit : « La loi n’a servi qu’à faire connaître le péché, au lieu que maintenant la justice de Dieu nous est révélée sans la loi, quoique attestée par la loi et les Prophètes. La justice de Dieu est manifestée par la foi en Jésus-Christ à tous ceux qui croient en lui. » Cette justice de Dieu appartient au Nouveau Testament et est confirmée par l’Ancien, c’est-à-dire par la loi et les Prophètes. Je dois donc exposer d’abord le point de la Cause pour produire ensuite les témoins. C’est Jésus-Christ lui-même qui nous apprend à observer cet ordre, lorsqu’il dit : « Un docteur bien instruit dans le royaume de Dieu est semblable à un père de famille qui tire de son trésor de nouvelles et de vieilles choses. » Il ne dit pas de vieilles et de nouvelles choses, ce qu’il n’aurait certainement pas manqué de faire, s’il n’avait eu plus d’égard au prix des choses qu’au temps.
\subsection[{Chapitre V}]{Chapitre V}

\begin{argument}\noindent Paroles du divin Sauveur qui annoncent qu’il y aura un jugement de Dieu à la fin des temps.
\end{argument}

\noindent Le Sauveur lui-même, reprochant leur incrédulité à quelques villes où il avait fait de grands miracles, et leur en préférant d’autres qu’il n’avait point visitées : « Je vous déclare, disait-il, qu’au jour du jugement, Tyr et Sidon seront traitées moins rigoureusement que vous. » Et quelque temps après, s’adressant à une autre ville : « Je t’assure, dit-il, qu’au jour du jugement, Sodome sera traitée moins rigoureusement que toi. » Il montre clairement par là que le jour du jugement doit arriver. Il dit encore ailleurs : « Les Ninivites s’élèveront, au jour du jugement, contre ce peuple et le condamneront, parce qu’ils ont fait pénitence à la prédication de Jonas, et qu’ici il y a plus que Jonas. La reine du Midi s’élèvera, au jour du jugement, contre ce peuple et le condamnera, parce qu’elle est venue des extrémités de la terre pour entendre la sagesse de Salomon, et qu’il y a ici plus que Salomon. » Ce passage nous apprend deux vérités : la première, que le jour du jugement viendra ; la seconde, que les morts ressusciteront en ce jour. Car en parlant des Ninivites et de la reine du Midi, Jésus parlait certainement d’hommes qui n’étaient plus, et il dit pourtant qu’ils revivront au jour du jugement. Et lorsqu’il dit qu’ils condamneront, ce n’est point qu’ils doivent juger eux-mêmes, mais c’est qu’en comparaison d’eux, les autres mériteront d’être condamnés.\par
Ailleurs, à propos du mélange des bons et des méchants en ce monde et de leur séparation au jour du jugement, il se sert de la parabole d’un champ semé de bon grain, où l’on répand de l’ivraie, et l’expliquant à ses disciples : « Celui qui sème le bon grain, dit-il, est le Fils de l’homme ; le champ, c’est le monde ; le bon grain, ce sont les enfants du royaume, et l’ivraie les enfants du diable ; l’ennemi qui l’a semée, c’est le diable ; la moisson, c’est la fin du monde ; les moissonneurs, ce sont les auges. Comme on amasse et comme on brûle l’ivraie, ainsi il sera fait à la fin du monde. Le Fils de l’homme enverra ses anges, et ils enlèveront de son royaume tous les scandales et tous ceux qui commettent l’iniquité, et ils les jetteront dans la fournaise ardente. Là il y aura des pleurs et des grincements de dents. Alors les justes brilleront comme le soleil dans le royaume de leur père. Que celui qui a des oreilles pour entendre, entende. » Il est vrai qu’il ne nomme pas ici le jour du jugement ; mais il l’exprime bien plus clairement par les choses mêmes, et prédit qu’il arrivera à la fin du monde.\par
Il parle de même à ses disciples : « Je vous dis, en vérité, que vous qui m’avez suivi,lorsqu’au temps de la régénération le Fils de l’homme sera assis sur le trône de sa gloire, vous serez assis, vous également, sur douze trônes, et vous jugerez les douze tribus d’Israël. » Ceci nous apprend que Jésus jugera avec ses disciples ; d’où vient qu’ailleurs il dit aux Juifs : « Si c’est au nom de Belzébuth que je chasse les démons, au nom de qui vos enfants les chassent-ils ? C’est pourquoi ils seront eux-mêmes vos juges. » Il ne faut point croire, parce que Jésus a parlé de douze trônes, qu’il ne jugera qu’avec douze disciples. Le nombre douze doit s’entendre comme exprimant la multitude de ceux qui jugeront avec lui, à cause du nombre sept qui marque d’ordinaire une grande multitude, et dont les deux parties, trois et quatre, multipliées l’une par l’autre, donnent douze. En effet, quatre fois trois et trois fois quatre font douze ; sans parler des autres raisons qui expliquent le choix de ce nombre. Autrement, comme l’apôtre Mathias a été mis à la place du traître Judas, il s’ensuivrait que l’apôtre saint Paul, qui a plus travaillé qu’eux tous, n’aurait point de trône pour juger. Or, il témoigne assez lui-même qu’il sera du nombre des juges, quand il dit : « Ne savez-vous pas que nous jugerons les anges ? » Il faut entendre dans le même sens le nombre douze appliqué à ceux qui seront jugés. Car bien qu’il ne soit question que des douze tribus d’Israël, il ne s’ensuit pas que Dieu ne jugera pas la tribu de Lévi, qui est la treizième, ni qu’il jugera le peuple d’Israël seul, et non les autres nations. Quant à la régénération dont il s’agit, nul doute qu’elle ne doive s’entendre de la résurrection des morts. Notre chair, en effet, sera régénérée par la foi.\par
Je laisse de côté beaucoup d’autres passages qui semblent faire allusion au dernier jugement, mais qui, considérés de près, se trouvent ambigus ou relatifs à un autre sujet, par exemple à cet avènement du Sauveur qui se fait tous les jours dans son Église (c’est-à-dire dans ses membres, où il se manifeste partiellement et peu à peu, parce que l’Église entière est son corps), ou bien à la destruction de la Jérusalem terrestre, dont il est parlé comme s’il s’agissait de la fin du monde et du jour de ce grand et dernier jugement. Ainsi on ne saurait entendre clairement ces passages, à moins de comparer ensemble ce qu’en disent les trois évangélistes, saint Matthieu, saint Marc et saint Luc. Tous trois, en effet, s’éclaircissent l’un l’autre, si bien que l’on voit mieux ce qui se rapporte à un même objet. C’est aussi ce que je me suis proposé dans une lettre que j’ai écrite à Hésychius d’heureuse mémoire, évêque de Salone, lettre que j’ai intitulée : {\itshape De la fin du siècle}.\par
J’arrive maintenant à ce passage de l’Évangile selon saint Matthieu, où il est parlé de la séparation des bons et des méchants par un jugement dernier et manifeste de Jésus-Christ : « Quand le Fils de l’homme, dit-il, viendra dans sa majesté, accompagné de tous ses anges, il s’assoira sur son trône, et tous les peuples de la terre seront assemblés en sa présence, et il les séparera les uns des autres, commue un berger sépare les brebis des boucs, et il mettra les brebis à sa droite et les boucs à sa gauche. Alors le roi dira à ceux qui seront à sa droite : Venez, vous que mon père a bénis, et prenez possession du royaume qui vous a été préparé dès le commencement du monde. Car j’ai eu faim, et vous m’avez donné à manger ; j’ai eu soif, et vous m’avez donné à boire ; j’ai eu besoin d’abri, et vous m’avez donné l’hospitalité ; j’étais nu, et vous m’avez vêtu ; j’étais malade, et vous m’avez soulagé ; j’étais prisonnier, et vous m’êtes venu voir. Alors les justes répondront et lui diront : Seigneur, quand vous avons-nous vu avoir faim et vous avons-nous donné à manger, ou avoir soif et vous avons-nous donné à boire ? quand vous avons-nous vu sans abri et vous avons-nous donné l’hospitalité, ou sans vêtement et vous avons-nous vêtu ? quand vous avons-nous vu malade et en prison, et sommes-nous venu vers vous ? Et le roi leur répondra : Je vous le dis, en vérité, toutes les fois que vous avez rendu un tel secours aux moindres de mes frères, c’est à moi que vous l’avez rendu. Il dira ensuite à ceux qui seront à sa gauche : Retirez-vous de moi, maudits, et allez au feu éternel, qui a été préparé pour le diable et pour ses anges. » Il leur reproche ensuite de n’avoir point fait pour lui les mêmes choses dont il a loué ceux qui étaient à sa droite ; et comme ils lui demandent : Quand donc vous avons-nous vu en avoir besoin ? il leur répond de même quetous les secours qu’ils ont refusés aux moindres de ses frères, c’est à lui qu’ils les ont refusés. Puis il conclut ainsi : « Et ceux-là iront au supplice éternel, et les justes à la vie éternelle. » Saint Jean l’évangéliste dit clairement que Jésus a fixé l’époque du jugement à l’heure où les morts ressusciteront. Après avoir dit que le Père ne juge personne, mais qu’il a donné au Fils tout pouvoir de juger, afin que tous honorent le Fils comme ils honorent le Père ; parce que celui qui n’honore pas le Fils n’honore pas le Père qui l’a envoyé, il ajoute aussitôt : « En vérité, en vérité, je vous dis que celui qui entend ma parole, et qui croit en celui qui m’a envoyé, possède la vie éternelle et ne viendra point en jugement, mais qu’il passera de la mort à la vie. » Il nous assure par ces paroles que les fidèles ne viendront point en jugement. Comment donc seront-ils séparés des méchants par le jugement et mis à sa droite, à moins qu’on ne prenne ici le jugement pour la condamnation ? Il est certain, en effet, que ceux qui entendent sa parole, et qui croient en celui qui l’a envoyé, ne seront pas condamnés.
\subsection[{Chapitre VI}]{Chapitre VI}

\begin{argument}\noindent De la première résurrection et de la seconde.
\end{argument}

\noindent Il poursuit en ces termes : « En vérité, en vérité, je vous dis que le temps vient, et qu’il est déjà venu, que les morts entendront la voix du Fils de Dieu, et que ceux qui l’entendront vivront ; car, comme le Père a la vie en lui-même, il a aussi donné au Fils d’avoir la vie en lui-même. » Il ne parle pas encore de la seconde résurrection, c’est-à-dire de celle des corps, qui doit arriver à la fin du monde, mais de la première, qui se fait maintenant. C’est pour distinguer celle-ci de l’autre qu’il dit : « Le temps vient, et il est déjà venu. » Or, cette résurrection ne regarde pas les corps, mais les âmes. Les âmes ont aussi leur mort, qui consiste dans l’impiété et dans le crime ; et c’est de celle-là que sont morts ceux dont le Seigneur a dit : « Laissez les morts « ensevelir leurs morts », c’est-à-dire laissez ceux qui sont morts de la mort de l’âme ensevelir ceux qui sont morts de la mort du corps. Il dit donc de ces morts que l’impiété et le crime ont fait mourir dans l’âme : « Le temps vient, et il est déjà venu, que les morts entendront la voix du Fils de Dieu, et ceux qui l’entendront vivront. » Ceux, dit-il, qui l’entendront, c’est-à-dire qui lui obéiront, qui croiront en lui et qui persévéreront jusqu’à la fin. Il ne fait ici aucune différence entre les bons et les méchants, parce qu’il est avantageux à tous d’entendre sa voix et de vivre, en passant de la mort de l’impiété à la vie de la grâce. C’est de cette mort que saint Paul dit : « Donc tous sont morts, et un seul est mort pour tous, afin que ceux qui vivent ne vivent plus pour eux-mêmes, mais pour celui qui est mort et ressuscité à cause d’eux. » Ainsi, tous sans exception sont morts par le péché, soit par le péché originel, soit par les péchés actuels qu’ils y ont ajoutés, par ignorance ou par malice, et un seul vivant, c’est-à-dire exempt de tout péché, est mort pour tous ces morts, afin que ceux qui vivent parce que leurs péchés leur ont été remis, ne vivent plus pour eux-mêmes, muais pour celui qui est mort pour tous à cause de nos péchés et qui est ressuscité pour notre justification, afin que, croyant en celui qui justifie l’impie et étant justifiés de notre impiété comme des morts qui ressuscitent, nous puissions appartenir à la première résurrection qui se fait maintenant. À celle-là n’appartiennent que ceux qui seront éternellement heureux, au lieu que l’Apôtre nous apprend que les bons et les méchants appartiendront à la seconde, dont il va parler tout à l’heure. Celle-ci est de miséricorde, et celle-là de justice ; ce qui fait dire au Psalmiste : « Seigneur, je chanterai votre miséricorde et votre jugement. »\par
C’est de ce jugement que saint Jean parle ensuite, quand il dit : « Et il lui a donné le pouvoir de juger, parce qu’il est le Fils de l’homme. » Il montre par là qu’il viendra juger, revêtu de la même chair dans laquelle il était venu pour être jugé. Et il dit pour cette raison : « Parce qu’il est le Fils de l’homme. » Puis, parlant de ce dont nous traitons : « Ne vous étonnez pas de cela, dit-il, car le temps viendra que tous ceux qui sont dans les sépulcres entendront la voix du Fils de l’homme ; et ceux qui auront bien vécu sortiront pour ressusciter à la vie, comme les autres pour ressusciter au jugement. » Voilà ce jugement dont il a parlé auparavant, pour désigner la condamnation, en ces termes : « Celui qui entend ma parole et qui croit en celui qui m’a envoyé, possède la vie éternelle, et ne viendra point en jugement, mais il est déjà passé de la mort à la vie. » Ce qui signifie qu’appartenant à la première résurrection, par laquelle on passe maintenant de la mort à la vie, il ne tombera point dans la damnation qu’il identifie avec le jugement, quand il dit : « Comme les autres pour ressusciter au jugement », c’est-à-dire pour être condamnés. Que celui donc qui ne veut pas être condamné à la seconde résurrection ressuscite à la première ; car : « Le temps vient, et il est déjà venu, que les morts entendront la voix du Fils de Dieu ; et ceux qui l’entendront vivront. » En d’autres termes, ils ne tomberont point dans la damnation que l’Écriture appelle la seconde mort et où seront précipités, après la seconde résurrection, qui est celle des corps, ceux qui n’auront pas ressuscité à la première, qui est celle des âmes. Il poursuit ainsi : « Le temps viendra » ; (et il n’ajoute pas : « et il est déjà venu », parce que celui-là ne viendra qu’à la fin du monde, au grand et dernier jugement de Dieu). — « Le temps, dit-il, viendra que tous ceux qui sont dans les sépulcres entendront sa voix et sortiront. » Il ne dit pas, comme lorsqu’il parle de la première résurrection, que ceux qui « l’entendront vivront ». En effet, tous ceux qui l’entendront ne vivront pas, au moins de la vie qui seule mérite ce nom, parce qu’elle est bienheureuse. S’ils n’avaient quelque sorte de vie, ils ne pourraient pas l’entendre, ni sortir de leur tombeau, lorsque leur corps ressuscitera. Or, il nous apprend ensuite pourquoi tous ne vivront pas : « Ceux, dit-il, qui ont bien vécu sortiront pour ressusciter à la vie », voilà ceux qui vivront ; « et les autres pour ressusciter au jugement », voilà ceux qui ne vivront pas, parce qu’ils mourront de la seconde mort. S’ils ont mal vécu, c’est qu’ils ne sont pas ressuscités à la première résurrection qui se fait maintenant, c’est-à-dire à celle des âmes, ou parce qu’ils n’y ont pas persévéré jusqu’à la fin. De même qu’il y a deux générations, dont j’ai déjà parlé ci-dessus, l’une selon la foi, qui se fait maintenant par le baptême, et l’autre selon la chair, qui se fera au dernier jugement, quand la chair deviendra immortelle et incorruptible, de même il y a deux résurrections. La première, qui est celledes âmes, se fait présentement ; elle empêche de tomber dans la seconde mort. L’autre ne se fera qu’à la fin du monde ; elle ne regarde pas les âmes, mais les corps, qu’elle enverra, par suite du jugement dernier, les uns dans la seconde mort, et les autres dans cette vie où il n’y a point de mort.
\subsection[{Chapitre VII}]{Chapitre VII}

\begin{argument}\noindent Ce qu’il faut entendre raisonnablement par les deux résurrections et par le règne de mille ans dont saint Jean parle dans son Apocalypse.
\end{argument}

\noindent Le même évangéliste parle de ces deux résurrections dans son Apocalypse, mais de telle sorte que quelques-uns des nôtres, n’ayant pas compris la première, ont donné dans des visions ridicules. Voici ce que dit l’apôtre saint Jean : « Je vis descendre du ciel un ange qui avait la clef de l’abîme, et une chaîne en sa main : et il prit le dragon, cet ancien serpent qu’on appelle le diable et Satan, et le lia pour mille ans. Puis l’ayant précipité dans l’abîme, il ferma l’abîme et le scella sur lui, afin qu’il ne séduisît plus les nations, jusqu’à ce que les mille ans fussent accomplis ; après quoi il doit être lié pour un peu de temps. Je vis aussi des trônes et des personnes assises dessus, à qui la puissance de juger fut donnée ; avec elles, les âmes de ceux qui ont été égorgés pour les témoignages qu’ils ont rendus à Jésus et pour la parole de Dieu, et tous ceux qui n’ont point adoré la bête ni son image, ni reçu son caractère sur le front ou dans leur main ; et ils ont régné pendant mille ans avec Jésus. Les autres n’ont point vécu jusqu’à ce que mille ans soient accomplis. Voilà la première résurrection. Heureux et saint est celui qui y a part ! La seconde mort n’aura point de pouvoir sur eux, mais ils seront prêtres de Dieu et de Jésus-Christ, et ils régneront mille ans avec lui. » Ceux à qui ces paroles ont donné lieu de croire que la première résurrection sera corporelle, ont surtout adopté cette opinion à cause du nombre de mille ans, dans la pensée que tout ce temps doit être comme le sabbat des saints, où ils se reposeront après les travaux de six mille ans qui seront écoulés depuis que l’homme a été créé et précipité de la félicité du paradis dans les misères de la vie mortelle, afin que, suivantcette parole : « Devant Dieu un jour est comme mille ans et mille ans comme un jour », six mille ans s’étant écoulés comme six jours, le septième, c’est-à-dire les derniers mille ans, tienne lieu de sabbat aux saints qui ressusciteront pour le solenniser. Tout cela serait jusqu’à un certain point admissible, si l’on croyait que durant ce sabbat les saints jouiront de quelques délices spirituelles, à cause de la présence du Sauveur, et j’ai moi-même autrefois été de ce sentiment. Mais comme ceux qui l’adoptent disent que les saints seront dans des festins continuels, il n’y a que des âmes charnelles qui puissent être de leur avis.\par
Aussi les spirituels leur ont-ils donné le nom de {\itshape chiliastes}, d’un mot grec qui peut se traduire littéralement par {\itshape millénaires}. Il serait trop long de les réfuter en détail ; j’aime mieux montrer comme on doit entendre ces paroles de l’Apocalypse.\par
Notre-Seigneur Jésus-Christ a dit lui-même : « Personne ne peut entrer dans la maison du fort et lui enlever ses biens qu’il ne l’ait lié auparavant. » Par le {\itshape fort}, il entend le diable, parce qu’il s’est assujetti le genre humain, et par ses {\itshape biens}, les fidèles qu’il tenait engagés dans l’impiété et dans le crime. C’était donc pour lier ce {\itshape fort} que saint Jean, selon l’Apocalypse, vit un ange descendre du ciel, qui tenait la clef de l’abîme et la chaîne. Et il prit, dit-il, le dragon, cet ancien serpent, que l’on nomme le diable et Satan, et il le lia pour mille ans ; c’est-à-dire qu’il l’empêcha de séduire et de s’assujettir ceux qui devaient être délivrés. Pour les mille ans, on peut les entendre de deux manières : ou bien parce que ces choses se passent dans les derniers mille ans, c’est-à-dire au sixième millénaire, dont les dernières années s’écoulent présentement pour être suivies du sabbat qui n’a point de soir, c’est-à-dire du repos des saints qui ne finira jamais, de sorte que l’Écriture appelle ici mille ans la dernière partie de ce temps, en prenant la partie pour le tout ; — ou bien elle se sert de ce nombre pour toute la durée du monde, employant ainsi un nombre parfait pour marquer la plénitude du temps. Le nombre de mille est le cube de dix, dix fois dix faisant cent ; mais c’est là une figure plane, et pour la rendre solide, il faut multiplier cent par dix et cela fait mille. D’ailleurs, si l’Écriture se sert de cent pour un nombre indéfini, comme lorsque Notre-Seigneur promet à celui qui quittera tout pour le suivre : « qu’il recevra le centuple dès cette vie », ce que l’Apôtre exprime en disant qu’un véritable chrétien possède toutes choses, bien qu’il semble qu’il n’ait rien, selon cette parole encore : « Le monde est le trésor du fidèle », combien plus le nombre de mille ans doit-il signifier l’universalité ! Aussi est-ce le meilleur sens qu’on puisse donner à ces paroles du psaume : « Il s’est toujours souvenu de son alliance et de la promesse qu’il a faite pour mille générations » ; c’est-à-dire pour toutes les générations.\par
Saint Jean poursuit : « Et il le précipita dans l’abîme » ; par cet abîme est marquée la multitude innombrable des impies, dont le cœur est un gouffre de malignité contre l’Église de Dieu ; non que le diable n’y fût déjà auparavant, mais parce qu’étant exclu de la Société des fidèles, il a commencé à posséder davantage les autres. Celui-là est plus possédé du diable, qui non seulement est éloigné de Dieu, mais qui hait même les serviteurs de Dieu sans raison. « Et il le ferma, dit-il, et le scella sur lui, afin qu’il ne séduisît plus les nations jusqu’à ce que mille ans fussent accomplis. » Il le ferma sur lui, c’est-à-dire il lui défendit d’en sortir. Ce qu’ajoute saint Jean, qu’il le scella, signifie, selon moi, que Dieu ne veut pas qu’on sache quels sont ceux qui appartiennent au démon ou ceux qui ne lui appartiennent pas, et c’est une chose tout à fait incertaine en cette vie, parce qu’il est incertain si celui qui semble être debout ne tombera point, et si celui qui semble être tombé ne se relèvera point. Or, le diable est ainsi lié et enfermé pour être incapable de séduire les nations qui appartiennent à Jésus-Christ et qu’il séduisait auparavant. « Dieu », comme dit l’Apôtre, « a résolu, avant la naissance du monde, de les délivrer de la puissance des ténèbres et de les faire passer dans le royaume du Fils de son amour. » Les fidèles ignorent-ils que maintenant même le démon séduit les nations et les entraîne avec lui au supplice éternel ? mais ce ne sont pas celles qui sont prédestinées à la vie bienheureuse.\par
Il ne faut pas s’arrêter à-ce que le diable séduit souvent ceux mêmes qui, régénérés en Jésus-Christ, marchent dans les voies de Dieu ; car « le Seigneur connaît ceux qui sont à lui » ; et de ceux-là, Satan n’en séduit aucun jusqu’à le faire tomber dans la damnation éternelle. Le Seigneur les connaît comme Dieu, c’est-à-dire comme celui à qui rien de ce qui doit arriver n’est caché, et non comme un homme, qui ne voit un autre homme que quand il est présent, si toutefois on peut dire qu’il voit celui dont il ne voit pas le cœur, et dont il ne sait pas ce qu’il doit devenir ensuite, non plus que lui-même. Le diable est donc lié et enfermé dans l’abîme, afin qu’il ne séduise pas les nations qui composent l’Église et qu’il séduisait auparavant, lorsque l’Église n’était pas encore. Il n’était pas dit, en effet, « afin qu’il ne séduisît plus personne », mais : « afin qu’il ne séduisît plus les nations », par lesquelles l’Apôtre a voulu sans doute qu’on entendît l’Église. — « Jusqu’à ce que mille ans fussent accomplis », c’est-à-dire ce qui reste du sixième jour qui est de mille ans, ou bien ce qui reste de la durée du monde.\par
Et ces mots : « Afin qu’il ne séduisît plus les nations, jusqu’à ce que mille ans fussent accomplis », il ne faut pas les entendre comme s’il devait plus tard séduire les nations qui composent l’Église des prédestinés. Car ou bien cette expression est semblable à celle-ci : « Nos yeux sont arrêtés sur le Seigneur notre Dieu, jusqu’à ce qu’il ait pitié de nous » ; où il est clair que, lorsque Dieu aura pris pitié de ses serviteurs, ils ne laisseront pas de jeter les yeux sur lui ; ou bien voici l’ordre de ces paroles : « Et il ferma l’abîme et il le scella sur lui, jusqu’à ce que mille ans fussent accomplis », de sorte que ce qu’il ajoute : « afin qu’il ne séduisît plus les nations », doit s’entendre, indépendamment du reste, comme si toute période était conçue ainsi : « Et il ferma l’abîme, et il le scella sur lui, jusqu’à ce que mille ans fussent accomplis, afin qu’il ne séduisît plus les nations. » En d’autres termes, c’est afin qu’il cesse de séduire les nations que l’abîme est fermé jusqu’à la révolution de mille ans.
\subsection[{Chapitre VIII}]{Chapitre VIII}

\begin{argument}\noindent Du diable enchaîné et délié de ses chaînes.
\end{argument}

\noindent « Après cela », dit saint Jean, « il doit être délié pour un peu de temps. » Si le diable est lié et enfermé, afin qu’il ne puisse pas séduire l’Église, sa délivrance consistera-telle à le pouvoir ? À Dieu ne plaise ! Il ne séduira jamais l’Église prédestinée et élue avant la création du monde, dont il est dit que : « Le Seigneur connaît ceux qui sont à lui. » Cependant il y aura ici-bas une Église, au temps que le diable doit être délié, comme il y en a toujours eu une depuis Jésus-Christ. Saint Jean dit un peu après, que le diable, une fois délié, portera les nations qu’il aura séduites dans le monde entier, à faire la guerre à l’Église, et que le nombre de ses ennemis égalera les sables de la mer : « Et ils se répandirent, dit-il, sur la terre, et ils environnèrent le camp des saints et la Cité bien-aimée de Dieu. Mais Dieu fit tomber un feu du ciel qui les dévora ; et le diable, qui les séduisait, fut jeté dans un étang de feu et de soufre avec la bête et le faux prophète, pour y être tourmentés jours et nuit dans les siècles des siècles. » Ce passage regarde le dernier jugement, et néanmoins j’ai été bien aise de le rapporter, de peur qu’on ne s’imagine que, dans le peu de temps que le diable doit être délié, il n’y aura point d’Église en ce monde, soit qu’il ne l’y trouve plus, soit qu’il la détruise par ses persécutions. Le diable n’a donc pas été lié dans tout ce temps que comprend l’Apocalypse, savoir : depuis le premier avènement de Jésus-Christ jusqu’à la fin du monde où se fera le second. Et c’est ce que saint Jean appelle mille ans, en sorte que l’Écriture entend par là que le diable ne séduira pas l’Église pendant cet intervalle, puisqu’il ne la séduira pas non plus lorsqu’il sera délié. En effet, il est indubitable que si c’est être lié pour lui que de pouvoir séduire l’Église, il le pourra faire quand il sera délié. Être lié par rapport au diable, c’est donc n’avoir pas permission de tenter les hommes autant qu’il peut, par adresse ou par violence, pour les faire passer à son parti. Si cela lui était permis pendant un si long espace de temps, la faiblesse des hommes est telle qu’il ferait tomber un grand nombre de fidèles et qu’il empêcherait beaucoup d’hommes de le devenir, ceque Dieu ne veut pas. Aussi est-ce pour l’en empêcher qu’il l’a lié.\par
Mais il sera délié quand il ne restera que peu de temps. L’Écriture nous apprend que le démon et ses complices tourneront toute leur rage contre l’Église pendant trois ans et demi ; et ceux à qui il aura affaire seront tels qu’il ne les pourra surmonter ni par force, ni par artifice. Or, s’il n’était jamais délié, on ne connaîtrait pas si bien sa puissance et sa malignité, ni la patience de la cité sainte, non plus que la sagesse admirable avec laquelle le Tout-Puissant a su se servir de la malice du diable, soit en ne l’empêchant pas de séduire les saints, afin d’exercer leur vertu, soit en ne lui permettant pas d’user de toute sa fureur, de peur qu’il ne triomphât d’une infinité d’hommes faibles qui devaient grossir les rangs de l’Église. Il sera donc délié sur la fin des temps, afin que la Cité de Dieu reconnaisse, à la gloire de son Rédempteur et de son Libérateur, quel adversaire elle aura surmonté. Que sommes-nous en comparaison des chrétiens qui seront alors, puisqu’ils surmonteront un ennemi déchaîné, que nous avons bien de la peine à combattre, tout lié qu’il est ? Néanmoins, il n’y a point de doute que pendant cet intervalle même, Dieu n’ait eu et n’ait encore des soldats si braves et si expérimentés que, fussent-ils vivants quand le diable sera délié, ils ne craindraient ni ses efforts, ni ses ruses.\par
Or, le diable n’a pas seulement été lié lorsque l’Église a commencé de se répandre de la Judée parmi les nations ; mais il l’est encore maintenant et le sera jusqu’à la fin des siècles, où il doit être délié. Nous voyons encore tous les jours des personnes quitter leur infidélité dans laquelle le démon les retenait, et embrasser la foi ; et il y en aura toujours sans doute qui se convertiront jusqu’à la fin du monde. Le {\itshape fort} est lié de même à l’égard de chacun des fidèles, lorsqu’ils lui sont enlevés comme sa proie ; comme, d’autre part, l’abîme où il a été enfermé n’a pas été détruit par la mort des premiers persécuteurs de l’Église ; mais à ceux-là d’autres ont succédé et leur succéderont jusqu’à la fin des siècles, afin qu’il soit toujours enfermé dans ces cœurs pleins de passion et d’aveuglement, comme en un abîme profond. Or, c’est une questionde savoir si, pendant ces trois dernières années et demie que le démon exercera toute sa fureur, il y aura encore quelques hommes, au milieu des fidèles, qui embrasseront la foi. Comment cette parole se justifierait-elle : « Personne ne peut entrer dans la maison du fort et lui « enlever ses biens, qu’il ne l’ait d’abord lié », si on les lui enlève lors même qu’il est délié ? Il semble donc que cela nous oblige à croire qu’en ce peu de temps l’Église ne fera aucune nouvelle conquête, mais que le diable combattra seulement contre ceux qui se trouveront déjà chrétiens ; et si quelques-uns de ceux-là sont vaincus, il faut dire qu’ils n’étaient pas du nombre des prédestinés. Ce n’est pas en vain que le même saint Jean, qui a écrit l’Apocalypse, a dit de quelques-uns dans une de ses Épîtres : « Ils sont sortis d’avec nous, mais ils n’étaient pas d’entre nous ; car s’ils eussent été d’entre nous, ils y seraient demeurés. » Mais que dirons-nous des petits enfants ? Il n’est pas croyable que cette dernière persécution n’en trouve point parmi les chrétiens qui ne soient pas baptisés, et que même il ne leur en naisse pendant ce temps, et en ce cas que leurs parents ne les baptisent. Comment donc enlèvera-t-on ces biens à Satan, puisqu’il sera délié, et que, selon la parole du Seigneur : « Personne n’entre en sa maison et ne lui enlève ses biens, qu’il ne l’ait lié auparavant » ? Croyons donc plutôt que, même pendant ce temps, les apostasies ne manqueront point, non plus que les conversions, et que les parents auront assez de courage pour baptiser leurs enfants, aussi bien que les nouveaux convertis, qu’ils vaincront ce fort, tout délié qu’il sera, c’est-à-dire quoiqu’il emploie contre eux des ruses et des manœuvres qu’il n’avait point encore mises en usage, tellement qu’ils lui seront encore enlevés, quoiqu’il ne soit pas lié. Néanmoins, la parole de l’Évangile subsistera toujours : « Que personne ne peut entrer dans la maison du fort, ni lui enlever ses biens, qu’il ne l’ait lié auparavant. » Cet ordre a été, en effet, observé. On a lié d’abord le fort, et on lui a ensuite enlevé ses biens dans toutes les nations, pour en composer l’Église, qui s’est depuis accrue et fortifiée au point de devenir capable de dépouiller le démon, lors même qu’il sera délié. De même qu’il faut avouer que la charité de plusieurs se refroidira,parce que le crime sera triomphant, et que plusieurs, qui ne sont pas écrits au livre de vie, succomberont sous les persécutions inouïes du diable déjà délié, de même il faut croire que non seulement les véritables chrétiens, mais que quelques-uns de ceux mêmes qui seront hors de l’Église, aidés de la grâce de Dieu et de l’autorité des Écritures, qui ont prédit la fin du monde qu’ils verront arriver, seront plus disposés à croire ce qu’ils ne croyaient pas, et plus forts pour vaincre le diable, tout déchaîné qu’il sera. Disons, dans cet état de choses, qu’il a été lié afin qu’on lui puisse enlever ses biens, lors même qu’il sera délié, suivant cette parole du Sauveur : « Comment peut-on entrer dans la maison du fort pour lui enlever ses biens, qu’on ne l’ait lié auparavant ? »
\subsection[{Chapitre IX}]{Chapitre IX}

\begin{argument}\noindent En quoi consiste le règne des saints avec Jésus-Christ, pendant mille ans, et en quoi il diffère du règne éternel.
\end{argument}

\noindent Pendant les mille ans que le diable est lié, c’est-à-dire pendant tout le temps qui s’écoule depuis le premier avènement du Sauveur jusqu’au second, les saints règnent avec lui. Et, en effet, si, outre le royaume dont il doit dire à la fin des siècles : « Venez, vous que mon Père a bénis, prenez possession du royaume qui vous a été préparé » ; ses saints, à qui il dit : « Je suis avec vous jusqu’à la fin du monde », n’en avaient, dès maintenant, un autre où ils règnent avec lui, certes l’Église ne serait pas appelée son royaume ou le royaume des cieux. Car c’est à cette heure que le docteur de la loi, dont parle l’Évangile, « qui tire de son trésor de nouvelles et de vieilles choses », est instruit dans le royaume de Dieu ; et c’est de l’Église que les moissonneurs doivent arracher l’ivraie que le père de famille avait laissé croître parmi le bon grain jusqu’à la moisson. Notre-Seigneur explique ainsi cette parabole : « La moisson, c’est la fin du siècle. Comme donc on ramasse l’ivraie et on la jette au feu la même chose arrivera à la fin du monde. Le Fils de l’homme enverra ses anges, et ils arracheront de son royaume tous les scandales. » Sera-ce du royaume où il n’y a pas de scandales ? Non, sans doute. Ce sera donc de celui d’ici-bas, qui est son Église. Il dit plus haut : « Celui qui violera l’un de ces moindres commandements et qui enseignera aux hommes à le suivre sera le dernier dans le royaume des cieux ; mais celui qui l’accomplira et qui l’enseignera sera grand dans les cieux. » Il les place tous deux dans le royaume des cieux, tant celui qui ne fait pas ce qu’il enseigne que celui qui le fait ; mais l’un est très petit et l’autre très grand. Il ajoute aussitôt : « Car je vous dis que si votre justice n’est pas plus grande que celle des Scribes et des Pharisiens (c’est-à-dire que la justice de ceux qui ne font pas ce qu’ils enseignent, puisqu’il déclare d’eux dans un autre endroit : Qu’ils disent ce qu’il faut faire et qu’ils ne le font pas), vous n’entrerez point dans le royaume des cieux. » Il faut donc entendre d’une autre manière le royaume des cieux où sont et celui qui ne pratique pas ce qu’il enseigne et celui qui le pratique, et le royaume où n’entre que celui qui pratique ce qu’il enseigne. Ainsi le premier, c’est l’Église d’ici-bas, et le second, c’est l’Église telle qu’elle sera, quand les méchants n’y seront plus. L’Église est donc maintenant le royaume de Jésus-Christ et le royaume des cieux, de sorte que dès à présent les saints de Dieu règnent avec lui, mais autrement qu’ils ne régneront plus tard. Néanmoins l’ivraie ne règne point avec lui, quoiqu’elle croisse dans l’Église avec le bon grain. Ceux-là seuls règnent avec lui qui font ce que dit l’Apôtre : « Si vous êtes ressuscités avec Jésus-Christ, goûtez les choses du ciel, où Jésus-Christ est assis à la droite de Dieu ; cherchez les choses du ciel et non celles de la terre. » Il dit d’eux encore que leur conversation est dans le ciel. Enfin, ceux-là règnent avec lui, qui sont tellement dans son royaume qu’ils sont eux-mêmes son royaume. Or, comment ceux-là sont-ils le royaume de Jésus-Christ, qui, bien qu’ils y soient jusqu’à la fin du monde et des scandales, y cherchent leurs intérêts et non pas ceux de Jésus-Christ ?\par
Voilà comment l’Apocalypse parle de ce royaume, où l’on a encore des ennemis à combattre ou à retenir dans le devoir, jusqu’à ce qu’on arrive dans le royaume paisible où l’on régnera sans trouble et sans traverses.\par
Voilà comment elle s’explique sur cette première résurrection qui se fait maintenant. Après avoir dit que le diable demeurera lié pendant mille ans, et qu’ensuite il doit être délié pour un peu de temps, aussitôt reprenant ce que l’Église fait pendant ces mille ans ou ce qui se passe dans l’Église : « Et je vis, dit-il, des trônes et des hommes assis sur ces trônes ; et on leur donna le pouvoir de juger. » Il ne faut pas s’imaginer que ceci soit dit du dernier jugement, mais il s’agit des trônes des chefs et des chefs qui gouvernent maintenant même l’Église. Quant au pouvoir de juger qui leur est donné, il semble qu’on ne le puisse mieux entendre que de cette promesse : « Ce que vous lierez sur la terre sera lié au ciel, et ce que vous délierez sur la terre sera délié au ciel. » Ce qui fait dire à l’Apôtre : « Qu’ai-je affaire de juger ceux qui sont hors de l’Église ? N’êtes-vous pas juges de ceux qui sont dedans ? » — « Et les âmes », continue saint Jean, « de ceux qui ont été mis à mort pour avoir rendu témoignage à Jésus ». Il faut sous-entendre ce qu’il dit ensuite : « Ont régné mille ans avec Jésus » ; c’est-à-dire : Les âmes des martyrs encore séparées de leur corps. En effet, les âmes des justes trépassés ne sont point séparées de l’Église, qui maintenant même est le royaume de Jésus-Christ. Autrement on n’en ferait point mémoire à l’autel dans la communion du corps de Jésus-Christ ; et il ne servirait de rien dans le danger de recourir à son baptême, pour ne pas sortir du monde sans l’avoir reçu, ou à la réconciliation, lorsqu’on a été séparé de ce même corps par la pénitence ou par la mauvaise vie. Pourquoi ces saintes pratiques, sinon parce que les fidèles, tout morts qu’ils sont, ne laissent pas d’être membres de l’Église ? Dès lors leurs âmes, quoique séparées de leurs corps, règnent déjà avec Jésus-Christ pendant ces mille ans ; d’où vient qu’on lit dans le même livre de l’Apocalypse : « Bienheureux sont les morts qui meurent dans le Seigneur ! l’Esprit leur dit déjà qu’ils se reposent de leurs travaux, car leurs œuvres les suivent. » L’Église commence donc par régner ici avec Jésus-Christ dans les vivants et dans les morts ; car, comme dit l’Apôtre : « Jésus-Christ est mort afin d’avoir empire sur les vivants et sur les morts. » Mais saint Jean ne fait mention que des âmes des martyrs, parce que ceux-là règnent principalement avec Jésus-Christ après leur mort, qui ont combattu jusqu’à la mort pour la vérité ; ce qui n’empêche point qu’en prenant la partie pour le tout, nous ne devions entendre que les autres morts appartiennent aussi à l’Église, qui est le royaume de Jésus-Christ.\par
Les paroles qui suivent : « Et tous ceux qui n’ont point adoré la bête ni son image, ni reçu son caractère sur le front ou dans leur main », doivent être entendues des vivants et des morts. Pour cette bête, quoique cela demande un plus long examen, on peut fort bien l’expliquer par la cité impie et par le peuple infidèle, contraires au peuple fidèle et à la Cité de Dieu. J’entends par son image le déguisement de ceux qui, faisant profession de foi, vivent comme des infidèles. Ils feignent d’être ce qu’ils ne sont pas, et ne sont chrétiens que de nom. En effet, non seulement les ennemis déclarés de Jésus-Christ et de sa cité appartiennent à la bête, mais encore l’ivraie qui doit être ôtée à la fin du monde de son royaume, qui est l’Église. Et qui sont ceux qui n’adorent ni la bête ni son image, sinon ceux qui font ce que dit l’Apôtre, et qui ne sont point attachés à un même joug avec les infidèles ? Ils n’adorent point, c’est-à-dire ils ne consentent point ; ils ne se soumettent point et ne reçoivent point le caractère, c’est-à-dire le sceau du crime, ni sur le front par leur profession, ni dans leurs mains par leurs actions. Ceux qui sont exempts de cette profanation, qu’ils vivent encore dans cette chair mortelle ou qu’ils soient morts, règnent dès maintenant avec Jésus-Christ pendant tout le temps désigné par mille ans.\par
« Les autres », dit saint Jean, « n’ont point vécu ; car c’est maintenant le temps que les morts entendront la voix du Fils de Dieu, et que ceux qui l’entendront vivront ; mais, pour les autres, ils ne vivront point. » Et quant à ce qu’il ajoute : « Jusqu’à ce que mille ans soient accomplis », il faut entendre par là qu’ils n’ont point vécu pendant le temps où ils devaient vivre, en passant de la mort à la vie. Ainsi, quand le temps de la résurrection des corps sera arrivé, ils ne sortiront point de leurs tombeaux pour vivre, mais pour être jugés et condamnés, ce quiconstitue la seconde mort. Car, jusqu’à ce que les mille ans soient accomplis, quiconque, pendant tout ce temps où se fait la première résurrection, n’aura point vécu, c’est-à-dire n’aura point entendu la voix du Fils de Dieu, ni passé de la mort à la vie, passera infailliblement à la seconde mort avec son corps dans la seconde résurrection, qui est celle des corps. Saint Jean ajoute : « Voilà la première résurrection. Heureux et saint est celui qui y participe ! » Or, celui-là seul y participe qui non seulement ressuscitera en sortant du péché, mais qui encore persévérera dans cet état de résurrection. « La seconde mort, dit-il, n’a point de pouvoir sur ceux-là » ; mais elle en a sur les autres, dont il a dit auparavant : « Les autres n’ont pas vécu, jusqu’à ce que mille ans soient accomplis. » Encore que dans cet espace qu’il nomme mille ans, ils aient vécu de la vie du corps, ils n’ont pas vécu de celle de l’âme en ressuscitant et en sortant de la mort du péché, afin d’avoir part à la première résurrection et de ne pas tomber sous l’empire de la seconde mort.
\subsection[{Chapitre X}]{Chapitre X}

\begin{argument}\noindent Ce qu’il faut répondre à ceux qui pensent que la résurrection regarde seulement les corps, et non les âmes.
\end{argument}

\noindent Il en est qui croient qu’on ne peut parler de résurrection qu’à l’égard des corps, et qui soutiennent que cette première résurrection dont parle saint Jean doit s’entendre de la résurrection des corps. Il n’appartient, disent-ils, de se relever qu’à ce qui tombe ; or, les corps tombent en mourant, d’où vient qu’on les appelle des cadavres ; donc ce ne sont pas les âmes qui ressuscitent, mais les corps. Mais que répondront-ils à l’Apôtre qui admet aussi une résurrection de l’âme ? Ceux-là étaient ressuscites selon l’homme intérieur, et non pas selon l’homme extérieur, à qui il dit « Si vous êtes ressuscités avec Jésus-Christ, ne goûtez plus que les choses du ciel. » C’est la même pensée qu’il exprime ailleurs en d’autres termes : « Afin, dit-il, qu’à l’exemple de Jésus-Christ qui est ressuscité des mortspour la gloire du Père, nous marchions aussi dans la vie nouvelle. » De là encore cette parole : « Levez-vous, vous qui dormez, levez-vous d’entre les morts, et Jésus-Christ vous éclairera. » Quand ces interprètes disent qu’il n’appartient qu’aux corps de tomber, ils n’entendent pas cette parole : « Ne vous éloignez point de lui, de peur que vous ne tombiez » ; ni celle-ci : « S’il tombe ou s’il demeure debout, c’est pour son maître » ; ni celle-ci encore : « Que celui qui se croit debout prenne garde de tomber. » Assurément cette chute s’entend de l’âme et non du corps.\par
Si donc c’est à ce qui tombe à ressusciter, et si les âmes tombent comme les corps, il faut convenir qu’elles ressuscitent aussi. Ce que saint Jean ajoute, après avoir dit que la seconde mort n’a point de pouvoir sur ceux-là, savoir, qu’ils seront prêtres de Dieu et de Jésus-Christ, et qu’ils régneront avec lui l’espace de mille ans, cela ne doit pas s’entendre des seuls évêques ou des seuls prêtres, mais de tous les fidèles qu’il nomme prêtres, parce qu’ils sont tous membres d’un seul grand-prêtre, de même qu’on les appelle tous chrétiens, à cause du chrême mystique auquel ils ont tous part. Aussi est-ce d’eux que l’apôtre saint Pierre a dit : « Le peuple saint et le sacerdoce royal. » Il est à remarquer d’ailleurs que saint Jean déclare, bien qu’en peu de mots et en passant, que Jésus-Christ est Dieu, lorsqu’il appelle les chrétiens {\itshape les prêtres de Dieu et de Jésus-Christ}, c’est-à-dire du Père et du Fils. Et de plus, Jésus-Christ, bien qu’il soit fils de l’homme, à cause de la forme d’esclave qu’il a prise, a été aussi fait prêtre éternel selon l’ordre de Melchisédech, comme nous l’avons dit plusieurs fois.
\subsection[{Chapitre XI}]{Chapitre XI}

\begin{argument}\noindent De Gog et de Magog que le diable, délié a l’approche de la fin des siècles, suscitera contre l’Église.
\end{argument}

\noindent « Et quand les mille ans seront révolus, Satan sera délivré de sa prison, et il sortira pour séduire les nations qui sont aux quatre coins du monde, Gog et Magog ; et il les portera à faire la guerre, et leur nombre égalera les grains de sable de la mer. » Il les séduira donc alors, pour les attirer dans cette guerre ; car auparavant il les séduisait aussi tant qu’il pouvait par une infinité d’artifices. Mais alors il sortira, c’est-à-dire qu’il fera éclater sa haine et persécutera ouvertement. Cette persécution sera la dernière que l’Église souffrira, mais dans toute la terre, c’est-à-dire que toute la cité de Dieu sera persécutée à travers toute la cité des impies. Il ne faut pas entendre par Gog et Magog des peuples barbares d’une certaine contrée du monde, comme ont fait ceux qui pensent que ce sont les Gètes elles Massagètes, à cause des premières lettres de ces noms. En effet, l’Écriture marque clairement qu’ils seront répandus dans tout l’univers, quand elle dit : « Les nations qui sont aux quatre coins de la terre » ; et elle ajoute que c’est Gog et Magog. Or, nous avons acquis la certitude que Gog signifie {\itshape toit}, et Magog, {\itshape du toit} ; comme qui dirait « la maison et celui qui en sort ». Ces nations sont donc, comme nous disions un peu plus haut, l’abîme où le diable est enfermé ; et c’est lui-même qui en sort de sorte qu’elles sont « la maison », et lui « celui qui sort de la maison ». Ou bien, si par ces deux mots nous voulons entendre les nations, « elles sont la maison », parce que le diable y est enfermé maintenant, et comme à couvert, et « elles sortiront de la maison », lorsqu’elles feront éclater la haine qu’elles couvent. Quant à ces paroles : « Et ils se répandirent sur la terre et environnèrent le camp des saints et la Cité bien-aimée », il ne faut pas les entendre comme si les ennemis étaient venus ou devaient venir en un lieu particulier et déterminé, puisque le camp des saints et la Cité bien-aimée ne sont autre chose que l’Église qui sera répandue sur toute la terre. C’est là qu’elle sera assiégée et pressée par ses ennemis, qui exciteront contre elle une cruelle persécution, et mettront en usage tout ce qu’ils auront de rage et de malice, sans pouvoir triompher de son courage, ni lui faire abandonner, comme le marque le texte sacré, son camp et ses étendards.
\subsection[{Chapitre XII}]{Chapitre XII}

\begin{argument}\noindent Si le feu que saint Jean vit descendre du ciel et dévorer les impies doit s’entendre du dernier supplice.
\end{argument}

\noindent Saint Jean ajoute : « Et un feu descendit du ciel, qui les dévora » ; il ne faut pas entendre cela du dernier supplice auquel ils seront voués, quand il leur sera dit : « Retirez-vous de moi, maudits, et allez au feu éternel. » Car alors ils seront envoyés dans le feu, et le feu ne tombera pas du ciel sur eux. Or, par le ciel, on peut fort bien entendre ici la fermeté des saints, qui les empêchera de succomber sous la violence de leurs persécuteurs. Le firmament est le ciel, et c’est cette fermeté céleste qui allume dans le cœur des méchants un zèle ardent, un zèle qui les désespère, quand ils se voient dans l’impuissance d’attirer les saints de Jésus-Christ au parti de l’Antéchrist. Voilà le feu qui les dévorera ; « ce feu qui vient de Dieu », parce que c’est sa grâce qui rend les saints invincibles, éternel sujet de tourments pour leurs ennemis. De même qu’il y a un bon zèle, comme celui dont parle le Psalmiste, quand il dit : « Le zèle de votre maison me dévore » ; il y en a aussi un mauvais, ainsi que le dit l’Écriture : « Le zèle s’est emparé d’une populace ignorante, et c’est maintenant le feu qui consume les impies » ; — maintenant, dit le texte sacré, et c’est sans préjudice du feu du dernier jugement. Si saint Jean a entendu par ce feu la plaie qui frappera les persécuteurs de l’Église à la venue de Jésus-Christ, lorsqu’il tuera l’Antéchrist du souffle de sa bouche, ce ne sera pas non plus le dernier supplice des impies, mais celui qu’ils doivent souffrir après la résurrection des corps.
\subsection[{Chapitre XIII}]{Chapitre XIII}

\begin{argument}\noindent Si le temps de la persécution de l’Antéchrist doit être compris dans les mille ans.
\end{argument}

\noindent Cette dernière persécution de l’Antéchrist doit durer trois ans et demi, selonl’Apocalypse et le prophète Daniel. Bien que ce temps soit court, on a raison de demander s’il sera compris ou non dans les mille ans de la captivité du diable et du règne des saints. S’il y est compris, le règne des saints s’étendra au-delà de la captivité du diable, et ils régneront avec leur roi, lors même que le diable sera délié et qu’il les persécutera de tout son pouvoir. Comment alors l’Écriture détermine-t-elle le règne des saints et la captivité du diable par le même espace de mille ans, si le diable doit être délié trois ans et demi avant que les saints cessent de régner ici-bas avec Jésus-Christ ? D’un autre côté, si nous disons que les trois ans et demi ne sont pas compris dans les mille ans, afin que le règne des saints cesse avec la captivité du diable, ce qui semble être le sens le plus naturel des paroles de l’Apocalypse, nous serons obligés d’avouer que les saints ne régneront point avec Jésus-Christ pendant cette persécution. Mais qui oserait dire que les membres du Sauveur ne régneront pas avec lui, lorsqu’ils lui seront le plus étroitement unis, et que la gloire des combattants sera d’autant plus grande et leur couronne plus éclatante, que le combat aura été plus rude et plus opiniâtre ? Ou si l’on prétend qu’il n’est pas convenable de dire qu’ils régneront alors, à cause des maux qu’ils souffriront, il faudra dire aussi que pendant les mille ans mêmes, tous les saints qui ont souffert ne régnaient pas avec Jésus-Christ au temps de leur souffrance, et qu’ainsi ceux qui ont été égorgés pour avoir rendu témoignage à Jésus-Christ et pour la parole de Dieu, ces martyrs dont l’auteur de l’Apocalypse dit qu’il a vu les âmes, ne régnaient pas avec ce Sauveur, quand ils enduraient la persécution, et qu’ils n’étaient pas son royaume, quand il les possédait d’une manière si excellente. Or, il n’est rien de plus faux, ni de plus absurde. An moins ne peut-on pas nier que les âmes des martyrs ne règnent pendant les mille ans avec Jésus-Christ, et qu’elles ne règnent même après avec lui, lorsque le diable sera délié. Il faut croire aussi, par conséquent, qu’après les mille ans, les saints régneront encore avec ce Sauveur, et qu’ainsi leur règne s’étendra de ces trois ans et demi au-delà de la captivité du diable. Lors donc que saint Jean dit : « Les prêtres de Dieu et de Jésus-Christ régnerontavec lui pendant mille ans ; et les mille ans finis, Satan sera délivré de sa prison » ; il faut entendre que les mille ans ne finiront pas le règne des saints, mais seulement la captivité du diable ; ou du moins, comme trois ans et demi sont peu considérables, en comparaison de tout le temps qui est marqué par mille ans, l’Écriture ne s’est pas mise en peine de les y comprendre. Nous avons déjà vu la même chose, au seizième livre de cet ouvrage, au sujet des quatre cents ans, bien qu’il y eût un peu plus : coutume assez fréquente dans les saintes Écritures, si l’on y veut faire attention.
\subsection[{Chapitre XIV}]{Chapitre XIV}

\begin{argument}\noindent De la damnation du diable et des siens, et récapitulation de ce qui a été dit sur la résurrection des corps et le jugement dernier.
\end{argument}

\noindent Après avoir parlé de la dernière persécution, saint Jean résume en peu de mots ce que le diable doit souffrir au dernier jugement avec la cité dont il est le prince : « Et le diable, dit-il, qui les séduisait, fut jeté dans un étang de feu et de soufre, où la bête et le faux prophète seront tourmentés jour et nuit, dans les siècles des siècles. » Nous avons dit plus haut que par la bête, on peut fort bien entendre la cité impie ; et quant à son faux prophète, c’est ou l’Antéchrist, ou cette image, ce fantôme dont nous avons parlé dans Je même endroit. L’Apôtre revient ensuite au dernier jugement qui se fera à la seconde résurrection des morts, c’est-à-dire à celle des corps, et déclare comment il lui a été révélé : « Je vis, dit-il, un grand trône blanc, et celui qui était assis dessus, devant qui le ciel et la terre s’enfuirent et disparurent. » Il ne dit pas : Je vis un grand trône blanc, et celui qui était assis dessus, et le ciel et la terre s’enfuirent devant lui, parce que cela n’arriva pas alors, c’est-à-dire avant qu’il eût jugé les vivants et les morts ; mais il dit qu’il vit assis sur le trône celui devant qui le ciel et la terre s’enfuirent dans la suite. Lorsque le jugement sera achevé, ce ciel et cette terre cesseront en effet d’exister, et il y aura un ciel nouveau et une terre nouvelle. Ce monde passera, non par destruction, mais par changement ; ce qui a fait dire à l’Apôtre : « La figure de ce mondepasse ; c’est pourquoi je désire que vous viviez sans soin et sans souci de ce monde » ; c’est donc la figure du monde qui passe, et non sa nature. Saint Jean, après avoir dit qu’il vit celui qui était assis sur le trône, devant qui s’enfuient le ciel et la terre, ce qui n’arrivera qu’après, ajoute : « Je vis aussi les morts, grands et petits ; et des livres furent ouverts ; et un autre livre fut ouvert, qui est le livre de la vie de chacun, et les morts furent jugés sur ce qui était écrit dans ces livres, chacun selon ses œuvres. » Il dit que des livres furent ouverts, ainsi qu’un autre, « qui est le livre de la vie de chacun ». Or, ces premiers livres sont l’Ancien et le Nouveau Testament, pour montrer les choses que Dieu a ordonné qu’on fit ; et cet autre livre particulier de la vie de chacun est là pour faire voir ce que chacun aura ou n’aura pas fait. À prendre ce livre matériellement combien faudrait-il qu’il fût grand et gros ? ou combien faudrait-il de temps pour lire un livre contenant la vie de chaque homme ? Est-ce qu’il y aura autant d’anges que d’hommes, et chacun entendra-t-il le récit de sa vie de la bouche de l’ange qui lui sera assigné ? À ce compte, il n’y aurait donc pas un livre pour tous, mais pour un chacun. Cependant l’Écriture n’en marque qu’un pour tous, quand elle dit : « Et un autre livre fut ouvert »… Il faut dès lors entendre par ce livre une vertu divine, par laquelle chacun se ressouviendra de toutes ses œuvres, tant bonnes que mauvaises, et elles lui seront toutes présentées en un instant, afin que sa conscience le condamne ou le justifie, et qu’ainsi tous les hommes soient payés en un moment, Si cette vertu divine est nommée un livre, c’est qu’on y lit, en quelque sorte, tout ce qu’on se souvient d’avoir fait. Pour montrer que les morts doivent être jugés, c’est-à-dire les grands et les petits, il ajoute, par forme de récapitulation et en reprenant ce qu’il avait omis, ou plutôt ce qu’il avait différé : « Et la mer présenta ses morts, et la mort et l’enfer rendirent les leurs » ; ce qui arriva sans doute avant que les morts fussent jugés, et cependant il ne le rapporte qu’après. Ainsi j’ai raison de dire qu’il reprend ce qu’il avait omis. Mais maintenant il garde l’ordre, et croit devoirrépéter ce qu’il avait déjà dit du jugement. Après ces paroles : « Et la mer rendit ses morts, et la mort et l’enfer rendirent les leurs », il ajoute aussitôt : « Et chacun fut jugé selon ses œuvres » ; et c’est ce qu’il avait dit avant : « Les morts furent jugés selon leurs œuvres. »
\subsection[{Chapitre XV}]{Chapitre XV}

\begin{argument}\noindent Des morts que vomit la mer pour le jugement, et de ceux que la mort et l’enfer rendirent.
\end{argument}

\noindent Mais quels sont ces morts que la nier contenait et qu’elle vomit ? Ceux qui meurent dans la mer échapperaient-ils à l’enfer ? ou bien est-ce que la mer conserve leurs corps ? ou bien, ce qui est encore plus absurde, la mer aurait-elle les bons et l’enfer les méchants ? qui le croira ? Il me semble donc que c’est avec quelque raison qu’on a entendu ici le siècle par la mer. Ainsi saint Jean, voulant dire que ceux que Jésus-Christ trouvera encore vivants seront jugés avec ceux qui doivent ressusciter, les appelle aussi morts, tant les bons que les méchants : les bons, à qui il est dit « Vous êtes morts, et votre vie est cachée en Dieu avec Jésus-Christ » ; et les méchants, dont il est dit : « Laissez les morts ensevelir leurs morts. » On peut aussi les appeler morts en ce qu’ils ont des corps mortels ; ce qui a donné lieu à cette parole de l’Apôtre : « Il est vrai que le corps est mort, à cause du péché ; mais l’esprit est vivant, à cause de la justice » ; montrant par là que l’un et l’antre est dans un homme vivant : un corps vivant et un esprit qui vit. Il ne dit pas toutefois le corps mortel, mais {\itshape le corps mort}, bien qu’il le dise ensuite, comme on a coutume de l’appeler communément. Ce sont ces morts que la {\itshape mer vomit} ; entendez que ce siècle présentera les hommes qu’il contenait, parce qu’ils n’étaient pas encore morts. « Et la mort et l’enfer, dit-il, rendirent aussi leurs morts. » La mer {\itshape les présenta}, selon la traduction littérale, parce qu’ils comparurent dans l’état où ils furent trouvés ; au lieu que la mort et l’enfer les rendirent, parce qu’ils les rappelèrent à la vie qu’ils avaient déjà quittée. Peut-être n’est-ce pas seulement {\itshape la mort}, mais encore {\itshape l’enfer} : la mort, pour marquer les justes qui l’ont seulement soufferte, sans aller en enfer ; et l’enfer, à cause des méchants qui y souffrent des supplices. S’il est au fond assez vraisemblable que les saints de l’Ancien Testament, qui ont cru à l’incarnation de Jésus-Christ, ont été, après la mort, dans des lieux, à la vérité, fort éloignés de ceux où les méchants sont tourmentés, mais néanmoins dans les enfers, jusqu’à ce qu’ils en fussent tirés par le sang du Sauveur et par la descente qu’il y fit certainement, les véritables chrétiens, après l’effusion de ce sang divin, ne vont point dans les enfers, en attendant qu’ils reprennent leur corps et qu’ils reçoivent les récompenses qu’ils méritent. Or, après avoir dit : « Et ils furent jugés chacun selon leurs œuvres », il ajoute en un mot quel fut ce jugement : « Et la mort, dit-il, et l’enfer furent jetés dans un étang de feu » ; désignant par là le diable et tous les démons, attendu que le diable est auteur de la mort et des peines de l’enfer. C’est même ce qu’il a dit avant plus clairement par anticipation : « Et le diable qui les séduisait fut jeté dans un étang de feu et de soufre. » Ce qu’il avait exprimé là plus obscurément : « Où la bête et le faux prophète, etc. », il l’éclaircit ici en ces termes : « Et ceux qui ne se trouvèrent pas écrits dans le livre de vie furent jetés dans l’étang de feu. » Ce livre n’est pas pour avertir Dieu, comme s’il pouvait se tromper par oubli ; mais il signifie la prédestination de ceux à qui la vie éternelle sera donnée. Dieu ne les lit pas dans ce livre, comme s’il ne les connaissait pas ; mais plutôt sa prescience infaillible est ce livre de vie dans lequel ils sont écrits, c’est-à-dire connus de toute éternité.
\subsection[{Chapitre XVI}]{Chapitre XVI}

\begin{argument}\noindent Du nouveau ciel et de la nouvelle terre.
\end{argument}

\noindent Après avoir parlé du jugement des méchants, saint Jean avait à nous dire aussi quelque chose de celui des bons. Il a déjà expliqué ce que Notre-Seigneur a exprimé en ce peu de mots : « Ceux-ci iront au supplice éternel » ; il lui reste à expliquer ce qui suit immédiatement : « Et les justes à la vie éternelle ». — « Et je vis, dit-il, un ciel nouveau et une terre nouvelle. Car le premier ciel et la première terre avaient disparu ; et il n’y avait plus de mer. » Cela arrivera dans l’ordre que j’ai marqué ci-dessus, à propos du passage où il dit avoir vucelui qui était assis sur le trône, devant qui le ciel et la terre s’enfuirent. Aussitôt que ceux qui ne sont pas écrits au livre de vie auront été jugés et envoyés au feu éternel, dont le lieu et la nature sont, à mon avis, inconnus à tous les hommes, à moins que Dieu ne le leur révèle, alors la figure du monde passera par l’embrasement de toutes choses, comme elle passa autrefois par le déluge. Cet embrasement détruira les qualités des éléments corruptibles qui étaient conformes au tempérament de nos corps corruptibles, pour leur en donner d’autres qui conviennent à des corps immortels, afin que le monde renouvelé soit en harmonie avec les corps des hommes qui seront renouvelés pareillement. Quant à ces paroles : « Il n’y aura plus de mer », il n’est pas aisé de décider si la mer sera desséchée par l’embrasement universel, ou bien si elle sera transformée. Nous lisons bien qu’il y aura un ciel nouveau et une terre nouvelle ; mais pour une mer nouvelle, je ne me souviens pas de l’avoir jamais lu. Il est vrai que, dans ce même livre, il est parlé d’une sorte de mer semblable à du cristal, mais il n’est pas là question de la fin du monde, et le texte ne dit pas que ce fut proprement une mer, mais une sorte de mer. Pourtant, à l’imitation des Prophètes, qui se plaisent à employer des métaphores pour voiler leur pensée, saint Jean, disant « qu’il n’y avait plus de mer », a peut-être voulu parler de cette même mer dont il avait dit auparavant que « la mer présenta les morts qui étaient dans son sein. » En effet, il n’y aura plus alors de siècle plein d’orages et de tempêtes, tel que le nôtre, qu’il a présenté sous l’image d’une mer.
\subsection[{Chapitre XVII}]{Chapitre XVII}

\begin{argument}\noindent De la glorification éternelle de l’Église, à la fin du monde.
\end{argument}

\noindent « Ensuite », dit l’Apôtre, « je vis descendre la grande cité, la nouvelle Jérusalem qui venait de Dieu, parée comme une jeune épouse, ornée pour son époux. Et j’entendis une grande voix qui sortait du trône et disait : Voici le tabernacle de Dieu avec les hommes, et il demeurera avec eux, et ils seront son peuple, et il sera leur Dieu. Il essuiera toutes les larmes de leurs yeux, et il n’yaura plus ni mort, ni deuil, ni cris, ni douleur, parce que le premier état sera fini. Et celui qui était assis sur le trône dit : Je m’en vais faire toutes choses nouvelles. » L’Écriture dit que cette Cité descendra du ciel, parce que la grâce de Dieu, qui l’a formée, en vient ; elle lui dit par la même raison dans Isaïe : « Je suis le Seigneur qui te forme. » Cette Cité, en effet, est descendue du ciel, dès qu’elle a commencé, depuis que ses concitoyens s’accroissent par la grâce du baptême, que leur a communiquée la venue du Saint-Esprit. Mais elle recevra une si grande splendeur à la venue de Jésus-Christ, qu’il ne lui restera aucune marque de vieillesse, puisque les corps mêmes passeront de la corruption et de la mortalité à un état d’incorruptibilité et d’immortalité. Il me semble qu’il y aurait trop d’impudence à soutenir que les paroles de saint Jean doivent s’entendre des mille ans que les saints régneront avec leur roi, attendu qu’il dit très clairement que « Dieu essuiera toutes les larmes de leurs yeux, et qu’il n’y aura plus ni mort, ni deuil, ni cris, ni douleur ». Et qui serait assez déraisonnable pour prétendre que, parmi les misères de cette vie mortelle, non seulement tout le peuple de Dieu, mais qu’aucun saint même soit exempt de larmes et d’ennui ? tandis qu’au contraire, plus on est saint et plein de bons désirs, plus on répand de pleurs dans la prière ! N’est-ce point la Cité sainte, la Jérusalem céleste, qui dit : « Mes larmes m’ont servi de nourriture jour et nuit » ; et encore : « Je tremperai mon lit de pleurs toute la nuit, je le baignerai de mes larmes » ; et ailleurs : « Mes gémissements ne vous sont point cachés » ; et enfin : « Ma douleur s’est renouvelée. » Ne sont-ce pas les enfants de la divine Jérusalem qui gémissent, parce qu’ils voudraient bien, non pas que leur corps fût anéanti, mais qu’il fût revêtu d’immortalité, en sorte que ce qu’il y a de mortel en eux fût absorbé par la vie ? ne sont-ce pas eux qui, possédant les prémices de l’Esprit, soupirent en eux-mêmes en attendant l’adoption divine, c’est-à-dire la rédemption de leur corps ? Et l’apôtre saint Paul n’était-il pas un citoyen de cette Jérusalem céleste, surtout quand il était saisi d’une profonde tristesse et percé jusqu’au cœur parune douleur poignante et continuelle à cause des Israélites, qui étaient ses frères selon la chair ? Quand donc la mort ne sera-t-elle plus dans cette Cité, sinon quand on dira : « Ô mort ! où est ta victoire ? ô mort ! où est ton aiguillon ? or, l’aiguillon de la mort, c’est le péché », lequel ne sera plus alors ; mais maintenant, ce n’est pas un habitant obscur de cette Cité, c’est saint Jean lui-même qui crie dans son épître : « Si nous disons que nous sommes sans péché, nous nous séduisons nous-mêmes, et la vérité n’est point en nous. » Je demeure d’accord que dans l’Apocalypse il y a beaucoup de choses obscures, propres à exercer l’esprit du lecteur, et un petit nombre de choses claires, propres à faire comprendre les autres, non sans prendre beaucoup de peine. La raison de cette obscurité, c’est surtout la coutume de l’auteur de dire les mêmes choses en tant de manières, qu’il semble qu’il veut parler de différentes choses, lorsque c’est toujours la même, diversement exprimée. Mais quant à ces paroles : « Dieu essuiera toutes les larmes de leurs yeux ; et il n’y aura plus ni mort, ni deuil, ni cris, ni douleur » ; elles regardent si évidemment le siècle à venir, l’immortalité et l’éternité des saints, qui seuls seront délivrés de ces misères, qu’il ne faut rien chercher de clair dans l’Écriture sainte, si l’on trouve ces paroles obscures.
\subsection[{Chapitre XVIII}]{Chapitre XVIII}

\begin{argument}\noindent Ce qu’annonce saint Pierre touchant le jugement dernier.
\end{argument}

\noindent Voyons maintenant ce que l’apôtre saint Pierre a écrit sur ce jugement : « Dans les derniers jours, dit-il, viendront des séducteurs pleins d’artifices, qui, marchant à la suite de leurs passions, diront : Qu’est devenue la promesse de son avènement ? car, depuis que nos pères sont morts, toutes choses se passent comme au commencement de la création. — Paroles d’insensés qui ne veulent pas savoir que les cieux furent d’abord dégagés des eaux par la parole de Dieu, aussi bien que la terre, et que le monde d’alors périt et fut submergé par les eaux. Mais les cieux et la terre qui existent à présent ont été rétablis par la même parole de Dieu, et sont destinés à être brûlés par le feu au jourdu jugement, lorsque les méchants périront. Or, apprenez, mes bien-aimés, que devant Dieu un jour est comme mille ans, et mille ans comme un jour. Ainsi le Seigneur ne diffère point l’accomplissement de sa promesse, comme quelques-uns se l’imaginent, mais il vous attend avec patience, parce qu’il veut, non pas qu’aucun périsse, mais que tous se repentent et se convertissent. Or, le jour du Seigneur viendra comme un larron, et alors les cieux passeront avec un grand fracas, les éléments seront dissous par la violence du feu, et la terre sera consumée avec tous ses ouvrages. Puisque toutes choses doivent périr, il vous convient d’attendre ce moment dans la sainteté et d’aller au-devant du jour du Seigneur, alors que les cieux embrasés seront dissous, et que les éléments périront par le feu. Mais nous attendrons, selon sa promesse, de nouveaux cieux et une nouvelle terre Où la justice régnera. » L’Apôtre ne dit rien ici de la résurrection des morts ; mais il s’étend beaucoup sur la ruine du monde, et, parce qu’il dit du déluge, il semble nous avertir de la manière dont l’univers doit périr un jour. Il dit, en effet, que le monde, qui était alors, périt, non seulement le globe de la terre, mais encore les cieux, c’est-à-dire les espaces-de l’air qui avaient été envahis par la crue des eaux. Il entend, en effet, par les cieux, ce lieu de l’air où souffle le vent, et seulement ce lieu, mais non les cieux supérieurs où sont placés le soleil, la lune et les étoiles. Ainsi toute cette région de l’air avait été changée par l’envahissement de l’eau, et elle périt ainsi, comme la terre avait péri avant elle par le déluge. « Mais, dit-il, les cieux et la terre d’à présent ont été rétablis par la même parole de Dieu, et sont réservés pour être brûlés par le feu, au jour du jugement, lorsque les méchants périront. » Ainsi le monde qui a été rétabli, c’est-à-dire ces cieux et cette terre, mis à la place du monde qui avait été détruit par le déluge, sont destinés à périr par le feu, au jour du jugement, quand les méchants périront. Il déclare, sans hésiter, que les méchants périront à cause du grand-changement qui leur arrivera, bien que leur nature doive toujours demeurer au milieu des supplices éternels. On dira peut-être : Si le monde est embrasé après le jugement, oùseront les saints lors de cet embrasement suprême, avant que Dieu ait remplacé le monde détruit par un ciel nouveau et une terre nouvelle ? car, puisqu’ils auront des corps, il faut bien qu’ils soient quelque part. Nous pouvons répondre qu’ils seront dans les hautes régions où le feu de l’embrasement n’atteindra pas, non plus qu’autrefois l’eau du déluge ; leurs corps seront tels alors qu’ils pourront demeurer où il leur conviendra. Ils ne craindront pas même le feu de cet embrasement, étant immortels et incorruptibles ; de même que les corps mortels et corruptibles des trois jeunes hommes purent vivre dans la fournaise ardente, sans être atteints par le feu.
\subsection[{Chapitre XIX}]{Chapitre XIX}

\begin{argument}\noindent De l’épître de saint Paul aux habitants de Thessalonique sur l’apparition de l’Antéchrist, après lequel viendra le jour du Seigneur.
\end{argument}

\noindent Je me vois dans la nécessité de négliger un grand nombre de témoignages des évangélistes et des Apôtres sur ce dernier jugement, craignant de donner trop d’étendue à ce livre. Mais je ne puis passer sous silence ce que dit saint Paul dans une épître écrite aux habitants de Thessalonique : « Nous vous prions, mes frères, par l’avènement de Notre-Seigneur Jésus-Christ et au nom de notre union en lui, de ne pas vous laisser ébranler légèrement, sur la foi de quelques fausses prophéties ou sur quelque discours et sur quelque lettre qu’on supposerait venir de nous, pour vous faire croire que le jour du Seigneur est proche ; Que personne ne vous trompe. Il faut auparavant que l’apostat vienne, et que l’homme de péché se manifeste, ce fils de perdition, qui s’opposera à Dieu, et qui s’élèvera au-dessus de tout ce qu’on appelle Dieu et qu’on adore, jusqu’à s’asseoir dans le temple de Dieu, voulant passer lui-même pour Dieu. Ne vous souvient-il pas que je vous disais tout cela, quand j’étais encore avec vous ? Vous savez bien aussi ce qui empêche qu’il ne vienne, afin qu’il paraisse en son temps. Car le mystère d’iniquité commence à se former. Seulement que celui qui tient maintenant tienne jusqu’à ce qu’il sorte ; et alors se « révélera ce méchant que le Seigneur tueradu souffle de sa bouche, et qu’il dissipera par l’éclat de sa présence ce méchant, dis-je, qui doit venir avec la puissance de Satan et faire une infinité de prodiges et de faux miracles qui séduiront ceux qui doivent périr pour n’avoir point aimé la vérité qui les eût sauvés. C’est pourquoi Dieu leur en verra un esprit d’erreur qui les fera croire au mensonge, afin que tous ceux qui n’ont point cru à la vérité, mais qui ont consenti à l’iniquité, soient condamnés. »\par
Il est hors de doute que saint Paul a dit ceci de l’Antéchrist et du jour du jugement, qu’il appelle le jour du Seigneur, pour expliquer que le Seigneur ne viendra point avant que celui qu’il appelle l’apostat ne soit venu. Que si l’on peut appeler avec raison tous les impies des apostats, à plus forte raison peut-on nommer ainsi l’Antéchrist. Mais quel est le temple de Dieu où il doit s’asseoir ? On ne peut décider si c’est dans les ruines du temple de Salomon ou dans l’Église. S’il s’agissait du temple d’une idole ou du démon, assurément l’Apôtre ne l’appellerait pas le temple de Dieu. Aussi a-t-on voulu que ce passage, qui a rapport à l’Antéchrist, s’entendît non seulement du prince des impies, mais cri quelque sorte de tout ce qui fait corps avec lui, c’est-à-dire de la multitude des hommes qui lui appartiennent ; et l’on a cru qu’il valait mieux suivre le texte grec et dire, non « dans le temple de Dieu », mais « en temple de Dieu », comme si l’Antéchrist était lui-même le temple de Dieu, qui n’est autre chose que l’Église. C’est ainsi que nous disons il « s’assied en ami », c’est-à-dire comme ami, et autres locutions du même genre. Quant à ces paroles : « Vous savez aussi ce qui empêche qu’il ne vienne maintenant », c’est-à-dire vous connaissez la cause du retard de sa venue, « c’est afin qu’il paraisse en son temps ». Comme il dit Vous le savez, il ne s’en est pas expliqué plus clairement ; mais nous qui l’ignorons, nous avons bien de la peine à comprendre ce qu’il veut dire, d’autant mieux que ce qu’il ajoute rend plus obscur encore le sens de ce passage. En effet, que signifient ces paroles : « Le mystère d’iniquité commence déjà à se former ; seulement que celui qui tient maintenant tienne jusqu’à ce qu’il sorte ; et alors le méchant se manifestera » ? J’avoue franchement ne pas comprendre ceque cela veut dire ; mais je ne passerai pas sous silence les conjectures de ceux que j’ai pu lire ou entendre.\par
Il en est qui pensent que saint Paul parle ici de l’empire romain, et que c’est la raison pour laquelle il a affecté d’être obscur, de crainte qu’on ne l’accusât de faire des imprécations contre un empire qu’on regardait comme éternel ; de sorte que par ces paroles « Le mystère d’iniquité commence à se former », il aurait eu en vue Néron, dont on regardait les œuvres comme celles de l’Antéchrist. D’autres pensent même que Néron n’a pas été tué, mais seulement enlevé, pour qu’on le crût mort, et qu’il est caché quelque part, vivant et dans la vigueur de l’âge qu’il avait quand on le crut mort, pour reparaître en son temps et être rétabli dans son royaume. Mais cette opinion me semble tout au moins fort singulière. Toutefois, ces paroles de l’Apôtre : « Seulement que celui qui tient maintenant tienne jusqu’à ce qu’il sorte ci, peuvent sans absurdité s’entendre de l’empire romain, comme s’il y avait : « Seulement que celui qui commande, commande jusqu’à ce qu’il sorte », c’est-à-dire jusqu’à ce qu’il soit retranché. — « Et alors le méchant se découvrira », c’est-à-dire l’Antéchrist, comme tout le monde en tombe d’accord.\par
Mais d’autres pensent que ces paroles : « Vous savez ce qui empêche qu’il ne vienne ; car le mystère d’iniquité commence déjà à se former », ne doivent s’appliquer qu’aux méchants et aux hypocrites qui sont dans l’Église, jusqu’à ce qu’ils soient en assez grand nombre pour fournir un grand peuple à l’Antéchrist, et que c’est ce qu’il appelle le « mystère d’iniquité ci, parce que c’est une chose cachée. Les paroles de l’Apôtre seraient donc une exhortation aux fidèles de demeurer fermes dans leur foi, quand il dit : « Seulement que celui qui tient maintenant tienne jusqu’à ce qu’il sorte », c’est-à-dire jusqu’à ce que le mystère d’iniquité sorte de l’Église, où il est maintenant caché. Ceux-là estiment que ce mystère d’iniquité est celui dont parle ainsi saint Jean dans son épître : « Mes enfants, voici la dernière heure ; car, commevous avez ouï dire que l’Antéchrist doit venir et qu’il y a déjà maintenant plusieurs Antéchrists, cela nous fait connaître que nous sommes arrivés maintenant à la dernière heure. Ils sont sortis d’avec nous, mais ils n’étaient pas des nôtres ; car s’ils eussent été des nôtres, ils seraient demeurés. » De même, disent-ils, que plusieurs hérétiques, que saint Jean appelle des Antéchrists, sont déjà sortis de l’Église, à cette heure, qu’il dit être la dernière, ainsi tous ceux qui n’appartiendront pas à Jésus-Christ, mais à l’Antéchrist, en sortiront alors, et c’est alors qu’il se manifestera.\par
C’est ainsi qu’on explique, ceux-ci d’une manière, ceux-là d’une autre, ces obscures paroles de saint Paul ; mais du moins on ne doute point qu’il n’ait dit que Jésus-Christ ne viendra pas juger les vivants et les morts avant que l’Antéchrist ne soit venu séduire ceux qui seront déjà morts dans l’âme, encore que cette séduction même appartienne au mystère des jugements de Dieu. « L’Antéchrist », comme dit l’Apôtre, « viendra avec la puissance de Satan, et fera une infinité de prodiges et de faux miracles pour séduire ceux qui doivent périr. » Alors en effet Satan sera délié et il agira de tout son pouvoir par d’Antéchrist, en faisant plusieurs miracles trompeurs. On a coutume de demander si l’Apôtre les appelle de faux miracles, parce que ce ne seront que des illusions et des prestiges, ou bien parce qu’ils entraîneront dans l’erreur ceux qui croiront ces prodiges au-dessus de la puissance du diable, faute de connaître ce qu’il peut et surtout ce qu’il pourra, alors qu’il recevra un pouvoir plus grand qu’il ne l’a jamais eu. En effet, lorsque le feu tomba du ciel et consuma la nombreuse famille de Job avec tant de troupeaux, et qu’un tourbillon de vent abattit la maison où étaient ses enfants et les écrasa sous ses ruines, ce n’étaient pas des illusions, et cependant c’étaient des œuvres de Satan, à qui Dieu avait donné ce pouvoir. Quoi qu’il en soit (car nous saurons mieux un jour pourquoi l’Apôtre les appelle de faux miracles), il est certain qu’ils séduiront ceux qui auront mérité d’être séduits, pour n’avoir pas aimé la vérité qui les eût sauvés. L’Apôtre ne dissimule pas que « Dieu leur enverra une erreur si forte et si spécieuse qu’ils auront foi dans le mensonge ! » Il la leur enverra, parce qu’il permettra au diable de faire ces prodiges, et il le lui permettra par un jugement très juste, bien que le dessein du diable en cela soit injuste et criminel : « Afin », ajoute-t-il, « que tous ceux qui n’ont point cru à la vérité, mais qui ont consenti à l’iniquité, soient condamnés. » Ainsi ils seront séduits par ces jugements de Dieu, également justes et cachés, qu’il n’a jamais cessé d’exercer sur les hommes depuis le péché du premier homme. Après avoir été séduits, ils seront condamnés dans le dernier et public jugement par Jésus-Christ, qui, condamné injustement par les hommes, les condamnera justement.
\subsection[{Chapitre XX}]{Chapitre XX}

\begin{argument}\noindent Ce que saint Paul a enseigné sur la résurrection des morts dans sa première épître aux habitants de Thessalonique.
\end{argument}

\noindent L’Apôtre ne parle pas ici de la résurrection des morts ; mais dans sa première épître aux mêmes habitants de Thessalonique, il dit : « Je ne veux pas, mes frères, que vous ignoriez ce qui regarde ceux qui dorment, de peur que vous ne vous affligiez comme font les autres hommes qui n’ont point d’espérance. Car si nous croyons que Jésus-Christ est mort et ressuscité, nous devons croire aussi que Dieu amènera avec Jésus ceux qui sont morts avec lui. Je vous déclare donc, selon la parole du Seigneur, que nous qui vivons et qui sommes réservés pour l’avènement du Seigneur, nous ne préviendrons point ceux qui sont déjà dans le sommeil de la mort ; mais à la voix de l’archange et au son de la trompette de Dieu, le Seigneur lui-même descendra du ciel ; et ceux qui seront morts en Jésus-Christ ressusciteront les premiers. Ensuite, nous qui sommes vivants et qui serons demeurés jusqu’alors, nous serons emportés avec eux dans les nues et au milieu des airs devant le Seigneur ; et ainsi nous serons pour jamais avec le Seigneur. » Ces paroles de l’Apôtre marquent clairement la résurrection future, lorsque Notre-Seigneur Jésus-Christ viendra juger les vivants et les morts.\par
Mais on a coutume de demander si ceux que le Seigneur trouvera vivants, et que saint Paul figure ici par lui-même et par ceux quivivaient alors, ne mourront point ; ou bien si, dans le moment où ils seront emportés dans l’air devant le Seigneur, ils passeront par la mort à l’immortalité. On aurait tort de croire que, pendant qu’ils seront portés dans l’air, ils ne pourront mourir et ressusciter. Aussi ne faut-il pas entendre ces paroles : « Et ainsi nous serons pour jamais avec le Seigneur », comme si saint Paul voulait dire par là que nous demeurerons toujours avec lui dans l’air, puisqu’il n’y demeurera pas lui-même, et qu’il y viendra seulement en passant ; mais nous serons pour jamais avec le Seigneur, en ce que nous aurons toujours des corps mortels, dans quelque lieu que nous soyons avec lui. Or, c’est l’Apôtre lui-même qui nous oblige en quelque sorte à croire que ceux que Notre-Seigneur trouvera vivants souffriront la mort et recevront l’immortalité incontinent, puisqu’il dit : « Tous vivront en Jésus-Christ » ; et encore « Ce qu’on sème dans la terre ne renaît pas, s’il ne meurt auparavant ». Comment donc ceux que Jésus-Christ trouvera vivants revivront-ils en lui par l’immortalité, s’ils ne meurent pas ? Il est vrai que si l’on ne peut pas dire proprement du corps d’un homme qu’il est semé, à moins qu’il ne retourne à la terre, selon la sentence portée par Dieu contre le premier pécheur : « Tu es terre, et tu retourneras à la terre » ; il faut avouer que ceux que Notre-Seigneur trouvera en vie, à son avènement, ne sont pas compris dans ces paroles de l’Apôtre, ni dans celles de la Genèse. Il est clair qu’étant enlevés dans les nues, ils ne seront pas semés en terre et n’y retourneront pas, soit qu’ils ne doivent pas mourir, soit qu’ils meurent momentanément dans l’air.\par
Mais, d’un autre côté, le même Apôtre, écrivant aux Corinthiens, dit : « Nous ressusciterons tous » ; ou, suivant d’autres leçons : « Nous dormirons tous. » Si donc on ne peut ressusciter sans avoir passé par la mort, comment tous ressusciteront-ils ou dormiront-ils, si tant d’hommes que Jésus-Christ trouvera vivants ne doivent ni dormir ni ressusciter ? J’estime donc qu’il faut nous en tenir à ce quenous venons de dire, que ceux que Jésus-Christ trouvera en vie, et qui seront emportés dans l’air, mourront en ce moment, pour reprendre aussitôt après leurs corps mortels. Pourquoi ne croirions-nous pas que cette multitude de corps puisse être semée en quelque sorte dans l’air et y reprendre à l’heure même une vie immortelle et incorruptible, lorsque nous croyons ce que nous dit le même Apôtre, que la résurrection se fera en un clin d’œil, et que la poussière des corps, répandue en cent lieux, sera rassemblée avec tant de facilité et de promptitude ? Quant à cette parole de la Genèse : « Tu es terre, et tu retourneras à la terre » ; il ne faut pas s’imaginer qu’elle ne s’accomplisse pas dans les saints qui mourront dans l’air, sous prétexte que leurs corps ne retomberont pas sur la terre, attendu que ces mots : « Tu retourneras à la terre », signifient : Tu iras, après avoir perdu la vie, là où tu étais avant de la recevoir ; c’est-à-dire, tu seras, quand tu auras perdu ton âme, comme tu étais avant d’en avoir une. L’homme n’était que terre, en effet, quand Dieu souffla sur sa face pour lui donner la vie. C’est donc comme s’il lui disait : Tu es une terre animée, ce que tu n’étais pas ; tu seras une terre sans âme, comme tu étais. Ce que sont tous les corps morts avant qu’ils ne pourrissent, ceux-là le seront s’ils meurent, quelque part qu’ils meurent. Ils retourneront donc à la terre, puisque d’hommes vivants. Ils redeviendront terre ; de même que ce qui devient cendre retourne en cendre, que ce qui devient vieux va à la vieillesse, que la boue qui durcit revient à l’état de pierre. Mais toutes nos réflexions à ce sujet ne sont que des conjectures ; et nous ne comprendrons bien qu’au jour suprême ce qui en est réellement. Si nous voulons être chrétiens, nous devons croire à la résurrection des corps, quand Jésus-Christ viendra juger les vivants et les morts. Et ici notre foi n’est pas vaine, bien que nous ne comprenions pas parfaitement ce qu’il en sera, pourvu que nous y croyions. Il nous reste à examiner, comme nous l’avons promis, ce que les livres prophétiques de l’Ancien Testament disent de ce dernier jugement de Dieu ; mais nous n’aurons pas besoin, pour être compris, de nous étendre beaucoup, si le lecteur veut bien se rappeler ce que nous venons de dire.
\subsection[{Chapitre XXI}]{Chapitre XXI}

\begin{argument}\noindent Preuves de la résurrection des morts et du jugement dernier, tirées du prophète Isaïe.
\end{argument}

\noindent Le prophète Isaïe a dit : « Les morts ressusciteront, et ceux qui sont dans les tombeaux en sortiront, et tous ceux qui sont sur la terre se réjouiront ; car la rosée qui vient de vous est leur santé ; mais la terre des impies tombera. » Tout le commencement du verset regarde la résurrection des bienheureux ; mais quand il dit : « La terre des impies tombera », il faut l’entendre des méchants qui tomberont dans la damnation. Pour ce qui regarde la résurrection des bons, si nous y voulons prendre garde, nous trouverons qu’il faut rapporter à la première ces paroles : « Les morts ressusciteront » ; et à la seconde celles-ci, qui viennent après : « Ceux qui sont dans les tombeaux ressusciteront aussi. » Ces mots : « Et tous ceux qui sont sur la terre se réjouiront ; car la rosée qui vient de vous est leur santé », s’appliquent aux saints que Jésus-Christ trouvera vivants à son avènement. Par la santé, nous ne pouvons entendre raisonnablement que l’immortalité ; car on peut dire qu’il n’y a point de santé plus parfaite que celle qui n’a pas besoin, pour se maintenir, de prendre tous les jours le remède des aliments. Le même Prophète parle encore ainsi du jour du jugement, après avoir donné de l’espérance aux bons et de la frayeur aux méchants : « Voici ce que dit le Seigneur : Je me détournerai sur eux comme un fleuve de paix et comme un torrent qui inondera la gloire des nations. Leurs enfants seront portés sur les épaules et caressés sur les genoux. Je vous caresserai comme une mère caresse son enfant, et ce sera dans Jérusalem que, vous recevrez cette consolation. Vous verrez, et votre cœur se réjouira, et vos os germeront comme l’herbe. On reconnaîtra la main du Seigneur qui va venir comme un feu ; et ses chariots seront comme la tempête, pour exercer sa vengeance dans sa colère et livrer tout en proie aux flammes. Car toute la terre sera jugée par le feu du Seigneur, et toute chair par son glaive. Plusieurs seront blessés par le Seigneur. » Le Prophète dit que le Seigneur se détournera sur les bons comme un fleuve de paix ; ce qui sansdoute leur promet une abondance de paix la plus grande qui puisse être. C’est cette paix dont nous jouirons à la fin et dont nous avons amplement parlé au livre précédent. Voilà le fleuve que le Seigneur détournera sur les bons, à qui il promet une si grande félicité, pour nous faire entendre que dans cette heureuse région, qui est le ciel, tous les désirs seront comblés par lui, Comme cette paix sera une source d’incorruptibilité et d’immortalité qui se répandra sur les corps mortels, il dit qu’il se détournera comme un fleuve sur eux, afin de se répandre d’en haut sur les choses les plus humbles et d’égaler les hommes aux anges. Et par la Jérusalem dont le Prophète parle, il ne faut point entendre celle qui est esclave, ainsi que ses enfants, mais au contraire, avec l’Apôtre, celle qui est libre et noire mère, et qui est éternelle dans les cieux, où nous serons consolés après les ennuis et les travaux de cette vie mortelle, et portés sur ses épaules et sur ses genoux comme de petits enfants. Nous serons, en quelque sorte, tout renouvelés pour une si grande félicité et pour les ineffables douceurs que nous goûterons dans son sein. Là nous verrons, et notre cœur se réjouira. Il ne dit point ce que nous verrons ; mais que sera-ce, sinon Dieu ? Alors s’accomplira en nous la promesse de l’Évangile : « Bienheureux ceux qui ont le cœur pur, parce qu’ils verront Dieu. » Que sera-ce, sinon toutes ces choses que nous ne voyons point maintenant, mais que nous croyons, et dont l’idée que nous nous formons, selon la faible portée de notre esprit, est infiniment au-dessous de ce qu’elles sont réellement : « Vous verrez, dit-il, et votre cœur se réjouira. » Ici vous croyez, là vous verrez.\par
Quand il a dit : « Et votre cœur se réjouira », craignant que nous ne pensions que ces biens de la Jérusalem céleste ne regardent que l’esprit, il ajoute « Et vos os germeront comme l’herbe », où il nous rappelle la résurrection des corps, comme s’il reprenait ce qu’il avait omis de dire. Cette résurrection ne se fera pas, en effet, lorsque nous aurons vu ; mais au contraire, c’est quand elle sera accomplie que nous verrons. En effet, le Prophète avait déjà parlé auparavant d’un ciel nouveau et d’une terre nouvelle, aussi bien que des promesses faites aux saints : « Il y aura un ciel nouveau et une terre nouvelle ; et ils ne trouveront que des sujets de joie dans cet heureux séjour. Je ferai que Jérusalem ne soit plus qu’une fête éternelle, et mon peuple la joie même. Et Jérusalem fera tout mon plaisir, et mon peuple toutes mes délices. On n’y entendra plus de pleurs ni de gémissements. » Puis vient le reste, que certains veulent faire rapporter au règne charnel des mille ans. Le Prophète mêle ici les expressions figurées avec les autres, afin que notre esprit s’exerce salutairement à y chercher un sens spirituel ; mais la paresse et l’ignorance s’arrêtent à la lettre, et ne vont pas plus loin. Pour revenir aux paroles du Prophète que nous avions commencé à expliquer, après avoir dit : « Et vos os germeront comme l’herbe », pour montrer qu’il ne parle que de la résurrection des bons, il ajoute : « Et l’on reconnaîtra la main du Seigneur envers ceux qui le servent. » Quelle est cette main, sinon celle qui distingue les hommes qui servent Dieu de ceux qui le méprisent ? Il parle ensuite de ces derniers dans les termes suivants : « Et il exécutera ses menaces contre les rebelles. Car voilà le Seigneur qui va venir comme un feu, et ses chariots seront comme la tempête, pour exercer sa vengeance dans sa colère, et donner tout en proie aux flammes. Car toute la terre sera jugée par le feu du Seigneur, et toute chair par son glaive, et plusieurs seront blessés par le Seigneur. » Par ces mots de {\itshape feu}, de {\itshape tempête}, et de {\itshape glaive}, il entend le supplice de l’enfer. Les chariots désignent le ministère des anges. Lorsqu’il dit que toute la terre et toute chair seront jugées par le feu du Seigneur et par son glaive, il faut excepter les saints et les spirituels, et n’y comprendre que les hommes terrestres et charnels, dont il est dit qu’ils ne goûtent que les choses de la terre, et que la sagesse selon la chair, c’est la mort et enfin ceux que Dieu appelle chair, quand il dit : « Mon esprit ne demeurera plus parmi ceux-ci, parce qu’ils ne sont que chair. » Quand il dit que « plusieurs seront blessés par le Seigneur », ces blessures doivent s’entendre de la seconde mort. Il est vrai qu’on peut prendre aussi en bonne part le feu, le glaive et les blessures. Notre-Seigneur dit lui-même qu’il est venu pour apporter le feu sur la terre.\par
Les disciples virent comme des langues de feu qui se divisèrent quand le Saint-Esprit descendit sur eux. Notre-Seigneur dit encore qu’il n’est pas venu sur la terre pour apporter la paix, mais le glaive. L’Écriture appelle la parole de Dieu un glaive à deux tranchants, à cause des deux Testaments et dans le Cantique des cantiques, l’Église s’écrie qu’elle est blessée d’amour comme d’un trait. Mais ici, où il est clair que Dieu vient pour exécuter ses vengeances, on voit de quelle façon toutes ces expressions doivent s’expliquer.\par
Après avoir brièvement indiqué ceux qui seront consumés par ce jugement, le Prophète, figurant les pécheurs et les impies sous l’image des viandes défendues par l’ancienne loi, dont ils ne se sont pas abstenus, revient à la grâce du Nouveau Testament, depuis le premier avènement du Sauveur jusqu’au jugement dernier, par lequel il termine sa prophétie. Il raconte que le Seigneur déclare qu’il viendra pour rassembler toutes les nations, et qu’elles seront témoins de sa gloire ; car, dit l’Apôtre : « Tous ont péché et tous ont besoin de la gloire de Dieu. » Isaïe ajoute qu’il fera devant eux tant de miracles qu’ils croiront en lui, qu’il enverra certains d’entre eux en différents pays et dans les îles les plus éloignées, où l’on n’a jamais ouï parler de lui, ni vu sa gloire, qu’ils amèneront à la foi les frères de ceux à qui le Prophète a parlé, c’est-à-dire les Israélites élus, en annonçant l’Évangile parmi toutes les nations, qu’ils amèneront un présent à Dieu, de toutes les contrées du monde, sur des chevaux et sur des chariots (qui sont les secours du ciel et qui se transmettent par le ministère des anges et des hommes), enfin qu’ils l’amèneront dans la sainte Cité de Jérusalem, qui maintenant est répandue par toute la terre dans la sainteté des fidèles. En effet, où ils se sentent aidés par un secours divin, les hommes croient, et où ils croient, ils viennent. Or, le Seigneur les compare aux enfants d’Israël qui lui offrent des victimes dans son temple, avec des cantiques de louange, comme l’Église le pratique déjà partout. De nos jours, ne choisit-on pas les prêtres et les lévites, non en regardant la race et le sang, comme cela se pratiquait d’abord dans le sacerdoce selon l’ordre d’Aaron, muais comme il convient à l’esprit du Nouveau Testament, où Jésus-Christ est le souverain prêtre selon l’ordre de Melchisédech, en considérant le mérite que la grâce divine donne à chacun ? ne choisit-on pas, dis-je, des prêtres et des lévites qu’il ne faut pas juger par la fonction dont ils sont souvent indignes, mais par la sainteté, qui ne peut être commune aux bons et aux méchants ?\par
Après avoir ainsi parlé de cette miséricorde de Dieu pour son Église, dont les effets nous sont si sensibles et si connus, Isaïe promet, de la part de Dieu, les fins où chacun arrivera lorsque le dernier jugement aura séparé les bons d’avec les méchants : « Car, de même que le nouveau ciel et la nouvelle terre demeureront en ma présence, dit le Seigneur, ainsi votre semence et votre nom demeureront devant moi ; et ils passeront de mois en mois et de sabbat en sabbat, et toute chair viendra m’adorer en Jérusalem ; et ils sortiront, et ils verront les membres des hommes prévaricateurs. Leur ver ne mourra point, et le feu qui les brûlera ne s’éteindra point ; et ils serviront de spectacle à toute chair. » C’est par là que le prophète Isaïe finit son livre, comme par là aussi le monde doit finir. Quelques versions, au lieu des « membres des hommes », portent les « cadavres des hommes », entendant évidemment par là la peine des corps damnés, quoique d’ordinaire on n’appelle cadavre qu’une chair sans âme, au lieu que les corps dont il parle seront animés, sans quoi ils ne pourraient souffrir aucun tourment. Cependant il est possible qu’on ait voulu entendre par ces mots des corps semblables à ceux des hommes qui passeront à la seconde mort, d’où vient cette parole du Prophète : « La terre des impies tombera. » Qui ne sait, en effet, que {\itshape cadavre} vient d’un mot latin qui signifie {\itshape tomber} ? De même il est assez clair que par le mot hommes le Prophète veut parler de toutes les créatures humaines en général ; car personne n’oserait soutenir que les femmes pécheresses ne subiront pas aussi leur supplice. Il faut le croire d’autant mieux que c’est de la femme elle-même que l’homme est sorti. Mais voici ce qui importe particulièrement à notre sujet, puisque le Prophète, en parlant des bons, dit : « Toute chair viendra », parce que le peuple chrétien sera composé de toutes les nations, et qu’en parlant des méchants, il les appelle {\itshape membres} ou {\itshape cadavres}, cela montre que le jugement qui enverra à leur fin les bons et les méchants aura lieu après la résurrection de la chair, dont il parle si clairement.
\subsection[{Chapitre XXII}]{Chapitre XXII}

\begin{argument}\noindent Comment il faut entendre que les bons sortiront pour voir le supplice des méchants.
\end{argument}

\noindent Mais comment les bons sortiront-ils pour voir le supplice des méchants ? Dirons-nous qu’ils quitteront réellement les bienheureuses demeures, pour passer aux lieux des supplices et être témoins des tourments des damnés ? À Dieu ne plaise ! c’est en esprit, c’est par la connaissance qu’ils sortiront. Ce mot {\itshape sortir} fait entendre que ceux qui seront tourmentés seront dehors : car Notre-Seigneur appelle aussi {\itshape ténèbres extérieures} ces lieux opposés à l’entrée qu’il annonce au bon serviteur, quand il lui dit : « Entre dans la joie de ton Seigneur » ; et loin que les méchants y entrent pour y être connus, ce sont plutôt les saints qui sortent en quelque façon vers eux par la connaissance qu’ils ont de leur malheur. Ceux qui seront dans les tourments ne sauront pas ce qui se passera au dedans, « dans la joie du Seigneur » ; mais ceux qui posséderont cette joie sauront tout ce qui se passera au dehors, dans « les ténèbres extérieures ». C’est pour cela qu’il est dit qu’ils sortiront, parce qu’ils connaîtront ce qui se fera à l’égard de ceux mêmes qui seront dehors. Si, en effet, les Prophètes ont pu connaître ces choses, quand elles n’étaient pas encore arrivées, par le peu que Dieu en révélait à des hommes mortels, comment les saints immortels les ignoreraient-ils, alors qu’elles seront accomplies et que Dieu sera tout en tous ? La semence et le nom des saints demeureront donc stables dans la plénitude de Dieu, j’entends cette semence dont saint Jean dit : « Et la semence de Dieu demeure en lui » ; et ce nom dont parle Isaïe : « Je leur donnerai un nom éternel, et ils passeront de mois en mois et de sabbat en sabbat », comme de lune en lune, et de repos en repos. Car les saints seront tout cela, alors que, de ces ombres anciennes et passagères, ils entreront dans les clartés nouvelles et éternelles. Quantà ce feu inextinguible et à ce ver immortel qui feront le supplice des réprouvés, on les explique diversement. Les uns rapportent l’un et l’autre au corps, et les autres à l’âme. D’autres disent que le feu tourmentera le corps, et le ver l’âme, et qu’ainsi il faut prendre le premier au propre et le second au figuré, ce qui ne paraît pas vraisemblable. Mais ce n’est pas ici le lieu de parler de cette différence, puisque nous avons destiné ce livre au dernier jugement qui fera la séparation des bons et des méchants. Nous parlerons en particulier de leurs peines et de leurs récompenses.
\subsection[{Chapitre XXIII}]{Chapitre XXIII}

\begin{argument}\noindent Prophétie de Daniel sur la persécution de l’Antéchrist, sur le jugement dernier et sur le règne des saints.
\end{argument}

\noindent Daniel prédit aussi ce dernier jugement, après l’avoir fait précéder de l’avènement de l’Antéchrist, et il conduit sa prophétie jusqu’au règne des saints. Ayant vu dans une extase prophétique quatre bêtes, qui figuraient quatre royaumes, dont le quatrième est conquis par un roi, qui est l’Antéchrist, et après cela, le royaume du Fils de l’homme, qui est celui de Jésus-Christ, il s’écrie : « Mon esprit fut saisi d’horreur ; moi, Daniel, je demeurai tout épouvanté, et les visions de ma tête me troublèrent. Je m’approchai donc de l’un de ceux qui étaient présents, et je lui demandai la vérité sur tout ce que je voyais, et il me l’apprit. Ces quatre bêtes immenses, me dit-il, sont quatre royaumes qui s’établiront sur la terre et qui ensuite seront détruits. Les saints du Très-Haut prendront leur place et régneront jusque dans le siècle et jusque dans le siècle des siècles. » — « Après cela, poursuit Daniel, je m’enquis avec soin quelle était la quatrième bête, si différente des autres, et beaucoup plus terrible, car ses dents étaient de fer, et ses ongles d’airain ; elle mangeait et dévorait tout, et foulait tout aux pieds. Je m’informai aussi des dix cornes qu’elle avait à la tête, et d’une autre qui en sortit et qui fit tomber les trois premières. Et cette corne avait des yeux, et une bouche qui disait de terribles choses ; et elle était plus grande que les autres. Je m’aperçus que cette corne faisait la guerre aux saints, et était plus forte qu’eux, jusqu’à ce que l’Ancien des jours vînt et donnât le royaume aux saints du Très-Haut. Ainsi, le temps étant venu, les saints furent mis en possession du royaume. Alors celui à qui je parlais me dit : La quatrième bête sera un quatrième royaume qui s’élèvera sur la terre et détruira tous les autres ; il dévorera toute la terre et la ravagera et la foulera aux pieds. Ces dix cornes sont dix rois, après lesquels il en viendra un plus méchant que tous les autres, qui en humiliera trois, vomira des blasphèmes contre le Très-Haut, et fera souffrir mille maux à ses saints. Il entre, prendra même de changer les temps et d’abolir la loi ; et on le laissera régner un temps, des temps, et la moitié d’un temps. Après viendra le jugement, qui lui ôtera l’empire « et l’exterminera pour jamais ; et toute la puissance, la grandeur, et la domination souveraine des rois sera donnée aux saints du Très-Haut. Son royaume sera éternel, et toutes ces puissances le serviront et lui obéiront. Voilà ce qu’il me dit. Cependant, j’étais extrêmement troublé, et mon visage en fut tout changé ; mais je ne laissai pas que de bien retenir ce qu’il m’avait dit. » Quelques-uns ont entendu par ces quatre royaumes ceux des Assyriens, des Perses, des Macédoniens et des Romains ; et si l’on veut en avoir la raison, on n’a qu’à lire les commentaires du prêtre Jérôme sur Daniel, qui sont écrits avec tout le soin et toute l’érudition désirables ; mais au moins ne peut-on douter que Daniel ne dise ici très clairement que la tyrannie de l’Antéchrist contre les fidèles, quoique courte, précédera le dernier jugement et le règne éternel des saints, Là suite du passage fait voir que le temps, les temps, et la moitié d’un temps signifient un an, deux ans, et la moitié d’un an, c’est-à-dire trois ans et demi. Il est vrai que les temps semblent marquer un temps indéfini ; mais l’hébreu ne désigne que deux temps, car on dit que les. Hébreux ont, aussi bien que les Grecs, le nombre duel, que les Latins n’ont pas. Pour les dix rois, je ne sais s’ils signifient dix rois qui existeront réellement dans J’empire romain, quand l’Antéchrist viendra, et j’ai peur que ce nombre ne nous trompe. Que savons-nous s’il n’est pas mis là pour signifier l’universalité de tous les rois qui doivent précéder son avènement, comme l’Écriture se sert assezsouvent du nombre de mille, de cent ou de sept, et de tant d’autres qu’il est inutile de rapporter, pour marquer l’universalité ?\par
Le même Daniel s’exprime ainsi dans un autre passage : « Le temps viendra où il s’élèvera une persécution si cruelle qu’il n’y en aura jamais eu de semblable sur la terre. En ce temps-là, tous ceux qui se trouveront écrits sur le livre seront sauvés, et plusieurs de ceux qui dorment sous un amas de terre ressusciteront, les uns pour la vie éternelle, les autres pour une confusion et un opprobre éternels. Or, les sages auront un éclat pareil à celui du firmament, et ceux qui enseignent la justice brilleront à jamais comme les étoiles. » Ce passage de Daniel est assez conforme à un autre de l’Évangile où il est aussi parlé de la résurrection du corps. Ceux que l’Évangéliste dit être « dans les sépulcres », Daniel dit qu’ils sont sous un « amas de terre », ou, comme d’autres traduisent « dans la poussière de la terre ». De même qu’il est dit là qu’ils « sortiront », ici il est dit qu’ils « ressusciteront ». Dans l’Évangile : « Ceux qui auront bien vécu sortiront de leur tombeau pour ressusciter à la vie, et ceux qui auront mal vécu pour ressusciter à la damnation. » Dans le Prophète ; « Les uns ressusciteront pour la vie éternelle, les autres pour une confusion et un opprobre éternels. » Que l’on ne s’imagine pas que l’Évangéliste et le Prophète diffèrent l’un de l’autre, sous prétexte que celui-là dit : « Tous ceux qui sont dans les sépulcres » ; et celui-ci : « Plusieurs de ceux qui sont sous un amas de terre » ; car quelquefois l’Écriture dit « plusieurs » pour « tous ». C’est ainsi qu’il est dit à Abraham « Je vous établirai père de plusieurs nations », bien qu’il lui soit dit ailleurs : « Toutes les nations seront bénies en votre semence. » Et il est dit encore un peu après à Daniel, au sujet de la même résurrection : « Et vous, venez, et reposez ; car il reste encore du temps jusqu’à la consommation des siècles ; et vous vous reposerez, et vous ressusciterez pour posséder votre héritage, à la fin les temps. »
\subsection[{Chapitre XXIV}]{Chapitre XXIV}

\begin{argument}\noindent Prophéties tirées des psaumes de David sur la fin du monde et sur le dernier jugement de dieu.
\end{argument}

\noindent Il y a dans les psaumes beaucoup de passages qui regardent le jugement dernier, mais on n’y en parle que d’une manière concise et rapide. Il ne faut pas toutefois que je passe sous silence ce qui y est dit en termes très clairs sur la fin du monde : « Seigneur », dit le Psalmiste, « vous avez créé la terre au commencement, et les cieux sont l’ouvrage de vos mains. Ils périront, mais pour vous, vous resterez. Ils vieilliront tous comme un vêtement. Vous les changerez de forme comme un manteau, et ils seront transformés. Mais vous, vous êtes toujours le même, et vos années ne finiront point. » D’où vient donc que Porphyre, qui loue la piété des Hébreux et les félicite d’adorer le grand et vrai Dieu, terrible aux dieux mêmes, accuse les chrétiens d’une extrême folie, sur la foi des oracles de ses dieux, parce qu’ils disent que le monde périra ? Voilà cependant que les saintes Lettres des Hébreux disent au Dieu devant qui toutes les autres divinités tremblent, de l’aveu même d’un si grand philosophe : « Les cieux sont l’ouvrage de vos mains, et ils périront. » Est-ce donc qu’au temps où les cieux périront, le monde, dont ils sont la partie la plus haute et la plus assurée, ne périra pas ? Si Jupiter ne goûte pas ce sentiment, s’il blâme les chrétiens par la voix imposante d’un oracle d’être trop crédules, comme l’assure notre philosophe, pourquoi ne traite-t-il pas aussi de folie la sagesse des Hébreux, qui ont inscrit ce même sentiment dans leurs livres sacrés ? Du moment donc que cette sagesse, qui plait tant à Porphyre qu’il la fait louer par la bouche de ses dieux, nous apprend que les cieux doivent périr, quelle aberration de faire du dogme de la fin du monde un grief contre la religion chrétienne, et le plus sérieux de tous, sous prétexte que les cieux ne peuvent périr que le monde entier ne périsse ? Il est vrai que dans les Écritures qui sont proprement les nôtres, et ne nous sont pas communes avec les Hébreux, c’est-à-dire dans l’Évangile et leslivres des Apôtres, on lit que : « La figure de ce monde passe » ; que : « Le monde passe » ; que : « Le ciel et la terre passeront » ; expressions plus douces, il faut en convenir, que celle des Hébreux, qui disent que le monde {\itshape périra}. De même, dans l’épître de saint Pierre, où il est dit que le monde qui existait alors périt par le déluge, il est aisé de voir quelle est la partie du monde que cet apôtre a voulu désigner, et comment il entend qu’elle a péri, et quels sont les cieux alors renouvelés qui ont été mis en réserve pour être brûles par le feu au jour du jugement dernier et de la ruine des méchants. Un peu après il s’exprime ainsi : « Le jour du Seigneur viendra comme un larron, et alors les cieux passeront avec grand fracas, leséléments embrasés se dissoudront, et la terre, avec ce qu’elle contient, sera consumée par le feu. » Et il ajoute : « Donc, puisque toutes ces choses doivent périr, quelle ne doit pas être votre piété ? » On peut fort bien entendre ici que les cieux qui périront sont ceux dont il dit qu’ils sont mis en réserve pour être brûlés par le feu, et que les éléments qui doivent se dissoudre par l’ardeur du feu sont ceux qui occupent cette basse partie du monde, exposée aux troubles et aux orages ; mais que les globes célestes, où sont suspendus les astres, demeureront intacts. Quant « à ces étoiles qui doivent tomber du ciel », outre qu’on peut donner à ces paroles un autre sens, meilleur que celui que porte la lettre, elles prouvent encore davantage la permanence des cieux, si toutefois les étoiles en doivent tomber. C’est alors une façon figurée de parler, ce qui est vraisemblable, ou bien cela doit s’entendre de quelques météores qui se formeront dans la moyenne région de l’air, comme celui dont parle Virgile :\par
 {\itshape « Une étoile, suivie d’une longue traînée de lumière, traversa le ciel et alla se perdre dans la forêt d’Ida. »} \par
Mais pour revenir au passage du Psalmiste, il semble qu’il n’excepte aucun des cieux, et qu’ils doivent tous périr, puisqu’il dit que les cieux sont l’ouvrage des mains de Dieu, et qu’ils périront. Or, puisqu’il n’y en a pas un qui ne soit l’ouvrage de ses mains, il semble aussi qu’il n’y en ait pas un qui ne doive périr. Je ne pense pas, en effet, que nos philosophes veuillent expliquer ces paroles du psaume par celles de saint Pierre, qu’ils haïssent tant, et prétendre que, comme cet apôtre a entendu les parties pour le tout, quand il a dit que le monde avait péri par le déluge, le Psalmiste de même n’a entendu parler que de la partie la plus basse des cieux, quand il a dit que les cieux périront. Puis donc qu’il n’y a pas d’apparence qu’ils en usent de la sorte, de peur d’approuver le sentiment de l’apôtre saint Pierre et d’être obligés de donner à ce dernier embrasement autant de pouvoir qu’il en donne au déluge, eux qui soutiennent qu’il est impossible que tout le genre humain périsse par les eaux et le feu, il ne leur reste autre chose à dire, sinon que leurs dieux ont loué la sagesse des Hébreux, parce qu’ils n’avaient pas lu ce psaume.\par
Le psaume quarante-neuf parle aussi du jugement dernier en ces termes : « Dieu viendra visible, notre Dieu viendra, et il ne se taira pas. Un feu dévorant marchera devant lui, et une tempête effroyable éclatera tout autour. Il appellera le ciel en haut et la terre, afin de discerner son peuple. Assemblez-lui ses saints, qui élèvent son testament au-dessus des sacrifices. » Nous entendons ceci de Notre-Seigneur Jésus-Christ, qui viendra du ciel, comme nous l’espérons, juger les vivants et les morts. Il viendra visible pour juger justement les bons et les méchants, lui qui est déjà venu caché pour être injustement jugé par les méchants. Il viendra visible, je le répète, et il ne se taira pas, c’est-à-dire qu’il parlera en juge, lui qui s’est tu devant son juge, lorsqu’il a été conduit à la mort comme une brebis qu’on mène à la boucherie, et qui est demeuré muet comme un agneau qui se laisse tondre, ainsi que nous le voyons annoncé dans Isaïe et accompli dans l’Évangile. Quant au {\itshape feu} et à la {\itshape tempête} qui accompagnent le Seigneur, nous avons déjà dit comment il faut entendre ces expressions, en expliquant les expressions semblables du prophète Isaïe. Par ces mots : « Il appellera le ciel en haut » ; comme les saints et les justes s’appellent avec raison le {\itshape ciel}, le Psalmiste veut dire sans doute ce qu’a dit l’Apôtre : que nous serons emportés dans les nues, pour aller au-devant du Seigneur, au milieu des airs : car à le comprendre selon la lettre, comment le ciel serait-il {\itshape appelé en haut}, puisqu’il ne peut être ailleurs ? À l’égard de ce qui suit : « Et la terre, pour faire la séparation de son peuple », si l’on sous-entend seulement il appellera, c’est-à-dire {\itshape il appellera} la terre, sans sous-entendre {\itshape en haut}, on peut fort bien penser que {\itshape le ciel} figure ceux qui doivent juger avec lui, et la terre ceux qui doivent être jugés ; et alors ces paroles : « Il appellera le ciel en haut », ne signifient pas qu’il enlèvera les saints dans les airs, mais qu’il les fera asseoir sur des trônes pour juger. Ces mots peuvent encore avoir le sens suivant : « Il appellera le ciel en haut », c’est-à-dire qu’il appellera les anges au plus haut des cieux, pour descendre en leur compagnie et juger le monde ; et « il appellera aussi la terre », c’est-à-dire les hommes qui doivent être jugés sur la terre. Mais si, lorsque le Psalmiste dit : « Et la terre, etc. », on sous-entend l’un ou l’autre, c’est-à-dire {\itshape qu’il appellera} et qu’il appellera {\itshape en haut}, je ne pense pas qu’on puisse mieux l’entendre que des hommes qui seront emportés dans les airs au-devant de Jésus-Christ, et qu’il appelle {\itshape le ciel}, à cause de leurs âmes, et {\itshape la terre}, à cause de leurs corps.\par
Or, qu’est-ce {\itshape discerner son peuple}, sinon séparer par le jugement les bons d’avec les méchants, comme les brebis d’avec les boucs ? Il s’adresse ensuite aux anges, et leur dit : « Assemblez-lui ses saints ci, parce que sans doute un acte aussi important se fera par le ministère des anges. » Que si nous demandons quels sont ces saints qu’ils lui doivent assembler : « Ceux, dit-il, qui élèvent son testament au-dessus des sacrifices. » Car voilà toute la vie des justes : élever le testament de Dieu au-dessus des sacrifices. En effet, ou les œuvres de miséricorde sont préférables aux sacrifices, selon cet oracle du ciel : « J’aime mieux la miséricorde que le sacrifice », ou au moins, en donnant un autre sens aux paroles du Psalmiste, les œuvres de miséricorde sont les sacrifices qui servent à apaiser Dieu, comme je me souviens de l’avoir dit au deuxième livre de cet ouvrage. Les justes accomplissentle testament de Dieu par ces œuvres, parce qu’ils les font à cause des promesses qui sont contenues dans son Nouveau Testament ; d’où vient qu’au dernier jugement, quand Jésus-Christ aura assemblé ses saints et les aura placés à sa droite, il leur dira : « Venez, vous que mon père a bénis, prenez possession du royaume qui vous est préparé dès le commencement du monde ; car j’ai eu faim, et vous m’avez donné à manger » ; et le reste au sujet des bonnes œuvres des justes et de la récompense éternelle qu’ils en recevront par la dernière sentence.
\subsection[{Chapitre XXV}]{Chapitre XXV}

\begin{argument}\noindent Prophétie de Malachie annonçant le dernier jugement de Dieu et la purification de quelques-uns par les peines du purgatoire.
\end{argument}

\noindent Le prophète Malachie ou Malachi, appelé aussi Ange, et qui, suivant quelques-uns, est le même qu’Esdras, dont il y a d’autres écrits reçus dans le canon des livres saints (tel est, d’après Jérémie, le sentiment des Hébreux), Malachie, dis-je, a parlé ainsi du jugementdernier : « Le voici qui vient, dit le Seigneur tout-puissant ; et qui soutiendra l’éclat de son avènement, ou qui pourra supporter ses regards ? Car il sera comme le feu d’une fournaise ardente et comme l’herbe des foulons ; et il s’assoira comme un fondeur qui affine et épure l’or et l’argent ; et il purifiera les enfants de Lévi, et il les fondra comme l’or et l’argent ; et ils offriront des victimes au Seigneur en justice. Et le sacrifice de Juda et de Jérusalem plaira au Seigneur, comme autrefois dans les premières années. Je m’approcherai de vous pour juger, et je serai un témoin fidèle contre les enchanteurs, les adultères et les parjures, contre ceux qui retiennent le salaire de l’ouvrier, qui oppriment les veuves par violence, outragent les orphelins, font injustice à l’étranger, et ne craignent point mon nom, dit le Seigneur tout-puissant. Car je suis le Seigneur votre Dieu, et je ne change point. » Ces paroles font voir clairement, à mon avis, qu’en ce jugement il y aura pour quelques-uns des peines purifiantes. Que peut-on entendre autre chose par ce qui suit : « Quisoutiendra l’éclat de son avènement, ou qui pourra supporter ses regards ? Car il sera comme le feu d’une fournaise ardente et comme l’herbe des foulons. Il s’assoira comme un fondeur qui affine et épure l’or et l’argent ; et il purifiera les enfants de Lévi, et il les fondra comme l’or et l’argent. » Isaïe dit quelque chose de semblable : « Le Seigneur fera disparaître les impuretés des fils et des filles de Sion, et ôtera le sang du milieu d’eux par le souffle du jugement et par le souffle du feu. » À moins qu’on ne veuille dire qu’ils seront purifiés et comme affinés, lorsque les méchants seront séparés d’eux par le jugement dernier, et que la séparation des uns sera la purification des autres, puisqu’à l’avenir ils vivront sans être mêlés ensemble. Mais, d’un autre côté, lorsque le Prophète ajoute « qu’il purifiera les enfants de Lévi, et les affinera comme on affine l’or et l’argent, qu’ils offriront des victimes au Seigneur en justice, et que le sacrifice de Juda et de Jérusalem plaira au Seigneur », il fait bien voir que ceux qui seront purifiés plairont à Dieu par des sacrifices de justice, et qu’ainsi ils seront purifiés de l’injustice qui était cause qu’ils lui déplaisaient auparavant. Or, eux-mêmes seront des victimes d’une pleine et parfaite justice, lorsqu’ils seront purifiés. Que pourraient-ils en cet état offrir à Dieu de plus agréable qu’eux-mêmes ? Mais nous parlerons ailleurs de ces peines purifiantes, afin d’en parler plus à fond. Au reste, par les enfants de Lévi, de Juda et de Jérusalem, il faut entendre l’Église de Dieu, composée non seulement des Juifs, mais des autres nations, non pas telle qu’elle est dans ce temps de pèlerinage, dans ce temps où : « Si nous disons que nous n’avons point de péché, nous nous séduisons nous-mêmes, et la vérité n’est point en nous », mais telle qu’elle sera alors, purifiée par le dernier jugement, comme une aire nettoyée par le van. Ceux mêmes qui ont besoin de cette purification ayant été purifiés par le feu, nul n’aura plus à offrir de sacrifice à Dieu pour ses péchés. Sans doute tous ceux qui sacrifient ainsi sont coupables de quelques péchés, et c’est pour en obtenir la rémission qu’ils sacrifient ; mais lorsqu’ils auront fait accepter leur sacrifice, Dieu les renverra purifiés.
\subsection[{Chapitre XXVI}]{Chapitre XXVI}

\begin{argument}\noindent Des sacrifices que les saints offriront à Dieu, et qui lui seront agréables, comme aux anciens jours, dans les premières années du monde.
\end{argument}

\noindent Or, Dieu, voulant montrer que sa Cité ne sera point alors en état de péché, dit que les enfants de Lévi offriront des sacrifices en justice. Ce ne sera donc pas en péché, ni pour le péché. D’où l’on peut conclure que ce qui suit : « Et le sacrifice de Juda et de Jérusalem plaira au Seigneur, comme aux anciens jours, dans les premières années », ne peut servir de fondement raisonnable aux Juifs pour prétendre qu’il y a là une promesse de ramener le temps des sacrifices de l’Ancien Testament. Ils n’offraient point alors de victimes en justice, mais en péché, puisqu’ils les offraient, surtout dans l’origine, pour leur péché spécialement. Cela est si vrai, que le grand-prêtre, qui était vraisemblablement plus juste que les autres, avait coutume, selon le commandement de Dieu, d’offrir d’abord pour ses péchés, ensuite pour ceux du peuple. Il faut dès lors expliquer le sens de ces paroles : « Comme aux anciens jours, dans les premières années ». Peut-être rappellent-elles le temps où les premiers hommes étaient dans le paradis ; et, en effet, c’est alors que, dans l’état de pureté et d’intégrité, exempts de toute souillure et de tout péché, ils s’offraient eux-mêmes à Dieu comme des victimes très pures. Mais depuis qu’ils en ont été chassés pour leur désobéissance, et que toute la nature humaine a été condamnée en eux, personne, à l’exception du Médiateur (et de quelques petits enfants, ceux qui ont été baptisés), « personne, dit l’Écriture, n’est exempt de péché ; pas même l’enfant « qui n’a qu’un jour de vie sur la terre ». Répondra-t-on que ceux-là peuvent passer pour offrir des sacrifices en justice, qui les offrent avec foi, puisque l’Apôtre a dit que « le juste vit de la foi » ; c’est oublier que, selon le même Apôtre, le juste se séduit lui-même, s’il se dit exempt de péché ; il se gardera donc bien de le dire et de le croire, lui qui vit de la foi. Peut-on comparer d’ailleurs le temps de la foi aux derniers temps, où ceux qui offriront des sacrifices en justice seront purifiés par le feu du dernier jugement ? Puisqu’ilfaut croire qu’après cette purification les justes n’auront aucun péché, ce temps ne peut assurément être comparé qu’avec celui où les premiers hommes, avant leur infidélité, menaient dans le paradis la vie la plus innocente et la plus heureuse. On peut donc très bien donner ce sens aux paroles de l’Écriture sur « les « anciens jours et les premières années ». Dans Isaïe, après la promesse d’un ciel nouveau et d’une terre nouvelle, entre autres images et paroles énigmatiques sur la félicité des saints, que nous n’avons point expliquées pour éviter d’être long, on lit : « Les jours de mon peuple seront comme l’arbre de vie. » Or, qui est assez peu versé dans les Écritures pour ignorer où Dieu avait planté l’arbre de vie, dont les premiers hommes furent sevrés, lorsque leur désobéissance les chassa du paradis et que Dieu plaça auprès de cet arbre un ange terrible avec une épée flamboyante ?\par
Si l’on soutient que ces jours de l’arbre de vie, rappelés par Isaïe, sont ceux de l’Église, qui s’écoulent maintenant, et que c’est Jésus-Christ que le Prophète appelle l’arbre de vie, parce qu’il est la Sagesse de Dieu, dont Salomon a dit : « Elle est un arbre de vie pour tous ceux qui l’embrassent » ; si l’on soutient que les premiers hommes ne passèrent pas des années dans le paradis et n’eurent pas le loisir d’y engendrer des enfants, de sorte qu’on ne puisse rapporter à ce temps les mots : « Comme aux anciens jours, dans les premières années », j’aime mieux laisser cette question, pour n’être point obligé d’entrer dans une trop longue discussion. Aussi bien, je vois un autre sens qui m’empêche de croire que le Prophète nous promette ici, comme un grand présent, le retour des sacrifices charnels des Juifs, aux anciens-jours, dans les premières années. En effet, ces victimes de l’ancienne loi, qui devaient être choisies saris tache et sans défaut dans chaque troupeau, représentaient les hommes justes, exempts de toute souillure, tel que Jésus-Christ seul a été. Or, comme après le jugement, ceux qui seront dignes de purification auront été purifiés par le feu, de telle sorte qu’ils s’offriront eux-mêmes en justice, comme des victimes pures de toute tache et de toute souillure, ils seront certainement semblables aux victimes des anciens jours et des premières années que l’on offrait en image de ces victimes futures.\par
En effet, la pureté que figurait le corps pur de ces animaux immolés sera alors réellement dans la chair et dans l’âme immortelle des saints. Ensuite le Prophète, s’adressant à ceux qui seront dignes, non de purification, mais de damnation, leur dit : « Je m’approcherai de vous pour juger, et je serai un prompt témoin contre les enchanteurs, contre les adultères, etc. » Et après avoir fait le dénombrement de beaucoup d’autres crimes damnables, il ajoute : « Car je suis le Seigneur votre Dieu, et je ne change point », comme s’il disait : Pendant que vous changez, par vos crimes, en pis, par ma grâce, en mieux, moi je ne change point. Il dit qu’il se portera pour témoin, parce qu’il n’a pas besoin, pour juger, d’autres témoins que de lui-même ; et qu’il sera un prompt témoin, ou bien parce qu’il viendra soudain et à l’improviste, quand on le croira encore éloigné, ou bien parce qu’il convaincra les consciences, sans avoir besoin de beaucoup de paroles, comme il est écrit : « Les pensées de l’impie déposeront contre lui » ; et selon l’Apôtre : « Les pensées des hommes les accuseront ou les excuseront au jour que Dieu jugera par Jésus-Christ de tout ce qui est caché dans le cœur. » C’est ainsi que Dieu sera un prompt témoin, parce qu’en un instant il rappellera de quoi convaincre et punir une conscience.
\subsection[{Chapitre XXVII}]{Chapitre XXVII}

\begin{argument}\noindent De la séparation des bons et des méchants au jour du jugement dernier.
\end{argument}

\noindent Ce que j’ai rapporté sommairement du même Prophète, au dix-huitième livre, regarde aussi le jugement dernier. Voici le passage : « Ils seront mon héritage, dit le Seigneur tout-puissant, au jour que j’agirai, et je les épargnerai, comme un père épargne un fils obéissant. Alors je me comporterai d’une autre sorte, et vous verrez la différence qu’il y a entre le juste et l’impie, entre celui qui sert Dieu et celui qui ne le sert pas. Car voici venir le jour allumé comme une fournaise ardente et il les consumera ; Tous les étrangers et tous les pécheurs seront comme du chaume, et le jour qui approche les brûlera tous, dit le Seigneur, sans qu’il reste d’eux ni branches, ni racines. Mais pour vousqui craignez mon nom, le soleil de justice se lèvera pour vous, et vous trouverez une abondance de tous biens, à l’ombre de ses ailes. Vous bondirez comme de jeunes taureaux échappés, et vous foulerez aux pieds les méchants, et ils deviendront cendres sous vos pas, dit le Seigneur tout-puissant. » Quand cette différence des peines et des récompenses qui sépare les méchants d’avec les bons, et qui ne se voit pas sous le soleil, dans la vanité de cette vie, paraîtra sous le soleil de justice qui éclairera la vie future, alors sera le dernier jugement.
\subsection[{Chapitre XXVIII}]{Chapitre XXVIII}

\begin{argument}\noindent Il faut interpréter spirituellement la loi de Moïse pour prévenir les murmures damnables des âmes charnelles.
\end{argument}

\noindent Le même prophète ajoute : « Souvenez-vous de la loi que j’ai donnée pour tout Israël à mon serviteur Moïse, sur la montagne de Choreb. » C’est fort à propos qu’il rappelle les commandements de Dieu, après avoir relevé, la grande différence qu’il y a entre ceux qui observent la loi et ceux qui la méprisent. Il le fait aussi afin d’apprendre aux Juifs à concevoir spirituellement la loi, et à y trouver Jésus-Christ, le juge qui doit faire le discernement des bons et des méchants. Ce n’est pas en vain que le même Seigneur dit aux Juifs : « Si vous aviez foi en Moïse, vous croiriez en moi aussi ; car c’est de moi qu’il a écrit. » En effet ; c’est parce qu’ils comprennent la loi charnellement, et qu’ils ne savent pas que ses promesses temporelles De sont que des figures des récompenses éternelles, c’est pour cela qu’ils sont tombés dans des murmures ; et qu’ils ont dit : « C’est une folie de servir Dieu ; que nous revient-il d’avoir observé ses commandements et de nous être humiliés en la présence du Seigneur tout-puissant ? N’avons-nous donc pas raison d’estimer heureux les méchants et les ennemis de Dieu ; puisqu’ils triomphent dans la gloire et l’opulence ? » Pour arrêter ces murmures, le Prophète a été obligé en quelque sorte de déclarer le dernier jugement, où les méchants ne posséderont pas même une fausse félicité, mais paraîtront évidemment malheureux, et où les bons neseront assujettis à aucune misère, mais jouiront avec éclat d’une éternelle béatitude. Il avait rapporté auparavant des plaintes semblables des Juifs : « Tout homme qui fait le mal est bon devant Dieu, et il n’y a que les méchants qui lui plaisent. » C’est donc en entendant charnellement la loi de Moïse qu’ils se sont portés à ces plaintes ; d’où vient, au psaume soixante-douze, ce cri de celui qui a chancelé, et qui a senti ses pieds défaillir en considérant la prospérité des méchants, de sorte qu’il a envié leur condition, jusqu’à proférer ces paroles : « Comment Dieu voit-il cela ? Le Très-Haut connaît-il ces choses ? » et encore : « C’est donc bien en vain que j’ai conservé purs mon cœur et mes mains. » Le Psalmiste avoue qu’il s’est vainement efforcé de comprendre pourquoi les bous paraissent misérables en cette vie, et les méchants heureux : « Je m’efforce en vain, dit-il, il faut que j’entre dans le sanctuaire de Dieu, et que j’y découvre la fin. » En effet, à la fin du monde, au dernier jugement, il n’en sera pas ainsi ; et les choses paraîtront tout autres, quand éclateront au grand jour la félicité des bons et la misère des méchants.
\subsection[{Chapitre XXIX}]{Chapitre XXIX}

\begin{argument}\noindent De la venue d’Élie avant le jugement, pour dévoiler le sens caché des Écritures et convertir les Juifs a Jésus-Christ.
\end{argument}

\noindent Après avoir averti les Juifs de se souvenir de la loi de Moïse, prévoyant bien qu’ils seraient encore longtemps sans la concevoir spirituellement, l’Écriture ajoute aussitôt : « Je vous enverrai Élie de Thesba, avant que ce grand et lumineux jour du Seigneur arrive, qui tournera le cœur du père vers le fils, et le cœur de l’homme vers son prochain, de peur qu’à mon avènement je ne détruise entièrement la terre. » C’est une croyance assez générale parmi les fidèles, qu’à la fin du monde, avant le jugement, les Juifs doivent croire au vrai Messie, c’est-à-dire en notre Christ, par le moyen de ce grand et admirable prophète Eue, qui leur expliquera la loi. Aussi bien, ce n’est pas sans raison que l’on espère en lui le précurseur de l’avènement de Jésus-Christ, puisque ce n’est pas sansraison que maintenant même on le croit vivant. Il est certain, en effet, d’après le témoignage même de l’Écriture, qu’il a été ravi dans un char de feu. Lorsqu’il sera venu, il expliquera spirituellement la loi que les Juifs entendent encore charnellement, et « il tournera le cœur du père vers le fils », c’est-à-dire le cœur des pères vers leurs enfants ; car les Septante ont mis ici le singulier pour le pluriel. Le sens est que les Juifs, qui sont les enfants des Prophètes, du nombre desquels était Moïse, entendront la loi comme leurs pères, et ainsi le cœur des pères se tournera vers les enfants et le cœur des enfants vers les pères, lorsqu’ils auront les mêmes sentiments. Les Septante ajoutent que « le cœur de l’homme se tournera vers son prochain », parce qu’il n’y a rien de plus proche que les pères et leurs enfants. On peut encore donner un autre sens plus relevé aux paroles des Septante, qui ont interprété l’Écriture en prophètes, et dire qu’Élie tournera le cœur de Dieu le Père vers le Fils, non en faisant qu’il l’aime, mais en instruisant les Juifs de cet amour, et les portant par là eux-mêmes à aimer notre Christ, qu’ils haïssaient auparavant. En effet, de notre temps, au regard des Juifs, Dieu a le cœur détourné de notre Christ, parce qu’ils ne croient pas qu’il soit Dieu, ni Fils de Dieu. Mais alors Dieu aura pour eux le cœur tourné vers son Fils, quand, leur cœur étant changé, ils verront l’amour du Père envers le Fils. Quant à ce qui suit : « Et le cœur de l’homme vers son prochain », comment pouvons-nous mieux interpréter ces paroles qu’en disant qu’Élie tournera le cœur de l’homme vers Jésus-Christ homme ? Car Jésus-Christ étant notre Dieu, sous la forme de Dieu, a pris la forme d’esclave, et a daigné devenir notre prochain. Voilà donc ce que fera Élie : « De peur, dit le Seigneur, qu’à mon avènement je ne détruise entièrement la terre. » C’est que ceux-là sont terre qui ne goûtent que les choses de la terre, comme les Juifs charnels ; et voilà ceux d’où viennent ces murmures contre Dieu : « Les méchants lui plaisent », et : « C’est une folie de le servir. »
\subsection[{Chapitre XXX}]{Chapitre XXX}

\begin{argument}\noindent Malgré l’obscurité de quelques passages de l’Ancien Testament, où la personne du Christ ne paraît pas en toute évidence, il faut, quand il est dit que Dieu viendra juger, entendre cela de Jésus-Christ.
\end{argument}

\noindent Il y a beaucoup d’autres témoignages de l’Écriture sur le dernier jugement, mais il serait trop long de les rapporter, et il nous suffit d’avoir prouvé qu’il a été annoncé par l’Ancien et par le Nouveau Testament. Mais l’Ancien ne déclare pas aussi formellement que le Nouveau que c’est Jésus-Christ qui doit rendre ce jugement. De ce qu’il y est dit que le Seigneur Dieu viendra, il ne s’ensuit pas que ce doive être Jésus-Christ, car cette qualification convient aussi bien au Père ou au Saint-Esprit qu’au Fils. Nous ne devons pas toutefois laisser passer ce point sans preuves. Il est nécessaire pour cela de montrer premièrement, comment Jésus-Christ parle dans ses prophètes, sous le nom de Seigneur Dieu, afin qu’aux autres endroits, où cela n’est point manifeste et où néanmoins il est dit que le Seigneur Dieu doit venir pour juger, on puisse l’entendre de Jésus-Christ. Il y a un passage dans le prophète Isaïe qui fait voir clairement ce dont il s’agit. Voici en effet comment Dieu parla par ce Prophète : « Écoutez-moi, Jacob et Israël que j’appelle. Je suis le premier et je suis pour jamais. Ma main a fondé la terre, et ma droite a affermi le ciel. Je les appellerai, et ils s’assembleront tous et ils entendront. Qui a annoncé ces choses ? Comme je vous aime, j’ai accompli votre volonté sur Babylone et exterminé la race des Chaldéens. J’ai parlé et j’ai appelé ; je l’ai amené, et je l’ai fait réussir dans ses entreprises. Approchez-vous de moi, et écoutez-moi. Dès le commencement, je n’ai point parlé en secret ; j’étais présent, lorsque ces choses se faisaient. Et maintenant le Seigneur Dieu m’a envoyé, et son Esprit. » C’est lui-même qui parlait tout à l’heure comme le Seigneur Dieu, et néanmoins on ne saurait pas que c’est Jésus-Christ, s’il n’ajoutait : « Et maintenant le Seigneur Dieu m’a envoyé, et son Esprit. » Il dit cela, en effet, selon la forme d’esclave, et parle d’une chose à venir, comme si elle était passée. De même, en cet autre passage du même prophète : « Ila été conduit à la mort, comme une brebis que l’on mène à la boucherie » ; il ne dit pas : « Il sera conduit », mais il se sert du passé pour le futur, selon le langage ordinaire des Prophètes. Il y a un autre passage dans Zacharie, où il dit clairement que le Tout-Puissant a envoyé le Tout-Puissant. Or, de qui peut-on entendre cela, sinon de Dieu le Père qui a envoyé Dieu le Fils ? Voici le passage : « Le Seigneur tout puissant a dit : Après la gloire, il m’a envoyé vers les nations, qui vous ont pillé. Car vous toucher, c’est toucher la prunelle de son œil. J’étendrai ma main sur eux, et ils deviendront les dépouilles de ceux qui étaient leurs esclaves et vous connaîtrez que c’est le Seigneur tout-puissant qui m’a envoyé. » Voilà le Seigneur tout puissant qui dit qu’il est envoyé par le Seigneur tout-puissant. Qui serait entendre ces paroles d’un autre que de Jésus-Christ, qui parle aux brebis égarées de la maison d’Israël ? Aussi dit-il dans l’Évangile. « Je n’ai été envoyé que pour les brebis perdues de la maison d’Israël », qu’il compare ici à la prunelle des yeux de Dieu, pour montrer combien il les chérit. Parmi ces brebis, il faut compter les Apôtres mêmes, mais « après la gloire », c’est-à-dire après sa résurrection glorieuse, car avant, comme dit saint Jean l’évangéliste : « Jésus n’était point encore glorifié. » Il fut aussi envoyé aux nations, en la personne de ses Apôtres ; et ainsi fut accompli ce qu’on lit dans le psaume : « Vous me délivrerez des rébellions de ce peuple ; vous m’établirez chef des nations » ; afin que ceux qui avaient pillé les Israélites, et dont les Israélites avaient été les esclaves, devinssent eux-mêmes les dépouilles des Israélites ; car c’est ce qu’il avait promis aux Apôtres en leur disant : « Je vous ferai pêcheurs d’hommes » ; et à l’un deux : « Dès ce moment ton emploi sera de prendre des hommes. » Ils deviendront donc les dépouilles, mais en un bon sens, comme sont celles qu’on enlève dans l’Évangile à ce Fort armé, après l’avoir lié de chaînes encore plus fortes que lui.\par
Le Seigneur parlant encore par les Prophètes : « En ce jour-là, dit-il, j’aurai soin d’exterminer toutes les nations quiviennent contre Jérusalem, et je verserai sur la maison de David et sur les habitants de Jérusalem l’esprit de grâce et de miséricorde ; ils jetteront les yeux sur moi, parce qu’ils m’ont insulté ; et ils se lamenteront, comme ils se lamenteraient au sujet d’un fils bien-aimé ; ils seront outrés de douleur, comme ils le seraient pour un fils unique. » À qui appartient-il, sinon à, Dieu seul, d’exterminer toutes les nations ennemies de la cité de Jérusalem, « qui viennent contre elle », c’est-à-dire qui lui sont contraires, ou, selon d’autres versions, qui « viennent sur elle », c’est-à-dire qui veulent l’assujettir ? et à qui appartient-il de répandre l’esprit de grâce et de miséricorde sur la maison de David et sur les habitants de Jérusalem ? Sans doute cela n’appartient qu’à Dieu ; et aussi est-ce à Dieu que le Prophète le fait dire. Et toutefois Jésus-Christ fait voir que c’est lui qui est ce Dieu qui a fait toutes ces merveilles, lorsqu’il ajoute : « Et ils jetteront les yeux sur moi, parce qu’ils m’ont insulté, et ils se lamenteront, comme ils se lamenteraient au sujet d’un fils bien-aimé, et ils seront outrés de douleur, comme ils le seraient pour un fils unique. » Car en ce jour-là, les Juifs mêmes, qui doivent recevoir l’esprit de grâce et de miséricorde, jetant les yeux sur Jésus-Christ, qui viendra dans sa majesté, et voyant que c’est, lui qu’ils ont méprisé dans son abaissement, en la personne de leurs pères, se repentiront de l’avoir insulté dans sa passion. Quant à leurs pères qui ont été les auteurs d’une si grande impiété, ils le verront bien aussi, quand ils ressusciteront ; mais ce ne sera que pour être punis de leur attentat, et non pour se convertir. Ce n’est donc pas d’eux qu’il faut entendre ces paroles : « Je répandrai sur la maison de David et sur les habitants de Jérusalem l’esprit de grâce et de miséricorde ; et ils jetteront les yeux sur moi, à cause qu’ils m’ont insulté » ; et pourtant, ceux qui croiront à la prédication d’Élie doivent descendre de leur race. Mais de même que nous disons aux Juifs : Vous avez fait mourir Jésus-Christ, quoique ce crime soit l’ouvrage de leurs ancêtres ; de même ceux dont parle le Prophète s’affligeront d’être en quelque sorte les auteurs du mal que d’autres ont accompli. Ainsi, bien qu’après avoir reçu l’esprit de grâce et de miséricorde, ils ne soient point enveloppés dans une même condamnation,ils ne laisseront pas de pleurer le crime de leurs pères, comme s’ils en étaient coupables. Au reste, tandis que les Septante ont traduit : « Ils jetteront les yeux sur moi, à cause qu’ils m’ont insulté », l’hébreu porte : « Ils jetteront les yeux sur moi qu’ils ont percé » ; expressions qui rappellent encore mieux Jésus-Christ crucifié. Toutefois « l’insulte », suivant l’expression adoptée par les Septante, embrasse en quelque sorte l’ensemble de la passion. En effet, Jésus-Christ fut insulté par les Juifs, et quand il fut pris, et quand il fut lié, et quand il l’ut jugé, et quand il fut revêtu du manteau d’ignominie, et quand il fut couronné d’épines, frappé sur la tête à coups de roseau, adoré dérisoirement le genou en terre, et quand il porta sa croix, et enfin quand il y fut attaché. Ainsi, en réunissant l’une et l’autre version, et en lisant qu’ils l’ont insulté et qu’ils l’ont percé, nous reconnaîtrons mieux la vérité de la passion du Sauveur.\par
Quand donc nous lisons dans les Prophètes que Dieu doit venir juger, il le faut entendre de Jésus-Christ ; car, bien que ce soit le Père qui doive juger, il ne jugera que par l’avènement du Fils de l’homme. Il ne jugera personne visiblement ; il a donné tout pouvoir de juger au Fils, qui viendra pour rendre le jugement, comme il est venu pour le subir. De quel autre que de lui peut-on entendre ce que Dieu dit par Isaïe, sous le nom de Jacob et d’Israël, dont le Christ est issu selon la chair : « Jacob est mon serviteur ; je le protègerai ; Israël est mon élu ; c’est pourquoi mon âme l’a choisi. Je lui ai donné mon esprit ; il prononcera le jugement aux nations. Il ne criera point, il ne se taira point ; et sa voix ne sera point entendue au dehors. Il ne brisera point le roseau cassé ; il n’éteindra point la lampe qui fume encore ; mais il jugera en vérité. Il sera resplendissant, et ne pourra être opprimé jusqu’à ce qu’il établisse le jugement sur la terre ; et les nations espéreront en lui. » L’hébreu ne porte pas Jacob et Israël ; mais les Septante, voulant nous montrer comment il faut entendre le mot de serviteur que porte le serviteur, c’est-à-dire le profond abaissement où a daigné se soumettre le Très-Haut, ont mis le nom de celui dans la postérité duquel il a pris cette forme de serviteur. Le Saint-Esprit lui a été donné, et nous le voyons descendre sur lui dans l’Évangile, sous la forme d’une colombe. Il a prononcé le jugement aux nations, parce qu’il a prédit l’accomplissement futur de ce qui leur était caché. Sa douceur l’a empêché de crier ; et toutefois il n’a pas cessé de prêcher la vérité. Mais sa voix n’a point été entendue au dehors, et ne l’est pas encore, parce que ceux qui sont retranchés de son corps ne lui obéissent pas. Il n’a point brisé ni éteint les Juifs, ses persécuteurs, qui sont comparés ici tour à tour à un roseau cassé, parce qu’ils ont perdu leur fermeté, et à une lampe fumante, parce qu’ils n’ont plus de lumière. Il les a épargnés, parce qu’il n’était pas encore venu pour les juger, mais pour être jugé par eux. Il a prononcé un jugement véritable, leur prédisant qu’ils seraient punis, s’ils persistaient en leur malice. Sa face a été resplendissante sur la montagne, et son nom célèbre dans l’univers ; et il n’a pu être opprimé par ses persécuteurs, ni dans sa personne, ni dans son Église. Ainsi, c’est en vain que ses ennemis disent : « Quand est-ce que son nom sera aboli et périra ? jusqu’à ce qu’il établisse le jugement sur la terre. » Voilà ce que nous cherchions et ce qui était caché car c’est le dernier jugement qu’il établira sur la terre, quand il descendra du ciel. Nous voyons déjà accompli ce que le Prophète ajoute : « Et les nations espéreront en son nom. » Que ce fait, qui ne peut pas être nié, soit donc une raison pour croire ce que l’on nie impudemment. Car qui eût osé espérer cette merveille dont sont témoins ceux-là mêmes qui refusent de croire en Jésus-Christ, et qui grincent des dents et sèchent de dépit, parce qu’ils ne peuvent les nier ? qui eût osé espérer que les nations espéreraient au nom de Jésus-Christ, quand on le prenait, quand on le liait et le bafouait, quand on l’insultaitet le crucifiait, et enfin quand ses disciples même avaient perdu l’espérance qu’ils commençaient à avoir en lui ? Ce qu’à peine un seul larron crut alors sur la croix, toutes les nations le croient maintenant, et, de peur de mourir à jamais, elles sont marquées du signe de cette croix sur laquelle Jésus-Christ est mort.\par
Il n’est donc personne qui doute de ce jugement dernier, annoncé dans les saintes Écritures, sinon ceux qui, par une incrédulité aveugle et opiniâtre, ne croient pas en ces Écritures mêmes, bien qu’elles aient déjà justifié devant toute la terre une partie des vérités qu’elles annoncent. Voilà donc les choses qui arriveront en ce jugement, ou vers cette époque : l’avènement d’Élie, la conversion des Juifs, la persécution de l’Antéchrist, la venue de Jésus-Christ pour juger, la résurrection des morts, la séparation des bons et des méchants, l’embrasement du monde et son renouvellement. Il faut croire que toutes ces choses arriveront ; mais comment et en quel ordre ? l’expérience nous l’apprendra mieux alors que toutes nos conjectures ne peuvent le faire maintenant. J’estime pourtant qu’elles arriveront dans le même ordre où je viens de les rappeler.\par
Il ne me reste plus que deux livres à écrire pour terminer cet ouvrage et m’acquitter de mes promesses avec l’aide de Dieu. Dans le premier des deux je traiterai du supplice des méchants ; dans l’autre, de la félicité des bons ; et j’y réfuterai les vains raisonnements des hommes qui se croient sages en se raillant des promesses de Dieu, et qui méprisent comme faux et ridicules les dogmes qui nourrissent notre foi. Mais pour ceux qui sont sages selon Dieu, sa toute-puissance est le grand argument qui leur fait croire toutes les vérités qui semblent incroyables aux hommes, et qui néanmoins sont contenues dans les saintes Écritures, dont la véracité a déjà été justifiée de tant de manières. Ils tiennent pour certain qu’il est impossible que Dieu ait voulu nous tromper, et qu’il peut faire ce qui parait impossible aux infidèles.
\section[{Livre vingt-et-unième. La réprobation des méchants}]{Livre vingt-et-unième. \\
La réprobation des méchants}\renewcommand{\leftmark}{Livre vingt-et-unième. \\
La réprobation des méchants}

\subsection[{Chapitre premier}]{Chapitre premier}

\begin{argument}\noindent L’ordre de la discussion veut que l’on traite du supplice éternel des damnés avant de parler de l’éternelle félicité des saints.
\end{argument}

\noindent Je me propose, avec l’aide de Dieu, de traiter dans ce livre du supplice que doit souffrir le diable avec tous ses complices, lorsque les deux cités seront parvenues à leurs fins par Notre-Seigneur Jésus-Christ, juge des vivants et des morts. Ce qui me décide à observer cet ordre et à ne parler qu’au livre suivant de la félicité des saints, c’est que, dans l’un et dans l’autre état, l’âme sera unie à un corps, et qu’il semble moins croyable que des corps puissent subsister parmi des tourments éternels, que dans une félicité éternelle, exempte de toute douleur. Ainsi, quand j’aurai établi le premier point, je prouverai plus aisément l’autre. L’Écriture sainte ne s’éloigne pas de cet ordre ; car, bien qu’elle commence quelquefois par la félicité des bons, comme dans ce passage : « Ceux qui ont bien vécu sortiront de leur tombeau pour ressusciter à la vie, et ceux qui ont mal vécu en sortiront pour être condamnés », il y a aussi d’autres passages où elle n’en parle qu’en second lieu, comme dans celui-ci : « Le Fils de l’homme enverra ses anges, qui ôteront tous les scandales de son royaume et les jetteront dans la fournaise ardente. C’est là qu’il y aura des pleurs et des grincements de dents. Alors les justes resplendiront comme le soleil dans le royaume de leur Père. » Et encore : « Ainsi les méchants iront au supplice éternel, et « les bons à la vie éternelle. » Si l’on y veut regarder, on trouvera aussi que les Prophètes ont suivi tantôt le premier ordre, tantôt le second. Mais il serait trop long de le prouver ici ; qu’il me suffise d’avoir rendu raison de l’ordre que j’ai choisi.
\subsection[{Chapitre II}]{Chapitre II}

\begin{argument}\noindent Si des corps peuvent vivre éternellement dans le feu.
\end{argument}

\noindent Que dirai-je pour prouver aux incrédules que des corps humains vivants et animés peuvent non seulement ne jamais mourir, mais encore subsister éternellement au milieu des flammes et des tourments ? Car ils ne veulent pas que notre démonstration se fonde sur la toute-puissance de Dieu, mais sur des exemples. Nous leur répondrons donc qu’il y a des animaux qui certainement sont corruptibles, puisqu’ils sont mortels, et qui ne laissent pas de vivre au milieu du feu, et de plus, que dans des sources d’eau chaude où on ne saurait porter la main sans se brûler, il se trouve une certaine sorte de vers qui non seulement y vivent, mais qui ne peuvent vivre ailleurs. Mais nos adversaires refusent de croire le fait, à moins de le voir ; ou si on le leur montre, du moins si on le leur prouve par des témoins dignes de foi, ils prétendent que cela ne suffit pas encore, sous prétexte que les animaux en question, d’une part, ne vivent pas toujours, et de l’autre, que, vivant dans le feu sans douleur, parce que cet élément est conforme à leur nature, ils s’y fortifient, bien loin d’y être tourmentés. Comme si le contraire n’était pas plus vraisemblable ! Car c’est assurément une chose merveilleuse d’être tourmenté par le feu, et néanmoins d’y vivre ; mais il est bien plus surprenant de vivre dans le feu et de n’y pas souffrir. Si donc on croit la première de ces choses, pourquoi ne croirait-on pas l’autre ?
\subsection[{Chapitre III}]{Chapitre III}

\begin{argument}\noindent La souffrance corporelle n’aboutit pas nécessairement à la dissolution des corps.
\end{argument}

\noindent Mais, disent-ils, il n’y a point de corps qui puisse souffrir sans pouvoir mourir. Qu’en savent-ils ? Car qui peut assurer que les démons ne souffrent pas en leur corps, quand ils avouent eux-mêmes qu’ils sont extrêmement tourmentés ? Que si l’on réplique qu’il n’y a point du moins de corps solide ou palpable, en un mot, qu’il n’y a point de chair qui puisse souffrir sans pouvoir mourir, il est vrai que l’expérience favorise cette assertion, car nous ne connaissons point de chair qui ne soit mortelle ; mais à quoi se réduit l’argumentation de nos adversaires ? à prétendre que ce qu’ils n’ont point expérimenté est impossible. Cependant, si l’on prend les choses en elles-mêmes, comment la douleur serait-elle une présomption de mort, puisqu’elle est plutôt une marque de vie ? Car l’on peut demander si ce qui souffre peut toujours vivre ; mais il est certain que tout ce qui souffre vit, et que la douleur ne se peut trouver qu’en ce qui a vie. Il est donc nécessaire que celui qui souffre vive ; et il n’est pas nécessaire que la douleur donne la mort, puisque toute douleur ne tue pas même nos corps, qui sont mortels et doivent mourir. Or, ce qui fait que la douleur tue en ce monde, c’est que l’âme est unie au corps de manière à ne pas résister aux grandes douleurs ; elle se retire donc, parce que la liaison des membres est si délicate que l’âme ne peut soutenir l’effort des douleurs aiguës. Mais, dans l’autre monde, l’âme sera tellement jointe au corps et le corps sera tel que cette union ne pourra être dissoute par aucun écoulement de temps, ni par quelque douleur que ce soit. Il est donc vrai qu’il n’y a point maintenant de chair qui puisse souffrir sans pouvoir mourir ; mais la chair ne sera pas alors telle qu’elle est, comme aussi la mort sera bien différente de celle que nous connaissons. Car il y aura bien toujours une mort, mais elle sera éternelle, parce que l’âme ne pourra, ni vivre étant séparée de Dieu, ni être délivrée par la mort des douleurs du corps. La première mort chasse l’âme du corps, malgré elle, etla seconde l’y retient malgré elle. L’une et l’autre néanmoins ont cela de commun que le corps fait souffrir à l’âme ce qu’elle ne veut pas.\par
Nos adversaires ont soin de remarquer qu’il n’y a point maintenant de chair qui puisse souffrir sans pouvoir mourir ; et ils ne prennent pas garde qu’il en arrive tout autrement dans une nature bien plus noble que la chair. Car l’esprit, qui par sa présence fait vivre et gouverne le corps, peut souffrir et ne pas mourir. Voilà un être qui a le sentiment de la douleur et qui est immortel. Or, ce que nous voyons maintenant se produire dans l’âme de chacun des hommes se produira alors dans le corps de tous les damnés. D’ailleurs, si nous voulons y regarder de plus près, nous trouvons que la douleur, qu’on appelle corporelle, appartient moins au corps qu’à l’âme ; car c’est l’âme qui souffre et non le corps, lors même que la douleur vient du corps, comme, par exemple, quand l’âme souffre à l’endroit où le corps est blessé. Et de même que nous disons que les corps sentent et vivent, quoique le sentiment et la vie du corps viennent de l’âme, de même nous disons que les corps souffrent, quoique la douleur du corps soit originairement dans l’âme. L’âme donc souffre avec le corps à l’endroit du corps où il se passe quelque chose qui la fait souffrir ; mais elle souffre seule aussi, bien qu’elle soit dans le corps, quand, par exemple, c’est une cause invisible qui l’afflige, le corps étant sain. Elle souffre même quelquefois hors du corps. Car le mauvais riche souffrait dans les enfers, quand il disait : « Je suis torturé dans cette flamme », Au contraire, le corps ne souffre point sans être animé, et du moment qu’il est animé, il ne souffre point sans avoir une âme, Si donc de la douleur à la mort, la conséquence était bonne, ce serait plutôt à l’âme de mourir, puisque c’est elle principalement qui souffre. Or, souffrant plus que le Corps, elle ne peut mourir ; comment donc conclure que les corps des damnés mourront, de ce qu’ils doivent être dans les souffrances ? Les Platoniciens ont cru que c’est de nos corps terrestres et de nos membres moribonds que les passions tirent leur origine : « Et de là, dit Virgile, nos craintes et nos désirs, nos douleurs et nos joies. » Mais nous avons établi, au quatorzième livre de cet ouvrage, que, du propre aveu des Platoniciens, les âmes, même purifiées de toute souillure, gardent un désir étrange de retourner dans des corps. Or, il est certain que ce qui est capable de désir est aussi capable de douleur, puisque le désir se tourne en douleur, lorsqu’il est frustré de son attente ou qu’il perd le bien qu’il avait acquis. Si donc l’âme ne laisse pas d’être immortelle, quoique ce soit elle qui souffre seule dans l’homme, ou du moins qui souffre le plus, il ne s’ensuit pas, de ce que les corps des damnés souffriront, qu’ils puissent mourir. Enfin, si les corps sont cause que les âmes souffrent, pourquoi ne leur causent-ils pas la mort aussi bien que la douleur, sinon parce qu’il est faux de conclure que ce qui fait souffrir doit faire mourir. Il n’y a donc rien d’incroyable à ce que ce feu puisse causer de la douleur aux corps des damnés sans leur donner la mort, puisque nous voyons que les corps mêmes font souffrir les âmes sans les tuer. Évidemment, la douleur n’est pas une présomption nécessaire de la mort.
\subsection[{Chapitre IV}]{Chapitre IV}

\begin{argument}\noindent Exemples tirés de la nature.
\end{argument}

\noindent Si donc la salamandre vit dans le feu, comme l’ont affirmé les naturalistes, si certaines montagnes célèbres de la Sicile, qui subsistent depuis tant de siècles au milieu des flammes qu’elles vomissent, sont une preuve suffisante que tout ce qui brûle ne se consume pas, comme d’ailleurs l’âme fait assez voir que tout ce qui est susceptible de souffrir ne l’est pas de mourir, pourquoi nous demande-t-on encore des exemples qui prouvent que les corps des hommes condamnés au supplice éternel pourront conserver leur âme au milieu des flammes ; brûler sans être consumés, et souffrir éternellement sans mourir ? Nous devons croire que la substance de la chair recevra cette propriété nouvelle de celui qui en a donné à tous les autres corps de si merveilleuses et que leur multitude seule nous empêche d’admirer. Car quel autre que le Dieu créateur de toutes choses a donnéà la chair du paon la propriété de ne point se corrompre après la mort ? Cela m’avait d’abord paru incroyable ; mais il arriva qu’on me servit à Carthage un oiseau de cette espèce. J’en fis garder quelques tranches prises sur la poitrine, et quand on me les rapporta après le temps suffisant pour corrompre toute autre viande, je trouvai celle-ci parfaitement saine ; un mois après, je la vis dans le même état ; au bout de l’année, elle était seulement un peu plus sèche et plus réduite. Je demande aussi qui a donné à la paille une qualité si froide qu’elle conserve la neige, et si chaude qu’elle mûrit les fruits verts.\par
Mais qui peut expliquer les merveilles du feu lui-même, qui noircit tout ce qu’il brûle, quoiqu’il soit lui-même du plus pur éclat, et qui, avec la plus belle couleur du monde, décolore la plupart des objets qu’il touche, et transforme en noir charbon une braise étincelante ? Et encore cet effet n’est-il pas régulier ; car les pierres cuites au feu blanchissent, et, bien que le feu soit rouge, il les rend blanches, tandis que le blanc s’accorde naturellement avec la lumière, comme le noir avec les ténèbres. Mais de ce que le feu brûle le bois et calcine la pierre, il ne faut pas conclure que ces effets contraires s’exercent sur des éléments contraires. Car le bois et la pierre sont des éléments différents, à la vérité, mais non pas contraires, comme le blanc et le noir. Et cependant le blanc est produit dans la pierre elle noir dans le bois par cette même cause, savoir le feu, qui rend le bois éclatant et la pierre sombre, et qui ne pourrait agir sur la pierre, s’il n’était lui-même alimenté par le bois. Que dirai-je du charbon lui-même ? N’est-ce pas une chose merveilleuse qu’il soit si fragile que le moindre choc suffit pour l’écraser, et si fort que l’humidité ne le peut corrompre, ni le temps le détruire ? C’est pourquoi ceux qui plantent des bornes mettent d’ordinaire du charbon dessous, pour le faire servir au besoin à prouver en justice à un plaideur de mauvaise foi, même après une longue suite d’années, que la borne est restée à la place convenue. Qui a pu préserver ce charbon de la corruption, dans uneterre où le bois pourrit, sinon ce feu même, qui pourtant corrompt toute chose ?\par
Considérons maintenant les effets prodigieux de la chaux. Sans répéter ce que j’ai déjà dit, que le feu la blanchit, lui qui noircit tout, n’a-t-elle pas la vertu de nourrir intérieurement le feu ? et lors même qu’elle ne nous Semble qu’une masse froide, ne voyons-nous pas que le feu est caché et comme assoupi en elle ? Voilà pourquoi nous lui donnons le nom de chaux vive, comme si le feu qu’elle recèle était l’âme invisible de ce corps. Mais ce qui est admirable, c’est qu’on l’allume quand on l’éteint. Car, pour en dégager le feu latent, on le couvre d’eau, et alors elle s’échauffe par le moyen même qui fait refroidir tout ce qui est chaud. Comme s’il abandonnait la chaux expirante, le feu caché en elle paraît et s’en va, et elle devient ensuite si froide par cette espèce de mort, que l’eau cesse de l’allumer, et qu’au lieu de l’appeler chaux vive, nous l’appelons chaux éteinte. Peut-on imaginer une chose plus étrange ? et néanmoins en voici une plus étonnante encore : au lieu d’eau, versez de l’huile sur la chaux, elle ne s’allumera point, bien que l’huile soit l’aliment du feu. Certes, si l’on nous racontait de pareils effets de quelque pierre de l’Inde, sans que nous en pussions faire l’expérience, nous n’en voudrions rien croire, ou nous serions étrangement surpris. Mais nous n’admirons pas les prodiges qui se font chaque jour sous nos yeux, non pas qu’ils soient moins admirables, mais parce que l’habitude leur ôte leur prix, comme il arrive de certaines raretés des Indes, qui, venues du bout du monde, ont cessé d’être admirées, dès qu’on a pu les admirer à loisir.\par
Bien des personnes, parmi nous, possèdent des diamants, et on en peut voir chez les orfèvres et les lapidaires. Or, on assure que cette pierre ne peut être entamée ni par le fer ni par le feu, mais seulement par du sang de bouc. Ceux qui possèdent et connaissentcette pierre l’admirent-ils comme les personnes à qui on en montre la vertu pour la première fois ? et celles qui n’ont pas vu l’expérience sont-elles bien convaincues du fait ? Si elles y croient, elles l’admirent comme une chose qu’on n’a jamais vue. Viennent-elles à faire l’expérience, l’habitude leur fait perdre insensiblement de leur admiration. Nous savons que l’aimant attire le fer, et la première fois que je fus témoin de ce phénomène, j’en demeurai vraiment stupéfait. Je voyais un anneau de fer enlevé par la pierre d’aimant, et puis, comme si elle eût communiqué sa vertu au fer, cet anneau en enleva un autre, celui-ci un troisième, de sorte qu’il y avait une chaîne d’anneaux suspendus en l’air, sans être intérieurement entrelacés. Qui ne serait épouvanté de la vertu de cette pierre, vertu qui n’était pas seulement en elle, mais qui passait d’anneau en anneau, et les attachait l’un à l’autre par un lien invisible ? Mais ce que j’ai appris par mon frère et collègue dans l’épiscopat, Sévère, évêque de Milévis, est bien étonnant. Il m’a raconté que, dînant un jour chez Bathanarius, autrefois comte d’Afrique, il le vit prendre une pierre d’aimant, et, après l’avoir placée sous une assiette d’argent où était un morceau de fer, communiquer au fer tous les mouvements que sa main imprimait à l’aimant et le faire aller et venir à son gré, sans que d’ailleurs l’assiette d’argent en reçut aucune impression. Je raconte ce que j’ai vu ou ce que j’ai entendu dire à une personne dont le témoignage est pour moi aussi certain que celui de mes propres yeux. J’ai lu aussi d’autres effets de la même pierre. Quand en place un diamant auprès, elle n’enlève plus le fer, et si déjà elle l’avait enlevé, à l’approche du diamant elle le laisse tomber. L’aimant nous vient des Indes ; or, si nous cessons déjà de l’admirer, parce qu’il nous est connu, que sera-ce des peuples qui nous l’envoient, eux qui se le procurent aisément ? Peut-être est-il chez eux aussi commun que l’est ici la chaux, que nous voyons sans étonnement s’allumer par l’action de l’eau, qui éteint le feu, et ne pas s’enflammer sous l’action de l’huile qui excite la flamme : tant ces effets nous sont devenus familiers par l’habitude !
\subsection[{Chapitre V}]{Chapitre V}

\begin{argument}\noindent Il y a beaucoup de choses dont nous ne pouvons rendre raison et qui n’en sont pas moins très certaines.
\end{argument}

\noindent Et cependant, lorsque nous parlons aux infidèles des miracles de Dieu, passés ou futurs, dont nous ne pouvons leur prouver la vérité par des exemples, ils nous en demandent la raison ; et comme nous ne saurions la leur donner, les miracles étant au-dessus de la portée de l’esprit humain, ils les traitent de fables. Qu’ils nous rendent donc raison eux-mêmes de tant de merveilles dont nous sommes ou dont nous pouvons être témoins ! S’ils avouent que cela leur est impossible, ils doivent convenir aussi qu’il ne faut pas conclure qu’une chose n’a point été ou ne saurait être, de ce qu’on n’en peut rendre raison. Sans m’arrêter à une foule de choses passées dont l’histoire fait foi, je veux seulement rapporter ici quelques faits dont on peut s’assurer sur les lieux mêmes. On dit que le sel d’Agrigente, en Sicile, fond dans le feu et pétille dans l’eau ; que chez les Garamantes il y a une fontaine si froide, le jour, qu’on n’en saurait boire, et si chaude, la nuit, qu’on n’y peut toucher. Oh en trouve une aussi dans l’Épire, où les flambeaux allumés s’éteignent et où les flambeaux éteints se rallument. En Arcadie, il y a une pierre qui, une fois échauffée, demeure toujours chaude, sans qu’on la puisse refroidir, et qu’on appelle pour cela {\itshape asbest}e. En Égypte, le bois d’un certain figuier ne surnage pas comme les autres bois, mais coule au fond de l’eau ; et, ce qui est plus étrange, c’est qu’après y avoir séjourné quelque temps, il remonte à la surface, bien qu’une fois pénétré par l’eau il dût être plus pesant. Aux environs de Sodome, la terre produit des fruits que leur apparente maturité invite à cueillir, et qui tombent en cendre sous la main ou sous la dent qui les touche. En Perse, il y a une pierre appelée {\itshape pyrite}, ainsi appelée parce qu’elle s’enflamme si on la presse fortement, et une autre nommée {\itshape sélénite}, dont la blancheur intérieure croît et diminue avec la lune. Les cavales de Cappadoce sont fécondées par le vent, et leurs poulains ne vivent pas plus de trois années. Dans l’Inde, le sol de l’île de Tylos est préféré à tous les autres, parce que les arbres n’y sont jamais dépouillés de leur feuillage.\par
Que ces incrédules qui ne veulent pas ajouter foi à l’Écriture sainte, sous prétexte qu’elle contient des choses incroyables, rendent raison, s’ils le peuvent, de toutes ces merveilles. Il n’y a aucune raison, disent-ils, qui fasse comprendre que la chair brûle sans être consumée, qu’elle souffre sans mourir. Grands raisonneurs, qui peuvent rendre raison de tout ce qu’il y a de merveilleux dans le monde ! qu’ils rendent donc raison de ce peu que je viens de rapporter. Je ne doute point que si les faits cités plus haut leur étaient restés inconnus et qu’on vînt leur dire qu’ils doivent arriver un jour, ils n’y crussent bien moins encore qu’ils ne font aux peines futures que nous leur annonçons. En effet, qui d’entre eux voudrait nous croire, si, au lieu d’affirmer que les corps des damnés vivront et souffriront éternellement dans les flammes, nous leur disions qu’il y aura un sel qui fondra au feu et qui pétillera dans l’eau, une fontaine si chaude, pendant la fraîcheur de la nuit, qu’on n’osera y toucher, et si froide, dans la grande chaleur du jour, que personne n’y voudra boire ; une pierre qui brûlera ceux qui la presseront, et une autre, qui, une fois enflammée, ne pourra s’éteindre ? Si nous annoncions toutes ces merveilles pour le siècle futur, les incrédules nous répondraient : Voulez-vous que nous y croyions ? rendez-nous-en raison. Ne faudrait-il pas alors avouer que cela n’est point en notre pouvoir, et que l’intelligence humaine est trop bornée pour pénétrer les causes de ces merveilleux ouvrages de Dieu ? Mais nous n’en sommes pas moins assurés que Dieu ne fait rien sans raison, que rien de ce qu’il veut ne lui est impossible, et nous croyons tout ce qu’il annonce, parce que nous ne pouvons croire qu’il soit menteur ou impuissant. Que répondent cependant ces détracteurs de notre foi, ces grands chercheurs de raisons, quand nous leur demandons raison des merveilles qui existent sous nos yeux et de ces prodiges que la raison naturelle ne peut comprendre, puisqu’ils semblent contraires à la nature même des choses ? Si nous les annoncions comme devant arriver, ne nous défieraient-ils pas d’en rendre raison, comme de tous les miracles que nous annonçons pour l’avenir ? Donc, puisque la raison détaille et que la parole expire devant ces ouvrages de Dieu, que nos adversaires cessent de dire qu’une chose n’est pas ou ne peut pas être parce que la raison de l’homme ne peut l’expliquer. Cela n’empêche pas les faits que nous avons cités de se produire : cela n’empêchera pas les prodiges annoncés par la foi de s’accomplir un jour.
\subsection[{Chapitre VI}]{Chapitre VI}

\begin{argument}\noindent Tous les miracles qu’on cite ne sont pas des faits naturels, mais la plupart sont des imaginations de l’homme ou des artifices des démons.
\end{argument}

\noindent Mais je les entends s’écrier : Tout cela n’est pas, nous n’en croyons rien ; ce qu’on a dit, ce qu’on a écrit sont autant de faussetés. S’il fallait y croire, il faudrait croire aussi les récits des mêmes auteurs : qu’il y a eu, par exemple, ou qu’il y a un certain temple de Vénus où l’on voit un candélabre surmonté d’une lampe qui brûle en plein air et que les vents ni les pluies ne peuvent éteindre, ce qui lui a valu, comme à la pierre dont nous parlions tout à l’heure, le nom d’{\itshape asbeste}, c’est-à-dire lumière inextinguible. — Je ne serais pas surpris que nos adversaires crussent par ce discours nous avoir fermé la bouche ; car si nous déclarons qu’il ne faut point croire à la lampe de Vénus, nous infirmons les autres merveilles que nous avons rapportées, et si nous admettons, au contraire, ce récit comme véritable, nous autorisons les divinités du paganisme. Mais, ainsi que je l’ai dit au dix-huitième livre de cet ouvrage, nous ne sommes pas obligés de croire tout ce que renferme l’histoire profane, les auteurs eux-mêmes qui l’ont écrite n’étant pas toujours d’accord, et, comme dit Varron, semblant conspirer à se contredire. Nous n’en croyons donc (et encore, si nous le jugeons à propos) que ce qui, n’est point contraire aux livres que nous devons croire, Et quant à ces merveilles de la nature dont nous nous servons pour persuader aux incrédules la vérité des merveilles à venir que la foi nous annonce, nous nous contentons de croire à celles dont nous pouvons nous-mêmes faire l’expérience, ou qu’il n’est pas difficile de justifier par de bons témoignages. Ce temple de Vénus, cette lampe qui ne peut s’éteindre, loin de nous embarrasser, nous donnerait beau jeu contre nos adversaires ; car nous la rangeons parmi tous les miracles de la magie, tant ceux que les démons opèrent par eux-mêmes que ceux qu’ils font par l’entremise des hommes. Et nous ne saurions nier ces miracles sans aller contre les témoignages de l’Écriture. Or, de trois choses l’une : ou l’industrie des hommes s’est servie de la pierre asbeste pour allumer cette lampe, ou c’est un ouvrage de la magie, ou quelque démon, sous le nom de Vénus, a produit cette merveille. En effet, les malins esprits sont attirés en certains lieux, non par des viandes, comme les animaux, mais par certains signes appropriés à leur goût, comme diverses sortes de pierres, d’herbes, de bois, d’animaux, de charmes et de cérémonies. Or, pour être ainsi attirés par les hommes, ils les séduisent d’abord, soit en leur glissant un poison secret dans le cœur, soit en nouant avec eux de fausses amitiés ; et ils font quelques disciples, qu’ils établissent maîtres de plusieurs. On n’aurait pu savoir au juste, si eux-mêmes ne l’avaient appris, quelles sont les choses qu’ils aiment ou qu’ils abhorrent, ce qui les attire ou les contraint de venir, en un mot, tout ce qui fait la science de la magie. Mais ils travaillent surtout à se rendre maîtres des cœurs, et c’est ce dont ils se glorifient le plus, quand ils essaient de se transformer en anges de lumière. Ils font donc beaucoup de choses, j’en conviens, et des choses dont nous devons d’autant plus nous défier que nous avouons qu’elles sont plus merveilleuses. Au surplus, elles-mêmes nous servent à prouver notre foi ; car si les démons impurs sont si puissants, combien plus puissants sont les saints anges ! combien aussi Dieu, qui a donné aux anges le pouvoir d’opérer tant de merveilles, est-il encore plus puissant qu’eux !\par
Qu’il soit donc admis que les créatures de Dieu produisent, par le moyen des arts mécaniques, tous ces prodiges, assez surprenantspour que ceux qui n’en ont pas le secret les croient divins, comme cette statue de fer suspendue en l’air dans un temple par des pierres d’aimant, ou comme cette lampe de Vénus citée tout à l’heure et dont peut-être tout le miracle consistait en une asbeste qu’on y avait adroitement adaptée. Si tout cela est admis comme vrai ; et si les ouvrages des magiciens, que l’Écriture appelle sorciers et enchanteurs, ont pu donner une telle renommée aux démons qu’un grand poète n’a pas hésité à dire d’une magicienne :\par
 {\itshape « Elle assure que ses enchantements peuvent à son gré délivrer les âmes ou leur envoyer de cruels soucis, arrêter le coure des fleuves et faire rétrograder les astres ; elle invoque tes mânes ténébreux ; la terre va mugir sous ses pieds et on verra les arbres descendre des montagnes… »} \par
combien est-il plus aisé à Dieu de faire des merveilles qui paraissent incroyables aux infidèles, lui qui a donné leur vertu aux pierres comme à tout le reste, lui qui a départi aux hommes le génie qui leur sert à modifier la nature en mille façons merveilleuses, lui qui a fait les anges, créatures plus puissantes que toutes les forces de la terre ! Son pouvoir est une merveille qui surpasse toutes les autres, et sa sagesse, qui agit, ordonne et permet, n’éclate pas moins dans l’usage qu’il fait de toutes choses que dans la création de l’univers.
\subsection[{Chapitre VII}]{Chapitre VII}

\begin{argument}\noindent La toute-puissance de Dieu est la raison suprême que doit faire croire aux miracles.
\end{argument}

\noindent Pourquoi donc Dieu ne pourrait-il pas faire que les corps des morts ressuscitent et que ceux des damnés soient éternellement tourmentés, lui qui a créé le ciel, la terre, l’air, les eaux et toutes les merveilles innombrables qui remplissent l’univers ? L’univers lui-même n’est-il point la plus grande et la plus étonnante des merveilles ? Mais nos adversaires, qui croient à un Dieu créateur de l’univers et qui le gouverne par le ministère des dieux inférieurs également créés de sa main, nos adversaires, dis-je, tout en se plaisant à exalter, bien loin de les méconnaître, les puissances qui opèrent divers effets surprenants (soit qu’elles agissent de heur propre gré, soit qu’on les contraigne d’agir par le moyen de certains rites ou même des invocations magiques), quand nous leur parlons de la vertumerveilleuse de plusieurs objets naturels, qui ne sont ni des animaux raisonnables, ni des esprits, ceux, par exemple, dont nous venons de faire mention, ils nous répondent : C’est leur nature ; la nature leur a donné cette propriété : ce ne sont là que les vertus naturelles des choses. Ainsi la seule raison pour laquelle le sel d’Agrigente fond dans le feu et pétille dans l’eau, c’est que telle est sa nature. Or, il semble plutôt que ce soit là un effet contre nature, puisque la nature a donné au feu, et non à l’eau, la propriété de faire pétiller le sel ; à l’eau, et non au feu, celle de le dissoudre. Mais, disent-ils, la nature de ce sel est d’être contraire au sel ordinaire. Voilà donc encore apparemment la belle explication qu’ils nous réservent de la fontaine des Garamantes, glacée dans le jour et bouillante pendant la nuit, et de cette source extraordinaire qui, froide à la main et éteignant comme toutes les autres les flambeaux allumés, allume les flambeaux éteints ; il en sera de même de la pierre asbeste, qui, sans avoir une chaleur propre, une fois enflammée, ne petit plus s’éteindre, et enfin, de tant d’autres phénomènes qu’il serait fastidieux de rappeler. Ils ont beau être contre nature, on les expliquera toujours en disant que telle est la nature des choses. Explication très courte, j’en conviens, et réponse très satisfaisante. Mais puisque Dieu est l’auteur de toutes les natures, d’où vient que nos adversaires, quand ils refusent de croire une chose que nous affirmons, sous prétexte qu’elle est impossible, ne veulent pas convenir que nous-en donnions une explication meilleure que la leur, en disant que telle est la volonté du Tout-Puissant ? car enfin Dieu n’est appelé de ce nom que parce qu’il peut faire tout ce qu’il veut. N’est-ce point lui qui a créé tant de merveilles surprenantes que j’ai rapportées, et qu’on croirait sans doute impossibles, si on ne les voyait de ses yeux, ou du moins s’il n’y en avait des preuves et des témoignages dignes de foi ? Car pour celles qui n’ont d’autres témoins que les auteurs qui les rapportent, lesquels ; n’étant pas inspirés des lumières divines, ont pu, comme tous les hommes, être induits en erreur, il est permis à chacun d’en croire ce qu’il lui plaît.\par
Pour moi, je ne veux pas qu’on croie légèrement les prodiges que j’ai rapportés, parce que je ne suis pas moi-même assure de leur existence, excepté ceux dont j’ai fait et dont chacun peut aisément faire l’expérience : ainsi, la chaux qui boue dans l’eau et demeure froide dans l’huile ; la pierre d’aimant, qui ne saurait remuer un fétu et qui enlève le fer ; la chair du paon, inaccessible à la corruption qui n’a pas épargné le corps de Platon ; la paille, si froide qu’elle conserve la neige, et si chaude qu’elle fait mûrir les fruits ; enfin le feu qui blanchit les pierres et noircit tous les autres objets. Il en est de même de l’huile qui fait des taches noires, quoiqu’elle soit claire et luisante, et de l’argent qui noircit ce qu’il touche, bien qu’il soit blanc. C’est encore un fait certain que la transformation du bois en charbon : brillant, il devient noir ; dur, il devient fragile ; sujet à corruption, il devient incorruptible. J’ai vu tous ces effets et un grand nombre d’autres qu’il est inutile de rappeler. Quant à ceux que je n’ai pas vus, et que j’ai trouvés dans les livres, j’avoue que je n’ai pu les contrôler par des témoignages certains, excepté pourtant cette fontaine où les flambeaux allumés s’éteignent et les flambeaux éteints se rallument, et aussi ces fruits de Sodome, beaux au dehors, au dedans cendre et fumée. Cette fontaine, toutefois, je n’ai rencontré personne qui m’ait dit l’avoir vue en Épire ; mais d’autres voyageurs m’ont assuré en avoir rencontré en Gaule une toute semblable, près de Grenoble. Et pour les fruits de Sodome, non seulement des historiens dignes de foi, mais une foule de voyageurs l’assurent si fermement que je n’en puis douter.\par
Je laisse les autres prodiges pour ce qu’ils sont ; je les ai-rapportés sur la foi des historiens de nos adversaires, afin de montrer avec quelle facilité on s’en rapporte à leur parole en l’absence de toute bonne raison, tandis qu’on ne daigne pas nous croire nous-mêmes quand nous annonçons des merveilles que Dieu doit accomplir, sous prétexte qu’elles sont au-dessus de l’expérience. Nous rendons pourtant, nous, raison de notre foi ; car quelle raison meilleure donner de ces merveilles qu’en disant : Le Tout-Puissant les a prédites dans les mêmes livres où il en a prédit beaucoup d’autres que nous avons vues s’accomplir ? Celui-là saura faire, selon ce qu’il a promis, des choses qu’on juge impossibles, qui a déjà promis et qui a fait que les nations incrédules croiraient des choses impossibles.
\subsection[{Chapitre VIII}]{Chapitre VIII}

\begin{argument}\noindent Ce n’est point une chose contre nature que la connaissance approfondie d’un objet fasse découvrir en lui des propriétés opposées à celles qu’on y avait aperçues auparavant.
\end{argument}

\noindent Mais, disent nos contradicteurs, ce qui nous empêche de croire que des corps humains puissent toujours brûler sans jamais mourir, c’est que nous savons que telle n’est point la nature des corps humains, au lieu que tous les faits merveilleux qui ont été rapportés tout à l’heure sont une suite de la nature des choses. Je réponds à cela que, selon nos saintes Écritures, la nature du corps de l’homme, avant le péché, était de ne pas mourir, et qu’à la résurrection des morts, il sera rétabli dans son premier état. Mais comme les incrédules ne veulent point admettre cette autorité, puisque s’ils la recevaient, nous ne serions plus en peine de leur prouver les tourments éternels des damnés, il faut produire ici quelques témoignages de leurs plus savants écrivains, qui fassent voir qu’une chose peut devenir, par la suite du temps, toute autre qu’on ne l’avait connue auparavant.\par
Voici ce que je trouve textuellement dans le livre de Varron, intitulé : {\itshape De l’origine du peuple romain} : « Il se produisit dans le ciel un étrange prodige. Castor atteste que la brillante étoile de Vénus, que Plaute appelle {\itshape Vesperugo}, et Homère {\itshape Hesperos}, changea de couleur, de grandeur, de figure et de mouvement, phénomène qui ne s’était jamais vu jusqu’alors. Adraste de Cyzique et Dion de Naples, tous deux mathématiciens célèbres, disent que cela arriva sous le règne d’Ogygès. » Varron, qui est un auteur considérable, n’appellerait pas cet accident un prodige, s’il ne lui eût semblé contre nature. Car nous disons que tous les prodiges sont contre nature ; mais cela n’est point vrai. En effet, comment appeler contraires à la nature des effets qui se font par la volonté de Dieu, puisque la volonté du Créateur fait seule la nature de chaque chose ? Les prodiges ne sont donc pas contraires à la nature, mais seulement à une certaine notion que nous avions auparavant de la nature des objets. Qui pourrait raconter la multitude innombrable de prodiges qui sont rapportés dans les auteurs profanes ? mais arrêtons-nous seulement à ce qui regarde notre sujet. Qu’y a-t-il de mieux réglé par l’auteur de la nature que le cours des astres ? qu’y a-t-il au monde qui soit établi sur des lois plus fixes et plus immuables ? Et toutefois, quand celui qui gouverne ses créatures avec un empire absolu l’a jugé convenable, une étoile, qui est remarquable entre toutes les autres par sa grandeur, par son éclat) a changé de couleur, de grandeur, de figure, et, ce qui est plus étonnant encore, de règle et de loi dans son cours. Certes, voilà un événement qui met en défaut toutes les tables astrologiques, s’il en existait déjà, et tous ces calculs des savants, si certains à leurs yeux et si infaillibles qu’ils ont osé avancer que cette métamorphose de Vénus ne s’était pas produite auparavant et ne s’est pas représentée depuis. Pour nous, nous lisons dans les Écritures que le soleil même s’arrêta au commandement de Jésus Navé, pour lui donner le temps d’achever sa victoire, et qu’il retourna en arrière pour assurer le roi Ézéchias des quinze années de vie que Dieu lui accordait ; mais quand les infidèles croient ces sortes de miracles accordés à la vertu des saints, ils les attribuent à la magie, comme je le disais tout à l’heure de cette enchanteresse de Virgile, « qui arrêtait le cours des rivières et faisait rétrograder les astres ». Nous lisons aussi dans l’Écriture que le Jourdain arrêta le cours de ses eaux et retourna en arrière, pour laisser passer le peuple de Dieu sous la conduite de Jésus Navé, et que la même chose arriva au prophète Élie et à son disciple Élisée nous y lisons aussi le miracle de la course rétrograde du soleil en faveur du roi Ézéchias. Mais ce prodige de l’étoile de Vénus, rapporté par Varron, nous ne voyons pas qu’il soit arrivé à la prière d’aucun homme.\par
Que les infidèles ne se laissent-donc point aveugler par cette prétendue connaissance de la nature des choses. Comme si Dieu n’y pouvait apporter des changements qu’ils ne connaissent pas ! et, à dire vrai, les choses lesplus ordinaires ne nous paraîtraient pas moins merveilleuses que les autres, si nous n’étions pas accoutumés à n’admirer que celles qui sont rares. Consultez la seule raison : qui n’admirera que, dans cette multitude infinie d’hommes, tous soient assez semblables les uns aux autres pour que leur nature les distingue de tous les autres animaux, et assez dissemblables pour se distinguer entre eux aisément ? Et cette différence est même encore plus admirable que leur ressemblance ; car il paraît assez naturel que des animaux d’une même espèce se ressemblent ; et pourtant, comme il n’y a pour nous de merveilleux que ce qui est rare, nous ne nous étonnons jamais plus qu’en voyant deux hommes qui se ressemblent si fort qu’on les prendrait l’un pour l’autre et qu’on s’y tromperait toujours.\par
Mais peut-être nos adversaires ne croiront-ils pas au phénomène que je viens de rapporter d’après Varron, bien que Varron soit un de leurs historiens et un très savant homme ; ou bien en seront-ils faiblement touchés, parce que ce prodige ne dura pas longtemps et que l’étoile reprit ensuite son cours ordinaire. Voici donc un autre prodige qui subsiste encore aujourd’hui, et qui, à mon avis, doit suffire pour les convaincre que, si clairement qu’ils se flattent de connaître la nature d’une chose, ce n’est pas une raison de défendre à Dieu de la transformer à son gré et de la rendre tout autre qu’ils ne la connaissaient. La terre de Sodome n’a pas toujours été ce qu’elle est aujourd’hui. Sa surface était semblable à celle des autres terres, et même plus fertile, car l’Écriture la compare au paradis terrestre. Cependant, depuis que le feu du ciel l’a touchée, l’aspect en est affreux, au témoignage même des historiens profanes, confirmé par le récit des voyageurs, et ses fruits, sous une belle apparence, ne renferment que cendre et fumée. Elle n’était pas telle autrefois, et voilà ce qu’elle est maintenant. L’auteur de toutes les natures a fait dans la sienne un changement si prodigieux qu’il dure encore, après une longue suite de siècles.\par
De même qu’il n’a pas été impossible à Dieu de créer les natures qu’il lui a plu, il ne lui est pas impossible non plus de les changer comme il lui plaît. De là vient ce nombre infini de choses extraordinaires qu’onappelle prodiges, monstres, phénomènes, et qu’il serait infiniment long de rapporter. On dit que les monstres sont ainsi nommés parce qu’ils montrent en quelque façon l’avenir, et on donne aussi aux autres mots une origine semblable. Mais que les devins prédisent ce qu’ils voudront, soit qu’ils se trompent, soit que Dieu permette en effet que les démons les inspirent pour les punir de leur curiosité et les aveugler davantage, soit enfin que les démons ne rencontrent juste que par hasard ; pour nous, nous pensons que ce qu’on appelle phénomènes contre nature, suivant une locution employée par saint Paul lui-même, quand il dit que l’olivier sauvage, enté contre nature sur le bon olivier, participe à son suc et à sa sève, nous pensons que ces phénomènes, au fond, ne sont rien moins que contre nature, et servent à Prouver clairement qu’aucun obstacle, aucune loi de la nature, n’empêchera Dieu de faire des corps des damnés ce qu’il a prédit. Or, comment l’a-t-il prédit ? c’est ce que je pense avoir montré suffisamment, au livre précédent, par les témoignages tirés de l’Ancien et du Nouveau Testament.
\subsection[{Chapitre IX}]{Chapitre IX}

\begin{argument}\noindent De la géhenne de feu et de la nature des peines éternelles.
\end{argument}

\noindent Il ne faut donc point douter que la sentence que Dieu a prononcée par son Prophète, touchant le supplice éternel des damnés, ne s’accomplisse exactement. Il est dit : « Leur ver ne mourra point, et le feu qui les brûlera ne s’éteindra point. » Et c’est pour nous faire mieux comprendre cette vérité que Jésus-Christ, quand il prescrit de retrancher les membres qui scandalisent l’homme, désignant par là les hommes mêmes que nous chérissons à l’égal de nos membres, s’exprime ainsi : « Il vaut mieux pour vous que vous entriez avec une seule main dans la vie, que d’en avoir deux et d’être jeté dans l’enfer, où leur ver ne meurt point et où le feu qui les consume ne s’éteint point. » Il en dit autant du pied : « Il vaut mieux pour vous entrer dans la vie éternelle n’ayant qu’unpied, que d’en avoir deux et d’être précipité dans l’enfer, où leur ver ne meurt point et où le feu qui les brûle ne s’éteint point. » Enfin il parle de l’œil dans les mêmes termes : « Il vaut mieux pour vous que vous entriez au royaume de Dieu n’ayant qu’un œil, que d’en avoir deux et d’être précipité dans l’enfer, où leur ver ne meurt point et où le feu qui les brûle ne s’éteint point. » Il ne s’est pas lassé de répéter trois fois la même chose au même lieu. Qui ne serait épouvanté de cette répétition et de cette menace sortie avec tant de force d’une bouche divine ?\par
Au reste, ceux qui veulent que ce ver et que ce feu ne soient pas des peines du corps, mais de l’âme, disent que les hommes séparés du royaume de Dieu seront brûlés dans l’âme jar une douleur et un repentir tardifs et inutiles, et qu’ainsi l’Écriture a fort bien pu se servir du mot feu pour marquer cette douleur cuisante d’où vient, ajoutent-ils, cette parole de l’Apôtre : « Qui est scandalisé, sans que je brûle ? » ils croient aussi que le ver figure la même douleur ; car il est écrit, disent-ils, que « comme la teigne ronge un habit, et le ver le bois, ainsi la tristesse afflige le cœur de l’homme ». Mais ceux qui ne doutent point que le corps ne soit tourmenté en enfer aussi bien que l’âme, soutiennent que le corps y sera brûlé par le feu, et l’âme rongée en quelque sorte par un ver de douleur. Bien que ce sentiment soit probable, car il est absurde de supposer que soit le corps, soit l’âme, ne souffrent pas ensemble dans l’enfer, je croirais cependant plus volontiers que le ver et le feu s’appliquent ici tous deux au corps, et non à l’âme. Je dirais donc que l’Écriture ne fait pas mention de la peine de l’âme, parce qu’elle est nécessairement impliquée dans celle du corps. En effet, on lit dans l’Ancien Testament : « Le supplice de la chair de l’impie sera le feu et le ver. » Il pouvait dire plus brièvement : « Le supplice de l’impie » ; pourquoi dit-il « le supplice de la chair de l’impie », sinon parce que le ver et le feu seront tous deux le supplice du corps ? Ou, s’il a parlé de la chair, parce que les hommes seront punis pour avoir vécu selon la chair, et tomberont dans la seconde mort que l’Apôtre a marquée ainsi : « Si vous vivez selon la chair, vousmourrez » ; que chacun choisisse, entre les deux sens, celui qu’il préfère, soit qu’il rapporte le feu au corps, et le ver à l’âme, soit qu’il les rapporte tous deux au corps. J’ai déjà montré que les animaux pouvaient vivre et souffrir dans le feu sans mourir et sans se consumer, par un miracle de la volonté de Dieu, à qui on ne saurait contester ce pouvoir sans ignorer qu’il est l’auteur de tout ce qu’on admire dans la nature. En effet, c’est lui qui a produit dans le monde et les merveilles que j’ai rappelées et tontes celles en nombre infini que j’ai passées sous silence, et ce monde enfin dont l’ensemble est plus merveilleux encore que tout ce qu’il contient. Ainsi donc, libre à chacun de choisir des deux sens celui qu’il préfère, et de rapporter le ver au corps, en prenant l’expression au propre, ou à l’âme, en prenant le sens au figuré. Quant à savoir qui a le mieux choisi, c’est ce que nous saurons mieux un jour, lorsque la science des saints sera si parfaite qu’ils n’auront pas besoin d’éprouver ces peines pour les connaître. « Car maintenant nous ne savons les choses que d’une façon partielle, jusqu’au jour où la plénitude s’accomplira. » Il suffit pour le moment de repousser cette opinion que les corps des damnés ne seront pas tourmentés par le feu.
\subsection[{Chapitre X}]{Chapitre X}

\begin{argument}\noindent Comment le feu de l’enfer, si c’est un feu corporel, pourra brûler les malins esprits, c’est-à-dire les démons qui n’ont point de corps.
\end{argument}

\noindent Ici se présente une question : si le feu de l’enfer n’est pas un feu immatériel, analogue à la doutent de l’âme, mais un feu matériel, brûlant au contact et capable de tourmenter les corps, comment pourra-t-il servir au supplice des démons qui sont des esprits ? car nous savons que le même feu doit servir de supplice aux démons et aux hommes, suivant cette parole de Jésus-Christ : « Retirez-vous de moi, maudits, et allez au feu éternel, qui a été préparé pour le diable et pour ses anges. » Il faut donc que les démons aient aussi, comme l’ont pensé de savants hommes, des corps composés de cet air grossier et humide qui se fait sentir à nous, quand il estagité par le vent. En effet, si cet élément ne pouvait recevoir aucune impression du feu, il ne deviendrait pas brûlant, lorsqu’il est échauffé dans un bain ; pour brûler, il faut qu’il soit brûlé lui-même, et il cause l’impression qu’il subit. Au surplus, si l’on veut que les démons n’aient point de corps, il est inutile de se mettre beaucoup en peine de prouver le contraire. Qui nous empêchera de dire que les esprits, même incorporels, peuvent être tourmentés par un feu corporel d’une manière très réelle, quoique merveilleuse, du moment que les esprits des hommes, qui certainement sont aussi incorporels, peuvent être actuellement enfermés dans des corps, et y sont unis alors par des liens indissolubles ? Si les démons n’ont point de corps, ils seront attachés à des feux matériels pour en être tourmentés ; non qu’ils animent ces feux de manière à former des animaux composés d’âme et de corps ; mais, comme je l’ai dit, cela se fera d’une manière merveilleuse ; et ils seront tellement unis à ces feux, qu’ils en recevront de la douleur sans leur communiquer la vie. Aussi bien, cette union même qui enchaîne actuellement les esprits aux corps, pour en faire des animaux, n’est-elle pas merveilleuse et incompréhensible à l’homme ? et cependant c’est l’homme même » Je dirais volontiers que ces esprits brûleront sans corps, comme le mauvais riche brûlait dans les enfers, quand il disait : « Je souffre beaucoup dans cette flamme » ; mais j’entends ce qu’on va m’objecter : que cette flamme était de même nature que les yeux que le mauvais riche éleva sur Lazare, que la langue qu’il voulait rafraîchir d’une goutte d’eau, et que le doigt de Lazare dont il voulait se servir pour cet office, bien que tout cela se fit dans un lieu, où les âmes n’avaient point de corps. Cette flamme qui le brûlait et cette goutte d’eau qu’il demandait étaient donc incorporelles, comme sont les choses que l’on voit en dormant ou dans l’extase, lesquelles, bien qu’incorporelles, apparaissent pourtant comme des corps. L’homme qui est en cet état, quoiqu’il n’y soit qu’en esprit, ne laisse pas de se voir si semblable à son corpsqu’il n’y peut trouver de différence. Mais cette géhenne, que l’Écriture appelle aussi un étang de feu et de soufre, sera un feu corporel, et tourmentera les corps des hommes et des démons ; ou bien, si ceux-ci n’ont point de corps, ils seront unis à ce feu, pour en souffrir de la douleur sans l’animer. Car il n’y aura qu’un feu pour les uns et pour les autres, comme l’a dit la Vérité.
\subsection[{Chapitre XI}]{Chapitre XI}

\begin{argument}\noindent S’il y aurait justice à ce que la durée des peines ne fut pas plus longue que la vie des pécheurs.
\end{argument}

\noindent Mais, parmi les adversaires de la Cité de Dieu, plusieurs prétendent qu’il est injuste de punir les péchés, si grands qu’ils soient, de cette courte vie par un supplice éternel. Comme si jamais aucune loi avait proportionné la durée de la peine à celle du crime ! Les lois, suivant Cicéron, établissent huit sortes de peines l’amende, la prison, le fouet, le talion, l’ignominie, l’exil, la mort, la servitude. Y a-t-il aucune de ces peines dont la durée se mesure à celle du crime, si ce n’est peut-être la peine du talion, qui ordonne que le criminel souffre le même mal qu’il a fait souffrir ; d’où vient cette parole de la loi : « Œil pour œil, dent pour dent. » Il est matériellement possible, en effet, que la justice arrache l’œil au criminel en aussi peu de temps qu’il l’a arraché à sa victime ; mais si la raison veut que celui qui a donné un baiser à la femme d’autrui soit puni du fouet, combien de temps ne souffrira-l-il pas pour une faute qui s’est passée en un moment ? La douceur d’une courte volupté n’est-elle pas punie en ce cas par une longue douleur ? Que dirai-je de la prison ? n’y doit-on demeurer qu’autant qu’a duré le délit qui vous y a fait condamner ? mais ne voyons-nous pas qu’un esclave demeuré plusieurs années dans les fers, pour avoir offensé son maître par une seule parole ou l’avoir blessé d’un coup dont la trace a passé en un instant ? Pour l’amende, l’ignominie, l’exil et la servitude, comme ces peines sont d’ordinaire irrévocables, ne sont-elles pas en quelquesorte semblables aux peines éternelles, eu égard à la brièveté de cette vie ? Elles ne peuvent pas être réellement éternelles, parce que la vie même où on les souffre ne l’est pas ; et toutefois des fautes que l’on punit par de si longs supplices se commettent en très peu de temps, sans que personne ait jamais cru qu’il fallût proportionner la longueur des tourments à la durée plutôt qu’à la grandeur des crimes. Se peut-il imaginer que les lois fassent consister le supplice des condamnés à mort dans le court moment que dure l’exécution ? elles le font consister à les supprimer pour jamais de la société des vivants. Or, ce qui se fait dans cette cité mortelle par le supplice de la première mort, se fera pareillement dans la cité immortelle par la seconde mort. De même que les lois humaines ne rendent jamais l’homme frappé du supplice capital à la société, ainsi les lois divines ne rappellent jamais le pécheur frappé de la seconde mort à la vie éternelle. Comment donc, dira-t-on, cette parole de votre Christ sera-t-elle vraie : « On vous mesurera selon la mesure que vous aurez appliquée aux autres », si un péché temporel est puni d’une peine éternelle ? Mais on ne prend pas garde que cette mesure dont il est parlé ici ne regarde pas le temps, mais le mal, ce qui revient à dire que celui qui aura fait le mal le subira. Au surplus, on peut fort bien entendre aussi cette parole de Jésus-Christ au sens propre, je veux dire au sens des jugements et des condamnations dont il est question en cet endroit. Ainsi, que celui qui juge et condamne injustement son prochain soit jugé lui-même et condamné justement, il est mesuré sur la même mesure, bien qu’il ne reçoive pas ce qu’il a donné : il est jugé comme il a jugé les autres ; mais la punition qu’il souffre est juste, tandis que celle qu’il avait infligée était injuste.
\subsection[{Chapitre XII}]{Chapitre XII}

\begin{argument}\noindent De la grandeur du premier péché, qui exigeait une peine éternelle pour tous les hommes, abstraction faite de la grâce du Sauveur.
\end{argument}

\noindent Mais une peine éternelle semble dure et injuste aux hommes, parce que, dans les misères de la vie terrestre, ils n’ont pas cette haute et pure sagesse qui pourrait leur faire sentir la grandeur de la prévarication primitive. Plus l’homme jouissait de Dieu, plus son crime a été grand de l’avoir abandonné, et il a mérité de souffrir un mal éternel pour avoir détruit en lui un bien qui pouvait aussi être éternel. Et, de là, la damnation de toute la masse du genre humain ; car le premier coupable a été puni avec toute sa postérité, qui était en lui comme dans sa racine. Aussi nul n’est exempt du supplice qu’il mérite, s’il n’en est délivré par une grâce qu’il ne mérite pas ; et tel est le partage des hommes que l’on voit en quelques-uns ce que peut une miséricorde gratuite, et, dans tout le reste, ce que peut une juste vengeance. L’une et l’autre ne sauraient paraître en tous, puisque, si tous demeuraient sous la peine d’une juste condamnation, on ne verrait dans aucun la miséricorde de Dieu ; et d’autre part, si tons étaient transportés des ténèbres à la lumière, on ne verrait dans aucun sa sévérité. Et s’il y en a plus de punis que de sauvés, c’est pour montrer ce qui était dû à tous. Car alors même que tous seraient enveloppés dans la vengeance, nul ne pourrait blâmer justement la justice du Dieu vengeur ; si donc un si grand nombre sont délivrés, que d’actions de grâce ne sont pas dues pour ce bienfait gratuit au divin libérateur !
\subsection[{Chapitre XIII}]{Chapitre XIII}

\begin{argument}\noindent Contre ceux qui croient que les méchants, après la mort, ne seront punis que de peines purifiantes.
\end{argument}

\noindent Les Platoniciens, il est vrai, ne veulent pas qu’une seule faute reste impunie mais ils ne reconnaissent que des peines qui servent à l’amendement du coupable, qu’elles soient infligées par les lois humaines ou par les lois divines, qu’on les souffre dès cette vie ou qu’on ait à les subir dans l’autre pour n’en avoir point souffert ici-bas ou n’en être pas devenu meilleur. De là vient que Virgile, après avoir parlé de ces corps terrestres, et de ces membres moribonds d’où viennent à l’âme :\par
{\itshape « Et ses craintes et les désirs, et ses douleurs et ses joies, enfermée qu’elle est dans une prison ténébreuse d’où elle ne peut contempler le ciel »} ; \par
Virgile ajoute :\par
 {\itshape « Et lorsqu’au dernier jour la vie abandonne les âmes, leurs misères ne sont pas finies et elles ne sont pas purifiées d’un seul coup de leurs souillures corporelles. Par une loi nécessaire, mille vices invétérés s’y attachent encore et y germent en mille façons. Elles sont donc soumises à des peines et expient dans les supplices leurs crimes passés : les unes suspendues dans le vide et livrées au souffle du vent, les autres plongées dans un abîme immense pour s’y laver de leurs souillures ou pour y être purifiées par le feu. »} \par
Ceux qui adoptent ce sentiment ne reconnaissent après la mort que des peines purifiantes ; et comme l’air, l’eau et le feu sont des éléments supérieurs à la terre, ils les font servir de moyens d’expiation pour purifier les âmes que le commerce de la terre a souillées. Aussi Virgile a-t-il employé ces trois éléments : l’air, quand il dit qu’elles sont livrées au souffle du vent ; l’eau, quand il les plonge dans un abîme immense ; le feu, quand il charge le feu de les purifier. Pour nous, nous reconnaissons qu’il y a dans cette vie mortelle quelques peines purifiantes, mais elles n’ont ce caractère que chez ceux qui en profitent pour se corriger, et non chez les autres, qui n’en deviennent pas meilleurs, ou qui n’en deviennent que pires. Toutes les autres peines, temporelles ou éternelles, que la providence de Dieu inflige à chacun par le ministère des hommes ou par celui des bons et des mauvais anges, ont pour objet, soit de punir les péchés passés ou présents, soit d’exercer et de manifester la vertu. Quand nous endurons quelque mal par la malice ou par l’erreur d’un autre, celui-là pèche qui nous cause ce mal ; mais Dieu, qui le permet par un juste et secret jugement, ne pèche pas. Les uns donc souffrent des peines temporelles en cette vie seulement, les autres après la mort ; et d’autres en cette vie et après la mort tout ensemble, bien que toujours avant le dernier jugement. Mais tous ceux qui souffrent des peines temporelles après la mort ne tombent point dans les éternelles. Nous avons déjà dit qu’il y en a à qui les peines ne sont pas remises en ce siècle et à qui elles seront remises en l’autre, afin qu’ils ne soient pas punis du supplice qui ne finit pas.
\subsection[{Chapitre XIV}]{Chapitre XIV}

\begin{argument}\noindent Des peines temporelles de cette vie, qui sont une suite de l’humaine condition.
\end{argument}

\noindent Ils sont bien rares ceux qui, dans cette vie, n’ont rien à souffrir en expiation de leurs péchés, et qui ne les expient qu’après la mort. Nous avons connu toutefois quelques personnes arrivées à une extrême vieillesse sans avoir eu la moindre fièvre, et qui ont passé leur vie dans une tranquillité parfaite. Cela n’empêche pas qu’à y regarder de près, la vie des hommes n’est qu’une longue peine, selon la parole de l’Écriture : « La vie humaine sur la terre est-elle autre chose qu’une tentation ? » La seule ignorance est déjà une grande peine, puisque, pour y échapper, on oblige les enfants, à force de châtiments, à apprendre les arts et les sciences. L’étude où on les contraint par, la punition est quelque chose de si pénible, qu’à l’ennui de l’étude ils préfèrent quelquefois l’ennui de la punition. D’ailleurs, qui n’aurait horreur de recommencer son enfance et n’aimerait mieux mourir ? Elle commence par les larmes, présageant ainsi, sans le savoir, les maux où elle nous engage. On dit cependant que Zoroastre, roi des Bactriens, rit en naissant ; mais ce prodige ne lui annonça rien de bon, car il passe pour avoir inventé la magie, qui, d’ailleurs, ne lui fut d’aucun secours contre ses ennemis, puisqu’il fut vaincu par Ninus, roi des Assyriens. Aussi nous lisons dans l’Écriture : « Un joug pesant est imposé aux enfants d’Adam, du jour où ils sortent du sein de leur mère jusqu’à celui où ils entrent dans le sein de la mère commune. » Cet arrêt est tellement inévitable, que les enfants mêmes, délivrés par le baptême du péché originel, le seul qui les rendit coupables, sont sujets à une infinité de maux, jusqu’à être tourmentés quelquefois par les malins esprits ; mais loin de nous la pensée que ces souffrances leur soient fatales, quand, par l’aggravation de la maladie, elles arrivent à séparer l’âme du corps.
\subsection[{Chapitre XV}]{Chapitre XV}

\begin{argument}\noindent La grâce de Dieu, qui nous fait revenir de la profondeur de notre ancienne misère, est un acheminement au siècle futur.
\end{argument}

\noindent Aussi bien, ce joug pesant qui a été imposé aux fils d’Adam, depuis leur sortie du sein de leur mère jusqu’au jour de leur ensevelissement au sein de la mère commune, est encore pour nous, dans notre misère, un enseignement admirable : il nous exhorte à user sobrement de toutes choses, et nous fait comprendre que cette vie de châtiment n’est qu’une suite du péché effroyable commis dans le Paradis, et que tout ce qui nous est promis par le Nouveau Testament ne regarde que la part que nous aurons à la vie future ; il faut donc accepter cette promesse comme un gage et vivre dans l’espérance, en faisant chaque jour de nouveaux progrès et mortifiant par l’esprit les mauvaises inclinations de la chair car « Dieu connaît ceux qui sont à lui » ; et « tous ceux qui sont conduits par l’esprit de Dieu sont enfants de Dieu » ; enfants par grâce, et non par nature, n’y ayant qu’un seul Fils de Dieu par nature, qui, par sa bonté, s’est fait fils de l’homme, afin que nous, enfants de l’homme par nature, nous devinssions par grâce enfants de Dieu. Toujours immuable, il s’est revêtu de notre nature pour nous sauver, et, sans perdre sa divinité, il s’est fait participant de notre faiblesse, afin que, devenant meilleurs, nous perdions ce que nous avons de vicieux et de mortel par la communication de sa justice et de son immortalité, et que nous conservions ce qu’il a mis de bon en nous dans la plénitude de sa bonté. De même que nous sommes tombés, par le péché d’un seul homme, dans une si déplorable misère, ainsi nous arrivons, par la grâce d’un seul homme, mais d’un homme-Dieu, à la possession d’un si grand bonheur. Et nul ne doit être assuré d’avoir passé du premier état au second, qu’il ne soit arrivé au lieu où il n’y aura plus de tentation, et qu’il ne possède cette paix qu’il poursuit à travers les combats que la chair livre contre l’esprit et l’esprit contre la chair. Or, une telle guerre n’aurait pas lieu, si l’homme, par l’usage de son libre arbitre, eût conservé sa droiture naturelle ; mais par son refus d’entretenir avec Dieu une paix quifaisait son bonheur, il est contraint de combattre misérablement contre lui-même. Toutefois cet état vaut mieux encore que celui où il se trouvait avant de s’être converti à Dieu : il vaut mieux combattre le vice que de le laisser régner sans combat, et la guerre, accompagnée de l’espérance d’une paix éternelle, est préférable à la captivité dont on n’espère point sortir. Il est vrai que nous souhaiterions bien de n’avoir plus cette guerre à soutenir, et qu’enflammés d’un divin amour, nous désirons ardemment cette paix et cet ordre accomplis, où les chosés d’un prix inférieur seront pour jamais subordonnées aux choses supérieures. Mais lors même, ce qu’à Dieu ne plaise, que nous n’aurions pas foi dans un si grand bien, nous devrions toujours mieux aimer ce combat, tout pénible qu’il puisse être, qu’une fausse paix achetée par l’abandon de notre âme à la tyrannie des passions.
\subsection[{Chapitre XVI}]{Chapitre XVI}

\begin{argument}\noindent Des lois de grâce qui s’étendent sur toutes les époques de la vie des hommes régénérés.
\end{argument}

\noindent Telle est la miséricorde de Dieu à l’égard des vases de miséricorde qu’il a destinés à la gloire, que la première et la seconde enfance de l’homme, l’une livrée sans défense à la domination de la chair, l’autre en qui la raison encore faible, quoique aidée de la parole, ne peut combattre les mauvaises inclinations, toutes deux ne laissent pas cependant de passer de la puissance des ténèbres au royaume de Jésus-Christ, sans même traverser le purgatoire, quand une créature humaine vient à mourir à cet âge où elle n’est pas encore capable d’accomplir les commandements de Dieu, pourvu qu’elle ait reçu les sacrements du Médiateur. Car la seule régénération spirituelle suffit pour rendre impuissante à nuire après la mort l’alliance que la génération charnelle avait contractée avec la mort. Mais quand on est arrivé à un âge capable de discipline, il faut commencer la guerre contre les vices, et s’y porter avec courage, de peur de tomber en des péchés qui méritent la damnation. Nos mauvaises inclinations sont plus faciles à surmonter, quand elles ne sont pas encore fortifiées par l’habitude ; si nous les laissons prendre empire sur nous et nousmaîtriser, la victoire est plus difficile, et on ne les surmonte véritablement que lorsqu’on le fait par amour de la véritable justice, qui ne se trouve qu’en la foi de Jésus-Christ. Car si la loi commande sans que l’esprit vienne à son secours, la défense qu’elle fait du péché ne sert qu’à en augmenter le désir ; si bien qu’on y ajoute encore par la violation de la loi. Quelquefois aussi on surmonte des vices manifestes par d’autres qui sont cachés et que l’on prend pour des vertus, quoique l’orgueil et une vanité périlleuse en soient les véritables principes. Les vices ne sont donc vraiment vaincus que lorsqu’ils le sont par l’amour de Dieu, amour que Dieu seul donne, et qu’il ne donne que par le Médiateur entre Dieu et les hommes, Jésus-Christ homme, qui a voulu participer à notre mortalité misérable pour nous faire participer à sa divinité. Or, ils sont en bien petit nombre ceux qui ont atteint l’adolescence sans commettre aucun péché mortel, sans tomber dans aucun excès, dans aucune impiété, assez heureux et assez forts pour avoir comprimé par la grâce abondante de l’esprit tous les mouvements déréglés de la convoitise. La plupart, après avoir reçu le commandement de la loi, l’ont violé, et, s’étant laissé emporter au torrent des vices, ont eu recours ensuite à la pénitence ; de la sorte, assistés de la grâce de Dieu, ils reprennent courage, et leur esprit soumis à Dieu parvient à soumettre la chair. Que celui donc qui veut se soustraire aux peines éternelles, ne soit pas seulement baptisé, mais justifié en Jésus-Christ, afin de passer véritablement de l’empire du diable sous la puissance du Sauveur. Et qu’il ne compte pas sur des peinés purifiantes, si ce n’est avant le dernier et redoutable jugement ! On ne saurait nier pourtant que le feu ; même éternel, ne fasse plus ou moins souffrir les damnés, selon la diversité de leurs crimes ; et qu’il ne doive être moins ardent pour les uns, plus ardent pour les autres, soit que son ardeur varie suivant l’énormité de la peine, soit qu’elle reste égale, mais que tous ne la sentent pas également.
\subsection[{Chapitre XVII}]{Chapitre XVII}

\begin{argument}\noindent De ceux qui pensent que nul homme n’aura à subir des peines éternelles.
\end{argument}

\noindent Il me semble maintenant à propos de combattre avec douceur l’opinion de ceux d’entre nous qui, par esprit de miséricorde, ne veulent pas croire au supplice éternel des damnés, et soutiennent qu’ils seront délivrés après un espace de temps plus ou moins long, selon la grandeur de leurs péchés. Les uns font cette grâce à tous les damnés, les autres la font seulement à quelques-uns. Origène est encore plus indulgent : il croit que le diable même et ses anges, après avoir longtemps souffert, seront à la fin délivrés de leurs tourments pour être associés aux saints anges. Mais l’Église l’a condamné justement pour cette erreur et pour d’autres encore, entre lesquelles je citerai surtout ces vicissitudes éternelles de félicité et de misère où il soumet les âmes. En cela, il se départ de cette compassion qu’il semble avoir pour les malheureux damnés, puisqu’il fait souffrir aux saints de véritables misères, en leur attribuant une béatitude où ils ne sont point assurés de posséder éternellement le bien qui les rend heureux. L’erreur de ceux qui restreignent aux damnés cette vicissitude et veulent que leurs supplices fassent place à une éternelle félicité est bien loin de celle d’Origène. Cependant, si leur opinion est tenue pour bonne et pour vraie, parce qu’elle est indulgente, elle sera d’autant meilleure et d’autant pins vraie qu’elle sera plus indulgente. Que cette source de bonté se répande donc jusque sur les anges réprouvés, au moins après plusieurs siècles de tortures. Pourquoi se répand-elle sur toute la nature humaine et vient-elle à tarir pour les auges ? Mais non, cette pitié n’ose aller aussi loin et s’étendre jusqu’au diable. Et pourtant, si un de ces miséricordieux se risquait à aller jusque-là, sa bonté n’en serait-elle pas plus grande ? mais aussi son erreur serait plus pernicieuse et plus opposée aux paroles de Dieu.
\subsection[{Chapitre XVIII}]{Chapitre XVIII}

\begin{argument}\noindent De ceux qui croient qu’aucun homme ne sera damné au dernier jugement, à cause de l’intercession des saints.
\end{argument}

\noindent D’autres encore, comme j’ai pu m’en assurer dans la conversation, sous prétexte de respecter l’Écriture, mais en effet dans leur propre intérêt, font Dieu encore plus indulgent envers les hommes. Ils avouent bien que les méchants et les infidèles méritent d’être punis, comme l’Écriture les en menace ; mais ils soutiennent que lorsque le jour du jugement sera venu, la clémence l’emportera, et que Dieu, qui est bon, rendra tous les coupables aux prières et aux intercessions des saints. Car, si les saints priaient pour eux, quand ils en étaient persécutés, que ne feront-ils point, quand ils les verront abattus, humiliés et suppliants ? Et comment croire que les saints perdent leurs entrailles de miséricorde, surtout en cet état de vertu consommée qui les met à l’abri de toutes les passions ? ou comment douter que Dieu ne les exauce, alors que leurs prières seront parfaitement pures ? L’opinion précédente, qui veut que les méchants soient à la fin délivrés de leurs tourments, allègue en leur faveur ce passage du psaume : « Dieu oubliera-t-il sa clémence ? et sa colère arrêtera-t-elle le cours de ses miséricordes ? » Mais nos nouveaux adversaires soutiennent que ce même passage favorise bien mieux encore leur opinion. La colère de Dieu, disent-ils, veut que tous ceux qui sont indignes de la béatitude éternelle souffrent un supplice éternel, mais pour permettre qu’ils en souffrent un quelconque, si court qu’il soit, ne faut-il pas que sa colère arrête le cours de ses miséricordes ? Et c’est pourtant ce que nie le Psalmiste. Car il ne dit pas : Sa colère arrêtera-t-elle longtemps le cours de ses miséricordes ? mais il dit qu’elle ne l’arrêtera nullement.\par
Si l’on répond qu’à ce compte les menaces de Dieu sont fausses, puisqu’il ne condamnera personne, ils répliquent qu’elles ne sont pas plus fausses que celle qu’il fit à Ninive de la détruire, ce qui pourtant n’arriva pas, bien qu’il l’en eût menacée sans condition. En effet, le Prophète ne dit pas : Ninive sera détruite, si elle ne se corrige et ne fait pénitence, mais il dit : « Encore quarante jours,et Ninive sera détruite. » Cette menace était donc vraie, ajoutent-ils, puisque les Ninivites méritaient ce châtiment ; mais Dieu ne l’exécuta point, parce que sa colère n’arrêta pas le cours de ses miséricordes, et qu’il se laisse fléchir à leurs cris et à leurs larmes. Si donc, disent-ils, il pardonna alors, bien que cela dût contrister son prophète, combien sera-t-il plus favorable encore, quand tous ses saints intercéderont pour des suppliants ? Objecte-t-on que l’Écriture n’a point parlé de ce pardon, c’est, à leur sens, afin d’effrayer un grand nombre de pécheurs par la crainte des supplices et de les obliger à se convertir, et aussi afin qu’il y en ait qui puissent prier pour ceux qui ne se convertiront pas. Ils ne prétendent pas néanmoins que l’Écriture n’ait rien laissé entrevoir à ce sujet. Car à quoi s’applique, disent-ils, cette parole du psaume : « Seigneur, que la douceur que vous avez cachée à ceux qui vous craignent est grande et abondante ! » Ne veut-elle pas nous faire entendre que cette douceur de la miséricorde de Dieu est cachée aux hommes pour les retenir dans la crainte ? Ils ajoutent que c’est pour cela que l’Apôtre a dit : « Dieu a permis que tous tombassent dans l’infidélité, afin de faire grâce à tous » ; montrant ainsi qu’il ne damnera personne. Toutefois ceux qui sont de cette opinion ne l’étendent pas jusqu’à Satan et à ses anges. Car ils ne sont touchés de compassion que pour leurs semblables ; et en cela ils plaident principalement leur cause, parce que, comme ils vivent dans le désordre et dans l’impiété, ils se flattent de profiter de cette impunité générale qu’ils couvrent du nom de miséricorde. Mais ceux qui l’étendent même au prince des démons et à ses satellites portent encore plus haut qu’eux la miséricorde de Dieu.
\subsection[{Chapitre XIX}]{Chapitre XIX}

\begin{argument}\noindent De ceux qui promettent l’impunité de tous leurs péchés, même aux hérétiques, à cause de leur participation au corps de Jésus-Christ.
\end{argument}

\noindent Il y en a d’autres qui ne promettent pas à tous les hommes cette délivrance des supplices éternels, mais seulement à ceux qui, ayant reçu le baptême, participent au corpsde Jésus-Christ, de quelque manière d’ailleurs qu’ils aient vécu, et en quelque hérésie, en quelque impiété qu’ils soient tombés. Et ils se fondent sur ce que le Sauveur a dit : « Voici le pain qui est descendu du ciel, afin que celui qui en mangera ne meure point. Je suis le pain descendu du ciel : si quelqu’un mange de ce pain, il vivra éternellement. » Il faut donc nécessairement, disent-ils, qu’à ce prix les hérétiques soient délivrés de la mort éternelle, et qu’ils passent quelque jour à l’éternelle félicité.
\subsection[{Chapitre XX}]{Chapitre XX}

\begin{argument}\noindent De ceux qui promettent l’indulgence de Dieu, non à tous les pêcheurs, mais a ceux qui se sont faits catholiques, dans quelques crimes et dans quelques erreurs qu’ils soient tombés par la suite.
\end{argument}

\noindent Quelques-uns ne font pas cette promesse à tous ceux qui ont reçu le baptême de Jésus-Christ et participé au sacrement de son corps, mais aux seuls catholiques, alors même d’ailleurs qu’ils vivent mal. Ceux-là, disent-ils, sont établis corporellement en Jésus-Christ, ayant mangé son corps, non pas seulement en sacrement, mais en réalité. Et comme dit l’Apôtre : « Nous ne sommes tous ensemble qu’un même pain et qu’un même corps » ; Or, bien que les catholiques tombent ensuite dans l’hérésie, ou même dans l’idolâtrie, par cela seul qu’ils ont reçu le baptême de Jésus-Christ étant dans son corps, c’est-à-dire dans l’Église catholique, et ayant mangé le corps du Sauveur, ils ne mourront point éternellement, mais ils jouiront quelque jour de l’éternelle félicité. Et la grandeur de leur impiété rendra sans doute leurs peines plus longues, mais elle ne les rendra pas éternelles.
\subsection[{Chapitre XXI}]{Chapitre XXI}

\begin{argument}\noindent De ceux qui croient au salut des catholiques qui auront persévéré dans leur foi, bien qu’ils aient très mal vécu et mérité par là le feu de l’enfer
\end{argument}

\noindent Mais d’autres, considérant cette parole de l’Écriture : « Celui qui persévérera jusqu’à la fin sera sauvé », ne promettent le salut qu’à ceux qui seront toujours demeurés dans l’Église catholique, quoiqu’ils aient d’ailleurs mal vécu. Ils disent qu’ils seront sauvés par l’épreuve du feu, en vertu de ce que dit l’Apôtre : « Personne ne peut établir d’autre fondement que celui qui est posé, savoir, Jésus-Christ. Or, on verra ce que chacun aura bâti sur ce fondement, si c’est de l’or, de l’argent et des pierres précieuses, ou du bois, du foin et de la paille ; car le jour du Seigneur le manifestera, et le feu fera connaître quel est l’ouvrage de chacun : celui dont l’ouvrage demeurera en recevra la récompense ; celui dont l’ouvrage sera brûlé en souffrira préjudice ; il ne laissera pas pourtant d’être sauvé, mais par l’épreuve du feu. » Ils disent donc qu’un chrétien catholique, quelque vie qu’il mène, a Jésus-Christ pour fondement, lequel manque à tout hérétique retranché de l’unité du corps ; et dès lors, dans quelque désordre qu’il ait vécu, comme il aura bâti sur le fondement de Jésus-Christ, bois, foin ou paille, peu importe, il sera sauvé par l’épreuve du feu, c’est-à-dire, après une peine passagère, délivré de ce feu éternel qui tourmentera les méchants au dernier jugement.
\subsection[{Chapitre XXII}]{Chapitre XXII}

\begin{argument}\noindent De ceux qui pensent que les fautes rachetées par des aumônes ne seront pas comptées au jour du jugement.
\end{argument}

\noindent J’en ai rencontré aussi plusieurs convaincus que les flammes éternelles ne seront que pour ceux qui négligent de racheter leurs péchés par des aumônes convenables, suivant cette parole de l’apôtre saint Jacques : « On jugera sans miséricorde celui qui aura été sans miséricorde. » Celui donc, disent-ils, qui aura fait l’aumône, tout en menant une vie déréglée, sera jugé avec miséricorde, si bien qu’il ne sera point puni, ou qu’il sera finalement délivré ; c’est pour cela, suivant eux, que le Juge même des vivants et des morts ne fait mention que des aumônes, lorsqu’il s’adresse à ceux qui sont à sa droite et à sa gauche. Ils prétendent aussi que cette demande que nous faisons tous les jours dans l’Oraison dominicale : « Remettez-nous nos offenses, comme nous les remettons à ceux qui nous ont offensés », doit être entendue dans le même sens. C’est faire l’aumône quede pardonner une offense. Notre-Seigneur lui-même a donné un si haut prix au pardon des injures, qu’il a dit : « Si vous pardonnez à ceux qui vous offensent, votre Père vous pardonnera vos péchés ; mais si vous ne leur pardonnez point, votre Père céleste ne vous pardonnera pas non plus. » À cette sorte d’aumône se rapporte aussi ce qui a été cité de saint Jacques, que celui qui n’aura point fait miséricorde sera jugé sans miséricorde. Notre-Seigneur n’a point distingué les grands des petits péchés, mais il a dit généralement : « Votre Père vous remettra vos péchés, si vous remettez vos offenses. » Ainsi, dans quelque désordre que vive un pécheur jusqu’à la mort, ils estiment que ses crimes lui sont remis tous les jours en vertu de cette oraison qu’il récite tous les jours, pourvu qu’il se souvienne de pardonner de bon cœur les offenses à qui lui en demande pardon. — Pour moi, je vais, avec l’aide de Dieu, réfuter toutes ces erreurs, et je mettrai fin à ce vingt-unième livre.
\subsection[{Chapitre XXIII}]{Chapitre XXIII}

\begin{argument}\noindent Contre ceux qui prétendent que ni les supplices du diable, ni ceux des hommes pervers ne seront éternels.
\end{argument}

\noindent Et premièrement, il faut s’enquérir et savoir pourquoi l’Église n’a pu souffrir l’opinion de ceux qui promettent au diable le pardon, même après de très grands et de très longs supplices. Car tant de saints si versés dans le Nouveau et dans l’Ancien Testament n’ont envié la béatitude à personne ; mais c’est qu’ils ont vu qu’ils ne pouvaient anéantir ni infirmer cet arrêt que le Sauveur déclare qu’il prononcera au jour du jugement : « Retirez-vous de moi, maudits, et allez dans le feu éternel préparé pour le diable et pour ses anges. » Ces paroles montrent clairement que le diable et ses anges brûleront dans le feu éternel, et c’est aussi ce qui résulte de ce passage de l’Apocalypse : « Le diable qui les séduisait fut jeté dans un étang de feu et de soufre, avec la bête et le faux prophète, et ils y seront tourmentés jour et nuit, dans les siècles des siècles. » L’Écriture disait tout à l’heure : « Le feu éternel » ; elle dit maintenant : « Pendant les siècles des siècles » : expressionssynonymes pour désigner une durée sans fin. Il n’y a donc pas à chercher d’autre raison, de raison plus juste et plus évidente que celle-là de cette croyance fixe et immuable de la véritable piété, qu’il n’y aura plus de retour à la justice et à la vie des saints pour le diable et pour ses anges. Cela sera ainsi, parce que l’Écriture qui ne trompe personne, dit que Dieu ne les a point épargnés, mais qu’il les a jetés dans les ténébreuses prisons de l’enfer, pour y être gardés jusqu’au dernier jugement, après lequel ils seront précipités dans le feu éternel et tourmentés durant les siècles des siècles. Et maintenant, comment prétendre que tous les hommes, ou même quelques-uns, seront délivrés de cette éternité de peines, après quelques longues souffrances que ce puisse être, sans porter atteinte à la foi qui nous fait croire que le supplice des démons sera éternel ? En effet, si parmi ceux à qui l’on dira : « Retirez-vous de moi, maudits, et allez au feu éternel préparé pour le diable et pour ses anges », il en est qui ne doivent pas toujours demeurer dans ce feu, pourquoi voudrait-on que le diable et ses anges y demeurassent éternellement ? Est-ce que la sentence que Dieu prononcera contre les anges et contre les hommes ne sera vraie que pour les anges ? Oui, si les conjectures des hommes l’emportent sur la parole de Dieu. Mais comme cela est absurde, ceux qui veulent se garantir du supplice éternel ne doivent pas perdre leur temps à disputer contre Dieu, mais accomplir ses commandements, tandis qu’il en est encore temps. D’ailleurs, quelle apparence y a-t-il d’entendre par ces mots : Supplice éternel, un feu qui doit durer longtemps, et, par vie éternelle, une vie qui doit durer toujours, alors que Jésus-Christ, au même lieu, et sans distinction, ni intervalle, a dit : « Ceux-ci iront au supplice éternel, et les justes dans la vie éternelle. » Si les deux destinées sont éternelles, on doit entendre ou que toutes deux dureront longtemps, mais pour finir un jour, ou que toutes deux dureront toujours, pour ne finir jamais. Car les deux choses sont corrélatives : d’un côté, le supplice éternel, de l’autre, la vie éternelle ; de sorte qu’on ne peut prétendre sans absurdité qu’une seule et même expression caractérise une vie éternelle qui n’aurait point de fin, et un suppliceéternel qui en aurait une. Puis donc que la vie éternelle des saints ne finira point, il en sera de même du supplice éternel des démons.
\subsection[{Chapitre XXIV}]{Chapitre XXIV}

\begin{argument}\noindent Contre ceux qui pensent qu’au jour du jugement Dieu pardonnera a tous les méchants sur l’intercession des saints.
\end{argument}

\noindent Or, ce raisonnement est aussi concluant contre ceux qui, dans leur propre intérêt, tâchent d’infirmer, les paroles de Dieu, sous prétexte d’une plus grande miséricorde, et qui prétendent que les paroles de l’Écriture sont vraies, non parce que les hommes doivent souffrir les peines dont il les a menacés, mais parce qu’ils méritent de les souffrir. Dieu se laissera fléchir, disent-ils, à l’intercession des saints, qui, priant alors d’autant plus pour leurs ennemis que leur sainteté sera plus grande, en obtiendront plus aisément le pardon. — Mais pourquoi donc, si leurs prières sont si efficaces, ne les emploieraient-ils pas de même pour les anges à qui le feu éternel est préparé, afin que Dieu révoque son arrêt contre eux et les préserve de ces flammes ? Quelqu’un sera-t-il assez hardi pour aller jusque-là et dire que les saints anges se joindront aux saints hommes, devenus égaux aux anges de Dieu, afin d’intercéder pour les anges et pour les hommes condamnés, et d’obtenir que la miséricorde de Dieu les dérobe aux vengeances de sa justice ? Voilà ce qu’aucun catholique n’a dit et ne dira jamais. Autrement il n’y a plus de raison pour que l’Église ne prie pas même dès maintenant pour le diable et pour ses anges, puisque Dieu, qui est son maître, lui a commandé de prier pour ses ennemis. La même raison donc qui empêche maintenant l’Église de prier pour les mauvais anges qu’elle sait être ses ennemis, l’empêchera alors de prier pour les hommes destinés aux flammes éternelles. Car maintenant elle prie pour les hommes qui sont ses ennemis, parce que c’est encore, le temps d’une pénitence utile. En effet, que demande-t-elle à Dieu pour eux, sinon, comme dit l’Apôtre : « Qu’ils fassent pénitence et qu’ils sortent des pièges du diable qui les tient captifs et en dispose à son gré » ? Que si l’Église connaissait ès à présent ceux qui sont prédestinés à aller avec le diable dansle feu éternel, elle prierait aussi peu pour eux que pour lui. Mais, comme elle n’en est pas assurée, elle prie pour tous ses ennemis qui sont ici-bas, quoiqu’elle ne soit pas exaucée pour tous. Car elle n’est exaucée que pour ceux qui, bien que ses ennemis, sont prédestinés à devenir ses enfants par le moyen de ses prières. Mais prie-t-elle pour les âmes de ceux qui meurent dans l’obstination et qui n’entrent point dans son sein ? Non, et pourquoi cela, sinon parce qu’elle compte déjà au nombre des complices du diable ceux qui pendant cette vie ne sont pas amis de Jésus-Christ ?\par
C’est donc, je le répète, la même raison qui empêche maintenant l’Église de prier pour les mauvais anges qui l’empêchera alors de prier pour les hommes destinés au feu éternel. Et c’est encore pour la même raison que tout en priant maintenant pour les morts en général, elle ne prie pas pourtant pour les méchants et les infidèles qui sont morts. Car, parmi les hommes qui meurent, il en est pour qui les prières de l’Église ou de quelques personnes pieuses sont exaucées ; mais ce sont-ceux qui ayant été régénérés en Jésus-Christ, n’ont pas assez mal vécu pour qu’on les juge indignes de cette assistance, ni assez bien pour qu’elle ne leur soit pas nécessaire. Il s’en trouvera aussi, après la résurrection des morts, à qui Dieu fera miséricorde et qu’il n’enverra point dans le feu éternel, à condition qu’ils auront souffert les peines que souffrent les âmes des trépassés. Car il ne serait pas vrai de dire de quelques-uns, qu’il ne leur sera pardonné ni en cette vie, ni dans l’autre, s’il n’y en avait à qui Dieu ne pardonne point en cette vie, mais à qui il pardonnera dans l’autre. Donc, puisque le Juge des vivants et des morts a dit : « Venez, vous que mon Père a bénis, prenez possession du royaume qui vous a été préparé dès la naissance du monde » ; et aux autres au contraire : « Retirez-vous de moi, maudits, et allez au feu éternel préparé pour le diable et ses anges » ; et : « Ceux-ci iront au supplice éternel et les justes à la vie éternelle », il y a trop de présomption à prétendre que le supplice ne sera éternel pour aucun de ceux que Dieu envoie au supplice éternel, et ce serait donner lieu de désespérer ou de douter de la vie éternelle.\par
Que personne n’explique donc ces paroles dupsaume : « Dieu oubliera-t-il sa clémence ? et sa colère arrêtera-t-elle le cours de ses miséricordes ? » comme si la sentence de Dieu était vraie à l’égard des bons et fausse à l’égard des méchants, ou vraie à l’égard des hommes de bien et des mauvais anges, et fausse à l’égard des hommes méchants. Ce que dit le psaume se rapporte aux vases de miséricorde et aux enfants de la promesse, du nombre desquels était ce prophète même qui, après avoir dit : « Dieu oubliera-t-il sa clémence ? et sa colère arrêtera-t-elle le cours de ses miséricordes ? » ajoute aussitôt : « Et j’ai dit : Je commence ; ce changement est un coup de la droite du Très-Haut » ; par où il explique sans doute ce qu’il venait de dire : « Sa colère arrêtera-t-elle le cours de ses miséricordes ? » Car cette vie mortelle où l’homme est devenu semblable à la vanité, et où ses jours passent comme une ombre, est un effet de la colère de Dieu. Et cependant, malgré cette colère, il n’oublie pas de montrer sa miséricorde, en faisant lever son soleil sur les bons et sur les méchants, et pleuvoir sur les justes et sur les injustes. Ainsi sa colère n’arrête pas le cours de ses miséricordes, surtout en ses changements dont parle la suite du psaume : « Je commence ; ce changement est un coup de la droite du Très-Haut. » Quelque misérable, en effet, que soit cette vie, Dieu ne laisse pas d’y changer en mieux les vases de miséricorde ; non que sa colère ne subsiste toujours au milieu de cette malheureuse corruption, mais elle n’arrête pas le cours de sa bonté. Et puisque la vérité du divin cantique se trouve ainsi accomplie, il n’est pas besoin d’en étendre le sens au châtiment de ceux qui n’appartiennent pas à la Cité de Dieu. Si donc l’on persiste à l’interpréter de la sorte, qu’on fasse du moins consister la miséricorde divine, non à préserver les damnés de ces peines ou à les en délivrer, mais à les leur rendre plus légères qu’ils ne le méritent : sentiment que je ne prétends pas d’ailleurs établir, me bornant à ne le point rejeter.\par
Quant à ceux qui ne voient qu’une menace au lieu d’un arrêt effectif dans ces paroles : « Retirez-vous de moi, maudits, et allez au feu éternel » ; et dans cet autre passage : « Ceux-ci iront au supplice éternel » ; et encore dans celui-ci : « Ils seront tourmentés dans les siècles des siècles » ; et enfin dans cet endroit : « Leur ver ne mourra point, et le feu qui les brûlera ne s’éteindra point » ; ce n’est pas moi qui les combats et qui les réfute, c’est l’Écriture sainte. En effet, les Ninivites ont fait pénitence en cette vie ; et cela leur a été utile, parce qu’ils ont semé dans ce champ où Dieu a voulu qu’on semât avec larmes pour y moissonner plus tard avec joie. Qui peut nier toutefois que la prédiction de Dieu n’ait été accomplie, à moins de ne pas considérer assez comment Dieu détruit les pécheurs non seulement quand il est en colère contre eux, mais aussi quand il leur fait miséricorde ? Il les détruit de deux manières : ou comme les habitants de Sodome, en punissant les hommes mêmes pour leurs péchés, ou comme les habitants de Ninive, en détruisant les péchés des hommes par la pénitence. Ce que Dieu avait annoncé est donc arrivé : la mauvaise Ninive a été renversée, et elle est devenue bonne, ce qu’elle n’était pas ; et, bien que ses murs et ses maisons soient demeurés debout, elle a été ruinée dans ses mauvaises mœurs. Ainsi, quoique le Prophète ait été contristé de ce que les Ninivites n’avaient pas ressenti l’effet qu’ils appréhendaient de ses menaces et de ses prédictions, néanmoins ce que Dieu avait prévu arriva, parce qu’il savait bien que cette prédiction devait être accomplie dans un plus favorable sens.\par
Mais afin que ceux que la miséricorde égare comprennent quelle est la portée de ces paroles de l’Écriture : « Seigneur, que la douceur que vous avez cachée à ceux qui vous craignent est grande et abondante ! » qu’ils lisent ce qui suit : « Mais vous l’avez consommée en ceux qui espèrent en vous. » Qu’est-ce à dire sinon que la justice de Dieu n’est pas douce à ceux qui ne le servent que par la crainte du châtiment, comme font ceux qui veulent établir leur propre justice en la fondant sur la loi ? Ne connaissant pas en effet la justice de Dieu, ils ne la peuvent goûter. Ils mettent leur espérance en eux-mêmes, au lieu de la mettre en lui ; aussil’abondance de la douceur de Dieu leur est cachée ; parce que, s’ils craignent Dieu c’est de cette crainte servile qui n’est point accompagnée d’amour, car l’amour parfait bannit la crainte. Dieu a donc consommé sa douceur en ceux qui espèrent en lui ; il l’a consommée en leur inspirant son amour, afin qu’étant remplis d’une crainte, chaste que l’amour ne bannit pas, mais qui demeure éternellement, ils ne s’en glorifient que dans le Seigneur. En effet, la justice de Dieu, c’est Jésus-Christ « qui nous a été donné de Dieu pour être notre sagesse, notre justice, notre sanctification et notre rédemption, afin que, comme il est écrit, celui qui se glorifie, se glorifie dans le Seigneur ». Cette justice de Dieu, qui est un don de la grâce et non l’effet de nos mérites, n’est pas connue de ceux qui, voulant établir leur propre justice, ne sont point soumis à la justice de Dieu, qui est Jésus-Christ. C’est dans cette justice que se trouve l’abondance de la douceur de Dieu. De là vient cette parole du psaume : « Goûtez et voyez combien le Seigneur est doux ! » En ce pèlerinage, nous le goûtons plutôt que nous ne pouvons nous en rassasier, ce qui excite plus fortement encore la faim et la soit que nous en avons, jusqu’au jour où nous le verrons tel qu’il est, et où cette parole du Psalmiste sera accomplie : « Je serai rassasié, quand votre gloire paraîtra. » C’est ainsi que Jésus-Christ consomme l’abondance de sa douceur en ceux qui espèrent en lui. Or, si Dieu cache à ceux qui le craignent l’abondance de cette douceur dans le sens où l’entendent nos adversaires, c’est-à-dire afin que la peur d’être damnés engage les impies à bien vivre, de sorte qu’il puisse y avoir des fidèles qui prient pour leurs frères qui vivent mal, comment alors Dieu a-t-il consommé sa douceur en ceux qui espèrent en lui, puisque, selon ces rêveries, c’est par cette douceur même qu’il ne doit pas damner ceux qui n’espèrent pas en lui ? Que le chrétien cherche donc cette douceur que Dieu consomme en ceux qui espèrent en lui, et non celle qu’on s’imagine qu’il consommera en ceux qui le méprisent et le blasphèment ; car c’est en vain qu’on cherche en l’autre vie ce qu’on a négligé d’acquérir en celle-ci. Cette parole de l’Apôtre : « Dieu a permisque tous tombassent dans l’infidélité, afin de faire miséricorde à tous », ne veut pas dire que Dieu ne damnera personne, et, après ce qui précède, le sens en est assez clair. Quand saint Paul écrit aux païens convertis, il leur dit, à propos des Juifs qui devaient se convertir dans la suite : « De même qu’autrefois vous n’aviez point foi en Dieu, et que maintenant vous avez obtenu miséricorde, tandis que les Juifs sont demeurés incrédules, ainsi les Juifs n’ont pas cru pendant que vous avez obtenu « miséricorde, afin qu’un jour ils l’obtiennent eux-mêmes. » Puis il ajoute ces paroles, dont ceux-ci se servent pour le tromper : « Car Dieu a permis que tous tombassent dans l’infidélité, afin de faire grâce à tous. » Qui donc tous, sinon ceux dont il parlait, c’est-à-dire vous et eux ? Dieu a donc laissé tomber dans l’infidélité tous les Gentils et tous les Juifs qu’il a connus et prédestinés pour être conformes à l’image de son fils, afin que, se repentant de leur infidélité et ayant recours à la miséricorde de Dieu, ils pussent s’écrier comme le Psalmiste : « Seigneur, que la douceur que vous avez cachée à ceux qui vous craignent est grande et abondante ! mais vous l’avez consommée en ceux qui espèrent, non en eux-mêmes, mais en vous. » Il fait donc miséricorde à tous les vases de miséricorde. Qu’est-ce à dire à tous ? évidemment, à ceux qu’il a prédestinés, appelés, justifiés et glorifiés d’entre les Gentils et d’entre les Juifs ; c’est de tous ces hommes, et non de tous les hommes, que nul ne sera damné.
\subsection[{Chapitre XXV}]{Chapitre XXV}

\begin{argument}\noindent Si ceux d’entre les hérétiques qui ont été baptisés, et qui sont devenus mauvais par la suite en vivant dans le désordre, et ceux qui, régénérés par la foi catholique, ont passé ensuite à l’hérésie et au schisme, et enfin ceux qui, sans renier la foi catholique, ont persisté dans le désordre, si tous ceux-là pourront échapper au supplice éternel par l’effet des sacrements.
\end{argument}

\noindent Répondons maintenant à ceux qui promettent la remise du feu éternel, non au diable et à ses anges, non à tous les hommes, mais seulement à ceux qui, ayant reçu le baptêmede Jésus-Christ, ont participé à son corps et à son sang, de quelque manière qu’ils aient vécu, et en quelque hérésie, en quelque impiété qu’ils soient tombés. L’Apôtre les réfute, lorsqu’il dit : « Les œuvres de la chair sont aisées à connaître, comme la fornication, l’impureté, l’impudicité, l’idolâtrie, les empoisonnements, les inimitiés, les contentions, les jalousies, les animosités, les divisions, les hérésies, l’envie, l’ivrognerie, la débauche, et autres crimes, dont je vous ai déjà dit et dont je vous dis encore, que ceux qui les commettent ne posséderont point le royaume de Dieu. » Cette menace de saint Paul est vaine, si des hommes qui ont commis ces crimes possèdent le royaume de Dieu, quelques souffrances qu’ils aient pu endurer auparavant. Mais comme cette menace a pour fondement la vérité, il s’ensuit qu’ils ne le posséderont point. Or, s’ils ne possèdent jamais le royaume de Dieu, ils seront condamnés au supplice éternel ; car il n’y a point de milieu entre le royaume de Dieu et l’enfer.\par
Il faut donc voir comment on doit entendre ce que dit Notre-Seigneur : « Voici le pain qui est descendu du ciel, afin que quiconque en mange ne meure point. Je suis le pain vivant descendu du ciel : si quelqu’un mange de ce pain, il vivra éternellement. » Les adversaires à qui nous aurons tout à l’heure à répondre, et qui ne promettent pas le pardon à tous ceux qui auront reçu le baptême et le corps de Jésus-Christ, mais seulement aux catholiques, quoiqu’ayant mal vécu, réfutent eux-mêmes ceux à qui nous répondons maintenant. Il ne suffit pas, disent-ils, pour être sauvé, d’avoir mangé le corps de Jésus-Christ sous la forme du sacrement, il faut l’avoir mangé en effet, il faut avoir été véritablement partie de son corps, dont l’Apôtre dit : « Nous ne sommes tous ensemble qu’un même pain et qu’un même corps. » Il n’y a donc que celui qui est dans l’unité du corps de Jésus-Christ, de ce corps dont les fidèles ont coutume de recevoir le sacrement à l’autel, c’est-à-dire membre de l’Église, dont on puisse dire qu’il mange véritablement le corps de Jésus-Christ et qu’il boit son sang. Ainsi les hérétiques et les schismatiques qui sont séparés de l’unité de ce corps peuvent bien recevoirle même sacrement, mais sans fruit, et même avec dommage, pour être condamnés plus sévèrement, et non pour être un jour délivrés ; car ils ne sont pas dans le lien de paix représenté par ce sacrement.\par
Mais, d’autre part, ces derniers interprètes, qui ont raison de soutenir que celui-là qui ne mange pas le corps de Jésus-Christ n’est pas dans le corps de Jésus-Christ, ont tort de promettre la délivrance des peines éternelles à ceux qui sortent de l’unité de ce corps pour se jeter dans l’hérésie ou dans l’idolâtrie. D’abord, il n’est pas supportable que ceux qui, sortant de l’Église catholique, ont formé des hérésies détestables, soient dans une condition meilleure que ceux qui, n’ayant jamais été catholiques, sont tombés dans les pièges des hérésiarques. Un déserteur est un ennemi de la foi pire que celui qui ne l’a jamais abandonnée, ne l’ayant jamais reçue. En second lieu, l’Apôtre réfute cette opinion, lorsqu’après avoir énuméré les œuvres de la chair, il ajoute : « Ceux qui commettent ces crimes ne posséderont pas le royaume de Dieu. »\par
C’est pourquoi ceux qui vivent dans le désordre, et qui, d’ailleurs, persévèrent dans la communion de l’Église, ne doivent pas se croire en sûreté, sous prétexte qu’il est dit : « Celui qui persévérera jusqu’à la fin sera sauvé. » Par leur mauvaise vie, en effet, ils abandonnent la justice qui donne la vie, et qui n’est autre que Jésus-Christ, soit en pratiquant la fornication, soit en déshonorant leur corps par d’autres impuretés que l’Apôtre n’a pas voulu nommer, soit enfin en commettant quelqu’une de ces œuvres dont il est dit : « Ceux qui les commettront ne posséderont pas le royaume de Dieu. » Or, ne devant pas être dans le royaume de Dieu, ils seront inévitablement dans le feu éternel. On ne peut pas dire, du moment qu’ils ont persévéré dans le désordre jusqu’à la fin de leur vie, qu’ils aient persévéré en Jésus-Christ jusqu’à la fin, puisque persévérer en Jésus-Christ, c’est persévérer dans la foi. Or, cette foi, selon la définition du même apôtre, opère par amour, et l’amour, comme il le dit encore ailleurs, ne fait point le mal. Il ne faut donc pas dire que ceux-ci même mangent le corps de Jésus-Christ, puisqu’ils ne doivent pas être comptés comme membres du corpsde Jésus-Christ. À part les autres raisons, ils ne sauraient être tout ensemble les membres de Jésus-Christ et les membres d’une prostituée. Enfin, lorsque Jésus-Christ lui-même dit : « Celui qui mange ma chair et boit mon sang demeure en moi, et moi en lui », il fait bien voir ce que c’est que manger son corps et boire son sang en vérité, et non pas seulement sous la forme du sacrement c’est demeurer en Jésus-Christ, afin que Jésus-Christ demeure aussi en nous. Comme s’il disait : Que celui qui ne demeure point en moi, et en qui je ne demeure point, ne prétende pas manger mon corps, ni boire mon sang. Ceux-là donc ne demeurent point en Jésus-Christ qui ne sont pas ses membres : or, ceux-là ne sont pas ses membres qui se font les membres d’une prostituée, à moins qu’ils ne renoncent au mal par la pénitence, et qu’ils reviennent au bien par cette réconciliation.
\subsection[{Chapitre XXVI}]{Chapitre XXVI}

\begin{argument}\noindent Ce qu’il faut entendre par ces paroles : être sauvé comme par le feu et avoir Jésus-Christ pour fondement.
\end{argument}

\noindent Mais les chrétiens catholiques, disent-ils, ont pour fondement Jésus-Christ, de l’unité duquel ils ne se sont pas séparés, quelque mauvaise vie qu’ils aient menée, c’est-à-dire quoiqu’ils aient bâti sur ce fondement une très mauvaise vie, comparée par l’Apôtre au bois, au foin, à la paille’. La vraie foi, qui fait qu’ils ont eu Jésus-Christ pour fondement, pourra les délivrer finalement de l’enfer, non toutefois sans qu’il y ait pour eux quelque punition, puisqu’il est écrit que ce qu’ils auront bâti sera brûlé. — Que l’apôtre saint Jacques leur réponde en peu de mots : « Si quelqu’un dit qu’il a la foi, et qu’il n’ait point les œuvres, la foi pourra-t-elle le sauver ? » Ils insistent et demandent quel est donc celui dont l’apôtre saint Paul dit : « Il ne laissera pas pourtant d’être sauvé, mais comme par le feu. » Voyons ensemble quel est celui-là ; mais toujours est-il très certain que ce n’est pas celui dont parle saint Jacques. Autrement ce serait mettre en opposition deux apôtres, puisque l’un dirait qu’encore qu’un homme ait de mauvaises œuvres, la foi ne le sauvera pas du feu, et l’autre : que la foi ne pourra sauver celui qui n’aura pas de bonnes œuvres.\par
Nous saurons quel est celui qui peut être sauvé par le feu, si nous connaissons auparavant ce que c’est que d’avoir Jésus-Christ pour fondement. Or, cette image même nous l’enseigne ; car il suffit de considérer que dans un édifice rien ne précède le fondement. Quiconque donc a de telle sorte Jésus-Christ dans le cœur, qu’il ne lui préfère point les choses terrestres et temporelles, pas même celles dont l’usage est permis, celui-là a Jésus-Christ pour fondement. Mais s’il lui préfère ces choses, bien qu’il semble avoir la foi de Jésus-Christ, il n’a pas Jésus-Christ pour fondement. Combien moins l’a-t-il donc, alors que, méprisant ses commandements salutaires, il ne songe qu’à satisfaire, ses passions ? Ainsi, quand un chrétien aime une femme de mauvaise vie, et, s’attachant à elle, devient un même corps avec elle, il n’a point Jésus-Christ pour fondement. Mais quand il aime sa femme légitime selon Jésus-Christ, qui doute qu’il ne puisse avoir Jésus-Christ pour fondement ? S’il l’aime selon le monde et charnellement, comme les Gentils qui ne connaissent pas Dieu, l’Apôtre lui permet encore cela par condescendance, ou plutôt c’est Jésus-Christ qui le lui permet. Dès lors il peut encore avoir Jésus-Christ pour fondement, puisque, s’il ne lui préfère point son amour et son plaisir, s’il bâtit sur ce fondement du bois, du foin et de la paille, il ne laissera pas d’être sauvé par le feu. Les afflictions, comme un feu, brûleront ses délices et ses amours, qui ne sont pas criminelles, à cause du mariage. Ce feu figure donc les veuvages, les pertes d’enfants, et toutes les autres calamités qui emportent ou traversent les plaisirs terrestres. Ainsi cet édifice fera tort à celui qui l’aura construit, parce qu’il n’aura pas ce qu’il a édifié, et qu’il sera affligé de la perte des choses dont la jouissance le charmait. Mais il sera sauvé par le feu à cause du fondement, parce que, si un tyran lui proposait le choix, il ne préférerait pas ces choses à Jésus-Christ. Voyez dans les écrits de l’Apôtre un homme qui édifie sur ce fondement de l’or, de l’argent et des pierres précieuses : « Celui, dit-il, qui n’a point de femme pense aux choses de Dieu et à plaire à Dieu. » Voyez-en un autre maintenant quiédifie du bois, du foin et de la paille : « Mais celui, dit-il, qui a une femme pense aux choses du monde et à plaire à sa femme. — On verra quel est l’ouvrage de chacun car le jour du Seigneur le fera connaître » ; entendez le jour d’affliction ; « car », ajoute l’Apôtre, « il sera manifesté par le feu ». Il donne ici à l’affliction le nom de feu, au même sens où il est dit ailleurs dans l’Écriture : « La fournaise ardente éprouve les vases du potier, et l’affliction les hommes justes. » Et encore : « Le feu découvrira quel est l’ouvrage de chacun. Celui dont l’ouvrage demeurera (car les pensées de Dieu et le soin de lui plaire demeurent) recevra récompense pour ce qu’il aura édifié » ; ce qui veut dire qu’il recueillera le fruit de ses pensées et de ses afflictions. « Mais celui dont l’ouvrage sera brûlé en souffrira la perte », parce qu’il avait aimé. « Il ne laissera pas pourtant d’être sauvé », parce qu’aucune affliction ne l’a séparé de ce fondement ; « mais comme par le feu » ; car il ne perdra pas sans douleur ce qu’il possédait avec affection. Nous avons trouvé, ce me semble, un feu qui ne damne aucun des deux hommes dont nous parlons, mais qui enrichit l’un, nuit à l’autre, et les éprouve tous deux.\par
Mais si nous voulons entendre dans le même sens le feu dont Notre-Seigneur dit à ceux qui sont à sa gauche : « Retirez-vous de moi, maudits, et allez au feu éternel » ; en sorte que nous embrassions dans cet arrêt ceux qui bâtissent sur le fondement du bois, du foin, de la paille, et que nous prétendions qu’ils sortiront du feu par la vertu de ce fondement, après avoir été tourmentés pendant quelque temps pour leurs péchés, que devons-nous penser de ceux qui sont à la droite de Jésus-Christ et à qui il dit : « Venez, vous que mon Père a bénis, prenez possession du royaume qui vous est préparé », sinon que ce sont ceux qui ont bâti sur le fondement de l’or, de l’argent et des pierres précieuses ? Si donc par le feu dont parle l’Apôtre, quand il dit : « Comme par le feu », nous entendons le feu d’enfer, il faudra dire que les uns et les autres, c’est-à-dire ceux qui sont à la droite et ceux qui sont à la gauche, y seront également envoyés. Le feu dont il est dit : « Le jour du Seigneur manifestera quel est l’ouvrage de chacun et le fera connaître », ce feu éprouvera les uns et les autres ; et par conséquent ce n’est pas le feu éternel, puisque celui dont l’ouvrage demeurera, c’est-à-dire ne sera pas consumé par ce feu, recevra récompense pour ce qu’il aura édifié, et que celui dont l’ouvrage sera brûlé trouvera son châtiment dans son regret. Ceux-là seuls qui seront à la gauche seront envoyés au feu éternel par une suprême et éternelle condamnation, au lieu que le feu dont parle saint Paul au passage cité éprouve ceux qui sont à la droite. Mais il les éprouve de telle sorte qu’il ne brûle point l’édifice des uns et brûle celui des autres, sans que cela empêche ces derniers même d’être sauvés, parce qu’ils ont établi Jésus-Christ pour leur fondement, et l’ont plus aimé que tout le reste. Or, s’ils sont sauvés, ils seront certainement assis à la droite et entendront avec les autres ces paroles « Venez, vous que mon Père a bénis, prenez possession du royaume qui vous est préparé », au lieu d’être à la gauche avec les réprouvés, à qui il sera dit : « Retirez-vous de moi, maudits, et allez au feu éternel. » Car nul de ces maudits ne sera délivré du feu ; ils iront tous au supplice éternel, ou leur ver ne mourra point, et où le feu qui les brûlera ne s’éteindra point, et où ils seront tourmentés jour et nuit, dans les siècles des siècles.\par
Maintenant si l’on dit que dans l’intervalle de temps qui se passera entre la mort de chacun et ce jour qui sera, après la résurrection des corps, le dernier jour de rémunération et de damnation, si l’on dit que les âmes seront exposées à l’ardeur d’un feu que ne sentiront point ceux « qui n’auront pas eu dans cette vie des mœurs et des affections charnelles, de telle sorte qu’ils n’aient point bâti un édifice de bois, de foin et de paille que le feu puisse consumer » ; mais que sentiront ceux qui auront bâti un semblable édifice, c’est-à-dire qui auront commis des péchés véniels, et qui devront pour cela être soumis à un supplice transitoire, je ne m’y oppose point, car cela peut être vrai. La mon même du corps, qui est une peine du premier péché et que chacun souffre en son temps, peut être une partie de ce feu. Les persécutions de l’Église, qui ont couronné tant de martyrs et qu’endurent tous ceux qui sontchrétiens, sont aussi comme un feu qui éprouve ces différents édifices, qui consume les uns avec leurs auteurs, lorsqu’il n’y trouve pas Jésus-Christ pour fondement, qui brûle les autres sans toucher à leurs auteurs, qui seront sauvés, quoiqu’après punition, et qui épargne absolument les autres, parce qu’ils sont bâtis pour durer éternellement. Il y aura aussi vers la fin du monde, au temps de l’Antéchrist, une persécution si horrible qu’il n’y en a jamais eu de semblable. Combien y aura-t-il alors d’édifices, soit d’or ou de foin, élevés sur le bon fondement, qui est Jésus-Christ, que ce feu éprouvera avec dommage pour les uns, avec joie pour les autres, mais sans perdre ni les uns ni les autres à cause de ce bon fondement ? Mais quiconque préfère à Jésus-Christ, je ne dis pas sa femme, dont il se sert pour la volupté charnelle, mais même d’autres personnes qu’on n’aime pas de cette sorte, comme sont les parents, celui-là n’a point pour fondement Jésus-Christ ; et ainsi il ne sera pas sauvé par le feu. Il ne sera point du tout sauvé, parce qu’il ne pourra demeurer avec le Sauveur, qui, parlant de cela très clairement, dit : « Celui qui aime son père ou sa mère plus que moi, n’est pas digne de moi ; et celui qui aime son fils et sa fille plus que moi, n’est pas non plus digne de moi. » Pour celui qui aime humainement ses parents, de sorte néanmoins qu’il ne les préfère pas à Jésus-Christ, et qui aimerait mieux les perdre que lui, si on le mettait à cette épreuve, celui-là sera sauvé par le feu, parce qu’il faut que la perte de ces choses humaines cause autant de douleur qu’on y trouvait de plaisir. Enfin, celui qui aime ses parents en Jésus-Christ, et qui les aide à s’unir à lui et à acquérir son royaume, ou qui ne les aime que parce qu’ils sont les membres de Jésus-Christ, à Dieu ne plaise qu’un amour de cette sorte soit un édifice de bois, de foin et de paille que le feu consumera ! C’est un édifice d’or, d’argent et de pierres précieuses. Eh ! comment pourrait-il aimer plus que Jésus-Christ ceux qu’il n’aime que pour Jésus-Christ ?
\subsection[{Chapitre XXVII}]{Chapitre XXVII}

\begin{argument}\noindent Contre ceux qui croient qu’ils ne seront pas damnés, quoiqu’ayant persévéré dans le péché, parce qu’ils ont pratiqué l’aumône.
\end{argument}

\noindent Nous n’avons plus à réfuter qu’un dernier système, savoir, que le feu éternel ne sera que pour ceux qui négligent de racheter leurs péchés par de convenables aumônes, suivant cette parole de l’apôtre saint Jacques : « On jugera sans miséricorde celui qui sera sans miséricorde. » Celui donc, disent-ils, qui a pratiqué la miséricorde, bien qu’il n’ait pas renoncé à sa mauvaise vie, sera jugé avec miséricorde, de sorte qu’il ne sera pas damné, mais délivré finalement de son supplice. Ils assurent que le discernement que Jésus-Christ fera entre ceux de sa droite et ceux de sa gauche, pour envoyer les uns au royaume de Dieu et les autres au supplice éternel, ne sera fondé que sur le soin qu’on aura mis ou non à faire des aumônes. Ils tâchent encore de prouver par l’Oraison dominicale, que les péchés qu’ils commettent tous les jours, quelque grands qu’ils soient, peuvent leur être remis en retour des œuvres de charité, De même, disent-ils, qu’il n’y a point de jour où les chrétiens ne récitent cette oraison, il n’y a point de crime commis tous les jours qu’elle n’efface, à condition qu’en disant : « Pardonnez-nous nos offenses », nous ayons soin de faire ce qui suit : « comme nous les pardonnons à ceux qui nous ont offensés ». Notre-Seigneur, ajoutent-ils, ne dit pas : Si vous pardonnez aux hommes les fautes qu’ils ont faites contre vous, votre Père vous pardonnera les péchés légers que vous commettrez tous les jours ; mais il dit : « Il vous pardonnera vos péchés. » Ils estiment donc qu’en quelque nombre et de quelque espèce qu’ils soient, quand même on les commettrait tous les jours et quand on mourrait sans y avoir renoncé auparavant, les aumônes en obtiendront le pardon.\par
Certes, ils ont raison de vouloir que ce soient de dignes aumônes ; car s’ils disaient que tous les crimes, en quelque nombre qu’ils soient, seront remis par toute sorte d’aumônes, ils seraient choqués eux-mêmes d’une proposition si absurde. En effet, ce serait dire qu’un homme très riche, endonnant tous les jours quelques pièces de monnaie aux pauvres, pourrait racheter des homicides, des adultères, et les autres crimes les plus énormes. Si l’on ne peut avancer cela sans folie, reste à savoir quelles sont ces dignes aumônes capables d’effacer les péchés, et dont le précurseur même de Jésus-Christ entendait parler ; quand il disait : « Faites de dignes fruits de pénitence. » On ne trouvera pas sans doute que ces dignes aumônes soient celles des gens qui commettent tous les jours des crimes. En effet, leurs rapines vont bien plus haut que le peu qu’ils donnent à Jésus-Christ en la personne des pauvres, afin d’acheter tous les jours de lui l’impunité de leurs actions damnables. D’ailleurs, quand fis donneraient tout leur bien aux membres de Jésus-Christ pour un seul crime, s’ils ne renonçaient à leurs désordres, touchés par cette charité dont il est dit que jamais elle ne fait le mal, cette libéralité leur serait inutile. Que celui donc qui fait de dignes aumônes pour ses péchés commence à les faire envers lui-même. Il n’est pas raisonnable d’exercer envers le prochain une charité qu’on n’exerce pas envers soi, puisqu’il est écrit : « Vous aimerez votre prochain comme vous-même » ; et encore : « Ayez pitié de votre âme, en vous rendant agréable à Dieu. » Celui donc qui ne fait pas à son âme cette aumône afin de plaire à Dieu, comment peut-on dire qu’il fait de dignes aumônes pour ses péchés ? C’est pour cela qu’il est écrit : « À qui peut être bon celui qui est méchant envers lui-même ? » Car les aumônes aident les prières ; et c’est encore pourquoi il faut se rendre attentif à ces paroles : « Mon fils, vous avez péché, ne péchez plus, et priez Dieu qu’il vous pardonne vos péchés passés. » Nous devons donc faire des aumônes pour être exaucés, lorsque nous prions pour nos péchés passés, et non pour obtenir la licence de mal faire.\par
Or, Notre-Seigneur a prédit qu’il imputera à ceux qui seront à la droite les aumônes qu’ils auront faites, et à ceux qui seront à la gauche celles qu’ils auront manqué de faire, voulant montrer ce que peuvent les aumônes pour effacer les péchés commis, et non pour les commettre sans cesse impunément. Mais il ne faut pas croire que ceux qui ne veulentpas changer de vie fassent de véritables aumônes ; car ce que Jésus-Christ même leur dit : « Quand vous avez manqué de rendre ces devoirs au moindre des miens, c’est à moi que vous avez manqué de les rendre », fait assez voir qu’ils ne les rendent pas, lors même qu’ils croient les rendre. En effet, quand ils donnent du pain à un chrétien qui a faim, s’ils le lui donnaient en tant qu’il est chrétien, certes, ils ne se refuseraient pas à eux-mêmes le pain de la justice, qui est Jésus-Christ ; car Dieu ne regarde pas à qui l’on donne, mais dans quel esprit on donne. Ainsi, celui qui aime Jésus-Christ dans un chrétien lui fait l’aumône dans le même esprit où il s’approche de ce Sauveur, au lieu que les autres ne cherchent qu’à s’en éloigner, puisqu’ils n’aspirent qu’à jouir de l’impunité : or, on s’éloigne d’autant plus de Jésus-Christ qu’on aime davantage ce qu’il condamne. En effet, que sert-il d’être baptisé, si l’on n’est justifié ? Celui qui a dit : « Si l’on ne renaît de l’eau et du Saint-Esprit, on ne saurait entrer dans le royaume de Dieu », n’a-t-il pas dit aussi : « Si votre justice n’est pas plus grande que celle des Scribes et des Pharisiens, vous n’entrerez point dans le royaume des cieux » ? Pourquoi plusieurs courent-ils au baptême pour éviter le premier arrêt, et pourquoi si peu se mettent-ils en peine d’être justifiés pour éviter le second ? De même que celui-là ne dit pas à son frère : Fou ! qui, lorsqu’il lui dit cette injure, n’est pas en colère contre son frère, mais contre ses défauts, car, autrement, il mériterait l’enfer, ainsi, celui qui donne l’aumône à un chrétien, et qui n’aime pas en lui Jésus-Christ, ne la donne pas à un chrétien. Or, celui-là n’aime pas Jésus-Christ qui refuse d’être justifié en Jésus-Christ ; et comme il servirait de peu à celui qui appellerait son frère fou par colère, et sans songer à le corriger, de faire des aumônes pour obtenir le pardon de cette faute, à moins de se réconcilier avec lui, suivant ce commandement qui nous est fait au même lieu : « Lorsque vous faites votre offrande à l’autel, si vous vous souvenez d’avoir offensé votre frère, laissez là votre offrande, et allez auparavant vous réconcilier avec lui, et puis vous reviendrez offrir votre présent » ; de même, il sert de peu de faire de grandesaumônes pour ses péchés, lorsqu’on demeure dans l’habitude du péché.\par
Quant à l’oraison de chaque jour que Notre-Seigneur lui-même nous a enseignée, d’où vient qu’on l’appelle dominicale, elle efface, il est vrai, les péchés de chaque jour, quand chaque jour on dit : « Pardonnez-nous nos offenses », et qu’on ne dit pas seulement, mais qu’on fait ce qui suit : « comme nous pardonnons à ceux qui nous ont offensés » ; mais on récite cette prière parce qu’on commet des péchés, et non pas pour en commettre. Notre Sauveur nous a voulu montrer par là que, quelque bonne vie que nous menions, dans les ténèbres et la langueur où nous sommes, nous commettons tous les jours des fautes pour lesquelles nous avons besoin de prier et de pardonner à ceux qui nous offensent, si nous voulons que Dieu nous pardonne. Lors donc que Notre-Seigneur dit : « Si vous pardonnez aux hommes les fautes qu’ils font contre vous, votre Père vous pardonnera aussi vos péchés », il n’a pas entendu nous donner une fausse confiance dans cette oraison pour commettre tous les jours des crimes, soit en vertu de l’autorité qu’on exerce en se mettant au-dessus des lois, soit par adresse en trompant les hommes ; mais il a voulu par là nous apprendre à ne pas nous croire exempts de péchés, quoique nous soyons exempts de crimes : avertissement que Dieu donna aussi autrefois aux prêtres de l’ancienne loi, en leur commandant d’offrir en premier lieu des sacrifices pour leurs péchés, et ensuite pour ceux du peuple. Aussi bien, si nous considérons attentivement les paroles de notre grand et divin Maître, nous trouverons qu’il ne dit pas : Si vous pardonnez aux hommes les fautes qu’ils font contre vous, votre Père vous pardonnera aussi tous vos péchés, quels qu’ils soient ; mais : « Votre Père vous pardonnera aussi vos péchés. » Il enseignait une prière de tous les jours, et parlait à ses disciples, qui étaient justes. Qu’est-ce donc à dire {\itshape vos péchés}, sinon ceux dont vous-mêmes, qui êtes justifiés et sanctifiés, ne serez pas exempts ? Nos adversaires, qui cherchent dans cette prière un prétexte pour commettre tous les jours des crimes, prétendent que Notre-Seigneur a voulu aussi parler des grands péchés, parce qu’il n’a pas dit : Il vous pardonnera les petitspéchés, mais : Il vous pardonnera vos péchés. Nous, au contraire, considérant ceux à qui il parlait, et lui entendant dire vos péchés, nous ne devons entendre par là que les petits, parce que ses disciples n’en commettaient point d’autres ; mais les grands mêmes, dont il se faut entièrement défaire par une véritable conversion, ne sont pas remis par la prière, si l’on ne fait ce qui est dit au même endroit : « comme nous pardonnons à ceux qui nous ont offensés ». Que si les fautes, même légères, dont les plus saints ne sont pas exempts en cette vie, ne se pardonnent qu’à cette condition, combien plus les crimes énormes, bien qu’on cesse de les commettre, puisque Notre-Seigneur a dit : « Mais si vous ne pardonnez pas les fautes qu’on commet contre vous, votre Père ne vous pardonnera pas non plus. » C’est ce que veut dire l’apôtre saint Jacques, lorsqu’il parle ainsi : « On jugera sans miséricorde celui qui aura été sans miséricorde. » On doit aussi se souvenir de ce serviteur, à qui son maître avait remis dix mille talents, qu’il l’obligea à payer ensuite, parce qu’il avait été inexorable envers un autre serviteur comme lui, qui lui devait cent deniers. Ces paroles de l’Apôtre : « La miséricorde l’emporte sur la justice », s’appliquent à ceux qui sont enfants de la promesse et vases de miséricorde. Les justes mêmes, qui ont vécu dans une telle sainteté qu’ils reçoivent dans les tabernacles éternels ceux qui ont acquis leur amitié par les richesses d’iniquité, ne sont devenus tels que par la miséricorde de celui qui justifie l’impie et qui lui donne la récompense selon la grâce, et non selon les mérites. Du nombre de ces impies justifiés est l’Apôtre, qui dit « J’ai obtenu miséricorde pour être fidèle. »\par
Ceux qui sont ainsi reçus dans les tabernacles éternels, il faut avouer que, comme ils n’ont pas assez bien vécu pour être sauvés sans le suffrage des saints, la miséricorde à leur égard l’emporte encore bien plus sur la justice. Et néanmoins, on ne doit pas s’imaginer qu’un scélérat impénitent soit reçu dans les tabernacles éternels pour avoir assisté les saints avec des richesses d’iniquité, c’est-à-dire avec des biens mal acquis, ou tout au moins avec de fausses richesses,mais que l’iniquité croit vraies, parce qu’elle ne connaît pas les vraies richesses qui rendent opulents ceux lui reçoivent les autres dans les tabernacles éternels. Il y a donc un certain genre de vie qui n’est pas tellement criminel que les aumônes y soient inutiles pour gagner le ciel, ni tellement bon qu’il suffise pour atteindre un si grand bonheur, à moins d’obtenir miséricorde par les mérites de ceux dont on s’est fait des amis par les aumônes. À ce propos, je m’étonne toujours qu’on trouve, même dans Virgile, cette parole du Seigneur : « Faites-vous des amis avec les richesses d’iniquité, afin qu’ils vous reçoivent dans les tabernacles éternels », ou bien en d’autres termes : « Celui qui reçoit un prophète, en qualité de prophète, recevra la récompense du prophète, et celui qui reçoit un juste, en qualité de juste, recevra la récompense du juste. » En effet, dans le passage où Virgile décrit les Champs-Élysées, que les païens croient être le séjour des bienheureux, non seulement il y place ceux qui y sont arrivés par leurs propres mérites, mais encore :\par
N’est-ce pas là ce mot que les chrétiens ont si souvent à la bouche, quand par humilité ils se recommandent à un juste : Souvenez-vous de moi, lui disent-ils, et ils cherchent par de bons offices à graver leur nom dans son souvenir ? Maintenant si nous revenons à la question de savoir quel est ce genre de vie et quels sont ces crimes qui ferment l’entrée du royaume de Dieu, et dont néanmoins on obtient le pardon, il est très difficile de s’en assurer et très dangereux de vouloir le déterminer. Pour moi, quelque soin que j’y ai mis jusqu’à présent, je ne l’ai pu découvrir. Peut-être cela est-il caché, de peur que nous n’en devenions moins courageux à éviter les péchés qu’on peut commettre sans péril de damnation. En effet, si nous les connaissions, il se pourrait que nous ne nous fissions pas scrupule de les commettre, sous prétexte que les aumônes suffisent pour nous en obtenir le pardon ; au lieu que, ne les connaissant pas, nous sommes plus obligés de nous tenir sur nos gardes, et de faire effort pour avancer dans la vertu, sans toutefois négliger de nous faire des amis parmi les saints au moyen des aumônes.\par
Mais cette délivrance qu’on obtient ou par ses prières, ou par l’intercession des saints, ne sert qu’à empêcher d’être envoyé au feu éternel ; elle ne servira pas à en faire sortir, quand on y sera déjà. Ceux mêmes qui pensent que ce qui est dit dans l’Évangile de ces bonnes terres qui rapportent des fruits en abondance, l’une trente, l’autre soixante, et l’autre cent pour un, doit s’entendre des saints, qui, selon la diversité de leurs mérites, délivreront les uns trente hommes, les autres soixante, les autres cent, ceux-là même croient qu’il en sera ainsi au jour du jugement, mais nullement après. On rapporte à ce sujet le mot d’une personne d’esprit qui, voyant les hommes se flatter d’une fausse impunité et croire que par l’intercession des saints tous les pécheurs peuvent être sauvés, répondit fort à propos qu’il était plus sûr de tâcher, par une bonne vie, d’être du nombre des intercesseurs, de peur que ce nombre soit si restreint qu’après qu’ils auront délivré l’un trente pécheurs, l’autre soixante, l’autre cent, il n’en reste encore un grand nombre pour lesquels ils n’auront plus le droit d’intercéder, et parmi eux celui qui aura mis vainement son espérance dans un autre. Mais j’ai suffisamment répondu à ceux qui, ne méprisant pas l’autorité de nos saintes Écritures, mais les comprenant mal, y trouvent, non pas le sens qu’elles ont, mais celui qu’ils veulent leur donner. Notre réponse faite, terminons cet avant-dernier livre, comme nous l’avons annoncé.
\section[{Livre vingt-deuxième. Bonheur des saints}]{Livre vingt-deuxième. \\
Bonheur des saints}\renewcommand{\leftmark}{Livre vingt-deuxième. \\
Bonheur des saints}

\subsection[{Chapitre premier}]{Chapitre premier}

\begin{argument}\noindent De la condition des anges et des hommes.
\end{argument}

\noindent Ce dernier livre, ainsi que je l’ai promis au livre précédent, roulera tout entier sur la question de la félicité de la Cité de Dieu : félicité éternelle, non parce qu’elle doit longtemps durer, mais parce qu’elle ne doit jamais finir, selon ce qui est écrit dans l’Évangile : « Son royaume n’aura point de fin. » La suite des générations humaines, dont les unes meurent pour être remplacées par d’autres, n’est que le fantôme de l’éternité, de même qu’on dit qu’un arbre est toujours vert, lorsque de nouvelles feuilles, succèdent à celles qui tombent, lui conservent toujours son ombrage. Mais la Cité de Dieu sera véritablement éternelle ; car tous ses membres seront immortels, et les hommes justes y acquerront ce que les anges n’y ont jamais perdu. Le Dieu tout-puissant, son fondateur, fera cette merveille ; car il l’a promis, et il ne peut mentir ; nous en avons pour gage tant d’autres promesses déjà accomplies, sans parler des merveilles accomplies sans avoir été promises.\par
C’est lui qui, dès le commencement, a créé ce monde, peuplé d’êtres visibles et intelligibles, tous excellents, mais entre lesquels nous ne voyons rien de meilleur que les esprits qu’il a créés intelligents et capables de le connaître et de le posséder, les unissant ensemble par les liens d’une société que nous appelons la Cité sainte et céleste, où le soutien de leur existence et le principe de leur félicité, c’est Dieu lui-même qui leur sert d’aliment et de vie. C’est lui qui a donné le libre arbitre à cette nature intelligente, à condition que si elle venait à abandonner Dieu, source de sa béatitude, elle tomberait aussitôt dans la plusprofonde misère. C’est lui qui, prévoyant que parmi les anges quelques-uns, enflés d’orgueil, mettraient leur félicité en eux-mêmes et perdraient ainsi le vrai bien, n’a pas voulu leur ôter cette puissance, jugeant qu’il était plus digne de sa propre puissance et de sa bonté de se bien servir du mal que de ne pas le permettre. En effet, le mal n’eût jamais été, si la nature muable, quoique bonne et créée par le Dieu suprême et immuablement bon qui a fait bonnes toutes ses œuvres, ne s’était elle-même rendue mauvaise par le péché. Aussi bien son péché même atteste son excellence primitive. Car si elle-même n’était un bien très grand, quoique inférieur à son divin principe, la perte qu’elle a faite de Dieu comme de sa lumière ne pourrait être un mal pour elle. De même, en effet, que la cécité est un vice de l’œil, et que ce vice non seulement témoigne que l’œil a été fait pour voir la lumière, mais encore fait ressortir l’excellence du plus noble des sens, ainsi la nature qui jouissait de Dieu nous apprend, par son désordre même, qu’elle a été créée bonne, puisque ce qui la rend misérable, c’est de ne plus jouir de Dieu. C’est lui qui a très justement puni d’une misère éternelle la chute volontaire des mauvais anges, et qui a donné aux autres, fidèlement attachés à leur souverain bien, l’assurance de ne jamais le perdre, comme prix de leur fidélité. C’est lui qui a créé l’homme dans la même droiture que les anges, avec le même libre arbitre, animal terrestre à la vérité, mais digne du ciel, s’il demeure attaché à son créateur ; et il l’a condamné aussi à la misère, s’il vient à s’en détacher. C’est lui qui, prévoyant que l’homme pècherait à son tour par la transgression de la loi divine et l’abandon de son Dieu, n’a pas voulu non plus lui ôter la puissance du libre arbitre, parce qu’il prévoyait aussi le bienqu’il pourrait tirer de ce mal ; et en effet, sa grâce a rassemblé parmi cette race mortelle justement condamnée un si grand peuple qu’elle en a pu remplir la place désertée par les anges prévaricateurs. Ainsi cette Cité suprême et bien-aimée, loin d’être trompée dans le compte de ses élus, se réjouira peut-être d’en recueillir une plus abondante moisson.
\subsection[{Chapitre II}]{Chapitre II}

\begin{argument}\noindent De l’éternelle et immuable volonté de Dieu.
\end{argument}

\noindent Les méchants, il est vrai, font beaucoup de choses qui sont contre la volonté de Dieu ; mais il est si puissant et si sage qu’il fait aboutir ce qui paraît contredire sa volonté aux fins déterminées par sa prescience. C’est pourquoi, lorsqu’on dit qu’il change de volonté, qu’il entre en colère, par exemple, contre ceux qu’il regardait d’un œil favorable, ce sont les hommes qui changent, et non pas lui. Leurs dispositions changeantes font qu’ils trouvent Dieu changé. Ainsi le soleil change pour des yeux malades ; il était doux et agréable, il devient importun et pénible, et cependant il est resté le même en soi. On appelle aussi volonté de Dieu celle qu’il forme dans les cœurs dociles à ses commandements, et voilà le sens de ces paroles de l’Apôtre : « C’est Dieu qui opère en nous le vouloir même. » De même que la justice de Dieu n’est pas seulement celle qui le fait juste en soi, mais encore celle qu’il produit dans l’homme justifié, ainsi la loi de Dieu est plutôt la loi des hommes, mais c’est Dieu qui la leur a donnée. En effet, c’est à des hommes que Jésus-Christ disait : « Il est écrit dans votre loi » ; et nous lisons encore autre part : « La loi de Dieu est gravée dans son cœur. » On parle de cette volonté que Dieu forme dans les hommes, quand on dit qu’il veut ce qu’en effet il ne veut pas lui-même, mais ce qu’il fait vouloir aux siens, comme on dit aussi qu’il connaît ce qu’il fait connaître à l’ignorance des hommes. Par exemple, quand l’Apôtre s’exprime ainsi : « Mais maintenant connaissant Dieu, ou plutôt étant connus de Dieu », il ne faut pas croire que Dieu commençât alors à les connaître, eux qu’il connaissait avant la création du monde ; mais il est dit qu’il les connut alors, parce qu’il leur donna alors le don de connaître. J’ai déjà touché un mot de ces locutions dans les livres précédents. Ainsi donc, selon cette volonté par laquelle nous disons que Dieu veut ce qu’il fait vouloir aux autres qui ne connaissent pas l’avenir, il veut plusieurs choses qu’il ne fait pas.\par
En effet, ses saints veulent souvent, d’une volonté sainte que lui-même inspire, beaucoup de choses qui n’arrivent pas ; ils prient Dieu, par exemple, en faveur de quelqu’un, et ils ne sont pas exaucés, bien que ce soit lui qui les ait portés à prier par un mouvement du Saint-Esprit. Ainsi, quand les saints inspirés de Dieu veulent et prient que chacun soit sauvé, nous pouvons dire : Dieu veut et ne fait pas. Mais, si l’on parle de cette volonté qui est aussi éternelle que sa prescience, il a certainement fait tout ce qu’il a voulu au ciel et sur la terre, et non seulement les choses passées ou présentes, mais même les choses à venir. Or, avant que le temps arrive où il a fixé l’accomplissement des choses qu’il a connues et ordonnées avant tous les temps, nous disons : Cela arrivera quand Dieu voudra. Mais quand nous ignorons non seulement à quelle époque une chose doit arriver, mais même si elle doit arriver en effet, nous disons : Cela arrivera si Dieu le veut. Ce n’est pas qu’il doive alors survenir en Dieu une volonté qu’il n’avait pas, mais c’est qu’alors arrivera ce qu’il avait prévu de toute éternité dans sa volonté immuable.
\subsection[{Chapitre III}]{Chapitre III}

\begin{argument}\noindent De la promesse d’une béatitude éternelle pour les saints et d’un supplice éternel pour les impies.
\end{argument}

\noindent Donc, pour ne rien dire de mille autres questions, de même que nous voyons maintenant s’accomplir en Jésus-Christ ce que Dieu promit à Abraham en lui disant : « Toutes les nations seront bénies en vous », ainsi s’accomplira ce qu’il a promis à cette même race, quand il a dit par son Prophète : « Ceux qui étaient dans les tombeaux ressusciteront » ; et encore : « Il y aura un ciel nouveau et une terre nouvelle, et ils ne se souviendront plus du passé, et ils en perdront entièrement la mémoire ; mais ils trouveront en elle des sujets de joie et d’allégresse. Et voici que je ferai de Jérusalem et de mon peuple une fête et une réjouissance, et je prendrai mon plaisir en Jérusalem et mon contentement en mon peuple, et l’on n’y entendra plus désormais ni plaintes ni soupirs. » Même prédiction par la bouche d’un autre prophète : « En ce temps-là, tout votre peuple qui se trouvera écrit dans le livre sera sauvé, et plusieurs de ceux qui dorment dans la poussière de la terre (ou, selon d’autres interprètes, sous un amas de terre) ressusciteront les uns pour la vie éternelle, et les autres pour recevoir un opprobre et une confusion éternelle. » Et ailleurs par le même prophète : « Les saints du Très-Haut recevront le royaume, et ils le posséderont jusque dans le siècle, et jusque dans les siècles des siècles » ; et un peu après : « Et son royaume sera éternel. » Ajoutez à cela tant d’autres promesses semblables que j’ai rapportées dans le vingtième livre, ou que j’ai omises et qui se trouvent néanmoins dans l’Écriture. Tout cela arrivera comme les merveilles dont l’accomplissement a déjà été un sujet d’étonnement pour les incrédules. C’est le même Dieu qui a promis, lui devant qui tremblent les divinités des païens, de l’aveu d’un éminent philosophe païen.
\subsection[{Chapitre IV}]{Chapitre IV}

\begin{argument}\noindent Contre les sages du monde qui pensent que les corps terrestres des hommes ne pourront être transportés dans le ciel.
\end{argument}

\noindent Mais ces personnages si remplis de science et de sagesse, et en même temps si rebelles à une autorité qui a soumis, comme elle l’avait annoncé bien des siècles à l’avance, tant de générations humaines, ces philosophes, dis-je, s’imaginent avoir trouvé un argument fort décisif contre la résurrection des corps, quand ils allèguent un certain passage de Cicéron, au troisième livre de sa République. Après avoir dit qu’Hercule et Romulus sont devenus des dieux, d’hommes qu’ils étaient auparavant, Cicéron ajoute : « Mais leurs corps n’ont pas été enlevés au ciel, la nature ne souffrant pas que ce qui est formé de la terre subsiste autre part que dans la terre. » Voilà le grand raisonnement de ces sagesdont le Seigneur connaît les pensées, et les connaît pour vaines. Car supposez que nous soyons ces esprits purs, c’est-à-dire des esprits sans corps, habitant le ciel sans savoir s’il existe des animaux terrestres, si l’on venait nous dire qu’un jour nous serons unis par un lien merveilleux aux corps terrestres pour les animer, n’aurions-nous pas beaucoup plus de sujet de n’en rien croire, et de dire que la nature ne peut souffrir qu’une substance incorporelle soit emprisonnée dans un corps ? Cependant la terre est pleine d’esprits à qui des corps terrestres sont unis par un lien mystérieux. Pourquoi donc, s’il plaît à Dieu, qui a fait tout cela, pourquoi un corps terrestre ne pourrait-il pas être enlevé parmi les corps célestes, puisqu’un esprit, plus excellent que tous les corps, et, par conséquent, qu’un corps céleste, a pu être uni à un corps terrestre ? Quoi donc ! une si petite particule de terre a pu retenir un être fort supérieur à un corps céleste, afin d’en recevoir la vie et le sentiment, et le ciel dédaignerait de recevoir ou ne pourrait retenir cette terre vivante et animée qui tire la vie et le sentiment d’une substance plus excellente que tout corps céleste ? Si cela ne se fait pas maintenant, c’est que le temps n’est pas venu, le temps, dis-je, déterminé par celui-là même qui a fait une chose beaucoup plus merveilleuse, mais que l’habitude a rendue vulgaire. Car enfin, que des esprits incorporels, plus excellents que tout corps céleste, soient unis à des corps terrestres, n’est-ce pas là un phénomène qui doit nous étonner plutôt que de voir des corps, quoique terrestres, être élevés à des demeures célestes, il est vrai, mais corporelles ? Mais nous sommes accoutumés à voir la première de ces merveilles, qui est nous-mêmes ; au lieu que nous n’avons jamais vu L’autre, qui n’est pas encore devenue notre propre nature. Certes, si nous consultons la raison, nous trouverons qu’il est beaucoup plus merveilleux de joindre des corps à des esprits que d’unir des corps à des corps, bien que ces corps soient différents, les uns étant célestes et les autres terrestres.
\subsection[{Chapitre V}]{Chapitre V}

\begin{argument}\noindent De la résurrection des corps, que certains esprits ne veulent pas admettre, bien que proclamée par le monde entier.
\end{argument}

\noindent Mais je veux que cela ait été autrefois incroyable. Voilà le monde qui croit maintenant que le corps de Jésus-Christ, tout terrestre qu’il est, a été emporté au ciel ; voilà les doctes et les ignorants qui croient que la chair ressuscitera-et qu’elle montera au ciel ; et il en est très peu qui demeurent incrédules. Or, de deux choses l’une : s’ils croient une chose croyable, que ceux qui ne la croient pas s’accusent eux-mêmes de stupidité ; et s’ils croient une chose incroyable, il n’est pas moins incroyable qu’on soit porté à croire une chose de cette espèce. Le même Dieu a donc prédit ces deux choses incroyables, que les corps ressusciteraient et que le monde le croirait ; et il les a prédites toutes deux, bien longtemps avant que l’une des deux arrivât. De ces deux choses incroyables, nous en voyons déjà une accomplie, qui est que le monde croirait une chose incroyable ; pourquoi désespérerions-nous de voir l’autre, puisque celle lui est arrivée n’est pas moins difficile à croire ? Et, si l’on y songe, la manière même dont le monde a cru est une chose encore plus incroyable. Jésus-Christ a envoyé un petit nombre d’hommes sans lumières et sans politesse, étrangers aux belles connaissances, ignorant les ressources de la grammaire, les armes de la dialectique, les artifices pompeux de la rhétorique, en un mot de pauvres pécheurs ; il les a envoyés à l’océan du siècle avec les seuls filets de la foi, et ils ont pris une infinité de poissons de toute espèce, de l’espèce même la plus merveilleuse et la plus rare, je veux parler des philosophes. Ajoutez, si vous voulez, ce troisième miracle aux deux autres. Voilà en tout trois choses incroyables qui néanmoins sont arrivées : il est incroyable que Jésus-Christ soit ressuscité en sa chair, et qu’avec cette même chair il soit monté au ciel ; il est incroyable que le monde ait cru une chose aussi incroyable ; il est incroyable enfin qu’un petit nombre d’hommes de basse condition, inconnus, ignorants, aient pu persuader une chose aussi incroyable au monde et aux savants du monde. De ces trois choses incroyables, nos adversaires ne veulent pas croire la première ; ils sont contraints de voir la seconde, et ils ne sauraient la comprendre, à moins de croire la troisième. En effet, la résurrection de Jésus-Christ, et son ascension au ciel en la chair où il est ressuscité, sont choses déjà prêchées et crues dans tout l’univers ; si elles ne sont pas croyables, d’où vient que l’univers les croit ? Admettez qu’un grand nombre de personnages illustres, doctes, puissants, aient déclaré les avoir vues et se soient chargés de les publier en tout lieu, il n’est plus étrange que le monde les ait crues ; et en ce cas il y a bien de l’opiniâtreté à ne pas les croire. Mais si, comme il est vrai, le monde a cru un petit nombre d’hommes inconnus et ignorants sur leur parole, comment se fait-il qu’une poignée d’incrédules entêtés ne veuille pas croire ce que le monde croit ? Et si le monde a cru à ce peu de témoins obscurs, infimes, ignorants, méprisables, c’est qu’en eux elle a vu paraître avec plus d’éclat la majesté de Dieu. Leur éloquence a été toute en miracles, et non en paroles ; et ceux qui n’avaient pas vu Jésus-Christ ressusciter et monter au ciel avec son corps, n’ont pas eu de peine à le croire, sur la foi de témoignages confirmés par une infinité de prodiges. En effet, des hommes qui ne pouvaient savoir au plus que deux langues, ils les entendaient parler soudain toutes les langues du monde. Ils voyaient un boiteux de naissance, après quarante ans d’infirmité, marcher d’un pas égal, à leur parole et au nom de Jésus-Christ ; les linges qu’ils avaient touchés guérissaient les malades ; et tandis que des milliers d’hommes infirmes se rangeaient sur leur passage, il suffisait que leur nombre les couvrît en passant pour les rendre à la santé. Et combien ne pourrais-je pas citer d’autres prodiges, sans parler même des morts qu’ils ont ressuscités au nom du Sauveur ! Si nos adversaires nous accordent la réalité de ces miracles, voilà bien des choses incroyables qui viennent s’ajouter aux trois premières ; et il faut être singulièrement opiniâtre pour ne pas croire une chose incroyable, telle que la résurrection du corps de Jésus-Christ et son ascension au ciel, du moment qu’elle est confirmée par tant d’autres choses non moins incroyables et pourtant réelles. Si, au contraire, ils ne croient pas que les Apôtres aient fait ces miracles pour établir la croyance à la résurrection et à l’ascension de Jésus-Christ, ce seul grand miracle nous suffit, que toute la terre ait cru sans miracles.
\subsection[{Chapitre VI}]{Chapitre VI}

\begin{argument}\noindent Rome a fait un dieu de Romulus, parce qu’elle aimait en lui son fondateur ; au lieu que l’Église a aimé Jésus-Christ, parce qu’elle l’a cru Dieu.
\end{argument}

\noindent Rappelons ici le passage où Cicéron s’étonne que la divinité de Romulus ait obtenu créance. Voici ses propres paroles : « Ce qu’il y a de plus admirable dans l’apothéose de Romulus, c’est que les autres hommes qui ont été faits dieux vivaient dans des siècles grossiers, où il était aisé de persuader aux peuples tout ce qu’on voulait. Mais il n’y a pas encore six cents ans qu’existait Romulus, et déjà les lettres et les sciences florissaient depuis longtemps dans le monde, et y avaient dissipé la barbarie. » Et un peu après il ajoute : « On voit donc que Romulus a existé bien des années après Homère, et que, les hommes commençant à être éclairés, il était difficile, dans un siècle déjà poli, de recourir à des fictions. Car l’antiquité a reçu des fables qui étaient quelquefois bien grossières ; mais le siècle de Romulus était trop civilisé pour rien admettre qui ne fût au moins vraisemblable. » Ainsi, voilà un des hommes les plus savants et les plus éloquents du monde, Cicéron, qui s’étonne qu’on ait cru à la divinité de Romulus, parce que le siècle où-il est venu était assez éclairé pour répudier des fictions. Cependant, qui a cru que Romulus était un dieu, sinon Rome, et encore Rome faible et naissante ? Les générations suivantes furent obligées de conserver la tradition des ancêtres ; et, après avoir sucé cette superstition avec le lait, elles la répandirent parmi les peuples que Rome fit passer sous son joug. Ainsi, toutes ces nations vaincues, sans ajouter foi à la divinité de Romulus, ne laissaient pas de la proclamer pour ne pas offenser la maîtresse du monde, trompée elle-même, sinon par amour de l’erreur, du moins par l’erreur de son amour. Combien est différente notre foi dans la divinité de Jésus-Christ !\par
Il est sans doute le fondateur de la Cité éternelle ; mais tant s’en faut qu’elle l’ait cru dieu, parce qu’il l’a fondée, qu’elle ne mérite d’être fondée que parce qu’elle le croit dieu. Rome, déjà bâtie et dédiée, a élevé à son fondateur un temple où elle l’a adoré comme un dieu ; la nouvelle Jérusalem, afin d’être bâtie et dédiée, a pris pour base de sa foi son fondateur, Jésus-Christ Dieu. La première, par amour pour Romulus, l’a cru dieu ; la seconde, convaincue que Jésus-Christ était Dieu, l’a aimé. Quelque chose a donc précédé l’amour de celle-là, et l’a portée à croire complaisamment à une perfection, même imaginaire, de celui qu’elle aimait ; et de même, quelque chose a précédé la foi de celle-ci, pour lui-faire aimer sans témérité un privilège très véritable dans celui en qui elle croit. Sans parler, en effet, de tant de miracles qui ont établi la divinité de Jésus-Christ, nous avions sur lui, avant qu’il ne parût sur la terré, des prophéties divines parfaitement dignes de foi et dont nous n’attendions pas l’accomplissement, comme nos pères, mais qui sont déjà accomplies. Il n’en est pas ainsi de Romulus. On sait par les historiens qu’il a bâti Rome et qu’il y a régné, sans qu’aucune prophétie antérieure eût rien annoncé de cela. Maintenant, qu’il ait été transporté parmi les dieux, l’histoire le rapporte comme une croyance, elle ne le prouve point comme un fait. Point de miracle pour témoigner de la vérité de cette apothéose. On parle d’une louve qui nourrit les deux frères comme d’une grande merveille. Mais qu’est-ce que cela pour prouver qu’un homme est un dieu ? Alors même que cette louve aurait été une vraie louve et non pas une courtisane, le prodige aunait été commun aux deux-frères, et cependant il n’y en a qu’un qui passe pour un dieu. D’ailleurs, à qui a-t-on défendu de croire et de dire que Romulus, Hercule et autres personnages semblables étaient des dieux ? Et qui a mieux aimé mourir que de cacher sa foi ? Ou plutôt se serait-il jamais rencontré une seule nation qui eût adoré Romulus sans la crainte du nom romain ? Et cependant qui pourrait compter tous ceux qui ont mieux aimé perdre la vie dans les plus cruels tourments que de nier la divinité de Jésus-Christ ? Ainsi la crainte, fondée ou non, d’encourir une légère indignation des Romains contraignait quelques peuples vaincus à adorer Romulus comme un dieu ; et la crainte des plus horribles supplices et de la mort même, n’a pu empêcher sur toute la terre un nombre immense de martyrs, non seulement d’adorer Jésus-Christ comme un dieu, mais de le confesser publiquement. La Cité de Dieu, étrangère encore ici-bas, mais qui avait déjà recruté toute une armée de peuples, n’a point alors combattu contre ses persécuteurs pour la conservation d’une vie temporelle ; mais au contraire elle ne leur a point résisté, afin d’acquérir la vie éternelle. Les chrétiens étaient chargés de chaînes, mis en prison, battus de verges, tourmentés, brûlés, égorgés, mis en pièces, et leur nombre augmentait. Ils ne croyaient pas combattre pour leur salut éternel, s’ils ne méprisaient leur salut éternel pour l’amour du Sauveur.\par
Je sais que Cicéron, dans sa {\itshape République}, au livre huitième, si je ne me trompe, soutient qu’un État bien réglé n’entreprend jamais la guerre que pour garder sa foi ou pour veiller à son salut. Et Cicéron explique ailleurs ce qu’il entend par le salut d’un État, lorsqu’il dit : « Les particuliers se dérobent souvent par une prompte mort à la pauvreté, à l’exil, à la prison, au fouet, et aux autres peines auxquelles les hommes les plus grossiers ne sont pas insensibles ; mais la mort même, qui semble affranchir de toute peine, est une peine pour un État, qui doit être constitué pour être éternel. Ainsi la mort n’est point naturelle à une république comme elle l’est à un individu, qui doit non seulement la subir malgré lui, mais souvent même la souhaiter. Lors donc qu’un État succombe, disparaît, s’anéantit, il nous est (si l’on peut comparer les petites choses aux grandes), il nous est une image de la ruine et de la destruction du monde entier. » Cicéron parle ainsi, parce qu’il pense, avec les Platoniciens, que le monde ne doit jamais périr. Il est donc avéré que, suivant Cicéron, un État doit entreprendre la guerre pour son salut, c’est-à-dire pour subsister éternellement ici-bas, tandis que ceux qui le composent, naissent et meurent par une continuelle révolution : comme un olivier, un laurier, ou tout autre arbre semblable, conserve toujours le même ombrage, malgré la chute et le renouvellement de ses feuilles. La mort, selon lui, n’est pas une peine pour les particuliers, puisqu’elle les délivre souvent de toute autre peine, mais elle est une peine pour un État. Ainsi l’on peut demander avec raison si les Sagontins firent bien d’aimer mieux que leur cité pérît que de manquer de foi aux Romains, car les citoyens de la cité de la terre les louent de cette action. Mais je ne vois pas comment ils pouvaient suivre cette maxime de Cicéron : qu’il ne faut entreprendre la guerre que pour sa foi ou son salut, Cicéron ne disant pas ce qu’il faut faire de préférence dans le cas où l’on ne pourrait conserver l’un de ces biens sans perdre l’autre. En effet, les Sagontins ne pouvaient se sauver sans trahir leur foi envers les Romains, ni garder cette foi sans périr, comme ils périrent en effet. Il n’en est pas de même du salut dans la Cité de Dieu : on le conserve, ou plutôt on l’acquiert avec ta foi et par la foi, et la perte de la foi entraîne celle du salut. C’est cette pensée d’un cœur ferme et généreux qui a fait un si grand nombre de martyrs, tandis que Romulus n’en a pu avoir un seul qui ait versé son sang pour confesser sa divinité.
\subsection[{Chapitre VII}]{Chapitre VII}

\begin{argument}\noindent Si le monde a cru en Jésus-Christ, c’est l’ouvrage d’une vertu divine, et non d’une persuasion humaine.
\end{argument}

\noindent Mais il est parfaitement ridicule de nous opposer la fausse divinité de Romulus, quand nous parlons de Jésus-Christ. Si, dès le temps de Romulus, c’est-à-dire six cents ans avant Cicéron, le monde était déjà tellement éclairé qu’il rejetait comme faux tout ce qui n’était pas vraisemblable, combien plutôt encore, au temps de Cicéron lui-même, et surtout plus tard, sous les règnes d’Auguste et de Tibère,époques de civilisation de plus en plus avancée, eût-on rejeté bien loin la résurrection de Jésus-Christ en sa chair et son ascension au ciel comme choses absolument impossibles ! Il a fallu, pour ouvrir l’oreille et le cœur des hommes à cette croyance, que la vérité divine ou la divinité véritable et une infinité de miracles eussent déjà démontré que de tels miracles pouvaient se faire et s’étaient effectivement accomplis. Voilà pourquoi, malgré tant de cruelles persécutions, on a cru et prêché hautement la résurrection et l’immortalité de la chair, lesquelles ont d’abord paru en Jésus-Christ pour se réaliser un jour en tous les hommes ; voilà pourquoi cette croyance a été semée par toute la terre pour croître et se développer de plus en plus par le sang fécond des martyrs ; car l’autorité des miracles venant confirmer l’autorité des prophéties, la vérité a pénétré enfin dans les esprits, et l’on a vu qu’elle était plutôt contraire à la coutume qu’à la raison, jusqu’au jour où le monde entier a embrassé par la foi ce qu’il persécutait dans sa fureur.
\subsection[{Chapitre VIII}]{Chapitre VIII}

\begin{argument}\noindent Des miracles qui ont été faits pour que le monde crût en Jésus-Christ et qui n’ont pas cessé depuis qu’il y croit.
\end{argument}

\noindent Pourquoi, nous dit-on, ces miracles qui, selon vous, se faisaient autrefois, ne se font-ils plus aujourd’hui ? Je pourrais répondre que les miracles étaient nécessaires avant que le monde crût, pour le porter à croire, tandis qu’aujourd’hui quiconque demande encore des miracles pour croire est lui-même un grand miracle de ne pas croire ce que toute la terre croit ; mais ils ne parlent ainsi que pour faire douter de la réalité des miracles. Or, d’où vient qu’on publie si hautement partout que Jésus-Christ est monté au ciel avec son corps ? d’où vient qu’en des siècles éclairés, où l’on rejetait tout ce qui paraissait impossible, le monde a cru sans miracles des choses tout à fait incroyables ? Aiment-ils mieux dire qu’elles étaient incroyables, et que c’est pour cela qu’on les a crues ? Que ne les croient-ils donc eux-mêmes ? Voici donc à quoi se réduit tout notre raisonnement : ou bien des choses incroyables que tout le monde voyait ont persuadé une chose incroyable que tout le monde ne voyait pas ; ou bien cette chose était tellement croyable qu’elle n’avait pas besoin de miracles pour être crue, et, dans ce dernier cas, où trouver une opiniâtreté plus extrême que celle de nos adversaires ? Voilà ce qu’on peut répondre aux plus obstinés. Que plusieurs miracles aient été opérés pour assurer ce grand et salutaire miracle par lequel Jésus-Christ est ressuscité et monté au ciel avec son corps, c’est ce que l’on ne peut nier. En effet, ils sont consignés dans les livres sacrés qui déposent tout ensemble et de la réalité de ces miracles et de la foi qu’ils devaient fonder. La renommée de ces miracles s’est répandue pour donner la foi, et la foi qu’ils leur ont donnée ajoute à leur renommée un nouvel éclat. On les lit aux peuples afin qu’ils croient, et néanmoins on ne les leur lirait pas, si déjà ils n’avaient été crus. Car il se fait encore des miracles au nom de Jésus-Christ, soit par les sacrements, soit par les prières et les reliques des saints, mais ils ne sont pas aussi célèbres que les premiers. Le canon des saintes Lettres, qui devait être fixé par l’Église, fait connaître ces premiers miracles en tous lieux et les confie à la mémoire des peuples. Au contraire, ceux-ci ne sont connus qu’aux lieux où ils se passent, et souvent à peine le sont-ils d’une ville entière, surtout quand elle est grande, ou d’un voisinage restreint. Ajoutez enfin que l’autorité de ceux qui les rapportent, tout fidèles qu’ils sont et s’adressant à des fidèles, n’est pas assez considérable pour ne laisser aucun doute aux bons esprits.\par
Le miracle qui eut lieu à Milan (j’y étais alors), quand un aveugle recouvra la vue, a pu être connu de plusieurs ; en effet, la ville est grande, l’empereur était présent, et ce miracle s’opéra à la vue d’un peuple immense accouru de tous côtés pour voir les corps des saints martyrs Gervais et Protais, qui avaient été découverts en songe à l’évêque Ambroise. Or, par la vertu de ces reliques, l’aveugle sentit se dissiper les ténèbres de ses yeux et recouvra la vue.\par
Mais qui, à l’exception d’un petit nombre, a entendu parler à Carthage de la guérison miraculeuse d’Innocentius, autrefois avocat de la préfecture, guérison que j’ai vue de mes propres yeux ? C’était un homme très pieux,ainsi que toute sa maison, et il nous avait reçus chez lui, mon frère Alype et moi, au retour de notre voyage d’outre-mer, quand nous n’étions pas encore clercs, mais engagés cependant au service de Dieu ; nous demeurions donc avec lui. Les médecins le traitaient de certaines fistules hémorroïdales qu’il avait en très grande quantité, et qui le faisaient beaucoup souffrir. Ils avaient déjà appliqué le fer et usé de tous les médicaments que leur conseillait leur art. L’opération avait été fort douloureuse et fort longue ; mais les médecins, par mégarde, avaient laissé subsister une fistule qu’ils n’avaient point vue entre toutes les autres. Aussi, tandis qu’ils soignaient et guérissaient toutes les fistules ouvertes, celle-là seule rendait leurs soins inutiles. Le malade, se défiant de ces longueurs, et appréhendant extrêmement une nouvelle incision, comme le lui avait fait craindre un médecin, son domestique, que les autres avaient renvoyé au moment de l’opération, ne voulant pas de lui, même comme simple témoin, et que son maître, après l’avoir chassé dans un accès de colère, n’avait consenti à recevoir qu’avec beaucoup de difficulté, le malade, dis-je, s’écria un jour, hors de lui : Est-ce que vous allez m’inciser encore ? et faudra-t-il que je souffre ce que m’a prédit celui que vous avez éloigné ? — Alors ils commencèrent à se moquer de l’ignorance de leur confrère et à rassurer le malade par de belles promesses. Cependant plusieurs jours se passent, et tout ce que l’un tentait était inutile. Les médecins persistaient toujours à dire qu’ils guériraient cette hémorroïde par la force de leurs médicaments, sans employer le fer. Ils appelèrent un vieux praticien, fameux par ces sortes de cures, nommé Ammonius, qui, après avoir examiné le mal, en porta le même jugement. Le malade, se croyant déjà hors d’affaire, raillait le médecin domestique, sur ce qu’il avait prédit qu’il faudrait une nouvelle opération. Que dirai-je de plus ? Après bien des jours, inutilement reculés, ils en vinrent à avouer, las et confus, que le fer pouvait seul opérer la guérison. Le malade épouvanté, pâlissant, aussitôt que son extrême frayeur lui eût permis de parler, leur enjoignit de se retirer et de ne plus revenir.\par
Cependant, après avoir longtemps pleuré, il n’eut d’autre ressource que d’appeler un certain Alexandrin, chirurgien célèbre, pour faire ce qu’il n’avait pas voulu que les autres fissent. Celui-ci vint donc ; mais après avoir reconnu par les cicatrices l’habileté de ceux qui l’avaient traité, il lui conseilla, en homme de bien, de les reprendre, et de ne pas les priver du fruit de leurs efforts. Il ajouta qu’Innocentius ne pouvait guérir, en effet, qu’en subissant une nouvelle incision, mais qu’il ne voulait point avoir l’honneur d’une cure si avancée, et dans laquelle il admirait l’adresse de ceux qui l’avaient précédé. Le malade se réconcilia donc avec ses médecins ; il fut résolu qu’ils feraient l’opération en présence de l’Alexandrin, et elle fut remise par eux au lendemain. Cependant, les médecins s’étant retirés, le malade tomba dans une si profonde tristesse que toute sa maison en fut remplie de deuil, comme s’il eût déjà été mort. Il était tous les jours visité par un grand nombre de personnes pieuses, et entre autres par Saturnin, d’heureuse mémoire, évêque d’Uzali, et par Gélose, prêtre, ainsi que par quelques diacres de l’Église de Carthage. De ce nombre aussi était l’évêque Aurélius, le seul de tous qui ait survécu, personnage éminemment respectable avec lequel nous nous sommes souvent entretenus de ce miracle de Dieu, dont il se souvenait parfaitement. Comme ils venaient, sur le soir, voir le malade, suivant leur ordinaire, il les pria de la manière la plus attendrissante d’assister le lendemain même à ses funérailles plutôt qu’à ses souffrances, car les incisions précédentes lui avaient causé tant de douleur qu’il croyait fermement mourir entre les mains des médecins. Ceux-ci le consolèrent du mieux qu’ils purent, et l’exhortèrent à se confier à Dieu et à se soumettre à sa volonté. Ensuite nous nous mîmes en prière ; et nous étant agenouillés et prosternés à terre, selon notre coutume, il s’y jeta lui-même avec tant d’impétuosité qu’il semblait que quelqu’un l’eût fait tomber rudement, et il commença à prier. Mais qui pourrait exprimer de quelle manière, avec quelle ardeur, quels transports, quels torrents de larmes, quels gémissements et quels sanglots, tellement enfin que tous ses membres tremblaient et qu’il était comme suffoqué ! Je ne sais si les autres priaient et si tout cela ne les détournait point ; pour moi, je ne le pouvais faire, et je dis seulement en moi-même ce peu de mots : Seigneur, quelles prières de vos serviteurs exaucerez-vous, si vous n’exaucez pas celles-ci ? Il me paraissait qu’on n’y pouvait rien ajouter, sinon d’expirer en priant. Nous nous levons, et, après avoir reçu la bénédiction de l’évêque, nous nous retirons, le malade priant les assistants de se trouver le lendemain matin chez lui, et nous, l’exhortant à avoir bon courage. Le jour venu, ce jour tant appréhendé, les serviteurs de Dieu arrivèrent, comme ils l’avaient promis. Les médecins entrent ; on prépare tout ce qui est nécessaire à l’opération, on tire les redoutables instruments ; chacun demeure interdit et en suspens. Ceux qui avaient le plus d’autorité encouragent le malade, tandis qu’on le met sur son lit dans la position la plus commode pour l’incision ; on délie les bandages, on met à nu la partie malade, le médecin regarde, et cherche de l’œil et de la main l’hémorroïde qu’il devait ouvrir. Enfin, après avoir exploré de toutes façons la partie malade, il finit par trouver une cicatrice très ferme. — Il n’y a point de paroles capables d’exprimer la joie, le ravissement, et les actions de grâces de tous ceux qui étaient présents. Ce furent des larmes et des exclamations que l’on peut s’imaginer, mais qu’il est impossible de rendre.\par
Dans la même ville de Carthage, Innocentia, femme très pieuse et du rang le plus distingué, avait au sein un cancer, mal incurable, à ce que disent les médecins. On a coutume de couper et de séparer du corps la partie où est le mal, ou, si l’on veut prolonger un peu la vie du malade, de n’y rien faire ; et c’est, dit-on, le sentiment d’Hippocrate. Cette dame l’avait appris d’un savant médecin, son ami, de sorte qu’elle n’avait plus recours qu’à Dieu. La fête de Pâques étant proche, elle fut avertie en songe de prendre garde à la première femme qui se présenterait à elle au sortir du baptistère, et de la prier de faire le signe de la croix sur son mal. Cette femme le fit, et Innocentia fut guérie à l’heure même. Le médecin qui luiavait conseillé de n’employer aucun remède, si elle voulait vivre un peu plus longtemps, la voyant guérie, lui demanda vivement ce qu’elle avait fait pour cela, étant bien aise sans doute d’apprendre un remède qu’Hippocrate avait ignoré. Elle lui dit ce qui en était, non sans craindre, à voir son visage méfiant, qu’il ne lui répondît quelque parole injurieuse au Christ : « Vraiment, s’écria-t-il, je pensais que vous m’alliez dire quelque chose de bien merveilleux ! » Et comme elle se révoltait déjà : « Quelle grande merveille, ajouta-t-il, que Jésus-Christ ait guéri un cancer au sein, lui qui a ressuscité un mort de quatre jours ? » Quand j’appris ce qui s’était passé, je ne pus supporter la pensée qu’un si grand miracle, arrivé dans une si grande ville, à une personne de si haute condition, pût demeurer caché ; je fus même sur le point de réprimander cette dame. Mais quand elle m’eut assuré qu’elle ne l’avait point passé sous silence, je demandai à quelques dames de ses amies intimes, qui étaient alors avec elle, si elles le savaient. Elles me dirent que non. « Voilà donc, m’écriai-je, de quelle façon vous le publiez ! vos meilleures amies n’en savent rien ! » Et comme elle m’avait rapporté le fait très brièvement, je lui en fis recommencer l’histoire tout au long devant ces dames, qui en furent singulièrement étonnées et en rendirent gloire à Dieu.\par
Un médecin goutteux de la même ville, ayant donné son nom pour être baptisé, vit en songe, la nuit qui précéda son baptême, des petits enfants noirs et frisés qu’il prit pour des démons, et qui lui défendirent de se faire baptiser cette année-là. Sur son refus de leur obéir, ils lui marchèrent sur les pieds, en sorte qu’il y sentit des douleurs plus cruelles que jamais. Cela ne l’empêcha point de se faire baptiser le lendemain, comme il l’avait promis à Dieu, et il sortit du baptistère non seulement guéri de ses douleurs extraordinaires, mais encore de sa goutte, sans qu’il en ait jamais rien ressenti, quoique ayant encore longtemps vécu. Qui a entendu parler de ce miracle ? Cependant nous l’avons connu, nous et un certain nombre de frères à qui le bruit en a pu parvenir.\par
Un ancien mime de Curube fut guéride même d’une paralysie et d’une hernie, et sortit du baptême comme s’il n’avait jamais rien eu. Qui connaît ce miracle, hors ceux de Curube, et peut-être un petit nombre de personnes ? Pour nous, quand nous l’apprîmes, nous fîmes venir cet homme à Carthage, par l’ordre du saint évêque Aurélius, bien que nous en eussions été informés par des personnes tellement dignes de foi que nous n’en pouvions douter.\par
Hespérius, d’une famille tribunitienne, possède dans notre voisinage un domaine sur les terres de Fussales, appelé Zubédi. Ayant reconnu que l’esprit malin tourmentait ses esclaves et son bétail, il pria nos prêtres, en mon absence, de vouloir bien venir chez lui afin d’en chasser les démons. L’un d’eux s’y rendit, et offrit le sacrifice du corps de Jésus-Christ, avec de ferventes prières, pour faire cesser cette possession. Aussitôt elle cessa par la miséricorde de Dieu. Or, Hespérius avait reçu d’un de ses amis un peu de la terre sainte de Jérusalem où Jésus-Christ fut enseveli et ressuscita le troisième jour. Il avait suspendu cette ferre dans sa chambre à coucher, pour se mettre lui-même à l’abri des obsessions du démon. Lorsque sa maison en fut délivrée, il se demanda ce qu’il ferait de cette terre qu’il ne voulait plus, par respect, garder dans sa chambre. Il arriva par hasard que mon collègue Maximin, évêque de Sinite, et moi, nous étions alors dans les environs. Hespérius nous fit prier de l’aller voir, et nous y allâmes. Il nous raconta tout ce qui s’était passé, et nous pria d’enfouir cette terre en un lieu où les chrétiens pussent s’assembler pour faire le service de Dieu. Nous y consentîmes. Il y avait près de là un jeune paysan paralytique, qui, sur cette nouvelle, pria ses parents de le porter sans délai vers ce saint lieu ; et à peine y fut-il arrivé et eut-il prié, qu’il put s’en retourner sur ses pieds, parfaitement guéri.\par
Dans une métairie nommée Victoriana, à trente milles d’Hippone, il y a un monument en l’honneur des deux martyrs de Milan, Gervais et Protais. On y porta un jeune homme qui, étant allé vers midi, pendant l’été, abreuver son cheval à la rivière, fut possédé par le démon. Comme il était étendu mourant et semblable à un mort, la maîtresse du lieu vint sur le soir, selon sa coutume, près dumonument, avec ses servantes et quelques religieuses, pour y chanter des hymnes et y faire sa prière. Alors le démon, frappé et comme réveillé par ces voix, saisit l’autel avec un frémissement terrible, et sans oser ou sans pouvoir le remuer, il s’y tenait attaché et pour ainsi dire lié. Puis, priant d’une voix gémissante, il suppliait qu’on lui pardonnât, et il confessa même comment et en quel endroit il était entré dans le corps de ce jeune homme. À la fin, promettant d’en sortir, il en nomma toutes les parties, avec menace de les couper, quand il sortirait, et, en disant cela, il se retira de ce jeune homme. Mais l’œil du malheureux tomba sur sa joue, retenu par une petite veine comme par une racine, et la prunelle devint toute blanche. Ceux qui étaient présents et qui s’étaient mis en prière avec les personnes accourues au bruit, touchés de ce spectacle et contents de voir ce jeune homme revenu à son bon sens, s’affligeaient néanmoins de la perte de son œil et disaient qu’il fallait appeler un médecin. Alors le beau-frère de celui qui l’avait transporté prenant la parole : « Dieu, dit-il, qui a chassé le démon à la prière de ces saints, peut bien aussi rendre la vue à ce jeune homme. » Là-dessus il remit comme il put l’œil à sa place et le banda avec son mouchoir ; sept jours après, il crut pouvoir l’enlever, et il trouva l’œil parfaitement guéri. D’autres malades encore trouvèrent en ce lieu leur guérison ; mais ce récit nous mènerait trop loin.\par
Je connais une fille d’Hippone, qui, s’étant frottée d’une huile où le prêtre qui priait pour elle avait mêlé ses larmes, fut aussitôt délivrée du malin esprit. Je sais que la même chose arriva à un jeune homme, la première fois qu’un évêque, qui ne l’avait point vu, pria pour lui.\par
Il y avait à Hippone un vieillard nommé Florentius, homme pauvre et pieux, qui vivait de son métier de tailleur. Ayant perdu l’habit qui le couvrait et n’ayant pas de quoi en acheter un autre, il courut au tombeau des Vingt. Martyrs, qui est fort célèbre chez nous, et les pria de le vêtir. Quelques jeunes gens qui se trouvaient là par hasard, et qui avaient envie de rire, l’ayant entendu, le suivirent quand il sortit et se mirent à le railler, comme s’il eûtdemandé cinquante oboles aux martyrs pour avoir un habit. Mais lui, continuant toujours son chemin sans rien dire, vit un grand poisson qui se débattait sur le rivage ; il le prit avec le secours de ces jeunes gens, et le vendit trois cents oboles à un cuisinier nommé Catose, chrétien zélé, à qui il raconta tout ce qui s’était passé. Il se disposait à acheter de la laine, afin que sa femme lui en fît tel habit qu’elle pourrait ; mais le cuisinier ayant ouvert le poisson, trouva dedans une bague d’or. Touché à la fois de compassion et de pieux effroi, il la porta à cet homme, en lui disant : Voilà comme les vingt Martyrs ont pris soin de vous vêtir.\par
L’évêque Projectus ayant apporté à Tibilis des reliques du très glorieux martyr saint Étienne, il se fit autour du reliquaire un grand concours de peuple. Une femme aveugle des environs pria qu’on la menât à l’évêque qui portait ce sacré dépôt, et donna des fleurs pour les faire toucher aux reliques. Quand on les lui eut rendues, elle les porta à ses yeux, et recouvra tout d’un coup la vue. Tous ceux qui étaient présents furent surpris de ce miracle ; mais elle, d’un air d’allégresse, se mit à marcher la première devant eux et n’eut plus besoin de guide.\par
Lucillus, évêque de Sinite, ville voisine d’Hippone, portait en procession les reliques du même martyr, fort révéré en ce lieu. Une fistule, qui le faisait beaucoup souffrir et que son médecin était sur le point d’ouvrir, fut tout d’un coup guérie par l’effet de ce pieux fardeau ; car il n’en souffrit plus désormais.\par
Eucharius, prêtre d’Espagne, qui habitait à Calame, fut guéri d’une pierre, qui le tourmentait depuis longtemps, par les reliques du même martyr, que l’évêque Possidius y apporta. Le même prêtre, étant en proie à une autre maladie qui le mit si bas qu’on le croyait mort et que déjà on lui avait lié les mains, revint par le secours du même martyr. On jeta sur les reliques sa robe de prêtre que l’on remit ensuite sur lui, et il fut rappelé à la vie.\par
Il y avait là un homme fort âgé, nommé Martial, le plus considérable de la ville, qui avait une grande aversion pour la religion chrétienne. Sa fille était chrétienne et songendre avait été baptisé la même année. Ceux-ci le voyant malade, le conjurèrent en pleurant de se faire chrétien ; mais il refusa, et les chassa avec colère d’auprès de lui. Son gendre trouva à propos d’aller au tombeau de saint Étienne, pour demander à Dieu la conversion de son beau-père. Il pria avec beaucoup de ferveur, et, prenant quelques fleurs de l’autel, les mit sur la tête du malade, comme il était déjà nuit. Le vieillard s’endormit ; mais il n’était pas jour encore qu’il cria qu’on allât chercher l’évêque qui se trouvait alors avec moi à Hippone. À son défaut, il fit venir des prêtres, à qui il dit qu’il était chrétien, et qui le baptisèrent, au grand étonnement de tout le monde. Tant qu’il vécut, il eut toujours ces mots à la bouche : « Seigneur Jésus, recevez mon esprit » ; sans savoir que ces paroles, les dernières qu’il prononça, avaient été aussi les dernières paroles de saint Étienne, quand il fut lapidé par les Juifs.\par
Deux goutteux, l’un citoyen et l’autre étranger, furent aussi guéris par le même saint :le premier fut guéri instantanément ; le second eut une révélation de ce qu’il devait faire, quand la douleur se ferait sentir ; il le fit et fut soulagé.\par
Audurus est une terre où il y a une église, et dans cette église une chapelle dédiée à saint Étienne. Il arriva par hasard que, pendant qu’un petit enfant jouait dans la cour, des bœufs qui traînaient un chariot, sortant de leur chemin, firent passer la roue sur lui et le tuèrent. Sa mère l’emporte et le place près du lieu consacré au saint ; or, non seulement il recouvra la vie, mais il ne parut pas même qu’il eût été blessé.\par
Une religieuse qui demeurait à Caspalium, terre située dans les environs, étant fort malade et abandonnée des médecins, on porta sa robe à la même chapelle ; mais la religieuse mourut avant qu’on eût eu le temps de la rapporter. Cependant ses parents en couvrirent son corps inanimé, et aussitôt elle ressuscita et fut guérie.\par
À Hippone, un nommé Bassus, de Syrie, priait devant les reliques du saint martyr pour sa fille, dangereusement malade ; il avait apporté avec lui la robe de son enfant. Tout à coup ses gens accoururent pour lui annoncer qu’elle était morte. Mais quelques-uns de ses amis, qu’ils rencontrèrent en chemin, les empêchèrent de lui annoncer cette nouvelle, de peur qu’il ne pleurât devant tout le monde. De retour chez lui, et quand la maison retentissait déjà des plaintes de ses domestiques, il jeta sur sa fille la robe qu’il apportait de l’église, et elle revint incontinent à la vie.\par
Le fils d’un certain Irénéus, collecteur des impôts, était mort dans la même ville. Pendant que l’on se préparait à faire ses funérailles, un des amis du père lui conseilla de faire frotter le corps de son fils de l’huile du même martyr. On le fit, et l’enfant ressuscita.\par
L’ancien tribun Eleusinus, qui avait mis son fils, mort de maladie, sur le tombeau du même martyr, voisin du faubourg où il demeurait, le remporta vivant, après avoir prié et versé des larmes pour lui.\par
Je pourrais encore rapporter un grand nombre d’autres miracles que je connais ; mais comment faire ? il faut bien, comme je l’ai promis, arriver à la fin de cet ouvrage. Je ne doute point que plusieurs des nôtres qui me liront ne soient fâchés que j’en aie omis beaucoup qu’ils connaissent aussi bien que moi ; mais je les prie de m’excuser, et de considérer combien il serait long de faire ce que je suis obligé de négliger. Si je voulais rapporter seulement toutes les guérisons qui ont été opérées à Calame et à Hippone par le glorieux martyr saint Étienne, elles contiendraient plusieurs volumes ; encore ne seraient-ce que celles dont on a écrit les relations pour les lire au peuple. Aussi bien, c’est par mes ordres que ces relations ont été dressées, quand j’ai vu se faire de notre temps plusieurs miracles semblables à ceux d’autrefois et dont il fallait ne pas laisser perdre la mémoire. Or, il n’y a pas encore deux ans que les reliques de ce martyr sont à Hippone ; et bien qu’on n’ait pas donné de relation de tous les miracles qui s’y sont faits, il s’en trouve déjà près de soixante-dix au moment où j’écris ceci. Mais à Calame, où les reliques de ce saint martyr sont depuis plus longtemps et où l’on a plus de soin d’écrire ces relations, le nombre en monte bien plus haut.\par
Nous savons encore que plusieurs miracles sont arrivés à Uzales, colonie voisine d’Utique, grâce aux reliques du même martyr, que l’évêque Evodius y avait apportées, bien avant qu’il y en eût à Hippone ; mais on n’a pascoutume en ce pays d’en écrire des relations, ou du moins cela ne se pratiquait pas autrefois. Peut-être le fait-on maintenant. Comme nous y étions, il n’y a pas longtemps, une dame de haute condition, nommée Pétronia, ayant été guérie miraculeusement d’une langueur qui avait épuisé tous les remèdes des médecins, nous l’exhortâmes, avec l’agrément de l’évêque, à en faire une relation qui pût être lue au peuple. Elle nous l’accorda fort obligeamment et y inséra une circonstance que je ne puis négliger ici, quoique pressé de passer à ce qui me reste à dire. Elle dit qu’un juif lui persuada de porter sur elle à nu une ceinture de cheveux où serait une bague dont le chaton avait été fait d’une pierre trouvée dans les reins d’un bœuf. Cette dame, portant cette ceinture sur elle, venait à l’église du saint martyr. Mais un jour partie de Carthage, comme elle s’était arrêtée dans une de ses terres sur les bords du fleuve Bagrada et qu’elle se levait pour continuer son chemin, elle fut tout étonnée de voir son anneau à ses pieds. Elle tâta sa ceinture pour voir si elle ne s’était pas détachée, et la trouvant bien liée, elle crut que l’anneau s’était rompu. Mais elle l’examina, le trouva parfaitement entier, et prit ce prodige pour une assurance de sa guérison. Elle délia donc sa ceinture et la jeta avec l’anneau dans le fleuve.\par
Ils ne croiront pas ce miracle ceux qui ne croient pas que le Seigneur Jésus-Christ soit sorti du sein de sa mère sans altérer sa virginité, et qu’il soit entré, toutes portes fermées, dans le lieu où étaient réunis ses disciples. Mais qu’ils s’informent au moins du fait que je viens de citer, et s’ils le trouvent vrai, qu’ils croient aussi le reste. C’est une dame illustre, de grande naissance, et mariée en haut lieu ; elle demeure à Carthage. La ville est grande, et la personne connue. Il est donc impossible que ceux qui s’enquerront de ce miracle n’apprennent pas ce qui en est. Tout au moins le martyr même, par les prières duquel elle a été guérie, a cru au fils d’une vierge, à celui qui est entré, les portes fermées, dans le lieu où étaient réunis ses disciples ; en un mot, et tout ce que nous disons présentement n’est que pour en venir là, il a cru en celui qui est monté au ciel avec le même corps dans lequel il est ressuscité ; et si tant de merveilles s’opèrent par l’intercession du saint martyr, c’est qu’il a donné sa vie pour maintenir sa foi. Il s’accomplit donc encore aujourd’hui beaucoup de miracles ; le même Dieu qui a fait les prodiges que nous lisons fait encore ceux-ci par les personnes qu’il lui plaît de choisir, et comme il lui plaît. Mais ces derniers ne sont pas aussi connus, parce qu’une fréquente lecture ne les imprime pas dans la mémoire aussi fortement que les autres. Aux lieux mêmes où l’on prend soin d’en écrire des relations, ceux qui sont présents, lorsqu’on les lit, ne les entendent qu’une fois, et il y a beaucoup d’absents. Les personnes mêmes qui les ont entendu lire ne les retiennent pas, et à peine s’en trouve-t-il une seule de celles-là qui les rapporte aux autres.\par
Voici un miracle qui est arrivé parmi nous et qui n’est pas plus grand que ceux dont j’ai fait mention ; mais il est si éclatant que je ne crois pas qu’il y ait à Hippone une personne qui ne l’ait vu, ou qui n’en ait ouï parler, et qui jamais puisse l’oublier : dix enfants, dont sept fils et trois filles, natifs de Césarée on Cappadoce, et d’assez bonne condition, ayant été maudits par leur mère pour quelque outrage qu’ils lui firent après la mort de son mari, furent miraculeusement frappés d’un tremblement de membres. Ne pouvant souffrir la confusion à laquelle ils étaient en butte dans leur pays, ils s’en allèrent, chacun de leur côté, errer dans l’empire romain. Il en vint deux à Hippone, un frère et une sœur, Paul et Palladia, déjà fameux en beaucoup d’endroits par leur disgrâce ; ils y arrivèrent quinze jours avant la fête de Pâques, et ils visitaient tous les jours l’Église où se trouvaient les reliques du glorieux saint Étienne, priant Dieu de s’apaiser à leur égard et de leur rendre la santé. Partout où ils allaient, ils attiraient les regards, et ceux qui les avaient vus ailleurs disaient aux autres la cause de leur tremblement. Le jour de Pâques venu, et comme déjà un grand concours de peuple remplissait l’église, le jeune homme, tenant les balustres du lieu où étaient les reliques du martyr, tomba tout d’un coup, et demeura par terre comme endormi, sans toutefois trembler, comme il faisait d’ordinaire, même en dormant. Cet accident étonna tout le monde, et plusieurs en furent touchés. Il s’en trouva qui voulurent le relever ; mais d’autres les en empêchèrent, et dirent qu’il valait mieux attendre la fin de son sommeil. Tout à coup le jeune homme se releva sur ses pieds sans trembler, car il était guéri, examinant tous ceux qui le regardaient. Qui put s’empêcher alors de rendre grâces à Dieu ? Toute l’église retentit de cris de joie, et l’on courut promptement à moi pour me dire l’événement, à l’endroit où j’étais assis, prêt à m’avancer vers le peuple. Ils venaient l’un sur l’autre, le dernier m’annonçant cette nouvelle, comme si je ne l’avais point apprise du premier. Tandis que je me réjouissais et rendais grâces à Dieu, le jeune homme guéri entra lui-même avec les autres, et se jeta à mes pieds ; je l’embrassai et le relevai. Nous nous avançâmes vers le peuple, l’église étant toute pleine, et l’on n’entendait partout que ces mots : Dieu soit béni ! Dieu soit béni ! Je saluai le peuple, et il recommença encore plus fort les mêmes acclamations. Enfin, comme chacun eut fait silence, on lut quelques leçons de l’Écriture. Quand le moment où je devais parler fut venu, je fis un petit discours, selon l’exigence du temps et la grandeur de cette joie, aimant mieux qu’ils goûtassent l’éloquence de Dieu dans une œuvre si merveilleuse, que dans mon propre discours. Le jeune homme dîna avec nous, et nous raconta en détail l’histoire de son malheur et celle de ses frères, de ses sœurs et de sa mère. Le lendemain, après le sermon, je promis au peuple de lui en lire le récit, au jour suivant. Le troisième jour donc après le dimanche de Pâques, comme on faisait la lecture promise, je fis mettre le frère et la sœur sur les degrés du lieu où je montais pour parler, afin qu’on pût les voir. Tout le peuple les regardait attentivement, l’un dans une attitude tranquille, l’autre tremblant de tous ses membres. Ceux qui ne les avaient pas vus ainsi apprenaient, par le malheur de la sœur, la miséricorde de Dieu pour le frère. Ils voyaient ce dont il fallait se réjouir pour lui et ce qu’il fallait demander pour elle. Quand on eut achevé de lire la relation, je les fis retirer. Je commençais à faire quelques observations sur cette histoire, lorsqu’on entendit de nouvelles acclamations qui venaient du tombeau du saint martyr. Toute l’assemblée se tourna de ce côté et s’y porta en masse. La jeune fille n’avait pas plus tôt descendu les degrés où je l’avais fait mettre, qu’elle avait couru se mettre en prières auprès du tombeau.\par
À peine en eut-elle touché les balustres qu’elle tomba comme son frère et se releva parfaitement guérie. Or, comme nous demandions ce qui était arrivé, et d’où venaient ces cris de joie, les fidèles rentrèrent avec elle dans la basilique où nous étions, la ramenant guérie du tombeau du martyr. Alors il s’éleva un si grand cri de joie de la bouche des hommes et des femmes, que l’on crut que les larmes et les acclamations ne finiraient point. Palladia fut conduite au même lieu où on l’avait vue un peu auparavant trembler de tous ses membres. Plus on s’était affligé de la voir moins favorisée que son frère, plus on se réjouissait de la voir aussi bien guérie que lui. On glorifiait la bonté de Dieu, qui avait entendu et exaucé les prières qu’on avait à peine eu le temps de faire pour elle. Aussi, il s’élevait de toute part de si grands cris d’allégresse qu’à peine nos oreilles pouvaient-elles les soutenir. Qu’y avait-il dans le cœur de tout ce peuple si joyeux, sinon cette foi du Christ, pour laquelle saint Étienne avait répandu son sang ?
\subsection[{Chapitre IX}]{Chapitre IX}

\begin{argument}\noindent Tous les miracles opérés par les martyrs au nom de Jésus-Christ sont autant de témoignages de la foi qu’ils ont eue en Jésus-Christ.
\end{argument}

\noindent À qui ces miracles rendent-ils témoignage, sinon à cette foi qui prêche Jésus-Christ ressuscité et monté au ciel en corps et en âme ? Les martyrs eux-mêmes ont été les martyrs, c’est-à-dire les témoins de cette foi c’est pour elle qu’ils se sont attiré la haine et la persécution du monde, et qu’ils ont vaincu, non en résistant, mais en mourant. C’est pour elle qu’ils sont morts, eux qui peuvent obtenir ces grâces du Seigneur au nom duquel ils sont morts. C’est pour elle qu’ils ont souffert, afin que leur admirable patience fût suivie de ces miracles de puissance. Car s’il n’était pas vrai que la résurrection de la chair s’est d’abord manifestée en Jésus-Christ et qu’elle doit s’accomplir dans tous les hommes telle qu’elle a été annoncée par ce Sauveur et prédite par les Prophètes, pourquoi les martyrs, égorgés pour cette foi qui prêche la résurrection, ont-ils, quoique morts, un sigrand pouvoir ? En effet, soit que Dieu fasse lui-même ces miracles, selon ce merveilleux mode d’action qui opère des effets temporels du sein de l’éternité, soit qu’il agisse par ses ministres, et, dans ce dernier cas, soit qu’il emploie le ministère des esprits des martyrs, comme s’ils étaient encore au monde, ou celui des anges, les martyrs y interposant seulement leurs prières, soit enfin qu’il agisse de quelque autre manière incompréhensible aux hommes, toujours faut-il tomber d’accord que les martyrs rendent témoignage à cette foi qui prêche la résurrection éternelle des corps.
\subsection[{Chapitre X}]{Chapitre X}

\begin{argument}\noindent Combien sont plus dignes d’être honorés les martyrs qui opèrent de tels miracles pour que l’on adore Dieu, que les démons qui ne font certains prodiges que pour se faire eux-mêmes adorer comme des dieux.
\end{argument}

\noindent Nos adversaires diront peut-être que leurs dieux ont fait aussi des miracles. À merveille, pourvu qu’ils en viennent déjà à comparer leurs dieux aux hommes qui sont morts parmi nous. Diront-ils qu’ils ont aussi des dieux tirés du nombre des morts, comme Hercule, Romulus et plusieurs autres qu’ils croient élevés au rang des dieux ? Mais nous ne croyons point, nous, que nos martyrs soient des dieux, parce que nous savons que notre Dieu est le leur ; et cependant, les miracles que les païens prétendent avoir été faits par les temples de leurs dieux ne sont nullement comparables à ceux qui se font par les tombeaux de nos martyrs. Ou s’il en est quelques-uns qui paraissent du même ordre, nos martyrs ne laissent pas de vaincre leurs dieux, comme Moïse vainquit les mages de Pharaon. En effet, les prodiges opérés par les démons sont inspirés par le même orgueil qui les a portés à vouloir être dieux ; au lieu que nos martyrs les font, ou plutôt Dieu les fait par eux et à leur prière, afin d’établir de plus en plus cette foi qui nous fait croire, non que les martyrs sont nos dieux, mais qu’ils n’ont avec nous qu’un même Dieu. Enfin, les païens ont bâti des temples aux divinités de leur choix, leur ont dressé des autels, donné des prêtres et fait des sacrifices ; mais nous, nous n’élevons point à nos martyrs des templescomme à des dieux, mais des tombeaux comme à des morts dont les esprits sont vivants devant Dieu. Nous ne dressons point d’autels pour leur offrir des sacrifices, mais nous immolons l’hostie à Dieu seul, qui est notre Dieu et le leur. Pendant ce sacrifice, ils sont nommés en leur lieu et en leur ordre, comme des hommes de Dieu qui, en confessant son nom, ont vaincu le monde ; mais le prêtre qui sacrifie ne les invoque point : c’est à Dieu qu’il sacrifie et non pas à eux, quoiqu’il sacrifie en mémoire d’eux ; car il est prêtre de Dieu et non des martyrs. Et en quoi consiste le sacrifice lui-même ? c’est le corps de Jésus-Christ, lequel n’est pas offert aux martyrs, parce qu’eux-mêmes sont aussi ce corps. À quels miracles croira-t-on de préférence ? aux miracles de ceux qui veulent passer pour dieux, ou aux miracles de ceux qui ne les font que pour établir la foi en la divinité de Jésus-Christ ? À qui se fier ? à ceux qui veulent faire consacrer leurs crimes ou à ceux qui ne souffrent pas même que l’on consacre leurs louanges, et qui veulent qu’on les rapporte à la gloire de celui en qui on les loue ? C’est en Dieu, en effet, que leurs âmes sont glorifiées. Croyons donc à la vérité de leurs discours et à la puissance de leurs miracles ; car c’est pour avoir dit la vérité qu’ils ont souffert la mort, et c’est la mort librement subie qui leur a valu le don des miracles. Et l’une des principales vérités qu’ils ont affirmées, c’est que Jésus-Christ est ressuscité des morts et qu’il a fait voir, en sa chair l’immortalité de la résurrection qu’il nous a promise au commencement du nouveau siècle ou à la fin de celui-ci.
\subsection[{Chapitre XI}]{Chapitre XI}

\begin{argument}\noindent Contre les Platoniciens qui prétendent prouver, par le poids des éléments, qu’un corps terrestre ne peut demeurer dans le ciel.
\end{argument}

\noindent À cette grâce signalée de Dieu, qu’opposent ces raisonneurs dont Dieu sait que les pensées sont vaines ? Ils argumentent sur le poids des éléments. Platon, leur maître, leur a enseigné en effet que deux des grands éléments du monde, et les plus éloignés l’un de l’autre, le feu et la terre, sont joints et unis par deux éléments intermédiaires, c’est-à-dire par l’airet par l’eau. Ainsi, disent-ils, puisque la terre est le premier corps en remontant la série, l’eau le second, l’air le troisième, et le ciel le quatrième, un corps terrestre ne peut pas être dans le ciel. Chaque élément, pour tenir sa place, est tenu en équilibre par son propre poids. Voilà les arguments dont la faiblesse présomptueuse des hommes se sert pour combattre la toute-puissance de Dieu, Que font donc tant de corps terrestres dans l’air, qui est le troisième élément au-dessus de la terre ? à moins qu’on ne veuille dire que celui qui a donné aux corps terrestres des oiseaux la faculté de s’élever en l’air par la légèreté de leurs plumes ne pourra donner aux hommes, devenus immortels, la vertu de résider même au plus haut des cieux ! À ce compte, les animaux terrestres qui ne peuvent voler, comme sont les hommes, devraient vivre sous la terre comme les poissons, qui sont des animaux aquatiques et vivent sous l’eau. Pourquoi un animal terrestre ne tire-t-il pas au moins sa vie du second élément, qui est l’eau, et ne peut-il y séjourner sans être suffoqué ; et pourquoi faut-il qu’il vive dans le troisième ? Y a-t-il donc erreur ici dans l’ordre des éléments, ou plutôt n’est-ce pas leur raisonnement, et non la nature, qui est en défaut ? Je ne reviendrai pas ici sur ce que j’ai déjà dit au troisième livre, comme par exemple qu’il y a beaucoup de corps terrestres pesants, tels que le plomb, auxquels l’art peut donner une certaine figure qui leur permet de nager sur l’eau. Et l’on refusera au souverain artisan le pouvoir de donner au corps humain une qualité qui l’élève et le retienne dans le ciel !\par
Il y a plus, et ces philosophes ne peuvent pas même se servir, pour me combattre, de l’ordre prétendu des éléments. Car si la terre occupe par son poids la première région, si l’eau vient ensuite, puis l’air, puis le ciel, l’âme est au-dessus de tout cela. Aristote en fait un cinquième corps, et Platon nie qu’ellesoit un corps. Or, si elle est un cinquième corps, assurément ce corps est au-dessus de tous les autres ; et si elle n’est point un corps, elle les surpasse tous à un titre encore plus élevé. Que fait-elle donc dans un corps terrestre ? que fait la chose la plus subtile, la plus légère, la plus active de toutes, dans une masse si grossière, si pesante et si inerte ? Une nature à ce point excellente ne pourra-t-elle pas élever son corps dans le ciel ? Et si maintenant des corps terrestres ont la vertu de retenir les âmes en bas, les âmes ne pourront-elles pas un jour élever en haut des corps terrestres ?\par
Passons à ces miracles de leurs dieux qu’ils opposent à ceux de nos martyrs, et nous verrons qu’ils nous justifient. Certes, si jamais les dieux païens ont fait quelque chose d’extraordinaire, c’est ce que rapporte Varron d’une vestale qui, accusée d’avoir violé son vœu de chasteté, puisa de l’eau du Tibre dans un crible et la porta à ses juges, sans qu’il s’en répandît une seule goutte. Qui soutenait sur le crible le poids de l’eau ? qui l’empêchait de fuir à travers tant d’ouvertures ? Ils répondront que c’est quelque dieu ou quelque démon. Si c’est un dieu, en est-il un plus puissant que celui qui a créé le monde ? et si c’est un démon, est-il plus puissant qu’un ange soumis au Dieu créateur du monde ? Si donc un dieu inférieur, ange ou démon, a pu tenir suspendu un élément pesant et liquide, en sorte qu’on eût dit que l’eau avait changé de nature, le Dieu tout-puissant, qui a créé tous les éléments, ne pourra-t-il ôter à un corps terrestre sa pesanteur, pour qu’il habite, renaissant et vivifié. Où il plaira à l’esprit qui le vivifie ?\par
D’ailleurs, puisque ces philosophes veulent que l’air soit entre le feu et l’eau, au-dessous de l’un et au-dessus de l’autre, d’où vient que nous le trouvons souvent entre l’eau et l’eau, ou entre l’eau et la terre ? Qu’est-ce que les nuées, selon eux ? de l’eau, sans doute ; et cependant, ne trouve-t-on pas l’air entre elles et les mers ? Par quel poids et quel ordre des éléments, des torrents d’eau, très impétueux et très abondants, sont-ils suspendus dans les nues, au-dessus de l’air, avant de courir au-dessous de l’air sur la terre ? Et enfin, pourquoi l’air est-il entre le ciel et la terre dans toutes les parties du monde, si sa place est entre le ciel et l’eau, comme celle de l’eau est entre l’air et la terre ?\par
Bien plus, si l’ordre des éléments veut, comme le dit Platon, que les deux extrêmes, c’est-à-dire le feu et la terre, soient unis par les deux autres qui sont au milieu, c’est-à-dire l’eau et le feu, et que le feu occupe le plus haut du ciel, et la terre la plus basse partie du monde comme une sorte de fondement, de telle sorte que la terre ne puisse être dans le ciel, pourquoi le feu est-il sur la terre ? Car enfin, dans leur système, ces deux éléments, la terre et le feu, le plus bas et le plus haut, doivent se tenir si bien, chacun à sa place, que ni celui qui doit être en bas ne puisse monter en haut, ni celui qui est en haut descendre en bas. Ainsi, puisqu’à leur avis il ne peut y avoir la moindre parcelle de feu dans le ciel, nous ne devrions pas voir non plus la moindre parcelle de feu sur la terre. Cependant le feu est si réellement sur la terre, et même sous la terre, que les sommets des montagnes le vomissent ; outre qu’il sert sur la terre aux différents usages des hommes, et qu’il naît même dans la terre, puisque nous le voyons jaillir du bois et du caillou, qui sont sans doute des corps terrestres. Mais le feu d’en liant, disent-ils, est un feu tranquille, pur, inoffensif et éternel, tandis que celui-ci est violent, chargé de vapeur, corruptible et corrompant. Il ne corrompt pourtant pas les montagnes et les cavernes, où il brûle continuellement. Mais je veux qu’il soit différent de l’autre, afin de pouvoir servir à nos besoins. Pourquoi donc ne veulent-ils pas que la nature des corps terrestres, devenue un jour incorruptible, puisse un jour se mettre en harmonie avec celle du ciel, comme aujourd’hui le feu corruptible s’unit avec la terre ? Ils ne sauraient donc tirer aucun avantage ni du poids, ni de l’ordre des éléments, pour montrer qu’il est impossible au Dieu tout-puissant de modifier nos corps de telle sorte qu’ils puissent demeurer dans le ciel.
\subsection[{Chapitre XII}]{Chapitre XII}

\begin{argument}\noindent Contre les calomnies et les railleries des infidèles au sujet de la résurrection des corps.
\end{argument}

\noindent Mais nos adversaires nous pressent de questions minutieuses et ironiques sur la résurrection de la chair ; ils nous demandent si les créatures avortées ressusciteront ; et comme Notre-Seigneur a dit : « En vérité, je vous le déclare, le moindre cheveu de votre tête ne périra pas » ; ils nous demandent encore si la taille et la force seront égales en tous, ou si les corps seront de différentes grandeurs. Dans le premier cas, d’où les êtres avortés, supposé qu’ils ressuscitent, prendront-ils ce qui leur manquait en naissant ? Et si l’on dit qu’ils ne ressusciteront pas, n’étant pas véritablement nés, la même difficulté s’élève touchant les petits enfants venus à terme, mais morts au berceau. En effet, nous ne pouvons pas dire que ceux qui n’ont pas été seulement engendrés, mais régénérés par le baptême, ne ressusciteront pas De plus, ils demandent de quelle stature seront les corps dans cette égalité de tous : s’ils ont tous la longueur et la largeur de ceux qui ont été ici les plus grands, où plusieurs prendront-ils ce qui leur manquait sur terre pour atteindre à cette hauteur ? Autre question : si, comme dit l’Apôtre, nous devons parvenir à « la plénitude de l’âge de Jésus-Christ » ; si, selon le même Apôtre, « Dieu nous a prédestinés pour être rendus conformes à l’image de son Fils » ; si, en d’autres termes, le corps de Jésus-Christ doit être la mesure de tous ceux qui seront dans son royaume, il faudra, disent-ils, retrancher de la stature de plusieurs hommes. Et alors comment s’accomplira cette parole : « Que le moindre cheveu de votre tête ne périra pas » ? Et au sujet des cheveux mêmes, ne demandent-ils pas encore si nous aurons tous ceux que le barbier nous a retranchés ? Mais dans ce cas, de quelle horrible difformité ne serions-nous pas menacés ! Car ce qui arrive aux cheveux ne manquerait pas d’arriver aux ongles. Où serait donc alors la bienséance, qui doit avoir ses droits en cet état bienheureux plus encore que dans cette misérable vie ? Dirons-nous que tout cela ne reviendra pas aux ressuscités ? Tout cela périra donc ; et alors,pourquoi prétendre qu’aucun des cheveux de notre tête ne périra ? Mêmes difficultés sur la maigreur et l’embonpoint : car si tous les ressuscités sont égaux, les uns ne seront plus maigres, et les autres ne seront plus gras. Il y aura à retrancher aux uns, à ajouter aux autres, Les uns gagneront ce qu’ils n’avaient pas, les autres perdront ce qu’ils avaient.\par
On ne soulève pas moins d’objections au sujet de la corruption et de la dissolution des corps morts, dont une partie s’évanouit en poussière et une autre s’évapore dans l’air ; de plus, les uns sont mangés par les bêtes, les autres consumés par le feu ; d’autres tombés dans l’eau par suite d’un naufrage ou autrement, se corrompent et se liquéfient. Comment croire que tout cela puisse se réunir pour reconstituer un corps ? — Ils se prévalent encore des défauts qui viennent de naissance ou d’accident ; ils allèguent les enfantements monstrueux, et demandent d’un air de dérision si les corps contrefaits ressusciteront dans leur même difformité. Répondons-nous que la résurrection fera disparaître tous ces défauts ? ils croient nous convaincre de contradiction par les cicatrices du Sauveur que nous croyons ressuscitées avec lui. Mais voici la question la plus difficile : À qui doit revenir la chair d’un homme, quand un autre homme affamé en aura fait sa nourriture ? Cette chair s’est assimilée à la substance de celui qui l’a dévorée et a rempli les vides qu’avait creusés chez lui la maigreur. On demande donc si elle retournera au premier homme qui la possédait, ou à celui qui s’en est nourri. C’est ainsi que nos adversaires prétendent livrer au ridicule la foi dans la résurrection, sauf à promettre à l’âme, avec Platon, une vicissitude éternelle de véritable misère et de fausse félicité, ou à soutenir avec Porphyre qu’après diverses révolutions à travers les corps, elle verra la fin de ses misères, non en prenant un corps immortel, mais en restant affranchie de toute espèce de corps.
\subsection[{Chapitre XIII}]{Chapitre XIII}

\begin{argument}\noindent Si les enfants avortés, étant compris au nombre des morts, ne le seront pas au nombre des ressuscités.
\end{argument}

\noindent Je vais répondre, avec l’aide de Dieu, aux objections que j’ai mises dans la bouche de nos adversaires. Je n’oserai nier, ni assurer que les enfants avortés, qui ont vécu dans le sein de leur mère et y sont morts, doivent ressusciter. Cependant je ne vois pas pourquoi, étant du nombre des morts, ils seraient exclus de la résurrection. En effet, ou bien tous les morts ne ressusciteront pas, et il y aura des âmes qui demeureront éternellement sans corps, comme celles qui n’en ont eu que dans le sein maternel ; ou bien, si toutes les âmes humaines reprennent les corps qu’elles ont eus, en quelque lieu qu’elles les aient laissés, je ne vois pas de raison pour exclure de la résurrection les enfants même qui sont morts dans le sein de leur mère. Mais à quelque sentiment qu’on s’arrête, tout au moins faut-il leur appliquer, s’ils ressuscitent, ce que nous allons dire des enfants déjà nés.
\subsection[{Chapitre XIV}]{Chapitre XIV}

\begin{argument}\noindent Si les enfants ressusciteront avec le même corps qu’ils avaient à l’âge où ils sont morts.
\end{argument}

\noindent Que dirons-nous donc des enfants, sinon qu’ils ne ressusciteront pas dans l’état de petitesse où ils étaient en mourant ? Ils recevront, en un instant, par la toute-puissance de Dieu, l’accroissement auquel ils devaient parvenir avec le temps. Quand Notre-Seigneur a dit : « Pas un cheveu de votre tête ne périra » ; il a entendu que nous ne perdrons rien de ce que nous avions, mais non pas que nous ne gagnerons rien de ce qui nous manquait. Or, ce qui manque à un enfant qui meurt, c’est le développement complet de son corps. Il a beau être parfait comme enfant, la perfection de la grandeur corporelle lui manque, et il ne l’atteindra que parvenu au terme de sa croissance. On peut dire en un sens que, dès qu’il est conçu, il possède tout ce qu’il doit acquérir : il le possède idéalement et en puissance, mais non en fait, de même que toutes les parties du corps humain sont contenues dans la semence, quoique plusieursmanquent aux enfants déjà nés, les dents, par exemple, et autres parties analogues. C’est dans cette raison séminale de la matière qu’est renfermé tout ce qu’on ne voit pas encore, tout ce qui doit paraître un jour. C’est en elle que l’enfant, qui sera un jour petit ou grand, est déjà grand ou petit. C’est par elle enfin qu’à la résurrection des corps, nous ne perdrons rien de ce que nous avions ici-bas ; et dussent les hommes ressusciter tous égaux et avec une taille de géants, ceux qui l’ont eue n’en perdront rien, puisque Jésus-Christ a dit : Aucun cheveu de votre-tête ne périra ; et, quant aux autres, l’admirable Ouvrier qui a tiré toutes choses du néant ne sera pas en peine de suppléer à ce qui leur manque.
\subsection[{Chapitre XV}]{Chapitre XV}

\begin{argument}\noindent Si la taille de Jésus-Christ sera le modèle de la taille de tous les hommes, lors de la résurrection.
\end{argument}

\noindent Il est certain que Jésus-Christ est ressuscité avec la même stature qu’il avait à sa mort, et ce serait se tromper que de croire qu’au jour de la résurrection générale, il prendra, pour égaler les plus hautes statures, une grandeur charnelle qu’il n’avait pas, quand il apparut à ses disciples sous la forme qui leur était connue. Maintenant, dirons-nous que les plus grands doivent être réduits à la mesure du Sauveur ? mais alors il serait beaucoup retranché du corps de plusieurs, ce qui va contre cette parole divine : « Pas un cheveu de votre tête ne périra. » Reste donc à dire que chacun prendra la taille qu’il avait dans sa jeunesse, bien qu’il soit mort vieux, ou celle qu’il aurait dû prendre un jour, si la mort ne l’eût prévenu. Quant à cette mesure de l’âge parfait de Jésus-Christ, dont parle l’Apôtre, ou bien il ne faut pas l’entendre à la lettre et dire que la mesure parfaite de ce chef mystique trouvera son accomplissement dans la perfection de ses membres ; ou, si nous l’entendons de la résurrection des corps, il faut croire que les corps ne ressusciteront ni au-dessus, ni au-dessous de la jeunesse, mais dans l’âge et dans la force où nous savons que Jésus-Christ était arrivé. Les plus savants même d’entre les païens ont fixé laplénitude de la jeunesse à l’âge de trente ans environ, après lequel l’homme commence à être sur le retour et incline vers la vieillesse. Aussi l’Apôtre n’a-t-il pas dit : À la mesure du corps ou de la stature ; mais : À la mesure de l’âge parfait de Jésus-Christ.
\subsection[{Chapitre XVI}]{Chapitre XVI}

\begin{argument}\noindent Comment il faut entendre que les saints seront rendus conformes à l’image du Fils de Dieu.
\end{argument}

\noindent Et quand l’Apôtre parle de ces « prédestinés qui seront rendus conformes à l’image du Fils de Dieu », on peut fort bien entendre qu’il s’agit de l’homme intérieur. C’est ainsi qu’il est dit dans un autre endroit : « Ne vous conformez point au siècle, mais réformez-vous par un renouvellement de votre esprit. » C’est par la même partie de notre être que nous devons réformer pour n’être pas conformes au siècle, que nous deviendrons conformes au Fils de Dieu. On peut encore entendre cette parole dans ce sens que, Dieu-lui-même s’étant rendu conforme à nous, quand il a pris la condition mortelle, de même nous lui serons conformes par l’immortalité, ce qui a rapport aussi à la résurrection des corps. Si l’on veut expliquer ces paroles par la forme sous laquelle les corps ressusciteront, cette conformité, aussi bien que la mesure dont parle l’Apôtre, ne regardera que l’âge, et non pas la taille. Chacun donc ressuscitera aussi grand qu’il était ou qu’il aurait été dans sa jeunesse, et quant à la forme, il importera peu que ce soit celle d’un vieillard ou d’un enfant, puisque ni l’esprit ni le corps ne seront plus sujets à aucune faiblesse. Si donc on s’avisait de soutenir que chacun ressuscitera dans la même conformation des membres qu’il avait à sa mort, il n’y aurait pas lieu à s’engager contre lui dans une laborieuse discussion.
\subsection[{Chapitre XVII}]{Chapitre XVII}

\begin{argument}\noindent Si les femmes, en ressuscitant, garderont leur sexe.
\end{argument}

\noindent De ces paroles : « Jusqu’à ce que nous par« venions tous à l’état d’homme parfait, à lamesure de la plénitude de l’âge de Jésus-Christ », et de celles-ci : « Rendus conformes à l’image du Fils de Dieu », quelques-uns ont conclu que les femmes ne ressusciteront point dans leur sexe, mais dans celui de l’homme, parce que Dieu a formé l’homme seul du limon de la terre, et qu’il a tiré la femme de l’homme. Pour moi, j’estime plus raisonnable de croire à la résurrection de l’un et de l’autre sexe. Car il n’y aura plus alors cette convoitise qui nous cause aujourd’hui de la confusion. Aussi bien, avant le péché, l’homme et la femme étaient nus, et ils n’en rougissaient pas. Le vice sera donc retranché de nos corps, mais leur nature subsistera. Or, le sexe de la femme n’est point en elle un vice ; c’est sa nature. D’ailleurs, il n’y aura plus alors ni commerce charnel ni enfantement, et la femme sera ornée d’une beauté nouvelle qui n’allumera pas la convoitise désormais disparue, mais qui glorifiera la sagesse et la bonté de Dieu, qui a fait ce qui n’était pas, et délivré de la corruption ce qu’il a fait. Il fallait, au commencement du genre humain, qu’une côte fût tirée du flanc de l’homme endormi pour en faire une femme ; car c’est là un symbole prophétique de Jésus-Christ et de son Église. Ce sommeil d’Adam était la mort du Sauveur, dont le côté fut percé d’une lance sur la croix, après qu’il eut rendu l’esprit ; il en sortit du sang et de l’eau, lesquels figurent les sacrements, sur lesquels l’Église est « édifiée » ; aussi l’Écriture s’est-elle servie de ce mot : car elle ne dit pas que Dieu forma ou façonna la côte du premier homme, mais qu’il « l’édifia en femme », d’où vient que l’Apôtre appelle l’Église l’édifice du corps de Jésus-Christ. La femme est donc la créature de Dieu aussi bien que l’homme, mais elle a été faite de l’homme, pour consacrer l’unité, et elle en a été faite de cette manière pour figurer Jésus-Christ et l’Église. Celui qui a créé l’un et l’autre sexe les rétablira tous deux. Aussi Jésus-Christ lui-même quand les Sadducéens, qui niaient la résurrection, lui demandèrent auquel des sept frères appartiendrait la femme qui les avait tous eus pour maris l’un après l’autre, chacun voulant, selon le précepte de la loi, perpétuerla postérité de son frère : « Vous vous trompez leur dit-il, faute de connaître les Écritures et le pouvoir de Dieu. » Et loin de dire comme c’était le moment : Que me demandez-vous ? celle dont vous me parlez ne sera plus une femme, mais un homme, il ajouta ; « Car à la résurrection on ne se mariera point et où n’épousera point ; mais tous seront comme les anges de Dieu dans le ciel. » Ils seront en effet égaux aux anges pour l’immortalité et la béatitude, mais non quant au corps, ni quant à la résurrection, dont les anges n’ont pas eu besoin, parce qu’ils n’ont pas pu mourir. Notre-Seigneur a donc dit qu’il n’y aura point de noces à la résurrection, mais non pas qu’il n’y aura point de femmes ; et il l’a dit en une occasion où la réponse naturelle était : Il n’y aura point de femmes, s’il avait prévu qu’il ne devait point y en avoir. Bien plus, il a déclaré que la différence des sexes subsisterait, en disant : « On ne s’y mariera point », ce qui regarde les femmes, et : « On n’y épousera point », ce qui regarde les hommes. Aussi celles qui se marient ici-bas, comme ceux qui y épousent, seront à la résurrection ; mais ils n’y feront point de telles alliances.
\subsection[{Chapitre XVIII}]{Chapitre XVIII}

\begin{argument}\noindent De l’homme parfait, c’est-à-dire de Jésus-Christ, et de son corps, c’est-à-dire de l’église, qui en est la plénitude.
\end{argument}

\noindent Pour comprendre ce que dit l’Apôtre, que nous parviendrons tous à l’état d’homme parfait, il faut examiner avec attention toute la suite de sa pensée. Il s’exprime ainsi : « Celui qui est descendu est celui-là même qui est monté au-dessus de tous les cieux, afin de consommer toutes choses. Lui-même en a établi quelques-uns apôtres, d’autres prophètes, ceux-ci évangélistes, ceux-là pasteurs et docteurs, pour la consommation des saints, l’œuvre du ministère et l’édifice du corps de Jésus-Christ, jusqu’à ce que nous parvenions tous à l’unité d’une même foi, à la connaissance du Fils de Dieu, à l’état d’homme parfait et à la mesure de la plénitude de l’âge de Jésus-Christ, afin que nous ne soyons plus comme des enfants, nous laissant aller à tout vent de doctrine et aux illusions des hommes fourbes quiveulent nous engager dans l’erreur, mais que, pratiquant la vérité par la charité, nous croissions en toutes choses dans Jésus-Christ, qui est la tête d’où tout le corps bien lié et bien disposé reçoit, selon la mesure et la force de chaque partie, le développement nécessaire pour s’édifier soi-même dans la charité. » Voilà quel est l’homme parfait : la tête d’abord, puis le corps composé de tous les membres, qui recevront la dernière perfection en leur temps. Chaque jour cependant, de nouveaux éléments se joignent à ce corps, tandis que s’édifie l’Église à qui l’on dit : « Vous êtes le corps de Jésus-Christ et ses membres » ; et ailleurs : « Pour son corps qui est l’Église » ; et encore : « Nous ne sommes tous ensemble qu’un seul pain et qu’un seul corps. » C’est de l’édifice de ce corps qu’il est dit ici : « Pour la consommation des saints, pour l’œuvre du ministère et l’édifice du corps de Jésus-Christ. » Puis l’Apôtre ajoute ce passage dont il est question : « Jusqu’à ce que nous parvenions tous à « l’unité d’une même foi, à la connaissance du Fils de Dieu, à l’état d’homme parfait et à la mesure de la plénitude de l’âge de Jésus-Christ » ; et le reste, montrant enfin de quel corps on doit entendre cette mesure par ces paroles : « Afin que nous croissions en toutes tout le corps bien lié et bien disposé reçoit, selon la mesure et la force de chaque partie, le développement qui lui convient. » Comme il y a une mesure de chaque partie, il y en a aussi une de tout le corps, composé de toutes ces parties ; et c’est la mesure de la plénitude dont il est dit : « À la mesure de la plénitude de l’âge de Jésus-Christ. » L’Apôtre fait encore mention de cette plénitude, lorsque, parlant de Jésus-Christ, il dit ; « Il l’a établi pour être le chef de toute l’Église, qui est son corps et sa plénitude, lui qui consomme tout en tous. » Mais, lors même qu’il faudrait entendre le passage dont il s’agit de la résurrection, qui nous empêcherait d’appliquer aussi à la femme ce qu’il dit de l’homme, en prenant l’{\itshape homme} pour tous les deux, comme dans ce verset du Psaume : « Bienheureux l’homme qui craint le Seigneur ! » Car assurément les femmes qui craignent le Seigneur sont comprises dans la pensée du Psalmiste.
\subsection[{Chapitre XIX}]{Chapitre XIX}

\begin{argument}\noindent Tous les défauts corporels, qui, pendant cette vie, sont contraires à la beauté de l’homme, disparaîtront à la résurrection, la substance naturelle du corps terrestre devant seule subsister, mais avec d’autres proportions d’une justesse accomplie.
\end{argument}

\noindent Est-il besoin de répondre maintenant aux objections tirées des ongles et des cheveux ? Si l’on a bien compris une fois qu’il ne périra rien de notre corps, afin qu’il n’ait rien de difforme, on comprendra aussi aisément que ce qui ferait une monstrueuse énormité sera distribué dans toute la masse du corps, et non pas accumulé à une place où la proportion des membres en serait altérée. Si, après avoir fait un vase d’argile, on le voulait défaire pour en recomposer un vase nouveau, il ne serait pas nécessaire que cette portion de terre qui formait l’anse ou le fond dans le premier vase, les formât aussi dans le second ; il suffirait que toute l’argile y fût employée. Si donc les ongles et les cheveux, tant de fois coupés, ne peuvent revenir à leur place qu’en produisant une difformité, ils n’y reviendront pas. Cependant ils ne seront pas anéantis, parce qu’ils seront changés en la même chair à laquelle ils appartenaient, afin d’y occuper une place où ils ne troublent pas l’économie générale des parties. Je ne dissimule pas, au surplus, que cette parole du Seigneur : « Pas un cheveu de votre tête ne périra », ne paraisse s’appliquer plutôt au nombre des cheveux qu’à leur longueur. C’est dans ce sens qu’il a dit aussi : « Tous les cheveux de votre tête sont comptés. » Je ne crois donc pas que rien doive périr de notre corps de tout ce qui lui était naturel ; je veux seulement montrer que tout ce qui en lui était défectueux, et servait à faire voir la misère de sa condition, sera rendu à sa substance transfigurée, le fond de l’être restant tout entier, tandis que la difformité seule périra. Si un artisan ordinaire, qui a mal fait une statue, peut la refondre si bien qu’il en conserve toutes les parties, sans y laisser néanmoins ce qu’elle avait de difforme, que ne faut-il pas attendre, je le demande, du suprême Artisan ? Ne pourra-t-il ôter et retrancher aux corps des hommes toutes les difformités naturelles ou monstrueuses, qui sont une condition de cette viemisérable, mais qui ne peuvent convenir à la félicité future des saints, comme ces accroissements naturels sans doute, mais cependant disgracieux, de notre corps, sans rien enlever pour cela de sa substance ?\par
Il ne faut point dès lors que ceux qui ont trop ou trop peu d’embonpoint appréhendent d’être au séjour céleste ce qu’ils ne voudraient pas être, même ici-bas. Toute la beauté du corps consiste, en effet, en une certaine proportion de ses parties, couvertes d’un coloris agréable. Or, quand cette proportion manque, ce qui choque la vue, c’est qu’il y a quelque chose qui fait défaut, ou quelque chose d’excessif. Ainsi donc, cette difformité qui résulte de la disproportion des parties du corps disparaîtra, lorsque le Créateur, par des moyens connus de lui, suppléera à ce qui manque ou ôtera le superflu. Et quant à la couleur des chairs, combien ne sera-t-elle pas vive et éclatante en ce séjour où : « Les justes brilleront comme le soleil dans le royaume de leur père » ? Il faut croire que Jésus-Christ déroba cet éclat aux yeux de ses disciples, quand il parut devant eux après sa résurrection ; car ils n’auraient pu le soutenir, et cependant ils avaient besoin de regarder leur maître pour le reconnaître. C’est pour cette raison qu’il leur fit toucher ses cicatrices, qu’il but et mangea avec eux, non par nécessité, mais par puissance. Quand on ne voit pas un objet présent, tout en voyant d’autres objets également présents, comme il arriva aux disciples qui ne virent pas alors l’éclat du visage de Jésus-Christ, quoique présent, et qui pourtant voyaient d’autres choses, les Grecs appellent cet état {\itshape aorasia}, mot que les Latins ont traduit dans la Genèse par {\itshape caecitas}, faute d’un autre équivalent. C’est l’{\itshape aveuglement} dont les Sodomites furent frappés, lorsqu’ils cherchaient la porte de Loth sans pouvoir la trouver. En effet, si c’eût été chez eux une véritable {\itshape cécité}, comme celle qui empêche de rien voir, ils n’auraient point cherché la porte pour entrer, mais des guides pour les ramener,\par
Or, je ne sais comment, l’affection que nous avons pour les bienheureux martyrs nous fait désirer de voir dans le ciel les cicatrices des plaies qu’ils ont reçues pour le nom de Jésus-Christ, et peut-être les verrons-nous. Ce ne sera pas une difformité dans leur corps, mais une marque d’honneur, qui donnera de l’éclat, non point à leur corps, mais à leur gloire. Il ne faut pas croire toutefois que les membres qu’on leur aura coupés leur manqueront à la résurrection, eux à qui il a été dit : « Pas un cheveu de votre tête ne périra. » Mais, s’il est à propos qu’on voie, dans le siècle nouveau, ces marques glorieuses de leur martyre gravées jusque dans leur chair immortelle, on doit penser que les endroits où ils auront été blessés ou mutilés conserveront seulement une cicatrice, en sorte qu’ils ne laisseront pas de recouvrer les membres qu’ils avaient perdus. La foi nous assure, il est vrai, que dans l’autre vie aucun des défauts de notre corps ne paraîtra plus ; mais ces marques de vertu ne peuvent être considérées comme des défauts.
\subsection[{Chapitre XX}]{Chapitre XX}

\begin{argument}\noindent Au jour de la résurrection, la substance de notre corps, de quelque manière qu’elle ait été dissipée, sera réunie intégralement.
\end{argument}

\noindent Loin de nous la crainte que la toute-puissance du Créateur ne puisse rappeler, pour ressusciter les corps, toutes les parties qui ont été dévorées par les bêtes, ou consumées par le feu, ou changées en poussière, ou dissipées dans l’air ! Loin de nous la pensée que rien soit tellement caché dans le sein de la nature, qu’il puisse se dérober à la connaissance ou au pouvoir du Créateur ! Cicéron, dont l’autorité est si grande pour nos adversaires, voulant définir Dieu autant qu’il en est capable : « C’est, dit-il, un esprit libre et indépendant, dégagé de toute composition mortelle, qui connaît et meut toutes choses, et qui a lui-même un mouvement éternel. » Cicéron s’inspire ici des plus grands philosophes. Eh bien ! pour parler selon leur sentiment, peut-il y avoir une chose qui reste inconnue à celui qui connaît tout, ou qui se dérobe pour jamais à celui qui meut tout ? Ceci me conduit â répondre à cette questionqui paraît plus difficile que toutes les autres : à qui, lors de la résurrection, appartiendra la chair d’un homme mort, devenue celle d’un homme vivant ? Supposez, en effet, qu’un malheureux, pressé par la faim, mange de la chair d’un homme mort, et c’est là une extrémité que nous rencontrons quelquefois dans l’histoire et dont nos misérables temps fournissent aussi plus d’un exemple, peut-on soutenir avec quelque raison que toute cette substance ait disparu par les sécrétions et qu’il ne s’en soit assimilé aucune partie à la chair de celui qui s’en est nourri, alors que l’embonpoint qu’il a recouvré montre assez quelles ruines il a réparées par ce triste secours ? Mais j’ai déjà indiqué plus haut le moyen de résoudre cette difficulté ; car toutes les chairs que la faim a consommées se sont évaporées dans l’air, et nous avons reconnu que la toute-puissance de Dieu en peut rappeler tout ce qui s’y est évanoui. Cette chair mangée sera donc rendue à celui en qui elle a d’abord commencé d’être une chair humaine, puisque l’autre ne l’a que d’emprunt, et c’est comme un argent prêté qu’il doit rendre. La sienne, que la faim avait amaigrie, lui sera rendue par celui qui peut rappeler à son gré tout ce qui a disparu ; et alors même qu’elle serait tout à fait anéantie et qu’il n’en serait rien resté dans les plus secrets replis de la nature, le Dieu tout-puissant saurait bien y suppléer par quelque moyen. La Vérité ayant déclaré que « pas un cheveu de votre tête ne périra », il serait absurde de penser qu’un cheveu ne puisse se perdre, et que tant de chairs dévorées ou consumées par la faim pussent périr.\par
De toutes ces questions que nous avons traitées et examinées selon notre faible pouvoir, il résulte que les corps auront, à la résurrection, la même taille qu’ils avaient dans leur jeunesse, avec la beauté et la proportion de tous leurs membres. Il est assez vraisemblable que, pour garder cette proportion, Dieu distribuera dans toute la masse du corps ce qui, placé en un seul endroit, serait disgracieux, et qu’ainsi il pourra même ajouter quelque chose à notre stature. Que si l’on prétend que chacun ressuscitera dans la même stature qu’il avait à la mort, à la bonne heure, pourvu qu’on bannisse toute difformité, toute faiblesse, toute pesanteur, toute corruption, et enfin tout autre défaut contraire à la beauté de ce royaume, où les enfants de la résurrection et de la promesse seront égaux aux anges de Dieu, sinon pour le corps et pour l’âge, au moins pour la félicité.
\subsection[{Chapitre XXI}]{Chapitre XXI}

\begin{argument}\noindent Du corps spirituel en qui sera renouvelée et transformée la chair des bienheureux.
\end{argument}

\noindent Tout ce qui s’est perdu des corps vivants ou des cadavres après la mort sera dès lors rétabli avec ce qui est demeuré dans les tombeaux, et ressuscitera en un corps nouveau et spirituel, revêtu d’incorruptibilité et d’immortalité. Mais alors même que, par quelque fâcheux accident ou par la cruauté de mains ennemies, un corps humain serait entièrement réduit en poudre, et que, dissipé en air et en eau, il ne se trouverait pour ainsi dire nulle part, il ne pourra néanmoins être soustrait à la toute-puissance du Créateur, et pas un cheveu de sa tête ne périra. La chair devenue spirituelle sera donc soumise à l’esprit ; mais ce sera une chair néanmoins, et non un esprit, tout comme quand l’esprit devenu charnel a été soumis à la chair, il reste un esprit, et non pas une chair. Nous avons donc de cela ici-bas une expérience qui est un effet de la peine du péché. En effet, ceux-là n’étaient pas charnels selon la chair, mais selon l’esprit, à qui l’Apôtre disait : « Je n’ai pu vous parler comme à des hommes spirituels, mais comme à des personnes qui sont encore charnelles. » Et l’homme spirituel, en cette mortelle vie, ne laisse pas d’être encore charnel selon le corps, et de voir en ses membres une loi qui résiste à la loi de son esprit. Mais il sera spirituel, même selon le corps, lorsque la chair sera ressuscitée et que cette parole de saint Paul se trouvera accomplie : « Le corps est semé animal, et il ressuscitera spirituel. » Or, quelles seront les perfections de ce corps spirituel ? Comme nous n’en avons pas encore l’expérience, j’aurais peur qu’il n’y eût de la témérité à en parler. Toutefois, puisqu’il y va de la gloire de Dieu de ne pas cacher la joie qu’allume en nous l’espérance, et que le Psalmiste, dans les plus violents transports d’unsaint et ardent amour, s’écrie : « Seigneur, j’ai aimé la beauté de votre maison ! » tâchons, avec son aide, de conjecturer, par les grâces qu’il fait aux bons et aux méchants en cette vie de misère, combien doit être grande celle dont nous ne pouvons parler dignement, faute de l’avoir éprouvée. Je laisse à part ce temps où Dieu créa l’homme droit ; je laisse à part la vie bienheureuse de ce couple fortuné dans les délices du paradis terrestre, puisqu’elle fut si courte que leurs enfants n’eurent pas le bonheur de la goûter. Je ne parle que de cette condition misérable que nous connaissons, en laquelle nous sommes, qui est exposée à une infinité de tentations, ou, pour mieux dire, qui n’est qu’une tentation continuelle, quelques progrès que nous fassions dans la vertu. Eh bien ! qui pourrait compter encore tous les témoignages que Dieu y donne aux hommes de sa bonté ?
\subsection[{Chapitre XXII}]{Chapitre XXII}

\begin{argument}\noindent Des misères et des maux de cette vie, qui sont des peines du péché du premier homme, et dont on ne peut être délivré que par la grâce de Jésus-Christ.
\end{argument}

\noindent \hspace{1em}Que toute la race des hommes ait été condamnée dans sa première origine, cette vie même, s’il faut l’appeler une vie, le témoigne assez par les maux innombrables et cruels dont elle est remplie. En effet, que veut dire cette profonde ignorance où naissent les enfants d’Adam, principe de toutes leurs erreurs, et dont ils ne peuvent s’affranchir sans le travail, la douleur et la crainte ? Que signifient tant d’affections vaines et nuisibles d’où naissent les cuisants soucis, les inquiétudes, les tristesses, les craintes, les fausses joies, les querelles, les procès, les guerres, les trahisons, les colères, les inimitiés, les tromperies, la fraude, la flatterie, les larcins, les rapines, la perfidie, l’orgueil, l’ambition, l’envie, les homicides, les parricides, la cruauté, l’inhumanité, la méchanceté, la débauche, l’insolence, l’impudence, l’impudicité, les fornications, les adultères, les incestes, les péchés contre nature de l’un et de l’autre sexe, et tant d’autres impuretés qu’on n’oserait seulement nommer : sacrilèges, hérésies, blasphèmes, parjures, oppression des innocents, calomnies, surprises, prévarications, fauxtémoignages, jugements injustes, violences brigandages, et autres malheurs semblable que ne saurait embrasser la pensée, mais qui remplissent et assiègent la vie ? Il est vrai que ces crimes sont l’œuvre des méchants ; mais ils ne laissent pas de venir tous de cette ignorance et de cet amour déréglé, comme d’une racine que tous les enfants d’Adam portent en eux en naissant. Qui en effet, ignore dans quelle ignorance manifeste chez les enfants, et dans combien de passions qui se développent au sortir même de l’enfance, l’homme vient au monde ! Certes, si on le laissait vivre à sa guise et faire ce qui lui plairait, il n’est pas un des crimes que j’ai nommés, sans parler de ceux que je n’ai pu nommer, où on ne le vît se précipiter.\par
Mais, par un conseil de la divine Providence, qui n’abandonne pas tout à fait ceux qu’elle a condamnés, et qui, malgré sa colère, n’arrête point le cours de ses miséricordes, la loi et l’instruction veillent contre ces ténèbres et ces convoitises dans lesquelles nous naissons. Bienfait inestimable, mais qui ne s’opère point sans peines et sans douleurs. Pourquoi, je vous le demande, toutes ces menaces que l’on fait aux enfants, pour les retenir dans le devoir ? pourquoi ces maîtres, ces gouverneurs, ces férules, ces fouets, ces verges dont l’Écriture dit qu’il faut souvent se servir envers un enfant qu’on aime, de peur qu’il ne devienne incorrigible et indomptable ? pourquoi toutes ces peines, sinon pour vaincre l’ignorance et réprimer la convoitise, deux maux qui avec nous entrent dans le monde ? D’où vient que nous avons de la peine à nous souvenir d’une chose, et que nous l’oublions sans peine ; qu’il faut beaucoup de travail pour apprendre, et point du tout pour ne rien savoir ; qu’il en coûte tant d’être diligent, et si peu d’être paresseux ? Cela ne dénote-t-il pas clairement à quoi la nature corrompue se porte par le poids de ses inclinations, et de quel secours elle a besoin pour s’en relever ? La paresse, la négligence, la lâcheté, la fainéantise, sont des vices qui fuient le travail, tandis que le travail même, tout bienfaisant qu’il puisse être, est une peine.\par
Mais outre les peines de l’enfance, sans lesquelles rien ne peut s’apprendre de ce queveulent les parents, qui veulent rarement quelque chose d’utile, où est la parole capable d’exprimer, où est la pensée capable de comprendre toutes celles où les hommes sont sujets et qui sont inséparables de leur triste condition ? Quelle appréhension et quelle douleur ne nous causent pas, et la mort des personnes qui nous sont chères, et la perte des biens, et les condamnations, et les supercheries des hommes, et les faux soupçons, et toutes les violences que l’on peut avoir à souffrir, comme les brigandages, les captivités, les fers, la prison, l’exil, les tortures, les mutilations, les infamies et les brutalités, et mille autres souffrances horribles qui nous accablent incessamment ? À ces maux ajoutez une multitude d’accidents auxquels les hommes ne contribuent pas : le chaud, le froid, les orages, les inondations, les foudres, la grêle, les tremblements de terre, les chutes de maison, les venins des herbes, des eaux, de l’air ou des animaux, les morsures des bêtes, ou mortelles ou incommodes, la rage d’un chien, cet animal naturellement ami de l’homme, devenu alors plus à craindre que les lions et les dragons, et qui rend un homme qu’il a mordu plus redoutable aux siens que les bêtes les plus farouches. Que ne souffrent point ceux qui voyagent sur mer et sur terre ? Qui peut se déplacer sans s’exposer à quelque accident imprévu ? Un homme qui se portait fort bien, revenant chez lui, tombe, se rompt la jambe et meurt. Le moyen d’être, en apparence, plus en sûreté qu’un homme assis dans sa chaise ! Héli tombe de la sienne et se tue. Quels accidents les laboureurs, ou plutôt tous les hommes, ne craignent-ils pas pour les biens de la campagne, tarit du côté du ciel et de la terre que du côté des animaux ? Ils ne sont assurés de la moisson que quand elle est dans la grange, et toutefois nous en savons qui l’ont perdue, même quand elle y était, par des tempêtes et des inondations. Qui se peut assurer sur son innocence d’être à couvert des insultes des démons, puisqu’on les voit quelquefois tourmenter d’une façon si cruelle les enfants nouvellement baptisés, que Dieu, qui le permet ainsi, nous apprend bien par là à déplorer la misère de cette vie et à désirer la félicité de l’autre ? Que dirai-je des maladies, qui sonten si grand nombre que même les livres des médecins ne les contiennent pas toutes ? la plupart des remèdes qu’on emploie pour les guérir sont autant d’instruments de torture, si bien qu’un homme ne peut se délivrer d’une douleur que par une autre. La soif n’a-t-elle pas contraint quelques malheureux à boire de l’urine ? la faim n’a-t-elle pas porté des hommes, non seulement à se nourrir de cadavres humains qu’ils avaient rencontrés, mais à tuer leurs semblables pour les dévorer ? N’a-t-on pas vu des mères, poussées par une faim exécrable, plonger le couteau dans le sein de leurs enfants ? Le sommeil même, qu’on appelle proprement repos, combien est-il souvent inquiet, accompagné de songes terribles et affreux, qui effraient l’âme et dont les images sont si vives qu’on ne les saurait distinguer des réalités de la veille ? En certaines maladies, ces visions fantastiques tourmentent même ceux qui veillent, sans parler des illusions dont les démons abusent les hommes en bonne santé, afin de troubler du moins les sens de leurs victimes, s’ils ne peuvent réussir à les attirer à leur parti.\par
Il n’y a que la grâce du Sauveur Jésus-Christ, notre Seigneur et notre Dieu, qui nous puisse délivrer de l’enfer de cette misérable vie. C’est ce que son nom même signifie :car Jésus veut dire Sauveur. Et nous lui devons demander surtout qu’après la vie actuelle, il nous délivre d’une autre encore plus misérable, qui n’est pas tant une vie qu’une mort. Ici-bas, bien que nous trouvions de grands soulagements à nos maux dans les choses saintes et dans l’intercession des saints, ceux qui demandent ces grâces ne les obtiennent pas toujours ; et la Providence le veut ainsi, de peur qu’un motif temporel ne nous porte à suivre une religion qu’il faut plutôt embrasser en vue de l’autre vie, où il aura plus de mal. C’est pour cela que la grâce aide les bons au milieu des maux, afin qu’ils les supportent d’autant plus constamment qu’ils ont plus de foi. Les doctes du siècle prétendent que la philosophie y fait aussi quelque chose, cette philosophie que les dieux, selon Cicéron, ont accordée dans sapureté à un petit nombre d’hommes. « Ils n’ont jamais fait, dit-il, et ne peuvent faire un plus grand présent aux hommes. »\par
Cela prouve que ceux mêmes que nous combattons ont été obligés de reconnaître en quelque façon que la grâce de Dieu est nécessaire pour acquérir la véritable philosophie. Et si la véritable philosophie, qui est l’unique secours contre les misères de la condition mortelle, a été donnée à un si petit nombre d’hommes, voilà encore une preuve que ces misères sont des peines auxquelles les hommes ont été condamnés. Or, comme nos philosophes tombent d’accord que le ciel ne nous a pas fait de don plus précieux, il faut croire aussi qu’il n’a pu venir que du vrai Dieu, de ce Dieu qui est reconnu comme le plus grand de tous par ceux-là mêmes qui en adorent plusieurs.
\subsection[{Chapitre XXIII}]{Chapitre XXIII}

\begin{argument}\noindent Des misères de cette vie qui sont propres aux bons indépendamment de celles qui leur sont communes avec les méchants.
\end{argument}

\noindent Outre les maux de cette vie qui sont communs aux bons et aux méchants, les bons ont des traverses particulières à essuyer dans la guerre continuelle qu’ils font à leurs passions. Les révoltes de la chair contre l’esprit sont tantôt plus fortes, tantôt moindres, mais elles ne cessent jamais ; de sorte que, ne faisant jamais ce que nous voudrions, il ne nous reste qu’à lutter contre toute concupiscence mauvaise, autant que Dieu nous en donne le pouvoir, et à veiller continuellement sur nous-mêmes, de crainte qu’une fausse apparence ne nous trompe, qu’un discours artificieux ne nous surprenne, que quelque erreur ne s’empare de notre esprit, que nous ne prenions un bien pour un mal, ou un mal pour un bien, que la crainte ne nous détournede faire ce qu’il faut, que la passion ne nous porte à faire ce qu’il ne faut pas, que le soleil ne se couche sur notre colère, que la peine ne nous entraîne à rendre le mal pour le mal, qu’une tristesse excessive ou déraisonnable ne nous accable, que nous ne soyons ingrats pour un bienfait reçu, que les médisances ne nous troublent, que nous ne portions des jugements téméraires, que nous ne soyons accablés de ceux que l’on porte contre nous, que le péché ne règne en notre corps mortel en secondant nos désirs, que nous ne fassions de nos membres des instruments d’iniquité pour le péché, que notre œil ne suive ses appétits déréglés, qu’un désir de vengeance ne nous entraîne, que nous n’arrêtions nos regards ni nos pensées sur des objets illégitimes, que nous ne prenions du plaisir à entendre quelque parole outrageuse ou déshonnête, que nous ne fassions ce qui n’est pas permis, quoique nous en soyons tentés, que, dans cette guerre pénible et pleine de dangers, nous ne nous promettions la victoire par nos propres forces, ou que nous cédions à l’orgueil de nous l’attribuer au lieu d’en faire honneur à celui dont l’Apôtre dit : « Grâces soient rendues à Dieu, qui nous donne la victoire par Notre-Seigneur Jésus-Christ » ; et ailleurs : « Nous demeurons victorieux au milieu de tous ces maux par la grâce de celui qui nous a aimés. » Sachons pourtant que, quelque résistance que nous opposions aux vices et quelque avantage que nous remportions sur eux, tant que nous sommes dans ce corps mortel, nous ne pouvons manquer de dire à Dieu : « Remettez-nous nos dettes. » Mais dans ce royaume où nous demeurerons éternellement, revêtus de corps immortels, nous n’aurons plus de guerre ni de dettes, comme nous n’en aurions jamais eu, si notre nature était demeurée dans sa première pureté. Ainsi cette guerre même, où nous sommes si exposés et dont nous désirons être délivrés par une dernière victoire, fait partie des maux de cette vie, qui, ainsi que nous venons de l’établir par le dénombrement de tant de misères, a été condamnée par un arrêt divin.
\subsection[{Chapitre XXIV}]{Chapitre XXIV}

\begin{argument}\noindent Des biens dont le Créateur a rempli cette vie, toute exposée qu’elle soit à la damnation.
\end{argument}

\noindent Cependant, il faut louer la justice de Dieu dans ces misères mêmes qui affligent le genre humain ; car de quelle multitude de biens sa bonté n’a-t-elle pas aussi rempli cette vie ! D’abord, il n’a pas voulu arrêter, même après le péché, l’effet de cette bénédiction qu’il a répandue sur les hommes, en leur disant : « Croissez et multipliez et remplissez la terre. » La fécondité est demeurée dans une race justement condamnée ; et bien que le péché nous ait imposé la nécessité de mourir, il n’a pas pu nous ôter cette vertu admirable des semences, ou plutôt cette vertu encore plus admirable qui les produit, et qui est profondément enracinée et comme entée dans la substance du corps. Mais dans ce fleuve ou ce torrent qui emporte les générations humaines, le mal et le bien se mêlent toujours : le mal que nous devons à notre premier père, le bien que nous devons à la bonté du Créateur. Dans le mal originel, il y a deux choses : le péché et le supplice ; et il y en a deux autres dans le bien originel : la propagation et la conformation. J’ai déjà parlé suffisamment de ce double mal, je veux dire du péché, qui vient de notre audace, et du supplice, qui est l’effet du jugement de Dieu, J’ai dessein maintenant de parler des biens que Dieu a communiqués ou communique encore à notre nature, toute corrompue et condamnée qu’elle est. En la condamnant, il ne lui a pas ôté tout ce qu’il lui avait donné : autrement, elle ne serait plus du tout ; et, en l’assujettissant au démon pour la punir, il ne s’est pas privé du pouvoir qu’il avait sur elle, puisqu’il a toujours conservé son empire sur le démon lui-même, qui d’ailleurs ne subsisterait pas un instant sans celui qui est l’être souverain et le principe de tous les êtres.\par
De ces deux biens qui se répandent du sein de sa bonté, comme d’une source féconde, sur la nature humaine, même corrompue et condamnée, le premier, la propagation, fut le premier don que Dieu accorda à l’homme en le bénissant, lorsqu’il fit les premiers ouvrages du monde, dont il se reposa le septième jour. Pour la conformation, il la lui donne sanscesse par son action continuellement créatrice. S’il venait à retirer à soi sa puissance efficace, ses créatures ne pourraient aller au-delà, ni accomplir la durée assignée à leurs mouvements mesurés, ni même conserver l’être qu’elles ont reçu. Dieu a donc créé l’homme de telle façon qu’il lui a donné le pouvoir de se reproduire, sans néanmoins l’y obliger ; et s’il a ôté ce pouvoir à quelques-uns, en les rendant stériles, il ne l’a pas ôté au genre humain. Toutefois, bien que cette faculté soit restée à l’homme, malgré son péché, elle n’est pas telle qu’elle aurait été, s’il n’avait jamais péché. Car depuis que l’homme est déchu par sa désobéissance de cet état de gloire où il avait été créé, il est devenu semblable aux bêtes et engendre comme elles, gardant toujours en lui cependant cette étincelle de raison qui fait qu’il est encore créé à l’image de Dieu. Mais si la conformation ne se joignait pas à la propagation, celle-ci demeurerait oisive et ne pourrait accomplir son ouvrage. Dieu en effet avait-il besoin pour peupler la terre que l’homme et la femme eussent commerce ensemble ? il lui suffisait de créer plusieurs hommes comme il avait créé le premier. Et maintenant même, le mâle et la femelle pourraient s’accoupler, et n’engendreraient rien, sans l’action créatrice de Dieu. De même que l’Apôtre a dit de l’institution spirituelle qui forme l’homme à la piété et à la justice : « Ce n’est ni celui qui plante, ni celui qui arrose, qui est quelque chose, mais Dieu, qui donne l’accroissement » ; ainsi l’on peut dire que ce n’est point l’homme, dans l’union conjugale, qui est quelque chose, mais Dieu qui donne l’être ; que ce n’est point la mère, bien qu’elle porte son fruit dans son sein et le nourrisse, qui est quelque chose, mais Dieu qui donne l’accroissement. Lui seul, par l’action qu’il exerce maintenant encore, fait que les semences se développent, et sortent de ces plis secrets et invisibles qui les tenaient cachées, pour exposer à nos yeux les beautés visibles que nous admirons. Lui seul, liant ensemble par des nœuds admirables la nature spirituelle et la nature corporelle, l’une pour commander, l’autre pour obéir, compose l’être animé, ouvrage si grand et si merveilleux, que non seulement l’homme, qui est un animal raisonnable, et par conséquent plus nobleet plus excellent que tous les animaux de la terre, mais la moindre petite mouche ne peut être attentivement considérée sans étonner l’intelligence et faire louer le Créateur.\par
C’est donc lui qui a donné à l’âme humaine cet entendement où la raison et l’intelligence sont comme assoupies dans les enfants, pour se réveiller et s’exercer avec l’âge, afin qu’ils soient capables de connaître la vérité et d’aimer le bien, et qu’ils acquièrent ces vertus de prudence, de force, de tempérance et de justice nécessaires pour combattre les erreurs et les autres vices, et pour les vaincre par le seul désir du Bien immuable et souverain. Que si cette capacité n’a pas toujours son effet dans la créature raisonnable, qui peut néanmoins exprimer ou seulement concevoir la grandeur du bien renfermé dans ce merveilleux ouvrage du Tout-Puissant ? Outre l’art de bien vivre et d’arriver à la félicité immortelle, art sublime qui s’appelle la vertu, et que la seule grâce de Dieu en Jésus-Christ donne aux enfants de la promesse et du royaume, l’esprit humain n’a-t-il pas inventé une infinité d’arts qui font bien voir qu’un entendement si actif, si fort et si étendu, même en les choses superflues ou nuisibles, doit avoir un grand fonds de bien dans sa nature, pour avoir pu y trouver tout cela ? Jusqu’où n’est pas allée l’industrie des hommes dans l’art de former des tissus, d’élever des bâtiments, dans l’agriculture et la navigation ? Que d’imagination et de perfection dans ces vases de toutes formes, dans cette multitude de tableaux et de statues ! Quelles merveilles ne se font pas sur la scène, qui semblent incroyables à qui n’en a pas été témoin ! Que de ressources et de ruses pour prendre, tuer ou dompter les bêtes farouches ! Combien de sortes de poisons, d’armes, de machines, les hommes n’ont-ils pas inventées contre les hommes mêmes ! combien de secours et de remèdes pour conserver la santé ! combien d’assaisonnements et de mets pour le plaisir de la bouche et pour réveiller l’appétit ! Quelle diversité de signes pour exprimer et faire agréer ses pensées, et au premier rang, la parole et l’écriture ! quelle richesse d’ornements dans l’éloquence et la poésie pour réjouir l’esprit et pour charmer l’oreille, sans parler de tant d’instruments de musique, de tant d’airs et de chants ! Quelle connaissance admirable des mesures et des nombres ! quelle sagacité d’esprit dans la découverte des harmonies et des révolutions des globes célestes ! Enfin, qui pourrait dire toutes les connaissances dont l’esprit humain s’est enrichi touchant les choses naturelles, surtout si on voulait insister sur chacune en particulier, au lieu de les rapporter en général ? Pour défendre même des erreurs et des faussetés, combien les philosophes et les hérétiques n’ont-ils pas fait paraître d’esprit ? car nous ne parlons maintenant que de la nature de l’entendement qui sert d’ornement à cette vie mortelle, et non de la foi et de la vérité par lesquelles on acquiert la vie immortelle. Certes une nature excellente, ayant pour auteur un Dieu également juste et puissant, qui gouverne lui-même tous ses ouvrages, ne serait jamais tombée dans ces misères, et de ces misères n’irait point (les seuls justes exceptés) dans tous les tourments éternels, si elle n’avait été corrompue originairement dans le premier homme, d’où sont sortis tous les autres, par quelque grand et énorme péché.\par
Si nous considérons notre corps même, bien qu’il meure comme celui des bêtes, qui l’ont souvent plus robuste que nous, quelle bonté et quelle providence de Dieu y éclatent de toutes parts ? Les organes des sens et les autres membres n’y sont-ils pas tellement dis-pesés, sa forme et sa stature si bien ordonnées, qu’il paraît clairement avoir été fait pour le service et le ministère d’une âme raisonnable ? L’homme n’a pas été créé courbé vers la terre, comme les animaux sans raison ; mais sa stature droite et élevée l’avertit de porter ses pensées et ses désirs vers le ciel. D’ailleurs cette merveilleuse vitesse donnée à la langue et à la main pour parler et pour écrire, et pour exécuter tant de choses, ne montre-t-elle pas combien est excellente l’âme qui a reçu un corps si bien fait pour serviteur ? que dis-je ? et quand bien même le corps n’aurait pas besoin d’agir, les proportions en sont observées avec tant d’art et de justesse, qu’il serait difficile de décider si, dans sa structure, Dieua eu plus d’égard à l’utilité qu’à la beauté. Au moins n’y voyons-nous rien d’utile qui ne soit beau tout à la fois : ce qui nous serait plus, évident encore, si nous connaissions les rapports et les proportions que toutes les parties ont entre elles, et dont nous pouvons découvrir quelque chose par ce que nous voyons au dehors. Quant à ce qui est caché, comme l’enlacement des veines, des nerfs, des muscles, des fibres, personne ne le saurait connaître. En effet, bien que les anatomistes aient disséqué des cadavres, et quelquefois même se soient cruellement exercés sur des hommes vivants pour fouiller dans les parties les plus secrètes du corps humain, et apprendre ainsi à les guérir, toutefois, comment aucun d’entre eux aurait-il trouvé cette proportion admirable dont nous parlons, et que les Grecs appellent {\itshape harmonie}, puisqu’ils ne l’ont pas seulement osé chercher ? Si nous pouvions la connaître dans les entrailles, qui n’ont aucune beauté apparente, nous y trouverions quelque chose de plus beau et qui satisferait plus notre esprit que tout ce qui flatte le plus agréablement nos yeux dans la figure extérieure du corps. Or, il y a certaines parties dans le corps qui ne sont que pour l’ornement et non pas pour l’usage, comme les mamelles de l’homme, et la barbe, qui n’est pas destinée à le défendre, puisque autrement les femmes, qui sont plus faibles, devraient en avoir. Si donc il n’y a aucun membre, de tous ceux qui paraissent, qui n’orne le corps autant qu’il le sert, et s’il y en a même qui ne sont que pour l’ornement et je pense que l’on comprend aisément que, dans la structure du corps, Dieu a eu plus d’égard à la beauté qu’à la nécessité. En effet, le temps de la nécessité passera, et il en viendra un autre, où nous ne jouirons que de la beauté de nos semblables, sans aucune concupiscence : digne sujet de louanges envers le Créateur, à qui il est dit dans le psaume : « Vous vous êtes revêtu de gloire et de splendeur ! »\par
Que dire de tant d’autres choses également belles et utiles qui remplissent l’univers et dont la bonté de Dieu a donné l’usage et le spectacle à l’homme, tout condamné qu’il soit à tant de peines et à tant de misères ? Parlerai-je de ce vif éclat de la lumière, de la magnificence du soleil, de la lune et des étoiles, de ces sombres beautés des forêts, des couleurs et des parfums des fleurs, de cette multitude d’oiseaux si différents de chant et de plumage, de cette diversité infinie d’animaux dont les plus petits sont les plus admirables ? car les ouvrages d’une fourmi et d’une abeille nous étonnent plus que le corps gigantesque d’une baleine. Parlerai-je de la mer, qui fournit toute seule un si grand spectacle à nos yeux, et des diverses couleurs dont elle se couvre comme d’autant d’habits différents, tantôt verte, tantôt bleue, tantôt pourprée ? Combien même y a-t-il de plaisir à la voir en courroux, pourvu que l’on se sente à l’abri de ses flots ? Que dire de cette multitude de mets différents qu’on a trouvés pour apaiser la faim, de ces divers assaisonnements que nous offre la libéralité de la nature contre le dégoût, sans recourir à l’art des cuisiniers, de cette infinité de remèdes qui servent à conserver ou à rétablir la santé, de cette agréable vicissitude des jours et des nuits, de ces doux zéphyrs qui tempèrent les chaleurs de l’été, et de mille sortes de vêtements que nous fournissent les arbres et les animaux ? Qui peut tout décrire ? et si je voulais même étendre ce peu que je me borne à indiquer, combien de temps ne me faudrait-il pas ? car il n’y a pas une de ces merveilles qui n’en comprenne plusieurs. Et ce ne sont là pourtant que les consolations de misérables condamnés et non les récompenses des bienheureux ; quelles seront donc ces récompenses ? qu’est-ce que Dieu donnera à ceux qu’il prédestine à la vie, s’il donne tant ici-bas à ceux qu’il a prédestinés à la mort ? de quels biens ne comblera-t-il point en la vie bienheureuse ceux pour qui il a voulu que son Fils unique souffrît tant de maux et la mort même en cette vie mortelle et misérable ? Aussi l’Apôtre, parlant de ceux qui sont prédestinés au royaume céleste « Que ne nous donnera-t-il point, dit-il, après « n’avoir pas épargné son propre Fils, et l’avoir « livré à la mort pour nous tous » ? Quand cette promesse sera accomplie, quels biens n’avons-nous pas à espérer dans ce royaume, ayant déjà reçu pour gage la mort d’un Dieu ? En quel état sera l’homme lorsqu’il n’aura plus de passions à combattre et qu’il sera dans une paix parfaite avec lui-même ? Ne connaîtra-t-il pas certainement toutes choses sanspeine et sans erreur, lorsqu’il puisera la sagesse de Dieu à sa source même ? Que sera son corps, lorsque, parfaitement soumis à l’esprit dont il tirera une vie abondante, il n’aura plus besoin d’aliments ? il ne sera plus animal, mais spirituel, gardant, il est vrai, la substance de la chair, mais exempt désormais de toute corruption charnelle.
\subsection[{Chapitre XXV}]{Chapitre XXV}

\begin{argument}\noindent De l’obstination de quelques incrédules qui ne veulent pas croire à la résurrection de la chair, admise aujourd’hui, selon les prédictions des livres saints, par le monde entier.
\end{argument}

\noindent Les plus fameux philosophes conviennent avec nous des biens dont l’âme heureuse jouira ; ils combattent seulement la résurrection de la chair et la nient autant qu’ils peuvent. Mais le grand nombre de ceux qui y croient a rendu imperceptible le nombre de ceux qui la nient ; et les savants et les ignorants, les sages du monde et les simples se sont rangés du côté de Jésus-Christ, qui a fait voir comme réel dans sa résurrection ce qu’une poignée d’incrédules trouve absurde. Le monde a cru ce que Dieu a prédit, et cette foi même du monde a été aussi prédite, sans qu’on en puisse attribuer la prédiction aux sortilèges de Pierre, puisqu’elle l’a précédé de tant d’années’. Celui qui a annoncé ces choses est le même Dieu devant qui tremblent toutes les autres divinités ; je l’ai déjà dit et je ne suis pas fâché de le répéter ; car ici Porphyre est d’accord avec moi, lui qui cherche dans les oracles mêmes de ses dieux des témoignages à l’honneur de notre Dieu, et va jusqu’à lui donner le nom de Père et de Roi. Or, gardons-nous d’entendre ce que Dieu a prédit comme l’entendent ceux qui ne partagent pas avec le monde cette foi du monde qu’il a prédite. Et pourquoi en effet ne pas l’entendre plutôt comme l’entend le monde dont la foi même a été prédite ? En effet, s’ils ne veulent l’entendre d’une autre manière que pour ne pas faire injure à ce Dieu à qui ils rendent un témoignage si éclatant, et pour ne pas dire que sa prédiction est vaine, n’est-ce pas lui faire une plus grande injure encore de dire qu’il la faut entendre autrement que le monde ne la croit, puisque lui-même a annoncé, loué, accompli la foi du monde ? Pourquoi ne peut-il pas faire que la chair ressuscite et vive éternellement ? est-ce là un mal et une chose indigne de lui ? — Mais nous avons déjà amplement parlé de sa toute-puissance qui a fait tant de choses incroyables. Voulez-vous savoir ce que ne peut le Tout-Puissant ? le voici : il ne peut mentir. Croyez donc ce qu’il peut en ne croyant pas ce qu’il ne peut. Ne croyant pas qu’il puisse mentir, croyez donc qu’il fera ce qu’il a promis, et croyez-le comme l’a cru le monde dont il a prédit la foi. Maintenant, comment nos philosophes montrent-ils que ce soit un mal ? Il n’y aura là aucune corruption, par conséquent, aucun mal du corps. D’ailleurs, nous avons parlé de l’ordre des éléments et des autres objections que l’on a imaginées à ce sujet, et nous avons fait voir, au treizième livre, combien les mouvements d’un corps incorruptible seront souples et aisés, à n’en juger que par ce que nous voyons maintenant, lorsque notre corps se porte bien, quoique sa santé actuelle la plus parfaite ne soit pas comparable à l’immortalité qu’il possédera un jour. Que ceux qui n’ont pas lu ce que j’ai dit ci-dessus, ou qui ne veulent pas s’en souvenir, prennent la peine de le relire.
\subsection[{Chapitre XXVI}]{Chapitre XXVI}

\begin{argument}\noindent Opinion de Porphyre sur le souverain bien.
\end{argument}

\noindent Mais, disent-ils, Porphyre assure qu’une âme, pour être heureuse, doit fuir toute sorte de corps. C’est donc en vain que nous prétendons que le corps sera incorruptible, si l’âme ne peut être heureuse qu’à condition de fuir le corps. J’ai déjà suffisamment répondu à cette objection, au livre indiqué. J’ajouterai ceci seulement : si les philosophes ont raison, que Platon, leur maître, corrige donc ses livres, et dise que les dieux fuiront leurs corps pour être bienheureux, c’est-à-dire qu’ils mourront, lui qui dit qu’ils sont enfermés dans des corps célestes et que néanmoins le dieu qui les a créés leur a promis qu’ils y demeureraient toujours, afin qu’ils pussent être assurés de leur félicité, quoique cela ne dût pas être naturellement. Il renverse en cela du même coup cet autre raisonnementqu’on nous oppose à tout propos : qu’il ne faut pas croire à la résurrection de la chair, parce qu’elle est impossible. En effet, selon ce même philosophe, lorsque le Dieu incréé a promis l’immortalité aux dieux créés, il leur a dit qu’il faisait une chose impossible. Voici le discours même que Platon prête à Dieu : « Comme vous avez commencé d’être, vous ne sauriez être immortels ni parfaitement indissolubles ; mais vous ne serez jamais dissous, et vous ne connaîtrez aucune sorte de mort, parce que la mort ne peut rien contre ma volonté, laquelle est un lien plus fort et plus puissant que ceux dont vous fûtes unis au moment de votre naissance. » Après cela, on ne peut plus douter, que, suivant Platon, le Dieu créateur des autres dieux ne leur ait promis ce qui est impossible. Celui qui dit : Vous ne pouvez à la vérité être immortels, mais vous le serez, parce que je le veux, — que dit-il autre chose, sinon : Je ferai que vous serez ce que vous ne pouvez être ? Celui-là donc ressuscitera la chair et la rendra immortelle, incorruptible et spirituelle, qui, selon Platon, a promis de faire ce qui est impossible. Pourquoi donc s’imaginer encore que ce que Dieu a promis de faire, ce que le monde entier croit sur sa parole, est impossible, surtout lorsqu’il a aussi promis que le monde le croirait ? Nous ne disons pas qu’un autre dieu le doive faire que celui qui, selon Platon, fait des choses impossibles. Il ne faut donc pas que les âmes fuient toutes sortes de corps pour être heureuses, mais il faut qu’elles en reçoivent un incorruptible. Et en quel corps incorruptible est-il plus raisonnable qu’elles se réjouissent, que dans le corps corruptible où elles ont gémi ? Ainsi elles n’auront pas ce désir que Virgile leur attribue, d’après Platon, de vouloir de nouveau retourner dans les corps a, puisqu’elles auront éternellement ces corps, et elles les auront si bien qu’elles ne s’en sépareront pas, même pendant le plus petit espace de temps.
\subsection[{Chapitre XXVII}]{Chapitre XXVII}

\begin{argument}\noindent Des opinions contraires de Platon et de Porphyre, lesquelles les eussent conduits à la vérité, si chacun d’eux avait voulu céder quelque chose à l’autre.
\end{argument}

\noindent Platon et Porphyre ont aperçu chacun certaines vérités qui peut-être en auraient fait des chrétiens, s’ils avaient pu se les communiquer l’un à l’autre. Platon avance que les âmes ne peuvent être éternellement sans corps, de sorte que celles même des sages retourneront à la vie corporelle, après un long espace de temps. Porphyre déclare que lorsque l’âme parfaitement purifiée sera retournée au Père, elle ne reviendra jamais aux misères de cette vie. Si Platon avait persuadé à Porphyre cette vérité, que sa raison avait conçue, que les âmes mêmes des hommes justes et sages retourneront en des corps humains ; et si Porphyre eût fait part à Platon de cette autre vérité, qu’il avait établie, que les âmes des saints ne reviendront jamais aux misères d’un corps corruptible, je pense qu’ils auraient bien vu qu’il s’ensuit de là que les âmes doivent retourner dans des corps, mais dans des corps immortels et incorruptibles. Que Porphyre dise donc avec Platon : elles retourneront dans des corps ; que Platon dise avec Porphyre : elles ne retourneront pas à leur première misère. Ils reconnaîtront alors tous deux qu’elles retourneront en des corps où elles ne souffriront plus rien. Ce n’est autre chose que ce que Dieu a promis, savoir l’éternelle félicité des âmes dans des corps immortels. Et maintenant ; une fois accordé que les âmes des saints retourneront en des corps immortels, je pense qu’ils n’auraient pas beaucoup de peine à leur permettre de retourner en ceux où ils ont souffert les maux de la terre, et où ils ont religieusement servi Dieu pour être délivrés de tout mal.
\subsection[{Chapitre XXVIII}]{Chapitre XXVIII}

\begin{argument}\noindent Comment Platon, Labéon et même Varron auraient pu voir la vérité de la résurrection de la chair, s’ils avaient réuni leurs opinions en une seule.
\end{argument}

\noindent Quelques-uns des nôtres, qui aiment Platonà cause de la beauté de son style et de quelques vérités répandues dans ses écrits, disent qu’il professe à peu près le même sentiment que nous sur la résurrection. Mais Cicéron, qui en touche un mot dans sa {\itshape République}, laisse voir que le célèbre philosophe a plutôt voulu se jouer que dire ce qu’il croyait véritable. Platon, en effet, introduit dans un de ses dialogues un homme ressuscité qui fait des récits conformes aux sentiments des Platoniciens. Labéon rapporte aussi que deux hommes morts le même jour se rencontrèrent dans un carrefour, et qu’ensuite, ayant reçu l’ordre de retourner dans leur corps, ils se jurèrent une parfaite amitié, qui dura jusqu’à ce qu’ils moururent de nouveau. Mais ces sortes de résurrections sont comme celles des personnes que nous savons avoir été de nos jours rendues à la vie, mais non pas pour ne plus mourir, Varron rapporte quelque chose de plus merveilleux dans son traité : {\itshape De l’origine du peuple romain}. Voici ses propres paroles : « Quelques astrologues ont écrit que les hommes sont destinés à une renaissance qu’ils appellent palingénésie, et ils en fixent l’époque à quatre cent quarante ans après la mort. À ce moment, l’âme reprendra le même corps qu’elle avait auparavant. » Ce que Varron et ces astrologues, je ne sais lesquels, car il ne les nomme point, disent ici, n’est pas absolument vrai, puisque, lorsque les âmes seront revenues à leurs corps, elles ne les quitteront plus ; mais au moins cela renverse-t-il beaucoup d’arguments que nos adversaires tirent d’une prétendue impossibilité. En effet, les païens qui ont été de ce sentiment n’ont donc pas estimé que des corps évaporés dans l’air, ou écoulés en eau, ou réduits en cendre et en poussière, ou passés dans la substance soit des bêtes, soit des hommes, ne puissent être rétablis en leur premier état. Si donc Platon et Porphyre, ou plutôt ceux qui les aiment et qui sont actuellement en vie, tiennent que les âmes purifiées retourneront dans des corps, comme le dit Platon, et que néanmoins elles ne reviendront point à leurs misères, comme le veut Porphyre, c’est-à-dire s’ils tiennent ce qu’enseigne notre religion, qu’elles rentreront dans des corps où elles demeureront éternellement sans souffrir aucun mal, il ne leur reste plus qu’à dire avec Varron qu’elles retourneront aux même corps qu’elles animaient primitivement, et toute la question de la résurrection sera résolue.
\subsection[{Chapitre XXIX}]{Chapitre XXIX}

\begin{argument}\noindent De la nature de la vision par laquelle les saints connaîtront Dieu dans la vie future.
\end{argument}

\noindent Voyons maintenant, autant qu’il plaira à Dieu de nous éclairer, ce que les saints feront dans leurs corps immortels et spirituels, alors que leur chair ne vivra plus charnellement, mais spirituellement. Pour avouer avec franchise ce qui en est, je ne sais quelle sera cette action, ou plutôt ce calme et ce repos dont ils jouiront. Les sens du corps ne m’en ont jamais donné aucune idée, et quant à l’intelligence, qu’est-ce que toute la nôtre, en comparaison d’un si grand objet ? C’est au séjour céleste que règne « cette paix de Dieu, qui », comme dit l’Apôtre, « surpasse tout entendement » : quel entendement, sinon le nôtre, ou peut-être même celui des anges ? mais elle ne surpasse pas celui de Dieu. Si donc les saints doivent vivre dans la paix de Dieu, assurément la paix où ils doivent vivre surpasse tout entendement. Qu’elle surpasse le nôtre, il n’en faut point douter ; mais si elle surpasse même celui des anges, comme il semble que l’Apôtre le donne à penser, qui dit tout n’exceptant rien, il faut appliquer ses paroles à la paix dont jouit Dieu, et dire que ni nous, ni les anges même ne la peuvent connaître comme Dieu la connaît. Ainsi elle surpasse tout autre entendement que le sien. Mais de même que nous participerons un jour, selon notre faible capacité, à cette paix, soit en nous-mêmes, soit en notre prochain, soit en Dieu, en tant qu’il est notre souverain bien, ainsi les anges la connaissent aujourd’hui autant qu’ils en sont capables, et les hommes aussi, mais beaucoup moins qu’eux, tout avancés qu’ils soient dans les voies spirituelles. Quel homme en effet peut surpasser celui qui a dit : « Nous connaissons en partie, et en partie nous devinons, jusqu’au jour où le parfait s’accomplira » ; et ailleurs : « Nous ne voyons maintenant que comme dans un miroir et en énigme ; mais alors nous verrons face à face. » C’est ainsi que voientdéjà les saints anges, qui sont aussi appelés nos anges, parce que, depuis que nous avons été délivrés de la puissance des ténèbres et transportés au royaume de Jésus-Christ, après avoir reçu le Saint-Esprit pour gage de notre réconciliation, nous commençons à appartenir à ces anges avec qui nous posséderons en commun cette sainte et chère Cité de Dieu, sur laquelle nous avons déjà écrit tant de livres. Les anges de Dieu sont donc nos anges, comme le Christ de Dieu est notre Christ. Ils sont les anges de Dieu, parce qu’ils ne l’ont point abandonné ; et ils sont nos anges, parce que nous commençons à être leurs concitoyens. C’est ce qui a fait dire à Notre-Seigneur : « Prenez bien garde de ne mépriser aucun de ces petits ; car je vous assure que leurs anges voient sans cesse la face de mon Père dans le ciel. » Nous la verrons, nous aussi, comme ils la voient, mais nous ne la voyons pas encore de cette façon, d’où vient cette parole de l’Apôtre, que j’ai rapportée : « Nous ne voyons maintenant que dans un miroir et en énigme ; mais alors nous verrons face à face. » Cette vision nous est réservée pour récompense de notre foi, et saint Jean parle ainsi : « Lorsqu’il paraîtra, nous serons semblables à lui, parce que nous le verrons tel qu’il est. » Il est clair que dans ces passages, par la face de Dieu, on doit entendre sa manifestation, et non cette partie de notre corps que nous appelons ainsi.\par
C’est pourquoi quand on me demande ce que feront les saints dans leur corps spirituel, je ne dis pas ce que je vois, mais ce que je crois, suivant cette parole du psaume : « J’ai cru, et c’est ce qui m’a fait parler. » Je dis donc que c’est dans ce corps qu’ils verront Dieu ; mais de savoir s’ils le verront par ce corps, comme maintenant nous voyons le soleil, la lune, les étoiles elles autres objets sensibles, ce n’est pas une petite question. Il est dur de dire que les saints ne pourront alors ouvrir et fermer les yeux quand il leur plaira, mais il est encore plus dur de dire que quiconque fermera les yeux ne verra pas Dieu. Si Élisée, quoique absent de corps, vit son serviteur Giezi qui prenait, se croyant inaperçu, des présents de Naaman le Syrien que le Prophète avait guéri de la lèpre, àcombien plus forte raison les saints verront-ils toutes choses dans ce corps spirituel, non seulement ayant les yeux fermés, mais même étant corporellement absents ! Ce sera alors le temps de cette perfection dont parle l’Apôtre, quand il dit : « Nous connaissons en partie et en partie nous devinons ; mais quand le parfait sera arrivé, le partiel sera aboli. » Pour montrer ensuite par une sorte de comparaison combien cette vie, quelque progrès qu’on y fasse dans la vertu, est différente de l’autre : « Quand j’étais enfant, dit-il, je jugeais en enfant, je raisonnais en enfant ; mais lorsque je suis devenu homme, je me suis défait de tout ce qui tenait de l’enfant. Nous ne voyons maintenant que comme dans un miroir et en énigme, mais alors nous verrons face à face. Je ne connais maintenant qu’en partie, mais je connaîtrai alors comme je suis connu. » Si donc en cette vie, où la connaissance des plus grands prophètes ne mérite pas plus d’être comparée à celle que nous aurons dans la vie future, qu’un enfant n’est comparable à un homme fait, Élisée tout absent qu’il était, vit son serviteur qui prenait des présents, dirons-nous que, lorsque le parfait sera arrivé et que le corps corruptible n’appesantira plus l’âme, les saints auront besoin pour voir des yeux dont le prophète Élisée n’eut pas besoin ? Voici comment ce Prophète parle à Giezi, selon la version des Septante : « Mon esprit n’allait-il pas avec toi, et ne sais-je pas que Naaman est sorti de son char au-devant de toi et que tu as accepté de l’argent ? » Ou comme le prêtre Jérôme traduit sur l’hébreu : « Mon esprit n’était-il pas présent, quand Naaman est descendu de son char pour aller au-devant de toi ? » Le Prophète dit qu’il vit cela avec son esprit, aidé sans doute surnaturellement d’en haut ; à combien plus forte raison, les saints recevront-ils cette grâce du ciel, lorsque Dieu sera tout en tous ! Toutefois les yeux du corps auront aussi leur fonction et seront à leur place, et l’esprit s’en servira par le ministère du corps spirituel. Bien que le prophète Élisée n’ait pas eu besoin de ses yeux pour voir son serviteur absent, ce n’est pas à dire qu’il ne s’en servit point pour voir les objets présents, qu’il pouvait néanmoins voir aussi avec son esprit, bien qu’il fermât ses yeux, comme il en vit qui étaient loin de lui. Gardons-nous donc dedire que les saints ne verront pas Dieu en l’autre vie les yeux fermés, puisqu’ils le verront toujours avec l’esprit.\par
La question est de savoir s’ils le verront aussi avec les yeux du corps, quand ils les auront ouverts. Si leurs yeux, tout spirituels qu’ils seront dans leur corps spirituel, n’ont pas plus de vertu que n’en ont les nôtres maintenant, il est certain qu’ils ne leur serviront point à voir Dieu. Ils auront donc une vertu infiniment plus grande, si, par leur moyen, on voit cette nature immatérielle qui n’est point contenue dans un lieu limité, mais qui est tout entière partout. Quoique nous disions en effet que Dieu est au ciel et sur la terre, selon ce qu’il dit lui-même par le Prophète : « Je remplis le ciel et le terre » ; il ne s’ensuit pas qu’il ait une partie de lui-même dans le ciel et une autre sur la terre mais il est tout entier dans le ciel et tout entier sur la terre, non en divers temps, mais à la fois, ce qui est impossible à toute nature corporelle. Les yeux des saints auront donc alors une infiniment plus grande vertu, par où je n’entends pas dire qu’ils auront la vue plus perçante que celle qu’on attribue aux aigles ou aux serpents ; car ces animaux, quelque clairvoyants qu’ils soient, ne sauraient voir que des corps, au lieu que les yeux des saints verront même des choses incorporelles. Telle était peut-être cette vertu qui fut donnée au saint homme Job, quand il disait à Dieu : « Auparavant je vous entendais, mais à cette heure mon œil vous voit ; c’est pourquoi je me suis méprisé moi-même ; je me suis comme fondu devant vous, et j’ai cru que je n’étais que cendre et que poussière. » Au reste, ceci se peut très bien entendre des yeux de l’esprit dont saint Paul dit : « Afin qu’il éclaire les yeux de votre cœur. » Or, que Dieu se voie de ces yeux-là, c’est ce dont ne doute aucun chrétien qui accepte avec foi cette parole de notre Dieu et maître : « Bienheureux ceux qui ont le cœur pur, parce qu’ils verront Dieu ! » mais il reste toujours à savoir si on le verra aussi des yeux du corps, et c’est ce que nous examinons maintenant.\par
Nous lisons dans l’Évangile : « Et toute chair verra le salut de Dieu » ; or, il n’y a aucun inconvénient à entendre ce passagecomme s’il y avait : Et tout homme verra le Christ de Dieu qui a été vu dans un corps, et qui sera vu sous la même forme, quand il jugera les vivants et les morts. — En effet, que le Christ soit {\itshape le salut de Dieu}, cela se justifie par plusieurs témoignages de l’Écriture, mais singulièrement par ces paroles du vénérable vieillard Siméon, qui, ayant pris Jésus enfant entre ses bras, s’écria : « C’est maintenant, Seigneur, que vous pouvez laisser aller en paix votre serviteur, selon votre parole, puisque mes yeux ont vu votre salut. » Quant à ce passage de Job, tel qu’il se trouve dans les exemplaires hébreux : « Je verrai Dieu dans ma chair », il faut croire sans doute que Job prophétisait ainsi la résurrection de la chair ; mais il n’a pas dit pourtant : Je verrai Dieu {\itshape par} ma chair. Et quand il l’aurait dit, on pourrait l’entendre de Jésus-Christ, qui est Dieu aussi, et qu’on verra dans la chair et par le moyen de la chair. Mais maintenant, en l’entendant de Dieu même, on peut fort bien l’expliquer ainsi : « Je verrai Dieu dans ma chair » c’est-à-dire, je serai dans ma chair, lorsque je verrai Dieu. De même ce que dit l’Apôtre : « Nous verrons face à face », ne nous oblige point à croire que nous verrons Dieu par cette partie du corps où sont les yeux corporels, lui que nous verrons sans interruption par les yeux de l’esprit. En effet, si l’homme intérieur n’avait aussi une face, l’Apôtre ne dirait pas : « Mais nous, contemplant à face dévoilée la gloire du Seigneur, nous sommes transformés en la même image, allant de clarté en clarté, comme par l’esprit du Seigneur. » Nous n’entendons pas autrement ces paroles du psaume : « Approchez-vous de lui, et vous serez éclairés, et vos faces ne rougiront point. » C’est par là foi qu’on approche de Dieu, et il est certain que la foi appartient au cœur et non au corps. Mais comme nous ignorons jusqu’à quel degré de perfection doit être élevé le corps spirituel des bienheureux, car nous parlons d’une chose dont nous n’avons point d’expérience et sur laquelle l’Écriture ne se déclare pas formellement, il faut de toute nécessité qu’il nous arrive ce qu’on lit dans la Sagesse : « Les pensées des hommes sont chancelantes, et leur prévoyance est incertaine. »\par
Si cette opinion des philosophes que les objets des sens et de l’esprit sont tellement partagés que l’on ne saurait voir les choses intelligibles par le corps, ni les corporelles par l’esprit, si cette opinion était vraie, assurément nous ne pourrions voir Dieu par les yeux d’un corps, même spirituel. Mais la saine raison et l’autorité des Prophètes se jouent de ce raisonnement. Qui, en effet, serait assez peu sensé pour dire que Dieu ne connaît pas les choses corporelles ? et cependant il n’a point de corps pour les voir. Il y a plus : ce que nous avons rapporté d’Élisée ne montre-t-il pas clairement qu’on peut voir les choses corporelles par l’esprit, sans avoir besoin du corps ? Quand Giezi prit les présents de Naaman, le fait se passa corporellement ; et cependant le Prophète ne le vit pas avec les yeux du corps, mais par l’esprit. De plus, puisqu’il est constant que les corps se voient par l’esprit, pourquoi ne se peut-il pas faire que la vertu d’un corps spirituel soit telle qu’on voie même un esprit par ce corps ? car Dieu est esprit. D’ailleurs, si chacun connaît par un sentiment intérieur, et non par les yeux du corps, la vie qui l’anime, il n’en est pas de même pour la vie de nos semblables :nous la voyons par le corps, quoique ce soit une chose invisible. Comment discernons-nous les corps vivants de ceux qui ne le sont pas, sinon parce que nous voyons en même temps et les corps et la vie que nous ne saurions voir que par le corps ? mais la vie sans le corps se dérobe aux yeux corporels.\par
C’est pourquoi il est possible et fort croyable que dans l’autre vie nous verrons de telle façon les corps du ciel nouveau et de la terre nouvelle que nous y découvrirons Dieu présent partout, non comme aujourd’hui, où ce qu’on peut voir de lui se voit, en quelque sorte, par les choses créées, comme dans un miroir et en énigme, et d’une façon partielles, et plus par la foi qu’autrement, mais comme nous voyons maintenant la vie des hommes qui se présentent à nos yeux. Nous ne croyons pas qu’ils vivent ; nous le voyons. Alors donc, ou bien les yeux du corps seront tellement perfectionnés qu’on verra Dieu avec leur aide, comme on le voit par l’esprit, supposition difficile ou même impossible à justifier par aucun témoignage de l’Écriture, on bien, ce qui est plus aisé à comprendre, Dieu noussera si connu et si sensible que nous le verrons par l’esprit au dedans de nous, dans les autres, dans lui-même, dans le ciel nouveau et dans la terre nouvelle, en un mot, dans tout être alors subsistant. Nous le verrons même par le corps dans tout corps, de quelque côté que nous jetions les yeux. Et nos pensées aussi deviendront visibles ; car alors s’accomplira ce que dit l’Apôtre : « Ne jugez point avant le temps, jusqu’à ce que le Seigneur vienne, et qu’il porte la lumière dans les plus épaisses ténèbres, et qu’il découvre les pensées des cœurs ; et chacun alors recevra de Dieu la louange qui lui est due. »
\subsection[{Chapitre XXX}]{Chapitre XXX}

\begin{argument}\noindent De l’éternelle félicité de la Cité de Dieu et du sabbat éternel.
\end{argument}

\noindent Qu’elle sera heureuse cette vie où tout mal aura disparu, où aucun bien ne sera caché, où l’on n’aura qu’à chanter les louanges de Dieu, qui sera tout en tous ! car que faire autre chose en un séjour où ne se peuvent rencontrer ni la paresse, ni l’indigence ? Le Psalmiste ne veut pas dire autre chose, quand il s’écrie : « Heureux ceux qui habitent votre maison, Seigneur ! ils vous loueront éternellement. » Toutes les parties de notre corps, maintenant destinées à certains usages nécessaires à la vie, n’auront point d’autre emploi que de concourir aux louanges de Dieu. Toute cette harmonie du corps humain dont j’ai parlé et qui nous est maintenant cachée, se découvrant alors à nos yeux avec une infinité d’autres choses admirables, nous transportera d’une sainte ardeur pour louer hautement le grand Ouvrier. Je n’oserais déterminer quels seront les mouvements de ces corps spirituels ; mais, à coup sûr, mouvement, altitude, expression, tout sera dans la convenance, en un lieu où rien que de convenable ne se peut rencontrer. Un autre point assuré, c’est que le corps sera incontinent où l’esprit voudra, et que l’esprit ne voudra rien qui soit contraire à la dignité du corps, ni à la sienne. Là régnera la véritable gloire, loin de l’erreur et de la flatterie. Là le véritable honneur, qui ne sera pas plus refusé à qui le mérite que déféré à qui ne le mérite pas, nul indigne n’y pouvant prétendre dans un séjour où le mérite seul donne accès. Là enfin la véritable paix où l’on ne souffrira rien de contraire, ni de soi-même, ni des autres. Celui-là même qui est l’auteur de la vertu en sera la récompense, parce qu’il n’y a rien de meilleur que lui et qu’il a promis de se donner à tous. Que signifie ce qu’il a dit par le prophète : « Je serai leur Dieu, et ils seront mon peuple », sinon : Je serai l’objet qui remplira tous leurs souhaits ; je serai tout ce que les hommes peuvent honnêtement désirer, vie, santé, nourriture, richesses, gloire, honneur, paix, en un mot tous les biens, afin que, comme dit l’Apôtre : « Dieu soit tout en tous. » Celui-là sera la fin de nos désirs, qu’on verra sans fin, qu’on aimera sans dégoût, qu’on louera sans lassitude : occupation qui sera commune à tous, ainsi que la vie éternelle.\par
Au reste, il n’est pas possible de savoir quel sera le degré de gloire proportionné aux mérites de chacun. Il n’y a point de doute pourtant qu’il n’y ait en cela beaucoup de différence. Et c’est encore un des grands biens de cette Cité, que l’on n’y portera point envie à ceux que l’on verra au-dessus de soi, comme maintenant les anges ne sont point envieux de la gloire des archanges. L’on souhaitera aussi peu de posséder ce qu’on n’a pas reçu, quoiqu’on soit parfaitement uni à celui qui a reçu, que le doigt souhaite d’être l’œil, bien que l’œil et le doigt entrent dans la structure du même corps. Chacun donc y possédera tellement son don, l’un plus grand, l’autre plus petit, qu’il aura en outre le don de n’en point désirer de plus grand que le sien.\par
Et il ne faut pas s’imaginer que les bienheureux n’auront point de libre arbitre, sous prétexte qu’ils ne pourront plus prendre plaisir au péché ; ils seront même d’autant plus libres qu’ils seront délivrés du plaisir de pécher pour prendre invariablement plaisir à ne pécher point. Le premier libre arbitre qui fut donné à l’homme, quand Dieu le créa droit, consistait à pouvoir ne pas céder au péché et aussi à pouvoir pécher. Mais ce libre arbitre supérieur, qu’il doit recevoir à la fin, sera d’autant plus puissant qu’il ne pourra plus pécher, privilège qu’il ne tiendra pas de lui-même, mais de la bonté de Dieu. Autre chose est d’être Dieu, autre chose est de participer de Dieu. Dieu, par nature, ne peut pécher ; mais celui qui participe de Dieu reçoitseulement de lui la grâce de ne plus pouvoir pécher. Or, cet ordre devait être gardé dans le bienfait de Dieu, de donner premièrement à l’homme un libre arbitre par lequel il pût ne point pécher, et ensuite de lui en donner un par lequel il ne puisse plus pécher : le premier pour acquérir le mérite, le second pour recevoir la récompense. Or, l’homme ayant péché lorsqu’il l’a pu, c’est par une grâce plus abondante qu’il est délivré, afin d’arriver à cette liberté où il ne pourra plus pécher. De même que la première immortalité qu’Adam perdit en péchant consistait à pouvoir ne pas mourir, et que la dernière consistera à ne pouvoir plus mourir, ainsi la première liberté de la volonté consistait à pouvoir ne pas pécher, la dernière consistera à ne pouvoir plus pécher. De la sorte, l’homme ne pourra pas plus perdre sa vertu que sa félicité. Et il n’en sera pourtant pas moins libre : car dira-t-on que Dieu n’a point de libre arbitre, sous prétexte qu’il ne saurait pécher ? Tous les membres de cette divine Cité auront donc une volonté parfaitement libre, exempte de tout mal, comblée de tout bien, jouissant des délices d’une joie immortelle, sans plus se souvenir de ses fautes ni de ses misères, et sans oublier néanmoins sa délivrance, pour n’être pas ingrate envers son libérateur.\par
L’âme se souviendra donc de ses maux passés, mais intellectuellement et sans les ressentir, comme un habile médecin qui connaît plusieurs maladies par son art, sans les avoir jamais éprouvées. De même qu’on peut connaître les maux de deux manières, par science ou par expérience, car un homme de bien connaît les vices autrement qu’un libertin, on peut aussi les oublier de deux matières. Celui qui les a appris par science ne les oublie pas de la même manière que celui qui les a soufferts ; car celui-là les oublie en abdiquant sa connaissance, et celui-ci en dépouillant sa misère. C’est de cette dernière façon que les saints ne se souviendront plus de leurs maux passés. Ils seront exempts de tous maux, sans qu’il leur en reste le moindre sentiment ; et toutefois, par le moyen de la science qu’ils posséderont au plus haut degré, ils ne connaîtront pas seulement leur misère passée, mais aussi la misère éternelle des damnés. En effet, s’ils ne se souvenaient lias d’avoir été misérables, comment, selon le Psalmiste, chanteraient-ils éternellement les miséricordes de Dieu ? or, nous savons que cette Cité n’aura pas de plus grande joie que de chanter ce cantique à la gloire du Sauveur qui nous a rachetés par son sang. Là cette parole sera accomplie : « Tenez-vous en repos, et reconnaissez que je suis Dieu. » Là sera vraiment le grand sabbat qui n’aura point de soir, celui qui est figuré dans la Genèse, quand il est dit : « Dieu se reposa de toutes ses œuvres le septième jour, et il le bénit et le sanctifia, parce qu’il s’y reposa de tous les ouvrages qu’il avait entrepris. » En effet, nous serons nous-mêmes le septième jour, quand nous serons remplis et comblés de la bénédiction et de la sanctification, de Dieu. Là nous nous reposerons, et nous reconnaîtrons que c’est lui qui est Dieu, qualité souveraine que nous avons voulu usurper, quand nous avons abandonné Dieu pour écouter cette parole du séducteur : « Vous serez comme des dieux » ; d’autant plus aveugles que nous aurions eu cette qualité en quelque sorte, par anticipation et par grâce, si nous lui étions demeurés fidèles au lieu de le quitter. Qu’avons-nous fait en le quittant, que mourir misérablement ? Mais alors, rétablis par sa bonté et remplis d’une grâce plus abondante, nous nous reposerons éternellement et nous verrons que c’est lui qui est Dieu ; car nous serons pleins de lui et il sera tout en tous. Nos bonnes œuvres mêmes, quand nous les croyons plus à lui qu’à nous, nous sont imputées pour obtenir ce sabbat ; au lieu que, si nous venons à nous les attribuer, elles deviennent des œuvres serviles, puisqu’il est dit du sabbat : « Vous n’y ferez aucune œuvre servile » ; d’où cette parole qui est dans le prophète Ézéchiel : « Je leur ai donné mes sabbats comme un signe d’alliance entre eux et moi, afin qu’ils apprissent que je suis le Seigneur qui les sanctifie. » Nous saurons cela parfaitement, quand nous serons parfaitement en repos et que nous verrons parfaitement que c’est lui qui est Dieu.\par
Ce sabbat paraîtra encore plus clairement, si l’on compte les âges, selon l’Écriture, comme autant de jours, puisqu’il se trouve justement le septième. Le premier âge, comme le premier jour, se compte depuis Adamjusqu’au déluge ; le second, depuis le déluge jusqu’à Abraham ; et, bien que celui-ci ne comprenne pas une aussi longue durée que le premier, il comprend autant de générations, depuis Abraham jusqu’à Jésus-Christ. L’évangéliste Matthieu compte trois âges qui comprennent chacun quatre générations : un d’Abraham à David, l’autre de David à la captivité de Babylone, le troisième de cette captivité à la naissance temporelle de Jésus-Christ. Voilà donc déjà cinq âges. Le sixième s’écoule maintenant et ne doit être mesuré par aucun nombre certain de générations, à cause de cette parole du Sauveur : « Ce n’est pas à vous de connaître les temps dont mon Père s’est réservé la disposition. » Après celui-ci, Dieu se reposera comme au septième jour, lorsqu’il nous fera reposer en lui, nous qui serons ce septième jour. Mais il serait troplong de traiter ici de ces sept âges. Qu’il suffise de savoir que le septième sera notre sabbat, qui n’aura point de soir, mais qui finira par le jour dominical, huitième jour et jour éternel, consacré par la résurrection de Jésus-Christ et figurant le repos éternel, non seulement de l’esprit, mais du corps. C’est là que nous nous reposerons et que nous verrons, que nous verrons et que nous aimerons, que nous aimerons et que nous louerons. Voilà ce qui sera à la fin sans fin. Et quelle autre fin nous proposons-nous que d’arriver au royaume qui n’a point de fin ?\par
Il me semble, en terminant ce grand ouvrage, qu’avec l’aide de Dieu je me suis acquitté de ma dette. Que ceux qui trouvent que j’en ai dit trop ou trop peu, me le pardonnent ; et que ceux qui pensent que j’en ai dit assez en rendent grâces, non à moi, mais à Dieu avec moi. Ainsi soit-il !
 


% at least one empty page at end (for booklet couv)
\ifbooklet
  \pagestyle{empty}
  \clearpage
  % 2 empty pages maybe needed for 4e cover
  \ifnum\modulo{\value{page}}{4}=0 \hbox{}\newpage\hbox{}\newpage\fi
  \ifnum\modulo{\value{page}}{4}=1 \hbox{}\newpage\hbox{}\newpage\fi


  \hbox{}\newpage
  \ifodd\value{page}\hbox{}\newpage\fi
  {\centering\color{rubric}\bfseries\noindent\large
    Hurlus ? Qu’est-ce.\par
    \bigskip
  }
  \noindent Des bouquinistes électroniques, pour du texte libre à participation libre,
  téléchargeable gratuitement sur \href{https://hurlus.fr}{\dotuline{hurlus.fr}}.\par
  \bigskip
  \noindent Cette brochure a été produite par des éditeurs bénévoles.
  Elle n’est pas faîte pour être possédée, mais pour être lue, et puis donnée.
  Que circule le texte !
  En page de garde, on peut ajouter une date, un lieu, un nom ; pour suivre le voyage des idées.
  \par

  Ce texte a été choisi parce qu’une personne l’a aimé,
  ou haï, elle a en tous cas pensé qu’il partipait à la formation de notre présent ;
  sans le souci de plaire, vendre, ou militer pour une cause.
  \par

  L’édition électronique est soigneuse, tant sur la technique
  que sur l’établissement du texte ; mais sans aucune prétention scolaire, au contraire.
  Le but est de s’adresser à tous, sans distinction de science ou de diplôme.
  Au plus direct ! (possible)
  \par

  Cet exemplaire en papier a été tiré sur une imprimante personnelle
   ou une photocopieuse. Tout le monde peut le faire.
  Il suffit de
  télécharger un fichier sur \href{https://hurlus.fr}{\dotuline{hurlus.fr}},
  d’imprimer, et agrafer ; puis de lire et donner.\par

  \bigskip

  \noindent PS : Les hurlus furent aussi des rebelles protestants qui cassaient les statues dans les églises catholiques. En 1566 démarra la révolte des gueux dans le pays de Lille. L’insurrection enflamma la région jusqu’à Anvers où les gueux de mer bloquèrent les bateaux espagnols.
  Ce fut une rare guerre de libération dont naquit un pays toujours libre : les Pays-Bas.
  En plat pays francophone, par contre, restèrent des bandes de huguenots, les hurlus, progressivement réprimés par la très catholique Espagne.
  Cette mémoire d’une défaite est éteinte, rallumons-la. Sortons les livres du culte universitaire, cherchons les idoles de l’époque, pour les briser.
\fi

\ifdev % autotext in dev mode
\fontname\font — \textsc{Les règles du jeu}\par
(\hyperref[utopie]{\underline{Lien}})\par
\noindent \initialiv{A}{lors là}\blindtext\par
\noindent \initialiv{À}{ la bonheur des dames}\blindtext\par
\noindent \initialiv{É}{tonnez-le}\blindtext\par
\noindent \initialiv{Q}{ualitativement}\blindtext\par
\noindent \initialiv{V}{aloriser}\blindtext\par
\Blindtext
\phantomsection
\label{utopie}
\Blinddocument
\fi
\end{document}
