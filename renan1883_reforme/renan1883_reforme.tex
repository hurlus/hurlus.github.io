%%%%%%%%%%%%%%%%%%%%%%%%%%%%%%%%%
% LaTeX model https://hurlus.fr %
%%%%%%%%%%%%%%%%%%%%%%%%%%%%%%%%%

% Needed before document class
\RequirePackage{pdftexcmds} % needed for tests expressions
\RequirePackage{fix-cm} % correct units

% Define mode
\def\mode{a4}

\newif\ifaiv % a4
\newif\ifav % a5
\newif\ifbooklet % booklet
\newif\ifcover % cover for booklet

\ifnum \strcmp{\mode}{cover}=0
  \covertrue
\else\ifnum \strcmp{\mode}{booklet}=0
  \booklettrue
\else\ifnum \strcmp{\mode}{a5}=0
  \avtrue
\else
  \aivtrue
\fi\fi\fi

\ifbooklet % do not enclose with {}
  \documentclass[french,twoside]{book} % ,notitlepage
  \usepackage[%
    papersize={105mm, 297mm},
    inner=12mm,
    outer=12mm,
    top=20mm,
    bottom=15mm,
    marginparsep=0pt,
  ]{geometry}
  \usepackage[fontsize=9.5pt]{scrextend} % for Roboto
\else\ifav
  \documentclass[french,twoside]{book} % ,notitlepage
  \usepackage[%
    a5paper,
    inner=25mm,
    outer=15mm,
    top=15mm,
    bottom=15mm,
    marginparsep=0pt,
  ]{geometry}
  \usepackage[fontsize=12pt]{scrextend}
\else% A4 2 cols
  \documentclass[twocolumn]{report}
  \usepackage[%
    a4paper,
    inner=15mm,
    outer=10mm,
    top=25mm,
    bottom=18mm,
    marginparsep=0pt,
  ]{geometry}
  \setlength{\columnsep}{20mm}
  \usepackage[fontsize=9.5pt]{scrextend}
\fi\fi

%%%%%%%%%%%%%%
% Alignments %
%%%%%%%%%%%%%%
% before teinte macros

\setlength{\arrayrulewidth}{0.2pt}
\setlength{\columnseprule}{\arrayrulewidth} % twocol
\setlength{\parskip}{0pt} % classical para with no margin
\setlength{\parindent}{1.5em}

%%%%%%%%%%
% Colors %
%%%%%%%%%%
% before Teinte macros

\usepackage[dvipsnames]{xcolor}
\definecolor{rubric}{HTML}{800000} % the tonic 0c71c3
\def\columnseprulecolor{\color{rubric}}
\colorlet{borderline}{rubric!30!} % definecolor need exact code
\definecolor{shadecolor}{gray}{0.95}
\definecolor{bghi}{gray}{0.5}

%%%%%%%%%%%%%%%%%
% Teinte macros %
%%%%%%%%%%%%%%%%%
%%%%%%%%%%%%%%%%%%%%%%%%%%%%%%%%%%%%%%%%%%%%%%%%%%%
% <TEI> generic (LaTeX names generated by Teinte) %
%%%%%%%%%%%%%%%%%%%%%%%%%%%%%%%%%%%%%%%%%%%%%%%%%%%
% This template is inserted in a specific design
% It is XeLaTeX and otf fonts

\makeatletter % <@@@


\usepackage{blindtext} % generate text for testing
\usepackage[strict]{changepage} % for modulo 4
\usepackage{contour} % rounding words
\usepackage[nodayofweek]{datetime}
% \usepackage{DejaVuSans} % seems buggy for sffont font for symbols
\usepackage{enumitem} % <list>
\usepackage{etoolbox} % patch commands
\usepackage{fancyvrb}
\usepackage{fancyhdr}
\usepackage{float}
\usepackage{fontspec} % XeLaTeX mandatory for fonts
\usepackage{footnote} % used to capture notes in minipage (ex: quote)
\usepackage{framed} % bordering correct with footnote hack
\usepackage{graphicx}
\usepackage{lettrine} % drop caps
\usepackage{lipsum} % generate text for testing
\usepackage[framemethod=tikz,]{mdframed} % maybe used for frame with footnotes inside
\usepackage{pdftexcmds} % needed for tests expressions
\usepackage{polyglossia} % non-break space french punct, bug Warning: "Failed to patch part"
\usepackage[%
  indentfirst=false,
  vskip=1em,
  noorphanfirst=true,
  noorphanafter=true,
  leftmargin=\parindent,
  rightmargin=0pt,
]{quoting}
\usepackage{ragged2e}
\usepackage{setspace} % \setstretch for <quote>
\usepackage{tabularx} % <table>
\usepackage[explicit]{titlesec} % wear titles, !NO implicit
\usepackage{tikz} % ornaments
\usepackage{tocloft} % styling tocs
\usepackage[fit]{truncate} % used im runing titles
\usepackage{unicode-math}
\usepackage[normalem]{ulem} % breakable \uline, normalem is absolutely necessary to keep \emph
\usepackage{verse} % <l>
\usepackage{xcolor} % named colors
\usepackage{xparse} % @ifundefined
\XeTeXdefaultencoding "iso-8859-1" % bad encoding of xstring
\usepackage{xstring} % string tests
\XeTeXdefaultencoding "utf-8"
\PassOptionsToPackage{hyphens}{url} % before hyperref, which load url package

% TOTEST
% \usepackage{hypcap} % links in caption ?
% \usepackage{marginnote}
% TESTED
% \usepackage{background} % doesn’t work with xetek
% \usepackage{bookmark} % prefers the hyperref hack \phantomsection
% \usepackage[color, leftbars]{changebar} % 2 cols doc, impossible to keep bar left
% \usepackage[utf8x]{inputenc} % inputenc package ignored with utf8 based engines
% \usepackage[sfdefault,medium]{inter} % no small caps
% \usepackage{firamath} % choose firasans instead, firamath unavailable in Ubuntu 21-04
% \usepackage{flushend} % bad for last notes, supposed flush end of columns
% \usepackage[stable]{footmisc} % BAD for complex notes https://texfaq.org/FAQ-ftnsect
% \usepackage{helvet} % not for XeLaTeX
% \usepackage{multicol} % not compatible with too much packages (longtable, framed, memoir…)
% \usepackage[default,oldstyle,scale=0.95]{opensans} % no small caps
% \usepackage{sectsty} % \chapterfont OBSOLETE
% \usepackage{soul} % \ul for underline, OBSOLETE with XeTeX
% \usepackage[breakable]{tcolorbox} % text styling gone, footnote hack not kept with breakable


% Metadata inserted by a program, from the TEI source, for title page and runing heads
\title{\textbf{ La Réforme intellectuelle et morale de la France }}
\date{1883}
\author{Ernest Renan}
\def\elbibl{Ernest Renan. 1883. \emph{La Réforme intellectuelle et morale de la France}}
\def\elsource{}

% Default metas
\newcommand{\colorprovide}[2]{\@ifundefinedcolor{#1}{\colorlet{#1}{#2}}{}}
\colorprovide{rubric}{red}
\colorprovide{silver}{lightgray}
\@ifundefined{syms}{\newfontfamily\syms{DejaVu Sans}}{}
\newif\ifdev
\@ifundefined{elbibl}{% No meta defined, maybe dev mode
  \newcommand{\elbibl}{Titre court ?}
  \newcommand{\elbook}{Titre du livre source ?}
  \newcommand{\elabstract}{Résumé\par}
  \newcommand{\elurl}{http://oeuvres.github.io/elbook/2}
  \author{Éric Lœchien}
  \title{Un titre de test assez long pour vérifier le comportement d’une maquette}
  \date{1566}
  \devtrue
}{}
\let\eltitle\@title
\let\elauthor\@author
\let\eldate\@date


\defaultfontfeatures{
  % Mapping=tex-text, % no effect seen
  Scale=MatchLowercase,
  Ligatures={TeX,Common},
}


% generic typo commands
\newcommand{\astermono}{\medskip\centerline{\color{rubric}\large\selectfont{\syms ✻}}\medskip\par}%
\newcommand{\astertri}{\medskip\par\centerline{\color{rubric}\large\selectfont{\syms ✻\,✻\,✻}}\medskip\par}%
\newcommand{\asterism}{\bigskip\par\noindent\parbox{\linewidth}{\centering\color{rubric}\large{\syms ✻}\\{\syms ✻}\hskip 0.75em{\syms ✻}}\bigskip\par}%

% lists
\newlength{\listmod}
\setlength{\listmod}{\parindent}
\setlist{
  itemindent=!,
  listparindent=\listmod,
  labelsep=0.2\listmod,
  parsep=0pt,
  % topsep=0.2em, % default topsep is best
}
\setlist[itemize]{
  label=—,
  leftmargin=0pt,
  labelindent=1.2em,
  labelwidth=0pt,
}
\setlist[enumerate]{
  label={\bf\color{rubric}\arabic*.},
  labelindent=0.8\listmod,
  leftmargin=\listmod,
  labelwidth=0pt,
}
\newlist{listalpha}{enumerate}{1}
\setlist[listalpha]{
  label={\bf\color{rubric}\alph*.},
  leftmargin=0pt,
  labelindent=0.8\listmod,
  labelwidth=0pt,
}
\newcommand{\listhead}[1]{\hspace{-1\listmod}\emph{#1}}

\renewcommand{\hrulefill}{%
  \leavevmode\leaders\hrule height 0.2pt\hfill\kern\z@}

% General typo
\DeclareTextFontCommand{\textlarge}{\large}
\DeclareTextFontCommand{\textsmall}{\small}

% commands, inlines
\newcommand{\anchor}[1]{\Hy@raisedlink{\hypertarget{#1}{}}} % link to top of an anchor (not baseline)
\newcommand\abbr[1]{#1}
\newcommand{\autour}[1]{\tikz[baseline=(X.base)]\node [draw=rubric,thin,rectangle,inner sep=1.5pt, rounded corners=3pt] (X) {\color{rubric}#1};}
\newcommand\corr[1]{#1}
\newcommand{\ed}[1]{ {\color{silver}\sffamily\footnotesize (#1)} } % <milestone ed="1688"/>
\newcommand\expan[1]{#1}
\newcommand\foreign[1]{\emph{#1}}
\newcommand\gap[1]{#1}
\renewcommand{\LettrineFontHook}{\color{rubric}}
\newcommand{\initial}[2]{\lettrine[lines=2, loversize=0.3, lhang=0.3]{#1}{#2}}
\newcommand{\initialiv}[2]{%
  \let\oldLFH\LettrineFontHook
  % \renewcommand{\LettrineFontHook}{\color{rubric}\ttfamily}
  \IfSubStr{QJ’}{#1}{
    \lettrine[lines=4, lhang=0.2, loversize=-0.1, lraise=0.2]{\smash{#1}}{#2}
  }{\IfSubStr{É}{#1}{
    \lettrine[lines=4, lhang=0.2, loversize=-0, lraise=0]{\smash{#1}}{#2}
  }{\IfSubStr{ÀÂ}{#1}{
    \lettrine[lines=4, lhang=0.2, loversize=-0, lraise=0, slope=0.6em]{\smash{#1}}{#2}
  }{\IfSubStr{A}{#1}{
    \lettrine[lines=4, lhang=0.2, loversize=0.2, slope=0.6em]{\smash{#1}}{#2}
  }{\IfSubStr{V}{#1}{
    \lettrine[lines=4, lhang=0.2, loversize=0.2, slope=-0.5em]{\smash{#1}}{#2}
  }{
    \lettrine[lines=4, lhang=0.2, loversize=0.2]{\smash{#1}}{#2}
  }}}}}
  \let\LettrineFontHook\oldLFH
}
\newcommand{\labelchar}[1]{\textbf{\color{rubric} #1}}
\newcommand{\milestone}[1]{\autour{\footnotesize\color{rubric} #1}} % <milestone n="4"/>
\newcommand\name[1]{#1}
\newcommand\orig[1]{#1}
\newcommand\orgName[1]{#1}
\newcommand\persName[1]{#1}
\newcommand\placeName[1]{#1}
\newcommand{\pn}[1]{\IfSubStr{-—–¶}{#1}% <p n="3"/>
  {\noindent{\bfseries\color{rubric}   ¶  }}
  {{\footnotesize\autour{ #1}  }}}
\newcommand\reg{}
% \newcommand\ref{} % already defined
\newcommand\sic[1]{#1}
\newcommand\surname[1]{\textsc{#1}}
\newcommand\term[1]{\textbf{#1}}

\def\mednobreak{\ifdim\lastskip<\medskipamount
  \removelastskip\nopagebreak\medskip\fi}
\def\bignobreak{\ifdim\lastskip<\bigskipamount
  \removelastskip\nopagebreak\bigskip\fi}

% commands, blocks
\newcommand{\byline}[1]{\bigskip{\RaggedLeft{#1}\par}\bigskip}
\newcommand{\bibl}[1]{{\RaggedLeft{#1}\par\bigskip}}
\newcommand{\biblitem}[1]{{\noindent\hangindent=\parindent   #1\par}}
\newcommand{\dateline}[1]{\medskip{\RaggedLeft{#1}\par}\bigskip}
\newcommand{\labelblock}[1]{\medbreak{\noindent\color{rubric}\bfseries #1}\par\mednobreak}
\newcommand{\salute}[1]{\bigbreak{#1}\par\medbreak}
\newcommand{\signed}[1]{\bigbreak\filbreak{\raggedleft #1\par}\medskip}

% environments for blocks (some may become commands)
\newenvironment{borderbox}{}{} % framing content
\newenvironment{citbibl}{\ifvmode\hfill\fi}{\ifvmode\par\fi }
\newenvironment{docAuthor}{\ifvmode\vskip4pt\fontsize{16pt}{18pt}\selectfont\fi\itshape}{\ifvmode\par\fi }
\newenvironment{docDate}{}{\ifvmode\par\fi }
\newenvironment{docImprint}{\vskip6pt}{\ifvmode\par\fi }
\newenvironment{docTitle}{\vskip6pt\bfseries\fontsize{18pt}{22pt}\selectfont}{\par }
\newenvironment{msHead}{\vskip6pt}{\par}
\newenvironment{msItem}{\vskip6pt}{\par}
\newenvironment{titlePart}{}{\par }


% environments for block containers
\newenvironment{argument}{\itshape\parindent0pt}{\vskip1.5em}
\newenvironment{biblfree}{}{\ifvmode\par\fi }
\newenvironment{bibitemlist}[1]{%
  \list{\@biblabel{\@arabic\c@enumiv}}%
  {%
    \settowidth\labelwidth{\@biblabel{#1}}%
    \leftmargin\labelwidth
    \advance\leftmargin\labelsep
    \@openbib@code
    \usecounter{enumiv}%
    \let\p@enumiv\@empty
    \renewcommand\theenumiv{\@arabic\c@enumiv}%
  }
  \sloppy
  \clubpenalty4000
  \@clubpenalty \clubpenalty
  \widowpenalty4000%
  \sfcode`\.\@m
}%
{\def\@noitemerr
  {\@latex@warning{Empty `bibitemlist' environment}}%
\endlist}
\newenvironment{quoteblock}% may be used for ornaments
  {\begin{quoting}}
  {\end{quoting}}

% table () is preceded and finished by custom command
\newcommand{\tableopen}[1]{%
  \ifnum\strcmp{#1}{wide}=0{%
    \begin{center}
  }
  \else\ifnum\strcmp{#1}{long}=0{%
    \begin{center}
  }
  \else{%
    \begin{center}
  }
  \fi\fi
}
\newcommand{\tableclose}[1]{%
  \ifnum\strcmp{#1}{wide}=0{%
    \end{center}
  }
  \else\ifnum\strcmp{#1}{long}=0{%
    \end{center}
  }
  \else{%
    \end{center}
  }
  \fi\fi
}


% text structure
\newcommand\chapteropen{} % before chapter title
\newcommand\chaptercont{} % after title, argument, epigraph…
\newcommand\chapterclose{} % maybe useful for multicol settings
\setcounter{secnumdepth}{-2} % no counters for hierarchy titles
\setcounter{tocdepth}{5} % deep toc
\markright{\@title} % ???
\markboth{\@title}{\@author} % ???
\renewcommand\tableofcontents{\@starttoc{toc}}
% toclof format
% \renewcommand{\@tocrmarg}{0.1em} % Useless command?
% \renewcommand{\@pnumwidth}{0.5em} % {1.75em}
\renewcommand{\@cftmaketoctitle}{}
\setlength{\cftbeforesecskip}{\z@ \@plus.2\p@}
\renewcommand{\cftchapfont}{}
\renewcommand{\cftchapdotsep}{\cftdotsep}
\renewcommand{\cftchapleader}{\normalfont\cftdotfill{\cftchapdotsep}}
\renewcommand{\cftchappagefont}{\bfseries}
\setlength{\cftbeforechapskip}{0em \@plus\p@}
% \renewcommand{\cftsecfont}{\small\relax}
\renewcommand{\cftsecpagefont}{\normalfont}
% \renewcommand{\cftsubsecfont}{\small\relax}
\renewcommand{\cftsecdotsep}{\cftdotsep}
\renewcommand{\cftsecpagefont}{\normalfont}
\renewcommand{\cftsecleader}{\normalfont\cftdotfill{\cftsecdotsep}}
\setlength{\cftsecindent}{1em}
\setlength{\cftsubsecindent}{2em}
\setlength{\cftsubsubsecindent}{3em}
\setlength{\cftchapnumwidth}{1em}
\setlength{\cftsecnumwidth}{1em}
\setlength{\cftsubsecnumwidth}{1em}
\setlength{\cftsubsubsecnumwidth}{1em}

% footnotes
\newif\ifheading
\newcommand*{\fnmarkscale}{\ifheading 0.70 \else 1 \fi}
\renewcommand\footnoterule{\vspace*{0.3cm}\hrule height \arrayrulewidth width 3cm \vspace*{0.3cm}}
\setlength\footnotesep{1.5\footnotesep} % footnote separator
\renewcommand\@makefntext[1]{\parindent 1.5em \noindent \hb@xt@1.8em{\hss{\normalfont\@thefnmark . }}#1} % no superscipt in foot
\patchcmd{\@footnotetext}{\footnotesize}{\footnotesize\sffamily}{}{} % before scrextend, hyperref


%   see https://tex.stackexchange.com/a/34449/5049
\def\truncdiv#1#2{((#1-(#2-1)/2)/#2)}
\def\moduloop#1#2{(#1-\truncdiv{#1}{#2}*#2)}
\def\modulo#1#2{\number\numexpr\moduloop{#1}{#2}\relax}

% orphans and widows
\clubpenalty=9996
\widowpenalty=9999
\brokenpenalty=4991
\predisplaypenalty=10000
\postdisplaypenalty=1549
\displaywidowpenalty=1602
\hyphenpenalty=400
% Copied from Rahtz but not understood
\def\@pnumwidth{1.55em}
\def\@tocrmarg {2.55em}
\def\@dotsep{4.5}
\emergencystretch 3em
\hbadness=4000
\pretolerance=750
\tolerance=2000
\vbadness=4000
\def\Gin@extensions{.pdf,.png,.jpg,.mps,.tif}
% \renewcommand{\@cite}[1]{#1} % biblio

\usepackage{hyperref} % supposed to be the last one, :o) except for the ones to follow
\urlstyle{same} % after hyperref
\hypersetup{
  % pdftex, % no effect
  pdftitle={\elbibl},
  % pdfauthor={Your name here},
  % pdfsubject={Your subject here},
  % pdfkeywords={keyword1, keyword2},
  bookmarksnumbered=true,
  bookmarksopen=true,
  bookmarksopenlevel=1,
  pdfstartview=Fit,
  breaklinks=true, % avoid long links
  pdfpagemode=UseOutlines,    % pdf toc
  hyperfootnotes=true,
  colorlinks=false,
  pdfborder=0 0 0,
  % pdfpagelayout=TwoPageRight,
  % linktocpage=true, % NO, toc, link only on page no
}

\makeatother % /@@@>
%%%%%%%%%%%%%%
% </TEI> end %
%%%%%%%%%%%%%%


%%%%%%%%%%%%%
% footnotes %
%%%%%%%%%%%%%
\renewcommand{\thefootnote}{\bfseries\textcolor{rubric}{\arabic{footnote}}} % color for footnote marks

%%%%%%%%%
% Fonts %
%%%%%%%%%
\usepackage[]{roboto} % SmallCaps, Regular is a bit bold
% \linespread{0.90} % too compact, keep font natural
\newfontfamily\fontrun[]{Roboto Condensed Light} % condensed runing heads
\ifav
  \setmainfont[
    ItalicFont={Roboto Light Italic},
  ]{Roboto}
\else\ifbooklet
  \setmainfont[
    ItalicFont={Roboto Light Italic},
  ]{Roboto}
\else
\setmainfont[
  ItalicFont={Roboto Italic},
]{Roboto Light}
\fi\fi
\renewcommand{\LettrineFontHook}{\bfseries\color{rubric}}
% \renewenvironment{labelblock}{\begin{center}\bfseries\color{rubric}}{\end{center}}

%%%%%%%%
% MISC %
%%%%%%%%

\setdefaultlanguage[frenchpart=false]{french} % bug on part


\newenvironment{quotebar}{%
    \def\FrameCommand{{\color{rubric!10!}\vrule width 0.5em} \hspace{0.9em}}%
    \def\OuterFrameSep{\itemsep} % séparateur vertical
    \MakeFramed {\advance\hsize-\width \FrameRestore}
  }%
  {%
    \endMakeFramed
  }
\renewenvironment{quoteblock}% may be used for ornaments
  {%
    \savenotes
    \setstretch{0.9}
    \normalfont
    \begin{quotebar}
  }
  {%
    \end{quotebar}
    \spewnotes
  }


\renewcommand{\headrulewidth}{\arrayrulewidth}
\renewcommand{\headrule}{{\color{rubric}\hrule}}

% delicate tuning, image has produce line-height problems in title on 2 lines
\titleformat{name=\chapter} % command
  [display] % shape
  {\vspace{1.5em}\centering} % format
  {} % label
  {0pt} % separator between n
  {}
[{\color{rubric}\huge\textbf{#1}}\bigskip] % after code
% \titlespacing{command}{left spacing}{before spacing}{after spacing}[right]
\titlespacing*{\chapter}{0pt}{-2em}{0pt}[0pt]

\titleformat{name=\section}
  [block]{}{}{}{}
  [\vbox{\color{rubric}\large\raggedleft\textbf{#1}}]
\titlespacing{\section}{0pt}{0pt plus 4pt minus 2pt}{\baselineskip}

\titleformat{name=\subsection}
  [block]
  {}
  {} % \thesection
  {} % separator \arrayrulewidth
  {}
[\vbox{\large\textbf{#1}}]
% \titlespacing{\subsection}{0pt}{0pt plus 4pt minus 2pt}{\baselineskip}

\ifaiv
  \fancypagestyle{main}{%
    \fancyhf{}
    \setlength{\headheight}{1.5em}
    \fancyhead{} % reset head
    \fancyfoot{} % reset foot
    \fancyhead[L]{\truncate{0.45\headwidth}{\fontrun\elbibl}} % book ref
    \fancyhead[R]{\truncate{0.45\headwidth}{ \fontrun\nouppercase\leftmark}} % Chapter title
    \fancyhead[C]{\thepage}
  }
  \fancypagestyle{plain}{% apply to chapter
    \fancyhf{}% clear all header and footer fields
    \setlength{\headheight}{1.5em}
    \fancyhead[L]{\truncate{0.9\headwidth}{\fontrun\elbibl}}
    \fancyhead[R]{\thepage}
  }
\else
  \fancypagestyle{main}{%
    \fancyhf{}
    \setlength{\headheight}{1.5em}
    \fancyhead{} % reset head
    \fancyfoot{} % reset foot
    \fancyhead[RE]{\truncate{0.9\headwidth}{\fontrun\elbibl}} % book ref
    \fancyhead[LO]{\truncate{0.9\headwidth}{\fontrun\nouppercase\leftmark}} % Chapter title, \nouppercase needed
    \fancyhead[RO,LE]{\thepage}
  }
  \fancypagestyle{plain}{% apply to chapter
    \fancyhf{}% clear all header and footer fields
    \setlength{\headheight}{1.5em}
    \fancyhead[L]{\truncate{0.9\headwidth}{\fontrun\elbibl}}
    \fancyhead[R]{\thepage}
  }
\fi

\ifav % a5 only
  \titleclass{\section}{top}
\fi

\newcommand\chapo{{%
  \vspace*{-3em}
  \centering % no vskip ()
  {\Large\addfontfeature{LetterSpace=25}\bfseries{\elauthor}}\par
  \smallskip
  {\large\eldate}\par
  \bigskip
  {\Large\selectfont{\eltitle}}\par
  \bigskip
  {\color{rubric}\hline\par}
  \bigskip
  {\Large TEXTE LIBRE À PARTICPATION LIBRE\par}
  \centerline{\small\color{rubric} {hurlus.fr, tiré le \today}}\par
  \bigskip
}}

\newcommand\cover{{%
  \thispagestyle{empty}
  \centering
  {\LARGE\bfseries{\elauthor}}\par
  \bigskip
  {\Large\eldate}\par
  \bigskip
  \bigskip
  {\LARGE\selectfont{\eltitle}}\par
  \vfill\null
  {\color{rubric}\setlength{\arrayrulewidth}{2pt}\hline\par}
  \vfill\null
  {\Large TEXTE LIBRE À PARTICPATION LIBRE\par}
  \centerline{{\href{https://hurlus.fr}{\dotuline{hurlus.fr}}, tiré le \today}}\par
}}

\begin{document}
\pagestyle{empty}
\ifbooklet{
  \cover\newpage
  \thispagestyle{empty}\hbox{}\newpage
  \cover\newpage\noindent Les voyages de la brochure\par
  \bigskip
  \begin{tabularx}{\textwidth}{l|X|X}
    \textbf{Date} & \textbf{Lieu}& \textbf{Nom/pseudo} \\ \hline
    \rule{0pt}{25cm} &  &   \\
  \end{tabularx}
  \newpage
  \addtocounter{page}{-4}
}\fi

\thispagestyle{empty}
\ifaiv
  \twocolumn[\chapo]
\else
  \chapo
\fi
{\it\elabstract}
\bigskip
\makeatletter\@starttoc{toc}\makeatother % toc without new page
\bigskip

\pagestyle{main} % after style

  \section[{Première partie, . le mal}]{Première partie,  \\
le mal}\renewcommand{\leftmark}{Première partie,  \\
le mal}

\noindent Ceux qui veulent à tout prix découvrir dans l’histoire l’application d’une rigoureuse justice distributive s’imposent une tâche assez rude. Si, en beaucoup de cas, nous voyons les crimes nationaux suivis d’un prompt châtiment, dans une foule de cas aussi nous voyons le monde régi par des jugements moins sévères ; beaucoup de pays ont pu être faibles et corrompus impunément. C’est certainement un des signes de grandeur de la France que cela ne lui ait pas été permis. Énervée par la démocratie, démoralisée par sa prospérité même, la France a expié de la manière la plus cruelle ses années d’égarement. La raison de ce fait est dans l’importance même de la France et dans la noblesse de son passé. Il y a une justice pour elle ; il ne lui est pas loisible de s’abandonner, de négliger sa vocation ; il est évident que la Providence l’aime ; car elle la châtie. Un pays qui a joué un rôle de premier ordre n’a pas le droit de se réduire au matérialisme bourgeois qui ne demande qu’à jouir tranquillement de ses richesses acquises. N’est pas médiocre qui veut. L’homme qui prostitue un grand nom, qui manque à une mission écrite dans sa nature, ne peut se permettre sans conséquence une foule de choses que l’on pardonne a l’homme ordinaire, qui n’a ni passé à continuer, ni grand devoir à remplir.\par
Pour voir en ces dernières années que l’état moral de la France était gravement atteint, il fallait quelque pénétration d’esprit, une certaine habitude des raisonnements politiques et historiques. Pour voir le mal aujourd’hui, il ne faut, hélas ! que des yeux. L’édifice de nos chimères s’est effondre comme les châteaux féériques qu’on bâtit en rêve. Présomption, vanité puérile, indiscipline, manque de sérieux, d’application, d’honnêteté, faiblesse de tête, incapacité de tenir à la fois beaucoup d’idées sous le regard, absence d’esprit scientifique, naïve. et grossière ignorance, voilà depuis un an l’abrégé de notre histoire. Cette armée, si fière et si prétentieuse, n’a pas rencontré une seule bonne chance. Ces hommes d’État, si sûrs de leur fait, se sont trouvés des enfants. Cette administration infatuée a été convaincue d’incapacité. Cette instruction publique, fermée à tout progrès, est convaincue d’avoir laissé l’esprit de la France s’abimer dans la nullité. Ce clergé catholique, qui prêchait hautement l’infériorité des nations protestantes, est resté spectateur atterré d’une ruine qu’il avait en partie faite. Cette dynastie, dont les racines dans le pays semblaient si profondes, n’eut pas le 4 septembre un seul défenseur. Cette opposition, qui prétendait avoir dans ses recettes révolutionnaires des remèdes à tous les maux, s’est trouvée au bout de quelques jours aussi impopulaire que la dynastie déchue. Ce parti républicain, qui, plein des funestes erreurs qu’on répand depuis un demi-siècle sur l’histoire de la Révolution, s’est cru capable de répéter une partie qui ne fut gagnée il y a quatre-vingts ans que par suite de circonstances tout à fait différentes de celles d’aujourd’hui, s’est trouvée n’être qu’un halluciné, prenant ses rêves pour des réalités. Tout a croulé comme en une vision d’Apocalypse. La légende même s’est vue blessée à mort. Celle de l’Empire a été détruite par Napoléon III ; celle de 1792 a reçu le coup de grâce de M. Gambetta ; celle de la Terreur (car la Terreur même avait chez nous sa légende ) a eu sa hideuse parodie dans la Commune ; celle de Louis XIV ne sera plus ce qu’elle était depuis le jour où le descendant de l’électeur de Brandebourg a relevé l’empire de Charlemagne dans la salle des fêtes de Versailles. Seul, Bossuet se trouve avoir été prophète, quand il dit : {\itshape Et nunc, reges, intelligite} ! \par
De nos jours (et cela rend la tâche des réformateurs difficile ), ce sont les peuples qui doivent comprendre. Essayons, par une analyse aussi exacte que possible, de nous rendre compte du mal de la France, pour tâcher de découvrir le remède qu’il convient d’y appliquer les forces du malade sont très grandes ; ses ressources sont comme infinies ; sa bonne volonté est réelle. C’est au médecin à ne pas se tromper ; car tel régime étroitement conçu, tel remède appliqué hors de propos, révolterait le malade, le tuerait ou aggraverait son mal.\par
\par
\subsection[{I}]{I}
\noindent L’histoire de France est un tout si bien lié dans ses parties, qu’on ne peut comprendre un seul de nos deuils contemporains sans en rechercher la cause dans le passé. Nous avons, il y a deux ans \footnote{Dans le travail sur la Monarchie constitutionnelle.}, exposé ce que nous regardons comme la marche régulière des États sortis de la féodalité du moyen âge, marche dont l’Angleterre est le type le plus parfait, puisque l’Angleterre, sans rompre avec sa royauté, avec sa noblesse, avec ses comtes, avec ses communes, avec son Église, avec ses universités, a trouvé moyen d’être l’État le plus libre, le plus prospère et le plus patriote qu’il y ait. Tout autre fut la marche de la société française depuis le \textsc{xii}\textsuperscript{e} siècle. La royauté capétienne, comme il arrive d’ordinaire aux grandes forces, porta son principe jusqu’à l’exagération. Elle détruisit la possibilité de toute vie provinciale, de toute représentation de la nation. Déjà, sous Philippe le Bel, le mal est évident. L‘élément qui a fait ailleurs la vie parlementaire, la petite noblesse de campagne, a perdu son importance. Le roi ne convoque les états généraux que pour qu’on le supplie de faire ce qu’il a déjà décidé. Comme instruments de gouvernement, il ne vent plus employer que ses parents, puissante aristocratie de princes du sang, assez égoïstes, et des gens de loi ou d’administration anoblis ({\itshape milites} {\itshape regis} ), serviteurs complaisants du pouvoir absolu. Cet état de choses se fait amnistier au \textsc{xvii}\textsuperscript{e} siècle par la grandeur incomparable qu’il donne à la France ; mais bientôt après le contraste devient criant. La nation la plus spirituelle de l’Europe n’a pour réaliser ses idées qu’une machine politique informe. Turgot considère les Parlements comme le principal obstacle à tout bien ; il n’espère rien des assemblées. Cet homme admirable, si dégagé de tout amour-propre, se trompait-il ? non. Il voyait juste, et ce qu’il voyait équivalait à dire que le mal était sans remède. Ajoutez à cela une profonde démoralisation du peuple ; le protestantisme, qui l’eût élevé, avait été expulsé ; le catholicisme n’avait pas fait son éducation. L’ignorance des basses classes était effroyable. Richelieu, l’abbé Fleury posent nettement en principe que le peuple ne doit savoir ni lire ni écrire. À côté de cette barbarie, une société charmante, pleine d’esprit, de lumières et de grâce. On ne vit jamais plus clairement les aptitudes intimes de la France, ce qu’elle peut et ce qu’elle ne peut pas. La France sait admirablement faire de la dentelle ; elle ne sait pas faire de la toile de ménage. Les besognes humbles, comme celle du magister, seront toujours chez nous pauvrement exécutées. La France excelle dans l ‘exquis ; elle est médiocre dans le commun. Par quel caprice est-elle avec cela démocratique ? Par le même caprice qui fait que Paris, tout en vivant de la cour et du luxe, est une ville socialiste, que Paris, qui passe son temps à persifler toute croyance et toute vertu, est intraitable, fanatique, badaud, quand il s’agit de sa chimère de république. Admirables assurément furent les débuts de la Révolution, et, si l’on s’était borné à convoquer les états généraux, à les régulariser, à les rendre annuels, on eût été parfaitement dans la vérité. Mais la fausse politique de Rousseau l’emporta. On voulut faire une constitution {\itshape a priori}. On ne remarqua pas que l’Angleterre, le plus constitutionnel des pays, n’a jamais eu de constitution écrite, strictement libellée. On se laissa déborder par le peuple ; on applaudit puérilement au désordre de la prise de la Bastille, sans songer que ce désordre emporterait tout plus tard. Mirabeau, le plus grand, le seul grand politique du temps, débuta par des imprudences qui l’eussent, probablement perdu, s’il eût vécu ; car, pour un homme d’État, il est bien plus avantageux d’avoir débuté par la réaction que par des complaisances pour l’anarchie. L’étourderie des avocats de Bordeaux, leurs déclamations creuses, leur légèreté morale achevèrent de tout ruiner. On se figura que l’État, qui s’était incarné dans le roi, pouvait se passer du roi, et que l’idée abstraite de la chose publique suffirait pour maintenir un pays où les vertus publiques font trop souvent défaut.\par
Le jour où la France coupa la tête à son roi, elle commit un suicide. La France ne peut être comparée à ces petites patries antiques, se composant le plus souvent d’une ville avec sa banlieue, où tout le monde était parent. La France était une grande société d’actionnaires formée par un spéculateur de premier ordre, la maison capétienne. Les actionnaires ont cru pouvoir se passer du chef, et puis continuer seuls les affaires. Cela ira bien, tant que les affaires seront bonnes ; mais, les affaires devenant mauvaises, il y aura des demandes de liquidation. La France avait été faite par la dynastie capétienne. en supposant que la vieille Gaule eût le sentiment de son unité nationale, la domination romaine, la conquête germanique avaient détruit ce sentiment. L ‘empire franc, soit sous les Mérovingiens, soit sous les Carlovingiens, est une construction artificielle dont l’unité ne gît que dans la force des conquérants. Le traité de Verdun, qui rompt cette unité, coupe l’empire franc du nord au sud en trois bandes, dont l’une, la part de Charles ou Carolingie, répond si peu à ce que nous appelons la France, que la Flandre entière et la Catalogne en font partie, tandis que vers l’est elle a pour limites la Saône et les Cévennes. La politique capétienne arrondit ce lambeau incorrect, et en huit cents ans fit la France comme nous l’entendons, la France qui a crée tout ce dont nous vivons, ce qui nous lie, ce qui est notre raison d’être. La France est de la sorte le résultat de la politique capétienne continuée avec une admirable suite. Pourquoi le Languedoc est-il réuni à la France du nord, union que ni la langue, ni la race, ni l’histoire, ni le caractère des populations n’appelaient ? Parce que les rois de Paris, pendant tout le \textsc{xiii}\textsuperscript{e} siècle, exercèrent sur ces contrées une action persistante et victorieuse. Pourquoi Lyon fait-il partie de la France ? Parce que Philippe le Bel, au moyen des subtilités de ses légistes, réussit à le prendre dans les mailles de son filet. Pourquoi les Dauphinois sont-ils nos compatriotes ? Parce que, le dauphin Humbert étant tombé dans une sorte de folie, le roi de France se trouva là pour acheter ses terres à beaux deniers comptants. Pourquoi la Provence a-t-elle été entraînée dans le tourbillon de la Carolingie, où rien ne semblait d’abord faire penser qu’elle dût être portée ? Grâce aux roueries de Louis XI et de son compère Palamède de Forbin. Pourquoi la Franche-Comté, l’Alsace, la Lorraine se sont-elles réunies à la Carolingie, malgré la ligne méridienne tracée par le traité de Verdun ? Parce que la maison de Bourbon retrouva pour agrandir le domaine royal le secret qu’avaient si admirablement pratiqué les premiers Capétiens. Pourquoi enfin Paris, ville si peu centrale, est-elle la capitale de la France ? Parce que Paris a été la ville des Capétiens, parce que l’abbé de Saint-Denis est devenu roi de France \footnote{ \noindent « … Challes, li rois de Saint-Denis. » \par
 
\bibl{(Roman de Roncevaux, laisse 40. ) }
\par
 \noindent Hugues le Blanc dut sa fortune à la possession des grandes abbayes de Saint-Denis, de Saint-Germain-des-Prés, de Saint-Martin de Tours, qui faisait de lui le tuteur de pays riches et prospères. La bannière du roi capétien, c’est la bannière de Saint Denis. Son cri de ralliement est {\itshape Montjoie Saint-Denis.} Les premiers Capétiens chantent au chœur à Saint-Denis.
 }. Naïveté sans égale ! Cette ville, qui réclame sur le reste de la France un privilège aristocratique de supériorité et qui doit ce privilège à la royauté, est en même temps le centre de l’utopie républicaine. Comment Paris ne voit-il pas qu’il n’est ce qu’il est que par la royauté, qu’il ne reprendra toute son importance de capitale que par la royauté, qu’une république, selon la règle posée par l’illustre fondateur des États-Unis d’Amérique, créerait nécessairement pour son gouvernement central, à Amboise ou à Blois, un petit Washington ? \par
Voilà ce que ne comprirent pas les hommes ignorants et bornés qui prirent en main les destinées de la France à la fin du dernier siècle. Ils se figurèrent qu’on pouvait se passer du roi ; ils ne comprirent pas que, le roi une fois supprimé, l’édifice dont le roi était la clef de voûte croulait. Les théories républicaines du \textsc{xviii}\textsuperscript{e} siècle avaient pu réussir en Amérique, parce que l’Amérique était une colonie formée par le concours volontaire d’émigrants cherchant la liberté ; elles ne pouvaient réussir en France, parce que la France avait été construite en vertu d’un tout autre principe. Une dynastie nouvelle faillit sortir de la convulsion terrible qui agitait la France ; mais on vit alors combien il est difficile aux nations modernes de se créer d’autres maisons souveraines que celles qui sont sorties de la conquête germanique. Le génie extraordinaire qui avait élevé Napoléon sur le pavois l’en précipita, et la vieille dynastie revint, en apparence décidée à tenter l’expérience de monarchie constitutionnelle qui avait si tristement échoué entre les mains du pauvre Louis XVI.\par
Il était écrit que, dans cette grande et tragique histoire de France, le roi et la nation rivaliseraient d’imprudence. Cette fois, les fautes de la royauté furent les plus graves. Les ordonnances de juillet 1830 peuvent vraiment être qualifiées de crime politique ; on ne les tira de l’article 14 de la Charte que par un sophisme évident. Cet article 14 n’avait nullement dans la pensée de Louis XVIII le sens que lui prêtèrent les ministres de Charles X. Il n’est pas admissible que l’auteur de la Charte eût mis dans la Charte un article qui en renversait toute l’économie. C’était le cas d’appliquer l’axiome : Contra {\itshape eum qui dicere potuit clarius prœsumptio est facienda. Si} avant M. de Polignac quelqu’un eût pu penser que cet article donnait au roi le droit de supprimer la Charte, c’eût été l’objet d’une perpétuelle protestation ; or personne ne protesta ; car personne ne pensa jamais que cet insignifiant article contînt le droit implicite des coups d’État. L’insertion de cet article ne vint pas de la royauté, qui s’y serait réservé un moyen d’éluder ses engagements ; il faisait partie du projet de constitution élaboré par les chambres de 1814, fort attentives à ne pas exagérer les droits du roi ; il ne donna lieu alors à aucune observation ; « on n’y voyait qu’une sorte de lieu commun emprunté aux constitutions antérieures, et personne n’y soupçonnait le sens redoutable et mystérieux qu’on a voulu depuis y attacher \footnote{ M. de Viel-Castel, Hist. de la {\itshape Restauration}, t. I, p. 429.}. » \par
Les députes de 1830 eurent donc raison de résister aux ordonnances, et les citoyens qui étaient à portée d’entendre leur appel firent bien de s’armer. La situation était celle du roi d’Angleterre, qui plus d’une fois s’est trouve en lutte avec son parlement. Mais, dès que le roi, vaincu, eut retire les ordonnances, il fallait s’arrêter et maintenir le roi dans son palais. Il lui convint d’abdiquer ; il fallait prendre celui en faveur de qui il abdiquait. On fit autrement. Hâtons-nous de dire que dix-huit années d’un règne plein de sagesse justifièrent à beaucoup d’égards le choix du 10 août 1830, et que ce choix pouvait s’autoriser de quelques-uns des précédents de la révolution de 1688 en Angleterre ; mais, pour qu’une substitution aussi hardie devînt légitime, il fallait qu’elle durât. Par une série d’impardonnables étourderies de la part de la nation et par suite d’une regrettable faiblesse de la dynastie nouvelle, cette consécration manqua. Le roi et ses fils, au lieu de maintenir leur droit par les armes, se retirèrent et laissèrent l’émeute parisienne violer outrageusement la volonté de la nation. Déchirure funeste faite a un titre un peu caduc en son origine et qui ne pouvait acquérir de force que par sa persistance. Une dynastie doit à la nation, qui toujours est censée l’appuyer, de résister à une minorité turbulente. L’humanité est satisfaite, pourvu qu’après la bataille le pouvoir vainqueur se montre généreux et traite les rebelles, non comme des coupables, mais comme des vaincus.\par
Nous entrions pour la plupart dans la vie publique, quand survint le néfaste incident du 24 février. Avec un instinct parfaitement juste, nous sentîmes que ce qui se passa ce jour-là était un grand malheur. Libéraux par principes philosophiques, nous vîmes bien que les arbres de la liberté qu’on plantait avec une joie si naïve ne verdiraient jamais ; nous comprimes que les problèmes sociaux qui se posaient d’une façon audacieuse étaient destines à jouer un rôle de premier ordre dans l’avenir du monde. Le baptême de sang des journées de juin, les réactions qui suivirent nous serrèrent le cœur ; il était clair que l’âme et l’esprit de la France couraient un véritable péril. La légèreté des hommes de : 1848 fut vraiment sans pareille. Ils donnèrent à la France, qui ne le demandait pas, le suffrage universel. Ils ne songèrent pas que ce suffrage ne bénéficierait qu’à cinq millions de paysans, étrangers à toute idée libérale. Je voyais assidûment à cette époque M. Cousin. Dans les longues promenades que ce profond connaisseur de toutes les gloires françaises me faisait faire dans les rues de Paris de la rive gauche, m’expliquant l’histoire de chaque maison et de ses propriétaires au \textsc{xvii}\textsuperscript{e} siècle, il me disait souvent ce mot : « Mon ami, on ne comprend pas encore quel crime a été la révolution de février ; le dernier terme de cette révolution sera peut-être le démembrement de la France. » \par
Le coup d’état du 2 décembre nous froissa profondément. Dix ans nous portâmes le deuil du droit ; nous protestâmes selon nos forces contre le système d’abaissement intellectuel savamment dirigé par, M. Fortoul, à peine mitigé par ceux qui lui succédèrent. Il arriva cependant ce qui arrive toujours. Le pouvoir inauguré par la violence s’améliorait en vieillissant ; il se prit à voir que le développement libéral de l’homme est un intérêt majeur pour tout gouvernement. le pays, d’un autre côte, était enchanté de ce gouvernement médiocre. Il avait ce qu’il voulait ; chercher à renverser, un tel gouvernement malgré le vœu évident du plus grand nombre eût été insensé. Ce qu’il y avait de plus sage était de tirer, du mal le meilleur parti possible, de faire comme les évêques du \textsc{v}\textsuperscript{e} siècle et du \textsc{vi}\textsuperscript{e} siècle, qui, ne pouvant repousser les barbares, cherchaient à les éclairer. Nous consentîmes donc à servir le gouvernement de l’empereur Napoléon III dans ce qu’il avait de bon, c’est-à-dire en tant qu’il touchait aux intérêts éternels de la science, de l’éducation publique, du progrès des lumières, à ces devoirs sociaux enfin qui ne chôment jamais.\par
Il est incontestable, d’ailleurs, que le règne de l’empereur Napoléon III, malgré ses immenses lacunes, avait résolu une moitié du problème. La majorité de la France était parfaitement contente. Elle avait ce qu’elle voulait, l’ordre et la paix. La liberté manquait, il est vrai ; la vie politique était des plus faibles ; mais cela ne blessait qu’une minorité d’un cinquième ou d’un sixième de la nation, et encore dans cette minorité faut-il distinguer un petit nombre d’hommes instruits, intelligents, vraiment libéraux, d’une foule peu réfléchie, animée de cet esprit séditieux qui a pour unique programme d’être toujours en opposition avec le gouvernement et de chercher à le renverser. L’administration était très mauvaise ; mais quiconque ne niait pas le principe des droits de la dynastie souffrait peu. Les hommes d’opposition eux-mêmes étaient plutôt gênes dans leur activité que persécutés. La fortune du pays s’augmentait dans des proportions inouïes. À la date du 8 mai 1870, après de très graves fautes commises, sept millions et demi d’électeurs se déclarèrent encore satisfaits. Il ne venait à l’esprit de presque personne qu’un tel état pût être expose à la plus effroyable des catastrophes. Cette catastrophe, en effet, ne sortit pas d’une nécessité générale de situation ; elle vint d’un trait particulier du caractère de l’empereur Napoléon III.
\subsection[{II}]{II}
\noindent L’empereur Napoléon III avait fondé sa fortune en répondant au besoin de réaction, d’ordre, de repos qui fut la conséquence de la révolution de 1848. Si l’empereur Napoléon III se fût renfermé dans ce programme, s’il se fût contenté de comprimer à l’intérieur toute idée, toute liberté politique, de développer les intérêts matériels, de s’appuyer sur un cléricalisme modéré et sans conviction, son règne et celui de sa dynastie eussent été assurés pour longtemps. Le pays s’enfonçait de plus en plus dans la vulgarité, oubliait sa vieille histoire ; la nouvelle dynastie était fondée. La France telle que l’a faite le suffrage universel est devenue profondément matérialiste ; les nobles soucis de la France d’autrefois, le patriotisme, l’enthousiasme du beau, l’amour de la gloire, ont disparu avec les classes nobles qui représentaient l’âme de la France. Le jugement et le gouvernement des choses ont été transportes à la masse ; or la masse est lourde, grossière, dominée par la vue la plus superficielle de l’intérêt. Ses deux pôles sont l’ouvrier et le paysan. L’ouvrier n’est pas éclairé ; le paysan veut avant tout acheter de la terre, arrondir son champ. Parlez au paysan, au socialiste de l’Internationale, de la France, de son passé, de son génie, il ne comprendra pas un tel langage. L’honneur militaire, de ce point de vue borné, paraît une folie ; le goût des grandes choses, la gloire de l’esprit sont des chimères ; l’argent dépensé pour l’art et la science est de l’argent perdu, dépense follement, pris dans la poche de gens qui se soucient aussi peu que possible d’art et de science. Voilà l’esprit provincial que l’empereur servit merveilleusement dans les premières années de son règne. S’il était reste le docile et aveugle serviteur de cette réaction mesquine, aucune opposition n’aurait réussi à l’ébranler. Toutes les oppositions réunies eussent trouvé leur limite en deux millions de voix tout au plus. Le chiffre des opposants augmentait chaque année ; d’où quelques personnes concluaient qu’il grandirait jusqu’à devenir majorité. Erreur ; ce chiffre eût rencontre un point d’arrêt qu’il n’eût pas dépassé. Disons-le, puisque nous avons la certitude que ces lignes ne seront lues que par des personnes intelligentes : un gouvernement qui aura pour unique désir de s’établir en France et de s’y éterniser aura désormais, je le crains, une voie bien simple à suivre : imiter le programme de Napoléon III, moins la guerre. De la sorte il amènera la France au degré d’abaissement où arrive toute société qui renonce aux hautes visées ; mais il ne mourra qu’avec le pays, de la mort lente de ceux qui s’abandonnent au courant de la destinée, sans jamais le contrarier.\par
Tel n’était pas l’empereur Napoléon III. Il était supérieur en un sens à la majorité du Pays ; il aimait le bien ; il avait un goût réel peu éclairé sans doute, cependant, de la noble culture de l’humanité. À plusieurs égards, il était en totale dissonance avec ceux qui l’avaient nommé. Il rêvait la gloire militaire ; le fantôme de Napoléon 1\textsuperscript{er} le hantait. Cela est d’autant plus étrange que l’empereur Napoléon III voyait fort bien qu’il n’avait ni aptitudes, ni pratique pour la guerre, et qu’il savait que la France avait perdu à cet égard toutes ses qualités. Mais l’idée innée l’emportait. L’empereur sentait si bien que ses vues personnelles à cet égard étaient une sorte de {\itshape nœvus} qu’il fallait cacher, que toujours, à l’époque de la fondation de son pouvoir, nous le voyons occupé à protester qu’il veut la paix. Il reconnaissait que c’était là le moyen de se rendre populaire. La guerre de Crimée ne fut acceptée dans l’opinion que parce qu’on la crut sans conséquence pour la paix générale. La guerre d’Italie ne fut pardonnée que quand on la vit tourner coud et rester à mi-chemin.\par
Le plus simple bon sens commandait à l’empereur Napoléon III de ne jamais faire la guerre. La France, il le savait, ne la désirait en aucune sorte \footnote{ Enquête des préfets. {\itshape Journal des Débats}, 3 et 4 Octobre 1870.}. En outre, un pays travaillé par les révolutions, qui a des divisions dynastiques, n’est pas capable d’un grand effort militaire. Le roi jean, Charles VII, François 1\textsuperscript{er} et même Louis XIV traversèrent des situations aussi critiques que celle de Napoléon III après la capitulation de Sedan ; ils ne furent pas pour cela renverses, ni même un moment ébranles. Le roi de Prusse Frédéric-Guillaume III, après la bataille d’Iéna, se trouva plus solide que jamais sur son trône ; mais Napoléon III ne pouvait supporter une défaite. Il était comme un joueur qui jouerait à la condition d’être fusillé s’il perd une partie. Un pays divisé sur les questions dynastiques doit renoncer a la guerre ; car, au premier échec, cette cause de faiblesse apparaît, et fait de tout accident un eu mortel. L’homme qui a une blessure mal cicatrisée peut se livrer aux actes de la vie ordinaire sans qu’on s’aperçoive de son infirmité ; mais tout exercice violent lui est interdit ; à la première fatigue sa blessure se rouvre, et il tombe. On ne conçoit pas que Napoléon III se soit fait une si complète illusion sur la solidité de l’édifice qu’il avait fait lui-même d’argile. Comment ne vit-il pas qu’un tel édifice ne résisterait pas à une secousse, et que le choc d’un ennemi puissant devait nécessairement le faire crouler ? \par
La guerre déclarée au mois de juillet 1870 est donc une aberration personnelle, l’explosion ou plutôt le retour offensif d’une idée depuis longtemps latente dans l’esprit de Napoléon III, idée que les goûts pacifiques du pays l’obligeaient de dissimuler, et à laquelle il semble qu’il avait lui-même presque renoncé. Il n’y a pas un exemple de plus complète trahison d’un État par son souverain, en prenant le mot trahison pour désigner l’acte du mandataire qui substitue sa volonté à celle du mandant. Est-ce à dire que le pays ne soit pas responsable de ce qui est arrive ? Hélas ! nous ne pouvons le soutenir. Le pays a été coupable de s’être donné un gouvernement peu éclairé et surtout une chambre misérable, qui, avec une légèreté dépassant toute imagination, vota sur la parole d’un ministre la plus funeste des guerres. Le crime de la France fut celui d’un homme riche qui choisit un mauvais gérant de sa fortune, et lui donne une procuration illimitée ; cet homme mérite d’être ruiné ; mais on n’est pas juste si l’on prétend qu’il a fait lui-même les actes que son fondé de pouvoirs a faits sans lui et malgré lui.\par
Quiconque connaît la France, en effet, dans son ensemble et dans ses variétés provinciales, n’hésitera pas à reconnaître que le mouvement qui emporte ce pays depuis un demi-siècle est essentiellement pacifique. La génération militaire, froissée par les défaites de 1814 et 1815, avait à peu près disparu sous la Restauration et sous le règne de Louis-Philippe. Un patriote profondément honnête, mais souvent superficiel, raconta nos anciennes victoires d’un ton de triomphe qui souvent put blesser l’étranger ; mais cette dissonance allait s’affaiblissant chaque jour. On peut dire qu’elle avait cessé depuis 1848. Deux mouvements commencèrent alors, qui devaient être la fin non-seulement de tout esprit guerrier, mais de tout patriotisme : je veux parler de l’éveil extraordinaire des appétits matériels chez les ouvriers et chez les paysans. Il est clair que le socialisme des ouvriers est l’antipode de l’esprit militaire c’est presque la négation de la patrie les doctrines de l’Internationale sont là pour le prouver. Le paysan, d’un autre côté, depuis qu’on lui a ouvert la voie de la richesse et qu’on lui a montré que son industrie est la plus sûrement lucrative, le paysan a senti redoubler son horreur pour la conscription. Je parle par expérience. Je fis la campagne électorale de mai 1869 dans une circonscription toute rurale de Seine-et-Marne ; je puis assurer que je ne trouvai pas sur mon chemin un seul élément de l’ancienne vie militaire du pays. Un gouvernement à bon marché, peu imposant, peu gênant, un honnête désir de liberté, une grande soif d’égalité, une totale indifférence à la gloire du pays, la volonté arrêtée de ne faire aucun sacrifice à des intérêts non palpables, voilà ce qui me parut l’esprit du paysan dans la partie de la France où le paysan est, comme on dit, le plus avance.\par
Je ne veux pas dire qu’il ne restât plus de traces du vieil esprit qui se nourrit des souvenirs du premier empire. Le parti très peu nombreux qu’on peut appeler bonapartiste, au sens propre, entourait l’empereur de déplorables excitations. Le parti catholique, par ses lieux communs erronés sur la prétendue décadence des nations protestantes, cherchait aussi à rallumer un feu presque éteint. Mais cela ne touchait nullement le pays. L’expérience de 1870 l’a bien montré ; l’annonce de la guerre fut accueillie avec consternation ; les sottes rodomontades des journaux, les criailleries des gamins sur le boulevard sont des faits dont l’histoire n’aura de compte à tenir que pour montrer à quel point une bande d’étourdis peut donner le change sur les vrais sentiments d’un pays. La guerre prouva jusqu’à l’évidence que nous n’avions plus nos anciennes facultés militaires. Il n’y a rien là qui doive étonner celui qui s’est fait une idée juste de la philosophie de notre histoire. La France du moyen âge est une construction germanique, élevée par une aristocratie militaire germanique avec des matériaux gallo-romains. Le travail séculaire, de la France a consisté à expulser de son sein tous les éléments déposés par l’invasion germanique, jusqu’à la Révolution, qui a été la dernière convulsion de cet effort. L’esprit militaire de la France venait de ce quelle avait de germanique ; en chassant violemment les éléments germaniques et en les remplaçant par une conception philosophique et égalitaire de la société, la France a rejeté du même coup tout ce qu’il y avait en elle d’esprit militaire. Elle est restée un pays riche, considérant la\par
comme une sotte carrière, très peu rémunératrice. La France est ainsi devenue le pays le plus pacifique du monde ; toute son activité s’est tournée vers les problèmes sociaux, vers l’acquisition de la richesse et les progrès de l’industrie. Les classes éclairées n’ont pas laissé dépérir le goût de l’art, de la science, de la littérature, d’un luxe élégant ; mais la carrière militaire a été abandonnée. Peu de familles de la bourgeoisie aisée, ayant à choisir un état pour leur fils, ont préféré aux riches perspectives du commerce et de l’industrie une profession dont elles ne comprennent pas l’importance sociale. L’école de Saint-Cyr n’a guère eu que le rebut de la jeunesse, jusqu’à ce que l’ancienne noblesse et le parti catholique aient commencé à la peupler, changement dont les conséquences n’ont pas encore eu le temps de se développer. Cette nation a été autrefois brillante et guerrière ; mais elle l’a été par sélection, si j’ose le dire. Elle entretenait et produisait une noblesse admirable, pleine de bravoure et d’éclat. Cette noblesse une fois tombée, il est reste un fond indistinct de médiocrité, sans originalité ni hardiesse, une roture ne comprenant ni le privilège de l’esprit ni celui de l’épée. Une nation ainsi faite peut arriver au comble de la prospérité matérielle ; elle n’a plus de rôle dans le monde, plus d’action à l’étranger. D’autre part, il est impossible de sortir d’un pareil état avec le suffrage universel. Car on ne dompte pas le suffrage universel avec lui-même ; on le trompe, on l’endort ; mais, tant qu’il règne, il oblige ceux qui relèvent de lui de pactiser avec lui et de subir sa loi. Il y a cercle vicieux à rêver qu’on peut réformer les erreurs d’une opinion inconvertissable en prenant son seul point d’appui dans l’opinion.\par
La France n’a fait, du reste, que suivre en cela le mouvement général de toutes les nations de l’Europe, la Prusse et la Russie exceptées. M. Cobden, que je vis vers 1857, était enchanté de nous. L’Angleterre nous avait devancés dans cette voie du matérialisme industriel et commercial ; seulement, bien plus sages que nous, les Anglais surent faire marcher leur gouvernement d’accord avec la nation, tandis que notre maladresse a été telle, que le gouvernement de notre choix a pu nous engager malgré nous dans la guerre. Je ne sais si je me trompe ; mais il y a une vue d’ethnographie historique qui s’impose de plus en plus à mon esprit. La similitude de l’Angleterre et de la France du Nord m’apparaît chaque jour davantage. Notre étourderie vient du Midi, et, si la {\itshape France} n’avait pas entraîné le Languedoc et la Provence dans son cercle d’activité, nous serions sérieux, actifs, protestants, parlementaires. Notre fond de race est le même que celui des Iles Britanniques ; l’action germanique, bien qu’elle ait été assez forte dans ces îles pour faire dominer un idiome germanique, n’a pas, en somme, été plus considérable sur l’ensemble des trois royaumes que sur l’ensemble de la France. Comme la France, l’Angleterre me paraît en train d’expulser son élément germanique, cette noblesse obstinée, fière, intraitable, qui la gouvernait du temps de Pitt, de Castlereagh, de Wellington. Que cette pacifique et toute chrétienne école d’économistes est loin de la passion des hommes de fer qui imposèrent à leur pays de si grandes choses ! L’opinion publique de l’Angleterre, telle qu’elle se produit depuis trente ans n’est nullement germanique ; on y sent l’esprit celtique, plus doux, plus sympathique, plus humain. Ces sortes d’aperçus doivent être pris d’une façon très large ; on peut dire cependant que ce qui reste encore d’esprit militaire dans le monde est un fait germanique. C’est probablement par la race germanique, en tant que féodale et militaire, que le socialisme et la démocratie égalitaire, qui chez nous autres Celtes ne trouveraient pas facilement leur limite, arriveront à être domptes, et cela sera conforme aux précédents historiques ; car un des traits de la race germanique a toujours été de faire marcher de pair l’idée de conquête et l’idée de garantie. ; en d’autres termes, de faire dominer le fait matériel et brutal de la propriété résultant de la conquête sur toutes les considérations des droits de l’homme et sur les théories abstraites de contrat social. La réponse à chaque progrès du socialisme pourra être de la sorte un progrès du germanisme, et on entrevoit le jour où tous les pays de socialisme seront gouvernés par des Allemands. L’invasion du IVe et du Ve siècle se fit par des raisons analogues, les pays romains étant devenus incapables de produire de bons gendarmes, de bons mainteneurs de propriété.\par
En réalité notre pays, surtout la province, allait vers une forme sociale qui, malgré la diversité des apparences, avait plus d’une analogie avec l’Amérique, vers une forme sociale où beaucoup de choses tenues autrefois pour choses d’État seraient laissées à l’initiative privée. Certes, on pouvait n’être pas le partisan d’un tel avenir ; il était clair que la France en se développant dans ce sens resterait fort au-dessous de l’Amérique. À son manque d’éducation, de distinction, à ce vide que laisse toujours dans un pays l’absence de cour, de haute société, d’anciennes institutions, l’Amérique supplée par le feu de sa jeune croissance, par son patriotisme, par la confiance exagérée peut-être qu’elle a dans sa force, par la persuasion qu’elle travaille à la grande œuvre de l’humanité, par l’efficacité de ses convictions protestantes, par sa hardiesse et son esprit d’entreprise, par l’absence presque totale de germes socialistes, par la facilité avec laquelle la différence du riche et du pauvre y est acceptée, par le privilège surtout qu’elle a de se développer à l’air libre, dans l’infini de l’espace et sans voisins. Privée de ces avantages, faisant son expérience, pour ainsi dire, en vase clos, à la fois trop pesante et trop légère, trop crédule et trop railleuse, la France n ‘aurait jamais été qu’une Amérique de second ordre, mesquine, médiocre, peut-être plus semblable au Mexique ou à l’Amérique du Sud qu’aux États-Unis. La royauté conserve dans nos vieilles sociétés une foule de choses bonnes à garder ; avec l’idée que j’ai de la vieille France et de son génie, j’appellerais cet adieu à la gloire et aux grandes choses : {\itshape Finis Franciæ.} Mais, en politique, il faut se garder de prendre ses sympathies pour ce qui doit être ; ce qui réussit en ce monde est d’ordinaire le rebours de nos instincts, à nous autres idéalistes, et presque toujours nous sommes autorisés à conclure, de ce qu’une chose nous déplaît, qu’elle sera. Ce désir d’un état politique impliquant le moins possible de gouvernement central est le vœu universel de la province. L’antipathie qu’elle témoigne contre Paris n’est pas seulement la juste indignation contre les attentats d’une minorité factieuse ; ce n’est pas seulement le Paris révolutionnaire, c’est le Paris gouvernant que la France n’aime pas. Paris est pour la France synonyme d’exigences gênantes. C’est Paris qui lève les hommes, qui absorbe l’argent, qui l’emploie à une foule de fins que la province ne comprend pas. Le plus capable des administrateurs du dernier règne me disait, à propos des élections de 1869, que ce qui lui paraissait le plus compromis en France était le système de l’impôt la province à chaque élection forçant ses élus à prendre des engagements, qu’il faudrait bien tenir tôt ou tard dans une certaine mesure et dont l’accomplissement serait la destruction des finances de l’État. La première fois que je rencontrai Prevost-Paradol, au retour de sa campagne électorale dans la Loire-Inférieure, je lui demandai son impression dominante : « Nous verrons bientôt la fin de l’État », me dit-il. C’est exactement ce que j’aurais répondu, s’il m’avait demandé mes impressions de Seine-et-Marne. Que le préfet se mêle d’aussi peu de choses que possible, que l’impôt et le service militaire soient aussi réduits que possible, et la province sera satisfaite. La plupart des gens n’y demandent guère qu’une seule chose, c’est qu’on les laisse tranquillement faire fortune. Seuls, les pays pauvres montrent encore de l’avidité pour les places ; dans les départements riches, les fonctions ne sont pas considérées et sont tenues pour un des emplois les moins avantageux qu’on ait à faire de son activité.\par
Tel est l’esprit de ce qu’on peut appeler la démocratie provinciale. Un pareil esprit, on le voit, diffère sensiblement de l’esprit républicain ; il peut s’accommoder de l’empire et de la royauté constitutionnelle aussi bien que de la république, et même mieux à quelques égards. Aussi indifférent à telle ou telle dynastie qu’à tout ce qui peut s’appeler gloire ou éclat, il préfère au fond avoir une dynastie, comme garantie d’ordre ; mais il ne veut faire aucun sacrifice à l’établissement de cette dynastie. C’est le pur matérialisme politique, l’antipode de la part d’idéalisme qui est l’âme des théories légitimistes et républicaines. Un tel parti, qui est celui de l’immense majorité des Français, est trop superficiel, trop borné pour pouvoir conduire les destinées d’un pays. L’énorme sottise qu’il fit à son point de vue quand il prit en 1848 le prince Louis-Napoléon pour gérant de ses affaires, il la renouvellera vingt fois. Son sort est d’être dupe sans fin, car il est défendu à l’homme bassement intéressé d’être habile ; la simple platitude bourgeoise ne peut susciter la quantité de dévouement nécessaire pour créer un ordre de choses et pour le maintenir.\par
Il y a du vrai, en effet, dans le principe germanique qu’une société n’a un droit plein à son patrimoine que tandis qu’elle peut le garantir. Dans un sens général, il n’est pas bon que celui qui possède soit incapable de défendre ce qu’il possède. Le duel des chevaliers du moyen âge, la menace de l’homme armé venant présenter la bataille au propriétaire qui s’endort dans la mollesse, était à quelques égards légitime. Le droit du brave a fondé la propriété ; l’homme d’épée est bien le créateur de toute richesse, puisqu’en défendant ce qu’il a conquis il assure le bien des personnes qui sont groupées sous sa protection. Disons au moins qu’un état comme celui qu’avait rêvé la bourgeoisie française, état où celui qui possédait et jouissait ne tenait pas réellement l’épée (par suite de la loi sur le remplacement ) pour défendre sa propriété, constituait un véritable {\itshape porte à faux} d’architecture sociale. Une classe possédante qui vit dans une oisiveté relative, qui rend peu de services publics, et qui se montre néanmoins arrogante, comme si elle avait un droit de naissance à posséder et comme si les autres avaient par naissance le devoir de la défendre, une telle classe, dis-je, ne possédera pas longtemps. Notre société devient trop exclusivement une association de faibles ; une telle société se défend mal ; il lui est difficile de réaliser ce qui est le grand {\itshape criterium} du droit et de la volonté qu’a une réunion d’hommes de vivre ensemble et de se garantir mutuellement, je veux dire une puissante force armée. L’auteur de la richesse est aussi bien celui qui la garantit par ses armes que celui qui la crée par son travail. L’économie politique, uniquement préoccupée de la création de la richesse par le travail, n’a jamais compris la féodalité, laquelle était au fond tout aussi légitime que la constitution de l’armée moderne. Les ducs, les marquis, les comtes, étaient au fond les généraux, les colonels, les commandants d’une {\itshape Landwehr}, dont les appointements consistaient en terres et en droits seigneuriaux.
\subsection[{III}]{III}
\noindent Ainsi la tradition d’une politique nationale se perdait de jour en jour. Le principe du goût que la majorité des Français a pour la monarchie étant essentiellement matérialiste, et aussi éloigné que possible de ce qui peut s’appeler fidélité, loyalisme, amour de ses princes, la France, tout en voulant une dynastie, se montre très coulante sur le choix de la dynastie elle-même. Le règne éphémère mais brillant de Napoléon I\textsuperscript{er} avait suffi pour créer un titre auprès de ce peuple, étranger, à toute idée de légitimité séculaire. Le prince Louis-Napoléon se présentant en 1848 comme héritier de ce titre, et paraissant fait exprès pour tirer la France d’un état qui lui est antipathique et dont elle s’exagérait les dangers, la France le saisit comme une bouée de sauvetage, l’aida dans ses entreprises les plus téméraires, se fit complice de ses coups d’État. Pendant près de vingt ans, les fauteurs du 10 décembre purent croire qu’ils avaient eu raison. La France développa prodigieusement ses ressources intérieures. Ce fut une vraie révélation. Grâce à l’ordre, à la paix, aux traités de commerce, Napoléon III apprit à la France sa propre richesse. L’abaissement politique intérieur mécontentait une fraction intelligente ; le reste avait trouve ce qu’il voulait, et il n’est pas douteux que le règne de Napoléon III restera pour certaines classes de la nation un véritable idéal. Je le répète, si Napoléon III eût voulu ne pas faire la guerre, la dynastie des Bonapartes était fondée pour des siècles. Mais telle est la faiblesse d’un État dénué de base morale, qu’un jour de folie suffit pour tout perdre. Comment l’empereur ne vit-il pas que la guerre avec l’Allemagne était une épreuve trop forte pour un pays aussi affaibli que la France ? Un entourage ignorant et sans sérieux, conséquence du péché d’origine de la monarchie nouvelle, une cour où il n’y avait qu’un seul homme intelligent (ce prince plein d’esprit et connaissant merveilleusement son siècle, que la fatalité de sa destinée laissa presque sans autorité ), rendaient possibles toutes les surprises, tous les malheurs.\par
Pendant que la fortune publique, en effet, prenait des accroissements inouïs, pendant que le paysan acquérait par ses économies des richesses qui n’élevaient en rien son état intellectuel, sa civilité, sa culture, l’abaissement de toute aristocratie se produisait en d’effrayantes proportions ; la moyenne intellectuelle du public descendait étrangement. Le nombre et la valeur des hommes distingués qui sortaient de la nation se maintenaient, augmentaient peut-être ; dans plus d’un genre de mérite, les nouveaux venus né le cédaient à aucun des noms illustres des générations écloses sous un meilleur soleil ; mais l’atmosphère s’appauvrissait ; on mourait de froid. L’Université, déjà faible, peu éclairée, était systématiquement affaiblie ; les deux seuls bons enseignements qu’elle possédât, celui de l’histoire et celui de la philosophie, furent à peu près supprimés. L’École polytechnique, l’École normale étaient découronnées. Quelques efforts d’amélioration qui se firent à partir de 1860 restèrent incohérents et sans suite. Les hommes de bonne volonté qui s’y compromirent ne furent pas soutenus. Les exigences cléricales auxquelles on se soumettait ne laissaient passer qu’une inoffensive médiocrité ; tout ce qui était un peu original se voyait condamné à une sorte de bannissement dans son propre pays. Le catholicisme restait la seule force organisée en dehors de l’État et confisquait à son profit l’action extérieure de la France. Paris était envahi par l’étranger viveur, par les provinciaux, qui n’y encourageaient qu’une petite presse ridicule et la sotte littérature, aussi peu parisienne que possible, du nouveau genre bouffon. Le pays, en attendant, s’enfonçait dans un matérialisme hideux. N’ayant pas de noblesse pour lui donner l’exemple, le paysan enrichi, content de sa lourde et triviale aisance, ne savait pas vivre, restait gauche, sans idées. {\itshape Oves non habentes pastorem}, telle était la France : un feu sans flamme ni lumière ; un cœur sans chaleur ; un peuple sans prophètes sachant dire ce qu’il sent ; une planète morte, parcourant son orbite d’un mouvement machinal.\par
La corruption administrative n’était pas le vol organisé, comme cela s’est vu à Naples, en Espagne ; c’était l’incurie, la paresse, un laisser-aller universel, une complète indifférence pour la chose publique. Toute fonction était devenue une sinécure, un droit à une rente pour ne rien faire. Avec cela, tout le monde était inattaquable. Grâce à la loi sur la diffamation qui a l’air d’avoir été faite pour protéger les moins honorables des citoyens, grâce surtout à l’universel discrédit où la presse tomba par sa vénalité, une prime énorme était assurée à la médiocrité et à la malhonnêteté. Celui qui hasardait quelque critique devenait vite un être à part et bientôt un homme dangereux. On ne le persécutait pas ; cela était bien inutile. Tout se perdait dans une mollesse générale, dans un manque complet d’attention et de précision. Quelques hommes d’esprit et de cœur, qui donnaient d’utiles conseils, étaient impuissants. L’impertinence vaniteuse de l’administration officielle, persuadée que l’Europe l’admirait et l’enviait, rendait toute observation inutile et toute réforme impossible.\par
L’opposition était-elle plus éclairée que le gouvernement ? à peine. Les orateurs de l’opposition se montraient, en ce qui concerne les affaires allemandes, plus étourdis encore que M. Rouher. En somme, l’opposition ne représentait nullement un principe supérieur de moralité. Étrangère à toute idée de politique savante, elle ne sortait pas de l’ornière du superficiel radicalisme français. À part quelques hommes de valeur, qu’on s’étonne de voir issus d’une source aussi trouble que le suffrage parisien, le reste n’était que déclamation, parti pris démocratique. La province valait mieux à quelques égards. Des besoins d’une vie locale régulière, d’une sérieuse décentralisation au profit de la commune, du canton, du département, le désir impérieux d’élections libres, la volonté arrêtée de réduire le gouvernement au strict nécessaire, de diminuer considérablement l’armée, de supprimer les sinécures, d’abolir l’aristocratie des fonctionnaires, constituaient un programme assez libéral, quoique mesquin, puisque le fond de ce programme était de payer le moins possible, de renoncer à tout ce qui peut s’appeler gloire, force, éclat. De ces vœux accomplis, fût résulté avec le temps une petite vie provinciale, matériellement très florissante, indifférente à l’instruction et à la culture intellectuelle, assez libre ; une vie de bourgeois aisés, indépendants les uns des autres, sans souci de la science, de l’art, de la gloire, du génie ; une vie, je le répète, assez semblable à la vie américaine, sauf la différence des mœurs et du tempérament.\par
Tel était l’avenir de la France, si Napoléon III n’eût volontairement couru à sa ruine. On allait à pleines voiles vers la médiocrité. D’une part, les progrès de la prospérité matérielle absorbaient la bourgeoisie ; de l’autre, les questions sociales étouffaient complètement les questions nationales et patriotiques. Ces deux ordres de questions se font en quelque sorte équilibre ; l’avènement des unes signale l’éclipse des autres. La grande amélioration qui s’était faite dans la situation de l’ouvrier était loin d’être favorable à son amélioration morale. Le peuple est bien moins capable que les classes élevées ou éclairées de résister à la séduction des plaisirs faciles, qui ne sont sans inconvénients que quand on est blasé sur leur compte. Pour que le bien-être ne démoralise pas, il faut y être habitué ; l’homme sans éducation s’abîme vite dans le plaisir, le prend lourdement au sérieux, ne s’en dégoûte pas. La moralité supérieure du peuple allemand vient de ce qu’il a été jusqu’à nos jours très maltraité. Les politiques qui soutiennent qu’il faut que le peuple souffre pour qu’il soit bon n’ont malheureusement pas tout à fait tort.\par
Le dirai-je ? notre philosophie politique concourait au même résultat. Le premier principe de notre morale, c’est de supprimer le tempérament, de faire dominer le plus possible la raison sur l’animalité ; or c’est là l’inverse de l’esprit guerrier. Quelle pouvait être notre règle de conduite, à nous autres libéraux, qui ne pouvons pas admettre le droit divin en politique, quand nous n’admettons pas le surnaturel en religion ? Un simple droit humain, un compromis entre le rationalisme absolu de Condorcet et du \textsc{xviii}\textsuperscript{e} siècle, ne reconnaissant que le droit de la raison à gouverner l’humanité, et les droits résultant de l’histoire, L’expérience manquée de la Révolution nom a guéris du culte de la raison ; mais, en y mettant toute la bonne volonté possible, nous n’avons pu en venir au culte de la force ou du droit fondé sur la force, qui est le résumé de la politique allemande. Le consentement des diverses parties d’un État nous paraît l’{\itshape ultima ratio} de l’existence de cet État. — Tels étaient nos principes, et ils avaient deux défauts essentiels : le premier, c’est qu’il se trouvait au monde des gens qui en avaient de tout autres, qui vivaient des dures doctrines de l’ancien régime, lequel faisait consister l’unité de la nation dans les droits du souverain, tandis que nous nous imaginions que le \textsc{xix}\textsuperscript{e} siècle avait inauguré un droit nouveau, le droit des populations ; le second défaut, c’est que ces principes, nous ne réussimes pas toujours à les faire prévaloir chez nous. Les principes que je disais tout à l’heure sont bien des principes français, en ce sens qu’ils sortent logiquement de notre philosophie, de notre révolution, de notre caractère national avec ses qualités et ses défauts. Malheureusement, le parti qui les professe n’est, comme tous les partis intelligents, qu’une minorité, et cette minorité a été trop souvent vaincue chez nous. L’expédition de Rome a été la plus évidente dérogation à la seule politique qui pouvait nous convenir. La tentative de nous immiscer dans les affaires allemandes a été une flagrante inconséquence, et celle-ci ne doit pas être mise uniquement à la charge du gouvernement déchu ; l’opposition n’avait cessé d’y pousser depuis Sadowa. Ceux qui ont toujours repoussé la politique de conquête ont le droit de dire : « Prendre l’Alsace malgré elle est un crime ; la céder autrement que devant une nécessité absolue serait un crime aussi. » Mais ceux qui ont prêche la doctrine des frontières naturelles et des convenances nationales n’ont pas le droit de trouver mauvais qu’on leur fasse ce qu’ils voulaient faire aux autres. La doctrine des frontières naturelles et celle du droit des populations ne peuvent être invoquées par la même bouche, sous peine d’une évidente contradiction.\par
Ainsi nous nous sommes trouvés faibles, désavoués par notre propre pays. La France pouvait se désintéresser de toute action extérieure comme le fit sagement Louis-Philippe. Des qu’elle agissait à l’étranger, elle ne pouvait servir que son propre principe, le principe des nations libres, composées de provinces libres, maîtresses de leurs destinées. C’est de ce point de vue que nous vîmes avec sympathie la guerre d’Italie de l’empereur Napoléon III, même à quelques égards la guerre de Crimée, et surtout l’aide qu’il donna à la formation d’une Allemagne du Nord autour de la Prusse. Nous crûmes un moment que notre rêve allait se réaliser, c’est-à-dire l’union politique et intellectuelle de l’Allemagne, de l’Angleterre et de la France, constituant à elles trois une force directrice de l’humanité et de la civilisation, faisant digne à la Russie, ou plutôt la dirigeant dans sa, voie et l’élevant. Hélas ! que faire avec un esprit étrange et inconsistant ? La guerre d’Italie eut pour contrepartie l’occupation prolongée de Rome, négation complète de tous les principes français ; la guerre de Crimée, qui n’eût été légitime que si elle avait abouti à émanciper les bonnes populations tenues dans la sujétion par la Turquie, n’eut pour résultat que de fortifier le principe ottoman ; l’expédition du Mexique fut un défi jeté à toute idée libérale. Les titres réels qu’on s’était acquis à la reconnaissance de l’Allemagne, on les perdit en prenant après Sadowa une attitude de mauvaise humeur et de provocation.\par
Il est injuste, disons-le encore, de rejeter toutes ces fautes sur le compte du dernier régime, et un des tours les plus dangereux que pourrait prendre l’amour-propre national serait de s’imaginer que nos malheurs n’ont eu pour cause que les fautes de Napoléon III, si bien que, Napoléon III une fois écarté, la victoire et le bonheur devraient nous revenir. La vérité est que toutes nos faiblesses eurent une racine plus profonde, une racine qui n’a nullement disparu, la démocratie mal entendue. Un pays démocratique ne peut être bien gouverné, bien administré, bien commandé. La raison en est simple. Le gouvernement, l’administration, le commandement sont dans une société le résultat d’une sélection qui tire de la masse un certain nombre d’individus qui gouvernent, administrent, commandent. Cette sélection peut se faire de quatre manières qui ont été appliquées tantôt isolement, tantôt concurremment dans diverses sociétés : 1º par la naissance ; 2º par le tirage au sort ; 3º par l’élection populaire ; 4º par les examens et les concours.\par
Le tirage au sort n’a guère été appliqué qu’à Athènes et à Florence, c’est-à-dire dans les deux seules villes où il y ait eu un peuple d’aristocrates, un peuple donnant par son histoire, au milieu des plus étranges écarts, le plus fin et le plus charmant spectacle. Il est clair que dans nos sociétés, qui ressemblent à de vastes Scythies, au milieu desquelles les cours, les grandes villes, les universités représentent des espèces de colonies grecques, un tel mode de sélection amènerait des résultats absurdes ; il n’est pas besoin de s’y arrêter.\par
Le système des examens et des concours n’a été appliqué en grand qu’en Chine. Il y a produit une sénilité générale et incurable. Nous avons été nous-mêmes assez loin dans ce sens, et ce n’est pas là une des moindres causes de notre abaissement.\par
Le système de l’élection ne peut être pris comme base unique d’un gouvernement. Appliquée au commandement militaire, en particulier, l’élection est une sorte de contradiction, la négation même du commandement, puisque, dans les choses militaires, le commandement est absolu ; or l’élu ne commande jamais absolument à son électeur. Appliquée au choix de la personne du souverain, l’élection encourage le charlatanisme, détruit d’avance le prestige de l’élu, l’oblige à s’humilier devant ceux qui doivent lui obéir. À plus forte raison ces objections s’appliquent-elles si le suffrage est universel. Appliqué au choix des députés, le suffrage universel n’amènera jamais, tant qu’il sera direct ’ que des choix médiocres. Il est impossible d’en faire sortir une chambre haute, une magistrature, ni même un bon conseil départemental on municipal. Essentiellement borné, le suffrage universel ne comprend pas la nécessité de la science, la supériorité du noble et du savant. Il ne peut être bon qu’à former un corps de notables, et encore à condition que l’élection se fasse dans une forme que nous spécifierons plus tard.\par
Il est incontestable que, s’il fallait s’en tenir à un moyen de sélection unique, la naissance vaudrait mieux que l’élection. Le hasard de la naissance est moindre que le hasard du scrutin. La naissance entraîne d’ordinaire des avantages d’éducation et quelquefois une certaine supériorité de race. Quand il s’agit de la désignation du souverain et des chefs militaires, le {\itshape criterium} de la naissance s’impose presque nécessairement. Ce {\itshape criterium}, après tout, ne blesse que le préjugé français, qui voit dans la fonction une rente à distribuer au fonctionnaire bien plus qu’un devoir public. Ce préjugé est l’inverse du vrai principe de gouvernement, lequel ordonne de ne considérer dans le choix du fonctionnaire que le bien de l’État ou, en d’autres termes, la bonne exécution de la fonction. Nul n’a droit à une place ; tous ont droit que les places soient bien remplies. Si l’hérédité de certaines fonctions était un gage de bonne gestion, je n’hésiterais pas à conseiller pour ces fonctions l’hérédité.\par
On comprend maintenant comment la sélection du commandement, qui, jusqu’à la fin du \textsc{xvii}\textsuperscript{e} siècle, s’est faite si remarquablement en France, est maintenant si abaissée, et a pu produire ce corps de gouvernants, de ministres, de députés, de sénateurs, de maréchaux, de généraux, d’administrateurs que nous avions au mois de juillet de l’année dernière, et qu’on peut regarder comme un des plus pauvres personnels d’hommes d’État que jamais pays ait vus en fonction. Tout cela venait du suffrage universel, puisque l’empereur, source de toute initiative, et le Corps législatif, seul contrepoids aux initiatives de l’empereur, en venaient. Ce misérable gouvernement était bien le résultat de la démocratie : la France l’avait voulu, l’avait tiré de ses entrailles. La France du suffrage universel n’en aura jamais de beaucoup meilleur. Il serait contre nature qu’une moyenne intellectuelle qui atteint à peine celle d’un homme ignorant et borné se fit représenter par un corps de gouvernement éclairé, brillant et fort. D’un tel procédé de sélection, d’une démocratie aussi mal entendue ne peut sortir qu’un complet obscurcissement de la conscience d’un pays. Le collège grand électeur formé par tout le monde est inférieur au plus médiocre souverain d’autrefois ; la cour de Versailles valait mieux pour les choix des fonctionnaires que le suffrage universel d’aujourd’hui ; ce suffrage produira un gouvernement inférieur à celui du \textsc{xviii}\textsuperscript{e} siècle à ses plus mauvais jours.\par
Un pays n’est pas la simple addition des individus qui le composent ; c’est une âme, une conscience, une personne, une résultante vivante. Cette âme peut résider en un fort petit nombre d’hommes ; il vaudrait mieux que tous pussent y participer ; mais ce qui est indispensable, c’est que, par la sélection gouvernementale, se forme une tête qui veille et pense pendant que le reste du pays ne pense pas et ne sent guère. Or la sélection française est la plus faible de toutes. Avec son suffrage universel non organisé, livré au hasard, la France ne peut avoir qu’une tête sociale sans intelligence ni savoir, sans prestige ni autorité. la France voulait la paix, et elle a si sottement choisi ses mandataires qu’elle a été jetée dans la guerre. La chambre d’un pays ultra-pacifique a voté d’enthousiasme la guerre la plus funeste. Quelques braillards de carrefour, quelques journalistes imprudents ont pu passer pour l’expression de l’opinion de la nation. Il y a en France autant de gens de cœur et de gens d’esprit que dans aucun autre pays ; mais tout cela n’est pas mis en valeur. Un pays qui n’a d’autre organe que le suffrage universel direct est dans son ensemble, quelle que soit la valeur des hommes qu’il possède, un être ignorant, sot, inhabile à trancher sagement une question quelconque. Les démocrates se montrent bien sévères pour l’ancien régime, qui amenait souvent au pouvoir des souverains incapables ou méchants. Sûrement les États qui font résider la conscience nationale dans une famille royale et son entourage ont des hauts et des bas ; mais prenons dans son ensemble la dynastie capétienne, qui a régné près de neuf cents ans ; pour quelques périodes de baisse — au \textsc{xiv}\textsuperscript{e}, au \textsc{xvi}\textsuperscript{e}, au \textsc{xviii}\textsuperscript{e} siècle, quelles admirables séries au \textsc{xii}\textsuperscript{e}, au \textsc{xiii}\textsuperscript{e}, au \textsc{xvii}\textsuperscript{e} siècle, de Louis le Jeune à Philippe le Bel, de Louis XIV à la deuxième moitié du règne de Louis XIV ! Il n’y a pas de système électif qui puisse donner une représentation comme celle-là. L’homme le plus médiocre est supérieur à la résultante collective qui sort de trente-six millions d’individus, comptant chacun pour une unité. Puisse l’avenir me donner tort ! Mais on peut craindre qu’avec des ressources infinies de courage, de bonne volonté, et même d’intelligence, la France ne s’étouffe comme un feu mal disposé. L’égoïsme, source du socialisme, la jalousie, source de la démocratie, ne feront jamais qu’une société faible, incapable de résister à de puissants voisins. Une société n’est forte qu’à la condition de reconnaître le fait des supériorités naturelles, lesquelles au fond se réduisent à une seule, celle de la naissance, puisque la supériorité intellectuelle et morale n’est elle-même que la supériorité d’un germe de vie éclos dans des conditions particulièrement favorisées.
\subsection[{IV}]{IV}
\noindent Si nous eussions été seuls au monde ou sans voisins, nous aurions pu continuer indéfiniment notre décadence et même nous y complaire ; mais nous n’étions pas seuls au monde. Notre passé de gloire et d’empire venait comme un spectre troubler notre fête. Celui dont les ancêtres ont été mêlés à de grandes luttes n’est pas libre de mener une vie paisible et vulgaire ; les descendants de ceux que ses pères ont tués viennent sans cesse le réveiller dans sa bourgeoise félicité et lui porter l’épée au front.\par
Toujours légère et inconsidérée, la France avait à la lettre oublié qu’elle avait insulté il y a un demi-siècle la plupart des nations de l’Europe, et en particulier la race qui offre en tout le contraire de nos qualités et de nos défauts. La conscience française est courte et vive ; la conscience allemande est longue, tenace et profonde. Le Français est bon, étourdi ; il oublie vite le mal qu’il a fait et celui qu’on lui a fait ; l’Allemand est rancunier, peu généreux ; il comprend médiocrement la gloire, le point d’honneur ; il ne connaît pas le pardon. Les revanches de 1814 et de 1815 n’avaient pas satisfait l’énorme haine que les guerres funestes de l’Empire avaient allumée dans le cœur de l’Allemagne. Lentement, savamment, elle préparait la vengeance d’injures qui pour nous étaient des faits d’un autre âge, avec lequel nous ne nous sentions aucun lien et dont nous ne croyions nullement porter la responsabilité.\par
Pendant que nous descendions insouciants la pente d’un matérialisme inintelligent ou d’une philosophie trop généreuse, laissant presque se perdre tout souvenir d’esprit national (sans songer que notre état social était si peu solide qu’il suffisait pour tout perdre du caprice de quelques hommes imprudents ), un tout autre esprit, le vieil esprit de ce que nous appelons l’ancien régime, vivait en Prusse, et à beaucoup d’égards en Russie. L’Angleterre et le reste de l’Europe, ces deux pays exceptés, étaient engagés dans la même voie que nous, voie de paix, d’industrie, de commerce, présentée par l’école des économistes et par la plupart des hommes d’État comme la voie même de la civilisation. Mais il y avait deux pays où l’ambition dans le sens d’autrefois, l’envie de s’agrandir, la foi nationale, l’orgueil de race duraient encore. La Russie, par ses instincts profonds, par son fanatisme à la fois religieux et politique, conservait le feu sacré des temps anciens, ce qu’on trouve bien peu chez un peuple usé comme le nôtre par l’égoïsme, c’est-à-dire la prompte disposition à se faire tuer pour une cause à laquelle ne se rattache aucun intérêt personnel. En Prusse, une noblesse privilégiée, des paysans soumis à un régime quasi-féodal, un esprit militaire et national poussé jusqu’à la rudesse, une vie dure, une certaine pauvreté générale, avec un peu de jalousie contre les peuples qui mènent une vie plus douce, maintenaient les conditions qui ont été jusqu’ici la force des nations. Là, l’état militaire, chez nous déprécié ou considéré comme synonyme d’oisiveté et de vie désœuvrée, était le principal titre d’honneur, une sorte de carrière savante. L’esprit allemand avait appliqué à l’art de tuer la puissance de ses méthodes. Tandis que, de ce côte du Rhin, tous nos efforts consistaient à extirper les souvenirs selon nous néfastes du premier empire, le vieil esprit des Blücher, des Scharnhorst vivait là encore. Chez nous, le patriotisme se rapportant aux souvenirs militaires était ridiculise sous le nom de {\itshape chauvinisme}, là-bas, tous sont ce que nous appelons des {\itshape chauvins}, et s’en font gloire. La tendance du libéralisme français était de diminuer l’État au profit de la liberté individuelle ; l’État en Prusse était bien plus tyrannique qu’il ne le fut jamais chez nous ; le Prussien, élevé, dressé, moralisé, instruit, enrégimenté, toujours surveillé par l’État, était bien plus gouverné (mieux gouverné aussi sans doute ) que nous ne le fûmes jamais, et ne se plaignait pas. Ce peuple est essentiellement monarchique ; il n’a nul besoin d’égalité ; il a des vertus, mais des vertus de classes. Tandis que parmi nous un même type d’honneur est l’idéal de tous, en Allemagne, le noble, le bourgeois, le professeur, le paysan, l’ouvrier, ont leur formule particulière du devoir ; les devoirs de l’homme, les droits de l’homme sont peu compris ; et c’est là une grande force, car l’égalité est la plus grande cause d’affaiblissement politique et militaire qu’il y ait. Joignez-y la science, la critique, l’étendue et la précision de l’esprit, toutes qualités que développe au plus haut degré l’éducation prussienne, et que notre éducation française oblitère ou ne développe pas ; joignez-y surtout les qualités morales et en particulier la qualité qui donne toujours la victoire à une race sur les peuples qui l’ont moins, la chasteté \footnote{Les femmes comptent en France pour une part énorme du mouvement social et politique ; en Prusse, elles comptent pour infiniment moins.}, et vous comprendrez que, pour quiconque a un peu de philosophie de l’histoire et a compris ce que c’est que la vertu des nations, pour quiconque a lu les deux beaux traités de Plutarque, {\itshape De la vertu et de la fortune d’Alexandre. De la vertu et de la fortune des Romains}, il ne pouvait y avoir de doute sur ce qui se préparait. Il était facile de voir que la révolution française, faiblement arrêtée un moment par les événements de 1814 et de 1815, allait une seconde fois voir se dresser devant elle son éternelle ennemie, la race germanique ou plutôt slavo-germanique du Nord, en d’autres termes, la Prusse, demeurée pays d’ancien régime, et ainsi préservée du matérialisme industriel, économique, socialiste, révolutionnaire, qui a dompté la virilité de tous les autres peuples. La résolution fixe de l’aristocratie prussienne de vaincre la révolution française a eu ainsi deux phases distinctes, l’une de 1792 à 1815, l’autre de 1848 à 1871, toutes deux victorieuses, et il en sera probablement encore ainsi à l’avenir, à moins que la révolution ne s’empare de son ennemi lui-même, ce à quoi l’annexion de l’Allemagne à la Prusse fournira de grandes facilites, mais non encore pour un avenir immédiat.\par
La guerre est essentiellement une chose d’ancien régime. Elle suppose une grande absence de réflexion égoïste, puisque, après la victoire, ceux qui ont le plus contribué à la faire remporter, je veux dire les morts, n’en jouissent pas ; elle est le contraire de ce manque d’abnégation, de cette âpreté dans la revendication des droits individuels, qui est l’esprit de notre moderne démocratie. Avec cet esprit-là il n’y a pas de guerre possible. La démocratie est le plus fort dissolvant de l’organisation militaire. L’organisation militaire est fondée sur la discipline ; la démocratie est la négation de la discipline. L’Allemagne a bien son mouvement démocratique ; mais ce mouvement est subordonné au mouvement patriotique national. La victoire de l’Allemagne ne pouvait donc manquer d’être complète ; car une force organisée bat toujours une force non organisée, même numériquement supérieure. La victoire de l’Allemagne a été la victoire de l’homme discipline sur celui qui ne l’est pas, de l’homme respectueux, soigneux, attentif, méthodique sur celui qui ne l’est pas ; ç’a été la victoire de la science et de la raison ; mais ç’a été en même temps la victoire de l’ancien régime, du principe qui nie la souveraineté du peuple et le droit des populations à régler leur sort. Ces dernières idées, loin de fortifier une race, la désarment, la rendent impropre à toute action militaire, et, pour comble de malheur, elles ne la préservent pas de se remettre entre les mains d’un gouvernement qui lui fasse faire les plus grandes fautes. L’acte inconcevable du mois de juillet 1870 nous jeta dans un gouffre. Tous les germes putrides qui eussent amené sans cela une lente consomption devinrent un accès pernicieux ; tous les voiles se déchirèrent ; des défauts de tempérament qu’on ne faisait que soupçonner apparurent d’une manière sinistre.\par
Une maladie ne va jamais seule ; car un corps affaibli n’a plus la force de comprimer les causes de destruction qui sont toujours à l’état latent dans l’organisme, et que l’état de santé empêche de faire éruption. L’horrible épisode de la Commune est venu montrer une plaie sous la plaie, un abîme au-dessous de l’abîme. Le 18 mars 1871 est, depuis mille ans, le jour où la conscience française a été le plus bas. Nous doutâmes un moment si elle se reformerait, si la force vitale de ce grand corps, atteinte au point même du cerveau où réside le {\itshape sensorium commune}, serait suffisante pour l’emporter, sur la pourriture qui tendait à l’envahir, L’œuvre des Capétiens parut compromise, et on put croire que la future formule philosophique de notre histoire clorait en 1871 le grand développement commencé par les ducs de France au \textsc{ix}\textsuperscript{e} siècle. Il n’en a pas été ainsi. La conscience française, quoique frappée d’un coup terrible, s’est retrouvée elle-même ; elle est sortie en trois ou quatre jours de son évanouissement. La France s’est reprise à la vie, le cadavre que les vers déjà se disputaient a retrouvé sa chaleur et son mouvement. Dans quelles conditions va se produire cette existence d’outre tombe ? Sera-ce le court éclair de la vie d’un ressuscité ? La France va-t-elle reprendre un chapitre interrompu de son histoire ? Ou bien va-t-elle entrer dans une phase entièrement nouvelle de ses longues et mystérieuses destinées ? Quels sont les vœux qu’un bon Français peut former en de telles circonstances ? Quels sont les conseils qu’il peut donner à son pays ? Nous allons essayer de le dire, non avec cette assurance qui serait en de pareils jours l’indice d’un esprit bien superficiel, mais avec cette réserve qui fait une large part aux hasards de tous les jours et aux incertitudes de l’avenir.
\section[{Deuxième partie, les remèdes}]{Deuxième partie,\\
 les remèdes}\renewcommand{\leftmark}{Deuxième partie,\\
 les remèdes}

\noindent Une chose connue de tous le monde est la facilite avec laquelle notre pays se réorganise. Des faits récents ont prouvé combien la France a été peu atteinte dans sa richesse. Quant aux pertes d’hommes, s’il était permis de parler d’un pareil sujet avec une froideur qui a l’air cruel, je dirais qu’elles sont à peine sensibles. Une question se pose donc à tout esprit réfléchi. Que va faire la France ? Va-t-elle se remettre sur la pente d’affaiblissement national et de matérialisme politique où elle était engagée avant la guerre de 1870, ou bien va-t-elle réagir énergiquement contre la conquête étrangère, répondre à l’aiguillon qui l’a piquée au vif, et, comme l’Allemagne de 1807, prendre dans sa défaite le point de départ d’une ère de rénovation ? — La France est très oublieuse. Si la Prusse n’avait pas exigé de cessions territoriales, je n’hésiterais pas à répondre que le mouvement industriel, économique, socialiste, eut repris son cours ; les pertes d’argent eussent été réparées au bout de quelques années ; le sentiment de la gloire militaire et de la vanité nationale se fût perdu de plus en plus. Oui, l’Allemagne avait entre les mains après Sedan le plus beau rôle de l’histoire du monde. En restant sur sa victoire, en ne faisant violence à aucune partie de la population française, elle enterrait la guerre pour l’éternité, autant qu’il est permis de parler d’éternité, quand il s’agit des choses humaines. Elle n’a pas voulu de ce rôle ; elle a pris violemment deux millions de Français, dont une très petite fraction peut être supposée consentante à une telle séparation. Il est clair que tout ce qui reste de patriotisme français n’aura de longtemps qu’un objectif, regagner les provinces perdues. Ceux même qui sont philosophes avant d’être patriotes ne pourront être insensibles au cri de deux millions d’hommes, que nous avons été obligés de jeter à la mer pour sauver le reste des naufragés, mais qui étaient liés avec nous pour la vie et pour la mort. La France a donc là une pointe d’acier enfoncée en sa chair, qui ne la laissera plus dormir. Mais quelle voie va-t-elle suivre dans l’œuvre de sa réforme ? En quoi sa renaissance ressemblera-t-elle à tant d’autres tentatives de résurrection nationale ? Quelle y sera la part de l’originalité française ? C’est ce qu’il faut rechercher, en tenant {\itshape a priori} pour probable qu’une conscience aussi impressionnable que la conscience française aboutira, sous l’étreinte de circonstances uniques, aux manifestations les plus inattendues.\par
\subsection[{I}]{I}
\noindent Il existe un modèle excellent de la manière dont une nation peut se relever des derniers désastres. C’est la Prusse elle-même qui nous l’a donné, et elle ne peut nous reprocher de suivre son exemple. Que fit la Prusse après la paix de Tilsitt ? Elle se résigna, se recueillit. Le territoire qui lui restait était tout au plus le cinquième de ce qui nous reste ; ce territoire était le plus pauvre de l’Europe, et les conditions militaires qui lui étaient faites semblaient de nature à le condamner pour jamais à l’impuissance. Il y avait de quoi décourager un patriotisme moins âpre. La Prusse s’organisa silencieusement ; loin de chasser sa dynastie, elle se serra autour d’elle adora son roi médiocre, sa reine Louise, qui pourtant avait été une des causes immédiates de la guerre. Toutes les capacités de la nation furent appelées ; Stein dirigea tout avec son ardeur concentrée. La réforme de l’armée fut un chef-d’œuvre d’étude et de réflexion ; l’université de Berlin fut le centre de la régénération de l’Allemagne ; une collaboration cordiale fut demandée aux savants, aux philosophes, qui ne mirent qu’une condition à leur concours, celle qu’ils mettent et doivent mettre toujours, leur liberté. De ce sérieux travail poursuivi pendant cinquante ans, la Prusse sortit la première nation de l’Europe. Sa régénération eut une solidité que ne saurait donner la simple vanité patriotique, elle eut une base morale ; elle fut fondée sur l’idée du devoir, sur la fierté que donne le malheur noblement supporté.\par
Il est clair que, si la France voulait imiter son exemple, elle serait prête en moins de temps. Si le mal de la France venait d’un épuisement profond, il n’y aurait rien à faire ; mais tel n’est pas le cas ; les ressources sont immenses ; il s’agit de les organiser. Il est incontestable aussi que les circonstances nous viendraient en aide. « La figure de ce monde passe », dit l’Écriture. Certaines personnes mourront ; les difficultés intérieures de l’Allemagne reviendront ; le parti catholique et le parti démocratique des deux Internationales (comme on dit en Prusse ) créeront à M. de Bismark et à ses successeurs de perpétuelles difficultés ; il faut songer que l’unité de l’Allemagne n’est nullement encore l’unité de la France ; il y a des parlements à Dresde, à Munich, à Stuttgart ; qu’on se figure Louis XIV dans de pareilles conditions. En Prusse, la rivalité du parti féodal et du parti libéral, habilement conjurée par M. de Bismark, éclatera ; le rayonnement fécond et pacifique du germanisme s’arrêtera. Le facteur de la conscience slave, c’est la conscience allemande ; la conscience des Slaves grandira et s’opposera de plus en plus à celle des Allemands ; l’inconvénient qu’il y a pour un État à détenir des pays malgré eux se révèlera de plus en plus ; la crise interminable de l’Autriche amènera les péripéties les plus dangereuses ; Vienne deviendra de toute manière un embarras pour Berlin ; quoi qu’on fasse, cet empire est, né bicéphale, il vivra difficilement. La roue de fortune tourne et tournera toujours, après avoir monté, on descend ; et voilà pourquoi l’orgueil est quelque chose de si peu raisonnable. Les organisations militaires sont comme les outillages industriels ; un outillage vieillit vite, et il est rare que l’industriel réforme de lui-même l’outillage qui est en sa possession ; cet outillage, en effet, représente un immense capital d’établissement ; on veut le garder ; on ne le change que si la concurrence vous y force. En ce cas, il arrive presque toujours que le concurrent a l’avantage ; car il construit à neuf, et n’a pas de concession à faire à un établissement antérieur. Sans le fusil à aiguille, la France n’eût jamais remplacé son fusil à piston ; mais le fusil à aiguille l’ayant mise en mouvement, elle a fait le chassepot. Les organisations militaires se succèdent de la sorte comme les machines de l’industrie. La machine militaire de Frédéric le Grand eut en son temps l’excellence ; en 1792, elle était totalement vieillie et impuissante. La machine de Napoléon eut ensuite la force ; de nos jours, la machine de M. de Moltke a prouvé son immense supériorité. Ou les choses humaines vont changer leur marche, ou ce qui est le meilleur aujourd’hui ne le sera pas demain. Les aptitudes militaires changent d’une génération à l’autre. Les armées de la République et de l’Empire succédèrent à celles qui furent battues à Rosbach. Une fois la France entraînée, une fois son embonpoint bourgeois et ses habitudes casanières secoués, impossible de dire ce qui arrivera.\par
Il est donc certain que, si la France veut se soumettre aux conditions d’une réforme sérieuse, elle peut très vite, reprendre sa place dans le concert européen. Je ne saurais croire qu’aucun homme d’État sérieux ait fait en Allemagne le raisonnement qu’ont sans cesse répète les journaux allemands : « Prenons l’Alsace et la Lorraine pour mettre la France hors d’état de recommencer. » S’il ne s’agit que de surface territoriale et de chiffres d’âmes, la France est à peine entamée. La question est de savoir si elle voudra entrer dans la voie d’une réforme sérieuse, en d’autres termes, imiter la conduite de la Prusse après Iéna.\par
Cette voie serait austère ; ce serait celle de la pénitence. En quoi consiste la vraie pénitence ? Tous les Pères de la vie spirituelle sont d’accord sur ce point : la pénitence ne consiste pas à mener une vie dure, à jeûner, à se mortifier. Elle consiste à se corriger de ses défauts, et parmi ses défauts à se corriger justement de ceux qu’on aime, de ce défaut favori qui est presque toujours le fond même de notre nature, le principe secret de nos actions. Quel est pour la France ce défaut favori, dont il importe avant tout qu’elle se corrige ? C’est le goût de la démocratie superficielle. La démocratie fait notre faiblesse militaire et politique ; elle fait notre ignorance, notre sotte vanité ; elle fait, avec le catholicisme arrière, l’insuffisance de notre éducation nationale. Je comprendrais donc qu’un bon esprit et un bon patriote, plus jaloux d’être utile à ses concitoyens que de leur plaire, s’exprimât à peu près en ces termes :\par

\begin{quoteblock}
 \noindent « Corrigeons-nous de la démocratie. Rétablissons la royauté, rétablissons dans une certaine mesure la noblesse ; fondons une solide instruction nationale primaire et supérieure ; rendons l’éducation plus rude, le service militaire obligatoire pour tous ; devenons sérieux, appliqués, soumis aux puissances, amis de la règle et de la discipline. Soyons humbles surtout. Défions-nous de la présomption. La Prusse a mis soixante-trois ans à se venger d’Iéna ; mettons-en, au moins vingt à nous venger de Sedan ; pendant dix ou quinze ans, abstenons-nous complètement des affaires du monde ; renfermons-nous dans le travail obscur de notre réforme intérieure. À aucun prix ne faisons de révolution, cessons de croire que nous avons en Europe le privilège de l’initiative ; renonçons à une attitude qui fait de nous une perpétuelle exception à l’ordre général De la sorte, il est incontestable que, les changements ordinaires du monde y aidant, nous aurons dans quinze ou vingt ans retrouvé notre rang.\par
 « Nous ne le retrouverions pas autrement. La victoire de la Prusse a été la victoire de la royauté de droit quasi-divin (de droit historique ) ; une nation ne saurait se réformer sur le type prussien sans la royauté historique et sans la noblesse. La démocratie ne discipline ni ne moralise. On ne se discipline pas soi-même ; des enfants mis ensemble sans maître ne s’élèvent pas ; ils jouent et perdent leur temps. De la masse ne peut émerger assez de raison pour gouverner et reformer un peuple. Il faut que la réforme et l’éducation viennent du dehors, d’une force n’ayant d’autre intérêt que celui de la nation, mais distincte de la nation et indépendante d’elle. Il y a quelque chose que la démocratie ne fera jamais, c’est la guerre, j’entends la guerre savante comme la Prusse l’a inaugurée. Le temps des volontaires indisciplinés et des corps francs est passé. Le temps des brillants officiers, ignorants, braves, frivoles, est passé aussi. La guerre est désormais un problème scientifique et d’administration, une œuvre compliquée que la démocratie superficielle n’est pas plus capable de mener à bonne fin que des constructeurs de barques ne sauraient faire une frégate cuirassée. La démocratie à la française ne donnera jamais assez d’autorité aux savants pour qu’ils puissent faire prévaloir une direction rationnelle. Comment les choisirait-elle, obsédée qu’elle est de charlatans et incompétente pour décider entre eux ? La démocratie, d’ailleurs, ne sera pas assez ferme pour maintenir longtemps l’effort énorme qu’il faut pour une grande guerre. Rien ne se fait en ces gigantesques entreprises communes, si chacun, selon une expression vulgaire, « en prend et en laisse » ; or la démocratie ne peut sortir de sa mollesse sans entrer dans la terreur. Enfin, la république doit toujours être en suspicion contre l’hypothèse d’un général victorieux. La monarchie est si naturelle à la France, que tout général qui aurait donné à son pays une éclatante victoire serait capable de renverser les institutions républicaines. La république ne peut exister que dans un pays vaincu ou absolument pacifié. Dans tout pays exposé à la guerre, le cri du peuple sera toujours le cri des Hébreux à Samuel : « Un roi qui marche à notre tête et fasse la guerre avec nous. » \par
 « La France s’est trompée sur la forme que peut prendre la conscience d’un peuple. Son suffrage universel est comme un tas de sable, sans cohésion ni rapport fixe entre les atomes. On ne construit pas une maison avec cela. La conscience d’une nation réside dans la partie éclairée de la nation, laquelle entraîne et commande le reste. La civilisation à l’origine a été une œuvre aristocratique, l’œuvre d’un tout petit nombre (nobles et prêtres ), qui l’on imposée par ce que les démocrates appellent force et imposture ; la conservation de la civilisation est une œuvre aristocratique aussi. Patrie, honneur, devoir, sont choses créées et maintenues par un tout petit nombre au sein d’une foule qui, abandonnée à elle-même, les laisse tomber. Que fût devenue Athènes, si on eût donné le suffrage à ses deux cent mille esclaves et noyé sous le nombre la petite aristocratie d’hommes libres qui l’avaient faite ce qu’elle était ? La France de même avait été créée par le roi, la noblesse, le clergé, le tiers état. Le peuple proprement dit et les paysans, aujourd’hui maîtres absolus de la maison, y sont en réalité des intrus, des frelons impatronisés dans une ruche qu’ils n’ont pas construite. L’âme d’une nation ne se conserve pas sans un collège officiellement charge de la garder. Une dynastie est la meilleure institution pour cela ; car, en associant les chances de la nation à celles d’une famille, une telle institution crée les conditions les plus favorables à une bonne continuité. Un sénat comme celui de Rome et de Venise remplit très bien le même office ; les institutions religieuses, sociales, pédagogiques, gymnastiques des Grecs y suffisaient parfaitement ; le prince électif à vie a même soutenu des états sociaux assez forts ; mais ce qui ne s’est jamais vu, c’est le rêve de nos démocrates, une maison de sable, une nation sans institutions traditionnelles, sans corps charge de faire la continuité de la conscience nationale, une nation fondée sur ce déplorable principe qu’une génération n’engage pas la génération suivante, si bien qu’il n’y a nulle chaîne des morts aux vivants, nulle sûreté pour l’avenir. Rappelez-vous ce qui a tué toutes les sociétés coopératives d’ouvriers : l’incapacité de constituer dans de telles sociétés une direction sérieuse, la jalousie contre ceux que la société avait revêtus d’un mandat quelconque, la prétention de les subordonner toujours à leurs mandants, le refus obstiné de leur faire une position digne. La démocratie française fera la même faute en politique ; il ne sortira jamais une direction éclairée de ce qui est la négation même de la valeur du travail intellectuel et de la nécessite d’un tel travail.\par
 « Et ne dites pas qu’une assemblée pourra remplir ce rôle des vieilles dynasties et des vieilles aristocraties. Le nom seul de république est une excitation à un certain développement démocratique malsain ; on le verra bien au progrès d’exaltation qui se manifestera dans les élections, comme cela eut lieu en 1850 et 1851. Pour arrêter ce mouvement, une assemblée se montrera impitoyable ; mais alors se dévoilera une autre tendance, celle qui porte à préférer une monarchie libérale à une république réactionnaire. La fatalité de la république est à la fois de provoquer l’anarchie et de la réprimer très durement. Une assemblée n’est jamais un grand homme. Une assemblée a les défauts qui chez un souverain sont les plus rédhibitoires : bornée, passionnée, emportée, décidant vite, sans responsabilité, sous le coup de l’idée du moment. Espérer qu’une assemblée composée de notabilités départementales, d’honnêtes provinciaux, pourra prendre et soutenir le brillant héritage de la royauté, de la noblesse françaises, est une chimère. Il faut un centre aristocratique permanent, conservant l’art, la science, le goût, contre le béotisme démocratique et provincial. Paris le sent bien ; jamais aristocratie n’a tenu à son privilège séculaire autant que Paris à ce privilège qu’il s’attribue d’être une institution de la France, d’agir à certains jours comme tête et souverain, et de réclamer l’obéissance du reste du pays ; mais que Paris, en réclamant son privilège de capitale, se prétende encore républicain et ait fondé le suffrage de tous, c’est là une des plus fortes inconséquences dont l’histoire des siècles ait garde le souvenir.\par
 « La synagogue de Prague a dans ses traditions une vieille légende qui m’a toujours paru un symbole frappant. Un cabbaliste du \textsc{xvi}\textsuperscript{e} siècle avait fait une statue si parfaitement conforme aux proportions de l’archétype divin, qu’elle vivait, agissait. En lui mettant sous la langue le nom ineffable de Dieu (le mystique tétragramme ), le cabbaliste conférait même à l’homme de plâtre la raison, mais une raison obscure, imparfaite, qui avait toujours besoin d’être guidée : il se servait de lui comme d’un domestique pour diverses besognes serviles ; le samedi, il lui ôtait de la bouche le talisman merveilleux, pour qu’il observât le saint repos. Or une fois il oublia cette précaution bien nécessaire. Pendant qu’on était au service divin en entendit dans le {\itshape ghetto} un bruit épouvantable ; c’était l’homme de plâtre qui cassait, brisait tout. On accourt, on se saisit de lui. À partir de ce moment, on lui ôta pour jamais le tétragramme, et on le mit sous clef dans le grenier de la synagogue, où il se voit encore. Hélas ! nous avions cru qu’en faisant balbutier quelques mots de raison à l’être informe que la lumière intérieure n’éclaire pas, nous en faisions un homme. Le jour où nous l’avons abandonné à lui-même, la machine brutale s’est détraquée ; je crains qu’il ne faille la remiser pour des siècles.\par
 « Relever un droit historique, en place de cette malheureuse formule du droit « divin » que les publicistes d’il y a cinquante ans mirent en vogue, serait donc la tâche qu’il faudrait se proposer. La monarchie, en liant les intérêts d’une nation à ceux d’une famille riche et puissante, constitue le système de plus grande fixité pour la conscience nationale. La médiocrité du souverain n’a même en un tel système que de faibles inconvénients. Le degré de raison nationale émanant d’un peuple qui n’a pas contracté un mariage séculaire avec une famille est, au contraire, si faible, si discontinu, si intermittent qu’on ne peut le comparer qu’à la raison d’un homme tout à fait inférieur ou même à l’instinct d’un animal. Le premier pas est donc évidemment que la France reprenne sa dynastie. Un pays n’a qu’une dynastie, celle qui a fait son unité au sortir d’un état de crise ou de dissolution. La famille qui a fait la France en neuf cents ans existe ; plus heureux que la Pologne, nous possédons notre vieux drapeau d’unité ; seulement, une déchirure funeste le dépare. Les pays dont l’existence est fondée sur la royauté souffrent toujours les maux les plus graves quand il y a des dissidences sur l’hérédité légitime. D’un autre côté, l’impossible est l’impossible... Sans doute on ne peut soutenir que la branche d’Orléans, depuis sa retraite sans combat en février (acte qui put être le fait de bons citoyens, mais ne fut pas celui de princes ), ait des droits royaux bien stricts ) mais elle a un titre excellent, le souvenir du règne de Louis-Philippe, l’estime et l’affection de la partie éclairée de la nation.\par
 « Il ne faut pas nier, d’un autre côté, que la Révolution et les années qui ont suivi furent à beaucoup d’égards une de ces crises génératrices où tous les casuistes politiques reconnaissent que se fonde le droit des dynasties. La maison Bonaparte émergea du chaos révolutionnaire qui accompagna et suivit la mort de Louis XVI, comme la maison capétienne sortit de l’anarchie qui accompagna en France la décadence de la maison carlovingienne. Sans les événements de 1814 et 1815, il est probable que la maison Bonaparte héritait du titre des Capétiens. La remise en valeur du titre bonapartiste à la suite de la révolution de 1848 lui a donné une réelle force. Si la révolution de la fin du dernier siècle doit un jour être considérée comme le point de départ d’une France nouvelle, il est possible que la maison Bonaparte devienne la dynastie de cette nouvelle France ; car Napoléon I\textsuperscript{er} sauva la révolution d’un naufrage inévitable, et personnifia très bien les besoins nouveaux. La France est certainement monarchique ; mais l’hérédité repose sur des raisons politiques trop profondes pour qu’elle les comprenne. Ce qu’elle veut, c’est une monarchie sans la loi bien fixe, analogue à celle des Césars romains. La maison de Bourbon ne doit pas se prêter à ce désir de la nation ; elle manquerait à tous ses devoirs si elle consentait jamais à jouer les rôles de podestats, de stathouders, de présidents provisoires de républiques avortées. On ne se taille pas un justaucorps dans le manteau de Louis XIV. La maison Bonaparte, au contraire, ne sort pas de son rôle en acceptant ces positions indécises, qui ne sont pas en contradiction avec ses origines et que justifie la pleine acceptation qu’elle a toujours faite du dogme de la souveraineté du peuple.\par
 « La France est dans la position de l’Hercule du sophiste Prodicus, {\itshape Hercules in bivio.} Il faut que d’ici à quelques mois elle décide de son avenir. Elle peut garder la république : mais qu’on ne veuille pas des choses contradictoires. Il y a des esprits qui se figurent une république puissante, influente, glorieuse. Qu’ils se détrompent et choisissent. Oui, la république est possible en France, mais une république à peine supérieure en importance à la confédération helvétique et moins considérée. La république ne peut avoir ni armée ni diplomatie ; la république serait un État militaire d’une rare nullité ; la discipline y serait très imparfaite ; car, ainsi que l’a bien montre M. Stoffel, il n’y a pas de discipline dans l’armée, s’il n’y en a pas dans la nation. Le principe de la république, c’est l’élection ; une société républicaine est aussi faible qu’un corps d’armée qui nommerait ses officiers ; la peur de n’être pas réélu paralyse toute énergie. M. de Savigny a montré qu’une société a besoin d’un gouvernement venant du dehors, d’au-delà, d’avant elle, que le pouvoir social n’émane pas tout entier de la société, qu’il y a un droit philosophique et historique (divin, si l’on veut ) qui s’impose à la nation. La royauté n’est nullement, comme affecte de le croire notre superficielle école constitutionnelle, une présidence héréditaire. Le président des États-Unis n’a pas fait la nation, tandis que le roi a fait la nation. Le roi n’est pas une émanation de la nation ; le roi et la nation sont deux choses ; le roi est en dehors de la nation. La royauté est ainsi un fait divin pour ceux qui croient au surnaturel, un fait historique pour ceux qui n’y croient pas. La volonté actuelle de la nation, le plébiscite, même sérieusement pratiqué, ne suffit pas. L’essentiel n’est pas que telle volonté particulière de la majorité se fasse ; l’essentiel est que la raison générale de là nation triomphe. La majorité numérique peut vouloir l’injustice, l’immoralité ; elle peut vouloir détruire son histoire, et alors la souveraineté de la majorité numérique n’est plus que la pire des erreurs.\par
 « C’est, en tout cas, l’erreur qui affaiblit le plus une nation. Une assemblée élue ne réforme pas. Donnez à la France un roi jeune, sérieux, austère en ses mœurs ; qu’il règne cinquante ans, qu’il groupe autour de lui des hommes âpres au travail, fanatiques de leur œuvre, et la France aura encore un siècle de gloire et de prospérité. Avec la république, elle aura l’indiscipline, le désordre, des francs tireurs, des volontaires cherchant à faire croire au pays qu’ils se vouent à la mort pour lui, et n’ayant pas assez d’abnégation pour accepter les conditions communes de la vie militaire. Ces conditions, obéissance, hiérarchie, etc., sont le contraire de tout ce que conseille le catéchisme démocratique, et voilà pourquoi une démocratie ne saurait vivre avec un état militaire considérable. Cet état militaire ne peut se développer sous un pareil régime, ou, s’il se développe, il absorbe la démocratie. On m’objectera l’Amérique ; mais, outre que l’avenir de ce pays est très obscur, il faut dire que l’Amérique, par sa position géographique, est placée, en ce qui concerne l’armée, dans une situation toute particulière, à laquelle la nôtre ne saurait être comparée.\par
 « Je ne conçois qu’une issue à ces hésitations, qui tuent le pays ; c’est un grand acte d’autorité nationale. On peut être royaliste sans admettre le droit divin, comme on peut être catholique sans croire à l’infaillibilité du pape, chrétien sans croire au surnaturel et à la divinité de Jésus-Christ. La dynastie est en un sens antérieure et supérieure à la nation, puisque c’est la dynastie qui a fait la nation ; mais elle ne peut rien contre la nation ni sans elle. Les dynasties ont des droits sur le pays qu’elles représentent historiquement ; mais le pays a aussi des droits sur elles, puisque les dynasties n’existent qu’en vue du pays. Un appel adressé au pays dans des circonstances extraordinaires pourrait constituer un acte analogue au grand fait national qui créa la dynastie capétienne, ou à la décision de l’université de Paris lors de l’avènement des Valois. Nos anciens théoriciens de la monarchie conviennent que la légitimité des dynasties s’établit à certains moments solennels, où il s’agit avant tout de tirer la nation de l’anarchie et de remplacer un titre dynastique périmé.\par
 « C’est également par le procédé historique, je veux dire en profitant habilement des pans de murs qui nous restent d’une plus vieille construction, et en développant ce qui existe, que l’on pourrait former quelque chose pour remplacer les anciennes traditions de famille. Pas de royauté sans noblesse ; ces deux choses reposent au fond sur le même principe, une sélection créant artificiellement pour le bien de la société une sorte de race à part. La noblesse n’a plus chez nous aucune signification de race, Elle résulte d’une cooptation presque fortuite, où l’usurpation des titres, les malentendus, les petites fraudes, et surtout l’idée puérile qui consiste à croire que la préposition {\itshape de} est une marque de noblesse, tiennent presque autant de place que la naissance et l’anoblissement légal. Le suffrage à deux degrés introduirait un principe aristocratique bien meilleur. L’armée serait un autre moyen d’anoblissement. L’officier de notre future {\itshape Landwehr}, milice locale sans cesse exercée, deviendrait vite un hobereau de village, et cette fonction aurait souvent une tendance à être héréditaire ; le capitaine cantonal, vers l’âge de cinquante ans, aimerait à transmettre son office à son fils, qu’il aurait formé et que tous connaîtraient. La même chose arriva au moyen âge par la nécessité de se défendre. Le {\itshape Ritter}, qui avait un cheval, sorte de brigadier de gendarmerie, devint un petit seigneur.\par
 « La base de la vie provinciale devrait ainsi être un honnête gentilhomme de village, bien loyal, et un bon curé de campagne tout entier dévoué à l’éducation morale du peuple. Le devoir est une chose aristocratique, il faut qu’il ait sa représentation spéciale. Le maître, dit Aristote, a plus de devoirs que l’esclave ; les classes supérieures en ont plus que les classes inférieures. Cette {\itshape gentry} provinciale ne doit pas être tout ; mais elle est une base nécessaire. Les universités, centres de haute culture intellectuelle, la cour, école de mœurs brillantes, Paris, résidence du souverain et ville de grand monde, corrigeront ce que la {\itshape gentry} provinciale a d’un peu lourd, et empêcheront que la bourgeoisie, trop fière de sa moralité, ne dégénère en pharisaïsme. Une des utilités des dynasties est justement d’attribuer aux choses exquises ou sérieuses une valeur que le public ne peut leur donner, de discerner certains produits particulièrement aristocratiques que la masse ne comprend pas. Il fut bien plus facile à Turgot d’être ministre en 1774 qu’il ne le serait de nos jours. De nos jours, sa modestie, sa gaucherie, son manque de talent comme orateur et comme écrivain l’eussent arrêté dès les premiers pas. Il y a cent ans, pour arriver, il lui suffit d’être compris et apprécié de l’abbé de Véry, prêtre philosophe, très écouté de madame de Maurepas.\par
 « Tout le monde est à peu près d’accord sur ce point qu’il nous faut une loi militaire calquée pour les lignes générales sur le système prussien. Il y aura dans le premier moment d’émotion des députés pour la faire. Mais, ce moment passé, si nous restons en république, il n’y aura pas de députés pour la maintenir ou la faire exécuter. À chaque élection, le député sera obligé de prendre à cet égard des engagements qui énerveront son action future. Si la Prusse avait le suffrage universel, elle n’aurait pas le service militaire universel, ni l’instruction obligatoire. Depuis longtemps la pression de l’électeur aurait fait alléger ces deux charges. Le système prussien n’est possible qu’avec des nobles de campagne, chefs-nés de leur village, toujours en contact avec leurs hommes, les formant de longue main, les réunissant en un clin d’œil. Un peuple sans nobles est au moment du danger un troupeau de pauvres affolés, vaincu d’avance par un ennemi organisé. Qu’est-ce que la noblesse, en effet, si ce n’est la fonction militaire considérée comme héréditaire et mise au premier rang des fonctions sociales ? Quand la guerre aura disparu du monde, la noblesse disparaîtra aussi ; non auparavant. On ne forme pas une armée, comme on forme une administration des domaines ou des tabacs, par le choix libre des familles et des jeunes gens. La carrière militaire entendue de la sorte est trop chétive pour attirer les bons sujets. La sélection militaire de la démocratie est misérable ; un Saint-Cyr formé sous un tel régime sera toujours excessivement faible. S’il y a, au contraire, une classe qui soit appliquée à la guerre par le fait de la naissance, cela donnera pour l’armée une moyenne de bons esprits, qui sans cela iraient à d’autres applications.\par
 « Sont-ce là des rêves ? Peut-être ; mais alors, je vous l’assure, la France est perdue. Elle ne le serait pas, si l’on pouvait croire que l’Allemagne sera entraînée à son tour dans la ronde du sabbat démocratique, où nous avons laissé toute notre vertu ; mais cela n’est pas probable. Ce peuple est soumis, résigné au-delà de tout ce qu’on peut croire. Son orgueil national est si fort exalté par ses victoires, que, pendant une ou deux générations encore, les problèmes sociaux n’occuperont qu’une part limitée de son activité. Un peuple, comme un homme, préfère toujours s’appliquer à ce en quoi il excelle ; or la race germanique sent sa supériorité militaire. Tant quelle sentira cela, elle ne fera ni révolution, ni socialisme. Cette race est vouée pour longtemps à la guerre et au patriotisme ; cela la détournera de la politique intérieure, de tout ce qui affaiblit le principe de hiérarchie et de discipline. S’il est vrai, comme il semble, que la royauté et l’organisation nobiliaire de l’armée sont perdues chez les peuples latins, il faut dire que les peuples latins appellent une nouvelle invasion germanique et la subiront. » 
 \end{quoteblock}

\subsection[{II}]{II}
\noindent Heureux qui trouve dans des traditions de famille ou dans le fanatisme d’un esprit étroit l’assurance qui seule tranche tous ces doutes ! Quant à nous, trop habitués à voir les différents cotés des choses pour croire à des solutions absolues, nous admettrions aussi qu’un très honnête citoyen parlât ainsi qu’il suit :\par

\begin{quoteblock}
 \noindent « La politique ne discute pas les solutions imaginaires. On ne change pas le caractère d’une nation. Il suffit que le plan de réforme que vous venez de tracer ait été celui de la Prusse pour que j’ose affirmer que ce ne sera pas celui de la France. Des réformes supposant que la France abjure ses préjugés démocratiques sont des réformes chimériques. La France, croyez-le, restera un pays de gens aimables, doux, honnêtes, droits, gais, superficiels, pleins de bon cœur, de faible intelligence politique ; elle conservera son administration médiocre, ses comités entêtés, ses corps routiniers, persuadée qu’ils sont les premiers du monde ; elle s’enfoncera de plus en plus dans cette voie de matérialisme, de républicanisme vulgaire vers laquelle tout le monde moderne, excepté la Prusse et la Russie, paraît se tourner. Cela veut-il dire qu’elle n’aura jamais sa revanche ? C’est peut-être justement par là qu’elle l’aura. Sa revanche serait alors un jour d’avoir devancé le monde dans la route qui conduit à la fin de toute noblesse, de toute vertu. Pendant que les peuples germaniques et slaves conserveraient leurs illusions de jeunes races, nous leur resterions inférieurs ; mais ces races vieilliront à leur tour ; elles entreront dans la voie de toute chair. Cela ne se fera pas aussi vite que le croit l’école socialiste, toujours persuadée que les questions qui la préoccupent absorbent le monde au même degré. Les questions de rivalité entre les races et les nations paraissent devoir longtemps encore l’emporter sur les questions de salaire et de bien-être, dans les parties de l’Europe qu’on peut appeler d’ancien monde ; mais l’exemple de la France est contagieux. Il n’y a jamais eu de révolution française qui n’ait eu son contrecoup à l’étranger. La plus cruelle vengeance que la France put tirer de l’orgueilleuse noblesse qui a été le principal instrument de sa défaite serait de vivre en démocratie, de démontrer par le fait la possibilité de la république. Il ne faudrait peut-être pas beaucoup attendre pour que nous pussions dire à nos vainqueurs comme les morts d’Isaïe : {\itshape Et tu vulneratus es sicut et nos, nostri similis effectus es} ! \par
 « Que la France reste donc ce qu’elle est ; qu’elle tienne sans défaillance le drapeau de libéralisme qui lui à fait un rôle depuis cent ans. Ce libéralisme est souvent une cause de faiblesse, c’est une raison pour que le monde y vienne ; car le monde va s’énervant et perdant de sa rigueur antique. La France en tout cas est plus sûre d’avoir sa revanche, si elle la doit à ses défauts, que si elle est réduite à l’attendre de qualités qu’elle n’a jamais eues. Nos ennemis peuvent être rassurés si le Français, pour reprendre sa place, doit préalablement devenir un Poméranien ou un Diethmarse. Ce qui a vaincu la France, c’est un reste de force morale, de rudesse de pesanteur et d’esprit d’abnégation qui s’est trouvé avoir encore résisté, sur un point perdu du monde, à l’effet délétère de la réflexion égoïste. Que la démocratie française réussisse à constituer un état viable, et ce vieux levain aura bien vite disparu sous l’action du plus énergique dissolvant de toute vertu que le monde ait connu jusqu’ici. » 
 \end{quoteblock}

\noindent Peut-être, en effet, le parti qu’a pris la France sur le conseil de quelques hommes d’État qui la connaissent bien, d’ajourner les questions constitutionnelles et dynastiques est-il le plus sage. Nous nous y conformerons. Sans sortir de ce programme, on peut indiquer quelques réformes qui, en toute hypothèse, doivent être méditées.
\subsection[{III}]{III}
\noindent Ceux mêmes qui n’admettent pas que la France se soit trompée en proclamant sans réserve la souveraineté du peuple ne peuvent nier au moins, s’ils ont quelque esprit philosophique, qu’elle n’ait choisi un mode de représentation nationale très imparfait \footnote{ J’ai été heureux de m’être rencontré, dans les vues qui suivent, avec quelques bons esprits qui cherchent en ce moment le remède à nos institutions si défectueuses. J. Foulon-Ménard {\itshape Fonctions de l’État.} Nantes, 1871 ; J. Guadet, {\itshape Du suffrage universel et de son application d’après un mode nouveau}, Bordeaux.}. La nomination des pouvoirs sociaux au suffrage universel direct est la machine politique la plus grossière qui ait jamais été employée. Un pays se compose de deux éléments essentiels : 1º les citoyens pris isolément comme de simples unités ; {\itshape 2}º les fonctions sociales, les groupes, les intérêts, la propriété. Deux chambres sont donc nécessaires et jamais gouvernement régulier, quel qu’il soit, ne vivra sans deux chambres. Une seule chambre nommée par le suffrage des citoyens, pris comme de simples unités pourra ne pas renfermer un seul magistrat, un seul général, un seul professeur, un seul administrateur. Une telle chambre pourra mal représenter la propriété, les intérêts, ce qu’on peut appeler les collèges moraux de la nation. Il est donc absolument nécessaire qu’à côté d’une assemblée élue par les citoyens sans distinction de professions, de titres, de classes sociales, il y ait une assemblée formée par un autre procédé, et représentant les capacités, les spécialités, les intérêts divers, sans lesquels il n’y a pas d’État organisé.\par
Est-il indispensable que la première, de ces deux chambres, pour être une vraie représentation des citoyens, soit nommée par l’universalité des citoyens ? Non certes, et le brusque établissement du suffrage universel en 1848 a été, de l’aveu de tous les politiques, une grande faute. Mais il ne s’agit plus de revenir sur ce fait. Toute mesure, comme la loi du 31 mai 1851, ayant pour but de priver des citoyens d’un droit qu’ils ont exercé depuis vingt-trois ans serait un acte blâmable. Ce qui est légitime, possible et juste, c’est de faire que le suffrage, tout en restant parfaitement universel, ne soit plus direct, c’est d’introduire des degrés dans le suffrage. Toutes les constitutions de la première république, hormis celle de 1793, qui ne fonctionna jamais, admirent ce principe élémentaire. Les deux degrés corrigeraient ce que le suffrage universel a nécessairement de superficiel ; la réunion des électeurs au second degré constituerait un public politique digne de candidats sérieux. On peut accorder que tout citoyen possède un certain droit à la direction de la chose publique ; mais il faut régler ce droit, en éclairer l’exercice. Que cent citoyens d’un même canton, en confiant leur procuration à un de leurs concitoyens habitant le même canton, le fassent électeur ; cela donnera environ quatre-vingt mille électeurs pour toute la France. Ces quatre mille électeurs formeraient des collèges départementaux, dont chaque fraction cantonale se réunirait au chef-lieu de canton, aurait ses assises libres, et voterait pour tout le département. Le scrutin de liste, si absurde avec le suffrage universel direct, aurait alors sa pleine raison d’être, surtout si le nombre des membres de la première chambre était réduit, comme il devrait l’être, à quatre on cinq cents. Dans ce système, les opérations pour le choix des électeurs du second degré seraient, il est vrai, publiques ; mais il y aurait là une garantie de moralité. La procuration électorale devrait être conférée pour quinze ou vingt ans ; si on forme le collège électoral en vue de chaque élection particulière, on perdra presque tous les avantages de la réforme dont il s’agit.\par
J’avoue que je préférerais un système plus représentatif encore, et où la femme, l’enfant fussent comptés. Je voudrais que, dans les élections primaires, l’homme marié votât pour sa femme (en d’autres termes, que sa voix comptât pour deux ), que le père votât pour ses enfants mineurs ; je concevrais même la mère, la sœur confiant leur pouvoir à un fils, à un frère majeurs. Il est sûrement impossible que la femme participe directement à la vie politique ; mais il est juste qu’elle soit comptée. Il y aurait trop d’inconvénients à ce qu’elle pût choisir la personne à laquelle elle donnerait sa procuration politique ; mais la femme qui a son mari, son père, ou bien un frère, un fils majeurs a des procureurs naturels, dont elle doit pouvoir, si j’ose le dire, doubler la personnalité le jour du scrutin. De la sorte, la société devient un ensemble lié, cimenté, où tout est devoir réciproque, responsabilité, solidarité. Les électeurs du second degré seraient des aristocrates locaux, des autorités, des notables nommés presque à vie. Ces électeurs pourraient être rassemblés par cantons en temps de crise ; ils seraient les gardiens des mœurs, les surveillants des deniers publics ; ils tiendraient école de gravité et de sérieux. Les conseils généraux de département émaneraient de procédés électoraux analogues, légèrement modifiés.\par
Tout autres et infiniment plus variés devraient être les moyens servant à composer la seconde chambre. Supposons que le nombre des membres soit de trois cent soixante. D’abord, il y faudrait une trentaine de sièges héréditaires, réservés aux survivants d’anciennes familles, dont les titres résisteraient à un travail historique et critique. Les membres à vie seraient nommés par des procédés divers. On pourrait faire désigner un membre par le conseil général de chaque département. Le chef de l’État nommerait cinquante membres ; la chambre haute elle-même se recruterait jusqu’à concurrence de trente membres ; la première chambre en nommerait trente autres. Les cent vingt ou cent trente membres restants représenteraient les corps nationaux, les fonctions sociales. L’armée et la marine y figureraient par les maréchaux et les amiraux ; la magistrature, les corps enseignants, les clergés y verraient siéger leurs chefs ; chaque classe de l’Institut nommerait un membre ; il en serait de même des corporations industrielles, des chambres de commerce, etc. Les grandes villes, enfin, sont des personnes morales, ayant un esprit propre. Je voudrais que toute grande ville de plus de cent mille âmes eût un élu dans la chambre haute ; Paris en aurait quatre ou cinq. Cette chambre représenterait ainsi tout ce qui est une individualité dans l’État ; ce serait vraiment un corps conservateur de tous les droits et de toutes les libertés.\par
Il est permis d’espérer que deux chambres ainsi formées serviraient au progrès libéral, et non à la révolution. Vu certaines particularités du caractère français, il serait bon d’interdire la publicité des séances, laquelle fait trop souvent dégénérer les débats en parade. On fonderait ainsi un genre d’éloquence simple et vrai, bien préférable au ton de nos harangues prolixes, déclamatoires, de mauvais goût. Le compte rendu a l’inconvénient de déplacer l’objectif de l’orateur, de le porter à viser le public plutôt que la Chambre et de faire servir le gouvernement du pays à l’agitation du pays. Si la France veut un avenir de réformes et de revanches, il faut qu’elle évite d’user ses forces en luttes parlementaires. Le gouvernement parlementaire est excellent pour les époques de prospérité ; il sert à faire éviter les fautes très graves et les excès, ce qui certes est capital : mais il n’excite pas les grands efforts moraux. La Prusse n’aurait pas accompli sa renaissance à la suite d’Iéna, si elle eût pratiqué la vie parlementaire. Elle traversa quarante ans de silence, qui servirent merveilleusement à tremper le caractère de la nation.\par
Il est incontestable que Paris est la seule capitale possible de la France ; mais ce privilège doit être payé par des charges. Non seulement il faut que Paris renonce à ses attentats sur la représentation de la France ; Paris, étant constitué pas la résidence des autorités centrales à l’état de ville à part, ne peut avoir les droits d’une ville ordinaire. Paris ne saurait avoir ni maire, ni conseil élu dans les conditions ordinaires, ni garde civique. Le souverain ne doit pas trouver dans la ville où il réside une autre souveraineté que la sienne. Les usurpations dont la commune de Paris s’est rendue coupable à toutes les époques ne justifient que trop les appréhensions à cet égard.\par
Avec de solides institutions, la liberté de la presse pourrait être laissée entière. Dans un état social vraiment assis, l’action de la presse est très utile comme contrôle ; sans la presse, des abus extrêmement graves sont inévitables. C’est aux classes honnêtes à décourager par leur mépris la presse scandaleuse. Quant à la liberté des clubs, l’expérience a montre que cette liberté n’a aucun avantage sérieux, et qu’elle ne vaut pas la peine qu’on y fasse des sacrifices.\par
La cause de la décentralisation administrative est trop complètement gagnée pour que nous y insistions. Que si l’on veut parler d’une décentralisation plus profonde, qui ferait de la France une fédération d’États analogue aux États-Unis d’Amérique, il faut s’entendre. Il n’y a pas d’exemple dans l’histoire d’un État unitaire et centralisé décrétant son morcellement. Un tel morcellement a failli se faire au mois de mars dernier ; il se ferait le jour où la France serait mise encore plus bas qu’elle ne l’a été par la guerre de 1870 et par la Commune ; il ne se fera jamais par mesure légale. Un pouvoir organisé ne cède que ce qu’on lui arrache. Quand de grandes machines de gouvernement, comme l’empire romain, l’empire franc, commencent à s’affaiblir, les parties disloquées de ces ensembles font leurs conditions au pouvoir central, se dressent des chartes, forcent le pouvoir, central à les signer. En d’autres termes, la formation d’une confédération (hors le cas des colonies ) est l’indice d’un empire qui s’effondre. Ajournons donc de tels propos, d’autant plus que, si les crocs de fer qui retiennent ensemble les pierres de la vieille construction se relâchaient, il n’est pas sur que ces pierres resteraient à leur place et ne se disjoindraient pas tout à fait.\par
La colonisation en grand est une nécessité politique tout à fait de premier ordre. Une nation qui ne colonise pas est irrévocablement vouée au socialisme, à la guerre du riche et du pauvre, la conquête d’un pays de race inférieure par une race supérieure, qui s’y établit pour le gouverner, n’a rien de choquant. L’Angleterre pratique ce genre de colonisation dans l’Inde, au grand avantage de l’Inde, de l’humanité en général, et à son propre avantage. La conquête germanique du Ve et du \textsc{vi}\textsuperscript{e} siècle est devenue en Europe la base de toute conservation et de toute légitimité. Autant les conquêtes entre races égales doivent être blâmées, autant la régénération, des races inférieures ou abâtardies par les races supérieures est dans l’ordre providentiel de l’humanité. L’homme du peuple est presque toujours chez nous un noble déclassé, sa lourde main est bien mieux faite pour manier l’épée que l’outil servile. Plutôt que de travailler, il choisit de se battre, c’est-à-dire qu’il revient à son premier état. {\itshape Regere imperio populos}, voilà notre vocation. Versez cette dévorante activité sur des pays qui, comme la Chine, appellent la conquête étrangère. Des aventuriers qui troublent la société européenne faites un {\itshape ver sacrum}, un essaim comme ceux des Francs, des Lombards, des Normands ; chacun sera dans son rôle. La nature a fait une race d’ouvriers, c’est la race chinoise, d’une dextérité de main merveilleuse sans presque aucun sentiment d’honneur ; gouvernez-la avec justice, en prélevant d’elle pour le bienfait d’un tel gouvernement un ample douaire au profit de la race conquérante, elle sera satisfaite ; — une race de travailleurs de la terre, c’est le nègre ; soyez pour lui bon et humain, et tout sera dans l’ordre ; — une race de maîtres et de soldats, c’est la race européenne. Réduisez cette noble race à travailler dans l’ergastule comme des nègres et des Chinois, elle se révolte. Tout révolté est chez nous, plus ou moins, un soldat qui a manque sa vocation, un être fait pour la vie héroïque, et que vous appliquez à une besogne contraire à sa race, mauvais ouvrier, trop bon soldat. Or la vie qui révolte nos travailleurs rendrait heureux un Chinois, un {\itshape fellah}, êtres qui ne sont nullement militaires. Que chacun fasse ce pour quoi il est fait, et tout ira bien. Les économistes se trompent en considérant le travail comme l’origine de la propriété. L’origine de la propriété, c’est la conquête et la garantie donnée par le conquérant aux fruits du travail autour de lui. Les Normands ont été en Europe les créateurs de la propriété ; car, le lendemain du jour où ces bandits eurent des terres, ils s’établirent pour eux et pour tous les gens de leur domaine un ordre social et une sécurité qu’on n’avait pas vus jusque-là.
\subsection[{IV}]{IV}
\noindent Dans la lutte qui vient de finir l’infériorité de la France a été surtout intellectuelle ; ce qui nous a manqué, ce n’est pas le cœur, c’est la tête. L’instruction publique est un sujet d’importance capitale ; l’intelligence française s’est affaiblie ; il faut la fortifier. Notre plus grande erreur est de croire que l’homme naît tout élevé ; l’Allemand, il est vrai, croit trop à l’éducation ; il en devient pédant ; mais nous y croyons trop peu. Le manque de foi à la science est le défaut profond de la France ; notre infériorité militaire et politique n’a pas d’autre cause ; nous doutons trop de ce que peuvent la réflexion, la combinaison savante. Notre système d’instruction a besoin de réformes radicales ; presque tout ce que le premier empire a fait à cet égard est mauvais. L’instruction publique ne peut être donnée directement par l’autorité centrale ; un ministère de l’instruction publique sera toujours une très médiocre machine d’éducation.\par
L’instruction primaire est la plus difficile à organiser. Nous envions à l’Allemagne sa supériorité à cet égard ; mais il n’est pas philosophique de vouloir les fruits sans le tronc et les racines. En Allemagne, l’instruction populaire est venue du protestantisme. Le luthéranisme ayant fait consister la religion à lire un livre, et plus tard ayant réduit la dogmatique chrétienne à une quintessence impalpable, a donné une importance hors de ligne à la maison d’école ; l’illettré a presque été chassé du christianisme ; la communion parfois lui est refusée. Le catholicisme, au contraire, faisant consister le salut en des sacrements et en des croyances surnaturelles, tient l’école pour chose secondaire. Excommunier celui qui ne sait ni lire ni écrire nous paraît impie.\par
L’école n’étant pas l’annexe de l’église est la rivale de l’église. Le curé s’en défie, la veut aussi faible que possible, l’interdit même si elle n’est pas toute cléricale. Or, sans la collaboration et la bonne volonté du curé, l’école de village ne prospérera jamais. Que ne pouvons-nous espérer que le catholicisme se réforme, qu’il se relâche de ses règles surannées ! Quels services ne rendrait pas un curé, pasteur catholique, offrant dans chaque village le type d’une famille bien réglée, surveillant l’école, presque maître d’école lui-même, donnant à l’éducation du paysan le temps qu’il consacre aux fastidieuses répétitions de son bréviaire ! En réalité, l’église et l’école sont également nécessaires ; une nation ne peut pas plus se passer de l’une que de l’autre ; quand l’église et l’école se contrarient, tout va mal.\par
Nous touchons ici à la question qui est au fond de toutes les autres. La France a voulu rester catholique ; elle en porte les conséquences. Le catholicisme est trop hiératique pour donner un aliment intellectuel et moral à une population ; il fait fleurir le mysticisme transcendant à côté de l’ignorance ; il n’a pas d’efficacité morale ; il exerce des effets funestes sur le développement du cerveau. Un élève des jésuites ne sera jamais un officier susceptible d’être opposé à un officier prussien ; un élève, des écoles élémentaires catholiques ne pourra jamais faire la guerre savante avec les armes perfectionnées. Les nations catholiques qui ne se réformeront pas seront toujours infailliblement battues par les nations protestantes. Les croyances surnaturelles sont comme un poison qui tue si on le prend à trop haute dose. Le protestantisme en mêle bien une certaine quantité à son breuvage ; mais la proportion est faible et devient alors bienfaisante. Le moyen âge avait créé deux maîtrises de la vie de l’esprit, l’Église, l’Université ; les pays protestants ont gardé ces deux cadres ; ils ont crée la liberté dans l’Église, la liberté dans l’Université, si bien que ces pays peuvent avoir à la fois des Églises établies, un enseignement officiel, et une pleine liberté de conscience et d’enseignement. Nous autres, pour avoir la liberté, nous avons été obliges de nous séparer de l’Église ; les jésuites avaient depuis longtemps réduit nos universités à un rôle secondaire. Aussi nos efforts ont été faibles, ne se rattachant à aucune tradition ni à aucune institution du passé.\par
Un libéral comme nous est ici fort embarrassé ; car notre premier principe est que, dans ce qui touche à la liberté de conscience, l’État ne doit se mêler de rien. La foi, comme toutes les choses exquises, est susceptible ; au moindre contact, elle crie à la violence. Ce qu’il faut désirer, c’est une réforme libérale du catholicisme, sans intervention de l’État. Que l’Église admette deux catégories de croyants, ceux qui sont pour la lettre et ceux qui s’en tiennent à l’esprit. À un certain degré de la culture rationnelle, la croyance au surnaturel devient pour plusieurs une impossibilité ; ne forcez pas ceux-là à porter une chape de plomb. Ne nous mêlez pas de ce que nous enseignons, de ce que nous écrivons, et nous ne vous disputerons pas le peuple ; ne nous contestez pas notre place à l’université, à l’académie, et nous vous abandonnerons sans partage l’école de campagne. L’esprit humain est une échelle où chaque degré est nécessaire ; ce qui est bon à tel niveau n’est pas bon à tel autre ; ce qui est funeste pour l’un ne l’est pas pour l’autre. Conservons au peuple son éducation religieuse, mais qu’on nous laisse libres. Il n’y a pas de fort développement de la tête sans liberté ; l’énergie morale n’est pas le résultat d’une doctrine en particulier, mais de la race et de la vigueur de l’éducation. Nous avait-on assez parlé de la décadence de cette Allemagne qu’on présentait comme une officine d’erreurs énervantes, de dangereuses subtilités. Elle était tuée, disait-on, par le sophisme, le protestantisme, le matérialisme, le panthéisme, le fatalisme. Je ne jurerais pas, en effet, que M. de Moltke ne professe quelqu’une de ces erreurs ; mais on avouera que cela ne l’empêche pas d’être un bon officier d’état-major. Renonçons à ces déclamations fades. La liberté de penser, alliée à la haute culture, loin d’affaiblir un pays, est une condition du grand développement de l’intelligence. Ce n’est pas telle ou telle solution qui fortifie l’esprit ; ce qui le fortifie, c’est la discussion, la liberté. On peut dire que pour l’homme cultive il n’y a pas de mauvaise doctrine ; car pour lui toute doctrine est un effort vers le vrai, un exercice utile à la santé de l’esprit. Vous voulez garder vos jeunes gens dans une sorte de gynécée intellectuel ; vous en ferez des hommes bornés. Pour former de bonnes têtes scientifiques, des officiers sérieux et appliqués, il faut une éducation ouverte à tout, sans dogme rétrécissant. La supériorité intellectuelle et militaire appartiendra désormais à la nation qui pensera librement. Tout ce qui exerce le cerveau est salutaire. Il y a plus : la liberté de penser dans les universités a cet avantage que le libre penseur, satisfait de raisonner à son aise dans sa chaire au milieu de personnes placées au même point de vue que lui, ne songe plus à faire de la propagande parmi les gens du monde et les gens du peuple. Les universités allemandes présentent à ce sujet le spectacle le plus curieux.\par
Notre instruction secondaire, quoique fort critiquable, est la meilleure partie de notre système d’enseignement. Les bons élèves d’un lycée de Paris sont supérieurs aux jeunes Allemands pour le talent d’écrire, l’art de la rédaction ; ils sont mieux préparés à être avocats on journalistes ; mais ils ne savent pas assez de choses. Il faut se persuader que la science prend de plus en plus le dessus sur ce qu’on appelle en France les lettres. L ‘enseignement doit surtout être scientifique ; le résultat de l’éducation doit être que le jeune homme sache le plus possible de ce que l’esprit humain a découvert sur la réalité de l’univers. Quand je dis scientifique, je ne dis pas pratique, professionnel ; l’État n’a pas à s’occuper des applications de métier ; mais il doit prendre garde que l’éducation qu’il donne ne se borne à une rhétorique creuse, qui ne fortifie pas l’intelligence. Chez nous, les dons brillants, le talent, l’esprit, le génie sont seuls estimes ; en Allemagne, ces dons sont rares, peut-être parce qu’ils ne sont pas fort prisés ; les bons écrivains y sont peu nombreux ; le journalisme, la tribune politique n’ont pas l’éclat qu’ils ont chez nous ; mais la force de tête, l’instruction, la solidité du jugement sont bien plus répandues, et constituent une moyenne de culture intellectuelle supérieure à tout ce qu’on avait pu obtenir jusqu’ici d’une nation.\par
C’est surtout dans l’enseignement supérieur qu’une reforme est urgente. Les écoles spéciales, imaginées par la Révolution, les chétives facultés créées par l’Empire, ne remplacent nullement le grand et beau système des universités autonomes et rivales, système que Paris a créé au moyen âge et que toute l’Europe a conservé, excepté justement la France qui l’a inauguré vers 1200. En y revenant, nous n’imiterons personne, nous ne ferons que reprendre notre tradition. Il faut créer en France cinq ou six universités, indépendantes les unes des autres, indépendantes des villes où elles seront établies, indépendantes du clergé. Il faut supprimer du même coup les écoles spéciales, École polytechnique, École normale, etc., institutions inutiles quand on possède un bon système d’universités, et qui empêchent les universités de se développer. Ces écoles ne sont, en effet, que des prélèvements funestes faits sur les auditeurs des universités \footnote{On n’entend pas nier l’utilité de tels établissements comme internats ou séminaires ; mais l’enseignement intérieur n’y devait pas dépasser la conférence entre élèves, selon les usages anciens.}. L’université enseigne tout, prépare à tout, et dans son sein toutes les branches de l’esprit humain se touchent et s’embrassent. À côte des universités, il peut, il doit y avoir des écoles d’application ; il ne peut y avoir des écoles d’État fermées et faisant concurrence aux universités. On se plaint que les facultés des lettres, des sciences, n’aient pas d’élèves assidus. Quoi de surprenant ? Leurs auditeurs naturels sont à l’École normale, à l’École polytechnique, où ils reçoivent le même enseignement, mais sans rien sentir du mouvement salutaire, de la communauté d’esprit que crée l’université.\par
Ces universités établies dans des villes de province \footnote{ Une circonstance  ; un autre ordre rendra l’application de ce système presque indispensable, c’est l’établissement du service obligatoire pour tous. Une telle organisation militaire n’est possible que si le jeune homme peut faire ses études d’université (droit, médecine, etc. ) en même temps que son service militaire, ainsi que cela se pratique en Allemagne. Cette combinaison suppose des villes d’étude régionales, qui soient en même temps des centres sérieux d’instruction militaire.}, sans préjudice naturellement de l’université de Paris et des grands établissements uniques, tels que le Collège de France, propres à Paris, me paraissent le meilleur moyen de réveiller l’esprit français. Elles seraient des écoles de sérieux, d’honnêteté, de patriotisme. Là se fonderait la vraie liberté de penser, qui ne pas sans de solides études. Là aussi se ferait un salutaire changement dans l’esprit de la jeunesse. Elle se formerait au respect ; elle prendrait, le sentiment de la valeur de la science. Un fait qui donne bien à réfléchir est celui-ci. Il est reconnu que nos écoles sont des foyers d’esprit démocratique peu réfléchi et d’une incrédulité portée vers une propagande populaire étourdie. C’est tout le contraire en Allemagne, où les universités sont des foyers d’esprit aristocratique, réactionnaire (comme nous disons ) et presque féodal, des foyers de libre pensée, mais non de prosélytisme indiscret. D’où vient cette différence ? De ce que la liberté de discussion, dans les universités allemandes, est absolue. Le rationalisme est loin de porter à la démocratie. La réflexion apprend que la raison n’est pas la simple expression des idées et des vœux de la multitude, qu’elle est le résultat des aperceptions d’un petit nombre d’individus privilégiés. Loin d’être portée à livrer la chose publique aux caprices de la foule, une génération aussi élevée sera jalouse de maintenir le privilège de la raison ; elle sera appliquée, studieuse et très peu révolutionnaire. La science sera pour elle comme un titre de noblesse, auquel elle ne renoncera pas facilement, et qu’elle défendra même avec une certaine âpreté. Des jeunes gens élevés dans le sentiment de leur supériorité se révolteront de ne compter que pour un comme le premier venu. Pleins du juste orgueil que donne la conscience de savoir la vérité que le vulgaire ignore, ils ne voudront pas être les interprètes des pensées superficielles de la foule. Les universités seront ainsi des pépinières d’aristocrates. Alors, l’espèce d’antipathie que le parti conservateur français nourrit contre la haute culture de l’esprit paraîtra le plus inconcevable des non-sens, la plus fâcheuse erreur.\par
Il va sans le dire qu’à côte de ces universités dotées par l’État, et où toutes les opinions savamment présentées auraient accès, une entière latitude serait laissée pour l’établissement d’universités libres. Je crois que ces universités libres produiraient de très médiocres résultats ; toutes les fois que la liberté existe réellement dans l’université, la liberté hors de l’université est de peu de conséquence ; mais, en leur permettant de s’établir, on aurait la conscience en règle et on fermerait la bouche aux personnes naïves toujours portées à croire que sans la tyrannie de l’État elles feraient des merveilles. Il est bien probable que les catholiques les plus fervents, un Ozanam, par exemple, préféreraient le champ libre des universités d’État, où tout se passerait au grand jour, à ces petites universités à huis clos, fondées par leur secte. En tout cas, ils auraient le choix. De quoi pourraient se plaindre avec un pareil régime les catholiques les plus portes à s’élever contre le monopole de l’État ? Personne ne serait exclu des chaires des universités à cause de ses opinions ; les catholiques y arriveraient comme tout le monde. Le système des {\itshape Privatdocent} permettrait en outre à toutes les doctrines de se produire en dehors des chaires dotées. Enfin les universités libres enlèveraient jusqu’au dernier prétexte aux récriminations. Ce serait l’inverse de notre système français, procédant par l’exclusion des sujets brillants. On croit avoir assez fait pour l’impartialité si, après avoir destitué ou refuse de nommer un libre penseur, on destitue ou refuse de nommer un catholique. En Allemagne, on les met tous deux face à face ; au lieu de ne servir que la médiocrité, un tel système sert à l’émulation et à l’éveil des esprits. En distinguant soigneusement le grade et le droit d’exercer une profession, comme on le fait en Allemagne, en établissant que l’université ne fait pas des médecins, des avocats, mais rend apte à devenir médecin, avocat, on lèverait les difficultés que certaines personnes trouvent à la collation des grades par l’État. L’État, en un tel système, ne salarie pas certaines opinions scientifiques ou littéraires ; il ouvre, dans un haut intérêt social et pour le bien de toutes les opinions, de grands champs clos, de vastes arènes, où les sentiments divers peuvent se produire, lutter entre eux et se disputer l’assentiment de la jeunesse, déjà mûre pour la réflexion, qui assiste à ces débats.\par
Former par les universités une tête de société rationaliste, régnant par la science, fière de cette science et peu disposée à laisser périr son privilège au profit d’une foule ignorante ; mettre (qu’on me permette, cette forme paradoxale d’exprimer ma pensée ) le pédantisme en honneur, combattre ainsi l’influence trop grande des femmes, des gens du monde, des Revues, qui absorbent tant de force vives ou ne leur offrent qu’une application superficielle ; donner plus à la spécialité, à la science, à ce que les Allemands appellent le {\itshape Fach}, moins à la littérature, au talent d’écrire et de parler ; compléter ce faite solide de l’édifice social par une cour et une capitale brillantes, d’où l’éclat d’un esprit aristocratique n’exclut pas la solidité et la forte culture de la raison ; en même temps, élever le peuple, raviver ’ses facultés un peu affaiblies, lui inspirer, avec l’aide d’un bon clergé dévoue à la patrie, l’acceptation d’une société supérieure, le respect de la science et de la vertu, l’esprit de sacrifice et de dévouement ; voilà ce qui serait l’idéal ; il sera beau du moins de chercher à en approcher.\par
J’ai dit à plusieurs reprises que ces reformes ne peuvent pas bien se faire sans la collaboration du clergé. Il est clair que notre principe théorique ne peut plus être que la séparation de l’Église et de l’État ; mais la pratique ne saurait être la théorie. Jusqu’ici, la France n’a connu que deux pôles, catholicisme, démocratie ; oscillant sans cesse de l’un à l’autre, elle ne se repose jamais entre les deux. Pour faire pénitence de ses excès démagogiques, la France se jette dans le catholicisme étroit ; pour réagir contre le catholicisme étroit, elle se jette dans la fausse démocratie. Il faudrait faire pénitence des deux à la fois, car la fausse démocratie et le catholicisme étroit s’opposent également à une réforme de la France sur le type prussien, je veux dire à une forte et saine éducation rationnelle. Nous sommes à l’égard du catholicisme dans cette situation étrange que nous ne pouvons vivre ni avec lui ni sans lui. L’Église est une pièce trop importante d’éducation pour qu’on se prive d’elle, si de son côte elle fait les concessions nécessaires et ne se rend pas, en exagérant ses doctrines, plus nuisible qu’utile. Si un mouvement gallican de reforme dans le genre de celui que rêve avec tant de candeur, de sincérité, de chaleur d’âme le P. Hyacinthe, si un mouvement de réforme, dis-je, entraînant le mariage des prêtres de campagne et le remplacement du bréviaire par un enseignement presque quotidien, était possible, il faudrait l’accueillir avec empressement ; mais je crains que l’Ëglise catholique ne se roidisse et n’aime mieux tomber que de se modifier. Un schisme m’y paraît plus probable que jamais ; ou plutôt le schisme est déjà fait ; de latent, il deviendra effectif. La haine des Allemands et des Français, l’occupation de Rome par le roi d’Italie, ont ajoute un élément explosible nouveau à ceux qu’avait entassés le concile. Si le pape reste dans Rome, capitale de l’Italie, les non-Italiens souffriront de voir leur chef spirituel ainsi subordonne à une nation particulière. Si le pape quitte Rome, les Italiens diront comme en 1378 : « Le pape est l’évêque de Rome ; qu’il revienne, ou nous allons choisir un évêque de Rome, lequel, par là même, sera le pape. » À vrai dire, un pape tel que l’a fait le concile ne peut résider nulle part ; il lui faudrait une ile escarpée et sans bords ; il n’a pas de place au monde ; or, si la papauté cesse d’avoir un petit territoire politiquement neutralise à son usage, elle verra briser son unité. Il me paraît donc presque inévitable que nous ayons bientôt deux papes et même trois, car il va être bien difficile que des Français, des Italiens et des Allemands soient de la même religion. Le principe des nationalités devait à la longue amener la ruine de la papauté. On dit souvent : « Les questions religieuses ont de nos jours trop peu d’importance pour amener des schismes. » C’est là une erreur ; des hérésies, des divisions sur les dogmes abstraits, il n’y en aura plus \footnote{Le dogme de l’infaillibilité fait exception ; car ce dogme est « pratique » au plus haut degré, et atteint toute l’organisation de l’Église catholique dans ses rapports avec l’ordre civil.} ; car on ne prend presque plus le dogme au sérieux ; mais des schismes dans le genre de celui d’Avignon, des divisions de personnes, des élections contestées et dont l’incertitude maintiendra longtemps affrontées des parties de la catholicité, cela est parfaitement possible, cela sera. Une fois le schisme fait sur les personnes, une fois les deux papes constitués, l’un à Rome, l’autre hors de l’Italie, la décomposition de la catholicité s’opérera par le choix des obédiences, comme celle de l’eau sous l’action de la pile électrique ; chacun des deux papes deviendra un pole qui attirera à lui les éléments qui lui seront homogènes ; l’un sera le pape du catholicisme rétrograde, l’autre le pape du catholicisme progressif ; car tous deux désireront avoir des partisans, et, pour avoir des partisans, il faut représenter quelque chose. Nous verrons Pierre de Lune prétendre encore enfermer l’Église universelle sur son rocher de Paniscole ; la ligne de séparation des obédiences pourrait même déjà être tracée. Une foule de réformes maintenant impraticables seront praticables alors, et l’horizon du catholicisme, maintenant si ferme, pourra s’ouvrir tout à coup et laisser voir des profondeurs inattendues.
\subsection[{V}]{V}
\noindent Avec des efforts sérieux, une renaissance serait donc possible, et je suis persuadé que, si la France marchait dix ans dans la voie que nous avons essaye d’indiquer, l’estime et la bienveillance du monde la dispenseraient de toute revanche. Oui, il serait possible qu’un jour cette guerre funeste dût être bénie et considérée comme le commencement d’une régénération. Ce n’est pas la seule fois que la guerre aurait été plus utile au vaincu qu’au vainqueur. Si la sottise, la négligence, la paresse, l’imprévoyance des États n’avaient pour conséquence de les faire battre, il est difficile de dire à quel degré d’abaissement pourrait descendre l’espèce humaine. La guerre est de la sorte une des conditions du progrès, le coup de fouet qui empêche un pays de s’endormir, en forçant la médiocrité satisfaite d’elle-même à sortir de son apathie. L’homme n’est soutenu que par l’effort et la lutte. La lutte contre la nature ne suffit pas ; l’homme finirait, au moyen de l’industrie, par la réduire à peu de chose. La lutte des races se dresse alors. Quand une population a fait produire à son fonds tout ce qu’à peut se produire, elle s’amollirait, si la terreur de son voisin ne la réveillait ; car le but de l’humanité n’est pas de jouir ; acquérir et créer est œuvre de force et de jeunesse : jouir est de la décrépitude. La crainte de la conquête est ainsi, dans les choses humaines, un aiguillon nécessaire. Le jour où l’humanité deviendrait un grand empire romain pacifie et n’ayant plus d’ennemis extérieurs serait le jour où la moralité et l’intelligence courraient les plus grands dangers.\par
Mais ces réformes s’accompliront-elles ? La France va-t-elle s’appliquer à corriger ses défauts, à reconnaître ses erreurs ? La question est complexe, et, pour la résoudre, il faut s’être fait une idée précise du mouvement qui semble emporter vers un but inconnu tout le monde européen.\par
Le \textsc{xix}\textsuperscript{e} siècle possède deux types de société qui ont fait leurs preuves, et que malgré les incertitudes qui peuvent peser sur leur avenir, auront une grande place dans l’histoire de la civilisation. L’un est le type américain, fonde essentiellement sur la liberté et la propriété, sans privilèges de classes, sans institutions anciennes, sans histoire, sans société aristocratique, sans cour, sans pouvoir brillant, sans universités sérieuses ni fortes institutions scientifiques, sans service militaire obligatoire pour les citoyens. Dans ce système, l’individu, très peu protège par l’État, aussi très peu gêne par l’État. Jeté sans patron dans la bataille de la vie, il s’en tire comme il peut, et s’enrichit, s’appauvrit, sans qu’il songe une seule fois à se plaindre du gouvernement, à le renverser, à lui demander quelque chose, à déclamer contre la liberté et la propriété. Le plaisir de déployer son activité à toute vapeur lui suffit, même quand les chances de la loterie ne lui ont pas été favorables. Ces sociétés manquent de distinction, de noblesse ; elles ne font guère d’œuvres originales en fait d’art et de science ; mais elles peuvent arriver à être très puissantes, et d’excellentes choses peuvent s’y produire. La grosse question est de savoir combien de temps elles dureront, quelles maladies particulières les affecteront, comment elles se comporteront à l’égard du socialisme, qui les a jusqu’ici peu atteintes.\par
Le second type de société que notre siècle voit exister avec éclat est celui que j’appellerai l’ancien régime développé et corrige. La Prusse en offre le meilleur modèle. Ici l’individu est pris, élevé, façonné, dresse, discipline, requis sans cesse par une société dérivant du passe, moulée dans de vieilles institutions, s’arrogeant une maîtrise de moralité et de raison. L’individu, dans ce système, donne énormément à l’État ; il reçoit en échange de l’État une forte culture intellectuelle et morale, ainsi que la joie de participer à une grande œuvre. Ces sociétés sont particulièrement nobles ; elles créent la science ; elles dirigent l’esprit humain ; elles font l’histoire ; mais elles sont de jour en jour affaiblies par les réclamations de l’égoïsme individuel, qui trouve le fardeau que l’État lui impose trop lourd à porter. Ces sociétés, en effet, impliquent des catégories entières de sacrifies, de gens qui doivent se résigner à une vie triste sans espoir d’amélioration. L’éveil de la conscience populaire et jusqu’à un certain point l’instruction du peuple minent ces grands édifices féodaux et les menacent de ruine. La France, qui était autrefois une société de ce genre, est tombée. L’Angleterre s’éloigne sans cesse du type que nous venons de décrire pour se rapprocher du type américain. L’Allemagne maintient ce grand cadre, non sans que des signes de révolte s’y fassent déjà entrevoir. Jusqu’à quel point cet esprit de révolte, qui n’est autre chose que la démocratie socialiste, envahira-t-il les pays germaniques à leur tour ? Voilà la question qui doit préoccuper le plus un esprit réfléchi. Nous manquons d’éléments Pour y répondre avec précision.\par
Si les nations d’ancien régime ne faisaient, quand leur vieil édifice est renverse, que passer au système américain, la situation serait simple ; on pourrait alors se reposer en cette philosophie de l’histoire de l’école républicaine, selon laquelle le type social américain est celui de l’avenir, celui auquel tous les pays en viendront tôt ou tard. Mais il n’en est pas ainsi. La partie active du parti démocratique qui maintenant travaille plus ou moins tous les États européens n’a nullement pour idéal la république américaine. À part quelques théoriciens, le parti démocratique a des tendances socialistes qui sont l’inverse des idées américaines sur la liberté et la propriété. La liberté du travail, la libre concurrence, le libre usage de la propriété, la faculté laissée à chacun de s’enrichir selon ses pouvoirs, sont justement ce dont ne veut pas la démocratie européenne. Résultera-t-il de ces tendances un troisième type social, où l’État interviendra dans les contrats, dans les relations industrielles et commerciales, dans les questions de propriété ? On ne peut guère le croire ; car aucun système socialiste n’a réussi jusqu’ici à se présenter avec les apparences de la possibilité. De là un doute étrange, qui en France atteint les proportions. du plus haut tragique et trouble notre vie à tous : d’une part, il semble bien difficile de faire tenir debout sous une forme quelconque les institutions de l’ancien régime ; d’une autre part, les aspirations du peuple ne sont nullement en Europe dirigées vers le système américain. Une série de dictatures ’instables, un césarisme de basse époque, voilà tout ce qui se montre comme ayant les chances de l’avenir.\par
La direction matérialiste de la France peut d’ailleurs faire contrepoids à tous les motifs virils de réforme qui sortent de la situation. Cette direction matérialiste dure depuis les années qui suivirent 1830. Sous la Restauration, l’esprit public était très vivant encore ; la société noble songeait à autre chose que jouir et s’enrichir. La décadence devint tout à fait sensible vers 1840. Le soubresaut de 1848 n’arrêta rien ; le mouvement des intérêts matériels était vers 1853 ce qu’il eût été si la révolution de février ne fût pas arrivée. Certes, la crise de 1870-1871 est bien plus profonde que celle de 1848 ; mais on peut craindre que le tempérament du pays ne prenne encore le dessus, que la masse de la nation, rentrant dans son indifférence, ne songe plus qu’à gagner de l’argent et à jouir. L’intérêt personnel ne conseille jamais le courage militaire ; car aucun des inconvénients qu’on encourt par la lâcheté n’équivaut à ce que l’on risque par le courage. Il faut, pour exposer sa vie, la foi à quelque chose d’immatériel ; or cette foi disparaît de jour en jour. Ayant détruit le principe de la légitimité dynastique, qui fait consister la raison d’être de l’union des provinces dans les droits du souverain, il ne nous restait plus qu’un dogme, savoir qu’une nation existe par le libre consentement de toutes ses parties. La dernière paix a porté à ce principe la blessure la plus grave. Enfin, loin de se relever, la culture intellectuelle a reçu des événements de l’année des coups sensibles ; l’influence du catholicisme étroit, qui sera le grand obstacle à la renaissance, ne paraît nullement en train de décroître ; la présomption d’une partie des personnes qui président à l’administration semble par moments avoir redoublé avec les défaites et les affronts.\par
On ne peut nier, d’ailleurs, que beaucoup des reformes que la Prusse nous impose ne doivent rencontrer chez nous de sérieuses difficultés. La base du programme conservateur de la France a toujours été d’opposer les parties sommeillantes de la conscience populaire aux parties trop éveillées, je veux dire l’armée au peuple. Il est clair que ce programme manquerait de base le jour où l’esprit démocratique pénétrerait l’armée elle-même. Entretenir une armée faisant un corps à part dans la nation et empêcher le développement de l’instruction primaire sont ainsi devenus dans un certain parti des articles de foi politique ; mais la France a pour voisine la Prusse, qui force indirectement la France, même conservatrice, à reculer sur ces deux principes. Le parti conservateur français ne s’est pas trompe en prenant le deuil le jour de la bataille de Sadowa. Ce parti avait pour maxime de calquer l’Autriche des Metternich, je veux dire de combattre l’esprit démocratique au moyen d’une armée disciplinée à part, d’un peuple de paysans tenus soigneusement dans l’ignorance, d’un clergé armé de puissants concordats. Ce régime énerve trop une nation qui doit lutter contre des rivaux. L’Autriche elle-même a dû y renoncer. C’est ainsi que, selon la thèse de Plutarque, le peuple le plus vertueux l’emporte toujours sur celui qui l’est moins, et que l’émulation des nations est la condition du progrès général. Si la Prusse réussit à échapper à la démocratie socialiste, il est possible qu’elle fournisse pendant une ou deux générations une protection à la liberté et à la propriété. Sans nul doute, les classes menacées par le socialisme feraient taire leurs antipathies patriotiques, le jour où elles ne pourraient plus tenir tête au flot montant, et où quelque État fort prendrait pour mission de maintenir l’ordre social européen. D’un autre côte, l’Allemagne trouverait dans l’accomplissement d’une telle œuvre (assez analogue à celle qu’elle exécuta au Ve siècle ) des emplois si avantageux de son activité, que le socialisme serait chez elle écarté pour longtemps. Riche, molle, peu laborieuse, la France se laissait aller depuis des années à faire exécuter toutes ses besognes pénibles, exigeant de l’application, par des étrangers qu’elle payait bien pour cela ; le gouvernement, en tant qu’il se confond avec le métier de gendarme, est à quelques égards une de ces besognes ennuyeuses pour lesquelles le Français, bon et faible, a peu d’aptitude ; le jour se laisse entrevoir où il payera des gens rogues, sérieux et durs pour cela, comme les Athéniens avaient des Scythes pour remplir les fonctions de sbires et de geôliers.\par
La gravité de la crise révélera peut-être des forces inconnues. L’imprévu est grand dans les choses humaines, et la France se plaît souvent à déjouer les calculs les mieux raisonnes. Étrange, parfois lamentable, la destinée de notre pays n’est jamais vulgaire. S’il est vrai que c’est le patriotisme français qui, à la fin du dernier siècle, a réveillé le patriotisme allemand, il sera peut-être vrai aussi de dire que le patriotisme allemand aura réveillé le patriotisme français sur le point de s’éteindre. Ce retour vers les questions nationales apporterait pour quelques années un temps d’arrêt aux, questions sociales. Ce qui s’est passe depuis trois mois, la vitalité que la France a montrée après l’effroyable syncope morale du 18 mars, sont des faits très consolants. On se prend souvent à craindre que la France et même l’Angleterre, au fond travaillée du même mal que nous (l’affaiblissement de l’esprit militaire, la prédominance des considérations commerciales et industrielles ), ne soient bientôt réduites à un rôle secondaire, et que la scène du monde européen n’en vienne a être uniquement occupée par deux colosses, la race germanique et la race slave, qui ont gardé la vigueur du principe militaire et monarchique, et dont la lutte remplira l’avenir. Mais on peut affirmer aussi que, dans un sens supérieur, la France aura sa revanche. On reconnaîtra un jour qu’elle était le sel de la terre, et que sans elle le festin de ce monde sera peu savoureux. On regrettera cette vieille France libérale, qui fut impuissante, imprudente, je l’avoue, mais qui aussi fut généreuse, et dont on dira un jour comme des chevaliers de l’Arioste :\par

On gran bontà de’ cavalieri antiqui !\\

\noindent Quand les vainqueurs du jour auront réussi à rendre le monde positif, égoïste, étranger à tout autre mobile que l’intérêt, aussi peu sentimental que possible, on trouvera qu’il fut heureux cependant pour l’Amérique que le marquis de Lafayette ait pensé autrement ; qu’il fut heureux pour l’Italie que, même à notre plus triste époque, nous ayons été capables d’une généreuse folie ; qu’il fut heureux pour la Prusse qu’en 1865, aux plans confus qui remplissaient la tête de l’empereur, se soit mêlée une vue de philosophie politique élevée.\par
Ne jamais trop espérer, ne jamais désespérer, doit être notre devise. Souvenons-nous que la tristesse seule est féconde en grandes choses, et que le vrai moyen de relever notre pauvre pays, c’est de lui montrer l’abîme où il est. Souvenons-nous surtout que les droits de la patrie sont imprescriptibles, et que le peu de cas qu’elle fait de nos conseils ne nous dispense pas de les lui donner. L’émigration à l’extérieur ou à l’intérieur est la plus mauvaise action qu’on puisse commettre. L’empereur romain qui, au moment de mourir, résumait son opinion sur la vie par ces mots : {\itshape Nil expedit}, n’en donnait pas moins pour mot d’ordre à ses officiers : {\itshape Laboremus.}
 


% at least one empty page at end (for booklet couv)
\ifbooklet
  \pagestyle{empty}
  \clearpage
  % 2 empty pages maybe needed for 4e cover
  \ifnum\modulo{\value{page}}{4}=0 \hbox{}\newpage\hbox{}\newpage\fi
  \ifnum\modulo{\value{page}}{4}=1 \hbox{}\newpage\hbox{}\newpage\fi


  \hbox{}\newpage
  \ifodd\value{page}\hbox{}\newpage\fi
  {\centering\color{rubric}\bfseries\noindent\large
    Hurlus ? Qu’est-ce.\par
    \bigskip
  }
  \noindent Des bouquinistes électroniques, pour du texte libre à participation libre,
  téléchargeable gratuitement sur \href{https://hurlus.fr}{\dotuline{hurlus.fr}}.\par
  \bigskip
  \noindent Cette brochure a été produite par des éditeurs bénévoles.
  Elle n’est pas faîte pour être possédée, mais pour être lue, et puis donnée.
  Que circule le texte !
  En page de garde, on peut ajouter une date, un lieu, un nom ; pour suivre le voyage des idées.
  \par

  Ce texte a été choisi parce qu’une personne l’a aimé,
  ou haï, elle a en tous cas pensé qu’il partipait à la formation de notre présent ;
  sans le souci de plaire, vendre, ou militer pour une cause.
  \par

  L’édition électronique est soigneuse, tant sur la technique
  que sur l’établissement du texte ; mais sans aucune prétention scolaire, au contraire.
  Le but est de s’adresser à tous, sans distinction de science ou de diplôme.
  Au plus direct ! (possible)
  \par

  Cet exemplaire en papier a été tiré sur une imprimante personnelle
   ou une photocopieuse. Tout le monde peut le faire.
  Il suffit de
  télécharger un fichier sur \href{https://hurlus.fr}{\dotuline{hurlus.fr}},
  d’imprimer, et agrafer ; puis de lire et donner.\par

  \bigskip

  \noindent PS : Les hurlus furent aussi des rebelles protestants qui cassaient les statues dans les églises catholiques. En 1566 démarra la révolte des gueux dans le pays de Lille. L’insurrection enflamma la région jusqu’à Anvers où les gueux de mer bloquèrent les bateaux espagnols.
  Ce fut une rare guerre de libération dont naquit un pays toujours libre : les Pays-Bas.
  En plat pays francophone, par contre, restèrent des bandes de huguenots, les hurlus, progressivement réprimés par la très catholique Espagne.
  Cette mémoire d’une défaite est éteinte, rallumons-la. Sortons les livres du culte universitaire, cherchons les idoles de l’époque, pour les briser.
\fi

\ifdev % autotext in dev mode
\fontname\font — \textsc{Les règles du jeu}\par
(\hyperref[utopie]{\underline{Lien}})\par
\noindent \initialiv{A}{lors là}\blindtext\par
\noindent \initialiv{À}{ la bonheur des dames}\blindtext\par
\noindent \initialiv{É}{tonnez-le}\blindtext\par
\noindent \initialiv{Q}{ualitativement}\blindtext\par
\noindent \initialiv{V}{aloriser}\blindtext\par
\Blindtext
\phantomsection
\label{utopie}
\Blinddocument
\fi
\end{document}
