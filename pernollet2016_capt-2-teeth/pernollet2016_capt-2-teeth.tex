%%%%%%%%%%%%%%%%%%%%%%%%%%%%%%%%%
% LaTeX model https://hurlus.fr %
%%%%%%%%%%%%%%%%%%%%%%%%%%%%%%%%%

% Needed before document class
\RequirePackage{pdftexcmds} % needed for tests expressions
\RequirePackage{fix-cm} % correct units

% Define mode
\def\mode{a4}

\newif\ifaiv % a4
\newif\ifav % a5
\newif\ifbooklet % booklet
\newif\ifcover % cover for booklet

\ifnum \strcmp{\mode}{cover}=0
  \covertrue
\else\ifnum \strcmp{\mode}{booklet}=0
  \booklettrue
\else\ifnum \strcmp{\mode}{a5}=0
  \avtrue
\else
  \aivtrue
\fi\fi\fi

\ifbooklet % do not enclose with {}
  \documentclass[twoside]{book} % ,notitlepage
  \usepackage[%
    papersize={105mm, 297mm},
    inner=12mm,
    outer=12mm,
    top=20mm,
    bottom=15mm,
    marginparsep=3pt,
    marginpar=7mm,
  ]{geometry}
  \usepackage[fontsize=9.5pt]{scrextend} % for Roboto
\else\ifav % A5
  \documentclass[twoside]{book} % ,notitlepage
  \usepackage[%
    a5paper
  ]{geometry}
  \usepackage[fontsize=12pt]{scrextend}
\else% A4 2 cols
  \documentclass[twocolumn]{report}
  \usepackage[%
    a4paper,
    inner=15mm,
    outer=10mm,
    top=25mm,
    bottom=18mm,
    marginparsep=0pt,
  ]{geometry}
  \setlength{\columnsep}{20mm}
  \usepackage[fontsize=9.5pt]{scrextend}
\fi\fi

%%%%%%%%%%%%%%
% Alignments %
%%%%%%%%%%%%%%
% before teinte macros

\setlength{\arrayrulewidth}{0.2pt}
\setlength{\columnseprule}{\arrayrulewidth} % twocol

%%%%%%%%%%
% Colors %
%%%%%%%%%%
% before Teinte macros

\usepackage[dvipsnames]{xcolor}
\definecolor{rubric}{HTML}{0c71c3} % the tonic
\def\columnseprulecolor{\color{rubric}}
\colorlet{borderline}{rubric!30!} % definecolor need exact code
\definecolor{shadecolor}{gray}{0.95}
\definecolor{bghi}{gray}{0.5}

%%%%%%%%%%%%%%%%%
% Teinte macros %
%%%%%%%%%%%%%%%%%
%%%%%%%%%%%%%%%%%%%%%%%%%%%%%%%%%%%%%%%%%%%%%%%%%%%
% <TEI> generic (LaTeX names generated by Teinte) %
%%%%%%%%%%%%%%%%%%%%%%%%%%%%%%%%%%%%%%%%%%%%%%%%%%%
% This template is inserted in a specific design
% It is XeLaTeX and otf fonts

\makeatletter % <@@@

\setlength{\parskip}{0pt} % 1pt allow better vertical justification
\setlength{\parindent}{1.5em}

\usepackage{alphalph} % for alph couter z, aa, ab…
\usepackage{blindtext} % generate text for testing
\usepackage{booktabs} % for tables: \toprule, \midrule…
\usepackage[strict]{changepage} % for modulo 4
\usepackage{contour} % rounding words
\usepackage[nodayofweek]{datetime}
\usepackage{enumitem} % <list>
\usepackage{etoolbox} % patch commands
\usepackage{fancyvrb}
\usepackage{fancyhdr}
\usepackage{float}
\usepackage{fontspec} % XeLaTeX mandatory for fonts
\usepackage{footnote} % used to capture notes in minipage (ex: quote)
\usepackage{graphicx}
\usepackage{lettrine} % drop caps
\usepackage{lipsum} % generate text for testing
\usepackage{relsize} % \smaller \larger (ex: quotes in body and footnotes)
\usepackage{manyfoot} % for parallel footnote numerotation
\usepackage[framemethod=tikz,]{mdframed} % maybe used for frame with footnotes inside
\usepackage[defaultlines=2,all]{nowidow} % at least 2 lines by par (works well!)
\usepackage{pdftexcmds} % needed for tests expressions
\usepackage{poetry} % <l>, bad for theater
\usepackage{polyglossia} % bug Warning: "Failed to patch part"
\usepackage[%
  indentfirst=false,
  vskip=1em,
  noorphanfirst=true,
  noorphanafter=true,
  leftmargin=\parindent,
  rightmargin=0pt,
]{quoting}
\usepackage{ragged2e}
\usepackage{setspace} % \setstretch for <quote>
\usepackage{scrextend} % KOMA-common, used for addmargin
\usepackage{tabularx} % <table>
\usepackage[explicit]{titlesec} % wear titles, !NO implicit
\usepackage{tikz} % ornaments
\usepackage{tocloft} % styling tocs
\usepackage[fit]{truncate} % used im runing titles
\usepackage{unicode-math}
\usepackage[normalem]{ulem} % breakable \uline, normalem is absolutely necessary to keep \emph
\usepackage{xcolor} % named colors
\usepackage{xparse} % @ifundefined
\XeTeXdefaultencoding "iso-8859-1" % bad encoding of xstring
\usepackage{xstring} % string tests
\XeTeXdefaultencoding "utf-8"

\defaultfontfeatures{
  % Mapping=tex-text, % no effect seen
  Scale=MatchLowercase,
  Ligatures={TeX,Common},
}
\newfontfamily\zhfont{Noto Sans CJK SC}

% Metadata inserted by a program, from the TEI source, for title page and runing heads
\title{Captain Two-Teeth et le Chat Signorita\par
\medskip
\emph{Histoire courte}}
\date{2016}
\author{Richard Pernollet}
\def\elbibl{Richard Pernollet. 2016. \emph{Captain Two-Teeth et le Chat Signorita}}
\def\elabstract{%
 
\labelblock{Chapeau de l’auteur}

 \noindent Au sud de la Thaïlande, une île va disparaître, submergée par les eaux.\par
 Une vieille femme fait un dernier voyage pour mourir avec elle.\par
 Le Capitaine Two-Teeth la conduit une nouvelle fois.\par
 Et, le mal de mer aidant, le passé revient.\par
 — « Son île ! » – de quel droit pouvait-elle dire « son île » ?\par
 
\labelblock{Avertissement de l’édition}

 \noindent L’auteur étant bien vivant, ce texte est protégé par le droit d’auteur. Il a exprimé l’intention que l’“œuvre“ de son esprit (ce sont les termes de la loi, pas du tout les siens) soit librement communicable et partageable, mais sans profit commercial, en gardant le nom de l’auteur, et sans modification. Le choix de cette \href{https://creativecommons.org/licenses/by-nc-nd/4.0/deed.fr}{\dotuline{license CC BY-NC-ND}}\footnote{\href{https://creativecommons.org/licenses/by-nc-nd/4.0/deed.fr}{\url{https://creativecommons.org/licenses/by-nc-nd/4.0/deed.fr}}} n’est pas seulement l’effet d’opinions politiques, elle est aussi conséquente avec son esthétique. Son texte a été longtemps mûri, puis jeté tout d’un trait, il témoigne d’un moment, signé, daté. Ce texte est un morceau de temps libre qui ne peut plus être modifié ; de son temps à lui et personne d’autre. Ce temps, il ne le vend pas, il le partage.
 \vfill

}
\def\elsource{ \href{https://labarresymbolique.files.wordpress.com/2016/10/capt-2-teeth-sept2016-new3.pdf}{\dotuline{LaBarreSymbolique}}\footnote{\href{https://labarresymbolique.files.wordpress.com/2016/10/capt-2-teeth-sept2016-new3.pdf}{\url{https://labarresymbolique.files.wordpress.com/2016/10/capt-2-teeth-sept2016-new3.pdf}}} }
\def\eltitlepage{%
{\centering\parindent0pt
  {\LARGE\addfontfeature{LetterSpace=25}\bfseries Richard Pernollet\par}\bigskip
  {\Large 2016\par}\bigskip
  {\LARGE
\bigskip\textbf{Captain Two-Teeth et le Chat Signorita}\par
\bigskip\emph{Histoire courte}\par

  }
}

}

% Default metas
\newcommand{\colorprovide}[2]{\@ifundefinedcolor{#1}{\colorlet{#1}{#2}}{}}
\colorprovide{rubric}{red}
\colorprovide{silver}{lightgray}
\@ifundefined{syms}{\newfontfamily\syms{DejaVu Sans}}{}
\newif\ifdev
\@ifundefined{elbibl}{% No meta defined, maybe dev mode
  \newcommand{\elbibl}{Titre court ?}
  \newcommand{\elbook}{Titre du livre source ?}
  \newcommand{\elabstract}{Résumé\par}
  \newcommand{\elurl}{http://oeuvres.github.io/elbook/2}
  \author{Éric Lœchien}
  \title{Un titre de test assez long pour vérifier le comportement d’une maquette}
  \date{1566}
  \devtrue
}{}
\let\eltitle\@title
\let\elauthor\@author
\let\eldate\@date




% generic typo commands
\newcommand{\astermono}{\medskip\centerline{\color{rubric}\large\selectfont{\syms ✻}}\medskip\par}%
\newcommand{\astertri}{\medskip\par\centerline{\color{rubric}\large\selectfont{\syms ✻\,✻\,✻}}\medskip\par}%
\newcommand{\asterism}{\bigskip\par\noindent\parbox{\linewidth}{\centering\color{rubric}\large{\syms ✻}\\{\syms ✻}\hskip 0.75em{\syms ✻}}\bigskip\par}%

% lists
\newlength{\listmod}
\setlength{\listmod}{\parindent}
\setlist{
  itemindent=!,
  listparindent=\listmod,
  labelsep=0.2\listmod,
  parsep=0pt,
  % topsep=0.2em, % default topsep is best
}
\setlist[itemize]{
  label=—,
  leftmargin=0pt,
  labelindent=1.2em,
  labelwidth=0pt,
}
\setlist[enumerate]{
  label={\arabic*°},
  labelindent=0.8\listmod,
  leftmargin=\listmod,
  labelwidth=0pt,
}
% list for big items
\newlist{decbig}{enumerate}{1}
\setlist[decbig]{
  label={\bf\color{rubric}\arabic*.},
  labelindent=0.8\listmod,
  leftmargin=\listmod,
  labelwidth=0pt,
}
\newlist{listalpha}{enumerate}{1}
\setlist[listalpha]{
  label={\bf\color{rubric}\alph*.},
  leftmargin=0pt,
  labelindent=0.8\listmod,
  labelwidth=0pt,
}
\newcommand{\listhead}[1]{\hspace{-1\listmod}\emph{#1}}

\renewcommand{\hrulefill}{%
  \leavevmode\leaders\hrule height 0.2pt\hfill\kern\z@}

% General typo
\DeclareTextFontCommand{\textlarge}{\large}
\DeclareTextFontCommand{\textsmall}{\small}

% commands, inlines
\newcommand{\anchor}[1]{\Hy@raisedlink{\hypertarget{#1}{}}} % link to top of an anchor (not baseline)
\newcommand\abbr[1]{#1}
\newcommand{\autour}[1]{\tikz[baseline=(X.base)]\node [draw=rubric,thin,rectangle,inner sep=1.5pt, rounded corners=3pt] (X) {\color{rubric}#1};}
\newcommand\corr[1]{#1}
\newcommand{\ed}[1]{ {\color{silver}\sffamily\footnotesize (#1)} } % <milestone ed="1688"/>
\newcommand\expan[1]{#1}
\newcommand\foreign[1]{\emph{#1}}
\newcommand\gap[1]{#1}
\renewcommand{\LettrineFontHook}{\color{rubric}}
\newcommand{\initial}[2]{\lettrine[lines=2, loversize=0.3, lhang=0.3]{#1}{#2}}
\newcommand{\initialiv}[2]{%
  \let\oldLFH\LettrineFontHook
  % \renewcommand{\LettrineFontHook}{\color{rubric}\ttfamily}
  \IfSubStr{QJ’}{#1}{
    \lettrine[lines=4, lhang=0.2, loversize=-0.1, lraise=0.2]{\smash{#1}}{#2}
  }{\IfSubStr{É}{#1}{
    \lettrine[lines=4, lhang=0.2, loversize=-0, lraise=0]{\smash{#1}}{#2}
  }{\IfSubStr{ÀÂ}{#1}{
    \lettrine[lines=4, lhang=0.2, loversize=-0, lraise=0, slope=0.6em]{\smash{#1}}{#2}
  }{\IfSubStr{A}{#1}{
    \lettrine[lines=4, lhang=0.2, loversize=0.2, slope=0.6em]{\smash{#1}}{#2}
  }{\IfSubStr{V}{#1}{
    \lettrine[lines=4, lhang=0.2, loversize=0.2, slope=-0.5em]{\smash{#1}}{#2}
  }{
    \lettrine[lines=4, lhang=0.2, loversize=0.2]{\smash{#1}}{#2}
  }}}}}
  \let\LettrineFontHook\oldLFH
}
\newcommand{\labelchar}[1]{\textbf{\color{rubric} #1}}
\newcommand{\lnatt}[1]{\reversemarginpar\marginpar[\sffamily\scriptsize #1]{}}
\newcommand{\milestone}[1]{\autour{\footnotesize\color{rubric} #1}} % <milestone n="4"/>
\newcommand\name[1]{#1}
\newcommand\orig[1]{#1}
\newcommand\orgName[1]{#1}
\newcommand\persName[1]{#1}
\newcommand\placeName[1]{#1}
\newcommand{\pn}[1]{\IfSubStr{-—–¶}{#1}% <p n="3"/>
  {\noindent{\bfseries\color{rubric}   ¶  }}
  {{\footnotesize\autour{#1}}}}
\newcommand\reg{}
% \newcommand\ref{} % already defined
\newcommand\sic[1]{#1}
\newcommand\surname[1]{\textsc{#1}}
\newcommand\term[1]{\textbf{#1}}
\newcommand\zh[1]{{\zhfont #1}}


\def\mednobreak{\ifdim\lastskip<\medskipamount
  \removelastskip\nopagebreak\medskip\fi}
\def\bignobreak{\ifdim\lastskip<\bigskipamount
  \removelastskip\nopagebreak\bigskip\fi}

% commands, blocks

\newcommand{\byline}[1]{\bigskip{\RaggedLeft{#1}\par}\bigskip}
% \setlength{\RaggedLeftLeftskip}{2em plus \leftskip}
\newcommand{\bibl}[1]{{\RaggedLeft\normalfont #1\par}}
\newcommand{\biblitem}[1]{{\noindent\hangindent=\parindent   #1\par}}
\newcommand{\castItem}[1]{{\noindent\hangindent=\parindent #1\par}}
\newcommand{\dateline}[1]{\medskip{\RaggedLeft{#1}\par}\bigskip}
\newcommand{\docAuthor}[1]{{\large\bigskip #1 \par\bigskip}}
\newcommand{\docDate}[1]{#1 \ifvmode\par\fi }
\newcommand{\docImprint}[1]{\ifvmode\medskip\fi #1 \ifvmode\par\fi }
\newcommand{\labelblock}[1]{\medbreak{\noindent\color{rubric}\bfseries #1}\par\mednobreak}
\newcommand{\question}[1]{\bigbreak{\RaggedRight\noindent\emph{#1}\par}\mednobreak}
\newcommand{\salute}[1]{\bigbreak{#1}\par\medbreak}
\newcommand{\signed}[1]{\medskip{\RaggedLeft #1\par}\bigbreak} % supposed bottom
\newcommand{\speaker}[1]{\medskip{\Centering\sffamily #1 \par\nopagebreak}} % supposed bottom
\newcommand{\stagescene}[1]{{\Centering\sffamily\textsf{#1}\par}\bigskip}
\newcommand{\stageblock}[1]{\begingroup\leftskip\parindent\noindent\it\sffamily\footnotesize #1\par\endgroup} % left margin, better than list envs
\newcommand{\lpar}[1]{\noindent\hangindent=2\parindent  #1\par} % sp/l
\newcommand{\trailer}[1]{{\Centering\bigskip #1\par}} % sp/l

% environments for blocks (some may become commands)
\newenvironment{borderbox}{}{} % framing content
\newenvironment{citbibl}{\ifvmode\hfill\fi}{\ifvmode\par\fi }
\newenvironment{msHead}{\vskip6pt}{\par}
\newenvironment{msItem}{\vskip6pt}{\par}


% environments for block containers
\newenvironment{argument}{\itshape\parindent0pt}{\bigskip}
\newenvironment{biblfree}{}{\ifvmode\par\fi }
\newenvironment{bibitemlist}[1]{%
  \list{\@biblabel{\@arabic\c@enumiv}}%
  {%
    \settowidth\labelwidth{\@biblabel{#1}}%
    \leftmargin\labelwidth
    \advance\leftmargin\labelsep
    \@openbib@code
    \usecounter{enumiv}%
    \let\p@enumiv\@empty
    \renewcommand\theenumiv{\@arabic\c@enumiv}%
  }
  \sloppy
  \clubpenalty4000
  \@clubpenalty \clubpenalty
  \widowpenalty4000%
  \sfcode`\.\@m
}%
{\def\@noitemerr
  {\@latex@warning{Empty `bibitemlist' environment}}%
\endlist}
\newenvironment{docTitle}{\LARGE\bigskip\bfseries\onehalfspacing}{\bigskip}
% leftskip makes big bugs in Lexmark printing \sffamily
\newenvironment{epigraph}{\begin{addmargin}[2\parindent]{0em}\sffamily\large\setstretch{0.95}}{\end{addmargin}\bigskip}
\newenvironment{quoteblock}
  {\begin{quoting}\setstretch{0.9}} %
  {\end{quoting}}
\newenvironment{frametext}
  {\begin{mdframed}[default]} %
  {\end{mdframed}}

\quotingsetup{vskip=0pt}
\newcommand{\quoteskip}{\medskip}
\newenvironment{titlePage}
  {\Centering}
  {}






% table () is preceded and finished by custom command
\renewcommand\tabularxcolumn[1]{m{#1}}% for vertical centering text in X column
\newcommand{\tableopen}[1]{%
  \ifnum\strcmp{#1}{wide}=0{%
    \begin{center}
  }
  \else\ifnum\strcmp{#1}{long}=0{%
    \begin{center}
  }
  \else{%
    \begin{center}
  }
  \fi\fi
}
\newcommand{\tableclose}[1]{%
  \ifnum\strcmp{#1}{wide}=0{%
    \end{center}
  }
  \else\ifnum\strcmp{#1}{long}=0{%
    \end{center}
  }
  \else{%
    \end{center}
  }
  \fi\fi
}


% text structure
\newcommand\chapteropen{} % before chapter title
\newcommand\chaptercont{} % after title, argument, epigraph…
\newcommand\chapterclose{} % maybe useful for multicol settings
\setcounter{secnumdepth}{-2} % no counters for hierarchy titles
\setcounter{tocdepth}{5} % deep toc
\renewcommand\tableofcontents{\@starttoc{toc}}
% toclof format
% \renewcommand{\@tocrmarg}{0.1em} % Useless command?
% \renewcommand{\@pnumwidth}{0.5em} % {1.75em}
\renewcommand{\@cftmaketoctitle}{}
\setlength{\cftbeforesecskip}{\z@ \@plus.2\p@}

\@ifclassloaded{article}{%
  \typeout{class: article}%
}{%
  \renewcommand{\cftchapfont}{}
  \renewcommand{\cftchapdotsep}{\cftdotsep}
  \renewcommand{\cftchapleader}{\normalfont\cftdotfill{\cftchapdotsep}}
  \renewcommand{\cftchappagefont}{\bfseries}
  \setlength{\cftbeforechapskip}{0pt}
  \setlength{\cftchapnumwidth}{1em}
}
\renewcommand{\cftsecfont}{\normalfont}
\renewcommand{\cftsecpagefont}{\normalfont}
% \renewcommand{\cftsubsecfont}{\small\relax}
\renewcommand{\cftsecdotsep}{\cftdotsep}
\renewcommand{\cftsecpagefont}{\normalfont}
\renewcommand{\cftsecleader}{\normalfont\cftdotfill{\cftsecdotsep}}
\setlength{\cftsecindent}{1em}
\setlength{\cftsubsecindent}{2em}
\setlength{\cftsubsubsecindent}{3em}
\setlength{\cftsecnumwidth}{1em}
\setlength{\cftsubsecnumwidth}{1em}
\setlength{\cftsubsubsecnumwidth}{1em}

% footnotes
\newif\ifheading
\newcommand*{\fnmarkscale}{\ifheading 0.70 \else 1 \fi}
\renewcommand\footnoterule{\vspace*{0.3cm}\hrule height \arrayrulewidth width 3cm \vspace*{0.3cm}}
\setlength\footnotesep{1.5\footnotesep} % footnote separator
\renewcommand\@makefntext[1]{\parindent 1.5em \noindent \hb@xt@1.8em{\hss{\normalfont\@thefnmark . }}#1} % no superscipt in foot
\patchcmd{\@footnotetext}{\footnotesize}{\footnotesize\sffamily}{}{} % before scrextend, hyperref
\DeclareNewFootnote{A}[alph] % for editor notes
\renewcommand*{\thefootnoteA}{\alphalph{\value{footnoteA}}} % z, aa, ab…

% poem
\setlength{\poembotskip}{0pt}
\setlength{\poemtopskip}{0pt}
\setlength{\poemindent}{0pt}
\setlength{\poemmaxlinewd}{\linewidth}
\poemlinenumsfalse

%   see https://tex.stackexchange.com/a/34449/5049
\def\truncdiv#1#2{((#1-(#2-1)/2)/#2)}
\def\moduloop#1#2{(#1-\truncdiv{#1}{#2}*#2)}
\def\modulo#1#2{\number\numexpr\moduloop{#1}{#2}\relax}

% orphans and widows, nowidow package in test
% from memoir package
\clubpenalty=9996
\widowpenalty=9999
\brokenpenalty=4991
\predisplaypenalty=10000
\postdisplaypenalty=1549
\displaywidowpenalty=1602
\hyphenpenalty=400
% report h or v overfull ?
\hbadness=4000
\vbadness=4000
% good to avoid lines too wide
\emergencystretch 3em
\pretolerance=750
\tolerance=2000
\def\Gin@extensions{.pdf,.png,.jpg,.mps,.tif}

\PassOptionsToPackage{hyphens}{url} % before hyperref and biblatex, which load url package
\usepackage{hyperref} % supposed to be the last one, :o) except for the ones to follow
\hypersetup{
  % pdftex, % no effect
  pdftitle={\elbibl},
  % pdfauthor={Your name here},
  % pdfsubject={Your subject here},
  % pdfkeywords={keyword1, keyword2},
  bookmarksnumbered=true,
  bookmarksopen=true,
  bookmarksopenlevel=1,
  pdfstartview=Fit,
  breaklinks=true, % avoid long links, overrided by url package
  pdfpagemode=UseOutlines,    % pdf toc
  hyperfootnotes=true,
  colorlinks=false,
  pdfborder=0 0 0,
  % pdfpagelayout=TwoPageRight,
  % linktocpage=true, % NO, toc, link only on page no
}
\urlstyle{same} % after hyperref



\makeatother % /@@@>
%%%%%%%%%%%%%%
% </TEI> end %
%%%%%%%%%%%%%%

\setmainlanguage{french}
%%%%%%%%%%%%%
% footnotes %
%%%%%%%%%%%%%
\renewcommand{\thefootnote}{\bfseries\textcolor{rubric}{\arabic{footnote}}} % color for footnote marks

%%%%%%%%%
% Fonts %
%%%%%%%%%
% \linespread{0.90} % too compact, keep font natural
\ifav % A5
  \usepackage{DejaVuSans} % correct
  \setsansfont{DejaVuSans} % seen, if not set, problem with printer
\else\ifbooklet
  \usepackage[]{roboto} % SmallCaps, Regular is a bit bold
  \setmainfont[
    ItalicFont={Roboto Light Italic},
  ]{Roboto}
  \setsansfont{Roboto Light} % seen, if not set, problem with printer
  \newfontfamily\fontrun[]{Roboto Condensed Light} % condensed runing heads
\else
  \usepackage[]{roboto} % SmallCaps, Regular is a bit bold
  \setmainfont[
    ItalicFont={Roboto Italic},
  ]{Roboto Light}
  \setsansfont{Roboto Light} % seen, if not set, problem with printer
  \newfontfamily\fontrun[]{Roboto Condensed Light} % condensed runing heads
\fi\fi
\renewcommand{\LettrineFontHook}{\bfseries\color{rubric}}
% \renewenvironment{labelblock}{\begin{center}\bfseries\color{rubric}}{\end{center}}

%%%%%%%%
% MISC %
%%%%%%%%

\setdefaultlanguage[frenchpart=false]{french} % bug on part


\newenvironment{quotebar}{%
    \def\FrameCommand{{\color{rubric!10!}\vrule width 0.5em} \hspace{0.9em}}%
    \def\OuterFrameSep{0pt} % séparateur vertical
    \MakeFramed {\advance\hsize-\width \FrameRestore}
  }%
  {%
    \endMakeFramed
  }
\renewenvironment{quoteblock}% may be used for ornaments
  {%
    \savenotes
    \setstretch{0.9}
    \begin{quotebar}
    \smallskip
  }
  {%
    \smallskip
    \end{quotebar}
    \spewnotes
  }


\renewcommand{\headrulewidth}{\arrayrulewidth}
\renewcommand{\headrule}{{\color{rubric}\hrule}}
\renewcommand{\lnatt}[1]{\marginpar{\sffamily\scriptsize #1}}

\titleformat{name=\chapter} % command
  [display] % shape
  {\vspace{1.5em}\centering} % format
  {} % label
  {0pt} % separator between n
  {}
[{\color{rubric}\huge\textbf{#1}}\bigskip] % after code
% \titlespacing{command}{left spacing}{before spacing}{after spacing}[right]
\titlespacing*{\chapter}{0pt}{-2em}{0pt}[0pt]

\titleformat{name=\section}
  [display]{}{}{}{}
  [\vbox{\color{rubric}\large\centering\textbf{#1}}]
\titlespacing{\section}{0pt}{0pt plus 4pt minus 2pt}{\baselineskip}

\titleformat{name=\subsection}
  [block]
  {}
  {} % \thesection
  {} % separator \arrayrulewidth
  {}
[\vbox{\large\textbf{#1}}]
% \titlespacing{\subsection}{0pt}{0pt plus 4pt minus 2pt}{\baselineskip}

\ifaiv
  \fancypagestyle{main}{%
    \fancyhf{}
    \setlength{\headheight}{1.5em}
    \fancyhead{} % reset head
    \fancyfoot{} % reset foot
    \fancyhead[L]{\truncate{0.45\headwidth}{\fontrun\elbibl}} % book ref
    \fancyhead[R]{\truncate{0.45\headwidth}{ \fontrun\nouppercase\leftmark}} % Chapter title
    \fancyhead[C]{\thepage}
  }
  \fancypagestyle{plain}{% apply to chapter
    \fancyhf{}% clear all header and footer fields
    \setlength{\headheight}{1.5em}
    \fancyhead[L]{\truncate{0.9\headwidth}{\fontrun\elbibl}}
    \fancyhead[R]{\thepage}
  }
\else
  \fancypagestyle{main}{%
    \fancyhf{}
    \setlength{\headheight}{1.5em}
    \fancyhead{} % reset head
    \fancyfoot{} % reset foot
    \fancyhead[RE]{\truncate{0.9\headwidth}{\fontrun\elbibl}} % book ref
    \fancyhead[LO]{\truncate{0.9\headwidth}{\fontrun\nouppercase\leftmark}} % Chapter title, \nouppercase needed
    \fancyhead[RO,LE]{\thepage}
  }
  \fancypagestyle{plain}{% apply to chapter
    \fancyhf{}% clear all header and footer fields
    \setlength{\headheight}{1.5em}
    \fancyhead[L]{\truncate{0.9\headwidth}{\fontrun\elbibl}}
    \fancyhead[R]{\thepage}
  }
\fi

\ifav % a5 only
  \titleclass{\section}{top}
\fi

\newcommand\chapo{{%
  \vspace*{-3em}
  \centering\parindent0pt % no vskip ()
  \eltitlepage
  \bigskip
  {\color{rubric}\hline}
  \bigskip
  {\Large TEXTE LIBRE À PARTICIPATIONS LIBRES\par}
  \centerline{\small\color{rubric} {\href{https://hurlus.fr}{\dotuline{hurlus.fr}}}, tiré le \today}\par
  \bigskip
}}

\newcommand\cover{{%
  \thispagestyle{empty}
  \centering\parindent0pt
  \eltitlepage
  \vfill\null
  {\color{rubric}\setlength{\arrayrulewidth}{2pt}\hline}
  \vfill\null
  {\Large TEXTE LIBRE À PARTICIPATIONS LIBRES\par}
  \centerline{\href{https://hurlus.fr}{\dotuline{hurlus.fr}}, tiré le \today}\par
}}

\begin{document}
\pagestyle{empty}
\ifbooklet{
  \cover\newpage
  \thispagestyle{empty}\hbox{}\newpage
  \cover\newpage\noindent Les voyages de la brochure\par
  \bigskip
  \begin{tabularx}{\textwidth}{l|X|X}
    \textbf{Date} & \textbf{Lieu}& \textbf{Nom/pseudo} \\ \hline
    \rule{0pt}{25cm} &  &   \\
  \end{tabularx}
  \newpage
  \addtocounter{page}{-4}
}\fi

\thispagestyle{empty}
\ifaiv
  \twocolumn[\chapo]
\else
  \chapo
\fi
{\it\elabstract}
\bigskip
\makeatletter\@starttoc{toc}\makeatother % toc without new page
\bigskip

\pagestyle{main} % after style
\setcounter{footnote}{0}
\setcounter{footnoteA}{0}
  
\section[{— 1 —}]{— 1 —}
\renewcommand{\leftmark}{— 1 —}

\noindent Elle savait qu’elle venait ici pour mourir, mais il n’y avait pas de bateaux, où étaient passés les bateaux ? Au bout du ponton la tête du capitaine Two-Teeth dansait. Un coup en haut, un coup en bas, il lui faisait signe. \emph{« Madame Élisa… Madame Élisa ! ».} Le bras levé disparaissant aussitôt comme s’il eût été monté sur ressorts.\par
À l’horizon le ciel était noir, plombé, et le soleil, caché derrière les nuages, dessinait une lame d’acier brillante juste au-dessus de la mer. Un clapot claque à ce moment-là sous le ponton faisant surgir entre deux planches un crachat d’écume qui lui arrose les pieds. \emph{Merde} ! Elle recule.\par
Tâte le ponton avec sa canne.\par
{\itshape Penser à ma cheville !\par}
Et, oubliant sa cheville, elle s’élance évitant les planches mouillées.\par
Arrivée au bout du ponton, elle comprend ce qui fait danser le Capitaine Two-Teeth. Il est là debout sous elle, torse nu, dans une barque ballottée par les vagues, et se tient à l’échelle pour l’empêcher de cogner.\par
C’est un bateau à longue queue comme on les appelle ici, c’est-à-dire à fond plat et muni d’un moteur dont l’hélice plonge au bout d’un long tube à deux ou trois mètres à l’arrière. L’ensemble, mobile, sert de gouvernail.\par
— Bonjour Captain, dit-elle.\par
— Bonjour Élisa, ça va ? dit-il avec son éternel sourire à deux dents, Désolé, peux pas l’attacher, trop de vagues, ça va ? Fait bon voyage ? C’est pas beau le temps ici, hein ? L’île, là-bas, elle s’enfonce.\par
Comment faisait-il pour rester aussi jeune ? se demanda-t-elle en le voyant sous elle lui sembler toujours le même malgré toutes ces années, gitan des mers, la peau recuite et les yeux bridés, tout mince, musclé, presque encore le corps d’un adolescent.\par
— Mm, fit-elle, j’ai vu ça à la télé, c’est pour ça que je viens. Ça me fait plaisir de vous revoir. Vous z'avez pas pu l’attacher ?\par
\emph{— Quoi, la barque} ? Impossible. Mer démontée. Ça va ?\par
— Ça va, juste un peu, c’est un peu dur pour arriver jusqu’ici, ça souffle. Quel temps ! Je vous tends ma canne.\par
— Vous voulez que je vous aide ?\par
— Non merci, je me sens encore capable de descendre toute seule !\par
Il réceptionne la canne et la lance au fond de la barque puis l’attend en lui tendant les bras. Elle descend. Ses grosses fesses en arrière. Elle sent le vent. Il la réceptionne juste au-dessus de la taille et l’aide à passer sur la barque et l’installe au milieu ; il rétablit l’équilibre, se dirige à l’arrière, écarte la barque de l’échelle d’un coup de pied. Au même moment surgit du fond de l’horizon une noria d’hélicoptères qui se dirigent vers eux, grossissent et leur passent au-dessus de la tête dans un fracas infernal de pales et rotors, et s’éloignent. Il explique : « Ils ramènent les gens de l’île, dit-il, vous savez, c’est l’évacuation, il y a que les vieux qui veulent rester là-bas, tout le monde fuit. Vous ne verrez pas votre copain Loum, il a fui aussi. Vous êtes sûre que vous voulez aller là-bas ?\par
— Absolument sûre ! » dit-elle en se retournant vers lui.\par
Debout à l’arrière de la barque, le gouvernail sous le bras, il met les gaz et se lance dans un demi-tour pour attaquer les vagues de biais. Derrière, l’hélice creuse son sillage. Il explique. La lagune s’est ensablée. C’est à cause de l’île là-bas qui s’enfonce. Le sable vient par là, amené par les courants. Ils n’auront bientôt plus rien là-bas. C’est pourquoi elle s’enfonce. Elle va disparaître.\par
Elle approuve. Elle l’a vu à la télé. C’est pour ça qu’elle revient.

\section[{— 2 —}]{— 2 —}
\renewcommand{\leftmark}{— 2 —}

\noindent Je veux mourir sur mon île, mm, mon île ? je dis mon île ? tu te prends pour qui, ma vieille ? Ton île. Pfff… ! « On peut plus mouiller ici, dit le Capitaine Two-Teeth, impossible, tous les bateaux sont dans une anse à côté, on y va. J’ai changé de bateau. Il est plus petit. Il y a moins de touristes ! »\par
C’est normal. La mer monte. Il n’y a plus de plage.\par
Elle regarde la rive. Le paysage défile. Les bungalows sont fermés, restos, bars, brasseries ont le rideau baissé. Un coup de vent, et tous les palmiers s’ébouriffent du même côté.\par
Le bateau monte, descend, de temps en temps le moteur s’emballe au sommet d’une vague, l’hélice sort de l’eau et ensuite c’est la glissade. Devant, la prochaine vague avance déjà comme une montagne.\par
Décontracte, imperturbable, la barre sous le bras, et forçant pour contrer les courants, le Capitaine Two-Teeth dit : « Vous ne verrez pas Loum, je vous l’ai dit, il est parti, mais son frère Ahmed est là, il veut rester. Les Anciens se sont installés en cercle sur la place du village. Ils attendent. Ils attendent la pleine lune de mai. Vous savez, on arrive à la pleine lune de mai, c’est dans six jours. Pour eux, ce sera leur dernière pleine lune ! »\par
Une vague. Le bateau plonge. « Je ne sais pas si je pourrais vous attendre, dit-il, le temps se gâte, ils annoncent un renforcement de la dépression. Je crains pour le retour…\par
— Ne vous inquiétez pas, mon intention est de rester là-bas.\par
— Vous en êtes sûre ?\par
— Absolument sûre ! »

\section[{— 3 —}]{— 3 —}
\renewcommand{\leftmark}{— 3 —}

\noindent Dormir. Le mal de mer me fait dormir. La pointe du bateau monte, descend et ça roule en plus, la mer est croisée. De temps en temps la proue s’enfonce dans un grand éclat de vague et une gerbe d’écume submerge le pont. Jusque-là il n’y en a pas une qui est venue jusqu’à mes pieds. Je me suis assise sous la cabine du Capitaine Two-Teeth. A l’avant. Pour prendre l’air. Car derrière, sous les bâches, ça pue vraiment trop le gasoil. Se souvenir de l’Aber Vrac’h, en Bretagne. Rémi le marin breton te dit : si tu veux pas avoir le mal de mer, ma vieille, regarde la ligne d’horizon, fixe tout ton regard sur elle, c’est la seule chose qui bouge pas en mer : la ligne d’horizon ! Tout bouge autour, mais elle, elle bouge pas. Regarde. Et quand ton œil se focalise dessus, il dit à ton estomac que, hé, dehors, non, ça ne tangue pas, c’est juste une mauvaise impression ; tout est stable dehors, en fait. Hein, tu sens que ça bouge ? Non ? Alors il n’y a pas de raison de s’inquiéter.\par
Se souvenir de la ligne d’horizon.\par
Ce Rémi breton savait tout. Son père avait été marin-pêcheur. Adolescent, il avait travaillé avec lui. La mer, la mer, oui, c’est beau, mais c’est dur. Tu te les cailles. Et t’es tout le temps mouillé, les mains dans l’eau. Des fois, quand je revenais de ma perm’ à terre, dès que je remontais sur le bateau, avec le gasoil, les odeurs, le mouvement, beurk, je me sentais pas bien. Mais bon tu t’amarines vite. Un coup ou deux à donner à manger aux poissons et puis la ligne d’horizon, la ligne d’horizon, tu regardes la ligne d’horizon, après ça se tasse, et tu ne gerbes plus ! Il connaissait. Pas envie de suivre de toute façon. Un bateau c’est trop cher. Bref, mon père est mort en mer, harponné par un cargo ou un sous-marin, on n’a jamais su, on n’a jamais su qui a tiré le filet qui les a envoyés par le fond. Plouf, ils ont plongé. On n’a jamais su le fin mot du mystère. Mais moi j’ai pas repris. Non. C’est pourquoi la mer…\par
Il savait naviguer, il savait barrer, voile ou moteur, peu importe, mais aujourd’hui la mer, Rémi, tu vois, je regarde la ligne d’horizon, est-ce que tu es toujours à l’Aber Vrac’h ?\par
Je me souviens, quand les gens descendent de bateau, ils ont un moment sur le ponton où on dirait qu’ils ont bu. Dans leur tête ça tangue toujours, le corps a pris des réflexes, mais d’un seul coup le pied est fixe et ils se sentent encore dans la nécessité de rétablir le roulis.

\section[{— 4 —}]{— 4 —}
\renewcommand{\leftmark}{— 4 —}

\noindent « CHI… CHI… CHIGNOLITA ! » Comment avez-vous dit, captain, il s’appelle comment votre chat ? Le Capitaine Two-Teeth répète avec ses deux dents qui sifflent. « CHI… Chignolita ! »\par
Ça veut dire quoi ?\par
Tout le monde rit. Un homme d’équipage traduit \emph{« Signolita ! ».}\par
Ah, Signorita, ah, ces Asiatiques, peuvent pas prononcer un « r » !\par
« MAIS C’EST UN GARÇON ! » je dis.\par
Avec le chat sur mes cuisses, je lui tâtais les boules.\par
Rires autour d’elle.\par
Il lui avait sauté dessus pendant qu’elle dormait, en lui plantant en plus les griffes dans les cuisses. Oh ! Et le chat, comme tous les chats, se la joue indifférent.\par
Elle allait le prendre par-derrière quand elle lui toucha les burnes.\par
« Signorita, ça sonne comme en espagnol, dit-elle. Mais c’est un mâle ! »\par
Rires autour d’elle. Est-il possible que dans les langues, masculin et féminin soient des nuances flottantes ? Parfois elles n’existent pas, tout est neutre.\par
— Enfin qui lui a donné ce nom ?\par
— Une touriste, elle était là le jour où le chat a sauté sur le bateau. Elle l’a pris d’abord pour une fille.\par
— Ah ?\par
Le chat Signorita avait été un chat pas ordinaire, disait le Capitaine Two-Teeth. Un jour, il était arrivé en courant, déboulant à fond de ballon sur le ponton comme s’il était poursuivi par une meute de matous et, sans hésiter, il avait sauté comme ça direct sur le bateau et il n’en était plus jamais redescendu ; jamais plus ; jusqu’à sa belle mort de chat il y a deux ans.\par
Vous l’avez-vu, vous, combien de fois le chat Signorita, mademoiselle Élisa ?\par
De temps en temps, il grimpait sur le mât pour se faire les griffes, rappelez-vous, mais sa place, le plus souvent, c’était toujours derrière la vitre de ma cabine, la cabine du Capitaine Two-Teeth, là, à côté de moi, fixe, stoïque, il regardait la mer. Il ne perdait pas de vue l’horizon. C’était marrant. Des fois j’ai dit à mes hommes, celui-là j’ai l’impression qu’il a une boussole dans la tête, je sens qu’il me donne la direction.\par
Oh, comme c’était ridicule de penser à ça maintenant !\par
Comme de se moquer de son accent !\par
Et de son impossibilité de dire les « s » ou les « r » !\par
Le chat Signorita était mort depuis longtemps.\par
Elle en conclut qu’elle venait de rêver sans doute et qu’elle avait peut-être même lancé un cri. Elle se leva.\par
Le moteur ronflait à mort à lutter contre le vent.\par
— Vous vous souvenez du chat Signorita, Capitaine ? dit-elle en arrivant à la porte de la cabine en essayant de se protéger.\par
— Mm, oui, bien sûr ! dit-il distrait, préoccupé par ce qui se passait devant.

\section[{— 5 —}]{— 5 —}
\renewcommand{\leftmark}{— 5 —}

\noindent \emph{La Pleine Lune de Mai} marque le changement de saison dans ce coin des Tropiques, après on passe à la Mousson, avec pluie au moins deux fois par jour, ce qui n’est pas gênant, car il fait toujours aussi chaud et on sèche vite, enfin le corps, car pour le lavage de ses petites n’affaires ça sèche pas du tout, c’est trop humide, y en a même qui mettent plus de slip pendant la Mousson, trop long à sécher, impossible même.\par
Quatre jours de fête en attendant.\par
Avant, on fait quatre jours de fête et, au quatrième, LA LUNE – LA LUNE ronde et magique – apparaît. Elle apparaît comme un gros ballon tout jaune au-dessus de la mer et elle paraît si proche de vous qu’on croirait presque qu’on peut la toucher. Elle est vraiment énorme, très proche. La musique s’arrête. On reste là sur la plage à côté des palmiers à la regarder. Cela fait quatre jours que l’on passe les mêmes cassettes. Cela fait quatre jours que l’on danse jour et nuit sur le même parquet bâché installé sur le sable. À côté on a construit un bateau en balsa qui ressemble à un petit chalutier de pêcheur côtier. Le dernier jour, les habitants de l’île viennent déposer leur offrande, le remplissent de sachets de riz, fleurs, couronnes, petits pâtés, biscuits, poissons séchés, liasses de gros billets de fausse monnaie.\par
Le matin, des pêcheurs sont allés repérer le sens des courants.\par
A midi, on envoie le bateau à la mer, des hommes tout habillés rentrent dans l’eau jusqu’aux épaules pour le lancer, il ne faut surtout pas qu’il revienne ; il emmène tous les péchés, tous les soucis, toutes les mauvaises pensées de l’année qu’on doit oublier pour la prochaine saison.\par
Et le petit bateau va, s’en va, s’en va, tout seul, il doit se débrouiller, juste porté par le courant, et tout le monde, de la plage, le regarde gagner le large.\par
Et il s’en va en franchissant les vagues, devenant de plus en plus petit, petite coquille de noix, jusqu’au moment où on ne le voit plus.\par
Et le soir, donc, la lune s’élève au-dessus de la mer, toute ronde et magique.\par
Et on remet la musique et on re-danse jusqu’au petit matin.\par
\bigbreak
\noindent Pourquoi revient-on dans les endroits qu’on a aimés ? Bien sûr, \emph{la mer, la mer toujours recommencée, quelle récompense après une pensée}, et aujourd’hui elle n’allait pas plus loin dans\emph{ « Le Cimetière Marin »} le fameux poème de Paul Valéry. Je crois que j’avais essayé de l’apprendre par cœur. L’ai-je jamais su ? La première fois que je suis venue ici, j’étais avec ma copine Mine, c’était mon premier voyage, et nous sommes arrivées avec nos sacs à dos. Les marins sur le ponton nous ont indiqué le ferry-boat du Capitaine Two-Teeth, c’est lui qui assurait la liaison avec l’île. Et nous avons vu le Capitaine Two-Teeth qui, déjà, n’avait plus que deux dents. Et c’est Mine, je crois, qui l’a appelé comme ça. Les hommes d’équipage qui ont compris ont souri, et il a souri. \emph{Vous voulez aller sur l’île} ? Il a dit que pour le moment il attendait. On partirait peut-être dans l’après-midi, ils avaient encore à charger les commandes pour les commerçants de l’île. Il y en a deux, j’ajoute, deux Chinois, qui font chacun épicerie et resto, il y en a un côté port, et l’autre, presque en face, de l’autre côté de la place. DEUX DENTS ! Il souriait tout le temps.

\section[{— 6 —}]{— 6 —}
\renewcommand{\leftmark}{— 6 —}

\noindent Pendant la Mousson, il n’y a plus de bateau pour l’île. Pour rejoindre le continent, on prend les barques et on y va à la pagaie.\par
{\centering\itshape \noindent SAM-PA-LO ! KAYO – KAYO !\par}
\noindent C’est ce que racontait Loum, dont le nom signifie « le vent » en parlé \emph{tchalé}, le parlé des Gitans des mers, adapté en Wind en anglais, quand il parle aux touristes.\par
Autrefois il fallait deux jours pour rejoindre le continent.\par
Tout à la pagaie !\par
{\centering\itshape \noindent SAM-PA-LO ! KAYO – KAYO !\par}
{\centering\itshape \noindent SAM-PA-LO ! KAYO – KAYO !\par}
{\centering\itshape \noindent SAM-PA-LO ! KAYO – KAYO !\par}
\bigbreak
\noindent Une fois, refaisant ce voyage seule, elle s’était fait une entorse et elle était restée au moins tout le début de la Mousson sur l’île, puis elle avait profité d’un voyage de Loum et des autres pêcheurs pour rentrer. C’avait été une expérience extraordinaire.\par
L’archipel de Taru Tao se situe dans la mer d’Andaman au sud de la Thaïlande. Ce fut pendant longtemps un bagne, interdit d’accès. C’est aujourd’hui un parc national protégé. Vivent ici des populations de pécheurs que l’on appelle les Gitans des mers et qui vont d’île en île, ou enfin allaient, entre la Thaïlande et la Malaisie.\par
C’est un endroit très prisé des amateurs de plongée. L’eau est d’un bleu immen­sément bleu et immensément clair, avec des poissons inouïs de toutes les couleurs.\par
Avec Mine, elles avaient craqué pour ce coin paumé de nature sauvage. Mine disait \emph{« J’ai l’impression d’être dans une carte postale, c’est réellement paradisiaque ! »} Elles étaient revenues plusieurs fois. Puis Élisa était revenue seule. Et une de ces fois elle s’était fait son entorse.

\section[{— 7 —}]{— 7 —}
\renewcommand{\leftmark}{— 7 —}

\noindent Loum ça veut dire vent en tchalé, c’est son neveu qui avait traduit. Il l’avait fait appeler. Avec Mine, elles se promenaient la nuit dans le village. Et lui était avec son frère Ahmed à discuter dehors sur la terrasse de sa maison en train en même temps d’écouter de la bonne musique sur un vieux transistor.\par
Ils avaient commencé à parler musique, mais Loum ne savait que quelques mots en anglais, et Ahmed se contentait de sourire.\par
Loum appela son fils pour aller chercher des bières en lui donnant l’ordre de ramener le neveu pour traduire.\par
Voilà, ils sont revenus à deux, plus une caisse de bières, et Loum a demandé à son fils de déboucher les bouteilles en l’appelant « Lakitine ! ».\par
Ça veut dire quoi ?\par
\emph{« Ouvre-bouteille en tchalé ».}\par
O.K. Et Lakitine a ouvert les bouteilles avec ses dents.\par
Oui, Loum, ça veut dire vent, mais Loum lui il ne s’est converti à rien, je reste un pêcheur, un gitan, disait-il, un îlien, un marin, je monte aux arbres ; la première fois que j’ai vu un Blanc je me suis enfui dans la forêt.\par
Ahmed son frère approuve. Le neveu confirme.\par
Loum c’est un guerrier, un marin, quelqu’un qui connaît bien la mer, et vous pouvez toujours aller le chercher dans la forêt, il sait comment faire.\par
Oui, approuve Loum, je chasse avec un arc.\par
Mine promène toujours en voyage de vieilles cassettes dans son sac.\par
Elle en sort une justement où c’est Brésil, Rock, Reggae, arabo-andalou.\par
C’est l’objet de la conversation de la nuit, la musique.\par
Loum passe la sienne, Mine sélectionne dans ses mix. Une petite lampe à huile éclaire la scène. On se croirait reparti en arrière. Tout est doux.\par
Ils se sont vus plusieurs fois ensuite et, lors des quatre jours de fête de pleine lune, ils ont été tout le temps ensemble, allant chercher ensemble des bières chez le Chinois, dansant ensemble et Mine a donné les cassettes à Loum et, finalement, ils sont allés sur la plage et ils ont regardé ensemble la pleine lune monter.\par
Loum ça veut dire « Le Vent ! » en tchalé.

\section[{— 8 —}]{— 8 —}
\renewcommand{\leftmark}{— 8 —}

\noindent Mais entrez donc, mademoiselle Élisa, entrez, ce n’est pas tous les jours qu’on a un iceberg de la taille de la Crète qui franchit les Tropiques. Nous vous attendions !\par
Une bonne trentaine de messieurs en costume gris se tiennent autour d’une table ovale et le président descend de l’estrade pour venir à sa rencontre.\par
\emph{Venez, venez, mademoiselle.} Je vous présente mademoi­selle Élisa Absalem, dit-il en direction des messieurs en gris. Nous sommes déjà parvenus à un consensus général qui donne un accord de principe (et elle assise maintenant à la droite du président) sur le fait que, maintenant, nous sommes à peu près tous d’accord pour envoyer sur zone des bâtiments de nos marines nationales. Nous ne savons pas quand cet iceberg va se désintégrer, mais nous allons réussir une belle évacuation de votre île.\par
— \emph{Mon île}, il ne faut pas exagérer, monsieur le président.\par
— Néanmoins je voudrais d’abord vous demander pourquoi ce revire­ment subit, vous arrivez avec une délégation, je le comprends, et vous dites que ces îliens veulent rester.\par
— Je le dis, mais c’est pas moi qui le dis, je cède la parole à mademoi­selle Taha.\par
\bigbreak
\noindent Mademoiselle Taha raconta plus tard qu’elle n’avait pas reconnu sa voix dans le micro et que, voyant cette assemblée, elle s’était demandée encore une fois ce qui pouvait bien distinguer un Blanc d’un autre. Je voudrais juste dire ce qui s’est passé, commença-t-elle, parce que je sais que dans cette assemblée il y a encore des gens qui doutent. Mais elle ne put continuer, car elle eut la gorge serrée en voyant en même temps sur l’écran les images de son île telle qu’elle était avant.\par
Élisa prend le relais. Vous voyez ce ponton ? dit-elle. C’est là où on accoste avec le Capitaine Two-Teeth, enfin avec le ferry-boat. Aujourd’hui il n’existe plus. Il n’y a plus de lagon non plus. L’eau a monté d’un coup. Les gens d’ici disent que l’île s’enfonce. Mais la réalité, nous le savons tous, c’est qu’elle va disparaître.\par
Il y a quelques années, je suis restée plusieurs semaines sur cette île, je m’étais faite une entorse et ces gens m’ont soignée ; je peux même dire qu’ils m’ont très bien soignée, puisque aujourd’hui je n’ai plus rien.\par
C’est un mélange de poivre et d’huile, on me mettait ça comme un cataplasme, sur la cheville. Ensuite un homme me massait matin et soir et m’a remis tout en place, je pouvais marcher. C’est pourquoi je suis attachée à cette île, ces gens ont de la magie dans leur rapport à la vie, à l’univers. \emph{Vous imaginez en plus ce ciel bleu plein d’étoiles et la lumière des bougies} ? Aujourd’hui, il n’y a pas plus de lagon.\par
— Ça me rappelle un film des années 60, dit le président, un film en noir et blanc. Il y avait un nuage nucléaire, la catastrophe s’était répandue sur toute la planète et, provisoirement, des survivants se retrouvaient dans un coin d’Australie où ils savaient que ce serait bientôt leur tour. Enfin aujourd’hui, chers collègues, nous sommes devant un réel problème : cet iceberg ne va pas partir, il faudrait le bombarder, le faire exploser, l’éloigner, et l’eau monte très vite. \emph{Oui, mademoiselle Taha, vous vouliez dire} ?\par
— Oui, je voulais dire que pour l’évacuation, les vieux, ils veulent pas partir. Il y a une légende chez nous qui dit que cette île est notre base pour notre voyage futur, une fenêtre de tir en quelque sorte. Avez-vous jamais assisté à la pleine lune de mai, monsieur le président ? C’est la dernière pleine lune avant la Mousson. Et quand vous voyez la lune alors, vous avez l’impression qu’elle est aussi grosse que la Terre et que vous pourriez presque la toucher. Toutes nos légendes disent qu’un jour nous allons mourir. Aujourd’hui c’est l’iceberg qui nous fait mourir, votre réchauffement climatique.\par
— Nous savons, fit le président, mais il y a aussi une Américaine là-bas.\par
— Oui, c’est exact, elle est revenue, elle a même lancé un appel pour qu’on vienne mourir avec elle.\par
— Pensez-vous que vous pourriez la convaincre ?\par
— Mais de quoi, monsieur le président ?

\section[{— 9 —}]{— 9 —}
\renewcommand{\leftmark}{— 9 —}

\noindent Mon cher Boldy, je t’écris cette lettre pour te dire que le nouveau monde qu’on nous avait promis n’est pas vraiment celui que nous cherchions. J’écris ces mots à la lampe à pétrole, car depuis notre dernière entrevue je n’ai pu retourner sur l’île, ils nous en ont empêchés. Ils nous ont parqués dans un centre d’accueil. Ici, ils nous coupent l’électricité à 8 heures. Je n’ai plus envie de chanter, plus personne n’a envie de chanter. Maintenant tout le monde a vu les images de notre île disparue, c’est fini. On ne s’occupe plus de nous. J’ai l’impression d’avoir servi de cobaye pour une cause perdue et inutile d’ailleurs. Comment vas-tu, toi ? Ne fais pas trop de body-building, sinon tu vas faire craquer ta peau. J’aperçois de temps en temps le Capitaine Two-Teeth. Il répare son bateau. Il n’y a plus de touristes. Nous, de ce côté-ci du continent, on a encore une plage, mais pour combien de temps encore ? Boldy, continue à t’élever très haut. Je t’envoie mille baisers. Ta grande sœur Taha.\par

\dateline{Lille sept.oct. 2016}
 


% at least one empty page at end (for booklet couv)
\ifbooklet
  \pagestyle{empty}
  \clearpage
  % 2 empty pages maybe needed for 4e cover
  \ifnum\modulo{\value{page}}{4}=0 \hbox{}\newpage\hbox{}\newpage\fi
  \ifnum\modulo{\value{page}}{4}=1 \hbox{}\newpage\hbox{}\newpage\fi


  \hbox{}\newpage
  \ifodd\value{page}\hbox{}\newpage\fi
  {\centering\color{rubric}\bfseries\noindent\large
    Hurlus ? Qu’est-ce.\par
    \bigskip
  }
  \noindent Des bouquinistes électroniques, pour du texte libre à participations libres,
  téléchargeable gratuitement sur \href{https://hurlus.fr}{\dotuline{hurlus.fr}}.\par
  \bigskip
  \noindent Cette brochure a été produite par des éditeurs bénévoles.
  Elle n’est pas faite pour être possédée, mais pour être lue, et puis donnée, ou déposée dans une boîte à livres.
  En page de garde, on peut ajouter une date, un lieu, un nom ;
  comme une fiche de bibliothèque en papier qui enregistre \emph{les voyages de la brochure}.
  \par

  Ce texte a été choisi parce qu’une personne l’a aimé,
  ou haï, elle a pensé qu’il partipait à la formation de notre présent ;
  sans le souci de plaire, vendre, ou militer pour une cause.
  \par

  L’édition électronique est soigneuse, tant sur la technique
  que sur l’établissement du texte ; mais sans aucune prétention scolaire, au contraire.
  Le but est de s’adresser à tous, sans distinction de science ou de diplôme.
  \par

  Cet exemplaire en papier a été tiré sur une imprimante personnelle
   ou une photocopieuse. Tout le monde peut le faire.
  Il suffit de
  télécharger un fichier sur \href{https://hurlus.fr}{\dotuline{hurlus.fr}},
  d’imprimer, et agrafer (puis lire et donner).\par

  \bigskip

  \noindent PS : Les hurlus furent aussi des rebelles protestants qui cassaient les statues dans les églises catholiques. En 1566 démarra la révolte des gueux dans le pays de Lille. L’insurrection enflamma la région jusqu’à Anvers où les gueux de mer bloquèrent les bateaux espagnols.
  Ce fut une rare guerre de libération dont naquit un pays toujours libre : les Pays-Bas.
  En plat pays francophone, par contre, restèrent des bandes de huguenots, les hurlus, progressivement réprimés par la très catholique Espagne.
  Cette mémoire d’une défaite est éteinte, rallumons-la. Sortons les livres du culte universitaire, débusquons les idoles de l’époque, pour les démonter.
\fi

\end{document}
