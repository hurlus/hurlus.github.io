%%%%%%%%%%%%%%%%%%%%%%%%%%%%%%%%%
% LaTeX model https://hurlus.fr %
%%%%%%%%%%%%%%%%%%%%%%%%%%%%%%%%%

% Needed before document class
\RequirePackage{pdftexcmds} % needed for tests expressions
\RequirePackage{fix-cm} % correct units

% Define mode
\def\mode{a4}

\newif\ifaiv % a4
\newif\ifav % a5
\newif\ifbooklet % booklet
\newif\ifcover % cover for booklet

\ifnum \strcmp{\mode}{cover}=0
  \covertrue
\else\ifnum \strcmp{\mode}{booklet}=0
  \booklettrue
\else\ifnum \strcmp{\mode}{a5}=0
  \avtrue
\else
  \aivtrue
\fi\fi\fi

\ifbooklet % do not enclose with {}
  \documentclass[french,twoside]{book} % ,notitlepage
  \usepackage[%
    papersize={105mm, 297mm},
    inner=12mm,
    outer=12mm,
    top=20mm,
    bottom=15mm,
    marginparsep=0pt,
  ]{geometry}
  \usepackage[fontsize=9.5pt]{scrextend} % for Roboto
\else\ifav
  \documentclass[french,twoside]{book} % ,notitlepage
  \usepackage[%
    a5paper,
    inner=25mm,
    outer=15mm,
    top=15mm,
    bottom=15mm,
    marginparsep=0pt,
  ]{geometry}
  \usepackage[fontsize=12pt]{scrextend}
\else% A4 2 cols
  \documentclass[twocolumn]{report}
  \usepackage[%
    a4paper,
    inner=15mm,
    outer=10mm,
    top=25mm,
    bottom=18mm,
    marginparsep=0pt,
  ]{geometry}
  \setlength{\columnsep}{20mm}
  \usepackage[fontsize=9.5pt]{scrextend}
\fi\fi

%%%%%%%%%%%%%%
% Alignments %
%%%%%%%%%%%%%%
% before teinte macros

\setlength{\arrayrulewidth}{0.2pt}
\setlength{\columnseprule}{\arrayrulewidth} % twocol
\setlength{\parskip}{0pt} % classical para with no margin
\setlength{\parindent}{1.5em}

%%%%%%%%%%
% Colors %
%%%%%%%%%%
% before Teinte macros

\usepackage[dvipsnames]{xcolor}
\definecolor{rubric}{HTML}{800000} % the tonic 0c71c3
\def\columnseprulecolor{\color{rubric}}
\colorlet{borderline}{rubric!30!} % definecolor need exact code
\definecolor{shadecolor}{gray}{0.95}
\definecolor{bghi}{gray}{0.5}

%%%%%%%%%%%%%%%%%
% Teinte macros %
%%%%%%%%%%%%%%%%%
%%%%%%%%%%%%%%%%%%%%%%%%%%%%%%%%%%%%%%%%%%%%%%%%%%%
% <TEI> generic (LaTeX names generated by Teinte) %
%%%%%%%%%%%%%%%%%%%%%%%%%%%%%%%%%%%%%%%%%%%%%%%%%%%
% This template is inserted in a specific design
% It is XeLaTeX and otf fonts

\makeatletter % <@@@


\usepackage{blindtext} % generate text for testing
\usepackage[strict]{changepage} % for modulo 4
\usepackage{contour} % rounding words
\usepackage[nodayofweek]{datetime}
% \usepackage{DejaVuSans} % seems buggy for sffont font for symbols
\usepackage{enumitem} % <list>
\usepackage{etoolbox} % patch commands
\usepackage{fancyvrb}
\usepackage{fancyhdr}
\usepackage{float}
\usepackage{fontspec} % XeLaTeX mandatory for fonts
\usepackage{footnote} % used to capture notes in minipage (ex: quote)
\usepackage{framed} % bordering correct with footnote hack
\usepackage{graphicx}
\usepackage{lettrine} % drop caps
\usepackage{lipsum} % generate text for testing
\usepackage[framemethod=tikz,]{mdframed} % maybe used for frame with footnotes inside
\usepackage{pdftexcmds} % needed for tests expressions
\usepackage{polyglossia} % non-break space french punct, bug Warning: "Failed to patch part"
\usepackage[%
  indentfirst=false,
  vskip=1em,
  noorphanfirst=true,
  noorphanafter=true,
  leftmargin=\parindent,
  rightmargin=0pt,
]{quoting}
\usepackage{ragged2e}
\usepackage{setspace} % \setstretch for <quote>
\usepackage{tabularx} % <table>
\usepackage[explicit]{titlesec} % wear titles, !NO implicit
\usepackage{tikz} % ornaments
\usepackage{tocloft} % styling tocs
\usepackage[fit]{truncate} % used im runing titles
\usepackage{unicode-math}
\usepackage[normalem]{ulem} % breakable \uline, normalem is absolutely necessary to keep \emph
\usepackage{verse} % <l>
\usepackage{xcolor} % named colors
\usepackage{xparse} % @ifundefined
\XeTeXdefaultencoding "iso-8859-1" % bad encoding of xstring
\usepackage{xstring} % string tests
\XeTeXdefaultencoding "utf-8"
\PassOptionsToPackage{hyphens}{url} % before hyperref, which load url package

% TOTEST
% \usepackage{hypcap} % links in caption ?
% \usepackage{marginnote}
% TESTED
% \usepackage{background} % doesn’t work with xetek
% \usepackage{bookmark} % prefers the hyperref hack \phantomsection
% \usepackage[color, leftbars]{changebar} % 2 cols doc, impossible to keep bar left
% \usepackage[utf8x]{inputenc} % inputenc package ignored with utf8 based engines
% \usepackage[sfdefault,medium]{inter} % no small caps
% \usepackage{firamath} % choose firasans instead, firamath unavailable in Ubuntu 21-04
% \usepackage{flushend} % bad for last notes, supposed flush end of columns
% \usepackage[stable]{footmisc} % BAD for complex notes https://texfaq.org/FAQ-ftnsect
% \usepackage{helvet} % not for XeLaTeX
% \usepackage{multicol} % not compatible with too much packages (longtable, framed, memoir…)
% \usepackage[default,oldstyle,scale=0.95]{opensans} % no small caps
% \usepackage{sectsty} % \chapterfont OBSOLETE
% \usepackage{soul} % \ul for underline, OBSOLETE with XeTeX
% \usepackage[breakable]{tcolorbox} % text styling gone, footnote hack not kept with breakable


% Metadata inserted by a program, from the TEI source, for title page and runing heads
\title{\textbf{ Traité de savoir-vivre à l’usage des jeunes générations }}
\date{1967}
\author{Vaneigem}
\def\elbibl{Vaneigem. 1967. \emph{Traité de savoir-vivre à l’usage des jeunes générations}}
\def\elsource{\{source\}}

% Default metas
\newcommand{\colorprovide}[2]{\@ifundefinedcolor{#1}{\colorlet{#1}{#2}}{}}
\colorprovide{rubric}{red}
\colorprovide{silver}{lightgray}
\@ifundefined{syms}{\newfontfamily\syms{DejaVu Sans}}{}
\newif\ifdev
\@ifundefined{elbibl}{% No meta defined, maybe dev mode
  \newcommand{\elbibl}{Titre court ?}
  \newcommand{\elbook}{Titre du livre source ?}
  \newcommand{\elabstract}{Résumé\par}
  \newcommand{\elurl}{http://oeuvres.github.io/elbook/2}
  \author{Éric Lœchien}
  \title{Un titre de test assez long pour vérifier le comportement d’une maquette}
  \date{1566}
  \devtrue
}{}
\let\eltitle\@title
\let\elauthor\@author
\let\eldate\@date


\defaultfontfeatures{
  % Mapping=tex-text, % no effect seen
  Scale=MatchLowercase,
  Ligatures={TeX,Common},
}


% generic typo commands
\newcommand{\astermono}{\medskip\centerline{\color{rubric}\large\selectfont{\syms ✻}}\medskip\par}%
\newcommand{\astertri}{\medskip\par\centerline{\color{rubric}\large\selectfont{\syms ✻\,✻\,✻}}\medskip\par}%
\newcommand{\asterism}{\bigskip\par\noindent\parbox{\linewidth}{\centering\color{rubric}\large{\syms ✻}\\{\syms ✻}\hskip 0.75em{\syms ✻}}\bigskip\par}%

% lists
\newlength{\listmod}
\setlength{\listmod}{\parindent}
\setlist{
  itemindent=!,
  listparindent=\listmod,
  labelsep=0.2\listmod,
  parsep=0pt,
  % topsep=0.2em, % default topsep is best
}
\setlist[itemize]{
  label=—,
  leftmargin=0pt,
  labelindent=1.2em,
  labelwidth=0pt,
}
\setlist[enumerate]{
  label={\bf\color{rubric}\arabic*.},
  labelindent=0.8\listmod,
  leftmargin=\listmod,
  labelwidth=0pt,
}
\newlist{listalpha}{enumerate}{1}
\setlist[listalpha]{
  label={\bf\color{rubric}\alph*.},
  leftmargin=0pt,
  labelindent=0.8\listmod,
  labelwidth=0pt,
}
\newcommand{\listhead}[1]{\hspace{-1\listmod}\emph{#1}}

\renewcommand{\hrulefill}{%
  \leavevmode\leaders\hrule height 0.2pt\hfill\kern\z@}

% General typo
\DeclareTextFontCommand{\textlarge}{\large}
\DeclareTextFontCommand{\textsmall}{\small}

% commands, inlines
\newcommand{\anchor}[1]{\Hy@raisedlink{\hypertarget{#1}{}}} % link to top of an anchor (not baseline)
\newcommand\abbr[1]{#1}
\newcommand{\autour}[1]{\tikz[baseline=(X.base)]\node [draw=rubric,thin,rectangle,inner sep=1.5pt, rounded corners=3pt] (X) {\color{rubric}#1};}
\newcommand\corr[1]{#1}
\newcommand{\ed}[1]{ {\color{silver}\sffamily\footnotesize (#1)} } % <milestone ed="1688"/>
\newcommand\expan[1]{#1}
\newcommand\foreign[1]{\emph{#1}}
\newcommand\gap[1]{#1}
\renewcommand{\LettrineFontHook}{\color{rubric}}
\newcommand{\initial}[2]{\lettrine[lines=2, loversize=0.3, lhang=0.3]{#1}{#2}}
\newcommand{\initialiv}[2]{%
  \let\oldLFH\LettrineFontHook
  % \renewcommand{\LettrineFontHook}{\color{rubric}\ttfamily}
  \IfSubStr{QJ’}{#1}{
    \lettrine[lines=4, lhang=0.2, loversize=-0.1, lraise=0.2]{\smash{#1}}{#2}
  }{\IfSubStr{É}{#1}{
    \lettrine[lines=4, lhang=0.2, loversize=-0, lraise=0]{\smash{#1}}{#2}
  }{\IfSubStr{ÀÂ}{#1}{
    \lettrine[lines=4, lhang=0.2, loversize=-0, lraise=0, slope=0.6em]{\smash{#1}}{#2}
  }{\IfSubStr{A}{#1}{
    \lettrine[lines=4, lhang=0.2, loversize=0.2, slope=0.6em]{\smash{#1}}{#2}
  }{\IfSubStr{V}{#1}{
    \lettrine[lines=4, lhang=0.2, loversize=0.2, slope=-0.5em]{\smash{#1}}{#2}
  }{
    \lettrine[lines=4, lhang=0.2, loversize=0.2]{\smash{#1}}{#2}
  }}}}}
  \let\LettrineFontHook\oldLFH
}
\newcommand{\labelchar}[1]{\textbf{\color{rubric} #1}}
\newcommand{\milestone}[1]{\autour{\footnotesize\color{rubric} #1}} % <milestone n="4"/>
\newcommand\name[1]{#1}
\newcommand\orig[1]{#1}
\newcommand\orgName[1]{#1}
\newcommand\persName[1]{#1}
\newcommand\placeName[1]{#1}
\newcommand{\pn}[1]{\IfSubStr{-—–¶}{#1}% <p n="3"/>
  {\noindent{\bfseries\color{rubric}   ¶  }}
  {{\footnotesize\autour{ #1}  }}}
\newcommand\reg{}
% \newcommand\ref{} % already defined
\newcommand\sic[1]{#1}
\newcommand\surname[1]{\textsc{#1}}
\newcommand\term[1]{\textbf{#1}}

\def\mednobreak{\ifdim\lastskip<\medskipamount
  \removelastskip\nopagebreak\medskip\fi}
\def\bignobreak{\ifdim\lastskip<\bigskipamount
  \removelastskip\nopagebreak\bigskip\fi}

% commands, blocks
\newcommand{\byline}[1]{\bigskip{\RaggedLeft{#1}\par}\bigskip}
\newcommand{\bibl}[1]{{\RaggedLeft{#1}\par\bigskip}}
\newcommand{\biblitem}[1]{{\noindent\hangindent=\parindent   #1\par}}
\newcommand{\dateline}[1]{\medskip{\RaggedLeft{#1}\par}\bigskip}
\newcommand{\labelblock}[1]{\medbreak{\noindent\color{rubric}\bfseries #1}\par\mednobreak}
\newcommand{\salute}[1]{\bigbreak{#1}\par\medbreak}
\newcommand{\signed}[1]{\bigbreak\filbreak{\raggedleft #1\par}\medskip}

% environments for blocks (some may become commands)
\newenvironment{borderbox}{}{} % framing content
\newenvironment{citbibl}{\ifvmode\hfill\fi}{\ifvmode\par\fi }
\newenvironment{docAuthor}{\ifvmode\vskip4pt\fontsize{16pt}{18pt}\selectfont\fi\itshape}{\ifvmode\par\fi }
\newenvironment{docDate}{}{\ifvmode\par\fi }
\newenvironment{docImprint}{\vskip6pt}{\ifvmode\par\fi }
\newenvironment{docTitle}{\vskip6pt\bfseries\fontsize{18pt}{22pt}\selectfont}{\par }
\newenvironment{msHead}{\vskip6pt}{\par}
\newenvironment{msItem}{\vskip6pt}{\par}
\newenvironment{titlePart}{}{\par }


% environments for block containers
\newenvironment{argument}{\itshape\parindent0pt}{\vskip1.5em}
\newenvironment{biblfree}{}{\ifvmode\par\fi }
\newenvironment{bibitemlist}[1]{%
  \list{\@biblabel{\@arabic\c@enumiv}}%
  {%
    \settowidth\labelwidth{\@biblabel{#1}}%
    \leftmargin\labelwidth
    \advance\leftmargin\labelsep
    \@openbib@code
    \usecounter{enumiv}%
    \let\p@enumiv\@empty
    \renewcommand\theenumiv{\@arabic\c@enumiv}%
  }
  \sloppy
  \clubpenalty4000
  \@clubpenalty \clubpenalty
  \widowpenalty4000%
  \sfcode`\.\@m
}%
{\def\@noitemerr
  {\@latex@warning{Empty `bibitemlist' environment}}%
\endlist}
\newenvironment{quoteblock}% may be used for ornaments
  {\begin{quoting}}
  {\end{quoting}}

% table () is preceded and finished by custom command
\newcommand{\tableopen}[1]{%
  \ifnum\strcmp{#1}{wide}=0{%
    \begin{center}
  }
  \else\ifnum\strcmp{#1}{long}=0{%
    \begin{center}
  }
  \else{%
    \begin{center}
  }
  \fi\fi
}
\newcommand{\tableclose}[1]{%
  \ifnum\strcmp{#1}{wide}=0{%
    \end{center}
  }
  \else\ifnum\strcmp{#1}{long}=0{%
    \end{center}
  }
  \else{%
    \end{center}
  }
  \fi\fi
}


% text structure
\newcommand\chapteropen{} % before chapter title
\newcommand\chaptercont{} % after title, argument, epigraph…
\newcommand\chapterclose{} % maybe useful for multicol settings
\setcounter{secnumdepth}{-2} % no counters for hierarchy titles
\setcounter{tocdepth}{5} % deep toc
\markright{\@title} % ???
\markboth{\@title}{\@author} % ???
\renewcommand\tableofcontents{\@starttoc{toc}}
% toclof format
% \renewcommand{\@tocrmarg}{0.1em} % Useless command?
% \renewcommand{\@pnumwidth}{0.5em} % {1.75em}
\renewcommand{\@cftmaketoctitle}{}
\setlength{\cftbeforesecskip}{\z@ \@plus.2\p@}
\renewcommand{\cftchapfont}{}
\renewcommand{\cftchapdotsep}{\cftdotsep}
\renewcommand{\cftchapleader}{\normalfont\cftdotfill{\cftchapdotsep}}
\renewcommand{\cftchappagefont}{\bfseries}
\setlength{\cftbeforechapskip}{0em \@plus\p@}
% \renewcommand{\cftsecfont}{\small\relax}
\renewcommand{\cftsecpagefont}{\normalfont}
% \renewcommand{\cftsubsecfont}{\small\relax}
\renewcommand{\cftsecdotsep}{\cftdotsep}
\renewcommand{\cftsecpagefont}{\normalfont}
\renewcommand{\cftsecleader}{\normalfont\cftdotfill{\cftsecdotsep}}
\setlength{\cftsecindent}{1em}
\setlength{\cftsubsecindent}{2em}
\setlength{\cftsubsubsecindent}{3em}
\setlength{\cftchapnumwidth}{1em}
\setlength{\cftsecnumwidth}{1em}
\setlength{\cftsubsecnumwidth}{1em}
\setlength{\cftsubsubsecnumwidth}{1em}

% footnotes
\newif\ifheading
\newcommand*{\fnmarkscale}{\ifheading 0.70 \else 1 \fi}
\renewcommand\footnoterule{\vspace*{0.3cm}\hrule height \arrayrulewidth width 3cm \vspace*{0.3cm}}
\setlength\footnotesep{1.5\footnotesep} % footnote separator
\renewcommand\@makefntext[1]{\parindent 1.5em \noindent \hb@xt@1.8em{\hss{\normalfont\@thefnmark . }}#1} % no superscipt in foot
\patchcmd{\@footnotetext}{\footnotesize}{\footnotesize\sffamily}{}{} % before scrextend, hyperref


%   see https://tex.stackexchange.com/a/34449/5049
\def\truncdiv#1#2{((#1-(#2-1)/2)/#2)}
\def\moduloop#1#2{(#1-\truncdiv{#1}{#2}*#2)}
\def\modulo#1#2{\number\numexpr\moduloop{#1}{#2}\relax}

% orphans and widows
\clubpenalty=9996
\widowpenalty=9999
\brokenpenalty=4991
\predisplaypenalty=10000
\postdisplaypenalty=1549
\displaywidowpenalty=1602
\hyphenpenalty=400
% Copied from Rahtz but not understood
\def\@pnumwidth{1.55em}
\def\@tocrmarg {2.55em}
\def\@dotsep{4.5}
\emergencystretch 3em
\hbadness=4000
\pretolerance=750
\tolerance=2000
\vbadness=4000
\def\Gin@extensions{.pdf,.png,.jpg,.mps,.tif}
% \renewcommand{\@cite}[1]{#1} % biblio

\usepackage{hyperref} % supposed to be the last one, :o) except for the ones to follow
\urlstyle{same} % after hyperref
\hypersetup{
  % pdftex, % no effect
  pdftitle={\elbibl},
  % pdfauthor={Your name here},
  % pdfsubject={Your subject here},
  % pdfkeywords={keyword1, keyword2},
  bookmarksnumbered=true,
  bookmarksopen=true,
  bookmarksopenlevel=1,
  pdfstartview=Fit,
  breaklinks=true, % avoid long links
  pdfpagemode=UseOutlines,    % pdf toc
  hyperfootnotes=true,
  colorlinks=false,
  pdfborder=0 0 0,
  % pdfpagelayout=TwoPageRight,
  % linktocpage=true, % NO, toc, link only on page no
}

\makeatother % /@@@>
%%%%%%%%%%%%%%
% </TEI> end %
%%%%%%%%%%%%%%


%%%%%%%%%%%%%
% footnotes %
%%%%%%%%%%%%%
\renewcommand{\thefootnote}{\bfseries\textcolor{rubric}{\arabic{footnote}}} % color for footnote marks

%%%%%%%%%
% Fonts %
%%%%%%%%%
\usepackage[]{roboto} % SmallCaps, Regular is a bit bold
% \linespread{0.90} % too compact, keep font natural
\newfontfamily\fontrun[]{Roboto Condensed Light} % condensed runing heads
\ifav
  \setmainfont[
    ItalicFont={Roboto Light Italic},
  ]{Roboto}
\else\ifbooklet
  \setmainfont[
    ItalicFont={Roboto Light Italic},
  ]{Roboto}
\else
\setmainfont[
  ItalicFont={Roboto Italic},
]{Roboto Light}
\fi\fi
\renewcommand{\LettrineFontHook}{\bfseries\color{rubric}}
% \renewenvironment{labelblock}{\begin{center}\bfseries\color{rubric}}{\end{center}}

%%%%%%%%
% MISC %
%%%%%%%%

\setdefaultlanguage[frenchpart=false]{french} % bug on part


\newenvironment{quotebar}{%
    \def\FrameCommand{{\color{rubric!10!}\vrule width 0.5em} \hspace{0.9em}}%
    \def\OuterFrameSep{\itemsep} % séparateur vertical
    \MakeFramed {\advance\hsize-\width \FrameRestore}
  }%
  {%
    \endMakeFramed
  }
\renewenvironment{quoteblock}% may be used for ornaments
  {%
    \savenotes
    \setstretch{0.9}
    \normalfont
    \begin{quotebar}
  }
  {%
    \end{quotebar}
    \spewnotes
  }


\renewcommand{\headrulewidth}{\arrayrulewidth}
\renewcommand{\headrule}{{\color{rubric}\hrule}}

% delicate tuning, image has produce line-height problems in title on 2 lines
\titleformat{name=\chapter} % command
  [display] % shape
  {\vspace{1.5em}\centering} % format
  {} % label
  {0pt} % separator between n
  {}
[{\color{rubric}\huge\textbf{#1}}\bigskip] % after code
% \titlespacing{command}{left spacing}{before spacing}{after spacing}[right]
\titlespacing*{\chapter}{0pt}{-2em}{0pt}[0pt]

\titleformat{name=\section}
  [block]{}{}{}{}
  [\vbox{\color{rubric}\large\raggedleft\textbf{#1}}]
\titlespacing{\section}{0pt}{0pt plus 4pt minus 2pt}{\baselineskip}

\titleformat{name=\subsection}
  [block]
  {}
  {} % \thesection
  {} % separator \arrayrulewidth
  {}
[\vbox{\large\textbf{#1}}]
% \titlespacing{\subsection}{0pt}{0pt plus 4pt minus 2pt}{\baselineskip}

\ifaiv
  \fancypagestyle{main}{%
    \fancyhf{}
    \setlength{\headheight}{1.5em}
    \fancyhead{} % reset head
    \fancyfoot{} % reset foot
    \fancyhead[L]{\truncate{0.45\headwidth}{\fontrun\elbibl}} % book ref
    \fancyhead[R]{\truncate{0.45\headwidth}{ \fontrun\nouppercase\leftmark}} % Chapter title
    \fancyhead[C]{\thepage}
  }
  \fancypagestyle{plain}{% apply to chapter
    \fancyhf{}% clear all header and footer fields
    \setlength{\headheight}{1.5em}
    \fancyhead[L]{\truncate{0.9\headwidth}{\fontrun\elbibl}}
    \fancyhead[R]{\thepage}
  }
\else
  \fancypagestyle{main}{%
    \fancyhf{}
    \setlength{\headheight}{1.5em}
    \fancyhead{} % reset head
    \fancyfoot{} % reset foot
    \fancyhead[RE]{\truncate{0.9\headwidth}{\fontrun\elbibl}} % book ref
    \fancyhead[LO]{\truncate{0.9\headwidth}{\fontrun\nouppercase\leftmark}} % Chapter title, \nouppercase needed
    \fancyhead[RO,LE]{\thepage}
  }
  \fancypagestyle{plain}{% apply to chapter
    \fancyhf{}% clear all header and footer fields
    \setlength{\headheight}{1.5em}
    \fancyhead[L]{\truncate{0.9\headwidth}{\fontrun\elbibl}}
    \fancyhead[R]{\thepage}
  }
\fi

\ifav % a5 only
  \titleclass{\section}{top}
\fi

\newcommand\chapo{{%
  \vspace*{-3em}
  \centering % no vskip ()
  {\Large\addfontfeature{LetterSpace=25}\bfseries{\elauthor}}\par
  \smallskip
  {\large\eldate}\par
  \bigskip
  {\Large\selectfont{\eltitle}}\par
  \bigskip
  {\color{rubric}\hline\par}
  \bigskip
  {\Large TEXTE LIBRE À PARTICPATION LIBRE\par}
  \centerline{\small\color{rubric} {hurlus.fr, tiré le \today}}\par
  \bigskip
}}

\newcommand\cover{{%
  \thispagestyle{empty}
  \centering
  {\LARGE\bfseries{\elauthor}}\par
  \bigskip
  {\Large\eldate}\par
  \bigskip
  \bigskip
  {\LARGE\selectfont{\eltitle}}\par
  \vfill\null
  {\color{rubric}\setlength{\arrayrulewidth}{2pt}\hline\par}
  \vfill\null
  {\Large TEXTE LIBRE À PARTICPATION LIBRE\par}
  \centerline{{\href{https://hurlus.fr}{\dotuline{hurlus.fr}}, tiré le \today}}\par
}}

\begin{document}
\pagestyle{empty}
\ifbooklet{
  \cover\newpage
  \thispagestyle{empty}\hbox{}\newpage
  \cover\newpage\noindent Les voyages de la brochure\par
  \bigskip
  \begin{tabularx}{\textwidth}{l|X|X}
    \textbf{Date} & \textbf{Lieu}& \textbf{Nom/pseudo} \\ \hline
    \rule{0pt}{25cm} &  &   \\
  \end{tabularx}
  \newpage
  \addtocounter{page}{-4}
}\fi

\thispagestyle{empty}
\ifaiv
  \twocolumn[\chapo]
\else
  \chapo
\fi
{\it\elabstract}
\bigskip
\makeatletter\@starttoc{toc}\makeatother % toc without new page
\bigskip

\pagestyle{main} % after style

  \section[{Introduction}]{Introduction}\renewcommand{\leftmark}{Introduction}

\noindent Ce qu’il y a de vécu dans ce livre, je n’ai pas l’intention de le rendre sensible à des lecteurs qui ne s’apprêtent en toute conscience à le revivre. J’attends qu’il se perde et se retrouve dans un mouvement général des esprits, comme je me flatte que les conditions présentes s’effaceront de la mémoire des hommes.\par
Le monde est à refaire : tous les spécialistes de son reconditionnement ne l’empêcheront pas. De ceux-là, que je ne veux pas comprendre, mieux vaut n’être pas compris.\par
Pour les autres, je sollicite leur bienveillance avec une humilité qui ne leur échappera pas. J’aurais souhaité qu’un tel livre fût accessible aux têtes les moins rompues au jargon des idées. J’espère n’avoir échoué qu’au second degré. De ce chaos sortiront quelque jour des formules qui tireront à bout portant sur nos ennemis. Entre-temps, que la phrase à relire fasse son chemin. La voie vers la simplicité est la plus complexe et, ici particulièrement, il était utile ne pas arracher aux banalités les multiples racines qui permettront de les transplanter dans un autre terrain, de les cultiver à notre profit.\par
Jamais je n’ai prétendu révéler du neuf, lancer de l’inédit sur le marché de la culture. Une infime correction de l’essentiel importe plus que cent innovations accessoires. Seul est nouveau le sens du courant qui charrie les banalités.\par
Depuis le temps qu’il y a des hommes, et qui lisent Lautréamont, tout est dit et peu sont venus pour en tirer profit. Parce que nos connaissances sont en soi banales, elles ne peuvent profiter qu’aux esprits qui ne le sont pas.\par
Le monde moderne doit apprendre ce qu’il sait déjà, devenir ce qu’il est, à travers une immense conjuration d’obstacles, par la pratique. On n’échappe à la banalité qu’en la manipulant, en la dominant, en la plongeant dans le rêve, en la livrant au bon plaisir de la subjectivité. J’ai fait la part belle à la volonté subjective, mais que personne ne m’en fasse grief avant d’avoir estimé tout de bon ce que peuvent, en faveur de la subjectivité, les conditions objectives que le monde réalise chaque jour. Tout part de la subjectivité et rien ne s’y arrête. Aujourd’hui moins que jamais.\par
La lutte du subjectif et de ce qui le corrompt élargit désormais les limites de la vieille lutte des classes. Elle la renouvelle et l’aiguise. Le parti pris de la vie est un parti pris politique. Nous ne voulons pas d’un monde où la garantie de ne pas mourir de faim s’échange contre le risque de mourir d’ennui.\par
L’homme de la survie, c’est l’homme émietté dans les mécanismes du pouvoir hiérarchisé, dans une combinaison d’interférences, dans un chaos de techniques oppressives qui n’attend pour s’ordonner que la patiente programmation des penseurs programmés.\par
L’homme de la survie, c’est aussi l’homme unitaire, l’homme du refus global. Il ne se passe un instant sans que chacun de nous ne vive contradictoirement, et à tous les degrés de la réalité, le conflit de l’oppression et de la liberté ; sans qu’il ne soit bizarrement déformé et comme saisi en même temps selon deux perspectives antagonistes : la perspective du pouvoir et la perspective du dépassement. Consacrées à l’analyse de l’une et l’autre, les deux parties qui composent le \emph{Traité de savoir-vivre} mériteraient donc d’être abordées non successivement, comme l’exige la lecture, mais simultanément, la description du négatif fondant le projet positif et le projet positif confirmant la négativité. Le meilleur ordre d’un livre, c’est de n’en avoir pas, afin que le lecteur y découvre le sien.\par
Ce qu’il y a de manqué dans l’écriture reflète aussi le manque chez le lecteur, en tant que lecteur et plus encore en tant qu’homme. Si la part d’ennui à l’écrire transparaît dans une certaine part d’ennui à le lire, ce ne sera là qu’un argument de plus pour dénoncer le manque à vivre. Pour le reste, que la gravité du temps excuse la gravité du ton. La légèreté est toujours en deçà ou au-delà des mots. L’ironie, ici, consiste à ne l’oublier jamais.\par
Le Traité de savoir-vivre entre dans un courant d’agitation dont on n’a pas fini d’entendre parler. Ce qu’il expose est une simple contribution parmi d’autres à la réédification du mouvement révolutionnaire international. Son importance ne devrait échapper à personne, car personne, avec le temps, n’échappera à ses conclusions.
\section[{I. L’insignifiant signifié}]{I. L’insignifiant signifié}\renewcommand{\leftmark}{I. L’insignifiant signifié}


\begin{argument}\noindent En se banalisant, la vie quotidienne a conquis peu à peu le centre de nos préoccupations (1). – Aucune illusion, ni sacrée ni désacralisée (2), – ni collective ni individuelle, ne peut dissimuler plus longtemps la pauvreté des gestes quotidiens (3). – L’enrichissement de la vie exige, sans faux-fuyants, l’analyse de la nouvelle pauvreté et le perfectionnement des armes anciennes du refus (4)
\end{argument}

\subsection[{1. La vie quotidienne}]{\textsc{1.} La vie quotidienne}
\noindent L’histoire présente évoque certains personnages de dessins animés, qu’une course folle entraîne soudain au-dessus du vide sans qu’ils s’en aperçoivent, de sorte que c’est la force de leur imagination qui les fait flotter à une telle hauteur ; mais viennent-ils à en prendre conscience, ils tombent aussitôt. \\
 Comme les héros de Bosustov, la pensée actuelle a cessé de flotter par la force de son propre mirage. Ce qui l’avait élevée l’abaisse aujourd’hui. À toute allure elle se jette au-devant de la réalité qui va la briser, la réalité quotidiennement vécue.\par

\astermono

\noindent La lucidité qui s’annonce est-elle d’essence nouvelle ? Je ne le crois pas. L’exigence d’une lumière plus vive émane toujours de la vie quotidienne, de la nécessité, ressentie par chacun, d’harmoniser son rythme de promeneur et la marche du monde. Il y a plus de vérités dans vingt-quatre heures de la vie d’un homme que dans toutes les philosophies. Même un philosophe ne réussit pas à l’ignorer, avec quelque mépris qu’il se traite ; et ce mépris, la consolation de la philosophie le lui enseigne. A force de pirouetter sur lui-même en se grimpant sur les épaules pour lancer de plus haut son message au monde, ce monde, le philosophe finit par le percevoir à l’envers ; et tous les êtres et toutes les choses vont de travers, la tête en bas, pour le persuader qu’il se tient debout, dans la bonne position. Mais il reste au centre de son délire ; ne pas en convenir lui rend simplement le délire plus inconfortable.\par
Les moralistes des XVI\textsuperscript{e} et XVII\textsuperscript{e} siècles règnent sur une resserre de banalités, mais tant est vif leur soin de le dissimuler qu’ils élèvent alentour un véritable palais de stuc et de spéculations. Un palais idéal abrite et emprisonne l’expérience vécue. De là une force de conviction et de sincérité que le ton sublime et la fiction de l’« homme universel » raniment, mais d’un perpétuel souffle d’angoisse. L’analyste, s’efforce d’échapper par une profondeur essentielle à la sclérose graduelle de l’existence ; et plus il s’abstrait de lui-même en s’exprimant selon l’imagination dominante de son siècle (le mirage féodal où s’unissent indissolublement Dieu, le pouvoir royal et le monde), plus sa lucidité photographie la face cachée de la vie, plus elle « invente » la quotidienneté.\par
La philosophie des Lumières accélère la descente vers le concret à mesure que le concret est en quelque sorte porté au pouvoir avec la bourgeoisie révolutionnaire. Des ruines de Dieu, l’homme tombe dans les ruines de sa réalité. Que s’est-il passé ? À peu près ceci : dix mille personnes sont là, persuadées d’avoir vu s’élever la corde d’un fakir, tandis qu’autant d’appareils photographiques démontrent qu’elle n’a pas remué d’un pouce. L’objectivité scientifique dénonce la mystification. Parfait mais pour montrer quoi ? Une corde enroulée, sans le moindre intérêt. J’incline peu à choisir entre le plaisir douteux d’être mystifié et l’ennui de contempler une réalité qui ne me concerne pas. Une réalité sur laquelle je n’ai prise, n’est-ce pas le vieux mensonge remis à neuf, le stade ultime de la mystification ?\par
Désormais, les analystes sont dans la rue. La lucidité n’est pas la seule arme. Leur pensée ne risque plus de s’emprisonner ni dans la fausse réalité des dieux, ni dans la fausse réalité des technocrates !
\subsection[{2. Aucune illusion}]{\textsc{2.} Aucune illusion}
\noindent Les croyances religieuses dissimulaient l’homme à lui-même, leur bastille l’emmurait dans un monde \emph{pyramidal} dont Dieu tenait lieu de sommet et le roi de hauteur. Hélas, il ne s’est pas trouvé le 14 juillet, assez de liberté sur les ruines du pouvoir unitaire pour empêcher les ruines elles-mêmes de s’édifier en prison. Sous le voile lacéré des superstitions n’apparut pas la vérité nue, comme le rêvait Meslier, mais bien la glu des idéologies. Les prisonniers du pouvoir parcellaire n’ont d’autre recours, contre la tyrannie que l’ombre de la liberté.\par
Pas un geste, pas une pensée qui ne s’empêtre aujourd’hui dans le filet des idées reçues. La retombée lente d’infimes fragments issus du vieux mythe explosé répand partout la poussière du sacré, une poussière qui silicose l’esprit et la volonté de vivre. Les contraintes sont devenues moins occultes, plus grossières, moins puissantes, plus nombreuses. La docilité n’émane plus d’une magie cléricale, elle résulte d’une foule de petites hypnoses : information, culture, urbanisme, publicité, suggestions conditionnantes au service de tout ordre établi et à venir. C’est, le corps entravé de toutes parts, Gulliver échoué sur le rivage de Lilliput, résolu à se libérer, promenant autour de lui son regard attentif ; le moindre détail, la moindre aspérité du sol, le moindre mouvement, il n’est rien qui ne revête l’importance d’un indice dont le salut va dépendre. Dans le familier naissent les chances de liberté les plus sûres. En fut-il jamais autrement ? L’art, l’éthique, la philosophie l’attestent : sous l’écorce des mots et des concepts, c’est toujours la réalité vivante de l’inadaptation au monde qui se tient tapie, prête à bondir. Parce que ni les dieux ni les mots ne parviennent aujourd’hui à la couvrir pudiquement, cette banalité-là se promène nue dans les gares et dans les terrains vagues ; elle vous accoste à chaque détour de vous-même, elle vous prend par l’épaule, par le regard ; et le dialogue commence. Il faut se perdre avec elle ou la sauver avec soi.
\subsection[{3. Ni collectif, ni individuel}]{\textsc{3. } Ni collectif, ni individuel}
\noindent Trop de cadavres parsèment les chemins de l’individualisme et du collectivisme. Sous deux raisons apparemment contraires sévissait un même brigandage, une même oppression de l’homme esseulé. La main qui étouffe Lautréamont, on le sait, étrangle aussi Serge Essénine. L’un meurt dans le garni du propriétaire Jules-François Dupuis, l’autre se pend dans un hôtel nationalisé. Partout se vérifie la loi :\par

\begin{quoteblock}
\noindent « Il n’est pas une arme de ta volonté individuelle qui, maniée par d’autres, ne se retourne aussitôt contre toi. ».\end{quoteblock}

\noindent Si quelqu’un dit ou écrit qu’il convient désormais de fonder la raison pratique sur les droits de l’individu et de l’individu seulement, il se condamne dans son propos s’il n’incite aussitôt son interlocuteur à fonder par lui-même la preuve de ce qu’il vient d’avancer. Or une telle preuve ne peut être que vécue, saisie par l’intérieur. C’est pourquoi il n’est rien dans les notes qui suivent qui ne doive être éprouvé et corrigé par l’expérience immédiate de chacun. Rien n’a tant de valeur qu’il ne doive être recommencé, rien n’a assez de richesses qu’il ne doive être enrichi sans relâche.\par

\astermono

\noindent De même que l’on distingue dans la vie privée ce qu’un homme pense et dit de lui, et ce qu’il est et fait réellement, de même il n’est personne qui n’ait appris à distinguer la phraséologie et les prétentions messianiques des partis, et leur organisation, leurs intérêts réels ; ce qu’ils croient être et ce qu’ils sont. L’illusion qu’un homme entretient sur lui et les autres n’est pas foncièrement différente de l’illusion que groupes, classes ou partis nourrissent autour d’eux et en eux. Bien plus, elles découlent d’une source unique : les idées dominantes, qui sont les idées de la classe dominante, même sous leur forme antagoniste.\par
Le monde des \emph{ismes}, qu’il enveloppe l’humanité tout entière ou chaque être particulier, n’est jamais qu’un monde vidé de sa réalité, une séduction terriblement réelle du mensonge. Le triple écrasement de la Commune, du Mouvement spartakiste et de Cronstadt-la-Rouge (1921) a montré une fois pour toutes les autres à quel bain de sang menaient trois idéologies de la liberté : le libéralisme, le socialisme, le bolchevisme. Il a cependant fallu, pour le comprendre et l’admettre universellement, que des formes abâtardies ou amalgamées de ces idéologies vulgarisent leur atrocité initiale par de pesantes démonstrations : les camps de concentration, l’Algérie de Lacoste, Budapest. Aux grandes illusions collectives, aujourd’hui exsangues à force d’avoir fait couler le sang des hommes, succèdent des milliers d’idéologies parcellaires vendues par la société de consommation comme autant de machines à décerveler portatives. Faudra-t-il autant de sang pour attester que cent mille coups d’épingle tuent aussi sûrement que trois coups de massue ?\par

\astermono

\noindent Qu’irais-je faire dans un groupe d’action qui m’imposerait de laisser au vestiaire, je ne dis pas quelques idées – car telles seraient mes idées qu’elles m’induiraient plutôt à rejoindre le groupe en question –, mais les rêves et les désirs dont je ne me sépare jamais, mais une volonté de vivre authentiquement et sans limites ? Changer d’isolement, changer de monotonie, changer de mensonge, à quoi bon ! Où l’illusion d’un changement réel est dénoncée, le simple changement d’illusion devient insupportable. Or telles sont les conditions actuelles : l’économie n’a de cesse de faire consommer davantage, et consommer sans relâche, c’est changer l’illusion à un rythme accéléré qui dissout peu à peu l’illusion du changement. On se retrouve seul, inchangé, congelé dans le vide produit par une cascade de \emph{gadgets}, de Volkswagen et de \emph{pocket books}.\par
Les gens sans imagination se lassent de l’importance conférée au confort, à la culture, aux loisirs, à ce qui détruit l’imagination. Cela signifie qu’on ne se lasse pas du confort, de la culture ou des loisirs, mais de l’usage qui en est fait et qui interdit précisément d’en jouir.\par
L’état d’abondance est un état de voyeurisme. À chacun son kaléidoscope ; un léger mouvement des doigts et l’image se transforme. On gagne à tous les coups : deux réfrigérateurs, une Dauphine, la TV., une promotion, du temps à perdre… Puis la monotonie des images consommées prend le dessus, renvoie à la monotonie du geste qui les suscite, à la légère rotation que le pouce et l’index impriment au kaléidoscope. Il n’y avait pas de Dauphine, seulement une idéologie sans rapport ou presque avec la machine automobile. Imbibé de « Johnny Walker, le whisky de l’Élite », on subissait dans une étrange mixture l’effet de l’alcool et de la lutte des classes. Plus rien de quoi s’étonner, voilà le drame ! La monotonie du spectacle idéologique renvoie maintenant à la passivité de la vie, à la survie. Par-delà les scandales préfabriqués – gaine Scandale et scandale de Panama – se révèle un scandale positif, celui des gestes privés de leurs substance au profit d’une illusion que son attrait perdu rend chaque jour plus odieuse. Gestes futiles et ternes à force d’avoir nourri de brillantes compensations imaginaires, gestes paupérisés à force d’enrichir de hautes spéculations où ils entraient comme valets à tout faire sous la catégorie infamante de « trivial » et de « banal », gestes aujourd’hui libérés et défaillants, prêts à s’égarer de nouveau, ou à périr sous le poids de leur faiblesse. Les voici, en chacun de vous, familiers, tristes, tout nouvellement livrés à la réalité immédiate et mouvante, qui est leur milieu « spontané ». Et vous voici égarés et engagés dans un nouveau prosaïsme, dans une perspective où proche et lointain coïncident.
\subsection[{4. Les armes anciennes}]{\textsc{4.} Les armes anciennes}
\noindent Sous une forme concrète et tactique, le concept de lutte des classes a constitué le premier regroupement des heurts et des dérèglements vécus individuellement par les hommes ; il est né du tourbillon de souffrances que la réduction des rapports humains à des mécanismes d’exploitation suscitait partout dans les sociétés industrielles. Il est issu d’une volonté de transformer le monde et de changer la vie\par
Une telle arme exigeait un perpétuel réajustement. Or ne voit-on pas la 1re Internationale tourner le dos aux artistes, en fondant exclusivement sur les revendications ouvrières un projet dont Marx avait cependant montré combien il concernait tous ceux qui cherchaient, dans le refus d’être esclaves, une vie riche et une humanité totale ? Lacenaire, Borel, Lassailly, Büchner, Baudelaire, Höderlin, n’était-ce pas aussi la misère et son refus radical ? Quoi qu’il en soit, l’erreur, – à l’origine excusable ? je ne veux pas le savoir – revêt des proportions délirantes dès l’instant où, moins d’un siècle plus tard, l’économie de consommation absorbant l’économie de production, l’exploitation de la force de travail est englobée par l’exploitation de la créativité quotidienne. Une même énergie arrachée au travailleur pendant ses heures d’usine ou ses heures de loisirs fait tourner les turbines du pouvoir, que les détenteurs de la vieille théorie lubrifient béatement de leur contestation formelle.\par
Ceux qui parlent de révolution et de lutte de classes sans se référer explicitement à la vie quotidienne, sans comprendre ce qu’il y a de subversif dans l’amour et de positif dans le refus des contraintes, ceux-là ont dans la bouche un cadavre.
\section[{II. L’humiliation}]{II. L’humiliation}\renewcommand{\leftmark}{II. L’humiliation}


\begin{argument}\noindent Fondée sur un échange permanent d’humiliation et d’attitudes agressives, l’économie de la vie quotidienne dissimule une technique d’usure, elle-même en butte au \emph{don} de destruction qu’elle appelle contradictoirement (1). – Plus l’homme est objet, plus il est aujourd’hui social (2). – La décolonisation n’a pas encore commencé (3). – elle se prépare à rendre une valeur nouvelle au vieux principe de souveraineté (4).
\end{argument}

\subsection[{1. Économie de la vie quotidienne : le don de destruction}]{\textsc{1.} Économie de la vie quotidienne : le \emph{don} de destruction}
\noindent Rousseau traversant une bourgade populeuse y fut insulté par un rustre dont la verve mit la foule en joie. Confus, décontenancé, Rousseau ne trouvant mot à lui opposer s’enfuit sous les quolibets. Quand son esprit enfin rasséréné eut fait moisson de réparties assez acerbes pour moucher d’un seul coup le railleur, on était à deux heures du lieu de l’incident.\par
Qu’est-ce le plus souvent que la trivialité quotidienne, sinon l’aventure dérisoire de Jean-Jacques, mais une aventure amenuisée, diluée, émiettée le temps d’un pas, d’un regard, d’une pensée, vécue comme un petit choc, une douleur fugitive presque inaccessible à la conscience et ne laissant à l’esprit qu’une sourde irritation bien en peine de découvrir son origine ? Engagées dans un chassé-croisé sans fin, l’humiliation et sa réplique impriment aux relations humaines un rythme obscène de déhanchements et de claudications. Dans le flux et le reflux des multitudes aspirées et foulées par le va-et-vient des trains de banlieue et envahissant les rues, les bureaux, les usines, ce ne sont que replis craintifs, attaques brutales, minauderies et coups de griffe sans raison avouée. Au gré des rencontres forcées, le vin change en vinaigre à mesure qu’on le déguste. Innocence et bonté des foules, allons donc ! Regardez-les comme ils se hérissent, menacés de toutes parts, lourdement présents sur le terrain de l’adversaire, loin, très loin d’eux-mêmes. Voici le lieu où, à défaut de couteau, ils apprennent à jouer des coudes et du regard.\par
Pas de temps mort, nulle trêve entre agresseurs et agressés. Un flux de signes à peine perceptibles assaille le promeneur, non solitaire. Propos, gestes, regards s’emmêlent, se heurtent, dévient de leur course, s’égarent à la façon des balles perdues, qui tuent plus sûrement par la tension nerveuse qu’elles excitent sans relâche. Nous ne faisons que fermer sur nous-mêmes d’embarrassantes parenthèses ; ainsi ces doigts (j’écris ceci à la terrasse d’un café), ces doigts qui repoussent la monnaie du pourboire et les doigts du garçon qui l’agrippent, tandis que le visage des deux hommes en présence, comme soucieux de masquer l’infamie consentie, revêt les marques de la plus parfaite indifférence.\par
Sous l’angle de la contrainte, la vie quotidienne est régie par un système économique où la production et la consommation de l’offense tendent à s’équilibrer. Le vieux rêve des théoriciens du libre-échange cherche ainsi sa perfection dans les voies d’une démocratie remise à neuf par le manque d’imagination qui caractérise la pensée de gauche. N’est-il pas étrange, au premier abord, l’acharnement des progressistes à décrier l’édifice en ruine du libéralisme, comme si les capitalistes, ses démolisseurs attitrés, n’étaient résolus à l’étatiser et à le planifier ? Pas si étrange en fait, car, polarisant l’attention sur des critiques déjà dépassées par les faits (comme s’il n’était pas établi partout que le capitalisme est lentement accompli par une économie planifiée dont le modèle soviétique aura été un primitivisme), on entend bien dissimuler que c’est précisément sur le modèle de cette économie périmée et soldée à bas prix que l’on reconstruit les rapports humains. Avec quelle persévérance inquiétante les pays « socialistes » ne persistent-ils pas à organiser la vie sur le mode bourgeois ? Partout, c’est le « présentez armes » devant la famille, le mariage, le sacrifice, le travail, l’inauthentique, tandis que des mécanismes homéostatiques simplifiés et rationalisés réduisent les rapports humains à des échanges « équitables » de respects et d’humiliations. Et bientôt, dans l’idéale démocratie des cybernéticiens, chacun gagnera sans fatigues apparentes une part d’indignité qu’il aura le loisir de distribuer selon les meilleures règles de justice ; car la justice distributive atteindra alors son apogée, heureux vieillards qui verrez ce jour-là !\par
Pour moi – et pour quelques autres, j’ose le croire – il n’y a pas d’équilibre dans le malaise. La planification n’est que l’antithèse du libre-échange. Seul l’échange a été planifié, et avec lui les sacrifices mutuels qu’il implique. Or s’il faut garder son sens au mot « nouveauté », ce ne peut être qu’en l’identifiant au dépassement, non au travestissement. Il n’y a, pour fonder une réalité nouvelle, d’autre principe en l’occurrence que le \emph{don}. En dépit de leurs erreurs et de leur pauvreté, je veux voir dans l’expérience historique des conseils ouvriers (1917, 1921, 1934, 1956) comme dans la recherche pathétique de l’amitié et de l’amour une seule et exaltante raison de ne pas désespérer des évidences actuelles. Mais tout s’acharne à tenir secret le caractère positif de telles expériences, le doute est savamment entretenu sur leur importance réelle, voire sur leur existence. Par hasard, aucun historien ne s’est donné la peine d’étudier comment les gens vivaient pendant les moments révolutionnaires les plus extrêmes. La volonté d’en finir avec le libre-échange des comportements humains se révèle donc spontanément par le biais du négatif. Le malaise mis en cause éclate sous les coups d’un malaise plus fort et plus dense.\par
En un sens négatif, les bombes de Ravachol ou, plus près de nous, l’épopée de Caraquemada dissipent la confusion qui règne autour du refus global – plus ou moins attesté mais attesté partout – des relations d’échange et de compromis. Je ne doute pas, pour l’avoir éprouvé maintes fois, que quiconque passe une heure dans la cage des rapports contraignants ne se sente une profonde sympathie pour Pierre-François Lacenaire et la passion du crime. Il ne s’agit nullement de faire ici l’apologie du terrorisme mais de reconnaître en lui le geste le plus pitoyable et le plus digne, susceptible de perturber, en le dénonçant, le mécanisme autorégulateur de la communauté sociale hiérarchisée. S’inscrivant dans la logique d’une société invivable, le meurtre ainsi conçu ne laisse pas d’apparaître comme la forme en creux du \emph{don}. Il est cette absence d’une présence intensément désirée dont parlait Mallarmé, le même qui, au procès des Trente, nomma les anarchistes des « anges de pureté ».\par
Ma sympathie pour le tueur solitaire s’arrête où commence la tactique, mais peut-être la tactique a-t-elle besoin d’éclaireurs poussés par le désespoir individuel. Quoi qu’il en soit, la tactique révolutionnaire nouvelle, celle qui va se fonder indissolublement sur la tradition historique et sur les pratiques, si méconnues et si répandues, de réalisation individuelle, n’a que faire de ceux qui rééditeraient le geste de Ravachol ou de Bonnot. Elle n’en a que faire, mais elle se condamne à l’hibernation théorique si par ailleurs elle ne séduit \emph{collectivement} des individus que l’isolement et la haine du mensonge collectif ont déjà gagnés à la décision rationnelle de tuer et de se tuer. Ni meurtrier, ni humaniste ! Le premier accepte la mort, le second l’impose. Que se rencontrent dix hommes résolus à la violence fulgurante plutôt qu’à la longue agonie de la survie, aussitôt finit le désespoir et commence la tactique. Le désespoir est la maladie infantile des révolutionnaires de la vie quotidienne.\par
L’admiration qu’adolescent j’entretenais pour les hors-la-loi, je la ressens aujourd’hui moins chargée de romantisme désuet que révélatrice des alibis grâce auxquels le pouvoir social s’interdit d’être mis \emph{directement} en cause. L’organisation sociale hiérarchisée est assimilable à un gigantesque \emph{racket} dont l’habileté, précisément percée à jour par le terrorisme anarchiste, consiste à se mettre hors d’atteinte de la violence qu’elle suscite, et à y parvenir en consumant dans une multitude de combats douteux les forces vives de chacun. (Un pouvoir « humanisé » s’interdira désormais de recourir aux vieux procédés de guerre et d’extermination raciste). Les témoins à charge sont peu suspects de sympathies anarchisantes. Ainsi, le biologiste Hans Seyle constate :\par

\begin{quoteblock}
\noindent « À mesure que les agents de maladies spécifiques disparaissent (microbes, sous-alimentation…), une proportion croissante de gens meurent de ce que l’on appelle les maladies d’usure ou maladies de dégénérescence provoquées par le stress, c’est-à-dire par l’usure du corps résultant de conflits, de chocs, de tensions nerveuses, de contrariétés, de rythmes débilitants… »\end{quoteblock}

\noindent Personne n’échappe désormais à la nécessité de mener son enquête sur le \emph{racket} qui le traque jusque dans ses pensées, jusque dans ses rêves. Les moindres détails revêtent une importance capitale. Irritation, fatigue, insolence, humiliation… \emph{cui prodest} ? À qui cela profite-t-il ? Et à qui profitent-elles, les réponses stéréotypées que le « \emph{Big Brother} Bon Sens » répand sous couvert de sagesse, comme autant d’alibis ? Irais-je me contenter d’explications qui me tuent quand j’ai tout à gagner là même où tout est agencé pour me perdre ?
\subsection[{2. Plus l’homme est objet, plus il est social}]{\textsc{2.} Plus l’homme est objet, plus il est social}
\noindent La poignée de main noue et dénoue la boucle des rencontres. Geste à la fois curieux et trivial dont on dit fort justement qu’il \emph{s’échange} ; n’est-il pas en effet la forme la plus simplifiée du contrat social ? Quelles garanties s’efforcent-elles d’assurer, ces mains serrées à droite, à gauche, au hasard, avec une libéralité qui semble suppléer à une nette absence de conviction ? Que l’accord règne, que l’entente sociale existe, que la vie en société est parfaite ? Il ne laisse pas de troubler, ce besoin de s’en convaincre, d’y croire par habitude, de l’affirmer à la force du poignet.\par
Ces complaisances, le regard les ignore, il méconnaît l’échange. Mis en présence, les yeux se troublent comme s’ils devinaient dans les pupilles qui leur font face leur reflet vide et privé d’âme ; à peine se sont-ils frôlés, déjà ils glissent et s’esquivent, leurs lignes de fuite vont en un point virtuel se croiser, traçant un angle dont l’ouverture exprime la divergence, le désaccord fondamentalement ressenti. Parfois l’accord s’accomplit, les yeux s’accouplent ; c’est le beau regard parallèle des couples royaux dans la statuaire égyptienne, c’est le regard embué, fondu, noyé d’érotisme des amants ; les yeux qui de loin se dévorent. Plus souvent, le faible accord scellé dans une poignée de main, le regard le dément. La grande vogue de l’accolade, de l’accord social énergiquement réitéré – dont l’emprunt « shake hand » dit assez l’usage commercial – ne serait-ce pas une ruse au niveau des sens, une façon d’émousser la sensibilité du regard et de l’adapter au vide du spectacle sans qu’il regimbe ? Le bon sens de la société de consommation a porté la vieille expression « voir les choses en face » à son aboutissement logique : ne voir en face de soi que des choses.\par
Devenir aussi insensible et partant aussi maniable qu’une brique, c’est à quoi l’organisation sociale convie chacun avec bienveillance. La bourgeoisie a su répartir plus équitablement les vexations, elle a permis qu’un plus grand nombre d’hommes y soient soumis selon des normes \emph{rationnelles}, au nom d’impératifs concrets et spécialisés (exigences économiques, sociale, politique, juridique…). Ainsi morcelées, les contraintes ont à leur tour émietté la ruse et l’énergie mises communément à les tourner ou à les briser. Les révolutionnaires de 1793 furent grands parce qu’ils osaient détruire l’emprise de Dieu dans le gouvernement des hommes ; les révolutionnaires prolétariens tirèrent de ce qu’ils défendaient une grandeur que l’adversaire bourgeois eût été bien en peine de leur conférer ; leur force, ils la tenaient d’eux seuls.\par
Toute une éthique fondée sur la valeur marchande, l’utile agréable, l’honneur du travail, les désirs mesurés, la survie, et sur leur contraire, la valeur pure, le gratuit, le parasitisme, la brutalité instinctive, la mort, voilà l’ignoble cuvée où les facultés humaines bouillonnent depuis bientôt deux siècles. Voilà de quels ingrédients sûrement améliorés les cybernéticiens méditent d’accommoder l’homme futur. Sommes-nous convaincus de n’atteindre pas déjà à la sécurité des êtres parfaitement adaptés, qui accomplissent leurs mouvements dans l’incertitude et l’inconscience des insectes ? On fait l’essai depuis assez longtemps d’une publicité invisible, par l’introduction dans un déroulement cinématographique d’images autonomes, au 1/24 de seconde, sensibles à la rétine mais restant en deçà d’une perception consciente. Les premiers slogans auguraient parfaitement la suite à prévoir. Ils disaient : « Conduisez moins vite ! », « Allez à l’église ! » Or que représente un petit perfectionnement de cet ordre en regard de l’immense machine à conditionner dont chaque rouage, urbanisme, publicité, idéologie, culture… est susceptible d’une centaine de perfectionnements identiques ? Encore une fois, la connaissance du sort qui va \emph{continuer} d’être fait aux hommes, si l’on n’y prend garde, offre moins d’intérêt que le sentiment vécu d’une telle dégradation. \emph{Le Meilleur des mondes} de Huxley, \emph{1984} d’Orwell et \emph{Le Cinquième Coup de trompette} de Touraine refoulent dans le futur un frisson qu’un simple coup d’œil sur le présent suffirait à provoquer ; et c’est le présent qui porte à maturation la conscience et la volonté de refus. Au regard de mon emprisonnement actuel, le futur est pour moi sans intérêt.\par

\astermono

\noindent Le sentiment d’humiliation n’est rien que le sentiment d’être objet. Il fonde, ainsi compris, une lucidité combative où la critique de l’organisation de la vie ne se sépare pas de la mise en œuvre immédiate d’un projet de vie autre. Oui, il n’y a de construction possible que sur la base du désespoir individuel et sur la base de son dépassement : les efforts entrepris pour maquiller ce désespoir et le manipuler sous un autre emballage suffiraient à le prouver.\par
Quelle est cette illusion qui séduit le regard au point de lui dissimuler l’effritement des valeurs, la ruine du monde, l’inauthenticité, la non-totalité ? Est-ce la croyance en mon bonheur ? Douteux ! Une telle croyance ne résiste ni à l’analyse, ni aux bouffées d’angoisse. J’y découvre plutôt la croyance au bonheur des autres, une source inépuisable d’envie et de jalousie qui fait éprouver par le biais du négatif le sentiment d’exister. J’envie, donc j’existe. Se saisir au départ des autres, c’est se saisir autre. Et l’autre, c’est l’objet, toujours. Si bien que la vie se mesure au degré d’humiliation vécue. Plus on choisit son humiliation, plus on « vit » ; plus on vit de la vie rangée des choses. Voilà la ruse de la réification, ce qui la fait passer comme l’arsenic dans la confiture.\par
La gentillesse prévisible des méthodes d’oppression n’est pas sans expliquer cette perversion qui m’empêche, comme dans le conte de Grimm, de m’écrier « le roi est nu » chaque fois que la souveraineté de ma vie quotidienne dévoile ma misère. Certes la brutalité policière sévit encore, et comment ! Partout où elle s’exerce, les bons esprits de gauche en dénoncent à juste titre l’infamie. Et puis après ? Incitent-ils les masses à s’armer ? Provoquent-ils de légitimes représailles ? Encouragent-ils à une chasse aux flics comme celle qui orna les arbres de Budapest des plus beaux fruits de l’AVO ? Non, ils organisent des manifestations pacifiques ; leur police syndicale traite de provocateurs quiconque résiste à ses mots d’ordre. La nouvelle police est là. Elle attend de prendre la relève. Les psychosociologues gouverneront sans coups de crosse, voire sans morgue. La violence oppressive amorce sa reconversion en une multitude de coups d’épingle raisonnablement distribués. Ceux qui dénoncent du haut de leurs grands sentiments le mépris policier exhortent à vivre déjà dans le mépris policé.\par
L’humanisme adoucit la machine décrite par Kafka dans \emph{La Colonie pénitentiaire}. Moins de grincements, moins de cris. Le sang affole ? Qu’à cela ne tienne, les hommes vivront exsangues. Le règne de la survie promise sera celui de la mort douce, c’est pour cette douceur de mourir que se battent les humanistes. Plus de Guernica, plus d’Auschwitz, plus d’Hiroshima, plus de Sétif. Bravo ! Mais la vie impossible, mais la médiocrité étouffante, mais l’absence de passions ? Et cette colère envieuse où la rancœur de n’être jamais soi invente le bonheur des autres ? Et cette façon de ne se sentir jamais tout à fait dans sa peau ? Que personne ne parle ici de détails, de points secondaires. Il n’y a pas de petites vexations, pas de petits manquements. Dans la moindre éraflure se glisse la gangrène. Les crises qui secouent le monde ne se différencient pas fondamentalement des conflits où mes gestes et mes pensées s’affrontent aux forces hostiles qui les freinent et les dévoient. (Comment ce qui vaut pour ma vie quotidienne cesserait-il de valoir pour l’histoire alors que l’histoire ne prend son importance, en somme, qu’au point d’incidence où elle rencontre mon existence individuelle ?) A force de morceler les vexations et de les multiplier, c’est à l’atome de réalité invivable que l’on va s’en prendre tôt ou tard, libérant soudain une énergie nucléaire que l’on ne soupçonnait plus sous tant de passivité et de morne résignation. Ce qui produit le bien général est toujours terrible.
\subsection[{3. La décolonisation n’a pas encore commencé}]{\textsc{3.} La décolonisation n’a pas encore commencé}
\noindent Le colonialisme a, des années 1945 à 1960, pourvu la gauche d’un père providentiel. Il lui a permis, en lui offrant un adversaire à la taille du fascisme, de ne pas se définir au départ d’elle-même, qui n’était rien, mais de s’affirmer par rapport à autre chose ; il lui a permis de s’accepter comme une chose, dans un ordre où les choses sont tout ou rien.\par
Personne n’a osé saluer la fin du colonialisme de peur de le voir sortir de partout, comme un diable de sa boîte mal fermée. Dès l’instant où le pouvoir colonial s’effondrant dénonçait le colonialisme du pouvoir exercé sur les hommes, les problèmes de couleur et de race prenaient l’importance d’une compétition de mots-croisés. À quoi servaient-elles, les marottes d’antiracisme et d’anti-antisémitisme brandies par les bouffons de la gauche ? En dernière analyse, à étouffer les cris de nègres et de Juifs tourmentés que poussaient tous ceux qui n’étaient ni nègres ni Juifs, à commencer par les Juifs et les nègres eux-mêmes ! Je ne songe évidemment pas à mettre en cause la part de généreuse liberté qui a pu animer les sentiments antiracistes dans le cours d’une époque assez récente encore. Mais le passé m’indiffère dès l’instant où je ne le choisis pas. Je parle aujourd’hui, et personne, au nom de l’Alabama ou de l’Afrique du Sud, au nom d’une exploitation spectaculaire, ne me convaincra d’oublier que l’épicentre de tels troubles se situe en moi et en chaque être humilié, bafoué par tous les égards d’une société soucieuse d’appeler « policé » ce que l’évidence des faits s’obstine à traduire policier\par
Je ne renoncerai pas à ma part de violence.\par
Il n’existe guère en matière de rapports humains d’état plus ou moins supportable, d’indignité plus ou moins admissible ; le quantitatif ne fait pas le compte. Des termes injurieux comme « macaque » ou « bicot » blesseraient-ils plus profondément qu’un rappel à l’ordre ? Qui oserait sincèrement l’assurer ? Interpellé, sermonné, conseillé par un flic, un chef, une autorité, qui ne se sent, au fond de soi et avec cette lucidité des réalités passagères, sans réserves « youpin, raton, chinetoque » ?\par
Quel beau portrait-robot nous offraient du pouvoir les vieux colons prophétisant la chute dans l’animalité et la misère pour ceux qui jugeraient leur présence indésirable ? Sécurité d’abord, dit le gardien au prisonnier. Les ennemis du colonialisme d’hier humanisent le colonialisme généralisé du pouvoir ; ils s’en font les chiens de garde de la manière la plus habile qui soit : en aboyant contre toutes les séquelles de l’inhumanité ancienne.\par
Avant de briguer la charge de président de la Martinique, Aimé Césaire constatait dans une phrase célèbre :\par

\begin{quoteblock}
\noindent « La bourgeoisie s’est trouvée incapable de résoudre les problèmes majeurs auxquels son existence a donné naissance : le problème colonial et le problème du prolétariat. »\end{quoteblock}

\noindent Il oubliait déjà d’ajouter : « car il s’agit là d’un même problème dont on se condamne à ne rien saisir dès l’instant où on les dissocie ».
\subsection[{4. La souveraineté}]{\textsc{4.} La souveraineté}
\noindent Je lis dans Gouy : « La moindre offense au roi coûtait aussitôt la vie » (\emph{Histoire de France}) ; dans la Constitution américaine : « Le peuple est souverain » ; chez Pouget : « Les rois vivaient grassement de leur souveraineté tandis que nous crevons de la nôtre » (\emph{Père Peinard}), et Corbon me dit : « Le peuple groupe aujourd’hui la foule des hommes à qui tous les égards sont refusés » (\emph{Secret du peuple}). En quelques lignes, voici reconstituées les mésaventures du principe de souveraineté.\par
La monarchie désignait sous le nom de « sujets » les \emph{objets} de son arbitraire. Sans doute s’efforçait-elle par là de modeler et d’envelopper l’inhumanité foncière de sa domination dans une humanité de liens idylliques. Le respect dû à la personne du roi n’est pas en soi critiquable. Il ne devient odieux que parce qu’il se fonde sur le droit d’humilier en subordonnant. Le mépris a pourri le trône des monarques. Mais que dire alors de la royauté citoyenne, j’entends : des droits multipliés par la vanité et la jalousie bourgeoises, de la souveraineté accordée comme un dividende à chaque individu ? Que dire du principe monarchique démocratiquement morcelé ?\par
La France compte aujourd’hui vingt-quatre millions de « mini-rois » dont les plus grands – les dirigeants – n’ont pour paraître tels que la grandeur du ridicule. Le sens du respect s’est déchu au point de se satisfaire en humiliant. Démocratisé en fonctions publiques et en rôles, le principe monarchique surnage le ventre en l’air comme un poisson crevé. Seul est visible son aspect le plus repoussant. Sa volonté d’être (sans réserve et absolument) supérieur, cette volonté a disparu. À défaut de fonder sa vie sur la souveraineté, on tente aujourd’hui de fonder sa souveraineté sur la vie des autres. Mœurs d’esclaves.
\section[{III. L’isolement}]{III. L’isolement}\renewcommand{\leftmark}{III. L’isolement}

\noindent Para no sentirme solo \\
Por los siglos de los siglos.

\begin{argument}\noindent Il n’y a de communautaire que l’illusion d’être ensemble. Et contre l’illusion des remèdes licites se dresse seule la volonté générale de briser l’isolement (1). – Les rapports neutres sont le \emph{no man’s land} de l’isolement. L’isolement est un arrêt de mort signé par l’organisation sociale actuelle et prononcé contre elle (2).
\end{argument}

\subsection[{1. Il n’y a de communautaire que l’illusion d’être ensemble}]{\textsc{1.} Il n’y a de communautaire que l’illusion d’être ensemble}
\noindent Ils étaient là comme dans une cage dont la porte eût été grande ouverte, sans qu’ils puissent s’en évader. Rien n’avait plus d’importance en dehors de cette cage, parce qu’il n’existait plus rien d’autre. Ils demeuraient dans cette cage, étrangers à tout ce qui n’était pas elle, sans même l’ombre d’un désir de tout ce qui était au-delà des barreaux. Il eût été anormal, impossible même de s’évader vers quelque chose qui n’avait ni réalité ni importance. Absolument impossible. Car à l’intérieur de cette cage où ils étaient nés et où ils mourraient, le seul climat d’expérience tolérable était le réel, qui était simplement un instinct irréversible de faire en sorte que les choses eussent de l’importance. Ce n’est que si les choses avaient quelque importance que l’on pouvait respirer et souffrir. Il semblait qu’il y eût un accord entre eux et les morts silencieux pour qu’il en fût ainsi, car l’habitude de faire en sorte que les choses eussent de l’importance était devenue un instinct humain et, aurait-on dit, éternel. La vie était ce qui avait de l’importance, et le réel faisait partie de l’instinct qui donnait à la vie un peu de sens. L’instinct n’envisageait pas ce qui pouvait exister au-delà du réel parce qu’au-delà il n’y avait rien. Rien qui eût de l’importance. La porte restait ouverte et la cage devenait plus douloureuse dans sa réalité qui importait pour d’innombrables raisons et d’innombrables manières.\par
Nous ne sommes jamais sortis du temps des négriers.\par
Les gens offrent, dans les transports en commun qui les jettent les uns contre les autres avec une indifférence statisticienne, une expression insoutenable de déception, de hauteur et de mépris, comme l’effet naturel de la mort sur une bouche sans dents. L’ambiance de la fausse communication fait de chacun le policier de ses propres rencontres. L’instinct de fuite et d’agression suit à la trace les chevaliers du salariat, qui n’ont plus, pour assurer leurs pitoyables errances, que le métro et les trains de banlieue. Si les hommes se transforment en scorpions qui se piquent eux-mêmes et les uns les autres, n’est-ce pas en somme parce qu’il ne s’est rien passé et que les humains aux yeux vides et au cerveau flasque sont devenus « mystérieusement » des ombres d’hommes, des fantômes d’hommes, et, jusqu’à un certain point, ne sont plus des hommes que de nom ?\par
Il n’y a de communautaire que l’illusion d’être ensemble. Certes l’amorce d’une vie collective authentique existe à l’état latent au sein même de l’illusion – il n’y a pas d’illusion sans support réel – mais la communauté véritable reste à créer. Il arrive que la force du mensonge efface de la conscience des hommes la dure réalité de leur isolement. Il arrive que l’on oublie dans une rue animée qu’il s’y trouve encore de la souffrance et des séparations. Et, parce que l’on oublie seulement par la force du mensonge, la souffrance et les séparations se durcissent ; et le mensonge aussi se brise les reins sur une telle pierre angulaire. Il n’y a plus d’illusion à la taille de notre désarroi.\par
Le malaise m’assaille à proportion de la foule qui m’entoure. Aussitôt, les compromis qu’au fil des circonstances j’accordai à la bêtise accourent à ma rencontre, affluent vers moi en vagues hallucinantes de têtes sans visage. Le tableau célèbre d’Edward Munch, \emph{Le Cri}, évoque pour moi une impression ressentie dix fois par jour. Un homme emporté par une foule, visible de lui seul, hurle soudain pour briser l’envoûtement, se rappeler à lui, rentrer dans sa peau. Acquiescements tacites, sourires figés, paroles sans vie, veulerie et humiliation émiettés sur ses pas se ramassent, s’engouffrent en lui, l’expulsent de ses désirs et de ses rêves, volatilisent l’illusion d’« être ensemble ». On se côtoie sans se rencontrer ; l’isolement s’additionne et ne se totalise pas ; le vide s’empare des hommes à mesure qu’ils s’accroissent en densité. La foule me traîne hors de moi, laissant s’installer dans ma présence vide des milliers de petits renoncements.\par
Partout les réclames lumineuses reproduisent dans un miroitement de néon la formule de Plotin :\par

\begin{quoteblock}
\noindent « Tous les êtres sont ensemble bien que chacun d’eux reste séparé. »\end{quoteblock}

\noindent Il suffit pourtant d’étendre la main pour se toucher, de lever les yeux pour se rencontrer, et, par ce simple geste, tout devient proche et lointain, comme par sortilège.\par

\astermono

\noindent À l’égal de la foule, de la drogue et du sentiment amoureux, l’alcool possède le privilège d’ensorceler l’esprit le plus lucide. Grâce à lui, le mur bétonné de l’isolement semble un mur de papier que les acteurs déchirent selon leur fantaisie, car l’alcool dispose tout sur un plan théâtral intime. Illusion généreuse et qui tue d’autant plus sûrement.\par
Dans un bar ennuyeux, où les gens se morfondent, un jeune homme ivre brise son verre, saisit une bouteille et la fracasse contre un mur. Personne ne s’émeut ; déçu dans son attente, le jeune homme se laisse jeter dehors. Pourtant, son geste était virtuellement dans toutes les têtes. Lui seul l’a concrétisé, lui seul a franchi la première ceinture radioactive de l’isolement : l’isolement intérieur, cette séparation introvertie du monde extérieur et du moi. Personne n’a répondu à un signe qu’il avait cru explicite. Il est resté seul comme reste le blouson noir qui brûle une église ou tue un policier, en accord avec lui-même mais voué à l’exil tant que les autres vivent exilés de leur propre existence. Il n’a pas échappé au champ magnétique de l’isolement, le voici bloqué dans l’apesanteur. Toutefois, du fond de l’indifférence qui l’accueille, il perçoit mieux les nuances de son cri ; même si cette révélation le torture, il sait qu’il faudra recommencer sur un autre ton, avec plus de force ; avec plus de \emph{cohérence}.\par
Il n’existera qu’une commune damnation tant que chaque être isolé refusera de comprendre qu’un geste de liberté, si faible et si maladroit soit-il, est toujours porteur d’une communication authentique, d’un message personnel adéquat. La répression qui frappe le rebelle libertaire s’abat sur tous les hommes. Le sang de tous les hommes s’écoule avec le sang des Durruti assassinés. Partout où la liberté recule d’un pouce, elle accroît au centuple le poids de l’ordre des \emph{choses}. Exclus de la participation authentique, les gestes de l’homme se dévoient dans la frêle illusion d’être ensemble ou dans son contraire, le refus brutal et absolu du social. Ils oscillent de l’un à l’autre dans un mouvement de balancier qui fait courir les heures sur le cadran de la mort.\par

\astermono

\noindent Et l’amour à son tour engrosse l’illusion d’unité. Et ce ne sont la plupart du temps qu’avortements et foutaises. La peur de refaire à deux ou à dix un chemin trop pareil et trop connu, celui de l’esseulement, menace les symphonies amoureuses de son accord glacé. Ce n’est pas l’immensité du désir insatisfait qui désespère mais la passion naissante confrontée à son vide. Le désir inextinguible de connaître passionnément tant de filles charmantes naît dans l’angoisse et dans la peur d’aimer, tant l’on craint de ne se libérer jamais des rencontres d’\emph{objets}. L’aube où se dénouent les étreintes est pareille à l’aube où meurent les révolutionnaires sans révolution. L’isolement à deux ne résiste pas à l’isolement de tous. Le plaisir se rompt prématurément, les amants se retrouvent nus dans le monde, leurs gestes devenus soudain ridicules et sans force. Il n’y a pas d’amour possible dans un monde malheureux.\par
La barque de l’amour se brise contre la vie courante.\par
Es-tu prêt, afin que jamais ton désir ne se brise, es-tu prêt à briser les récifs du vieux monde ? Il manque aux amants d’aimer leur plaisir avec plus de conséquence et de poésie. Le prince Shekour, dit-on, s’empara d’une ville et l’offrit à sa favorite pour le prix d’un sourire. Nous voici quelques-uns épris du plaisir d’aimer sans réserve, assez passionnément pour offrir à l’amour le lit somptueux d’une révolution.
\subsection[{2. Les rapports neutres}]{\textsc{2.} Les rapports neutres}
\noindent S’adapter au monde est un jeu de pile ou face où l’on décide \emph{a priori} que le négatif devient positif, que l’impossibilité de vivre fonde les conditions \emph{sine qua non} de la vie. Jamais l’aliénation ne s’incruste si bien que lorsqu’elle se fait passer pour un bien inaliénable. Muée en positivité, la conscience de l’isolement n’est autre que la conscience privée, ce morceau d’individualisme incessible que les braves gens traînent avec eux comme leur propriété, encombrante et chère. C’est une sorte de plaisir-angoisse qui empêche à la fois que l’on se fixe à demeure dans l’illusion communautaire et que l’on reste bloqué dans les sous-sols de l’isolement.\par
Le \emph{no man’s land} des rapports neutres étend son territoire entre l’acceptation béate des fausses collectivités et le refus global de la société. C’est la morale de l’épicier, les « il faut bien s’entraider », « il y a des honnêtes gens partout », « tout n’est pas si mauvais, tout n’est pas si bon, il suffit de choisir », c’est la politesse, l’art pour l’art du malentendu.\par
Reconnaissons-le, les rapports humains étant ce que la hiérarchie sociale en fait, les rapports neutres offrent la forme la moins fatigante du mépris ; ils permettent de passer sans frictions inutiles à travers les trémies des contacts quotidiens. Ils n’empêchent pas de rêver, bien loin de là, à des formes de civilités supérieures, telle la courtoisie de Lacenaire, la veille de son exécution, pressant un ami :\par

\begin{quoteblock}
 \noindent « Surtout, je vous prie, portez mes remerciements à M. Scribe. Dites-lui qu’un jour, contraint par la faim, je me suis rendu chez lui pour lui soutirer de l’argent. Il a accédé à ma demande avec beaucoup de déférence ; il s’en souviendra, je pense. Dites-lui aussi qu’il a bien fait, car j’avais dans ma poche, à portée de la main, de quoi priver la France d’un auteur dramatique. »
 \end{quoteblock}

\noindent Mais l’innocuité des rapports neutres n’est qu’un temps mort dans la lutte incessante contre l’isolement, un lieu de passage rapide qui conduit la communication, et bien plus fréquemment, d’ailleurs, vers l’illusion communautaire. J’expliquerais assez ma répugnance d’arrêter un inconnu pour lui demander l’heure, un renseignement, deux mots… par cette façon douteuse de rechercher le contact : la gentillesse des rapports neutres construit lourdement sur le sable ; le temps vide ne me profite jamais.\par
L’impossibilité de vivre est partout garantie avec un tel cynisme que le plaisir-angoisse équilibré des rapports neutres participe au mécanisme général de démolition des hommes. Il semble en fin de compte préférable d’entrer sans atermoiements dans le refus radical tactiquement élaboré que de frapper gentiment à toutes les portes où s’échange une survie contre une autre.\par
« Je serais ennuyé de mourir si jeune », écrivait Jacques Vaché, deux ans avant de se suicider. Si le désespoir de survivre ne s’unit à la nouvelle prise de conscience pour bouleverser les années qui vont suivre, il ne restera que deux « excuses » à l’homme isolé : la chaise percée des partis et des sectes pataphysico-religieuses, ou la mort immédiate avec Umour. Un assassin de seize ans déclarait récemment : « J’ai tué parce que je m’ennuyais. » Quiconque a déjà senti monter en lui la force de sa propre destruction sait avec quelle négligente lassitude il pourrait lui arriver de tuer les organisateurs de l’ennui. Un jour. Par hasard.\par
Enfin, qu’un individu refuse également la violence de l’inadapté et l’adaptation à la violence du monde, où trouvera-t-il sa voie ? S’il n’élève au niveau d’une théorie et d’une pratique cohérentes sa volonté de parfaire l’unité avec le monde et avec soi, le grand silence des espaces sociaux bâtit pour lui le palais des délires solipsistes.\par
Les condamnés aux maladies mentales jettent, du fond de leur prison, les cris d’une révolte sabrée dans le négatif. Quel Fourier savamment mis à mort dans ce malade dont l’aliéniste Volnat :\par

\begin{quoteblock}
\noindent « En lui commençait une indifférence entre son moi et le monde extérieur. Tout ce qui se passait dans le monde se passait aussi dans son corps. Il ne pouvait placer une bouteille entre deux rayons d’un placard, car les rayons se rapprochant pouvaient briser la bouteille. Et ça lui serrait dans la tête. C’était comme si sa tête était serrée entre les rayons du placard. Il ne pouvait fermer une valise, car pressant les objets dans la valise, ça lui pressait dans la tête comme dans la valise. S’il sortait dans la rue après avoir fermé les portes et les fenêtres de sa maison, il se trouvait incommodé, son cerveau était compressé par l’air, et il devait retourner chez lui pour ouvrir une porte ou une fenêtre. « Pour que je sois à mon aise, disait-il, il me faudrait l’étendue, le champ libre. […] Il faudrait que je sois \emph{libre de mon espace}. C’est la lutte avec les \emph{choses} qui sont autour de moi. »\end{quoteblock}

\noindent Le Consul s’arrêta. Il lut l’inscription : « No se puede vivir sin amor » (Lowry : \emph{Au-dessous du volcan}).
\section[{IV. La souffrance}]{IV. La souffrance}\renewcommand{\leftmark}{IV. La souffrance}


\begin{argument}\noindent La souffrance de l’aliénation naturelle a fait place à la souffrance de l’aliénation sociale, tandis que les remèdes devenaient des justifications (1) – Où la justification manque, les exorcismes suppléent (2) – Mais aucun subterfuge ne dissimule désormais l’existence d’une organisation de la souffrance, tributaire d’une organisation fondée sur la répartition des contraintes (3). – La conscience réduite à la conscience des contraintes est l’antichambre de la mort. Le désespoir de la conscience fait les meurtriers de l’ordre, la conscience du désespoir, les meurtriers du désordre (4).
\end{argument}

\subsection[{1. De l’aliénation naturelle à l’aliénation sociale}]{\textsc{1.} De l’aliénation naturelle à l’aliénation sociale}
\noindent La symphonie des cris et des paroles offre au décor des rues une dimension mouvante. Sur une base continue se modulent des thèmes graves ou légers, voix éraillées, appels chantants, éclats nostalgiques de phrase sans fin. Une architecture sonore se superpose au tracé des rues et des façades, elle complète ou corrige la note attrayante ou répulsive d’un quartier. Cependant, de la Contrescarpe aux Champs-Élysées, les accords de base sonnent partout les mêmes : leur résonance sinistre s’est si bien incrustée dans toutes les oreilles qu’elle a cessé d’étonner. « C’est la vie », « on ne changera pas l’homme », « ça va comme ça va », « il faut se faire une raison », « ce n’est pas drôle tous les jours »… Ce lamento dont la trame unifie les conversations les plus diverses a si bien perverti la sensibilité qu’il passe pour la tournure la plus commune des dispositions humaines. Là où il n’est pas accepté, le désespoir tend le plus souvent à n’être plus perceptible. La joie absente depuis deux siècles de la musique européenne semble n’inquiéter personne, c’est tout dire. Consommer, consumer : la cendre est devenue norme du feu.\par
D’où tire-t-elle son origine, cette importance usurpée par la souffrance et par ses rites d’exorcisme ? Sans doute des dures conditions de survie imposées aux premiers hommes dans une nature hostile, parcourue de forces brutales et mystérieuses. Face aux dangers, la faiblesse des hommes découvrait dans l’agglomérat social non seulement une protection mais une manière de coopérer avec la nature, de pactiser avec elle et même de la transformer. Dans la lutte contre l’aliénation naturelle (la mort, la maladie, la souffrance), l’aliénation est devenue sociale. Et à leur tour, la mort, la maladie, la souffrance devinrent – quoi qu’on en pense – sociales. On échappait aux rigueurs du climat, à la faim, à l’inconfort pour tomber dans les pièges de l’esclavage. L’esclavage aux dieux, aux hommes, au langage. Et pourtant, un tel esclavage comportait sa part de victoire, il y avait de la grandeur à vivre dans la terreur d’un dieu qui vous rendait par ailleurs invincible. Ce brassage de l’humain et de l’inhumain suffirait certes à expliquer l’ambiguïté de la souffrance, sa façon d’apparaître tout au long de l’histoire des hommes à la fois comme un mal honteux et comme un mal salutaire, un bien, en quelque sorte. Il faut cependant compter ici avec l’ignoble tare des religions, avec la mythologie chrétienne surtout, qui mit son génie à porter au plus haut point de perfection cette suggestion morbide et dépravée : prémunis-toi contre la mutilation par la mutilation volontaire !\par
« Depuis la venue du Christ, nous sommes délivrés non du mal à souffrir mais du mal de souffrir inutilement », écrit fort justement le P. Charles de la Compagnie de Jésus. Le problème du pouvoir n’a jamais été de se supprimer mais de se donner une raison afin de ne pas opprimer « inutilement ». En mariant la souffrance à l’homme, sous prétexte de grâce divine ou de loi naturelle, le christianisme, cette thérapeutique maladive, a réussi son « coup de maître ». Du prince au manager, du prêtre au spécialiste, du directeur de conscience au psychologique, c’est toujours le principe de la souffrance utile et du sacrifice consenti qui constitue la base la plus solide du pouvoir hiérarchisé. Quelle que soit sa raison invoquée, monde meilleur, au-delà, société socialiste ou futur enchanteur, la souffrance acceptée est toujours chrétienne, toujours. À la vermine cléricale succèdent aujourd’hui les zélateurs d’un Christ passé au rouge. Partout les revendications officielles portent en filigrane la dégoûtante effigie de l’homme en croix, partout les camarades sont priés d’arborer la stupide auréole du militant martyr. Les malaxeurs de la bonne Cause préparent avec le sang versé les cochonnailles du futur : moins de chair à canon, plus de chair à principe !\par

\astermono

\noindent À première vue, l’idéologie bourgeoise paraissait résolue à traquer la souffrance avec autant d’opiniâtreté qu’elle en mettait à poursuivre les religions de sa haine. Entichée de progrès, de confort, de profit, de bien-être, de raison, elle possédait assez d’armes – sinon les armes réelles, du moins celles de l’illusion – pour convaincre de sa volonté d’en finir scientifiquement avec le mal de souffrir et le mal de croyance. Elle ne devait, on le sait, qu’inventer de nouveaux anesthésiques, de nouvelles superstitions.\par
On ôta Dieu, et la souffrance devint « naturelle », inhérente à la « nature humaine » ; on en venait à bout, mais par d’autres souffrances compensatoires : les martyrs de la science, les victimes du progrès, les générations sacrifiées. Or, dans ce mouvement même, la notion de souffrance naturelle dévoilait sa racine sociale. On ôta la Nature humaine, et la souffrance devint sociale, inhérente à l’être-dans-la-société. Mais, bien entendu, les révolutions démontrèrent que le mal social n’était pas un principe métaphysique ; qu’il pouvait exister une forme de société d’où le mal de vivre serait exclu. L’histoire brisait l’ontologie sociale, mais voici que la souffrance, loin de disparaître, trouvait de nouvelles raisons dans les exigences de l’histoire, soudain figée à son tour dans son fameux sens unique. La Chine prépare les enfants à la société sans classe en leur enseignant l’amour de la patrie, l’amour de la famille et l’amour du travail. L’ontologie historique ramasse les résidus de tous les systèmes métaphysiques passés, tous les en-soi, Dieu, la Nature, l’Homme, la Société. Désormais, les hommes font l’histoire contre l’Histoire elle-même, parce que l’Histoire est devenue le dernier rempart ontologique du pouvoir, la ruse ultime où il dissimule, sous la promesse d’un long week-end, sa volonté de \emph{durer} jusqu’au samedi qui ne viendra jamais. Au-delà de l’histoire fétichisée, la souffrance se révèle dépendante de l’organisation sociale hiérarchisée. Et quand la volonté d’en finir avec le pouvoir hiérarchisé aura suffisamment chatouillé la conscience des hommes, chacun conviendra que la liberté armée et le poids des contraintes n’ont rien de métaphysique.
\subsection[{2. Où la justification manque, les exorcismes suppléent}]{\textsc{2.} Où la justification manque, les exorcismes suppléent}
\noindent Tout en mettant à l’ordre du jour le bonheur et la liberté, la civilisation technicienne inventait l’\emph{idéologie} du bonheur et de la liberté. Elle se condamnait donc à ne rien créer qu’une liberté d’apathie, un bonheur dans la passivité. Du moins l’invention, toute pervertie qu’elle soit, avait suffi pour nier universellement qu’il y ait une souffrance inhérente à la condition d’être humain, qu’il puisse exister de toute éternité une condition humaine. C’est pourquoi la pensée bourgeoise échoue à vouloir consoler de la souffrance : aucune de ses justifications n’atteint à la force d’espérance que suscita jadis son pari fondamental sur la technique et le bien-être.\par
La fraternité désespérée dans la maladie est ce qui peut arriver de pire à une civilisation. C’est moins la mort qui épouvante les hommes du XX\textsuperscript{e} siècle que l’absence de vraie vie. Chaque geste mort, mécanisé, spécialisé, ôtant une part de vie cent fois, mille fois par jour jusqu’à l’épuisement de l’esprit et du corps, jusqu’à cette fin qui n’est plus la fin de la vie mais une absence arrivée à saturation, voilà qui risque de donner du charme aux apocalypses, aux destructions géantes, aux anéantissements complets, aux morts brutales, totales et propres. Auschwitz et Hiroshima sont bien le « réconfort du nihilisme ». Il suffit que l’impuissance à vaincre la souffrance devienne un sentiment collectif, et l’exigence de souffrir et de mourir peut s’emparer soudain d’une communauté. Consciemment ou non, la plupart des gens préfèrent mourir plutôt que de ressentir en permanence l’insatisfaction de vivre. J’ai toujours vu dans les cortèges anti-atomiques – si j’excepte une minorité agissante de radicaux – une majorité de pénitents cherchant à exorciser leur propre désir de disparaître avec l’humanité tout entière. Ils s’en défendent évidemment, mais leur peu de joie – il n’y a de vraie joie que révolutionnaire – témoigne contre eux, sans appel.\par
Peut-être est-ce aux fins d’éviter qu’un universel désir de périr ne s’empare des hommes qu’un véritable spectacle s’organise autour des misères et des douleurs particulières. Une sorte de philanthropie d’utilité publique pousse chacun à se réconforter de ses propres infirmités au spectacle de celles des autres.\par
Cela va des photos de catastrophe, du drame du chanteur cocu, des rengaines à la Berthe Sylva, de la vidange dérisoire de \emph{France-Soir}, aux hôpitaux, aux asiles, aux prisons, véritables musées de consolation à l’usage de ceux que leur crainte d’y entrer fait se réjouir de n’y être pas. J’ai le sentiment parfois d’une telle souffrance diffuse, éparse en moi, qu’il m’arrive de regarder comme un soulagement le malheur occasionnel qui la concrétise, la justifie, lui offre un exutoire licite. Rien ne me dissuadera de cette conviction : ma tristesse éprouvée lors d’une rupture, d’un échec, d’un deuil, ne m’atteint pas de l’extérieur comme une flèche mais sourd de moi telle une source qu’un glissement de terrain vient de libérer. Il y a des blessures qui permettent à l’esprit de pousser un cri longtemps contenu. Le désespoir ne lâche jamais sa proie ; c’est seulement la proie qui voit le désespoir dans la fin d’un amour ou la mort d’un enfant, là où il n’y a que son ombre portée. Le deuil est un prétexte, une façon commode d’éjaculer le néant à petits coups. Les pleurs, les cris, les hurlements de l’enfance restent emprisonnés dans le cœur des hommes. À jamais ? En toi aussi le vide ne cesse de gagner.
\subsection[{3. Organisation de la souffrance : répartition des contraintes}]{\textsc{3.} Organisation de la souffrance : répartition des contraintes}
\noindent Je dirai un mot encore des alibis du pouvoir. Supposons qu’un tyran prenne plaisir à jeter dans une étroite cellule des prisonniers préalablement pelés vifs, qu’entendre leurs cris atroces et les voir se battre chaque fois qu’ils se frôlent le divertisse fort, tout en l’incitant à méditer sur la nature humaine et le curieux comportement des hommes. Supposons qu’à la même époque et dans le même pays il se trouve des philosophes et des savants pour expliquer au monde de la science et des arts que la souffrance tient à la mise en commun des hommes, à l’inévitable présence des Autres, à la société en tant que telle, ne serait-on pas fondé à considérer ces gens comme les chiens de garde du tyran ? En répandant pareilles thèses, une certaine conception existentialiste a, par ricochet, frappé d’évidence et d’une pierre deux coups la collusion des intellectuels de gauche avec le pouvoir et la ruse grossière par laquelle une organisation sociale inhumaine attribue à ses propres victimes la responsabilité de ses cruautés. Un publiciste écrivait au XIX\textsuperscript{e} siècle :\par

\begin{quoteblock}
\noindent « On trouve à chaque pas, dans la littérature de nos jours, la tendance à regarder les souffrances individuelles comme un mal social et à rendre l’organisation de notre société responsable de la misère et de la dégradation de ses membres. Voilà une idée profondément nouvelle. On a cessé de prendre ses maux comme venant de la fatalité. »\end{quoteblock}

\noindent Une « nouveauté » si actuelle semble n’avoir pas troublé outre mesure les bons esprits confits de fatalité : Sartre et l’enfer des autres, Freud et l’instinct de mort, Mao et la nécessité historique. Quelle différence après tout avec le stupide : « Les hommes sont ainsi faits » ?\par
L’organisation sociale hiérarchisée est comparable à un système de trémies et de lames effilées. En nous écorchant vifs, le pouvoir met son point d’habileté à nous persuader que nous nous écorchons mutuellement. Se borner à l’écrire risque, il est vrai, de nourrir une nouvelle fatalité : mais j’entends bien, en l’écrivant, que personne ne se borne à le lire.\par

\astermono

\noindent L’altruisme se situe au verso de l’« enfer des autres », la mystification s’offrant cette fois sous le signe du positif. Qu’on en finisse une fois pour toutes avec cet esprit d’ancien combattant ! Pour que les autres m’intéressent, il faut d’abord que je trouve en moi la force d’un tel intérêt. Il faut que ce qui me lie aux autres apparaisse à travers ce qui me lie à la part la plus riche et la plus exigeante de ma volonté de vivre. Non l’inverse. Dans les autres, c’est toujours moi que je cherche, et mon enrichissement, et ma réalisation. Que chacun en prenne conscience et le « chacun pour soi » mené à ses conséquences ultimes débouchera sur le « tous pour chacun ». La liberté de l’un sera la liberté de tous. Une communauté qui ne s’érige pas au départ des exigences individuelles et de leur dialectique ne peut que renforcer la violence oppressive du pouvoir. L’Autre où je ne me saisis pas n’est qu’une chose et c’est bien à l’amour des choses que l’altruisme me convie. À l’amour de mon isolement.\par
Vu sous l’angle de l’altruisme ou de la solidarité – cet altruisme de gauche – le sentiment d’égalité marche la tête en bas. Qu’est-ce d’autre que l’angoisse commune aux sociétaires isolés, humiliés, baisés, battus, cocus, contents, l’angoisse de parcelles séparées, aspirant à se rejoindre non dans la réalité mais dans une unité mystique, n’importe quelle unité, celle de la nation ou celle du mouvement ouvrier, peu importe pourvu qu’on s’y sente comme dans les soirs de grandes beuveries « tous frères » ? L’égalité dans la grande famille des hommes exalte l’encens des mystifications religieuses. Il faut avoir les narines obturées pour ne pas s’en trouver mal.\par
Pour moi, je ne reconnais d’autre égalité que celle que ma volonté de vivre selon mes désirs reconnaît dans la volonté de vivre des autres. L’égalité révolutionnaire sera indissolublement individuelle et collective.
\subsection[{4. Désespoir de la conscience et conscience du désespoir}]{\textsc{4.} Désespoir de la conscience et conscience du désespoir}
\noindent Dans la perspective du pouvoir, un seul horizon : la mort. Et tant va la vie à ce désespoir qu’à la fin elle s’y noie. Partout où vient à stagner l’eau vive du quotidien les traits du noyé reflètent le visage des vivants, le positif est, à y bien regarder, négatif, le jeune est déjà le vieux et ce qui se construit atteint l’ordre des ruines. Au royaume du désespoir, la lucidité aveugle à l’égal du mensonge. On meurt de ne pas savoir, frappé par-derrière. Par ailleurs, la conscience de la mort qui guette accroît la torture et précipite l’agonie. L’usure des gestes freinés, entravés, interdits, ronge plus sûrement qu’un cancer, mais rien ne généralise le « cancer » comme la conscience claire d’une telle usure. Rien, j’en reste persuadé, ne peut sauver de l’anéantissement un homme à qui l’on poserait sans relâche la question : « As-tu repéré la main qui, avec tous les égards, te tue ? » Évaluer l’impact de chaque brimade, estimer au pèse-nerf le poids de chaque contrainte, cela suffit pour acculer l’individu le plus solide à un sentiment unique et envahissant, le sentiment d’une faiblesse atroce et d’une impuissance totale. C’est du fond de l’esprit que monte la vermine des contraintes, à laquelle rien d’humain ne résiste.\par
Parfois il me semble que le pouvoir me rend pareil à lui : une grande force sur le point de s’effondrer, une rage impuissante à sévir, un désir de totalité soudain racorni. Un ordre impuissant ne règne qu’en assurant l’impuissance de ses esclaves ; Franco et Battista, émasculant les prisonniers révolutionnaires, ont su le démontrer avec brio. Les régimes plaisamment baptisés « démocratiques » ne font qu’humaniser la castration : provoquer le vieillissement précoce paraît à première vue moins féodal que la technique du couteau et de la ligature. À première vue seulement, car sitôt qu’un esprit lucide a compris que par l’esprit venait désormais l’impuissance, on peut allégrement déclarer que la partie est perdue !\par
Il existe une prise de conscience admise par le pouvoir parce qu’elle sert ses desseins. Emprunter sa lucidité à la lumière du pouvoir, c’est rendre lumineuse l’obscurité du désespoir, c’est nourrir sa vérité de mensonge. Le stade esthétique se définit : ou la mort contre le pouvoir, ou la mort dans le pouvoir ; Arthur Cravan et Jacques Vaché, d’une part, le SS, le para, le tueur à gages de l’autre. La mort est chez eux un aboutissement logique et naturel, la confirmation suprême d’un état de fait permanent, le dernier point de suspension d’une ligne de vie où rien en fin de compte ne fut écrit. Ce qui n’échappe pas à l’attraction presque universelle du pouvoir tombe uniformément. C’est toujours le cas de la bêtise et de la confusion mentale, c’est souvent le cas de l’intelligence. La fêlure est la même chez Drieu et Jacques Rigaux, mais elle est de signe contraire, l’impuissance du premier est taillée dans la soumission et la servilité, la révolte du second se brise prématurément sur l’impossible. Le désespoir de la conscience fait les meurtriers de l’ordre, la conscience du désespoir, les meurtriers du désordre. A la chute dans le conformisme des prétendus anarchistes de droite répond, par l’effet d’une gravitation identique, la chute des archanges damnés dans les dents d’acier de la souffrance. Au fond du désespoir résonnent les crécelles de la contre-révolution.\par
La souffrance est le mal des contraintes. Une parcelle de joie pure, si infime soit-elle, la tient en respect. Renforcer la part de joie et de fête authentiques ressemble à s’y méprendre aux apprêts d’une insurrection générale.\par
De nos jours, les gens sont invités à une gigantesque chasse aux mythes et aux idées reçues mais, qu’on ne s’y trompe pas, on les envoie sans armes ou pis, avec les armes en papier de la spéculation pure, dans le marécage des contraintes où ils achèvent de s’enliser. C’est pourquoi la joie naîtra peut-être d’abord de pousser, les premiers en avant, les idéologues de la démystification, afin qu’observant comment ils se tirent d’affaire on puisse tirer parti de leurs actes ou avancer sur leurs corps.\par
Les hommes sont, comme l’écrit Rosanov, écrasés par l’armoire. Si l’on ne soulève pas l’armoire, il est impossible de délivrer d’une souffrance éternelle et insupportable des peuples entiers. Il est terrible d’écraser, ne fût-ce qu’un seul homme. Voici qu’il veut respirer et qu’il ne peut plus respirer. L’armoire recouvre tous les hommes et cependant chacun reçoit sa part incessible de souffrance. Et tous les hommes s’efforcent de soulever l’armoire, mais pas avec la même conviction, pas avec la même force. Étrange civilisation gémissante.\par
Les penseurs s’interrogent : « Des hommes sous l’armoire ! Comment se sont-ils mis là-dessous ? » Néanmoins, ils s’y sont mis. Et si quelqu’un vient au nom de l’objectivité démontrer qu’on n’arrive pas à bout d’un tel fardeau, chacune de ses phrases, chacune de ses paroles accroît le poids de l’armoire, de cet objet qu’il entend représenter par l’universalité de « sa conscience objective ». Et tout l’esprit chrétien est là, qui s’est donné rendez-vous, il caresse la souffrance comme un bon chien, il diffuse la photo d’hommes écrasés et souriants. « La raison de l’armoire est toujours la meilleure » laissent entendre des milliers de livres publiés chaque jour pour être rangés dans l’armoire. Et cependant tout le monde veut respirer et personne ne peut respirer, et beaucoup disent : « Nous respirerons plus tard », et la plupart ne meurent pas, car ils sont déjà morts.\par
Ce sera maintenant ou jamais.
\section[{V. Déchéance du travail}]{V. Déchéance du travail}\renewcommand{\leftmark}{V. Déchéance du travail}


\begin{argument}\noindent L’obligation de produire aliène la passion de créer. Le travail productif relève des procédés de maintien de l’ordre. Le temps de travail diminue à mesure que croît l’empire du conditionnement.
\end{argument}

\noindent Dans une société industrielle qui confond travail et productivité, la nécessité de produire a toujours été antagoniste au désir de créer. Que reste-t-il d’étincelle humaine, c’est-à-dire de créativité possible, chez un être tiré du sommeil à six heures chaque matin, cahoté dans les trains de banlieue, assourdi par le fracas des machines, lessivé, bué par les cadences, les gestes privés de sens, le contrôle statistique, et rejeté vers la fin du jour dans les halls de gares, cathédrales de départ pour l’enfer des semaines et l’infime paradis des week-ends, où la foule communie dans la fatigue et l’abrutissement ? De l’adolescence à l’âge de la retraite, les cycles de vingt-quatre heures font succéder leur uniforme émiettement de vitre brisée : fêlure du rythme figé, fêlure du temps – qui-est-de-l’argent, fêlure de la soumission aux chefs, fêlure de l’ennui, fêlure de la fatigue. De la force vive déchiquetée brutalement à la déchirure béante de la vieillesse, la vie craque de partout sous les coups du travail forcé. Jamais une civilisation n’atteignit à un tel mépris de la vie ; noyé dans le dégoût, jamais une génération n’éprouva à ce point le goût enragé de vivre. Ceux qu’on assassine lentement dans les abattoirs mécanisés du travail, les voici qui discutent, chantent, boivent, dansent, baisent, tiennent la rue, prennent les armes, inventent une poésie nouvelle. Déjà se constitue le front contre le travail forcé, déjà les gestes de refus modèlent la conscience future. Tout appel à la productivité est, dans les conditions voulues par le capitalisme et l’économie soviétisée, un appel à l’esclavage.\par
La nécessité de produire trouve si aisément ses justifications que le premier Fourastié venu en farcit dix livres sans peine. Par malheur pour les néo-penseurs de l’économisme, ces justifications sont celles du XIX\textsuperscript{e} siècle, d’une époque où la misère des classes laborieuses fit du droit au travail l’homologue du droit à l’esclavage, revendiqué à l’aube des temps par les prisonniers voués au massacre. Il s’agissait avant tout de ne pas disparaître physiquement, de survivre. Les impératifs de productivité sont des impératifs de survie ; or les gens veulent désormais vivre, non seulement survivre.\par
Le \emph{tripalium} est un instrument de torture. \emph{Labor} signifie « peine ». Il y a quelque légèreté à oublier l’origine des mots « travail » et « labeur ». Les nobles avaient du moins la mémoire de leur dignité comme de l’indignité qui frappait leurs esclavages. Le mépris aristocratique du travail reflétait le mépris du maître pour les classes dominées ; le travail était l’expiation à laquelle les condamnait de toute éternité le décret divin qui les avait voulues, pour d’impénétrables raisons, inférieures. Le travail s’inscrivait, parmi les sanctions de la Providence, comme la punition du pauvre, et parce qu’elle régissait aussi le salut futur, une telle punition pourrait revêtir les attributs de la joie. Au fond, le travail importait moins que la soumission.\par
La bourgeoisie ne domine pas, elle exploite. Elle soumet peu, elle préfère \emph{user}. Comment n’a-t-on pas vu que le principe du travail productif se substituait simplement au principe d’autorité féodal ? Pourquoi n’a-t-on pas voulu le comprendre ?\par
Est-ce parce que le travail améliore la condition des hommes et sauve les pauvres, illusoirement du moins, de la damnation éternelle ? Sans doute, mais il appert aujourd’hui que le chantage sur les lendemains meilleurs succède docilement au chantage sur le salut de l’au-delà. Dans l’un et l’autre cas, le présent est toujours sous le coup de l’oppression.\par
Est-ce parce qu’il transforme la nature ? Oui, mais que ferais-je d’une nature ordonnée en termes de profits dans un ordre de choses où l’inflation technique couvre la déflation sur l’emploi de la vie ? D’ailleurs, de même que l’acte sexuel n’a pas pour fonction de procréer mais engendre très accidentellement des enfants, c’est par surcroît que le travail organisé transforme la surface des continents, par prolongement et non par motivation. Travailler pour transformer le monde ? Allons donc ! Le monde se transforme dans le sens où il existe un travail forcé ; et c’est pourquoi il se transforme si mal.\par
L’homme se réaliserait-il dans son travail forcé ? Au XIX\textsuperscript{e} siècle, il subsistait dans la conception du travail une trace infime de créativité. Zola décrit un concours de cloutiers où les ouvriers rivalisent d’habileté pour parfaire leur minuscule chef-d’œuvre. L’amour du métier et la recherche d’une créativité cependant malaisée permettaient sans conteste de supporter dix à quinze heures auxquelles personne n’aurait pu résister s’il n’était glissé quelque façon de plaisir. Une conception encore artisanale dans son principe laissait à chacun le soin de se ménager un confort précaire dans l’enfer de l’usine. Le taylorisme assena le coup de grâce à une mentalité précieusement entretenue par le capitalisme archaïque. Inutile d’espérer d’un travail à la chaîne ne serait-ce qu’une caricature de créativité. L’amour du travail bien fait et le goût de la promotion dans le travail sont aujourd’hui la marque indélébile de la veulerie et de la soumission la plus stupide. C’est pourquoi, partout où la soumission est exigée, le vieux pet idéologique va son chemin, de l’\emph{Arbeit macht frei} des camps d’extermination aux discours de Henry Ford et de Mao Tsé-toung.\par
Quelle est donc la fonction du travail forcé ? Le mythe du pouvoir exercé conjointement par le chef et par Dieu trouvait dans l’unité du système féodal sa force de coercition. En brisant le mythe unitaire, le pouvoir parcellaire de la bourgeoisie ouvre, sous le signe de la crise, le règne des idéologies qui jamais n’atteindront ni seules, ni ensemble, au quart de l’efficacité du mythe. La dictature du travail productif prend opportunément la relève. Il a pour mission d’affaiblir biologiquement le plus grand nombre des hommes de les châtrer collectivement et de les abrutir afin de les rendre réceptifs aux idéologies les moins prégnantes, les moins viriles, les plus séniles qui furent jamais dans l’histoire du mensonge.\par
Le prolétariat du début du XIX\textsuperscript{e} siècle compte une majorité de diminués physiques, d’hommes brisés systématiquement par la torture de l’atelier. Les révoltes viennent de petits artisans, de catégories privilégiées ou de sans travail, non d’ouvriers assommés par quinze heures de labeur. N’est-il pas troublant de constater que l’allégement du nombre d’heures de prestations intervient au moment où le spectacle de variétés idéologiques mis au point par la société de consommation paraît de nature à remplacer efficacement les mythes féodaux détruits par la jeune bourgeoisie ? (Des gens ont vraiment travaillé pour un réfrigérateur, pour une voiture, pour un récepteur de télévision. Beaucoup continuent à le faire, « invités » qu’ils sont à consommer la passivité et le temps vide que leur « offre » la « nécessité » de produire.)\par
Des statistiques publiés en 1938 indiquent qu’une mise en œuvre des techniques de production contemporaines réduiraient la durée des prestations nécessaires à trois heures par jour. Non seulement nous sommes loin du compte avec nos sept heures de travail, mais après avoir usé des générations de travailleurs en leur promettant le bien-être qu’elle leur vend aujourd’hui à crédit, la bourgeoisie (et sa version soviétisée) poursuit sa destruction de l’homme en dehors du travail. Demain elle appâtera ses cinq heures d’usure quotidienne exigées par un temps de créativité qui croîtra dans la mesure où elle pourra l’emplir d’une impossibilité de créer (la fameuse organisation des loisirs).\par
On a écrit justement : « La Chine fait face à des problèmes économiques gigantesques ; pour elle, la productivité est une question de vie ou de mort. » Personne ne songe à le nier. Ce qui me paraît grave ne tient pas aux impératifs économiques, mais à la façon d’y répondre. L’armée Rouge de 1917 constituait un type nouveau d’organisation. L’armée Rouge de 1960 est une armée comme on en rencontre dans les pays capitalistes. Les circonstances ont prouvé que son efficacité restait loin au-dessous des possibilités des milices révolutionnaires. De même l’économie chinoise planifiée, en refusant d’accorder à des groupes fédérés l’organisation autonome de leur travail, se condamne à rejoindre une forme de capitalisme perfectionné, nommé socialisme. A-t-on pris la peine d’étudier les modalités de travail des peuples primitifs, l’importance du jeu et de la créativité, l’incroyable rendement obtenu par des méthodes qu’un appoint des techniques modernes rendrait cent fois plus efficaces encore ? Il ne semble pas. Tout appel à la productivité vient du haut. Or la créativité seule est spontanément riche. Ce n’est pas de la productivité qu’il faut attendre une vie riche, ce n’est pas de la productivité qu’il faut espérer une réponse collective et enthousiaste à la demande économique. Mais que dire de plus quand on sait de quel culte le travail est honoré à Cuba comme en Chine, et avec quelle aisance les pages vertueuses de Guizot passeraient désormais dans un discours du 1er Mai ?\par
À mesure que l’automation et la cybernétique laissent prévoir le remplacement massif des travailleurs par des esclaves mécaniques, le travail forcé révèle sa pure appartenance aux procédés barbares du maintien de l’ordre. Le pouvoir fabrique ainsi la dose de fatigue nécessaire à l’assimilation passive de ses diktats télévisés. Pour quel appât travailler désormais ? La duperie est épuisée ; il n’y a plus rien à perdre, pas même une illusion. L’organisation du travail et l’organisation des loisirs referment les ciseaux castrateurs chargés d’améliorer la race des chiens soumis. Verra-t-on quelque jour les grévistes, revendiquant l’automation et la semaine de dix heures, choisir, pour débrayer, de faire l’amour dans les usines, les bureaux et les maisons de la culture ? Il n’y aurait que les programmateurs, les managers, les dirigeants syndicaux et les sociologues pour s’en étonner et s’en inquiéter. Avec raison peut-être. Après tout, il y va de leur peau.
\section[{VI. Décompression et troisième force}]{VI. Décompression et troisième force}\renewcommand{\leftmark}{VI. Décompression et troisième force}


\begin{argument}\noindent Jusqu’à présent la tyrannie n’a fait que changer de mains. Dans le respect commun de la fonction dirigeante, les forces antagonistes n’ont cessé d’entretenir les germes de leur coexistence future. (Quand le meneur de jeu prend le pouvoir d’un chef, la révolution meurt avec les révolutionnaires.) Les antagonismes non résolus pourrissent en dissimulant les vraies contradictions. La décompression est le contrôle permanent des antagonistes par la caste dominante. La troisième force radicalise les contradictions et les mène à leur dépassement, au nom de la liberté individuelle et contre toutes les formes de contrainte. Le pouvoir n’a d’autre recours que d’écraser ou de récupérer la troisième force sans en reconnaître l’existence.
\end{argument}

\noindent Faisons le point. Quelques millions d’hommes vivaient dans une immense bâtisse sans porte ni fenêtre. D’innombrables lampes à huile rivalisaient sur leur maigre lumière avec les ténèbres qui régnaient en permanence. Comme il était d’usage, depuis la plus sage Antiquité, leur entretien incombait aux pauvres, aussi le cours de l’huile épousait-il fidèlement le cours sinueux de la révolte et de l’accalmie. Un jour une insurrection générale éclata, la plus violente que ce peuple eût connue. Les meneurs exigeaient une juste répartition des frais d’éclairage ; un grand nombre de révolutionnaires revendiquaient la gratuité de ce qu’ils appelaient un service d’utilité publique ; quelques extrémistes allaient jusqu’à réclamer la destruction d’une demeure prétendue insalubre et inadaptée à la vie commune. Selon la coutume, les plus raisonnables se trouvèrent désarmés devant la brutalité des combats. Au cours d’un engagement particulièrement vif avec les forces de l’ordre, un boulet mal dirigé creva dans le mur d’enceinte une brèche par où la lumière se coula. Le premier moment de stupeur passé, cet afflux de lumière fut salué par des cris de victoire. La solution était là : il suffisait désormais d’aménager d’autres brèches. Les lampes furent mises au rebut ou rangées dans des musées, le pouvoir échut aux perceurs de fenêtre. On oublia les partisans d’une destruction radicale et même leur liquidation discrète passa, semble-t-il, presque inaperçue. (On se querellait sur le nombre et l’emplacement des fenêtres.) Puis leurs noms revinrent en mémoire, un siècle ou deux plus tard, alors que, accoutumé à voir de larges baies vitrées, le peuple, cet éternel mécontent, s’était mis à poser d’extravagantes questions. « Traîner ses jours dans une serre climatisée, est-ce une vie ? », demanda-t-il.\par

\astermono

\noindent La conscience contemporaine est tantôt celle de l’emmuré, tantôt celle du prisonnier. L’oscillation lui tient lieu de liberté ; il va, comme le condamné, du mur blanc de sa cellule à la fenêtre grillagée de l’évasion. Que l’on perce une ouverture dans le caveau de l’isolement, et l’espoir filtre avec la lumière. De l’espoir d’évasion qu’entretiennent les prisons dépend la docilité du prisonnier. Acculé à un mur sans issue, un homme ne connaît par contre que la rage de l’abattre ou de s’y briser la tête, ce qui ne laisse pas d’être regrettable au regard d’une bonne organisation sociale (même si le suicidé n’a pas l’heureux esprit d’entrer dans la mort à la manière des princes orientaux, en immolant tous ses esclaves : juges, évêques, généraux, policiers, psychiatres, philosophes, managers, spécialistes et cybernéticiens).\par
L’emmuré vif a tout à gagner, le prisonnier, lui, peut perdre encore l’espoir. L’espoir est la laisse de la soumission. Dès que le pouvoir risque d’éclater, il fait jouer la soupape de sûreté, il diminue la pression interne. On dit qu’il change ; en vérité il n’a fait que s’adapter en résolvant ses difficultés.\par
Il n’est pas d’autorité qui ne voie se dresser contre elle une autorité similaire et de signe contraire. Or, rien de plus périlleux pour le principe de gouvernement hiérarchisé que l’affrontement sans merci de deux forces antagonistes animées d’une rage d’anéantissement total. Dans pareil conflit, le raz de marée du fanatisme emporte les valeurs les plus stables, le \emph{no man’s land} s’étend partout, instaurant l’interrègne du « rien n’est vrai, tout est permis ». L’histoire, il est vrai, n’offre pas d’exemple d’un combat titanesque qui ne fût opportunément désamorcé et transformé en conflit d’opérette. D’où vient la décompression ? De l’accord de principe implicitement conclu entre les forces en présence.\par
Le principe hiérarchique reste en effet commun aux forcenés des deux camps. On ne s’affronte jamais impunément, ni innocemment. Face au capitalisme des Lloyd George et des Krupp s’érige l’anticapitalisme de Lénine et de Trotsky. Dans le miroir des maîtres du présent se reflètent déjà les maîtres futurs. Comme l’écrit Henri Heine :\par


\begin{verse}
Lächelnd sheidet der Tyran\\
Denn er weiss, nach seinem Tode\\
Wechselt Willkür nur die Hände\\
Und die Knechtschaft hat kein Ende.\\!
\end{verse}
\begin{quoteblock}
\noindent {\itshape Le tyran meurt en souriant ; car il sait qu’après sa mort la tyrannie changera seulement de mains, et que l’esclavage est sans fin.}
\end{quoteblock}

\noindent Les chefs diffèrent comme diffèrent leurs modes de domination, mais il reste des chefs, des propriétaires d’un pouvoir exercé à titre privé. (La grandeur de Lénine tient sans conteste à son refus romantique d’assumer la fonction de maître absolu qu’impliquait son organisation très hiérarchisée du groupe bolchevik ; c’est par ailleurs à cette grandeur-là que le mouvement ouvrier est redevable de Cronstadt 21, de Budapest 56 et du \emph{batiouchka} Staline.)\par
Dès lors, le point commun va devenir point de décompression. Identifier l’adversaire avec le Mal et se nimber de l’auréole du Bien offre assurément l’avantage stratégique d’assurer l’unité d’action en polarisant l’énergie des combattants. Mais la manœuvre exige du même coup l’anéantissement de l’adversaire. Une telle perspective a de quoi faire hésiter les modérés. D’autant que détruire \emph{radicalement} l’adversaire pousse jusque dans le camp ami la destruction de cette part commune aux antagonistes. La logique bolchevique devait obtenir la tête des chefs sociaux-démocrates. Ceux-ci s’empressèrent de trahir, et ils le firent en tant que chefs. La logique anarchiste devait obtenir la liquidation du pouvoir bolchevik. Celui-ci s’empressa de les écraser, et le fit en tant que pouvoir hiérarchisé. La même chaîne de trahisons prévisibles jeta au-devant des fusils de l’union républicaine, socialiste et stalinienne, les anarchistes de Durruti.\par
Dès que le meneur de jeu se mue en dirigeant, le principe hiérarchique sauve sa peau, la révolution s’assied pour présider au massacre des révolutionnaires. Il faut le rappeler sans trêve : le projet insurrectionnel n’appartient qu’aux masses, le meneur le renforce, le chef le trahit. C’est entre le meneur et le chef que la lutte authentique se déroule d’abord.\par
Pour le révolutionnaire spécialisé, le rapport de force se mesure en quantité, de même que le nombre d’hommes commandés indique, pour n’importe quel militaire, la hauteur du grade. Les chefs de partis insurrectionnels ou prétendus tels perdent le qualitatif au nom de la clairvoyance quantitative. Eussent-ils bénéficié de 500 000 hommes supplémentaires et d’armements modernes, les « Rouges » n’en auraient pas moins perdu la révolution espagnole. Elle était morte sous la botte des commissaires du peuple. Les discours de la Pasionaria résonnaient déjà comme une oraison funèbre ; les clameurs pathétiques étouffaient le langage des faits, l’esprit des collectivités aragonaises ; l’esprit d’une minorité radicale résolue à trancher d’un seul coup toutes les têtes de l’hydre, non seulement sa tête fasciste.\par
Jamais, et pour cause, un affrontement absolu n’est arrivé à terme. La lutte finale n’a connu jusqu’à présent que de faux départs. Tout est à reprendre au début. La seule justification de l’histoire est de nous y aider.\par

\astermono

\noindent Soumis à la décompression, les antagonismes, irréductibles au premier abord, vieillissent côte à côte, ils se figent dans une opposition formelle, ils perdent leur substance, se neutralisent, mélangent leurs moisissures. Le bolchevik au couteau entre les dents, qui le reconnaîtrait dans le gagarinisme de Moscou la gâteuse ? Par la grâce du miracle œcuménique, le « prolétaire de tous les pays unissez-vous ! » cimente aujourd’hui l’union de tous les dirigeants. Tableau touchant. La part commune aux antagonismes, embryon de pouvoir qu’une lutte radicale eût extirpé, la voici qui réconcilie les frères ennemis.\par
Est-ce si simple ? Non pas. La farce manquerait de ressort. Sur la scène internationale, capitalisme et anticapitalisme sénescents donnent en spectacle leur spirituel marivaudage. Que les spectateurs frémissent à la pensée d’un désaccord, qu’ils trépignent de joie quand la paix vient bénir les peuples enlacés ! L’intérêt faiblit-il ? Une pierre est ajoutée au mur de Berlin ; l’affreux Mao grince des dents, tandis qu’un chœur de petits Chinois célèbre la patrie, la famille et le travail. Ainsi rafistolé, le vieux manichéisme va son chemin. Le spectacle idéologique crée, pour se renouveler, la mode des antagonismes désamorcés : êtes-vous pour ou contre Brigitte Bardot, Johnny Hallyday, la 3 CV Citroën, les jeunes, la nationalisation, les spaghetti, les vieux, l’ONU, les jupes courtes, le Pop Art, la guerre thermonucléaire, l’auto-stop ? Il n’est personne qui ne soit, à un moment de la journée, interpellé par une affiche, une information, un stéréotype, sommé de prendre parti sur les détails préfabriqués qui obturent patiemment toutes les sources de la créativité quotidienne. Dans les mains du pouvoir, ce fétiche glacé, les miettes d’antagonismes forment un anneau magnétique chargé de dérégler les boussoles individuelles, d’abstraire chacun de soi et de dévier les lignes de force.\par
La décompression n’est en somme que la manipulation des antagonismes par le pouvoir. Le conflit de deux termes prend son sens dans l’intervention d’un troisième. S’il n’existe que deux pôles, l’un et l’autre s’annulent car chacun emprunte sa valeur à l’autre. Impossible de juger, on entre dans le règne de la tolérance et de la relativité chères à la bourgeoisie. Comme on comprend l’intérêt porté par la hiérarchie apostolique et romaine à la querelle du manichéisme et du trinitarisme ! Dans un affrontement sans merci entre Dieu et Satan, que fût-il resté de l’autorité ecclésiastique ? Rien, les crises millénaristes l’ont prouvé. C’est pourquoi le bras séculier exerce son saint office, c’est pourquoi les bûchers flambent pour les mystiques de Dieu ou du diable, pour les théologiens téméraires qui mettent en question le principe du « trois en un ». Seuls les maîtres temporels du christianisme se veulent habilités à traiter le différend opposant le maître du Bien au maître du Mal. Ils sont les grands intermédiaires par qui le choix de l’un ou l’autre camp passe obligatoirement, ils contrôlent la voie du salut et celle de la damnation et ce contrôle importe plus pour eux que le salut ou la damnation mêmes. Sur terre, ils s’instituèrent juges sans appel, puisque aussi bien ils avaient choisi d’être jugés dans un au-delà dont ils inventaient les lois.\par
Le mythe chrétien désamorça l’âpre conflit manichéen en offrant au croyant la possibilité du salut individuel. C’était la brèche ouverte par le Poilu de Nazareth. L’homme échappait ainsi à la rigueur d’un affrontement entraînant nécessairement la destruction des valeurs, le nihilisme. Mais du même coup lui échappait la chance de se reconquérir à la faveur d’un bouleversement général, la chance de prendre sa place dans l’univers en chassant les dieux et leurs fléaux. De sorte que le mouvement de décompression semble avoir une fonction essentielle d’entraver la volonté la plus irréductible de l’homme, la volonté d’être soi sans partage.\par
De tous les conflits qui poussent un camp contre un autre, une part irrépressible de revendications individuelles entre en jeu, imposant souvent ses exigences menaçantes. À tel point qu’on est fondé à parler d’une \emph{troisième force}. La troisième force serait à la perspective individuelle ce que la force de décompression est à la perspective du pouvoir. Appoint spontané de toutes les luttes, elle radicalise les insurrections, dénonce les faux problèmes, menace le pouvoir dans sa structure même. Sa racine est partout dans la vie quotidienne. C’est à elle que Bretch fait allusion dans une des histoires de M. Keuner :\par

\begin{quoteblock}
\noindent « Comme on demandait à un prolétaire assigné en justice s’il voulait prêter serment sous la forme laïque ou ecclésiastique, il répondit : « Je suis chômeur ». »\end{quoteblock}

\noindent La troisième force amorce non le dépérissement des contraires, mais leur dépassement. Écrasée prématurément ou récupérée, elle devient, par un mouvement inverse, force de décompression. Ainsi, le salut de l’âme n’est autre que la volonté de vivre récupérée par le mythe, médiatisée, vidée de son contenu réel. Par contre, la revendication péremptoire d’une vie riche explique la haine dont furent l’objet certaines sectes gnostiques ou les Frères du Libre Esprit. Au déclin du christianisme, le combat que se livrent Pascal et les Jésuites oppose à la nécessité de réaliser Dieu dans le bouleversement nihiliste du monde la doctrine réformiste du salut et des accommodements avec le ciel. Enfin, débarrassée de sa gangue théologique, c’est elle toujours qui anime la lutte babouviste contre le million doré, le projet marxiste de l’homme total, les rêveries de Fourier, le déchaînement de la Commune, la violence anarchiste.\par

\astermono

\noindent Individualisme, alcoolisme, collectivisme, activisme… la variété des idéologies l’atteste : il y a cent façons d’être aux côtés du pouvoir. Il n’y a qu’une façon d’être radical. Le mur à abattre est immense, mais tant de brèches l’ont ébranlé qu’il suffira bientôt d’un seul cri pour le voir s’effondrer. Que sorte enfin des brumes historiques la formidable réalité de la troisième force, ce qu’il y avait de passions individuelles dans les insurrections ! On verra bien que la vie quotidienne renferme une énergie qui déplace les montagnes et supprime les distances. La longue révolution se prépare à écrire dans les faits la geste dont les auteurs anonymes ou inconnus rejoindront pêle-mêle Sade, Fourier, Babeuf, Marx, Lacenaire, Stirner, Lautréamont, Léhautier, Vaillant, Henry, Villa, Zapata, Makhno, les Fédérés, ceux de Hambourg, de Kiel, de Cronstadt, des Asturies, ceux qui n’ont pas fini de jouer, avec nous qui commençons à peine le grand jeu sur la liberté.
\section[{VII. L’ère du bonheur}]{VII. L’ère du bonheur}\renewcommand{\leftmark}{VII. L’ère du bonheur}


\begin{argument}\noindent Le \emph{Welfare State} contemporain correspond anachroniquement aux garanties de survie exigées par les déshérités de l’ancienne société de production (1). – La richesse de survie implique la paupérisation de la vie (2). – Le pouvoir d’achat est la licence d’acheter du pouvoir, de devenir objet dans l’ordre des choses. Opprimés et oppresseurs tendent à tomber, mais à des vitesses inégales, sous une même dictature du consommable (3).
\end{argument}

\subsection[{1. Le Welfare State}]{\textsc{1.} Le \emph{Welfare State}}
\noindent Le visage du bonheur a cessé d’apparaître en filigrane dans les œuvres de l’art et de la littérature depuis qu’il s’est multiplié à perte de vue le long des murs et des palissades, offrant à chaque passant particulier l’image universelle où il est invité à se reconnaître.\par


\begin{verse}
Avec Volkswagen, plus de problèmes !\\
Vivez sans souci avec Balamur !\\
Cet homme de goût est aussi un sage. Il choisit Mercedes Benz.\\
\end{verse}

\noindent Le bonheur n’est pas un mythe, réjouissez-vous, Adam Smith et Bentham Jérémie ! « Plus nous produirons, mieux nous vivrons », écrit l’humaniste Fourastié, tandis qu’un autre génie, le général Eisenhower, répond comme en écho : « Pour sauver l’économie, il faut acheter, acheter n’importe quoi. » Production et consommation sont les mamelles de la société moderne. Allaitée de pareille façon, l’humanité croît en force et beauté : élévation du niveau de vie, facilités sans nombre, divertissements variés, culture pour tous, confort de rêve. À l’horizon du rapport Khrouchtchev, l’aube radieuse et communiste se lève enfin, inaugurant son règne par deux décrets révolutionnaires : la suppression des impôts et les transports gratuits. Oui, l’âge d’or est en vue, à un jet de salive.\par
Dans ce bouleversement, un grand disparu : le prolétariat. S’est-il évanoui ? A-t-il pris le maquis ? Le relègue-t-on dans un musée ? \emph{Sociologi disputant}. Dans les pays hautement industrialisés, le prolétaire a cessé d’exister, assurent certains. L’accumulation de réfrigérateurs, de TV, de Dauphine, d’HLM., de théâtres populaires l’atteste. D’autres, par contre, s’indignent, dénoncent le tour de passe-passe, le doigt braqué sur une frange de travailleurs dont les bas salaires et les conditions misérables évoquent indéniablement le XIX\textsuperscript{e} siècle. « Secteurs retardataires, rétorquent les premiers, poches en voie de résorption ; nierez-vous que le sens de l’évolution économique aille vers la Suède, vers la Tchécoslovaquie, vers le Welfare State, et non vers l’Inde.\par
Le rideau noir se lève : la chasse aux affamés et au dernier prolétaire est ouverte. C’est à qui lui vendra sa voiture et son \emph{mixer}, son bar et sa bibliothèque. C’est à qui l’identifiera au personnage souriant d’une affiche bien rassurante : « Heureux qui fume une Lucky Strike. »\par
Et heureuse, heureuse humanité qui va, dans un futur rapproché, réceptionner les colis dont les insurgés du XIX\textsuperscript{e} siècle ont arraché, au prix des luttes que l’on sait, les ordres de la livraison. Les révoltés de Lyon et de Fourmies ont bien de la chance à titre posthume. Des millions d’êtres humains fusillés, torturés, emprisonnés, affamés, abrutis, ridiculisés savamment ont du moins, dans la paix des charniers et des fosses communes, la garantie historique d’être morts pour qu’isolés dans des appartements à air conditionné leurs descendants apprennent à répéter, sur la foi des émissions télévisées quotidiennement, qu’ils sont heureux et libres.\par

\begin{quoteblock}
\noindent « Les communards se sont fait tuer jusqu’au dernier pour que toi aussi tu puisses acheter une chaîne stéréophonique Philips haute fidélité. »\end{quoteblock}

\noindent Un bel avenir qui aurait fait la joie du passé, on n’en doute pas.\par
Le présent seul n’y trouve pas son compte. Ingrate et inculte, la jeune génération veut tout ignorer de ce glorieux passé offert en prime à tout consommateur d’idéologie trotskisto-réformiste. Elle prétend que revendiquer, c’est revendiquer pour l’immédiat. Elle rappelle que la raison des luttes passées est ancrée dans le présent des hommes qui les ont menées et que ce présent-là, en dépit des conditions historiques différentes, est aussi le sien. En bref, il y aurait, à la croire, un projet constant qui animerait les courants révolutionnaires radicaux : le projet de l’homme total, une \emph{volonté de vivre totalement} à laquelle Marx le premier aurait su donner une tactique de réalisation scientifique. Mais ce sont là d’abominables théories que les Églises chrétiennes et staliniennes n’ont jamais manqué de flétrir avec assiduité. Augmentation de salaires, de réfrigérateurs, de saints sacrements et de TNP., voilà qui devrait rassasier la fringale révolutionnaire actuelle.\par
Sommes-nous condamnés à l’état de bien-être ? Les esprits pondérés ne manqueront pas de regretter la forme sous laquelle est menée la contestation d’un programme qui, de Khrouchtchev au docteur Schweitzer, du pape à Fidel Castro, d’Aragon à feu Kennedy, fait l’unanimité.\par
En décembre 1956, un millier de jeunes gens se déchaînent dans les rues de Stockholm, incendiant les voitures, brisant les enseignes lumineuse, lacérant les panneaux publicitaires, saccageant les grands magasins. A Merlebach, lors d’une grève déclenchée pour décider le patronat à remonter les corps de sept mineurs tués par un éboulement, les ouvriers s’en prennent aux voitures en stationnement devant les bâtiments. En janvier 1961, les grévistes de Liège mettent à sac la gare des Guillemins et détruisent les installations du journal \emph{La Meuse}. Sur les côtes belges et anglaises, et à l’issue d’une opération concertée, quelques centaines de blousons noirs dévastent les installations balnéaires, en mars 1964. A Amsterdam (1966), les ouvriers tiennent la rue pendant plusieurs jours. Pas un mois ne s’écoule sans qu’une grève sauvage n’éclate, dressant les travailleurs à la fois contre les patrons et les dirigeants syndicaux. Welfare State. Le quartier de Watts a répondu.\par
Un ouvrier d’Espérance-Longdoz résumait comme suit son désaccord avec les Fourastié, Berger, Armand, Moles et autres chiens de garde du futur :\par

\begin{quoteblock}
\noindent « Depuis 1936, je me suis battu pour des revendications de salaire ; mon père, avant moi, s’est battu pour des revendications de salaires. J’ai la TV, un réfrigérateur, une Volkswagen. Au total, je n’ai jamais cessé d’avoir une vie de con. »\end{quoteblock}

\noindent En paroles ou en gestes, la nouvelle poésie s’accommode mal du Welfare State.
\subsection[{2. L’opulence de survie paupérise la vie}]{\textsc{2.} L’opulence de survie paupérise la vie}


\begin{verse}
Les plus beaux modèles \emph{de radio à la portée de tous} (1).\\
Vous aussi entrez dans la \emph{grande famille} des DAFistes (2).\\
Carven vous offre la qualité. Choisissez \emph{librement} dans la gamme de ses produits (3).\\
\end{verse}

\noindent Dans le royaume de la consommation, le citoyen est roi. Une royauté démocratique : égalité devant la consommation (1), fraternité dans la consommation (2), liberté selon la consommation (3). La dictature du consommable a parfait l’effacement des barrières de sang, de lignage ou de race ; il conviendrait de s’en réjouir sans réserve si elle n’avait interdit par la logique des choses toute différenciation qualitative, pour ne plus tolérer entre les valeurs et les hommes que des différences de quantité.\par
Entre ceux qui possèdent beaucoup et ceux qui ne possèdent peu, mais toujours davantage, la distance n’a pas changé, mais les degrés intermédiaires se sont multipliés, rapprochant en quelque sorte les extrêmes, dirigeants et dirigés, d’un même centre de médiocrité. Être riche se réduit aujourd’hui à un grand nombre d’objets pauvres.\par
Les biens de consommation tendent à n’avoir plus de valeur d’usage. Leur nature est d’être consommable à tout prix. (On connaît la vogue récente aux USA du \emph{nothing box}, un objet parfaitement impropre à quelque utilisation que ce soit.) Et comme l’expliquait très sincèrement le général Dwight Eisenhower, l’économie actuelle ne peut se sauver qu’en transformant l’homme en consommateur, en l’identifiant à la plus grande quantité possible de valeurs consommables, c’est dire de non-valeurs ou de valeurs vides, fictives, abstraites. Après avoir été le « capital le plus précieux », selon l’heureuse expression de Staline, l’homme doit devenir le bien de consommation le plus apprécié. L’image, le stéréotype de la vedette, du pauvre, du communiste, du meurtrier par amour, de l’honnête citoyen, du révolté, du bourgeois, va substituer à l’homme un système de catégories mécanographiquement rangées selon la logique irréfutable de la robotisation. Déjà la notion de \emph{teen-ager} tend à conformer l’acheteur au produit acheté, à réduire sa variété à une gamme variée, mais limitée d’objets à vendre (disque, guitare, \emph{blue-jeans}…). On n’a plus l’âge du cœur ou de la peau, mais l’âge de ce que l’on achète. Le temps de production qui était, disait-on, de l’argent, va devenir, en se mesurant au rythme de succession des produits achetés, usés, jetés, un temps de consommation et de consomption, un temps de vieillissement précoce, qui est l’éternelle jeunesse des arbres et des pierres.\par
Le concept de paupérisation trouve aujourd’hui son éclatante démonstration non, comme le pensait Marx, dans le cadre des biens nécessaires à la survie, puisque ceux-ci, loin de se raréfier, n’ont cessé d’augmenter, mais bien dans la survie elle-même, toujours antagoniste à la vraie vie. Le confort, dont on espérait un enrichissement de la vie déjà vécue richement par l’aristocratie féodale, n’aura été que l’enfant de la productivité capitaliste, un enfant prématurément destiné à vieillir sitôt que le circuit de la distribution l’aura métamorphosé en simple objet de consommation passive. Travailler pour survivre, survivre en consommant et pour consommer, le cycle infernal est bouclé. Survivre est, sous le règne de l’économisme, à la fois nécessaire et suffisant. C’est la vérité première qui fonde l’ère bourgeoise. Et il est vrai qu’une étape historique fondée sur une vérité aussi antihumaine ne peut constituer qu’une étape de transition, un passage entre la vie obscurément vécue des maîtres féodaux et la vie rationnellement et passionnellement construite des maîtres sans esclaves. Il reste une trentaine d’années pour empêcher que l’ère transitoire des esclaves sans maîtres ne dure deux siècles.
\subsection[{3. Le pouvoir d’achat : licence d’acheter du pouvoir ; dictature du consommable}]{\textsc{3.} Le pouvoir d’achat : licence d’acheter du pouvoir ; dictature du consommable}
\noindent La révolution bourgeoise prend, au regard de la vie quotidienne, des allures de contre-révolution. Rarement, sur le marché des valeurs humaines, dans la conception de l’existence, pareille dévaluation fut à ce point ressentie. La promesse, – jetée comme un défi à l’univers, – d’instaurer le règne de la liberté et du bien-être, rendait plus sensible encore la médiocrité d’une vie que l’aristocratie avait su enrichir de passions et d’aventures et qui, enfin accessible à tous, n’était plus guère qu’un palais loti en chambres de bonnes.\par
On allait désormais vivre moins de haine que de mépris, moins d’amour que d’attachement, moins de ridicule que de stupidité, moins de passions que de sentiments, moins de désirs que d’envie, moins de raison que de calcul et moins de goût de vivre que d’empressement à survivre. La morale du profit, parfaitement méprisable, remplaçait la morale de l’honneur, parfaitement haïssable ; au mystérieux pouvoir du sang, parfaitement ridicule, succédait le pouvoir de l’argent, parfaitement ubuesque. Les héritiers de la nuit du 4 août élevaient à la dignité de blason le compte en banque et le chiffre d’affaires, comptabilisant le mystère.\par
Où réside le mystère de l’argent ? Évidemment, en ce qu’il représente une somme d’êtres et de choses appropriables. Le blason nobiliaire exprime le choix de Dieu et le pouvoir réel exercé par l’élu ; l’argent est seulement le signe de ce qui peut être acquis, il est une traite sur le pouvoir, un choix possible. Le Dieu des féodaux, base apparente de l’ordre social, en est véritablement le prétexte et le couronnement luxueux. L’argent, ce dieu sans odeur des bourgeois, est lui aussi une médiation ; un contrat social. C’est un dieu maniable non plus par prières ou serments, mais par science et techniques spécialisées. Son mystère n’est plus dans une totalité obscure, impénétrable mais dans une somme de certitudes partielles en nombre infini ; plus dans une qualité de maître, mais dans la qualité d’êtres et de choses vénales (ce que 10 millions de francs mettent, par exemple, à la portée de son possesseur).\par
Dans l’économie dominée par les impératifs de production du capitalisme de libre-échange, la richesse confère à elle seule la puissance et les honneurs. Maîtresse des instruments de production et de la force de travail, elle assure conjointement, par le développement des forces productives et des biens de consommation, la richesse de son choix virtuel parmi la ligne infinie du progrès. Toutefois, à mesure que ce capitalisme se transforme en son contraire, l’économie planifiée de type étatique, le prestige du capitaliste jetant sur le marché le poids de sa fortune tend à disparaître et, avec lui, la caricature du marchand de chair humaine, cigare au bec et ventre redondant. Le manager tire aujourd’hui son pouvoir de ses facultés d’organisateur ; et les machines ordonnatrices sont déjà présentes pour lui donner, à sa dérision, un modèle qu’il n’atteindra jamais. Mais l’argent qu’il possède en propre, en fera-t-il étalage, prendra-t-il plaisir à lui faire signifier la richesse de ses choix virtuels ; construire un Xanadou, entretenir un harem, cultiver des filles-fleurs ? Hélas, où la richesse est sollicitée, pressée par les impératifs de consommation, comment conserverait-on sa valeur représentative ? Sous la dictature du consommable, l’argent va fondre comme neige au soleil. Son importance va décroître au profit d’objets plus représentatifs plus tangibles, mieux adaptés au spectacle du Welfare State. Son emploi n’est-il pas déjà contingenté par le marché des produits de consommation qui deviennent, enrobés d’idéologie, les vrais signes du pouvoir ? Sa dernière justification résidera avant peu dans la quantité d’objets et de \emph{gadgets} qu’il permettra d’acquérir et d’user à un rythme accéléré ; dans leur quantité et dans leur succession exclusivement, puisque aussi bien la distribution de masse et la standardisation leur ôtent automatiquement l’attrait de la rareté et de la qualité. La faculté de consommer beaucoup et à une cadence rapide, en changeant de voiture, d’alcool, de maison, de radio, de fille, indique désormais sur l’échelle hiérarchique le degré de pouvoir auquel chacun peut prétendre. De la supériorité du sang au pouvoir de l’argent, de la supériorité de l’argent au pouvoir du \emph{gadget}, la civilisation chrétienne et socialiste atteint son stade ultime : une civilisation du prosaïsme et du détail vulgaire. Un nid pour les petits hommes dont parlait Nietzsche.\par
Le pouvoir d’achat est la licence d’acheter du pouvoir. L’ancien prolétariat vendait sa force de travail pour subsister ; son maigre temps de loisir, il le vivait tant bien que mal en discussion, querelles, jeux de bistrot et de l’amour, trimard, fêtes et émeutes. Le nouveau prolétariat vend sa force de travail pour consommer. Quand il ne cherche pas dans le travail forcé une promotion hiérarchique, le travailleur est invité à s’acheter des objets (voiture, cravate, culture…) qui l’indexeront sur l’échelle sociale. Voici le temps où l’idéologie de la consommation devient consommation d’idéologie. Que personne ne sous-estime les échanges Est-Ouest ! D’un côté, l’\emph{hommo consomator} achète un litre de wisky et reçoit en prime le mensonge qui l’accompagne. De l’autre, l’homme communiste achète de l’idéologie et reçoit en prime un litre de vodka. Paradoxalement, les régimes soviétisés et les régimes capitalistes empruntent une voie commune, les premiers grâce à leur économie de production, les seconds par leur économie de consommation.\par
En URSS, le sur-travail des travailleurs n’enrichit pas directement, à proprement parler, le camarade directeur du trust. Il lui confère simplement un pouvoir renforcé d’organisateur et de bureaucrate. Sa plus-value est une plus-value de pouvoir. (Mais cette plus-value de type nouveau ne cesse pas pour autant d’obéir à la baisse tendancielle du taux de profit. Les lois de Marx pour la vie économique démontrent aujourd’hui leur véracité dans l’économie de la vie). Il la gagne, non au départ d’un capital-argent, mais sur une accumulation primitive de capital-confiance qu’une docile absorption de matière idéologique lui a value. La voiture et la datcha ajoutées de surcroît en récompense des services rendus à la patrie, au prolétariat, au rendement, à la Cause, laissent bien prévoir une organisation sociale où l’argent disparaîtrait, faisant place à des distinctions honorifiques, à des grades, à un mandarinat du biceps et de la pensée spécialisée. (Que l’on songe aux droits accordés aux émules de Stakhanov, aux « héros de l’espace », aux gratteurs de cordes et de bilans.)\par
En pays capitalistes, le profit matériel du patron, dans la production comme dans la consommation, se distingue encore du profit idéologique que le patron n’est plus seul, cette fois, à tirer de l’organisation de la consommation. C’est bien ce qui empêche encore de ne voir entre le manager et l’ouvrier qu’une différence entre la Ford renouvelée chaque année et la Dauphine entretenue amoureusement pendant cinq ans. Mais reconnaissons que la planification, vers laquelle tout concourt confusément aujourd’hui, tend à quantifier les différences sociales selon les possibilités de consommer et de faire consommer. Les degrés devenant plus nombreux et plus petits, l’écart entre les riches et les pauvres diminue de fait, amalgamant l’humanité dans les seules variations de pauvreté. Le point culminant serait la société cybernéticienne composée de spécialistes hiérarchisés selon leur aptitude à consommer et à faire consommer les doses de pouvoir nécessaires au fonctionnement d’une gigantesque machine sociale dont ils seraient à la fois le programme et la réponse. Une société d’exploiteurs-exploités dans une inégalité d’esclavage.\par
Reste le « tiers monde ». Restent les formes anciennes d’oppression. Que le serf des \emph{latifundia} soit le contemporain du nouveau prolétariat me paraît composer à la perfection le mélange explosif d’où naîtra la révolution totale. Qui oserait supposer que l’Indien des Andes déposera les armes après avoir obtenu la réforme agraire et la cuisine équipée, alors que les travailleurs les mieux payés d’Europe exigent un changement radical de leur mode de vie ? Oui, la révolte dans l’état de bien-être fixe désormais le degré d’exigences minimales pour toutes les révolutions du monde. À ceux qui l’oublieront, ne sera que plus dure la phrase de Saint Just :\par

\begin{quoteblock}
\noindent « Ceux qui font les révolutions à moitié n’ont fait que creuser un tombeau. »\end{quoteblock}

\section[{VIII. Échange et don.}]{VIII. Échange et don.}\renewcommand{\leftmark}{VIII. Échange et don.}


\begin{argument}\noindent La noblesse et le prolétariat conçoivent les rapports humains sur le modèle du \emph{don}, mais le don selon le prolétariat est le dépassement du don féodal. La bourgeoisie, ou classe des \emph{échanges}, est le levier qui permet le renversement du projet féodal et son dépassement par la longue révolution (1). – L’histoire est la transformation permanente de l’aliénation naturelle en aliénation sociale, et contradictoirement le renforcement d’une contestation qui va la dissoudre, en désaliénant. La lutte historique contre l’aliénation naturelle transforme l’aliénation en aliénation sociale, mais le mouvement de désaliénation historique atteint à son tour l’aliénation sociale et en dénonce la magie fondamentale. Cette magie tient à l’appropriation privative. Elle s’exprime par le sacrifice. Le sacrifice est la forme archaïque de l’échange. L’extrême quantification des échanges réduit l’homme à un pur objet. De ce point zéro peut naître un nouveau type de relations humaines sans échange ni sacrifice (2).
\end{argument}

\subsection[{1. La noblesse ou le prolétariat, modèle du don ; la bourgeoisie, classe des échanges}]{\textsc{1.} La noblesse ou le prolétariat, modèle du \emph{don} ; la bourgeoisie, classe des \emph{échanges}}
\noindent La bourgeoisie assure un interrègne précaire et peu glorieux entre la hiérarchie sacrée des féodaux et l’ordre anarchique des futures sociétés sans classes. Avec elle, le \emph{no man’s land} des échanges devient le lieu inhabitable qui sépare le vieux plaisir malsain du don de soi, auquel se livraient les aristocrates, et le plaisir de donner par amour de soi, auquel s’adonnent peu à peu les nouvelles générations de prolétaires.\par
Le donnant-donnant est la redondance favorite du capitalisme et de ses prolongements antagonistes. L’URSS. « offre » ses hôpitaux et ses techniciens, comme les USA. « offrent » leurs investissements et leurs bons offices, comme les pâtes Moles « offrent » leurs cadeaux-surprises.\par
Reste que le sens du don a été extirpé de la mentalité, des sentiments, des gestes. On songe à Breton et à ses amis offrant une rose à chaque jolie passante du boulevard Poissonnière et suscitant aussitôt la méfiance et l’animosité du public.\par
Le pourrissement des rapports humains par l’échange et la contrepartie est évidemment lié à l’existence de la bourgeoisie. Que l’échange persiste dans une partie du monde où la société sans classe serait, dit-on réalisé, atteste du moins que l’ombre de la bourgeoisie continue de régner aux pieds du drapeau rouge. D’autant que partout où vit une population industrielle, le plaisir de donner délimite très clairement la frontière entre le monde du calcul et le monde de l’exubérance, de la fête. Sa façon de donner ne laisse pas de trancher avec le don de prestige tel que le pratiquait la noblesse, irrémédiablement prisonnière de la notion de sacrifice. Vraiment, le prolétariat porte le projet de plénitude humaine, de vie totale. Ce projet, l’aristocratie avait réussi seulement à le mener jusqu’à son échec le plus riche. Reconnaissons néanmoins qu’un tel avenir devient accessible au prolétariat par la présence historique de la bourgeoisie, et par son entremise. N’est-ce pas grâce au progrès technique et aux forces productives développées par le capitalisme que le prolétariat se dispose à réaliser, dans le projet scientifiquement élaboré d’une société nouvelle, les rêveries égalitaires, les utopies de toute-puissance, la volonté de vivre sans temps mort ? Tout confirme aujourd’hui la mission, ou mieux la chance historique, du prolétariat : il lui appartient de détruire la féodalité en la dépassant. Et il le fera en foulant aux pieds la bourgeoisie vouée à ne représenter, dans le développement de l’homme, qu’une étape transitoire, mais une étape transitoire sans laquelle aucun dépassement du projet féodal ne se pourrait concevoir, une étape essentielle donc, qui créa l’indispensable levier sans lequel le pouvoir unitaire n’eût jamais été jeté à bas ; et surtout n’eût jamais été renversé et corrigé dans le sens de l’homme total. Le pouvoir unitaire était déjà, comme l’invention de Dieu l’atteste, un monde pour l’homme total, pour un homme total marchant sur la tête. Il n’y manquait que le renversement.\par
Il n’y a pas de libération possible en deçà de l’économique ; il n’y a sous le règne de l’économique qu’une hypothétique économie de survie. C’est sous l’aiguillon de ces deux vérités que la bourgeoisie pousse les hommes vers un dépassement de l’économique, vers un au-delà de l’histoire. Avoir mis la technique au service d’une poésie nouvelle n’aura pas été son moindre mérite. Jamais la bourgeoisie n’aura été si grande qu’en disparaissant.
\subsection[{2. L’histoire : transformation de l’aliénation naturelle en aliénation sociale}]{\textsc{2.} L’histoire : transformation de l’aliénation naturelle en aliénation sociale}
\noindent L’échange est lié à la survie des hordes primitives, au même titre que l’appropriation privative ; tous deux constituent le postulat sur lequel s’est construite l’histoire des hommes jusqu’à nos jours.\par
En assurant aux premiers hommes une sécurité accrue contre la nature hostile, la formation de réserves de chasse jetait les bases d’une organisation sociale qui n’a cessé de nous emprisonner. (Cf. Raoul et Laura Makarius : \emph{Totem et exogamie}.) L’unité de l’homme primitif et de la nature est d’essence magique. L’homme ne se sépare vraiment de la nature qu’en la transformant par la technique et, la transformant, il la désacralise. Or l’emploi de la technique est subordonné à une organisation sociale. La société naît avec l’outil. Bien plus, l’organisation est la première technique cohérente de lutte contre la nature. L’organisation sociale – hiérarchisée puisque fondée sur l’appropriation privative – détruit peu à peu le lien magique existant entre l’homme et la nature, mais à son tour elle se charge de magie, elle crée entre elle et les hommes une unité mythique calquée sur leur participation au mystère de la nature. Encadrée par les relations « naturelles » de l’homme préhistorique, elle va dissoudre lentement ce cadre qui la définit et l’emprisonne. L’histoire n’est de ce point de vue que la transformation de l’aliénation naturelle en aliénation sociale : une désaliénation devient aliénation sociale, un mouvement libérateur se freine jusqu’à ce que, le freinage l’emportant, la volonté d’émancipation humaine s’en prenne directement à l’ensemble des mécanismes paralysants, c’est-à-dire à l’organisation sociale fondée sur l’appropriation privative. C’est là le mouvement de désaliénation qui va défaire l’histoire, la réaliser dans les nouveaux modes de vie.\par
En effet, l’accession de la bourgeoisie au pouvoir annonce la victoire de l’homme sur les forces naturelles. Du même coup, l’organisation sociale hiérarchisée, née des nécessités de lutte contre la faim, la maladie, l’inconfort…, perd sa justification et ne peut qu’endosser la responsabilité du malaise dans les civilisations industrielles. Les hommes attribuent aujourd’hui leur misère non plus à l’hostilité de la nature mais à la tyrannie d’une forme sociale parfaitement inadaptée, parfaitement anachronique. En détruisant le pouvoir magique des féodaux, la bourgeoisie a condamné la magie du pouvoir hiérarchisé. Le prolétariat exécutera la sentence. Ce que la bourgeoisie a commencé par l’histoire va maintenant s’achever contre sa conception étroite de l’histoire. Et ce sera encore une lutte historique, une lutte des classes qui \emph{réalisera} l’histoire.\par
Le principe hiérarchique est le principe magique qui a résisté à l’émancipation des hommes et à leurs luttes historiques pour la liberté. Aucune révolution ne sera désormais digne de ce nom si elle n’implique au moins l’élimination radicale de toute hiérarchie.\par

\astermono

\noindent Dès l’instant où les membres d’une horde délimitent une réserve de chasse, dès l’instant donc où ils s’en assurent la propriété à titre privé, ils se trouvent confrontés à un type d’hostilité qui n’est plus l’hostilité des bêtes fauves, du climat, des régions inhospitalières, de la maladie, mais celle des groupes humains exclus de la jouissance du terrain de chasse. Le génie de l’homme va lui permettre d’échapper à l’alternative du règne animal : ou écraser le groupe rival ou être écrasé par lui. Le pacte, le contrat, l’échange fonde les chances d’existence des communautés primitives. La survie des clans antérieurs aux sociétés agricoles, et postérieurs aux hordes de la période dite « de la cueillette », passe nécessairement par un triple échange : échange des femmes, échange de nourriture, échange de sang. Participant de la mentalité magique, l’opération suppose un ordonnateur suprême, un maître des échanges, une puissance située au-delà et au-dessous des contractants. La naissance des dieux coïncide avec la naissance gémellaire du mythe sacré et du pouvoir hiérarchisé.\par
L’échange est loin d’accorder aux deux clans un avantage égal. Ne s’agit-il pas avant tout de s’assurer de la neutralité des exclus sans jamais leur permettre d’accéder à la réserve ? La tactique s’affine au stade des sociétés agricoles. Tenanciers avant d’être esclaves, les exclus entrent dans le groupe des possédants, non comme propriétaires, mais comme leur reflet dégradé (le mythe fameux de la Chute originelle), comme la médiation entre la terre et ses maîtres. Comment s’effectue la soumission des exclus ? Par l’emprise cohérente d’un mythe qui dissimule, – non par une volonté délibérée des maîtres, car ce serait leur supposer une rationalité, qui leur était encore étrangère, – la ruse des échanges, le déséquilibre des sacrifices consentis de part et d’autre. Au propriétaire, les exclus sacrifient \emph{réellement} une fraction importante de leur vie : ils acceptent son autorité et travaillent pour lui. Aux dominés, le maître sacrifie \emph{mythiquement} son autorité et son pouvoir de propriétaire : il est prêt à payer pour le salut commun de son peuple. Dieu est le garant de l’échange et le gardien du mythe. Il punit les manquements au contrat et récompense en conférant le pouvoir : un pouvoir mythique pour ceux qui se sacrifient réellement, un pouvoir réel pour ceux qui se sacrifient mythiquement. (Les faits historiques et mythologiques attestent que le sacrifice du maître au principe mythique a pu aller jusqu’à la mort.) Payer le prix de l’aliénation qu’il imposait aux autres renforçait par ailleurs le caractère divin du maître. Mais très tôt, semble-t-il, une mise à mort scénique ou par substitution décharge le maître d’une aussi redoutable contrepartie. Le Dieu des chrétiens déléguant son fils sur la terre donne à des générations de dirigeants une copie conforme à laquelle il leur suffira de se référer pour authentifier leur sacrifice.\par
Le sacrifice est la forme archaïque de l’échange. Il s’agit d’un échange magique, non quantifié, non rationnel. Il domine les rapports humains, y compris les rapports commerciaux, jusqu’à ce que le capitalisme marchand et son argent-mesure-de-toute-chose aient pris une telle extension dans le cadre esclavagiste, féodal, puis bourgeois, que l’économie apparaisse comme une zone particulière, un domaine séparé de la vie. Ce qu’il y avait d’échange dans le don féodal l’emporte dès l’apparition de la monnaie. Le don-sacrifice, le \emph{potlatch}, – ce \emph{jeu} d’échange et de qui-perd-gagne où l’ampleur du sacrifice accroît le poids du prestige – n’avait guère de place dans une économie de troc rationalisé. Chassé des secteurs dominés par les impératifs économiques, il va se trouver réinvesti dans des valeurs telles que l’hospitalité, l’amitié et l’amour, officiellement condamnés à disparaître à mesure que la dictature de l’échange quantifié (la valeur marchande) colonise la vie quotidienne et la transforme en marché.\par
Le capitalisme marchand et le capitalisme industriel accélèrent la quantification des échanges. Le don féodal se rationalise sur le modèle rigoureux des échanges commerciaux. Le jeu sur l’échange cesse d’être un jeu, devient calcul. Le ludique présidait à la promesse romaine d’immoler un coq aux dieux en échange d’un heureux voyage. La disparité des matières échangées échappait à la mesure mercantile. On comprend qu’il existe, dans une époque où Fouquet se ruine pour briller davantage aux yeux de ses contemporains et de Louis, le plus illustre d’entre eux, une poésie que ne connaît plus notre temps accoutumé à prendre modèle de rapports humains l’échange de 12,80 francs contre un filet de 750 grammes.\par
Par voie de conséquence, on en est arrivé à quantifier le sacrifice, à le rationaliser, à le peser, à le coter en bourse. Mais que devient la magie du sacrifice dans le règne des valeurs marchandes ? Et que devient la magie du pouvoir, la terreur sacrée qui pousse l’employé modèle à saluer respectueusement son chef de service ?\par
Dans une société où la quantité de \emph{gadgets} et d’idéologies traduit la quantité de pouvoir consommée, assumée, consumée, les rapports magiques s’évaporent, laissant le pouvoir hiérarchisé au centre de la contestation. La chute du dernier bastion sacré sera la fin d’un monde ou la fin du monde. Il s’agit de l’abattre avant qu’il n’entraîne l’humanité dans sa chute.\par
Rigoureusement quantifié (par l’argent puis par la quantité de pouvoir, par ce que l’on pourrait appeler des « unités sociométriques de pouvoir »), l’échange salit tous les rapports humains, tous les sentiments, toutes les pensées. Partout où il domine, il ne reste en présence que des choses ; un monde d’homme-objets figés dans les organigrammes du pouvoir cybernéticien en instance de régner ; le monde de la réification. Mais c’est aussi, contradictoirement, la chance d’une restructuration radicale de nos schèmes de vie et de pensée. Un point zéro où \emph{tout} peut vraiment commencer.\par

\astermono

\noindent La mentalité féodale semblait concevoir le don comme une sorte de refus hautain de l’échange, une volonté de nier l’interchangeable. Le refus allait de pair avec le mépris de l’argent et de la commune mesure. Certes, le sacrifice exclut le don pur mais tel fut bien souvent l’empire du jeu, du gratuit, de l’humain, que l’inhumanité, la religion, le sérieux purent passer pour accessoires dans des préoccupations comme la guerre, l’amour, l’amitié, le service d’hospitalité.\par
Par le don de soi, la noblesse scellait son pouvoir à la totalité des forces cosmiques et prétendait du même coup au contrôle de la totalité sacralisée par le mythe. En échangeant l’être contre l’avoir, le pouvoir bourgeois perd l’unité mythique de l’être et du monde ; la totalité s’émiette. L’échange semi-rationnel de la production égalise implicitement la créativité réduite à la force de travail et un taux de salaire horaire. L’échange semi-rationnel de la consommation égalise implicitement le vécu consommable (la vie réduite à l’activité de consommation) et une somme de pouvoir susceptible d’indexer le consommateur dans l’organigramme hiérarchique. Au sacrifice du maître succède le stade ultime du sacrifice, le sacrifice du spécialiste. Pour consommer, le spécialiste fera consommer selon un programme cybernéticien où l’hyperrationalité des échanges supprimera le sacrifice. Et l’homme du même coup ! Si l’échange pur règle un jour les modalités d’existence des citoyens-robots de la démocratie cybernétique, le sacrifice cessera d’exister. Pour obéir, les objets n’ont pas besoin de justification. Le sacrifice est exclu du programme des machines comme de son projet antagoniste, le projet de l’homme total.\par

\astermono

\noindent L’effritement des valeurs humaines prises en charge par les mécanismes d’échange entraîne l’effritement de l’échange même. L’insuffisance du don aristocratique engage à fonder de nouveaux rapports humains sur le don pur. Il faut retrouver le plaisir de donner ; donner par excès de richesse ; donner parce que l’on possède en surabondance. Quels beaux \emph{potlatchs} sans contrepartie la société de bien-être va, bon gré, mal gré, susciter quand l’exubérance des jeunes générations découvrira le don pur ! (La passion, de plus en plus répandue chez les jeunes, de voler livres, manteaux, sacs de dames, armes et bijoux pour le seul plaisir de les offrir laisse heureusement présager l’emploi que la volonté de vivre réserve à la société de consommation).\par
Aux besoins préfabriqués répond le besoin unitaire d’un nouveau style de vie. L’art, cette économie des moments vécus, a été absorbé par le marché des affaires. Les désirs et les rêves travaillent pour le \emph{marketing}. La vie quotidienne s’émiette en une suite d’instants interchangeables comme les \emph{gadgets} qui y correspondent (mixer, Hi-Fi, pessaire, euphorimètre, somnifère). Partout des parcelles égales entre elles s’agitent dans la lumière équitablement répartie du pouvoir. Égalité, justice. Échange de néants, de limites et d’interdictions. Il n’y a de succession que de temps morts.\par
Il faut renouer avec l’imperfection féodale, non pour la parfaire mais la dépasser. Il faut renouer avec l’harmonie de la société unitaire en la libérant du fantôme divin et de la hiérarchie sacrée. La nouvelle innocence n’est pas si loin des ordalies et des jugements de Dieu ; l’inégalité du sang est, plus que l’égalité bourgeoise, proche de l’égalité d’individus libres et irréductibles les uns aux autres. Le style contraint de la noblesse n’est qu’une esquisse grossière du grand style que connaîtront les maîtres sans esclaves. Mais quel monde entre un style de vie et la manière de survie qui ravage tant d’existences contemporaines.
\section[{IX. La technique et son usage médiatisé}]{IX. La technique et son usage médiatisé}\renewcommand{\leftmark}{IX. La technique et son usage médiatisé}


\begin{argument}\noindent La technique désacralise à l’encontre des intérêts de ceux qui en contrôlent l’emploi. – Le règne démocratique de la consommation ôte toute valeur magique aux \emph{gadgets}. De même le règne de l’organisation (une technique des techniques nouvelles) prive les nouvelles forces de production de leur pouvoir de bouleversement et de séduction. – L’organisation est ainsi dénoncée comme pure organisation de l’autorité (1). – Les médiations aliénées affaiblissent l’homme en se rendant indispensables. – Un masque social recouvre les êtres et les objets. Dans l’état actuel de l’appropriation privative, ce masque transforme ce qu’il recouvre en choses mortes, en marchandises. Il n’y a plus de nature. – Retrouver la nature, c’est la réinventer comme adversaire valable en construisant de nouveaux rapports sociaux. – L’excroissance de l’équipement matériel crève la peau de la vieille société hiérarchisée (2).
\end{argument}

\subsection[{1. Gadgets, le progrès de l’inutile}]{\textsc{1.} \emph{Gadgets}, le progrès de l’inutile}
\noindent Une égale carence frappe les civilisations non industrielles, où l’on meurt encore de faim, et les civilisations automatisées, où l’on meurt déjà d’ennui. Tout paradis est artificiel. Riche en dépit des tabous et des rites, la vie d’un Trobriandais est à la merci d’une épidémie de variole ; pauvre en dépit du confort, la vie d’un Suédois moyen est à la merci du suicide et du mal de survie.\par
Rousseauisme et bergeries accompagnent les premiers vrombissements de la machine industrielle. Telle qu’on la trouve chez Smith ou Condorcet, l’idéologie du progrès ressortit d’ailleurs du vieux mythe des quatre âges. L’âge du fer précédant l’âge d’or, il paraît « naturel » que le progrès s’accomplisse aussi comme une récurrence : il faut rejoindre l’état d’innocence antérieur à la Chute.\par
La croyance au pouvoir magique des techniques n’est pas sans aller de pair avec son contraire, le mouvement de désacralisation. La machine est le modèle de l’intelligible. Ses courroies, ses transmissions, ses réseaux, rien n’y est obscur ni mystérieux, tout s’y explique parfaitement mais la machine est aussi le miracle qui doit faire accéder l’humanité au règne du bonheur et de la liberté. Du reste, l’ambiguïté sert ses maîtres : la mystique des lendemains qui chantent justifie à divers degrés de référence l’exploitation rationnelle des hommes d’aujourd’hui. C’est donc moins la logique désacralisante qui ébranle la foi dans le progrès, que l’emploi inhumain de potentiel technique, que la mystique grinçante de cet emploi. Tant que les classes laborieuses et les peuples sous-développés offrirent le spectacle de leur misère matérielle lentement décroissante, l’enthousiasme pour le progrès se nourrit amplement à la mangeoire de l’idéologie libérale et de son prolongement, le socialisme. Mais, un siècle après la démystification spontanée des ouvriers lyonnais brisant les métiers à tisser, la crise générale éclate, issue cette fois de la crise de la grande industrie. C’est la répression fasciste, le rêve débile d’un retour à l’artisanat et au corporatisme, l’ubuesque « bon sauvage » aryen.\par
Les promesses de la vieille société de production tombent aujourd’hui en une avalanche de biens consommables que personne ne risque d’attribuer à la manne céleste. Célébrer la magie des \emph{gadgets} comme on a célébré la magie des forces productives est une entreprise vouée à l’échec. Il existe une littérature admirative sur le marteau-pilon. On ne l’imagine pas sur le mixer. La multiplication des instruments de confort – tous également révolutionnaires si l’on en croit la publicité – a donné au plus rustre des hommes le droit de porter sur les merveilles de l’invention technique un jugement aussi familièrement admiratif que la main qu’il porte aux fesses d’une fille complaisante. Les premiers hommes foulant le sol de Mars n’interrompront pas une fête de village.\par
Le collier d’attelage, la machine à vapeur, l’électricité, l’énergie nucléaire surgissant, il faut bien l’avouer, presque accidentellement, perturbaient et modifiaient l’infrastructure des sociétés. Il serait vain d’attendre aujourd’hui de forces productives nouvelles qu’elles bouleversent les modes de production. L’épanouissement des techniques a vu naître une super-technique de synthèse, aussi importante peut-être que la communauté sociale, cette première synthèse technicienne fondée à l’aube de l’humanité. Plus importante même, car, arrachée à ses maîtres, il est possible que la cybernétique libère les groupes humains du travail et de l’aliénation sociale. Le projet de Charles Fourier n’est rien d’autre, à une époque où l’utopie reste possible.\par
Cela dit, il y a de Fourier aux cybernéticiens, qui contrôlent l’organisation opérationnelle des techniques, la distance de la liberté à l’esclavage. Sans doute le projet cybernéticien prétend-il atteindre déjà une perfection suffisante pour résoudre l’ensemble des problèmes posés par l’apparition d’une technique nouvelle. Rien n’est moins sûr :\par
1° Plus rien à attendre des forces productives en évolution permanente, plus rien à attendre des biens de consommation en multiplication croissante. Plus d’ode dithyrambique au climatiseur musical, plus de cantate au nouveau four solaire ! Voilà une lassitude à venir et déjà si manifestement présente qu’elle risque de se convertir tôt ou tard en critique de l’organisation elle-même.\par
2° Toute la souplesse de la synthèse cybernéticienne ne réussira jamais à dissimuler qu’elle n’est que la synthèse dépassante des différents gouvernements qui se sont exercés sur les hommes ; et leur stade ultime. Comment masquerait-elle la fonction aliénante qu’aucun pouvoir n’a pu soustraire aux armes de la critique et à la critique des armes ? Le pagayeur n’a que faire de crocodiles plus intelligents. En fondant le pouvoir parfait, les cybernéticiens vont promouvoir l’émulation et la perfection du refus. Leur programmation des techniques nouvelles se brisera sur ces mêmes techniques, détournées par une autre organisation. Une organisation révolutionnaire.
\subsection[{2. La nature est recouverte de marchandise}]{\textsc{2.} La nature est recouverte de marchandise}
\noindent L’organisation technocratique hausse la médiation technique à son plus haut point de cohérence. On sait depuis longtemps que le maître s’approprie le monde objectif à l’aide de l’esclave ; que l’outil n’aliène le travailleur qu’à l’instant où le maître le détient. De même, dans la consommation, les biens n’ont en soi rien d’aliénant, mais le choix conditionné et l’idéologie qui les enrobe déterminent l’aliénation de leurs acheteurs. L’outil dans la production, le choix conditionné dans la consommation deviennent le support du mensonge, les médiations qui, incitant l’homme, producteur et consommateur, à \emph{agir} illusoirement dans une \emph{passivité} réelle, le transforment en être essentiellement dépendant. Les médiations usurpées séparent l’individu de lui-même, de ses désirs, de ses rêves, de sa volonté de vivre ; ainsi s’accrédite la légende selon laquelle nul ne peut se passer d’elles ni de ce qui les gouverne. Où le pouvoir échoue à paralyser par les contraintes, il paralyse par suggestion : en imposant à chacun des béquilles dont il s’assure le contrôle et la propriété. Somme de médiations aliénantes, le pouvoir attend du baptême cybernéticien qu’il le fasse accéder à l’état de totalité. Mais il n’y a pas de pouvoir total, il n’y a que des pouvoirs totalitaires. On ne sacralise pas une organisation avec le ridicule de ses prêtres.\par
A force d’être saisi par des médiations aliénées (outil, pensée, besoins falsifiés), le monde objectif (ou la nature, comme on veut) a fini par s’entourer d’une sorte d’écran qui le rend paradoxalement étranger à l’homme à mesure que l’homme le transforme et se transforme. Le voile des rapports sociaux enveloppe inextricablement le domaine naturel. Ce que l’on appelle aujourd’hui « naturel » est aussi artificiel que le fond de teint « naturel » des parfumeurs. Les instruments de la \emph{praxis} n’appartiennent pas en propre aux tenants de la \emph{praxis}, aux travailleurs, et c’est évidemment pourquoi la zone d’opacité qui sépare l’homme de lui-même et de la nature fait partie de l’homme et de la nature. Il n’y a pas une nature à retrouver mais une nature à refaire, à reconstruire.\par
La quête de la vraie nature, de la vie naturelle opposée brutalement au mensonge de l’idéologie sociale représente une des naïvetés les plus touchantes d’une bonne partie du prolétariat révolutionnaire, des anarchistes, et d’esprits aussi remarquables que le jeune Wilhelm Reich, par exemple.\par
Sous le règne de l’exploitation de l’homme par l’homme, la transformation réelle de la nature passe par la transformation réelle du mensonge social. Jamais dans leur lutte, la nature et l’homme n’ont été réellement face à face. La médiation du pouvoir social hiérarchisé et son organisation de l’apparence les unissaient et les séparaient. Transformer la nature, c’était la socialiser, mais on a mal socialisé la nature. Il n’y a de nature que sociale puisque l’histoire n’a jamais connu de société sans pouvoir.\par
Un tremblement de terre est-il un phénomène naturel ? Atteignant les hommes, il ne les atteint que dans la sphère du social aliéné. Qu’est-ce qu’un tremblement de terre-en-soi ? Si, à l’instant où j’écris, une secousse sismique qui restera ignorée de toute éternité ébranle le relief de Sirius, que puis-je faire d’autre que de l’abandonner aux résidus métaphysiques des universités et des centres de pensée pure ?\par
Et la mort, elle aussi, frappe les hommes socialement. Non seulement parce que l’énergie et la richesse absorbées par le gâchis militaire et l’anarchie capitaliste ou bureaucratique offriraient à la lutte scientifique contre la mort un appoint particulièrement nécessaire, mais surtout parce que le bouillon de culture où se développent les germes de la mort s’entretient, avec la bénédiction de la science, dans le gigantesque laboratoire de la société. (\emph{Stress}, usure nerveuse, conditionnement, envoûtement, thérapeutiques maladives.) Seules les bêtes ont encore droit à la mort naturelle, et encore…\par
Se dégageant de l’animalité supérieure par l’histoire, les hommes en arriveraient-ils à regretter le contact animal avec la nature ? C’est, je crois, le sens puéril qu’il convient d’attribuer à la recherche du naturel. Mais, enrichi et renversé, un tel désir signifie le dépassement de 30 000 ans d’histoire.\par
La tâche est actuellement de saisir une nature nouvelle comme adversaire valable, c’est-à-dire de la resocialiser en libérant l’appareillage technique de la sphère d’aliénation, en l’ôtant des mains des dirigeants et des spécialistes. La nature ne prendra le sens d’adversaire valable qu’au terme d’une désaliénation sociale, au sein d’une civilisation « mille fois supérieure » où la créativité de l’homme ne rencontrera pas, comme premier obstacle à son expansion, l’homme lui-même.\par

\astermono

\noindent L’organisation technique ne succombe pas sous la pression d’une force extérieure. Sa faillite est l’effet d’un pourrissement interne. Loin de subir le châtiment d’une volonté prométhéenne, elle crève au contraire de ne s’être jamais émancipée de la dialectique du maître et de l’esclave. Même s’ils régnaient un jour, les cybernéticiens gouverneraient toujours trop près du bord. Leurs plus étincelantes prospectives appellent déjà ces mots d’un ouvrier noir à un patron blanc. (\emph{Présence africaine}, 1956) :\par

\begin{quoteblock}
\noindent « Quand nous avons vu vos camions, vos avions, nous avons cru que vous étiez des dieux et puis, après des années, nous avons appris à conduire vos camions, bientôt nous apprendrons à conduire vos avions, et nous avons compris que ce qui vous intéressait le plus, c’était de fabriquer les camions et les avions et de gagner de l’argent. Nous, ce qui nous intéresse, c’est de nous en servir. Maintenant, vous êtes nos forgerons. »\end{quoteblock}

\section[{X. Le règne du quantitatif}]{X. Le règne du quantitatif}\renewcommand{\leftmark}{X. Le règne du quantitatif}


\begin{argument}\noindent Les impératifs économiques tentent d’imposer à l’ensemble des comportements humains la mesure étalonnée des marchandises. La très grande quantité devrait tenir lieu de qualitatif, mais même la quantité est contingentée, économisée. Le mythe se fonde sur la qualité, l’idéologie sur la quantité. La saturation idéologique est un morcellement en petites quantités contradictoires, incapables de ne pas se détruire et de n’être pas détruites par la négativité qualitative du refus populaire (1). – Quantitatif et linéaire sont indissociables. Ligne et mesure du temps, ligne et mesure de la vie définissent la survie ; une suite d’instants interchangeables. Ces lignes entrent dans la géométrie confuse du pouvoir (2).
\end{argument}

\subsection[{1. La quantité pour seule qualité}]{\textsc{1.} La quantité pour seule qualité}
\noindent Le système des échanges commerciaux a fini par gouverner les relations quotidiennes de l’homme avec lui-même et avec ses semblables. Sur l’ensemble de la vie publique et privée, le quantitatif règne.\par

\begin{quoteblock}
\noindent « Je ne sais pas ce que c’est qu’un homme, avouait le marchant de \emph{L’Exception et la règle}, je ne connais que son prix. »\end{quoteblock}

\noindent Dans la mesure où les individus acceptent et font exister le pouvoir, le pouvoir aussi les réduit à sa mesure, il les étalonne. Pour le système autoritaire, qu’est-ce que l’individu ? Un point dûment situé dans sa perspective. Un point qu’il reconnaît certes, mais à travers une mathématique, sur un diagramme où les éléments, portés en abscisses et ordonnées, lui assignent sa place exacte.\par
La capacité chiffrée de produire et de faire produire, de consommer et de faire consommer, concrétise à merveille cette expression si chère aux philosophes (et par ailleurs si révélatrice de leur mission) : la mesure de l’homme. Il n’est pas jusqu’à l’humble plaisir d’une randonnée en voiture qui ne s’évalue communément sur le nombre de kilomètres parcourus, la vitesse atteinte, et la consommation d’essence. À la cadence où les impératifs économiques s’approprient les sentiments, les passions, les besoins, payant comptant leur falsification, il ne restera bientôt plus à l’homme que le souvenir d’avoir été. L’histoire, où l’on vivra rétrospectivement, consolera de survivre. Comment la vraie joie tiendrait-elle dans un espace-temps mesurable et mesuré ? Même pas un rire franc. Tout au plus l’épais contentement de celui-qui-a-pour-son-argent, et existe à ce taux. Il n’y a de mesurable que l’objet, c’est pourquoi tout échange réifie.\par

\astermono

\noindent Ce qui subsistait de tension passionnelle entre la jouissance et sa recherche aventureuse achève de se désagréger en une succession haletante de gestes reproduits mécaniquement, et sur un rythme dont on attend vainement qu’il hausse, ne serait-ce, qu’à un semblant d’orgasme. L’Éros quantitatif de la vitesse, du changement rapide, de l’amour contre la montre déforme partout le visage authentique du plaisir.\par
Le qualitatif revêt lentement l’aspect d’un infini quantitatif, une série sans fin et dont la fin temporaire est toujours la négation du plaisir, une insatisfaction de base, comme dans le donjuanisme. Encore si la société actuelle encourageait une insatisfaction de ce genre, si elle laissait à la soif insatiable d’absolu licence d’exercer ses ravages et son attrait délirant ! Qui refuserait d’accorder quelque charme à la vie d’un oisif, un tant soit peu désabusé, mais jouissant à loisir de tout ce qui rend la passivité délicieuse : sérail de jolies filles et de beaux esprits, drogues raffinées, mets recherchés, liqueurs brutales, parfums suaves ; à un homme, dis-je, moins enclin à changer la vie qu’à chercher refuge dans ce qu’elle offre de plus accueillant ; à un jouisseur de grand style (les porcs n’ont que la manière de jouir) ? Mais quoi ! Il n’est aujourd’hui personne qui détienne un tel choix : la quantité même est contingentée par les sociétés de l’Est et de l’Ouest. Un magnat de la finance à qui il ne resterait qu’un mois à vivre refuserait encore d’engloutir le tout de sa fortune dans une immense orgie. La morale du profit et de l’échange ne lâche pas sa proie ; l’économie capitaliste à l’usage des familles s’appelle parcimonie.\par
Et pourtant, quelle aubaine pour la mystification que d’emprisonner le quantitatif dans la peau du qualitatif, je veux dire de laisser à la multiplicité des possibles l’illusion prestigieuse de fonder un monde à plusieurs dimensions. Englober les échanges dans le don, laisser entre la Terre et le Ciel s’épanouir toutes les aventures (celle de Gilles de Rais, celle de Dante), c’est cela précisément qui était interdit à la classe bourgeoise, c’est cela qu’elle détruisait au nom du commerce et de l’industrie. Et à quelle nostalgie elle se condamnait ainsi ! Pauvre et précieux catalyseur – à la fois tout et rien –, grâce auquel la société sans classe et sans pouvoir autoritaire réalisera les rêves de son enfance aristocratique.\par
Les sociétés unitaires féodales et tribales tenaient en l’acte de foi un élément qualitatif mythique et mystifiant de première importance. A peine la bourgeoisie a-t-elle brisé l’unité du pouvoir et de Dieu qu’elle s’efforce d’enrober d’esprit unitaire ce qui n’est plus entre ses mains que parcelles et miettes de pouvoir. Hélas, sans unité, pas de qualitatif ! La démocratie triomphe avec l’atomisation sociale. La démocratie est le pouvoir limité du plus grand nombre et le pouvoir du plus grand nombre limité. Très tôt, les grandes idéologies lâchent la foi pour le nombre. Qu’est-ce que la patrie ? Aujourd’hui quelques milliers d’anciens combattants. Et ce que Marx et Engels appelaient « notre parti » ? Aujourd’hui quelques milliers de voix électorales, quelques milliers de colleurs d’affiches ; un parti de masse.\par
En fait, l’idéologie tire son essence de la quantité, elle n’est rien qu’une idée reproduite un grand nombre de fois dans le temps (le conditionnement pavlovien) et dans l’espace (la prise en charge par les consommateurs). L’idéologie, l’information, la culture tendent de plus en plus à perdre leur contenu pour devenir du quantitatif pur. Moins une information a d’importance, plus elle est répétée et mieux elle éloigne les gens de leurs véritables problèmes. Mais nous sommes loin du gros mensonge dont Goebbels dit qu’il passe mieux que tout autre. La surenchère idéologique étale avec la même force de conviction cent bouquins, cent poudres à lessiver, cents conceptions politiques dont elle a successivement fait admettre l’incontestable supériorité. Même dans l’idéologie, la quantité se détruit par la quantité ; les conditionnements s’usent à force de se heurter. Comment retrouverait-on de la sorte la vertu du qualitatif, qui soulève des montagnes.\par
Au contraire, les conditionnements contradictoires risquent d’aboutir à un trauma, à une inhibition, à un refus radical du décervelage. Certes, il existe une parade : laisser au conditionné le soin de juger entre deux mensonges quel est le plus vrai, poser de fausses questions, susciter de faux dilemmes. Reste que la vanité de telles diversions pèse peu au regard du mal de survie auquel la société de consommation expose ses membres. De l’ennui peut naître à chaque instant l’irrésistible refus de l’uniformité. Les événements de Watts, de Stockholm et d’Amsterdam ont montré de quel prétexte infime pouvait jaillir le trouble salutaire. Quelle quantité de mensonges réitérés un seul geste de poésie révolutionnaire, n’est-il pas capable d’anéantir ? De Villa à Lumumba, de Stockholm à Watts, l’agitation qualitative, celle qui radicalise les masses parce qu’elle est issue du radicalisme des masses, corrige les frontières de la soumission et de l’abrutissement.
\subsection[{2. Pouvoir des linéaires}]{\textsc{2.} Pouvoir des linéaires}
\noindent Sous les régimes unitaires, le sacré cimentait la pyramide sociale où, du seigneur au serf, chaque être particulier tenait sa place selon le vœu de la Providence, l’ordre du monde et le bon plaisir du roi. La cohésion de l’édifice, corrodée par la critique dissolvante de la jeune bourgeoisie, disparaîtra sans que s’efface, on le sait, l’ombre de la hiérarchie divine. La dislocation de la pyramide, loin de supprimer l’humain, l’émiette. On voit s’absolutiser de petits être particuliers, de petits « citoyens » rendus disponibles par l’atomisation sociale ; l’imagination boursouflée de l’égocentrisme érige en univers ce qui tient en un point, tout pareil à des milliers d’autres points, grains de sable libres, égaux et fraternels, s’affairant çà et là comme autant de fourmis dont on vient bouleverser le savant labyrinthe. Ce ne sont que des lignes devenues folles depuis que Dieu a cessé de leur offrir un point de convergence, des lignes qui s’entrelacent et se brisent dans un apparent désordre ; car nul ne s’y trompe : en dépit de l’anarchie concurrentielle et de l’isolement individualiste, des intérêts de classe et de castes se nouent, structurant une géométrie rivale de la géométrie divine, mais bien impatiente d’en reconquérir la cohérence.\par
Or la cohérence du pouvoir unitaire, bien que fondée sur le principe divin, est une cohérence sensible, intimement vécue par chacun. Le principe matériel du pouvoir parcellaire n’autorise, paradoxalement, qu’une cohérence abstraite. Comment l’organisation de la survie économique se substituerait-elle sans heurt à ce Dieu immanent, partout présent, partout pris à témoin jusque dans les gestes les plus dénués d’importance (couper du pain, éternuer…) ? Supposons même que le gouvernement laïcisé des hommes puisse, avec l’aide des cybernéticiens, égaler la toute-puissance (d’ailleurs parfaitement relative) du mode de domination féodal, qui suppléera – et comment ? – à l’ambiance mythique et poétique enveloppant la vie des communautés socialement solidaires et lui assurant, en quelque sorte, une troisième dimension ? La bourgeoisie est bel et bien prise au piège de sa demi-révolution.\par

\astermono

\noindent Quantitatif et linéaire se confondent. Le qualitatif est plurivalent, le quantitatif univoque. La vie brisée, c’est la ligne de vie.\par
L’ascension radieuse de l’âme vers le ciel fait place à la prospection bouffonne du futur. Aucun moment ne s’irradie plus dans le temps cyclique des vieilles sociétés ; le temps est un fil ; de la naissance à la mort, de la mémoire du passé au futur attendu, une éternelle survie étire sa succession d’instants et de présents hybrides également grignotés par le temps qui fuit, par le temps qui vient. Le sentiment de vivre en symbiose avec les forces cosmiques – ce sens du simultané – révélait aux Anciens des joies que notre \emph{écoulement dans le monde} est bien en peine de nous accorder. Que reste-t-il d’une telle joie ? Le vertige de passer, la hâte de marcher au même pas que le temps. Être de son temps, comme disent ceux qui en font commerce.\par
Il ne s’agit pas de regretter le temps cyclique, le temps de l’effusion mystique, mais bien de le corriger, de le centrer sur l’homme, non sur l’animal divin. L’homme n’est pas le centre du temps actuel ; seulement un point. Le temps se compose d’une succession de points, chacun pris indépendamment des autres, comme un absolu, amis un absolu répété, rabâché. Parce qu’ils se situent sur une ligne unique, tous les gestes, tous les instants prennent une égale importance. C’est cela le prosaïsme. Le règne du quantitatif est le règne du pareil au même. Les parcelles absolutisées ne sont-elles pas interchangeables ? Dissociés les uns des autres – et donc séparés de l’homme lui-même – les instants de la survie se suivent et se ressemblent, comme se suivent et se ressemblent les attitudes spécialisées qui leur répondent, les rôles. On fait l’amour comme on fait de la moto. Chaque instant a son stéréotype, et les fragments de temps emportent les fragments d’hommes vers un incorrigible passé.\par
À quoi bon enfiler des perles dans l’espoir d’un collier de souvenirs ! Encore si la profusion de perles détruisait le collier, mais non. Instant par instant, le temps fait son puits, tout se perd, rien ne se crée…\par
Je ne désire pas une suite d’instants mais un grand moment. Une totalité vécue, et qui ne connaît pas de durée. Le temps pendant lequel je dure n’est que le temps de mon vieillissement. Et cependant, parce qu’il faut aussi survivre pour vivre, en ce temps-là s’enracinent nécessairement les moments virtuels, les possibles. Fédérer les instants, les alléger de plaisir, en dégager la promesse de vie, c’est déjà apprendre à construire une « situation ».\par

\astermono

\noindent Les lignes de survie individuelles s’entrecroisent, se heurtent, se coupent. Chacune assigne à la liberté de l’autre ses limites, les projets s’annulent au nom de leur autonomie. Ainsi se fonde la géométrie du pouvoir parcellaire.\par
On croit vivre dans le monde et l’on se range en fait dans une perspective. Non plus la perspective simultanée des peintres primitifs mais celle des rationalistes de la Renaissance. Les regards, les pensées, les gestes échappent avec peine à l’attraction du lointain point de fuite qui les ordonne et les corrige ; les situe dans son spectacle. Le pouvoir est le plus grand urbaniste. Il lotit la survie en parcelles privée et publique, il rachète à bas prix les terrains défrichés, interdit de construire sans passer par ses normes. Lui-même construit pour exproprier chacun de sa peau. Il construit avec une lourdeur que lui envient ses singes bâtisseurs de villes, traduisant en zones de dirigeants, en quartiers de cadres, en blocs de travailleurs (comme à Mourenx) le vieux grimoire de la sainte hiérarchie.\par
Reconstruire la vie, rebâtir le monde : une même volonté.
\section[{XI. Abstraction médiatisée et médiation abstraite}]{XI. Abstraction médiatisée et médiation abstraite}\renewcommand{\leftmark}{XI. Abstraction médiatisée et médiation abstraite}


\begin{argument}\noindent La réalité est aujourd’hui emprisonnée dans la métaphysique comme elle l’était jadis dans la vision théologique. La façon de voir, imposée par le pouvoir, « abstrait » les médiations de leur fonction initiale, qui est de prolonger dans le réel les exigences du vécu. Mais la médiation ne perd jamais tout à fait le contact avec le vécu, elle résiste à l’attraction du champ autoritaire. Le point de résistance est l’observatoire de la subjectivité. Jusqu’à présent, les métaphysiciens n’ont fait qu’organiser le monde, il s’agit maintenant de le transformer contre eux (1). – Le règne de la survie garantie fait lentement s’effriter la croyance au pouvoir nécessaire (2). – Ainsi s’annonce un refus croissant des formes qui nous gouvernent, un refus de leur principe ordonnateur (3). – La théorie radicale, seule garantie du refus cohérent, pénètre les masses parce qu’elle prolonge leur créativité spontanée. L’idéologie « révolutionnaire » est la théorie récupérée par les dirigeants. – Les mots existent à la frontière de la volonté de vivre et de sa répression ; leur emploi décide de leur sens ; l’histoire contrôle les modalités d’emploi. La crise historique du langage annonce un \emph{dépassement} possible vers la poésie des gestes, vers le grand jeu sur les signes (4).
\end{argument}

\subsection[{1. La métaphysique organise le monde}]{\textsc{1.} La métaphysique organise le monde}
\noindent Quel est ce détour par où, me poursuivant, j’achève de me perdre ? Quel écran me sépare de moi sous couvert de me protéger ? Et comment me retrouver dans cet émiettement qui me compose ? J’avance vers je ne sais quelle incertitude de me saisir jamais. Tout se passe comme si mes pas me précédaient, comme si pensées et affects épousaient les contours d’un paysage mental qu’ils imaginent créer, qui les modèle en fait. Une force absurde – d’autant plus absurde qu’elle inscrit dans la rationalité du monde et paraît incontestable – me contraint de sauter sans relâche pour atteindre un sol que mes pieds n’ont jamais quitté. Et par ce bond inutile vers moi, mon présent m’est volé ; je vis le plus souvent en décalage avec ce que je suis, au rythme du temps mort.\par
On s’étonne beaucoup trop peu à mon sens de voir le monde emprunter, à certaines époques, les \emph{formes} de la métaphysique dominante. La croyance au diable et à Dieu, si farfelue soit-elle, fait de l’un et l’autre fantômes une réalité vivante sitôt qu’une collectivité les juge assez présents pour inspirer des textes de lois. De même la stupide distinction entre cause et effet a pu régir une société où les comportements humains et les phénomènes en général étaient analysés en termes de cause et d’effet. Et aujourd’hui encore, personne ne peut sous-estimer la dichotomie aberrante entre pensée et action, théorie et pratique, réel et imaginaire… Ces idées-là sont des forces d’organisation. Le monde du mensonge est un monde réel, on y tue et on y est tué, il est préférable de ne pas l’oublier. On a beau ironiser sur le pourrissement de la philosophie, les philosophes contemporains se retirent avec un sourire entendu derrière leur médiocrité de pensée : ils savent au moins que le monde reste une construction philosophique, un grand débarras idéologique. Nous survivons dans un paysage métaphysique. La médiation abstraite et aliénante qui m’éloigne de moi est terriblement concrète.\par
Part de Dieu accordée à l’homme, la Grâce a survécu à Dieu lui-même. Elle est laïcisée. Quittant la théologie pour la métaphysique, elle est restée incrustée dans l’homme individuel comme un guide, un mode de gouvernement intériorisé. Quand l’imagerie freudienne accroche au-dessus de la porte du moi le monstre du Superego, elle succombe moins à la tentation d’une simplification abusive qu’à un refus d’enquêter plus avant sur l’origine sociale des contraintes. (Ce que Reich a bien compris.) C’est parce que les hommes sont divisés, non seulement entre eux mais aussi en eux, que l’oppression règne. Ce qui sépare de soi et affaiblit unit par de faux liens au pouvoir, ainsi renforcé et choisi comme protecteur, comme \emph{père}.\par

\begin{quoteblock}
\noindent « La médiation dit Hegel, est l’égalité avec soi-même-se-mouvant. »\end{quoteblock}

\noindent Mais se mouvoir peut-être aussi se perdre. Et lorsqu’il ajoute :\par

\begin{quoteblock}
\noindent « C’est le moment du \emph{meurs} et du \emph{deviens} »\end{quoteblock}

\noindent …il n’y a pas un mot à changer pour que le sens diffère radicalement selon la perspective où l’on se place, celle du pouvoir totalitaire ou celle de l’homme total.\par
La médiation échappe-t-elle à mon contrôle, c’est aussitôt vers l’étrange et l’inhumain que m’entraîne une démarche que je crois mienne. Engels montrait judicieusement qu’une pierre, un fragment de la nature étrangère à l’homme, devenait humaine sitôt qu’elle prolongeait la main en servant d’outil (et la pierre humanise à son tour la main de l’hominien). Mais approprié par un maître, un patron, une commission de \emph{planning}, une organisation dirigeante, l’outil change de sens, il dévie vers d’autres prolongements les gestes de celui qui en use. Ce qui est vrai pour l’outil vaut pour toutes les médiations.\par
De même que Dieu régnait en conseiller des Grâces, le magnétisme du principe gouvernant s’empare du plus grand nombre possible de médiations. Le pouvoir est la somme des médiations aliénées et aliénantes. La science (\emph{scientia theologioe ancilla}) a opéré la reconversion du mensonge divin en information opérationnelle, en abstraction organisée, rendant au mot son sens étymologique, \emph{ab-trahere}, tirer hors de.\par
L’énergie dépensée par l’individu pour se réaliser, pour se prolonger dans le monde selon ses désirs et ses rêves, est soudain freinée, mise en suspens, aiguillée vers d’autres voies, récupérée. La phase normale de l’accomplissement change de plan, quitte le vécu, s’enfonce dans la transcendance.\par
Or le mécanisme d’abstraction n’obéit pas purement et simplement au principe autoritaire. Tout amoindri qu’il soit par sa médiation volée, l’homme entre dans le labyrinthe du pouvoir avec les armes de la volonté agressive de Thésée. S’il arrive qu’il s’y perde, c’est d’avoir auparavant perdu Ariane, doux lien qui l’attache à la vie, volonté d’être soi. Car seule l’incessante relation de la théorie et de la praxis vécue permet d’espérer la fin de toutes les dualités, le règne de la totalité, la fin du pouvoir de l’homme sur l’homme.\par
Le sens de l’humain n’est pas dévoyé vers l’inhumain sans résistance, sans combat. Où se situe le champ d’affrontement ? Toujours dans le prolongement immédiat du vécu, dans la spontanéité. Non que j’oppose ici à la médiation abstraite une sorte de spontanéité brute, disons instinctive, ce serait reproduire à un niveau supérieur le choix imbécile entre la spéculation pure et l’activisme borné, la disjonction entre théorie et pratique. La tactique adéquate consiste plutôt à déclencher l’attaque à l’endroit précis où s’embusquent les détrousseurs du vécu, à la frontière du geste amorcé et de son prolongement perverti, au moment même où le geste spontané est aspiré par le contresens et le malentendu. On dispose là, pendant un infime laps de temps, d’un panorama qui embrasse à la fois, dans la même prise de conscience, les exigences du vouloir-vivre et ce que l’organisation sociale se prépare à en faire ; le vécu et sa récupération par les machines autoritaires. Le point de résistance est l’observatoire de la subjectivité. Pour des raisons identiques, ma connaissance du monde n’existe valablement qu’à l’instant où je le transforme.
\subsection[{2. Le pouvoir n’est plus nécessaire à la survie}]{\textsc{2.} Le pouvoir n’est plus nécessaire à la survie}
\noindent La médiation du pouvoir exerce un chantage permanent sur l’immédiat. Certes, l’idée qu’un geste ne peut s’achever dans la totalité de ses implications reflète exactement la réalité du monde déficitaire, d’un monde de la non-totalité ; mais elle renforce du même coup le caractère métaphysique des faits leur falsification officielle. Le sens commun a fait siennes des allégations comme : « Les chefs sont toujours nécessaires », « Ôtez l’autorité, vous précipitez l’humanité dans la barbarie et le chaos » et \emph{tutti quanti}. La coutume, il est vrai, a si bien mutilé l’homme, qu’il croit, se mutilant, obéir à la loi naturelle. Peut-être est-ce l’oubli de sa propre perte qui l’accroche le mieux au pilori de la soumission. Quoi qu’il en soit, il entre bien dans la mentalité d’un esclave d’associer le pouvoir à la seule forme de vie possible, à la survie. Et il entre bien dans les desseins du maître d’encourager tel sentiment.\par
Dans la lutte de l’espèce humaine pour sa survie, l’organisation sociale hiérarchisée a marqué indéniablement une étape décisive. La cohésion d’une collectivité autour de son chef a représenté à un moment de l’histoire la chance de salut la plus sûre, sinon la seule. Mais la survie était garantie au prix d’une aliénation nouvelle ; ce qui la sauvegardait l’emprisonnait, ce qui la maintenait en vie lui interdisait de croître. Les régimes féodaux étaient crûment la contradiction : des serfs, mi-hommes mi-bêtes, voisinent avec une poignée de privilégiés dont certains s’efforcent d’accéder individuellement à l’exubérance et à la puissance de vivre.\par
La conception féodale se soucie peu de la survie proprement dite : les famines, les épidémies, les massacres ôtent du meilleur des mondes des millions d’êtres sans émouvoir outre mesure des générations de lettrés et de fins jouisseurs. Au contraire, la bourgeoisie trouve dans la survie la matière première de ses intérêts économiques. Le besoin de se nourrir et de subsister matériellement motive forcément le commerce et l’industrie. Si bien qu’il n’est pas abusif de voir dans le primat de l’économie, ce dogme de l’esprit bourgeois, la source même de son célèbre humanisme. Si les bourgeois préfèrent l’homme à Dieu, c’est qu’il produit et consomme, achète et fournit. L’univers divin, qui est en deçà de l’économie, a tout lieu de leur déplaire autant que le monde de l’homme total, qui en est l’au-delà.\par
À rassasier la survie, à la gonfler artificiellement, la société de consommation suscite un nouvel appétit de vivre. Partout où la survie est aussi garantie que le travail, les anciennes protections se transforment en obstacles. Non seulement la lutte pour survivre empêche de vivre mais, devenue lutte sans revendication réelle, elle corrode jusqu’à la survie même, elle rend précaire ce qui était dérisoire. Si la survie ne mue pas, elle crèvera, nous étouffant tous dans sa peau trop étroite.\par
La protection des maîtres a perdu sa raison d’être depuis que la sollicitude mécanique des \emph{gadgets} a mis fin théoriquement à la nécessité de l’esclave. Désormais, la terreur savamment entretenue d’une apothéose thermonucléaire est l’\emph{ultima ratio} des dirigeants. Le pacifisme de la coexistence garantit leur existence. Mais l’existence des dirigeants ne garantit plus celle des hommes. Le pouvoir ne protège plus, il se protège contre chacun. Création spontanée de l’inhumain par l’humain, il n’est plus aujourd’hui que l’inhumaine interdiction de créer.
\subsection[{3. Le pouvoir est forme pure}]{\textsc{3.} Le pouvoir est forme pure}
\noindent Chaque fois qu’est différé l’achèvement total et immédiat d’un geste, le pouvoir se renforce dans sa fonction de grand médiateur. Au contraire, la poésie spontanée est l’anti-médiation par excellence.\par
De façon schématique, on est fondé d’admettre que l’aspect « somme des contraintes » caractérisant les pouvoirs parcellaires de type bourgeois ou soviétique se résorbe peu à peu dans une organisation axée davantage sur les médiations aliénantes. La fascination idéologique remplace la baïonnette. Ce mode perfectionné de gouvernement n’est pas sans évoquer les ordinateurs de la cybernétique. Planifiant et supprimant, selon les directives prudentes de la gauche technocratique et spécialisée, les petits intermédiaires (chefs spirituels, généraux putschistes, stalino-franquistes et autres enfants d’Ubu), l’Argus électronique construit son absolutisme et l’état de bien-être. Mais plus il aliène les médiations, plus la soif de l’immédiat devient insatiable, plus la poésie sauvage des révolutionnaires abolit les frontières.\par
L’autorité, à son stade ultime, va culminer dans l’union de l’abstrait et du concret. Le pouvoir abstrait déjà comme on guillotine encore. La face du monde éclairée par lui s’ordonne selon une métaphysique du réel ; et c’est pain bénit que de voir les fidèles philosophes rempiler à son service avec un grade de technocrate, de sociologue, de spécialistes à tout crin.\par
La forme pure qui hante l’espace social est le visage discernable de la mort des hommes. Elle est la névrose avant la nécrose, le mal de survie qui s’étend à mesure qu’au vécu se substituent des images, des formes, des objets, que la médiation aliénée transmute le vécu en chose, le madréporise. C’est un homme ou un arbre ou une pierre… prophétise Lautréamont.\par
Gombrowicz, lui, rend un hommage mérité à la Forme, à la vieille entremetteuse du pouvoir, aujourd’hui promue au rang d’honneur des instances gouvernantes :\par

\begin{quoteblock}
 \noindent « Vous n’avez jamais su apprécier comme il se doit, et faire comprendre aux autres, quelle importance considérable a le rôle de la Forme dans notre vie. Même dans la psychologie, vous n’avez pas su assurer à la Forme la place qui lui convient. Jusqu’à maintenant, nous continuons à juger que ce sont les sentiments, les instants ou les idées qui commandent notre conduite, alors que nous considérons la Forme tout au plus comme un inoffensif ornement accessoire. Et quand la veuve, accompagnant le corbillard de son mari, pleure tendrement, nous pensons qu’elle pleure parce qu’elle ressent douloureusement sa perte. Lorsque quelque ingénieur, médecin ou avocat assassine son épouse, ses enfants ou un ami, nous estimons qu’il se laisse porter à l’assassinat par les instincts sanguinaires et violents. Lorsque quelque politicien s’exprime niaisement, trompeusement ou mesquinement dans un discours public, nous disons qu’il est sot parce qu’il s’exprime sottement. Mais, dans la réalité, l’affaire se présente ainsi : l’être humain ne s’extériorise pas d’une manière immédiate et conforme à sa nature, mais toujours à travers une Forme définie et cette Forme, cette manière d’être, cette manière de parler et de réagir ne proviennent pas uniquement de lui-même, mais lui sont imposés de l’extérieur.\par
 
\begin{quoteblock}
\noindent « Et voilà que ce même homme peut se manifester tantôt avec sagesse, tantôt sottement ou sanguinairement ou angéliquement, mûrement ou non, suivant la forme qui se présente à lui et selon la pression du conditionnement… Quand vous opposerez-vous consciemment à la Forme ? Quand cesserez-vous de vous identifier à ce qui vous définit ? »\end{quoteblock}

 \end{quoteblock}

\subsection[{4. La crise historique du langage annonce un dépassement possible vers la poésie des gestes}]{\textsc{4.} La crise historique du langage annonce un \emph{dépassement} possible vers la poésie des gestes}
\noindent Dans \emph{Critique de la Philosophie du droit de Hegel}, Marx écrit :\par

\begin{quoteblock}
\noindent « La théorie devient force matérielle lorsqu’elle pénètre les masses. La théorie est capable de pénétrer les masses dès qu’elle fait des démonstrations \emph{ad hominem} et elle fait des démonstrations \emph{ad hominem} dès qu’elle devient radicale. Être radical, c’est prendre les choses par la racine. Et la racine de l’homme, c’est l’homme lui-même. »\end{quoteblock}

\noindent En somme, la théorie radicale pénètre les masses parce qu’elle en est d’abord l’émanation. Dépositaire d’une créativité spontanée, elle a pour mission d’en assurer la force de frappe. Elle est la technique révolutionnaire au service de la poésie. Une analyse des insurrections passées et présentes, qui s’exprime hors de la volonté de reprendre la lutte avec plus de cohérence et d’efficacité, sert fatalement l’ennemi, elle se range dans la culture dominante. On ne peut parler opportunément des moments révolutionnaires sans les donner à vivre à brève échéance. Simple critère pour marquer les penseurs errants et tintinnabulants de la gauche planétaire.\par
Ceux qui savent terminer une révolution se trouvent toujours au premier plan pour l’expliquer à ceux qui l’ont faite. Ils disposent de raisons aussi excellentes pour l’expliquer que pour la terminer ; c’est le moins que l’on puisse dire. Quand la théorie échappe aux artisans d’une révolution, elle finit par se dresser contre eux. Elle ne les pénètre plus, elle les domine, elle les conditionne. Ce que le peuple n’accroît plus par la force de ses armes accroît la force de ceux qui le désarment. Le léninisme, c’est aussi la révolution expliquée à coups de fusil aux marins de Cronstadt et aux partisans de Makhno. Une idéologie.\par
Quand les dirigeants s’emparent de la théorie, elle se change entre leurs mains en idéologie, en une argumentation \emph{ad hominem} contre l’homme lui-même. La théorie radicale émane de l’individu, de l’être en tant que sujet ; elle pénètre les masses par ce qu’il y a de plus créatif dans chacun, par la subjectivité, par la volonté de réalisation. Au contraire, le conditionnement idéologique est le maniement technique de l’inhumain, du poids des choses. Il change les hommes en objets qui n’ont d’autre sens que l’Ordre où ils se rangent. Il les assemble pour les isoler, fait de la foule une multiplication de solitaires.\par
L’idéologie est le mensonge du langage ; la théorie radicale est la vérité du langage ; leur conflit, qui est celui de l’homme et de la part d’inhumain qu’il sécrète, préside à la transformation du monde en réalités humaines, comme à sa transmutation en réalités métaphysiques. Tout ce que les hommes font et défont passe par la médiation du langage. Le champ sémantique est un des principaux champs de bataille où s’affronte la volonté de vivre et l’esprit de soumission.\par

\astermono

\noindent Le conflit est inégal. Les mots servent le pouvoir mieux que les hommes ne se servent d’eux ; ils le servent plus fidèlement que la plupart des hommes, plus scrupuleusement que les autres médiations (espace, temps, technique…). C’est que toute transcendance prend sa source dans le langage, s’élabore dans un système de signes et de symboles (mots, danse, rite, musique, sculpture, bâtisse…). À l’instant où le geste soudain suspendu, inachevé, cherche à se prolonger sous une forme qui tôt ou tard le fasse s’achever, se réaliser, – de même qu’un générateur transforme son énergie mécanique en énergie électrique acheminée à des kilomètres de distance jusqu’à un autre moteur où elle se reconvertit en énergie mécanique, – le langage s’empare du vécu, l’emprisonne, le vide de sa substance, l’\emph{abstrait}. Et les catégories sont prêtes, condamnant à l’incompréhension, au non-sens, ce qui n’entre pas dans leurs schèmes, appelant à l’existence-dans-le-pouvoir ce qui gît dans le néant, ce qui n’a pas encore sa place au sein de l’Ordre. La répétition des signes reconnus fonde l’idéologie.\par
Et, cependant, les hommes se servent aussi des mots et de signes pour tenter de parfaire leurs gestes interrompus. Et parce qu’ils le font, il existe un langage poétique ; un langage du vécu qui, pour moi, se confond avec la théorie radicale, avec la théorie pénétrant les masses, devenant force matérielle. Même récupérée et dirigée contre son but initial, la poésie trouve tôt ou tard à s’accomplir. Le « Prolétaires de tous les pays… », qui a fait l’État stalinien, réalisera un jour la société sans classes. Aucun signe poétique n’est jamais accaparé définitivement par l’idéologie.\par
Le langage qui détourne de leur réalisation les gestes radicaux, les gestes créatifs, gestes humains par excellence, entre dans l’antipoésie, définit la fonction linguistique du pouvoir, sa science informationnelle. Cette information est le modèle de la fausse communication, de la communication de l’inauthentique, du non-vécu. Un principe me paraît bien établi : dès qu’un langage cesse d’obéir à la volonté de réalisation, il falsifie la communication ; il ne communique plus que cette abusive promesse de vérité qui s’appelle mensonge. Mais ce mensonge est la vérité de ce qui me détruit, me corrompt, me soumet. Les signes sont ainsi les points de fuite d’où divergent les perspectives antagonistes qui se partagent le monde et le construisent : la perspective du pouvoir et la perspective du vouloir-vivre. Chaque mot, chaque idée, chaque symbole possèdent une fiche d’agent double. Certains, comme le mot « patrie » ou l’uniforme de gendarme, servent le plus souvent l’autorité ; mais que l’on ne s’y trompe pas, le heurt des idéologies rivales ou leur simple usure peuvent faire un bon anarchiste du pire mercenaire (je pense ici au beau titre choisi par Bellegarigue pour sa publication : \emph{L’Anarchie, journal de l’Ordre}).\par
Pour le système sémiologique dominant, – qui est celui des castes dominantes, – il n’y a que des signes mercenaires, et le roi, dit Humpty-Dumpty, paie double ceux qu’il emploie beaucoup. Mais au fond, il n’y a pas de mercenaire qui ne se réjouisse un jour de tuer le roi. Condamnés que nous sommes au mensonge, il faut apprendre à y glisser une part de vérité corrosive. L’agitateur n’agit pas autrement ; il donne à ses mots et à ses signes un poids de réalité vécue qui entraîne tous les autres dans leur sillage. Il \emph{détourne}.\par
D’une manière générale, le combat pour le langage est le combat pour la liberté de vivre. Pour le renversement de perspective. En lui s’affrontent les faits métaphysiques et la réalité des faits ; je veux dire : les faits saisis de façon statique dans un système d’interprétation du monde et les faits saisis dans leur devenir, dans la \emph{praxis} qui les transforme.\par
On ne renversera pas le pouvoir comme on renverse un gouvernement. Le front uni contre l’autorité couvre l’étendue de la vie quotidienne et engage l’immense majorité des hommes. Savoir vivre, c’est savoir ne pas reculer d’un pouce dans sa lutte contre le renoncement. Que personne ne sous-estime l’habileté du pouvoir à gaver ses esclaves de mots jusqu’à en faire les esclaves de ses mots.\par
De quelles armes chacun dispose-t-il pour assurer sa liberté ? On peut en citer trois :\par
1. L’information corrigée dans le sens de la poésie : décryptage de nouvelles, traduction de termes officiels (« société » devenant dans la perspective opposée au pouvoir, « racket » ou « lieu du pouvoir hiérarchisé »), éventuellement glossaire ou encyclopédie (Diderot en avait parfaitement compris l’importance ; les situationnistes aussi).\par
2. Le dialogue ouvert, langage de la dialectique ; la palabre, et toute forme de discussion non spectaculaire.\par
3. Ce que Jacob Boehme appelle le « langage sensuel » (\emph{sensualische Sprache}) « parce qu’il est un miroir limpide de nos sens ». Et l’auteur de la \emph{Voie vers Dieu} précise :\par

\begin{quoteblock}
\noindent « Dans le langage sensuel, tous les esprits conversent entre eux, ils n’ont besoin d’aucun langage, car c’est le langage de la nature. »\end{quoteblock}

\noindent Si l’on se reporte à ce que j’ai nommé la recréation de la nature, le langage dont parle Boehme apparaît nettement comme le langage de la spontanéité, du « faire », de la poésie individuelle et collective ; le langage situé sur l’axe du projet de réalisation, conduisant le vécu hors « des cavernes de l’histoire ». A cela se rattache aussi ce que Paul Brousse et Ravachol entendaient par « la propagande par le fait ».\par

\astermono

\noindent Il existe une communication silencieuse. Elle est bien connue des amants. À ce stade, semble-t-il, le langage perd son importance de médiation essentielle, la pensée cesse de distraire (au sens d’éloigner de soi), les mots et les signes sont donnés par surcroît, comme un luxe, une exubérance. Que l’on songe à ces minauderies, à ce baroque de cris et de caresses si étonnamment ridicules pour qui ne partage pas l’ivresse des amants. Mais c’est aussi à la communication directe que renvoie la réponse de Léhautier, à qui le juge demandait quels compagnons anarchistes il connaissait à Paris : « les anarchistes n’ont pas besoin de se connaître pour penser la même chose ». Pour les groupes radicaux qui sauront s’élever à la plus haute cohérence théorique et vécue, les mots \emph{parfois} atteindront à ce privilège de jouer et de faire l’amour. Identité de l’érotique et de la communication.\par
J’ouvre ici une parenthèse. On a souvent remarqué que l’histoire se faisait à revers ; le problème du langage devenu superflu, du langage-jeu, l’atteste une fois de plus. Un courant baroque parcourt l’histoire de la pensée, se jouant des mots et des signes avec l’intention subversive de troubler l’ordre sémiologique et l’Ordre en général. Or la série d’attentats contre le langage, qui va de fatrasies à Jean-Pierre Brisset en passant par les hordes iconoclastes, tire sa vraie lumière de l’explosion dadaïste. La volonté d’en découdre avec les signes, la pensée, les mots, correspond pour la première fois en 1916, à une vraie crise de la communication. La liquidation du langage si souvent entreprise spéculativement trouvait enfin à se réaliser historiquement.\par
Tant qu’une époque gardait toute sa foi en la transcendance du langage et en Dieu, le maître de toute transcendance, le doute entretenu quant aux signes relevait de l’activité terroriste. Lorsque la crise des rapports humains eut brisé le réseau unitaire de communication mythique, l’attentat contre le langage prit l’allure d’une révolution. Si bien qu’il est presque engageant d’avancer, à la manière de Hegel, que la décomposition du langage a choisi le mouvement Dada pour se révéler à la conscience des hommes. Sous le régime unitaire, la même volonté de jouer avec les signes est restée sans écho, trahie en quelque sorte par l’histoire. En dénonçant la communication falsifiée, Dada amorçait le stade de dépassement du langage, la recherche de la poésie. Le langage du mythe et le langage du spectacle se rendent aujourd’hui à la réalité qui les sous-tend : le langage des faits. Ce langage portant la critique de tous les modes d’expression porte en lui sa propre critique. Pauvres sous-dadaïstes ! Pour n’avoir rien compris au dépassement nécessaire impliqué par Dada, ils continuent d’annoner que nos dialogues sont des dialogues de sourds. Aussi ont-ils leur mangeoire bien garnie dans le spectacle de la décomposition culturelle.\par

\astermono

\noindent Le langage de l’homme total sera le langage total ; peut-être la fin du vieux langage des mots. Inventer ce langage c’est reconstruire l’homme jusque dans son inconscient. Dans le mariage brisé des pensées, des mots, des gestes, la totalité se cherche à travers la non-totalité. Il faudra parler encore jusqu’au moment où les faits permettront de se taire.
\section[{XII. Le sacrifice}]{XII. Le sacrifice}\renewcommand{\leftmark}{XII. Le sacrifice}


\begin{argument}\noindent Il existe un réformisme du sacrifice qui n’est qu’un sacrifice au réformisme. L’automutilation humaniste et l’autodestruction fasciste ôtent jusqu’au choix de la mort. – Toutes les causes sont également inhumaines. – La volonté de vivre s’affirme à l’encontre de l’épidémie masochiste partout où paraissent des prétextes de révolte ; sous d’apparentes revendications parcellaires, elle prépare la révolution sans nom, la révolution de la vie quotidienne (1). – Le refus du sacrifice est le refus de la contrepartie ; l’individu ne s’échange pas. – Trois replis stratégiques sont d’ores et déjà ménagés pour le sacrifice volontaire : l’art, les grands sentiments humains, le présent (2).
\end{argument}

\subsection[{1. La révolution sans nom, la révolution de la vie quotidienne}]{\textsc{1.} La révolution sans nom, la révolution de la vie quotidienne}
\noindent Où la force et le mensonge échouent à briser l’homme et à le domestiquer, la séduction s’y emploie. Qu’est-ce que la séduction déployée par le pouvoir ? La contrainte intériorisée et drapée dans la bonne conscience du mensonge ; le masochisme de l’honnête homme. Il a bien fallu appeler le don de soi ce qui n’était que castration, peindre aux couleurs de la liberté le choix de plusieurs servitudes. Le « sentiment du devoir accompli » fait de chacun l’honorable bourreau de soi-même.\par
J’ai montré dans « Banalités de base » (\emph{Internationale situationniste}, n° 7 et 8) comment la dialectique du maître et de l’esclave impliquait que fût englobé par le sacrifice mythique du maître le sacrifice réel de l’esclave – l’un sacrifiant spirituellement son pouvoir réel à l’intérêt général, l’autre sacrifiant matériellement sa vie réelle à un pouvoir qu’il ne partage qu’apparemment. Le réseau d’\emph{apparence généralisée} ou, comme on voudra, le mensonge essentiel exigé initialement par le mouvement d’appropriation privative (appropriation des choses par l’appropriation des êtres) appartient indissolublement à la dialectique du sacrifice et fonde ainsi la fameuse séparation. L’erreur des philosophes fut de construire une ontologie et une idée d’homme éternel sur ce qui n’était qu’un accident social, une nécessité contingente. L’histoire s’efforce de liquider l’appropriation privative depuis qu’elle a cessé de répondre aux conditions qui l’ont fait naître, mais l’erreur, entretenue métaphysiquement, continue de profiter aux maîtres, à l’« éternelle » minorité dominante.\par

\astermono

\noindent La mésaventure du sacrifice se confond avec celle du mythe. La pensée bourgeoise en révèle la matérialité, le désacralise, l’émiette ; sans toutefois le liquider, car ce serait pour la bourgeoisie cesser d’exploiter, c’est-à-dire cesser d’être. Le spectacle parcellaire n’est qu’une phase de la décomposition du mythe ; une décomposition qu’accélère aujourd’hui la dictature du consommable. De même le vieux sacrifice-don lié aux forces cosmiques achève de se perdre dans un sacrifice-échange tarifié selon le barème de la Sécurité sociale et des lois démocratiques. Le sacrifice fanatise d’ailleurs de moins en moins, comme séduit de moins en moins le lamentable \emph{show} des idéologies. On ne remplace pas impunément le grand rut du salut éternel par de petites masturbations privées. On ne compense pas l’espoir insensé de l’au-delà par un calcul de promotion. Héros de la patrie, héros du travail, héros du frigidaire et de la pensée à tempérament… La gloire des potiches est fêlée.\par
Il n’empêche. La fin prochaine d’un mal ne me consolera jamais d’avoir à le subir dans l’immédiat. La vertu du sacrifice est partout prônée. Aux prêtres rouges s’unissent les bureaucrates œcuméniques. Vodka et lacrima-christi. Entre les dents, plus de couteau, la bave du Christ ! Sacrifiez-vous dans la joie, mes frères ! Pour la Cause, pour l’Ordre, pour la Révolution, pour le Parti, pour l’Union, pour le bœuf en daube !\par
Les vieux socialistes avaient eu ce mot célèbre :\par

\begin{quoteblock}
\noindent « On croit mourir pour la patrie, on meurt pour le capital. »\end{quoteblock}

\noindent Leurs héritiers sont maintenant fustigés de semblables formules :\par

\begin{quoteblock}
\noindent « On croit lutter pour le prolétariat, on meurt pour ses dirigeants », « on croit bâtir pour l’avenir, on entre avec l’acier dans un plan quinquennal. »\end{quoteblock}

\noindent Et, après avoir assené ces slogans, que font les jeunes turcs de la gauche en révolte ? Ils entrent au service d’une Cause ; la « meilleure » des Causes. Leur temps de créativité, ils le passent à distribuer des tracts, à coller des affiches, à manifester, à prendre à partie le président de l’assemblée régionale. Ils militent. Il faut bien agir, puisque les autres pensent pour eux. Le tiroir du sacrifice n’a pas de fond.\par
La meilleure des Causes est celle où l’on se perd le mieux corps et âme. Les lois de la mort ne sont que les lois niées de la volonté de vivre. La part de mort l’emporte sur la part de vie ; il n’y a pas d’équilibre, pas de compromis possible au niveau de la conscience. Il faut défendre tout l’un ou tout l’autre. Les frénétiques de l’Ordre absolu – Chouans, Nazis, Carlistes – ont su prouver avec une belle conséquence qu’ils étaient du parti de la mort. Du moins la ligne du \emph{Viva la muerte} ! est nette, sans bavure. Les réformistes de la mort à petite dose – les socialistes de l’ennui – n’ont même pas l’honneur absurde d’une esthétique de la destruction totale. Ils savent seulement modérer la passion de vivre, la racornir en sorte que, se tournant contre elle-même, elle devienne passion de détruire et de se détruire. Adversaires du camp d’extermination, ils le sont au nom de la mesure : au nom du pouvoir mesuré, au nom de la mort mesurée.\par
Les partisans du sacrifice absolu à l’État, à la Cause, au Führer, ces grands contempteurs de la vie, ont en commun, avec ceux qui opposent aux morales et aux techniques de renoncement leur rage de vivre, un sens antagoniste mais semblablement aigu de la fête. Il semble que la vie soit si spontanément une fête que, torturée par un monstrueux ascétisme, elle mette à se terminer d’un seul coup tout l’éclat qui lui fut dérobé. La fête que connaissent à l’instant de mourir les légions ascétiques, les mercenaires, les fanatiques, les flics du combat à outrance est une fête macabre, figée comme devant l’éternité d’un flash photographique, esthétisée. Les \emph{paras} dont parle Bigeard entrent dans la mort par l’esthétique, statufiés, madréporisés, conscients peut-être de leur ultime hystérie. L’esthétique est bien la fête sclérosée, privée de mouvement, séparée de la vie comme une tête de Jivaro ; la fête de la mort. La part d’esthétique, la part de la \emph{pose}, correspond d’ailleurs à la part de mort que secrète la vie quotidienne. Toute apocalypse est belle d’une beauté morte. O chanson des Gardes suisses, que Louis-Ferdinand Céline nous fit aimer.\par
La fin de la Commune n’est pas une apocalypse. Il y a, des Nazis rêvant d’entraîner le monde dans leur chute aux Communards livrant Paris aux flammes, la distance de la mort totale brutalement affirmée à la vie totale brutalement niée. Les premiers se bornent à déclencher le processus d’anéantissement logique mis en place par les humanistes qui enseignent la soumission et le renoncement. Les seconds savent qu’une vie passionnément construite ne peut plus se défaire ; qu’il y a plus de plaisir à la détruire tout entière qu’à la voir mutiler ; que mieux vaut disparaître dans un feu de joie vive que céder sur toute la ligne en cédant d’un pouce. Débarrassé de son emphase, le cri abusivement proféré par la stalinienne Ibarruri : « Plutôt mourir debout que vivre à genoux », me paraît prononcer souverainement pour un certain mode de suicide, pour une heureuse façon de prendre congé. Ce qui fut valable pour la Commune le reste pour un individu.\par
Contre le suicide de lassitude, contre un renoncement couronnant les autres. Un dernier éclat de rire, à la Cravan. Une dernière chanson, à la Ravachol.\par

\astermono

\noindent La révolution cesse dès l’instant où il faut se sacrifier pour elle. Se perdre et la fétichiser. Les moments révolutionnaires sont les fêtes où la vie individuelle célèbre son union avec la société régénérée. L’appel au sacrifice y sonne comme un glas. Vallès reste en deçà de son propos :\par

\begin{quoteblock}
\noindent « Si la vie des résignés ne dure pas plus que celle des rebelles, autant être rebelle au nom d’une idée. »\end{quoteblock}

\noindent Un militant n’est jamais révolutionnaire qu’à l’encontre des idées qu’il accepte de servir. Le Vallès combattant pour la Commune est d’abord cet enfant, puis ce bachelier qui rattrape en un long dimanche les éternelles semaines du passé. L’idéologie est la pierre sur la tombe de l’insurgé. Elle veut l’empêcher de ressusciter.\par
Quand l’insurgé commence à croire qu’il lutte pour un bien supérieur, le principe autoritaire cesse de vaciller. L’humanité n’a jamais manqué de raisons pour faire renoncer à l’humain. À tel point qu’il existe chez certains un véritable réflexe de soumission, une peur irraisonnée de la liberté, un masochisme partout présent dans la vie quotidienne. Avec quelle amère facilité on abandonne un désir, une passion, la part essentielle de soi. Avec quelle passivité, quelle inertie on accepte de vivre pour quelque chose, d’agir pour quelque chose, tandis que le mot « chose » l’emporte partout de son poids mort. Parce qu’il n’est pas facile d’être soi, on abdique allégrement ; au premier prétexte venu, l’amour des enfants, de la lecture, des artichauts. Le désir du remède s’efface sous la généralité abstraite du mal.\par
Pourtant, le réflexe de liberté sait, lui aussi, se frayer un chemin à travers les prétextes. Dans la grève pour l’augmentation de salaire, dans l’émeute, n’est-ce pas l’esprit de la fête qui s’éveille et prend consistance ? À l’heure où j’écris, des milliers de travailleurs débraient ou prennent les armes, obéissent à des consignes ou à un principe, et au fond c’est à changer l’emploi de leur vie qu’ils s’appliquent passionnément. Transformer le monde et réinventer la vie est le mot d’ordre effectif des mouvements insurrectionnels. La revendication qu’aucun théoricien ne crée puisqu’elle est seule à fonder la création poétique. La révolution se fait tous les jours contre les révolutionnaires spécialisés, une révolution sans nom, comme tout ce qui ressortit du vécu, préparant, dans la clandestinité quotidienne des gestes et des rêves, sa cohérence explosive.\par
Aucun problème ne vaut pour moi celui qui pose à longueur de journée la difficulté d’inventer une passion, d’accomplir un désir, de construire un rêve comme il s’en construit dans mon esprit, la nuit. Mes gestes inachevés me hantent et non pas l’avenir de la race humaine, ni l’état du monde en l’an 2000, ni le futur conditionnel, ni les ratons laveurs de l’abstrait. Si j’écris, ce n’est pas, comme on dit, « pour les autres », ni pour m’exorciser de leurs fantômes ! Je noue les mots bout à bout pour sortir du puits de l’isolement, d’où il faudra bien que les autres me tirent. J’écris par impatience et avec impatience. Pour vivre sans temps mort. Des autres, je ne veux rien savoir qui ne me concerne d’abord. Il faut qu’ils se sauvent de moi comme je me sauve d’eux. Notre projet est commun. Il est exclu que le projet de l’homme total se fonde jamais sur une réduction de l’individu. Il n’y a pas de castration plus ou moins valable. La violence apolitique des jeunes générations, leur mépris pour les rayons à prix unique de la culture, de l’art, de l’idéologie le confirment dans les faits : la réalisation individuelle sera l’œuvre du « chacun pour soi » compris collectivement. Et de façon radicale.\par
À ce stade de l’écriture où l’on cherchait jadis l’explication, je veux désormais que l’on trouve le règlement de compte.
\subsection[{2. Le refus du sacrifice}]{\textsc{2.} Le refus du sacrifice}
\noindent Le refus du sacrifice est le refus de la contrepartie. Il n’est rien dans l’univers des choses monnayables ou non qui puisse servir d’équivalence à l’être humain. L’individu est irréductible ; il change, mais ne s’échange pas. Or un simple coup d’œil sur les mouvements de réformation sociale suffit pour en convaincre : ils n’ont jamais revendiqué qu’un assainissement de l’échange et du sacrifice, mettant leur point d’honneur à humaniser l’inhumain et à le rendre séduisant. Chaque fois que l’esclave rend son esclavage supportable, il vole au secours du maître.\par
Voie vers le socialisme : plus les rapports sordides de la réification enchaînent les hommes, plus s’exacerbe la tentation humanitaire de mutiler \emph{égalitairement}. Alors que l’incessante dégradation de la vertu d’abnégation et de dévouement pousse au refus radical, il se trouve aujourd’hui quelques sociologues, ces policiers de la société moderne, pour chercher une parade dans l’exaltation d’une forme plus subtile de sacrifice : l’art.\par

\astermono

\noindent Les grandes religions avaient su transformer la misérable vie terrestre en une voluptueuse attente ; la vallée de larmes débouchait sur la vie éternelle en Dieu. L’art, selon sa conception bourgeoise, assume mieux que Dieu le privilège de conférer la gloire éternelle. À l’art-dans-la-vie-et-en-Dieu des régimes unitaires (la statuaire égyptienne, l’art nègre…) succède un art complémentaire de la vie, un art qui supplée à l’absence de Dieu (IV\textsuperscript{e} siècle grec, Horace, Ronsard, Malherbe, les Romantiques…) Les bâtisseurs de cathédrale se souciaient aussi peu que Sade de passer à la postérité. Ils assuraient leur salut en Dieu comme Sade en lui-même, non leur conservation dans les musées de l’histoire. Ils travaillaient pour un état suprême de l’être, non pour une durée d’années et de siècles admiratifs.\par
L’histoire est le paradis terrestre de la spiritualité bourgeoise. On y accède non par la marchandise, mais par une apparente gratuité, par le sacrifice de l’œuvre d’art, par ce qui échappe à la nécessité immédiate d’accroître le capital : œuvre de bienfaisance pour le philanthrope, œuvre d’héroïsme pour le patriote, œuvre de victoire pour le militaire, œuvre littéraire ou scientifique pour le poète ou le savant… Mais l’expression « faire œuvre d’art » est elle-même ambivalente. Elle comprend l’expérience vécue de l’artiste et l’abandon de cette expérience vécue pour une abstraction de la substance créatrice : la forme esthétique. Ainsi l’artiste sacrifie ce qu’il crée, au souvenir impérissable de son nom, à son entrée dans la gloire funèbre des musées. N’est-ce pas pourtant la volonté de faire œuvre durable qui l’empêche de créer le moment impérissable de la vie ?\par
En vérité, sauf dans l’académisme, l’artiste ne succombe pas intégralement à la récupération esthétique. Sacrifiant son vécu immédiat pour la belle apparence, l’artiste, et quiconque essaie de vivre est artiste, obéit aussi au désir d’accroître sa part de rêves dans le monde objectif des autres hommes. En ce sens, il assigne à la chose créée la mission d’achever sa propre réalisation individuelle dans la collectivité. La créativité est par essence révolutionnaire.\par
La fonction du spectacle idéologique, artistique, culturel, consiste à changer les loups de la spontanéité en bergers du savoir et de la beauté. Les anthologies sont pavées de textes d’agitation, les musées d’appels insurrectionnels ; l’histoire les conserve si bien dans le jus de leur durée qu’on en oublie de les voir ou de les entendre. Et c’est ici que la société de consommation agit soudain comme un dissolvant salutaire. L’art n’érige plus aujourd’hui que des cathédrales en plastique. Il n’y a plus d’esthétique qui, sous la dictature du consommable, ne disparaisse avant d’avoir connu ses œuvres maîtresses. L’immaturé est la loi du consommable. L’imperfection d’une voiture permet son renouvellement rapide. La seule condition d’un soudain éclat esthétique tient à la surenchère momentanée qu’une œuvre introduit dans le spectacle de la décomposition artistique. Bernard Buffet, Georges Mathieu, Alain Robbe-Grillet, Pop Art et Yé-Yé s’achètent les yeux fermés aux grands magasins du Printemps. Il serait aussi impensable de miser sur la pérennité d’une œuvre que sur les valeurs éternelles de la Standard Oil.\par
Quand les sociologues les plus évolués ont fini par comprendre comment l’objet d’art devenait une valeur marchande, par quel biais la fameuse créativité de l’artiste se pliait à des normes de rentabilité, il leur est apparu qu’il fallait en revenir à la source de l’art, à la vie quotidienne, non pour la changer, car telle n’est pas leur attribution, mais pour en faire la matière d’une esthétique nouvelle qui, réfractaire à l’empaquetage, échapperait donc au mécanisme de l’achat et de la vente. Comme s’il n’existait pas une façon de consommer sur place ! On connaît le résultat : socio-drames et \emph{happenings}, en prétendant organiser une participation immédiate des spectateurs, ne participent en fait que de l’esthétique du néant. Sur le mode du spectacle, seul le vide de la vie quotidienne est exprimable. En fait de consommable, qu’y a-t-il de mieux que l’esthétique du vide ? À mesure qu’elle s’accélère, la décomposition des valeurs ne devient-elle pas la seule forme de distraction possible ? Le gag consiste à transformer les spectateurs du vide culturel et idéologique en ses organisateurs ; à remplir l’inanité du spectacle par la participation obligatoire du spectateur, de l’agent passif par excellence. Le \emph{happening} et ses dérivés ont quelque chance de fournir à la société d’esclaves sans maîtres, que les cybernéticiens nous préparent, le spectacle sans spectateur qu’elle requiert. Pour les artistes, au sens strict du terme, la voie de la récupération absolue est toute tracée. Ils entreront avec les Lapassade et consorts dans la grande corporation des spécialistes. Le pouvoir saura les récompenser d’ainsi déployer leur talent pour habiller de couleurs neuves et séduisantes le vieux conditionnement à la passivité.\par
Vue dans la perspective du pouvoir, la vie quotidienne n’est qu’un tissu de renoncements et de médiocrité. Elle est vraiment le vide. Une esthétique de la vie quotidienne ferait de chacun les artistes organisateurs de ce vide. Le dernier sursaut de l’art officiel va s’efforcer de modeler sous une forme thérapeutique ce que Freud avait appelé par une simplification suspecte l’« instinct de mort », c’est-à-dire la soumission joyeuse au pouvoir. Partout où la volonté de vivre n’émane pas spontanément de la poésie individuelle, s’étend l’ombre du crapaud crucifié de Nazareth. Sauver l’artiste qui vit en chaque être humain ne se fera pas en régressant vers des formes artistiques dominées par l’esprit de sacrifice. Tout est à reprendre à la base.\par

\astermono

\noindent Les surréalistes, certains du moins, avaient compris que le seul dépassement valable de l’art était dans le vécu : une œuvre qu’aucune idéologie ne récupère dans la cohérence de son mensonge. On sait à quel abandon les a menés docilement leur complaisance envers le spectacle culturel. La décomposition contemporaine en matière de pensée et d’art offre, il est vrai, de moindres risques de récupération esthétique qu’au cours des années 1930. La conjoncture actuelle ne peut que renforcer l’agitation situationniste.\par
On a beaucoup épilogué, – précisément depuis les surréalistes, – sur la disparition de certains rapports idylliques comme l’amitié, l’amour, l’hospitalité. Qu’on ne s’y trompe pas : la nostalgie de vertus plus humaines dans le passé ne fait qu’obéir à la nécessité future d’aviver la notion de sacrifice, par trop contestée. Désormais il ne peut plus y avoir ni d’amitié, ni d’amour, ni d’hospitalité, ni de solidarité où il y a abnégation. Sous peine de renforcer la séduction de l’inhumain. Bretch l’exprime à la perfection dans l’anecdote suivante : comme exemple de la bonne manière de rendre service à des amis, M. K., pour le plus grand plaisir de ceux qui l’écoutaient racontait l’histoire suivante. Trois jeunes gens arrivèrent chez un vieil Arabe et lui dirent : « Notre père est mort. Il nous a laissé dix-sept chameaux et dans son testament il ordonne que l’aîné en ait la moitié, le cadet un tiers et le plus jeune un neuvième. Nous n’arrivons pas à nous mettre d’accord sur le partage. À toi de prendre la décision. » L’Arabe réfléchit et dit : « Je constate que, pour pouvoir partager, il vous manque un chameau. J’ai le mien, je n’ai que celui-là, mais il est à votre disposition. Prenez-le, faites le partage et ne me ramenez que ce qui restera. » Ils le remercièrent pour ce service d’ami, emmenèrent le chameau et partagèrent les dix-huit bêtes : l’aîné en reçut la moitié, ce qui fit neuf, le cadet un tiers, ce qui fit six, et le plus jeune un neuvième, ce qui fit deux. À leur étonnement lorsqu’il eurent écarté leurs chameaux il en restait un. Ils le rendirent à leur vieil ami, en renouvelant leurs remerciements. M. K. disait que cette manière de rendre un service d’ami était bonne, parce qu’elle ne demandait de sacrifice à personne. L’exemple vaut d’être étendu à l’ensemble de la vie quotidienne avec la force d’un principe indiscutable.\par
Il ne s’agit pas de choisir l’art du sacrifice contre le sacrifice de l’art, mais bien la fin du sacrifice comme art. La promotion d’un savoir-vivre, d’une construction de situations vécues est partout présente, partout dénaturée par les falsifications de l’humain.\par

\astermono

\noindent Le sacrifice du présent sera peut-être le stade ultime d’un rite qui a mutilé l’homme depuis ses origines. Chaque minute s’effrite en bribes de passé et de futur. Jamais, sauf dans la jouissance, nous ne sommes adonnés à ce que nous faisons. Ce que nous allons faire et ce que nous avons fait bâtit le présent sur fond d’éternel déplaisir. Dans l’histoire collective comme dans l’histoire individuelle, le culte du passé et le culte du futur sont également réactionnaires. Tout ce qui doit se construire se construit dans le présent. Une croyance populaire veut qu’un noyé revoie à l’instant de mourir tout le film de sa vie. Je tiens pour assuré qu’il existe d’intenses lueurs où la vie se condense et se refait. Avenir, passé, pions dociles de l’histoire ne couvrent que le sacrifice du présent. Ne rien échanger ni contre une chose, ni contre le passé, ni contre le futur. Vivre intensément, pour soi, dans le plaisir sans fin et la conscience que ce qui vaut \emph{radicalement} pour soi vaut pour tous. Et par-dessus tout cette loi :\par

\begin{quoteblock}
\noindent « Agis comme s’il ne devait jamais exister de futur. »\end{quoteblock}

\section[{XIII. La séparation}]{XIII. La séparation}\renewcommand{\leftmark}{XIII. La séparation}


\begin{argument}\noindent Base de l’organisation sociale, l’appropriation privative tient les hommes séparés d’eux-mêmes et des autres. Des paradis unitaires artificiels s’efforcent de dissimuler la séparation en récupérant avec plus ou moins de bonheur les rêveries d’unité prématurément brisées. En vain. – Du plaisir de créer au plaisir de détruire, il n’y a qu’une oscillation qui détruit le pouvoir.
\end{argument}

\noindent Les hommes vivent séparés les uns des autres, séparés de ce qu’ils sont dans les autres, séparés d’eux-mêmes. L’histoire des hommes est l’histoire d’une séparation fondamentale qui provoque et conditionne toutes les autres : la distinction sociale en maîtres et esclaves. Par l’histoire, les hommes s’efforcent de se rejoindre et d’atteindre à l’unité. La lutte de classes n’est qu’une phase, mais une phase décisive, de la lutte pour l’homme total.\par
De même que la classe dominante a les meilleures raisons du monde de nier la lutte des classes, de même l’histoire de la séparation ne peut manquer de se confondre avec l’histoire de sa dissimulation. Mais un tel enténèbrement procède moins d’une volonté délibérée que d’un long combat douteux où le désir d’unité se mue le plus souvent en son contraire. Ce qui ne supprime pas radicalement la séparation la renforce. En accédant au pouvoir, la bourgeoisie jette une lumière plus vive sur ce qui divise aussi essentiellement les hommes, elle fait prendre conscience du caractère social et de la matérialité de la séparation.\par

\astermono

\noindent Qu’est-ce que Dieu ? Le garant et la quintessence du mythe où se justifie la domination de l’homme par l’homme. La dégoûtante invention n’a pas d’autre excuse. À mesure que le mythe en se décomposant passe au stade de spectacle, le Grand Objet Extérieur, comme dit Lautréamont, s’émiette au vent de l’atomisation sociale, il dégénère en un Dieu à usage intime, sorte de badigeon pour maladies honteuses.\par
Au plus fort de la crise ouverte par la fin de la philosophie et du monde antiques, le génie du christianisme va subordonner la refonte d’un nouveau système mythique à un principe fondamental : le trinitarisme. Que signifie le dogme des trois personnes en Dieu, qui fera couler tant d’encre et tant de sang ?\par
Par l’âme, l’homme appartient à Dieu, par le corps au chef temporel, par l’esprit à lui-même ; son salut est dans l’âme, sa liberté dans l’esprit, sa vie terrestre dans le corps. L’âme enveloppe le corps et l’esprit, sans elle ils ne sont rien. N’est-ce pas, à y regarder de plus près, l’union du maître et de l’esclave dans le principe de l’homme envisagé comme créature divine ? L’esclave est le corps, la force du travail que le seigneur s’approprie ; le maître est l’esprit qui, gouvernant le corps, lui concède une parcelle de son essence supérieure. L’esclave se sacrifie donc par le corps à la puissance du maître cependant que le maître se sacrifie par l’esprit à la communauté de ses esclaves (le roi au service du peuple, de Gaulle au service de la France, le lavement des pieds dans l’Église…). Le premier offre sa vie terrestre, en échange il reçoit la conscience d’être libre, c’est-à-dire l’esprit du maître en lui descendu. La conscience mystifiée est la conscience du mythe. Le second offre idéalement son pouvoir de maître à l’ensemble de ceux qu’il dirige ; en noyant l’aliénation des corps dans l’aliénation plus subtile de l’esprit, il économise sur la dose de violence nécessaire au maintien de l’esclavage. Par son esprit, l’esclave s’identifie, ou du moins peut s’identifier, au maître auquel il livre sa force de vie ; mais à qui s’identifiera le maître ? Pas aux esclaves en tant que choses possédées, en tant que corps ; plutôt aux esclaves en tant qu’émanation de l’esprit du maître en soi, du maître suprême. Puisque le maître particulier se sacrifie sur le plan spirituel, il doit chercher dans la cohérence du mythe un répondant à son sacrifice, une idée en soi de maîtrise à laquelle il participe et se soumette. C’est pourquoi la classe contingente des maîtres a créé un Dieu devant lequel elle s’agenouille spirituellement pour s’identifier à lui. Dieu authentifie le sacrifice mythique du maître au bien public, et le sacrifice réel de l’esclave au pouvoir privé et privatif du maître. Dieu est le principe de toute soumission, la nuit qui légalise tous les crimes. Le seul crime illégal est le refus d’accepter un maître. Dieu est l’harmonie du mensonge ; une forme idéale où s’unissent le sacrifice volontaire de l’esclave (le Christ), le sacrifice consenti du maître (le Père ; l’esclave est le fils du maître) et leur lien indissoluble (le Saint-Esprit). L’homme idéal, créature divine, unitaire et mythique où l’humanité est invitée à se reconnaître réalise le même modèle trinitaire : un \emph{corps} soumis à l’\emph{esprit} qui le guide pour la plus grande gloire de l’\emph{âme}, la synthèse englobante.\par
Voici donc un type de relation où deux termes tirent leur sens d’un principe absolu, se mesurent à l’obscur, à la norme inaccessible, à l’indiscutable transcendance (Dieu, le sang, la sainteté, la grâce…). Pendant des siècles, d’innombrables dualités mijoteront, comme en un bon bouillon, au feu de l’unité mythique. Et tirant le bouillon du feu, la bourgeoisie ne gardera qu’une nostalgie de la chaleur unitaire et une série de froides abstractions sans saveur : corps et esprit, être et conscience, individu et collectivité, privé et public, général et particulier… Paradoxalement, la bourgeoisie, mue par ses intérêts de classe, détruit à son désavantage l’unitaire et sa structure tridimensionnelle. L’aspiration à l’unité si habilement satisfaite par la pensée mythique des régimes unitaires, loin de disparaître avec elle s’exacerbe, au contraire, à mesure que la matérialité de la séparation s’empare de la conscience. Dévoilant les fondements économico-sociaux de la séparation, la bourgeoisie fournit les armes qui doivent en assurer la fin. Mais la fin de la séparation implique la fin de la bourgeoisie et la fin de tout pouvoir hiérarchisé. C’est pourquoi toute classe ou caste dirigeante se trouve incapable d’opérer la reconversion de l’unité féodale en unité réelle, en participation sociale authentique. Seul le nouveau prolétariat a mission d’arracher aux dieux la troisième force, la création spontanée, la poésie, pour la garder vivante dans la vie quotidienne de tous. L’ère transitoire du pouvoir parcellaire n’aura été qu’une insomnie dans le sommeil, l’indispensable point zéro dans le renversement de perspective, le nécessaire appel du pied avant le bond du dépassement.\par

\astermono

\noindent L’histoire atteste la lutte menée contre le principe unitaire ; et comment transparaît la réalité dualiste. Initialement mené dans un langage théologique, qui est le langage officiel du mythe, l’affrontement s’exprime ensuite dans un langage idéologique, qui est celui du spectacle. Manichéens, cathares, hussites, calvinistes… rejoignent par leurs préoccupations, Jean de Meung, La Boétie, Vanino Vanini. Ne voit-on pas Descartes accrocher \emph{in extremis} à la glande pinéale une âme dont il ne sait que faire ? Tandis qu’au sommet d’un monde parfaitement intelligible son Dieu funambule garde un équilibre parfaitement incompréhensible, le Dieu de Pascal se dissimule, privant l’homme et le monde d’un support sans lequel ils sont réduits à se contester mutuellement, à n’être jugés que l’un par rapport à l’autre, à se peser au néant.\par
Dès la fin du XVIII\textsuperscript{e} siècle, la dissociation paraît partout sur la scène, l’émiettement s’accélère. L’ère des petits hommes concurrentiels est ouverte. Des morceaux d’êtres humains s’absolutisent : matière, esprit, conscience, action, universel, particulier… Quel Dieu ramasserait cette porcelaine ?\par
L’esprit de domination trouvait à se justifier dans une transcendance. On n’imagine pas un Dieu capitaliste. La domination suppose un système trinitaire. Or les rapports d’exploitation sont dualistes. De plus ils sont indissociables de la matérialité des rapports économiques. L’économique n’a pas de mystère ; du miracle il ne conserve que le \emph{hasard} du marché ou le parfait agencement programmatique des ordinatrices de \emph{plannings}. Le Dieu rationnel de Calvin séduit bien moins que le prêt à intérêt qu’il autorise impunément. Quant au Dieu des anabaptistes de Münster et des paysans révolutionnaires de 1525, il est déjà, sous une forme archaïque, l’élan irrépressible des masses vers une société de l’homme total.\par
Le chef mystique ne devient pas simplement le chef du travail. Le seigneur ne se transforme pas en patron. Supprimez la supériorité mystérieuse du sang et du lignage, il ne reste qu’un mécanisme d’exploitation, une course au profit qui n’a d’autre justification qu’elle-même. Une différence quantitative d’argent ou de pouvoir sépare le patron du travailleur, non plus la barrière qualitative de la race. C’est le caractère odieux de l’exploitation, qu’elle s’exerce entre « égaux ». La bourgeoisie justifie, – bien malgré elle, on s’en doute – toutes les révolutions. Quand les peuples cessent d’être abusés, ils cessent d’obéir.\par

\astermono

\noindent Le pouvoir parcellaire fragmente jusqu’à l’inconsistance les êtres sur lesquels il règne. Et simultanément se fragmente le mensonge unitaire. La mort de Dieu vulgarise la conscience de la séparation. Le désespoir romantique n’exprima-t-il pas le cri d’une déchirure douloureusement ressentie ? La fêlure est partout : dans l’amour, dans le regard, dans la nature, dans le rêve, dans la réalité… Le drame de la conscience, dont parle Hegel, est bien davantage la conscience du drame. Une telle conscience est révolutionnaire chez Marx. Quand Peter Schlemihl part à la recherche de son ombre pour oublier qu’il est, de fait, une ombre à la recherche de son corps, la démarche offre assurément moins de risques pour le pouvoir. Dans un réflexe d’autodéfense, la bourgeoisie « invente » des paradis unitaires artificiels en récupérant avec plus ou moins de bonheur les désenchantements et les rêves d’unité prématurément brisés.\par
À côté des masturbations collectives : idéologies, illusion d’être ensemble, éthique du troupeau, opium du peuple, il y a toute la gamme des produits marginaux, à la frontière du licite et de l’illicite : idéologie individuelle, obsession, monomanie, passion unique, donc aliénante, drogue et ses substituts (alcool, illusion de la vitesse et du changement rapide, sensation, rare…). ceci permet de se perdre totalement sous couvert de s’atteindre, c’est vrai, mais l’activité dissolvante procède surtout de l’usage parcellaire qui en est fait. La passion du jeu cesse d’être aliénante si celui qui s’y livre recherche le jeu dans la totalité de la vie : dans l’amour, dans la pensée, dans la construction des situations. De même le désir de tuer n’est plus une monomanie s’il s’allie à la conscience révolutionnaire.\par
Pour le pouvoir, le danger des palliatifs unitaires est donc double. D’une part, ils laissent insatisfaits, d’autre part, ils débouchent sur la volonté de construire une unité sociale réelle. L’élévation mystique vers l’unité n’avait d’autre fin que Dieu ; la progression horizontale, dans l’histoire, vers une problématique unité spectaculaire est un fini infini. Elle provoque une soif insatiable d’absolu, or le quantitatif est en lui-même une limite. La course folle ne peut précipiter ainsi que dans le qualitatif, soit par la voie négative, soit, si la prise de conscience s’établit, par la transformation de la négativité en positivité. Par la voie négative, certes, on ne s’atteint pas soi-même, on s’abîme dans sa propre dissolution. Le délire provoqué, la volupté du crime et de la cruauté, l’éclair convulsif de la perversité sont des chemins qui convient à se perdre sans réticence. On ne fait là qu’obéir avec un zèle étonnant à la gravitation du pouvoir qui disloque et détruit. Mais le pouvoir ne \emph{durerait} guère s’il ne freinait sa force de décomposition. Le général tue ses soldats jusqu’à un certain point seulement. Reste à savoir si le néant se distille au compte-gouttes. Le plaisir limité de se détruire risque fort de détruire en fin de compte le pouvoir qui le limite. On l’a bien vu dans les émeutes de Stockholm ou de Watts. Il suffit d’un coup de pouce pour que le plaisir devienne total, pour que la violence négative \emph{libère sa positivité}. Je tiens qu’il n’y a pas de plaisir qui ne cherche à s’assouvir totalement, dans tous les domaines, unitairement ; Huysmans n’a pas, j’imagine, l’humour d’y songer quand il écrit gravement d’un homme en érection qu’il « s’insurge ».\par
Le déchaînement du plaisir sans restriction est la voie la plus sûre vers la révolution de la vie quotidienne, vers la construction de l’homme total.
\section[{XIV. L’organisation de l’apparence}]{XIV. L’organisation de l’apparence}\renewcommand{\leftmark}{XIV. L’organisation de l’apparence}


\begin{argument}\noindent L’organisation de l’apparence est un système de protection des faits. Un \emph{racket}. Elle les présente dans la réalité médiate pour que la réalité immédiate ne les présente pas. Le mythe est l’organisation de l’apparence du pouvoir parcellaire. Contesté, la cohérence du mythe devient mythe de la cohérence. Accrue historiquement, l’incohérence du spectacle devient spectacle de l’incohérence : le \emph{Pop Art} est l’actuel pourrissement consommable et le pourrissement du consommable actuel (1). – La pauvreté du « drame » comme genre littéraire va de pair avec la reconquête de l’espace social par les attitudes théâtrales. Le théâtre s’appauvrit sur la scène et s’enrichit de la vie quotidienne, dont il s’efforce de dramatiser les conduites. – Les rôles sont les moules idéologiques du vécu. La mission de les parfaire appartient aux spécialistes (2).
\end{argument}

\subsection[{1. La décomposition du spectacle devient le spectacle de la décomposition}]{\textsc{1.} La décomposition du spectacle devient le spectacle de la décomposition}

\begin{quoteblock}
\noindent « On a, dit Nietzsche, imaginé par un mensonge le monde idéal, on a enlevé à la réalité sa valeur, sa signification, sa véracité. Le mensonge de l’idéal a été jusqu’à présent la malédiction suspendue au-dessus de la réalité. L’humanité elle-même, à force de se pénétrer de ce mensonge a été faussée et falsifiée jusque dans ses instincts les plus profonds, jusqu’à l’adoration des valeurs opposées à celles qui garantissaient le développement, le présent en devenir. »\end{quoteblock}

\noindent Qu’est-ce donc que le mensonge de l’idéal sinon la vérité des maîtres ? Quand le vol a besoin d’assises légales, quand l’autorité se couvre de l’intérêt général pour s’exercer impunément à des fins privées, comment voudrait-on que le mensonge ne fascine les esprits, ne les plie à ses lois jusqu’à faire de ce pli comme une disposition quasi naturelle de l’homme ? Et il est vrai que l’homme ment parce que dans un monde régi par le mensonge, il ne lui est pas possible d’agir autrement ; il est lui-même mensonge, lié par son propre mensonge. Le sens commun ne contresigne jamais que le décret promulgué au nom de tous contre la vérité. Il est une codification vulgarisée du mensonge.\par
Et cependant, personne ne reste grimaçant vingt-quatre heures par jour sous le poids de l’inauthentique. De même que chez les penseurs les plus radicaux le mensonge des mots porte en soi la lumière qui le fait transparaître, de même il est peu d’aliénations quotidiennes qui ne se brisent, l’espace d’une seconde ou d’une heure ou d’un rêve, sur leur désaveu subjectif. Personne n’est tout à fait dupe de ce qui le détruit pas plus que les mots n’obéissent tout à fait au pouvoir. Il s’agit seulement d’étendre les moments de vérité, les icebergs subjectifs qui couleront les \emph{Titanic} du mensonge.\par

\astermono

\noindent La vague de matérialité emporte au large les débris du mythe qu’elle a brisé. La bourgeoisie, qui en fut le mouvement et n’en est plus que l’écume, disparaît avec eux. Montrant par quel prévisible choc en retour le roi dicte au tueur à gages les ordres qui demain seront exécutoires sur sa propre personne, Shakespeare semble décrire anticipativement le sort promis à la classe déicide. La machine à tuer ne connaît plus ses maîtres dès l’instant où les assassins de l’ordre cessent d’obéir à la foi du mythe ou, si l’on veut, au Dieu qui légalise leurs crimes. Ainsi la révolution est-elle la plus belle invention de la bourgeoisie, le nœud coulant grâce auquel elle va se balancer dans le néant. On comprend que la pensée bourgeoise, tout entière suspendue à la corde radicale qu’elle a tressée, s’accroche avec l’énergie du désespoir à toutes les solutions réformistes, à tout ce qui peut prolonger sa durée, même si son poids l’entraîne irrésistiblement vers la dernière convulsion. Le fascisme est en quelque sorte le porte-parole de la chute irrémédiable, esthète rêvant de précipiter l’univers dans le gouffre, logicien de la mort d’une classe et sophiste de la mort universelle. Cette mise en scène de la mort choisie et refusée est aujourd’hui au centre du spectacle de l’incohérence.\par
L’organisation de l’apparence se veut, comme l’ombre de l’oiseau qui vole, immobile. Mais son immobilité, liée aux efforts de la classe dominante pour asseoir son pouvoir, n’est qu’un vain espoir d’échapper à l’histoire qui l’entraîne. Cependant, il existe entre le mythe et son état parcellaire et désacralisé, le spectacle, une différence notable dans leur résistance à la critique des faits. L’importance variable prise dans les civilisations unitaires par les artisans, les marchands, les banquiers, explique la permanence d’une oscillation entre la \emph{cohérence du mythe} et le \emph{mythe de la cohérence}. Tandis que le triomphe de la bourgeoisie, en introduisant l’histoire dans l’arsenal des apparences, rend l’apparence à l’histoire et donne un sens irréversible à l’évolution qui va de l’\emph{incohérence du spectacle} au \emph{spectacle de l’incohérence}.\par
Chaque fois que la classe commerçante, peu respectueuse des traditions, menace de désacraliser les valeurs, le mythe de la cohérence succède à la cohérence du mythe. Qu’est-ce à dire ? Ce qui, jusque-là, allait de soi a soudainement besoin d’être réaffirmé avec force, la foi spontanée le cède à la profession de foi, le respect des grands de ce monde s’affermit dans le principe de la monarchie autoritaire. Je souhaite que l’on étudie de plus près le paradoxe de ces interrègnes du mythe où l’on voit les éléments bourgeois sacraliser leur importance par une religion nouvelle, par l’anoblissement… dans le même temps que les nobles, d’un mouvement inverse, s’adonnent au grand jeu de l’impossible dépassement (la Fronde, mais aussi la dialectique héraclitéenne et Gilles de Rais). L’aristocratie a su tourner en mot d’esprit le mot de sa fin ; la bourgeoisie n’aura pour disparaître que la gravité de sa pensée. Pour les forces révolutionnaires du dépassement, n’y aurait-il pas plus à tirer de la légèreté de mourir que du poids de la survie ?\par
Sapé par la critique des faits, le mythe de la cohérence n’a pu fonder une nouvelle cohérence mythique. L’apparence, ce miroir où les hommes se dissimulent à eux-mêmes leurs propres décisions, s’émiette et tombe dans le domaine public de l’offre et de la demande individuelle. Sa disparition sera celle du pouvoir hiérarchisé, cette façade « derrière laquelle il n’y a rien ». La progression ne laisse pas de doute. Au lendemain de la grande révolution, les succédanés de Dieu font prime sur le marché du laissé pour compte, Être suprême et concordat bonapartiste ouvrent la série, suivis de près par le nationalisme, l’individualisme, le socialisme, le national-socialisme, les néo\emph{ismes}, sans compter les résidus individualisés de toutes les \emph{Weltanschauung} en solde et les milliers d’idéologies portatives offertes aujourd’hui comme prime à tout acheteur de TV., de culture, de poudre à lessiver. La décomposition du spectacle passe désormais par le spectacle de la décomposition. Il est dans la logique des choses que le dernier comédien filme sa propre mort. En l’occurrence, la logique des choses est celle du consommable, de ce qui se vend en se consumant. La pataphysique, le sous-dadaïsme, la mise en scène de la pauvreté quotidienne vont border la route qui conduit en hésitant vers les derniers cimetières.
\subsection[{2. Banalité du drame à sensation}]{\textsc{2.} Banalité du drame à sensation}
\noindent L’évolution du théâtre comme genre littéraire ne laisse pas d’éclairer l’organisation de l’apparence. Après tout, n’en est-il pas la forme la plus simple, la notice explicative ? Originellement confondu avec elle en des représentations sacrées révélant aux hommes le mystère de la transcendance, il élabore en se désacralisant le modèle de futures constructions de type spectaculaire. Hormis les machines de guerre, les machines anciennes trouvent leur origine dans le théâtre ; grues, poulies, mécanismes hydrauliques appartiennent au magasin des accessoires avant de bouleverser les rapports de production. Le fait vaut d’être signalé : si loin que l’on remonte, la domination de la terre et des hommes relève partout de techniques mises invariablement au service du travail et de l’illusion.\par
La naissance de la tragédie rétrécit déjà le champ où les hommes primitifs et les dieux s’affrontaient dans un dialogue cosmique. La participation magique est distancée, mise en suspens ; elle s’ordonne selon les lois de réfraction des rites initiaux, non plus selon ces rites eux-mêmes ; elle devient un \emph{spectaculum}, une chose vue, tandis que les dieux, relégués peu à peu parmi les décors inutiles semblent prévenir leur élimination graduelle de toute la scène sociale. Quand la désacralisation aura dissous les relations mythiques, le drame succédera à la tragédie. La comédie atteste bien de la transition ; son humour corrosif attaque avec l’énergie des forces nouvelles un genre désormais sénile. Le \emph{Dom Juan} de Molière, la parodie de Haendel dans \emph{L’Opéra des Gueux} de John Gay sont, à ce titre, éloquents.\par
Avec le drame, la société des hommes prend la place des dieux. Or, si le théâtre n’est au XIX\textsuperscript{e} siècle qu’un divertissement parmi d’autres, qu’on ne s’y trompe pas : en fait, débordant la scène traditionnelle, il reconquiert tout l’espace social. La banalité consistant à assimiler la vie à une comédie dramatique appartient à ce type d’évidence qui semble dispenser de l’analyse. De la confusion savamment entretenue entre le théâtre et la vie, il paraît bon de ne pas discuter ; comme s’il était naturel que cent fois par jour, je cesse d’être moi-même pour me glisser dans la peau de personnages dont je ne veux assumer ni les préoccupations, ni la signification. Certes, il peut m’arriver de me comporter librement en acteur, de tenir un rôle par jeu, par plaisir. Le rôle n’est pas là. L’acteur chargé de figurer un condamné à mort dans une pièce réaliste a toute latitude de rester lui-même – n’est-ce pas le paradoxe du bon comédien ? – mais s’il jouit d’une telle liberté, c’est évidemment que le cynisme de ses bourreaux ne l’atteint pas dans sa chair, frappe seulement l’image stéréotypée qu’il incarne à force de technique et de sens dramatique. Dans la vie quotidienne, les rôles imprègnent l’individu, ils le tiennent éloigné de ce qu’il est et de ce qu’il veut être authentiquement ; ils sont l’aliénation incrustée dans le vécu. Là, les jeux sont faits, c’est pourquoi ils ont cessé d’être des jeux. Les stéréotypes dictent à chacun en particulier, on pourrait presque dire « intimement », ce que les idéologies imposent collectivement.\par

\astermono

\noindent Un conditionnement parcellaire a remplacé l’ubiquité du conditionnement divin et le pouvoir s’efforce d’atteindre, par une grande quantité de petits conditionnements, à la qualité de l’ancien service d’Ordre. Cela signifie que la contrainte et le mensonge s’individualisent, cernent de plus près chaque être particulier pour mieux le transvaser dans une forme abstraite. Cela signifie aussi qu’en un sens, celui du gouvernement des hommes, le progrès des connaissances humaines perfectionne l’aliénation ; plus l’homme se connaît par la voie officielle, plus il s’aliène. La science est l’alibi de la police. Elle enseigne jusqu’à quel degré l’on peut torturer sans entraîner la mort, elle enseigne surtout jusqu’à quel point l’on peut devenir l’\emph{héautontimorouménos}, l’honorable bourreau de soi-même. Comment devenir une chose en gardant l’apparence humaine et au nom d’une certaine apparence humaine.\par
Le cinéma ou sa forme individualisée, la télévision, ne remporte pas ses plus belles victoires sur le terrain de la pensée. Il dirige bien peu l’opinion. Son influence s’exerce autrement. D’une scène de théâtre, un personnage frappe le spectateur par la ligne générale de son attitude et par la force de conviction de ce qu’il récite ; sur le grand ou le petit écran, le même personnage se décompose en une suite de détails précis qui agissent sur l’œil du spectateur comme autant de subtiles impressions. C’est une école du regard, une leçon d’art dramatique où une crispation du visage, un mouvement de la main traduisent pour des milliers de spectateurs la façon adéquate d’exprimer un sentiment, un désir… A travers la technique encore rudimentaire de l’image, l’individu apprend à modeler ses attitudes existentielles sur les portraits-robots que la psychosociologie moderne trace de lui. Il entre dans les schémas du pouvoir à la faveur même de ses tics et de ses manies. La misère de la vie quotidienne atteint son comble en se mettant en scène. De même que la passivité du consommateur est une passivité active, de même, la passivité du spectateur est sa fonction d’assimiler des rôles pour les tenir ensuite selon les normes officielles. Les images répétées, les stéréotypes offrent une série de modèles où chacun est invité à se tailler un rôle. Le spectacle est un musée des images, un magasin d’ombres chinoises. Il est aussi un théâtre d’essai. L’homme-consommateur se laisse conditionner par les stéréotypes (côté passif) sur lesquels il modèle ses différents comportements (côté actif). Dissimuler la passivité en renouvelant les formes de participation spectaculaire et la variété des stéréotypes, c’est à quoi s’emploient aujourd’hui les fabricants de \emph{happenings}, de \emph{Pop Art} et de socio-drames. Les machines de la société de production tendent à devenir à part entière les machines de la société de spectacle ; on peut exposer un cerveau électronique. On en revient à une conception originelle du théâtre, la participation générale des hommes au mystère de la divinité, mais à l’étage supérieur, avec l’appui de la technique. Et du même coup avec des chances de dépassement qui ne pouvaient exister dans la plus haute antiquité.\par
Les stéréotypes ne sont rien que les formes dégénérées des anciennes catégories éthiques (le chevalier, le saint, le pêcheur, le héros, le félon, le féal, l’honnête homme…). Les images qui agissaient au sein de l’apparence mythique par la force du qualitatif ne puisent leur rayonnement au sein de l’apparence spectaculaire que grâce à leur reproduction rapide et conditionnante (le slogan, la photo, la vedette, les mots…). J’ai montré plus haut que la production technique de relations magiques telles que la croyance ou l’identification dissolvait en fin de compte la magie. Ceci, ajouté à la fin des grandes idéologies, a précipité le chaos des stéréotypes et des rôles. D’où les conditions nouvelles imposées au spectacle.\par
Des évènements, nous ne possédons qu’un scénario vide. Leur forme nous atteint, non leur substance ; elle nous atteint avec plus ou moins de force, selon son caractère répétitif et selon la place qu’elle occupe dans la structure de l’apparence. Car en tant que système organisé, l’apparence est un gigantesque classeur où les évènements sont morcelés, isolés, étiquetés et rangés (affaires du cœur, domaine politique, secteur gastronomique…). Boulevard Saint-Germain, un jeune blouson noir assassine un passant. Qu’est-ce au juste que la nouvelle diffusée par la presse ? Un schéma préétabli chargé de susciter la pitié, l’indignation, le dégoût, l’envie ; un fait décomposé en ses parties abstraites, elles-mêmes distribuées selon les rubriques adéquates (la jeunesse, la délinquance, la violence, l’insécurité…). L’image, la photo, le style, construits et coordonnés selon des techniques combinatoires, constituent une sorte de distributeur automatique d’explications toutes faites et de sentiments contrôlés. Des individus réels réduits à des rôles servent d’appâts : l’étrangleur, le prince de Galles, Louison Bobet, Brigitte Bardot, Mauriac divorcent, font l’amour, pensent et se curent le nez pour des milliers de gens. La promotion du détail prosaïque spectaculairement signifié aboutit à la multiplication des rôles inconsistants. Le mari jaloux et meurtrier a sa place au côté du pape agonisant, la veste de Johnny Hallyday rejoint le soulier de Khrouchtchev, l’envers vaut l’endroit, le spectacle de l’incohérence est permanent. C’est qu’il existe une crise des structures. Les thèmes sont trop abondants, le spectacle est partout, dilué, inconsistant. La vieille relation si souvent employée, le manichéisme, tend à disparaître ; le spectacle est en deçà du bien et du mal. En 1930, les surréalistes saluant le geste d’un exhibitionniste s’illusionnaient sur la portée de leur éloge. Ils apportaient au spectacle de la morale le piment nécessaire à sa régénération. La presse à sensation n’agit pas autrement. Le scandale est une nécessité de l’information, au même titre que l’humour noir et le cynisme. Le vrai scandale est dans le refus du spectacle, dans son sabotage. Le pouvoir ne l’évitera qu’en renouvelant et en rajeunissant les structures de l’apparence. Ce pourrait bien être la fonction principale, en dernier ressort, des structuralistes. Mais on n’enrichit pas la pauvreté en la multipliant. Le spectacle se dégrade par la force des choses, ainsi s’effrite le poids qui entraîne à la passivité ; les rôles par la force de résistance du vécu, ainsi la spontanéité crève l’abcès de l’inauthentique et de la fausse activité.
\section[{XV. Le rôle}]{XV. Le rôle}\renewcommand{\leftmark}{XV. Le rôle}


\begin{argument}\noindent Les stéréotypes sont les images dominantes d’une époque, les images du spectacle dominant. Le stéréotype est le modèle du rôle, le rôle est un comportement modèle. La répétition d’une attitude crée le rôle, la répétition d’un rôle crée le stéréotype. Le stéréotype est une forme objective dans laquelle le rôle est chargé d’introduire. L’habileté à tenir et à traiter les rôles détermine la place occupée dans le spectacle hiérarchique. La décomposition spectaculaire multiplie les stéréotypes et les rôles, mais ceux-ci tombent dans le dérisoire et frôlent de trop près leur négation, le geste spontané (1, 2) – L’identification est le mode d’entrée dans le rôle. La nécessité de s’identifier importe plus pour la tranquillité du pouvoir que les choix des modèles d’identification. – L’identification est un état maladif, mais seuls les accidents d’identification tombent dans la catégorie officielle appelée « maladie mentale ». – Le rôle a pour fonction de vampiriser la volonté de vivre (3). – Le rôle représente le vécu en le transformant en chose, il console de la vie qu’il appauvrit. Il devient aussi un plaisir substitutif et névrotique. – Il importe de se détacher des rôles et les rendre au ludique (4). – La réussite du rôle assure la promotion spectaculaire, le passage d’une catégorie à une catégorie supérieure ; c’est l’initiation, concrétisée notamment par le culte du nom et de la photographie. Les spécialistes sont les initiés maîtres de l’initiation. La somme de leurs inconséquences définit la conséquence du pouvoir qui détruit en se détruisant (5). – La décomposition du spectacle rend les rôles interchangeables. La multiplication des faux changements crée les conditions d’un changement unique et réel, les conditions d’un changement radical. Le poids de l’inauthentique suscite une réaction violente et quasi biologique du vouloir-vivre.
\end{argument}

\subsection[{1. La représentation}]{\textsc{1.} La représentation}
\noindent Nos efforts, nos ennuis, nos échecs, l’absurdité de nos actes proviennent la plupart du temps de l’impérieuse nécessité où nous sommes de figurer des personnages hybrides, hostiles à nos vrais désirs sous couvert de les satisfaire.\par

\begin{quoteblock}
\noindent « Nous voulons vivre, dit Pascal, dans l’idée des autres, dans une vie imaginaire et nous nous efforçons pour cela de paraître. Nous travaillons à embellir et à conserver cet être imaginaire et nous négligeons le véritable. »\end{quoteblock}

\noindent Originale au XVII\textsuperscript{e} siècle, en un temps où le paraître se porte bien, où la crise de l’apparence organisée affleure à la seule conscience des plus lucides, la remarque de Pascal relève aujourd’hui, à l’heure où les valeurs se décomposent, de la banalité, de l’évidence pour tous. Par quelle magie attribuons-nous à des formes sans vie la vivacité de passions humaines ? Comment succombons-nous à la séduction d’attitudes empruntées ? Qu’est-ce que le rôle ?\par
Ce qui pousse l’homme à rechercher le pouvoir, est-ce rien d’autre que la faiblesse à laquelle ce pouvoir le réduit ? Le tyran s’irrite des devoirs que la soumission même de son peuple lui impose. La consécration divine de son autorité sur les hommes, il la paie d’un perpétuel sacrifice mythique, d’une humiliation permanente devant Dieu. Quittant le service de Dieu, il quitte dans le même mouvement le service d’un peuple aussitôt dispensé de le servir. Le \emph{vox populi, vox Dei} doit s’interpréter : « Ce que Dieu veut, le peuple le veut. » L’esclave s’irriterait bientôt d’une soumission que ne compenserait en échange une bribe d’autorité. De fait, toute soumission donne droit à quelque pouvoir et il n’y a de pouvoir qu’au prix d’une soumission ; c’est pourquoi certains acceptent si facilement d’être gouvernés. Le pouvoir s’exerce partout partiellement, à tous les niveaux de la cascade hiérarchique. C’est là sa contestable ubiquité.\par
Le rôle est une consommation de pouvoir. Il situe dans la \emph{représentation} hiérarchique, dans le spectacle donc ; en haut, en bas, au milieu, jamais en deçà ni au-delà. En tant que tel, il introduit dans le mécanisme culturel : il est \emph{initiation}. Le rôle est aussi la monnaie d’échange du sacrifice individuel ; en tant que tel, il exerce une fonction \emph{compensatoire}. Résidu de la séparation, il s’efforce enfin de créer une unité comportementale ; en tant que tel, il fait appel à l’identification.
\subsection[{2. La pose}]{\textsc{2.} La pose}
\noindent L’expression « jouer un rôle dans la société » montre assez par son premier usage restrictif que le rôle fut une distinction réservée à un certain nombre d’élus. L’esclave romain, le serf du Moyen Âge, le journalier agricole, le prolétaire abruti par treize heures de travail quotidien, ceux-là ne tiennent pas des rôles, ou ils les tiennent à un degré si rudimentaire que les gens policés voient dans ces êtres plus des animaux que des hommes. Il existe en effet une misère d’être en deçà de la misère du spectacle. Dès le XIX\textsuperscript{e} siècle, la notion du bon et de mauvais ouvrier se vulgarise comme la notion de maître-esclave s’était répandue dans le mythe avec le Christ. Elle se vulgarise avec moins de moyens et moins d’importance, encore que Marx ait jugé bon de la railler. Ainsi, le rôle, comme le sacrifice mythique, se démocratise. L’inauthentique à la portée de tous ou le triomphe du socialisme.\par
Voici un homme de trente-cinq ans. Chaque matin, il prend sa voiture, entre au bureau, classe des fiches, déjeune en ville, joue au poker, reclasse des fiches, quitte le travail, boit deux Ricard, rentre chez lui, retrouve sa femme, embrasse ses enfants, mange un steak sur un fond de TV, se couche, fait l’amour, s’endort. Qui réduit la vie d’un homme à cette pitoyable suite de clichés ? Un journaliste, un policier, un enquêteur, un romancier populiste ? Pas le moins du monde. C’est lui-même, c’est l’homme dont je parle qui s’efforce de décomposer sa journée en une suite de poses choisies plus ou moins inconsciemment parmi la gamme des stéréotypes dominants. Entraîné à corps et conscience perdus dans une séduction d’images successives, il se détourne du plaisir authentique pour gagner, par une ascèse passionnellement injustifiable, une joie frelatée, trop démonstrative pour n’être pas de façade. Les rôles assumés l’un après l’autre lui procurent un chatouillement de satisfaction quand il réussit à les modeler fidèlement sur les stéréotypes. La satisfaction du rôle bien rempli, il la tire de sa véhémence à s’éloigner de soi, à se nier, à se sacrifier.\par
Omnipotence du masochisme ! Comme d’autres étaient comte de Sandomir, palatin de Smirnov, margrave de Thorn, duc de Courlande, il charge d’une majesté toute personnelle ses façons d’automobiliste, d’employé, de chef, de subordonné, de collègue, de client, de séducteur, d’ami, de philatéliste, d’époux, de père de famille, de téléspectateur, de citoyen… Et pourtant il n’est pas cette mécanique imbécile, ce pantin amorphe. En de brefs instants, sa vie quotidienne libère une énergie qui, si elle n’était pas récupérée, dispersée, gaspillée dans les rôles, suffirait à bouleverser l’univers de la survie. Qui dira la force de frappe d’une rêverie passionnée, du plaisir d’aimer, d’un désir naissant, d’un élan de sympathie ? Ces moments de vie authentique, chacun cherche spontanément à les accroître afin qu’ils gagnent l’intégralité de la quotidienneté, mais le conditionnement réduit la plupart des hommes à les poursuivre à revers, par le biais de l’inhumain ; à les perdre à jamais à l’instant de les atteindre.\par

\astermono

\noindent Il existe une vie et une mort des stéréotypes. Telle image séduit, sert de modèle à des milliers de rôles individuels puis s’effrite et disparaît selon la loi du consommable, renouvellement et caractère périssable. Où la société du spectacle puise-t-elle ses nouveaux stéréotypes ? Dans la part de créativité qui empêche certains rôles de se conformer au stéréotype vieillissant (de même que le langage se renouvelle au contact de formes populaires), dans la part de jeu qui transforme les rôles.\par
Dans la mesure où le rôle se conforme à un stéréotype, il tend à se figer, à prendre le caractère statique de son modèle. Il n’a ni présent, ni passé, ni futur parce qu’il est un temps de \emph{pose} et, pour ainsi dire, une pause du temps. Du temps comprimé dans l’espace-temps dissocié, qui est l’espace-temps du pouvoir (toujours selon cette logique que la force du pouvoir réside dans sa force conjointe de séparer réellement et d’unir faussement). On est fondé de le comparer à l’image cinématographique, ou mieux à un de ses éléments, à une de ces attitudes prédéterminées qui, reproduites rapidement et un grand nombre de fois avec des variations minimes, donnent un plan. La reproduction est ici assurée par les rythmes de publicité et d’information, par la faculté de faire parler du rôle ; et par conséquent sa possibilité de s’ériger un jour en stéréotype (le cas Bardot, Sagan, Buffet, James Dean…). Mais, quelque poids qu’il atteigne dans la balance des opinions dominantes, le rôle a surtout pour mission d’adapter aux normes de l’organisation sociale, d’intégrer au monde paisible des choses. C’est pourquoi l’on voit les caméras de la renommée s’embusquer partout, s’emparer d’existences banales, faire du cœur une affaire de courrier et des poils superflus une question de beauté. Habillant un amant délaissé en Tristan au rabais, un vieillard délabré en symbole du passé et une ménagère en bonne fée du foyer, le spectacle greffé sur la vie quotidienne a de longue date devancé le \emph{Pop Art}. Il était prévisible que d’aucuns prendraient modèles sur les collages – à tous les coups rémunérateurs – de sourires conjugaux, d’enfants éclopés et de génies bricoleurs. Il n’en reste pas moins que le spectacle atteint là l’étage critique, le dernier avant la présence effective du quotidien. Les rôles frôlent de trop près leur négation. Le raté tient son rôle médiocrement, l’inadapté le refuse. À mesure que l’organisation spectaculaire s’effrite, elle englobe les secteurs défavorisés, elle se nourrit de ses propres résidus. Chanteurs aphones, artistes minables, lauréats malheureux, vedettes insipides traversent périodiquement le ciel de l’information avec une fréquence qui détermine leur place dans la hiérarchie.\par
Restent les irrécupérables, ceux qui refusent les rôles, ceux qui élaborent la théorie et la pratique de ce refus. C’est sans conteste de l’inadaptation à la société du spectacle que viendra une nouvelle poésie du vécu, une réinvention de la vie. Vivre intensément est-ce autre chose que détourner le cours du temps, perdu dans l’apparence ? Et la vie n’est-elle pas dans ses moments les plus heureux un présent dilaté qui refuse le temps accéléré du pouvoir, le temps qui s’écoule en ruisseaux d’années vides, le temps du vieillissement ?
\subsection[{3. L’identification}]{\textsc{3.} L’identification}
\noindent On connaît le principe du test de Szoudi. Invité à choisir, parmi quarante-huit photos de malades en état de crise paroxystique, les visages qui lui inspirent sympathie ou aversion, le patient accorde immanquablement sa préférence aux individus présentant une pulsion qu’il accepte tandis qu’il rejette les porteurs de pulsions qu’il refoule. Il se définit par identification positives et négatives. Du choix opéré, le psychiatre tire un profil pulsionnel dont il se sert pour élargir son patient ou le diriger vers le crématoire climatisé des asiles.\par
Que l’on considère maintenant les impératifs de la société de consommation, une société où l’être de l’homme est de consommer ; consommer du Coca-Cola, de la littérature, des idées, des sentiments, de l’architecture, de la TV, du pouvoir. Les produits de consommation, les idéologies, les stéréotypes sont les photos d’un formidable test de Szondi auquel chacun de nous est instamment convié de prendre part, non par un simple choix mais par un engagement, par une activité pratique. La nécessité d’écouler objets, idées, comportements modèles, implique un centre de décryptage où une sorte de profil pulsionnel des consommateurs servirait à rectifier les choix et à créer de nouveaux besoins mieux adaptés aux biens consommables. On peut considérer que les études de marché, la technique des motivations, les sondages d’opinions, les enquêtes sociologiques, le structuralisme entrent anarchiquement et avec bien des faiblesses dans un tel projet. Coordination et rationalisation font défaut ? Les cybernéticiens arrangeront cela, si nous leur prêtons vie.\par
À première vue, le choix de l’« image consommable » semble primordial. La ménagère-qui-lave-son-linge-avec-Omo diffère, et c’est une question de chiffre d’affaires, de la ménagère-qui-lave-son-linge-avec-Sunil. De même l’électeur démocrate diffère de l’électeur républicain, le communiste du chrétien. Mais la frontière est de moins en moins perceptible. Le spectacle de l’incohérence en vient à valoriser le degré zéro des valeurs. Si bien que l’identification à n’importe quoi l’emporte peu à peu, comme la nécessité de consommer n’importe quoi, sur l’importance d’être constant dans le choix d’une voiture, d’une idole ou d’un homme politique. L’essentiel, après tout, n’est-il pas de rendre l’homme étranger à ses propres désirs et de le loger dans le spectacle, en zone contrôlée ? Bon ou mauvais, honnête ou criminel, de gauche ou de droite, peu importe la forme pourvu que l’on s’y perde. A Khrouchtchev son Evtoutchenko, et les \emph{hooligans} seront bien gardés. La troisième force seule n’a rien à quoi s’identifier, ni opposant, ni chef prétendument révolutionnaire. Elle est la force d’identité, celle où chacun se reconnaît et se trouve. Là, personne ne décide pour moi ni en mon nom, là, ma liberté est celle de tous.\par

\astermono

\noindent La maladie mentale n’existe pas. Elle est une catégorie commode pour ranger et tenir à l’écart les accidents d’identification. Ceux que le pouvoir ne peut ni gouverner, ni tuer, il les taxe de folie. On y trouve les extrémistes et les monomaniaques du rôle. On y trouve aussi ceux qui se moquent du rôle ou le refusent. Leur isolement est le critère qui les condamne. Qu’un général s’identifie à la France avec la caution de millions d’électeurs et il se trouve une opposition pour lui contester sérieusement d’y prétendre. Ne voit-on pas avec le même succès Hörbiger inventer une physique nazie ; le général Walker et Barry Goldwater opposer l’homme supérieur, blanc, divin et capitaliste, et l’homme inférieur, noir, démoniaque et communiste ; Franco se recueillir et demander à Dieu la sagesse d’opprimer l’Espagne, et partout dans le monde les dirigeants prouver par un délire à froid que l’homme est une machine à gouverner ? L’identification fait la folie, et non point l’isolement.\par
Le rôle est cette caricature de soi que l’on mène en tous lieux, et qui en tous lieux introduit dans l’absence. Mais l’absence est ordonnée, habillée, fleurie. Paranoïaques, schizophrènes, tueurs sadiques dont le rôle n’est pas reconnu d’utilité publique (n’est pas distribué sous le label du pouvoir comme l’est celui de flic, de chef, de militaire) trouvent leur utilité dans des endroits spéciaux, asiles, prisons, sorte de musées dont le gouvernement tire un double profit, en y éliminant de dangereux concurrents et en enrichissant le spectacle de stéréotypes négatifs. Les mauvais exemples et leur punition exemplaire donnent du piquant au spectacle et le protègent. Il suffit simplement d’encourager l’identification en accentuant l’isolement pour détruire la fausse distinction entre l’aliénation mentale et l’aliénation sociale.\par
À l’autre pôle de l’identification absolue, il existe une façon de mettre entre le rôle et soi une distance, une zone ludique qui est un véritable nid d’attitudes rebelles à l’ordre spectaculaire. On ne se perd jamais tout à fait dans un rôle. Même inversée, la volonté de vivre garde un potentiel de violence toujours près de dévier des chemins qu’on lui trace. Le larbin fidèle qui s’identifie au maître peut aussi l’égorger en temps opportun. Il arrive un instant où son privilège de mordre comme un chien excite son désir de frapper comme un homme. Diderot l’a fort bien montré dans \emph{Le Neveu de Rameau}, et les sœurs Papin mieux encore. C’est que l’identification prend, comme toute inhumanité, sa source dans l’humain. La vie authentique se nourrit de désirs authentiques ressentis. Et l’identification par le rôle fait coup double : elle récupère le jeu des métamorphoses, le plaisir de se masquer et d’être partout sous toutes les formes du monde ; elle fait sienne la vieille passion labyrinthique de se perdre pour mieux se retrouver, le jeu de dérive et de métamorphoses. Elle récupère aussi le réflexe d’identité, la volonté de trouver chez les autres hommes la part la plus riche et la plus authentique de soi. Le jeu cesse alors d’être un jeu, se fige, perd le choix de ses propres règles. La recherche de l’identité devient l’identification.\par
Mais renversons la perspective. Un psychiatre a pu écrire :\par

\begin{quoteblock}
\noindent « La reconnaissance par la société amène l’individu à dépenser ses pulsions sexuelles dans un but culturel, qui est le meilleur moyen de se défendre contre elles. »\end{quoteblock}

\noindent En clair, cela signifie qu’on assigne au rôle la mission d’absorber l’énergie vitale, de réduire la force érotique en l’usant par une sublimation permanente. Moins il y a de réalité érotique, plus il y a de formes sexualisées dans le spectacle. Le rôle – Wilhelm Reich dirait « la carapace » – garantit l’impuissance de jouir. Contradictoirement, le plaisir, la joie de vivre, la jouissance effrénée brisent la carapace, brisent le rôle. Si l’individu voulait considérer le monde non plus dans la perspective du pouvoir mais dans une perspective dont il soit le point de départ, il aurait tôt fait de déceler les actes qui le libèrent vraiment, les moments les plus authentiquement vécus, qui sont comme des trous de lumière dans la grisaille des rôles. Observer les rôles à la lumière du vécu authentique, les radiographier si l’on veut, permettrait d’en détourner l’énergie qui s’y est investie, de sortir la vérité du mensonge. Travail à la fois individuel et collectif. Également aliénants, les rôles n’offrent pas pour autant la même résistance. On se sauve plus aisément d’un rôle de séducteur qu’un d’un rôle de flic, de dirigeant, de prêtre. C’est ce qu’il convient pour chacun d’étudier de très près.
\subsection[{4. La compensation}]{\textsc{4.} La compensation}
\noindent Pourquoi les hommes accordent-ils aux rôles un prix parfois supérieur au prix qu’ils accordent à leur propre vie ? En vérité parce que leur vie n’a pas de prix, l’expression signifiant ici dans son ambiguïté que la vie est au-delà de toute estimation publique, de tout étalonnage ; et aussi qu’une telle richesse est, au regard du spectacle et de ses critères, une pauvreté insoutenable. Pour la société de consommation, la pauvreté est ce qui échappe au consommable. Réduire l’homme au consommateur passe donc pour un enrichissement, du point de vue spectaculaire. Plus on a de choses et de rôles, plus on est ; ainsi en décide l’organisation de l’apparence. Mais du point de vue de la réalité vécue, ce qui se gagne en degré de pouvoir se perd d’autant dans la réalisation authentique. Ce qui se gagne en paraître se perd en être et en devoir-être.\par
Ainsi le vécu offre-t-il toujours la matière première du contrat social, il paie le droit d’entrée. C’est lui qu’on sacrifie tandis que la compensation réside en brillants agencements de l’apparence. Et plus la vie quotidienne est pauvre, plus s’exacerbe l’attrait de l’inauthentique. Et plus l’illusion l’emporte, plus la vie quotidienne s’appauvrit. Délogée de l’essentiel à force d’interdits, de contraintes et de mensonges, la réalité vécue paraît si peu digne d’intérêt que les chemins de l’apparence accaparent tous les soins. On vit son rôle mieux que sa propre vie. La compensation donne, dans l’\emph{état des choses}, le privilège de peser davantage. Le rôle supplée à un manque : tantôt l’insuffisance de vie, tantôt à l’insuffisance d’un autre rôle. Tel ouvrier dissimule son éreintement sous le titre d’OS. 2 et la pauvreté même de ce rôle sous l’apparence incomparablement supérieure d’un propriétaire de 403. Mais chaque rôle se paie en mutilations (surcroîts de travail, aliénation du confort, survie). Chaque rôle remplit comme une étoupe inconsistante le vide laissé par l’expulsion du moi et de la vraie vie. Enlève-t-on brutalement l’étoupe, il reste une plaie béante. Le rôle était simultanément menace et protection. Mais la menace est seulement ressentie dans le négatif, elle n’existe pas officiellement. Officiellement, il y a menace quand le rôle risque d’être perdu ou dévalorisé, quand on perd l’honneur ou la dignité, quand, selon l’expression si joliment précise, on perd la \emph{face}. Et cette ambiguïté du rôle explique à mon sens pourquoi les gens s’y accrochent, pourquoi il colle à la peau, pourquoi on y engage sa vie : appauvrissant l’expérience vécue, il la protège contre la révélation de son insupportable misère. Un individu isolé ne survit pas à une révélation aussi brutale. Et le rôle participe de l’isolement organisé, de la séparation de la fausse unité. La compensation, comme l’alcool, fournit le \textbf{doping} nécessaire à la réalisation du pouvoir-être inauthentique. Il existe une ivresse de l’identification.\par
La survie et ses illusions protectrices forment un tout indissoluble. Les rôles s’éteignent évidemment quand disparaît la survie, bien que certains morts puissent lier leur nom à un stéréotype. La survie sans les rôles est une mort civile. De même que nous sommes condamnés à la survie, nous sommes condamnés à faire « bonne figure » dans l’inauthentique. L’armure empêche la liberté des gestes et amortit les chocs. Sous la carapace tout est vulnérable. Reste donc la solution ludique du « faire comme si » ; ruser avec les rôles.\par
Il convient d’adopter la suggestion de Rosanov :\par

\begin{quoteblock}
\noindent « Extérieurement, je suis déclinable. Subjectivement, je suis absolument indéclinable. Je ne m’accorde pas. Un adverbe en quelque sorte. »\end{quoteblock}

\noindent En dernier ressort, c’est le monde qui doit se modeler sur le subjectif ; s’accorder avec moi afin que je m’accorde avec lui. Rejeter les rôles comme un paquet d’habits sales reviendrait à nier la séparation et à verser dans le mystique ou le solipsisme. Je suis chez l’ennemi et l’ennemi est chez moi. Il ne faut pas qu’il me tue, c’est pourquoi je m’abrite sous la carapace des rôles. Et je travaille, et je consomme, et je sais me montrer poli, et je ne fais pas d’outrages aux mœurs. Mais il faut cependant détruire un monde aussi factice, c’est pourquoi les gens avisés laissent jouer les rôles entre eux. Passer pour un irresponsable, voilà la meilleure façon d’être responsable pour soi. Tous les métiers sont sales, faisons-les salement, tous les rôles sont mensonges, laissons-les se démentir ! J’aime la superbe de Jacques Vaché écrivant :\par

\begin{quoteblock}
\noindent « Je promène de ruines en village mon monocle de Crystal et une théorie de peinture inquiétante. J’ai successivement été un littérateur couronné, un dessinateur pornographe connu et un peintre cubiste scandaleux. Maintenant, je reste chez moi et laisse aux autres le soin d’expliquer et de discuter ma personnalité d’après celles indiquées. »\end{quoteblock}

\noindent Il me suffit d’être totalement vrai avec ceux de mon bord, avec les défenseurs de la vie authentique.\par
Plus on se détache du rôle, mieux on le manipule contre l’adversaire. Mieux on se garde du poids des choses, plus on conquiert la légèreté du mouvement. Les amis ne s’encombrent guère de \emph{formes}, ils polémiquent à découvert, sachant qu’ils ne peuvent se blesser. Où la communication se veut réelle, le malentendu n’est pas un crime. Mais si tu m’abordes armé de pied en cap, m’imposant le combat pour chercher un accord en mode de victoire, tu ne trouveras de moi qu’une \emph{pose} évasive, un silence habillé pour te signifier la fin du dialogue. La contention des rôles ôte de prime abord tout intérêt à la discussion, seul l’ennemi recherche la rencontre sur le terrain des rôles, dans la lice du spectacle. Tenir en respect ses fantômes, à longueur de journée, n’est-ce pas suffisant sans que de prétendues amitiés n’y contraignent de surcroît ? Encore, si mordre et aboyer pouvaient donner conscience de la chiennerie des rôles, éveiller soudainement à l’importance de soi…\par
Fort heureusement, le spectacle de l’incohérence introduit forcément du \emph{jeu} dans les rôles. La morale de « l’envers vaut l’endroit » dissout l’esprit de sérieux. L’attitude ludique laisse flotter les rôles dans son indifférence. C’est pourquoi la réorganisation de l’apparence s’efforce, avec si peu de bonheur, d’accroître la part du jeu (concours \emph{Intervilles, Quitte ou Double}…), de mettre la désinvolture au service du consommable. La distanciation s’affirme avec la décomposition du paraître. Certains rôles sont douteux, ambigus ; ils contiennent leur propre critique. Rien ne peut empêcher désormais la reconversion du spectacle en un jeu collectif dont la vie quotidienne va créer par ses moyens de bord les conditions d’expansion permanente.
\subsection[{5. L’initiation}]{\textsc{5.} L’initiation}
\noindent En protégeant la misère de la survie et en protestant contre elle, le mouvement de compensation distribue à chaque être un certain nombre de possibilités formelles de participer au spectacle, sortes de laissez-passer qui autorisent la représentation scénique d’une ou plusieurs tranches de vie, publique ou privée, peu importe. De même que Dieu conférait la grâce à tous les hommes, laissant à chacun la liberté du salut ou de la damnation, l’organisation sociale donne à chacun le droit de réussir ou de rater son entrée dans les cercles du monde. Mais tandis que Dieu aliénait globalement la subjectivité, la bourgeoisie l’émiette dans un ensemble d’aliénations partielles. En un sens, la subjectivité, qui n’était rien, devient quelque chose, elle a sa vérité, son mystère, ses passions, sa raison, ses droits. Sa reconnaissance officielle passe par sa division en éléments étalonnés et homologués selon les normes du pouvoir. Le subjectif entre dans ces formes objectives que sont les stéréotypes par le moyen de l’identification. Il y entre en miettes, en fragments absolutisés, décortiqué de façon ridicule (le traitement grotesque du moi chez les romantiques, et son contrepoison, l’humour).\par
Être, c’est posséder des représentations du pouvoir. Pour être quelqu’un, l’individu doit, comme on dit, faire la part des choses, entretenir ses rôles, les polir, les remettre sur le métier, s’initier progressivement jusqu’à mériter la promotion spectaculaire. Les usines scolaires, la publicité, le conditionnement de tout Ordre aide avec sollicitude l’enfant, l’adolescent, l’adulte à gagner leur place dans la grande famille des consommateurs.\par
Il existe des paliers d’initiation. Tous les groupes socialement reconnus ne disposent pas de la même dose de pouvoir, et cette dose, ils ne la répartissent pas uniformément entre leurs membres. Entre le président et ses militants, le chanteur et ses \emph{fans}, le député et ses électeurs s’étendent les chemins de la promotion. Certains groupes sont solidement structurés, d’autres ont les contours très lâches ; cependant, tous se construisent grâce à l’illusoire sentiment de participer que partagent leurs membres, sentiments que l’on peut entretenir par des réunions, des insignes, des menus travaux, des responsabilités… Cohérence mensongère et souvent friable. Il y a, dans cet effarant scoutisme à tous les niveaux, des stéréotypes du cru : martyrs, héros, modèles, génie, penseur, dévoué de service et grand homme à succès. Par exemple : Danielle Casanova, Cienfuegos, Brigitte Bardot, Mathieu Axelos, le vétéran des sociétés de pétanque et Wilson. Le lecteur reconstituera par lui-même les groupes concernés.\par
La mise en collectivité des rôles remplacera-t-elle le vieux pouvoir déchu des grandes idéologies ? On ne peut oublier que le pouvoir est lié à son organisation de l’apparence. La retombée du mythe en fragments idéologiques s’étale aujourd’hui en une poussière de rôles. Cela signifie aussi que la misère du pouvoir n’a plus pour se dissimuler que la misère de son mensonge en miettes. Le prestige d’une vedette, d’un père de famille ou d’un chef d’État ne vaut même plus un pet de mépris. Rien n’échappe à la décomposition nihiliste, sinon son dépassement. Même une victoire technocratique interdisant ce dépassement livrerait les hommes à une activité vide, à un rite initiatique sans objet, à un sacrifice pur, à un enrôlement sans rôle, à une spécialisation de principe.\par
De fait, le spécialiste préfigure cet être fantomatique, ce rouage, cette chose mécanique logée dans la rationalité d’une organisation sociale, dans l’ordre parfait des zombies. On le rencontre partout, dans la politique comme dans le \emph{hold-up}. En un sens, la spécialisation est la science du rôle, elle donne au paraître le brillant que lui conféraient jadis la noblesse, l’esprit, le luxe ou le compte en banque. Mais le spécialiste fait plus. Il s’enrôle pour enrôler les autres ; il est ce chaînon entre la technique de production et de consommation et la technique de la représentation spectaculaire, mais c’est un chaînon isolé, une monade en quelque sorte. Connaissant le tout d’une parcelle, il engage les autres à produire et à consommer dans les limites de cette parcelle de telle sorte qu’il recueille une plus-value de pouvoir et accroisse sa part de représentation dans la hiérarchie. Il sait, au besoin, renoncer à la multiplicité des rôles pour n’en conserver qu’un, condenser son pouvoir au lieu de l’essaimer, réduire sa vie à l’unilinéaire. Il devient alors, un manager. Le malheur veut que le cercle où son autorité s’exerce soit toujours trop étroit, trop parcellaire. Il est dans la situation du gastro-entérologue qui guérit les maladies qu’il considère et empoisonne le reste du corps. Assurément l’importance du groupe où il sévit peut lui laisser l’illusion de son pouvoir, mais l’anarchie est telle, et les intérêts parcellaires si contradictoires et si concurrentiels, qu’il finit par prendre conscience de son impuissance. De même que se paralysent mutuellement les chefs d’État détenteurs de la force nucléaire, de même par leurs interférences les spécialistes élaborent et actionnent en dernier ressort une gigantesque machine – le pouvoir, l’organisation sociale – qui les domine tous et les écrase avec plus ou moins de ménagement, selon leur emplacement dans les rouages. Ils l’élaborent et l’actionnent aveuglément, car elle est l’ensemble de leurs interférences. Il faut donc attendre de la plupart des spécialistes que la soudaine conscience d’une passivité aussi désastreuse, et pour laquelle ils s’affairent si obstinément, les rejette avec autant de fougue vers la volonté de vivre authentiquement. Comme il est prévisible qu’un certain nombre d’entre eux, exposés plus longuement ou avec plus d’intensité aux radiations de la passivité autoritaire, doivent à la façon de l’officier, dans \emph{La Colonie pénitentiaire} de Kafka, mourir avec la machine, torturés par ses derniers soubresauts. Les interférences des gens de pouvoir, des spécialistes, font et défont chaque jour la majesté chancelante du pouvoir. On connaît le résultat. Que l’on imagine maintenant à quel cauchemar glacé nous condamnerait une organisation rationnelle, un \emph{pool} de cybernéticiens réussissant à éliminer les interférences ou du moins les contrôler. Il ne resterait que les tenants du suicide thermonucléaire pour leur disputer le prix Nobel.\par

\astermono

\noindent L’usage le plus commun du nom et de la photo, tel que l’ont répandu les papiers curieusement appelés d’« identité », montrent assez leur collusion avec l’organisation policière des sociétés contemporaines. Non seulement avec la basse police, celle des perquisitions, des filatures, des passages à tabac, des assassinats méthodiques, mais aussi avec les forces plus secrètes de l’ordre. Le passage répété d’un nom, d’une photo dans les réseaux de l’information écrite et orale indique à quel niveau hiérarchique et catégoriel l’individu se situe. Il va de soi que le nom le plus souvent prononcé dans un quartier, dans une ville, dans un pays, dans le monde, exerce un pouvoir de fascination. Une étude statistique menée sur cette base dans un espace-temps déterminé dresserait aisément une sorte de carte en relief du pouvoir.\par
Cependant, la détérioration du rôle va historiquement de pair avec l’insignifiance du nom. Pour l’aristocrate, le nom contient en résumé le mystère de la naissance et de la race. Dans la société de consommation, la mise en évidence publicitaire du nom de Bernard Buffet transforme en peintre célèbre un dessinateur médiocre. La manipulation du nom sert à fabriquer des dirigeants comme elle fait vendre une lotion capillaire. Cela signifie aussi qu’un nom célèbre n’appartient plus à celui qui le porte. Sous l’étiquette Buffet, il n’y a qu’une chose dans un bas de soie. Un morceau de pouvoir.\par
N’est-il pas comique d’entendre les humanistes protester contre la réduction des hommes à des numéros, à des matricules ? Comme si la destruction de l’homme sous l’originalité faisandée du nom ne valait pas l’inhumanité d’une série de chiffres. J’ai déjà dit que la lutte confuse entre les prétendus progressistes et les réactionnaires tournait toujours autour de la question : faut-il briser l’homme à coups de trique ou à coups de récompense ? C’est une belle récompense que d’avoir un nom connu.\par
Mais tant vont les noms aux choses que les êtres les perdent. Renversant la perspective, j’aime prendre conscience qu’aucun nom n’épuise ni ne recouvre ce qui est moi. Mon plaisir n’a pas de nom. Les trop rares moments où je me construis n’offrent aucune poignée par où l’on puisse les manipuler de l’extérieur. Seule la dépossession de soi s’empêtre dans le nom des choses qui nous écrasent. Je souhaite que l’on comprenne aussi dans ce sens, et pas seulement dans le simple refus du contrôle policier, le geste d’Albert Libertad brûlant ses papiers d’identité, cessant d’avoir un nom pour en choisir mille, geste que rééditeront en 1959 les travailleurs noirs de Johannesburg. Admirable dialectique du changement de perspective : puisque l’état des choses m’interdit de porter un nom qui soit comme pour les féodaux l’émanation de ma force, je renonce à toute appellation ; j’entre dans la forêt sans nom où la biche de Lewis Carroll explique à Alice :\par

\begin{quoteblock}
\noindent « Imagine que la maîtresse d’école désire t’interpeller. Plus de nom, la voilà qui crie hé ! ho ! mais personne ne s’appelle de la sorte, personne ne doit donc répondre. »\end{quoteblock}

\noindent Heureusement forêt de la subjectivité radicale.\par
Giorgio de Chirico me paraît rejoindre avec une belle conséquence le chemin qui mène à la forêt d’Alice. Ce qui est vrai pour le nom reste valable pour la représentation du visage. La photo exprime essentiellement le rôle, la pose. L’âme y est emprisonnée, soumise à l’interprétation ; c’est pourquoi une photo a toujours l’air triste. On l’examine comme on examine un objet. Et d’ailleurs, n’est-ce pas se faire objet que de s’identifier à une gamme d’expressions, si variées soient-elles ? Le Dieu des mystiques savait du moins éviter cet écueil. Mais j’en reviens à Chirico. À peu près contemporain de Libertad (s’il était homme, le pouvoir ne se féliciterait jamais assez des rencontres qu’il a su empêcher), ses personnages à tête vide dressent bien le bilan accusateur de l’inhumanité. Les places désertes, le décor pétrifié montrent l’homme déshumanisé par les choses qu’il a crées et qui, figées dans un urbanisme où se condense la force oppressive des idéologies, le vident de sa substance, le vampirisent ; je ne sais plus qui parle, à propos d’une toile, de paysage vampirique – Breton peut-être. Par ailleurs, l’absence de traits appelle en creux la présence d’un visage nouveau, une présence qui humanise les pierres elles-mêmes. Ce visage est pour moi celui de la création collective. Parce qu’il n’a le visage de personne, le personnage de Chirico a le visage de tous.\par
Tandis que la culture contemporaine se donne beaucoup de peine pour signifier son néant, tire une sémiologie de sa propre nullité, voici une peinture où l’absence s’ouvre de façon explicite vers la poésie des faits, vers la réalisation de l’art, de la philosophie, de l’homme. Trace d’un monde réifié, l’espace blanc, introduit dans la toile à l’endroit essentiel, indique aussi que le visage a quitté le lieu des représentations et des images, qu’il va maintenant s’intégrer dans la \emph{praxis} quotidienne.\par
La période 1910-1920 révélera un jour son incomparable richesse. Pour la première fois, avec beaucoup d’incohérence et de génie, un pont fut projeté entre l’art et la vie. J’ose dire qu’il n’existe rien, l’aventure surréaliste exceptée, dans la période qui va de cette avant-garde du dépassement à l’actuel projet situationniste. Le désenchantement de la vieille génération qui piétine depuis quarante ans, que ce soit dans le domaine de l’art ou de la révolution, ne me démentira pas. Le mouvement Dada, le carré blanc de Malévitch, \emph{Ulysses}, les toiles de Chirico fécondent, par la présence de l’homme total, l’absence de l’homme réduit à l’état de chose. Et l’homme total n’est rien d’autre aujourd’hui que le projet que le plus grand nombre des hommes élabore au nom de la créativité interdite.
\subsection[{6. La subversion}]{\textsc{6.} La subversion}
\noindent Dans le monde unitaire, sous le regard immobile des dieux, l’aventure et le pèlerinage définissent le changement à l’intérieur de l’immuable. Il n’y a rien à découvrir, car le monde est donné de toute éternité, mais la révélation attend le pèlerin, le chevalier, l’errant à la croisée des chemins. En vérité la révélation est en chacun : parcourant le monde, ou la cherchant en soi, on la cherche au loin et elle jaillit soudain, source miraculeuse que la pureté d’un geste fait sourdre à l’endroit même où le chercheur disgracié n’aurait rien deviné. La \emph{source} et le \emph{château} dominent l’imagination créatrice du Moyen Âge. Leur symbolisme est clair : sous le mouvement, trouver l’immuable ; sous l’immuable, trouver le mouvement.\par
Qu’est-ce qui fait la grandeur d’Héliogabale, de Tamerlan, de Gilles de Rais, de Tristan, de Perceval ? Ils se retirent vaincus dans un Dieu vivant ; ils s’identifient au démiurge, délaissant leur humanité insatisfaite pour régner et mourir sous le masque de la divine épouvante. Cette mort des hommes, qu’est le Dieu de l’immuable, laisse la vie fleurir à l’ombre de sa faux. Le Dieu mort pèse plus lourd que l’ancien Dieu vivant ; en vérité la bourgeoisie ne nous a pas débarrassés de Dieu, elle a seulement climatisé son cadavre. Le romantisme est l’odeur de Dieu pourrissant, le reniflement de dégoût devant les conditions de survie.\par
Classe déchirée par les contradictions, la bourgeoisie fonde sa domination sur la transformation du monde mais refuse sa propre transformation. Elle est un mouvement qui veut échapper au mouvement. Dans le régime unitaire, l’image de l’immuable contenait le mouvement. Dans le régime parcellaire, le mouvement va s’efforcer de reproduire l’immuable. (Il y aura toujours des guerres, des pauvres, des esclaves.) La bourgeoisie au pouvoir ne tolère que le changement vide, abstrait, coupé de la totalité. C’est un changement partiel et un changement de parcelle. Mais l’habitude du changement est dans son principe chargé de subversion. Or le changement est l’impératif qui domine la société consommation. Il faut que les gens changent de voitures, de mode, d’idées. Il le faut pour qu’un changement radical ne vienne mettre un terme à une forme d’autorité qui n’a plus d’autre issue pour s’exercer encore que de s’offrir en consommation, de se consumer en consumant chacun. Par malheur, dans cette fuite en avant vers la mort, dans cette course qui ne veut pas finir, il n’y a pas d’avenir réel, il n’y a qu’un passé habillé de neuf à la hâte et jeté dans le futur. Depuis près d’un quart de siècle, les mêmes nouveautés se succèdent sur le marché du gadget et des idées, à peine maquillées de la veille. De même sur le marché des rôles. Comment disposerions-nous d’une variété telle que l’ancienne \emph{qualité} du rôle, le rôle selon la conception féodale, puisse s’en trouver compensée ? Alors que :\par
1° le quantitatif est en soi une limite et appelle la reconversion en qualitatif ;\par
2° le mensonge du renouvellement transparaît dans la pauvreté du spectacle. L’enrôlement successif use les travestis. La multiplication des changements de détails exacerbe le désir de changer sans jamais le satisfaire. En précipitant le changement d’illusions, le pouvoir ne peut échapper à la réalité du changement radical.\par
Non seulement la multiplication des rôles tend à les rendre équivalents, mais encore elle les fragmente, elle les rend dérisoires. La quantification de la subjectivité a créé des catégories spectaculaires pour les gestes les plus prosaïques ou les dispositions les plus communes : une façon de sourire, un tour de poitrine, une coupe de cheveux… Il y a de moins en moins de grands rôles, de plus en plus de figurations. Même les Ubu-Staline, Hitler, Mussolini n’ont plus que de pâles descendants. La plupart des gens connaissent bien le malaise d’entrer dans un groupe et de prendre contact, c’est l’angoisse du comédien, la peur de tenir mal son rôle. Il faut attendre de l’émiettement des attitudes et des poses officiellement contrôlables que cette angoisse redécouvre sa source : non pas la maladresse du rôle mais la perte de soi dans le spectacle, dans l’ordre des choses. Dans son livre \emph{Médecine et Homme total}, le docteur Solié constate à propos de l’extension effarante des maladies nerveuses :\par

\begin{quoteblock}
\noindent « Il n’y a pas de maladie en soi, de même qu’il n’y a pas de malade en soi, il n’y a qu’un être-dans-le-monde authentique ou inauthentique. »\end{quoteblock}

\noindent La reconversion de l’énergie volée par le paraître en volonté de vivre authentiquement s’inscrit dans la dialectique de l’apparence. Déclenchant une réaction de défense quasi biologique, le refus de l’inauthentique a toutes les chances de détruire dans sa violence ceux qui n’ont cessé d’organiser le spectacle de l’aliénation. Ceux qui se font aujourd’hui une gloire d’être idoles, artistes, sociologues, penseurs, spécialistes de toutes les mises en scène devraient y réfléchir. Les explosions de colère populaire ne sont pas des accidents au même titre que l’éruption du Krakatoa.\par

\astermono

\noindent Un philosophe chinois a dit :\par

\begin{quoteblock}
\noindent « La confluence est l’approche du néant. Dans la confluence totale, la présence remue. »\end{quoteblock}

\noindent L’aliénation s’étend à toutes les activités de l’homme en les dissociant à l’extrême mais, se dissociant du même coup, elle devient partout plus vulnérable. Dans la désagrégation du spectacle, comme l’écrivait Marx\par

\begin{quoteblock}
\noindent « La vie nouvelle prend conscience de soi, ruine ce qui était ruiné, et rejette ce qui était rejeté. »\end{quoteblock}

\noindent Sous la dissociation, il y a l’unité ; sous l’usure, la concentration d’énergie ; sous l’émiettement de soi, la subjectivité radicale. Le qualitatif. Mais il ne suffit pas de vouloir refaire le monde comme on fait l’amour avec la fille que l’on aime.\par
Plus s’épuise ce qui a pour fonction de dessécher la vie quotidienne, plus la puissance de vie l’emporte sur le pouvoir du rôle. Ainsi s’amorce le renversement de perspective. C’est à ce niveau que la nouvelle théorie révolutionnaire doit se concentrer afin d’ouvrir la brèche du dépassement. À l’ère du calcul et à l’ère du soupçon inaugurées par le capitalisme et le stalinisme s’oppose et se construit dans un phase clandestine de tactique l’\emph{ère du jeu}.\par
L’état de dégradation du spectacle, les expériences individuelles, les manifestations collectives de refus doivent préciser dans les faits le maniement tactique du rôle. Collectivement, il est possible de supprimer les rôles. La créativité spontanée et le sens de la fête qui se donnent libre cours dans les moments révolutionnaires en offrent de nombreux exemples. Quand la joie occupe le cœur du peuple, il n’y a ni chef ni mise en scène qui puisse s’en emparer. C’est seulement en affamant leur joie que l’on se rend maître des masses révolutionnaires ; en les empêchant d’aller plus loin et d’étendre leurs conquêtes. Dans l’immédiat, un groupe d’action théorique et pratique comme celui que constituent les situationnistes est déjà capable d’entrer dans le spectacle politico-culturel en tant que subversion.\par
Individuellement, et donc de façon transitoire, il faut savoir nourrir ses rôles sans jamais les engraisser à ses dépens. Se protéger par eux en se protégeant contre eux ; récupérer l’énergie qu’ils absorbent, le pouvoir qu’ils donnent illusoirement. Jouer le jeu de Jacques Vaché.\par
Si ton rôle en impose aux autres, deviens ce pouvoir qui n’est pas toi, puis laisse errer son fantôme. On succombe toujours dans une lutte de prestige, ne te fatigue pas. Pas de vaines querelles, pas de discussion oiseuses, pas de forum, pas de colloques, pas de semaines pour la pensée marxiste ! Quand il faudra frapper pour te libérer vraiment, frappe pour tuer ! Les mots ne tuent pas.\par
Des gens t’entourent, ils veulent discuter. Ils t’admirent ? Crache-leur au visage ; ils se moquent de toi ? Aide-les à se trouver dans leur rire. Le rôle porte en soi le ridicule. Il n’y a que des rôles autour de toi ? Jettes-y ta désinvolture, ton humour, ta distanciation ; joue avec eux comme le chat avec la souris ; il se peut qu’à ce traitement, l’un ou l’autre de tes proches s’éveille à lui-même, découvre les conditions du dialogue. Également aliénants, tous les rôles ne sont d’ailleurs pas également méprisables. Dans l’échantillonnage des conduites formalisées, quelques-unes dissimulent à peine le vécu et ses exigences aliénées. Des alliances temporaires sont, me semble-t-il, permises avec certaines attitudes, avec certaines images révolutionnaires pour autant qu’à travers l’idéologie qu’elles supposent, il y ait promesse de radicalité. Je pense notamment au culte de Lumumba chez les jeunes révolutionnaires congolais. Celui qui garde présent à l’esprit que le seul traitement valable pour les autres et pour soi est l’accroissement de la dose de radicalité ne peut ni se tromper ni se perdre.
\section[{XVI. La fascination du temps}]{XVI. La fascination du temps}\renewcommand{\leftmark}{XVI. La fascination du temps}


\begin{argument}\noindent Par un gigantesque envoûtement, la croyance au temps de l’écoulement fonde la réalité de l’écoulement du temps. Le temps est l’usure de l’adaptation à laquelle l’homme doit se résoudre chaque fois qu’il échoue à transformer le monde. L’âge est un rôle, une accélération du temps « vécu » sur le plan de l’apparence, un attachement aux choses.
\end{argument}

\noindent L’accroissement du malaise dans la civilisation infléchit aujourd’hui les thérapeutiques dans la voie d’une nouvelle démonologie. De même que l’invocation, l’envoûtement, la possession, l’exorcisme, l’orgie sabbatique, la métamorphose, le talisman possédaient le privilège ambigu de guérir ou de faire souffrir, de même il arrive aujourd’hui, toujours plus sûrement, que la consolation de l’homme opprimé (médecine, idéologie, compensation du rôle, \emph{gadgets} de confort, méthodes de transformation du monde…) nourrisse l’oppression elle-même. Il existe un \emph{ordre des choses} maladif, voilà ce que les dirigeants veulent à tout prix dissimuler. Wilhelm Reich explique dans une belle page de \emph{La Fonction de l’orgasme} comment il parvint après de longs mois de traitement psychanalytique à guérir une jeune ouvrière viennoise. Elle souffrait d’une dépression due à ses conditions de vie et de travail. Guérie, Reich la renvoya dans son milieu. Quinze jours plus tard, elle se suicidait. On sait que la lucidité et l’honnêteté de Reich devait le condamner à l’exclusion des cercles psychanalytiques, à l’isolement, au délire et à la mort ; on ne dévoile pas impunément la duplicité des démonologues.\par
Ceux qui organisent le monde organisent la souffrance et son anesthésie ; c’est connu. La plupart des gens vivent en somnambules, partagés entre la crainte et le désir de s’éveiller ; coincés entre leur état névrotique et le traumatisme d’un retour au vécu. Cependant, voici l’époque où la survie sous anesthésie exige des doses qui vont, saturant l’organisme, déclencher ce que l’on nomme dans l’opération magique un « choc en retour ». C’est l’imminence de ce bouleversement et sa nature qui permettent de parler du conditionnement des hommes comme d’un gigantesque envoûtement.\par
L’envoûtement suppose l’existence d’un espace-réseau reliant les objets les plus éloignés à l’aide d’une sympathie dirigée par des lois spécifiques, analogie formelle, coexistence organique, symétrie fonctionnelle, alliance des symboles… Les correspondances s’établissent en associant un nombre incalculable de fois une conduite et l’apparition d’un signal. Il s’agit en somme d’un conditionnement généralisé. Or on peut se demander si la mode aujourd’hui très répandue de dénoncer un certain conditionnement, propagande, publicité, \emph{mass media}, n’agit pas comme un exorcisme partiel qui maintient en place et hors de soupçon un envoûtement plus vaste, plus essentiel. Il est facile de railler l’outrance de \emph{France-Soir} pour tomber dans le mensonge distingué du \emph{Monde}. L’information, le langage, le temps ne sont-ils pas les tenailles gigantesques avec lesquelles le pouvoir travaille l’humanité et la range dans sa perspective ? Une emprise maladroite, il est vrai, mais dont la force est d’autant plus prégnante que les hommes n’ont pas conscience de savoir lui résister et qu’ils ignorent souvent dans quelle mesure ils lui résistent déjà spontanément.\par
Les grands procès staliniens ont démontré qu’il suffisait d’un peu de patience et d’obstination pour faire s’accuser un homme de tous les crimes et l’envoyer en public implorer sa mise à mort. Aujourd’hui conscient d’une telle technique et mis en garde contre elle, comment pourrait-on ignorer que l’ensemble des mécanismes qui nous dirigent décrète avec la même insidieuse persuasion mais avec plus de moyens et plus de constance : « Tu es faible, tu dois vieillir, tu dois mourir. » La conscience obéit, puis le corps. J’aime comprendre en matérialiste la phrase d’Antonin Artaud.\par

\begin{quoteblock}
\noindent « On ne meurt pas parce qu’il faut mourir ; on meurt parce que c’est un pli auquel on a contraint la conscience un jour, il n’y a pas si longtemps. »\end{quoteblock}

\noindent En terrain non propice, une plante meurt. L’animal s’adapte au milieu, l’homme le transforme. La mort n’est donc pas, selon qu’il s’agit d’une plante, d’un animal ou d’un homme, un phénomène identique. En terrain favorable, la plante se trouve dans les conditions de l’animal, elle peut s’adapter. Dans la mesure où l’homme échoue à transformer son milieu ambiant, il se trouve lui aussi dans les conditions de l’animal. L’adaptation est la loi du monde animal.\par
Le syndrome général d’adaptation dit Hans Selye, le théoricien du \emph{Stress}, passe par trois phases : la réaction d’alarme, le stade de résistance, le stade de l’épuisement. Sur le plan du paraître, l’homme a su lutter pour l’éternité mais, sur le plan de la vie authentique, il en reste à l’adaptation animale : réaction spontanée de l’enfance, consolidation de l’âge adulte, épuisement de la vieillesse. Et plus il veut aujourd’hui paraître, plus le caractère éphémère et incohérent du spectacle lui remontre qu’il vit comme un chien et meurt comme une touffe d’herbe sèche. Car enfin, se résoudra-t-on bientôt à admettre que l’organisation sociale que l’homme s’est créée pour transformer le monde au mieux de ses désirs a désormais cessé de l’aider ; n’est plus, entré dans l’usage, que l’interdiction d’employer selon les règles d’une organisation supérieure encore à créer, les techniques de libération et de réalisation individuelles qu’il s’est forgées à travers l’histoire de l’appropriation privative, de l’exploitation de l’homme par l’homme, du pouvoir hiérarchisé ?\par
Nous vivons désormais dans un système clos, étouffant. Ce qui se gagne d’un côté se reperd de l’autre. Vaincue quantitativement par les progrès en matière sanitaire, la mort s’introduit qualitativement dans la survie. L’adaptation est démocratisée, rendue plus facile pour tous, et l’on perd à ce prix l’essentiel, qui est d’adapter le monde à l’humain.\par
Certes il existe une lutte contre la mort, mais elle prend place à l’intérieur même du syndrome d’adaptation ; ce qui revient à mêler la mort au remède. Il est d’ailleurs significatif que les recherches thérapeutiques portent surtout sur le stade d’épuisement, comme si l’on voulait, jusque dans la vieillesse, prolonger le stade de résistance. On applique le traitement de choc quand la faiblesse et l’impuissance ont déjà fait leur œuvre ; un traitement de choc chargé d’empêcher l’usure d’adaptation impliquerait trop assurément, comme l’avait compris Reich, que l’on s’en prenne directement à l’organisation sociale, à ce qui interdit de dépasser le stade d’adaptation. On préfère les guérisons partielles, du moins l’ensemble n’en souffre pas. Mais que se passera-t-il quand la vie quotidienne se trouvera, à force de guérisons partielles, atteinte dans son ensemble par le malaise de l’inauthentique ? Quand l’exorcisme et l’envoûtement auront dévoilé à tous leur apport commun à la société du malaise ?\par

\astermono

\noindent On ne pose pas la question « Quel âge avez-vous ? » sans se référer aussitôt au pouvoir. La date repère y contraint déjà. Ne mesure-t-on pas le temps au départ d’une manifestation d’autorité : agrégation d’un Dieu, d’un messie, d’un chef, d’une ville conquérante ? Dans l’esprit aristocratique, le temps accumulé est un gage d’autorité : la vieillesse, mais aussi la série des ancêtres, accroissent la prépotence du noble. En mourant, l’aristocrate lègue à sa descendance une vitalité tonifiée par le passé. Au contraire, la bourgeoisie n’a pas de passé ; elle n’en reconnaît pas du moins, son pouvoir en miettes n’obéit plus à l’hérédité. Elle refait parodiquement le chemin de la noblesse : l’identification au temps cyclique, au temps de l’éternel retour, se satisfait dans une identification aveugle à des morceaux de temps linéaire, à des passages successifs et rapides.\par
Le rapport de l’âge avec l’indice de départ du temps mesurable n’est pas la seule allusion indiscrète au pouvoir. Je soutiens que l’âge mesuré n’est rien d’autre qu’un rôle, une accélération du temps vécu sur le mode du non-vécu, donc sur le plan de l’apparence et selon les lois de l’adaptation. En prenant du pouvoir, on prend de l’âge. Jadis, seuls les gens âgés, c’est-à-dire d’ancienne noblesse ou d’expérience ancienne exerçaient le pouvoir. Aujourd’hui l’on étend aux jeunes le privilège douteux de vieillir. La société de consommation mène au vieillissement précoce ; n’a-t-elle pas trouvé sous l’étiquette \emph{teen-ager} un groupe nouveau à convertir en consommateurs ? Celui qui consomme se consume en inauthentique ; il nourrit le paraître au profit du spectacle et aux dépens de la vraie vie. Il meurt où il s’accroche parce qu’il s’accroche à des choses mortes ; à des marchandises, à des rôles.\par
Tout ce que tu possèdes te possède en retour. Tout ce qui te rend propriétaire t’adapte à la nature des choses ; te vieillit. \emph{Le temps qui s’écoule est ce qui remplit l’espace vide laissé par l’absence du moi}. Si tu cours après le temps, le temps court plus vite encore : c’est la loi du consommable. Veux-tu le retenir ? Il t’essouffle et te vieillit d’autant. Il faut le prendre sur le fait, dans le présent ; mais le présent est à construire.\par
Nous étions nés pour ne jamais vieillir, pour ne mourir jamais. Nous n’aurons que la conscience d’être venus trop tôt ; et un certain mépris du futur qui nous assure déjà une belle tranche de vie.
\section[{XVII. Le mal de survie}]{XVII. Le mal de survie}\renewcommand{\leftmark}{XVII. Le mal de survie}


\begin{argument}\noindent Le capitalisme a démystifié la survie. Il a rendu insupportable la pauvreté de la vie quotidienne confrontée à l’enrichissement des possibilités techniques. La survie est devenue une économie de la vie. La civilisation de la survie collective multiplie les temps morts de la vie individuelle, si bien que la part de mort risque de l’emporter sur la survie collective elle-même. À moins que la rage de détruire ne se reconvertisse en rage de vivre.
\end{argument}

\noindent Jusqu’à présent, les hommes n’ont fait que s’adapter à un \emph{système} de transformation du monde. Il s’agit maintenant d’adapter le système à la transformation du monde.\par
L’organisation des sociétés humaines a changé le monde, et le monde en changeant a bouleversé l’organisation des sociétés humaines. Mais tandis que l’organisation hiérarchisée s’empare de la nature et se transforme dans la lutte, la part de liberté et de créativité réservée aux individus se trouve absorbée par la nécessité de s’adapter aux normes sociales et à leurs variations ; du moins en l’absence de moments révolutionnaires généralisés.\par
Le temps de l’individu dans l’histoire est en majeure partie un temps mort. Que ceci nous soit devenu insupportable date d’une prise de conscience assez récente. D’une part, la bourgeoisie prouve par sa révolution que les hommes \emph{peuvent} accélérer la transformation du monde, qu’ils \emph{peuvent} individuellement améliorer leur vie, l’amélioration étant ici comprise comme une accession à la classe dominante, à la richesse, au succès capitaliste. D’autre part, elle annule par interférence la liberté des individus, elle accroît les temps morts dans la vie quotidienne (nécessité de produire, de consommer, de calculer), elle s’incline devant les lois hasardeuses du marché, devant les inévitables crises cycliques avec leur lot de guerres et de misère, devant les barrières de bon sens (on ne changera pas l’homme, il y aura toujours des pauvres…). La politique de la bourgeoisie, et des séquelles socialistes, est un politique de coups de frein dans une voiture dont l’accélérateur est bloqué à fond de course. Plus l’accélération augmente, plus les coups de frein sont brusques, dangereux et inopérants. La vitesse du consommable est la vitesse de désagrégation du pouvoir ; et simultanément, l’élaboration imminente d’un monde nouveau, d’une nouvelle dimension, d’un univers parallèle né dans l’effondrement du Vieux Monde.\par
Le passage du système d’adaptation aristocratique au système d’adaptation « démocratique » élargit brutalement l’écart existant entre la passivité de la soumission individuelle et le dynamisme social qui transforme la nature, entre l’impuissance des hommes et la puissance des techniques nouvelles. L’attitude contemplative sied parfaitement au mythe féodal, à un monde quasi immobile serti par ses Dieux éternels. Mais comment l’esprit de soumission s’accommoderait-il de la vision dynamique des marchands, des manufacturiers, des banquiers, des découvreurs de richesse, de ceux qui connaissent, non la révélation de l’immuable, mais celle du mouvement économique, la soif insatiable de profit, le besoin d’un renouvellement permanent ? Pourtant, partout où elle vulgarise et valorise le passager, le transitoire, l’espoir, la bourgeoisie en tant que pouvoir s’efforce d’y emprisonner les hommes réels. Elle substitue à l’immobilisme théologique une métaphysique du mouvement ; l’une et l’autre représentations entravent la réalité mouvante, mais la première avec plus de bonheur et d’harmonie que la seconde ; avec plus de cohérence et plus d’unité. L’idéologie du progrès et du changement mise au service de l’immuable, voilà le paradoxe que rien ne peut désormais, ni dissimuler à la conscience, ni justifier devant elle. On voit, dans cet univers en expansion de la technique et du confort, les êtres se replier sur eux-mêmes, se racornir, vivre petitement, mourir pour des détails. Le cauchemar offre à la promesse d’une liberté totale un mètre cube d’autonomie individuelle, rigoureusement contrôlée par les voisins. Un espace-temps de la mesquinerie et de la pensée basse.\par
La mort dans un Dieu vivant donnait à la vie quotidienne sous l’Ancien Régime une dimension illusoire qui atteignait la richesse d’une réalité multiple. Disons que jamais on ne s’est mieux réalisé dans l’inauthentique. Mais que dire de la vie sous un Dieu mort, sous le Dieu pourrissant qu’est le pouvoir parcellaire ? La bourgeoisie a fait l’économie d’un Dieu en économisant sur la vie des hommes. Elle a aussi fait de l’économique un impératif sacré et de la vie un système économique. C’est ce schéma que les programmateurs du futur s’apprêtent à rationaliser, à planifier, à humaniser, quoi. Et que l’on se rassure, la programmation cybernéticienne aura l’irresponsabilité du cadavre de Dieu.\par
Kierkegaard exprime bien le mal de survie lorsqu’il écrit :\par

\begin{quoteblock}
\noindent « Laissons les autres gémir sur la méchanceté de leur époque. Moi je me plains de sa mesquinerie ; car elle est sans passion… Ma vie se résout en une seule couleur. »\end{quoteblock}

\noindent La survie est la vie réduite à l’essentiel, à la forme abstraite, au ferment nécessaire pour que l’homme participe à la production et à la consommation. Pour l’esclave romain, le repos et la nourriture. Pour les bénéficiaires des Droits de l’Homme, de quoi se nourrir et se cultiver, assez de conscience pour tenir un rôle, d’initiative pour gagner du pouvoir, de passivité pour en arborer les signes. La liberté de s’adapter d’une façon \emph{supérieurement animale}.\par
La survie est une vie au ralenti. Le paraître implique de telles dépenses ! Elle a son hygiène intime amplement vulgarisée par l’information : éviter les émotions fortes, surveiller sa tension, manger peu, boire raisonnablement, survivre en bonne santé pour mieux vivre son rôle. « Le surmenage, maladie des dirigeants », titrait \emph{Le Monde} dans une de ses rubriques. Il faut ménager la survie, car elle est usure ; il faut la vivre peu, car elle est à la mort. On mourait jadis en la mort faite vie, en Dieu. Aujourd’hui le respect de la vie interdit de la toucher, de l’éveiller, de la sortir de sa léthargie. On meurt par inertie, quand la quantité de mort que l’on porte en soi atteint son point de saturation. Quelle académie des sciences révélera le taux de radiations mortelles qui tuent nos gestes quotidiens ? A force de s’identifier à ce qui n’est pas soi, à passer d’un rôle à l’autre, d’un pouvoir à l’autre, d’un âge à l’autre, comment n’être pas enfin ce passage éternel qu’est la décomposition ?\par
La présence, au sein de la vie même, d’une mort mystérieuse et tangible, a pu abuser Freud au point de l’inciter à reconnaître une malédiction ontologique, un prétendu instinct de mort. Déjà annoncée par Reich, l’erreur de Freud transparaît aujourd’hui, clarifiée par le phénomène de consommation. Les trois éléments de l’instinct de mort, nirvana, tendance à la répétition, masochisme, ne traduisent rien d’autre que les trois styles d’emprise du pouvoir : la contrainte assumée passivement, la séduction coutumière, la médiation perçue comme une loi inéluctable.\par
On le sait, la consommation de biens – qui est toujours dans l’état actuel une consommation de pouvoir – porte en elle sa propre destruction et ses conditions de dépassement. La satisfaction du consommateur ne peut ni ne doit jamais être atteinte ; la logique du consommable exige que soient créés de nouveaux besoins, mais il est aussi vrai que l’accumulation de ces besoins falsifiés accentue le malaise de l’homme maintenu, de plus en plus malaisément, dans son unique état de consommateur. De plus, la richesse en biens de consommation appauvrit le vécu authentique. Elle l’appauvrit doublement ; d’abord en lui donnant sa contrepartie en \emph{choses} ; ensuite parce qu’il est impossible, même si on le voulait, de s’attacher à ces choses puisqu’il faut les consommer, c’est-à-dire les détruire. De là un manque à vivre sans cesse plus exigeant, une insatisfaction qui se dévore elle-même. Or ce besoin de vivre est ambivalent ; il est un point du renversement de perspective.\par
Dans l’optique orientée du consommateur, dans la vision conditionnée, le manque à vivre apparaît comme un manque à consommer du pouvoir et à se consumer pour le pouvoir. À l’absence de vraie vie est offert le palliatif d’une mort à tempérament. Un monde qui condamne à mourir exsangue est bien forcé de propager le goût du sang. Où règne le mal de survie, le désir de vivre prend spontanément les armes de la mort : meurtre gratuit, sadisme… Si l’on détruit la passion, elle renaît dans la passion de détruire. Personne, à ces conditions, ne survivra à l’ère de la survie. Et déjà le désespoir actuel atteint un tel degré que beaucoup de gens peuvent reprendre à leur compte le propos d’Antonin Artaud :\par

\begin{quoteblock}
\noindent « Je suis stigmatisé par une mort pressante où la mort véritable est pour moi sans terreur. »\end{quoteblock}

\noindent L’homme de la survie est l’homme du plaisir-angoisse, de l’inachevé, de la mutilation. Où irait-il se retrouver dans cette perte infinie de soi où tout l’engage ? Son errance est un labyrinthe privé de centre, un labyrinthe rempli de labyrinthes. Il se traîne dans un monde d’équivalences. Se tuer ? Pour se tuer, il faut sentir une résistance, posséder en soi une valeur à détruire. S’il n’y a rien, les gestes de destruction eux-mêmes s’effritent, volent en éclats. On ne jette pas du vide dans le vide. « Si une pierre tombait et me tuait, ce serait un expédient », écrit Kierkegaard. Il n’est aujourd’hui personne, je crois, qui n’ait ressenti l’épouvante d’une telle pensée. C’est l’inertie qui tue le plus sûrement, inertie de ceux qui choisissent le gâtisme à dix-huit ans, se plongent huit heures par jour dans un travail abrutissant, se nourrissent d’idéologies. Sous le pitoyable clinquant du spectacle, il n’y a que des êtres décharnés, souhaitant et redoutant l’expédient de Kierkegaard pour n’avoir plus jamais à souhaiter ce qu’ils redoutent, pour n’avoir plus jamais à redouter ce qu’ils souhaitent.\par
Parallèlement, la rage de vivre apparaît comme une existence biologique, le revers de la rage de détruire et de se laisser détruire.\par

\begin{quoteblock}
\noindent « Tant que nous ne serons pas parvenus à supprimer aucune des causes du désespoir humain, nous n’aurons pas le droit d’essayer de supprimer les moyens par lesquels l’homme essaie de se débarrasser du désespoir. »\end{quoteblock}

\noindent Le fait est que l’homme dispose à la fois des moyens de supprimer les causes du désespoir et de la force qu’il est capable de déployer pour s’en débarrasser. Personne n’a le droit d’ignorer que l’emprise du conditionnement l’accoutume à survivre sur une centième de ses possibilités de vivre. Il y a trop d’unité dans le mal de survie pour que le vécu rende plus compact n’unisse à son tour le plus grand nombre des hommes dans la volonté de vivre. Pour que le refus du désespoir ne devienne la construction d’une vie nouvelle. Pour que l’économie de la vie ne s’ouvre sur la mort de l’économie ; au-delà de la survie.
\section[{XVIII. Le refus en porte-à-faux}]{XVIII. Le refus en porte-à-faux}\renewcommand{\leftmark}{XVIII. Le refus en porte-à-faux}


\begin{argument}\noindent Il existe un moment de dépassement, un moment historiquement défini par la force et la faiblesse du pouvoir ; par le morcellement de l’individu jusqu’à l’atome subjectif ; par la familiarité de la vie quotidienne avec ce qui la détruit. Le dépassement sera général, unitaire et subjectif-construit (1). – Abandonnant leur radicalité, les éléments initialement révolutionnaires se condamnent au réformisme. Aujourd’hui l’abandon quasi général de l’esprit révolutionnaire définit les réformes de survie. – Une organisation révolutionnaire nouvelle doit isoler les noyaux de dépassement dans les grands mouvements du passé, elle doit reprendre et réaliser notamment : le projet de la liberté individuelle perverti par le libéralisme ; le projet de la liberté collective, perverti par le socialisme ; le projet de retrouver la nature, perverti par le fascisme ; le projet de l’homme total, perverti par les \emph{idéologies} marxistes, ce projet qui anime, sous le langage théologique du temps, les grandes hérésies du Moyen Age et leur rage anticléricale si opportunément exhumée par notre siècle, où les clercs s’appellent « spécialistes » (2). – L’homme du ressentiment est le parfait survivant, l’homme privé de la conscience du dépassement possible, l’homme de la décomposition (3). – Quand l’homme du ressentiment prend conscience de la décomposition spectaculaire, il devient nihiliste. Le nihilisme actif est prérévolutionnaire. Il n’y a pas de conscience du dépassement nécessaire sans conscience de la décomposition. – Les blousons noirs sont les héritiers légitimes de Dada (4).
\end{argument}

\subsection[{1. La question du dépassement.}]{\textsc{1.} La question du dépassement.}
\noindent Le refus est multiple, le dépassement est un. Confrontée à l’insatisfaction contemporaine et par celle appelée à témoigner, l’histoire humaine se confond avec l’histoire d’un refus radical toujours porteur du dépassement, toujours porté vers sa propre négation ; un refus dont les aspects multiples ne dissimulent jamais ce qu’il y a d’essentiellement commun entre la dictature d’un Dieu, d’un roi, d’un chef, d’une classe, d’une organisation. Quel imbécile a parlé d’une ontologie de révolte ? En transformant l’aliénation naturelle en aliénation sociale, le mouvement historique enseigne aux hommes la liberté dans l’esclavage, il leur apprend simultanément la révolte et la soumission. La révolte a moins besoin de métaphysique que les métaphysiciens de révolte. L’existence, attestée depuis des millénaires, d’un pouvoir hiérarchisé, suffit parfaitement à expliquer la permanence d’une contestation, et de la répression qui la brise.\par
Le renversement de la féodalité et la réalisation du maître sans esclave forment un seul et même projet. L’échec partiel de ce projet, lors de la révolution française, n’a cessé de le rendre plus familier et plus désirable à mesure que d’autres révolutions avortées – à titres divers, la Commune et la révolution bolchevique – le précisaient et en différaient l’accomplissement.\par
Les philosophies de l’histoire ont, toutes, partie liée avec cet échec. C’est pourquoi la conscience de l’histoire est aujourd’hui indissociable de la conscience du dépassement nécessaire.\par
Le point de dépassement est de mieux en mieux repérable sur l’écran social. Pourquoi ? La question du dépassement est une question tactique. Dans ses grandes lignes, elle se présente comme suit :\par
\labelchar{1. –} Ce qui ne tue pas le pouvoir le rend plus fort, mais ce que le pouvoir ne tue pas l’affaiblit à son tour.\par
— Plus les impératifs de consommation englobent les impératifs de production, plus le gouvernement par contrainte cède le pas au gouvernement par séduction.\par
— Démocratiquement réparti, le privilège de consommer étend au plus grand nombre des hommes le privilège d’autorité (à des degrés divers, s’entend).\par
— Les hommes s’affaiblissent, leur refus s’anémie, dès qu’ils cèdent aujourd’hui à la séduction de l’Autorité. Le pouvoir se renforce donc mais, réduit par ailleurs à l’état de consommable, il se consume, il s’use, il devient vulnérable \emph{par nécessité}.\par
Le point de dépassement est un moment dans cette dialectique de la force et de la faiblesse. S’il appartient sans doute à la critique radicale de le localiser et de le renforcer tactiquement, en échange, les faits sont là partout pour susciter la critique radicale. Le dépassement chevauche la contradiction qui hante le monde actuel, défraie l’information quotidienne et caractérise la plupart des comportements :\par
1° le refus débile, c’est-à-dire le réformisme ;\par
2° le refus extravagant, c’est-à-dire le nihilisme (dont il faut distinguer la forme passive de la forme active)\par
\labelchar{2. –} En s’émiettant, le pouvoir hiérarchisé gagne en ubiquité et perd sa fascination. Moins de gens vivent en marge de la société, en trimardeurs, et moins de gens se montrent respectueux d’un patron, d’un prince, d’un dirigeant, d’un rôle ; plus de gens survivent dans la société et plus de gens vouent l’organisation sociale à l’exécration. Chacun est, \emph{dans sa vie quotidienne}, au centre du conflit. De là une double conséquence :\par
1° Victime de l’atomisation sociale, l’individu est aussi victime du pouvoir parcellaire. Mise en évidence et menacée, la subjectivité devient la revendication essentielle. Désormais, pour élaborer une collectivité harmonieuse, la théorie révolutionnaire devra se fonder non plus sur la base du communautaire mais sur la subjectivité, sur les cas spécifiques, sur le vécu particulier.\par
2° Morcelé à l’extrême, le refus recrée contradictoirement les conditions d’un refus global. Comment va se créer la nouvelle collectivité révolutionnaire ? Par une explosion en chaîne, de subjectivité à subjectivité. La construction d’une communauté d’individus à part entière amorcera le renversement de perspective sans lequel il n’est pas de dépassement possible.\par
\labelchar{3. –} Enfin, la notion même du renversement de perspective se vulgarise. Chacun côtoie de trop près sa propre négation. Le vivant se rebelle. L’enchantement des lointains disparaît quand l’œil approche trop ; la perspective aussi. En emprisonnant les hommes dans son décor de choses, en s’introduisant maladroitement en eux, le pouvoir répand le trouble et le malaise. Le regard et la pensée s’embrouillent, les valeurs s’estompent, les formes se diluent, les anamorphoses inquiètent, comme lorsqu’on se tient le nez collé à un tableau. Le changement de perspective picturale – Ucello, Kandisky – est d’ailleurs contemporain d’un changement de perspective sociale. Le rythme de consommation précipite l’esprit dans cet interrègne où proche et lointain coïncident. C’est avec l’appui des faits eux-mêmes que la plupart des hommes vont bientôt expérimenter cet état de liberté auquel aspiraient, mais sans les moyens de le réaliser, les hérétiques de Souabe en 1270 :\par

\begin{quoteblock}
 \noindent « S’étant élevé au-dessus de Dieu et ayant atteint le degré de la perfection divine, ils avaient abandonné Dieu ; il n’était pas rare, assure Cohn, qu’un adepte, homme ou femme, affirmât \emph{n’avoir plus du tout besoin de Dieu} »\par
 
\bibl{(\emph{Les fanatiques de l’Apocalypse})}
 \end{quoteblock}

\subsection[{2.Abandon de la misère et misère de l’abandon.}]{\textsc{2.}Abandon de la misère et misère de l’abandon.}
\noindent Il n’y a guère de mouvement révolutionnaire qui ne porte en soi la volonté d’un changement total, il n’y en a guère à ce jour qui n’ait fait sa victoire d’un changement de détail. Dès que le peuple en armes renonce à sa propre volonté pour suivre celle de ses conseillers, il perd l’emploi de sa liberté et couronne, sous le titre ambigu de dirigeants révolutionnaires, ses oppresseurs de demain. Telle est en quelque sorte la « ruse » du pouvoir parcellaire : il engendre des révolutions parcellaires, dissociées du renversement de perspective, coupées de la totalité ; détachées paradoxalement du prolétariat qui les fait. Comment voudrait-on que la totalité des libertés revendiquées s’accommode des quelques parcelles de libertés conquises sans faire aussitôt les frais d’un régime totalitaire ? On a cru y voir une malédiction : la révolution dévorant ses enfants : comme si la défaite de Makhno, l’écrasement de Cronstadt, l’assassinat de Durruti n’étaient impliqués déjà par la structure des noyaux bolchéviks initiaux, peut-être même par les attitudes autoritaires de Marx dans la Ire Internationale. Nécessité historique et raison d’État ne sont que nécessité et raison de dirigeants appelés à cautionner leur abandon du projet révolutionnaire, leur abandon de la radicalité.\par
L’abandon, c’est le non-dépassement. Et la contestation parcellaire, le refus partiel, la revendication en miettes, est précisément ce qui interdit le dépassement. La pire inhumanité n’est jamais qu’une volonté d’émancipation cédant aux compromis et se fossilisant sous la couche de ses renoncements successifs. Libéralisme, socialisme, bolchevisme se construisent de nouvelles prisons sous l’enseigne de la liberté. La gauche lutte pour un confort accru dans l’aliénation, mais elle a l’indigente habileté de le faire au nom des barricades, au nom du drapeau rouge et des plus beaux moments révolutionnaires. Fossilisée et déterrée comme appât, la radicalité originelle est trahie doublement, abandonnée deux fois. Prêtres-ouvriers, curés-blousons noirs, généraux communistes, princes rouges, dirigeants « révolutionnaires », l’élégance radicale se porte bien, elle s’harmonise au goût d’une société qui sait vendre un rouge à lèvres sous le slogan « Révolution en rouge, révolution avec Redflex ». La manœuvre n’est pas sans risque. À se caricaturer sans fin selon les normes de la publicité, la volonté la plus authentiquement révolutionnaire en vient à se raviser par contrecoup, à se purifier. Les allusions ne sont jamais perdues !\par
La nouvelle vague insurrectionnelle rallie aujourd’hui des jeunes gens qui se sont tenus à l’écart de la politique spécialisée, qu’elle soit de gauche ou de droite, ou qui y sont passés rapidement, le temps d’une erreur de jugement ou d’une ignorance excusables. Dans le raz de marée nihiliste, tous les fleuves se confondent. L’au-delà de cette confusion importe seul. La révolution de la vie quotidienne sera la révolution de ceux qui, retrouvant avec plus ou moins d’aisance les germes de réalisation totale conservés, contrariés, dissimulés dans les idéologies de tout genre, auront aussitôt cessé d’être mystifiés et mystificateurs.\par

\astermono

\noindent Même s’il a jamais existé un esprit de révolte au sein du christianisme, je dénie le droit et la capacité de le comprendre à un homme qui continue de s’affubler du nom de chrétien. Il n’y a plus aujourd’hui d’hérétiques. Le langage théologique dans lequel s’exprimèrent tant d’admirables soulèvements fut la marque d’une époque, le seul langage possible, sans plus. Il faut désormais traduire. Et la traduction va de soi. Compte tenu de mon temps, et de l’aide objective qu’il m’apporte, qu’ai-je dit de plus au XX\textsuperscript{e} siècle que des Frères du Libre Esprit ? Ils déclaraient au XIII\textsuperscript{e} :\par

\begin{quoteblock}
\noindent « On peut être à ce point uni à Dieu que, quoi qu’on fasse, on ne puisse pécher. J’appartiens à la liberté de la Nature et je satisfais tous les désirs de ma nature. L’homme libre a parfaitement raison de faire tout ce qui lui procure du plaisir. Que le monde entier soit détruit et périsse totalement plutôt qu’un homme libre s’abstienne de faire une seule action que sa nature le pousse à accomplir. »\end{quoteblock}

\noindent Et comment ne pas saluer Johann Hartmann :\par

\begin{quoteblock}
\noindent « L’homme véritablement libre est roi et seigneur de toutes les créatures. Toutes choses lui appartiennent, et il a le droit de se servir de toutes celles qui lui plaisent. Si quelqu’un l’en empêche, l’homme libre a le droit de le tuer et de prendre ses biens. »\end{quoteblock}

\noindent Ou encore ce Jean de Brünn :\par
« Toutes les choses que Dieu a créées sont communes à tous. Ce que l’œil voit et convoite, que la main s’en saisisse »\par
Il se justifiait ainsi d’avoir pratiqué la ruse, le brigandage et le vol à main armée. Ou les Pifles d’Arnold, purs à ce point que quoi qu’ils fissent, ils ne pouvaient pécher (1157) ? Ces diamants du christianisme ont toujours brillé d’un éclat trop vif aux yeux chassieux des chrétiens. Quand l’anarchiste Pauwels dépose, le 15 mars 1894, une bombe à l’église de la Madelaine, quand le jeune Robert Burger égorge un prêtre le 11 août 1963, c’est la grande tradition hérétique qui se perpétue pauvrement mais dignement dans leur geste. Le curé Meslier et le curé Jacques Roux, fomentant jacqueries et émeutes, ont montré, à mon sens, la dernière reconversion possible du prêtre sincèrement attaché aux fondements révolutionnaires de la religion. Mais c’est ce que n’ont pas compris les sectateurs de cet œcuménisme contemporain qui va de Rome à Moscou et de la canaille cybernéticienne aux créatures de l’Opus Dei. À l’image de ce nouveau clergé, on devine sans peine ce que sera le dépassement des hérésies.\par

\astermono

\noindent Personne ne conteste au libéralisme la gloire d’avoir répandu les ferments de liberté aux quatre coins du monde. En un sens la liberté de presse, de pensée, de création a du moins l’avantage de dénoncer la duperie du libéralisme ; et n’est-ce pas au fond sa plus belle oraison funèbre ? Car le système est habile, qui emprisonne la liberté au nom de la liberté. L’autonomie des individus se détruit par interférence, la liberté de l’un commence où finit la liberté de l’autre. Ceux qui refusent le principe sont détruits par le fer, ceux qui l’acceptent sont détruits par la justice. Personne n’a les mains sales : on pousse sur un bouton, le couperet de la police et de l’intervention étatique tombe, et c’est bien regrettable. L’État est la mauvaise conscience du libéral, l’instrument d’une répression nécessaire qu’au fond du cœur il désavoue. Pour les affaires courantes, la liberté du capitaliste se charge de rappeler ses limites à la liberté du travailleur. C’est ici que le bon socialiste entre en scène et dénonce l’hypocrisie.\par
Qu’est-ce que le socialisme ? Une façon de sortir le libéralisme de sa contradiction, c’est-à-dire de la sauvegarde et de la destruction simultanées de la liberté individuelle. Empêcher les individus de se nier par interférence, la résolution est louable, mais le socialisme aboutit à une autre solution. Il supprime les interférences sans libérer l’individu ; bien plus, il fond la volonté individuelle dans la médiocrité collective. Seul, il est vrai, le secteur économique fait l’objet de sa réforme, et l’arrivisme, le libéralisme de la vie quotidienne s’accommode assez d’une planification bureaucratique, contrôlant l’ensemble des activités, promotion du militant, rivalités de dirigeants… On empêche l’interférence dans un domaine, on détruit la concurrence économique et la libre entreprise mais la course à la consommation de pouvoir reste la seule forme de liberté autorisée. L’amusante querelle que celle où s’opposent les tenants d’une liberté autolimitative, les libéraux de la production et les libéraux de la consommation !\par
L’ambiguïté du socialisme, la radicalité et son abandon, apparaît parfaitement dans ces deux interventions rapportées l’une et l’autre dans le compte rendu des débats de la I\textsuperscript{ère} Internationale. En 1867, Chémalé rappelle :\par

\begin{quoteblock}
\noindent « le produit s’\emph{échange} contre un produit d’égale valeur, ou bien il y a tromperie, escroquerie, vol. »\end{quoteblock}

\noindent Il s’agit donc, selon Chémalé de rationaliser l’échange, de le rendre équitable. Le socialisme corrige le capitalisme, le rend humain, le planifie, le vide de sa substance (le profit) ; et qui profite de la fin du capitalisme ? Cependant, contemporain de ce socialisme, il en existe un autre. Au congrès de Genève de la même Association internationale des Travailleurs, en 1866, Varlin, le futur communard, déclare :\par

\begin{quoteblock}
\noindent « Tant qu’une entrave empêchera l’\emph{emploi de soi-même}, la liberté n’existera pas. »\end{quoteblock}

\noindent Qui oserait entreprendre aujourd’hui de libérer la liberté contenue dans le socialisme sans lutte de toutes ses forces contre le socialisme ?\par
Faut-il épiloguer encore sur l’abandon, par toutes les variétés de marxisme actuel, du projet de Marx ? En URSS, en Chine, à Cuba, qu’y a-t-il de commun avec la construction de l’homme total ? Parce que la misère où se nourrissait la volonté révolutionnaire d’un dépassement et d’un changement radical s’est atténuée, une nouvelle misère est venue, faite de renoncements et de compromissions. Abandon de la misère et misère de l’abandon. N’est-ce pas le sentiment d’avoir laissé son projet initial se fragmenter et se réaliser par morceau qui justifie la boutade désabusée de Marx : « Moi, je ne suis pas marxiste » ?\par
Et même le fascisme immonde est une volonté de vivre niée, retournée, la chair d’un ongle incarné. Une volonté de vivre devenue volonté de puissance, une volonté de puissance devenue volonté d’obéissance passive, une volonté d’obéissance passive devenue volonté de mort ; car céder d’un pouce sur le qualitatif, c’est céder sur la totalité du qualitatif.\par
Brûler le fascisme, soit, mais que la même flamme embrase les idéologies, toutes les idéologies sans exception, et leurs valets.\par

\astermono

\noindent Partout la force poétique est, par la force des choses, abandonnée ou poussée à l’abandon. L’homme isolé abandonne sa volonté individuelle, sa subjectivité, pour briser l’isolement : il y gagne l’illusion communautaire et un goût plus aigu de la mort. L’abandon est le premier pas vers la récupération par les mécanismes du pouvoir.\par
Pas une technique, pas une pensée dont le premier mouvement n’obéisse à une volonté de vivre ; pas une technique, pas une pensée officiellement accréditée qui n’incite à mourir. Les traces de l’abandon sont les signes d’une histoire encore mal connue des hommes. Les étudier, c’est déjà forger les armes du dépassement total. Où se trouve le noyau radical, le qualitatif ? Telle est la question qui doit dissoudre les habitudes de pensée et de vie ; telle est la question qui entre dans la stratégie du dépassement, dans la construction de nouveaux réseaux de radicalité. Ceci vaut pour la philosophie : l’ontologie trahit l’abandon de l’être en devenir. Pour la psychanalyse : technique de libération, elle « libère » surtout du besoin d’attaquer l’organisation sociale. Pour les rêves et les désirs volés, violés, falsifiés par le conditionnement. Pour la radicalité des actes spontanés d’un homme, et que contredit la plupart du temps ce qu’il pense de lui-même et du monde. Pour le jeu : distribué en catégories de jeux licites – de la roulette à la guerre, en passant par le lynch – il tient quitte de jouer authentiquement sur les moments de la vie quotidienne. Pour l’amour, inséparable de la révolution et si pauvrement dépris du plaisir de donner…\par
Ôtez le qualitatif, il reste le désespoir ; toutes les formes de désespoir disponibles pour une organisation de la mort des hommes, pour le pouvoir hiérarchisé : réformisme, fascisme, apolitisme crétin, médiocratie, activisme et passivité, boy-scoutisme et masturbation idéologique. Un ami de Joyce racontait :\par

\begin{quoteblock}
\noindent « Je ne me souviens pas qu’une seule fois en toutes ces années, Joyce ait dit un mot des événements publics, proféré le nom de Poincaré, de Roosevelt, de Valera, de Staline, émis une allusion à Genève ou à Locarno, à l’Abyssinie, à l’Espagne, à la Chine, au Japon, à l’affaire Prince, à Violette Nozière… »\end{quoteblock}

\noindent À vrai dire, que pouvait-il ajouter à \emph{Ulysses}, à \emph{Finnegans Wake} ? Après le \emph{Das Kapital} de la créativité individuelle, il importait que les Leopold Bloom du monde entier s’unissent pour se défaire de leur pauvre survie, et pour introduire dans la réalité vécue de leur existence la richesse et la variété de leur « monologue intérieur ». Joyce ne faisait pas le coup de feu avec Durruti, il ne s’était trouvé ni au côté des Asturiens, ni au côté des ouvriers viennois ; du moins avait-il la décence de ne pas commenter des informations, à l’anonymat desquelles il abandonnait \emph{Ulysses} – ce monument de culture, comme a dit un critique – en s’abandonnant, lui Joyce, l’homme de la subjectivité totale. Sur la veulerie de l’homme de lettres, c’est \emph{Ulysses} qui témoigne. Et contre la veulerie de l’abandon, c’est toujours le moment radical « oublié » qui témoigne. Ainsi révolutions et contre-révolutions se succèdent en l’espace de vingt-quatre heures, en l’espace d’un jour, fût-il le plus dénué d’événements. La conscience du geste radical et de son abandon s’affine et s’étend sans cesse. Comment en irait-il autrement ? La survie est aujourd’hui le non-dépassement devenu invivable.
\subsection[{3. L’homme du ressentiment}]{\textsc{3.} L’homme du ressentiment}
\noindent Plus le pouvoir se dispense en fragments consommables, plus se restreint le lieu de la survie ; jusqu’à ce monde de reptation où le plaisir, l’effort de libération et l’agonie s’expriment par le même soubresaut. La pensée basse et la vue courte ont depuis longtemps marqué l’appartenance de la bourgeoisie à une civilisation de troglodytes en progrès, à une civilisation de la survie qui découvre aujourd’hui sa finalité dans le confort des abris antiatomiques. Sa grandeur fut une grandeur empruntée, conquise moins sur l’ennemi qu’à son contact ; une ombre de la vertu féodale, de Dieu, de la Nature… Sitôt ces obstacles ôtés à son emprise immédiate, la bourgeoisie s’est trouvée réduite à se contester sur des détails ; à se porter des coups qui ne mettent pas son existence en péril. Flaubert, raillant le bourgeois, en appelle aux armes contre la Commune.\par
La noblesse rendait la bourgeoisie agressive, le prolétariat l’accule sur des positions de défense. Qu’est-ce que le prolétariat pour elle ? Même pas un adversaire, une mauvaise conscience tout au plus, et qu’elle s’efforce de dissimuler. Repliée sur elle-même, offrant le moins de surface vulnérable, proclamant la seule légitimité des réformes, elle a fait de l’envie cauteleuse et du ressentiment l’étoffe habituelle de ses révolutions parcellaires.\par
J’ai déjà dit qu’à mon sens aucune insurrection n’était parcellaire dans sa volonté initiale, qu’elle le devenait sitôt qu’à la poésie des agitateurs et meneurs de jeu se substituait l’autorité de dirigeants. L’homme du ressentiment est la version officielle du révolutionnaire : un homme privé de la conscience du dépassement possible ; un homme à qui échappe la nécessité d’un renversement de perspective et qui, rongé par l’envie, la haine et le désespoir, s’acharne à détruire par l’envie, la haine et le désespoir un monde si bien fait pour le brimer. Un homme isolé. Un réformiste coincé entre le refus global du pouvoir et son acceptation absolue. Refusant la hiérarchie par dépit de ne s’y trouver installé, un tel homme est tout préparé pour servir dans sa révolte les desseins de ses maîtres improvisés. Le pouvoir n’a pas de meilleur soutien que l’arrivisme déçu ; c’est pourquoi il s’emploie à consoler les vaincus de la course aux honneurs, il leur donne ses privilégiés à haïr.\par
En deçà du renversement de perspective, donc, la haine du pouvoir est encore une façon de lui reconnaître la primauté. Celui qui passe sous une échelle afin de prouver son mépris des superstitions leur fait trop d’honneurs en leur subordonnant sa liberté d’action. La haine obsessionnelle et la soif insatiable des charges autoritaires usent et appauvrissent sinon pareillement – car il y a plus d’humanité à lutter contre le pouvoir qu’à s’y prostituer – du moins dans une égale mesure. Il y a un monde entre lutter pour vivre et lutter pour ne pas mourir. Les révoltes de survie s’étalonnent sur les normes de la mort. C’est pourquoi elles exigent avant tout l’abnégation des militants, leurs renoncements \emph{a priori} au vouloir-vivre pour lequel il n’est personne qui ne lutte \emph{en fait}.\par
Le révolté sans autre horizon que le mur des contraintes risque de s’y briser la tête ou de le défendre un jour avec une bêtise opiniâtre. Car s’appréhender dans la perspective des contraintes, c’est toujours regarder dans le sens voulu par le pouvoir, qu’on le repousse ou qu’on accepte. Voici l’homme au point zéro, couvert de vermine, comme dit Rozanov. Limité de toutes parts, il se ferme à toute intrusion, il veille sur soi, jalousement, sans s’apercevoir qu’il est devenu stérile ; un cimetière en quelque sorte. Il introvertit sa propre existence. Il fait sienne l’impuissance du pouvoir pour lutter contre lui. Il pousse le \emph{fair-play} jusque-là. À ce prix, il lui coûte plus d’être pur, de jouer la pureté. Comme les gens les plus voués aux compromissions se font toujours une gloire incommensurable d’être restés intègres sur un ou deux points précis ! Le refus d’un grade à l’armée, la distribution de tracts dans une grève, une altercation avec les flics… s’harmonisent toujours avec le militantisme le plus obtus dans les partis communistes et leurs séquelles.\par
Ou encore, l’homme au point zéro se découvre un monde à conquérir, il a besoin d’un espace vital, d’une ruine plus vaste qui l’englobe. Le refus du pouvoir se confond vite avec le refus de ce dont le pouvoir s’approprie, le propre moi du révolté par exemple. À se définir de façon antagoniste aux contraintes et aux mensonges, il arrive que les contraintes et le mensonge entrent dans l’esprit comme une part caricaturale de révolte, et la plupart du temps, l’ironie n’est plus là pour aérer un peu. Aucun lien n’est plus difficile à rompre que celui où l’individu se détient lui-même par l’obscurcissement du refus. S’il se sert de la force de la liberté au profit de la non-liberté, il accroît par l’effort conjugué la force de la non-liberté, qui le rend esclave. Or il se peut que rien ne ressemble plus à la non-liberté que l’effort vers la liberté, mais la non-liberté a ceci de particulier, une fois achetée, elle n’a plus de valeur bien qu’on la paie aussi cher que la liberté.\par
Le resserrement des murs rend l’atmosphère irrespirable ; et plus les gens s’efforcent de respirer dans ces conditions, plus l’air est irrespirable. L’ambiguïté des signes de vie et de liberté, passant du positif au négatif selon les nécessaires déterminations de l’oppression globale, généralise la confusion où l’on défait d’une main ce que l’on fait de l’autre. L’incapacité de se saisir soi-même incite à saisir les autres au départ de leurs représentations négatives, de leurs rôles ; à les jauger comme des objets. Les vieilles filles, les bureaucrates, et tous ceux qui ont réussi leur survie ne connaissent sentimentalement d’autres raisons d’exister. Faut-il le dire, le pouvoir fonde sur ce malaise partagé ses meilleurs espoirs de récupération. Et plus la confusion mentale est grande, plus la récupération est aisée.\par
La myopie et le voyeurisme définissent inséparablement l’adaptation d’un homme à la mesquinerie sociale de notre époque. Contempler le monde par le trou de la serrure ! À défaut des premiers rôles, il réclame les premières loges au spectacle. Il a besoin d’évidences minuscules à se mettre sous la dent ; que les politiciens sont des salauds, que de Gaulle est un grand homme et la Chine la patrie des travailleurs. Il veut un adversaire vivant à déchirer, des mains de dignitaires à révérer ; pas un système. Comme on comprend le succès de représentations aussi grossières que le Juif ignoble, le nègre voleur, les deux cents familles. L’ennemi avait un visage et du même coup les traits de la foule se modelaient sur le visage, admirable celui-ci, du défenseur, du chef, du leader.\par
L’homme du ressentiment est disponible mais l’emploi de cette disponibilité, passe obligatoirement par une prise de conscience larvée : l’homme du ressentiment devient nihiliste. S’il ne tue pas les organisateurs de son ennui, les gens qui lui apparaissent comme tels en gros plan, dirigeants, spécialistes, propagateurs d’idéologies… il tuera au nom d’une autorité, au nom d’une raison d’État, au nom de la consommation idéologique. Et si l’état des choses n’incite pas à la violence et à l’explosion brutale, il continuera dans la crispation monotone du mécontentement à se démener parmi les rôles, à répandre son conformisme en dents de scie, applaudissant indifféremment à la révolte et à la répression, sensible à la seule et incurable confusion.
\subsection[{4. Le nihiliste}]{\textsc{4.} Le nihiliste}
\noindent Qu’est-ce que le nihilisme ? Rozanov répond parfaitement à la question quand il écrit :\par

\begin{quoteblock}
\noindent « La représentation est terminée. Le public se lève. Il est temps d’enfiler son manteau et de rentrer à la maison. On se retourne : plus de manteau ni de maison. »\end{quoteblock}

\noindent Dès qu’un système mythique entre en contradiction avec la réalité économico-sociale, un espace vide s’ouvre entre la façon de vivre des gens et l’explication dominante du monde, soudain inadéquate, loin en retrait. Un tourbillon se creuse, les valeurs traditionnelles s’y engouffrent et se brisent. Privée de ses prétextes et de ses justifications, dépouillée de toute illusion, la faiblesse des hommes apparaît nue, désarmée. Mais comme le mythe, qui protège et dissimule une telle faiblesse, est aussi cause de cette impuissance, son éclatement ouvre une voie nouvelle aux possibles. Sa disparition laisse le champ libre à la créativité et à l’énergie, longtemps détournées de l’authenticité vécue par la transcendance et par l’abstraction. Entre la fin de la philosophie antique et l’érection du mythe chrétien, la période d’interrègne connaît une floraison extraordinaire de pensées et d’actions toutes plus riches les unes que les autres. Récupérant les unes, étouffant les autres, c’est sur leur cadavre que Rome posera sa pierre. Et plus tard, au XVI\textsuperscript{e} siècle, l’effondrement du mythe chrétien déclenchera de même une frénésie d’expérimentations et de recherches. Mais l’analogie diffère cette fois sur un point : après 1789, la reconstitution d’un mythe est devenue rigoureusement impossible.\par
Si le christianisme désamorça le nihilisme de certaines sectes gnostiques et s’en fit un revêtement de protection, le nihilisme né de la révolution bourgeoise est, lui, un nihilisme de fait. Irrécupérable. La réalité de l’échange, comme je l’ai montré, domine toute tentative de dissimulation, tous les artifices de l’illusion. Jusqu’à son abolition, le spectacle ne sera jamais que le spectacle du nihilisme. La vanité du monde dont le Pascal des Pensées souhaitait propager la conscience pour la plus grande gloire de Dieu, la voici propagée par la réalité historique ; et en l’absence de Dieu, précisément victime de l’éclatement du mythe. Le nihilisme a tout vaincu, y compris Dieu.\par
Depuis un siècle et demi, la part la plus lucide de l’art et de la vie est le fruit d’investigations libres dans le champ des valeurs abolies. La raison passionnelle de Sade, le sarcasme de Kierkegaard, l’ironie vacillante de Nietzsche, la violence de Maldoror, la froideur mallarméenne, l’Umour de Jarry, le négativisme de Dada, voilà les forces qui se sont déployées sans limites pour introduire dans la conscience des hommes un peu de la moisissure des valeurs pourrissantes. Et, avec elle, l’espoir d’un dépassement total, d’un renversement de perspective.\par
Paradoxe.\par
1° Aux grands propagateurs du nihilisme, il a manqué une arme essentielle : le sens de la réalité historique, le sens de cette réalité qui était celle de la décomposition, de l’effritement du parcellaire.\par
2° La conscience aiguë du mouvement dissolvant de l’histoire à l’époque bourgeoise a toujours fait cruellement défaut aux meilleurs praticiens de l’histoire. Marx renonce à analyser le Romantisme et le phénomène artistique en général. Lénine ignore presque systématiquement l’importance de la vie quotidienne, les futuristes, Maïakovsky et les dadaïstes.\par
La conscience de la montée nihiliste et la conscience du devenir historique paraissent étrangement décalées. Dans l’intervalle laissé par ce décalage défile la foule des liquidateurs passifs, aplanissant du poids de sa bêtise les valeurs mêmes au nom desquelles elle manifeste. Bureaucrates, communistes, brutes fascistes, idéologues, politiciens véreux, écrivains sous-joyciens, penseurs néo-dadaïstes, prêtres du parcellaire, tous travaillent pour le grand Rien au nom de l’ordre familial, administratif, moral, national, cybernétique révolutionnaire (!). Tant que l’histoire n’avait pas marché assez loin, peut-être le nihilisme ne pouvait-il prendre l’allure d’une vérité générale, d’une banalité de base. Aujourd’hui, l’histoire a marché. Le nihilisme est lui-même sa propre matière, la voie du feu vers la cendre. La réification imprime le vide dans la réalité quotidienne. Nourrissant sous la vieille étiquette du \emph{moderne} la fabrication intensive de valeurs consommables et « futurisées », le passé des valeurs anciennes aujourd’hui ruinées nous rejette inévitablement vers un présent à \emph{construire}, c’est-à-dire vers le dépassement du nihilisme. Dans la conscience désespérée de la jeune génération, \emph{le mouvement dissolvant et le mouvement réalisant de l’histoire} se réconcilient lentement. Le nihilisme et le dépassement se rejoignent, c’est pourquoi le dépassement sera total. C’est là sans aucun doute la seule richesse de la société de l’abondance.\par
Quand l’homme du ressentiment prend conscience de l’irrécouvrable manque à gagner de la survie, il devient nihiliste. Il saisit l’impossibilité de vivre à un degré mortel pour la survie elle-même. L’angoisse nihiliste est invivable ; le vide absolu désintègre. Le tourbillon passé-futur met le présent au point zéro. C’est de ce point mort que partent les deux voies du nihilisme, ce que j’appellerai nihilisme passif et nihilisme actif.\par

\astermono

\noindent La passivité nihiliste unit sous le signe de la compromission et de l’indifférence la conscience des valeurs abolies et le choix délibéré, souvent intéressé, de l’une ou l’autre de ces valeurs démonétisées que l’on se propose de défendre envers et contre tout, « gratuitement », pour l’Art. Rien n’est vrai, donc quelques gestes sont honorables. Maurassiens farfelus, pataphysiciens, nationalistes, esthètes de l’acte gratuit, mouchards, OAS, pop-artistes, ce joli monde applique à sa façon le credo quia absurdum : on n’y croit pas, on le fait quand même, on finit par y prendre goût. Le nihilisme passif est un bond vers le conformisme.\par
D’ailleurs le nihilisme n’est jamais qu’un passage, un lieu d’ambiguïté, une oscillation dont l’un des pôles mène à la soumission servile et l’autre à l’insurrection permanente. Entre les deux, c’est le \emph{no man’s land}, le terrain vague du suicidé ou du tueur solitaire, de ce criminel dont Bettina dit fort justement qu’il est le crime de L’État. Jack l’Éventreur est de toute éternité insaisissable. Insaisissable par les mécanismes du pouvoir hiérarchisé, insaisissable par la volonté révolutionnaire. Un en-soi en quelque sorte ! Il gravite autour d’un point zéro où la destruction, cessant de prolonger la destruction opérée par le pouvoir, la prévient plutôt, la devance, l’accélère et fait, par trop de précipitation, voler en éclats la machine de \emph{La Colonie pénitentiaire}. L’être maldororien porte la fonction dissolvante de l’organisation sociale à son paroxysme ; jusqu’à l’autodestruction. L’absolu refus du social par l’individu réplique ici à l’absolu refus de l’individu par le social. N’est-ce pas là le moment fixe, le point d’équilibre du renversement de perspective, l’endroit précis où le mouvement n’existe pas, ni la dialectique, ni le temps ? Midi et éternité du grand refus. En deçà, les pogroms ; au-delà, la nouvelle innocence. Le sang des Juifs ou le sang des flics.\par

\astermono

\noindent Le nihilisme actif joint à la conscience de la désagrégation le désir d’en dénoncer les causes en précipitant le mouvement. Le désordre fomenté n’est que le reflet du désordre régnant sur le monde. Le nihilisme actif est prérévolutionnaire ; le nihilisme passif, contre-révolutionnaire. Et il arrive souvent que le commun des hommes se sente entraîné vers l’une et l’autre attitude par une perpétuelle oscillation, par une valse-hésitation à la fois dramatique et bouffonne. Comme ce soldat Rouge, – dont parle je ne sais quel auteur soviétique, Victor Chlovsky peut-être, – qui ne chargeait jamais sans crier « Vive le Tsar ! ». Mais il faut bien que les circonstances cautionnent tôt ou tard, fermant soudain la barrière tandis que l’on se trouve d’un côté ou de l’autre.\par

\astermono

\noindent C’est toujours sur le contre-pied du monde officiel que l’on apprend à danser pour soi. Encore faut-il aller jusqu’au bout de ses exigences, ne pas abandonner sa radicalité au premier tournant. Le renouvellement essoufflé des motivations auquel se condamne la course au consommable tire habilement profit de l’insolite, du bizarre, du choquant. L’humour noir et l’atroce entrent dans la salade publicitaire. Une certaine façon de danser dans le non-conformisme participe elle aussi des valeurs dominantes. La conscience du pourrissement des valeurs trouve sa place dans la stratégie de la vente. La décomposition est une valeur marchande. La nullité bruyamment affirmée s’achète ; qu’il s’agisse d’idées ou d’objets. Quant à la salière Kennedy, avec les trous percés aux points d’impact des balles meurtrières, elle démontrerait, s’il était nécessaire, avec quelle facilité une plaisanterie qui aurait en son temps fait la joie d’Émile Ponget et de son Père Peinard nourrit aujourd’hui la rentabilité.\par
Le mouvement Dada a poussé la conscience du pourrissement à son plus haut degré. Dada contenait vraiment les germes du dépassement du nihilisme, mais il les a laissés pourrir à leur tour. Toute l’équivoque surréaliste vient d’une juste critique émise inopportunément. Qu’est-ce à dire ? Ceci : le surréalisme critique à bon droit le dépassement raté par Dada mais lorsqu’il entreprend, lui, de dépasser Dada, il le fait sans repartir du nihilisme originel, sans prendre appui sur Dada-anti-Dada, sans l’accrocher à l’histoire. Et comme l’histoire a été le cauchemar dont ne s’éveillèrent jamais les surréalistes, désarmés devant le parti communiste, pris de court par la guerre d’Espagne, grognant toujours mais suivant la gauche en chiens fidèles !\par
Un certain romantisme avait déjà prouvé, sans que Marx ni Engels ne songent à s’en inquiéter, que l’art, c’est-à-dire le pouls de la culture et de la société, révèle en premier l’état de décomposition des valeurs. Un siècle plus tard, tandis que Lénine jugeait la question frivole, les dadaïstes voyaient dans l’abcès artistique le symptôme d’un cancer généralisé, d’une maladie de la société entière. Le déplaisant dans l’art ne reflète que l’art du déplaisir institué partout comme la loi du pouvoir. Voilà ce que les dadaïstes de 1916 avaient établi clairement. L’au-delà d’une telle analyse renvoyait directement à la lutte armée. Les larves néo-dadaïstes du Pop Art qui prolifèrent aujourd’hui sur le fumier de la consommation ont trouvé mieux à faire.\par
Travaillant, avec en somme plus de conséquence que Freud, à se guérir et à guérir leurs contemporains du déplaisir à vivre, les dadaïstes ont édifié le premier laboratoire d’assainissement de la vie quotidienne. Le geste allait bien au-delà de la pensée.\par

\begin{quoteblock}
\noindent « Ce qui comptait, a dit le peintre Grosz, c’était travailler pour ainsi dire dans l’obscurité la plus profonde. Nous ne savions pas ce que nous faisions. »\end{quoteblock}

\noindent Le groupe Dada était l’entonnoir où s’engouffraient les innombrables banalités, la notable quantité d’importance nulle du monde. Par l’autre bout, tout sortait transformé, original, neuf. Les êtres et les objets restaient les mêmes, et cependant, tout changeait de sens et de signe. Le renversement de perspective s’amorçait dans la magie du vécu retrouvé. Le détournement, qui est la tactique du renversement, bouleversait le cadre immuable du vieux monde. \emph{La poésie faite par tous} prenait dans ce bouleversement son véritable sens, bien éloigné de l’esprit littéraire auquel les surréalistes finirent par succomber piteusement.\par
La faiblesse initiale de Dada, il convient de la chercher dans son incroyable humilité. Pitre sérieux comme un pape, le Tzara qui, chaque matin, dit-on, répétait la phrase de Descartes « Je ne veux même pas savoir qu’il y eut des hommes avant moi », ce Tzara est bien celui qui, dédaignant des hommes comme Ravachol, Bonnot et les compagnons de Makhno, rejoindrait plus tard le troupeau de Staline. Si le mouvement Dada s’est disloqué devant l’impossible dépassement, c’est qu’il lui manqua l’instinct de rechercher dans l’histoire les diverses expériences de dépassement possible, les moments où les masses en révolte prennent leur destinée en main\par
Le premier abandon est toujours terrible. Du surréalisme au néo-dadaïsme, l’erreur initiale se multiplie et se répercute sans fin. Le surréalisme en appelle au passé, mais de quelle façon ? Sa volonté de corriger rend l’erreur plus troublante encore quand, faisant choix d’individualités parfaitement admirables (Sade, Fourier, Lautréamont…), il en écrit tant et si bien qu’il obtient pour ses protégés une mention honorable dans le panthéon des programmes scolaires. Une promotion littéraire, pareille à la promotion que les néo-dadaïstes décrochent pour leurs ancêtres dans l’actuel spectacle de la décomposition.\par

\astermono

\noindent S’il existe aujourd’hui un phénomène international assez semblable au mouvement Dada, il faut le reconnaître dans les plus belles manifestations de blousons noirs. Même mépris de l’art et des valeurs bourgeoises, même refus des idéologies, même volonté de vivre. Même ignorance de l’histoire, même révolte rudimentaire, même absence de tactique\par
Au nihiliste, il manque la conscience du nihilisme des autres ; et le nihilisme des autres s’inscrit désormais dans la réalité historique contemporaine ; il manque au nihilisme la conscience possible du dépassement possible. Cependant, cette survie où l’on parle tant de progrès parce que l’on désespère de progresser est aussi le fruit de l’histoire, elle procède de tous les abandons de l’humain qui jalonnent les siècles. J’ose dire que l’histoire de la survie est le mouvement historique qui va défaire l’histoire. Car la conscience claire de la survie et de ses conditions insupportables fusionne avec la conscience des abandons successifs, et conséquemment avec le vrai désir de reprendre le mouvement de dépassement partout \emph{dans l’espace et le temps}, où il a été prématurément interrompu. Le dépassement, c’est-à-dire la révolution de la vie quotidienne, va consister à reprendre les noyaux de radicalité abandonnés et à les valoriser avec la violence inouïe du ressentiment. L’explosion en chaîne de la créativité clandestine doit renverser la perspective du pouvoir. \emph{Les nihilistes sont, en dernier ressort, nos seuls alliés}. Ils vivent dans le désespoir du non-dépassement ? Une théorie cohérente peut, leur démontrant la fausseté de leur vue, mettre au service de leur volonté de vivre le potentiel énergétique de leurs rancœurs accumulées. Avec ces deux notions fondamentales – l’abandon du radical et la conscience historique de la décomposition – il n’est personne qui ne puisse mener à bien le combat pour la vie quotidienne et la transformation radicale du monde. Nihilistes, aurait dit Sade, encore un effort si vous voulez être révolutionnaires !
\section[{XIX. Le renversement de perspective}]{XIX. Le renversement de perspective}\renewcommand{\leftmark}{XIX. Le renversement de perspective}


\begin{argument}\noindent La lumière du pouvoir assombrit. Les yeux de l’illusion communautaire sont les trous du masque auxquels ne s’adaptent pas les yeux de la subjectivité individuelle. Il faut que le point de vue individuel l’emporte sur le point de vue de la fausse participation collective. Dans un esprit de totalité, aborder le social avec les armes de la subjectivité, tout reconstruire au départ de soi. Le renversement de perspective est la positivité du négatif, le fruit qui va faire éclater la bogue du Vieux Monde (1-2).
\end{argument}

\subsection[{1. Pierres blanches, pierres noires}]{\textsc{1.} Pierres blanches, pierres noires}
\noindent Comme on demandait à M. Keuner ce qu’il fallait entendre au juste par « renversement de perspective », il raconta l’anecdote suivante :\par
Deux frères très attachés l’un à l’autre avaient une curieuse manie. Ils indiquaient d’une pierre les événements de la journée, une pierre blanche pour les moments heureux, une pierre noire pour les instants de malheur et les déplaisirs. Or, le soir venu, lorsqu’ils comparaient le contenu de leur jarre, l’un ne trouvait que pierres blanches, l’autre que pierres noires. Intrigués par une telle constance dans la façon de vivre aussi différemment le même sort, ils furent de commun accord prendre conseil auprès d’un homme renommé pour la sagesse de ses paroles. « Vous ne vous parlez pas assez, dit le sage. Que chacun motive les raisons de son choix, qu’il en recherche les causes. » Ainsi firent-ils dès lors.\par
Comme ils constatèrent vite, le premier restait fidèle aux pierres blanches et le second aux pierres noires, mais, dans l’une et l’autre jarre, le nombre de pierres avait diminué. Au lieu d’une trentaine, on n’en comptait plus guère que sept ou huit. Peu de temps s’était écoulé lorsque le sage vit revenir les deux frères. Leurs traits portaient la marque d’une grande tristesse.\par
« Il n’y a pas si longtemps, dit l’un, ma jarre s’emplissait de cailloux couleur de nuit, le désespoir m’habitait en permanence, j’en étais réduit, je l’avoue, à vivre par inertie. Maintenant, j’y dépose rarement plus de huit pierres, mais ce que représentent ces huit signes de misère m’est à ce point intolérable que je ne puis vivre désormais dans pareil état. »\par
Et l’autre : « Pour moi, j’amoncelais chaque jour des pierres blanches. Aujourd’hui, j’en compte seulement sept ou huit, mais celles-là me fascinent tant qu’il ne m’arrive d’évoquer ces heureux instants sans désirer aussitôt les revivre plus intensément, et pour tout dire, éternellement. Ce désir me tourmente. »\par
Le sage souriait en les écoutant. « Allons, tout va bien, les choses prennent tournure. Persévérez. Encore un mot. À l’occasion, posez-vous la question : pourquoi le jeu de la jarre et des pierres nous passionne-t-il de la sorte ? »\par
Quand les deux frères rencontrèrent à nouveau le sage, ce fut pour déclarer : « Nous nous sommes posé la question ; pas de réponse. Alors nous l’avons posé à tout le village. Vois l’animation qui y règne. Le soir, accroupis devant leur maison, des familles entières discutent de pierres blanches et de pierres noires. Seuls les chefs et les notables se tiennent à l’écart. Noire ou blanche, une pierre est une pierre et toutes se valent, disent-ils en se moquant. »\par
Le vieillard ne dissimulait pas son contentement. « L’affaire suit son cours comme prévu. Ne vous inquiétez pas. Bientôt la question ne se posera plus ; elle est devenue sans importance, et peut-être un jour douterez-vous de l’avoir posée. »\par
Peu après, les prévisions du vieillard furent confirmées de la manière suivante : une grande joie s’était emparée des gens du village ; à l’aube d’une nuit agitée, le soleil éclaira, fichées sur les pieux acérés d’une palissade, les têtes fraîchement coupées des notables et des chefs.
\subsection[{2. Géométrie du pouvoir}]{\textsc{2.} Géométrie du pouvoir}
\noindent Le monde a toujours été une géométrie. Sous quel angle et dans quelle perspective les hommes doivent se voir, se parler, se représenter, les dieux des époques unitaires en ont d’abord décidé souverainement. Puis, les hommes, les hommes de la bourgeoisie, leur ont joué ce vilain tour : ils les ont mis en perspective, il les ont rangés dans un devenir historique où ils naissaient, se développaient, mouraient. L’histoire a été le crépuscule des dieux.\par
Historicisé, Dieu se confond avec la dialectique de sa matérialité, avec la dialectique du maître et de l’esclave ; l’histoire de la lutte des classes, l’histoire du pouvoir social hiérarchisé. En un sens, donc, la bourgeoisie amorce un renversement de perspective, mais pour le limiter aussitôt à l’apparence, Dieu aboli, ses poutres de soutènement se dressent encore vers le ciel vide. Et comme si l’explosion dans la cathédrale du sacré se propageait en très lentes ondes de choc, l’effritement du plâtras mythique s’achève aujourd’hui, près de deux siècles après l’attentat, dans l’émiettement du spectacle. La bourgeoisie n’est qu’une phase du dynamitage de Dieu, ce Dieu qui va maintenant disparaître radicalement, disparaître jusqu’à effacer les traces de ses origines matérielles : la domination de l’homme par l’homme.\par
Les mécanismes économiques, dont la bourgeoisie possédait partiellement le contrôle et la force, révélaient la matérialité du pouvoir, le tenant quitte du fantôme divin. Mais à quel prix ? Tandis que Dieu offrait dans sa grande négation de l’humain une sorte de refuge où les hommes de foi avaient paradoxalement licence, en opposant le pouvoir absolu de Dieu au pouvoir « usurpé » des prêtres et des chefs, de s’affirmer contre l’autorité temporelle, comme firent si souvent les mystiques, c’est aujourd’hui le pouvoir qui s’approche des hommes, leur fait ses avances, se rend consommable. Il pèse de plus en plus lourdement, ramène l’espace de vie à la simple survie, comprime le temps en une épaisseur de « rôle ». Pour recourir à un schématisme facile, on pourrait comparer le pouvoir à un angle. Un angle aigu à l’origine, le sommet perdu dans les profondeurs du ciel, puis s’élargissant peu à peu tandis que le sommet s’abaisse devient visible, descend encore jusqu’à s’aplatir, étendre ses côtés en une ligne droite et se confondre avec une succession de points équivalents et sans force. Au-delà de cette ligne, qui est celle du nihilisme, commence une perspective nouvelle, non le reflet de l’ancienne, non son involution. Plutôt un ensemble de perspective individuelles harmonisées, n’entrant jamais en conflit, mais construisant le monde selon les principes de cohérence et de collectivité. La totalité de ces angles, tous différents, s’ouvrent néanmoins dans la même direction, la volonté individuelle se confondant désormais avec la volonté collective.\par
Le conditionnement a pour fonction de placer et de déplacer chacun le long de l’échelle hiérarchique. Le renversement de perspective implique une sorte d’anticonditionnement, non pas un conditionnement d’un type nouveau, mais une tactique ludique : \emph{le détournement}.\par
Le renversement de perspective remplace la connaissance par la praxis, l’espérance par la liberté, la médiation par la volonté de l’immédiat. Il consacre le triomphe d’un ensemble de relations humaines fondées sur trois pôles inséparables : la \emph{participation}, la \emph{communication}, la \emph{réalisation}.\par
Renverser la perspective, c’est cesser de voir avec les yeux de la communauté, de l’idéologie, de la famille, des autres. C’est se saisir soi-même solidement, se choisir comme point de départ et comme centre. Tout fonder sur la subjectivité et suivre sa volonté subjective d’être tout. Dans la ligne de mire de mon insatiable désir de vivre, la totalité du pouvoir n’est qu’une cible particulière dans un horizon plus vaste. Son déploiement de force ne m’obstrue pas la vue, je le repère, j’en estime le danger, j’étudie les parades. Si pauvre qu’elle soit, ma créativité m’est un guide plus sûr que toutes les connaissances acquises par contrainte. Dans la nuit du pouvoir, sa petite lueur tient à distance les forces hostiles : conditionnement culturel, spécialisations de tout ordre, \emph{Weltanschauungen} inévitablement totalitaires. Chacun détient ainsi l’arme absolue. Encore faut-il, comme il en va de certains charmes, s’en servir à bon escient. L’aborde-t-on par le biais du mensonge et de l’oppression, à rebours, elle n’est plus qu’une lamentable bouffonnerie : une consécration artistique. Les gestes qui détruisent le pouvoir et les gestes qui construisent la libre volonté individuelle sont les mêmes, mais leur portée est différente ; comme en stratégie, la préparation de la défense diffère évidemment de la préparation de l’offensive.\par
Nous n’avons pas choisi le renversement de perspective par je ne sais quel volontarisme, c’est lui qui nous a choisis. Engagés comme nous le sommes dans la phase historique du RIEN, le pas suivant ne peut être qu’un changement du TOUT. La conscience d’une révolution totale, de sa nécessité, est notre dernière façon d’être historique, notre dernière chance de défaire l’histoire dans certaines conditions. Le jeu où nous entrons est le jeu de notre créativité. Ses règles s’opposent radicalement aux règles et aux lois qui régissent notre société. C’est un jeu de qui-perd-gagne : ce qui est tu est plus important que ce qui est dit, ce qui est vécu, plus important que ce qui est représenté sur le plan des apparences. Ce jeu, il faut le jouer jusqu’au bout. Celui qui a ressenti l’oppression jusqu’à ce que ses os ne la supportent plus, comment ne se jetterait-il pas vers la volonté de \emph{vivre sans réserve}, comme vers son dernier recours ? Malheur à celui qui abandonne en chemin sa violence et ses exigences radicales. Les vérités tuées deviennent vénéneuses, a dit Nietzsche. Si nous ne renversons pas la perspective, c’est la perspective du pouvoir qui achèvera de nous tourner définitivement contre nous-mêmes. Le fascisme allemand est né dans le sang de Spartakus. Dans chaque renoncement quotidien, la réaction ne prépare rien d’autre que notre mort totale.
\section[{XX. Créativité, spontanéité et poésie}]{XX. Créativité, spontanéité et poésie}\renewcommand{\leftmark}{XX. Créativité, spontanéité et poésie}


\begin{argument}\noindent Les hommes vivent en état de créativité vingt-quatre heures sur vingt-quatre. Percé à jour, l’usage combinatoire que les mécanismes de domination font de la liberté renvoie par contrecoup à la conception d’une liberté vécue, indissociable de la créativité individuelle. L’invitation à produire, à consommer, à organiser, échoue désormais à récupérer la passion de créer, où va se dissoudre la conscience des contraintes (1). – La spontanéité est le mode d’être de la créativité, non pas un état isolé, mais l’expérience immédiate de la subjectivité. La spontanéité concrétise la passion créatrice, elle amorce sa réalisation pratique, elle rend donc possible la poésie, la volonté de changer le monde selon la subjectivité radicale (2). – Le qualitatif est la présence attestée de la spontanéité créatrice, une communication directe de l’essentiel, la chance offerte à la poésie. Il est un condensé de possibles, un multiplicateur de connaissances et d’efficacité, le mode d’emploi de l’intelligence ; son propre critère. Le choc qualitatif provoque une réaction en chaîne observable dans tous les moments révolutionnaires ; il faut susciter une telle réaction par le scandale positif de la créativité libre et totale (3). – La poésie est l’organisation de la spontanéité créative en tant qu’elle la prolonge dans le monde. La poésie est l’acte qui engendre des réalités nouvelles. Elle est l’accomplissement de la théorie radicale, le geste révolutionnaire par excellence.
\end{argument}

\subsection[{1. Créativité}]{\textsc{1.} Créativité}
\noindent Dans ce monde fractionnaire dont le pouvoir social hiérarchisé fut, au cours de l’histoire, le dénominateur commun, il n’y eut jamais qu’une liberté tolérée, une seule : le changement de numérateur, l’immuable choix de se donner un maître. Pareil usage de la liberté a fini par lasser d’autant plus vite que les pires États totalitaires de l’Est et de l’Ouest ne cessent de s’en réclamer. Or le refus, actuellement généralisé, de changer d’employeur coïncide aussi avec un renouveau de l’organisation étatique. Tous les gouvernements du monde industrialisé ou en passe de l’être tendent à se modeler, à des degrés variables d’évolution, sur une forme commune, rationalisant les vieux mécanismes de domination, les automatisant en quelque sorte. Et ceci constitue la première chance de la liberté. Les démocraties bourgeoises ont montré qu’elles toléraient les libertés individuelles dans la mesure où elles se limitaient et se détruisaient réciproquement ; la démonstration faite, il est devenu impossible pour un gouvernement, si perfectionné soit-il, d’agiter la \emph{muleta} de la liberté sans que chacun ne devine l’épée qui y est cachée. Sans que, par contrecoup, la liberté ne retrouve sa racine, la créativité individuelle, et se refuse violemment à n’être que le permis, le licite, le tolérable, le sourire de l’autorité.\par
La deuxième chance de la liberté enfin ramenée à son authenticité créatrice tient aux mécanismes même du pouvoir. Il est évident que les systèmes abstraits d’exploitation et de domination sont des créations humaines, tirent leur existence et leurs perfectionnements d’une créativité dévoyée, récupérée. De la créativité, l’autorité ne peut et ne veut connaître que les diverses formes récupérables par le spectacle. Mais ce que les gens font officiellement n’est rien à côté de ce qu’ils font en se cachant. On parle de créativité à propos d’une œuvre d’art. Qu’est-ce que cela représente à côté de l’énergie créative qui agite un homme mille fois par jour, bouillonnement de désirs insatisfaits, rêveries qui se cherchent à travers le réel, sensations confuses et pourtant lumineusement précises, idées et gestes porteurs de bouleversement sans nom. Le tout voué à l’anonymat et à la pauvreté des moyens, enfermé dans la survie ou contraint de perdre sa richesse qualitative pour s’exprimer selon les catégories du spectacle. Que l’on pense au palais du facteur Cheval, au système génial de Fourier, à l’univers illustré du douanier Rousseau. Que chacun pense, plus précisément, à l’incroyable diversité de ses rêves, paysages autrement colorés que les plus belles toiles de Van Gogh. Qu’il pense au monde idéal bâti sans relâche sous son regard intérieur tandis que ses gestes refont le chemin du banal.\par
Il n’est personne, si aliéné soit-il, qui ne possède et ne se reconnaisse une part irréductible de créativité, une \emph{camera obscura} protégée contre toute intrusion du mensonge et des contraintes. Le jour où l’organisation sociale étendrait son contrôle sur cette part de l’homme, elle ne régnerait plus que sur des robots ou des cadavres. Et c’est en un sens pourquoi la conscience de la créativité s’accroît contradictoirement à mesure que se multiplient les essais de récupération auxquels se livre la société de consommation.\par
Argus est aveugle devant la menace la plus proche. Sous le règne du quantitatif, le qualitatif n’a pas d’existence légalement reconnue. C’est précisément ce qui le sauvegarde et l’entretient. Que la poursuite effrénée du quantitatif développe contradictoirement, par l’insatisfaction qu’elle nourrit, un désir absolu de qualitatif, j’ai eu l’occasion d’en parler plus haut. Plus la contrainte s’exerce au nom de la liberté de consommer, plus le malaise d’une telle contradiction fait naître la soif d’une liberté totale. Ce qu’il y avait de créativité opprimée dans l’énergie déployée par le travailleur a été révélé dans la crise de la société de production. Marx a dénoncé une fois pour toutes l’aliénation de la créativité dans le travail forcé, dans l’exploitation du producteur. À mesure que le système capitaliste et ses séquelles (même antagonistes) perdent sur le front de la production, ils s’efforcent de compenser par le biais de la consommation. Selon leurs directives, il faut que l’homme, se libérant de ses fonctions de producteur, s’englue dans une nouvelle fonction, celle de consommateur. Offrant à la créativité, enfin permise par la diminution des heures de travail, le terrain vague des loisirs, les bons apôtres de l’humanisme ne lèvent en fait qu’une armée prête à évoluer sur le champ de manœuvre de l’économie de consommation. À présent que l’aliénation du consommateur est percée à jour par la dialectique même du consommable, quelle prison prépare-t-on pour la très subversive créativité individuelle ? J’ai déjà dit que la dernière chance des dirigeants était de faire de chacun l’\emph{organisateur} de sa propre passivité.\par
Dewitt Peters explique, avec une candeur touchante, que « si l’on mettait simplement à la disposition des gens que la chose amuserait des couleurs, des pinceaux et des toiles, il pourrait en sortir quelque chose de curieux ». Tant que l’on appliquera cette politique pour une dizaine de domaines bien contrôlés comme le théâtre, la peinture, la musique, l’écriture… et en général pour des secteurs soigneusement isolés, on gardera quelque chance de donner aux gens une conscience d’artiste, une conscience d’homme qui fait profession d’exposer sa créativité dans les musées et les vitrines de la culture. Et plus une telle culture sera populaire, plus cela signifiera que le pouvoir a gagné. Mais les chances de « culturiser » de la sorte les hommes d’aujourd’hui sont minces. Espère-t-on vraiment, du côté des cybernéticiens, qu’un homme va accepter d’expérimenter librement dans des limites fixées autoritairement ? Croit-on vraiment que des hommes enfin conscients de leur force de créativité vont badigeonner les murs de leur prison et s’arrêter là ? Qu’est-ce qui les empêcherait d’expérimenter aussi avec les armes, les désirs, les rêves, les techniques de réalisation ? D’autant plus que les agitateurs sont déjà répandus dans la foule. La dernière récupération possible de la créativité – l’organisation de la passivité artistique – est éventée.\par

\begin{quoteblock}
\noindent « Je cherche, écrivait Paul Klee, un point lointain, à l’origine de la création, où je pressens une formule unique pour l’homme, l’animal, la plante, le feu, l’eau, l’air et toutes les forces qui nous entourent. »\end{quoteblock}

\noindent Lointain, un tel point ne l’est que dans la perspective mensongère du pouvoir. En fait, l’origine de toute création réside dans la créativité individuelle ; c’est de là que tout s’ordonne, les êtres et les choses, dans la grande liberté poétique. Point de départ de la nouvelle perspective, pour laquelle il n’est personne qui ne lutte de toutes ses forces et à chaque instant de son existence. « La subjectivité est le seul vrai » (Kierkegaard).\par
La vraie créativité est irrécupérable pour le pouvoir. A Bruxelles, en 1869, la police crut mettre la main sur le fameux trésor de l’Internationale, qui tracassait tant les capitalistes. Elle saisit une caisse colossale et solide, cachée dans un endroit obscur. On l’ouvrit, elle ne contenait que du charbon. La police ignorait que, touché par des mains ennemies, l’or pur de l’Internationale se convertit en charbon.\par
Dans les laboratoires de la créativité individuelle, une alchimie révolutionnaire transmute en or les métaux les plus vils de la quotidienneté. Il s’agit avant tout de dissoudre la conscience des contraintes, c’est-à-dire le sentiment d’impuissance, dans l’exercice attractif de la créativité ; les fondre dans l’élan de la puissance créatrice, dans l’affirmation sereine de son génie. La mégalomanie, par ailleurs stérile sur le plan du prestige et du spectacle, représente ici une étape importante dans la lutte qui oppose le moi aux forces coalisées du conditionnement. Dans la nuit du nihilisme aujourd’hui triomphant, l’étincelle créatrice, qui est l’étincelle de la vraie vie, brille avec plus d’éclat. Et tandis que le projet d’une meilleure organisation de la survie avorte, il y a, dans la multiplication de ces étincelles se fondant peu à peu dans une lumière unique, la promesse d’une nouvelle organisation fondée cette fois sur l’harmonie des volontés individuelles. Le devenir historique nous a conduits au croisement où la subjectivité radicale rencontre la possibilité de transformer le monde. Ce moment privilégié est le renversement de perspective.
\subsection[{2. La spontanéité}]{\textsc{2.} La spontanéité}
\noindent La spontanéité est le mode d’être de la créativité individuelle. Elle est son premier jaillissement, encore immaculé ; ni corrompu à la source, ni menacé de récupération. Si la créativité est la chose du monde la mieux partagée, la spontanéité, au contraire, semble relever d’un privilège. Seuls la détiennent ceux qu’une longue résistance au pouvoir a chargés de la conscience de leur propre valeur d’individu : le plus grand nombre des hommes dans les moments révolutionnaires, et plus qu’on ne croît, dans un temps où la révolution se construit tous les jours. Partout où la lueur de créativité subsiste, la spontanéité garde ses chances.\par

\begin{quoteblock}
\noindent « L’artiste nouveau proteste, écrivait Tsara en 1918, il ne peint plus, mais crée directement. »\end{quoteblock}

\noindent L’immédiateté est certainement la revendication la plus sommaire, mais aussi la plus radicale, qui doit définir ces nouveaux artistes que seront les constructeurs de situations à vivre. Sommaire, car enfin il ne convient pas de se laisser abuser par le mot spontanéité. Cela seul est spontané qui n’émane pas d’une contrainte intériorisée jusque dans le subconscient, et qui échappe au surplus à l’emprise de l’abstraction aliénante, à la récupération spectaculaire. On voit bien que la spontanéité est une conquête plus qu’un donné. La restructuration de l’individu (cf. la construction des rêves).\par
Ce qui a manqué jusqu’à présent à la créativité, c’est la conscience claire de sa poésie. Le sens commun a toujours voulu la décrire comme un état primaire, un stade antérieur auquel devait succéder une correction théorique, un transfert sur l’abstrait. C’était là isoler la spontanéité, en faire un en-soi et, partant, ne la reconnaître que falsifiée dans les catégories spectaculaires, dans l’\emph{action painting}, par exemple. Or la créativité spontanée porte en elle les conditions de son prolongement adéquat. Elle détient sa propre poésie.\par
Pour moi, la spontanéité constitue une expérience immédiate, une conscience du vécu, de ce vécu cerné de toutes parts, menacé d’interdits et cependant non encore aliéné, non encore réduit à l’inauthentique. Au centre de l’expérience vécue, chacun se trouve le plus près de lui-même. En cet espace-temps privilégié, je le sens bien, être réel me dispense d’être nécessaire. Et c’est toujours la conscience d’une nécessité qui aliène. On m’avait appris à me saisir, selon l’expression juridique, par défaut ; la conscience d’un moment de vie authentique élimine les alibis. L’absence de futur rejoint dans le même néant l’absence de passé. La conscience du présent s’harmonise à l’expérience vécue comme une sorte d’improvisation. Ce plaisir, pauvre parce qu’encore isolé, riche parce que déjà tendu vers le plaisir identique des autres, je ne puis m’empêcher de l’assimiler au plaisir du jazz. Le style d’improvisation de la vie quotidienne dans ses meilleurs moments rejoint ce que Dauer écrit du jazz :\par

\begin{quoteblock}
\noindent « La conception africaine du rythme diffère de la nôtre en ceci que nous le percevons auditivement tandis que les Africains le perçoivent à travers le mouvement corporel. Leur technique consiste essentiellement à introduire la discontinuité au sein de l’équilibre statique imposé par le rythme et le mètre à l’écoulement du temps. Cette discontinuité résultant de la présence de centres de gravité extatiques à contretemps, de l’accentuation propre au rythme et au mètre crée constamment des tensions entre les accents statiques et les accents extatiques qui leur sont imposés. »\end{quoteblock}

\noindent Le moment de la spontanéité créatrice est la plus infime présence du renversement de perspective. C’est un moment unitaire, c’est-à-dire un et multiple. L’explosion du plaisir vécu fait que, me perdant, je me trouve ; oubliant qui je suis, je me réalise. La conscience de l’expérience immédiate n’est rien d’autre que ce jazz, que ce balancement. Au contraire, la pensée qui s’attache au vécu dans un but analytique en reste séparée ; c’est le cas de toutes les études sur la vie quotidienne et, en un sens donc, de celle-ci – ce pourquoi je m’efforce d’y inclure à chaque instant sa propre critique, de peur qu’elle ne soit, comme beaucoup, aisément récupérable. Le voyageur qui fixe sa pensée sur la longueur du chemin à parcourir se fatigue plus que son compagnon qui laisse au gré de la marche errer son imagination ; de même la réflexion attentive à la démarche du vécu l’entrave, l’abstrait, le réduit à de futurs souvenirs.\par
Pour qu’elle se fonde vraiment dans le vécu, il faut que la pensée soit libre. Il suffit de penser \emph{autre} dans le sens du \emph{même}. Tandis que tu te fais, rêve d’un autre toi-même qui, un jour, te fera à son tour. Ainsi m’apparaît la spontanéité. La plus haute conscience de moi inséparable du moi et du monde.\par
Cependant, il faut retrouver les pistes de la spontanéité que les civilisations industrielles ont rendue sauvage. Il n’est pas facile de reprendre la vie par le bon bout. L’expérience individuelle est aussi une proie pour la folie, un prétexte. Les conditions sont celles dont parle Kierkegaard : « S’il est vrai que je porte une ceinture, toutefois, je ne vois pas la perche qui doit me soutenir. » Certes, la perche existe, et peut-être chacun pourrait-il la saisir, mais si lentement il est vrai que beaucoup mourront d’angoisse avant d’admettre qu’elle existe. Cependant, elle existe. C’est la subjectivité radicale : la conscience que tous les hommes obéissent à une même volonté de réalisation authentique, et que leur subjectivité se renforce de cette volonté subjective perçue chez les autres. Cette façon de partir de soi et de rayonner, moins vers les autres que vers ce que l’on découvre de soi en eux, donne à la spontanéité créatrice une importance stratégique semblable à celle d’une base de lancement. Les abstractions, les notions qui nous dirigent, il convient désormais de les ramener à leur source, à l’expérience vécue, non pour les justifier, mais pour les corriger au contraire, pour les inverser, les rendre au vécu dont elles sont issues et dont elles n’auraient jamais dû sortir ! C’est à cette condition que les hommes reconnaîtront sous peu que leur créativité individuelle ne se distingue pas de la créativité universelle. Il n’y a pas d’autorité en dehors de ma propre expérience vécue ; c’est ce que chacun doit prouver à tous.
\subsection[{3. Le qualitatif}]{\textsc{3.} Le qualitatif}
\noindent J’ai dit que la créativité, également répartie chez tous les individus, ne s’exprimait directement, \emph{spontanément}, qu’à la faveur de certains moments privilégiés. Ces états prérévolutionnaires, d’où irradie la poésie qui change la vie et transforme le monde, n’est-on pas fondé à les placer sous le signe de cette grâce moderne, le qualitatif ? De même que la présence de l’abomination divine se trahissait par la suavité spirituelle, soudain conférée aux rustres comme aux natures les plus fines – à Claudel, ce crétin, comme à Jean de la Croix –, de même un geste, une attitude, un mot parfois, atteste de façon indéniable la présence de la chance offerte à la poésie, c’est-à-dire à la construction totale de la vie quotidienne, au renversement global de perspective, à la révolution. Le qualitatif est un raccourci, un condensé, une communication directe de l’essentiel.\par
Kagame entendit un jour une vieille femme du Rwanda, qui ne savait ni lire ni écrire, dire : « Vraiment, les Blancs sont d’une naïveté désarmante ! Ils n’ont pas d’intelligence ! » Comme il lui répliquait : « Comment pouvez-vous dire une aussi grosse sottise ? Avez-vous pu comme eux inventer tant de merveilles qui dépassent notre imagination ? » Elle répondit avec un sourire compatissant : « Écoutez bien ceci, mon enfant ! Ils ont appris tout cela, mais ils n’ont pas d’intelligence ! Ils ne comprennent rien ! » De fait, la malédiction de la civilisation de la technique, de l’échange quantifié et de la connaissance scientifique, est de n’avoir rien créé qui encourage et libère \emph{directement} la créativité spontanée des hommes, au contraire, ni même qui leur permette de comprendre immédiatement le monde. Ce qu’exprimait la vieille femme rwandaise – cet être que l’administrateur blanc devait, du haut de sa spiritualité belge, regarder comme une bête sauvage – apparaissait chargé de culpabilité et de mauvaise conscience, c’est-à-dire entaché d’une bêtise ignoble, dans le vieux propos : « J’ai beaucoup étudié et c’est pourquoi je sais que je ne sais rien. » Car il est faux, en un sens, qu’une étude ne nous apprenne rien, si elle n’abandonne pas le point de vue de la totalité. Ce qui fut appelé rien, c’étaient les étages successifs du qualitatif ; ce qui, à des niveaux divers, restait dans la ligne du qualitatif. Que l’on me permette une image. Supposons plusieurs pièces situées exactement les unes au-dessus des autres, réunies par un ascenseur qui les traverse en leur milieu et communiquant par l’extérieur grâce à des volées d’escaliers en colimaçon. Entre les gens qui habitent les différentes pièces, la liaison est directe mais comment communiqueraient-ils avec ceux qui se trouvent engagés à l’extérieur, dans l’escalier ? Entre les détenteurs du qualitatif et les détenteurs de la connaissance à crémaillère, il n’y a pas de dialogue. Incapables pour la plupart de lire le manifeste de Marx et Engels, les ouvriers de 1848 possédaient en eux l’essentiel du texte. C’est d’ailleurs en cela que la théorie marxiste était radicale. La condition ouvrière et ses implications, que le Manifeste exprimait théoriquement à l’étage supérieur, permettaient aux plus ignorants des prolétaires d’accéder \emph{immédiatement}, le moment venu, à la compréhension de Marx. L’homme cultivé et usant de sa culture comme d’un lance-flammes est fait pour s’entendre avec l’homme inculte mais qui ressent dans la réalité vécue quotidiennement ce que l’autre exprime savamment. Il faut bien que les armes de la critique rejoignent la critique des armes.\par
Seul le qualitatif permet de passer d’un bond à l’étage supérieur. C’est la pédagogie du groupe en péril, la pédagogie de la barricade. Mais le graduel du pouvoir hiérarchisé ne conçoit semblablement qu’une hiérarchie de connaissance graduelles ; des gens dans l’escalier, spécialisés dans la nature et la quantité des marches, se rencontrent, se croisent, se heurtent, s’insultent. Quelle importance ? En bas l’autodidacte farci de bon sens, en haut l’intellectuel collectionnant les idées se renvoient l’image inverse d’un même ridicule. Miguel de Unamuno et l’ignoble Millan Astray, le salarié de la pensée et son contempteur, s’affrontent en vain ; hors du qualitatif, l’intelligence n’est qu’une marotte d’imbéciles.\par
Les alchimistes appelaient \emph{materia prima} les éléments indispensables au Grand Œuvre. Et ce que Paracelse en écrit s’applique parfaitement au qualitatif :\par

\begin{quoteblock}
\noindent « Il est manifeste que les pauvres en ont davantage que les riches. Les gens en gaspillent la bonne part et n’en retiennent que la mauvaise part. Elle est visible et invisible, et les enfants jouent avec elle dans la rue. Mais les ignorants la foulent aux pieds quotidiennement. »\end{quoteblock}

\noindent Or la conscience de \emph{materia prima} qualitative doit sans cesse s’affiner dans la plupart des esprits, à mesure que s’effondrent les bastions de la pensée spécialisée et de la connaissance graduelle. La prolétarisation accule désormais au même nihilisme ceux qui font profession de créer et ceux que leur profession empêche de créer, les artistes et les travailleurs. Et cette prolétarisation qui va de pair avec son refus, c’est-à-dire avec le refus des formes récupérées de la créativité, s’effectue dans un tel encombrement de biens culturels – disques, livres de poche – que ceux-ci vont, une fois arrachés au consommable, passer sans délais au service de la vraie créativité. Ainsi le sabotage des mécanismes de la consommation économique et culturelle trouve-t-il à s’illustrer de façon exemplaire chez ces jeunes gens qui volent les livres dont ils attendent confirmation de leur radicalité.\par
Réinvesties sous le signe du qualitatif, les connaissances les plus diverses créent un réseau aimanté capable de soulever les plus lourdes traditions. Le savoir est multiplié par la puissance exponentielle de la simple créativité spontanée. Avec des moyens de fortune et pour un prix dérisoire, un ingénieur allemand a mis au point un appareil qui réalise les mêmes opérations que le cyclotron. Si la créativité individuelle, aussi médiocrement stimulée, arrive à de pareils résultats, que ne faut-il espérer de chocs qualitatifs, de réactions en chaîne où l’esprit de la liberté qui s’est maintenu vivant dans les individus reparaîtrait collectivement pour célébrer, dans le feu de joie et la rupture d’interdits, la grande fête sociale ?\par
Il ne s’agit plus, pour un groupe révolutionnaire cohérent, de créer un conditionnement de type nouveau, mais au contraire d’établir des zones de protection où l’intensité du conditionnement tende vers zéro. Rendre chacun conscient de son potentiel de créativité est une tentative vouée à l’échec si elle ne recourt pas à l’éveil par le choc qualitatif. Il n’y a plus rien à attendre des partis de masses et des groupes fondés sur le recrutement quantitatif. Par contre, une microsociété dont les membres se seraient reconnus sur la base d’un geste ou d’une pensée radicale, et qu’un filtrage théorique serré maintiendrait dans un état de pratique efficace permanent, un tel noyau, donc, réunirait toutes les chances de rayonner un jour avec suffisamment de force pour libérer la créativité du plus grand nombre des hommes. Il faut changer en espoir le désespoir des terroristes anarchistes ; corriger dans le sens d’une stratégie moderne leur tactique de guerrier médiéval.
\subsection[{4. La poésie}]{\textsc{4.} La poésie}
\noindent Qu’est-ce que la poésie ? La poésie est l’organisation de la spontanéité créative, l’exploitation du qualitatif selon les lois intrinsèques de cohérence. Ce que les Grecs nommaient POIEN, qui est le « faire » ici rendu à la pureté de son jaillissement originel et, pour tout dire, à la totalité.\par
Où le qualitatif manque, nulle poésie possible. Dans le vide laissé par la poésie s’installe son contraire : l’information, le programme transitoire, la spécialisation, la réforme ; bref le parcellaire sous ses diverses formes. Toutefois, la présence du qualitatif n’implique pas fatalement un prolongement poétique. Il peut se faire qu’une grande richesse de signes et de possibles s’égare dans la confusion, se perde faute d’une cohérence, s’émiette par interférences. Or le critère d’efficacité prédomine toujours. La poésie, c’est donc aussi la théorie radicale digérée par les actes ; le couronnement de la tactique et de la stratégie révolutionnaire ; l’apogée du grand jeu sur la vie quotidienne.\par
Qu’est-ce que la poésie ? En 1895, lors d’une grève mal engagée et vouée, semble-t-il, à l’échec, un militant du Syndicat national des Chemins de Fer prit la parole et fit allusion à un moyen ingénieux et peu coûteux :\par

\begin{quoteblock}
\noindent « Avec deux sous d’une certaine matière utilisée à bon escient, déclara-t-il, il nous est possible de mettre une locomotive dans l’impossibilité de fonctionner. »\end{quoteblock}

\noindent Les milieux gouvernementaux et capitalistes cédèrent aussitôt. Ici la poésie est nettement l’acte qui engendre des réalités nouvelles, l’acte du renversement de perspective. La \emph{materia prima} est à la portée de tous. Sont poètes ceux qui en connaissent l’usage, savent l’employer efficacement. Et que dire d’une matière de deux sous quand l’existence quotidienne offre à profusion une énergie disponible et sans pareille : volonté de vivre, désir effréné, passion de l’amour, amour des passions, force de peur et d’angoisse, gonflement de la haine et retombées de la rage de détruire ? Quels bouleversements poétiques n’est-on pas fondé d’espérer de sentiments aussi universellement ressentis que ceux de la mort, de l’âge, de la maladie ? C’est de cette conscience encore marginale que doit partir la longue révolution de la vie quotidienne, la seule poésie faite par tous, non par un.\par
Qu’est-ce que la poésie ? demandent les esthètes. Et il faut alors leur rappeler cette évidence : la poésie est devenue rarement poème. La plupart des œuvres d’art trahissent la poésie. Comment en serait-il autrement, puisque la poésie et le pouvoir sont inconciliables ? Au mieux, la créativité de l’artiste se donne une prison, elle se cloître en attendant son heure dans une œuvre qui n’a pas dit son dernier mot ; mais bien que l’auteur en attende beaucoup, ce dernier mot – celui qui précède la communication parfaite – elle ne le prononcera jamais tant que la révolte de la créativité n’aura pas mené l’art jusqu’à sa réalisation.\par
L’œuvre d’art africaine, qu’il s’agisse d’un poème ou d’une musique, d’une sculpture ou d’un masque, n’est considérée comme achevée que lorsqu’elle est verbe créateur, parole agissante ; que si elle fonctionne. Or ceci ne vaut pas seulement pour l’art africain. Pas un art au monde qui ne s’efforce de fonctionner ; et de fonctionner, même au niveau des récupérations ultérieures, comme une seule et même volonté initiale : une volonté de vivre dans l’exubérance du moment créatif. Comprend-on pourquoi les meilleures œuvres n’ont pas de fin ? Elles ne font qu’exiger sur tous les tons le droit de se réaliser, d’entrer dans le monde du vécu. La décomposition de l’art actuel est l’arc idéalement bandé pour une telle flèche.\par
Rien ne sauvera de la culture du passé le passé de la culture, sinon les tableaux, les écrits, les architectures musicales ou lithiques dont le qualitatif nous atteint, libéré de sa forme aujourd’hui contaminée par le dépérissement de toutes les formes de l’art. Sade, Lautréamont, mais aussi Villon, Lucrèce, Rabelais, Pascal, Fourier, Bosch, Dante, Bach, Swift, Shakespeare, Uccello… se dépouillent de leur enveloppe culturelle, sortent des musées où l’histoire les avait colloqués et entrent comme de la mitraille meurtrière dans les marmites à renversement des réalisateurs de l’art. À quoi juge-t-on de la valeur d’une œuvre ancienne ? À la part de théorie radicale qu’elle contient, au noyau de spontanéité créative que les nouveaux créateurs s’apprêtent à libérer pour et par une poésie inédite.\par
La théorie radicale excelle à différer l’acte amorcé par la spontanéité créative, sans l’altérer ni le dévoyer de sa course. De même, dans ses meilleurs moments, la démarche artistique tente d’imprimer au monde le mouvement d’une subjectivité toujours tentaculaire, toujours assoiffée de créer et de se créer. Mais tandis que la théorie radicale colle à la réalité poétique, à la réalité qui se fait, au monde que l’on transforme, l’art s’engage dans une démarche identique avec un risque beaucoup plus grand de se perdre et de se corrompre. Seul l’art armé contre lui-même, contre ce qu’il a de plus faible – de plus esthétique – résiste à la récupération.\par
On le sait, la société de consommation réduit l’art à une variété de produit consommable. Et plus la réduction se vulgarise, plus la décomposition s’accélère, plus s’accroissent les chances d’un dépassement. La communication si impérativement désirée par l’artiste est interrompue et interdite jusque dans les rapports les plus simples de la vie quotidienne. Si bien que la recherche de nouveaux modes de communication, loin d’être réservée aux peintres ou aux poètes, participe aujourd’hui d’un effort collectif. Ainsi prends fin la vieille spécialisation de l’art. Il n’y a plus d’artistes, car tous le sont. L’œuvre d’art à venir, c’est la construction d’une vie passionnante.\par
La création importe moins que le processus qui engendre l’œuvre, que l’acte de créer. L’état de créativité fait l’artiste, et non pas le musée. Malheureusement, l’artiste se reconnaît rarement comme créateur. La plupart du temps, il pose devant un public, il donne à voir. L’attitude contemplative devant l’œuvre d’art a été la première pierre jetée au créateur. Cette attitude, il l’a provoquée et elle le tue aujourd’hui depuis que, réduite au besoin de consommer, elle relève des impératifs économiques les plus grossiers. C’est pourquoi il n’y a plus d’œuvre d’art, au sens classique du terme. Il ne peut plus y avoir d’œuvre d’art, et c’est très bien ainsi. La poésie est ailleurs, dans les faits, dans l’événement que l’on crée. La poésie des faits, qui a été de tout temps traitée marginalement, réintègre aujourd’hui le centre de tous les intérêts, la vie quotidienne qu’à vrai dire elle n’a jamais quittée.\par
La vraie poésie se moque de la poésie. Mallarmé, en quête du Livre, ne désire rien tant qu’abolir le poème, et comment abolir un poème sinon en le réalisant ? Or, cette nouvelle poésie, quelques contemporains de Mallarmé en usent avec éclat. Lorsqu’il les appela des « anges de pureté », l’auteur d’Hérodiade prit-il conscience que les agitateurs anarchistes offraient au poète une clé que, muré dans son langage, il ne pouvait employer ?\par
La poésie est toujours quelque part. Vient-elle à déserter les arts, on voit mieux qu’elle réside avant tout dans les gestes, dans un style de vie, dans une recherche de ce style. Partout réprimée, cette poésie-là fleurit partout. Brutalement refoulée, elle reparaît dans la violence. Elle consacre les émeutes, épouse la révolte, anime les grandes fêtes sociales avant que les bureaucrates l’assignent à résidence dans la culture hagiographique.\par
La poésie vécue a su prouver au cours de l’histoire, même dans la révolte parcellaire, même dans le crime – cette révolte d’un seul, comme dit Cœurderoy – qu’elle protégeait par-dessus tout ce qu’il y a d’irréductible dans l’homme : la spontanéité créative. La volonté de créer l’unité de l’homme et du social, non sur la base de la fiction communautaire, mais au départ de la subjectivité, voilà ce qui fait de la poésie nouvelle une arme dont chacun doit apprendre le maniement par soi-même. L’expérience poétique désormais fait prime. L’organisation de la spontanéité sera l’œuvre de la spontanéité elle-même.
\section[{XXI. Les maîtres sans esclaves}]{XXI. Les maîtres sans esclaves}\renewcommand{\leftmark}{XXI. Les maîtres sans esclaves}


\begin{argument}\noindent Le pouvoir est l’organisation sociale par laquelle les maîtres entretiennent les conditions d’esclavages. Dieu, L’État, l’Organisation : ces trois mots montrent assez ce qu’il y a d’autonomie et de détermination historique dans le pouvoir. Trois principes ont exercé successivement leur prépondérance : le principe de domination (pouvoir féodal), le principe d’exploitation (pouvoir bourgeois), le pouvoir d’organisation (pouvoir cybernétisé) (2). – L’organisation sociale hiérarchisée s’est perfectionnée en se désacralisant et en se mécanisant, mais ses contradictions se sont accrues. Elle s’est humanisée à mesure qu’elle vidait les hommes de leur substance humaine. Elle a gagné en autonomie aux dépens des maîtres (les dirigeants sont aux commandes, mais ce sont les leviers qui les gouvernent). Les chargés de pouvoir perpétuent aujourd’hui la race des esclaves soumis, ceux dont Théognis dit qu’ils naissent avec la nuque inclinée. Ils ont perdu jusqu’au plaisir malsain de dominer. Face aux maîtres-esclaves se dressent les hommes du refus, le nouveau prolétariat, riche de ses traditions révolutionnaires. De là sortiront les maîtres sans esclaves et un type de société supérieure où se réaliseront le projet vécu de l’enfance et le projet historique des grands aristocrates (1) (3).
\end{argument}

\subsection[{1. Les dirigeants sont aux commandes, mais ce sont les leviers qui les gouvernent}]{\textsc{1.} Les dirigeants sont aux commandes, mais ce sont les leviers qui les gouvernent}
\noindent Platon écrit dans le \emph{Théagès} :\par

\begin{quoteblock}
\noindent « Chacun de nous voudrait être si possible le maître de tous les hommes, ou mieux encore Dieu. »\end{quoteblock}

\noindent  Ambition médiocre si l’on se réfère à la faiblesse des maîtres et des dieux. Car enfin, si la petitesse des esclaves vient de ce qu’ils s’inféodent à des gouvernants, la petitesse des chefs et de Dieu lui-même tient à la nature déficitaire des gouvernés. Le maître connaît l’aliénation sous son pôle positif, l’esclave sous son pôle négatif ; à l’un comme à l’autre la maîtrise totale est également refusée.\par
Comment le féodal se comporte-t-il dans cette dialectique du maître et de l’esclave ? Esclave de Dieu et maître d’hommes – et maître d’hommes parce qu’esclave de Dieu, selon les exigences du mythe – le voici condamné à mêler intimement l’exécration et l’intérêt respectueux qu’il porte à Dieu, car c’est à Dieu qu’il doit obéissance et c’est de lui qu’il détient son pouvoir sur les hommes. En somme, il reproduit entre Dieu et lui le type de rapports existant entre les nobles et le roi. Qu’est-ce qu’un roi ? Un élu parmi les élus, et dont la succession se présente la plupart du temps comme un jeu où les égaux rivalisent. Les féodaux servent le roi, mais ils le servent comme ses égaux en puissance. Ainsi se soumettent-ils à Dieu, mais en rivaux, en concurrents.\par
On comprend l’insatisfaction des maîtres anciens. Par Dieu, ils entrent dans le pôle négatif de l’aliénation, par ceux qu’ils oppriment, dans son rôle positif. Quel désir auraient-ils d’être Dieu, puisqu’ils connaissent l’ennui de l’aliénation positive ? Et dans le même temps, comment ne souhaiteraient-ils pas en finir avec Dieu, leur tyran ? Le \emph{to be or not to be} des Grands s’est toujours traduit par la question, insoluble à l’époque, de nier et de conserver Dieu, c’est-à-dire de le dépasser, de le réaliser.\par
L’histoire atteste deux tentatives pratiques d’un tel dépassement, celle des mystiques et celles des grands négateurs. Maître Eckhart déclarait : « Je prie Dieu qu’il me fasse quitte de Dieu. » Semblablement, les hérétiques de Souabe disaient en 1270 qu’ils s’étaient élevés au-dessus de Dieu et que, ayant atteint le degré le plus élevé de la perfection divine, ils avaient abandonné Dieu. Par une autre voie, la voie négative, certaines individualités fortes, comme Héliogabale, Gilles de Rais, Erszebet Bathory, s’efforcent, on le voit bien, d’atteindre à la maîtrise totale sur le monde en liquidant les intermédiaires, ceux qui les aliènent positivement, leurs esclaves. Ils vont vers l’homme total par le biais de l’inhumanité totale. À rebours. De sorte que la passion de régner sans borne et le refus absolu des contraintes forment un seul et même chemin, une route ascendante et descendante où Caligula et Spartacus, Gilles de Rais, et Dosza Gyorgy se côtoient, ensemble et séparés. Mais il ne suffit pas de dire que la révolte intégrale des esclaves – la révolte intégrale, j’insiste, et non ses formes déficitaires, chrétiennes, bourgeoise ou socialiste – rejoint la révolte extrême des maîtres anciens. De fait, la volonté d’abolir l’esclavage et toutes ses séquelles (le prolétaire, l’exécution, l’homme soumis et passif) offre une chance unique à la volonté de régner sur le monde sans autre limite que la nature enfin réinventée, que la résistance offerte par les objets à leur transformation.\par
Cette chance-là s’inscrit dans le devenir historique. L’histoire existe parce qu’il existe des opprimés. La lutte contre la nature, puis contre les diverses organisations sociales de lutte contre la nature, est toujours la lutte pour l’émancipation humaine, pour l’homme total. Le refus d’être esclave est vraiment ce qui change le monde.\par
Quel est donc le but de l’histoire ? Elle est faite « dans certaines conditions » (Marx) par les esclaves et contre l’esclavage, elle ne peut donc que poursuivre une fin : la destruction des maîtres. De son côté, le maître n’a de cesse qu’il échappe à l’histoire, qu’il la refuse en massacrant ceux qui la font, et la font contre lui.\par
Et voici les paradoxes :\par
1° L’aspect le plus humain des maîtres anciens réside dans leur prétention à l’absolue maîtrise. Un tel projet implique le blocage absolu de l’histoire, donc le refus extrême du mouvement d’émancipation, c’est-à-dire l’inhumanité totale.\par
2° La volonté d’échapper à l’histoire rend vulnérable. À la fuir, on se découvre devant elle, on tombe plus sûrement sous ses coups ; le parti pris d’immobilisme ne résiste pas aux vagues d’assaut de réalités vécues, pas plus qu’à la dialectique de forces productives. Les maîtres sont les sacrifiés de l’histoire ; ils sont broyés par elle selon ce que la contemplation de trois mille ans donne, du haut de la pyramide du présent, pour un véritable \emph{planning}, un programme rigoureux, une ligne de force qui incite à parler d’un Sens de l’Histoire (fin du monde esclavagiste, fin du monde féodal, fin du monde bourgeois).\par
Parce qu’ils s’efforcent d’y échapper, les maîtres se rangent en ordre utile dans les tiroirs de l’histoire, ils entrent dans l’évolution temporelle linéaire en dépit qu’ils en aient. Au contraire, ceux qui font l’histoire – les révolutionnaires, les esclaves ivres d’une liberté totale – ceux-là paraissent agir \emph{sub specie aeternitatis}, sous le signe de l’intemporel, mus par la soif insatiable d’une vie intense et poursuivant leur but à travers les diverses conditions historiques. Peut-être la notion philosophique d’éternité est-elle liée aux tentatives historiques d’émancipation, peut-être cette notion va-t-elle un jour être réalisée, comme la philosophie, par ceux qui portent en eux la liberté totale et la fin de l’histoire traditionnelle ?\par
3° La supériorité du pôle négatif de l’aliénation sur le pôle positif, c’est que sa révolte intégrale rend seule possible le projet de maîtrise absolue. Les esclaves en lutte pour la suppression des contraintes dénouent le mouvement par lequel l’histoire dissout les maîtres, et par-delà l’histoire, c’est la possibilité d’un nouveau pouvoir sur les choses qu’ils rencontrent, un pouvoir qui ne s’approprie plus les objets en s’appropriant les êtres. Mais dans le cours même de l’histoire lentement élaborée, il est arrivé forcément que les maîtres, au lieu de disparaître, ont dégénéré, qu’il n’y a plus eu de maîtres mais seulement des esclaves-consommateurs de pouvoir, divergeant entre eux par le degré et la quantité de pouvoir consommé.\par
Il était fatal que la transformation du monde par les forces productives doive réaliser lentement, passant au préalable par l’étape bourgeoise, les conditions matérielles d’une émancipation totale. Aujourd’hui que l’automation et la cybernétique appliquées dans le sens de l’humain permettraient la construction du rêve des maîtres anciens et des esclaves de tous les temps, il n’y a plus qu’un magma socialement informe où la confusion mêle, en chaque être particulier, des parcelles dérisoires de maître et d’esclave. C’est cependant de ce \emph{règne des équivalences} que vont sortir les nouveaux maîtres sans esclaves.\par
Je veux au passage saluer Sade. Il est, par son apparition privilégiée à un tournant de l’histoire autant que par son étonnante lucidité, le dernier des grands seigneurs révoltés. Comment les maîtres du château de Selling assurent-ils leur maîtrise absolue ? Ils massacrent tous leurs serviteurs, accédant par ce geste à une éternité de délices. C’est le sujet des \emph{Cent vingt journées de Sodome}.\par
Marquis et sans-culottes, DAF. de Sade unit la parfaite logique hédoniste du grand seigneur méchant homme et la volonté révolutionnaire de jouir sans limite d’une subjectivité enfin dégagée du cadre hiérarchique. L’effort désespéré qu’il tente pour abolir le pôle positif et le pôle négatif de l’aliénation le range d’emblée parmi les théoriciens les plus importants de l’homme total. Il est bien temps que les révolutionnaires lisent Sade avec autant de soin qu’ils en mettent à lire Marx. (De Marx, il est vrai, les spécialistes de la révolution connaissent surtout ce qu’il a écrit sous le pseudonyme de Staline, ou au mieux de Lénine et Trotsky.) De toute façon, aucun désir de changer radicalement la vie quotidienne ne pourra désormais se passer ni des grands négateurs du pouvoir, ni de ces maîtres anciens qui surent se sentir à l’étroit dans la puissance que Dieu leur accordait.
\subsection[{2. Dieu, L’État, l’Organisation}]{\textsc{2.} Dieu, L’État, l’Organisation}
\noindent Le pouvoir bourgeois s’est nourri des miettes du pouvoir féodal. Il est le pouvoir féodal en miettes. Rongée par la critique révolutionnaire, piétinée et mise en pièces – sans que cette liquidation atteigne jamais ses conséquences ultimes : la fin du pouvoir hiérarchisé – l’autorité aristocratique survit sous une forme parodique, comme une grimace d’agonie, à la mort de l’aristocratie. Engoncés dans leur pouvoir parcellaire, faisant de leur parcelle une totalité (et le totalitaire n’est rien d’autre), les dirigeants bourgeois étaient condamnés à voir leur prestige tomber en lambeaux, gangrené par la décomposition du spectacle. Sitôt que vinrent à manquer le sérieux du mythe et la foi en l’autorité, il n’y eut plus, en mode de gouvernement, que la terreur bouffonne et les âneries démocratiques. Ah ! les beaux enfants de Bonaparte ! Louis-Philippe, Napoléon III, Thiers, Alphonse XIII, Hitler, Mussolini, Staline, Franco, Salazar, Nasser, Mao, de Gaulle… Ubus prolifiques enfantant aux quatre coins du monde des avortons de plus en plus débiles. Hier brandissant, telles des foudres jupitériennes, leurs allumettes d’autorité, les singes du pouvoir ne recueillent plus désormais sur la scène sociale que des succès d’estime. Il n’y a plus pour eux que des rôles de second plan. Assurément, le ridicule de Franco tue encore – personne ne songe à l’oublier – mais qu’on le sache aussi : bientôt la bêtise du pouvoir tuera plus sûrement que la bêtise au pouvoir.\par
La machine à décerveler de notre colonie pénitentiaire, c’est le spectacle. Les maîtres-esclaves d’aujourd’hui sont ses fidèles servants, figurants et metteurs en scène. Qui souhaitera les juger ? Ils plaideront non coupables. De fait, ils sont non coupables. Ils ont besoin moins de cynisme que d’aveux spontanés, de terreur que de victimes consentantes, de force que de troupeaux masochistes. L’alibi des gouvernants est dans la lâcheté des gouvernés. Mais voici que tous sont gouvernés, manipulés comme des choses par un pouvoir abstrait, par une organisation en-soi dont les lois s’imposent aux prétendus dirigeants. On ne juge pas les choses, on les empêche de nuire.\par
En octobre 1963, M. Fourastié, s’interrogeant sur le chef de demain, aboutit aux conclusions suivantes :\par

\begin{quoteblock}
\noindent « Le chef a perdu son pouvoir presque \emph{magique} ; il est et sera un homme capable de \emph{provoquer des actions}. Enfin, le règne des groupes de travail se développera pour préparer les décisions. Le chef sera un président de commission, mais qui saura \emph{conclure et trancher} » [souligné par moi].\end{quoteblock}

\noindent On retrouve là les trois phases historiques qui caractérisent l’évolution du maître :\par
1° Le principe de domination, lié à la société féodale ;\par
2° Le principe d’exploitation, lié à la société bourgeoise ;\par
3° Le principe d’organisation, lié à la société cybernétisée.\par
En fait, les trois éléments sont indissociables ; on ne domine pas sans exploiter ni organiser simultanément ; mais leur importance varie selon les époques. À mesure que l’on passe d’un stade à l’autre, l’autonomie et la part du maître se réduisent, rapetissent. L’humanité du maître tend vers zéro tandis que l’inhumanité du pouvoir désincarné tend vers l’infini.\par
Selon le \emph{principe de domination}, le maître refuse aux esclaves une existence qui limiterait la sienne. Dans le \emph{principe d’exploitation}, le patron accorde aux travailleurs une existence qui nourrit et accroît la sienne. Le \emph{principe d’organisation} classe les existences individuelles comme des fractions, selon le taux de capacité dirigeante ou exécutante qu’elles comportent (un chef d’atelier serait par exemple défini aux termes de longs calculs sur son rendement, sa représentation, etc, par 56 \% de fonction dirigeante, 40 \% de fonction exécutante et 4 \% d’ambigu, comme dirait Fourier).\par
La domination est un droit, l’exploitation un contrat, l’organisation un ordre des choses. Le tyran domine selon sa volonté de puissance, le capitaliste exploite selon les lois du profit, l’organisateur planifie et est planifié. Le premier se veut arbitraire, le deuxième juste, le troisième rationnel et objectif. L’inhumanité du seigneur est une humanité qui se cherche ; l’inhumanité de l’exploiteur tente de se dédouaner par la séduction qu’exercent sur l’humain le progrès technique, le confort, la lutte contre la faim et la maladie ; l’inhumanité du cybernéticien est une inhumanité qui s’accepte. Ainsi l’inhumanité du maître est devenue de moins en moins humaine. Car il y a plus d’atrocité dans un camp d’extermination systématique que dans la fureur meurtrière des féodaux se livrant une guerre sans cause. Et quel lyrisme encore dans les massacres d’Auschwitz quand on les compare aux mains glacées du conditionnement généralisé que tend vers la société, future et si proche, l’organisation technocratique des cybernéticiens ! Que l’on comprenne bien, il ne s’agit pas de choisir entre l’« humanité » d’une lettre de cachet et l’« humanité » d’un lavage de cerveau. Autant choisir entre la potence et la guillotine ! J’entends simplement que le plaisir douteux de dominer et d’écraser tend à disparaître. Le capitalisme a inauguré la nécessité d’exploiter les hommes sans en tirer de jouissance passionnelle. sans sadisme, sans cette joie négative d’exister qui consiste à faire souffrir, sans même une perversion de l’humain, sans même l’humain à rebours. Le règne des choses s’accomplit. En renonçant au principe hédoniste, les maîtres ont renoncé à la maîtrise. Cet abandon, il appartient aux maîtres sans esclaves de le corriger.\par
Ce que la société de production avait amorcé, la dictature du consommable l’achève aujourd’hui. Le principe d’organisation vient parfaire la véritable maîtrise des objets morts sur les hommes. La part de pouvoir qui restait aux possesseurs des instruments de production disparaît dès l’instant où les machines, échappant aux propriétaires, passent sous le contrôle des techniciens qui en organisent l’emploi. Tandis que les organisateurs eux-mêmes sont lentement digérés par les schémas et les programmes qu’ils ont élaborés. La machine simple aura été la dernière justification du chef, le dernier support de sa trace ultime d’humanité. L’organisation cybernéticienne de la production et de la consommation passe obligatoirement par le contrôle, la planification, la rationalisation de la vie quotidienne.\par
Les spécialistes sont ces maîtres en miettes, ces maîtres-esclaves qui prolifèrent sur le territoire de la vie quotidienne. Leur chance, on peut l’assurer, est nulle. Déjà en 1867, au Congrès de Bâle, Francau, de la Ire Internationale, déclarait :\par

\begin{quoteblock}
\noindent « Trop longtemps, nous avons été à la remorque des marquis du diplôme et des princes de la science. Faisons nos affaires nous-mêmes et, si inhabiles que nous puissions être, nous ne les ferons jamais plus mal qu’on ne les fit en notre nom. »\end{quoteblock}

\noindent Paroles pleines de sagesse, et dont le sens se renforce avec la prolifération des spécialistes et leur incrustation dans la vie individuelle. Le partage s’opère nettement entre ceux qui obéissent à l’attraction magnétique qu’exerce la grande machine kafkaïenne de la cybernétique et ceux qui, obéissant à leurs propres impulsions, s’efforcent de lui échapper. Ceux-ci sont dépositaires de la totalité de l’humain, puisque personne ne peut désormais y prétendre dans l’ancien clan des maîtres. Il n’y a plus, d’un côté, que des choses qui tombent à la même vitesse dans le vide, de l’autre, que le vieux projet des esclaves ivres d’une liberté totale.
\subsection[{3. Le maître sans esclaves ou le dépassement aristocratique de l’aristocratie}]{\textsc{3.} Le maître sans esclaves ou le dépassement aristocratique de l’aristocratie}
\noindent Le maître s’est perdu par les mêmes voies que Dieu. Il s’effondre comme un Golem dès qu’il cesse d’aimer les hommes, dès qu’il cesse, donc, d’aimer le plaisir qu’il s’offre à les opprimer. Dès qu’il abandonne le principe hédoniste. Il y a peu de plaisir à déplacer des choses, à manipuler des êtres passifs et insensibles comme des briques. Dans son raffinement, Dieu recherche des créatures vivantes, de la bonne chair palpitante, des âmes frissonnant de terreur et de respect. Il a besoin pour éprouver sa propre grandeur de sentir la présence de sujets ardents à la prière, à la contestation, à la ruse, à l’insulte même. Le Dieu catholique s’entend à prêter de la vraie liberté, mais à la façon des prêteurs sur gages. Les hommes, il les laisse aller, comme le chat la souris, jusqu’au jugement dernier où il va les croquer. Puis, vers la fin du Moyen Âge, avec l’entrée en scène de la bourgeoisie, le voici qui s’humanise lentement, paradoxalement, car il devient objet, et les hommes aussi deviennent des objets. En condamnant les hommes à la prédestination, le Dieu de Calvin perd le plaisir de l’arbitraire, il n’est plus libre d’écraser qui il veut, ni quand il veut. Dieu des transactions commerciales, sans fantaisie, mesurable et froid comme un taux d’escompte, il a honte, il se cache. \emph{Deus absconditus}. Le dialogue est rompu. Pascal désespère. De l’âme soudain sans attache, Descartes ne sait que faire. Plus tard – trop tard – Kierkegaard s’efforcera de ressusciter le dieu subjectif en ressuscitant la subjectivité des hommes. Mais rien ne peut ranimer Dieu devenu dans l’esprit des hommes le « grand objet extérieur » ; il est mort définitivement, transformé en pierre, madréporisé. D’ailleurs, saisis dans le glacis de sa dernière étreinte (la Forme hiérarchisée du pouvoir), les hommes paraissent voués à la réification, à la mort de l’humain. La perspective du pouvoir n’offre à contempler que des choses, des fragments de la grande pierre divine. N’est-ce pas selon cette perspective que la sociologie, la psychologie, l’économie et les sciences dites humaines – si soucieuses d’observer « objectivement » – braquent leur microscope ?\par
Par quelle raison le maître est-il contraint d’abandonner l’exigence hédoniste ? Qu’est-ce qui l’empêche d’atteindre à la jouissance totale, si ce n’est sa condition de maître, son parti pris de supériorité hiérarchique. Et l’abandon s’accroît à mesure que la hiérarchie se morcelle, que les maîtres se multiplient en rapetissant, que l’histoire démocratise le pouvoir. La jouissance imparfaite des maîtres est devenue jouissance des maîtres imparfaits. On a vu les maîtres bourgeois, plébéiens ubuesques, couronner leur révolte de brasserie par la fête funèbre du fascisme. Mais il n’y aura même plus de fête chez les maîtres-esclaves, chez les derniers hommes hiérarchisés ; seulement la tristesse des choses, une sérénité morose, le malaise du rôle, la conscience du « rien-être ».\par
Qu’adviendra-t-il de ces choses qui nous gouvernent ? Faudra-t-il les détruire ? Dans l’affirmative, les mieux préparés à liquider les esclaves au pouvoir sont ceux qui luttent depuis toujours contre l’esclavage. La créativité populaire, que n’ont brisée ni l’autorité des seigneurs ni celle des patrons, ne s’inféodera jamais à des nécessités programmatiques, à des plannings de technocrates. On dira qu’il y a, dans la liquidation d’une forme abstraite et d’un système, moins de passions et d’enthousiasme en œuvre que dans la mise à mort de maîtres abhorrés : c’est là envisager le problème dans le mauvais sens, le sens du pouvoir. Contrairement à la bourgeoisie, le prolétariat ne se définit pas par son adversaire de classe, il porte la fin de la distinction en classes et la fin de la hiérarchie. Le rôle de la bourgeoisie fut uniquement négatif. Saint-Just le rappelle superbement :\par

\begin{quoteblock}
\noindent « Ce qui constitue une république, c’est la destruction totale de ce qui lui est opposé. »\end{quoteblock}

\noindent Si la bourgeoisie se contente de forger des armes contre la féodalité, et partant, contre elle-même, le prolétariat au contraire contient en lui son dépassement possible. Il est la poésie momentanément aliénée par la classe dominante ou par l’organisation technocratique, mais toujours sur le point d’éclater. Unique dépositaire de la volonté de vivre, parce qu’il a connu jusqu’au paroxysme le caractère insupportable de la seule survie, le prolétariat brisera le mur des contraintes par le souffle de son plaisir et la violence spontanée de sa créativité. Toute la joie à prendre, tout le rire à s’offrir, il les détient déjà. C’est de lui-même qu’il tire sa force et sa passion. Ce qu’il s’apprête à construire détruira \emph{par surcroît} tout ce qui s’y oppose, comme, sur une bande magnétique, un enregistrement en efface un autre. La force des choses, le prolétariat, s’abolissant du même coup comme prolétariat, l’abolira par un geste de luxe, une sorte de nonchalance, une grâce que sait s’arroger celui qui prouve sa supériorité. Du nouveau prolétariat sortiront les maîtres sans esclaves, non les conditionnés de l’humanisme dont rêvent les onanistes de la gauche prétendument révolutionnaire. La violence insurrectionnelle des masses n’est qu’un aspect de la créativité du prolétariat, son impatience à se nier, comme il est impatient d’exécuter la sentence que la survie prononce contre elle-même.\par
Il me plaît de distinguer – distinction spécieuse – trois passions prédominantes, dans la destruction de l’ordre réifié. \emph{La passion de la puissance absolue}, une passion s’exerçant sur les objets mis immédiatement au service des hommes ; sans la médiation des hommes eux-mêmes. La destruction, donc, de ceux qui s’accrochent à l’ordre des choses, des esclaves possesseurs de pouvoir en miettes.\par

\begin{quoteblock}
 \noindent « Parce que nous n’en supportons plus l’aspect, nous supprimons les esclaves »\par
 
\bibl{(Nietzsche)}
 \end{quoteblock}

\noindent \emph{La passion de détruire les contraintes}, de briser les chaînes. C’est ce que dit Sade :\par

\begin{quoteblock}
\noindent « Les jouissances permises peuvent-elles se comparer aux jouissances qui réunissent à des attraits bien plus piquants ceux inappréciables de la rupture des freins sociaux et du renversement de toutes les lois ? »\end{quoteblock}

\noindent \emph{La passion de corriger un passé malheureux}, de revenir sur les espoirs déçus, tant dans la vie individuelle que dans l’histoire des révolutions écrasées. Comme il fut légitime de punir Louis XVI des crimes de ses prédécesseurs, il ne manque pas de raisons passionnantes, puisqu’il n’y a pas de vengeance possible sur des choses, pour effacer de la mémoire le souvenir, douloureux pour tout esprit libre, des fusillés de la Commune, des paysans torturés de 1525, des ouvriers assassinés, des révolutionnaires traqués et massacrés, de civilisations anéanties par le colonialisme, de tant de misères passées que le présent n’a jamais abolies. Il est devenu passionnant, parce que possible, de corriger l’histoire ; de noyer le sang de Babeuf, de Lacenaire, de Ravachol, de Bonnot dans le sang des obscurs descendants de ceux qui, esclaves d’un ordre fondé sur le profit et les mécanismes économiques, surent freiner cruellement l’émancipation humaine.\par
Le plaisir de jeter à bas le pouvoir, d’être maître sans esclave et de corriger le passé accorde à la subjectivité de chacun une place prépondérante. Dans le moment révolutionnaire, chaque homme est invité à faire lui-même sa propre histoire. \emph{La cause de la liberté de réalisation}, cessant du même coup d’être une cause, \emph{épouse toujours la subjectivité}. Seule une telle perspective permet l’ivresse des possibles, le vertige de toutes les jouissances mises à portée de tous.\par

\astermono

\noindent Éviter que le vieil ordre des choses ne s’effondre sur la tête de ses démolisseurs. L’avalanche du consommable risque de nous entraîner vers la chute finale, si nul ne veille à ménager des abris collectifs contre le conditionnement, le spectacle, l’organisation hiérarchique ; des abris d’où partiront les futures offensives. Les microsociétés actuellement en gestation vont réaliser le projet des maîtres anciens en le libérant de sa gangue hiérarchique. Le dépassement du « grand seigneur méchant homme » appliquera à la lettre l’admirable principe de Keats :\par

\begin{quoteblock}
\noindent « Tout ce qui peut être anéanti doit être anéanti pour que les enfants puissent être sauvés de l’esclavage. »\end{quoteblock}

\noindent Ce dépassement doit s’opérer simultanément sur trois points :\par
1° dépassement de l’organisation patriarcale ;\par
2° dépassement du pouvoir hiérarchisé ;\par
3° dépassement de l’arbitraire subjectif, du caprice autoritaire.\par
\labelchar{1. –} Le lignage contient la force magique de l’aristocratie, l’énergie transmise de génération en génération. En sapant la maîtrise féodale, la bourgeoisie est amenée, contre son gré, à saper la famille. Elle n’agit pas autrement envers l’organisation sociale… Cette négativité, je l’ai déjà dit, représente sûrement son aspect le plus riche, le plus « positif ». Mais ce qui manque à la bourgeoisie, c’est la possibilité de dépassement. Que sera le dépassement de la famille de type aristocratique ? Il faut répondre : la constitution de groupes cohérents où la créativité individuelle se trouve investie totalement dans la créativité collective, renforcée par elle ; où l’immédiateté du présent vécu prenne en charge le potentiel énergétique qui, chez les féodaux, provenait du passé. L’impuissance relative du maître immobilisé par son système hiérarchique ne laisse pas d’évoquer la faiblesse de l’enfant maintenu dans le cadre de la famille bourgeoise.\par
L’enfant acquiert une expérience subjective de la liberté, inconnue à toute espèce animale, mais il reste par ailleurs dans la dépendance objective de ses parents ; il a besoin de leurs soins, de leur sollicitude. Ce qui différencie l’enfant de l’animal tient à ce que l’enfant possède le sens de la transformation du monde, c’est-à-dire la poésie, à un degré illimité. En même temps, on lui interdit l’accès à des techniques que les adultes emploient la plupart du temps contre une telle poésie, et par exemple contre les enfants, en les conditionnant. Et quand les enfants accèdent enfin aux techniques, ils ont, sous le poids des contraintes, perdu dans leur maturité ce qui faisait la supériorité de leur enfance. L’univers des maîtres anciens tombe sous la même malédiction que l’univers des enfants : il n’a pas accès aux techniques de libération. Dès lors, il est condamné à rêver d’une \emph{transformation} du monde et à vivre selon les lois de l’\emph{adaptation} au monde. Dès l’instant où la bourgeoisie développe à un degré très élevé les techniques de transformation du monde, l’organisation hiérarchisée – que l’on est en droit de tenir pour le meilleur type de concentration d’énergie sociale dans un monde où l’énergie n’a pas le précieux appui des machines – apparaît comme un archaïsme, comme un frein au développement de la puissance humaine sur le monde. Le système hiérarchique, le pouvoir de l’homme sur l’homme, empêche de reconnaître les adversaires valables, il interdit la transformation réelle du milieu ambiant, pour l’emprisonner dans les nécessités d’adaptation à ce milieu et d’intégration à l’état de chose. C’est pourquoi :\par
\labelchar{2. –} Afin de détruire l’écran social qui aliène notre regard sur le monde, il importe de poser comme postulat le refus absolu de toute hiérarchie à l’intérieur du groupe. La notion même de dictature du prolétariat mérite une mise au point. La dictature du prolétariat est devenue la plupart du temps une dictature sur le prolétariat, elle est devenue une institution. Or, comme l’écrivait Lénine :\par
« La dictature du prolétariat est une lutte acharnée, sanglante et non sanglante, violente et pacifique, militaire et économique, pédagogique et administrative contre les forces et les traditions du Vieux Monde ».\par
Le prolétariat ne peut instaurer une domination durable, il ne peut exercer une dictature acceptée. Par ailleurs, la nécessité impérative de briser l’adversaire l’oblige à concentrer entre ses mains un pouvoir de répression fortement cohérent. Il s’agit donc de passer par une dictature qui se nie elle-même, comme le parti « dont la victoire doit être aussi la perte », comme le prolétariat lui-même. Le prolétariat doit, par sa dictature, mettre aussitôt sa négation à l’ordre du jour. Il n’a d’autre recours que de liquider en un bref laps de temps – aussi sanglant et aussi peu sanglant que les circonstances l’exigent – ceux qui entravent son projet de libération totale, ceux qui s’opposent à sa fin en tant que prolétariat. Il doit les détruire totalement, comme on détruit une vermine particulièrement prolifique. Et jusque dans chaque individu, il doit détruire les moindres velléités de prestige, les moindres prétentions hiérarchiques, susciter contre elles, c’est-à-dire contre les rôles, une sereine impulsion vers la vie authentique.\par
\labelchar{3. –} La fin des rôles implique le triomphe de la subjectivité. Et cette subjectivité enfin reconnue, et mise au centre des préoccupations, fait apparaître contradictoirement une nouvelle objectivité. Un nouveau monde des objets – une nouvelle nature, si l’on veut – va se reconstituer au départ des exigences de la subjectivité individuelle. Ici aussi, le rapport s’établit entre la perspective de l’enfance et celle des maîtres féodaux. Dans l’un et l’autre cas, bien que sur un mode différent, les possibles sont masqués par l’écran de l’aliénation sociale.\par
Qui ne se souvient ? Les solitudes enfantines s’ouvraient sur les immensités primitives, toutes les baguettes étaient magiques. Puis il a fallu s’adapter, devenir social et sociable. La solitude s’est dépeuplée, les enfants ont choisi malgré eux de vieillir, l’immensité s’est refermée comme un livre de contes. Personne en ce monde ne sort définitivement des cloaques de la puberté. Et l’enfance elle-même est lentement colonisée par la société de consommation. Les moins de dix ans vont-ils rejoindre les \emph{teen-agers} dans la grande famille des consommateurs, vont-ils vieillir plus vite dans une enfance « consommable » ? Impossible à ce stade de ne pas ressentir ce qu’il y a de similaire dans la déchéance historique des maîtres anciens et dans la déchéance croissante du royaume de l’enfance. Jamais la corruption de l’humain n’a atteint un tel paroxysme. Jamais nous n’avons été si lointainement proches de l’homme total.\par
Le caprice du maître ancien, du seigneur, a, sur le caprice de l’enfant, l’odieuse infériorité d’exiger l’oppression des autres hommes. Ce qu’il y a de subjectivité dans l’arbitraire féodal – selon mon gré, je te donne la richesse ou la mort – est gâché et entravé par la misère de sa réalisation. La subjectivité du maître ne se réalise en effet qu’en niant la subjectivité des autres, donc en se couvrant elle-même de chaînes ; s’enchaînant en enchaînant les autres.\par
L’enfant n’a pas ce privilège de l’imperfection. C’est d’un seul coup qu’il perd le droit à la subjectivité pure. On le taxe de puérilité, on l’incite à se conduire comme une grande personne. Et chacun grandit, refoulant son enfance jusqu’à ce que le gâtisme et l’agonie le persuadent qu’il a réussi à vivre en adulte.\par
Le jeu de l’enfant comme le jeu du grand seigneur a besoin d’être libéré, remis en honneur. Aujourd’hui, le moment est historiquement favorable. Il s’agit de sauver l’enfance en réalisant le projet des maîtres anciens ; l’enfance et sa subjectivité souveraine, l’enfance et ce rire qui est comme le bruissement de la spontanéité, l’enfance et cette façon de se brancher sur soi pour éclairer le monde, et cette façon d’illuminer les objets d’une lumière étrangement familière.\par
Nous avons perdu la beauté des choses, leur façon d’exister, en les laissant mourir entre les mains du pouvoir et des dieux. En vain la magnifique rêverie du surréalisme s’efforçait-elle de les ranimer par une irradiation poétique : la puissance de l’imaginaire ne suffit pas pour briser la gangue d’aliénation sociale qui emprisonne les choses ; elle n’arrive pas à les rendre au libre jeu de la subjectivité. Vus sous l’angle du pouvoir, une pierre, un arbre, un mixer, un cyclotron sont des objets morts, des croix plantées dans la volonté de les voir autres et de les changer. Et pourtant, au-delà de ce qu’on leur fait signifier, je sais que je les retrouverai exaltantes. Je sais ce qu’une machine peut susciter de passion dès qu’elle est mise au service du jeu, de la fantaisie, de la liberté. Dans un monde où tout est vivant, y compris les arbres et les pierres, il n’y a plus de signes contemplés passivement. Tout parle de joie. Le triomphe de la subjectivité rendra la vie aux choses ; et que les choses mortes dominent aujourd’hui insupportablement la subjectivité, n’est-ce pas, au fond, la meilleure chance historique d’arriver à un état de vie supérieur ?\par
De quoi est-il question ? De réaliser dans le langage actuel, c’est-à-dire dans la \emph{praxis}, ce qu’un hérétique déclarait à Ruysbroeck :\par

\begin{quoteblock}
\noindent « Dieu ne peut rien savoir, désirer ou faire sans moi. Avec Dieu, je me suis créé et j’ai créé toutes les choses, et c’est ma main qui soutient le ciel, la terre et toutes les créatures. Sans moi, rien n’existe. »\end{quoteblock}


\astermono

\noindent Il faut redécouvrir d’autres limites. Celles de l’aliénation sociale ont cessé, sinon de nous emprisonner, du moins de nous abuser. Pendant des siècles, les hommes sont restés devant une porte vermoulue, y perçant de petits trous d’épingle avec une facilité croissante. Un coup d’épaule suffit aujourd’hui pour l’abattre, c’est au-delà seulement que tout commence. Le problème du prolétariat n’est plus de prendre le pouvoir mais d’y mettre fin définitivement. De l’autre côté du monde hiérarchisé, les possibles viennent à notre rencontre. Le primat de la vie sur la survie sera le mouvement historique qui défera l’histoire. Nos adversaires valables sont encore à inventer ; à nous de chercher le contact, de les joindre sous le puéril revers des choses.\par
Verra-t-on les hommes renouer avec le cosmique un dialogue assez semblable à celui que durent connaître les premiers habitants de la terre, mais le renouer cette fois à l’étage supérieur, à l’étage surplombant la préhistoire, sans le respectueux tremblement des primitifs désarmés devant son mystère ? Imposer en somme au cosmos une signification humaine qui vienne avantageusement remplacer la signification divine dont il s’était chargé à l’aube des temps.\par
Et cet autre infini qu’est l’homme réel, ce corps, ces influx nerveux, ce travail de muscles, cette errance des rêves, se peut-il qu’il ne les gouverne un jour ? Se peut-il que la volonté individuelle enfin libérée par la volonté collective ne dépasse pas en prouesses le contrôle déjà sinistrement superbe que le conditionnement policier sait imposer à l’être humain ? D’un homme on fait un chien, une brique, un para, et l’on ne saurait faire un homme ?\par
Nous ne nous sommes jamais assez considérés comme infaillibles. Cette prétention, nous l’avons laissée – par orgueil peut-être – à des formes figées, à de grandes rides : le pouvoir, Dieu, le pape, le chef, les autres. Et pourtant, chaque fois que nous nous référions à la Société, à Dieu, à la Justice toute-puissante, c’est de notre pouvoir que nous parlions, mais si mal, il est vrai, si indirectement. Nous voici un étage au-dessus de la préhistoire. Une autre organisation humaine s’annonce, une organisation sociale où la créativité individuelle va laisser libre cours à son énergie, imprimer au monde les contours rêvés par chacun et harmonisés par tous.\par
Utopie ? Allons donc ? Quels sont ces reniflements de la condescendance ? Je ne connais pas un homme qui ne s’accroche à ce monde-là comme à ce qu’il a de plus cher. Et sans doute, beaucoup, lâchant prise, mettent à tomber autant d’ardeur désespérée qu’ils en mettaient à se cramponner. Chacun veut faire triompher sa subjectivité : il faut donc fonder l’union des hommes sur ce désir commun. Personne ne peut renforcer sa subjectivité dans l’aide des autres, sans l’aide d’un groupe devenu lui-même un centre de subjectivité, un reflet fidèle de la subjectivité de ses membres. L’Internationale situationniste est jusqu’à présent le seul groupe qui soit décidé à défendre la subjectivité radicale.
\section[{XXII. L’espace-temps du vécu et la correction du passé}]{XXII. L’espace-temps du vécu et la correction du passé}\renewcommand{\leftmark}{XXII. L’espace-temps du vécu et la correction du passé}


\begin{argument}\noindent La dialectique du pourrissement et du dépassement est la dialectique de l’espace-temps dissocié et de l’espace-temps unitaire (1). – Le nouveau prolétariat porte en soi la réalisation de l’enfance, son espace-temps (2). – L’histoire des séparations se résout lentement dans la fin de l’histoire « historisante » (3). – Temps cyclique et temps linéaire. – L’espace-temps vécu est l’espace-temps de la transformation, et l’espace-temps du rôle, celui de l’adaptation. – Quelle est la fonction du passé et de sa projection dans le futur ? Interdire le présent. L’idéologie historique est l’écran qui se dresse entre la volonté de réalisation individuelle et la volonté de construire l’histoire ; elle les empêche de fraterniser et de se confondre (4). – Le présent est l’espace-temps à construire ; il implique la correction du passé (5).
\end{argument}

\subsection[{1. L’espace-temps}]{\textsc{1.} L’espace-temps}
\noindent À mesure que les spécialistes organisent la survie de l’espèce et laissent à de savants schémas le soin de programmer l’histoire, la volonté de changer de vie en changeant le monde s’accroît partout dans le peuple. Si bien que chaque être particulier se voit confronté, au même titre que l’humanité tout entière, à un désespoir unanime au-delà duquel il n’y a que l’anéantissement ou le dépassement. Voici l’époque où l’évolution historique et l’histoire d’un individu tendent à se confondre parce qu’elles vont vers un commun aboutissement : l’état de chose et son refus. Et l’on dirait que l’histoire de l’espèce et les myriades d’histoires individuelles se rassemblent pour mourir ensemble ou pour ensemble recommencer TOUT. Le passé reflue vers nous avec ses germes de mort et ses ferments de vie. Et notre enfance est aussi au rendez-vous, menacée du mal de Loth.\par
Du péril suspendu au-dessus de l’enfance viendra, je veux le croire, le sursaut de révolte contre l’effroyable vieillissement auquel condamne la consommation forcée d’idéologies et de \emph{gadgets}. Il me plaît de souligner l’analogie de rêves et de désirs qui présentent indiscutablement la volonté féodale et la volonté subjective des enfants. En réalisant l’enfance, n’allons-nous pas réaliser le projet des maîtres anciens, nous les adultes de l’âge technocratique, riches de ce qui manque aux enfants, forts de ce qui fit défaut aux plus grands conquérants ? N’allons-nous pas identifier l’histoire et la destinée individuelle mieux que n’osèrent l’imaginer les fantaisies les plus débridées de Tamerlan et d’Héliogabale ?\par
Le primat de la vie sur la survie est le mouvement historique qui défera l’histoire. Construire la vie quotidienne, réaliser l’histoire, ces deux mots d’ordre, désormais, n’en font qu’un. Que sera la construction conjuguée de la vie et de la société nouvelle, que sera la révolution de la vie quotidienne ? Rien d’autre que le dépassement remplaçant le dépérissement, à mesure que la conscience du dépérissement effectif nourrit la conscience du dépassement nécessaire.\par
Si loin qu’ils remontent dans l’histoire, les essais de dépassement entrent dans l’actuelle poésie du renversement de perspective. Ils y entrent immédiatement, franchissant les barrières du temps et de l’espace, les brisant même. À coup sûr, la fin des séparations commence par la fin d’une séparation, celle de l’espace et du temps. Et de ce qui précède, il ressort que la reconstitution de cette unité primordiale passe par l’analyse critique de l’espace-temps des enfants, de l’espace-temps des sociétés unitaires, et de l’espace-temps des sociétés parcellaire porteuses de la décomposition et du dépassement enfin possible.
\subsection[{2. L’enfance}]{\textsc{2.} L’enfance}
\noindent Si nul n’y prend garde, le mal de survie fera bientôt d’un jeune homme un vieillard faustien encombré de regrets, aspirant à une jeunesse qu’il a traversée sans la reconnaître. Le \emph{teen-ager} porte les premières rides du consommateur. Peu de choses le séparent du sexagénaire ; il consomme plus et plus vite, gagnant une vieillesse précoce au rythme de ses compromis avec l’inauthentique. S’il tarde à se ressaisir, le passé se refermera derrière lui ; il ne reviendra plus sur ce qui a été fait, même pas pour le refaire. Beaucoup de choses le séparent des enfants auxquels hier encore il se mêlait. Il est entré dans la trivialité du marché, acceptant d’échanger contre sa représentation dans la société du spectacle la poésie, la liberté, la richesse subjective de l’enfance. Et cependant, s’il se reprend, s’il sort du cauchemar, quel ennemi vont devoir affronter les forces de l’ordre ? On le verra défendre les droits de son enfance avec les armes les plus redoutables de la technocratie sénile. On sait par quels prodiges les jeunes Simbas de la révolution lumumbiste s’illustrèrent, malgré leur équipement dérisoire ; que ne faut-il donc attendre d’une jeune génération pareillement colérée, mais armée avec plus de conséquence, et lâchée sur un théâtre d’opérations qui recouvre tous les aspects de la vie quotidienne ?\par
Car tous les aspects de la vie quotidienne sont en quelque sorte vécus d’une vie gestative dans l’enfance. L’accumulation d’événements vécus en peu de jours, en peu d’heures, empêche le temps de s’écouler. Deux mois de vacances sont une éternité. Deux mois pour le vieillard tiennent en une poignée de minutes. Les journées de l’enfant échappent au temps des adultes, elles sont du temps gonflé par la subjectivité, par la passion, par le rêve habité de réel. Au-dehors, les éducateurs veillent, ils attendent, montre en main, que l’enfant entre dans la ronde des heures. Ils \emph{ont} le temps. Et d’abord, l’enfant ressent comme une intrusion étrangère l’imposition par les adultes de leur temps à eux ; puis il finit par y succomber, il consent à vieillir. Ignorant tout des méthodes de conditionnement, il se laisse prendre au piège, comme un jeune animal. Quand, détenteur des armes de la critique, il voudra les braquer contre le temps, les années l’auront emporté loin de la cible. Il aura l’enfance au cœur comme une plaie ouverte.\par
Nous voici hantés par l’enfance tandis que, scientifiquement, l’organisation sociale la détruit. Les psychosociologues sont aux aguets, les prospecteurs de marché s’écrient déjà : « Regardez tous ces gentils petits dollars » (cité par Vance Packard). Un nouveau système décimal.\par
Dans les rues, des enfants jouent. L’un d’eux soudain se détache du groupe, s’avance vers moi, portant les plus beaux rêves de ma mémoire. Il m’enseigne – car mon ignorance sur ce point fut seule cause de ma perte – ce qui détruit la notion d’âge : la possibilité de vivre beaucoup d’événements ; non pas de les voir défiler, mais de les vivre, de les recréer sans fin. Et maintenant, à ce stade où tout m’échappe, où tout m’est révélé, comment ne surgirait-il pas sous tant de faux désirs une sorte d’instinct sauvage de totalité, une puérilité rendue redoutable par les leçons de l’histoire et de la lutte des classes ? La réalisation de l’enfance dans le monde adulte, comment le nouveau prolétariat n’en serait-il pas le plus pur détenteur ?\par
Nous sommes les découvreurs d’un monde neuf et cependant connu, auquel manque l’unité du temps et de l’espace ; un monde encore imprégné de séparations, encore morcelé. La semi-barbarie de notre corps, de nos besoins, de notre spontanéité (cette enfance enrichie de la conscience) nous procure des accès secrets qu’ont toujours ignorés les siècles aristocratiques, et que la bourgeoisie n’a jamais soupçonnés. Ils nous font pénétrer au labyrinthe des civilisations inachevées, et de tous les dépassements fœtalisés que l’histoire a conçus en se cachant. Nos désirs d’enfance retrouvés retrouvent l’enfance de nos désirs. Des profondeurs sauvages d’un passé, qui nous est tout proche et comme encore inaccompli, se dégage une nouvelle géographie des passions.
\subsection[{3. La fin de l’histoire}]{\textsc{3.} La fin de l’histoire}
\noindent Mobile dans l’immobile, le temps des sociétés unitaires est cyclique. Les êtres et les choses suivent leur cours en se déplaçant le long d’une circonférence dont le centre est Dieu. Ce Dieu-pivot, immuable bien que nulle part et partout, mesure la durée d’un pouvoir éternel. Il est sa propre norme et la norme de ce qui, gravitant à égale distance de lui, se déroule et revient sans s’écouler définitivement, sans se dénouer en fait jamais. « La treizième revient, c’est encore la première. »\par
L’espace des sociétés unitaires s’organise en fonction du temps. Comme il n’y a d’autre temps que celui de Dieu, il semble n’exister d’autre espace que l’espace contrôlé par Dieu. Cet espace s’étend du centre à la circonférence, du ciel à la terre, de l’Un au multiple. À première vue, le temps ne fait rien à l’affaire, il n’éloigne ni ne rapproche de Dieu. Par contre, l’espace est le chemin vers Dieu : la voie ascendante de l’élévation spirituelle et de la promotion hiérarchique. Le temps appartient en propre à Dieu, mais l’espace accordé aux hommes garde un caractère spécifiquement humain, irréductible. En effet, l’homme peut monter ou descendre, s’élever socialement ou déchoir, assurer son salut ou risquer la damnation. L’espace, c’est la présence de l’homme, le lieu de sa relative liberté, tandis que le temps l’emprisonne dans sa circonférence. Et qu’est-ce que le Jugement dernier, sinon Dieu ramenant le temps vers lui, le centre aspirant la circonférence et ramassant en son point immatériel la totalité de l’espace imparti à ses créatures. Annihiler la \emph{matière} humaine (son occupation de l’espace), c’est bien là le projet d’un maître impuissant à posséder tout à fait son esclave, donc incapable de ne pas se laisser partiellement posséder par lui.\par
La durée tient l’espace en laisse ; elle nous entraîne vers la mort, elle ronge l’espace de notre vie. Toutefois, la distinction n’apparaît pas si clairement au cours de l’histoire. Au même titre que les sociétés bourgeoises, les sociétés féodales sont, elles aussi, des sociétés de séparations, puisque la séparation tient à l’appropriation privative, mais elles possèdent sur les premières l’avantage d’une force de dissimulation étonnante.\par
La puissance du mythe réunit les éléments séparés, elle fait vivre unitairement, sur le mode de l’inauthentique, certes, mais dans un monde où l’inauthentique est Un et admis unanimement par une communauté cohérente (tribu, clan, royaume). Dieu est l’image, le symbole du dépassement de l’espace et du temps dissociés. Tous ceux qui « vivent » en Dieu participent de ce dépassement. La plupart y participent médiatement, c’est-à-dire qu’ils se conforment, dans l’espace de leur vie quotidienne, aux organisateurs de l’espace dûment hiérarchisé, du simple mortel à Dieu, aux prêtres, aux chefs. Pour prix de leur soumission, ils reçoivent l’offre d’une durée éternelle, la promesse d’une durée sans espace, l’assurance d’une pure temporalité en Dieu.\par
D’autres font peu de cas d’un tel échange. Ils ont rêvé d’atteindre au présent éternel que confère la maîtrise absolue sur le monde. On ne laisse pas d’être frappé par l’analogie entre l’espace-temps ponctuel des enfants et la volonté d’unité des grands mystiques. Ainsi Grégoire de Palamas (1341) peut-il décrire l’Illumination comme une sorte de conscience immatérielle de l’unité : « La lumière existe en dehors de l’espace et du temps. […] Celui qui participe de l’énergie divine devient lui-même en quelque sorte Lumière ; il est uni à la Lumière et, avec la Lumière, il voit en pleine conscience tout ce qui reste caché à ceux qui n’ont pas eu cette grâce. »\par
Cet espoir confus, qui ne pouvait être qu’indécis, voire indicible, l’ère transitoire bourgeoise l’a vulgarisé et précisé. Elle l’a concrétisé en donnant le coup de grâce à l’aristocratie et à sa spiritualité, elle l’a rendu possible en poussant à l’extrême sa propre décomposition. L’histoire des séparations se résout lentement dans la fin des séparations. L’illusion unitaire féodale s’incarne peu à peu dans l’unité libertaire de la vie à construire, dans un au-delà de la survie matériellement garantie.
\subsection[{4. Le présent interdit}]{\textsc{4.} Le présent interdit}
\noindent Einstein spéculant sur l’espace et le temps rappelle à sa façon que Dieu est mort. Sitôt que le mythe cesse de l’englober, la dissociation de l’espace et du temps jette la conscience dans un malaise qui fait les beaux jours du Romantisme (attrait des pays lointains, regret du temps qui fuit…).\par
Selon l’esprit bourgeois, qu’est-ce que le temps ? Le temps de Dieu ? Non plus, mais le temps du pouvoir, le temps du pouvoir parcellaire. Un temps en miettes dont l’unité de mesure est l’instant – cet instant qui essaie de se souvenir du temps cyclique. Non plus une circonférence mais une ligne droite finie et infinie ; non plus un synchronisme réglant chaque homme sur l’heure de Dieu mais une succession d’états où chacun se poursuit sans se rattraper, comme si la malédiction du Devenir le vouait à ne jamais se saisir que de dos, la face humaine restant inconnue, inaccessible, éternellement future ; non plus un espace circulaire embrassé par l’œil central du Tout-Puissant, mais une série de petits points autonomes en apparence, et en réalité s’intégrant, selon un certain rythme de succession, à la ligne qu’ils tracent chaque fois que l’un s’ajoute à l’autre.\par
Dans le sablier du Moyen Âge, le temps s’\emph{écoule} mais c’est le même sable qui passe et repasse d’un globe à l’autre. Sur le cadran \emph{circulaire} des montres, le temps s’égrène, ne revient jamais. Ironie des formes : l’esprit nouveau emprunte la sienne à une réalité morte, et c’est la mort du temps, la mort de son temps que la bourgeoisie habillait ainsi, du bracelet-montre à la pacotille de ses rêveries humanistes, d’une apparence cyclique.\par
Mais rien n’y fait, voici le temps des horlogers. L’impératif économique convertit chaque homme en chronomètre vivant, signe distinctif au poignet. Voici le temps du travail, du progrès, du rendement, le temps de production, de consommation, de \emph{planning} ; le temps du spectacle, le temps d’un baiser, le temps d’un cliché, un temps pour chaque chose (\emph{time is money}). Le temps-marchandise. Le temps de la survie.\par
L’espace est un point dans la ligne du temps, dans la machine qui transforme le futur en passé. Le temps contrôle l’espace vécu, mais il contrôle par l’extérieur, en le faisant passer, en le rendant transitoire. Pourtant, l’espace de la vie individuelle n’est pas un espace pur, et le temps qui l’entraîne n’est pas davantage une pure temporalité. Il vaut la peine d’examiner la question de plus près.\par
Chaque point qui termine la ligne du temps est unique, particulier, et cependant, que s’y ajoute le point suivant, le voilà noyé dans la ligne uniforme, digéré par un passé qui en a vu d’autres. Impossible de le distinguer. Chaque point fait donc progresser la ligne qui le fait disparaître.\par
C’est sur ce modèle, en détruisant et en remplaçant, que le pouvoir assure sa \emph{durée} mais, simultanément, les hommes incités à consommer le pouvoir le détruisent et le renouvellent en \emph{durant}. Si le pouvoir détruit tout, il se détruit ; s’il ne détruit rien, il est détruit. Il n’a de durée qu’entre les deux pôles de cette contradiction que la dictature du consommable rend de jour en jour plus aiguë. Et sa durée est subordonnée à la simple durée des hommes, c’est-à-dire à la permanence de leur survie. C’est pourquoi le problème de l’espace-temps dissocié se pose aujourd’hui en termes révolutionnaires.\par
L’espace vécu a beau être un univers de rêves, de désirs, de créativité prodigieuse, il n’est, dans l’ordre de la durée, qu’un point succédant à un autre point ; il s’écoule selon un sens précis, celui de sa destruction. Il paraît, s’accroît, disparaît dans la ligne anonyme du passé où son cadavre offre matière aux sautes de mémoire et aux historiens.\par
L’avantage du point d’espace vécu, c’est d’échapper en partie au système de conditionnement généralisé ; son inconvénient, de n’être rien par soi-même. L’espace de la vie quotidienne détourne un peu de temps à son profit, il l’emprisonne et le fait sien. En contrepartie, le temps de l’écoulement pénètre dans l’espace vécu et introvertit le sentiment de passage, de destruction, de mort. Je m’explique.\par
L’espace ponctuel de la vie quotidienne dérobe une parcelle de temps « extérieur », grâce auquel il se crée un petit espace-temps unitaire : c’est l’espace-temps des moments, de la créativité, du plaisir, de l’orgasme. Le lieu d’une telle alchimie est minuscule mais l’intensité vécue est telle qu’elle exerce sur la plupart des gens une fascination sans égale. Vu par les yeux du pouvoir, observé de l’extérieur, le moment passionné n’est qu’un point dérisoire, un instant drainé du futur au passé. Du présent comme présence subjective immédiate, la ligne du temps objectif ne sait rien et ne veut rien savoir. Et à son tour, la vie subjective resserrée en l’espace d’un point – ma joie, mon plaisir, mes rêveries – voudrait ne rien savoir du temps de l’écoulement, du temps linéaire, du temps des choses. Elle désire, au contraire, tout apprendre sur son présent, puisque après tout, elle n’est qu’un présent.\par
Au temps qui l’entraîne, l’espace vécu enlève donc une parcelle dont il fait son présent, dont il tente de faire son présent, car le présent est toujours à construire. C’est l’espace-temps unitaire de l’amour, de la poésie, du plaisir, de la communication… C’est le vécu sans temps morts. D’autre part, le temps linéaire, le temps objectif, le temps de l’écoulement pénètre à son tour dans l’espace imparti à la vie quotidienne. Il s’y introduit comme temps négatif, comme temps mort, comme reflet du temps de destruction. C’est le temps du \emph{rôle}, le temps qui à l’intérieur même de la vie incite à se désincarner, à répudier l’espace authentiquement vécu, à le restreindre et à lui préférer le paraître, la fonction spectaculaire. L’espace-temps créé par ce mariage hybride n’est autre que l’espace-temps de la survie.\par
Qu’est-ce que la vie privée ? L’amalgame, en un instant, en un \emph{point} entraîné vers sa destruction le long de la ligne de survie, d’un espace-temps réel (le moment) et d’un espace-temps falsifié (le rôle). Bien entendu, la structure de la vie privée n’obéit pas à une telle dichotomie. Il existe une interaction permanente. Ainsi les interdits qui cernent le vécu de toutes parts, et le refoulent dans un espace trop exigu, l’incitent à se changer en rôle, à entrer comme marchandise dans le temps de l’écoulement, à devenir du pur répétitif et à créer, comme temps accéléré, l’espace fictif du paraître. Tandis que simultanément, le malaise né de l’inauthentique, espace faussement vécu, renvoie à la recherche d’un temps réel, du temps de la subjectivité, du présent. De sorte que la vie privée est dialectiquement : \emph{un espace réel vécu} + \emph{un temps fictif spectaculaire} + \emph{un espace fictif spectaculaire} + \emph{un temps réel vécu}.\par
Plus le temps fictif compose avec l’espace fictif qu’il crée, plus on s’achemine vers l’état de chose, vers la pure valeur d’échange. Plus l’espace du vécu authentique compose avec le temps réellement vécu, plus la maîtrise de l’homme s’affermit. L’espace-temps unitairement vécu est le premier foyer de guérilla, l’étincelle du qualitatif dans la nuit qui dissimule encore la révolution de la vie quotidienne.\par
Non seulement, donc, le temps objectif s’acharne à détruire l’espace ponctuel en le précipitant dans le passé, mais encore il le ronge intérieurement en y introduisant ce rythme accéléré qui crée l’épaisseur du rôle (l’espace fictif du rôle résulte en effet de la répétition rapide d’une attitude, comme la répétition d’une image filmique donne l’apparence de vie). Le rôle installe dans la conscience subjective le temps de l’écoulement, du vieillissement, de la mort. Voilà le « pli auquel on a contraint la conscience » dont parle Antonin Artaud. Dominée extérieurement par le temps linéaire et intérieurement par le temps du rôle, la subjectivité n’a plus qu’à devenir une chose, une marchandise précieuse. Le processus s’accélère d’ailleurs historiquement. En effet, le rôle est désormais une consommation de temps dans une société où le temps reconnu est le temps de la consommation. Et une fois de plus, l’unité de l’oppression fait l’unité de la contestation. Qu’est-ce que la mort aujourd’hui ? L’absence de subjectivité et l’absence de présent.\par
La volonté de vivre réagit toujours unitairement. La plupart des individus se livrent, au profit de l’espace vécu, à un véritable détournement du temps. Si leurs efforts pour renforcer l’intensité du vécu, pour accroître l’espace-temps de l’authentique ne se perdaient dans la confusion et ne se fragmentaient dans l’isolement, qui sait si le temps objectif, le temps de la mort, ne se briserait pas ? Le moment révolutionnaire n’est-il pas une éternelle jeunesse ?\par

\astermono

\noindent Le projet d’enrichir l’espace-temps du vécu passe par l’analyse de ce qui l’appauvrit. Le temps linéaire n’a d’emprise sur les hommes que dans la mesure où il leur interdit de \emph{transformer} le monde, dans la mesure où il les contraint donc à s’\emph{adapter}. Pour le pouvoir, l’ennemi numéro UN, c’est la créativité individuelle s’irradiant librement. Et la force de la créativité est dans l’unitaire. Comment le pouvoir s’efforce-t-il de briser l’unité de l’espace-temps vécu ? En transformant le vécu en marchandise, en le jetant sur le marché du spectacle au gré de l’offre et de la demande des rôles et des stéréotypes. C’est ce que j’ai étudié dans les pages consacrées au rôle (paragraphe XV). En recourant à une forme particulière d’identification : l’attraction conjuguée du passé et du futur, qui annihile le présent. Enfin, en essayant de récupérer dans une idéologie de l’histoire la volonté de construire l’espace-temps unitaire du vécu (autrement dit, de construire des situations à vivre). J’examinerai ces deux derniers points.\par

\astermono

\noindent Sous l’angle du pouvoir, il n’y a pas de moments vécus (le vécu n’a pas de nom), mais seulement des instants qui se succèdent, tous égaux dans la ligne du passé. Tout un système de conditionnement vulgarise cette façon de voir, toute une persuasion clandestine l’introjecte. Le résultat est là. Où est-il, ce présent dont on parle ? Dans quel coin perdu de l’existence quotidienne ?\par
Tout est mémoire et anticipation. Le fantôme du rendez-vous à venir rejoint pour me hanter le fantôme du rendez-vous passé. Chaque seconde m’entraîne de l’instant qui fut à l’instant qui sera. Chaque seconde m’abstrait de moi-même ; il n’y a jamais de maintenant. Une agitation creuse met son point d’efficacité à rendre chacun passager, à faire passer le temps, comme on dit si joliment, et même à faire passer le temps dans l’homme de part en part. Quand Schopenhauer écrit :\par

\begin{quoteblock}
\noindent « Avant Kant, nous étions dans le temps ; depuis Kant, c’est le temps qui est en nous »,\end{quoteblock}

\noindent  il traduit bien l’irrigation de la conscience par le temps du vieillissement et de la décrépitude. Mais il n’entre pas dans l’esprit de Schopenhauer que l’écartèlement de l’homme, sur le chevalet d’un temps réduit à la divergence apparente d’un futur et d’un passé, soit la raison qui le pousse, en tant que philosophe, à édifier sa mystique du désespoir.\par
Et quel n’est pas le désespoir et le vertige d’un être distendu entre deux instants qu’il poursuit en zigzags, sans jamais les atteindre, sans jamais se saisir. Encore, s’il s’agissait de l’attente passionnée : vous voici sous le charme d’un moment passé, un moment d’amour, par exemple, et la femme aimée va paraître, vous la devinez, vous pressentez ses caresses… L’attente passionnée est, en somme, l’ombre de la situation à construire. Mais la plupart du temps, il faut bien l’avouer, la ronde du souvenir et de l’anticipation empêche l’attente et le sentiment du présent, elle précipite la course folle des temps morts et des instants vides.\par
Dans la lutte du pouvoir, il n’y a de futur que de passé réitéré. Une dose d’inauthentique connu est propulsée par ce que l’on appelle l’imagination prospective dans un temps qu’elle remplit à l’avance de sa parfaite vacuité. Les seuls souvenirs sont ceux des rôles qu’on a tenus, le seul futur un éternel \emph{remake}. La mémoire des hommes ne doit qu’obéir à la volonté du pouvoir de s’affirmer dans le temps comme une constante mémorisation de sa présence. Un \emph{nihil novi sub sole}, vulgairement traduit par « il faut toujours des dirigeants ».\par
L’avenir proposé sous l’étiquette de « temps autre » répond dignement à l’espace autre où l’on m’invite à m’épancher. Changer de temps, changer de peau, changer d’heure, changer de rôle ; seule l’aliénation ne change pas. Chaque fois que \emph{je} est un autre, il va et vient du passé au futur. Les rôles n’ont jamais de présent. Comment voudrait-on que les rôles se portent bien ? Si je rate mon présent, – ici étant toujours ailleurs, – pourrais-je me trouver environné de passé et de futur agréable ?\par

\astermono

\noindent L’identification au passé-futur trouve son couronnement dans le coup de l’idéologie historique, qui fait avancer sur la tête la volonté individuelle et collective de dominer l’histoire.\par
Le temps est une forme de perception de l’esprit, non pas évidemment une invention de l’homme mais un rapport dialectique avec la réalité extérieure, une relation tributaire par conséquent de l’aliénation et de la lutte des hommes dans et contre cette aliénation.\par
Absolument soumis à l’adaptation, l’animal ne possède pas la conscience du temps. L’homme, lui, refuse l’adaptation, il prétend transformer le monde. Chaque fois qu’il échoue dans sa volonté de démiurge, il connaît l’angoisse de s’adapter, l’angoisse de se sentir réduit à la passivité de l’animal. La conscience de l’adaptation nécessaire est la conscience du temps qui s’écoule. C’est pourquoi le temps est lié à l’angoisse humaine. Et plus la nécessité de s’adapter aux circonstances l’emporte sur le désir et la possibilité de les changer, plus la conscience du temps prend l’homme à la gorge. Le mal de survie est-il rien d’autre que la conscience aiguë de l’écoulement dans le temps et dans l’espace de l’autre, la conscience de l’aliénation ? Refuser la conscience du vieillissement et les conditions objectives du vieillissement de la conscience implique une exigence plus grande à vouloir faire l’histoire, avec plus de conséquence et selon les vœux de la subjectivité de tous.\par
La seule raison d’une idéologie historique est d’empêcher les hommes de faire l’histoire. Comment distraire les hommes de leur présent, sinon en les attirant dans les zones d’écoulement du temps ? Ce rôle incombe à l’historien. L’historien organise le passé, il le fragmente selon la ligne officielle du temps, puis il range les événements dans les catégories \emph{ad hoc}. Ces catégories, d’un emploi aisé, mettent l’événement en quarantaine. De solides parenthèses l’isolent, le contiennent, l’empêchent de prendre vie, de ressusciter, de déferler à nouveau dans les rues de notre quotidienneté. L’événement est congelé. Défense de le rejoindre, de le refaire, de le parfaire, de le mener vers son dépassement. Il est là, suspendu de toute éternité pour l’admiration des esthètes. Un léger changement d’indice et le voici transporté du passé dans le futur. L’avenir n’est qu’une redite des historiens. Le futur qu’ils annoncent est un collage de souvenirs, de leurs souvenirs. Vulgarisée par les penseurs staliniens, la notion fameuse du sens de l’Histoire a fini par vider de toute humanité l’avenir comme le passé.\par
Pressé de s’identifier à un autre temps et à un autre personnage, l’individu contemporain a réussi à se laisser voler son présent sous le patronage de l’historicisme. Il perd dans un espace-temps spectaculaire (« Vous entrez dans l’Histoire, camarades ! »), le goût de vivre authentiquement. Du reste, à ceux qui refusent l’héroïsme de l’engagement historique, le secteur psychologique apporte sa mystification complémentaire. Les deux catégories s’épaulent, elles fusionnent dans l’extrême misère de la récupération. On choisit l’histoire ou la petite vie tranquille.\par
Historique ou non, tous les rôles pourrissent. La crise de l’histoire et la crise de la vie quotidienne se confondent. Le mélange sera détonant. Il s’agit désormais de détourner l’histoire à des fins subjectives ; avec l’appui de tous les hommes. Marx, en somme, n’a rien voulu d’autre.
\subsection[{5. Le présent reconstruit, par correction du passé}]{\textsc{5.} Le présent reconstruit, par correction du passé}
\noindent Depuis près d’un siècle, les mouvements picturaux significatifs n’ont cessé de se donner comme un jeu – voire comme une plaisanterie – sur l’espace. Rien ne pouvait mieux que la créativité artistique exprimer la recherche inquiète et passionnée d’un nouvel espace vécu. Et comment traduire, sinon par l’humour (je pense aux débuts de l’impressionnisme, au pointillisme, au fauvisme, au cubisme, aux collages dadaïstes, aux premiers abstraits) le sentiment que l’art n’apportait plus guère de solution valable ?\par
Le malaise, d’abord sensible chez l’artiste, a gagné à mesure que l’art se décomposait, la conscience d’un nombre croissant de gens. Construire un art de vivre est aujourd’hui une revendication populaire. Il faut concrétiser dans un espace-temps passionnément vécu les recherches de tout un passé artistique, vraiment abandonnées de façon inconsidérée.\par
Les souvenirs ici sont souvenirs de blessures mortelles. Ce qui ne s’achève pas pourrit. On a fait du passé l’irrémédiable et, par un comble d’ironie, ceux qui en parlent comme d’un donné définitif ne cessent de le triturer, de le falsifier, de l’arranger au goût du jour à la façon du pauvre Wilson récrivant, dans le \emph{1984} d’Orwell, des articles de journaux officiels anciens, contredits par l’évolution postérieure des événements.\par
Il n’existe qu’un type d’oubli admissible, celui qui efface le passé en le réalisant. Celui qui sauve de la décomposition par le dépassement. Les faits, si loin qu’ils se situent, n’ont jamais dit leur dernier mot. Il suffit d’un changement radical dans le présent pour qu’ils dévalent de leur étagère et tombent à nos pieds. Sur la correction du passé, je ne connais guère de témoignage plus touchant que celui rapporté par Victor Serge dans \emph{Ville conquise}. Je ne veux pas en connaître de plus exemplaire.\par
À l’issue d’une conférence sur la Commune de Paris, donnée au plus fort de la révolution bolchevique, un soldat se lève lourdement de son fauteuil de cuir, au fond de la salle. « On l’entendit très bien murmurer d’un ton de commandement :\par
« — Racontez l’exécution du docteur Millière.\par
« Debout, massif, le front penché de sorte que l’on ne voyait de son visage que les grosses joues poilues, les lèvres boudeuses, le front bosselé et ridé – il ressemblait à certains masques de Beethoven –, il écouta ce récit : le docteur Millière, en redingote bleu foncé et chapeau haut de forme, conduit sous la pluie à travers les rues de Paris, – agenouillé de force sur les marches du Panthéon, – criant : « Vive l’humanité ! » – le mot du factionnaire versaillais accoudé à la grille quelques pas plus loin : « On va t’en foutre, de l’humanité ! »\par
« Dans la nuit noire de la rue sans lumière le bonhomme de terre rejoignit le conférencier. […] Il avait un secret au bord des lèvres. Son mutisme d’un instant fut chargé.\par
«- J’ai aussi été dans le gouvernement de Perm, l’an dernier, quand les koulaks se sont soulevés. […] Moi, j’avais lu en route la brochure d’Arnould : \emph{Les Morts de la Commune}. Une belle brochure. Je pensais à Millière. Et j’ai vengé Millière, citoyen ! C’est un beau jour dans ma vie qui n’en a pas beaucoup. Point par point, je l’ai vengé. J’ai fusillé comme ça, sur le seuil de l’église, le plus gros propriétaire de l’endroit ; je ne sais plus son nom et je m’en fous…\par
« Il ajouta après un court silence :\par
«- Mais c’est moi qui ai crié : « Vive l’humanité ! »\par
Les révoltes anciennes prennent dans mon présent une dimension nouvelle, celle d’une réalité immanente à construire sans tarder. Les allées du Luxembourg, le square de la Tour Saint-Jacques résonnent encore des fusillades et des cris de la Commune écrasée. Mais d’autres fusillades viendront, et d’autres charniers effaceront jusqu’au souvenir du premier. Pour laver le mur des Fédérés avec le sang des fusilleurs, les révolutionnaires de tous les temps rejoindront quelque jour les révolutionnaires de tous les pays.\par
Construire le présent, c’est corriger le passé, changer les signes du paysage, libérer de leur gangue les rêves et les désirs inassouvis, laisser les passions individuelles s’harmoniser dans le collectif. Des insurgés de 1525 aux rebelles mulélistes, de Spartacus à Pancho Villa, de Lucrèce à Lautréamont, il n’y a que le temps de ma volonté de vivre.\par
L’espérance de lendemain trouble nos fêtes. L’avenir est pire que l’Océan ; il ne contient rien. Planification, perspective, plan à long terme… autant spéculer sur le toit de la maison alors que le premier étage n’existe pas. Et pourtant, si tu construis bien le présent, le reste viendra de surcroît.\par
Seul m’intéresse le vif du présent, sa multiplicité. Je veux, en dépit des interdits, m’environner d’aujourd’hui comme d’une grande lumière ; ramener le temps autre et l’espace des autres à l’immédiateté de l’expérience quotidienne. Concrétiser la formule de Schwester Katrei :\par

\begin{quoteblock}
\noindent « Tout ce qui est en moi est en moi, tout ce qui est en moi est en dehors de moi, tout ce qui est en moi est partout autour de moi, tout ce qui est en moi est à moi et je ne vois partout que ce qui est en moi. »\end{quoteblock}

\noindent  Car il n’y a là que le juste triomphe de la subjectivité, tel que l’histoire le permet aujourd’hui ; pour peu que l’on détruise les bastilles du futur, pour peu que l’on restructure le passé, pour peu que l’on vive chaque seconde comme si, à la faveur d’un éternel retour, elle devait en des cycles sans fin se répéter exactement.\par
Il n’y a que le présent qui puisse être total. Un point d’une densité incroyable. Il faut apprendre à ralentir le temps, à vivre la passion permanente de l’expérience immédiate. Un champion de tennis a raconté que, au cours d’un match âprement disputé, il reçut une balle très difficile à reprendre. Soudain, il la vit s’approcher de lui au ralenti, si lentement qu’il eut le temps de juger la situation, de prendre une décision adéquate et de porter un coup de grande maîtrise. Dans l’espace de la création, le temps se dilate. Dans l’inauthenticité, le temps s’accélère. À qui possédera la poétique du présent adviendra l’aventure du petit Chinois amoureux de la Reine des Mers. Il partit à sa recherche au fond des océans. Quand il revint sur terre, un très vieil homme qui taillait des roses lui dit : « Mon grand-père m’a parlé d’un petit garçon disparu en mer, qui portait justement votre nom. »\par
« La ponctualité est la réserve du temps », dit la tradition ésotérique. Quant à cette phrase de la \emph{Pistis Sophia} : « Un jour de lumière est un millier d’années du monde », elle s’est traduite précisément dans le bain révélateur de l’histoire par le mot de Lénine constatant qu’il y a des journées révolutionnaires qui valent des siècles.\par
Il s’agit toujours de résoudre les contradictions du présent, de ne pas s’arrêter à mi-chemin, de ne pas se laisser « distraire », d’aller vers le dépassement. Œuvre collective, œuvre de passion, œuvre de poésie, œuvre du jeu (l’éternité est le monde du jeu, dit Boehme). Si pauvre soit-il, le présent contient toujours la vraie richesse, celle de la construction possible. Mais ce poème ininterrompu qui me réjouit, vous savez assez – vous vivez assez – tout ce qui me l’arrache des mains.\par
Succomber au tourbillon des temps morts, vieillir, s’user jusqu’au vide du corps et de l’esprit ? Plutôt disparaître comme un défi à la durée. Le citoyen Anquetil rapporte dans son \emph{Précis de l’histoire universelle}, paru à Paris en l’an VII de la République, qu’un prince persan, blessé par la vanité du monde, se retira dans un château, accompagné de quarante courtisanes, parmi les plus belles et les plus lettrées du royaume. Il y mourut au bout d’un mois dans l’excès des plaisirs. Mais qu’est-ce que la mort au regard d’une telle éternité ? S’il faut que je meure, que ce soit du moins comme il m’est arrivé d’aimer.
\section[{XXIII. La triade unitaire : réalisation – communication – participation}]{XXIII. La triade unitaire : réalisation – communication – participation}\renewcommand{\leftmark}{XXIII. La triade unitaire : réalisation – communication – participation}


\begin{argument}\noindent L’unité répressive du pouvoir dans sa triple fonction de contrainte, de séduction et de médiation n’est que la forme, inversée et pervertie par les techniques de dissociation, d’un triple projet unitaire. La société nouvelle, telle qu’elle s’élabore confusément dans la clandestinité, tend à se définir pratiquement comme une transparence de rapports humains favorisant la participation réelle de tous à la réalisation de chacun. – La passion de la création, la passion de l’amour, et la passion du jeu sont à la vie ce que le besoin de se nourrir et le besoin de se protéger sont à la survie (1). – La passion de créer fonde le projet de réalisation (2), la passion d’aimer fonde le projet de communication (4), la passion de jouer fonde le projet de participation (6). – Dissociés, ces trois projets renforcent l’unité répressive du pouvoir. – La subjectivité radicale est la présence – actuellement repérable chez la plupart des hommes – d’une même volonté de se construire une vie passionnante (3). L’érotique est la cohérence spontanée qui donne son unité pratique à l’enrichissement du vécu (5).
\end{argument}

\subsection[{1. Passions de la vie, besoins de survie}]{\textsc{1.} Passions de la vie, besoins de survie}
\noindent La construction de la vie quotidienne réalise au plus haut degré l’unité du rationnel et du passionnel. Le mystère entretenu depuis toujours sur la vie relève de l’obscurantisme où se dissimule la trivialité de la survie. De fait, la volonté de vivre est inséparable d’une certaine volonté d’organisation. L’attrait qu’exerce sur chaque individu la promesse d’une vie riche et multiple prend nécessairement l’allure d’un projet soumis en tout ou partie au pouvoir social chargé de le refréner. De même que le gouvernement des hommes recourt essentiellement à un triple mode d’oppression : la contrainte, la médiation aliénante et la séduction magique ; de même la volonté de vivre puise sa force et sa cohérence dans l’unité de trois projets indissociables : la réalisation, la communication, la participation.\par
Dans une histoire des hommes qui ne se réduirait pas à l’histoire de leur survie, sans par ailleurs s’en dissocier, la dialectique de ce triple projet, alliée à la dialectique des forces productives, rendrait compte de la plupart des comportements. Pas une émeute, pas une révolution, qui ne révèle une recherche passionnée de la vie exubérante, d’une transparence dans les rapports humains et d’un mode collectif de transformation du monde. Si bien qu’en deçà de l’évolution historique, il semble que l’on puisse déceler trois passions fondamentales, qui sont à la vie ce que le besoin de se nourrir et de se protéger est à la survie. La passion de la création, la passion de l’amour, la passion du jeu agissent en interaction avec le besoin de se nourrir et de se protéger, comme la volonté de vivre interfère sans cesse avec la nécessité de survivre. Bien entendu, ces éléments ne prennent une importance que dans le cadre historique, mais c’est précisément l’histoire de leur dissociation qui est ici mise en cause, au nom de leur totalité toujours revendiquée.\par
Le Welfare State tend à englober la question de la survie dans une problématique de la vie. Je l’ai montré plus haut. Dans cette conjecture historique où l’économie de la vie absorbe peu à peu l’économie de survie, la dissociation des trois projets, et des passions qui les sous-tendent, apparaît distinctement comme un prolongement de la distinction aberrante entre vie et survie. Entre la séparation, qui est le fief du pouvoir, et l’unité, qui est le domaine de la révolution, l’existence n’a la plupart du temps que l’ambiguïté pour s’exprimer : je parlerai donc séparément et unitairement de chaque projet.\par

\astermono

\noindent Le projet de réalisation naît de la passion de créer, dans le moment où la subjectivité se gonfle et veut régner partout. Le projet de communication naît de la passion de l’amour, chaque fois que des êtres découvrent en eux une volonté identique de conquêtes. Le projet de participation naît de la passion du jeu, quand le groupe aide à la réalisation de chacun.\par
Isolées, les trois passions se pervertissent. Dissociés, les trois projets se falsifient. La volonté de réalisation devient volonté de puissance ; sacrifiée au prestige et au rôle, elle règne dans un univers de contraintes et d’illusions. La volonté de communication tourne au mensonge objectif ; fondée sur des rapports d’objets, elle distribue aux sémiologues les signes qu’ils habillent d’une apparence humaine. La volonté de participation organise l’isolement de tous dans la foule, elle crée la tyrannie de l’illusion communautaire.\par
Coupée des autres, chaque passion s’intègre dans une vision métaphysique qui l’absolutise et la rend, comme telle, inaccessible. Les hommes de pensée ne manquent pas d’humour : ils déconnectent les éléments du circuit puis annoncent que le courant ne passera pas. Ils peuvent alors affirmer, sans filet, que la réalisation totale est un leurre, la transparence une chimère, l’harmonie sociale une lubie. Où la séparation règne, chacun est vraiment tenu à l’impossible. La manie cartésienne de morceler et de progresser par degrés garantit toujours l’inaccompli et le boiteux. Les armées de l’Ordre ne recrutent que des mutilés.
\subsection[{2. Le projet de réalisation}]{\textsc{2. }Le projet de réalisation}

\begin{argument}\noindent L’assurance d’une sécurité d’existence laisse sans emploi une grande quantité d’énergie jadis absorbée par la nécessité de survivre. La volonté de puissance tente de récupérer, au profit de l’esclavage hiérarchisé, cette énergie disponible pour la libre expansion de la vie individuelle (1). Le conditionnement de l’oppression généralisée provoque chez la plupart des hommes un repli stratégique vers ce qu’ils sentent en eux d’irréductibles : leur subjectivité. La révolution de la vie quotidienne se doit de concrétiser l’offensive que le centre subjectif lance cent fois par jour en direction du monde objectif (2).
\end{argument}

\subsubsection[{1. La volonté de puissance}]{\textsc{1.} La volonté de puissance}
\noindent La phase historique de l’appropriation privative a empêché l’homme d’être le Dieu créateur qu’il a dû se résoudre à créer idéalement pour homologuer son échec. Le désir d’être Dieu est au cœur de chaque homme, mais ce désir s’est exercé jusqu’à présent contre l’homme lui-même. J’ai montré comment l’organisation sociale hiérarchisée construit le monde en détruisant les hommes, comment le perfectionnement de son mécanisme et de ses réseaux la fait fonctionner comme un ordinateur géant dont les programmateurs sont aussi programmés, comment, enfin, le plus froid des monstres froids trouve son accomplissement dans le projet d’État cybernétisé.\par
Dans ces conditions, la lutte pour le pain quotidien, le combat contre l’inconfort, la recherche d’une stabilité d’emploi et d’une sécurité d’existence sont, sur le front social, autant de raids offensifs qui prennent lentement mais sûrement l’allure d’engagements d’arrière-garde (ceci dit sans en sous-estimer l’importance). La nécessité de survivre absorbait et continue d’absorber une dose d’énergie et de créativité dont l’état de bien-être va hériter comme d’une meute de loups déchaînés. En dépit de faux engagements et d’activités illusoires, l’énergie créatrice sans cesse stimulée ne se dissout plus assez vite sous la dictature du consommable. Qu’adviendra-t-il de cette exubérance soudain disponible, de ce surplus de robustesse et de virilité que ni les contraintes, ni le mensonge ne réussissent à user vraiment ? Non récupérée par la consommation artistique et culturelle – par le spectacle idéologique – la créativité se tourne spontanément contre les conditions et les garanties de survie.\par
Les hommes de la contestation n’ont à perdre que leur survie. Toutefois, ils peuvent la perdre de deux façons : en perdant la vie ou en la construisant. Puisque la survie est une sorte de mort lente, il existe une tentation, non dépourvue de raisons passionnées, de précipiter le mouvement et de mourir plus vite, un peu comme on pousse sur l’accélérateur d’une voiture de sport. On « vit » alors dans le négatif la négation de la survie. Ou, au contraire, les gens peuvent s’efforcer de survivre comme antisurvivants, en concentrant leur énergie sur l’enrichissement de leur vie quotidienne. Ils nient la survie mais en l’englobant dans une fête constructiviste. On reconnaîtra dans ces deux tendances la voie Une et contradictoire du pourrissement et du dépassement.\par
Le projet de réalisation est inséparable du dépassement. Le refus désespéré reste, quoi qu’il en est, prisonnier du dilemme autoritaire : la survie ou la mort. Ce refus acquiesçant, cette créativité sauvage et si aisément domptée par l’ordre des choses, c’est la \emph{volonté de puissance}.\par

\astermono

\noindent La volonté de puissance est le projet de réalisation falsifié, coupé de la participation et de la communication. C’est la passion de créer et de se créer, emprisonnée dans le système hiérarchique, condamnée à faire tourner les meules de la répression et de l’apparence. Prestige et humiliation, autorité et soumission, voilà le pas de manœuvre de la volonté de puissance. Le héros est celui qui sacrifie à la promotion du rôle et du muscle. Quand il est fatigué, il se range au conseil de Voltaire, il cultive son jardin. Et sa médiocrité sert encore de modèle, sous sa forme pataude, au commun des mortels.\par
Que de renoncements à la volonté de vivre chez le héros, le dirigeant, la vedette, le \emph{play-boy}, le spécialiste… Que de sacrifices pour imposer à des gens, – qu’ils soient deux ou des millions, – que l’on tient pour de parfaits imbéciles, à moins de l’être soi-même, sa photo, son nom, une teinture de respect !\par
Pourtant, la volonté de puissance contient, sous son emballage protecteur, une dose certaine de volonté de vivre. Je pense à la \emph{virtû} du \emph{condottiere}, à l’exubérance des géants de la Renaissance. Mais de nos jours, il n’y a plus de \emph{condottieri}. Tout au plus des capitaines d’industrie, des escrocs, des marchands de canon et de culture, des mercenaires. L’aventurier et l’explorateur s’appellent Tintin et Schweitzer. Et c’est avec ces gens-là que Zarathoustra médite de peupler les hauteurs de Sils-Maria, c’est dans ces avortons qu’il prétend découvrir le signe d’une race nouvelle. En vérité, Nietzsche est le dernier maître, crucifié par sa propre illusion. Sa mort réédite, en plus piquant, en plus spirituel, la comédie du Golgotha. Elle donne un sens à la disparition des maîtres, comme le Christ donnait un sens à la disparition de Dieu. Nietzsche a beau être sensible au dégoût, l’odeur ignoble du christianisme ne l’empêche pas de respirer à pleins poumons. Et comme il feint de ne pas comprendre que le christianisme, grand contempteur de la volonté de puissance, en est le meilleur protecteur, son \emph{racketter} le plus fidèle, puisque empêchant l’apparition des maîtres sans esclaves, Nietzsche consacre la permanence du monde hiérarchisé, où la volonté de vivre se condamne à n’être jamais que volonté de puissance. La formule « Dionysos le Crucifié », dont il signe ses derniers écrits, trahit bien l’humilité de celui qui n’a fait que chercher un maître à son exubérance mutilée. On n’approche pas impunément le sorcier de Bethléem.\par
Le nazisme est la logique nietzschéenne rappelée à l’ordre par l’histoire. La question était : que peut devenir le dernier des maîtres dans une société où les vrais maîtres ont disparu ? La réponse fut : un super-valet. Même l’idée de surhomme, si pauvre soit-elle chez Nietzsche, jure violemment avec ce que nous savons des larbins qui dirigèrent le III\textsuperscript{e} Reich. Pour le fascisme, un seul surhomme, L’État.\par
Le surhomme étatique est la force des faibles. C’est pourquoi les revendications de l’individu isolé s’accommodent toujours d’un rôle impeccablement tenu dans le spectacle officiel. La volonté de puissance est une volonté spectaculaire. L’homme seul déteste les autres, méprise les hommes tout en étant l’homme de la foule, l’homme méprisable par excellence. Son agressivité se plaît à faire fond sur l’illusion communautaire la plus grossière, sa combativité s’exerce dans la chasse aux promotions.\par
Le manager, le chef, le dur, le caïd a dû trimer, encaisser, tenir bon. Sa morale est celle des pionniers, des scouts, des armées, des groupes de choc du conformisme. « Ce que j’ai fait, aucune bête au monde ne l’aurait fait… » Une volonté de paraître à défaut d’être, une façon d’ignorer le vide de son existence en affirmant rageusement que l’on existe, voilà ce qui définit le caïd. Seuls les valets, s’enorgueillissent de leurs sacrifices. La part des choses est ici souveraine : tantôt l’artifice du rôle, tantôt l’authenticité de l’animal. Ce que l’homme refuse d’accomplir, la bête le fait. Les héros qui défilent, musique en tête, armée Rouge, S. S., paras, sont les mêmes qui torturèrent à Budapest, à Varsovie, à Alger. La fureur des troufions fait la discipline des armées ; la chiennerie policière connaît le temps de mordre et de ramper.\par
La volonté de puissance est une prime à l’esclavage. Elle est aussi une haine de l’esclavage. Jamais les grandes individualités du passé ne se sont identifiés à une Cause. Elles ont préféré assimiler la Cause à leur propre désir de puissance. Les grandes causes disparues, émiettées, les individualités se sont pareillement décomposées. Néanmoins le jeu reste. Les gens adoptent une Cause parce qu’ils n’ont pu s’adopter, eux et leurs désirs ; mais à travers la Cause et le sacrifice exigé, c’est leur volonté de vivre qu’ils poursuivent à rebours.\par
Parfois le sens de la liberté et du jeu s’éveille chez les irréguliers de l’Ordre. Je pense à Giulano, avant sa récupération par les propriétaires terriens, à « Billy the Kid », à des gangsters proches par instants des terroristes. On a vu des légionnaires et des mercenaires passer du côté des rebelles algériens ou congolais, choisissant ainsi le parti de l’insurrection ouverte et menant le goût du jeu jusqu’à ses conséquences extrêmes : la rupture de tous les interdits et le postulat de la liberté totale.\par
Je pense aussi aux blousons noirs. Leur volonté de puissance puérile a souvent su sauvegarder une volonté de vivre quasi intacte. Certes, la récupération menace le blouson noir : comme consommateur d’abord, parce qu’il en vient à désirer les objets qu’il n’a pas le moyen d’acheter, comme producteur ensuite, quand il vieillit, mais le jeu garde au sein des groupes un attrait si vif qu’il a quelque chance d’aboutir un jour à une conscience révolutionnaire. Si la violence inhérente aux groupes de JV. cessait de se dépenser en attentats spectaculaires et souvent dérisoires pour atteindre à la poésie des émeutes, le jeu devenant insurrectionnel provoquerait sans doute une réaction en chaîne, une onde de choc qualitative. La plupart des gens sont en effet sensibilisés au désir de vivre authentiquement, au refus des contraintes et des rôles. Il suffit d’une étincelle, et d’une tactique adéquate. Si les blousons noirs arrivent jamais à une conscience révolutionnaire par la simple analyse de ce qu’ils sont déjà et par la simple exigence d’être plus, ils détermineront vraisemblablement l’épicentre du renversement de perspective. Fédérer leurs groupes serait l’acte qui, à la fois, manifesterait cette conscience et la permettrait.
\subsubsection[{2. La révolution de la vie quotidienne}]{\textsc{2.} La révolution de la vie quotidienne}
\noindent Jusqu’à présent le centre a toujours été autre que l’homme, la créativité est demeurée marginale, suburbaine. L’urbanisme reflète bien les aventures de l’axe autour duquel la vie s’organise depuis des millénaires. Les villes anciennes s’élèvent autour d’une place forte ou d’un lieu sacré, temple ou église, point de jonction entre la terre et le ciel. Les cités ouvrières entourent de leurs rues tristes l’usine ou le combinat, tandis que les centres administratifs contrôlent des avenues sans âme. Enfin, les villes nouvelles, comme Sarcelles ou Mourenx, n’ont plus de centre. Cela simplifie : le point de référence qu’elles proposent est \emph{partout ailleurs}. Dans ces labyrinthes où il est permis seulement de se perdre, l’interdiction de jouer, de se rencontrer, de vivre se dissimule derrière des kilomètres de baies vitrées, dans le réseau quadrillé des artères, au sommet de blocs de béton habitables.\par
Il n’y a plus de centre d’oppression, car l’oppression est partout. Positivité d’une telle désagrégation : chacun prend conscience, dans l’extrême isolement, de la nécessité de se sauver d’abord, de se choisir comme centre, de construire au départ du subjectif un monde où l’on puisse être partout chez soi.\par
Le retour lucide à soi est le retour à la source des autres, à la source du social. Tant que la créativité individuelle ne sera pas mise au centre de l’organisation de la société, il n’y aura pas d’autres libertés pour les hommes que de détruire et d’être détruits. Si tu penses pour les autres, ils penseront pour toi. Celui qui pense pour toi te juge, il te réduit à sa norme, il t’abêtit, car la bêtise ne naît pas d’un manque d’intelligence, comme le croient les imbéciles, elle commence avec le renoncement, avec l’abandon de soi. C’est pourquoi, quiconque te demande raison et exige des comptes, traite-le en juge, c’est-à-dire en ennemi.\par
« Je veux des héritiers, je veux des enfants, je veux des disciples, je veux un père, je ne me veux pas moi-même », ainsi parlent les intoxiqués du christianisme, qu’ils soient de Rome ou de Pékin. Partout où règne un tel esprit, il n’y a que misères et névroses. La subjectivité m’est trop chère pour que je pousse la désinvolture au point de solliciter ou de refuser l’aide des autres hommes. Il ne s’agit pas de se perdre dans les autres, ni davantage de se perdre en soi. Quiconque sait qu’il doit compter avec la collectivité doit d’abord se trouver, sans quoi il ne tirera des autres que sa propre négation.\par
Le renforcement du centre subjectif offre un caractère si particulier qu’il est malaisé d’en parler. Le cœur de chaque être humain dissimule une chambre secrète, une \emph{camera obscura}. Seuls l’esprit et le rêve y accèdent. Cercle magique où le monde et le moi se rejoignent, il n’est pas un désir, pas un souhait qui n’y soit aussitôt exaucé. Les passions y croissent, belles fleurs vénéneuses où se prend l’air du temps. Pareil à un Dieu fantasque et tyrannique, je me crée un univers et règne sur des êtres qui ne vivront jamais que pour moi. L’humoriste James Thurber a montré en quelques pages charmantes comment le paisible Walter Mitty s’illustrait tour à tour comme capitaine intrépide, éminent chirurgien, assassin désinvolte, héros des tranchées ; tout en conduisant sa vieille Buick et en achetant des biscuits de chien.\par
L’importance du centre subjectif s’évalue aisément au discrédit qui le frappe. On aime y voir un havre de compensation, un repli méditatif, une sous-préfecture poétique, le signe de l’intériorité. La rêverie, dit-on, est sans conséquence. Pourtant, n’est-ce pas au départ des phantasmes et des représentations capricieuses de l’esprit que sont fomentés les plus beaux attentats contre la morale, l’autorité, le langage, l’envoûtement ? La richesse subjective n’est-elle pas la source de toute créativité, le laboratoire de l’expérience immédiate, la tête de pont implantée dans le Vieux Monde, et d’où partiront les prochaines invasions ?\par
Pour qui sait recueillir les messages et les visions laissés par le centre subjectif, le monde s’ordonne différemment, les valeurs changent, les choses perdent leur aura, deviennent de simples instruments. Dans la magie de l’imaginaire, rien n’existe que pour être à mon gré manipulé, caressé, brisé, recréé, modifié. Le primat de la subjectivité reconnue délie de l’envoûtement des choses. Au départ des autres, on se poursuit sans s’atteindre jamais, on répète les mêmes gestes privés de sens. Au départ de soi, au contraire, les gestes ne sont pas répétés mais repris, corrigés, idéalement réalisés.\par
L’onirisme latent sécrète une énergie qui ne demande qu’à faire tourner les circonstances comme des turbines. De même qu’il rend l’utopie impossible, le haut degré de technicité auquel atteint l’époque actuelle supprime le caractère purement féerique des rêves. Tous mes désirs sont réalisables dès l’instant où l’équipement matériel contemporain se met à leur service.\par
Et dans l’immédiat, même privée de ces techniques, est-ce que la subjectivité se trompe jamais ? Ce que j’ai rêvé d’être, il ne m’est pas impossible de l’objectiver. Chaque individu a au moins réussi une fois dans sa vie l’opération de Lassailly ou de Netchaïev ; le premier se faisant passer pour l’auteur d’un livre, non écrit, finit par devenir un authentique écrivain, le père des \emph{Roueries de Trialph} ; le second extorquant de l’argent à Bakounine au nom d’une organisation terroriste inexistante, en arrive à diriger un véritable groupe de nihilistes. Il faut bien que je sois quelque jour comme j’ai voulu que l’on me croie ; il faut bien que l’image privilégiée dans le spectacle par mon vouloir-être accède à l’authenticité. La subjectivité détourne ainsi à son profit le rôle et le mensonge spectaculaire, elle réinvestit l’apparence dans le réel.\par
La démarche purement spirituelle de l’imagination subjective cherche toujours sa réalisation pratique. Il n’est pas douteux que l’attraction du spectacle artistique – surtout celui qui raconte des histoires – joue sur cette tendance de la subjectivité à se réaliser, mais en fait, elle la capte, elle la fait couler dans les turbines de l’identification passive. C’est ce que souligne justement Debord dans son film d’agitation \emph{Critique de la séparation} :\par

\begin{quoteblock}
\noindent « Généralement, les événements qui arrivent dans l’existence individuelle telle qu’elle est organisée, ceux qui nous concernent réellement, et exigent notre adhésion, sont précisément ceux qui ne méritent rien de plus que de nous trouver spectateurs distants et ennuyés, indifférents. Au contraire, la situation qui est vue à travers une transposition artistique quelconque est assez souvent ce qui attire, ce qui mériterait que l’on devînt acteur, participant. Voilà un paradoxe à renverser, à remettre surs ses pieds. »\end{quoteblock}

\noindent  Il faut dissoudre les forces du spectacle artistique pour faire passer leur équipement à l’armement des rêves subjectifs. Quand ils seront armés, on ne risquera plus à les traiter de phantasmes. Le problème de réaliser l’art ne se pose pas en d’autres termes.
\subsection[{3. La subjectivité radicale}]{\textsc{3. }La subjectivité radicale}
\noindent Toutes les subjectivités diffèrent entre elles bien qu’elles obéissent toutes à une identique volonté de réalisation. Il s’agit de mettre leur variété au service de cette commune inclination, de créer un front uni de la subjectivité. Le projet de construire une société nouvelle ne peut perdre de vue cette double exigence : la réalisation de la subjectivité individuelle sera collective ou ne sera pas :\par

\begin{quoteblock}
 \noindent « chacun combat pour ce qu’il aime : voilà ce qui s’appelle parler de bonne foi. Combattre pour tous n’est que la conséquence »\par
 
\bibl{(Saint-Just).}
 \end{quoteblock}

\noindent Ma subjectivité se nourrit d’événements. D’événements les plus divers, une émeute, une peine d’amour, une rencontre, un souvenir, une rage de dents. Les ondes de choc de ce qui compose la réalité en devenir se répercutent dans les cavernes du subjectif. La trépidation des faits me gagne malgré moi ; tous ne m’impressionnent pas également mais leur contradiction m’atteint à tous coups, car mon imagination a beau s’en emparer, ils échappent la plupart du temps à ma volonté de les changer réellement. Le centre subjectif enregistre simultanément la transmutation du réel en imaginaire et le reflux des faits réintégrant le cours incontrôlable des choses. D’où la nécessité de jeter un pont entre la construction imaginaire et le monde objectif. Seule une théorie radicale peut conférer à l’individu des droits imprescriptibles sur le milieu et les circonstances. La théorie radicale saisit les hommes à la racine et la racine des hommes, c’est leur subjectivité – cette zone irréductible qu’ils possèdent en commun.\par
On ne se sauve pas seul, on ne se réalise pas isolément. Se peut-il qu’atteignant à quelque lucidité sur lui et sur le monde, un individu ne remarque pas chez ceux qui l’entourent une volonté identique à la sienne, une même recherche au départ du même point d’appui. ?\par
Toutes les formes de pouvoir hiérarchisé diffèrent entre elles et présentent cependant une identité dans leurs fonctions oppressives. De même toutes les subjectivités diffèrent entre elles et présentent cependant une identité dans leur volonté de réalisation intégrale. C’est à ce titre qu’il convient de parler d’une véritable « subjectivité radicale ».\par
Il existe une racine commune à toutes les subjectivités uniques et irréductibles : la volonté de se réaliser en transformant le monde, la volonté de vivre toutes les sensations, toutes les expériences, tous les possibles. À différents degrés de conscience et de résolution, elle est présente en chaque homme. Son efficacité tient évidemment à l’unité collective qu’elle atteindra sans perdre sa multiplicité. La conscience de cette unité nécessaire naît d’une sorte de réflexe d’identité, mouvement inverse de l’identification. Par l’identification, on perd son unicité dans la pluralité des rôles ; par le réflexe d’identité, on renforce sa plurivalence dans l’unité des subjectivités fédérées.\par
Le réflexe d’identité fonde la subjectivité radicale. Le regard qui vient de soi se chercher partout chez les autres. « Lorsque j’étais en mission dans L’État de Tchou, dit Confucius, je vis de petits cochons tétant leur mère morte. Bientôt ils tressaillirent et s’en allèrent. Ils sentaient qu’elle ne les voyait plus et qu’\emph{elle n’était plus semblable à eux}. Ce qu’ils aimaient dans leur mère, ce n’était pas son corps, mais ce qui rendait le corps vivant. » De même, ce que je recherche chez les autres, c’est la part la plus riche de moi qu’ils entretiennent en eux. Le réflexe d’identité va-t-il se propager inéluctablement ? Cela ne va pas de soi. Cependant, les conditions historiques actuelles y prédisposent.\par
Personne n’a jamais mis en doute l’intérêt que les hommes prennent à être nourris, logés, soignés, protégés des intempéries et des revers. Ce souhait commun, les imperfections de la technique, très tôt transformées en imperfections sociales, en ont retardé l’accomplissement. Aujourd’hui, l’économie planifiée laisse prévoir la solution finale des problèmes de survie. Maintenant que les besoins de survie sont en passe d’êtres satisfaits, dans les pays hyper-industrialisés tout au moins, on s’aperçoit qu’il existe aussi des passions de vie à satisfaire, que la satisfaction de ces passions touche l’ensemble des hommes et, bien plus, qu’un échec dans ce secteur remettrait en cause tous les acquis de la survie. Les problèmes de survie lentement mais sûrement résolus tranchent de plus en plus avec les problèmes de vie, lentement et sûrement sacrifiés aux impératifs de survie. Cette séparation facilite les choses : la planification socialiste s’oppose désormais à l’harmonisation sociale.\par

\astermono

\noindent La subjectivité radicale est le front commun de l’identité retrouvée, ceux qui sont incapables de reconnaître leur présence chez les autres se condamnent à être toujours étrangers à eux-mêmes. Je ne peux rien pour les autres s’ils ne peuvent rien pour eux-mêmes. C’est dans cette optique qu’il faut revoir des notions comme celles de « connaissance » et de « reconnaissance », de « sympathique » et de « sympathisant ».\par
La connaissance n’a de valeur que si elle débouche sur la reconnaissance du projet commun ; sur le réflexe d’identité. Le style de réalisation implique des connaissances multiples, mais ces connaissances ne sont rien sans le style de réalisation. Comme les premières années de l’Internationale situationniste l’ont montré, les principaux adversaires d’un groupe révolutionnaire cohérent sont les plus proches par la connaissance, et les plus éloignés par le vécu et le sens qu’ils lui donnent. De même, les sympathisants s’identifient au groupe et, du même coup, l’entravent. Ils comprennent tout, sauf l’essentiel, sauf la radicalité. Ils revendiquent la connaissance parce qu’ils sont incapables de se revendiquer, eux.\par
En me saisissant, je dessaisis les autres de leur emprise sur moi, je les laisse donc se reconnaître en moi. Nul ne s’accroît librement sans répandre la liberté dans le monde.\par
Je fais mien sans réserve le propos de Cœurderoy :\par

\begin{quoteblock}
\noindent « J’aspire à être moi, à marcher sans entrave, à m’affirmer seul dans ma liberté. Que chacun fasse comme moi. Et ne vous tourmentez plus alors du salut de la révolution, elle sera mieux entre les mains de tout le monde qu’entre les mains des partis. »\end{quoteblock}

\noindent Rien ne m’autorise à parler au nom des autres, je ne suis délégué que de moi-même et, pourtant, je suis constamment dominé par cette pensée que mon histoire n’est pas seulement une histoire personnelle, mais que je sers les intérêts d’hommes innombrables en vivant comme je vis et en m’efforçant de vivre plus intensément, plus librement. Chacun de mes amis est une collectivité qui a cessé de s’ignorer, chacun de nous sait qu’il agit pour les autres en agissant pour lui-même. C’est seulement dans ces conditions de transparence que peut se renforcer la participation authentique.
\subsection[{4. Le projet de communication}]{\textsc{4. }Le projet de communication}
\noindent La passion de l’amour offre le modèle le plus pur et le plus répandu de la communication authentique. En s’accentuant, la crise de la communication risque bien de la corrompre. La réification la menace. Il faut veiller à ce que la \emph{praxis} amoureuse ne devienne une rencontre d’objets, il faut éviter que la séduction n’entre dans les conduits spectaculaires. Hors de la voie révolutionnaire, il n’y a pas d’amour heureux.\par
Également importante, les trois passions qui sous-tendent le triple projet de réalisation, de communication, de participation, ne sont cependant pas également réprimées. Alors que le jeu et la passion créatrice tombent sous le coup d’interdits et de falsification, l’amour, sans échapper à l’oppression, reste toutefois l’expérience la plus répandue et la plus accessible à tous. La plus démocratique, en somme.\par
La passion de l’amour porte en soi le modèle d’une communication parfaite : l’orgasme, l’accord des partenaires dans l’acmé. Elle est, dans l’obscutité de la survie quotidienne, la lueur intermittente du qualitatif. L’intensité vécue, la spécificité, l’exaltation des sens, la motilité des affects, le goût du changement et de la variété, tout prédispose la passion de l’amour à repassionner les déserts du Vieux Monde. D’une survie sans passion ne peut naître que la passion d’une vie une et multiple. Les gestes de l’amour résument et condensent le désir et la réalité d’une telle vie. L’univers que les vrais amants édifient de rêves et d’enlacements est un univers de transparence ; les amants veulent être partout chez eux.\par
Mieux que les autres passions, l’amour a su préserver sa dose de liberté. La création et le jeu ont toujours « bénéficié » d’une représentation officielle, d’une reconnaissance spectaculaire qui les aliénait, pour ainsi dire, à la source. L’amour ne s’est jamais départi d’une certaine clandestinité, baptisée intimité. Il s’est trouvé protégé par la notion de vie privée, expulsé du jour (réservé au travail et à la consommation) et refoulé dans les recoins de la nuit, dans les lumières tamisées. Ainsi a-t-il échappé en partie à la grande récupération des activités diurnes. On ne peut en dire autant du projet de communication. L’étincelle de la passion amoureuse disparaît sous les cendres de la fausse communication. En s’accentuant sous le poids du consommable, la falsification risque aujourd’hui d’atteindre les simples gestes de l’amour.\par

\astermono

\noindent Ceux qui parlent de communication quand il n’y a que des rapports de choses répandent le mensonge et le malentendu qui réifient davantage. Entente, compréhension, accord… Que signifient ces mots alors que je ne vois autour de moi qu’exploiteurs et exploités, dirigeants et exécutants, acteurs et spectateurs, tous gens manipulés comme une grenaille par les machines du pouvoir ?\par
Ce n’est pas que les choses n’expriment rien. Si quelqu’un prête à un objet sa propre subjectivité, l’objet devient humain. Mais dans un monde régi par l’appropriation privative, la seule fonction de l’objet, c’est de justifier le propriétaire. Si ma subjectivité s’empare de ce qui l’environne, si mon regard fait sien un paysage, ce ne peut être qu’idéalement, sans conséquence matérielle, ni juridiques. Dans la perspective du pouvoir, les êtres, les idées et les choses ne sont pas là pour mon agrément, mais pour servir un maître ; rien n’est réellement, tout est fonction d’un ordre d’appartenance.\par
Il n’y a pas de communication authentique dans un monde où les fétiches gouvernent la plupart des comportements. Entre les êtres et les choses, l’espace est contrôlé par les médiations aliénantes. À mesure que le pouvoir devient une fonction abstraite, la confusion et la multiplication de ses signes a besoin de scribes, sémanticiens et mythologues, qui s’en fassent les interprètes. Dressé à ne voir autour de lui que des objets, le propriétaire a besoin de serviteurs objectifs et objectivés. Les spécialistes de la communication organisent le mensonge au profit des gardiens de cadavres. Seule la vérité subjective, armée par les conditions historiques, peut leur résister. C’est de l’expérience immédiate qu’il faut partir si l’on veut briser les pointes de pénétration les plus avancées des forces oppressives.\par

\astermono

\noindent La bourgeoisie n’a connu d’autre plaisir que celui de les dégrader tous. Il ne lui a pas suffi d’emprisonner la liberté d’aimer dans l’appropriation sordide d’un contrat de mariage, et de la sortir à heure fixe pour les besoins de l’adultère ; elle ne s’est pas contenté de la jalousie et du mensonge pour empoisonner la passion ; elle a réussi à désunir les amants dans l’enlacement de leurs gestes.\par
Le désespoir amoureux ne procède pas de ce que les amants ne peuvent s’obtenir, mais bien plutôt de ce que, mêlés dans l’étreinte, ils risquent de ne se rencontrer jamais ; de se saisir mutuellement comme objets. Déjà les conceptions hygiénistes de la social-démocratie suédoise ont popularisé cette caricature de la liberté d’aimer, l’amour manipulé comme un jeu de cartes.\par
L’écœurement qui naît d’un monde dépossédé de son authenticité ranime le désir insatiable de contacts humains. Quel heureux hasard que l’amour ! Parfois il m’arrive de penser qu’il n’existe pas d’autre réalité immédiate, pas d’autre humanité tangible que la caresse d’une chair féminine, douceur de la peau, tiédeur du sexe. Qu’il n’existe rien d’autre, mais ce rien s’ouvre sur une totalité qu’une vie éternelle ne tarirait pas.\par
Puis il arrive, au plus intime de la passion, que la masse inerte des objets exerce une attraction occulte. La passivité d’un partenaire dénoue soudain les liens qui se tressaient, le dialogue s’interrompt sans avoir vraiment commencé. La dialectique de l’amour se fige, il n’y a plus côte à côte que des gisants. Il n’y a plus que des rapports d’objets.\par
Bien que l’amour naisse toujours \emph{de} et \emph{dans} la subjectivité – une fille est belle parce qu’elle me plaît – mon désir ne peut s’empêcher d’objectiver ce qu’il convoite. Le désir objective toujours la personne aimée. Or si je laisse mon désir transformer l’être aimé en objet, ne suis-je pas condamné à me heurter à cet objet et, l’habitude aidant, à me détacher de lui ?\par
Qu’est-ce qui assure la parfaite communication amoureuse ? L’union des contraires :\par
— plus je me détache de l’objet de mon désir, et plus je donne de force objective à mon désir, plus je suis un désir insouciant de son objet ;\par
— plus je me détache de mon désir en tant qu’objet, et plus je donne de force objective à l’objet de mon désir, plus mon désir tire sa justification de l’être aimé.\par
Sur le plan social, ce jeu d’attitudes pourrait se traduire par le changement de partenaires et l’attachement simultané à un partenaire pivotal. Et toutes ces rencontres impliqueraient ce dialogue, qui n’est qu’un seul propos ressenti en commun et à la réalisation duquel je n’ai jamais cessé d’aspirer : « Je sais que tu m’aimes pas, car tu n’aimes personne hormis de toi-même. Je suis comme toi. Aime-moi ! »\par
Il n’y a pas d’amour possible hors de la subjectivité radicale. Il faut en finir avec l’amour chrétien, l’amour sacrifice, l’amour militant. À travers les autres n’aimer que soi, être aimé des autres à travers l’amour qu’ils se doivent. C’est ce qu’enseigne la passion de l’amour, c’est ce que commandent les conditions de la communication authentique.\par

\astermono

\noindent Et l’amour est aussi une aventure, une approche à travers l’inauthentique. Aborder une femme par le biais du spectaculaire, c’est se condamner dès l’abord à des rapports d’objets. C’est en quoi le \emph{play-boy} est un spécialiste. Le véritable choix se pose entre la séduction spectaculaire – le baratin – et la séduction du qualitatif – l’être séduisant parce qu’il ne se soucie pas de séduire.\par
Sade analyse deux comportements possibles : les libertins des \emph{Cent vingt journées de Sodome} ne jouissent réellement qu’en mettant à mort, dans d’affreuses tortures, l’objet de leur séduction (et quel hommage plus agréable à un objet que de le faire souffrir ?). Les libertins de la \emph{Philosophie dans le boudoir}, aimables et enjoués, se font une fête d’accroître à l’extrême leurs jouissances mutuelles. Les premiers sont les maîtres anciens, vibrants de haine et de révolte ; les seconds, les maîtres sans esclaves ne découvrant l’un dans l’autre que l’écho de leurs propres plaisirs.\par
Aujourd’hui, le véritable séducteur est le sadique, celui qui ne pardonne pas à l’être désiré d’être un objet. Au contraire, l’homme séduisant contient en lui la plénitude du désir, il rejette le rôle et sa séduction naît de ce refus. C’est Dolmancé, c’est Eugénie, c’est Madame de Saint-Ange. Pour l’être désiré, toutefois cette plénitude n’existe que s’il peut reconnaître en celui qui l’incarne sa propre volonté de vivre. La vraie séduction n’a pour séduire que sa vérité. Ne mérite pas d’être séduit qui veut. C’est en ce sens que parlent les \emph{Béguines} de Schweidnitz et leurs compagnons (XIII\textsuperscript{e} siècle) quand ils affirment que toute résistance à des avances sexuelles est la marque d’un esprit grossier. Les Frères du Libre Esprit expriment la même idée :\par

\begin{quoteblock}
\noindent « Tout homme qui connaît le Dieu qui l’habite porte en lui son propre ciel. En revanche, l’ignorance de sa propre divinité constitue un péché mortel en vérité. Telle est la signification de l’enfer que l’on transporte également avec soi dans la vie d’ici-bas. »\end{quoteblock}

\noindent L’enfer est le vide laissé par la séparation, l’angoisse des amants d’être côte à côte sans être ensemble. La non-communication est toujours un peu comparable à l’échec d’un mouvement révolutionnaire. La volonté de mort s’installe où la volonté de vivre échoue.\par

\astermono

\noindent Il faut débarrasser l’amour de ses mythes, de ses images, de ses catégories spectaculaires ; renforcer son authenticité, le rendre à sa spontanéité. Il n’y a pas d’autre façon de lutter contre sa récupération dans le spectacle et contre son objectivation. L’amour ne supporte ni l’isolement ni le morcellement, il déborde sur la volonté de transformer l’ensemble des conduites humaines, sur la nécessité de construire une société où les amants se sentent partout en liberté.\par
La naissance et le dissolution du moment de l’amour sont liés à la dialectique du souvenir et du désir. \emph{In status nascendi}, le désir et l’évocation des premiers désirs comblés (la non-résistance aux approches) se renforcent mutuellement. Dans le moment proprement dit, souvenir et désir coïncident. Le moment de l’amour est un espace-temps de vécu authentique, un présent où se condensent le souvenir du passé et l’arc du désir tendu vers l’avenir. Dans la \emph{phase de rupture}, le souvenir prolonge le moment passionnant mais le désir décroît peu à peu. Le présent se décompose, le souvenir se tournant nostalgiquement vers le bonheur passé tandis que le désir appréhende le malaise à venir. Dans la \emph{dissolution}, la séparation est effective. Le souvenir porte l’échec du passé récent et achève d’affaiblir le désir.\par
Dans le dialogue comme dans l’amour, dans la passion d’aimer comme dans le projet de communication, le problème consiste à éviter la phase de rupture. À cette fin, on peut envisager :\par
— d’étendre le moment de l’amour à tous ses prolongements, autrement dit de ne pas le dissocier ni des autres passions ni des autres projets, et de l’élever de l’état de moment à une véritable construction de situation ;\par
— de favoriser les expériences collectives de réalisation individuelle, et de multiplier ainsi les rencontres amoureuses en réunissant une grande variété de partenaires valables ;\par
— de maintenir en permanence le principe du plaisir, qui garde aux projets de réalisation, de communication et de participation leur caractère passionnant. Le plaisir est le principe d’unification. L’amour est la passion d’unité dans un \emph{moment} commun ; l’amitié, la passion d’unité dans un \emph{projet} commun.
\subsection[{5. L’érotique ou dialectique du plaisir}]{\textsc{5. }L’érotique ou dialectique du plaisir}
\noindent Il n’y a pas de plaisir qui ne soit à la recherche de sa cohérence. Son interruption, son non-assouvissement provoquent un trouble semblable à la stase dont parle Reich. Les mécanismes oppressifs du pouvoir entretiennent une crise permanente dans le comportement des hommes. Le plaisir et l’angoisse née de son absence ont donc essentiellement une fonction sociale. L’érotique est le mouvement des passions devenant unitaires, un jeu sur l’unité et le multiple, sans lequel il n’y a pas de cohérence révolutionnaire (« l’ennui est toujours contre révolutionnaire » – I. S., n° 3).\par
Wilelm Reich attribue la plupart des dérèglements comportementaux aux troubles de l’orgasme, à ce qu’il appelle l’« impuissance orgastique ». Selon lui, l’angoisse naît d’un orgasme incomplet, d’une décharge où n’aurait pas été liquidé totalement l’ensemble des excitations, caresses, jeux érotiques… qui ont préparé et rendu possible l’union sexuelle. La théorie reichienne considère que l’énergie accumulée et non dépensée devient flottante, et se change en pulsion d’angoisse. L’angoisse du plaisir inassouvi entrave les déclenchements orgastiques futurs.\par
Or le problème des tensions et de leur liquidation ne se pose pas seulement sur le plan de la sexualité. Il caractérise tous les rapports humains. Bien que Reich l’ait pressenti, il ne montre pas assez que la crise sociale actuelle est aussi une crise de type orgastique. Si « la source d’énergie de la névrose se trouve dans la marge qui sépare l’accumulation et la décharge de l’énergie sexuelle », il me semble que la source d’énergie de nos névroses se trouve aussi dans la marge qui sépare l’accumulation et la décharge de l’énergie mise en œuvre dans les rapports humains. La jouissance totale est encore possible dans le moment de l’amour, mais sitôt que l’on s’efforce de prolonger ce moment, de lui donner une extension sociale, on n’échappe pas à ce que Reich appelle la « stase ». Le monde du déficitaire et de l’inaccompli est le monde de la crise permanente. Que serait donc une société sans névrose ? Une fête permanente. Il n’y a pas d’autre guide que le plaisir.\par

\astermono


\begin{quoteblock}
\noindent « Tout est femme dans ce que l’on aime, dit La Mettrie, l’empire de l’amour ne reconnaît d’autres bornes que celles du plaisir. »\end{quoteblock}

\noindent Mais le plaisir lui-même ne veut pas reconnaître de bornes. Le plaisir qui ne s’accroît pas disparaît. Le répétitif le tue, il ne s’accommode pas du parcellaire. Le principe du plaisir est inséparable de la totalité.\par
L’érotique est le plaisir qui cherche sa cohérence. Il est le mouvement des passions devenant communicantes, inséparables, unitaires. Il s’agit de recréer dans la vie sociale les conditions qui sont celles de la jouissance parfaite dans le moment de l’amour. Des conditions qui permettent le jeu sur l’unité et le multiple, c’est-à-dire la libre participation à l’accomplissement de la transparence.\par
Freud définit le but d’Éros : l’unification ou la recherche de l’union. Mais quand il prétend que la peur d’être séparé et expulsé du groupe provient de l’angoisse de castration, sa proposition doit être renversée. C’est l’angoisse de castration qui provient de la peur d’être exclu, non l’inverse. Cette angoisse s’accentue à mesure que s’accentue l’isolement des individus dans l’illusion communautaire.\par
Tout en recherchant l’unification, Éros est essentiellement narcissique, amoureux de soi. Il désire un univers à aimer comme il s’aime lui-même. Norman Brown relève la contradiction dans \emph{Éros et Thanatos}. Comment, se demande-t-il, une orientation narcissique pourrait-elle conduire à l’union avec les êtres dans le monde ? Il répond :\par

\begin{quoteblock}
\noindent « L’antinomie abstraite du Moi et de l’Autre dans l’amour peut être vaincue si nous revenons à la réalité concrète du plaisir et à la définition essentielle de la sexualité comme étant l’activité agréable du corps, et si nous considérons l’amour comme le rapport entre le Moi et les sources du plaisir. »\end{quoteblock}

\noindent Encore faut-il préciser : la source du plaisir est moins dans le corps que dans une possibilité d’expansion dans le monde. La réalité concrète du plaisir tient à la liberté de s’unir à tous les êtres qui permettent de s’unir à soi-même. La réalisation du plaisir passe par le plaisir de la réalisation, le plaisir de la communication par la communication du plaisir, la participation au plaisir par le plaisir de la participation. C’est en cela que le narcissisme tourné vers l’extérieur, dont parle Brown, implique un bouleversement total des structures sociales.\par
Plus le plaisir gagne en intensité, plus il revendique la totalité du monde. C’est pourquoi il me plaît de saluer comme un slogan révolutionnaire l’exhortation de Breton :\par

\begin{quoteblock}
\noindent « Amants, faites-vous de plus en plus jouir ! »\end{quoteblock}

\noindent La civilisation occidentale est une civilisation du travail et, comme dit Diogène :\par

\begin{quoteblock}
\noindent « L’amour est l’occupation des paresseux. »\end{quoteblock}

\noindent Avec la disparition graduelle du travail forcé, l’amour est appelé à reconquérir le terrain perdu. Et cela ne va pas sans danger pour toutes les formes d’autorité. Parce que l’érotique est unitaire, il est aussi la liberté du multiple. Il n’y a pas de meilleure propagande pour la liberté que la liberté sereine de jouir. C’est pourquoi le plaisir est la plupart du temps confiné dans la clandestinité, l’amour dans une chambre, la créativité sous l’escalier de la culture, l’alcool et la drogue à l’ombre des lois…\par
La morale de la survie a condamné la diversité des plaisirs comme elle condamne la multiplicité unitaire au profit du répétitif. Si le plaisir-angoisse se satisfait du répétitif, le vrai plaisir s’accommode seulement de la diversité dans l’unité. Le modèle le plus simple de l’érotique est sans doute le couple pivotal. Les deux partenaires vivent leurs expériences dans une transparence et une liberté aussi complète que possible. Cette complicité rayonnante a le charme de relations incestueuses. La multiplicité des expériences vécues en commun fonde entre les partenaires un lien de frère et sœur. Les grandes amours ont toujours quelque chose d’incestueux ; de là à déduire que les amours entre frères et sœurs partent privilégiés, et devraient être favorisés, il n’y a qu’un pas qu’il serait bon de franchir en bousculant une fois pour toutes un des plus vieux et des plus ridicules tabous. On pourrait parler de \emph{sororisation}. Une épouse-sœur dont les amies soient mes épouses et mes sœurs.\par
Dans l’érotique, il n’y a d’autre perversion que la négation du plaisir, que sa falsification dans le plaisir-angoisse. Qu’importe la source pourvu que l’eau s’écoule. Les Chinois disent : immobiles l’un dans l’autre, le plaisir nous emporte.\par
Enfin la recherche du plaisir est la meilleure garantie du ludique. Elle sauvegarde la participation authentique, elle la protège contre le sacrifice, la contrainte, le mensonge. Les différents degrés d’intensité du plaisir marquent l’emprise de la subjectivité sur le monde. Ainsi, le caprice est le jeu du désir naissant ; le désir, le jeu de passion naissante. Et le jeu de la passion trouve sa cohérence dans la poésie révolutionnaire.\par
Est-ce à dire que la recherche du plaisir exclut le déplaisir ? Il s’agit plutôt de la réinventer. Le plaisir-angoisse n’est ni un plaisir, ni un déplaisir, mais une façon de se gratter qui irrite davantage. Qu’est-ce alors que le déplaisir authentique ? Un raté dans le jeu du désir ou de la passion ; un déplaisir positif, tendu d’autant plus passionnément vers un autre plaisir à construire.
\subsection[{6. Le projet de participation}]{\textsc{6. }Le projet de participation}

\begin{argument}\noindent Du jeu, l’organisation de la survie ne tolère que les falsifications spectaculaires. Mais la crise du spectacle fait que, traquée de toutes parts, la passion du jeu ressurgit partout. Elle prend désormais le visage du bouleversement social, et fonde, par-delà sa négativité, une société de participation réelle. La \emph{praxis} ludique implique le refus du chef, le refus du sacrifice, le refus du rôle, la liberté de réalisation individuelle, la transparence des rapports sociaux (1). – La tactique est la phase polémique du jeu. La créativité individuelle a besoin d’une organisation qui la concentre et lui donne plus de force. La tactique est inséparable d’un certain calcul hédoniste. Toute action parcellaire doit avoir pour but la destruction totale de l’ennemi. Il faut étendre aux sociétés industrialisées les formes adéquates de guérilla (2). – Le détournement est le seul usage révolutionnaire des valeurs spirituelles et matérielles distribuées par la société de consommation ; l’arme absolue du dépassement (3).
\end{argument}

\subsubsection[{1. La praxis ludique}]{\textsc{1.} La \emph{praxis} ludique}
\noindent Les nécessités de l’économie s’accommodent mal du ludique. Dans les transactions financières, tout est sérieux : on ne badine pas avec l’argent. La part de jeu encore englobée par l’économie féodale a été éliminée peu à peu par la rationalité des échanges monétaires. Le jeu sur les échanges permettait en effet de troquer des produits, sinon sans commune mesure, du moins non étalonnés rigoureusement. Or aucune fantaisie ne sera tolérée dès l’instant où le capitalisme impose ses rapports mercantiles, et l’actuelle dictature du consommable prouve suffisamment qu’il s’entend à les imposer partout, à tous les niveaux de la vie.\par
Dans le haut Moyen Âge, les rapports idylliques infléchissent dans le sens d’une certaine liberté les impératifs purement économiques de l’organisation seigneuriale des campagnes ; le ludique présidait souvent aux corvées, aux jugements, aux règlements de comptes. En précipitant dans la bataille de la production et de la consommation la presque totalité de la vie quotidienne, le capitalisme refoule la propension au ludique, tandis qu’il s’efforce en même temps de la récupérer dans la sphère du rentable. Ainsi a-t-on vu en quelques dizaines d’années les joies de l’évasion se muer en tourisme, l’aventure tourner en mission scientifique, le jeu guerrier devenir stratégie opérationnelle, le goût du changement se satisfaire d’un changement de goût…\par
En général, l’organisation sociale actuelle interdit le jeu authentique. Elle en réserve l’usage à l’enfance, à laquelle, soit dit en passant, elle propose avec une insistance croissante des sortes de jouets-gadgets, véritables primes à la passivité. L’adulte, lui, n’a droit qu’à des formes falsifiées et récupérées : compétition, jeux télévisés, élections, casino… Il va de soi que la pauvreté de ces expédients n’étouffe pas la richesse spontanée de la passion du jeu, surtout dans un temps où le ludique a bien des chances de trouver historiquement réunies ses conditions les plus favorables d’expansion.\par
Le sacré ménage le jeu profane et désacralisant : témoins les chapiteaux irrévérencieux, les sculptures obscènes des cathédrales. L’Église englobe sans les dissimuler le rire négateur, la fantaisie caustique, la critique nihiliste. Sous son manteau, le jeu démoniaque est sauf. Au contraire, le pouvoir bourgeois met le jeu en quarantaine, il l’isole dans un secteur particulier comme s’il voulait en préserver les autres activités humaines. L’art constitue ce domaine privilégié, et quelque peu méprisé, du non-rentable. Il le restera jusqu’à ce que l’impérialisme économique le convertisse à son tour en usine de consommation. Désormais traquée de toutes parts, la passion du jeu resurgit partout.\par
Dans la coche d’interdits qui recouvre l’activité ludique, une faille s’ouvre à l’endroit le moins résistant, la zone où le jeu s’est maintenu le plus longtemps, le secteur artistique. L’éruption du nom Dada. « Les représentations dadaïstes firent résonner dans les auditeurs l’instinct joueur primitif-irrationnel qui avait été submergé », dit Hugo Ball. Sur la pente fatale du canular et de la plaisanterie, l’art allait entraîner dans sa chute l’édifice que l’esprit de sérieux avait bâti à la gloire de la bourgeoisie. De sorte que le jeu emprunte aujourd’hui le visage de l’insurrection. Le jeu total et la révolution de la vie quotidienne se confondent désormais.\par
Chassée de l’organisation sociale hiérarchisée, la passion du jeu fonde, en la détruisant, une société de type nouveau, une société de la participation réelle. Sans présumer de ce que sera une organisation de rapports humains ouverte sans réserve à la passion du jeu, on peut s’attendre à ce qu’elle présente les caractéristiques suivantes :\par

\begin{itemize}[itemsep=0pt,]
\item refus du chef et de toute hiérarchie ;
\item refus du sacrifice ;
\item refus du rôle ;
\item liberté de réalisation authentique ;
\item transparence des rapports sociaux.
\end{itemize}


\astermono

\noindent Le jeu ne se conçoit ni sans règles ni sans jeu sur les règles. Voyez les enfants. Ils connaissent les règles du jeu, ils s’en souviennent très bien, mais ils trichent sans cesse, ils inventent ou imaginent des tricheries. Cependant, pour eux, tricher n’a pas le sens que lui attribuent les adultes. La tricherie fait partie de leur jeu, ils jouent à tricher, complices jusque dans leurs disputes. Ainsi recherchent-ils un jeu nouveau. Et parfois, cela réussit : un nouveau jeu se crée et se développe. Sans discontinuer, ils ravivent leur conscience ludique.\par
Dès qu’une autorité se fige, devient irrévocable, se pare d’un attrait magique, le jeu cesse. Pourtant, la légèreté ludique ne se départit jamais d’un esprit d’organisation, avec ce que cela implique de discipline. Mais même s’il faut un meneur de jeu investi d’un pouvoir de décision, ce pouvoir n’est jamais dissocié des pouvoirs dont chaque dispose de façon autonome, il est le point de concentration de toutes les volontés individuelles, le double collectif de chaque exigence particulière. Le projet de participation implique donc une cohérence telle que les décisions de chacun soient les décisions de tous. Ce sont évidemment les groupes numériquement faibles, les microsociétés, qui présentent les meilleures garanties d’expérimentation. Là, le jeu réglera souverainement les mécanismes de vie en commun, l’harmonisation des caprices, des désirs, des passions. D’autant plus que ce jeu correspondra au jeu insurrectionnel mené par le groupe et rendu nécessaires par la volonté de vivre hors des normes officielles.\par
La passion du jeu exclut le recours au sacrifice. On peut perdre, payer, subir la loi, passer un mauvais quart d’heure, c’est la logique du jeu, non la logique d’une Cause, non la logique du sacrifice. Quand apparaît la notion de sacrifice, le jeu se sacralise, ses règles deviennent des rites. Dans le jeu, les règles sont données avec la façon de les tourner et de jouer avec elles. Dans le sacré, au contraire, le rituel ne se laisse pas jouer, il faut le briser, transgresser l’interdit (mais profaner une hostie est encore une façon de rendre hommage à l’Église). Seul le jeu désacralise, seul il s’ouvre sur une liberté sans limite. Il est le principe du détournement, la liberté de changer le sens de tout ce qui sert le pouvoir ; la liberté, par exemple, de transformer la cathédrale de Chartres en lunapark, en labyrinthe, en champ de tir, en décor onirique…\par
Dans un groupe axé sur la passion du jeu, les corvées et les besognes ennuyeuses trouveront à se répartir par exemple, à la suite d’une erreur ou d’une défaite ludique. Ou, plus simplement, elles rempliront les temps morts, les repos passionnels prenant par contraste, une valeur d’excitant et rendant plus piquants les moments à venir. Les situations à construire vont nécessairement se fonder sur la dialectique de la présence et de l’absence, de la richesse et de la pauvreté, du plaisir et du déplaisir, l’intensité d’un ton aiguisant l’intensité de l’autre.\par
Par ailleurs, les techniques employées dans une ambiance de sacrifice et de contrainte perdent beaucoup de leur efficacité. Leur valeur instrumentale se double en effet d’une fonction répressive ; et la créativité opprimée diminue le rendement des machines oppressives. Seule l’attraction ludique garantit un travail non aliénant, un travail productif.\par
Le rôle dans le jeu ne se conçoit pas sans un jeu sur le rôle. Le rôle spectaculaire exige une adhésion ; le rôle ludique, au contraire, postule une distance, un recul d’où l’on s’appréhende jouant et libre, à la façon de ces comédiens éprouvés qui, entre deux tirades dramatiques, échangent des plaisanteries. L’organisation spectaculaire ne résiste pas à ce type de comportement. Les Marx Brothers ont montré ce qu’un rôle devenait quand le ludique s’en emparait, et ce n’est là qu’un exemple encore récupéré, à la limite, par le cinéma. Qu’en serait-il d’un jeu sur les rôles prenant son épicentre dans la vie réelle ?\par
Si quelqu’un entre dans le jeu avec un rôle fixe, un rôle sérieux, ou il est perdu, ou il corrompt le jeu. C’est le cas du provocateur. Le provocateur est un spécialiste du jeu collectif. Il en a la technique mais non la dialectique. Peut-être serait-il capable de traduire les aspirations du groupe en matière offensive – le provocateur pousse toujours à l’attaque – si, tenu pour son malheur à ne défendre jamais que son rôle, que sa mission, il n’était de ce fait incapable de représenter l’intérêt défensif du groupe. Cette incohérence entre l’offensif et le défensif dénonce tôt ou tard le provocateur, est cause de sa triste fin. Quel est le meilleur provocateur ? Le meneur de jeu devenu dirigeant.\par
Seule la passion du jeu est de nature à fonder une communauté dont les intérêts s’identifient à ceux de l’individu. À la différence du provocateur, le traître apparaît spontanément dans un groupe révolutionnaire. Il surgit chaque fois que la passion du jeu a disparu et que, du même coup, le projet de participation a été falsifié. Le traître est un homme qui, ne trouvant pas à se réaliser authentiquement selon le mode de participation qui lui est proposé, décide de « jouer » contre une telle participation, non pour la corriger, mais pour la détruire. Le traître est la maladie sénile des groupes révolutionnaires. L’abandon du ludique est la trahison qui les autorise toutes.\par
Enfin, portant la conscience de la subjectivité radicale, le projet de participation accroît la transparence des rapports humains. Le jeu insurrectionnel est inséparable de la communication.
\subsubsection[{2.  La tactique}]{\textsc{2. } La tactique}
\noindent La tactique est la phase polémique du jeu. Entre la poésie à l’état naissant (le jeu) et l’organisation de la spontanéité (la poésie), la tactique assure la continuité nécessaire. Essentiellement technique, elle empêche la spontanéité de se disperser, de se perdre dans la confusion. On sait aussi avec quelle désinvolture l’historien traite les révolutions spontanées. Pas une étude sérieuse, pas une analyse méthodique, rien qui rappelle de près ou de loin le livre de Clausewitz sur la guerre. À croire que les révolutionnaires mettent à ignorer les batailles de Makhno avec autant d’application qu’un général de Napoléon.\par
Quelques remarques, à défaut d’analyses plus fouillées.\par
Une armée bien hiérarchisée peut gagner une guerre, pas une révolution ; une horde indisciplinée ne remporte la victoire ni dans la guerre, ni dans la révolution. Il s’agit d’organiser sans hiérarchiser, autrement dit de veiller à ce que le meneur de jeu ne devienne un chef. L’esprit ludique est la meilleure garantie contre la sclérose autoritaire. Rien ne résiste à la créativité armée. On a vu les troupes villistes et makhnovistes venir à bout des corps d’armée les plus aguerris. Au contraire, si le jeu se fige, la bataille est perdue. La révolution périt pour que le leader soit infaillible. Pourquoi Villa échoue-t-il à Celaya ? Parce qu’il a négligé de renouveler son jeu stratégique et tactique. Sur le plan technique du combat, enivré par le souvenir de Ciudad Juarez, où, perçant les murs et progressant ainsi de maison en maison, il prit l’ennemi à revers et l’écrasa, Villa dédaigne les innovations militaires de la guerre de 1914-1918, nids de mitrailleuses, mortiers, tranchées. Sur le plan politique, une certaine étroitesse de vue l’a tenu à l’écart du prolétariat industriel. Il est significatif que l’armée d’Obregon, qui anéantit les Dorados de Villa, comportait des milices ouvrières et des conseillers militaires allemands.\par
La créativité fait la force des armées révolutionnaires. Souvent, les mouvements insurrectionnels remportent dès l’abord d’éclatantes victoires parce qu’ils brisent les règles du jeu observées par l’adversaire ; parce qu’ils inventent un jeu nouveau ; parce que chaque combattant participe à part entière à l’élaboration ludique. Mais si la créativité ne se renouvelle pas, si elle tend vers le répétitif, si l’armée révolutionnaire prend la forme d’une armée régulière, on voit peu à peu l’enthousiasme et l’hystérie suppléer vainement à la faiblesse combative et le souvenir des victoires anciennes préparer de terribles défaites. La magie de la Cause et du chef prend le pas sur l’unité consciente de la volonté de vivre et la volonté de vaincre. Après avoir tenu les princes en échec pendant deux ans, 40 000 paysans pour qui le fanatisme religieux tient lieu de tactique se font tailler en pièces à Frankenhaussen en 1525 ; l’armée féodale perd trois hommes. En 1964, à Stanleyville, des centaines de mulélistes, convaincus de leur invincibilité, se laissent massacrer en se jetant sur un pont contrôlé par deux mitrailleuses. Ce sont pourtant les mêmes qui s’emparèrent des camions et des armes de l’ANC en ravinant les routes de pièges à éléphants.\par
L’organisation hiérarchisée occupe avec son contraire, l’indiscipline et l’incohérence, le lieu commun de l’inefficacité. Dans une guerre classique, l’inefficacité d’un camp l’emporte sur l’inefficacité de l’autre grâce à une inflation technique ; dans la guerre révolutionnaire, la poétique des insurgés ôte à l’adversaire les armes et le temps de s’en servir, le privant ainsi de sa seule supériorité possible. Si l’action des guérilleros tombe dans le répétitif, l’ennemi apprend à jouer selon les règles du combattant révolutionnaire ; il est alors à craindre que la contre-guérilla parvienne sinon à détruire, du moins à enrayer la créativité populaire déjà freinée.\par

\astermono

\noindent Comment maintenir, dans une troupe qui refuse d’obéir servilement à un chef, la discipline nécessaire au combat ? Comment éviter le manque de cohésion ? La plupart du temps, les armées révolutionnaires tombent de Charybe en Scylla en passant de l’inféodation à une Cause à la recherche inconséquente du plaisir, ou l’inverse.\par
L’appel au sacrifice et au renoncement fonde, au nom de la liberté, un esclave futur. Par contre, la fête prématurée et la recherche d’un plaisir parcellaire précèdent toujours de peu la répression et les semaines sanglantes de l’ordre. Le principe du plaisir doit donner sa cohésion au jeu et le discipliner. La recherche du plus grand plaisir englobe le risque du déplaisir : c’est le secret de sa force. Où puisaient-ils leur ardeur, ces soudards de l’Ancien Régime montant à l’assaut d’une ville, dix fois repoussés, dix fois reprenant le combat ? Dans l’attente passionnée de la fête, – en l’occurrence, du pillage et de l’orgie, dans un plaisir d’autant plus vif qu’il se construit lentement. La meilleure tactique sait ne faire qu’un avec le calcul hédoniste. La volonté de vivre, brutale, effrénée, est pour le combattant l’arme secrète la plus meurtrière. Une telle arme se retourne contre ceux qui la mettent en péril : pour défendre sa peau, le soldat a tout intérêt à tirer sur ses supérieurs ; pour les mêmes raisons, les armées révolutionnaires gagnent à faire de chaque homme un habile tacticien et son propre maître ; quelqu’un qui sache construire son plaisir avec conséquence.\par
Dans les luttes à venir, la volonté de vivre intensément va remplacer l’ancienne motivation du pillage. La tactique va se confondre avec la science du plaisir, tant il est vrai que la recherche du plaisir est déjà plaisir lui-même. Cette tactique-là s’apprend tous les jours. Le jeu avec les armes ne diffère pas essentiellement de la liberté du jeu, celle que les hommes poursuivent plus ou moins consciemment dans chaque instant de leur vie quotidienne. Si quelqu’un ne dédaigne pas d’apprendre dans la simple quotidienneté ce qui le tue et ce qui le rend plus fort en tant qu’individu libre, il conquiert lentement son brevet de tacticien.\par
Cependant, il n’y a pas de tacticien isolé. La volonté de détruire la vieille société implique une fédération de tacticiens de la vie quotidienne. C’est une fédération de ce type que l’Internationale situationniste se propose dès maintenant d’équiper techniquement. La stratégie construit collectivement le plan incliné de la révolution, sur la tactique de la vie quotidienne individuelle.\par

\astermono

\noindent La notion ambiguë d’humanité provoque parfois un certain flottement dans les révolutions spontanées. Trop souvent le désir de placer l’homme au centre des revendications fait la part belle à un humanisme paralysant. Que de fois le parti de la révolution n’a-t-il épargné ses propres fusilleurs, que de fois n’a-t-il accepté une trêve où le parti de l’ordre puisait de nouvelles forces ? L’idéologie de l’humain est une arme pour la réaction, celle qui sert à justifier toutes les inhumanités (les paras belges à Stanleyville).\par
Il n’y a pas d’accommodement possible avec les ennemis de la liberté, pas d’humanité qui tienne pour les oppresseurs de l’homme. L’anéantissement des contre-révolutionnaires est le seul geste humanitaire qui prévienne la cruauté de l’humanisme bureaucratisé.\par
Enfin, un des problèmes de l’insurrection spontanée tient dans le paradoxe suivant : il faut, sur la base d’actions \emph{parcellaires}, détruire \emph{totalement} le pouvoir. La lutte pour la seule émancipation économique a rendu la survie possible pour tous en imposant la survie à tous. Or il est certain que les masses luttaient pour un objectif plus large, pour le changement global des conditions de vie. Par ailleurs, la volonté de changer d’un seul coup la totalité du monde participe de la pensée magique. C’est pourquoi elle tourne si facilement au plat réformisme. La tactique apocalyptique et celle des revendications graduelles se rejoignent tôt ou tard dans le mariage des antagonismes réconciliés. Les partis faussement révolutionnaires n’ont-ils pas fini par identifier tactique et compromission ?\par
Le plan incliné de la révolution se garde également de la conquête partielle et de l’attaque frontale. La guerre de guérilla est une guerre totale. C’est dans cette voie que s’engage l’Internationale situationniste, dans un harcèlement calculé sur tous les fronts – culturel, politique, économique, social. Le champ de la vie quotidienne assure l’unité du combat.
\subsubsection[{3. Le détournement}]{\textsc{3. }Le détournement}
\noindent Au sens large du terme, le détournement est une \emph{remise en jeu} globale. C’est le geste par lequel l’unité ludique s’empare des êtres et des choses figées dans un ordre de parcelles hiérarchisées.\par
Il nous est arrivé, le soir tombant, de pénétrer, mes amis et moi, dans le Palais de Justice de Bruxelles. On connaît le mastodonte écrasant de son énormité les quartiers pauvres en contrebas, protégeant cette riche avenue Louise dont nous ferons quelque jour un passionnant terrain vague. Au gré d’une longue dérive dans un dédale de couloirs, d’escaliers, de pièces en enfilade, nous supputions les aménagements possibles du lieu, nous occupions le territoire conquis, nous transformions par la grâce de l’imagination l’endroit patibulaire en un champ de foire fantastique, en un palais des plaisirs, où les aventures les plus piquantes acquiesceraient au privilège d’être réellement vécues. En somme, le détournement est la manifestation la plus élémentaire de la créativité. La rêverie subjective détourne le monde. Les gens détournent, comme Monsieur Jourdain et James Joyce faisaient l’un de la prose et l’autre \emph{Ulysses} ; c’est-à-dire spontanément et avec beaucoup de réflexion.\par
En 1955, Debord, frappé par l’emploi systématique du détournement chez Lautréamont, attirait l’attention sur la richesse d’une technique dont Jorn devait écrire en 1960 :\par

\begin{quoteblock}
\noindent « Le détournement est un jeu dû à la capacité de dévalorisation. Tous les éléments du passé culturel doivent être réinvestis ou disparaître. »\end{quoteblock}

\noindent Enfin, dans la revue \emph{Internationale situationniste} (n° 3), Debord, revenant sur la question, précisait :\par

\begin{quoteblock}
\noindent « Les deux lois fondamentales du détournement sont la perte d’importance, allant jusqu’à la déperdition de son sens premier, de chaque élément autonome détourné ; et en même temps, l’organisation d’un autre ensemble signifiant, qui confère à chaque élément sa nouvelle portée. »\end{quoteblock}

\noindent Les conditions historiques actuelles viennent apporter leur caution aux remarques précitées. Il est désormais évident que :\par
— partout où s’étend le marais de la décomposition, le détournement prolifère spontanément. L’ère des valeurs consommables renforce singulièrement la possibilité d’organiser de nouveaux ensembles signifiants ;\par
— le secteur culturel n’est plus un secteur privilégié. L’art du détournement s’étend à tous les refus attestés par la vie quotidienne ;\par
— la dictature du parcellaire fait du détournement la seule technique au service de la totalité. Le détournement est le geste révolutionnaire le plus cohérent, le plus populaire et le mieux adapté à la pratique insurrectionnelle. Par une sorte de mouvement naturel – la passion du jeu – il entraîne vers l’extrême radicalisation.\par

\astermono

\noindent Dans la décomposition qui atteint l’ensemble des conduites spirituelles et matérielles – décomposition liée aux impératifs de la société de consommation – la phase de dévalorisation du détournement est en quelque sorte prise en charge et assurée par les conditions historiques. La négativité incrustée dans la réalité des faits tend ainsi à assimiler le détournement à une tactique de dépassement, à un acte essentiellement positif.\par
Si l’abondance de biens de consommation est saluée partout comme une évolution heureuse, l’emploi social de ces biens, on le sait, en corrompt le bon usage. Parce que le \emph{gadget} est avant tout prétexte à profit pour le capitalisme et les régimes bureaucratiques, il se doit d’être inutilisable à d’autres fins. L’idéologie du consommable agit comme un défaut de fabrication, elle sabote la marchandise enrobée par elle ; elle introduit dans l’équipement matériel du bonheur un nouvel esclavage. Dans ce contexte, le détournement vulgarise un autre mode d’emploi, il invente un \emph{usage} supérieur où la subjectivité manipulera à son avantage ce qui est vendu pour être manipulé contre elle. La crise du spectacle va précipiter les forces du mensonge dans le camp de la vérité vécue. L’art de retourner contre l’ennemi les armes que les nécessités commerciales lui ordonnent de distribuer est la question dominante des problèmes de tactique et de stratégie. Il faut propager les méthodes de détournement comme \emph{A B C} du consommateur qui voudrait cesser de l’être.\par
Le détournement, qui a fait ses premières armes dans l’art, est maintenant devenu l’art du maniement de toutes les armes. Apparu initialement dans les remous de la crise culturelle des années 1910-1925, il s’est étendu peu à peu à l’ensemble des secteurs touchés par la décomposition. Il n’empêche que le domaine de l’art offre encore aux techniques de détournement un champ d’expérimentation valable ; qu’il faut savoir tirer les leçons du passé. Ainsi, l’opération de réinvestissement prématuré à laquelle les surréalistes se livrèrent, en englobant dans un contexte parfaitement valable les anti-valeurs dadaïstes imparfaitement réduites à zéro, montre bien que la tentative de construire au départ d’éléments mal dévalorisés conduit toujours à la récupération par les mécanismes dominants de l’organisation sociale. L’attitude « combinatoire » des actuels cybernéticiens à propos de l’art va jusqu’à la fière accumulation insignifiante d’éléments quelconques, qui n’ont été \emph{aucunement dévalorisés}. Pop Art et Jean-Luc Godard, c’est l’apologétique du déchet.\par
L’expression artistique permet également de chercher, à tâtons et prudemment, de nouvelles formes d’agitation et de propagande. Dans cet ordre d’idées, les compositions de Michèle Bernstein en 1963 (plâtre modelé où s’incrustent des miniatures telles que soldats de plomb, voiture, tanks…) incitent, avec des titres comme « Victoire de la Bande à Bonnot », « Victoire de la Commune de Paris », « Victoire des Conseils ouvriers de Budapest », à corriger dans le sens du mieux certains événements figés artificiellement dans le passé ; à refaire l’histoire du mouvement ouvrier et, dans le même temps, à réaliser l’art. Si limitée qu’elle soit, si spéculative qu’elle demeure, une telle agitation ouvre la voie à la spontanéité créatrice de tous, ne serait-ce qu’en prouvant, dans un secteur particulièrement falsifié, que le détournement est le seul langage, le seul geste qui porte en soi sa propre critique.\par
La créativité n’a pas de limite, le détournement n’a pas de fin.
\section[{XXIV. L’intermonde et la nouvelle innocence}]{XXIV. L’intermonde et la nouvelle innocence}\renewcommand{\leftmark}{XXIV. L’intermonde et la nouvelle innocence}


\begin{argument}\noindent L’intermonde est le terrain vague de la subjectivité, le lieu où les résidus du pouvoir et de sa corrosion se mêlent à la volonté de vivre (1). – La nouvelle innocence libère les monstres de l’intériorité, elle projette la violence trouble de l’intermonde contre le vieil ordre des choses qui en est cause (2).
\end{argument}

\subsection[{1. L’intermonde}]{\textsc{1. }L’intermonde}
\noindent Il existe une frange de subjectivité troublée, rongée par le mal du pouvoir. Là s’agitent les haines indéfectibles, les dieux de vengeance, la tyrannie des envies, les renâclements de la volonté frustrée. C’est une corruption marginale qui menace de toutes parts ; un intermonde.\par
L’intermonde est le terrain vague de la subjectivité. Il contient la cruauté essentielle, celle du flic et celle de l’insurgé, celle de l’oppression et celle de la poésie de la révolte. À mi-chemin entre la récupération spectaculaire et l’usage insurrectionnel, le super-espace-temps du rêveur s’élabore monstrueusement selon les normes de la volonté individuelle et dans la perspective du pouvoir. L’appauvrissement croissant de la vie quotidienne a fini par en faire un domaine public ouvert à toutes les investigations, un lieu de lutte en terrain découvert entre la spontanéité créatrice et sa corruption. En bon explorateur de l’esprit, Artaud rend parfaitement compte de ce combat douteux :\par

\begin{quoteblock}
\noindent « L’inconscient ne m’appartient pas, sauf en rêve, et puis, tout ce que je vois en lui et qui traîne, est-ce une forme marquée pour naître ou du malpropre que j’ai rejeté ? Le subconscient est ce qui transpire des prémisses de ma volonté intérieure, mais je ne sais pas très bien qui y règne, et je crois bien que ce n’est pas moi, mais le flot des volontés adverses qui, je ne sais pourquoi, pense en moi et n’a jamais eu d’autres préoccupations au monde et d’autre idée que de prendre ma place, à moi, dans mon corps et dans mon moi. Mais dans le préconscient où leurs tentations me malmènent, toutes ces mauvaises volontés, je les revois, mais armé cette fois de toute ma conscience, et qu’elles déferlent contre moi, que m’importe puisque maintenant, je me sens là… J’aurai donc senti qu’il fallait remonter le courant et me distendre dans ma préconscience jusqu’au point où je me verrai évoluer et \emph{désirer}. »\end{quoteblock}

\noindent Et Artaud dira plus loin :\par

\begin{quoteblock}
\noindent « Le peyotl m’y a mené. »\end{quoteblock}

\noindent L’aventure du solitaire de Rodez résonne comme un avertissement. Sa rupture avec le mouvement surréaliste est significative. Il reproche au groupe de s’intégrer au bolchevisme ; de se mettre au service d’une révolution – qui, soit dit en passant, traîne après elle les fusillés de Cronstadt – au lieu de mettre la révolution à son service. Artaud a mille fois raison de s’en prendre à l’incapacité du mouvement de fonder sa cohérence révolutionnaire sur ce qu’il contenait de plus riche, le primat de la subjectivité. Mais, sitôt consommée la rupture avec le surréalisme, on le voit s’égarer dans le délire solipsiste et dans la pensée magique. Réaliser la volonté subjective en transformant le monde, il n’en est plus question. Au lieu d’extérioriser l’intériorité dans les faits, il va au contraire la sacraliser, découvrir dans le monde figé des analogies la permanence d’un mythe fondamental, à la révélation duquel accèdent seules les voies de l’impuissance. Ceux qui hésitent à jeter au-dehors l’incendie qui les dévore n’ont que le choix de brûler, de se consumer, selon les lois du consommable, dans la tunique de Nessus des idéologies – que ce soit l’idéologie de la drogue, de l’art, de la psychanalyse, de la théosophie ou de la révolution, voilà précisément ce qui ne change rien à l’histoire.\par

\astermono

\noindent L’imaginaire est la science exacte des solutions possibles. Il n’est pas un monde parallèle laissé à l’esprit pour le dédommager de ses échecs dans la réalité extérieure. Il est une force destinée à combler le fossé qui sépare l’intériorité de l’extériorité. Une \emph{praxis} condamnée à l’inaction.\par
Avec ses hantises, ses obsessions, ses flambées de haine, son sadisme, l’intermonde semble une cache aux fauves, rendus furieux par leur séquestration. Chacun est libre d’y descendre à la faveur du rêve, de la drogue, de l’alcool, du délire des sens. Il y a là une violence qui ne demande qu’à être libérée, un climat où il est bon de se plonger, ne serait-ce qu’afin d’atteindre à cette conscience qui danse et tue, et que Norman Brown a appelée la conscience dionysiaque.
\subsection[{2. La nouvelle innocence}]{\textsc{2.} La nouvelle innocence}
\noindent L’aube rouge des émeutes ne dissout pas les créatures monstrueuses de la nuit. Elle les habille de lumière et de feu, les répand par les villes, par les campagnes. La nouvelle innocence, c’est le rêve maléfique devenant réalité. La subjectivité ne se construit pas sans anéantir ses obstacles ; elle puise dans l’intermonde la violence nécessaire à cette fin. La nouvelle innocence est la construction lucide d’un anéantissement.\par
L’homme le plus paisible est couvert de rêveries sanglantes. Comme il est difficile de traiter avec sollicitude ceux qu’on ne peut abattre sur-le-champ, de désarmer par la gentillesse ceux qu’il est inopportun de désarmer par la force. À ceux qui ont failli me gouverner, je dois beaucoup de haine. Comment liquider la haine sans liquider sa cause ? La barbarie des émeutes, le pétrolage, la sauvagerie populaire, les excès que flétrissent les historiens bourgeois, c’est précisément le vaccin contre la froide atrocité des forces de l’ordre et de l’oppression hiérarchisée.\par
Dans la nouvelle innocence, l’intermonde, se débondant soudain, submerge les structures oppressives. Le jeu de la violence pure est englobé par la pure violence du jeu révolutionnaire.\par
Or le choc de la liberté fait des miracles. Il n’est rien qui lui résiste, ni les maladies de l’esprit, ni les remords, ni la culpabilité, ni le sentiment d’impuissance, ni l’abrutissement que crée l’environnement du pouvoir. Quand une canalisation d’eau creva dans le laboratoire de Pavlov, aucun des chiens qui survécurent à l’inondation ne garda la moindre trace de son long conditionnement. Le raz de marée des grands bouleversements sociaux aurait-il moins d’effet sur les hommes qu’une inondation sur les chiens ? Reich préconise de favoriser chez les névrosés affectivement bloqués et musculairement hypertoniques des explosions de colère. Ce type de névrose me paraît particulièrement répandu aujourd’hui : c’est le mal de survie. Et l’explosion la plus cohérente de colère a beaucoup de chance de ressembler à une insurrection générale.\par
Trois mille ans d’enténèbrement ne résisteront pas à dix jours de violence révolutionnaire. La reconstruction sociale va pareillement reconstruire l’inconscient individuel de tous.\par

\astermono

\noindent La révolution de la vie quotidienne liquidera les notions de justice, de châtiment, de supplice, notions subordonnées à l’échange et au parcellaire. Nous ne voulons pas être des justiciers, mais des maîtres sans esclaves, retrouvant, par-delà la destruction de l’esclavage, une nouvelle innocence, une grâce de vivre. Il s’agit de détruire l’ennemi, non de le juger. Dans les villages libérés par sa colonne, Durruti rassemblait les paysans, leur demandait de désigner les fascistes et les fusiller sur-le-champ. La prochaine révolution refera le même chemin. Sereinement. Nous savons qu’il n’y aura plus personne pour nous juger, que les juges seront à jamais absents, parce qu’on les aura mangés.\par
La nouvelle innocence implique la destruction d’un ordre de choses qui n’a fait qu’entraver de tout temps l’art de vivre, et menace aujourd’hui ce qui reste d’authenticité vécue. Je n’ai nul besoin de raisons pour défendre ma liberté. À chaque instant le pouvoir me place en état de légitime défense. Dans ce bref dialogue entre l’anarchiste Duval et le policier chargé de l’arrêter, la nouvelle innocence peut reconnaître sa jurisprudence spontanée :\par
— Duval, je vous arrête au nom de la Loi.\par
— Et moi je te supprime au nom de la Liberté.\par
Les objets ne saignent pas. Ceux qui pèsent du poids mort des choses mourront comme des choses. Comme ces porcelaines que les révolutionnaires brisaient, au sac Razoumovskoé – on leur en fit grief, ils répondirent, rapporte Victor Serge :\par

\begin{quoteblock}
\noindent « Nous briserons toutes les porcelaines du monde pour transformer la vie. Vous aimez trop les choses et pas assez les hommes… Vous aimez trop les hommes comme les choses, et pas assez l’homme. »\end{quoteblock}

\noindent Ce qu’il n’est pas nécessaire de détruire mérite d’être sauvé : c’est la forme la plus succincte de notre futur code pénal.
\section[{XXV. Suite de « Vous foutez-vous de nous ? » Vous ne vous en foutrez pas longtemps}]{XXV. Suite de « Vous foutez-vous de nous ? » Vous ne vous en foutrez pas longtemps}\renewcommand{\leftmark}{XXV. Suite de « Vous foutez-vous de nous ? » Vous ne vous en foutrez pas longtemps}

{\citbibl (Adresse des Sans-Culottes de la rue Mouffetard à la Convention, 9 décembre 1792.)}\noindent A Los Angeles, à Prague, à Stockholm, à Stanleyville, à Turin, à Mieres, à Saint-Domingue, à Amsterdam, partout où le geste et la conscience du refus suscitent de passionnants débrayages dans les usines d’illusions collectives, la révolution de la vie quotidienne est en marche. La contestation s’enrichit à mesure que la misère s’universalise. Ce qui fut longtemps la raison d’affrontements particuliers, la faim, la contrainte, l’ennui, la maladie, l’angoisse, l’esseulement, le mensonge, dévoile aujourd’hui sa rationalité fondamentale, sa forme vide et enveloppante, son abstraction terriblement oppressive. C’est au monde du pouvoir hiérarchisé, de l’État, du sacrifice, de l’échange, du quantitatif, – à la marchandise comme volonté et comme représentation du monde, – que s’en prennent les forces agissantes d’une société entièrement nouvelle, encore à inventer et cependant déjà présente. Il n’est plus une région du globe où la \emph{praxis} révolutionnaire n’agisse désormais comme révélateur, changeant le négatif en positif, illuminant dans le feu des insurrections la face cachée de la terre, dressant la carte de sa conquête.\par
Seule la \emph{praxis} révolutionnaire réelle apporte aux instructions pour une prise d’armes la précision sans laquelle les meilleures propositions restent contingentes et partielles. Mais la même \emph{praxis} montre aussi qu’elle est éminemment corruptible dès qu’elle rompt avec sa propre rationalité, – une rationalité non plus abstraite mais concrète, dépassement de la forme vide et universelle de la marchandise, – qui seule permet une objectivation non aliénante : la réalisation de l’art et de la philosophie dans le vécu individuel. La ligne de force et d’expansion d’une telle rationalité naît de la rencontre non fortuite de deux pôles sous tension. Elle est l’étincelle entre la subjectivité puisant dans le totalitarisme des conditions oppressives la volonté d’être tout, et le dépérissement qui atteint par l’histoire le système généralisé de la marchandise.\par
Les conflits existentiels ne se différencient pas qualitativement des conflits inhérents à l’ensemble des hommes. C’est pourquoi les hommes ne peuvent espérer contrôler les lois qui dominent leur histoire générale s’ils ne contrôlent en même temps leur histoire individuelle. Ceux qui s’approchent de la révolution en s’éloignant d’eux-mêmes – tous les militants – la font le dos tourné, à rebours. Contre le volontarisme et contre la mystique d’une révolution historiquement fatale, il faut répandre l’idée d’un plan d’accès, d’une construction à la fois rationnelle et passionnelle où s’unissent dialectiquement les exigences subjectives immédiates et les conditions objectives contemporaines. Le \emph{plan incliné de la révolution} est, dans la dialectique du partiel et de la totalité, le projet de construire la vie quotidienne dans et par la lutte contre la forme marchande, en sorte que chaque stade particulier de la révolution représente son aboutissement final. Ni programme maximum, ni programme minimum, ni programme transitoire, mais une stratégie d’ensemble fondée sur les caractères essentiels du système à détruire, et contre lesquels porteront les premiers coups.\par
Dans le moment insurrectionnel, et donc aussi dès maintenant, les groupes révolutionnaires devront poser globalement les problèmes imposés par la diversité des circonstances, de même que le prolétariat les résoudra globalement en se défaisant. Citons entre autres : comment dépasser concrètement le travail, sa division, l’opposition travail-loisir (problème de la reconstruction des rapports humains par une \emph{praxis} passionnante et consciente touchant tous les aspects de la vie sociale, etc.) ? Comment dépasser concrètement l’échange (problème de la dévalorisation de l’argent, y compris de la subversion par la fausse monnaie, des relations détruisant la vieille économie, de la liquidation des secteurs parasitaires, etc.) ? Comment dépasser concrètement l’État et toute forme de communauté aliénante (problème de la construction de situations, des assemblées d’autogestion, d’un droit positif cautionnant toutes les libertés et permettant la suppression des secteurs retardataires, etc.) ? Comment organiser l’extension du mouvement au départ de zones-clés afin de révolutionner l’ensemble des conditions établies partout (auto-défense, rapports avec les régions non libérées, vulgarisation de l’usage et de la fabrication d’armes, etc.) ?\par
Entre la vieille société en désorganisation et la société nouvelle à organiser, l’Internationale situationniste offre un exemple de groupe à la recherche de sa cohérence révolutionnaire. Son importance, comme celle de tout groupe porteur de la poésie, c’est qu’elle va servir de modèle à la nouvelle organisation sociale. Il faut donc empêcher que l’oppression extérieure (hiérarchie, bureaucratisation…) se reproduise à l’intérieur du mouvement. Comment ? En exigeant que la participation soit subordonnée au maintien de l’égalité réelle entre tous les membres, non comme un droit métaphysique mais au contraire comme la norme à atteindre. C’est précisément pour éviter l’autoritarisme et la passivité (les dirigeants et les militants) que le groupe doit sans hésiter sanctionner toute baisse de niveau théorique, tout abandon pratique, toute compromission. Rien n’autorise à tolérer des gens que le régime dominant sait fort bien tolérer. L’exclusion et la rupture sont les seules défenses de la cohérence en péril.\par
De même, le projet de centraliser la poésie éparse implique la faculté de reconnaître ou de susciter des groupes autonomes révolutionnaires, de les radicaliser, de les fédérer sans en assumer jamais la direction. La fonction de L’Internationale situationniste est une fonction axiale : être partout comme un axe que l’agitation populaire fait tourner et qui propage à son tour, en le multipliant, le mouvement initialement reçu. Les situationnistes reconnaîtront les leurs sur le critère de la cohérence révolutionnaire.\par
La longue révolution nous achemine vers l’édification d’une société parallèle, opposée à la société dominante et en passe de la remplacer ; ou mieux, vers la constitution de micro-sociétés coalisées, véritables foyers de guérilla, en lutte pour l’\emph{autogestion généralisée}. La radicalité effective autorise à toutes les variantes, est la garantie de toutes les libertés. Les situationnistes n’arrivent donc pas face au monde avec un nouveau type de société : voici l’organisation idéale, à genoux ! Ils montrent seulement en combattant pour eux-mêmes, et avec la plus haute conscience de ce combat, pourquoi les gens se battent vraiment, et pourquoi la conscience d’une telle bataille doit être acquise.\par

\dateline{(1963-1965)}
\section[{Toast aux ouvriers révolutionnaires}]{Toast aux ouvriers révolutionnaires}\renewcommand{\leftmark}{Toast aux ouvriers révolutionnaires}

\noindent La critique radicale n’a fait qu’analyser le vieux monde et ce qui le nie. Elle doit maintenant se réaliser dans la pratique des masses révolutionnaires ou se renier contre elle.\par
Tant que le projet de l’homme total restera le fantôme qui hante l’absence de réalisation individuelle immédiate, tant que le prolétariat n’aura pas arraché de fait la théorie à ceux qui l’apprennent de son propre mouvement, le pas en avant de la radicalité sera toujours suivi de deux pas en arrière de l’idéologie.\par
En incitant les prolétaires à s’emparer de la théorie tirée du vécu et du non-vécu quotidien, le \emph{Traité} prenait, en même temps que le parti du dépassement, le risque de toutes les falsifications auxquelles l’exposait le retard de sa mise en œuvre insurrectionnelle. Dès l’instant qu’elle échappe au mouvement de la conscience révolutionnaire soudain freinée par l’histoire, la théorie radicale devient autre en restant elle-même, elle n’échappe pas tout à fait au mouvement similaire et inverse, à la régression vers la pensée séparée, vers le spectacle. Et qu’elle porte en soi sa propre critique ne l’expose jamais qu’à supporter en plus de la vermine idéologique – dont la variété s’étend ici du subjectivisme au nihilisme, en passant par le communautaire et l’hédonisme apolitique – les grenouilles boursouflées de la critique-critique.\par
Les atermoiements d’une action ouvrière radicale, qui mettra bientôt au service des passions et des besoins individuels les aires de production et de consommation qu’elle est \emph{initialement} seule à pouvoir détourner, ont montré que la fraction du prolétariat sans emprise directe sur les mécanismes économiques réussissait seulement, dans sa phase ascendante, à formuler et à diffuser une théorie qu’incapable de réaliser et de corriger par elle-même elle transforme, dans sa phase de défaite, en une régression intellectuelle. La conscience sans usage n’a plus qu’à se justifier comme conscience usagée.\par
Ce que l’expression subjective du projet situationniste a pu donner de meilleur dans la préparation de mai 1968 et dans la prise de conscience des nouvelles formes d’exploitation est ensuite devenu le pire dans la lecture intellectualisée à laquelle s’est résignée l’impuissance d’un grand nombre à détruire ce que seuls pouvaient détruire, moins du reste par occupation que par sabotage et détournement, les travailleurs responsables des secteurs clés de la production et de la consommation.\par
Parce que le projet situationniste a été la pensée pratique la plus avancée de ce prolétariat sans mainmise sur les centres moteurs du processus marchand, et aussi parce qu’il n’a jamais cessé de se donner pour tâche unique d’anéantir l’\emph{organisation sociale de survie au profit de l’autogestion généralisée}, il ne peut tôt ou tard que reprendre son mouvement réel en milieu ouvrier, laissant au spectacle et à ses flatulences critiques le soin de le découvrir ou de l’augmenter de scolies.\par
La théorie radicale appartient à qui la rend meilleure. La défendre contre le livre, contre la marchandise culturelle où elle reste trop souvent et trop longtemps \emph{exposée}, ce n’est pas en appeler à l’ouvrier anti-travail, anti-sacrifice, anti-hiérarchie contre le prolétaire réduit à la conscience, désarmée, des mêmes refus ; c’est exiger de ceux qui sont à la base de la lutte unitaire contre la société de survie qu’ils aient recours aux modes d’expression dont ils disposent avec le plus d’efficacité, aux actes révolutionnaires qui créent leur langage dans les conditions elles-mêmes créées pour empêcher tout retour en arrière. Le sabotage du travail forcé, la destruction du processus de production et de reproduction de la marchandise, le détournement des stocks et des forces productives au profit des révolutionnaires et de tous ceux qui les rejoindront par attraction passionnelle, voilà ce qui peut mettre fin non seulement à la réserve bureaucratique que constituent les ouvriers intellectualistes et les intellectuels ouvriéristes, mais à la séparation entre intellectuels et manuels, à toutes les séparations. Contre la division du travail et l’usine universelle, unité du non-travail et autogestion généralisée !\par
L’évidence des principales thèses du \emph{Traité} doit maintenant se manifester dans les mains de ses anti-lecteurs sous forme de résultats concrets. Non plus dans une agitation d’étudiants mais dans la révolution totale. Il faut que la théorie porte la violence où la violence est déjà. Ouvriers des Asturies, du Limbourg, de Poznan, de Lyon, de Detroit, de Gsepel, de Leningrad, de Canton, de Buenos Aires, de Johannesburg, de Liverpool, de Kiruna, de Coïmbra, il vous appartient d’accorder au prolétariat tout entier le pouvoir d’étendre au plaisir de la révolution faite pour soi et pour tous le plaisir pris chaque jour à l’amour, à la destruction des contraintes, à la jouissance des passions.\par
Sans la critique des armes, les armes de la critique sont les armes du suicide. Quand ils ne tombent pas dans le désespoir du terrorisme ou dans la misère de la contestation, bon nombre de prolétaires deviennent les voyeurs de la classe ouvrière, les spectateurs de leur propre efficacité différée. Contents d’être révolutionnaires par procuration à force d’avoir été cocus et battus comme révolutionnaires sans révolution, ils attendent que se précipite la baisse tendancielle de pouvoir des cadres bureaucratiques pour proposer leur médiation et se conduire en chefs au nom de leur impuissance objective à briser le spectacle. C’est pourquoi il importe tant que l’organisation des ouvriers insurgés – la seule nécessaire aujourd’hui – soit l’œuvre des ouvriers insurgés eux-mêmes, afin qu’elle serve de modèle d’organisation au prolétariat tout entier dans sa lutte pour l’autogestion généralisée. Avec elle prendront fin définitivement les organisations répressives (États, partis, syndicats, groupes hiérarchisés) et leur complément critique, le fétichisme organisationnel qui sévit dans le prolétariat non producteur. Elle corrigera dans la pratique immédiate la contradiction du volontarisme et du réalisme par laquelle l’IS (J’ai quitté l’IS et sa croissante quantité d’importance nulle en novembre 1970), en ne disposant que de l’exclusion et de la rupture pour empêcher l’incessante reproduction du monde dominant dans le groupe, a montré ses limites et démontré son incapacité d’harmoniser les accords et les discords intersubjectifs. Elle prouvera enfin que la fraction du prolétariat séparée des possibilités concrètes de détourner les moyens de production a besoin non d’organisation mais d’individus agissant pour leur compte, se fédérant occasionnellement en commandos de sabotage (neutralisation des réseaux répressifs, occupation de la radio, etc.), intervenant où et quand l’opportunité leur offre des garanties d’efficacité tactique et stratégique, n’ayant d’autre souci que de jouir sans réserves et \emph{inséparablement} d’attiser partout les étincelles de la guérilla ouvrière, le feu négatif et positif qui, venu de la base du prolétariat, est aussi la seule base de liquidation du prolétariat et de la société de classes.\par
S’il manque aux ouvriers la cohérence de leur efficacité possible, du moins sont-ils assurés de la conquérir pour tous et de façon décisive, car à travers l’expérience des grèves sauvages et des émeutes se manifeste clairement la résurgence des assemblées de conseils, le retour des Communes, dont les apparitions soudaines ne surprendront – le temps d’une contre-attaque répressive sans comparaison avec la répression des mouvements intellectuels – que ceux qui ne voient pas sous la diversité de l’immobilité spectaculaire le progrès unitaire de la vieille taupe, la lutte clandestine du prolétariat pour l’appropriation de l’histoire et le bouleversement global de toutes les conditions de la vie quotidienne. Et la nécessité de l’\emph{histoire-pour-soi} dévoile aussi son ironie dans la cohérence négative à laquelle aboutit au mieux le prolétariat désarmé, une cohérence en creux partout présente comme une mise en garde objective contre ce qui menace par l’intérieur la radicalité ouvrière : l’intellectualisation, avec sa régression de la conscience au savoir et à la culture ; les médiateurs non contrôlés et leur bureaucratie critique ; les obsédés du prestige, plus soucieux du renouvellement des rôles que de leur disparition dans l’émulation ludique de la guérilla de base ; le renoncement à la subversion concrète, à la conquête révolutionnaire du territoire et à son mouvement unitaire-international vers la fin des séparations, du sacrifice, du travail forcé, de la hiérarchie, de la marchandise sous toutes ses formes.\par
Le défi que la réification lance à la créativité de chacun n’est plus dans les « que faire ? » théoriques mais dans la pratique du fait révolutionnaire. Quiconque ne découvre pas dans la révolution la passion pivotale qui permet toutes les autres n’a que les ombres du plaisir. En ce sens, le \emph{Traité} est le chemin le plus court de la subjectivité individuelle à sa réalisation dans l’histoire faite par tous. Au regard de la longue révolution, il n’est qu’un petit point, mais un des points de départ du mouvement communaliste d’autogestion généralisée, comme il n’est qu’une esquisse, mais du jugement de mort que la société de survie prononce contre elle-même et que l’internationale des usines, des campagnes et des rues exécutera sans appel.\par
Pour un monde de jouissance à gagner, nous n’avons à perdre que l’ennui.\par

\dateline{Octobre 1972.}
 


% at least one empty page at end (for booklet couv)
\ifbooklet
  \pagestyle{empty}
  \clearpage
  % 2 empty pages maybe needed for 4e cover
  \ifnum\modulo{\value{page}}{4}=0 \hbox{}\newpage\hbox{}\newpage\fi
  \ifnum\modulo{\value{page}}{4}=1 \hbox{}\newpage\hbox{}\newpage\fi


  \hbox{}\newpage
  \ifodd\value{page}\hbox{}\newpage\fi
  {\centering\color{rubric}\bfseries\noindent\large
    Hurlus ? Qu’est-ce.\par
    \bigskip
  }
  \noindent Des bouquinistes électroniques, pour du texte libre à participation libre,
  téléchargeable gratuitement sur \href{https://hurlus.fr}{\dotuline{hurlus.fr}}.\par
  \bigskip
  \noindent Cette brochure a été produite par des éditeurs bénévoles.
  Elle n’est pas faîte pour être possédée, mais pour être lue, et puis donnée.
  Que circule le texte !
  En page de garde, on peut ajouter une date, un lieu, un nom ; pour suivre le voyage des idées.
  \par

  Ce texte a été choisi parce qu’une personne l’a aimé,
  ou haï, elle a en tous cas pensé qu’il partipait à la formation de notre présent ;
  sans le souci de plaire, vendre, ou militer pour une cause.
  \par

  L’édition électronique est soigneuse, tant sur la technique
  que sur l’établissement du texte ; mais sans aucune prétention scolaire, au contraire.
  Le but est de s’adresser à tous, sans distinction de science ou de diplôme.
  Au plus direct ! (possible)
  \par

  Cet exemplaire en papier a été tiré sur une imprimante personnelle
   ou une photocopieuse. Tout le monde peut le faire.
  Il suffit de
  télécharger un fichier sur \href{https://hurlus.fr}{\dotuline{hurlus.fr}},
  d’imprimer, et agrafer ; puis de lire et donner.\par

  \bigskip

  \noindent PS : Les hurlus furent aussi des rebelles protestants qui cassaient les statues dans les églises catholiques. En 1566 démarra la révolte des gueux dans le pays de Lille. L’insurrection enflamma la région jusqu’à Anvers où les gueux de mer bloquèrent les bateaux espagnols.
  Ce fut une rare guerre de libération dont naquit un pays toujours libre : les Pays-Bas.
  En plat pays francophone, par contre, restèrent des bandes de huguenots, les hurlus, progressivement réprimés par la très catholique Espagne.
  Cette mémoire d’une défaite est éteinte, rallumons-la. Sortons les livres du culte universitaire, cherchons les idoles de l’époque, pour les briser.
\fi

\ifdev % autotext in dev mode
\fontname\font — \textsc{Les règles du jeu}\par
(\hyperref[utopie]{\underline{Lien}})\par
\noindent \initialiv{A}{lors là}\blindtext\par
\noindent \initialiv{À}{ la bonheur des dames}\blindtext\par
\noindent \initialiv{É}{tonnez-le}\blindtext\par
\noindent \initialiv{Q}{ualitativement}\blindtext\par
\noindent \initialiv{V}{aloriser}\blindtext\par
\Blindtext
\phantomsection
\label{utopie}
\Blinddocument
\fi
\end{document}
