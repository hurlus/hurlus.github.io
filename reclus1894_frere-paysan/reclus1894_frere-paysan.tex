%%%%%%%%%%%%%%%%%%%%%%%%%%%%%%%%%
% LaTeX model https://hurlus.fr %
%%%%%%%%%%%%%%%%%%%%%%%%%%%%%%%%%

% Needed before document class
\RequirePackage{pdftexcmds} % needed for tests expressions
\RequirePackage{fix-cm} % correct units

% Define mode
\def\mode{a4}

\newif\ifaiv % a4
\newif\ifav % a5
\newif\ifbooklet % booklet
\newif\ifcover % cover for booklet

\ifnum \strcmp{\mode}{cover}=0
  \covertrue
\else\ifnum \strcmp{\mode}{booklet}=0
  \booklettrue
\else\ifnum \strcmp{\mode}{a5}=0
  \avtrue
\else
  \aivtrue
\fi\fi\fi

\ifbooklet % do not enclose with {}
  \documentclass[french,twoside]{book} % ,notitlepage
  \usepackage[%
    papersize={105mm, 297mm},
    inner=12mm,
    outer=12mm,
    top=20mm,
    bottom=15mm,
    marginparsep=0pt,
  ]{geometry}
  \usepackage[fontsize=9.5pt]{scrextend} % for Roboto
\else\ifav
  \documentclass[french,twoside]{book} % ,notitlepage
  \usepackage[%
    a5paper,
    inner=25mm,
    outer=15mm,
    top=15mm,
    bottom=15mm,
    marginparsep=0pt,
  ]{geometry}
  \usepackage[fontsize=12pt]{scrextend}
\else% A4 2 cols
  \documentclass[twocolumn]{report}
  \usepackage[%
    a4paper,
    inner=15mm,
    outer=10mm,
    top=25mm,
    bottom=18mm,
    marginparsep=0pt,
  ]{geometry}
  \setlength{\columnsep}{20mm}
  \usepackage[fontsize=9.5pt]{scrextend}
\fi\fi

%%%%%%%%%%%%%%
% Alignments %
%%%%%%%%%%%%%%
% before teinte macros

\setlength{\arrayrulewidth}{0.2pt}
\setlength{\columnseprule}{\arrayrulewidth} % twocol
\setlength{\parskip}{0pt} % classical para with no margin
\setlength{\parindent}{1.5em}

%%%%%%%%%%
% Colors %
%%%%%%%%%%
% before Teinte macros

\usepackage[dvipsnames]{xcolor}
\definecolor{rubric}{HTML}{800000} % the tonic 0c71c3
\def\columnseprulecolor{\color{rubric}}
\colorlet{borderline}{rubric!30!} % definecolor need exact code
\definecolor{shadecolor}{gray}{0.95}
\definecolor{bghi}{gray}{0.5}

%%%%%%%%%%%%%%%%%
% Teinte macros %
%%%%%%%%%%%%%%%%%
%%%%%%%%%%%%%%%%%%%%%%%%%%%%%%%%%%%%%%%%%%%%%%%%%%%
% <TEI> generic (LaTeX names generated by Teinte) %
%%%%%%%%%%%%%%%%%%%%%%%%%%%%%%%%%%%%%%%%%%%%%%%%%%%
% This template is inserted in a specific design
% It is XeLaTeX and otf fonts

\makeatletter % <@@@


\usepackage{blindtext} % generate text for testing
\usepackage[strict]{changepage} % for modulo 4
\usepackage{contour} % rounding words
\usepackage[nodayofweek]{datetime}
% \usepackage{DejaVuSans} % seems buggy for sffont font for symbols
\usepackage{enumitem} % <list>
\usepackage{etoolbox} % patch commands
\usepackage{fancyvrb}
\usepackage{fancyhdr}
\usepackage{float}
\usepackage{fontspec} % XeLaTeX mandatory for fonts
\usepackage{footnote} % used to capture notes in minipage (ex: quote)
\usepackage{framed} % bordering correct with footnote hack
\usepackage{graphicx}
\usepackage{lettrine} % drop caps
\usepackage{lipsum} % generate text for testing
\usepackage[framemethod=tikz,]{mdframed} % maybe used for frame with footnotes inside
\usepackage{pdftexcmds} % needed for tests expressions
\usepackage{polyglossia} % non-break space french punct, bug Warning: "Failed to patch part"
\usepackage[%
  indentfirst=false,
  vskip=1em,
  noorphanfirst=true,
  noorphanafter=true,
  leftmargin=\parindent,
  rightmargin=0pt,
]{quoting}
\usepackage{ragged2e}
\usepackage{setspace} % \setstretch for <quote>
\usepackage{tabularx} % <table>
\usepackage[explicit]{titlesec} % wear titles, !NO implicit
\usepackage{tikz} % ornaments
\usepackage{tocloft} % styling tocs
\usepackage[fit]{truncate} % used im runing titles
\usepackage{unicode-math}
\usepackage[normalem]{ulem} % breakable \uline, normalem is absolutely necessary to keep \emph
\usepackage{verse} % <l>
\usepackage{xcolor} % named colors
\usepackage{xparse} % @ifundefined
\XeTeXdefaultencoding "iso-8859-1" % bad encoding of xstring
\usepackage{xstring} % string tests
\XeTeXdefaultencoding "utf-8"
\PassOptionsToPackage{hyphens}{url} % before hyperref, which load url package

% TOTEST
% \usepackage{hypcap} % links in caption ?
% \usepackage{marginnote}
% TESTED
% \usepackage{background} % doesn’t work with xetek
% \usepackage{bookmark} % prefers the hyperref hack \phantomsection
% \usepackage[color, leftbars]{changebar} % 2 cols doc, impossible to keep bar left
% \usepackage[utf8x]{inputenc} % inputenc package ignored with utf8 based engines
% \usepackage[sfdefault,medium]{inter} % no small caps
% \usepackage{firamath} % choose firasans instead, firamath unavailable in Ubuntu 21-04
% \usepackage{flushend} % bad for last notes, supposed flush end of columns
% \usepackage[stable]{footmisc} % BAD for complex notes https://texfaq.org/FAQ-ftnsect
% \usepackage{helvet} % not for XeLaTeX
% \usepackage{multicol} % not compatible with too much packages (longtable, framed, memoir…)
% \usepackage[default,oldstyle,scale=0.95]{opensans} % no small caps
% \usepackage{sectsty} % \chapterfont OBSOLETE
% \usepackage{soul} % \ul for underline, OBSOLETE with XeTeX
% \usepackage[breakable]{tcolorbox} % text styling gone, footnote hack not kept with breakable


% Metadata inserted by a program, from the TEI source, for title page and runing heads
\title{\textbf{ À mon Frère le paysan }}
\date{1925}
\author{Élisée Reclus}
\def\elbibl{Élisée Reclus. 1925. \emph{À mon Frère le paysan}}
\def\elsource{ \href{http://gallica.bnf.fr/ark:/12148/bpt6k818649}{\dotuline{http://gallica.bnf.fr/ark:/12148/bpt6k818649}}\footnote{\href{http://gallica.bnf.fr/ark:/12148/bpt6k818649}{\url{http://gallica.bnf.fr/ark:/12148/bpt6k818649}}}  \href{http://efele.net/ebooks/livres/000157}{\dotuline{http://efele.net/ebooks/livres/000157}}\footnote{\href{http://efele.net/ebooks/livres/000157}{\url{http://efele.net/ebooks/livres/000157}}} }

% Default metas
\newcommand{\colorprovide}[2]{\@ifundefinedcolor{#1}{\colorlet{#1}{#2}}{}}
\colorprovide{rubric}{red}
\colorprovide{silver}{lightgray}
\@ifundefined{syms}{\newfontfamily\syms{DejaVu Sans}}{}
\newif\ifdev
\@ifundefined{elbibl}{% No meta defined, maybe dev mode
  \newcommand{\elbibl}{Titre court ?}
  \newcommand{\elbook}{Titre du livre source ?}
  \newcommand{\elabstract}{Résumé\par}
  \newcommand{\elurl}{http://oeuvres.github.io/elbook/2}
  \author{Éric Lœchien}
  \title{Un titre de test assez long pour vérifier le comportement d’une maquette}
  \date{1566}
  \devtrue
}{}
\let\eltitle\@title
\let\elauthor\@author
\let\eldate\@date


\defaultfontfeatures{
  % Mapping=tex-text, % no effect seen
  Scale=MatchLowercase,
  Ligatures={TeX,Common},
}


% generic typo commands
\newcommand{\astermono}{\medskip\centerline{\color{rubric}\large\selectfont{\syms ✻}}\medskip\par}%
\newcommand{\astertri}{\medskip\par\centerline{\color{rubric}\large\selectfont{\syms ✻\,✻\,✻}}\medskip\par}%
\newcommand{\asterism}{\bigskip\par\noindent\parbox{\linewidth}{\centering\color{rubric}\large{\syms ✻}\\{\syms ✻}\hskip 0.75em{\syms ✻}}\bigskip\par}%

% lists
\newlength{\listmod}
\setlength{\listmod}{\parindent}
\setlist{
  itemindent=!,
  listparindent=\listmod,
  labelsep=0.2\listmod,
  parsep=0pt,
  % topsep=0.2em, % default topsep is best
}
\setlist[itemize]{
  label=—,
  leftmargin=0pt,
  labelindent=1.2em,
  labelwidth=0pt,
}
\setlist[enumerate]{
  label={\bf\color{rubric}\arabic*.},
  labelindent=0.8\listmod,
  leftmargin=\listmod,
  labelwidth=0pt,
}
\newlist{listalpha}{enumerate}{1}
\setlist[listalpha]{
  label={\bf\color{rubric}\alph*.},
  leftmargin=0pt,
  labelindent=0.8\listmod,
  labelwidth=0pt,
}
\newcommand{\listhead}[1]{\hspace{-1\listmod}\emph{#1}}

\renewcommand{\hrulefill}{%
  \leavevmode\leaders\hrule height 0.2pt\hfill\kern\z@}

% General typo
\DeclareTextFontCommand{\textlarge}{\large}
\DeclareTextFontCommand{\textsmall}{\small}

% commands, inlines
\newcommand{\anchor}[1]{\Hy@raisedlink{\hypertarget{#1}{}}} % link to top of an anchor (not baseline)
\newcommand\abbr[1]{#1}
\newcommand{\autour}[1]{\tikz[baseline=(X.base)]\node [draw=rubric,thin,rectangle,inner sep=1.5pt, rounded corners=3pt] (X) {\color{rubric}#1};}
\newcommand\corr[1]{#1}
\newcommand{\ed}[1]{ {\color{silver}\sffamily\footnotesize (#1)} } % <milestone ed="1688"/>
\newcommand\expan[1]{#1}
\newcommand\foreign[1]{\emph{#1}}
\newcommand\gap[1]{#1}
\renewcommand{\LettrineFontHook}{\color{rubric}}
\newcommand{\initial}[2]{\lettrine[lines=2, loversize=0.3, lhang=0.3]{#1}{#2}}
\newcommand{\initialiv}[2]{%
  \let\oldLFH\LettrineFontHook
  % \renewcommand{\LettrineFontHook}{\color{rubric}\ttfamily}
  \IfSubStr{QJ’}{#1}{
    \lettrine[lines=4, lhang=0.2, loversize=-0.1, lraise=0.2]{\smash{#1}}{#2}
  }{\IfSubStr{É}{#1}{
    \lettrine[lines=4, lhang=0.2, loversize=-0, lraise=0]{\smash{#1}}{#2}
  }{\IfSubStr{ÀÂ}{#1}{
    \lettrine[lines=4, lhang=0.2, loversize=-0, lraise=0, slope=0.6em]{\smash{#1}}{#2}
  }{\IfSubStr{A}{#1}{
    \lettrine[lines=4, lhang=0.2, loversize=0.2, slope=0.6em]{\smash{#1}}{#2}
  }{\IfSubStr{V}{#1}{
    \lettrine[lines=4, lhang=0.2, loversize=0.2, slope=-0.5em]{\smash{#1}}{#2}
  }{
    \lettrine[lines=4, lhang=0.2, loversize=0.2]{\smash{#1}}{#2}
  }}}}}
  \let\LettrineFontHook\oldLFH
}
\newcommand{\labelchar}[1]{\textbf{\color{rubric} #1}}
\newcommand{\milestone}[1]{\autour{\footnotesize\color{rubric} #1}} % <milestone n="4"/>
\newcommand\name[1]{#1}
\newcommand\orig[1]{#1}
\newcommand\orgName[1]{#1}
\newcommand\persName[1]{#1}
\newcommand\placeName[1]{#1}
\newcommand{\pn}[1]{\IfSubStr{-—–¶}{#1}% <p n="3"/>
  {\noindent{\bfseries\color{rubric}   ¶  }}
  {{\footnotesize\autour{ #1}  }}}
\newcommand\reg{}
% \newcommand\ref{} % already defined
\newcommand\sic[1]{#1}
\newcommand\surname[1]{\textsc{#1}}
\newcommand\term[1]{\textbf{#1}}

\def\mednobreak{\ifdim\lastskip<\medskipamount
  \removelastskip\nopagebreak\medskip\fi}
\def\bignobreak{\ifdim\lastskip<\bigskipamount
  \removelastskip\nopagebreak\bigskip\fi}

% commands, blocks
\newcommand{\byline}[1]{\bigskip{\RaggedLeft{#1}\par}\bigskip}
\newcommand{\bibl}[1]{{\RaggedLeft{#1}\par\bigskip}}
\newcommand{\biblitem}[1]{{\noindent\hangindent=\parindent   #1\par}}
\newcommand{\dateline}[1]{\medskip{\RaggedLeft{#1}\par}\bigskip}
\newcommand{\labelblock}[1]{\medbreak{\noindent\color{rubric}\bfseries #1}\par\mednobreak}
\newcommand{\salute}[1]{\bigbreak{#1}\par\medbreak}
\newcommand{\signed}[1]{\bigbreak\filbreak{\raggedleft #1\par}\medskip}

% environments for blocks (some may become commands)
\newenvironment{borderbox}{}{} % framing content
\newenvironment{citbibl}{\ifvmode\hfill\fi}{\ifvmode\par\fi }
\newenvironment{docAuthor}{\ifvmode\vskip4pt\fontsize{16pt}{18pt}\selectfont\fi\itshape}{\ifvmode\par\fi }
\newenvironment{docDate}{}{\ifvmode\par\fi }
\newenvironment{docImprint}{\vskip6pt}{\ifvmode\par\fi }
\newenvironment{docTitle}{\vskip6pt\bfseries\fontsize{18pt}{22pt}\selectfont}{\par }
\newenvironment{msHead}{\vskip6pt}{\par}
\newenvironment{msItem}{\vskip6pt}{\par}
\newenvironment{titlePart}{}{\par }


% environments for block containers
\newenvironment{argument}{\itshape\parindent0pt}{\vskip1.5em}
\newenvironment{biblfree}{}{\ifvmode\par\fi }
\newenvironment{bibitemlist}[1]{%
  \list{\@biblabel{\@arabic\c@enumiv}}%
  {%
    \settowidth\labelwidth{\@biblabel{#1}}%
    \leftmargin\labelwidth
    \advance\leftmargin\labelsep
    \@openbib@code
    \usecounter{enumiv}%
    \let\p@enumiv\@empty
    \renewcommand\theenumiv{\@arabic\c@enumiv}%
  }
  \sloppy
  \clubpenalty4000
  \@clubpenalty \clubpenalty
  \widowpenalty4000%
  \sfcode`\.\@m
}%
{\def\@noitemerr
  {\@latex@warning{Empty `bibitemlist' environment}}%
\endlist}
\newenvironment{quoteblock}% may be used for ornaments
  {\begin{quoting}}
  {\end{quoting}}

% table () is preceded and finished by custom command
\newcommand{\tableopen}[1]{%
  \ifnum\strcmp{#1}{wide}=0{%
    \begin{center}
  }
  \else\ifnum\strcmp{#1}{long}=0{%
    \begin{center}
  }
  \else{%
    \begin{center}
  }
  \fi\fi
}
\newcommand{\tableclose}[1]{%
  \ifnum\strcmp{#1}{wide}=0{%
    \end{center}
  }
  \else\ifnum\strcmp{#1}{long}=0{%
    \end{center}
  }
  \else{%
    \end{center}
  }
  \fi\fi
}


% text structure
\newcommand\chapteropen{} % before chapter title
\newcommand\chaptercont{} % after title, argument, epigraph…
\newcommand\chapterclose{} % maybe useful for multicol settings
\setcounter{secnumdepth}{-2} % no counters for hierarchy titles
\setcounter{tocdepth}{5} % deep toc
\markright{\@title} % ???
\markboth{\@title}{\@author} % ???
\renewcommand\tableofcontents{\@starttoc{toc}}
% toclof format
% \renewcommand{\@tocrmarg}{0.1em} % Useless command?
% \renewcommand{\@pnumwidth}{0.5em} % {1.75em}
\renewcommand{\@cftmaketoctitle}{}
\setlength{\cftbeforesecskip}{\z@ \@plus.2\p@}
\renewcommand{\cftchapfont}{}
\renewcommand{\cftchapdotsep}{\cftdotsep}
\renewcommand{\cftchapleader}{\normalfont\cftdotfill{\cftchapdotsep}}
\renewcommand{\cftchappagefont}{\bfseries}
\setlength{\cftbeforechapskip}{0em \@plus\p@}
% \renewcommand{\cftsecfont}{\small\relax}
\renewcommand{\cftsecpagefont}{\normalfont}
% \renewcommand{\cftsubsecfont}{\small\relax}
\renewcommand{\cftsecdotsep}{\cftdotsep}
\renewcommand{\cftsecpagefont}{\normalfont}
\renewcommand{\cftsecleader}{\normalfont\cftdotfill{\cftsecdotsep}}
\setlength{\cftsecindent}{1em}
\setlength{\cftsubsecindent}{2em}
\setlength{\cftsubsubsecindent}{3em}
\setlength{\cftchapnumwidth}{1em}
\setlength{\cftsecnumwidth}{1em}
\setlength{\cftsubsecnumwidth}{1em}
\setlength{\cftsubsubsecnumwidth}{1em}

% footnotes
\newif\ifheading
\newcommand*{\fnmarkscale}{\ifheading 0.70 \else 1 \fi}
\renewcommand\footnoterule{\vspace*{0.3cm}\hrule height \arrayrulewidth width 3cm \vspace*{0.3cm}}
\setlength\footnotesep{1.5\footnotesep} % footnote separator
\renewcommand\@makefntext[1]{\parindent 1.5em \noindent \hb@xt@1.8em{\hss{\normalfont\@thefnmark . }}#1} % no superscipt in foot
\patchcmd{\@footnotetext}{\footnotesize}{\footnotesize\sffamily}{}{} % before scrextend, hyperref


%   see https://tex.stackexchange.com/a/34449/5049
\def\truncdiv#1#2{((#1-(#2-1)/2)/#2)}
\def\moduloop#1#2{(#1-\truncdiv{#1}{#2}*#2)}
\def\modulo#1#2{\number\numexpr\moduloop{#1}{#2}\relax}

% orphans and widows
\clubpenalty=9996
\widowpenalty=9999
\brokenpenalty=4991
\predisplaypenalty=10000
\postdisplaypenalty=1549
\displaywidowpenalty=1602
\hyphenpenalty=400
% Copied from Rahtz but not understood
\def\@pnumwidth{1.55em}
\def\@tocrmarg {2.55em}
\def\@dotsep{4.5}
\emergencystretch 3em
\hbadness=4000
\pretolerance=750
\tolerance=2000
\vbadness=4000
\def\Gin@extensions{.pdf,.png,.jpg,.mps,.tif}
% \renewcommand{\@cite}[1]{#1} % biblio

\usepackage{hyperref} % supposed to be the last one, :o) except for the ones to follow
\urlstyle{same} % after hyperref
\hypersetup{
  % pdftex, % no effect
  pdftitle={\elbibl},
  % pdfauthor={Your name here},
  % pdfsubject={Your subject here},
  % pdfkeywords={keyword1, keyword2},
  bookmarksnumbered=true,
  bookmarksopen=true,
  bookmarksopenlevel=1,
  pdfstartview=Fit,
  breaklinks=true, % avoid long links
  pdfpagemode=UseOutlines,    % pdf toc
  hyperfootnotes=true,
  colorlinks=false,
  pdfborder=0 0 0,
  % pdfpagelayout=TwoPageRight,
  % linktocpage=true, % NO, toc, link only on page no
}

\makeatother % /@@@>
%%%%%%%%%%%%%%
% </TEI> end %
%%%%%%%%%%%%%%


%%%%%%%%%%%%%
% footnotes %
%%%%%%%%%%%%%
\renewcommand{\thefootnote}{\bfseries\textcolor{rubric}{\arabic{footnote}}} % color for footnote marks

%%%%%%%%%
% Fonts %
%%%%%%%%%
\usepackage[]{roboto} % SmallCaps, Regular is a bit bold
% \linespread{0.90} % too compact, keep font natural
\newfontfamily\fontrun[]{Roboto Condensed Light} % condensed runing heads
\ifav
  \setmainfont[
    ItalicFont={Roboto Light Italic},
  ]{Roboto}
\else\ifbooklet
  \setmainfont[
    ItalicFont={Roboto Light Italic},
  ]{Roboto}
\else
\setmainfont[
  ItalicFont={Roboto Italic},
]{Roboto Light}
\fi\fi
\renewcommand{\LettrineFontHook}{\bfseries\color{rubric}}
% \renewenvironment{labelblock}{\begin{center}\bfseries\color{rubric}}{\end{center}}

%%%%%%%%
% MISC %
%%%%%%%%

\setdefaultlanguage[frenchpart=false]{french} % bug on part


\newenvironment{quotebar}{%
    \def\FrameCommand{{\color{rubric!10!}\vrule width 0.5em} \hspace{0.9em}}%
    \def\OuterFrameSep{\itemsep} % séparateur vertical
    \MakeFramed {\advance\hsize-\width \FrameRestore}
  }%
  {%
    \endMakeFramed
  }
\renewenvironment{quoteblock}% may be used for ornaments
  {%
    \savenotes
    \setstretch{0.9}
    \normalfont
    \begin{quotebar}
  }
  {%
    \end{quotebar}
    \spewnotes
  }


\renewcommand{\headrulewidth}{\arrayrulewidth}
\renewcommand{\headrule}{{\color{rubric}\hrule}}

% delicate tuning, image has produce line-height problems in title on 2 lines
\titleformat{name=\chapter} % command
  [display] % shape
  {\vspace{1.5em}\centering} % format
  {} % label
  {0pt} % separator between n
  {}
[{\color{rubric}\huge\textbf{#1}}\bigskip] % after code
% \titlespacing{command}{left spacing}{before spacing}{after spacing}[right]
\titlespacing*{\chapter}{0pt}{-2em}{0pt}[0pt]

\titleformat{name=\section}
  [block]{}{}{}{}
  [\vbox{\color{rubric}\large\raggedleft\textbf{#1}}]
\titlespacing{\section}{0pt}{0pt plus 4pt minus 2pt}{\baselineskip}

\titleformat{name=\subsection}
  [block]
  {}
  {} % \thesection
  {} % separator \arrayrulewidth
  {}
[\vbox{\large\textbf{#1}}]
% \titlespacing{\subsection}{0pt}{0pt plus 4pt minus 2pt}{\baselineskip}

\ifaiv
  \fancypagestyle{main}{%
    \fancyhf{}
    \setlength{\headheight}{1.5em}
    \fancyhead{} % reset head
    \fancyfoot{} % reset foot
    \fancyhead[L]{\truncate{0.45\headwidth}{\fontrun\elbibl}} % book ref
    \fancyhead[R]{\truncate{0.45\headwidth}{ \fontrun\nouppercase\leftmark}} % Chapter title
    \fancyhead[C]{\thepage}
  }
  \fancypagestyle{plain}{% apply to chapter
    \fancyhf{}% clear all header and footer fields
    \setlength{\headheight}{1.5em}
    \fancyhead[L]{\truncate{0.9\headwidth}{\fontrun\elbibl}}
    \fancyhead[R]{\thepage}
  }
\else
  \fancypagestyle{main}{%
    \fancyhf{}
    \setlength{\headheight}{1.5em}
    \fancyhead{} % reset head
    \fancyfoot{} % reset foot
    \fancyhead[RE]{\truncate{0.9\headwidth}{\fontrun\elbibl}} % book ref
    \fancyhead[LO]{\truncate{0.9\headwidth}{\fontrun\nouppercase\leftmark}} % Chapter title, \nouppercase needed
    \fancyhead[RO,LE]{\thepage}
  }
  \fancypagestyle{plain}{% apply to chapter
    \fancyhf{}% clear all header and footer fields
    \setlength{\headheight}{1.5em}
    \fancyhead[L]{\truncate{0.9\headwidth}{\fontrun\elbibl}}
    \fancyhead[R]{\thepage}
  }
\fi

\ifav % a5 only
  \titleclass{\section}{top}
\fi

\newcommand\chapo{{%
  \vspace*{-3em}
  \centering % no vskip ()
  {\Large\addfontfeature{LetterSpace=25}\bfseries{\elauthor}}\par
  \smallskip
  {\large\eldate}\par
  \bigskip
  {\Large\selectfont{\eltitle}}\par
  \bigskip
  {\color{rubric}\hline\par}
  \bigskip
  {\Large TEXTE LIBRE À PARTICPATION LIBRE\par}
  \centerline{\small\color{rubric} {hurlus.fr, tiré le \today}}\par
  \bigskip
}}

\newcommand\cover{{%
  \thispagestyle{empty}
  \centering
  {\LARGE\bfseries{\elauthor}}\par
  \bigskip
  {\Large\eldate}\par
  \bigskip
  \bigskip
  {\LARGE\selectfont{\eltitle}}\par
  \vfill\null
  {\color{rubric}\setlength{\arrayrulewidth}{2pt}\hline\par}
  \vfill\null
  {\Large TEXTE LIBRE À PARTICPATION LIBRE\par}
  \centerline{{\href{https://hurlus.fr}{\dotuline{hurlus.fr}}, tiré le \today}}\par
}}

\begin{document}
\pagestyle{empty}
\ifbooklet{
  \cover\newpage
  \thispagestyle{empty}\hbox{}\newpage
  \cover\newpage\noindent Les voyages de la brochure\par
  \bigskip
  \begin{tabularx}{\textwidth}{l|X|X}
    \textbf{Date} & \textbf{Lieu}& \textbf{Nom/pseudo} \\ \hline
    \rule{0pt}{25cm} &  &   \\
  \end{tabularx}
  \newpage
  \addtocounter{page}{-4}
}\fi

\thispagestyle{empty}
\ifaiv
  \twocolumn[\chapo]
\else
  \chapo
\fi
{\it\elabstract}
\bigskip
\makeatletter\@starttoc{toc}\makeatother % toc without new page
\bigskip

\pagestyle{main} % after style

   
\begin{quoteblock}
\noindent « Est-il vrai », m’as-tu demandé, « est-il vrai que tes camarades, les ouvriers des villes, pensent à me prendre la terre, cette douce terre que j’aime et qui me donne des épis, bien avarement, il est vrai, mais qui me les donne pourtant ? Elle a nourri mon père et le père de mon père ; et mes enfants y trouveront peut-être un peu de pain. Est-il vrai que tu veux me prendre la terre, me chasser de ma cabane et de mon jardinet ? Mon arpent ne sera-t-il plus à moi ? »\end{quoteblock}

\noindent Non, mon frère, ce n’est pas vrai. Puisque tu aimes le sol et que tu le cultives, c’est bien à toi qu’appartiennent les moissons. C’est toi qui fais naître le pain, nul n’a le droit d’en manger avant toi, avant ta femme qui s’est associée à ton sort, avant l’enfant qui est né de votre union. Garde tes sillons en toute tranquillité, garde ta bêche et ta charrue pour retourner la terre durcie, garde la semence pour féconder le sol. rien n’est plus sacré que ton labeur, et mille fois maudit celui qui voudrait t’enlever le sol devenu   nourricier par tes efforts !\par
Mais ce que je dis à toi, je ne le dis pas à d’autres qui se prétendent cultivateurs et qui ne le sont pas. Quels sont-ils ces soi-disant travailleurs, ces engraisseurs du sol ? L’un est né grand seigneur. Quand on l’a placé dans son berceau, tout enveloppé de laines fines et de soies douces à toucher et à voir, le prêtre, le magistrat, le notaire et d’autres personnages sont venus saluer le nouveau-né comme un futur maître de la terre. Des courtisans, hommes et femmes, sont accourus de toutes parts pour lui apporter des présents, des étoffes brochées d’argent et des hochets d’or ; pendant qu’on le comble de cadeaux, des scribes enregistrent en de grands livres que le poupon possède ici des sources et là des rivières, plus loin des bois, des champs et des prairies, puis ailleurs des jardins et encore d’autres champs, d’autres bois, d’autres pâturages. Il en a dans la montagne, il en a dans la plaine ; même sous la terre il est aussi maître de grands domaines où des hommes travaillent, par centaines ou par milliers. Quand il sera devenu grand, peut-être, un jour, ira-t-il visiter ce dont il hérita au sortir du ventre maternel ; peut-être ne se donnera-t-il pas même la peine de voir toutes ces choses ; mais il en fera recueillir  et vendre les produits. De tous côtés, par routes et par chemins de fer, par barques de rivières et par navires sur l’océan, on lui apportera de grands sacs d’argent, revenus de toutes ses campagnes. Eh bien, quand nous aurons la force, laisserons-nous tous ces produits du labeur humain, les laisserons-nous dans les coffres-forts de l’héritier, aurons-nous le respect de cette propriété ? non, mes amis, nous prendrons tout cela. Nous déchirerons ces papiers et plans, nous briserons les portes de ces châteaux, nous saisirons ces domaines. « Travaille, si tu veux manger ! » dirons-nous à ce prétendu cultivateur ! Rien de toutes ces richesses n’est plus à toi ! »\par
Et cet autre seigneur né pauvre, sans parchemin, que nul flatteur ne vint admirer dans la cabane ou la mansarde maternelle, mais qui eut la chance de s’enrichir par son travail probe ou improbe ? Il n’avait pas une motte de terre où reposer sa tête, mais il a su, par des spéculations ou des économies, par les faveurs des maîtres ou du sort, acquérir d’immenses étendues qu’il enclôt maintenant de murs et de barrières : il récolte où il n’a point semé, il mange et grappille le pain qu’un autre a gagné par son travail. Respecterons-nous cette deuxième propriété, celle de l’enrichi qui ne travaille point  sa terre, mais qui la fait labourer par des mains esclaves et qui la dit sienne ? Non, cette deuxième propriété, nous ne la respecterons pas plus que la première. Ici encore, quand nous en aurons la force, nous viendrons mettre la main sur ces domaines et dire à celui qui s’en croit maître :\par

\begin{quoteblock}
\noindent « En arrière, parvenu ! Puisque tu as su travailler, continue ! Tu auras le pain que te donnera ton labeur, mais la terre que d’autres cultivent n’est plus à toi. Tu n’es plus le maître du pain. »\end{quoteblock}

\noindent Ainsi nous prendrons la terre, oui, nous la prendrons, mais à ceux qui la détiennent sans la travailler, pour la rendre à ceux auxquels il était interdit d’y toucher. Toutefois, ce n’est point pour qu’ils puissent à leur tour exploiter d’autres malheureux. La mesure de la terre à laquelle l’individu, le groupe familial ou la communauté d’amis ont naturellement droit, est embrassée par leur travail individuel ou collectif. Dès qu’un morceau de terre dépasse l’étendue de ce qu’ils peuvent cultiver, ils n’ont aucune raison naturelle de revendiquer ce lambeau ; l’usage en appartient à d’autres travailleurs. La limite se trace diversement entre les cultures des individus ou des groupes, suivant la mise en état de la production. Ce que tu cultives, mon frère, est à toi, et nous t’aiderons  à le garder par tous les moyens en notre pouvoir ; mais ce que tu ne cultives pas est à un compagnon. Fais-lui de la place. Lui aussi saura féconder la terre.\par
Mais si l’un et l’autre vous avez droit à votre part de terre, aurez-vous l’imprudence de rester isolés ? Seul, trop seul, le petit paysan cultivateur est trop faible pour lutter à la fois contre la nature avare et contre l’oppresseur méchant. S’il réussit à vivre, c’est par un prodige de volonté. Il faut qu’il s’accommode à tous les caprices du temps et se soumette en mille occasions à la torture volontaire. Que la gelée fende la pierre, que le soleil brûle, que la pluie tombe ou que le vent hurle, il est toujours à l’œuvre ; que l’inondation noie ses récoltes, que la chaleur les calcine, il moissonne tristement ce qui reste et qui ne suffira guère à le nourrir. Qu’arrive le jour des semailles, il se retirera le grain de la bouche pour le jeter dans le sillon. Dans son désespoir, l’âpre foi lui reste : il sacrifie une partie de la pauvre moisson, si nécessaire, dans la confiance qu’après le rude hiver, après le brûlant été, le blé mûrira pourtant et doublera, triplera la semence, la décuplera peut-être. Quel amour intense il ressent pour cette terre, qui le fait tant peiner par le travail, tant souffrir par la crainte et les déceptions,  tant exulter de joie quand les lignes ondulent à pleins épis. Aucun amour n’est plus fort que celui du paysan pour le sol qu’il défonce et qu’il ensemence, duquel il est né et dans lequel il retournera ! Et pourtant que d’ennemis l’entourent et lui envient la possession de cette terre qu’il adore ! Le percepteur d’impôts taxe sa charrue et lui prend une part de son blé ; le marchand en saisit une autre part ; le chemin de fer le frustre aussi dans le transport de la denrée. De toutes parts, il est trompé. Et nous avons beau lui crier : « Ne paie pas l’impôt, ne paie pas la rente », il paie quand même parce qu’il est seul, parce qu’il n’a pas confiance dans ses voisins, les autres petits paysans, propriétaires ou métayers, et n’ose se concerter avec eux. On les tient asservis, lui et tous les autres, par la peur et la désunion.\par
Il est certain que si tous les paysans d’un même district avaient compris combien l’union peut accroître la force contre l’oppression, ils n’auraient jamais laissé périr les communautés des temps primitifs, les « groupes d’amis », comme on les appelle en Serbie et autres pays slaves. Le propriété collective de ces associations n’est point divisée en d’innombrables enclos par des haies, des murs et des fossés. Les compagnons n’ont point à se  disputer pour savoir si un épi poussé à droite ou à gauche du sillon est bien à eux. Pas d’huissier, pas d’avoué, pas de notaire pour régler les intérêts entre les camarades. Après la récolte, avant l’époque du nouveau labour, ils se réunissent pour discuter les affaires communes. Le jeune homme qui s’est marié, la famille qui s’est accrue d’un enfant ou chez laquelle est entré un hôte, exposent leur situation nouvelle et prennent une plus large part de l’avoir commun pour satisfaire leurs besoins plus grands. On resserre ou l’on éloigne les distances suivant l’étendue du sol et le nombre de membres, et chacun besogne dans son champ, heureux d’être en paix avec les frères qui travaillent à leur côté sur la terre mesurée aux besoins de tous. Dans les circonstances urgentes, les camarades s’entr’aident : un incendie a dévoré telle cabane, tous s’occupent à la reconstruire ; une ravine d’eau a détruit un bout de champ, on en prépare un autre pour le détenteur lésé. Un seul paît les troupeaux de la communauté, et le soir, les brebis, les vaches savent reprendre le chemin de leur étable sans qu’on les y pousse. La commune est à la fois la propriété de tous et de chacun.\par
Oui, mais la commune, de même que l’individu, est bien faible si elle reste dans l’isolement.  Peut-être n’a-t-elle pas assez de terres pour l’ensemble des participants, et tous doivent souffrir de la faim ! Presque toujours elle se trouve en lutte avec un seigneur plus riche qu’elle, qui prétend à la possession de tel ou tel champ, de telle forêt ou de tel terrain de pâture. Elle résiste bien, et si le seigneur était seul, elle aurait bien vite triomphé de l’insolent personnage ; mais le seigneur n’est pas seul, il a pour lui le gouverneur de la province et le chef de la police, pour lui les prêtres et les magistrats, pour lui le gouvernement tout entier avec ses lois et son armée. Au besoin, il dispose du canon pour foudroyer ceux qui lui disputent le sol débattu. Ainsi, la commune pourrait avoir cent fois raison, elle a toutes les chances que les puissants lui donnent tort. Et nous avons beau lui crier, comme à l’imposable isolé : « Ne cède pas ! », elle doit céder, victime de son isolement et de sa faiblesse.\par
Vous êtes donc faibles, vous tous, petits propriétaires, isolés ou associés en communes, vous êtes bien faibles contre tous ceux qui cherchent à vous asservir, accapareurs de terre qui en veulent à votre petit lopin, gouvernants qui cherchent à en prélever tout le produit. Si vous ne savez pas vous unir, non seulement d’individu à individu et de  commune à commune, mais aussi de pays à pays, en une grande internationale de travailleurs, vous partagerez bientôt le sort de millions et de millions d’hommes qui sont déjà dépouillés de tous droits aux semailles et à la récolte et qui vivent dans l’esclavage du salariat, trouvant l’ouvrage quand des patrons ont intérêt à leur en donner, toujours obligés de mendier sous mille formes, tantôt demandant humblement d’être embauchés, tantôt même en avançant la main pour implorer une avare pitance. Ceux-ci ont été privés de la terre, et vous pouvez l’être demain. Y a-t-il une si grande différence entre leur sort et le vôtre ? La menace les atteint déjà ; elle vous épargne encore pour un jour ou deux. Unissez-vous tous dans votre malheur ou votre danger. Défendez ce qui vous reste et reconquérez ce que vous avez perdu.\par
Sinon votre sort à venir est horrible, car nous sommes dans un âge de science et de méthode et nos gouvernants, servis par l’armée des chimistes et des professeurs, vous préparent une organisation sociale dans laquelle tout sera réglé comme dans une usine, où la machine dirigera tout, même les hommes ; où ceux-ci seront de simples rouages que l’on changera comme de vieux fer quand  ils se mêleront de raisonner et de vouloir.\par
C’est ainsi que dans les solitudes du Grand-Ouest Américain, des compagnies de spéculateurs, en fort bons termes avec le gouvernement, comme le sont tous les riches ou ceux qui ont l’espoir de le devenir, se sont fait concéder des domaines immenses dans les régions fertiles et en font à coups d’hommes et de capitaux des usines à céréales. Tel champ de culture a la superficie d’une province. Ce vaste espace est confié à une sorte de général, instruit, expérimenté, bon agriculteur et bon commerçant, habile dans l’art d’évaluer à sa juste valeur la force de rendement des terrains et des muscles. Notre homme s’installe dans une maison commode au centre de sa terre. Il a dans ses hangars cent charrues, cent machines à semer, cent moissonneuses, vingt batteuses ; une cinquantaine de wagons traînés par des locomotives vont et viennent incessamment sur des lignes de rails entre les gares du champ et le port le plus voisin dont les embarcadères et les navires lui appartiennent aussi. Un réseau de téléphones va de la maison palatiale à toutes les constructions du domaine ; la voix du maître est entendue de partout ; il a l’oreille à tous les bruits, le regard à tous les actes ; rien ne se fait sans ses ordres et loin  de sa surveillance.\par
Et que devient l’ouvrier, le paysan dans ce monde si bien organisé ? Machines, chevaux et hommes sont utilisés de la même manière : on voit en eux autant de forces, évaluées en chiffres, qu’il faut employer au mieux du bénéfice patronal, avec le plus de produit et le moins de dépenses possible. Les écuries sont disposées de telle sorte qu’au sortir même de l’édifice, les animaux commencent à creuser le sillon de plusieurs kilomètres de long qu’ils ont à tracer jusqu’au bout du champ : chacun de leurs pas est calculé, chacun rapporte au maître. De même les mouvements des ouvriers sont réglés à l’issue du dortoir commun. Là, point de femmes ni d’enfants qui viennent troubler la besogne par une caresse ou par un baiser. Les travailleurs sont groupés par escouades ayant leurs sergents, leurs capitaines et l’inévitable mouchard. Le devoir est de faire méthodiquement le travail commandé, d’observer le silence dans les rangs. Qu’une machine se détraque, on la jette au rebut, s’il n’est pas possible de la réparer. Qu’un cheval tombe et se casse un membre, on lui tire un coup de revolver dans l’oreille et on le traîne au charnier. Qu’un homme succombe à la peine, qu’il se brise un membre ou se laisse envahir par la fièvre, on daigne bien  ne pas l’achever, mais on s’en débarrasse tout de même : qu’il meure à l’écart sans fatiguer personne de ses plaintes. À la fin des grands travaux, quand la nature se repose, le directeur se repose aussi et licencie son armée. L’année suivante, il trouvera toujours une quantité suffisante d’os et de muscles à embaucher, mais il se gardera bien d’employer les mêmes travailleurs que l’année précédente. Ils pourraient parler de leur expérience, s’imaginer qu’il en savent autant que le maître, obéir de mauvaise grâce, qui sait ? S’attacher peut-être à la terre cultivée par eux et se figurer qu’elle leur appartient !\par
Certes, si le bonheur de l’humanité consistait à créer quelques milliardaires thésaurisant au profit de leurs passions et de leurs caprices les produits entassés par tous les travailleurs asservis, cette exploitation scientifique de la terre par une chiourme de galériens serait l’idéal rêvé. Prodigieux sont les résultats financiers de ces entreprises, quand la spéculation ne ruine pas ce que la spéculation crée. Telle quantité de blé obtenue par le travail de cinq cents hommes pourrait en nourrir cinquante mille ; à la dépense faite par un salaire avare correspond un rendement énorme de denrées qu’on expédie par chargement de navires et qui se vendent dix fois la valeur de production.  Il est vrai que si la masse des consommateurs manquant d’ouvrage et de salaire devient trop pauvre, elle ne pourra plus acheter tous ces produits et, condamnée à mourir de faim, elle n’enrichira plus les spéculateurs. Mais ceux-ci ne s’occupent point du lointain avenir : gagner d’abord, marcher sur un chemin pavé d’argent, et l’on verra plus tard ; les enfants se débrouilleront ! « Après nous le déluge ! »\par
Voilà, camarades travailleurs qui aimez le sillon où vous avez vu pour la première fois le mystère de la tigelle de froment perçant la dure motte de terre, voilà quelle destinée l’on vous prépare ! On vous prendra le champ et la récolte, on vous prendra vous-mêmes, on vous attachera à quelque machine de fer, fumante et stridente, et tout enveloppés de la fumée de charbon, vous aurez à balancer vos bras sur un levier dix ou douze mille fois par jour. C’est là ce qu’on appelle l’agriculture. Et ne vous attardez pas alors à faire l’amour quand le cœur vous dira de prendre femme ; ne tournez pas la tête vers la jeune fille qui passe : le contremaître n’entend pas qu’on fraude le travail du patron.\par
S’il convient à celui-ci de vous permettre le mariage pour créer progéniture, c’est qu’il vous trouvera bien à son gré ; vous aurez cette âme d’esclave qu’il aura voulu façonner ; vous  serez assez vil pour qu’il autorise la race d’abjection à se perpétuer. L’avenir qui vous attend est celui de l’ouvrier, de l’ouvrière, de l’enfant d’usine ! Jamais esclavage antique n’a plus méthodiquement pétri et façonné la matière humaine pour la réduire à l’état d’outil. Que reste-t-il d’humain dans l’être hâve, déjeté, scrofuleux qui ne respire jamais d’autre atmosphère que celle des suints, des graisses et des poussières ?\par
Evitez cette mort à tout prix, camarades. Gardez jalousement votre terre, vous qui en avez un lopin ; elle est votre vie et celle de la femme, des enfants que vous aimez. Associez-vous aux compagnons dont la terre est menacée comme la vôtre par les usiniers, les amateurs de chasse, les prêteurs d’argent ; oubliez toutes vos petites rancunes de voisin à voisin, et groupez-vous en communes où tous les intérêts soient solidaires, où chaque motte de gazon ait tous les communiers pour défenseurs. À cent, à mille, à dix mille, vous serez déjà bien forts contre le seigneur et ses valets ; mais vous ne serez pas encore assez forts contre une armée. Associez-vous donc de commune à commune et que la plus faible dispose de la force de toutes. Bien plus, faites appel à ceux qui n’ont rien, à ces gens deshérités des villes  qu’on vous a peut-être appris à haïr, mais qu’il faut aimer parce qu’ils vous aideront à garder la terre et à reconquérir celle qu’on vous a prise. Avec eux, vous attaquerez, vous renverserez les murailles d’enclos ; avec eux, vous fonderez la grande commune des hommes, où l’on travaillera de concert à vivifier le sol, à l’embellir et à vivre heureux, sur cette bonne terre qui nous donne le pain.\par
Mais si vous ne faites pas cela, tout est perdu. Vous périrez esclaves et mendiants : « Vous avez faim », disait récemment un maire d’Alger à une députation d’humbles sans-travail, « vous avez faim ?... eh bien, mangez-vous les uns les autres ! »\par

\byline{Elisée RECLUS}
 


% at least one empty page at end (for booklet couv)
\ifbooklet
  \pagestyle{empty}
  \clearpage
  % 2 empty pages maybe needed for 4e cover
  \ifnum\modulo{\value{page}}{4}=0 \hbox{}\newpage\hbox{}\newpage\fi
  \ifnum\modulo{\value{page}}{4}=1 \hbox{}\newpage\hbox{}\newpage\fi


  \hbox{}\newpage
  \ifodd\value{page}\hbox{}\newpage\fi
  {\centering\color{rubric}\bfseries\noindent\large
    Hurlus ? Qu’est-ce.\par
    \bigskip
  }
  \noindent Des bouquinistes électroniques, pour du texte libre à participation libre,
  téléchargeable gratuitement sur \href{https://hurlus.fr}{\dotuline{hurlus.fr}}.\par
  \bigskip
  \noindent Cette brochure a été produite par des éditeurs bénévoles.
  Elle n’est pas faîte pour être possédée, mais pour être lue, et puis donnée.
  Que circule le texte !
  En page de garde, on peut ajouter une date, un lieu, un nom ; pour suivre le voyage des idées.
  \par

  Ce texte a été choisi parce qu’une personne l’a aimé,
  ou haï, elle a en tous cas pensé qu’il partipait à la formation de notre présent ;
  sans le souci de plaire, vendre, ou militer pour une cause.
  \par

  L’édition électronique est soigneuse, tant sur la technique
  que sur l’établissement du texte ; mais sans aucune prétention scolaire, au contraire.
  Le but est de s’adresser à tous, sans distinction de science ou de diplôme.
  Au plus direct ! (possible)
  \par

  Cet exemplaire en papier a été tiré sur une imprimante personnelle
   ou une photocopieuse. Tout le monde peut le faire.
  Il suffit de
  télécharger un fichier sur \href{https://hurlus.fr}{\dotuline{hurlus.fr}},
  d’imprimer, et agrafer ; puis de lire et donner.\par

  \bigskip

  \noindent PS : Les hurlus furent aussi des rebelles protestants qui cassaient les statues dans les églises catholiques. En 1566 démarra la révolte des gueux dans le pays de Lille. L’insurrection enflamma la région jusqu’à Anvers où les gueux de mer bloquèrent les bateaux espagnols.
  Ce fut une rare guerre de libération dont naquit un pays toujours libre : les Pays-Bas.
  En plat pays francophone, par contre, restèrent des bandes de huguenots, les hurlus, progressivement réprimés par la très catholique Espagne.
  Cette mémoire d’une défaite est éteinte, rallumons-la. Sortons les livres du culte universitaire, cherchons les idoles de l’époque, pour les briser.
\fi

\ifdev % autotext in dev mode
\fontname\font — \textsc{Les règles du jeu}\par
(\hyperref[utopie]{\underline{Lien}})\par
\noindent \initialiv{A}{lors là}\blindtext\par
\noindent \initialiv{À}{ la bonheur des dames}\blindtext\par
\noindent \initialiv{É}{tonnez-le}\blindtext\par
\noindent \initialiv{Q}{ualitativement}\blindtext\par
\noindent \initialiv{V}{aloriser}\blindtext\par
\Blindtext
\phantomsection
\label{utopie}
\Blinddocument
\fi
\end{document}
