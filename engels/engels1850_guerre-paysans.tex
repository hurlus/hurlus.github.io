%%%%%%%%%%%%%%%%%%%%%%%%%%%%%%%%%
% LaTeX model https://hurlus.fr %
%%%%%%%%%%%%%%%%%%%%%%%%%%%%%%%%%

% Needed before document class
\RequirePackage{pdftexcmds} % needed for tests expressions
\RequirePackage{fix-cm} % correct units

% Define mode
\def\mode{a4}

\newif\ifaiv % a4
\newif\ifav % a5
\newif\ifbooklet % booklet
\newif\ifcover % cover for booklet

\ifnum \strcmp{\mode}{cover}=0
  \covertrue
\else\ifnum \strcmp{\mode}{booklet}=0
  \booklettrue
\else\ifnum \strcmp{\mode}{a5}=0
  \avtrue
\else
  \aivtrue
\fi\fi\fi

\ifbooklet % do not enclose with {}
  \documentclass[french,twoside]{book} % ,notitlepage
  \usepackage[%
    papersize={105mm, 297mm},
    inner=12mm,
    outer=12mm,
    top=15mm,
    bottom=15mm,
    marginparsep=0pt,
  ]{geometry}
  \usepackage[fontsize=9.5pt]{scrextend} % for Roboto
\else\ifav
  \documentclass[french,twoside]{book} % ,notitlepage
  \usepackage[%
    a5paper,
    inner=25mm,
    outer=15mm,
    top=15mm,
    bottom=15mm,
    marginparsep=0pt,
  ]{geometry}
  \usepackage[fontsize=12pt]{scrextend}
\else% A4 2 cols
  \documentclass[twocolumn]{book}
  \usepackage[%
    a4paper,
    inner=15mm,
    outer=10mm,
    top=25mm,
    bottom=18mm,
    marginparsep=0pt,
  ]{geometry}
  \setlength{\columnsep}{20mm}
  \usepackage[fontsize=9.5pt]{scrextend}
\fi\fi

%%%%%%%%%%%%%%
% Alignments %
%%%%%%%%%%%%%%
% before teinte macros

\setlength{\arrayrulewidth}{0.2pt}
\setlength{\columnseprule}{\arrayrulewidth} % twocol
\setlength{\parskip}{0pt} % classical para with no margin
\setlength{\parindent}{1.5em}

%%%%%%%%%%
% Colors %
%%%%%%%%%%
% before Teinte macros

\usepackage[dvipsnames]{xcolor}
\definecolor{rubric}{HTML}{902c20} % the tonic
\def\columnseprulecolor{\color{rubric}}
\colorlet{borderline}{rubric!30!} % definecolor need exact code
\definecolor{shadecolor}{gray}{0.95}
\definecolor{bghi}{gray}{0.5}

%%%%%%%%%%%%%%%%%
% Teinte macros %
%%%%%%%%%%%%%%%%%
%%%%%%%%%%%%%%%%%%%%%%%%%%%%%%%%%%%%%%%%%%%%%%%%%%%
% <TEI> generic (LaTeX names generated by Teinte) %
%%%%%%%%%%%%%%%%%%%%%%%%%%%%%%%%%%%%%%%%%%%%%%%%%%%
% This template is inserted in a specific design
% It is XeLaTeX and otf fonts

\makeatletter % <@@@


\usepackage{blindtext} % generate text for testing
\usepackage{contour} % rounding words
\usepackage[nodayofweek]{datetime}
\usepackage{DejaVuSans} % font for symbols
\usepackage{enumitem} % <list>
\usepackage{etoolbox} % patch commands
\usepackage{fancyvrb}
\usepackage{fancyhdr}
\usepackage{fontspec} % XeLaTeX mandatory for fonts
\usepackage{footnote} % used to capture notes in minipage (ex: quote)
\usepackage{framed} % bordering correct with footnote hack
\usepackage{graphicx}
\usepackage{lettrine} % drop caps
\usepackage{lipsum} % generate text for testing
\usepackage[framemethod=tikz,]{mdframed} % maybe used for frame with footnotes inside
\usepackage{pdftexcmds} % needed for tests expressions
\usepackage{polyglossia} % non-break space french punct, bug Warning: "Failed to patch part"
\usepackage[%
  indentfirst=false,
  vskip=1em,
  noorphanfirst=true,
  noorphanafter=true,
  leftmargin=\parindent,
  rightmargin=0pt,
]{quoting}
\usepackage{ragged2e}
\usepackage{setspace}
\usepackage{tabularx} % <table>
\usepackage[explicit]{titlesec} % wear titles, !NO implicit
\usepackage{tikz} % ornaments
\usepackage{tocloft} % styling tocs
\usepackage[fit]{truncate} % used im runing titles
\usepackage{unicode-math}
\usepackage[normalem]{ulem} % breakable \uline, normalem is absolutely necessary to keep \emph
\usepackage{verse} % <l>
\usepackage{xcolor} % named colors
\usepackage{xparse} % @ifundefined
\XeTeXdefaultencoding "iso-8859-1" % bad encoding of xstring
\usepackage{xstring} % string tests
\XeTeXdefaultencoding "utf-8"
\PassOptionsToPackage{hyphens}{url} % before hyperref, which load url package
\usepackage{hyperref} % supposed to be the last one, :o) except for the ones to follow
\urlstyle{same} % after hyperref

% TOTEST
% \usepackage{hypcap} % links in caption ?
% \usepackage{marginnote}
% TESTED
% \usepackage{background} % doesn’t work with xetek
% \usepackage{bookmark} % prefers the hyperref hack \phantomsection
% \usepackage[color, leftbars]{changebar} % 2 cols doc, impossible to keep bar left
% \usepackage[utf8x]{inputenc} % inputenc package ignored with utf8 based engines
% \usepackage[sfdefault,medium]{inter} % no small caps
% \usepackage{firamath} % unavailable in Ubuntu 21-04
% \usepackage{flushend} % bad for last notes, supposed flush end of columns
% \usepackage[stable]{footmisc} % BAD for complex notes https://texfaq.org/FAQ-ftnsect
% \usepackage{helvet} % not for XeLaTeX
% \usepackage{multicol} % not compatible with too much packages (longtable, framed, memoir…)
% \usepackage{sectsty} % \chapterfont OBSOLETE
% \usepackage{soul} % \ul for underline, OBSOLETE with XeTeX
% \usepackage[breakable]{tcolorbox} % text styling gone, footnote hack not kept with breakable



% Metadata inserted by a program, from the TEI source, for title page and runing heads
\title{\textbf{ La guerre des paysans en Allemagne }\\ \medskip
\textit{ traduction 1977, Bottigelli }}
\date{1850}
\author{Engels, Friedrich}
\def\elbibl{Engels, Friedrich. 1850. \emph{La guerre des paysans en Allemagne}}
\def\elsource{\href{https://www.marxists.org/francais/marx/works/1850/00/fe1850h.htm}{\dotuline{marxists.org}}\footnote{\href{https://www.marxists.org/francais/marx/works/1850/00/fe1850h.htm}{\url{https://www.marxists.org/francais/marx/works/1850/00/fe1850h.htm}}}}

% Default metas
\newcommand{\colorprovide}[2]{\@ifundefinedcolor{#1}{\colorlet{#1}{#2}}{}}
\colorprovide{rubric}{red}
\colorprovide{silver}{Gray}
\@ifundefined{syms}{\newfontfamily\syms{DejaVu Sans}}{}
\newif\ifdev
\@ifundefined{elbibl}{% No meta defined, maybe dev mode
  \newcommand{\elbibl}{Titre court ?}
  \newcommand{\elbook}{Titre du livre source ?}
  \newcommand{\elabstract}{Résumé\par}
  \newcommand{\elurl}{http://oeuvres.github.io/elbook/2}
  \author{Éric Lœchien}
  \title{Un titre de test assez long pour vérifier le comportement d’une maquette}
  \date{1566}
  \devtrue
}{}
\let\eltitle\@title
\let\elauthor\@author
\let\eldate\@date


\defaultfontfeatures{
  % Mapping=tex-text, % no effect seen
  Scale=MatchLowercase,
  Ligatures={TeX,Common},
}

\@ifundefined{\columnseprulecolor}{%
    \patchcmd\@outputdblcol{% find
      \normalcolor\vrule
    }{% and replace by
      \columnseprulecolor\vrule
    }{% success
    }{% failure
      \@latex@warning{Patching \string\@outputdblcol\space failed}%
    }
}{}

\hypersetup{
  % pdftex, % no effect
  pdftitle={\elbibl},
  % pdfauthor={Your name here},
  % pdfsubject={Your subject here},
  % pdfkeywords={keyword1, keyword2},
  bookmarksnumbered=true,
  bookmarksopen=true,
  bookmarksopenlevel=1,
  pdfstartview=Fit,
  breaklinks=true, % avoid long links
  pdfpagemode=UseOutlines,    % pdf toc
  hyperfootnotes=true,
  colorlinks=false,
  pdfborder=0 0 0,
  % pdfpagelayout=TwoPageRight,
  % linktocpage=true, % NO, toc, link only on page no
}


% generic typo commands
\newcommand{\astermono}{\medskip\centerline{\color{rubric}\large\selectfont{\syms ✻}}\medskip\par}%
\newcommand{\astertri}{\medskip\par\centerline{\color{rubric}\large\selectfont{\syms ✻\,✻\,✻}}\medskip\par}%
\newcommand{\asterism}{\bigskip\par\noindent\parbox{\linewidth}{\centering\color{rubric}\large{\syms ✻}\\{\syms ✻}\hskip 0.75em{\syms ✻}}\bigskip\par}%

% lists
\newlength{\listmod}
\setlength{\listmod}{\parindent}
\setlist{
  itemindent=!,
  listparindent=\listmod,
  labelsep=0.2\listmod,
  parsep=0pt,
  % topsep=0.2em, % default topsep is best
}
\setlist[itemize]{
  label=—,
  leftmargin=0pt,
  labelindent=1.2em,
  labelwidth=0pt,
}
\setlist[enumerate]{
  label={\bf\color{rubric}\arabic*.},
  labelindent=0.8\listmod,
  leftmargin=\listmod,
  labelwidth=0pt,
}
\newlist{listalpha}{enumerate}{1}
\setlist[listalpha]{
  label={\bf\color{rubric}\alph*.},
  leftmargin=0pt,
  labelindent=0.8\listmod,
  labelwidth=0pt,
}
\newcommand{\listhead}[1]{\hspace{-1\listmod}\emph{#1}}

\renewcommand{\hrulefill}{%
  \leavevmode\leaders\hrule height 0.2pt\hfill\kern\z@}

% General typo
\DeclareTextFontCommand{\textlarge}{\large}
\DeclareTextFontCommand{\textsmall}{\small}


% commands, inlines
\newcommand{\anchor}[1]{\Hy@raisedlink{\hypertarget{#1}{}}} % link to top of an anchor (not baseline)
\newcommand\abbr{}
\newcommand{\autour}[1]{\tikz[baseline=(X.base)]\node [draw=rubric,thin,rectangle,inner sep=1.5pt, rounded corners=3pt] (X) {#1};}
\newcommand\corr{}
\newcommand{\ed}[1]{ {\color{silver}\sffamily\footnotesize (#1)} } % <milestone ed="1688"/>
\newcommand\expan{}
\newcommand\gap{}
\renewcommand{\LettrineFontHook}{\color{rubric}}
\newcommand{\initial}[2]{\lettrine[lines=2, loversize=0.3, lhang=0.3]{#1}{#2}}
\newcommand{\initialiv}[2]{%
  \let\oldLFH\LettrineFontHook
  % \renewcommand{\LettrineFontHook}{\color{rubric}\ttfamily}
  \IfSubStr{Q}{#1}{
    \lettrine[lines=4, lhang=0.2, loversize=-0.1, lraise=0.2]{\smash{#1}}{#2}
  }{\IfSubStr{É}{#1}{
    \lettrine[lines=4, lhang=0.2, loversize=-0, lraise=0]{\smash{#1}}{#2}
  }{\IfSubStr{ÀÂ}{#1}{
    \lettrine[lines=4, lhang=0.2, loversize=-0, lraise=0, slope=0.6em]{\smash{#1}}{#2}
  }{\IfSubStr{A}{#1}{
    \lettrine[lines=4, lhang=0.2, loversize=0.2, slope=0.6em]{\smash{#1}}{#2}
  }{\IfSubStr{V}{#1}{
    \lettrine[lines=4, lhang=0.2, loversize=0.2, slope=-0.5em]{\smash{#1}}{#2}
  }{
    \lettrine[lines=4, lhang=0.2, loversize=0.2]{\smash{#1}}{#2}
  }}}}}
  \let\LettrineFontHook\oldLFH
}
\newcommand{\labelchar}[1]{\textbf{\color{rubric} #1}}
\newcommand{\milestone}[1]{\autour{\footnotesize\color{rubric} #1}} % <milestone n="4"/>
\newcommand\name{}
\newcommand\orig{}
\newcommand\orgName{}
\newcommand\persName{}
\newcommand\placeName{}
\newcommand{\pn}[1]{{\sffamily\textbf{#1.}} } % <p n="3"/>
\newcommand\reg{}
% \newcommand\ref{} % already defined
\newcommand\sic{}
\def\mednobreak{\ifdim\lastskip<\medskipamount
  \removelastskip\nopagebreak\medskip\fi}
\def\bignobreak{\ifdim\lastskip<\bigskipamount
  \removelastskip\nopagebreak\bigskip\fi}

% commands, blocks
\newcommand{\byline}[1]{\bigskip{\RaggedLeft{#1}\par}\bigskip}
\newcommand{\bibl}[1]{{\RaggedLeft{#1}\par\bigskip}}
\newcommand{\biblitem}[1]{{\noindent\hangindent=\parindent   #1\par}}
\newcommand{\dateline}[1]{\medskip{\RaggedLeft{#1}\par}\bigskip}
\newcommand{\labelblock}[1]{\bigbreak{\color{rubric}\noindent\textbf{#1}\par}\bignobreak}
\newcommand{\salute}[1]{\bigbreak{#1}\par\medbreak}
\newcommand{\signed}[1]{\bigbreak\filbreak{\raggedleft #1\par}\medskip}

% environments for blocks (some may become commands)
\newenvironment{borderbox}{}{} % framing content
\newenvironment{citbibl}{\ifvmode\hfill\fi}{\ifvmode\par\fi }
\newenvironment{docAuthor}{\ifvmode\vskip4pt\fontsize{16pt}{18pt}\selectfont\fi\itshape}{\ifvmode\par\fi }
\newenvironment{docDate}{}{\ifvmode\par\fi }
\newenvironment{docImprint}{\vskip6pt}{\ifvmode\par\fi }
\newenvironment{docTitle}{\vskip6pt\bfseries\fontsize{18pt}{22pt}\selectfont}{\par }
\newenvironment{msHead}{\vskip6pt}{\par}
\newenvironment{msItem}{\vskip6pt}{\par}
\newenvironment{titlePart}{}{\par }


% environments for block containers
\newenvironment{argument}{\fontlight\parindent0pt}{\vskip1.5em}
\newenvironment{biblfree}{}{\ifvmode\par\fi }
\newenvironment{bibitemlist}[1]{%
  \list{\@biblabel{\@arabic\c@enumiv}}%
  {%
    \settowidth\labelwidth{\@biblabel{#1}}%
    \leftmargin\labelwidth
    \advance\leftmargin\labelsep
    \@openbib@code
    \usecounter{enumiv}%
    \let\p@enumiv\@empty
    \renewcommand\theenumiv{\@arabic\c@enumiv}%
  }
  \sloppy
  \clubpenalty4000
  \@clubpenalty \clubpenalty
  \widowpenalty4000%
  \sfcode`\.\@m
}%
{\def\@noitemerr
  {\@latex@warning{Empty `bibitemlist' environment}}%
\endlist}
\newenvironment{quoteblock}% may be used for ornaments
  {\begin{quoting}}
  {\end{quoting}}

% table () is preceded and finished by custom command
\newcommand{\tableopen}[1]{%
  \ifnum\strcmp{#1}{wide}=0{%
    \begin{center}
  }
  \else\ifnum\strcmp{#1}{long}=0{%
    \begin{center}
  }
  \else{%
    \begin{center}
  }
  \fi\fi
}
\newcommand{\tableclose}[1]{%
  \ifnum\strcmp{#1}{wide}=0{%
    \end{center}
  }
  \else\ifnum\strcmp{#1}{long}=0{%
    \end{center}
  }
  \else{%
    \end{center}
  }
  \fi\fi
}


% text structure
\newcommand\chapteropen{} % before chapter title
\newcommand\chaptercont{} % after title, argument, epigraph…
\newcommand\chapterclose{} % maybe useful for multicol settings
\setcounter{secnumdepth}{-2} % no counters for hierarchy titles
\setcounter{tocdepth}{5} % deep toc
\markright{\@title} % ???
\markboth{\@title}{\@author} % ???
\renewcommand\tableofcontents{\@starttoc{toc}}
% toclof format
% \renewcommand{\@tocrmarg}{0.1em} % Useless command?
% \renewcommand{\@pnumwidth}{0.5em} % {1.75em}
\renewcommand{\@cftmaketoctitle}{}
\setlength{\cftbeforesecskip}{\z@ \@plus.2\p@}
\renewcommand{\cftchapfont}{}
\renewcommand{\cftchapdotsep}{\cftdotsep}
\renewcommand{\cftchapleader}{\normalfont\cftdotfill{\cftchapdotsep}}
\renewcommand{\cftchappagefont}{\bfseries}
\setlength{\cftbeforechapskip}{0em \@plus\p@}
% \renewcommand{\cftsecfont}{\small\relax}
\renewcommand{\cftsecpagefont}{\normalfont}
% \renewcommand{\cftsubsecfont}{\small\relax}
\renewcommand{\cftsecdotsep}{\cftdotsep}
\renewcommand{\cftsecpagefont}{\normalfont}
\renewcommand{\cftsecleader}{\normalfont\cftdotfill{\cftsecdotsep}}
\setlength{\cftsecindent}{1em}
\setlength{\cftsubsecindent}{2em}
\setlength{\cftsubsubsecindent}{3em}
\setlength{\cftchapnumwidth}{1em}
\setlength{\cftsecnumwidth}{1em}
\setlength{\cftsubsecnumwidth}{1em}
\setlength{\cftsubsubsecnumwidth}{1em}

% footnotes
\newif\ifheading
\newcommand*{\fnmarkscale}{\ifheading 0.70 \else 1 \fi}
\renewcommand\footnoterule{\vspace*{0.3cm}\hrule height \arrayrulewidth width 3cm \vspace*{0.3cm}}
\setlength\footnotesep{1.5\footnotesep} % footnote separator
\renewcommand\@makefntext[1]{\parindent 1.5em \noindent \hb@xt@1.8em{\hss{\normalfont\@thefnmark . }}#1} % no superscipt in foot


% orphans and widows
\clubpenalty=9996
\widowpenalty=9999
\brokenpenalty=4991
\predisplaypenalty=10000
\postdisplaypenalty=1549
\displaywidowpenalty=1602
\hyphenpenalty=400
% Copied from Rahtz but not understood
\def\@pnumwidth{1.55em}
\def\@tocrmarg {2.55em}
\def\@dotsep{4.5}
\emergencystretch 3em
\hbadness=4000
\pretolerance=750
\tolerance=2000
\vbadness=4000
\def\Gin@extensions{.pdf,.png,.jpg,.mps,.tif}
% \renewcommand{\@cite}[1]{#1} % biblio

\makeatother % /@@@>
%%%%%%%%%%%%%%
% </TEI> end %
%%%%%%%%%%%%%%


%%%%%%%%%%%%%
% footnotes %
%%%%%%%%%%%%%
\renewcommand{\thefootnote}{\bfseries\textcolor{rubric}{\arabic{footnote}}} % color for footnote marks

%%%%%%%%%
% Fonts %
%%%%%%%%%
\usepackage[]{roboto} % SmallCaps, Regular is a bit bold
% \linespread{0.90} % too compact, keep font natural
\newfontfamily\fontrun[]{Roboto Condensed Light} % condensed runing heads
\ifav
  \setmainfont[
    ItalicFont={Roboto Light Italic},
  ]{Roboto}
\else\ifbooklet
  \setmainfont[
    ItalicFont={Roboto Light Italic},
  ]{Roboto}
\else
  \setmainfont[
    ItalicFont={Roboto Italic},
    % BoldFont={Roboto},
  ]{Roboto Light}
\fi\fi
\renewcommand{\LettrineFontHook}{\bfseries\color{rubric}}
% \renewenvironment{labelblock}{\begin{center}\bfseries\color{rubric}}{\end{center}}

%%%%%%%%
% MISC %
%%%%%%%%

\setdefaultlanguage[frenchpart=false]{french} % bug on part


\newenvironment{quotebar}{%
    \def\FrameCommand{{\color{rubric!10!}\vrule width 0.5em} \hspace{0.9em}}%
    \def\OuterFrameSep{\itemsep} % séparateur vertical
    \MakeFramed {\advance\hsize-\width \FrameRestore}
  }%
  {%
    \endMakeFramed
  }
\renewenvironment{quoteblock}% may be used for ornaments
  {%
    \savenotes
    \setstretch{0.9}
    \begin{quotebar}
  }
  {%
    \end{quotebar}
    \spewnotes
  }


\renewcommand{\pn}[1]{{\footnotesize\autour{\color{rubric} #1}}} % <p n="3"/>
\renewcommand{\headrulewidth}{\arrayrulewidth}
\renewcommand{\headrule}{\color{rubric}\hrule}

% delicate tuning, image has produce line-height problems in title on 2 lines
\titleformat{name=\chapter} % command
  [display] % shape
  {\vspace{1.5em}\centering} % format
  {} % label
  {0pt} % separator between n
  {}
[{\color{rubric}\huge\textbf{#1}}\bigskip] % after code
% \titlespacing{command}{left spacing}{before spacing}{after spacing}[right]
\titlespacing*{\chapter}{0pt}{-2em}{0pt}[0pt]

\titleformat{name=\section}
  [block]{}{}{}{}
  [\vbox{\color{rubric}\large\raggedleft\textbf{#1}}]
\titlespacing{\section}{0pt}{0pt plus 4pt minus 2pt}{\baselineskip}

\titleformat{name=\subsection}
  [block]
  {}
  {} % \thesection
  {} % separator \arrayrulewidth
  {}
[\vbox{\large\textbf{#1}}]
% \titlespacing{\subsection}{0pt}{0pt plus 4pt minus 2pt}{\baselineskip}

\ifaiv
  \fancypagestyle{main}{%
    \fancyhf{}
    \setlength{\headheight}{1.5em}
    \fancyhead{} % reset head
    \fancyfoot{} % reset foot
    \fancyhead[L]{\truncate{0.45\headwidth}{\fontrun\elbibl}} % book ref
    \fancyhead[R]{\truncate{0.45\headwidth}{ \fontrun\nouppercase\leftmark}} % Chapter title
    \fancyhead[C]{\thepage}
  }
  \fancypagestyle{plain}{% apply to chapter
    \fancyhf{}% clear all header and footer fields
    \setlength{\headheight}{1.5em}
    \fancyhead[L]{\truncate{0.9\headwidth}{\fontrun\elbibl}}
    \fancyhead[R]{\thepage}
  }
\else
  \fancypagestyle{main}{%
    \fancyhf{}
    \setlength{\headheight}{1.5em}
    \fancyhead{} % reset head
    \fancyfoot{} % reset foot
    \fancyhead[RE]{\truncate{0.9\headwidth}{\fontrun\elbibl}} % book ref
    \fancyhead[LO]{\truncate{0.9\headwidth}{\fontrun\nouppercase\leftmark}} % Chapter title, \nouppercase needed
    \fancyhead[RO,LE]{\thepage}
  }
  \fancypagestyle{plain}{% apply to chapter
    \fancyhf{}% clear all header and footer fields
    \setlength{\headheight}{1.5em}
    \fancyhead[L]{\truncate{0.9\headwidth}{\fontrun\elbibl}}
    \fancyhead[R]{\thepage}
  }
\fi

\ifav % a5 only
  \titleclass{\section}{top}
\fi

\newcommand\chapo{{%
  \vspace*{-3em}
  \centering % no vskip ()
  {\Large\addfontfeature{LetterSpace=25}\bfseries{\elauthor}}\par
  \smallskip
  {\large\eldate}\par
  \bigskip
  {\Large\selectfont{\eltitle}}\par
  \bigskip
  {\color{rubric}\hline\par}
  \bigskip
  {\Large LIVRE LIBRE À PRIX LIBRE, DEMANDEZ AU COMPTOIR\par}
  \centerline{\small\color{rubric} {hurlus.fr, tiré le \today}}\par
  \bigskip
}}


\begin{document}
\pagestyle{empty}
\ifbooklet{
  \thispagestyle{empty}
  \centering
  {\LARGE\bfseries{\elauthor}}\par
  \bigskip
  {\Large\eldate}\par
  \bigskip
  \bigskip
  {\LARGE\selectfont{\eltitle}}\par
  \vfill\null
  {\color{rubric}\setlength{\arrayrulewidth}{2pt}\hline\par}
  \vfill\null
  {\Large LIVRE LIBRE À PRIX LIBRE, DEMANDEZ AU COMPTOIR\par}
  \centerline{\small{hurlus.fr, tiré le \today}}\par
  \newpage\null\thispagestyle{empty}\newpage
  \addtocounter{page}{-2}
}\fi

\thispagestyle{empty}
\ifaiv
  \twocolumn[\chapo]
\else
  \chapo
\fi
\elabstract
\bigskip
\makeatletter\@starttoc{toc}\makeatother % toc without new page
\bigskip

\pagestyle{main} % after style

  
\chapteropen
\renewcommand{\leftmark}{Note préliminaire (1870 puis 1874)}
\chapter[Note préliminaire (1870 puis 1874)]{Note préliminaire (1870 puis 1874)}

\chaptercont
\section[1)]{1)}
\noindent Le présent ouvrage a été écrit à Londres pendant l’été de 1850, sous l’impression directe de la contre-révolution qui venait à peine de s’achever ; il parut dans les numéros 5 et 6 de la \emph{Neue Rheinische Zeitung Politisch-ökonomische} Revue dirigée par Karl Marx, Hambourg 1850. Mes amis politiques en Allemagne désirent le réimprimer et j’acquiesce à leur demande, car il est, aujourd’hui encore, malheureusement d’actualité.\par
Ce travail ne prétend pas fournir une documentation résultant d’une recherche personnelle ; au contraire, tous les matériaux relatifs aux soulèvements paysans et à Thomas Münzer ont été empruntés à Zimmermann. Son livre, quoique présentant ici et là des lacunes, reste encore le meilleur recueil des faits. Le vieux Zimmermann aimait d’ailleurs vivement son sujet. Ce même instinct révolutionnaire qui se manifeste partout ici en faveur de la classe opprimée fit de lui un des meilleurs représentants de l’extrême-gauche à Francfort. Depuis, il doit certes avoir un peu vieilli.\par
Si par contre dans l’exposé de Zimmermann l’enchaînement interne fait défaut, s’il n’arrive pas à présenter les controverses religieuses et politiques de l’époque comme le reflet des luttes de classes contemporaines, s’il ne voit dans ces luttes que des oppresseurs et des opprimés, des méchants et des bons, et finalement le triomphe des méchants, si sa compréhension des rapports sociaux qui déterminèrent aussi bien l’explosion que l’issue de la lutte est tout à fait déficiente, la faute en est à l’époque où ce livre parut. On peut même dire que, pour son temps, ce livre est encore très réaliste et constitue une louable exception parmi les ouvrages des historiens idéalistes allemands.\par
Mon exposé cherchait, en n’esquissant le cours historique de la lutte que dans ses grandes lignes, à expliquer l’origine de la guerre des paysans, la position prise par les divers partis qui y participèrent, les théories politiques et religieuses par lesquelles ils cherchèrent à se l’expliquer et enfin le résultat de la lutte à partir des conditions d’existence historique de ces classes. En d’autres termes, je cherchais à montrer que la Constitution politique de l’Allemagne, les soulèvements contre elle, les théories politiques et religieuses de l’époque n’étaient pas des causes, mais des résultats du degré de développement auquel étaient arrivés, dans ce pays, l’agriculture, l’industrie, les voies de communication, le commerce des marchandises et de l’argent. Cette conception – qui est la seule conception matérialiste de l’histoire – provient de Marx et non de moi ; on la retrouve dans ses travaux sur la révolution française de 1848-49, publiés dans cette même Revue et dans son \emph{18 brumaire} de Louis Bonaparte.\par
Le parallèle entre la révolution allemande de 1525 et celle de 1848-49 était trop proche pour pouvoir être écarté à l’époque. Toutefois, à côté de la similitude du cours général des événements, qui fait qu’ici comme là ce fut toujours une seule et même armée de princes qui écrasa l’une après l’autre les diverses insurrections locales, à côté de la ressemblance, poussée parfois jusqu’au ridicule, dans la conduite de la bourgeoisie urbaine dans l’un et l’autre cas, il y a aussi des différences parfaitement claires et nettes :\par

\begin{quoteblock}
 \noindent « Qui profita de la révolution de 1525 ? Les princes. Qui profita de la révolution de 1848 ? Les grands souverains, l’Autriche et la Prusse. Derrière les petits princes de 1525 il y avait, liés à eux par le paiement des impôts, les petits bourgeois ; derrière les grands princes de 1850, derrière l’Autriche et la Prusse, il y a les grands bourgeois modernes qui se les soumettent rapidement au moyen de la dette d’État. Et derrière les grands bourgeois il y a les prolétaires. »
\end{quoteblock}

\noindent Je regrette d’être obligé de dire que, dans cette phrase, on faisait bien trop d’honneur à la grande bourgeoisie allemande. Elle a bien eu l’occasion, en Autriche comme en Prusse, « de se soumettre rapidement » la monarchie « au moyen de la dette d’État » ; mais jamais ni nulle part elle n’a profité de cette occasion.\par
La guerre de 1866 a fait tomber l’Autriche comme un don du ciel entre les mains de la bourgeoisie ; mais celle-ci ne sait pas régner, elle est impuissante et incapable de quoi que ce soit. Elle ne sait qu’une chose : sévir contre les travailleurs dès qu’ils bougent. Elle ne reste plus à la barre que parce que les Hongrois en ont besoin.\par
Et en Prusse ? Il est vrai, la dette d’État s’est vertigineusement accrue, le déficit est proclamé en permanence, les dépenses publiques augmentent chaque année, les bourgeois ont la majorité à la Chambre, sans eux on ne peut ni augmenter les impôts, ni obtenir de nouveaux emprunts – mais où est donc leur pouvoir sur l’État ? Il y a quelques mois à peine, lorsque le budget était de nouveau en déficit, ils avaient une position excellente. Il leur suffisait d’un peu de ténacité pour arracher de jolies concessions. Or que font-ils ? Ils considèrent comme une concession suffisante le fait que le gouvernement daigne leur permettre de mettre à ses pieds 9 millions, et cela non seulement pour un an, mais annuellement et pour toute la suite.\par
Je ne veux pas blâmer ces pauvres « nationaux-libéraux » de la Chambre plus qu’ils ne le méritent. Je sais qu’ils sont abandonnés par ceux qui sont derrière eux, par la masse de la bourgeoisie ; celle-ci ne veut pas régner : le souvenir de 1848 est encore trop vif en elle.\par
Nous verrons plus loin pourquoi la bourgeoisie allemande manifeste une lâcheté aussi remarquable.\par
Pour le reste, la phrase citée plus haut s’est trouvée entièrement confirmée. Depuis 1850, nous voyons les petits États, qui ne servent plus que de leviers pour les intrigues prussiennes ou autrichiennes, passer de plus en plus résolument à l’arrière-Plan, des luttes toujours plus vives pour l’hégémonie entre l’Autriche et la Prusse, enfin l’explication par la force de 1866, après laquelle l’Autriche conserve ses propres provinces, la Prusse se soumet, directement ou indirectement, tout le Nord, tandis que les trois États du Sud-Ouest sont provisoirement flanqués à la porte.\par
Pour la classe ouvrière allemande tous ces grands événements historiques ne présentent que l’importance suivante :\par
Premièrement, grâce au suffrage universel, les ouvriers ont obtenu le pouvoir de se faire représenter directement à l’Assemblée législative.\par
Deuxièmement, la Prusse a donné la première le bon exemple en escamotant trois autres couronnes de droit divin. Même les nationaux-libéraux ne croient plus, après cette pratique, que leur pays possède encore la même couronne immaculée de droit divin qu’il s’attribuait auparavant.\par
Troisièmement, il n’y a plus, en Allemagne, qu’un adversaire sérieux de la révolution : le gouvernement prussien.\par
Et quatrièmement, les Autrichiens-Allemands doivent enfin se demander une bonne fois ce qu’ils veulent être : Allemands ou Autrichiens, à quel parti ils préfèrent appartenir : à l’Allemagne ou à leurs annexes de Transleithanie ? Qu’ils doivent abandonner l’un ou l’autre était évident depuis longtemps, mais la démocratie petite-bourgeoise l’a toujours dissimulé.\par
En ce qui concerne les autres litiges importants nés de 1866 et discutés depuis jusqu’à satiété entre les « nationaux-libéraux » d’une part et les « populistes » de l’autre, l’histoire des années à venir pourrait bien prouver que ces deux points de vue ne se combattent avec tant de violence que parce qu’ils sont les pôles opposés d’un même esprit borné.\par
L’année 1866 n’a presque rien changé aux rapports sociaux en Allemagne. Les quelques réformes bourgeoises – système uniforme des poids et mesures, liberté de circuler, liberté professionnelle, etc., tout cela adapté à des limites bureaucratiques – n’atteignent même pas au niveau de ce qui a été conquis, depuis longtemps, par la bourgeoisie d’autres pays de l’Europe occidentale et laissent intact le principal fléau, le système bureaucratique des licences. Du reste, pour le prolétariat, toutes ces lois sur la liberté de circuler, sur le droit de cité, sur la suppression des passeports, etc., sont rendues parfaitement illusoires par les pratiques policières en usage.\par
Ce qui est bien plus important que les événements historiques de 1866, c’est le développement depuis 1848 en Allemagne de l’industrie et du commerce, des chemins de fer, des télégraphes et de la navigation transatlantique à vapeur. Si loin que soient ces progrès de ceux accomplis dans le même laps de temps en Angleterre et même en France, ils sont cependant inouïs pour l’Allemagne et ont réalisé, au cours de ces vingt années, bien plus que tout un siècle d’une autre période n’a jamais réalisé. Ce n’est qu’à présent que l’Allemagne se trouve vraiment et irrévocablement entraînée dans le commerce mondial. Les capitaux des industriels se sont rapidement accrus, la position sociale de la bourgeoisie s’est élevée en conséquence. Le symptôme le plus certain de la prospérité industrielle, la spéculation, fleurit abondamment, et enchaîne comtes et ducs à son char triomphal. Le capital allemand – que la terre lui soit légère ! – construit maintenant des chemins de fer russes et roumains, alors qu’il y a à peine quinze ans les chemins de fer allemands mendiaient le concours d’entrepreneurs anglais. Comment est-il donc possible que la bourgeoisie n’ait pas conquis aussi la domination politique, qu’elle se conduise aussi lâchement à l’égard du gouvernement ?\par
La bourgeoisie allemande a le malheur – selon la manière chérie des Allemands – d’arriver trop tard. Sa prospérité tombe dans une période où la bourgeoisie des autres pays d’Europe occidentale est politiquement à son déclin. En Angleterre elle n’a pu faire entrer son propre représentant, Bright, au gouvernement qu’au prix d’une extension du droit électoral, fait qui, par ses conséquences, mettra nécessairement fin à toute la domination bourgeoise. En France, où la bourgeoisie comme telle, en tant que classe, n’a régné que deux années sous la république, entre 1849 et 1850, elle n’a pu prolonger son existence sociale qu’en cédant sa domination politique à Louis Bonaparte et à l’armée. Et, étant donné l’infini renforcement de l’action réciproque des trois pays européens les plus avancés, il n’est plus possible, aujourd’hui, que la bourgeoisie puisse tranquillement instaurer sa domination politique en Allemagne, alors qu’elle se survit déjà en Angleterre et en France.\par
Ce qui distingue la bourgeoisie de toutes les classes qui régnèrent jadis, c’est cette particularité que, dans son développement, il y a un tournant à partir duquel tout accroissement de ses moyens de puissance, donc en premier lieu de ses capitaux, ne fait que contribuer à la rendre de plus en plus inapte à la domination politique. « Derrière les grands bourgeois il y a les prolétaires. » Dans la mesure même où elle développe son industrie, son commerce et ses moyens de communication, la bourgeoisie engendre le prolétariat. Et, à un certain moment – qui n’est pas nécessairement le même partout et ne se présente pas forcément au même degré de développement – elle commence à s’apercevoir que son double, le prolétariat, devient plus fort qu’elle. À partir de ce moment elle perd la force d’exercer exclusivement sa domination politique ; elle cherche des alliés avec lesquels elle partage son pouvoir ou auxquels elle le cède complètement, selon les circonstances.\par
En Allemagne, ce tournant a été atteint par la bourgeoisie dès 1848. Et, à ce moment-là, la bourgeoisie allemande prit peur bien plus du prolétariat français que du prolétariat allemand. Les combats de juin 1848 à Paris lui montrèrent ce qui l’attendait. Le prolétariat allemand était juste assez agité pour lui prouver qu’ici aussi la semence était jetée pour la même récolte ; et à partir de ce jour, l’action politique de la bourgeoisie avait perdu son mordant. Elle chercha des alliés, se vendit à eux à n’importe quel prix… et, aujourd’hui encore, elle n’a pas avancé d’un pas.\par
Ces alliés sont tous de nature réactionnaire : la royauté avec son armée et sa bureaucratie, la grande aristocratie féodale, les petits hobereaux, et même la prêtraille. La bourgeoisie a pactisé et s’est unie avec tout ce monde rien que pour sauver sa précieuse peau, jusqu’à ce qu’il ne lui reste plus rien à vendre. Et plus le prolétariat se développait, plus il commençait à se sentir comme une classe, à agir en tant que classe, et plus les bourgeois devenaient pusillanimes. Lorsque la stratégie prodigieusement mauvaise des Prussiens triompha, à Sadowa, de celle plus prodigieusement mauvaise encore des Autrichiens, il était bien difficile de dire lequel, du bourgeois prussien qui, lui aussi, avait été battu à Sadowa, ou de l’Autrichien, respira avec plus de soulagement.\par
Nos grands bourgeois agissent en 1870 exactement comme les bourgeois moyens en 1525. Quant aux petits bourgeois, aux artisans et aux boutiquiers, ils resteront toujours égaux à eux-mêmes. Ils espèrent s’élever en bluffant au rang de la grande bourgeoisie, ils redoutent d’être précipités dans le prolétariat. Pris entre la peur et l’espoir, ils sauveront leur peau pendant la lutte, et après ils se joindront au vainqueur. C’est leur nature.\par
L’action politique et sociale du prolétariat est allée du même pas que l’essor industriel depuis 1848. Le rôle que les ouvriers allemands jouent aujourd’hui dans leurs syndicats, coopératives, organisations et réunions politiques, aux élections et au prétendu Reichstag, montre déjà quelle transformation l’Allemagne a subie, insensiblement, ces vingt dernières années. C’est le plus grand honneur des ouvriers allemands d’avoir, eux seuls, réussi à envoyer au Parlement des ouvriers et des représentants des ouvriers, alors que ni les Français, ni les Anglais n’y sont encore parvenus.\par
Mais le prolétariat n’échappe pas encore, lui non plus, au parallèle avec l’année 1525. La classe réduite exclusivement et pour toute sa vie au salaire est encore bien loin de constituer la majorité du peuple allemand. Elle est donc également réduite à chercher des alliés. Et ceux-ci ne peuvent être cherchés que parmi les petits bourgeois, parmi le Lumpenproletariat des villes, parmi les petits paysans et les journaliers agricoles.\par
Nous avons déjà parlé des petits bourgeois. Ils sont très peu sûrs, sauf après la victoire, et alors ils poussent des cris de triomphe assourdissants, dans les bistrots. Cependant il y a parmi eux de très bons éléments qui se joignent spontanément aux ouvriers.\par
Le Lumpenproletariat, cette lie d’individus dévoyés de toutes les classes, qui établit son quartier général dans les grandes villes est, de tous les alliés possibles, le pire. Cette racaille est absolument vénale et importune. Quand les ouvriers français écrivaient sur les maisons, à chaque révolution, l’inscription : « Mort aux voleurs ! » et qu’ils en fusillaient même plus d’un, ce n’était certes pas par enthousiasme pour la propriété, mais bien parce qu’ils savaient très justement qu’il fallait avant tout se débarrasser de cette bande. Tout chef ouvrier qui emploie ces vagabonds comme gardes du corps, ou qui s’appuie sur eux, prouve déjà par là qu’il n’est qu’un traître au mouvement.\par
Les petits paysans – car les gros font partie de la bourgeoisie – sont de diverses catégories. Ou bien ce sont des paysans féodaux qui ont encore des corvées à faire pour leur noble maître. Après que la bourgeoisie a manqué à son devoir de libérer ces gens du servage, il ne sera pas difficile de les persuader qu’ils ne peuvent plus attendre leur libération que de la classe ouvrière.\par
Ou bien ce sont des métayers. Dans ce cas existe en général la même situation qu’en Irlande. Le fermage est si élevé que lorsque la récolte est moyenne, le paysan avec sa famille peut tout juste subsister, et lorsqu’elle est mauvaise il meurt presque de faim, n’est pas en état de payer le fermage et devient de ce fait totalement dépendant de la faveur du propriétaire foncier. Pour ces sortes de gens la bourgeoisie ne fait quelque chose que lorsqu’elle y est contrainte. De qui peuvent-ils donc attendre leur salut si ce n’est des ouvriers ?\par
Restent les paysans qui cultivent leur propre lopin de terre. Ceux-ci sont, le plus souvent, tellement grevés d’hypothèques qu’ils dépendent de l’usurier tout autant que le métayer du propriétaire foncier. À eux aussi il ne reste qu’un misérable salaire, et de plus tout à fait incertain, car il dépend de la bonne ou de la mauvaise récolte. Ce sont eux qui peuvent encore le moins attendre de la bourgeoisie, car ils sont précisément pressurés par les bourgeois, les capitalistes usuriers. Cependant ils sont le plus souvent très attachés à leur propriété, quoique eu réalité elle ne leur appartienne pas à eux, mais à l’usurier. On peut néanmoins les persuader qu’ils ne seront délivrés de l’usurier que lorsqu’un gouvernement dépendant du peuple transformera toutes les dettes hypothécaires en une dette unique due à l’État, et abaissera ainsi le taux de l’intérêt. Et cela, seule la classe ouvrière peut le réaliser.\par
Partout où dominent la grande et la moyenne propriété, les journaliers agricoles constituent la classe la plus nombreuse dans les campagnes. C’est le cas dans toute l’Allemagne du Nord et de l’Est et c’est là que les ouvriers industriels de la ville trouvent leurs alliés naturels les plus nombreux. Tout comme le capitaliste s’oppose à l’ouvrier d’industrie, le propriétaire foncier ou le gros fermier s’oppose au journalier agricole. Les mêmes mesures qui aident l’un doivent aider l’autre aussi. Les ouvriers industriels ne peuvent s’affranchir qu’en transformant le capital des bourgeois, c’est-à-dire les matières premières, les machines et les outils, les vivres nécessaires à la production, en propriété de la société, c’est-à-dire en leur propriété utilisée par eux en commun. De même les ouvriers agricoles ne peuvent être délivrés de leur terrible misère que si, avant tout, le principal objet de leur travail, la terre, est arrachée à la propriété privée des gros paysans et des seigneurs féodaux plus gros encore, transformée en propriété sociale et cultivée pour leur compte collectif par des coopératives d’ouvriers agricoles. Et ici nous en arrivons à la célèbre résolution du Congrès ouvrier international de Bâle : la société a intérêt à transformer la propriété foncière en propriété collective, nationale. Cette résolution concernait surtout les pays où existe la grande propriété foncière et l’exploitation de vastes domaines, et sur ces grands domaines un seul maître et beaucoup de journaliers. Or cette situation prédomine toujours dans l’ensemble en Allemagne, et c’est pourquoi la résolution en question était particulièrement opportune pour ce pays après l’Angleterre. Le prolétariat des champs, les salariés agricoles constituent la classe où se recrutent, dans leur grande masse, les armées des souverains. C’est la classe qui, en vertu du suffrage universel, envoie maintenant an Parlement toute cette foule de féodaux et de hobereaux ; mais c’est aussi la classe qui est la plus proche des ouvriers industriels des villes, qui partage avec eux les mêmes conditions d’existence, qui est dans une misère plus profonde même que la leur. Cette classe est impuissante parce qu’elle est émiettée et dispersée ; mais le gouvernement et l’aristocratie en connaissent si bien la force cachée, qu’ils laissent à dessein dépérir les écoles afin qu’elle reste ignorante. La tâche la plus urgente et la première du mouvement ouvrier allemand est de rendre cette classe vivante et de l’entraîner dans le mouvement. Le jour où la masse des ouvriers agricoles aura compris ses propres intérêts, un gouvernement réactionnaire, féodal, bureaucratique ou bourgeois sera impossible en Allemagne.
\section[2)]{2)}
\noindent Les lignes qui précèdent ont été écrites il y a plus de quatre ans ; elles conservent aujourd’hui encore toute leur valeur. Ce qui était vrai après Sadowa et le partage de l’Allemagne se trouve confirmé après Sedan et la fondation du Saint-Empire allemand prussien. Si infimes sont les changements que peuvent imprimer à la direction du mouvement historique, les événements « ébranlant le monde » de ce qu’on appelle la grande politique !\par
Ce que ces événements historiques peuvent par contre, c’est accélérer la rapidité de ce mouvement. Et, à cet égard les auteurs des « événements ébranlant le monde » ci-dessus ont eu involontairement des succès qu’ils n’ont certainement pas souhaités le moins du monde, mois auxquels, bon gré mal gré, ils sont obligés de se résigner.\par
La guerre de 1866 avait déjà ébranlé la vieille Prusse dans ses fondements. Il en avait déjà coûté beaucoup de peine pour ramener à la vieille discipline, après 1848, les industriels rebelles – bourgeois aussi bien que prolétaires – des provinces occidentales ; cependant on y était arrivé, et de nouveau les intérêts des hobereaux des provinces orientales et ceux de l’armée prédominaient dans l’État. En 1866, presque toute l’Allemagne du Nord-Ouest devint prussienne. Abstraction faite du tort moral irréparable que la couronne prussienne de droit divin s’était fait en engloutissant trois autres couronnes de droit divin, le centre de gravité de la monarchie se déplaça alors considérablement vers l’Ouest. Les cinq millions de Rhénans et de Westphaliens furent renforcés tout d’abord directement par les 4 millions, puis, indirectement, par les 6 millions d’Allemands annexés par la Confédération de l’Allemagne du Nord. Et, en 1870, s’y ajoutèrent encore les 8 millions d’Allemands du Sud-Ouest, de sorte que désormais, dans le « nouvel empire », aux 14 millions et demi de vieux Prussiens (des six provinces à l’Est de l’Elbe parmi lesquels il y avait en outre 2 millions de Polonais), s’opposaient les 25 millions qui avaient depuis longtemps échappé au féodalisme prussien endurci des hobereaux. Ainsi ce sont justement les victoires de l’armée prussienne qui déplacèrent toute la base de l’édifice de l’État prussien ; la domination des hobereaux devint de plus en plus intolérable même au gouvernement. Mais, en même temps, le développement industriel extrêmement rapide avait remplacé la lutte entre hobereaux et bourgeois par la lutte entre bourgeois et ouvriers, de sorte qu’à l’intérieur aussi les bases sociales du vieil État ont subi un bouleversement total. La monarchie, qui se décomposait lentement depuis 1840, avait eu pour condition fondamentale d’existence la lutte entre l’aristocratie et la bourgeoisie, lutte dans laquelle elle maintenait l’équilibre ; à partir du moment où il importait de protéger, non plus l’aristocratie contre la pression de la bourgeoisie, mais toutes les classes possédantes contre la pression de la classe ouvrière, la vieille monarchie absolue dut passer entièrement à la forme d’État spécialement élaborée à cette fin : la monarchie bonapartiste. J’ai analysé ailleurs (La Question du logement, 2º fasc., pp. 26 et suiv.) ce passage de la Prusse au bonapartisme. Ce que je n’avais pas à faire ressortir là, mais qui est essentiel ici, c’est que ce passage fut le plus grand pas en avant que la Prusse eût fait depuis 1848, tant la Prusse était restée en arrière du développement moderne. C’était encore un État mi-féodal, tandis que le bonapartisme est, en tout cas, une forme moderne d’État qui suppose l’abolition du féodalisme. La Prusse doit donc se décider à en finir avec ses nombreux restes de féodalité, à sacrifier ses hobereaux comme tels. Naturellement, cela s’accomplit sous les formes les plus atténuées et selon le proverbe : « Qui va doucement, va sûrement ». Il en fut ainsi, par exemple, pour la fameuse organisation des cercles. On supprime les privilèges féodaux du hobereau individuel sur sa terre, mais c’est pour les rétablir comme privilèges de l’ensemble des grands propriétaires fonciers sur tout le cercle. La chose subsiste, seulement on la traduit du dialecte féodal dans l’idiome bourgeois. On transforme d’office le hobereau de type prussien endurci en quelque chose comme un \emph{squire} anglais ; et il n’a pas tellement eu besoin de regimber, car l’un est aussi sot que l’autre.\par
C’est ainsi donc que l’étrange destinée de la Prusse voulut qu’elle achevât vers la fin de ce siècle, sous la forme agréable du bonapartisme, sa révolution bourgeoise qu’elle avait commencée en 1808-1813 et continuée quelque peu en 1848. Et si tout va bien, si le monde reste bien gentiment tranquille et si nous devenons tous assez vieux, nous pourrons peut-être voir, en 1900, que le gouvernement de Prusse a vraiment supprimé toutes les institutions féodales, que la Prusse en est arrivée enfin au point où en était la France en 1792.\par
L’abolition du féodalisme, si nous voulons nous exprimer positivement, signifie l’instauration du régime bourgeois. Au fur et à mesure que les privilèges aristocratiques tombent, la législation devient bourgeoise. Et ici nous nous trouvons au cœur même des rapports de la bourgeoisie allemande avec le gouvernement. Nous avons vu que le gouvernement est contraint d’introduire ces lentes et mesquines réformes. Mois à la bourgeoisie il présente chacune de ces petites concessions comme un sacrifice fait aux bourgeois, comme une concession arrachée à grand-peine à la couronne, concession en échange de laquelle les bourgeois devraient à leur tour accorder quelque chose au gouvernement. Et les bourgeois, quoique sachant assez bien à quoi s’en tenir là-dessus, acceptent la duperie. D’où cette convention tacite qui à Berlin est la base implicite de tous les débats au Reichstag et à la Chambre : d’une part le gouvernement, à une allure d’escargot, réforme les lois dans le sens des intérêts de la bourgeoisie, écarte les obstacles créés au développement de l’industrie par la féodalité et le particularisme des petits États, unifie monnaies, poids et mesures, introduit la liberté professionnelle, met avec la liberté de circulation à la disposition complète et illimitée du capital la main-d’œuvre de l’Allemagne, favorise le commerce et la spéculation ; d’autre part, la bourgeoisie abandonne au gouvernement tout le pouvoir politique réel, vote les impôts et les emprunts, lui accorde les soldats et l’aide à donner aux nouvelles réformes une forme légale telle que le vieux pouvoir policier à l’égard des personnes mal vues du gouvernement reste intact. La bourgeoisie achète son émancipation sociale graduelle au prix d’une renonciation immédiate à son propre pouvoir politique. Naturellement, le principal motif qui rend un tel accord acceptable pour la bourgeoisie n’est pas la peur du gouvernement, mais celle du prolétariat.\par
Cependant, si lamentable que soit le comportement de notre bourgeoisie dans le domaine politique, il est indéniable que, sur le plan industriel et commercial, elle fait enfin son devoir. L’essor de l’industrie et du commerce, que nous avons signalé dans l’introduction à la deuxième édition du présent ouvrage, s’est développé, depuis, avec bien plus de force encore. Ce qui s’est produit à cet égard, depuis 1868, dans la région industrielle rhéno-westphalienne, est vraiment inouï pour l’Allemagne et rappelle l’essor des districts usiniers anglais au début de ce siècle. Et il en aura été de même en Saxe et en Haute-Silésie, à Berlin, à Hanovre et dans les villes maritimes. Nous avons enfin un commerce mondial, une vraie grande industrie, une vraie bourgeoisie moderne ; en revanche, nous avons également connu un véritable krach et nous avons aussi gagné un prolétariat véritable et puissant.\par
Pour l’historien de l’avenir le grondement des canons de Spickeren, Mars-la-Tour et Sedan, et tout ce qui s’ensuit aura bien moins d’importance dans l’histoire de l’Allemagne de 1869-1874 que le développement discret, calme, mais ininterrompu du prolétariat allemand. Dès 1870, les ouvriers allemands durent affronter une rude épreuve : la provocation bonapartiste à la guerre et son effet naturel, l’enthousiasme national général en Allemagne. Les ouvriers socialistes allemands ne se laissèrent pas égarer un instant. Ils ne manifestèrent pas la moindre trace de chauvinisme national. Au milieu de l’ivresse de victoire la plus déchaînée, ils restèrent froids, exigèrent « une paix équitable avec la République française et sans annexions » ; et même l’état de siège ne put les réduire au silence. Ils ne se laissèrent toucher ni par la gloire des combats, ni par les bavardages sur la « magnificence de l’empire » allemand ; leur seul but resta l’affranchissement de tout le prolétariat européen. On peut bien le dire : les ouvriers d’aucun autre pays n’ont été, jusqu’à présent, soumis à une épreuve aussi lourde, aussi brillamment subie.\par
L’état de siège du temps de guerre fut suivi des procès de Haute trahison, de lèse-majesté et d’offense aux fonctionnaires, des tracasseries policières sans cesse aggravées du temps de paix. Le Volksstaat avait, en règle générale, trois à quatre rédacteurs simultanément en prison. Les autres journaux à l’avenant. Tout orateur du Parti quelque peu connu devait, au moins une fois par an, comparaître devant les tribunaux où il était presque régulièrement condamné. Les uns après les autres, bannissements, confiscations, dispersions des réunions, tombaient comme grêle ; mais tout cela en vain. Chaque militant arrêté ou expulsé était aussitôt remplacé par un autre ; pour chaque réunion dissoute on en convoquait deux autres ; d’un lieu à l’autre, on lassa l’arbitraire policier par la persévérance et par la stricte observation des lois. Toutes les persécutions produisirent le contraire du but poursuivi ; loin de briser ou seulement de faire plier le parti ouvrier, elles lui amenèrent au contraire, sans cesse, de nouvelles recrues et renforcèrent son organisation. Dans leur lutte contre les autorités, comme contre les bourgeois individuellement, les ouvriers se montrèrent partout, intellectuellement et moralement, supérieurs à eux et prouvèrent, surtout dans leurs conflits avec les “employeurs”, que ce sont eux, les ouvriers, qui sont à présent des hommes cultivés tandis que les capitalistes sont des lourdauds. Et, avec cela, ils mènent leurs luttes avec un humour qui prouve combien ils sont sûrs de leur cause et conscients de leur supériorité. Une lutte ainsi conduite, sur un terrain historiquement préparé, doit donner de grands résultats. Les succès obtenus aux élections de janvier restent uniques jusqu’à présent dons l’histoire du mouvement ouvrier moderne et la stupéfaction qu’ils suscitèrent dans toute l’Europe était parfaitement justifiée.\par
Les ouvriers allemands ont, sur ceux du reste de l’Europe, deux avantages essentiels. Premièrement, ils appartiennent au peuple le plus théoricien de l’Europe et ils ont conservé le sens théorique qui a si complètement disparu chez ceux qu’on appelle les gens “cultivés” en Allemagne. S’il n’y avait pas eu précédemment la philosophie allemande, en particulier celle de Hegel, le socialisme scientifique allemand – le seul socialisme scientifique qui ait jamais existé – n’eût jamais été fondé. Sans le sens théorique parmi les ouvriers, ce socialisme scientifique ne se serait jamais aussi profondément ancré en eux. Et ce qui prouve quel avantage infini c’est là, c’est d’une part l’indifférence à l’égard de toute théorie, une des causes principales du peu de progrès du mouvement ouvrier anglais, malgré l’excellente organisation des divers syndicats, et d’autre part le désordre et la confusion provoqués par le proudhonisme dans sa forme initiale chez les Français et les Belges, dans sa forme caricaturée dans la suite par Bakounine chez les Espagnols et les Italiens.\par
Le second avantage c’est que chronologiquement les Allemands sont venus dans le mouvement ouvrier à peu près les derniers. De même que le socialisme allemand théorique n’oubliera jamais qu’il s’est élevé sur les épaules de Saint-Simon, de Fourier et d’Owen, trois hommes qui, malgré toutes leurs idées chimériques et leurs vues utopiques, comptent parmi les plus grands cerveaux de tous les temps et ont anticipé génialement d’innombrables choses dont nous démontrons à présent scientifiquement la justesse – de même le mouvement ouvrier allemand pratique ne doit jamais oublier qu’il s’est développé sur les épaules du mouvement anglais et français, qu’il a pu simplement profiter de leurs expériences chèrement acquises et éviter maintenant leurs erreurs, alors inévitables pour la plupart. Sans le passé des trade-unions anglaises et des luttes politiques ouvrières françaises, sans l’impulsion gigantesque donnée particulièrement par la Commune de Paris, où en serions-nous aujourd’hui ?\par
Il faut reconnaître que les ouvriers allemands ont su profiter des avantages de leur situation avec une rare intelligence. Pour la première fois, depuis qu’il existe un mouvement ouvrier, la lutte est menée dans ses trois directions – théorique, politique et économique pratique résistance contre les capitalistes) – avec harmonie, cohésion, et méthode. C’est dans cette attaque concentrique, pour ainsi dire, qu’est la force invincible du mouvement allemand.\par
D’une part, en raison de leur position avantageuse, de l’autre, par suite des particularités insulaires du mouvement anglais et de la violente répression du mouvement français, les ouvriers allemands sont pour le moment placés à l’avant-garde de la lutte prolétarienne. On ne saurait prédire combien de temps les événements leur laisseront ce poste d’honneur. Mais tant qu’ils l’occuperont, ils rempliront leur devoir comme il convient, il faut l’espérer. Pour cela ils devront redoubler d’efforts dans tous les domaines de la lutte et de l’agitation. Ce sera en particulier le devoir des chefs de s’éclairer de plus en plus sur toutes les questions théoriques, de se libérer de plus en plus de l’influence de la phraséologie reçue, qui appartient à la conception du monde du passé et de ne jamais oublier que le socialisme, depuis qu’il est devenu une science, veut être pratiqué comme une science, c’est-à-dire étudié. Il importera donc de répandre avec un zèle accru parmi les masses ouvrières, la compréhension ainsi acquise et de plus en plus claire et de consolider de plus en plus l’organisation du Parti et celle des syndicats. Quoique les voix socialistes exprimées en janvier représentent déjà une assez belle armée, elles sont bien loin encore de constituer la majorité de la classe ouvrière allemande ; et, si encourageants que soient les succès de la propagande parmi la population rurale, il reste encore infiniment à faire précisément sur ce terrain. Il ne s’agit donc pas de relâcher le combat ; il faut, au contraire, arracher à l’ennemi une ville, une circonscription électorale après l’autre ; mais, avant tout, il s’agit maintenir le véritable esprit international qui n’admet aucun chauvinisme patriotique et qui salue avec joie tout nouveau progrès du mouvement prolétarien, de quelque nation qu’il provienne. Si les ouvriers allemands continuent à agir ainsi, je ne dis pas qu’ils marcheront à la tête du mouvement – il n’est pas dans l’intérêt du mouvement que les ouvriers d’une seule nation quelconque marchent à sa tête – mais ils occuperont une place honorable sur la ligne de combat ; et ils seront armés et prêts, lorsque des épreuves d’une dureté inattendue ou bien de grands événements exigeront d’eux encore plus de courage, de décision, d’énergie.
\chapterclose


\chapteropen
\renewcommand{\leftmark}{I. La situation économique et la structure sociale de l’Allemagne}
\chapter[I. La situation économique et la structure sociale de l’Allemagne]{I. La situation économique et la structure sociale de l’Allemagne}

\chaptercont
\noindent Le peuple allemand a, lui aussi, ses traditions révolutionnaires. Il fut un temps où l’Allemagne a produit des hommes qu’on peut comparer aux meilleurs révolutionnaires des autres pays, où le peuple allemand fit preuve d’une endurance et d’une énergie qui, dans une nation centralisée, eussent donné les résultats les plus grandioses, où les paysans et les plébéiens allemands caressèrent des idées et des projets devant lesquels leurs descendants frémissent assez souvent d’horreur aujourd’hui encore.\par
Le moment est venu, en face du relâchement actuel qui se manifeste presque partout après deux années de luttes, de présenter à nouveau au peuple allemand les figures rudes, mais vigoureuses et tenaces de la grande Guerre des paysans. Trois siècles se sont écoulés depuis, et bien des choses ont changé et cependant, la Guerre des paysans n’est pas si loin de nos luttes d’aujourd’hui et les adversaires à combattre sont en grande partie restés les mêmes qu’autrefois. Les classes et fractions de classes qui ont trahi partout en 1848 et 1849, nous les retrouvons, dans le même rôle de traîtres, déjà en 1525, quoique à une étape inférieure de développement. Et si le robuste vandalisme de la Guerre des paysans ne s’est dans le mouvement des dernières années manifesté que par endroits, dans l’Odenwald, dans la Forêt-Noire, en Silésie, cela ne constitue pas en tout cas un privilège de l’insurrection moderne.\par

\astertri

\noindent Jetons tout d’abord un rapide coup d’œil sur la situation de l’Allemagne au début du XVIᵉ siècle.\par
L’industrie allemande avait connu, au cours des XIVᵉ et XVᵉ siècles, un essor considérable. À l’industrie locale, rurale et féodale s’était substituée l’industrie corporative des villes, produisant pour un marché élargi et même pour des marchés assez lointains. Le tissage de lainages grossiers et de la toile était devenu une industrie permanente et très répandue. On fabriquait même déjà, à Augsbourg, des tissus de laine assez fins ainsi que des soieries. À côté du tissage, s’était développée en particulier cette industrie proche de l’art qui trouvait un aliment dans le luxe ecclésiastique et séculier de la fin du moyen âge : celle des joailliers, des statuaires, des sculpteurs, des graveurs sur cuivre et sur bois, des armuriers, des médailleurs, des tourneurs, etc. Toute une série d’inventions plus ou moins importantes, dont historiquement les apogées sont celle de la poudre et de l’imprimerie, avaient contribué considérablement au développement de l’industrie.\par
Le commerce se développait au même pas que l’industrie. Grâce à son monopole séculaire de la mer, la Hanse avait assuré l’élévation de toute l’Allemagne du Nord au-dessus de la barbarie moyenâgeuse. Et, quoiqu’elle commençât à partir de la fin du XVᵉ siècle, à succomber rapidement à la concurrence des Anglais et des Hollandais, la grande voie commerciale de l’Inde vers le Nord continuait malgré les découvertes de Vasco de Gama, à passer par l’Allemagne. Augsbourg était toujours le grand entrepôt des soieries italiennes, des épices de l’Inde et de tous les produits du Levant. Les villes de l’Allemagne du Sud, surtout Augsbourg et Nuremberg, étaient le centre d’une richesse et d’un luxe considérables pour l’époque. La production des matières premières s’était, de même, considérablement développée. Les mineurs allemands étaient au XVᵉ siècle les plus adroits du monde, et d’autre part, l’épanouissement des villes avait arraché même l’agriculture à la barbarie du moyen âge. Non seulement on avait défriché des surfaces immenses, mais on cultivait des plantes tinctoriales et d’autres plantes importées, dont la culture qui demandait plus de soins exerça, en général, une influence favorable sur l’agriculture.\par
Malgré tout, l’essor de la production nationale de l’Allemagne était resté inférieur à celui des autres pays. L’agriculture était très en retard sur celle de l’Angleterre et des Pays-Bas ; l’industrie sur celle de l’Italie, des Flandres et de l’Angleterre et, dans le domaine du commerce maritime, les Anglais, et en particulier les Hollandais, commençaient déjà à évincer les Allemands. La population était encore très disséminée. La civilisation en Allemagne n’existait qu’à l’état sporadique, groupée autour de centres industriels et commerciaux isolés. Les intérêts de ces divers centres divergeaient eux-mêmes considérablement, et c’est à peine s’ils avaient, ça et là, un point de contact. Le Sud avait de tout autres liaisons commerciales et de tout autres débouchés que le Nord l’Est et l’Ouest étaient presque à l’écart de toute circulation. Aucune ville n’eut l’occasion de devenir le centre industriel et commercial de tout le pays, comme Londres, par exemple, l’était déjà pour l’Angleterre. Les communications intérieures se limitaient presque exclusivement à la navigation côtière et fluviale et aux quelques grandes voies commerciales, menant d’Augsbourg et de Nuremberg, par Cologne aux Pays-Bas, et par Erfurt vers le Nord. Loin des fleuves et des routes commerciales, se trouvaient un certain nombre de petites villes qui à l’écart des grandes voies de communication, continuaient à végéter tranquillement dans les conditions d’existence de la fin du moyen âge, ne consommant que très peu de produits étrangers et ne fournissant qu’une petite quantité de produits d’exportation. Dans la population rurale, seule la noblesse était en contact avec des cercles plus larges et des besoins nouveaux. Quant aux paysans, dans leur grande masse, ils ne dépassèrent jamais le cadre des relations de voisinage et de leur étroit horizon local.\par
Tandis qu’en Angleterre et en France le développement du commerce et de l’industrie entraînait une étroite connexion des intérêts dans tout le pays et, par suite, la centralisation politique, l’Allemagne ne parvenait qu’à grouper les intérêts par provinces, autour de centres purement locaux, qu’au morcellement politique, morcellement qui devint peu après immuable, lorsque l’Allemagne fut exclue du commerce mondial. Au fur et à mesure de la désagrégation de l’\emph{Empire purement féodal}, l’union même que constituait l’Empire se désagrégea, les grands vassaux de l’Empire se transformèrent en princes à peu près indépendants, et les villes impériales d’une part et les chevaliers d’Empire d’autre part s’unirent tantôt les uns contre les autres, tantôt contre les princes ou contre l’empereur. Le pouvoir impérial ne sachant plus que penser de sa propre situation, ballottait entre les différents éléments qui composaient l’Empire et perdait de plus en plus de son autorité. Sa tentative de centralisation à la Louis XI ne réussit, malgré toutes les intrigues et toutes les violences, qu’à maintenir la cohésion des terres autrichiennes héréditaires. Ceux qui tirèrent et devaient finalement tirer profit de ce désordre général, de cet enchevêtrement de conflits innombrables, ce furent les représentants de la centralisation à l’intérieur du morcellement, les représentants de la centralisation locale et provinciale, \emph{les princes}, à côté desquels l’empereur lui-même devint de plus en plus un prince comme les autres.\par
Dans ces conditions, la situation des classes héritées du moyen âge s’était modifiée fondamentalement et de nouvelles classes s’étaient constituées à côté des anciennes.\par
Les \emph{princes} étaient issus de la haute noblesse. Ils étaient déjà à peu près complètement indépendants de l’empereur, et en possession de la plupart des droits souverains. Ils faisaient la guerre et la paix de leur propre chef, entretenaient des armées permanentes, convoquaient des diètes et imposaient des contributions. Ils avaient déjà soumis à leur autorité une grande partie de la petite noblesse et des villes. Ils employaient constamment tous les moyens en leur pouvoir pour annexer à leurs territoires le reste des villes et des baronnies non médiatisées. Par rapport à celles-ci, ils faisaient œuvre de centralisation, comme ils faisaient œuvre de décentralisation par rapport au pouvoir d’Empire. À l’intérieur, leur gouvernement était déjà très arbitraire. La plupart du temps ils ne convoquaient les états que lorsqu’ils ne pouvaient pas se tirer d’affaire autrement. Ils décrétaient des impôts et des emprunts selon leur bon plaisir. Le droit pour les états de voter l’impôt était rarement reconnu et plus rarement encore exercé. Et même alors, le prince avait ordinairement la majorité, grâce aux deux états qui n’étaient pas soumis à l’impôt, mais qui en profitaient, la chevalerie et le clergé. Le besoin d’argent des princes augmentait avec le luxe et le train grandissant de leur Cour, avec la constitution des armées permanentes et les dépenses croissantes du gouvernement. Les impôts devinrent de plus en plus lourds. Les villes en étaient, la plupart du temps, garanties par leurs privilèges. Tout le poids en retombait sur les paysans, tant sur ceux des domaines du prince que sur les serfs, les corvéables et les tenanciers des chevaliers vassaux. Quand les impôts indirects ne suffisaient pas, on faisait appel aux impôts indirects. Les manœuvres les plus raffinées de l’art financier étaient employées pour combler les trous du fisc. Quand tout cela ne suffisait pas encore, quand on ne pouvait plus rien mettre en gage, et qu’aucune ville libre impériale ne voulait plus donner de crédit, on recourait aux pires des opérations frauduleuses, on frappait de la monnaie frelatée, on établissait des cours forcées, hauts ou bas, selon que cela convenait au fisc. Le commerce des privilèges citadins ou autres, qu’on reprenait ensuite de force pour les revendre au prix fort, l’exploitation de toute tentative d’opposition comme prétexte à toute sorte de rançons et de pillages, etc., étaient également, à cette époque, des sources de revenus fructueuses et quotidiennes pour les princes. Enfin, la justice était également pour eux un article de commerce permanent et non négligeable. Bref, les sujets de cette époque, qui avaient, en outre, à satisfaire la cupidité des baillis et autres fonctionnaires du prince, jouissaient pleinement des bienfaits du système « paternel » de gouvernement.\par
La noblesse moyenne avait presque complètement disparu de la hiérarchie féodale du moyen âge. Une partie de ses membres étaient devenus de petits princes indépendants les autres étaient tombés dans les rangs de la petite noblesse. La petite noblesse, \emph{les chevaliers}, allait rapidement à sa ruine. Une grande partie était déjà complètement réduite à la misère et vivait seulement du service des princes, dans des emplois militaires ou civils. Une autre était dans la vassalité et dans la dépendance des princes. La minorité était dans la dépendance directe de l’Empire. Le développement de la technique militaire, le rôle croissant de l’infanterie, le progrès des armes à feu diminuèrent de plus en plus son importance militaire en tant que cavalerie lourde et mirent fin en même temps à l’inexpugnabilité de ses châteaux forts. Tout comme celle des artisans de Nuremberg, l’existence des chevaliers fut rendue superflue par les progrès de l’industrie. Leurs besoins d’argent contribuèrent considérablement à leur ruine. Le luxe déployé dans les châteaux, la splendeur dont on rivalisait dans les tournois et les fêtes, le prix des armes et des chevaux, augmentèrent avec les progrès du développement social, alors que les sources de revenus des chevaliers et des barons n’augmentaient que très peu, ou même pas du tout. Les guerres privées, avec leurs inévitables pillages et rançons, le brigandage de grands chemins et autres nobles occupations de ce genre devinrent, avec le temps, par trop dangereux. Les redevances et les prestations des sujets rapportaient à peine plus qu’autrefois. Pour subvenir à leurs besoins croissants, les seigneurs durent recourir aux mêmes moyens que les princes. L’exploitation des paysans par la noblesse s’aggrava d’année en année. Les serfs furent pressurés jusqu’à la dernière limite, les corvéables chargés, sous toutes sortes de prétextes et d’étiquettes, de nouvelles taxes et prestations. Les corvées, cens, redevances, droits de tenure, mainmorte, droit d’aubaine, etc., furent augmentés arbitrairement, en violation de tous les anciens contrats. On refusait de rendre la justice ou bien on la vendait, et quand le chevalier ne trouvait plus aucun prétexte pour tirer de l’argent du paysan, il le jetait en prison sans autre forme de procès, et l’obligeait à racheter sa liberté.\par
Avec les autres ordres, la petite noblesse ne vivait pas non plus en bonne intelligence. La noblesse vassale s’efforçait de devenir noblesse d’Empire. Celle-ci, à son tour, cherchait à conserver son indépendanee. D’où des différends continuels avec les princes. Les chevaliers enviaient le clergé, qui, bouffi d’orgueil comme il l’était alors, leur apparaissait comme un ordre superflu, ses grands domaines et ses immenses richesses indivisibles grâce au célibat et à la constitution de l’Église. Avec les villes, ils étaient sans cesse aux prises. Ils leur devaient de l’argent, vivaient du pillage de leur territoire, du détroussement de leurs marchands et de la rançon de leurs citoyens faits prisonniers au cours des guerres. Et la lutte de la chevalerie contre tous ces ordres se faisait d’autant plus violente que pour elle aussi la question d’argent devenait davantage une question vitale.\par
Le \emph{clergé}, représentant de l’idéologie féodale du moyen âge, ne se ressentait pas moins du bouleversement historique. L’invention de l’imprimerie et l’extension des besoins du commerce lui avaient enlevé le monopole, non seulement de la lecture et de l’écriture, mais aussi de la culture supérieure. La division du travail fit son apparition également dans le domaine intellectuel. Le clergé se vit évincer par la nouvelle caste des juristes de toute une série de postes extrêmement influents. Lui aussi, il commença à devenir en grande partie superflu, ce qu’il confirmait d’ailleurs lui-même par sa paresse et son ignorance sans cesse croissantes. Mais plus il devenait superflu, plus il devenait nombreux – grâce à ses énormes richesses qu’il augmentait encore constamment par tous les moyens possibles.\par
Le clergé se divisait lui-même en deux classes tout à fait distinctes. La hiérarchie ecclésiastique féodale constituait la classe \emph{aristocratique} : les évêques et les archevêques, les abbés, les prieurs et autres prélats. Ces hauts dignitaires de l’église étaient ou des princes d’Empire eux-mêmes, ou des seigneurs féodaux dominant, sous la suzeraineté d’autres princes, de vastes territoires avec de nombreux serfs et corvéables. Non seulement ils exploitaient leurs sujets aussi impitoyablement que la noblesse et les princes, mais ils s’y prenaient d’une façon plus cynique encore. Outre la violence directe, ils employaient toutes les chicaneries de la religion, outre les horreurs de la torture, toutes les horreurs de l’excommunication et du refus de l’absolution, toutes les intrigues du confessionnal pour arracher à leurs sujets leur dernier liard ou augmenter le patrimoine de l’Église. La falsification de documents était pour ces dignes hommes un moyen courant et familier d’extorsion. Mais quoique, outre les prestations et les taxes féodales ordinaires, ils touchassent également la dîme, tous ces revenus ne leur suffisaient pas encore. Pour soutirer encore plus d’argent au peuple, ils eurent recours à la confection d’images saintes et de reliques miraculeuses, à l’organisation de lieux de prière qui procuraient le salut, au trafic des indulgences, et cela pendant longtemps avec le plus grand succès.\par
Ce sont ces prélats et leur nombreuse gendarmerie de moines, constamment renforcée à mesure que se répandent les tracasseries politiques et religieuses, sur qui se concentra la haine des curés, non seulement dans le peuple, mais aussi dans la noblesse. Dans la mesure où ils dépendaient directement de l’Empire, ils constituaient un obstacle pour les princes. Le train de vie joyeux des évêques et des abbés ventripotents et de leur armée de moines provoquait la jalousie de la noblesse et indignait d’autant plus le peuple qui devait en supporter les frais, qu’il était en contradiction plus criante avec leurs sermons.\par
La fraction \emph{plébéienne} du clergé se composait des curés des villages et des villes. Ils étaient en dehors de la hiérarchie féodale de l’Eglise, et ne participaient en aucune façon à ses richesses. Leur travail était moins contrôlé, et quelque important qu’il fût pour l’Église, bien moins indispensable pour le moment que ne l’étaient les services policiers des moines encasernés. C’est pourquoi ils étaient beaucoup plus mal payés et leurs prébendes étaient la plupart du temps très minces. D’origine bourgeoise ou plébéienne, ils étaient assez près de la situation matérielle de la masse pour conserver, malgré leur état de prêtres, des sympathies bourgeoises et plébéiennes. La participation aux mouvements de l’époque, qui n’était qu’exception chez les moines, était de règle chez eux. Ils fournirent les théoriciens et les idéologues au mouvement, et un grand nombre d’entre eux, représentants des plébéiens et des paysans, moururent pour ce fait sur l’échafaud. Aussi la haine du peuple contre les prêtres ne se tourne-t-elle que rarement contre eux.\par

\hline
\noindent De même qu’au-dessus des princes et de la noblesse se trouvait l’empereur, de même au-dessus du haut et du bas clergé se trouvait le \emph{pape}. De même que l’empereur recevait le « pfennig ordinaire » (\emph{der gemeine Pfennig}), les taxes d’Empire, de même le pape prélevait des taxes ecclésiastiques générales, à l’aide desquelles il subvenait au luxe de la Cour de Rome. En aucun pays, ces impôts d’Église n’étaient perçus – grâce à la puissance et au nombre des prêtres – plus consciencieusement et plus strictement qu’en Allemagne. C’était le cas particulièrement pour les annates perçues à l’occasion de la vacance des évêchés. Avec les besoins croissants, on inventa de nouveaux moyens de se procurer de l’argent : commerce des reliques et des indulgences, dons jubilaires, etc. C’est ainsi que des sommes considérables passaient chaque année d’Allemagne à Rome et l’oppression accrue qui en résultait, non seulement augmentait la haine contre les prêtres, mais renforçait encore le sentiment national, particulièrement dans la noblesse, l’ordre le plus national de l’époque.\par
———\par
\noindent De la primitive petite-bourgeoisie des villes du moyen âge ils s’était formé, peu à peu, avec le développement du commerce et de l’industrie, à trois fractions bien distinctes.\par
À la tête de la société urbaine se trouvait le \emph{patriciat}, ceux qu’on appelait les \emph{« Honorables » (die Ehrbarkeit)}. Il se composait des familles les plus riches. Elles seules siégeaient au Conseil et occupaient toutes les fonctions municipales. Ainsi elles n’administraient pas seulement les revenus de la ville, elles en vivaient également. Fortes de leurs richesses, de leur situation aristocratique traditionnelle, reconnues par l’empereur et par l’Empire, elles exploitaient de toutes les manières la commune, aussi bien que les paysans sujets de la ville. Elles pratiquaient l’usure des grains et de l’argent, s’attribuaient toute sorte de monopoles, enlevaient à la commune, les uns après les autres, les droits d’affouage et de pacage, s’appropriaient directement la jouissance des forêts et prairies communales, décrétaient arbitrairement des péages sur les chemins, les ponts, les portes et autres charges, et trafiquaient des privilèges corporatifs, des droits de citoyenneté et de maîtrise et de la justice. Les paysans des alentours n’étaient pas mieux traités par le patriciat que par la noblesse ou les curés. Au contraire, les baillis des villes et les ammans (\emph{Amtmänner}) des villages, tous patriciens, ajoutaient encore à la dureté et à la cupidité aristocratiques une certaine minutie bureaucratique dans le recouvrement des droits. Les revenus municipaux ainsi recueillis étaient administrés de la façon la plus arbitraire. La comptabilité des livres municipaux, pure formalité, était négligée et confuse au possible. Les malversations et les déficits de trésorerie étaient à l’ordre du jour. On comprend combien il devait être facile à une caste peu nombreuse, entourée de toute part de privilèges, étroitement unie par les liens de parenté et les intérêts communs, de s’enrichir énormément au moyen des revenus des villes quand on pense aux abus de confiance que l’année 1848 a mis à jour dans un si grand nombre d’administrations municipales.\par
Les patriciens avaient fait le nécessaire pour laisser partout s’éteindre les droits de la commune urbaine, surtout en matière de finances. Ce n’est que plus tard, lorsque les escroqueries de ces messieurs devinrent par trop graves, que les communes se mirent de nouveau en mouvement pour se saisir au moins du contrôle de l’administration municipale. Et elles réussirent en fait à recouvrer leurs droits dans la plupart des villes. Mais, grâce aux dissensions perpétuelles entre les corporations, grâce à la ténacité des patriciens et à la protection qu’ils trouvaient auprès de l’Empire et des gouvernements des villes alliées les conseillers patriciens arrivèrent bientôt à rétablir, en fait, leur ancienne autocratie, soit par la ruse, soit par la violence. Au début du XVIᵉ siècle, dans toutes les villes, la commune se trouvait de nouveau dans l’opposition.\par
———\par
\noindent Cette opposition contre le patriciat se divisait en deux fractions qui se manifestent très nettement au cours de la Guerre des paysans.\par
L’\emph{opposition bourgeoise}, la devancière de nos libéraux actuels, groupait les bourgeois riches et moyens, ainsi qu’une partie plus ou moins importante selon les localités dés petits bourgeois. Ses revendications se maintenaient exclusivement sur le terrain constitutionnel. Elle réclamait le droit de contrôle sur l’administration municipale et une participation au pouvoir législatif, soit par l’intermédiaire de l’assemblée communale elle-même, soit par l’intermédiaire d’une représentation communale (grand Conseil, municipalité) de plus, la limitation du despotisme patricien et de l’oligarchie d’un petit nombre de familles qui se manifestait de plus en plus nettement même à l’intérieur du patriciat. Tout au plus exigeait-elle, en outre, que quelques sièges au Conseil fussent occupés par des bourgeois pris dans son sein. Ce parti auquel se ralliait, çà et là, la fraction mécontente et déclassée du patriciat, constituait la grande majorité dans toutes les assemblées communales ordinaires et dans les corporations. Les partisans du Conseil et l’opposition plus radicale constituaient une infime minorité parmi les \emph{bourgeois} proprement dits.\par
Nous aurons l’occasion de voir comment, au cours du mouvement du XVIᵉ siècle, cette opposition « modérée » « légale », « aisée », « intelligente », joua exactement le même rôle, et exactement avec le même succès, que son héritier, le parti constitutionnel, dans le mouvement des années 1848 et 1849.\par
Pour le reste, l’opposition bourgeoise vitupérait encore très vigoureusement les prêtres, dont la vie de paresse et les mœurs relâchées soulevaient sa réprobation. Elle réclamait des mesures contre le genre de vie scandaleuse de ces dignes hommes. Elle demandait la suppression de la juridiction particulière et de l’exemption d’impôts dont jouissaient les prêtres et, d’une façon générale, la limitation du nombre des moines.\par
L’\emph{opposition plébéienne} se composait des bourgeois déclassés et de la masse des citadins privés des droits civiques : les compagnons, les journaliers et les nombreux éléments embryonnaires du Lumpenproletariat, cette racaille que l’on trouve même aux degrés les plus bas du développement des villes. Le Lumpenproletariat constitue d’ailleurs un phénomène qu’on retrouve plus ou moins développé dans presque toutes les phases de la société passée. La masse des gens sans gagne-pain bien défini ou sans domicile fixe était, précisément à cette époque, considérablement augmentée par la décomposition du féodalisme dans une société où chaque profession, chaque sphère de la vie était retranchée derrière une multitude de privilèges. Dans tous les pays développés, jamais le nombre de vagabonds n’avait été aussi considérable que dans la première moitié du XVIᵉ siècle. De ces vagabonds, les uns s’engageaient, pendant les périodes de guerre, dans les armées d’autres parcouraient le pays en mendiant d’autres enfin s’efforçaient, dans les villes, de gagner misérablement leur vie par des travaux à la journée ou d’autres occupations non accaparées par des corporations. Ces trois éléments jouent un rôle dans la Guerre des paysans : le premier, dans les armées des princes, devant lesquelles succombèrent les paysans le deuxième, dans les conjurations et les armées paysannes, où son influence démoralisante se manifeste à chaque instant le troisième, dans les luttes des partis citadins. Il ne faut d’ailleurs pas oublier qu’une grande partie de cette classe, surtout l’élément des villes, possédait encore à l’époque un fonds considérable de saine nature paysanne et était encore loin d’avoir atteint le degré de vénalité et de dépravation du Lumpenproletariat civilisé d’aujourd’hui.\par
On voit que l’opposition plébéienne des villes, à cette époque, se composait d’éléments très mélangés. Elle groupait les éléments déclassés de la vieille société féodale et corporative et les éléments prolétariens non développés encore, à peine embryonnaires, de la société bourgeoise moderne en train de naître. D’un côté, des artisans appauvris, liés encore à l’ordre bourgeois existant par les privilèges des corporations de l’autre, des paysans chassés de leurs terres et des gens de service licenciés qui ne pouvaient pas encore se transformer en prolétaires. Entre eux les compagnons, placés momentanément en dehors de la société officielle et qui, par leurs conditions d’existence, se rapprochaient du prolétariat autant que le permettaient l’industrie de l’époque et les privilèges des corporations, mais qui en même temps étaient presque tous de futurs maîtres et de futurs bourgeois, en raison précisément de ces privilèges. C’est pourquoi la position politique de ce mélange d’éléments divers était nécessairement très peu sûre et différente selon les localités. Jusqu’à la Guerre des paysans, l’opposition plébéienne ne participe pas aux luttes politiques en tant que parti. Elle ne se manifeste que comme prolongement de l’opposition bourgeoise, appendice bruyant, avide de pillages, se vendant pour quelques tonneaux de vin. Ce sont les soulèvements des paysans qui la transforment en un parti, et même alors elle reste presque partout, dans ses revendications et dans son action, dépendante des paysans – ce qui prouve de façon curieuse à quel point les villes dépendaient encore à cette époque de la campagne. Dans la mesure où elle a une attitude indépendante, elle réclame l’établissement des monopoles industriels de la ville à la campagne, s’oppose à la réduction des revenus de la ville par la suppression des charges féodales pesant sur les paysans de la banlieue, etc. en un mot, dans cette mesure elle est réactionnaire, se subordonne à ses propres éléments petits-bourgeois, fournissant ainsi un prélude caractéristique à la tragi-comédie que joue depuis trois ans, sous la raison sociale de la démocratie, la petite bourgeoisie moderne.\par
Ce n’est qu’en Thuringe, sous l’influence directe de Münzer, et en divers autres lieux, sous celle de ses disciples, que la fraction plébéienne des villes fut entraînée par la tempête générale au point que l’élément prolétarien embryonnaire l’emporta momentanément sur toutes les autres fractions du mouvement. Cet épisode, qui constitue le point culminant de toute la Guerre des paysans et se ramasse autour de sa figure la plus grandiose, celle de \emph{Thomas Münzer}, est en même temps le plus court. Il est compréhensible que cet élément devait s’effondrer le plus rapidement, revêtir un caractère surtout fantastique, et que l’expression de ses revendications devait rester extrêmement confuse, car c’est lui qui rencontrait, dans les conditions de l’époque, le terrain le moins solide.\par
Au-dessous de toutes ces classes, à l’exception de la dernière, se trouvait la grande masse exploitée de la nation : les \emph{paysans}. C’est sur eux que pesait toute la structure des couches sociales : princes, fonctionnaires, nobles, curés, patriciens et bourgeois. Qu’il appartint à un prince, à un baron d’Empire, à un évêque, à un monastère ou à une ville, le paysan était partout traité comme une chose, comme une bête de somme, et même souvent pis. Serf, son maître pouvait disposer de lui à sa guise. Corvéable, les prestations légales contractuelles suffisaient déjà à l’écraser, mais ces prestations elles-mêmes s’accroissaient de jour en jour. La plus grande partie de son temps, il devait l’employer à travailler sur les terres de son maître. Sur ce qu’il gagnait dans ses rares heures disponibles, il devait payer cens, dîmes, redevances, taille, viatique (impôt militaire), impôts d’État et taxes d’Empire. Il ne pouvait ni se marier, ni même mourir sans payer un droit à son maître. Outre les corvées féodales ordinaires, il devait pour son maître récolter la paille, les fraises, les myrtilles, ramasser des escargots, rabattre le gibier à la chasse, fendre du bois, etc. Le droit de pêche et de chasse appartenait au maître, et le paysan devait regarder tranquillement le gibier détruire sa récolte. Les pâturages et les bois communaux des paysans leur avaient été presque partout enlevés de force par les seigneurs. Et de même qu’il disposait de la propriété, le seigneur disposait à son gré de la personne du paysan, de celle de sa femme et de ses filles. Il avait le droit de cuissage. Il pouvait, quand il voulait, faire jeter le paysan en prison, où la torture l’attendait aussi sûrement qu’aujourd’hui le juge d’instruction. Il le faisait assommer ou décapiter, selon son bon plaisir. De ces édifiants chapitres de la \emph{Carolina}, qui traitaient de la façon de « couper les oreilles », de « couper le nez », « crever les yeux », de « trancher les doigts et les mains », de « décapiter », de « rouer », de « brûler », de « pincer avec des tenailles brûlantes », d’ « écarteler », etc., il n’en est pas un seul que les nobles seigneurs et protecteurs n’aient employé à leur gré contre les paysans. Qui les aurait défendus ? Dans les tribunaux siégeaient des barons, des prêtres, des patriciens ou des juristes, qui savaient parfaitement pour quel travail ils étaient payés. Car tous les ordres officiels de l’Empire vivaient de l’exploitation des paysans.\par
Cependant, quoique grinçant des dents sous le joug qui les accablait, les paysans étaient très difficiles à soulever. Leur dispersion leur rendait extrêmement difficile tout entente commune. La longue accoutumance de générations successives à la soumission, la perte de l’habitude de l’usage des armes dans beaucoup de régions, la dureté de l’exploitation, tantôt atténuant, tantôt s’aggravant, selon la personne des seigneurs, contribuaient à maintenir les paysans dans le calme. C’est pourquoi on trouve au moyen âge quantité de révoltes paysannes locales, mais, en Allemagne tout au moins, on ne trouve pas avant la Guerre des paysans une seule insurrection générale nationale de la paysannerie. Il faut ajouter à cela que les paysans n’étaient pas capables à eux seuls de faire une révolution tant qu’ils trouvaient en face d’eux le bloc de la puissance organisée des princes, de la noblesse et des villes, unis en une alliance solide. Seule une alliance avec d’autres ordres pouvait leur donner une chance de vaincre, mais comment s’allier avec d’autres, quand tous les exploitaient également ?\par
Ainsi au début du XVIᵉ siècle les différents ordres de l’Empire : princes, noblesse, prélats, patriciens, bourgeois, plébéiens et paysans, composaient une masse extrêmement confuse, aux besoins les plus variés et se contrecarrant en tous sens. Chaque ordre faisait obstacle à l’autre, et était engagé dans une lutte permanente, tantôt ouverte, tantôt masquée, avec tous les autres. Cette division de toute la nation en deux grands camps, qui existait au début de la première Révolution française et que l’on constate maintenant aussi à une étape supérieure du développement dans les pays les plus avancés, était purement impossible dans les conditions d’alors. Elle ne pouvait s’établir, même très approximativement, que lorsque la couche inférieure de la nation, les paysans et les plébéiens, exploitée par tous les autres ordres, se souleva. On se rendra facilement compte de la confusion des intérêts des vues et des aspirations de cette époque, si on se rappelle quelle confusion a provoquée au cours de ces deux dernières années la composition actuelle, beaucoup moins complexe cependant de la nation allemande en noblesse féodale, bougeoisie, petite bourgeoisie, paysannerie et prolétariat.
\chapterclose


\chapteropen
\renewcommand{\leftmark}{II. Les grands groupements d’opposition et leurs idéologies. Luther et Münzer }
\chapter[II. Les grands groupements d’opposition et leurs idéologies. Luther et Münzer]{II. Les grands groupements d’opposition et leurs idéologies. Luther et Münzer }

\chaptercont
\noindent Le groupement des ordres alors si multiples en unités plus importantes était déjà presque impossible à cause de la décentralisation et l’indépendance locale et provinciale, l’isolement commercial et industriel des différentes provinces entre elles, les communications difficiles. Ce groupement ne se constitue qu’avec la diffusion générale d’idées révolutionnaires, religieuses et politiques sous la Réforme. Les différents ordres qui adhèrent à ces idées ou les rejettent concentrent la nation, à vrai dire d’une façon tout à fait malaisée et approximative, en trois grands camps : le camp catholique ou réactionnaire, le camp luthérien bourgeois-réformateur et le camp révolutionnaire. Si dans ce grand morcellement de la nation on trouve peu de logique, si l’on rencontre parfois les mêmes éléments dans les deux premiers camps, cela s’explique par l’état de décomposition dans lequel se trouvaient, à cette époque, la plupart des ordres officiels hérités du moyen âge, et par la décentralisation qui, dans des régions différentes, poussait momentanément les mêmes ordres dans des directions opposées. Nous avons eu si souvent l’occasion, au cours de ces dernières années en Allemagne de constater des phénomènes analogues qu’un tel pêle-mêle apparent des ordres et des classes dans les conditions beaucoup plus complexes du XVIᵉ siècle ne saurait nous étonner.\par
Malgré les expériences récentes, l’idéologie allemande continue à ne voir dans les luttes auxquelles a succombé le moyen âge que de violentes querelles théologiques. Si les gens de cette époque avaient seulement pu s’entendre au sujet des choses célestes, ils n’auraient eu, de l’avis de nos historiens et hommes d’État nationaux, aucune raison de se disputer sur les choses de ce monde. Ces idéologues sont assez crédules pour prendre pour argent comptant toutes les illusions qu’une époque se fait sur elle-même, ou que les idéologues d’une époque se font sur elle. Cette sorte de gens ne voient, par exemple, dans la Révolution de 1789 qu’un débat un peu bouillant sur les avantages de la monarchie constitutionnelle par rapport à la monarchie absolue dans la Révolution de juillet, qu’une controverse pratique sur l’impossibilité de défendre le droit « divin » ; dans la Révolution de février, qu’une tentative de résoudre la question « république ou monarchie », etc. Les \emph{luttes de classes} qui se poursuivent à travers tous ces bouleversements, et dont la phraséologie politique inscrite sur les drapeaux des parties en lutte n’est que l’expression, ces luttes entre classes, nos idéologues aujourd’hui encore, les soupçonnent à peine, quoique la nouvelle non seulement leur en vienne assez distinctement de l’étranger mais retentisse aussi dans le grondement et la colère de milliers et de milliers de prolétaires de chez nous.\par
Même dans ce que l’on appelle les guerres de religion du XVIᵉ siècle, il s’agissait avant tout de très positifs intérêts matériels de classes, et ces guerres étaient des luttes de classes, tout autant que les collisions intérieures qui se produisirent plus tard en Angleterre et en France. Si ces luttes de classes portaient, à cette époque, un signe de reconnaissance religieux, si les intérêts, les besoins, les revendications des différentes classes se dissimulaient sous le masque de la religion, cela ne change rien à l’affaire et s’explique facilement par les conditions de l’époque.\par
Le moyen âge était parti des tout premiers éléments. De la civilisation, de la philosophie, de la politique et de la jurisprudence antiques, il avait fait table rase pour tout recommencer par le début. Il n’avait repris du vieux monde disparu que le christianisme, ainsi qu’un certain nombre de villes à demi-détruites, dépouillées de toute leur civilisation. Le résultat fut que, de même qu’à toutes les étapes primitives de développement, les prêtres reçurent le monopole de la culture intellectuelle, et la culture elle-même prit un caractère essentiellement théologique. Entre les mains des prêtres, la politique et la jurisprudence restèrent, comme toutes les autres sciences, de simples branches de la théologie et furent traitées d’après les principes en vigueur dans celle-ci. Les dogmes de l’Église étaient en même temps des axiomes politiques, et les passages de la Bible avaient force de loi devant tous les tribunaux. Même lorsque se constitua une classe indépendante de juristes, la jurisprudence resta longtemps encore sous la tutelle de la théologie. Et cette souveraineté de la théologie dans tout le domaine de l’activité intellectuelle était en même temps la conséquence nécessaire de la situation de l’Église, synthèse la plus générale et sanction de la domination féodale régnante.\par
Il est clair que toutes les attaques dirigées en général contre le féodalisme devaient être avant tout des attaques contre l’Église, toutes les doctrines révolutionnaires sociales et politiques devaient être en même temps et principalement des hérésies théologiques. Pour pouvoir toucher aux conditions sociales existantes, il fallait leur enlever leur auréole sacrée.\par
L’opposition révolutionnaire contre la féodalité se poursuit pendant tout le moyen âge. Elle apparaît, selon les circonstances, tantôt sous forme de mystique, tantôt sous forme d’hérésie ouverte, tantôt sous forme d’insurrection armée. En ce qui concerne la mystique, on sait à quel point en dépendaient les réformateurs du XVIᵉ siècle. Münzer lui-même lui doit beaucoup. Les hérésies étaient soit l’expression de la réaction des bergers des Alpes, aux habitudes de vie patriarcales, contre la féodalité pénétrant jusqu’à eux (les Vaudois), soit l’expression de l’opposition au féodalisme des villes qui lui avaient échappé (Albigeois, Arnaud de Brescia, etc.), soit directement des insurrections paysannes (John Ball, le maître de Hongrie en Picardie, etc.). Nous pouvons laisser ici de côté les hérésies patriarcales des Vaudois ainsi que l’insurrection des Suisses, comme étant, d’après leur forme et leur contenu, des tentatives réactionnaires de s’opposer au mouvement de l’histoire et comme n’ayant qu’une importance locale. Dans les deux autres formes d’hérésie moyenâgeuse, nous trouvons au XIIᵉ siècle les précurseurs du grand antagonisme entre l’opposition bourgeoise et l’opposition paysanne-plébéienne, qui fut la principale cause de l’échec de la Guerre des paysans. Cet antagonisme se poursuit à travers toute la fin du moyen âge.\par
L’hérésie des villes – et c’est l’hérésie à proprement parler officielle du moyen âge – se tournait principalement contre les prêtres, dont elle attaquait les richesses et la position politique. De même que la bourgeoisie réclame maintenant un gouvernement à bon marché, de même les bourgeois du moyen âge réclamaient une Église à bon marché. Réactionnaire dans sa forme, comme toute hérésie qui ne voit dans le développement de l’Église et des dogmes qu’une dégénérescence, l’hérésie bourgeoise réclamait le rétablissement de la constitution simple de l’Église primitive et la suppression de l’ordre exclusif du clergé. Cette institution à bon marché aurait eu pour résultat de supprimer les moines, les prélats, la cour romaine, bref, tout ce qui coûtait cher dans l’Église. Etant elles-mêmes des républiques, bien qu’elles étaient placées sous la protection de monarques, les villes par leurs attaques contre la papauté exprimaient pour la première fois sous une forme générale cette vérité que la forme normale de la domination de la bourgeoisie, c’est la république. Leur opposition à toute une série de dogmes et de lois de l’Église s’explique en partie par ce qui précède, en partie par leurs autres conditions d’existence. Pourquoi, par exemple, elles s’élevaient si violemment contre le célibat des prêtres, nul ne l’explique mieux que Boccace. Arnaud de Brescia en Italie et en Allemagne, les Albigeois dans le Midi de la France, John Wyclif en Angleterre, Hus et les calixtins en Bohème, furent les principaux représentants de cette tendance. Si l’opposition au féodalisme ne se manifeste ici que comme opposition à la féodalité \emph{ecclésiastique}, la raison en est tout simplement que partout les villes constituaient déjà un ordre reconnu, et qu’elles avaient avec leurs privilèges, leurs armes ou dans les assemblées des états, des moyens suffisants pour lutter contre la féodalité laïque.\par
Ici aussi, nous voyons déjà, tant dans le Midi de la France qu’en Angleterre et en Bohème, la plus grande partie de la petite noblesse s’allier aux villes dans la lutte contre les prêtres et dans l’hérésie – phénomène qui s’explique par la dépendance de la petite noblesse à l’égard des villes et par sa solidarité d’intérêts avec ces dernières contre les princes et les prélats nous le retrouverons dans la Guerre des paysans.\par
Tout autre était le caractère de l’hérésie qui était l’expression directe des besoins des paysans et plébéiens, et qui était presque toujours liée à une insurrection. Elle comportait, certes, toutes les revendications de l’hérésie bourgeoise concernant les prêtres, la papauté et le rétablissement de la constitution de l Église primitive, mais elle allait aussi infiniment plus loin. Elle voulait que les conditions d’égalité du christianisme primitif soient rétablies entre les membres de la communauté et reconnues également comme norme pour la société civile. De « l’égalité des enfants de Dieu », elle faisait découler l’égalité civile, et même en partie déjà l’égalité des fortunes. Mise sur pied d’égalité de la noblesse et des paysans, des patriciens, des bourgeois privilégiés et des plébéiens, suppression des corvées féodales, du cens, des impôts, des privilèges et en tout cas des différences de richesse les plus criantes, telles étaient les revendications posées avec plus ou moins de netteté et soutenues comme découlant nécessairement de la doctrine chrétienne primitive. Cette hérésie paysanne-plébéienne, qu’il était encore difficile, à l’époque de l’apogée du féodalisme, par exemple chez les Albigeois, de séparer de l’hérésie bourgeoise, se transforme, au XIVᵉ et au XVᵉ siècle en un point de vue de parti nettement distinct, et apparaît habituellement de façon tout à fait indépendante à côté de l’hérésie bourgeoise. Tel fut John Ball, le prédicateur de l’insurrection de Wat Tyler en Angleterre, à côté du mouvement de Wyclif tels les Taborites, à côté des calixtins en Bohème. Chez les Taborites, la tendance républicaine apparaissait déjà sous les enjolivures théocratiques, tendance qui fut développée à la fin du XVᵉ et au début du XVIᵉ siècle par les représentants des plébéiens en Allemagne.\par
À cette forme d’hérésie se rattache l’exaltation des sectes mystiques, flagellants, lollards, etc. qui, pendant les périodes de réaction, perpétuent la tradition révolutionnaire.\par
Les plébéiens constituaient, à l’époque, la seule classe placée tout à fait en dehors de la société officielle. Ils étaient en dehors de l’association féodale comme de l’association bourgeoise. Ils n’avaient ni privilèges ni propriété, et ne possédaient même pas, comme les paysans et les petits bourgeois, un bien, fût-il grevé de lourdes charges. Ils étaient sous tous les rapports sans bien et sans droits. Leurs conditions d’existence ne les mettaient même pas en contact direct avec les institutions existantes, qui les ignoraient complètement. Ils étaient le symptôme vivant de la décomposition de la société féodale et corporative bourgeoise, et en même temps les précurseurs de la société bourgeoise moderne.\par
C’est cette situation qui explique pourquoi, dès cette époque, la fraction plébéienne ne pouvait pas se limiter à la simple lutte contre le féodalisme et la bourgeoisie privilégiée : elle devait, du moins en imagination, dépasser la société bourgeoise moderne qui pointait à peine. Elle explique pourquoi cette fraction, exclue de toute propriété, devait déjà mettre en question des institutions, des conceptions et des idées qui sont communes à toutes les formes de société reposant sur les antagonismes de classe. Les exaltations chiliastiques du christianisme primitif offraient pour cela un point de départ commode. Mais, en même temps, cette anticipation par-delà non seulement le présent, mais même l’avenir ne pouvait avoir qu’un caractère violent, fantastique, et devait, à la première tentative de réalisation pratique, retomber dans les limites restreintes imposées par les conditions de l’époque. Les attaques contre la propriété privée, la revendication de la communauté des biens devaient se résoudre en une organisation grossière de bienfaisance. La vague égalité chrétienne pouvait, tout au plus, aboutir à « l’égalité civile devant la loi » ; la suppression de toute autorité devient, en fin de compte, la constitution de gouvernements républicains élus par le peuple. L’anticipation en imagination du communisme était en réalité une anticipation des conditions bourgeoises modernes.\par
Cette anticipation de l’histoire ultérieure, violente, mais cependant très compréhensible étant donné les conditions d’existence de la fraction plébéienne, nous la rencontrons tout d’abord en \emph{Allemagne}, chez \emph{Thomas Münzer} et ses partisans. Il y avait bien eu déjà, chez les Taborites, une sorte de communauté millénariste des biens, mais seulement comme une mesure d’ordre exclusivement militaire. Ce n’est que chez Münzer que ces résonances communistes deviennent l’expression des aspirations d’une fraction réelle de la société. C’est chez lui seulement qu’elles sont formulées avec une certaine netteté, et après lui nous les retrouvons dans chaque grand soulèvement populaire, jusqu’à ce qu’elles se fondent peu à peu avec le mouvement prolétarien moderne tout comme au moyen âge les luttes des paysans libres contre la féodalité, qui les enserre de plus en plus dans ses filets, se fondent avec les luttes des serfs et des corvéables pour le renversement complet de la domination féodale.\par
Tandis que le premier des trois grands camps entre lesquels se divisait la nation, le camp \emph{conservateur-catholique}, groupait tous les éléments intéressés au maintien de l’ordre existant : pouvoir d’Empire, clergé et une partie des princes séculiers, noblesse riche, prélats et patriciat des villes, sous la bannière de la Réforme luthérienne \emph{bourgeoise-modérée} se rassemblent les éléments possédants de l’opposition, la masse de la petite noblesse, la bourgeoisie, et même une partie des princes séculiers, qui espéraient s’enrichir par la confiscation des biens de l’Église et voulaient profiter de l’occasion pour conquérir une indépendance plus grande à l’égard de l’Empire. Enfin, les paysans et les plébéiens constituaient le parti \emph{révolutionnaire}, dont les revendications et les doctrines furent exprimées avec le plus d’acuité par Thomas Münzer.\par
Tant d’après leurs doctrines que d’après leur caractère et leur action, \emph{Luther} et Münzer représentent totalement le parti que chacun d’eux dirigeait.\par
De 1517 à 1525, Luther a connu exactement la même évolution que les constitutionalistes allemands modernes de 1848 à 1849 et que connaît chaque parti bourgeois qui, après avoir été un moment à la tête du mouvement, se voit dans ce mouvement lui-même débordé par le parti plébéien ou prolétarien qui le soutenait jusqu’alors.\par
Lorsque, en 1517, Luther attaqua tout d’abord les dogmes et la constitution de l’Église catholique, son opposition n’avait pas encore de caractère bien déterminé. Sans dépasser les revendications de l’ancienne hérésie bourgeoise, elle n’excluait aucune tendance plus radicale et ne le pouvait d’ailleurs pas. Car il fallait unir tous les éléments d’opposition, déployer l’énergie la plus résolument révolutionnaire et représenter l’ensemble de l’hérésie antérieure en face de l’orthodoxie catholique. C’est précisément en ce sens que nos libéraux bourgeois étaient encore révolutionnaires en 1847, qu’ils se disaient socialistes et communistes et s’enthousiasmaient pour l’émancipation de la classe ouvrière. La forte nature paysanne de Luther se manifesta au cours de cette première période de son activité de la manière la plus impétueuse.\par

\begin{quoteblock}
 \noindent « Si le déchaînement de leur furie devait continuer, écrivait-il en parlant des prêtres romains, il me semble qu’il n’y aurait certes meilleur moyen et remède pour le faire cesser que de voir les rois et les princes intervenir par la violence, attaquer cette engeance néfaste qui empoisonne le monde et mettre fin à leur entreprise par les armes et non par la parole. De même que nous châtions les voleurs par la corde, les assassins par l’épée, les hérétiques par le feu, pourquoi n’attaquons-nous pas plutôt ces néfastes professeurs de ruine, les papes, les cardinaux, les évêques et toute la horde de la Sodome romaine, avec toutes les armes dont nous disposons, et ne lavons-nous pas nos mains dans leur sang ? »
\end{quoteblock}

\noindent Mais cette première ardeur révolutionnaire ne dura pas longtemps. La foudre que Luther avait lancée porta. Le peuple allemand tout entier se mit en mouvement. D’une part, les paysans et les plébéiens virent dans ses appels à la lutte contre les prêtres, dans ses prédications sur la liberté chrétienne le signal de l’insurrection de l’autre, les bourgeois modérés et une grande partie de la petite noblesse se rallièrent à lui, entraînant même avec eux un certain nombre de princes. Les uns crurent le moment venu de régler leurs comptes avec tous leurs oppresseurs les autres désiraient seulement mettre un terme à la toute-puissance des prêtres, à la dépendance vis-à-vis de Rome et de la hiérarchie catholique et s’enrichir grâce à la confiscation des biens de l’Église. Les partis se séparèrent et trouvèrent leur porte-parole. Luther eut à choisir entre ces partis. Protégé de l’électeur de Saxe, éminent professeur de l’université de Wittenberg, ayant acquis du jour au lendemain notoriété et puissance, entouré d’un cercle de créatures à sa dévotion et de flatteurs, ce grand homme n’hésita pas une minute. Il laissa tomber les éléments populaires du mouvement et rallia le parti de la noblesse, de la bourgeoisie et des princes. Les appels à la guerre d’extermination contre Rome s’éteignirent. Luther prêchait maintenant \emph{l’évolution pacifique} et la \emph{résistance passive} (voir par exemple l’appel à la noblesse allemande, 1520, etc.).\par
À l’invitation qui lui fut faite par Ulrich von Hutten de se rendre auprès de lui et de Sickingen, à Ebernbourg, centre de la conjuration de la noblesse contre le clergé et les princes, Luther répondit :\par

\begin{quoteblock}
 \noindent « Je ne suis pas pour que l’on gagne la cause de l’Évangile par la violence et les effusions de sang. C’est par la parole que le monde a été vaincu, c’est par la parole que l’Église s’est maintenue, c’est par la parole qu’elle sera remise en état, et de même que l’Antéchrist s’en est emparé sans violence, il tombera aussi sans violence. »
\end{quoteblock}

\noindent C’est du jour où la tendance de Luther prit cette tournure, ou plutôt se fixa de cette façon plus précise, que datent ces tractations autour des institutions et des dogmes à conserver ou à réformer, ce manège répugnant de diplomatie, de concessions, d’intrigues et d’accords, qui aboutit à la Confession d’Augsbourg, la constitution, enfin acquise au prix de marchandages de l’Église bourgeoise réformée. C’est exactement le même trafic sordide qui s’est répété récemment, et jusqu’à l’écœurement, sous la forme politique dans les assemblées nationales allemandes, les assemblées d’entente, les chambres de révision et autres Parlements d’Erfurt. C’est au cours de ces négociations que se manifesta le plus ouvertement le caractère petit-bourgeois de la Réforme officielle.\par
Que Luther, désormais représentant déclaré de la Réforme bourgeoise, prêchât le progrès dans le cadre de la loi, il y avait à cela de bonnes raisons. La plupart des villes s’étaient prononcées en faveur de la Réforme modérée, la petite noblesse s’y ralliait de plus en plus. Une partie des princes y adhéra, l’autre hésitait. Son succès était autant dire assuré, du moins dans une grande partie de l’Allemagne. Si les choses continuaient à se développer pacifiquement, les autres régions ne pouvaient pas, à la longue, résister à la poussée de l’opposition modérée. Mais tout ébranlement violent devait mettre le parti modéré en conflit avec le parti extrême des plébéiens et des paysans, éloigner du mouvement les princes, la noblesse et un certain nombre de villes, et ne laisser finalement d’autre alternative que le débordement du parti bourgeois par le parti paysan et plébéien ou l’écrasement de tous les partis du mouvement par la restauration catholique. Et de quelle façon les partis bourgeois, dès qu’ils ont obtenu le moindre succès, s’efforcent au moyen du progrès dans le cadre de la loi de louvoyer entre le Scylla de la révolution et le Charybde de la restauration, c’est ce que les événements récents nous ont suffisamment montré.\par
Comme par suite des conditions générales, sociales et politiques de l’époque, les résultats de toute transformation devaient nécessairement profiter aux princes, la Réforme bourgeoise devait tomber de plus en plus sous le contrôle des princes réformés, au fur et à mesure qu’elle se séparait plus nettement des éléments plébéiens et paysans. Luther lui-même devint de plus en plus leur valet, et le peuple savait très bien ce qu’il faisait, lorsqu’il l’accusait d’être devenu un courtisan comme les autres, et lorsqu’il le pourchassait comme à Orlamunde à coups de pierres.\par
Lorsque la Guerre des paysans éclata, et qui plus est dans des régions où les princes et la noblesse étaient en majorité catholiques, Luther s’efforça de jouer un rôle de médiateur. Il attaqua résolument les gouvernements. Il déclara qu’ils étaient responsables de l’insurrection par leurs vexations. Ce n’étaient pas les paysans qui se levaient contre eux, c’était Dieu lui-même. Cependant la révolte elle aussi était impie et contraire aux préceptes de l’évangile. Finalement, il conseilla aux deux parties adverses de céder et de conclure un accord amiable.\par
Mais l’insurrection s’étendit rapidement, malgré ces propositions bien intentionnées de médiation, elle s’étendit même à des régions protestantes qui se trouvaient sous l’autorité de princes, de nobles et de villes luthériens et dépassa rapidement la « prudente » Réforme bourgeoise. C’est dans le proche voisinage de Luther, en Thuringe, que la fraction la plus décidée des insurgés, sous la direction de Münzer, établit son quartier général. Encore quelques succès, et toute l’Allemagne était en flammes, Luther était encerclé, peut-être éxécuté à coups de pique comme traître, et la Réforme bourgeoise emportée par le raz de marée de la révolution plébéienne et paysanne. Il n’y avait donc plus à hésiter. En face de la révolution toutes les vieilles inimitiés furent oubliées. En comparaison des armées paysannes, les valets de la Sodome romaine étaient des agneaux innocents, de doux enfants de Dieu. Bourgeois et princes, noblesse et clergé, Luther et le pape s’unirent « contre les armées paysannes, pillardes et tueuses ».\par

\begin{quoteblock}
 \noindent « Il faut les mettre en pièces, les étrangler, les égorger, en secret et publiquement, comme on abat des chiens enragés ! s’écria Luther. C’est pourquoi, mes chers seigneurs, égorgez-les, abattez-les, étranglez-les, libérez ici, sauvez là ! Si vous tombez dans la lutte, vous n’aurez jamais de mort plus sainte ! »
\end{quoteblock}

\noindent Pas de fausse pitié pour les paysans ! Ils se mêlent eux-mêmes au insurgés, ceux qui ont pitié de ceux dont Dieu lui-même n’a pas pitié, mais qu’il veut au contraire punir et anéantir. Après, les paysans apprendront eux-mêmes à remercier Dieu, s’ils sont obligés de céder une de leurs vaches pour pouvoir garder l’autre en paix et l’insurrection montrera au princes quel est l’esprit du peuple, qu’on ne peut gouverner que par la force.\par

\begin{quoteblock}
 \noindent « Le sage dit : Cibus, onus et virgam asino (*). Qu’on donne de la paille d’avoine aux paysans ils n’entendent point les paroles de Dieu, ils sont stupides c’est pourquoi il faut leur faire entendre le fouet, l’arquebuse et c’est bien fait pour eu. Prions pour eux qu’ils obéissent. Sinon, pas de pitié ! Faites parler les arquebuses, sinon ce sera bien pis. »
\end{quoteblock}

———\par
\noindent L’ane a besoin de nourriture, fardeau et fouet.\par
C’est exactement ainsi que parlèrent nos ci-devant bourgeois socialistes et philanthropes lorsque le prolétariat, au lendemain des journées de mars, vint réclamer sa part des fruits de la victoire.\par
Avec sa traduction de la Bible, Luther avait donné au mouvement plébéien une arme puissante. Dans la Bible, il avait opposé au christianisme féodalisé de l’époque l’humble christianisme des premiers siècles à la société féodale en décomposition, le tableau d’une société qui ignorait la vaste et ingénieuse hiérarchie féodale. Les paysans avaient utilisé cet outil en tous sens contre les princes, la noblesse et le clergé. Maintenant, Luther le retournait contre eux et tirait de la Bible un véritable hymne aux autorités établies par Dieu, tel que n’en composa jamais aucun lèche-bottes de la monarchie absolue ! Le pouvoir princier de droit divin, l’obéissance passive, même le servage trouvèrent leur sanction dans la Bible. Ainsi se trouvaient reniées non seulement l’insurrection des paysans, mais toute la révolte de Luther contre les autorités spirituelles et temporelles. Ainsi étaient trahis, au profit des princes, non seulement le mouvement populaire, mais même le mouvement bourgeois.\par
Est-il nécessaire de nommer les bourgeois qui, eux aussi nous ont récemment donné une fois de plus des exemples de ce reniement de leur propre passé ?\par
Opposons maintenant au réformateur bourgeois Luther le révolutionnaire plébéien Münzer.\par
\emph{Thomas Münzer} était né à \emph{Stolberg} près du Hartz, vers l’année 1498. Son père était mort pendu, victime de l’arbitraire des comtes de Stolberg. Dés sa quinzième année, Münzer fonda à l’école, à Halle une ligue secrète contre l’archevêque de Magdebourg et l’Église romaine en général. Sa connaissance profonde de la théologie de l’époque lui permit d’obtenir de bonne heure le grade de docteur et une place de chapelain dans un couvent de religieuses à Halle. Il y traitait déjà avec le plus grand mépris les dogmes et les rites de l’Église, supprimait complètement à la messe les paroles de la transsubstantiation, et, ainsi que le rapporte Luther, avalait les hosties non consacrées. Il étudiait principalement les mystiques du moyen âge, en particulier, les écrits millénaristes de Joachim le Calabrais. L’heure du millénium, de la condamnation de l’Église dégénéré et du monde corrompu, que cet écrivain annonce et dépeint, sembla à Münzer être venue avec la Réforme et l’agitation générale de l’époque. Il prêcha dans la région avec beaucoup de succès. En 1520, il alla comme premier prédicateur évangélique à Zwickau. Là, il trouva une de ces sectes millénaristes exaltées qui continuaient à vivre dans le silence dans un grand nombre de régions, et derrière la modestie et la réserve momentanée desquelles s’était cachée l’opposition grandissante des couches sociales inférieures contre l’état de choses existant maintenant, avec l’agitation croissante, elles se produisaient d’une manière de plus en plus ouverte et opiniâtre. C’était la secte des anabaptistes, à la tête de laquelle se trouvait \emph{Niklas Storch}. Ils prêchaient l’approche du Jugement dernier et du millénium ils avaient « des visions, des extases et l’esprit de prophétie ». Ils entrèrent rapidement en conflit avec le Conseil de Zwickau. Münzer les défendit sans jamais se rallier complètement à eux, mais en les gagnant de plus en plus à son influence. Le Conseil intervint énergiquement contre eux ils durent quitter la ville, et Münzer avec eux. C’était à la fin de 1521.\par
Münzer se rendit à Prage et s’efforça d’y prendre pied en partant des restes du mouvement hussite, mais sa proclamation n’eut d’autre résultat que de l’obliger à fuir encore de Bohème. En 1522, il fut nommé prédicateur à Allstedt, en Thuringe. Là il commença par réformer le culte. Avant même que Luther osât aller jusque-là, il supprima complètement l’emploi du latin et fit lire toute la Bible, et non pas seulement les Evangiles et les épîtres préscrites aux offices du dimanche. En même temps il organisa la propagande dans la région. Le peuple accourut à lui de tous côtés, et bientôt Allstedt devint le centre du mouvement populaire anticlérical.\par
À cette époque, Münzer était encore avant tout théologien ses attaques étaient encore presque exclusivement dirigées contre les prêtres. Mais il ne prêchait pas, comme Luther le faisait déjà, les discussions paisibles et l’évolution pacifique. Il continuait les anciens prêches violents de Luther et appelait les princes saxons et le peuple à la lutte armée contre les prêtres romains.\par
« Le Christ ne dit-il pas : je ne suis pas venu vous apporter la paix, mais l’épée ? Mais qu’allez-vous [princes saxons] en faire ? L’employer à supprimer et à anéantir les méchants qui font obstacle à l’Évangile, si vous voulez être de bons serviteurs de Dieu. Le Christ à très solennellement ordonné (Saint Luc, 19,27) : saissez-vous de mes ennemies et étranglez-les devant mes yeux… Ne nous objectez pas ces \\
fades niaiseries que la puissance de Dieu le fera sans le secours de votre épée autrement elle pourrait se rouiller dans le fourreau. Car ceux qui sont opposés à la révélation de Dieu, il faut les exterminer sans merci de même qu’Ezéchias, Cyrus, Josias, Daniel et Elie ont exterminé les prêtres de Baal. Il n’est pas possible autrement de faire revenir l’Église chrétienne à son origine. Il faut arracher les mauvaises herbes des vignes de Dieu à l’époque de la récolte. Dieu a dit (Moise, 5,7) : vous ne devez pas avoir pitié des idolâtres. Détruisez leurs autels, brisez leurs images et brûlez-les afin que mon courroux ne s’abatte pas sur vous ! »\par
Mais ces appels aux princes n’eurent aucun résultat, alors que la Sèvre révolutionnaire croissait de jour en jour dans le peuple. Münzer dont les idées élaborées avec de plus en plus d’acuité devenaient chaque jour plus hardies, se sépara résolument de la Réforme bourgeoise et joua désormais directement le rôle d’un agitateur politique.\par
Sa doctrine théologique et philosophique attaquait somme toute, tous les points fondamentaux non seulement du catholicisme, mais aussi du christianisme. Il enseignait sous des formes chrétiennes, un panthéisme qui présente une ressemblance curieuse avec les conditions spéculatives modernes et frise même par moments l’athéisme. Il rejetait la Bible comme révélation tant unique qu’infaillible. La véritable révélation vivante c’est, disait Münzer, la raison, révélation qui a existé de tous temps et chez tous les peuples et qui existe encore. Opposer la Bible à la raison, c’est tuer l’esprit par la lettre. Car le Saint-Esprit dont parle la Bible n’existe pas en dehors de nous. Le Saint-Esprit, c’est précisément la raison. La foi n’est pas autre chose que l’incarnation de la raison dans l’homme et c’est pourquoi les païens peuvent aussi avoir la foi. C’est cette foi, c’est la raison devenue vivante qui divinise l’homme et le rend bienheureux. C’est pourquoi le ciel n’est pas quelque chose de l’au-delà, c’est dans cette vie même qu’il faut le chercher et la vocation des croyants est précisément d’établir ce ciel, le royaume de Dieu, sur la terre. De même qu’il n’existe pas de ciel dans l’au-delà, de même il n’y existe pas d’enfer ou de damnation. De même, il n’y a d’autre diable que les désirs et les appétits mauvais des hommes. Le Christ a été un homme comme les autres, un prophète et un maître, et la cène a été un simple repas commémoratif, où le pain et le vin étaient consommés sans rien y ajouter de mystique.\par
Münzer enseignait cette doctrine en la dissimulant la plupart du temps sous les mêmes tournures chrétiennes, sous lesquelles la philosophie moderne à du se cacher pendant un certain temps. Mais la pensée profondément hérétique ressort partout de ses écrits, et l’on s’aperçoit qu’il prenait beaucoup moins au sérieux le masque biblique que maints disciples de Hegel aujourd’hui. Et cependant, trois cents ans séparent Münzer de la philosophie moderne.\par
Sa doctrine politique se rattachait exactement à cette conception religieuse révolutionnaire et dépassait tout autant les rapports sociaux et politiques existants que sa théologie dépassait les conceptions religieuses de l’époque. De même que la philosophie religieuse de Münzer frisait l’athéisme, son programme politique frisait le communisme, et plus d’une secte communiste moderne, encore à la veille de la révolution de mars, ne disposait pas d’un arsenal théorique plus riche que celui des sectes « münzériennes » du XVIᵉ siècle. Ce programme, qui était moins la synthèse des revendications des plébéiens de l’époque, qu’une anticipation géniale des conditions d’émancipation des éléments prolétariens en germe parmi ces plébéiens, exigeait l’instauration immédiate sur terre du royaume de Dieu, du millénium des prophètes, par le retour de l’Église à son origine et par la suppression de toutes les institutions en contradiction avec cette Église soi-disant primitive, mais en réalité toute nouvelle. Pour Münzer, le royaume de Dieu n’était pas autre chose qu’une société où il n’y aurait plus aucune différence de classes, aucune propriété privée, aucun pouvoir d’État autonome, étranger aux membres de la société. Toutes les autorités existantes, si elles refusaient de se soumettre et d’adhérer à la révolution, devaient être renversées tous les travaux et les biens devaient être mis en commun et l’égalité la plus complète régner. Une ligue devait être fondée pour réaliser ce programme non seulement dans toute l’Allemagne, mais dans l’ensemble de la chrétienté. Les princes et les nobles seraient invités à se joindre à elle s’ils s’y refusaient, la ligue, à la première occasion, les renverserait les armes à la main ou les tuerait.\par
Münzer se mit immédiatement à l’œuvre pour organiser cette ligue. Ses prêches prirent un caractère encore plus violent, plus révolutionnaire. Ne se bornant plus à attaquer les prêtres, il tonnait avec la même fougue contre les princes, la noblesse, le patriciat. Il dépeignait sous les couleurs les plus ardentes l’oppression existante et y opposait le tableau imaginaire du règne millénaire de l’égalité sociale et républicaine. En même temps, il publiait un pamphlet révolutionnaire après l’autre et envoyait des émissaires dans toutes les directions, pendant que lui-même organisait la ligue à Allstedt et dans les environs.\par
Le premier résultat de cette propagande fut la destruction de la chapelle de la Vierge à Mellerbach, près d’Allstedt, d’après le commandement : « Vous détruirez leurs autels, briserez leurs colonnes, et brûlerez leurs idoles, car vous êtes un peuple saint » (Deutéronome, 7,6). Les princes saxons se rendirent eux-mêmes à Allstedt pour calmer la révolte et convoquèrent Münzer à leur château. Il s’y rendit et y fit un sermon comme ils n’en avaient certainement jamais entendu de semblable de la bouche de Luther, « la viande douillette de Wittenberg », comme l’appelait Münzer. Il déclara, s’appuyant sur le Nouveau Testament, qu’il fallait tuer les souverains impies, surtout les prêtres et les moines, qui traitent l’évangile comme une hérésie. Car les impies n’ont aucun droit à la vie, et ils ne vivent que par la grâce des élus. Si les princes n’exterminent pas les impies, Dieu leur retirera l’épée, \emph{car la puissance de l’épée appartient à la communauté}. La sentine de l’usure, du vol et du brigandage, ce sont les princes et les seigneurs qui font de toutes les créatures vivantes leur propriété : les poissons dans l’eau, les oiseaux dans le ciel, les plantes sur la terre. Et ensuite, ils prêchent aux pauvres le commandement : Tu ne voleras pas ! mais eux mêmes s’emparent de tout ce qui tombe entre leurs mains ils grugent et exploitent le paysan et l’artisan cependant dès qu’un pauvre s’en prend à quoi que ce soit, il est pendu, et, à tout cela, le docteur « Menteur » dit : Amen !\par

\begin{quoteblock}
 \noindent « Les seigneurs font eux-mêmes que les pauvres deviennent leurs ennemis. Ils ne veulent pas supprimer la cause de la révolte. Comment cela peut il finir bien à la longue ? Ah ! mes chers seigneurs, comme le Seigneur frappera joliment parmi les vieux pots avec une barre de fer ! En disant cela, je dois être rebelle. Soit ! »
\end{quoteblock}

\noindent Münzer fit imprimer son sermon. Par punition, son imprimeur d’Allstedt fut contraint par le duc Jean de Saxe à quitter le pays quant à Münzer lui-même, ses écrits durent désormais être obligatoirement soumis à la censure du gouvernement du duc de Weimar. Mais il ne tint aucun compte de cet ordre. Aussitôt après, il fit imprimer dans la ville impériale de Mülhausen un manifeste d’une violence extrême, où il demandait au peuple d’\par
« ouvrir tout grand le trou, afin que le monde entier puisse se rendre compte qui sont nos gros bonnets qui ont assez blasphémé Dieu pour en faire un petit bonhomme peint », et qu’il terminait par ces paroles : « Le monde entier doit supporter un grand choc. Il va commencer un jeu tel que les impies seront renversés et que les humbles seront élevés ».\par
En guise d’exergue à son manifeste, « Thomas Münzer au marteau » écrivait :\par

\begin{quoteblock}
 \noindent « Ecoute, j’ai placé mes paroles dans ta bouche, je t’ai placé aujourd’hui au-dessus des hommes et au-dessus des empires afin que tu déracines, brises, disperses et renverses, que tu construises et que tu plantes. Un mur de fer contre les rois, les princes, les prêtres et contre le peuple est érigé. Qu’ils se battent ! La victoire est merveilleuse qui entraîne la ruine des puissants tyrans impies. »
\end{quoteblock}

\noindent La rupture de Münzer avec Luther et son parti était depuis longtemps un fait accompli. Luther avait même été obligé d’accepter un certain nombre de réformes du culte que Münzer avait introduites de lui-même, sans le consulter. Il observait l’activité de Münzer avec la méfiance soupçonneuse du réformateur modéré à l’égard du parti plus énergique qui pousse en avant. Dès le printemps de 1524, Münzer avait écrit à Melanchton, ce modèle du casanier maladif, du philistin que ni lui, ni Luther ne comprenaient rien au mouvement, et qu’ils s’efforçaient de l’étouffer dans la croyance littérale en la Bible. Toute leur doctrine était vermoulue.\par

\begin{quoteblock}
 \noindent « Chers frères, assez d’attente et d’hésitation ! Il est temps. L’été frappe à nos portes. Rompez votre amitié avec les impies, ils empêchent la parole de Dieu d’agir avec toute sa force. Ne flattez pas vos princes, sinon vous vous condamnerez à la ruine avec eux. Doux savants, ne m’en veuillez pas, il m’est impossible de parler autrement. »
\end{quoteblock}

\noindent À maintes reprises, Luther provoque Münzer à là controverse orale, mais celui-ci, prêt à entreprendre à n’importe quel moment la lutte devant le peuple, n’avait pas la moindre envie de se laisser entraîner à une dispute théologique devant le public partial de l’université de Wittenberg. Il ne voulait pas « porter témoignage de l’Esprit uniquement devant l’Université ». Si Luther était sincère, à n’avait qu’à utiliser l’influence dont il disposait pour faire cesser les chicanes contre les imprimeurs de Münzer et mettre fin à la censure qui pesait sur ses écrits, afin que la lutte pût se poursuivre librement dans la presse.\par
Cette fois, après la parution du pamphlet révolutionnaire de Münzer cité plus haut, Luther le dénonça publiquement. Dans sa Lettre aux princes de Saxe contre l’esprit rebelle, il proclama que Münzer était un instrument de Satan et demanda aux princes d’intervenir et de chasser du pays les fomentateurs de révoltes, étant donné qu’ils ne se contentaient pas de répandre leur mauvais enseignements, mais appelaient à l’insurrection et à la résistance armée contre les autorités.\par
Le 1ᵉʳ août, Münzer fut convoqué au château de Weimar pour répondre devant les princes de l’accusation de menées séditieuses. Il y avait à sa charge un certain nombre de faits extrêmement compromettants. On avait découvert sa ligue secrète, on avait décelé son activité dans les associations de mineurs et de paysans. On le menaça de bannissement à peine de retour à Allstedt, il apprit que le duc Georges de Saxe exigeait qu’on le lui livrât. Des lettres de la ligue écrites de sa main avaient été saisies, lettres dans lesquelles il appelait les sujets de Georges à la résistance armée contre les ennemis de l’Évangile. S’il n’avait pas quitté la ville à temps, le Conseil l’eût livré.\par
Entre temps, l’agitation croissante parmi les plébéiens et les paysans avait considérablement facilité la propagande de Münzer, pour laquelle il avait trouvé de très précieux agents chez les anabaptistes. Cette secte, sans dogmes positifs bien définis, dont l’hostilité commune à toutes les classes dominantes et le symbole commun du second baptême maintenaient seuls la cohésion, d’une rigueur ascétique dans ses mœurs, inlassable, fanatique, menant sans crainte l’agitation, s’était de plus en plus groupée autour de Münzer. Exclus par les persécutions de toute résidence fixe, les anabaptistes parcouraient toute l’Allemagne et proclamaient partout la nouvelle doctrine, avec laquelle Münzer leur avait donné conscience de leurs besoins et de leurs aspirations. On ne saurait compter ceux qui furent torturés, brûlés ou exécutés, mais leur courage et la ténacité de ces émissaires restèrent inébranlables, et le succès de leur activité, étant donné l’agitation croissante du peuple, fut immense. C’est ce qui explique qu’au moment de sa fuite de Thuringe Münzer trouva partout le terrain préparé. Il pouvait désormais aller où il lui plaisait.\par
Près de Nuremberg, où il se rendit tout d’abord, une révolte paysanne venait d’être, un mois à peine auparavant, étouffée dans l’œuf. Münzer y fit de l’agitation clandestine. Bientôt entrèrent en scène des hommes qui défendirent ses idées théologiques les plus hardies sur le caractère non obligatoire de la Bible et sur la nullité des sacrements, affirmèrent que le Christ n’était qu’un homme, et déclarèrent impie le pouvoir temporel. « On reconnaît là l’action de Satan, l’Esprit d’Allstedt ! » s’écria Luther. C’est à Nuremberg que Münzer fit imprimer sa réponse à Luther. Il l’accusait directement de flatter les princes et de soutenir, en fait, par ses hésitations, le parti réactionnaire. « Mais, ajoutait-il, le peuple se libérera cependant, et, à ce moment-là, le docteur Luther sera comme un renard pris au piège. » – Par ordre du Conseil cet écrit fut confisqué, et Münzer dut quitter Nuremberg.\par
Traversant la Souabe, il se rendit en Alsace, puis en Suisse, et revint dans le sud de la Forêt-Noire où l’insurrection avait éclaté depuis plusieurs mois déjà, hâtée en grande partie par ses émissaires anabaptistes. Ce voyage de propagande de Münzer contribua manifestement d’une façon essentielle à l’organisation du Parti populaire, à la fixation nette de ses revendications et finalement à l’insurrection générale d’avril 1525. La double activité de Münzer, pour le peuple d’une part, auquel il s’adressait dans le langage du prophétisme religieux, le seul qu’il fût capable de comprendre à l’époque, et d’autre part pour les initiés, avec lesquels il pouvait ouvertement s’entretenir de ses véritables buts, se manifeste ici très nettement. Si déjà en Thuringe il avait groupé autour de lui et placé à la tête de la ligue secrète un groupe d’hommes des plus décidés, issus non seulement du peuple, mais aussi du bas clergé, dans la Forêt-Noire, il devient le centre de tout le mouvement révolutionnaire de l’Allemagne du Sud-Ouest. Il organise la liaison de la Saxe et de la Thuringe, par la Franconie et la Souabe, jusqu’en Alsace et à la frontière suisse, et compte parmi ses disciples et parmi les chefs de la ligue les agitateurs de l’Allemagne du Sud, tels que Hubmaier à Waldshut, Konrad Grebel à Zurich, Franz Rabmann à Griessen, Schappeler à Memmingen, Jakob Wehe à Leipheim, le docteur Mantel à Stuttgart, la plupart ecclésiastiques révolutionnaires. Lui-même résidait généralement à Griessen, à la limite du canton de Schaffhouse, d’où il entreprenait des tournées à travers le Hegau, le Klettgau, etc. Les persécutions sanglantes que les princes et les seigneurs inquiets entreprirent partout contre cette nouvelle hérésie plébéienne contribuèrent fortement à attiser l’esprit de rébellion et à renforcer l’association. C’est ainsi que Münzer fit de l’agitation pendant cinq mois environ dans l’Allemagne du Sud. Quelque temps avant qu’éclatât la conspiration, il revint en Thuringe, d’où il voulait diriger la révolte et où nous le retrouverons plus tard.\par
Nous verrons à quel point le caractère et la conduite des deux chefs de partis reflètent exactement l’attitude de leurs partis réciproques comment l’indécision de Luther, sa crainte devant le sérieux que prenait le mouvement, sa lâche servilité devant les princes, correspondaient parfaitement à la politique hésitante, équivoque de la bourgeoisie et comment l’énergie et la fermeté révolutionnaire de Münzer se retrouvent dans la fraction la plus avancée des plébéiens et des paysans. La seule différence est que, tandis que Luther se contentait d’exprimer les conceptions et les aspirations de la majorité de sa classe, et d’acquérir ainsi auprès d’elle une popularité à bon compte, Münzer, au contraire, dépassait de beaucoup les idées et les revendications immédiates des paysans et des plébéiens. Il forma avec l’élite des éléments révolutionnaires un parti qui, d’ailleurs, dans la mesure où il était a la taille de ses idées et possédait son énergie, ne représenta jamais qu’une petite minorité dans la masse des insurgés.
\chapterclose


\chapteropen
\renewcommand{\leftmark}{III. Précurseurs de la guerre des paysans entre 1476 et 1517}
\chapter[III. Précurseurs de la guerre des paysans entre 1476 et 1517]{III. Précurseurs de la guerre des paysans entre 1476 et 1517}

\chaptercont
\noindent Cinquante ans environ après la répression du mouvement hussite, se manifestèrent les premiers symptômes de l’esprit révolutionnaire qui germait chez les paysans allemands.\par
C’est dans l’évêché de Wurzbourg, région que la guerre des hussites, « les mauvais gouvernements, les nombreux impôts, les taxes, les dissensions, les hostilités, la guerre, l’incendie, le meurtre, la prison, etc. » avaient déjà appauvrie et que continuellement les évêques, les prêtres et les nobles pillaient sans vergogne, qu’éclata en 1476 la première révolte paysanne. Un jeune berger et musicien, \emph{Hans Böheim de Nicklashausen}, appelé également Jean le Timbalier et \emph{Jean le Joueur de fifre}, entra subitement en scène comme prophète dans la vallée de la Tauber. Il racontait que la Vierge Marie lui était apparue et qu’elle lui avait ordonné de brûler son tambourin, de cesser de s’adonner à la danse et aux autres plaisirs coupables, et d’exhorter au contraire le peuple à la pénitence. Chacun devait renoncer à ses pêchés et aux vanités de ce monde, quitter toute parure et tout ornement, et se rendre en pèlerinage auprés de la Vierge, à Niklashausen, pour obtenir le pardon de ses pêchés.\par
Nous trouvons déjà ici, chez le premier précurseur du mouvement, cet ascétisme que nous rencontrons dans toutes les révoltes teintées de religion du moyen âge, ainsi que dans les temps modernes au début de chaque mouvement prolétarien. Cette rigueur de mœurs ascétique, cette exigence de renonciation à toutes les jouissances et à tous les plaisirs de l’existence établissent d’une part, en face des classes dominantes, le principe de l’égalité spartiate, et constituent d’autre part une étape de transition nécessaire, sans laquelle la couche inférieure de la société ne peut jamais se mettre en mouvement. Pour développer son énergie révolutionnaire, pour acquérir une conscience claire de sa position hostile à l’égard de tous les autres éléments de la société, pour se concentrer elle-même en tant que classe, elle doit commencer par rejeter tout ce qui pourrait la réconcilier avec le régime social existant, renoncer aux rares plaisirs qui lui font encore momentanément supporter son existence opprimée, et que même l’oppression la plus dure ne peut lui arracher. Cet \emph{ascétisme plébéien et prolétarien} se distingue absolument par sa forme farouchement fanatique comme par son contenu, de l’ascétisme bourgeois, tel que le prêchaient la morale bourgeoise luthérienne et les puritains anglais (à la différence des indépendants et des sectes plus avancées), et dont tout le secret réside dans l’esprit \emph{d’épargne bourgeois}. Il va d’ailleurs de soi que cet ascétisme plébéien et prolétarien perd son caractère révolutionnaire au fur et à mesure que, d’une part, le développement des forces de production modernes multiplie à l’infini les objets de jouissance, rendant ainsi superflue l’égalité spartiate, et que, d’autre part, la situation sociale du prolétariat, et par conséquent le prolétariat lui-même, deviennent de plus en plus révolutionnaires. Cet ascétisme disparaît dès lors peu à peu dans les masses et se perd dans les sectes qui s’y obstinent, soit directement dans la ladrerie bourgeoise, soit dans une emphatique chevalerie de la vertu, qui en pratique aboutit également à une avarice de petits bourgeois ou d’artisans bornés. Il est d’autant moins nécessaire de prêcher la renonciation à la masse des prolétaires qu’ils ne possèdent presque plus rien à quoi ils puissent encore renoncer.\par
Les sermons de pénitence de Jean le Joueur de fifre eurent un succès considérable. Tous les prophètes de l’insurrection commencèrent par là et en effet, seul un effort violent, une renonciation brusque à son genre de vie habituel pouvaient mettre en mouvement cette masse paysanne dispersée, clairsemée, grandie dans la soumission aveugle. Les pèlerinages à Niklashausen commencèrent et prirent rapidement une grande extension et plus le peuple y affluait en masse, plus le jeune rebelle exposait ouvertement ses projets. La Vierge de Niklashausen lui avait annoncé, disait-il, que dorénavant il ne devait plus y avoir d’empereur, ni de prince, ni de pape, ni d’autres autorités spirituelles ou temporelles. Les hommes seraient désormais des frères. Ils gagneraient leur pain grâce au travail de leurs mains et aucun ne posséderait plus que son voisin. Tous les cens, redevances, corvées, douanes, impôts et autres taxes et prestations seraient supprimés pour l’éternité, et les bois, les rivières et les prairies seraient partout libres.\par
Le peuple accueillit avec enthousiasme le nouvel Évangile. La renommée du prophète, « le message de Notre-Dame », se répandit rapidement au loin de l’Odenwald, du Main, du Kocher et de la Jagst, même de la Bavière, de la Souabe et du Rhin affluèrent vers lui des foules de pèlerins. On se racontait les miracles qu’il aurait accomplis, on s’agenouillait devant lui et on l’adorait comme un saint On s’arrachait les touffes de son bonnet comme si c’étaient des reliques ou des amulettes. En vain les prêtres intervenaient-ils contre lui et décrivaient-ils ses visions comme des fantasmagories diaboliques et ses miracles comme des duperies infernales. La masse des croyants s’accroissait impétueusement. La secte révolutionnaire commença à se constituer. Les prêches dominicaux du pâtre rebelle réunirent À Niklashausen des assemblées de plus de 40 000 personnes.\par
Pendant plusieurs mois, Jean le Joueur de fifre prêcha devant les masses. Mai il n’avait pas l’intention de s’en tenir aux sermons. Il entretenait des rapports secrets avec le curé de Niklashausen et avec deux chevaliers, Kunz von Thunfeld et son fils, qui s’étaient ralliés à la nouvelle doctrine et devaient être les chefs militaires de l’insurrection projetée. Enfin, le dimanche avant la Saint-Cilian, sa puissance lui paraissant suffisamment établie, il donna le signal du mouvement. En terminant son prêche, il déclara :\par

\begin{quoteblock}
 \noindent « Et maintenant, rentrez chez vous et réfléchissez à ce que la très sainte Mère de Dieu vous a annoncé : samedi prochain, laissez à la maison les femmes, les enfants et les vieillards, mais vous, les hommes, revenez ici à Niklashausen, le jour de la sainte Marguerite, c’est-à-dire samedi prochain, et amenez avec vous vos frères et vos amis, quel qu’en soit le nombre. Cependant ne venez pas avec votre bâton de pèlerin, mais en armes, dans une main le cierge, dans l’autre l’épée, la pique ou la hallebarde. Et la sainte Vierge vous dira ce qu’elle veut que vous fassiez. »
\end{quoteblock}

\noindent Toutefois avant l’arrivée en masse des paysans, les cavaliers de l’évêque avaient déjà enlevé pendant la nuit le prophète de l’insurrection et l’avaient interné au château de Wurzbourg. Au jour fixé, près de 34000 paysans en armes se rassemblèrent, mais la nouvelle de cet enlèvement les abattit. La plus grande partie se dispersa. Les initiés en groupèrent environ 16000 et se rendirent avec eux devant le château, sous la direction de Kunz von Thunfeld et de son fils Michael. L’évêque les amena, à force de promesses, à se retirer mais à peine avaient-ils commencé à se disperser qu’ils furent traîtreusement assaillis par les cavaliers de l’évêque, et un certain nombre d’entre eux faits prisonniers. Deux furent décapités, Jean le Joueur de fifre lui-même fut brûlé. Kunz von Thunfeld dut s’enfuir et n’obtint la permission de rentrer au pays qu’en cédant la totalité de ses biens à l’évêché. Les pèlerinages à Niklashausen continuèrent quelque temps encore, mais furent finalement interdits aussi.\par
Après cette première tentative, l’Allemagne se calma pendant assez longtemps. Ce n’est qu’à la fin des années 1490-1500 que commencèrent de nouvelles révoltes et conjurations paysannes.\par
Nous ne nous étendrons pas sur la révolte paysanne hollandaise de 1491-92 qui ne fut enfin réprimée que par le duc Albert de Saxe, à la bataille de Heemskerk ni sur la révolte qui éclata au cours de la même année chez les paysans de l’abbaye de Kempten, en Haute-Souabe ni sur l’insurrection des paysans frisons sous la direction de Syaard Aylva, en 1497, qui fut également réprimée par Albert de Saxe. Ces révoltes, ou bien éclatèrent trop loin du théâtre de la véritable Guerre des paysans, ou bien ne furent que de simples luttes des paysans libres contre la tentative de les soumettre de force au féodalisme. Nous passerons immédiatement aux deux grandes conjurations qui préparèrent la Guerre des paysans : celles du \emph{Bundschuh} et du \emph{Pauvre Konrad}.\par
La même disette qui avait provoqué, dans les Pays-Bas, la révolte des paysans, amena en 1493 la constitution en Alsace d’une ligue secrète de paysans et de plébéiens à laquelle participèrent également des éléments de l’opposition purement bourgeoise, et avec laquelle sympathisait même plus ou moins une partie de la petite noblesse. Le siège de la ligue était la région de Sélestat, Soultz, Dambach, Rosheim, Scherwiller, etc. Les conjurés demandaient le pillage et l’extermination des Juifs, dont l’usure pressurait déjà, à cette époque comme aujourd’hui encore, les paysans alsaciens l’introduction d’une année jubilaire, où toutes les dettes seraient annulées la suppression des droits de douanes, accises et autres charges fiscales, de la justice ecclésiastique et du tribunal impérial de Rottweil le droit de vote des impôts la réduction du revenu des prêtres à une prébende de 50 à 60 florins la suppression de la confession auriculaire et le droit pour chaque communauté d’élire ses propres tribunaux. Le plan des conjurés était de s’emparer par surprise, dès qu’on serait suffisamment forts, de la forteresse de Sélestat, de confisquer les caisses de la ville et des monastères, et de soulever de là toute l’Alsace. La bannière de l’association, qui devait être déployée au moment de l’insurrection, portait un soulier de paysan, avec de longues lanières : le \emph{Bundschuh}, qui désormais donna son nom et son symbole à toutes les conjurations paysannes des 20 années suivantes.\par
Les conjurés avaient coutume de tenir leurs réunions la nuit sur le mont solitaire Hungerberg. L’entrée dans la ligue était accompagnée des cérémonies les plus mystérieuses et des menaces de châtiments les plus sévères contre les traîtres. Cependant, l’affaire s’ébruita juste à la veille du coup contre Sélestat, pendant la semaine sainte de l’année 1493. Les autorités intervinrent rapidement. Un grand nombre de conjurés furent arrêtés et mis à la torture. Les uns furent écartelés ou décapités, les autres furent amputés des mains ou des doigts et chassés du pays. Un grand nombre s’enfuit en Suisse.\par
Mais ce premier démantèlement n’anéantit en aucune façon le Bundschuh. Au contraire, il continua à subsister clandestinement, et les nombreux réfugiés dispersés à travers la Suisse et l’Allemagne du Sud devinrent autant d’émissaires qui, rencontrant partout, avec la même oppression, le même penchant à la révolte, étendirent le Bundschuh à tout le territoire badois actuel. On ne peut s’émpêcher d’admirer la ténacité et la constance avec lesquelles les paysans de l’Allemagne du Sud conspirèrent pendant prés de trente ans, à partir de 1493, surmontèrent toutes les difficultés provenant de leur état de dispersion et s’opposant à la constitution d’une vaste organisation centralisée, et, après de nombreux démantèlements, défaites et exécutions de leurs chefs, renouèrent chaque fois les fils de la conspiration, jusqu’au jour de l’insurrection générale.\par
En 1502, se manifestèrent, dans l’évêché de Spire, qui englobait alors également la région de Bruchsal, les symptômes d’un mouvement clandestin parmi les paysans. Le Bundschuh s’y était réorganisé avec un succès vraiment considérable. Près de 7000 hommes faisaient partie de l’organisation, dont le centre était à Untergrombach, entre Bruchsal et Weingarten, et dont les ramifications s’étendaient en descendant le Rhin jusqu’au Main et jusqu’au-delà du margraviat de Bade. Son programme contenait les points suivants : refus du paiement du cens, des dîmes, des impôts et douanes aux princes, seigneurs et prêtres abolition du servage \emph{confiscation des cloîtres et autres biens ecclésiastiques et partage de ceux-ci entre les gens du peuple plus d’autre maître que l’empereur}.\par
Nous trouvons ici pour la première fois, chez les paysans, la revendication de la sécularisation des biens du clergé au profit du peuple et celle de la monarchie allemande une et indivisible, deux revendications qui dès lors réapparaissent régulièrement au sein de la fraction la plus avancée des paysans et des plébéiens, jusqu’à ce que Thomas Münzer substitue au partage des biens d’Eglise leur \emph{confiscation} au profit de la \emph{communauté des biens} et à l’\emph{empire} allemand uni, la \emph{République} une et indivisible.\par
Le Bundschuh rénové avait, comme l’ancien, son lieu de réunion secret, son serment de discrétion absolue, ses cérémonies d’admission et son étendard portant l’inscription : « \emph{Rien que la justice de Dieu ! »} Le plan d’action était à peu près le même que celui des Alsaciens : on s’emparerait par surprise de Bruchsal, dont la majorité de la population faisait partie de la ligue, on y organiserait une armée qui serait ensuite envoyée comme centre de ralliement mobile dans les principautés environnantes.\par
Le plan fut trahi par un ecclésiastique auquel un des conjurés s’était confessé. Les gouvernements prirent immédiatement des mesures de représailles. À quel point l’organisation avait des ramifications lointaines, c’est ce qui ressort de la panique qui s’empara des divers États impériaux d’Alsace et de la Ligue souabe. On rassembla des troupes et l’on procéda à des arrestations en masse.\par
L’empereur Maximilien, le « dernier chevalier », édicta les ordonnances les plus sanguinaires pour châtier l’entreprise inouïe des paysans. Cà et là, il y eut des attroupements et de la résistance armée, mais les armées paysannes dispersées ne tinrent pas longtemps. Une partie des conjurés furent exécutés, un grand nombre s’enfuirent mais le secret fut si bien gardé que la plupart, même le chef, purent rester tranquillement soit dans leur propre localité soit dans les pays voisins.\par
Après cette nouvelle défaite un calme apparent assez long se produisit dans les luttes de classes. Mais le travail se poursuivit secrètement. En Souabe, se constitua dès les premières années du XVIᵉ siècle, manifestement en liaison avec les membres dispersés du Bundschuh, une nouvelle ligue, le \emph{Pauvre Konrad}. Dans la Forêt-Noire, le Bundschuh continua à subsister dans un certain nombre de petits cercles isolés, jusqu’à ce qu’au bout de dix ans un chef paysan énergique réussit à grouper à nouveau, en une grande conjuration, les différents fils du mouvement. Les deux conspirations apparurent au grand jour, peu de temps l’une après l’autre, au cours des années mouvementées de 1513 à 1515, au cours desquelles les paysans suisses, hongrois et slovènes déclenchèrent une série de soulèvements importants.\par
Le réorganisateur du Bundschuh de la région du Rhin supérieur était \emph{Joss Fritz}, d’Untergrombach, un des réfugiés de la conspiration de 1502, ancien soldat et caractère remarquable à tous points de vue. Depuis sa fuite, il avait séjourné en différentes localités situées entre le lac de Constance et la Forêt-Noire, et s’était finalement établi à Lehen, près Fribourg-en-Brisgau, où il était même devenu garde-champètre ! Comment, de là, il réorganisa l’association et avec quelle habilité il sut y faire entrer les gens les plus différents, sur ce point les pièces de l’instruction contiennent les détails les plus intéressants. Grâce à son talent diplomatique et à son inlassable persévérance, ce conspirateur modèle réussit à impliquer dans la ligue un nombre incroyable de gens appartenant aux catégories sociales les plus diverses : chevaliers, prêtres, bourgeois, plébéiens et paysans. Il semble à peu près certain qu’il organisa même plusieurs degrés, plus ou moins nettement séparés, dans la conspiration. Tous les éléments utilisables furent employés avec une prudence et une habileté extraordinaires. Outre les émissaires plus initiés qui parcouraient le pays sous les déguisements les plus divers, les chemineaux et mendiants furent employés aux missions subalternes. Joss Fritz était en rapport direct avec les rois des mendiants et tenait dans sa main, par leur intermédiaire, toute la nombreuse population des vagabonds. Ces rois des mendiants jouent un rôle considérable dans sa conspiration. C’étaient des figures extrêmement originales. L’un parcourait le pays avec une petite fille, soi-disant infirme des jambes, pour laquelle il mendiait. Il portait plus de huit insignes à son chapeau, les quatorze saints martyrs, sainte Odile, Notre-Dame, etc., avec cela, une longue barbe rousse et un grand bâton noueux, avec un poignard et un aiguillon. Un autre, qui quêtait pour saint Valentin, vendait des aromates et du semen-contra, portait une robe longue couleur de fer, une barrette rouge avec l’enfant de Trente, une épée au côté et plusieurs couteaux ainsi qu’un poignard à la ceinture. D’autres arboraient des blessures qu’ils entretenaient artificiellement et portaient des accoutrements fantastiques du même genre. Ils étaient au moins dix. Ils devaient, pour 2000 florins, soulever en même temps l’Alsace, le margraviat de Bade et le Brisgau et se rassembler avec au moins 2000 des leurs, le jour de la fête patronale de Saverne, à Rosen, sous le commandement de Georg Schneider, un ancien capitaine de lansquenets, pour s’emparer de la ville. Entre les membres de la ligue proprement dits on avait établi de station à station un service d’estafettes. Joss Fritz et son principal émissaire, Stoffel de Fribourg, chevauchaient continuellement d’un endroit à l’autre, et passaient en revue pendant la nuit les nouvelles recrues. Sur l’extension de la ligue sur le cours supérieur du Rhin et dans la Forêt-Noire, les actes de l’instruction fournissent des preuves suffisantes. Ils contiennent le nom et le signalement d’un nombre incalculable d’adhérents provenant des localités les plus différentes de la région. La plupart sont des artisans, puis viennent les paysans et les aubergistes, quelques nobles, des prêtres (entre autres celui de Lehen) et des soldats licenciés. On se rend compte, rien que d’après cette composition, du caractère bien plus développé que le Bundschuh avait pris sous la direction de Joss Fritz. L’élément plébéien des villes commençait de plus en plus à compter. Les ramifications de la conspiration s’étendaient sur toute l’Alsace, le Bade actuel et jusque dans le Wurtemberg et au Main. De temps en temps de grandes assemblées se tenaient sur des montagnes écartées, sur le Kniebis, etc., où l’on discutait des affaires de la ligue. Les réunions des chefs, auxquelles assistaient fréquemment les membres de la localité ainsi que des délégués des localités éloignées, avaient lieu sur la Hartmatte, près de Lehen, et c’est là que furent adoptés les quatorze articles de l’association. Pas d’autre maître que l’empereur et (d’après quelques-uns) le pape abolition du tribunal de Rottweil, limitation de la justice ecclésiastique aux affaires spirituelles suppression de tous les cens qui seraient payés jusqu’à concurrence de la valeur du capital réduction à cinq pour cent du taux maximum de l’intérêt liberté de la chasse, de la pêche, du pâturage et du ramassage du bois limitation des bénéfices à un par prêtre confiscation des biens ecclésiastiques et des joyaux des monastères au profit du trésor de guerre de la ligue suppression de tous les impôts et douanes iniques paix éternelle pour toute la chrétienté lutte énergique contre tous les ennemis de la ligue établissement d’un impôt spécial au profit de l’organisation prise d’une ville forte, Fribourg, où serait établi le siège de l’association ouverture de négociations avec l’empereur, dès que les troupes de la ligue seraient réunies, et avec la Suisse, en cas de refus de l’empereur. Tels sont les points sur lesquels on se mit d’accord. On se rend compte, d’après ce qui précède, combien d’une part les revendications des paysans et des plébéiens avaient pris un caractère de plus en plus net et ferme, et comment d’autre part on avait dû faire, dans la même mesure, des concessions aux modérés et aux hésitants.\par
L’insurrection devait éclater vers l’automne de 1513. Il ne manquait plus que la bannière de la ligue, et Joss Fritz se rendit à Heilbronn pour la faire peindre. Outre toutes sortes d’emblèmes et d’images, elle portait le Bundschuh, avec l’inscription suivante : « Seigneur, soutiens ta justice divine ! » Mais pendant son absence une tentative trop précipitée en vue de s’emparer par surprise de Fribourg fut faite et prématurément découverte. Quelques indiscrétions dans la propagande mirent le Conseil de Fribourg et le margrave de Bade sur les traces de la conspiration la trahison de deux conjurés compléta la série des révélations. Immédiatement le margrave, le Conseil de Fribourg et le gouvernement impérial à Ensisheim mirent en campagne leurs sbires et leurs soldats. Un certain nombre de membres du Bundschuh furent arrêtés, mis à la torture et exécutés. Cependant, cette fois encore, la plupart réussirent à s’enfuir et en particulier Joss Fritz. Les gouvernements suisses poursuivirent cette fois les réfugiés très énergiquement et en exécutèrent même plusieurs. Mais pas plus que leurs voisins ils ne purent empêcher que la plus grande partie des fugitifs restassent constamment à proximité de leur domicile et même y retournassent peu à peu. C’est le gouvernement alsacien d’Ensisheim qui se déchaîna le plus sur son ordre beaucoup de conjurés furent décapités, roués ou écartelés. Joss Fritz lui-même séjournait la plupart du temps sur la rive suisse du Rhin, mais il passait fréquemment en Forêt-Noire, sans qu’on pût jamais s’emparer de lui.\par
Les raisons pour lesquelles les Suisses s’étaient, cette fois, unis au gouvernements voisins contre les adhérents du Bundschuh s’expliquent par l’insurrection paysanne qui éclata l’année suivante, en 1514, à Berne, à Soleure et à Lucerne et qui eut en général pour résultat une épuration des gouvernements aristocratiques et du patriciat. En outre, les paysans réussirent à obtenir un certain nombre de privilèges. Si ces insurrections locales réussirent, cela est dû simplement au fait qu’en Suisse la concentration était encore plus faible qu’en Allemagne. En 1525, les paysans vinrent également partout facilement à bout de leurs seigneurs locaux, mais ils succombèrent devant les armées bien organisées des princes et précisément de telles armées n’existaient pas en Suisse.\par
En même temps que le Bundschuh dans le Bade et manifestement en liaison directe avec lui, s’était constituée dans le Wurtemberg une seconde conspiration. Les documents prouvent qu’elle existait déjà en 1503, et comme le nom de Bundschuh était trop dangereux depuis le démantèlement du centre d’Untergrombach, elle prit le nom de Pauvre Conrad. Son siège principal était la vallée de la Rems, au pied des montagnes de Hohenstaufen. Son existence n’était plus un secret depuis longtemps, du moins parmi le peuple. L’oppression impitoyable que faisait peser sur les paysans le gouvernement du duc Ulrich, ainsi qu’une série d’années de famine, qui contribuèrent puissamment à provoquer les mouvements de 1513 et de 1514, avaient renforcé le nombre des ligueurs. Les nouveaux impôts sur le vin, la viande et le pain, ainsi qu’un impôt sur le capital d’un Pfennig par an et par florin, provoquèrent l’explosion. On devait tout d’abord s’emparer de la ville de Schorndorf, où se réunissaient les chefs de la conspiration, dans la maison du coutelier Kaspar Pregizer. L’insurrection éclata au printemps de 1514. Trois mille paysans, 5000 d’après d’autres, marchèrent sur la ville, mais les bonnes promesses des fonctionnaires du duc les décidèrent à se retirer. Le duc Ulrich accourut avec quatre-vingts cavaliers, après avoir promis la suppression des nouveaux impôts, et il trouva du fait de cette promesse le calme qui régnait partout. Il promit également de convoquer une diète chargée d’examiner toutes les doléances. Mais les chefs de l’association se rendaient parfaitement compte qu’Ulrich n’avait d’autre but que de maintenir le calme dans le peuple, jusqu’à ce qu’il eût recruté et rassemblé suffisamment de troupes pour pouvoir violer sa parole et prélever de force les impôts. C’est pourquoi, de la maison de Kaspar Pregizer, « la chancellerie du Pauvre Conrad », ils lancèrent des appels à un congrès de la ligue, qu’appuyèrent des émissaires envoyés dans toutes les directions. Le succès du premier soulèvement dans la vallée de la Rems avait eu pour résultat de développer partout le mouvement dans le peuple. C’est pourquoi les messages et les émissaires rencontrèrent partout un accueil favorable et le congrès, qui se réunit le 28 mai à Unterturkheim, groupa de nombreux délégués venus de toutes les parties du Wurtemberg. On décida de poursuivre sans trêve l’agitation et, à la première occasion, de frapper un grand coup dans la vallée de la Rems pour de là étendre l’insurrection dans tout le pays. Tandis que Bantelhans de Dettingen, un ancien soldat, et Singerhans de Wurtingen, un paysan notable, apportaient à la ligue l’adhésion de la Haute-Souabe, l’insurrection éclatait déjà de tous côtés. Singerhans fut certes surpris et fait prisonnier, mais les villes de Backnang, Winnenden, Markgroenningen tombèrent aux mains des paysans alliés aux plébéiens, et toute la contrée, de Weinsberg à Blaubeuren, et de là jusqu’à la frontière badoise, était en insurrection ouverte. Ulrich dut céder. Tout en convoquant la diète pour le 25 juin, il écrivit en même temps aux princes voisins et aux villes libres pour leur demander leur appui contre l’insurrection, qui mettait en danger tous les princes et les autorités et notabilités de l’Empire et « affectait une ressemblance étrange avec le Bundschuh ».\par
Entre temps, la diète, c’est-à-dire les représentants des villes et un grand nombre de délégués des paysans, qui réclamaient également des sièges au Landtag, se réunit dès le 18 juin à Stuttgart. Les prélats n’étaient pas encore arrivés. Quant aux chevaliers, ils n’avaient même pas été invités. L’opposition citadine de Stuttgart, ainsi que deux armées de paysans menaçantes à proximité, à Leonberg et dans la vallée de la Rems, appuyèrent les revendications des paysans. L’on décida d’accepter leurs délégués et de destituer et de châtier les trois conseillers haïs du duc : Lamparter, Thumb et Lorcher, d’adjoindre au duc un conseil composé de quatre chevaliers, de quatre bourgeois et de quatre paysans, de lui accorder une liste civile fixe et de confisquer, au profit du trésor de l’État, les cloîtres et les abbayes.\par
Le duc Ulrich répondit à ces décisions révolutionnaires par un coup de force. Le 21 juin il se rendit avec ses chevaliers et ses conseillers à Tubingen, où les prélats le suivirent, ordonna aux bourgeois de s’y rendre également, ce qu’ils firent, et y continua les séances de la diète sans les paysans. Là les bourgeois, placés sous la terreur militaire, trahirent leurs alliés les paysans. Le 8 juillet fut conclu l’accord de Tubingen, qui imposait au pays la charge de près d’un million de dettes ducales, au duc certaines restrictions qu’il ne respecta d’ailleurs pas, et payait les paysans de quelques minces et vagues promesses et d’une loi pénale très positive contre la sédition et les associations. De représentation des paysans à la diète il ne fut naturellement plus question. Les paysans crièrent à la trahison, mais comme le duc, depuis le transfert de ses dettes aux états, avait de nouveau du crédit, il rassembla rapidement des troupes, auxquelles s’adjoignirent celles que lui envoyèrent ses voisins, particulièrement l’électeur du Palatinat. C’est ainsi que, avant la fin de juillet, l’accord de Tubingen fut accepté par tout le pays et le nouveau serment de fidélité prêté. Ce n’est que dans la vallée de la Rems que le Pauvre Conrad opposa quelque résistance. Le duc, qui s’y rendit de nouveau lui-même, faillit être assassiné et un camp de paysans se constitua sur le Kappelberg. Mais comme l’affaire traînait en longueur, la plupart des insurgés se dispersèrent par suite du manque de vivres, et peu après, le reste s’éloigna également après avoir signé un accord équivoque avec quelques membres de la diète. Ulrich, dont l’armée s’était renforcée entre temps des contingents mis à sa disposition par les villes qui, leurs revendications étant maintenant satisfaites, se retournaient avec fanatisme contre les paysans, se jeta malgré l’accord sur la vallée de la Rems, dont les villes et les villages furent livrés au pillage : 1600 paysans furent faits prisonniers, 16 d’entre eux furent immédiatement décapités, les autres condamnés pour la plupart à des amendes considérables au profit d’Ulrich. Un grand nombre furent maintenus longtemps en prison. On édicta des lois sévères pour empêcher la reconstitution de l’association et de toutes les assemblées paysannes, et la noblesse souabe conclut une alliance spéciale pour réprimer toute tentative d’insurrection. Cependant, les principaux chefs du Pauvre Conrad avaient réussi à s’enfuir en Suisse, d’où ils retournèrent chez eux pour la plupart isolément, au bout de quelques années.\par
En même temps que se produisait le mouvement wurtembergeois, se manifestaient des symptômes de nouvelles menées du Bundschuh dans le Brisgau et le margraviat de Bade. À Buhl, une tentative d’insurrection eut lieu au mois de juin, mais elle fut immédiatement dispersée par le margrave Philippe. Le chef du mouvement, Gugel-Bastian, fut arrêté et décapité à Fribourg.\par
Cette même année 1514, au printemps également, une guerre générale des paysans éclata en Hongrie. On prêchait une croisade contre les Turcs et, comme d’habitude, on promit la liberté aux serfs et aux corvéables qui accepteraient de s’engager parmi les croisés. Près de 60000 paysans se réunirent et furent placés sous le commandement de Georg Dósza, un Székler, qui s’était déjà distingué au cours de précédentes campagnes contre les Turcs et y avait gagné un titre de noblesse. Mais les chevaliers et les magnats hongrois voyaient d’un mauvais œil cette croisade qui menaçait de leur enlever leur propriété, leurs serfs. Ils poursuivirent les armées paysannes isolées et ramenèrent de force leurs serfs chez eux en les maltraitant. Lorsque la nouvelle en parvint à l’armée des croisés, là fureur des paysans opprimés éclata. Deux des plus ardents des prédicateurs de la croisade, Laurentius et Barnabas attisèrent dans l’armée, par leurs discours révolutionnaires, la haine contre la noblesse. Dósza lui-même partageait l’indignation de ses troupes à l’égard de la trahison des nobles. L’armée des croisés se transforma en une armée révolutionnaire, et il se mit à la tête de ce nouveau mouvement.\par
Il établit avec ses paysans son camp sur le Rakos, près de Pest. Les hostilités furent ouvertes par des bagarres avec des gens du parti de la noblesse dans les villages environnants et dans les faubourgs de Pest. Bientôt on en vint à des escarmouches, puis finalement à des vêpres siciliennes pour tous les nobles qui tombèrent entre les mains des paysans et à l’incendie de tous les châteaux environnants. La cour menaça, mais en vain. Lorsque la première sentence de la justice populaire eut été exécutée sur des nobles sous les murs de la capitale, Dósza passa à d autres opérations. Il partagea son armée en cinq colonnes. Deux furent envoyées dans les montagnes de la Haute-Hongrie, pour soulever tout le pays et anéantir la noblesse. La troisième, sous le commandement d’Ambros Szaleresi, un bourgeois de Pest, resta sur le Rakos pour surveiller la capitale. La quatrième et la cinquième furent conduites par Dósza et son frère Gregor contre Szegedin.\par
Entre temps, la noblesse s’était rassemblée à Pest. Elle appela à l’aide le voïvode de Transylvanie, Johann Zápolya. La noblesse alliée aux bourgeois de Budapest, battit et anéantit le corps campé sur le Rákos, après que Száleresi, avec les éléments bourgeois de l’armée paysanne, fut passé à l’ennemi. Une foule de prisonniers furent exécutés de la façon la plus barbare, les autres renvoyés chez eux, le nez et les oreilles coupés.\par
Dósza échoua devant Szegedin et marcha sur Csanád dont il s’empara, après avoir battu une armée de la noblesse commandée par Bâtory Istvân et l’évêque Csaky, et exercé sur les prisonniers, parmi lesquels se trouvaient également l’évêque et le trésorier du roi Teleki, des représailles sanglantes pour les cruautés commises sur le Rákos. À Csanád, il proclama la République, l’abolition de la noblesse, l’égalité de tous et la souveraineté du peuple, et marcha ensuite sur Temesvár, où Bátory s’était réfugié. Mais tandis qu’il assiégeait pendant deux mois cette forteresse et qu’il recevait l’appoint d’une nouvelle armée commandée par Anton Hosszu, les deux armées envoyées par lui dans la Haute-Hongrie furent battues en plusieurs combats par la noblesse et Johann Zápolya marcha contre lui à la tête de l’armée transylvanienne. Les paysans furent surpris et dispersés par Zápolya, Dósza lui-même fait prisonnier, rôti sur un trône ardent et mangé tout vif par ses propres partisans, qui n’eurent la vie sauve qu’à cette condition. Les paysans dispersés, rassemblés à nouveau par Laurentius et Hosszu, furent à nouveau battus, et tous ceux qui tombaient aux mains de l’ennemi furent empalés ou pendus. Les cadavres de paysans pendaient par milliers le long des routes ou à l’entrée des villages incendiés. Prés de 60 000 auraient été soit tués, soit massacrés. Mais la noblesse eut cependant soin de confirmer à nouveau, à la diète suivante, le servage des paysans comme loi du pays.\par
L’insurrection paysanne dans la « marche slovène », c’est-à-dire en Carinthie, en Carniole et en Styrie, qui éclata à la même époque, avait pour base une conjuration du genre Bundschuh, qui s’était déjà constituée dès 1503 dans cette région pressurée par la noblesse et les fonctionnaires de l’Empire, ravagée par la famine et les invasions des Turcs et qui avait déjà provoqué un soulèvement. Les paysans slovènes de cette région, comme les paysans allemands, levèrent à nouveau dès 1513 l’étendard des stara prawa (anciens droits) et si cette année encore ils se laissèrent apaiser, si en 1514 où ils se rassemblèrent encore plus massivement, ils furent amenés à se disperser par la promesse formelle de l’empereur Maximilien de rétablir les anciens droits, la guerre de revanche du peuple toujours trompé éclata d’autant plus furieusement au printemps 1515. De même qu’en Hongrie, les châteaux et les monastères furent partout détruits et les nobles faits prisonniers, jugés et décapités par des jurys de paysans. En Styrie et en Carinthie, le capitaine des troupes impériales Dietrichstein réussit à étouffer rapidement l’insurrection. Dans la Carniole, elle ne fut réprimée qu’à la suite de la prise de la ville de Rain (automne 1516) et des cruautés autrichiennes sans nombre qui s’ensuivirent, digne pendant aux atrocités commises par la noblesse hongroise.\par
On comprend qu’après une série de défaites aussi décisives et après ces atrocités massives de la noblesse, les paysans allemands soient restés assez longtemps tranquilles. Et cependant, ni les conspirations, ni les révoltes locales ne cessèrent jamais complètement. Dès 1516, la plupart des réfugiés du Bundschuh et du Pauvre Conrad retournèrent en Souabe et sur le cours supérieur du Rhin, et en 1517 le Bundschuh était déjà complètement reconstitué dans la Forêt-Noire. Joss Fritz lui-même, qui tenait toujours caché sur sa poitrine le vieux étendard de Bundschuh de 1513, parcourait à nouveau la Forêt-Noire et déployait une grande activité. La conjuration se réorganisa à nouveau. Comme quatre ans auparavant, on annonça de nouveau des assemblées sur le Kniebis. Mais le secret ne fut pas gardé. Les gouvernements eurent vent de la chose et intervinrent. Un certain nombre de conjurés furent pris et exécutés. Les membres les plus actifs et les plus intelligents durent s’enfuir, parmi eux Joss Fritz, dont on ne put, une fois de plus, s’emparer, mais qui semble être mort peu de temps après en Suisse, car à partir de ce moment son nom n’apparaît plus nulle part.
\chapterclose


\chapteropen
\renewcommand{\leftmark}{IV. La révolte de la noblesse}
\chapter[IV. La révolte de la noblesse]{IV. La révolte de la noblesse}

\chaptercont
\noindent À l’époque même où la quatrième conspiration du Bundschuh était réprimée dans la Forêt-Noire, Luther lança à Wittenberg le signal du mouvement qui devait entraîner dans son tourbillon tous les ordres et ébranler tout l’Empire. Les thèses de l’augustin de Thuringe firent l’effet de la foudre dans un baril de poudre. Elles donnèrent dès l’abord aux aspirations multiples et contradictoires des chevaliers comme des bourgeois, des paysans comme des plébéiens, des princes avides d’indépendance comme du bas clergé, des sectes mystiques clandestines comme de l’opposition littéraire des érudits et des satiristes burlesques une expression générale commune, autour de laquelle ils se groupèrent avec une rapidité surprenante. Cette alliance soudaine de tous les éléments d’opposition, si courte que fut sa durée, révéla brusquement la force immense du mouvement et le fit progresser d’autant plus rapidement.\par
Mais précisément, ce progrès rapide du mouvement devait aussi développer très vite les germes de dissension qu’il contenait en lui, et tout au moins détacher les uns des autres les différents éléments de la masse agitée, directement opposés les uns aux autres par leur situation sociale et les amener à leur position d’hostilité normale. Cette polarisation de la masse confuse de l’opposition autour de deux centres d’attraction se manifesta dès le début du mouvement de la Réforme : les nobles et les bourgeois se groupèrent sans réserve autour de Luther les paysans et les plébéiens, sans voir encore en lui un adversaire direct, continuèrent à former comme auparavant un parti d’opposition révolutionnaire indépendant. La seule différence fut que le mouvement était maintenant beaucoup plus général, beaucoup plus profond qu’avant Luther, d’où la nécessité d’une opposition nettement exprimée, d’une lutte directe entre les deux partis. Cet antagonisme direct se manifesta rapidement. Luther et Münzer se combattirent par la presse et en chaire, de même que les armées des princes, des chevaliers et des villes, composées en majeure partie d’éléments luthériens, ou du moins inclinant au luthéranisme, dispersèrent les armées des paysans et des plébéiens.\par
À quel point divergeaient les intérêts et les besoins des différents éléments qui avaient accepté la Réforme, c’est ce que montre, déjà avant la Guerre des paysans, la tentative faite par la noblesse pour faire triompher ses revendications en face des princes et du clergé.\par
Nous avons déjà vu plus haut quelle position prit la noblesse allemande au début du XVIᵉ siècle. Elle était sur le point de perdre son indépendance au profit des princes séculiers et ecclésiastiques, dont la puissance grandissait de jour en jour. Elle voyait en même temps, au fur et à mesure qu’elle sombrait elle-même, décliner la puissance de l’Empire, et l’Empire lui-même se décomposer en un certain nombre de principautés souveraines. Son déclin devait coïncider pour elle avec le déclin de l’Allemagne en tant que nation. À cela s’ajoutait le fait que la noblesse, et particulièrement la noblesse non médiatisée, était l’ordre qui, tant par sa vocation militaire que par sa position en face des princes, représentait l’Empire et la puissance impériale. Elle constituait l’ordre le plus national, et plus le pouvoir d’Empire était puissant, moins les princes étaient forts et nombreux, plus l’Allemagne était unie et plus cet ordre avait de force. De là, le mécontentement général de la chevalerie causé par la situation politique pitoyable de l’Allemagne, par l’impuissance extérieure de l’Empire, qui augmentait dans la mesure où la maison impériale rattachait au Reich par héritage une province après l’autre, par les intrigues des puissances étrangères à l’intérieur de l’Empire et les complots organisés contre le pouvoir impérial par les princes allemands alliés à l’étranger. C’est pourquoi les revendications de la noblesse devaient nécessairement se résumer avant tout en une réforme de l’Empire aux dépens des princes et du haut clergé. La synthèse en fut assumée par \emph{Ulrich von Hutten}, le théoricien de la noblesse allemande, en collaboration avec \emph{Franz von Sickingen}, son représentant militaire et politique.\par
Hutten a formulé d’une façon très nette et conçu d’une façon très radicale sa réforme de l’Empire exigée au nom de la noblesse. Il ne s’agissait de rien de moins que de la suppression de tous les princes, de la sécularisation de toutes les principautés et de tous les biens ecclésiastiques, de l’établissement d’une \emph{démocratie aristocratique}, ayant à sa tête un monarque, dans le genre de la défunte République polonaise à ses meilleurs jours. Grâce au rétablissement de la domination de la noblesse, classe militaire par excellence, à l’élimination des princes, représentants du morcellement grâce à la destruction de la puissance des prêtres et en arrachant l’Allemagne à la domination spirituelle de Rome, Hutten et Sickingen croyaient pouvoir rendre à l’Empire son unité, son indépendance et sa force\par
La démocratie aristocratique reposant sur le servage, telle qu’elle a existé en Pologne et, sous une forme un peu différente, dans les royaumes des premiers siècles de la conquête germanique, est l’une des formes sociales les plus primitives, et elle continue à évoluer tout naturellement vers la hiérarchie féodale développée, qui représente déjà une étape beaucoup plus élevée. Cette démocratie aristocratique pure était par conséquent impossible au XVIᵉ siècle. Elle était déjà impossible pour cette première raison que l’Allemagne comprenait à cette époque de grandes et puissantes villes. Mais, d’autre part, l’alliance de la petite noblesse et des villes, qui amena en Angleterre la transformation de la monarchie féodale en une monarchie constitutionnelle bourgeoise, était également impossible. En Allemagne la vieille noblesse s’était maintenue, tandis qu’en Angleterre elle avait été complètement détruite, à l’exception de 28 familles, par la guerre des Deux-Roses, et remplacée par une nouvelle noblesse, d’origine et de tendances bourgeoises. En Allemagne le servage s’était maintenu, et la noblesse disposait de revenus féodaux, alors qu’en Angleterre il avait presque complètement disparu, et que la noblesse était composée de simples propriétaires fonciers bourgeois, disposant d’un \emph{revenu bourgeois} : la rente foncière. Enfin la centralisation de la monarchie absolue qui, par suite de l’antagonisme entre la noblesse et la bourgeoisie, existait en France depuis Louis XI et s’y développait de plus en plus, était également impossible en Allemagne, pour cette simple raison que les conditions de la centralisation nationale y étaient inexistantes ou seulement en germe.\par
Plus Hutten s’efforçait dans ces conditions, de réaliser son idéal, plus il devait faire de concessions, et plus les contours de sa réforme d’Empire devenaient indécis. La noblesse, à elle seule, n’était pas assez forte pour réaliser cette entreprise, et c’est ce que prouvait sa faiblesse croissante vis-à-vis des princes. Elle devait se trouver des alliés, et les seuls alliés possibles étaient les villes, les paysans et les théoriciens influents du mouvement de la Réforme. Mais les villes connaissaient suffisamment la noblesse pour n’avoir aucune confiance en elle et repousser toute alliance avec elle. Les paysans voyaient avec raison dans la noblesse qui les pressurait et les maltraitait leur adversaire le plus acharné. Et les théoriciens de la Réforme étaient ou avec les bourgeois et les princes, ou avec les paysans. Que pouvait en effet promettre de positif la réforme de l’Empire, proposée par la noblesse aux bourgeois et aux paysans et dont le but principal était toujours le relèvement de la noblesse ? Dans ces conditions, il n’y avait pas d’autre issue pour Hutten que de ne pas dire grand-chose, ou de ne rien dire du tout dans ses écrits de propagande sur la situation réciproque future de la noblesse, des villes et des paysans, de rejeter tout le mal sur les princes, les prêtres et la dépendance de Rome, et de démontrer aux bourgeois que leur intérêt leur commandait de conserver tout au moins une attitude de neutralité dans la lutte imminente entre les princes et la noblesse. De la suppression du servage et des charges que la noblesse faisait peser sur les paysans, il n’est question nulle part chez Hutten.\par
La position de la noblesse allemande à l’égard des paysans était, à l’époque, exactement la même que celle de la noblesse polonaise à l’égard des siens dans les insurrections de 1830 à 1846. Tout comme les insurrections polonaises modernes, le mouvement n’était alors possible en Allemagne que par une alliance de tous les partis d’opposition, et surtout de la noblesse avec les paysans. Mais précisément cette alliance était dans les deux cas \emph{impossible}. La noblesse n’était pas plus dans la nécessité de renoncer à ses privilèges politiques et à ses droits féodaux vis-à-vis des paysans, que les paysans révolutionnaires n’étaient disposés à accepter une alliance avec la noblesse, c’est-à-dire avec l’ordre qui les opprimait le plus, pour des perspectives générales et imprécises. De même qu’en Pologne en 1830, la noblesse ne pouvait plus dans l’Allemagne de 1522 gagner les paysans. Seules, l’abolition complète du servage et de la sujétion, la renonciation à tous les privilèges de la noblesse eussent pu unir les paysans à la noblesse. Mais celle-ci, comme tout ordre privilégié, n’avait pas la moindre envie de renoncer volontairement à ses privilèges, à toute sa situation d’exception et à la plus grande partie de ses revenus.\par
C’est pourquoi, lorsque la lutte éclata, la noblesse se trouva finalement seule en face des princes. Il était donc à prévoir que les princes qui, depuis deux siècles, avaient constamment gagné du terrain sur elle, l’écraseraient cette fois encore facilement.\par
On connaît le déroulement de la lutte elle-même. Sickingen, qui était déjà reconnu comme le chef politique et militaire de la noblesse de l’Allemagne moyenne, et Hutten fondèrent en 1522, à Landau, une ligue de la noblesse rhénane, souabe et franconienne pour une durée de six années, soi-disant dans un but défensif. Sickingen rassembla une armée, en partie avec ses propres moyens, en partie avec l’aide des chevaliers des environs, recrutant des renforts en Franconie, sur le cours inférieur du Rhin, dans les Pays-Bas et en Westphalie et ouvrit en septembre 1522 les hostilités par une déclaration de guerre à l’électeur-archevêque de Trèves. Mais tandis qu’il assiégeait Trèves, ses renforts furent coupés par une intervention rapide des princes. Le landgrave de Hesse et l’électeur du Palatinat accoururent au secours de l’archevêque de Trèves, et Sickingen fut obligé de se réfugier dans son château fort de Landstuhl. Malgré tous les efforts de Hutten et de ses autres amis, la noblesse alliée, intimidée par l’action rapide et concentrée des princes, l’abandonna à son sort. Lui-même fut mortellement blessé rendit Landstuhl et mourut aussitôt après. Hutten dut s’enfuir en Suisse et mourut quelques mois plus tard dans l’île d’Ufnau, sur le lac de Zurich.\par
À la suite de cette défaite et de la mort de ses deux chefs, la puissance de la noblesse en tant que corps indépendant des princes fut brisée. Désormais la noblesse n’apparaît plus qu’au service et sous la direction des princes. La Guerre des paysans qui éclata aussitôt après, l’obligea davantage encore à se placer directement ou indirectement sous la protection des princes, et montra en même temps que la noblesse allemande préférait continuer à exploiter les paysans sous la tutelle des princes plutôt que renverser les princes et les prêtres au moyen d’une alliance ouverte avec les paysans \emph{émancipés}.
\chapterclose


\chapteropen
\renewcommand{\leftmark}{V. La guerre des paysans en Souabe et en Franconie}
\chapter[V. La guerre des paysans en Souabe et en Franconie]{V. La guerre des paysans en Souabe et en Franconie}

\chaptercont
\noindent À partir du moment où la déclaration de guerre de Luther à la hiérarchie catholique mit en mouvement tous les éléments d’opposition en Allemagne, il ne se passa pas d’année sans que les paysans ne réapparaissent également avec leurs revendications. De 1518 à 1523, les insurrections paysannes locales se succédèrent dans la Forêt-Noire et dans la Haute-Souabe. À partir du printemps de 1524, ces insurrections prirent un caractère systématique. Au mois d’avril de cette année-là les paysans de l’abbaye de Marchthal refusèrent les corvées et prestations féodales. Au mois de mai les paysans de Saint-Blasien refusèrent de payer les taxes de servage. En juin, les paysans de Steinheim, près de Memmingen, déclarèrent ne plus vouloir payer ni dîmes ni autres taxes. En juillet et en août les paysans de Thurgovie se soulevèrent et furent ramenés au calme, soit grâce à l’intervention des Zurichois, soit par la brutalité du gouvernement fédéral, qui en fit exécuter un certain nombre. Enfin éclata dans le landgraviat de Stuhlingen une insurrection plus nette, qui peut être considérée comme le véritable \emph{début de la Guerre des paysans}.\par
Les paysans de Stuhlingen refusèrent brusquement toute prestation au landgrave, se réunirent en fortes bandes et marchèrent, le 24 août 1524, sous la direction de \emph{Hans Müller de Bulgenbach} sur Waldshut, où ils fondèrent en collaboration avec les bourgeois une confrérie évangélique. Les bourgeois adhérèrent d’autant plus volontiers à cette confrérie qu’ils étaient alors en conflit avec le gouvernement de la Haute-Autriche, à cause des persécutions religieuses contre \emph{Balthasar Hubmaier}, leur prédicateur, ami et disciple de Thomas Münzer. On imposa donc une taxe pour la ligue de trois kreuzers par semaine – ce qui était énorme pour l’époque. On envoya des émissaires en Alsace, sur la Moselle, sur tout le cours supérieur du Rhin et en Franconie, pour faire adhérer partout les paysans à la ligue et on proclama comme but de l’association l’abolition de la domination féodale, la destruction de tous les châteaux et de tous les monastères, et la suppression de tous les seigneurs, en dehors de l’empereur. Le drapeau de la ligue fut le \emph{drapeau tricolore allemand}.\par
L’insurrection s’étendit rapidement à tout l’actuel haut pays badois. Une terreur panique s’empara de la noblesse de la Haute-Souabe, dont presque toutes les troupes guerroyaient en Italie contre le roi de France François Iᵉʳ. Elle n’avait d’autre issue que de faire traîner l’affaire en longueur en négociant et, entre temps, de rassembler de l’argent et de recruter des troupes, jusqu’à ce qu’elle fût assez forte pour châtier les paysans de leur témérité « par le sac et l’incendie, le pillage et le meurtre ». C’est à partir de cette époque que commencèrent ces trahisons méthodiques, ces ruses, ces violations systématiques de la parole donnée, par lesquelles la noblesse et les princes se distinguèrent pendant toute la durée de la Guerre des paysans, et qui constituaient leur arme la plus forte contre les paysans dispersés et difficiles à organiser. La Ligue souabe, qui groupait les princes, la noblesse et les villes libres de l’Allemagne du Sud-Ouest, intervint, mais sans garantir aux paysans de concession positive. Ceux-ci restèrent donc en mouvement. Hans Müller de Bulgenbach parcourut, du 30 septembre à la mi-octobre, la Forêt-Noire jusqu’à Urach et Furtwangen, porta les effectifs de son armée à 3500 hommes et prit position près d’Ewatingen (non loin de Stuhlingen). La noblesse ne disposait que de 1700 hommes, et de plus ces troupes étaient dispersées. Elle fut donc obligée d’accepter un armistice, qui fut aussi réellement conclu au camp d’Ewatingen. On promit aux paysans qu’un accord amiable, soit conclu directement entre les parties intéressées, soit par l’intermédiaire d’arbitres, et l’étude de leurs doléances par le tribunal de Stockach. Les troupes de la noblesse, comme celles des paysans, se dispersèrent.\par
Les paysans se mirent d’accord sur 16 articles à présenter à l’approbation du tribunal de Stockach. Ces articles étaient très modérés. Abolition du droit de chasse, des corvées, des impôts les plus lourds et, d’une façon générale, des privilèges seigneuriaux garanties contre les arrestations arbitraires et contre les tribunaux jugeant selon leur bon plaisir ce fut tout ce qu’ils réclamèrent.\par
La noblesse, par contre, exigea, dès que les paysans furent rentrés chez eux, l’exécution immédiate de toutes les prestations litigieuses jusqu’à ce que le tribunal eût rendu sa sentence. Naturellement, les paysans refusèrent et renvoyèrent les seigneurs au tribunal. Le conflit éclata à nouveau. Les paysans se rassemblèrent à nouveau les princes et les nobles concentrèrent leurs troupes. Cette fois le mouvement s’étendit jusqu’au-delà du Brisgau et pénétra profondément dans le Wurtemberg. Les troupes, sous le commandement de Georg Truchsess de Waldburg, ce duc d’Albe de la Guerre des paysans, se contentèrent de les observer, battirent isolément des colonnes de renfort, mais n’osèrent attaquer le gros de l’armée paysanne. Georg Truchsess négocia avec les chefs paysans et réussit à conclure çà et là quelques accords.\par
À la fin du mois de décembre commencèrent les discussions devant le tribunal de Stockach. Les paysans protestèrent contre la composition du tribunal, formé uniquement de nobles. On leur lut en réponse une lettre d’installation de l’empereur. Les négociations traînèrent en longueur. Entre temps, la noblesse, les princes, la Ligue souabe s’armèrent. L’archiduc Ferdinand, qui outre les domaines héréditaires restés autrichiens dominait également le Wurtemberg, la Forêt-Noire badoise et le Sud de l’Alsace, ordonna la plus grande sévérité contre les paysans rebelles. Il fallait se saisir d’eux, les mettre à la torture et les tuer sans merci, il fallait les perdre par tous les moyens, incendier et ravager leurs biens et chasser du pays leurs femmes et leurs enfants. On voit comment les princes et les seigneurs respectaient l’armistice et ce qu’ils entendaient par accord amiable et examen des doléances. L’archiduc Ferdinand, auquel la maison Welser d’Augsbourg avait avancé de l’argent, arma en toute hâte. La Ligue souabe imposa un certain contingent d’argent et de troupes à fournir en trois échéances.\par
Toutes ces insurrections coïncident avec le séjour, qui dura cinq mois, de Thomas Münzer dans le Haut-Pays. Nous n’avons, à vrai dire, aucune preuve directe de l’influence qu’il exerça sur l’explosion et la marche du mouvement, mais indirectement cette influence est tout à fait établie. Les révolutionnaires les plus décidés parmi les paysans sont pour la plupart ses disciples et représentent ses idées. Les douze articles, de même que la lettre-article des paysans du Haut-Pays lui sont attribués par tous ses contemporains, quoique, du moins en ce qui concerne les premiers, il n’en soit certainement pas l’auteur. Encore sur le chemin du retour en Thuringe, il adressa un appel résolument révolutionnaire aux paysans insurgés.\par
En même temps, le duc Ulrich, chassé du Wurtemberg depuis 1519, intriguait pour rentrer en possession de son domaine avec l’aide des paysans. C’est un fait qu’il s’efforçait depuis le début de son exil d’utiliser le parti révolutionnaire et qu’il le soutint constamment. Son nom est impliqué dans la plupart des insurrections locales qui se sont succédé entre 1520 et 1524 dans la Forêt-Noire et le Wurtemberg et maintenant il armait ouvertement pour, de son château de Hohentwiel, faire une incursion dans le Wurtemberg. Cependant il ne fut qu’utilisé par les paysans, n’eut jamais aucune influence sur eux et encore moins leur confiance.\par
C’est ainsi que l’hiver s’écoula sans qu’aucune des deux parties entreprit rien de décisif. Les nobles princes se cachaient, l’insurrection paysanne s’étendait. Au mois de janvier 1525, tout le pays situé entre le Danube, le Rhin et la Lech était en pleine fermentation et au mois de février l’orage éclata.\par
Tandis que \emph{l’armée de la Forêt-Noire et de l’Hegau}, sous la direction de Hans Müller de Bulgenbach, conspirait avec Ulrich de Wurtemberg et participait pour une part à son entreprise malheureuse sur Stuttgart (février et mars 1525), les paysans du Ried, au-dessus d’Ulm, se soulevèrent le 9 février, se rassemblèrent dans un camp adossé aux marais près de Baltringen, arborèrent le \emph{drapeau rouge}, et constituèrent sous le commandement de Ulrich Schmid \emph{l’armée de Baltringen}, forte de 10 à 12 000 hommes.\par
Le 25 février, \emph{l’armée du Haut Allgäu}, forte de 7000 hommes, se rassembla sur les rives du Schussen, parce que le bruit courait que les troupes marchaient contre les mécontents apparus ici également. Les habitants de Kempten, qui avaient été tout l’hiver en conflit avec leur archevêque, se rassemblèrent le 26 et s’unirent à eux. Les villes de Memmingen et de Kaufbeuren adhérèrent au mouvement sous certaines conditions. Mais déjà se manifestait ici le caractère équivoque de la position prise par les villes dans la lutte. Le 7 mars, les douze articles de Memmingen y furent adoptés par tous les paysans de l’Oberallgäu.\par
Sur message des paysans de l’Allgäu, se forma au bord du lac de Constance \emph{l’armée du Lac}, sous la direction d’Eitel Hans. Ce groupe se renforça rapidement lui aussi. Son quartier général était à Bermatingen.\par
De même dans le Bas-Allgäu, dans la région d’Ochsenhausen et de Schellenberg, dans celle de Zeil et de Waldenbourg, seigneuries du sénéchal de Waldburg, les paysans se soulevèrent, et même dès les premiers jours de mars. Cette armée du \emph{Bas Allgäu}, forte de sept mille hommes, dressa son camp à Wurzach.\par
Ces quatre armées adoptèrent tous les articles de Memmingen, qui étaient d’ailleurs beaucoup plus modérés que ceux de l’Hegau, et qui, même sur les points concernant l’attitude des armées en armes à l’égard de la noblesse et des gouvernements, faisaient montre d’une curieuse absence de fermeté. La détermination, là où elle apparut, ne se manifesta qu’au cours de la guerre, après que les paysans eurent fait l’expérience de la façon d’agir de leurs adversaires.\par
En même temps que ces armées, il s’en constituait une sixième sur le Danube. De toute la région d’Ulm à Donauwoerth, des vallées de l’Iller, de la Roth et de la Biber, les paysans se rendirent à Leipheim et y établirent un camp. Quinze localités envoyèrent tous leurs hommes valides, 117 des contingents. Le chef de \emph{l’armée de Leipheim} était Ulrich Schön, son prédicateur Jakob Wehe, pasteur de Leipheim.\par
C’est ainsi qu’au début de mars il y avait dans la Haute-Souabe 30 à 40000 paysans insurgés en armes, répartis en six camps. Le caractère de ces armées paysannes était très varié. Le parti révolutionnaire (le parti de Münzer) y était partout en minorité. Cependant, il constituait partout le noyau et la force des camps de paysans. La masse des paysans était toujours prête à conclure un accord avec les seigneurs, à condition qu’on leur assurât les concessions qu’ils espéraient arracher par leur attitude menaçante. De plus, lorsque l’affaire traîna en longueur et que les armées des princes approchèrent, ils se lassèrent de faire la guerre et ceux qui avaient encore quelque chose à perdre rentrèrent pour la plupart chez eux. Avec cela, les armées s’étaient renforcées des vagabonds du Lumpenproletariat qui rendaient la discipline plus difficile, démoralisaient les paysans et s’en allaient aussi facilement qu’ils étaient venus. Cela suffit pour expliquer pourquoi, au début, les armées paysannes restèrent partout sur la défensive, pourquoi elles se démoralisèrent dans les camps et pourquoi, sans parler déjà de leur infériorité tactique et de la rareté de bons chefs, elles n’étaient en aucune façon capables de tenir tête aux armées des princes.\par
Pendant que les armées en étaient encore à se rassembler, le duc Ulrich quitta Hohentwiel à la tête de troupes recrutées et de quelques paysans de l’Hegau et pénétra dans le Wurtemberg. La Ligue souabe était perdue si de leur côté les paysans avaient marché contre les troupes du sénéchal de Waldburg. Mais l’attitude purement défensive des armées paysannes permit bientôt à ce dernier de conclure un armistice avec les gens de Baltringen, de l’Allgäu et du Lac, d’entamer des pourparlers et de fixer au dimanche de la Passion, le 2 avril, le règlement définitif de l’affaire. Pendant ce temps il put marcher contre le duc Ulrich, occuper Stuttgart et l’obliger dès le 17 mars à quitter à nouveau le Wurtemberg. Puis il allait se tourner contre les paysans, quand une mutinerie éclata brusquement dans son armée, les mercenaires se refusant à marcher contre ceux-ci. Mais le sénéchal réussit à apaiser les mutins et il marcha sur Ulm, où se rassemblaient de nouveaux renforts, laissant à Kirchheim-sous-Teck un camp d’observation.\par
La Ligue souabe, qui avait enfin les mains libres et avait réussi à rassembler ses premiers contingents, jeta alors le masque et déclara qu’elle « était décidée à prévenir par la force des armes et avec l’aide de Dieu toute action que les paysans oseraient entreprendre de leur propre volonté ».\par
Cependant les paysans avaient respecté scrupuleusement l’armistice. Ils avaient arrêté leurs revendications, les célèbres douze articles, pour les négociations fixées au dimanche de la Passion. Ils réclamaient pour les paroisses le droit d’élire et de révoquer elles-mêmes leurs pasteurs, la suppression de la petite dîme et l’utilisation, une fois payés les traitements des prêtres, de la grande dîme pour des buts d’utilité publique, l’abolition du servage, du droit de chasse et de pêche et de la mainmorte, la réduction des corvées, impôts et redevances excessifs, la restitution des bois, des prairies et privilèges arrachés par la violence aux communes et aux particuliers, et la suppression de l’arbitraire dans la justice et l’administration. On voit que le parti modéré, conciliant, prédominait encore parmi les armées paysannes. Le parti révolutionnaire avait déjà, dans la « \emph{lettre-article} », établi précédemment son programme. Cette lettre ouverte à toutes les communautés paysannes leur demandait « d’adhérer à l’association et confrérie chrétienne » pour abolir toutes les charges, soit à l’amiable, « ce qui sans doute ne sera pas possible », soit par la violence, et menaçait tous les récalcitrants du « ban séculier », c’est-à-dire de l’exclusion de la société et de la rupture de toutes relations avec les membres de la ligue. Tous les châteaux, monastères et abbayes devaient également être frappés du ban séculier, à moins que la noblesse, les prêtres et les moines les quittent volontairement et aillent habiter dans des maisons ordinaires, comme tout le monde, et adhèrent à l’association chrétienne. – Dans ce manifeste radical, rédigé visiblement avant l’insurrection du printemps 1525, il s’agit donc avant tout de la révolution, de la victoire complète sur les classes encore dominantes et le « ban séculier » vise seulement les oppresseurs et les traîtres qu’il fallait abattre, les châteaux qu’il fallait brûler, les monastères et les abbayes qu’il fallait confisquer et dont les trésors devaient être transformés en argent.\par
Mais avant que les paysans aient eu même le temps de présenter leurs douze articles aux arbitres qui avaient été convoqués dans ce but, leur parvint la nouvelle de la rupture de l’accord par la Ligue souabe et de l’approche des troupes. Ils prirent immédiatement les mesures nécessaires. Une assemblée générale des armées de Baltringen, de l’Allgäu et du Lac se réunit à Gaisbeuren. Les quatre groupes furent réunis en un seul, et l’on constitua avec eux quatre nouvelles colonnes. On décida la confiscation des biens ecclésiastiques, la vente de leurs joyaux au profit du trésor de guerre et l’incendie des châteaux. C’est ainsi qu’à côté des douze articles officiels la lettre-article devint leur règle de conduite de la guerre et que le dimanche de la Passion, jour précédemment fixé pour la conclusion de la paix, fut la date du \emph{soulèvement général}.\par
L’agitation partout croissante, les conflits locaux continuels des paysans avec la noblesse, la nouvelle de l’insurrection progressant depuis six mois dans la Forêt-Noire et de son extension jusqu’au Danube et à la Lech suffisent, certes, pour expliquer la succession rapide des insurrections paysannes dans les deux tiers de l’Allemagne. Mais le fait de la simultanéité de toutes ces insurrections potentielles prouve que le mouvement était dirigé par des gens qui l’avaient organisé à l’aide de leurs émissaires anabaptistes ou autres. Au cours de la seconde quinzaine de mars, des troubles avaient déjà éclaté dans le Wurtemberg, sur le Neckar inférieur, dans l’Odenwald et dans la Basse et Moyenne-Franconie, mais partout, le 2 avril, le dimanche de la Passion, était indiqué à l’avance comme la date du soulèvement général, partout le coup décisif, l’insurrection en masse éclata dans la première semaine d’avril. De même les paysans de l’Allgäu, de l’Hegau et du Lac convoquèrent au camp, le 1ᵉʳ avril, tous les hommes valides à l’aide du tocsin et d’assemblées de masse et ouvrirent, en même temps que ceux de Baltringen, les hostilités contre les châteaux et les monastères.\par
En \emph{Franconie}, où le mouvement se groupait autour de six centres principaux, l’insurrection éclata partout aux premiers jours d’avril. Près de \emph{Nördlingen} se constituèrent ces jours-là deux camps de paysans, avec l’aide desquels le parti révolutionnaire de la ville, dirigé par \emph{Anton Forner}, prit le dessus, fit nommer Forner bourgmestre et fit adopter l’adhésion de la cité au mouvement paysan. Dans la région d’Ansbach, les paysans se soulevèrent partout du 1ᵉʳ au 7 avril, et l’insurrection s’étendit de là jusqu’en Bavière. Dans la région de \emph{Rothenbourg}, les paysans étaient déjà en armes depuis le 22 mars dans la ville de Rothenbourg les petits bourgeois et les plébéiens, dirigés par Stephan von \emph{Menzingen}, renversèrent le 27 mars la domination du patriciat. Mais comme précisément les prestations des paysans constituaient les ressources principales de la ville, même le nouveau gouvernement eut une attitude très vacillante et très équivoque à l’égard de ceux-ci. Dans l’évêché de \emph{Wurzbourg}, les paysans et les petites villes se soulevèrent en bloc dès le début d’avril, et dans l’évêché de \emph{Bamberg}, l’insurrection générale obligea en moins de cinq jours l’évêque à céder. Enfin, dans le Nord, à la frontière de la Thuringe, se constitua le puissant \emph{camp de Bildhausen}.\par
Dans l’\emph{Odenwald}, où \emph{Wendel Hipler}, un noble, ancien chancelier dès comtes de Hohenlohe, et \emph{Georg Metzler}, aubergiste à Ballenberg, près de Krautheim, étaient à la tête du parti révolutionnaire, l’orage éclata dès le 26 mars. Les paysans accoururent de tous côtés vers la Tauber 2000 hommes du camp situé aux portes de Rothenbourg se joignirent aussi à eux. Georg Metzler en prit le commandement et marcha le 4 avril, après que tous les renforts furent arrivés, sur le monastère de Schoenthal, sur la Jagst, où il fut rejoint par des paysans de la \emph{vallée du Neckar}. Ces derniers, dirigés par \emph{Jäcklein Rohrbach}, aubergiste à Boeckingen, près d’Heilbronn, avaient le dimanche de la Passion proclamé l’insurrection à Fleim, à Sontheim, etc., pendant qu’au même moment Wendel Hipler, à la tête d’un certain nombre de conjurés, s’était emparé par surprise de Oehringen et avait gagné au mouvement les paysans des environs. À Schoenthal, les deux colonnes paysannes, réunies pour former l’« \emph{armée claire} », avaient adopté les douze articles et organisé des raids contre les châteaux et les cloîtres. La grande armée comprenait 8000 hommes et avait des canons et 3000 arquebuses. \emph{Florian Geyer}, un chevalier franconien, se joignit à eux et constitua la légion noire, un corps d’élite, recruté spécialement parmi les soldats de Rothenbourg et d’Oehringen.\par
Le bailli wurtembergeois de Neckarsulm, le comte Ludwig von Helfenstein, ouvrit les hostilités. Il fit abattre sans autre forme de procès tous les paysans qui tombèrent entre ses mains. La grande armée marcha à sa rencontre. Ces massacres, ainsi que la nouvelle parvenue entre temps de la défaite de l’armée de Leipheim, de l’exécution de Jakob Wehe et des atrocités commises par le sénéchal exaspérèrent les paysans. Ils attaquèrent le comte de Helfenstein, qui s’était jeté dans Weinsberg. Le château fut pris d’assaut par Florian Geyer, la ville occupée après une longue lutte, et le comte Ludwig fait prisonnier avec plusieurs chevaliers. Le lendemain, 17 avril, Jaeklein Rohrbach, assisté des plus résolus de l’armée, fit passer les prisonniers en jugement et fit exécuter quatorze d’entre eux à coups de pique, le comte de Helfenstein en tête. C’était la mort la plus ignominieuse qu’il pouvait leur infliger. La prise de Weinsberg et la vengeance terrible exercée par Jaeklein sur le comte de Helfenstein ne manquèrent pas leur effet sur la noblesse. Les comtes de Löwenstein adhérèrent à l’association paysanne, ceux de Hohenlohe, qui y avaient déjà adhéré plus tôt, mais qui n’avaient encore apporté aucun secours, envoyèrent immédiatement les canons et la poudre demandés.\par
Les chefs se concertèrent pour savoir s’ils choisiraient Götz von Berlichingen comme commandant, « étant donné qu’il pouvait leur apporter l’appui de la noblesse ». La proposition fut bien accueillie. Mais Florian Geyer, qui voyait dans cet état d’esprit des paysans et de leurs chefs le commencement de la réaction, quitta le groupe à la tête de sa légion noire, parcourut de son propre chef d’abord la région du Neckar, puis celle de Wurzbourg, et détruisit partout châteaux et nids de prêtres.\par
Le reste de l’armée marcha tout d’abord sur Heilbronn. Dans cette puissante ville libre, les notables avaient contre eux, comme presque partout, une opposition bourgeoise et une opposition révolutionnaire. Celle-ci secrètement d’accord avec les paysans, ouvrit dès le 17 avril, au cours d’une émeute, les portes de la ville à Georg Metzler et à Jäcklein Rohrbach. Les chefs paysans prirent avec leurs gens possession de la ville, qui adhéra à la confrérie, versa une somme de 1200 florins et offrit une compagnie de volontaires. Seuls, le clergé et les possessions de l’Ordre teutonique furent rançonnés. Le 22, les paysans quittèrent la ville, où ils laissèrent une petite garnison. Heilbronn devait devenir le centre des différentes armées, qui y envoyèrent d’ailleurs des délégués et délibérèrent sur une action commune et des revendications communes des paysans. Mais l’opposition bourgeoise et le patriciat, qui s’était allié à elle depuis l’arrivée des paysans, avaient réussi, entre temps, à reprendre le dessus et, s’opposant à toute action énergique, attendirent l’approche de l’armée des princes pour trahir définitivement les paysans.\par
Ceux-ci marchèrent sur l’Odenwald. Le 24 avril, Götz von Berlichingen qui, quelques jours plus tôt, avait offert ses services d’abord à l’électeur du Palatinat, puis aux paysans, puis de nouveau à l’électeur, dut adhérer à la confrérie évangélique et prendre le commandement de l’armée claire \emph{lumineuse} (ainsi appelée par opposition à l’armée \emph{noire} de Florian Geyer). Mais il était en même temps prisonnier des paysans qui le surveillaient de très près et le soumettaient au contrôle du conseil des chefs, sans lequel il ne pouvait rien faire. Götz et Metzler se rendirent alors, avec le gros des paysans, à Amorbach, en passant par Buchen. Ils y restèrent du 30 avril au 5 mai, et soulevèrent toute la région de Mayence. La noblesse fut partout contrainte d’adhérer au mouvement pour sauver ses châteaux. Seuls les monastères furent incendiés et livrés au pillage. La armée s’était de plus en plus démoralisée. Les éléments les plus énergiques étaient partis avec Florian Geyer ou avec Jaeklein Rohrbach, car ce dernier s’en était également séparé après la prise de Heilbronn, manifestement parce qu’ayant fait exécuter le comte Helfenstein il ne pouvait rester plus longtemps dans une armée qui était prête à pactiser avec la noblesse. Cette insistance à s’entendre avec celle-ci était déjà par elle-même un signe de démoralisation. Quelque temps après Wendel Hipler proposa une réorganisation très opportuné de l’armée. On devait engager les mercenaires qui offraient tous les jours leurs services et cesser de renouveler chaque mois l’armée en licenciant les anciens contingents et en en recrutant de nouveaux, mais conserver les hommes se trouvant déjà sous les armes et exercés. Mais l’assemblée de la commune repoussa ces deux propositions. Les paysans étaient déjà devenus présomptueux et ne voyaient plus dans la guerre qu’une razzia, où la concurrence des mercenaires ne leur disait rien et ils voulaient conserver la liberté de rentrer chez eux, dès qu’ils auraient rempli leurs poches. À Amorbach, le conseiller de Heilbronn, Hans Berlin, réussit même à faire adopter par les chefs et les conseils de l’armée la « déclaration des douze articles », document qui enlevait leur dernier mordant aux douze articles et qui mettait dans la bouche des paysans un langage d’humbles suppliants. Mais cette fois c’en était trop pour les paysans. Ils repoussèrent la déclaration au milieu d’un grand tumulte et déclarèrent vouloir s’en tenir aux douze articles primitifs.\par
Entre temps, la situation s’était complètement modifiée dans l’évêché de Wurzbourg. L’évêque qui, lors de la première insurrection des paysans au début d’avril, s’était retiré dans son château fort de Frauenberg, près de Wurzbourg et avait vainement appelé à l’aide de tous les côtés, avait été finalement contraint de céder momentanément. Le 2 mai, il ouvrit une diète, à laquelle participèrent également les représentants des paysans. Mais avant qu’on eût pu aboutir à un accord quelconque, on s’empara de lettres prouvant les manœuvres de trahison de l’évêque. La diète se sépara immédiatement et les hostilités commencèrent entre les citadins et les paysans insurgés d’une part, et les troupes de l’évêque d’autre part. L’évêque lui-même s’enfuit, le 5 mai, à Heidelberg. Dès le lendemain, Florian Geyer arriva à Wurzbourg, à la tête de sa légion noire, ainsi que l’\emph{armée franconienne de la Tauber}, composée de paysans de Mergentheim, de Rothenbourg et d’Ansbach. Le 7 mai arriva Götz von Berlichingen, à la tête de la grande armée claire, et le siège du château de Frauenberg commença.\par
Dans la région de Limpourg et dans celle d’Ellwangen et de Hall, se constitua dès la fin de mars et début d’avril une autre armée, celle de Gaildorf ou l’\emph{armée claire ordinaire}. Elle entra très violemment en scène, souleva tout le pays, incendia un grand nombre de monastères et de châteaux, entre autres celui d’Hohenstaufen, obligea tous les paysans à se joindre à elle et contraignit tous les nobles, et même les échansons de Limpourg, à adhérer à la confrérie chrétienne. Au début de mai, elle fit une incursion dans le Wurtemberg, mais fut convaincue de se retirer. Le particularisme des petits États allemands de l’époque ne permettait pas plus alors qu’en 1848 une action commune des révolutionnaires appartenant à des états différents. Les paysans de Gaildorf, réduits à un faible territoire, devaient nécessairement se disloquer, après avoir vaincu toute résistance sur ce territoire. Ils conclurent un accord avec la ville de Gmund, et se dispersèrent, après y avoir laissé un corps de 500 hommes armés.\par
Dans le \emph{Palatinat}, des armées de paysans s’étaient constituées vers la fin d’avril sur les deux rives du Rhin. Elles détruisirent un grand nombre de châteaux et de monastères et s’emparèrent, le 1ᵉʳ mai, de Neustadt-sous-la-Hardt après que les paysans venus du Bruchrain eurent la veille obligé Spire à un accord. Le maréchal Habern ne put rien entreprendre contre eux, avec le peu de troupes ducales dont il disposait, et le 10 mai l’électeur fut obligé de conclure un accord avec les paysans insurgés dans lequel il leur garantissait qu’une diète mettrait fin aux causes de leurs doléances.\par
Dans le \emph{Wurtemberg} enfin, l’insurrection avait déjà éclaté de bonne heure dans un certain nombre de régions. Sur l’Alpe d’Urach, les paysans avaient dès le mois de février conclu une alliance contre les prêtres et les seigneurs, et à la fin du mois de mars les paysans de Blaubeuren, d’Urach, de Munsingen, de Balingen et de Rosenfeld se soulevèrent. La armée de Gaildorf à Goeppingen, Jäcklein Rohrbach à Brakenheim, les débris de l’armée battue de Leipheim à Pfullingen pénétrèrent en territoire wurttembergeois et soulevèrent toute la population des campagnes. Des troubles sérieux éclatèrent également dans d’autres régions. Dès le 6 avril, Pfullingen dut capituler devant les paysans. Le gouvernement de l’archiduc autrichien était dans le plus extrême embarras. Il n’avait pas d’argent et très peu de troupes. Les villes et les châteaux étaient dans le plus mauvais état et n’avaient ni garnison ni munitions. Même l’Asperg était presque sans défense.\par
La tentative du gouvernement de rassembler contre les paysans les contingents des villes amena sa défaite momentanée. Le 16 avril, le contingent de Bottwar refusa de marcher et, au lieu de Stuttgart, se rendit sur le Wunnenstein, près de Bottwar, où il constitua le noyau d’un camp de paysans et de bourgeois qui grandit rapidement. Le même jour, l’insurrection éclata dans le Zabergäu. Le monastère de Maulbronn fut livré au pillage et un certain nombre de cloîtres et de châteaux complètement dévastés. De la région voisine du Bruchrain accoururent des renforts paysans.\par
La armée du Wunnenstein fut placée sous le commandement de \emph{Matern Feuerbacher}, conseiller de Bottwar, l’un des chefs de l’opposition bourgeoise, mais suffisamment compromis pour se voir obligé de faire cause commune avec les paysans. Cependant il resta toujours très modéré, s’opposa à l’application de la lettre-article aux châteaux, et s’efforça en toute occasion de s’entremettre entre les paysans et les bourgeois modérés. Il empêcha la jonction des paysans wurtembergeois avec la grande armée claire, et plus tard décida également l’armée de Gaildorf à sortir du Wurtemberg. Il fut, dès le 19 avril, destitué à cause de ses tendances bourgeoises, mais fut nommé à nouveau, le lendemain, commandant de la armée. Il était indispensable, et même lorsque Jaeklein Rohrbach, le 22, vint se joindre aux Wurtembergeois à la tête de ses 200 hommes résolus, il ne lui resta qu’à le laisser à son poste et à se contenter de le surveiller étroitement.\par
Le 18 avril, le gouvernement essaya d’entamer des négociations avec les paysans du Wunnenstein. Ceux-ci exigèrent qu’il acceptât les douze articles, ce que ne pouvaient naturellement pas les plénipotentiaires. La armée se mit alors en mouvement. Le 20, elle était à Lauffen, où les délégués du gouvernement furent éconduits une dernière fois. Le 22, elle arriva, forte de 6000 hommes, à Bietigheim et menaça Stuttgart. Dans cette ville, la majorité du Conseil avait pris la fuite et un comité de bourgeois s’était emparé du pouvoir. La bourgeoisie y était, comme partout ailleurs, divisée en patriciat, opposition bourgeoise et plébéiens révolutionnaires. Ces derniers ouvrirent, le 25 avril, les portes de la ville aux paysans, qui l’occupèrent immédiatement. L’organisation de l’\emph{armée claire chrétienne}, ainsi que se nommaient maintenant les insurgés wurtembergeois, fut réalisée jusque dans le détail, et la solde, la répartition du butin et le ravitaillement, etc. furent réglés d’une façon définitive. Une compagnie de Stuttgart, commandée par Theus Gerber, se joignit aux paysans.\par
Le 29 avril, Feuerbacher marcha, à la tête de sa armée. contre ceux de Gaildorf qui avaient pénétré dans le Wurtemberg à Schorndorf, rallia toute la région et les amena ainsi à se retirer. Il empêchait ainsi un dangereux renforcement des éléments révolutionnaires de sa armée dirigés par Rohrbach, renforcement qui aurait résulté de l’amalgame avec les paysans radicaux de Gaildorf. De Schorndorf, à la nouvelle de l’approche du sénéchal de Waldburg, il marcha contre ce dernier et établit son camp le 1ᵉʳ mai à Kirchheim-sous-Teck.\par
Nous avons ainsi décrit la naissance et le développement de l’insurrection dans cette partie de l’Allemagne que nous devons considérer comme le terrain du premier groupe des armées paysannes. Avant de passer aux autres groupes (Thuringe et Hesse, Alsace, Autriche et Alpes), nous devons suivre la campagne du sénéchal, au cours de laquelle, seul tout d’abord, puis soutenu par un certain nombre de princes et de villes, il anéantit le premier groupe d’insurgés.\par
Nous avons laissé le sénéchal prés d’Ulm, sur lequel il marcha vers la fin mars, après avoir laissé à Kirchheim-sous-Teck un corps d’observation, commandé par Dietrich Spät. Le corps du sénéchal, renforcé par les troupes de la Ligue concentrées à Ulm, comptait près de 10000 hommes, dont 7200 fantassins. C’était la seule armée avec laquelle on put entreprendre une offensive contre les paysans. Les renforts ne se concentraient que très lentement à Ulm, soit à cause des difficultés du recrutement dans des pays insurgés, soit à cause du manque d’argent des gouvernements, et parce que partout les quelques troupes disponibles étaient plus qu’indispensables pour tenir les forteresses et les châteaux. Nous avons déjà vu à quel point les troupes dont disposaient les princes et les villes qui n’appartenaient pas à la Ligue souabe étaient peu nombreuses. Tout dépendait donc des succès que remporterait Georg Truchsess, à la tête de l’armée de la Ligue.\par
Le sénéchal se porta d’abord contre l’\emph{armée de Baltringen}, qui avait commencé entre temps à dévaster châteaux et cloîtres des environs du Ried. Les paysans, qui s’étaient retirés à l’approche des troupes de la Ligue, furent chassés des marais par une manœuvre d’enveloppement, traversèrent le Danube et se jetèrent dans les gorges et les forêts de l’Alpe souabe. Là où la cavalerie et l’artillerie, qui constituaient la principale force de l’armée de la Ligue, étaient impuissantes contre eux, le sénéchal ne les poursuivit pas. Il se tourna alors contre les paysans de Leipheim, qui se tenaient au nombre de 5000 à Leipheim, de 4000 dans la vallée de la Mindel et de 6000 à Illertissen, soulevaient toute la contrée, détruisaient cloîtres et châteaux, et se préparaient, les trois colonnes réunies, à marcher sur Ulm. Il semble qu’ici également une certaine démoralisation se soit produite chez les paysans et ait détruit la valeur militaire de l’armée, car d’emblée Jakob Wehe essaya d’entamer des négociations avec le sénéchal. Mais ce dernier qui disposait maintenant d’une armée suffisante refusa tous pourparlers, attaqua le 4 avril le gros de l’armée près de Leipheim et la dispersa entièrement. Jakob Wehe et Ulrich Schön, ainsi que deux autres chefs paysans, furent faits prisonniers et décapités. Leipheim capitula et, après quelques incursions dans la région, toute la contrée fut soumise.\par
Une nouvelle rébellion des mercenaires qui réclamaient le droit de piller et une solde supplémentaire arrêta à nouveau Truchsess jusqu’au 10 avril. Puis il se porta dans la direction du Sud-Ouest conte l’armée de \emph{Baltringen}, qui avait entre temps envahi ses domaines de Waldbourg, de Zeil et de Wolfegg et assiégeait ses châteaux. Là aussi il trouva les paysans dispersés et les battit les 11 et 12 avril, les uns après les autres, au cours de combats séparés à la suite desquels l’armée de Baltringen fut également complètement désagrégée. Le reste, sous la direction du prêtre Florian, se replia sur l’armée du Lac. C’est contre celle-ci que se tourna alors le sénéchal. L’\emph{armée du Lac}, qui entre temps, non seulement avait fait des coups de main, mais avait aussi amené les villes de Buchhorn (Friedrichshafen) et de Wollmatingen à adhérer à la confrérie chrétienne, tint le 13 mai un grand conseil de guerre au monastère de Salem et décida de marcher contre le sénéchal.\par
On sonna immédiatement le tocsin partout, et 10 000 hommes, auxquels se joignirent encore les restes de la armée de Baltringen, se rassemblèrent au camp de Bermatingen. Ils soutinrent, le 15 avril, un combat heureux contre le sénéchal qui ne voulait pas mettre ici son armée en jeu en engageant une bataille décisive et préféra entamer des négociations, d’autant plus qu’il venait d’apprendre que les paysans de l’Allgäu et du Hegau approchaient. C’est pourquoi il conclut, le 17 avril, à Weingarten, avec les paysans du Lac et ceux de Baltringen, un accord en apparence assez favorable pour eux, qu’ils acceptèrent sans hésitation. Il réussit également à faire accepter cet accord par les délégués du Haut et du Bas-Allgäu et se retira ensuite vers le Wurtemberg.\par
La ruse du sénéchal le sauva d’une défaite certaine. S’il n’avait pas réussi à séduire les paysans faibles, bornés, en grande partie déjà démoralisés, ainsi que leurs chefs pour la plupart incapables, timides et corruptibles, il eut été enfermé, avec sa petite armée, entre quatre colonnes, fortes d’au moins 25 à 30000 hommes, et irrémédiablement perdu. Mais l’étroitesse bornée de ses adversaires, toujours inévitable chez les masses paysannes, lui permit de leur échapper précisément au moment où ils pouvaient d’un seul coup mettre fin à la guerre du moins en Souabe et en Franconie. Les paysans du Lac observèrent l’accord dans lequel, bien entendu, ils étaient finalement bernés avec tant de scrupule qu’ils prirent plus tard les armes contre leurs propres alliés, les hommes du Hegau. Quant à ceux de l’Allgäu, entraînés dans la trahison par leurs chefs, ils le répudièrent certes aussitôt, mais le sénéchal était déjà hors de danger.\par
Les paysans du Hegau, quoique non inclus dans l’accord de Weingarten, donnèrent immédiatement après une nouvelle preuve de l’étroitesse locale sans limite, du provincialisme obstiné qui causa la ruine de toute la Guerre des paysans. Après que le sénéchal eut négocié en vain avec eux et fut parti dans la direction du Wurtemberg, ils le suivirent et restèrent constamment sur ses flancs. Mais l’idée ne leur vint pas de s’unir à la grande armée chrétienne du Wurtemberg, sous prétexte que les paysans du Wurtemberg et de la vallée du Neckar avaient aussi une fois refusé de leur venir en aide. C’est pourquoi, lorsque le sénéchal se fut assez éloigné de leur pays, ils revinrent tranquillement sur leurs pas et marchèrent sur Fribourg.\par
Nous avons laissé les Wurtembergeois sous le commandement de Matern Feuerbacher à Kirchheim-sous-Teck, d’où le corps d’observation laissé par le sénéchal et commandé par Dietrich Spät s’était retiré sur Urach. Après une tentative infructueuse sur cette ville, Feuerbacher se tourna vers Nurtingen et écrivit à toutes les armées d’insurgés des environs pour leur demander des renforts en vue de la bataille décisive. Il reçut, en effet, des renforts considérables du bas pays wurtembergeois, ainsi que du Gau. Les paysans du Gau surtout, qui s’étaient groupés autour des débris de l’armée de Leipheim revenus dans l’ouest du Wurtemberg et avaient soulevé toute la haute vallée du Nagold et du Neckar, jusqu’à Boeblingen et Leonberg, accoururent en deux fortes colonnes et firent leur jonction le 5 mai à Nurtingen, avec Feuerbacher. À Boeblingen, le sénéchal se heurta aux deux bandes réunies. Leur nombre, l’importance de leur artillerie et la force de leur position le surprirent. Il commença immédiatement, selon sa méthode habituelle, à engager des négociations avec les paysans et conclut avec eux un armistice. À peine leur avait-il donné par là la sécurité qu’il les attaqua, le 12 mai, \emph{pendant l’armistice}, et les contraignit à livrer une bataille décisive. Les paysans résistèrent longtemps et vaillamment, jusqu’à ce qu’enfin Boeblingen fût livrée au sénéchal par la trahison de la bourgeoisie de la ville. L’aile gauche des paysans, ainsi privée de son point d’appui, fut repoussée, tournée. Dès lors, le sort de la bataille était décidé. Le désordre s’introduisit dans les rangs des paysans indisciplinés, désordre qui se transforma rapidement en une fuite éperdue. Tous ceux qui ne furent pas massacrés ou faits prisonniers par les cavaliers de la Ligue jetèrent leurs armes et retournèrent chez eux. La « grande armée chrétienne », et avec elle toute l’insurrection wurtembergeoise, était complètement désagrégée. Theus Gerber réussit à s’enfuir à Esslingen, Feuerbacher s’enfuit en Suisse, Jaeklein Rohrbach fut fait prisonnier et traîné enchaîné à Neckargartach, où le sénéchal le fit attacher à un poteau, amasser du bois autour et rôtir ainsi tout vif à petit feu, pendant que lui-même, buvant avec ses chevaliers, se délectait à ce spectacle bien chevaleresque.\par
De Neckargartach, le sénéchal soutint par une incursion dans le Kraichgau les opérations de l’électeur palatin. Ce dernier, qui avait réussi entre temps à rassembler des troupes, rompit l’accord qu’il avait conclu avec les paysans dès qu’il apprit les succès remportés par le sénéchal, envahit le 23 mai le Bruchrain, prit et incendia Malsch après une résistance acharnée, dévasta un certain nombre de villages et occupa Bruchsal. En même temps le sénéchal attaqua par surprise Eppingen et fit prisonnier le chef de l’insurrection locale, Anton Eisenhut, que l’électeur fit immédiatement exécuter avec une dizaine d’autres chefs paysans. Le Bruchrain et le Kraichgau ainsi pacifiés furent contraints de verser une rançon d’environ 40000 florins. Les deux armées, celle du sénéchal, réduite à 6000 hommes par suite des pertes subies, et celle de l’électeur, forte de 6500 hommes, se réunirent et marchèrent à la rencontre des paysans de l’Odenwald.\par
La nouvelle de la défaite de Boeblingen avait partout répandu la terreur parmi les insurgés. Les villes libres, qui avaient senti peser sur elles la lourde main des paysans, respirèrent soudain. Heilbronn fut la première à entreprendre des démarches pour se réconcilier avec la Ligue souabe. Heilbronn était le siège de la chancellerie paysanne et des délégations des différentes armées insurgées, chargées d’élaborer les propositions à présenter à l’empereur et à l’Empire au nom de tous les paysans insurgés. Ces négociations, qui devaient avoir un résultat général, valable pour toute l’Allemagne, montrèrent une fois de plus qu’aucun ordre, pas plus celui des paysans que les autres, n’était suffisamment développé pour donner en partant de son point de vue une forme nouvelle à l’ensemble de l’état de choses en Allemagne. Il apparut immédiatement qu’il fallait gagner dans ce but la noblesse et surtout la bourgeoisie. C’est pourquoi on confia à \emph{Wendel Hipler} la direction des négociations. Wendel Hipler était incontestablement, de tous les chefs du mouvement, celui qui comprenait le mieux la situation de l’époque. Ce n’était ni un révolutionnaire à larges vues, comme Münzer, ni un représentant des paysans, comme Metzler ou Rohrbach. Sa vaste expérience, sa connaissance pratique de la position respective des différents ordres, l’empêchaient de représenter l’un ou l’autre des ordres impliqués dans le mouvement à l’exclusion de tous les autres. De même que Münzer, représentant la classe placée complètement en dehors de la société officielle antérieure, les premiers éléments du prolétariat, fut amené à pressentir le communisme, de même Wendel Hipler, représentant pour ainsi dire la moyenne de tous les éléments progressifs de la nation, en arriva à pressentir la \emph{société bourgeoise moderne}. Les principes qu’il défendait, les revendications qu’il posait, n’étaient certes pas ce qu’il était possible d’obtenir immédiatement, mais le résultat nécessaire, quelque peu idéalisé, de la décomposition réelle de la société féodale et les paysans, dès qu’ils se préoccupaient de faire des projets de loi valables pour tout l’Empire, étaient obligés de les admettre. C’est ainsi que la centralisation demandée par les paysans prit ici, à Heilbronn, une forme plus positive, très différente cependant de ce que se représentaient les paysans. Elle se précisa en réclamant par exemple l’unification de la monnaie, des poids et mesures, la suppression des douanes intérieures, etc., bref, en des revendications qui étaient beaucoup plus dans l’intérêt de la bourgeoisie des villes que des paysans. C’est ainsi qu’on fit à la noblesse des concessions qui se rapprochaient considérablement des libérations du servage moderne et qui aboutissaient finalement à la transformation de la propriété foncière féodale en propriété bourgeoise. En un mot, dès que les revendications des paysans furent groupées sous forme d’une « réforme de l’Empire », elles durent se subordonner, non aux revendications momentanées, mais aux intérêts définitifs de la bourgeoisie.\par
Pendant qu’on discutait encore cette réforme de l’Empire à Heilbronn, Hans Berlin, l’auteur de la « Déclaration des douze articles », se rendait déjà au-devant du sénéchal pour négocier avec lui au nom du patriciat et de la bourgeoisie, la reddition de la ville. Des mouvements réactionnaires, qui éclatèrent dans la ville, appuyèrent la trahison et Wendel Hipler dut s’enfuir avec les paysans. Il se rendit à Weinsberg, où il s’efforça de rassembler les débris des Wurtembergois et les quelques troupes mobiles de l’armée de Gaildorf. Mais l’approche de l’électeur du Palatinat et du sénéchal l’obligea à s’enfuir plus loin, et c’est ainsi qu’il dut se rendre à Wurzbourg pour mettre en mouvement la grande armée claire. Pendant ce temps, les troupes de la Ligue et de l’électeur soumirent toute la région du Neckar, obligèrent les paysans à prêter à nouveau serment à leurs seigneurs, incendièrent un grand nombre de villages et massacrèrent ou pendirent tous les paysans fuyards dont ils purent s’emparer. En manière de représailles pour l’exécution du comte de Helfenstein, Weinsberg fut incendiée.\par
Les armées réunies devant Wurzbourg avaient, entre temps, assiégé le Frauenberg et le 15 mai, avant même que la brèche fut ouverte, tenté une attaque vaillante, mais vaine. Quatre cents guerriers des plus valeureux, appartenant pour la plupart à la légion de Florian Geyer, restèrent morts ou blessés dans les fossés. Deux jours plus tard, le 17, Wendel Hipler arriva et fit tenir conseil de guerre. Il proposa de ne laisser que 4000 hommes devant le Frauenberg, et avec le gros de l’armée, fort de 20000 hommes, d’établir à Krautheim-sur-la-Jagst, sous les yeux mêmes du sénéchal, un camp où l’on concentrerait tous les renforts. Le plan était excellent, car ce n’est que grâce à la concentration de toutes les forces et à leur supériorité numérique sur l’adversaire que l’on pouvait désormais espérer vaincre l’armée des princes, forte de 13000 hommes. Mais la démoralisation et le découragement avaient déjà fait trop de ravages parmi les paysans pour permettre encore une action énergique quelconque. Il semble que Götz von Berlichingen, qui se démasqua bientôt comme un traître, ait également contribué à retenir l’armée, et ainsi le plan de Hipler ne fut jamais exécuté. Tout au contraire, les armées continuèrent à être éparpillées. Ce n’est que le 23 mai que la grande armée claire se mit en mouvement, après que les Franconiens eurent promis de la suivre le plus rapidement possible. Le 26 mai, les contingents du margraviat d’Ansbach campés à Wurzbourg furent rappelés chez eux à la nouvelle que le margrave avait ouvert les hostilités contre les paysans. Le reste de l’armée des assiégeants, avec la légion noire de Florian Geyer, prit position près de Heidingsfeld, non loin de Wurzbourg.\par
La grande armée claire arriva le 24 mai à Krautheim assez peu prête à se battre. Là, un grand nombre de paysans apprirent que leurs villages avaient entre temps prêté serment au sénéchal de Waldburg, et ils profitèrent de ce prétexte pour rentrer chez eux. La armée continua sa route sur Neckarsulm et entama, le 28, des négociations avec le sénéchal. En même temps on envoya des émissaires aux Franconiens, aux Alsaciens et aux paysans de la Forêt-Noire et de l’Hegau, pour leur demander d’envoyer rapidement des renforts. De Neckarsulm, Götz von Berlichingen revint à Oehringen. L’armée fondait à vue d’œil. Götz lui-même disparut pendant la marche. Il était rentré chez lui, après avoir négocié avec le sénéchal, par l’intermédiaire de son vieux camarade de combat Dietrich Spät, son passage à la cause des princes. À Oehringen, par suite de fausses nouvelles sur l’approche de l’ennemi, une terreur panique s’empara de la masse des paysans déconcertés et découragés. L’armée se dispersa dans un désordre complet, et ce ne fut qu’à grand-peine que Metzler et Wendel arrivèrent à rassembler 2000 hommes, qu’ils ramenèrent à Krautheim. Entre temps, le contingent franconien fort de 5000 hommes était arrivé, mais une marche de flanc sur Oehringen par Loewenstein, manifestement ordonnée par Götz dans un but de trahison, lui fit manquer l’armée claire et il marcha sur Neckarsulm. Cette petite ville, occupée par quelques compagnies de la grande armée claire, fut assiégée par le sénéchal. Les Franconiens arrivèrent dans la nuit et virent les feux de camp des troupes de la Ligue, mais leurs chefs n’osèrent attaquer et se retirèrent sur Krautheim, où ils trouvèrent enfin le reste de la grande armée claire. Ne voyant arriver aucun secours, Neckarsulm se rendit, le 29, aux troupes de la Ligue. Le sénéchal fit immédiatement exécuter treize paysans et se porta à la rencontre de l’armée, en incendiant, pillant et massacrant tout sur son passage. Dans toute la vallée du Neckar, du Kocher et de la Jagst, les monceaux de décombres et les cadavres des paysans pendus aux arbres marquaient son passage.\par
À Krautheim, l’armée de la Ligue se heurta aux paysans qui, contraints par un mouvement de flanc du sénéchal, s’étaient retirés à Koenigshofen, sur la Tauber. Ils s’y retranchèrent, forts de 8000 hommes et de 32 canons. Le sénéchal s’approcha d’eux en s’abritant derrière les collines et les bois, fit avancer des colonnes d’enveloppement et attaqua le 2 juin avec une telle supériorité de forces et une telle vigueur que, malgré la résistance opiniâtre de plusieurs colonnes, qui se poursuivit jusque dans la nuit, les paysans furent complètement battus et dispersés. Là comme ailleurs, la cavalerie de la Ligue, la « mort des paysans », contribua principalement à l’anéantissement de l’armée des insurgés se jetant sur les paysans ébranlés par les salves d’artillerie et d’arquebuses et les attaques à la lance, elle les mit complètement en déroute et les massacra les uns après les autres. Ce qui montre de quelle façon le sénéchal faisait la guerre, c’est le sort des 300 bourgeois de Koenigshofen qui se trouvaient dans l’armée paysanne. Ils furent tous tués au cours de la bataille, à l’exception de quinze, et sur ces quinze, quatre encore furent décapités par la suite.\par
Après s’être ainsi débarrassé des paysans de l’Odenwald, de la vallée du Neckar et de la Basse-Franconie, le sénéchal soumit toute la région en procédant à des incursions, à l’incendie de villages entiers et à d’innombrables exécutions. Il marcha ensuite sur Wurzbourg. En chemin, il apprit que la deuxième armée Franconienne, sous le commandement de Florian Geyer et de Gregor von Burgbernheim, se trouvait à Sulzdorf, et il se porta immédiatement à sa rencontre.\par
Florian Geyer, qui, depuis l’échec de son attaque contre le Frauenberg, était occupé principalement de négocier avec les princes et les villes, surtout avec la ville de Rothenbourg et le margrave Kasimir d’Ansbach, leur adhésion à la confrérie des paysans, fut brusquement rappelé par la nouvelle de la défaite de Koenigshofen. Il se joignit avec son armée à celle d’Ansbach, conduite par Gregor von Burgbernheim. Celle-ci s’était tout nouvelle ment constituée. Le margrave Kasimir, en véritable Hohenzollern, avait réussi à tenir en échec l’insurrection paysanne sur son territoire, soit par des promesses, soit par des concentrations de troupes menaçantes. Il observa une neutralité complète à l’égard de toutes les armées paysannes, tant qu’elles n’attirèrent aucun sujet d’Ansbach. Il s’efforça surtout de diriger la haine des paysans vers les domaines ecclésiastiques, de la confiscation finale desquels il espérait s’enrichir. En même temps, il ne cessait de poursuivre ses préparatifs militaires en attendant les événements. Dès que fut parvenue la nouvelle de la bataille de Boeblingen, il ouvrit immédiatement les hostilités contre ses paysans rebelles, pilla et incendia leurs villages et en fit pendre et massacrer un grand nombre. Cependant les paysans se rassemblèrent rapidement et le battirent sous le commandement de Gregor von Burgbernheim, le 29 mai, à Windsheim. Pendant qu’ils étaient encore à sa poursuite, ils reçurent l’appel au secours des paysans de l’Odenwald serrés de près, et se rendirent aussitôt à Heidingsfeld, et de là, avec Florian Geyer de nouveau à Wurzbourg (2 juin). Toujours sans nouvelles de l’armée de l’Odenwald, ils y laissèrent un corps de 5000 hommes, et avec 4000 hommes – le reste s’était dispersé – ils partirent à la poursuite des autres. Rendus confiants par de fausses nouvelles concernant l’issue de la bataille de Koenigshofen, ils furent surpris à \emph{Sulzdorf}, par le sénéchal de Waldburg, et complètement battus. Là comme ailleurs, les cavaliers et les mercenaires du sénéchal se livrèrent à un massacre épouvantable. Florian Geyer rassembla les restes de la légion noire, soit 600 hommes en tout, et réussit à passer jusqu’au village d’Ingolstadt 200 hommes occupèrent le cimetière et l’église, 400 le château. Les troupes du Palatin l’y poursuivirent. Une colonne de 1200 hommes s’empara du village et mit le feu à l’église. Tout ce qui ne périt pas dans les flammes fut massacré. Puis les troupes palatines ouvrirent une brèche dans les murs délabrés du château et tentèrent l’assaut. Deux fois repoussés par les paysans qui se tenaient à l’abri d’un mur intérieur, ils démolirent ce mur à coups de canon et tentèrent un troisième assaut, qui cette fois réussit. La moitié des gens de Geyer furent massacrés, lui-même réussit à s’échapper avec deux cents survivants. Mais dès le lendemain (qui était le lundi de la Pentecôte) son refuge fut découvert. Les Palatins cernèrent le bois où il s’était caché et sabrèrent toute l’armée. Pendant ces deux journées, ils ne firent que 17 prisonniers. Florian Geyer réussit, encore une fois, à s’échapper avec quelques-uns de ses hommes les plus résolus. Il se rendit auprès de l’armée de Gaildorf, qui s’était de nouveau rassemblée, au nombre de 7000 hommes. Mais lorsqu’il arriva, il la trouva, par suite des nouvelles désastreuses qui parvenaient maintenant de tous côtés, à nouveau dissoute en majorité. Il essaya encore une fois de rassembler les hommes dispersés dans les forêts, mais il fut surpris le 9 juin à Hall par les troupes de la Ligue et tomba en combattant.\par
Le sénéchal qui, immédiatement après la victoire de Koenigshofen, en avait informé les assiégés de Frauenberg, marcha alors sur Wurzbourg. Le Conseil de cette ville s’entendit secrètement avec lui, de telle sorte que l’armée de la Ligue put, la nuit du 7 juin, cerner la ville avec les 5000 paysans qui s’y trouvaient et, dès le lendemain matin, pénétrer sans coup férir dans Wurzbourg dont les portes lui avaient été ouvertes par le Conseil. Grâce à cette trahison des notables de Wurzbourg, la dernière armée franconienne fut désarmée et tous ses chefs faits prisonniers. Le sénéchal en fit immédiatement décapiter 81. À Wurzbourg arrivèrent l’un après l’autre les différents princes franconiens, l’évêque de Wurzbourg lui-même, celui de Bamberg et le margrave de Brandebourg-Ansbach. Ces seigneurs se partagèrent les rôles. Le sénéchal partit avec l’évêque de Bamberg, qui rompit immédiatement l’accord qu’il avait conclu avec ses paysans et livra son pays aux hordes déchaînées d’incendiaires et d’assassins de l’armée de la Ligue. Le margrave Kasimir dévasta son propre pays. Deiningen fut incendiée. De nombreux villages furent livrés au pillage et aux flammes. En outre, le margrave tint dans chaque village une juridiction criminelle. À Neustadt-sur-Aisch, il fit décapiter dix-huit rebelles, et, à Bergel, quarante-trois. De là, il se rendit à Rothenbourg, où le patriciat avait déjà fait une contre-révolution et fait arrêter Stephan von Menzingen. Les petits bourgeois et les plébéiens de Rothenbourg durent alors payer cher d’avoir eu une attitude aussi équivoque à l’égard des paysans, de leur avoir jusqu’au dernier moment refusé tout secours, d’avoir, dans un esprit d’égoïsme local étroit, contribué à l’écrasement de l’industrie rurale au bénéfice des corporations de la ville, et de n’avoir renoncé que de mauvaise grâce aux revenus municipaux provenant des prestations féodales des paysans. Le margrave en fit décapiter seize, et en premier lieu naturellement, Menzingen. – L’évêque de Wurzbourg parcourut de la même façon son territoire, pillant, dévastant et incendiant. Dans sa marche victorieuse, il fit exécuter 256 rebelles et couronna son œuvre, à son retour à Wurzbourg, en y faisant décapiter encore treize habitants.\par
Dans la région de Mayence, le gouverneur évêque Guillaume de Strasbourg rétablit l’ordre sans rencontrer de résistance. Il ne fit exécuter que quatre personnes. Le Rheingau, qui s’était également agité, mais où, depuis longtemps, les paysans étaient rentrés chez eux, fut attaqué par surprise par Frowin von Hutten, le cousin d’Ulrich, et « pacifié » complètement par l’exécution de douze meneurs. Francfort, qui avait aussi connu d’importants mouvements révolutionnaires, avait été tenue en bride, d’abord par les concessions faites par le Conseil, plus tard par l’arrivée des troupes recrutées par la ville. Dans le Palatinat rhénan, à la suite de la rupture par l’électeur de l’accord qu’il avait signé, environ 8000 paysans s’étaient de nouveau rassemblés et avaient recommencé à incendier monastères et châteaux. L’archevêque de Trèves appela à son secours le maréchal Habern et les battit, dès le 23 mai à Pfeddersheim. Toute une série d’actes de cruauté (à Pfeddersheim seulement 82 paysans furent exécuté) et la prise de Wissembourg mirent fin, dès le 7 juillet, à l’insurrection.\par
De toutes les armées paysannes, il n’en restait plus que deux à vaincre, celle de l’Hegau et de la Forêt-Noire et celle de l’Allgäu. Avec toutes deux, l’archiduc Ferdinand avait intrigué. De même que le margrave Kasimir et d’autres princes avaient cherché à utiliser l’insurrection pour confisquer des terres et des principautés ecclésiastiques, de même l’archiduc s’était efforcé de s’en servir pour agrandir les domaines de la maison d’Autriche. Il avait entamé des négociations avec Walter Bach, le chef des paysans de l’Allgäu, et avec Hans Müller de Bulgenbach, qui commandait l’armée de l’Hegau, afin d’amener les paysans à se déclarer partisans de l’annexion à l’Autriche. Mais quoique l’un et l’autre fussent corruptibles, il leur fut impossible d’obtenir de leurs armées plus que la conclusion par les hommes de l’Allgäu d’un armistice avec l’archiduc et le respect de la neutralité vis-à-vis de l’Autriche.\par
Au cours de leur évacuation du Wurtemberg, les paysans de l’\emph{Hegau} avaient détruit un certain nombre de châteaux et reçu des renforts provenant des territoires du margrave de Bade. Le 13 mai, ils marchèrent sur Fribourg, qu’ils bombardèrent à partir du 16 et dans laquelle ils pénétrèrent, le 23, drapeaux déployés, après la capitulation de la ville. De là ils marchèrent sur Stockach et Radolfzell et guerroyèrent longtemps sans résultat contre les garnisons de la ville. Celles-ci, ainsi que la noblesse et les villes environnantes, appelèrent à leur secours, en vertu du traité de Weingarten, les paysans du Lac et les anciens insurgés de l’armée du Lac se levèrent au nombre de 5000 contre leurs propres alliés. Telle était l’étroitesse stupide de ces paysans que 600 seulement refusèrent de marcher, essayant de se joindre aux hommes de l’Hegau et furent massacrés. Mais les paysans de l’Hegau, à l’instigation de Hans Müller de Bulgenbach qui avait été acheté, avaient levé le siège et s’étaient dispersés pour la plupart lorsque Hans Müller aussitôt après prit la fuite. Les autres se retranchèrent sur les pentes d’Hiltzing, où ils furent, le 16 juillet, défaits et anéantis par les troupes devenues entre temps disponibles. Les villes suisses s’entremirent pour obtenir un accord pour les paysans de l’Hegau, accord qui n’empêcha pas qu’Hans Müller de Bulgenbach, malgré sa trahison, fut arrêté et décapité à Laufenbourg. Dans le Brisgau, Fribourg se détacha (17 juillet) de la ligue paysanne et envoya des troupes contre elle. Mais ici aussi à cause de la faiblesse numérique des forces des princes, un accord fut signé le 18 septembre à Offenbourg, dans lequel on engloba également le Sundgau. Les huit unions de la Forêt-Noire et les paysans du Klettgau, qui n’avaient pas encore été désarmés, furent encore une fois poussés à l’insurrection par la tyrannie du comte de Soultz, et battus en octobre. Le 13 novembre les paysans de la Forêt-Noire furent contraints de signer un accord, et le 6 décembre Waldshut, le dernier rempart de l’insurrection sur le Rhin supérieur, tomba.\par
Après la retraite du sénéchal, les paysans de l’\emph{Allgäu} avaient repris leur campagne contre les monastères et les châteaux et exercé des représailles énergiques pour les dévastations commises par les troupes de la Ligue. On ne leur opposa que peu de troupes, qui entreprirent contre eux de petites attaques isolées, mais ne purent jamais les suivre dans les forêts. En juin, un mouvement éclata à Memmingen qui s’était montrée relativement neutre, contre le patriciat, mouvement que seul permit de réprimer le voisinage de quelques troupes de la Ligue, qui purent venir à temps porter secours aux notables. Schappeler, prédicateur et chef du mouvement plébéien, s’enfuit à Saint-Gall. Les paysans marchèrent sur la ville et ils se préparaient à ouvrir une brèche dans les murs lorsqu’ils apprirent que le sénéchal approchait venant de Wurzbourg. Le 27 juin, ils marchèrent à sa rencontre, en deux colonnes, par Babenhausen et Obergunzbourg. L’archiduc Ferdinand fit alors une nouvelle tentative pour gagner les paysans à la maison d’Autiche. S’appuyant sur l’armistice qu’il avait conclu avec eux, il donna l’ordre au sénéchal d’arrêter les hostilités contre les paysans. Mais la Ligue souabe lui ordonna de les attaquer et d’arrêter seulement les massacres et les incendies. Cependant le sénéchal était bien trop intelligent pour renoncer à sa meilleure arme, même s’il lui eût été possible de brider ses mercenaires, qu’il avait conduits d’exactions en exactions du lac de Constance jusqu’au Main. Les paysans, au nombre de 23000, prirent position derrière la Iller et la Leubas. Le sénéchal se plaça en face d’eux, à la tête d’une armée de 11000 hommes. Les positions des deux armées étaient fortes. La cavalerie ne pouvait pas agir sur ce terrain, et si les mercenaires du sénéchal étaient supérieurs aux paysans au point de vue de l’organisation, des moyens militaires et de la discipline, les paysans de l’Allgäu comptaient dans leurs rangs un grand nombre de soldats aguerris et de capitaines expérimentés et disposaient d’une artillerie nombreuse et bien servie. Le 19 juillet, les troupes de la Ligue ouvrirent un feu d’artillerie, qui se poursuivit le 20 des deux côtés, mais sans résultat.\par
Le 21, Georg von Frundsberg, avec 300 mercenaires, se joigna au sénéchal. Il connaissait personnellement un grand nombre des chefs paysans, qui avaient servi sous ses ordres dans les campagnes d’Italie et entama des pourparlers avec eux. La trahison réussit là où les moyens militaires avaient échoué. Walter Bach, plusieurs autres capitaines et commandants d’artillerie se laissèrent acheter. Ils firent incendier toute la réserve de poudre des paysans et décidèrent ces derniers à faire une tentative d’enveloppement. Mais à peine les paysans avaient-ils quitté leurs solides positions qu’ils tombèrent dans le guet-apens que le sénéchal, d’accord avec Bach et les autres chefs traîtres, leur avait tendu. Ils pouvaient d’autant moins se défendre que leurs chefs les avaient quittés, sous le prétexte d’une reconnaissance à faire et se trouvaient déjà en route pour la Suisse. Deux des colonnes paysannes furent ainsi complètement détruites, la troisième put encore se retirer en bon ordre sous la direction de Knopf de Leubas. Elle se rétablit sur le Kollenberg près de Kempten, où le sénéchal la cerna. Mais là non plus il n’osa pas l’attaquer. Il se contenta de lui couper le ravitaillement et s’efforça de la démoraliser en faisant incendier près de 200 villages des environs. La famine et le spectacle de leurs habitations en flammes décidèrent finalement les paysans à se rendre (25 juillet). Plus de vingt d’entre eux furent immédiatement exécutés. Knopf de Leubas, le seul chef de cette armée qui n’eût pas trahi son drapeau, réussit à s’enfuir à Bregenz. Mais là il fut arrêté et pendu après un long séjour en prison.\par
Ainsi se termina la Guerre des paysans souabes et franconiens.
\chapterclose


\chapteropen
\renewcommand{\leftmark}{VI. La guerre des paysans en Thuringe, en Alsace et en Autriche}
\chapter[VI. La guerre des paysans en Thuringe, en Alsace et en Autriche]{VI. La guerre des paysans en Thuringe, en Alsace et en Autriche}

\chaptercont
\noindent Dès le début des premiers mouvements en Souabe, \emph{Thomas Münzer} était revenu en toute hâte en \emph{Thuringe} et s’était établi, fin février où début de mars, dans la ville libre de \emph{Mulhausen} où ses partisans étaient le plus nombreux. Il tenait dans ses mains les fils de tout le mouvement. Il savait quelle tempête générale était sur le point de se déchaîner sur l’Allemagne du Sud et avait pris sur lui de faire de la Thuringe le centre du mouvement pour l’Allemagne du Nord. Il trouva un terrain tout à fait fécond. La Thuringe elle-même, principal centre de la Réforme, était extrêmement agitée. La misère matérielle des paysans, ainsi que les doctrines révolutionnaires religieuses et politiques en circulation, avaient également préparé les contrées voisines : la Hesse, la Saxe et le Hartz à une insurrection générale. À Mulhausen en particulier, la grande masse de la petite bourgeoisie était gagnée aux idées extrêmes de Münzer et était impatiente de faire valoir sa supériorité numérique sur l’orgueilleux patriciat. Münzer lui-même se vit contraint, pour ne pas anticiper sur le moment opportun, de modérer l’ardeur de ses partisans. Mais son disciple Pfeifer, qui dirigeait le mouvement à Mulhausen, s’était déjà tellement compromis qu’il se trouvait dans l’impossibilité de retarder l’explosion, et c’est pourquoi, dès le 17 mars 1525, avant même le soulèvement général dans l’Allemagne du Sud, Mulhausen fit sa révolution. Le vieux Conseil patricien fut renversé et le gouvernement de la ville confié au nouveau « Conseil éternel », dont Münzer fut nommé président.\par
C’est le pire qui puisse arriver au chef d’un parti extrême que d’être obligé d’assumer le pouvoir à une époque où le mouvement n’est pas encore mûr pour la domination de la classe qu’il représente et pour l’application des mesures qu’exige la domination de cette classe. Ce qu’il \emph{peut} faire ne dépend pas de sa volonté, mais du stade où en est arrivé l’antagonisme des différentes classes et du degré de développement des conditions d’existence matérielles et des rapports de production et d’échange, qui déterminent, à chaque moment donné, le degré de développement des oppositions de classes. Ce qu’il \emph{doit} faire, ce que son propre parti exige de lui ne dépend pas non plus de lui pas plus que du degré de développement de la lutte de classe et de ses conditions. Il est lié aux doctrines qu’il a enseignées et aux revendications qu’il a posées jusque-là, doctrines et revendications qui ne sont pas issues de la position momentanée des classes sociales en présence et de l’état momentané, plus ou moins contingent, des rapports de production et d’échange, mais de sa compréhension plus ou moins grande des résultats généraux du mouvement social et politique. Il se trouve ainsi nécessairement placé devant un dilemme insoluble : ce qu’il \emph{peut} faire contredit toute son action passée, ses principes et les intérêts immédiats de son parti et ce qu’il doit faire est irréalisable. En un mot, il est obligé de ne pas représenter son parti, sa classe, mais la classe pour la domination de laquelle le mouvement est précisément là. Il est obligé, dans l’intérêt de tout le mouvement, de réaliser les intérêts d’une classe qui lui est étrangère et de payer sa propre classe de phrases, de promesses et de l’assurrance que les intérêts de cette classe étrangère sont ses propres intérêts. Quiconque tombe dans cette situation fausse est irrémédiablement perdu. Nous en avons eu encore tout récemment des exemples. Rappelons seulement la position qu’adoptèrent les représentants du prolétariat dans le dernier gouvernement provisoire français, quoiqu’ils ne représentent eux-mêmes qu’un stade très inférieur du développement du prolétariat. Quiconque, après l’expérience du gouvernement de février, – pour ne rien dire de nos nobles gouvernements provisoires allemands et de nos régences d’Empire – peut spéculer encore sur des positions officielles, doit où bien être borné au-delà de toute mesure où n’appartenir qu’en paroles seulement au parti révolutionnaire extrême.\par
La position de Münzer à la tête du Conseil éternel de Mulhausen était cependant beaucoup plus risquée encore que celle de n’importe quel gouvernant révolutionnaire moderne. Non seulement le mouvement de l’époque, mais aussi son siècle n’étaient pas encore mûr pour la réalisation des idées qu’il avait seulement commencé lui-même à pressentir confusément. La classe qu’il représentait, bien loin d’être complètement développée et capable de dominer et de transformer toute la société, ne faisait que de naître. La transformation sociale qui hantait son imagination était encore si peu fondé dans les conditions matérielles de l’époque que ces dernières préparaient même un ordre social qui était précisément le contraire de celui qu’il rêvait d’instituer. Cependant, il restait lié a ses anciens prêches sur égalité chrétienne et la communauté évangélique des biens. Il devait donc tout au moins essayer de les mettre en application. C’est pourquoi il proclama la communauté des biens, l’obligation au travail égale pour tous et la suppression de toute autorité. Mais en réalité Mulhausen resta une ville libre républicaine, avec une Constitution un peu démocratisée, avec un Sénat élu au suffrage universel soumis au contrôle de l’assemblée des citoyens et un système de ravitaillement des pauvres improvisé à la hâte. La révolution sociale, qui épouvantait à tel point les contemporains bourgeois protestants, n’alla jamais en fait au-delà d’une faible et iconsciente tentative pour instaurer prématurément la future société bourgeoise.\par
Münzer lui-même semble avoir senti l’abîme existant entre ses théories et la réalité qu’il avait directement devant lui, abîme qui pouvait d’autant moins lui rester caché que le reflet de ses conceptions géniales devait être plus déformé dans les têtes incultes de la grande masse de ses partisans. Il se lança avec un zèle inouï, même chez lui, dans l’extension et l’organisation du mouvement. Il rédigeait des messages et envoyait des courriers et des émissaires dans toutes les directions. Ses écrits et ses prêches respirent un fanatique révolutionnaire qui étonne, même après ses précédents ouvrages. L’humour jeune et naïf de ses pamphlets révolutionnaires a complètement disparu. Le langage serein et didactique du penseur, qui ne lui était pas étranger jusque-là, n’apparaît plus. Münzer est désormais tout entier le prophète de la révolution. Il attise sans arrêt la haine contre les classes dominantes, il excite les passions les plus effrénées et n’emploie plus que les tournures violentes que met dans sa bouche le délire religieux et national des prophètes de l’Ancien Testament. On se rend compte, d’après le style qu’il devait dès lors s’approprier, quel était le niveau de culture du public sur lequel il devait agir.\par
L’exemple de Mulhausen et l’agitation de Münzer agirent bien vite au loin. En \emph{Thuringe}, dans l’\emph{Eichsfeld}, le \emph{Hartz}, les \emph{duchés saxons}, dans la \emph{Hesse} et la \emph{Fulda}, en \emph{Haute-Franconie} et dans le \emph{Vogtland}, les paysans se soulevèrent partout, se rassemblèrent en armées et incendièrent châteaux et monastères. Münzer était plus ou moins reconnu comme le chef de tout le mouvement et Mulhausen en resta le point central tandis qu’à Erfurt triomphait un mouvement purement bourgeois et que le parti dominant y observa constamment une attitude équivoque à l’égard des paysans.\par
Les princes se trouvèrent au début en Thuringe tout aussi décontenancés et impuissants en face des paysans qu’en Franconie et en Souabe. Ce n’est que dans les derniers jours d’avril que le landgrave de Hesse réussit à rassembler une armée – ce même landgrave Philippe, dont les historiens protestants et bourgeois de la Réforme ne savent pas assez vanter la piété. Nous allons donner immédiatement un petit aperçu des infamies commises par lui contre les paysans. Le landgrave Philippe soumit bientôt, au moyen de quelques rapides expéditions et d’une action énergique, la plus grande partie de son pays, leva de nouveaux contingents et pénétra ensuite sur le territoire de l’abbé de Fulda, jusqu’alors son suzerain. Le 3 mai, il battit sur le Frauenberg la armée des paysans de Fulda, soumit toute la contrée et saisit l’occasion, non seulement pour se défaire de la suzeraineté de l’abbé, mais même pour transformer l’abbaye de Fulda en un fief hessois – sous réserve naturellement de sa sécularisation ultérieure. Puis il s’empara d’Eisenach et de Langensalza, et marcha, après avoir fait sa jonction avec les troupes du duc de Saxe, contre le centre principal de la rébellion, Mulhausen. Münzer rassembla ses forces, environ 8000 hommes avec quelques canons, à Frankenhausen. La armée thuringienne était très loin de posséder la combativité dont firent preuve une partie des armées de la Haute-Souabe et de Franconie en face du sénéchal de Waldburg. Elle était mal armée et peu disciplinée, comptait peu de soldats aguerris et manquait absolument de chefs. Münzer lui-même ne possédait manifestement pas les moindres connaissances militaires. Cependant les princes trouvèrent opportun d’employer également ici la tactique qui avait permis si souvent au sénéchal de remporter la victoire : le parjure. Le 16 mai ils entamèrent des négociations avec les paysans, conclurent avec eux un armistice et les assaillirent brusquement, avant même qu’il fut expiré.\par
Münzer se tenait avec les siens sur la hauteur appelée encore aujourd’hui le Schlachtberg, retranché derrière une barricade de chariots. Le découragement grandissait déjà considérablement parmi les paysans. Les princes promirent une amnistie générale si les paysans consentaient à leur livrer Münzer vivant. Ce dernier fit former un cercle pour discuter les propositions des princes. Un chevalier et un prêtre se prononcèrent pour la capitulation. Münzer les fit immédiatement placer au milieu du cercle et décapiter. Cet acte d’énergie terroriste, accueilli avec enthousiasme par les révolutionnaires résolus, raffermit le moral des paysans. Mais ils se seraient malgré tout finalement dispersés pour la plupart sans opposer de résistance, s’ils n’avaient pas remarqué que les mercenaires des princes, après avoir cerné toute la hauteur, avançaient en colonnes serrées, malgré l’armistice. Rapidement les paysans se mirent en ordre de bataille derrière les chariots. Mais déjà les balles et les obus attaignaient les paysans à demi désarmées et inexpérimentés, déjà les mercenaires arrivaient au niveau de la barricade de chariots. Après une courte résistance, la ligne de chariots fut forcée, les canons des paysans furent pris et eux-mêmes dispersés. Ils s’enfuirent dans un désordre épouvantable, pour tomber d’autant plus sûrement aux mains des colonnes d’enveloppement et de la cavalerie, qui en firent un massacre inouï. Sur huit mille paysans, cinq mille furent massacrés. Le reste entra dans Frankenhausen, suivi de près par la cavalerie des princes. La ville fut prise. Münzer blessé à la tête fut découvert dans une maison et fait prisonnier. Le 25 mai, Mulhausen se rendit également. Pfeifer qui y était resté, réussit à s’enfuir, mais il fut arrêté sur le territoire d’Eisenach.\par
Münzer fut mis à la torture en présence des princes et décapité. Il se rendit sur le lieu du supplice avec le même courage qu’il avait manifesté toute sa vie. Il avait tout au plus vingt-huit ans lorsqu’il fut exécuté. Pfeifer aussi fut décapité mais outre ces deux-là, encore une foule d’autres. À Fulda l’homme de Dieu, Philippe de Hesse, avait commencé à exercer sa justice sanglante. Les princes saxons et lui firent entre autres exécuter 24 personnes à Eisenach, 41 à Langensalza, 300 après la bataille de Frankenhausen, à Mulhausen plus de 100, à Goermar 26, à Tungeda 50, à Sangerhausen 12, à Leipzig 8, pour ne rien dire des mutilations et autres moyens plus bénins, des pillages et incendies de villages et de villes.\par
Mulhausen perdit sa qualité de ville libre et fut annexée aux territoires saxons, tout comme l’abbaye de Fulda au landgraviat de Hesse.\par
Les princes franchirent ensuite la forêt de Thuringe, où des paysans franconiens du camp de Bildhausen s’étaient joints aux paysans thuringiens et avaient incendié un grand nombre de châteaux. La bataille s’engagea devant Meiningen. Les paysans furent battus et se replièrent sur la ville. Celle-ci leur ferma brusquement ses portes et menaça de les attaquer par derrière. Les paysans, acculés par cette trahison de leur alliés, capitulèrent devant les princes et se dispersèrent sans attendre la fin des négociations. Le camp de Bildhausen s’était depuis longtemps dispersé, et c’est ainsi qu’avec la mise en déroute de cette armée fut anéanti le dernier reste des insurgés de la Saxe, de la Hesse, de la Thuringe et de la Haute-Franconie.\par
En \emph{Alsace} l’insurrection avait éclaté plus tard que sur la rive droite du Rhin. Ce n’est que vers la mi-avril que les paysans de l’évêché de Strasbourg, puis peu après ceux de la Haute-Alsace et du Sundgau se soulevèrent. Le 18 avril une armée de paysans de la Basse Alsace pilla le monastère d’Altdorf. D’autres armées de paysans se constituèrent près d’Ebersheim et de Barr, ainsi que dans la vallée de la Willer et de l’Urbis. Toutes ces armées s’unirent pour former la grande armée de la Basse-Alsace, organisèrent la prise des villes et des bourgs et la destruction des monastères. Partout on mobilisa un homme sur trois. Les douze articles de cette armée sont beaucoup plus radicaux que ceux des paysans souabes et franconiens.\par
Tandis qu’une colonne de paysans de Basse-Alsace se rassemblait au début de mai près de Saint-Hippolyte et, après une vaine tentative de s’emparer de cette ville, entrait le 10 mai à Barken, le 13 à Ribeauvillé, le 14 à Riquewihr, de connivence avec les bourgeois de ces différentes villes, une deuxième colonne, ayant à sa tête Erasmus Gerber, marcha sur Strasbourg pour s’en emparer par surprise. La tentative échoua. La colonne se tourna alors dans la direction des Vosges, détruisit le cloître de Marmoutier et assiégea Saverne, qui se rendit le 13 mai. De là, elle marcha sur la frontière lorraine et souleva la partie contiguë du duché, tandis qu’en même temps elle fortifiait les défilés de la montagne. De grands camps furent constitués à Herbitzheim, sur la Sarre et à Neubourg. À Sarreguemines 4000 paysans lorrains-allemands se retranchèrent. Deux colonnes avancées, celle de Kolben dans les Vosges à Sturzelbronn, et celle de Kleeburgen, à Wissembourg, couvrirent le front et le flanc droit tandis que le flanc gauche s’appuyait sur les paysans de la Haute-Alsace.\par
Ces derniers, en mouvement depuis le 20 avril, avaient obligé le 10 mai Soultz, le 12 Guebwiller, le 15 Cernay et les environs à adhérer à la confrérie paysanne. Le gouvernement autrichien et les villes libres des environs se liguèrent certes immédiatement contre eux, mais furent hors d’état de leur opposer une résistance sérieuse, à plus forte raison de les attaquer. C’est ainsi qu’à l’exception d’un petit nombre de villes l’Alsace tout entière était, vers le milieu de mai, aux mains des insurgés.\par
Mais déjà s’approchait l’armée qui devait châtier l’audace impie des paysans alsaciens. Ce furent des \emph{Français} qui restaurèrent ici la domination féodale. Dès le 6 mai le duc Antoine de Lorraine se mit en marche à la tête d’une armée de 30000 hommes, comprenant la fleur de la noblesse française ainsi que des mercenaires espagnols, piémontais, lombards, grecs et albanais. Le 16 mai, il rencontra à La Petite Pierre 4000 paysans, qu’il battit sans difficulté, et dès le 17 il oblige la ville de Saverne occupée par les paysans à capituler. Mais pendant que les troupes lorraines pénétraient dans la ville et désarmaient les paysans, la capitulation était violée. Les paysans désarmés furent assaillis par les mercenaires et massacrés pour la plupart. Les autres colonnes de la Basse-Alsace se dispersèrent, et le duc Antoine marcha à la rencontre des armées de la Haute-Alsace. Ces dernières, qui s’étaient refusées à envoyer des secours aux paysans de la Basse-Alsace à Saverne, furent attaquées à Scherwiller par toutes les forces lorraines réunies. Les paysans se défendirent vaillamment, mais l’énorme supériorité numérique de l’advesaire – 30000 contre 7000 – et la trahison d’un certain nombre de chevaliers, particulièrement celle du bailli de Riquewihr, rendit toute bravoure inutile. Ils furent complètement battus et mis en déroute. Le duc soumit toute l’Alsace avec la cruauté habituelle aux princes. Sa présence fut épargnée au seul Sundgau. Ici le gouvernement autrichien décida les paysans, sous menace d’appeler le duc dans le pays, à conclure au début de juin l’accord d’Ensisheim. Mais il rompit aussitôt lui-même cet accord et fit pendre en masse les prédicateurs et les chefs du mouvement. Les paysans firent, là-dessus, une nouvelle insurrection, qui se termina finalement par l’inclusion des paysans du Sundgau dans le traité d’Offenbourg (18 septembre).\par
Il ne nous reste plus maintenant qu’à dire quelques mots de la Guerre des paysans dans les \emph{Alpes autrichiennes}. Ces contrées, ainsi que l’\emph{evêché de Salzbourg} y attenant, se trouvaient depuis les stara prava en opposition perpétuelle avec le gouvernement et la noblesse, et les doctrines de la Réforme avaient là aussi touvé un terrain favorable. Des persécutions religieuses et des impôts arbitraires et accablants provoquèrent l’insurrection.\par
La ville de \emph{Salzbourg}, soutenue par les paysans et les mineurs des environs, était depuis 1522 en conflit avec l’archevêque au sujet de ses privilèges municipaux et de l’excercice du culte. À la fin de 1524, l’archevêque attaqua la ville par surprise avec des mercenaires qu’il avait recrutés, utilisa les canons du château pour la terroriser et poursuivit les prédicateurs hérétiques. En même temps il édicta de nouveaux et lourds impôts et excita ainsi à l’extrême la population. Au printemps de 1525, en même temps qu’éclatait l’insurrection des paysans souabes, franconiens et thuringiens, les paysans et les mineurs de toute la contrée se soulevèrent brusquement, s’organisèrent en armées, sous le commandement des capitaines \emph{Prassler} et \emph{Weitmoser} délivrant la ville et assiégèrent le château de Salzbourg. Ils fondèrent, comme les paysans de l’Allemagne occidentale, une ligue chrétienne et résumèrent leurs revendications en articles ici au nombre de quatorze.\par
De même, en \emph{Styrie}, dans la \emph{Haute-Autriche}, en \emph{Carinthie} et en \emph{Carniole}, où de nouveau impôts, droits de douanes et décrets illégaux avaient lourdement atteint le peuple dans ses intérêts les plus directs, l’insurrection des paysans éclate au printemps de 1525. Ils s’emparèrent d’un certain nombre de châteaux et battirent le vainqueur des stara prawa, le vieux capitaine Dietrichstein, à Griess. Quoique le gouvernement réussit par des promesses illusoires à calmer une partie des insurgés, la plupart d’entre eux restèrent rassemblés et se joignirent aux insurgés de Salzbourg, de telle sorte que toute la région de Salzbourg et la plus grande partie de la Haute-Autriche, de la Styrie, de la Carinthie et de la Carniole étaient entre les mains des paysans et des mineurs.\par
Au Tyrol, les doctrines de la Réforme avaient trouvé également de nombreux adhérents. Plus encore que dans les autres régions des Alpes autrichiennes, des émissaires de Münzer y avaient déployé une activité couronnée de succès. L’archiduc Ferdinand poursuivit là aussi les prédicateurs de la nouvelle doctrine et porta également, par de nouveaux règlements fiscaux arbitraires, atteinte aux privilèges de la population. La conséquence fut que l’insurrection éclata, comme partout ailleurs au printemps de 1525. Les insurgés, dont le chef suprême Geismaier était un disciple de Münzer, le seul chef paysan qui possédait un éminent talent militaire, s’emparèrent d’une grande quantité de châteaux et entreprirent une action très énergique contre les prêtres, surtout dans le Sud, dans la région de l’Adige. Les paysans du Vorarlberg, se soulevèrent également et s’unirent aux paysans de l’Allgäu.\par
L’archiduc acculé de tous côtés fit concession sur concession aux rebelles, qu’il voulait encore exterminer quelque temps auparavant par le fer et par le feu. Il convoqua les diètes des pays héréditaires et conclut, jusqu’au moment de leur réunion, un armistice avec les paysans. Entre temps, il s’armait énergiquement pour pouvoir, le plus rapidement possible, parler un autre langage avec les scélérats.\par
L’armistice ne fut naturellement pas respecté longtemps. Dans les duchés, Dietrichstein, qui commençait à manquer d’argent, se mit à rançonner. Ses troupes slaves et magyares se permirent en outre les plus infâmes atrocités contre la population. C’est pourquoi les Styriens se soulevèrent à nouveau, assaillirent dans la nuit du 2 au 3 juillet le capitaine Dietrichstein à Schladming et massacrèrent tout ce qui ne parlait pas allemand. Dietrichstein lui-même fut fait prisonnier. Le 3 au matin, les paysans réunirent un tribunal qui condamna à mort 40 nobles tchèques et croates parmi les prisonniers. Ils furent décapités sur-le-champ. Le résultat ne se fit pas attendre. L’archiduc accepta immédiatement toutes les revendications des états des cinq duchés (Haute et Basse-Autriche, Styrie, Carinthie et Carniole).\par
Au Tyrol également les revendications de la diète furent acceptées, ce qui entraîna la pacification du Nord. Mais le Sud, qui maintenait ses premières revendications en face des décisions plus modérées de la diète, resta en armes. Ce n’est qu’en décembre que l’archiduc réussit à y rétablir l’ordre par la violence. Il n’oublia pas de faire exécuter un grand nombre de meneurs et de chefs de l’insurrection qui tombèrent entre ses mains.\par
En avril 10000 soldats bavarois, sous le commandement de Georg von Frundsberg, marchèrent sur Salzbourg. Celle force imposante, ainsi que les dissensions qui éclatèrent entre les paysans, amenèrent les Salzbourgeois à conclure avec l’archevêque un accord qui fut signé le 1ᵉʳ septembre et qu’accepta également l’archéduc. Mais les deux princes, qui avaient entre temps suffisamment renforcé leurs troupes, rompirent bientôt cet accord et poussèrent ainsi les paysans de Salzbourg à se soulever de nouveau. Les insurgés tinrent tout l’hiver. Au printemps, Geismaier se joignit à eux et engagea une brillante campagne contre les troupes qui approchaient de tous les côtés. Dans une série de combats extrêmement remarquables, il battit les unes après les autres – en mai et en juin 1526 – les troupes bavaroises, autrichiennes et celles de la Ligue souabe, ainsi que les mercenaires de l’archevêque de Salzbourg, et réussit longtemps à empêcher la jonction des différents corps. Avec cela, il trouva encore le temps d’assiéger la ville de Radstadt. Finalement, cerné de tous côtés par des troupes supérieures en nombre, il fut obligé de se retirer, réussit à passer au travers des lignes ennemies et conduisit les débris de son armée par-dessus les Alpes autrichiennes en territoire vénitien. La République de Venise et la Suisse offraient à l’infatigable chef paysan des points d’appui pour de nouvelles intrigues. Il s’efforça pendant une année encore, de les entraîner dans une guerre contre l’Autriche qui devait lui offrir l’occasion de provoquer une nouvelle insurrection paysanne. Mais au cours de ces négociations il tomba sous les coups d’un assassin. L’archiduc Ferdinand et l’archevêque de Salzbourg n’étaient pas tranquilles tant que Geismaier était encore en vie. Ils payèrent un bandit pour l’assassiner et ce dernier réussit à supprimer en 1527 le redoutable rebelle.
\chapterclose


\chapteropen
\renewcommand{\leftmark}{VII. Les conséquences de la guerre des paysans}
\chapter[VII. Les conséquences de la guerre des paysans]{VII. Les conséquences de la guerre des paysans}

\chaptercont
\noindent La retraite de Geismaier en territoire vénitien mit fin au dernier épilogue de la Guerre des paysans. Les paysans étaient retombés partout sous la dépendance de leurs maîtres ecclésiastiques, nobles ou patriciens. Les accords qui avaient été conclus ici et là avec eux furent rompus. Les anciennes charges furent aggravées par les rançons énormes qu’imposaient les vainqueurs aux vaincus. La plus grandiose tentative révolutionnaire du peuple allemand se termina par une défaite honteuse et une oppression momentanément redoublée. Mais avec le temps cependant, la situation de la paysannerie ne fut pas aggravée par l’écrasement de l’insurrection. Tout ce que la noblesse les princes et les prêtres pouvaient leur arracher bon an, mal an, ils le leur arrachaient déjà avant la guerre. Le paysan allemand de l’époque avait ceci de commun avec le prolétaire moderne que sa part des produits du travail se réduisit au minimum des moyens de subsistance nécessaires à son entretien et à la reproduction de la race paysanne. En moyenne, on ne pouvait pas leur prendre davantage. Certes, un grand nombre de paysans moyens assez aisés furent complètement ruinés, une foule de corvéables réduits au servage, des zones entières de terres communales furent confisquées, un grand nombre de paysans furent jetés, par suite de la destruction de leurs maisons et de la dévastation de leurs champs ainsi que du désordre général, dans le vagabondage ou dans la plèbe des villes. Mais les guerres et les dévastations étaient des phénomènes quotidiens à l’époque, et d’une façon générale, la situation de la classe paysanne était beaucoup trop misérable pour qu’on pût l’aggraver encore de façon durable par l’augmentation des impôts. Les guerres de religion, qui survinrent ensuite, et enfin la guerre de Trente ans, avec ses dévastations et ses dépopulations massives et répétées, frappèrent les paysans beaucoup plus durement que la Guerre des paysans. En particulier la guerre de Trente ans anéantit la plus grande partie des forces productives employées dans l’agriculture et ainsi elle jeta pour longtemps, et aussi par suite de la destruction simultanée d’un grand nombre de villes, les paysans, les plébéiens et les bourgeois ruinés dans la misère la plus effroyable, une vraie misère de paysans irlandais.\par
Ceux qui souffrirent le plus des conséquences de la Guerre des paysans, ce fut le \emph{clergé}. Leurs monastères et leurs abbayes avaient été incendiés, leurs objets précieux pillés, vendus à l’étranger ou fondus leurs réserves dévorées. Ce sont eux qui partout avaient pu opposer le moins de résistance, et c’est sur eux qu’était tombé le plus durement tout le poids de la haine populaire. Les autres ordres, princes, noblesse et bourgeoisie des villes, se réjouissaient même en secret des malheurs dont étaient victimes les prélats qu’ils haïssaient. La Guerre des paysans avait rendu populaire la sécularisation des biens ecclésiastiques au profit des paysans. Les princes séculiers et une partie des villes se mirent en devoir de réaliser cette sécularisation à \emph{leur} profit et bientôt, dans les pays protestants, les domaines des prélats furent aux mains des princes et du patriciat des villes. Mais la domination des princes ecclésiastiques avait aussi été fortement ébranlée et les princes séculiers s’entendirent à exploiter dans ce sens la haine populaire. Nous avons vu comment l’abbé de Fulda tomba, du rang de suzerain, au rang de vassal de Philippe de Hesse. C’est ainsi également que la ville de Kempten obligea le prince-abbé à lui vendre à un prix dérisoire toute une série de précieux privilèges qu’il possédait dans la ville.\par
La \emph{noblesse} avait également beaucoup souffert. La plupart de ses châteaux étaient anéantis, un certain nombre de familles les plus considérables étaient ruinées et ne pouvaient plus subsister qu’en entrant au service des princes. Son impuissance vis-à-vis des paysans était établie : elle avait été partout battue et contrainte à capituler. Seules les armées des princes l’avaient sauvée. Elle devait perdre de plus en plus son importance d’ordre dépendant directement de l’empereur et tomber sous la domination des princes.\par
Les \emph{villes} n’avaient également, en général, tiré aucun avantage de la Guerre des paysans. La domination des notables fut presque partout renforcée. L’opposition des bourgeois en fut brisée pour longtemps. La vieille routine patricienne continua à se traîner ainsi, entravant de tous côtés le commerce et l’industrie, jusqu’à la Révolution française. De plus, ce sont les villes que les princes rendirent responsables des succès momentanés que le parti bourgeois ou plébéien avaient obtenus chez elles au cours de la lutte. Des villes appartenant déjà aux domaines des princes furent lourdement rançonnées, privées de leurs privilèges et exposées sans défense à leur arbitraire cupide (Frankenhausen, Arnstadt, Schmalkalden, Wurzbourg, etc.). Les villes d’Empire furent incorporées aux domaines princiers (Mulhausen) ou placées sous la dépendance morale des princes voisins, comme ce fut le cas d’un grand nombre en Franconie.\par
Ceux qui, dans ces conditions, furent les seuls à tirer bénéfice de la Guerre des paysans, ce furent les \emph{princes}. Nous avons déjà vu, au début de cet ouvrage, comment l’insuffisance du développement industriel, commercial et agricole de l’Allemagne rendait impossible toute centralisation des Allemands en \emph{nation}, comment elle n’avait permis qu’une centralisation locale et provinciale, et comment, par conséquent, les représentants de cette centralisation à l’intérieur du morcellement, les princes, constituaient le seul ordre qui devait profiter de toute modification des relations sociales et politiques existantes. Le degré de développement de l’Allemagne de l’époque était tellement bas et d’autre part si inégal dans les diverses provinces, qu’à côté des principautés séculaires pouvaient encore exister des principautés ecclésiastiques, des républiques citadines et des comtes et barons souverains. Mais il tendait en même temps, quoique très lentement et très mollement, à une centralisation \emph{provinciale}, c’est-à-dire à la subordination aux princes des ordres d’Empire qui subsistaient encore. C’est pourquoi seuls les princes pouvaient tirer un bénéfice quelconque de la Guerre des paysans. C’est aussi ce qui se produisit. Ils en tirèrent un avantage non seulement relatif, du fait que leurs concurrents, le clergé, la noblesse et les villes en sortirent affaiblis, mais aussi absolu, en ce sens qu’ils remportèrent les le butin principal de tous les autres ordres. Les biens ecclésiastiques furent sécularisés à leur profit. Une partie de la noblesse, à demi ou complètement minée, dut se soumettre peu à peu à leur domination. L’argent des rançons imposées aux villes et aux communautés paysannes afflua dans les caisses de leur fisc qui d’ailleurs, par suite de la suppression de tant de privilèges municipaux acquit une plus grande liberté de mouvement pour ses chères opérations financières.\par
La Guerre des paysans ne fit qu’aggraver et consolider l’état de division de l’Allemagne, qui fut une des principales causes de son échec.\par
Nous avons vu comment l’Allemagne était morcelée non seulement en d’innombrables provinces indépendantes, presque totalement étrangères les unes aux autres, mais encore comment la nation, dans chacune de ces provinces, était divisée en une hiérarchie complexe d’ordres et de fractions d’ordre. Outre les princes et les prêtres, nous rencontrons la noblesse et les paysans à la campagne, les patriciens, les bourgeois et les plébéiens dans les villes, tous ordres dont les intérêts étaient totalement étrangers les uns aux autres, quand ils ne s’enchevêtraient pas ou même se contredisaient. Au-dessus de tous ces intérêts complexes, il y avait encore ceux de l’empereur et ceux du pape. Nous avons vu comment ces différents intérêts se constituaient en fin de compte péniblement, d’une façon incomplète et variable selon les localités, en trois grands groupes comment, malgré ce groupement pénible, chaque ordre s’opposait à la direction donnée par les conditions de l’époque au développement national, agissait indépendamment, entrait ainsi en conflit non seulement avec tous les éléments conservateurs, mais aussi avec tous les autres éléments d’opposition, et devait finalement succomber dans cette lutte. Ce fut le cas de la noblesse dans la révolte de Sickingen, de la paysannerie dans la Guerre des paysans, et de la bourgeoisie dans tout le mouvement timide de la Réforme. C’est ainsi que, dans la plupart des régions de l’Allemagne, les paysans et les plébéiens eux-mêmes ne purent arriver à une action commune et se firent obstacle réciproquement. Nous avons vu également quelles furent les causes qui déterminaient cet émiettement de la lutte des classes, l’échec complet du mouvement révolutionnaire qu’il entraînait et le demi-échec du mouvement bourgeois.\par
Comment le morcellement local et provincial et l’étroitesse locale et provinciale, qui en résulta nécessairement, ruinèrent le mouvement comment ni les bourgeois, ni les paysans, ni les plébéiens ne réussirent à mener une action nationale coordonné comment les paysans agirent dans chaque province de leur propre chef, refusèrent constamment de venir en aide aux paysans insurgés des régions voisines et furent ainsi anéantis successivement dans des combats isolés par des armées, dont la force numérique souvent ne représentait même pas le dixième de celle des paysans insurgés, c’est ce que chacun comprendra maintenant d’après ce qui précède. Les différents armistices et accords conclus par les diverses armées avec leurs adversaires représentent autant d’actes de trahison à l’égard de la cause commune, et le fait que le seul groupement des différentes armées qui fut possible ait eu pour base non la communauté plus ou moins grande de leur propre action mais la communauté de l’adversaire particulier devant lequel elles succombèrent, est la preuve la plus éclatante du degré de singularité à l’égard les uns des autres des paysans des différentes provinces.\par
Ici également, apparaît l’analogie avec le mouvement de 1848 – 50. En 1848 aussi, les intérêts des différentes classes de l’opposition entrèrent en conflit, chacune agissant pour soi. La bourgeoisie, trop développée pour pouvoir supporter plus longtemps le joug de l’absolutisme féodal et bureaucratique, n’était cependant pas suffisamment forte pour subordonner immédiatement à ses exigences propres celles des autres classes de la société. Le prolétariat, beaucoup trop faible pour pouvoir espérer sauter rapidement par-dessus la période bourgeoise et compter conquérir lui-même rapidement le pouvoir, avait déjà, sous l’absolutisme, trop bien appris à connaître les douceurs du régime bourgeois et était somme toute beaucoup trop développé pour pouvoir, même pour un temps très court, voir dans l’émancipation de la bourgeoisie sa propre émancipation. La masse de la nation, petits bourgeois, artisans et paysans, fut abandonnée par son allié naturel le plus proche, la bourgeoisie, comme déjà trop révolutionnaire, et aussi en partie par le prolétariat, comme pas encore suffisamment avancée. Divisée à son tour en plusieurs fractions, elle ne put rien réaliser et s’opposa à ses compagnons d’opposition de droite et de gauche. Quant à l’étroitesse locale enfin, elle ne peut pas avoir été plus grande en 1525, chez les paysans, qu’elle ne le fut dans toutes les classes qui participèrent au mouvement de 1848. Les cent révolutions locales, suivies d’autant de réactions locales qui ne rencontrèrent pas plus d’obstacle que les premières, le maintien des petits États, etc. parlent un langage suffisamment clair. \emph{Quiconque, après les deux révolutions allemandes de 1525 et de 1848 et les résultats qu’elles ont obtenus, peut encore radoter sur la république fédérative, est digne de l’asile d’aliénés.}\par
Mais les deux révolutions, celle du XVIᵉ et celle de 1848 – 50, malgré toutes leurs ressemblances, sont cependant essentiellement différentes l’une de l’autre. La révolution de 1848 prouve, sinon le progrès de l’Allemagne, du moins le progrès de l’Europe.\par
Qui profita de la révolution de 1525 ? Les princes. Qui profita de la révolution de 1848 ? Les \emph{grands} princes : l’Autriche et la Prusse. Derrière les petits princes de 1525 il y avait, liés à eux par le paiement des impôts, les petits bourgeois. Derrière les grands princes de 1850, derrière l’Autriche et la Prusse, il y a les grands bourgeois modernes qui se les soumettent rapidement au moyen de la dette d’État. Et, derrière les grands bourgeois, il y a les prolétaires.\par
La révolution de 1525 a été une affaire locale allemande. Les Anglais, les Français, les Tchèques, les Hongrois avaient déjà fait leur guerre des paysans lorsque les Allemands firent la leur. Si l’Allemagne était morcelée, l’Europe l’était encore bien davantage. La révolution de 1848 ne fut pas une affaire locale allemande, elle fut une partie isolée d’un grand événement européen. Ses causes motrices pendant tout son déroulement ne sont pas comprimées dans le cadre étroit de tel ou tel pays, pas même d’un seul continent. On peut même dire que les pays qui furent le véritable théâtre de cette révolution sont ceux qui ont le moins de part à cette révolution. Ils sont des matières premières plus ou moins conscientes et passives, qui sont transformées au cours d’un mouvement qui certes, dans les conditions sociales actuelles, ne peut nous apparaître que comme une force étrangère, quoiqu’il ne soit, en fin de compte, que notre propre mouvement. C’est pourquoi la révolution de 1848-50 ne peut pas se terminer comme celle de 1525.
\chapterclose

 


% at least one empty page at end (for booklet couv)
\ifbooklet
  \newpage\null\thispagestyle{empty}\newpage
\fi

\ifdev % autotext in dev mode
\fontname\font — \textsc{Les règles du jeu}\par
(\hyperref[utopie]{\underline{Lien}})\par
\noindent \initialiv{A}{lors là}\blindtext\par
\noindent \initialiv{À}{ la bonheur des dames}\blindtext\par
\noindent \initialiv{É}{tonnez-le}\blindtext\par
\noindent \initialiv{Q}{ualitativement}\blindtext\par
\noindent \initialiv{V}{aloriser}\blindtext\par
\Blindtext
\phantomsection
\label{utopie}
\Blinddocument
\fi
\end{document}
