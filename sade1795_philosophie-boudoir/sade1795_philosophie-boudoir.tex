%%%%%%%%%%%%%%%%%%%%%%%%%%%%%%%%%
% LaTeX model https://hurlus.fr %
%%%%%%%%%%%%%%%%%%%%%%%%%%%%%%%%%

% Needed before document class
\RequirePackage{pdftexcmds} % needed for tests expressions
\RequirePackage{fix-cm} % correct units

% Define mode
\def\mode{a4}

\newif\ifaiv % a4
\newif\ifav % a5
\newif\ifbooklet % booklet
\newif\ifcover % cover for booklet

\ifnum \strcmp{\mode}{cover}=0
  \covertrue
\else\ifnum \strcmp{\mode}{booklet}=0
  \booklettrue
\else\ifnum \strcmp{\mode}{a5}=0
  \avtrue
\else
  \aivtrue
\fi\fi\fi

\ifbooklet % do not enclose with {}
  \documentclass[french,twoside]{book} % ,notitlepage
  \usepackage[%
    papersize={105mm, 297mm},
    inner=12mm,
    outer=12mm,
    top=20mm,
    bottom=15mm,
    marginparsep=0pt,
  ]{geometry}
  \usepackage[fontsize=9.5pt]{scrextend} % for Roboto
\else\ifav
  \documentclass[french,twoside]{book} % ,notitlepage
  \usepackage[%
    a5paper,
    inner=25mm,
    outer=15mm,
    top=15mm,
    bottom=15mm,
    marginparsep=0pt,
  ]{geometry}
  \usepackage[fontsize=12pt]{scrextend}
\else% A4 2 cols
  \documentclass[twocolumn]{report}
  \usepackage[%
    a4paper,
    inner=15mm,
    outer=10mm,
    top=25mm,
    bottom=18mm,
    marginparsep=0pt,
  ]{geometry}
  \setlength{\columnsep}{20mm}
  \usepackage[fontsize=9.5pt]{scrextend}
\fi\fi

%%%%%%%%%%%%%%
% Alignments %
%%%%%%%%%%%%%%
% before teinte macros

\setlength{\arrayrulewidth}{0.2pt}
\setlength{\columnseprule}{\arrayrulewidth} % twocol
\setlength{\parskip}{0pt} % classical para with no margin
\setlength{\parindent}{1.5em}

%%%%%%%%%%
% Colors %
%%%%%%%%%%
% before Teinte macros

\usepackage[dvipsnames]{xcolor}
\definecolor{rubric}{HTML}{800000} % the tonic 0c71c3
\def\columnseprulecolor{\color{rubric}}
\colorlet{borderline}{rubric!30!} % definecolor need exact code
\definecolor{shadecolor}{gray}{0.95}
\definecolor{bghi}{gray}{0.5}

%%%%%%%%%%%%%%%%%
% Teinte macros %
%%%%%%%%%%%%%%%%%
%%%%%%%%%%%%%%%%%%%%%%%%%%%%%%%%%%%%%%%%%%%%%%%%%%%
% <TEI> generic (LaTeX names generated by Teinte) %
%%%%%%%%%%%%%%%%%%%%%%%%%%%%%%%%%%%%%%%%%%%%%%%%%%%
% This template is inserted in a specific design
% It is XeLaTeX and otf fonts

\makeatletter % <@@@


\usepackage{blindtext} % generate text for testing
\usepackage[strict]{changepage} % for modulo 4
\usepackage{contour} % rounding words
\usepackage[nodayofweek]{datetime}
% \usepackage{DejaVuSans} % seems buggy for sffont font for symbols
\usepackage{enumitem} % <list>
\usepackage{etoolbox} % patch commands
\usepackage{fancyvrb}
\usepackage{fancyhdr}
\usepackage{float}
\usepackage{fontspec} % XeLaTeX mandatory for fonts
\usepackage{footnote} % used to capture notes in minipage (ex: quote)
\usepackage{framed} % bordering correct with footnote hack
\usepackage{graphicx}
\usepackage{lettrine} % drop caps
\usepackage{lipsum} % generate text for testing
\usepackage[framemethod=tikz,]{mdframed} % maybe used for frame with footnotes inside
\usepackage{pdftexcmds} % needed for tests expressions
\usepackage{polyglossia} % non-break space french punct, bug Warning: "Failed to patch part"
\usepackage[%
  indentfirst=false,
  vskip=1em,
  noorphanfirst=true,
  noorphanafter=true,
  leftmargin=\parindent,
  rightmargin=0pt,
]{quoting}
\usepackage{ragged2e}
\usepackage{setspace} % \setstretch for <quote>
\usepackage{tabularx} % <table>
\usepackage[explicit]{titlesec} % wear titles, !NO implicit
\usepackage{tikz} % ornaments
\usepackage{tocloft} % styling tocs
\usepackage[fit]{truncate} % used im runing titles
\usepackage{unicode-math}
\usepackage[normalem]{ulem} % breakable \uline, normalem is absolutely necessary to keep \emph
\usepackage{verse} % <l>
\usepackage{xcolor} % named colors
\usepackage{xparse} % @ifundefined
\XeTeXdefaultencoding "iso-8859-1" % bad encoding of xstring
\usepackage{xstring} % string tests
\XeTeXdefaultencoding "utf-8"
\PassOptionsToPackage{hyphens}{url} % before hyperref, which load url package

% TOTEST
% \usepackage{hypcap} % links in caption ?
% \usepackage{marginnote}
% TESTED
% \usepackage{background} % doesn’t work with xetek
% \usepackage{bookmark} % prefers the hyperref hack \phantomsection
% \usepackage[color, leftbars]{changebar} % 2 cols doc, impossible to keep bar left
% \usepackage[utf8x]{inputenc} % inputenc package ignored with utf8 based engines
% \usepackage[sfdefault,medium]{inter} % no small caps
% \usepackage{firamath} % choose firasans instead, firamath unavailable in Ubuntu 21-04
% \usepackage{flushend} % bad for last notes, supposed flush end of columns
% \usepackage[stable]{footmisc} % BAD for complex notes https://texfaq.org/FAQ-ftnsect
% \usepackage{helvet} % not for XeLaTeX
% \usepackage{multicol} % not compatible with too much packages (longtable, framed, memoir…)
% \usepackage[default,oldstyle,scale=0.95]{opensans} % no small caps
% \usepackage{sectsty} % \chapterfont OBSOLETE
% \usepackage{soul} % \ul for underline, OBSOLETE with XeTeX
% \usepackage[breakable]{tcolorbox} % text styling gone, footnote hack not kept with breakable


% Metadata inserted by a program, from the TEI source, for title page and runing heads
\title{\textbf{ La philosophie dans le boudoir }\\ \medskip
\textit{ ou les instituteurs immoraux. Dialogues destinés à l’éducation des jeunes demoiselles. }}
\date{1795}
\author{Sade, Donatien Alphonse François de (1740, 1814)}
\def\elbibl{Sade, Donatien Alphonse François de (1740, 1814). 1795. \emph{La philosophie dans le boudoir}}

% Default metas
\newcommand{\colorprovide}[2]{\@ifundefinedcolor{#1}{\colorlet{#1}{#2}}{}}
\colorprovide{rubric}{red}
\colorprovide{silver}{lightgray}
\@ifundefined{syms}{\newfontfamily\syms{DejaVu Sans}}{}
\newif\ifdev
\@ifundefined{elbibl}{% No meta defined, maybe dev mode
  \newcommand{\elbibl}{Titre court ?}
  \newcommand{\elbook}{Titre du livre source ?}
  \newcommand{\elabstract}{Résumé\par}
  \newcommand{\elurl}{http://oeuvres.github.io/elbook/2}
  \author{Éric Lœchien}
  \title{Un titre de test assez long pour vérifier le comportement d’une maquette}
  \date{1566}
  \devtrue
}{}
\let\eltitle\@title
\let\elauthor\@author
\let\eldate\@date


\defaultfontfeatures{
  % Mapping=tex-text, % no effect seen
  Scale=MatchLowercase,
  Ligatures={TeX,Common},
}


% generic typo commands
\newcommand{\astermono}{\medskip\centerline{\color{rubric}\large\selectfont{\syms ✻}}\medskip\par}%
\newcommand{\astertri}{\medskip\par\centerline{\color{rubric}\large\selectfont{\syms ✻\,✻\,✻}}\medskip\par}%
\newcommand{\asterism}{\bigskip\par\noindent\parbox{\linewidth}{\centering\color{rubric}\large{\syms ✻}\\{\syms ✻}\hskip 0.75em{\syms ✻}}\bigskip\par}%

% lists
\newlength{\listmod}
\setlength{\listmod}{\parindent}
\setlist{
  itemindent=!,
  listparindent=\listmod,
  labelsep=0.2\listmod,
  parsep=0pt,
  % topsep=0.2em, % default topsep is best
}
\setlist[itemize]{
  label=—,
  leftmargin=0pt,
  labelindent=1.2em,
  labelwidth=0pt,
}
\setlist[enumerate]{
  label={\bf\color{rubric}\arabic*.},
  labelindent=0.8\listmod,
  leftmargin=\listmod,
  labelwidth=0pt,
}
\newlist{listalpha}{enumerate}{1}
\setlist[listalpha]{
  label={\bf\color{rubric}\alph*.},
  leftmargin=0pt,
  labelindent=0.8\listmod,
  labelwidth=0pt,
}
\newcommand{\listhead}[1]{\hspace{-1\listmod}\emph{#1}}

\renewcommand{\hrulefill}{%
  \leavevmode\leaders\hrule height 0.2pt\hfill\kern\z@}

% General typo
\DeclareTextFontCommand{\textlarge}{\large}
\DeclareTextFontCommand{\textsmall}{\small}

% commands, inlines
\newcommand{\anchor}[1]{\Hy@raisedlink{\hypertarget{#1}{}}} % link to top of an anchor (not baseline)
\newcommand\abbr[1]{#1}
\newcommand{\autour}[1]{\tikz[baseline=(X.base)]\node [draw=rubric,thin,rectangle,inner sep=1.5pt, rounded corners=3pt] (X) {\color{rubric}#1};}
\newcommand\corr[1]{#1}
\newcommand{\ed}[1]{ {\color{silver}\sffamily\footnotesize (#1)} } % <milestone ed="1688"/>
\newcommand\expan[1]{#1}
\newcommand\foreign[1]{\emph{#1}}
\newcommand\gap[1]{#1}
\renewcommand{\LettrineFontHook}{\color{rubric}}
\newcommand{\initial}[2]{\lettrine[lines=2, loversize=0.3, lhang=0.3]{#1}{#2}}
\newcommand{\initialiv}[2]{%
  \let\oldLFH\LettrineFontHook
  % \renewcommand{\LettrineFontHook}{\color{rubric}\ttfamily}
  \IfSubStr{QJ’}{#1}{
    \lettrine[lines=4, lhang=0.2, loversize=-0.1, lraise=0.2]{\smash{#1}}{#2}
  }{\IfSubStr{É}{#1}{
    \lettrine[lines=4, lhang=0.2, loversize=-0, lraise=0]{\smash{#1}}{#2}
  }{\IfSubStr{ÀÂ}{#1}{
    \lettrine[lines=4, lhang=0.2, loversize=-0, lraise=0, slope=0.6em]{\smash{#1}}{#2}
  }{\IfSubStr{A}{#1}{
    \lettrine[lines=4, lhang=0.2, loversize=0.2, slope=0.6em]{\smash{#1}}{#2}
  }{\IfSubStr{V}{#1}{
    \lettrine[lines=4, lhang=0.2, loversize=0.2, slope=-0.5em]{\smash{#1}}{#2}
  }{
    \lettrine[lines=4, lhang=0.2, loversize=0.2]{\smash{#1}}{#2}
  }}}}}
  \let\LettrineFontHook\oldLFH
}
\newcommand{\labelchar}[1]{\textbf{\color{rubric} #1}}
\newcommand{\milestone}[1]{\autour{\footnotesize\color{rubric} #1}} % <milestone n="4"/>
\newcommand\name[1]{#1}
\newcommand\orig[1]{#1}
\newcommand\orgName[1]{#1}
\newcommand\persName[1]{#1}
\newcommand\placeName[1]{#1}
\newcommand{\pn}[1]{\IfSubStr{-—–¶}{#1}% <p n="3"/>
  {\noindent{\bfseries\color{rubric}   ¶  }}
  {{\footnotesize\autour{ #1}  }}}
\newcommand\reg{}
% \newcommand\ref{} % already defined
\newcommand\sic[1]{#1}
\newcommand\surname[1]{\textsc{#1}}
\newcommand\term[1]{\textbf{#1}}

\def\mednobreak{\ifdim\lastskip<\medskipamount
  \removelastskip\nopagebreak\medskip\fi}
\def\bignobreak{\ifdim\lastskip<\bigskipamount
  \removelastskip\nopagebreak\bigskip\fi}

% commands, blocks
\newcommand{\byline}[1]{\bigskip{\RaggedLeft{#1}\par}\bigskip}
\newcommand{\bibl}[1]{{\RaggedLeft{#1}\par\bigskip}}
\newcommand{\biblitem}[1]{{\noindent\hangindent=\parindent   #1\par}}
\newcommand{\dateline}[1]{\medskip{\RaggedLeft{#1}\par}\bigskip}
\newcommand{\labelblock}[1]{\medbreak{\noindent\color{rubric}\bfseries #1}\par\mednobreak}
\newcommand{\salute}[1]{\bigbreak{#1}\par\medbreak}
\newcommand{\signed}[1]{\bigbreak\filbreak{\raggedleft #1\par}\medskip}

% environments for blocks (some may become commands)
\newenvironment{borderbox}{}{} % framing content
\newenvironment{citbibl}{\ifvmode\hfill\fi}{\ifvmode\par\fi }
\newenvironment{docAuthor}{\ifvmode\vskip4pt\fontsize{16pt}{18pt}\selectfont\fi\itshape}{\ifvmode\par\fi }
\newenvironment{docDate}{}{\ifvmode\par\fi }
\newenvironment{docImprint}{\vskip6pt}{\ifvmode\par\fi }
\newenvironment{docTitle}{\vskip6pt\bfseries\fontsize{18pt}{22pt}\selectfont}{\par }
\newenvironment{msHead}{\vskip6pt}{\par}
\newenvironment{msItem}{\vskip6pt}{\par}
\newenvironment{titlePart}{}{\par }


% environments for block containers
\newenvironment{argument}{\itshape\parindent0pt}{\vskip1.5em}
\newenvironment{biblfree}{}{\ifvmode\par\fi }
\newenvironment{bibitemlist}[1]{%
  \list{\@biblabel{\@arabic\c@enumiv}}%
  {%
    \settowidth\labelwidth{\@biblabel{#1}}%
    \leftmargin\labelwidth
    \advance\leftmargin\labelsep
    \@openbib@code
    \usecounter{enumiv}%
    \let\p@enumiv\@empty
    \renewcommand\theenumiv{\@arabic\c@enumiv}%
  }
  \sloppy
  \clubpenalty4000
  \@clubpenalty \clubpenalty
  \widowpenalty4000%
  \sfcode`\.\@m
}%
{\def\@noitemerr
  {\@latex@warning{Empty `bibitemlist' environment}}%
\endlist}
\newenvironment{quoteblock}% may be used for ornaments
  {\begin{quoting}}
  {\end{quoting}}

% table () is preceded and finished by custom command
\newcommand{\tableopen}[1]{%
  \ifnum\strcmp{#1}{wide}=0{%
    \begin{center}
  }
  \else\ifnum\strcmp{#1}{long}=0{%
    \begin{center}
  }
  \else{%
    \begin{center}
  }
  \fi\fi
}
\newcommand{\tableclose}[1]{%
  \ifnum\strcmp{#1}{wide}=0{%
    \end{center}
  }
  \else\ifnum\strcmp{#1}{long}=0{%
    \end{center}
  }
  \else{%
    \end{center}
  }
  \fi\fi
}


% text structure
\newcommand\chapteropen{} % before chapter title
\newcommand\chaptercont{} % after title, argument, epigraph…
\newcommand\chapterclose{} % maybe useful for multicol settings
\setcounter{secnumdepth}{-2} % no counters for hierarchy titles
\setcounter{tocdepth}{5} % deep toc
\markright{\@title} % ???
\markboth{\@title}{\@author} % ???
\renewcommand\tableofcontents{\@starttoc{toc}}
% toclof format
% \renewcommand{\@tocrmarg}{0.1em} % Useless command?
% \renewcommand{\@pnumwidth}{0.5em} % {1.75em}
\renewcommand{\@cftmaketoctitle}{}
\setlength{\cftbeforesecskip}{\z@ \@plus.2\p@}
\renewcommand{\cftchapfont}{}
\renewcommand{\cftchapdotsep}{\cftdotsep}
\renewcommand{\cftchapleader}{\normalfont\cftdotfill{\cftchapdotsep}}
\renewcommand{\cftchappagefont}{\bfseries}
\setlength{\cftbeforechapskip}{0em \@plus\p@}
% \renewcommand{\cftsecfont}{\small\relax}
\renewcommand{\cftsecpagefont}{\normalfont}
% \renewcommand{\cftsubsecfont}{\small\relax}
\renewcommand{\cftsecdotsep}{\cftdotsep}
\renewcommand{\cftsecpagefont}{\normalfont}
\renewcommand{\cftsecleader}{\normalfont\cftdotfill{\cftsecdotsep}}
\setlength{\cftsecindent}{1em}
\setlength{\cftsubsecindent}{2em}
\setlength{\cftsubsubsecindent}{3em}
\setlength{\cftchapnumwidth}{1em}
\setlength{\cftsecnumwidth}{1em}
\setlength{\cftsubsecnumwidth}{1em}
\setlength{\cftsubsubsecnumwidth}{1em}

% footnotes
\newif\ifheading
\newcommand*{\fnmarkscale}{\ifheading 0.70 \else 1 \fi}
\renewcommand\footnoterule{\vspace*{0.3cm}\hrule height \arrayrulewidth width 3cm \vspace*{0.3cm}}
\setlength\footnotesep{1.5\footnotesep} % footnote separator
\renewcommand\@makefntext[1]{\parindent 1.5em \noindent \hb@xt@1.8em{\hss{\normalfont\@thefnmark . }}#1} % no superscipt in foot
\patchcmd{\@footnotetext}{\footnotesize}{\footnotesize\sffamily}{}{} % before scrextend, hyperref


%   see https://tex.stackexchange.com/a/34449/5049
\def\truncdiv#1#2{((#1-(#2-1)/2)/#2)}
\def\moduloop#1#2{(#1-\truncdiv{#1}{#2}*#2)}
\def\modulo#1#2{\number\numexpr\moduloop{#1}{#2}\relax}

% orphans and widows
\clubpenalty=9996
\widowpenalty=9999
\brokenpenalty=4991
\predisplaypenalty=10000
\postdisplaypenalty=1549
\displaywidowpenalty=1602
\hyphenpenalty=400
% Copied from Rahtz but not understood
\def\@pnumwidth{1.55em}
\def\@tocrmarg {2.55em}
\def\@dotsep{4.5}
\emergencystretch 3em
\hbadness=4000
\pretolerance=750
\tolerance=2000
\vbadness=4000
\def\Gin@extensions{.pdf,.png,.jpg,.mps,.tif}
% \renewcommand{\@cite}[1]{#1} % biblio

\usepackage{hyperref} % supposed to be the last one, :o) except for the ones to follow
\urlstyle{same} % after hyperref
\hypersetup{
  % pdftex, % no effect
  pdftitle={\elbibl},
  % pdfauthor={Your name here},
  % pdfsubject={Your subject here},
  % pdfkeywords={keyword1, keyword2},
  bookmarksnumbered=true,
  bookmarksopen=true,
  bookmarksopenlevel=1,
  pdfstartview=Fit,
  breaklinks=true, % avoid long links
  pdfpagemode=UseOutlines,    % pdf toc
  hyperfootnotes=true,
  colorlinks=false,
  pdfborder=0 0 0,
  % pdfpagelayout=TwoPageRight,
  % linktocpage=true, % NO, toc, link only on page no
}

\makeatother % /@@@>
%%%%%%%%%%%%%%
% </TEI> end %
%%%%%%%%%%%%%%


%%%%%%%%%%%%%
% footnotes %
%%%%%%%%%%%%%
\renewcommand{\thefootnote}{\bfseries\textcolor{rubric}{\arabic{footnote}}} % color for footnote marks

%%%%%%%%%
% Fonts %
%%%%%%%%%
\usepackage[]{roboto} % SmallCaps, Regular is a bit bold
% \linespread{0.90} % too compact, keep font natural
\newfontfamily\fontrun[]{Roboto Condensed Light} % condensed runing heads
\ifav
  \setmainfont[
    ItalicFont={Roboto Light Italic},
  ]{Roboto}
\else\ifbooklet
  \setmainfont[
    ItalicFont={Roboto Light Italic},
  ]{Roboto}
\else
\setmainfont[
  ItalicFont={Roboto Italic},
]{Roboto Light}
\fi\fi
\renewcommand{\LettrineFontHook}{\bfseries\color{rubric}}
% \renewenvironment{labelblock}{\begin{center}\bfseries\color{rubric}}{\end{center}}

%%%%%%%%
% MISC %
%%%%%%%%

\setdefaultlanguage[frenchpart=false]{french} % bug on part


\newenvironment{quotebar}{%
    \def\FrameCommand{{\color{rubric!10!}\vrule width 0.5em} \hspace{0.9em}}%
    \def\OuterFrameSep{\itemsep} % séparateur vertical
    \MakeFramed {\advance\hsize-\width \FrameRestore}
  }%
  {%
    \endMakeFramed
  }
\renewenvironment{quoteblock}% may be used for ornaments
  {%
    \savenotes
    \setstretch{0.9}
    \normalfont
    \begin{quotebar}
  }
  {%
    \end{quotebar}
    \spewnotes
  }


\renewcommand{\headrulewidth}{\arrayrulewidth}
\renewcommand{\headrule}{{\color{rubric}\hrule}}

% delicate tuning, image has produce line-height problems in title on 2 lines
\titleformat{name=\chapter} % command
  [display] % shape
  {\vspace{1.5em}\centering} % format
  {} % label
  {0pt} % separator between n
  {}
[{\color{rubric}\huge\textbf{#1}}\bigskip] % after code
% \titlespacing{command}{left spacing}{before spacing}{after spacing}[right]
\titlespacing*{\chapter}{0pt}{-2em}{0pt}[0pt]

\titleformat{name=\section}
  [block]{}{}{}{}
  [\vbox{\color{rubric}\large\raggedleft\textbf{#1}}]
\titlespacing{\section}{0pt}{0pt plus 4pt minus 2pt}{\baselineskip}

\titleformat{name=\subsection}
  [block]
  {}
  {} % \thesection
  {} % separator \arrayrulewidth
  {}
[\vbox{\large\textbf{#1}}]
% \titlespacing{\subsection}{0pt}{0pt plus 4pt minus 2pt}{\baselineskip}

\ifaiv
  \fancypagestyle{main}{%
    \fancyhf{}
    \setlength{\headheight}{1.5em}
    \fancyhead{} % reset head
    \fancyfoot{} % reset foot
    \fancyhead[L]{\truncate{0.45\headwidth}{\fontrun\elbibl}} % book ref
    \fancyhead[R]{\truncate{0.45\headwidth}{ \fontrun\nouppercase\leftmark}} % Chapter title
    \fancyhead[C]{\thepage}
  }
  \fancypagestyle{plain}{% apply to chapter
    \fancyhf{}% clear all header and footer fields
    \setlength{\headheight}{1.5em}
    \fancyhead[L]{\truncate{0.9\headwidth}{\fontrun\elbibl}}
    \fancyhead[R]{\thepage}
  }
\else
  \fancypagestyle{main}{%
    \fancyhf{}
    \setlength{\headheight}{1.5em}
    \fancyhead{} % reset head
    \fancyfoot{} % reset foot
    \fancyhead[RE]{\truncate{0.9\headwidth}{\fontrun\elbibl}} % book ref
    \fancyhead[LO]{\truncate{0.9\headwidth}{\fontrun\nouppercase\leftmark}} % Chapter title, \nouppercase needed
    \fancyhead[RO,LE]{\thepage}
  }
  \fancypagestyle{plain}{% apply to chapter
    \fancyhf{}% clear all header and footer fields
    \setlength{\headheight}{1.5em}
    \fancyhead[L]{\truncate{0.9\headwidth}{\fontrun\elbibl}}
    \fancyhead[R]{\thepage}
  }
\fi

\ifav % a5 only
  \titleclass{\section}{top}
\fi

\newcommand\chapo{{%
  \vspace*{-3em}
  \centering % no vskip ()
  {\Large\addfontfeature{LetterSpace=25}\bfseries{\elauthor}}\par
  \smallskip
  {\large\eldate}\par
  \bigskip
  {\Large\selectfont{\eltitle}}\par
  \bigskip
  {\color{rubric}\hline\par}
  \bigskip
  {\Large TEXTE LIBRE À PARTICPATION LIBRE\par}
  \centerline{\small\color{rubric} {hurlus.fr, tiré le \today}}\par
  \bigskip
}}

\newcommand\cover{{%
  \thispagestyle{empty}
  \centering
  {\LARGE\bfseries{\elauthor}}\par
  \bigskip
  {\Large\eldate}\par
  \bigskip
  \bigskip
  {\LARGE\selectfont{\eltitle}}\par
  \vfill\null
  {\color{rubric}\setlength{\arrayrulewidth}{2pt}\hline\par}
  \vfill\null
  {\Large TEXTE LIBRE À PARTICPATION LIBRE\par}
  \centerline{{\href{https://hurlus.fr}{\dotuline{hurlus.fr}}, tiré le \today}}\par
}}

\begin{document}
\pagestyle{empty}
\ifbooklet{
  \cover\newpage
  \thispagestyle{empty}\hbox{}\newpage
  \cover\newpage\noindent Les voyages de la brochure\par
  \bigskip
  \begin{tabularx}{\textwidth}{l|X|X}
    \textbf{Date} & \textbf{Lieu}& \textbf{Nom/pseudo} \\ \hline
    \rule{0pt}{25cm} &  &   \\
  \end{tabularx}
  \newpage
  \addtocounter{page}{-4}
}\fi

\thispagestyle{empty}
\ifaiv
  \twocolumn[\chapo]
\else
  \chapo
\fi
{\it\elabstract}
\bigskip
\makeatletter\@starttoc{toc}\makeatother % toc without new page
\bigskip

\pagestyle{main} % after style

  {\raggedleft \noindent La mère en prescrira la lecture à sa fille.}
\section[{Aux libertins}]{Aux libertins}\phantomsection
\label{libertins}\renewcommand{\leftmark}{Aux libertins}

\noindent Voluptueux de tous les âges et de tous les sexes, c’est à vous seuls que j’offre cet ouvrage ; nourrissez-vous de ses principes, ils favorisent vos passions, et ces passions, dont de froids et plats moralistes vous effraient, ne sont que les moyens que la nature emploie pour faire parvenir l’homme aux vues qu’elles a sur lui ; n’écoutez que ces passions délicieuses, leur organe est le seul qui doive vous conduire au bonheur.\par
Femmes lubriques, que la voluptueuse Saint-Ange soit votre modèle ; méprisez, à son exemple, tout ce qui contrarie les lois divines du plaisir qui l’enchaînèrent toute sa vie.\par
Jeunes filles trop longtemps contenues dans les liens absurdes et dangereux d’une vertu fantastique et d’une religion dégoûtante, imitez l’ardente Eugénie, détruisez, foulez aux pieds, avec autant de rapidité qu’elle, tous les préceptes ridicules inculqués par d’imbéciles parents.\par
Et vous, aimables débauchés, vous qui, depuis votre jeunesse, n’avez plus d’autres freins que vos désirs, et d’autres lois que vos caprices, que le cynique Dolmancé vous serve d’exemple ; allez aussi loin que lui, si, comme lui, vous voulez parcourir toutes les routes de fleurs que la lubricité vous prépare ; convainquez-vous à son école que ce n’est qu’en étendant la sphère de ses goûts et de ses fantaisies, que ce n’est qu’en sacrifiant tout à la volupté, que le malheureux individu connu sous le nom d’homme, et jeté malgré lui sur ce triste univers, peut réussir à semer quelques roses sur les épines de la vie.
\section[{Premier dialogue}]{Premier dialogue}\phantomsection
\label{d1}\renewcommand{\leftmark}{Premier dialogue}

\textit{MME DE SAINT-ANGE, LE CHEVALIER DE MIRVEL}\par
\noindent MME DE SAINT-ANGE : Bonjour, mon frère, eh bien, M. Dolmancé ?\par
LE CHEVALIER : Il arrivera à quatre heures précises, nous ne dînons qu’à sept, nous aurons, comme tu vois, tout le temps de jaser.\par
MME DE SAINT-ANGE : Sais-tu, mon frère, que je me repens un peu, et de ma curiosité, et de tous les projets obscènes formés pour aujourd’hui ? En vérité, mon ami, tu es trop indulgent ; plus je devrais être raisonnable, plus ma maudite tête s’irrite et devient libertine : tu me passes tout, cela ne sert qu’à me gâter… À vingt-six ans, je devrais être déjà dévote, et je ne suis encore que la plus débordée des femmes… On n’a pas idée de ce que je conçois, mon ami, de ce que je voudrais faire. J’imaginais qu’en m’en tenant aux femmes, cela me rendrait sage ; … que mes désirs concentrés dans mon sexe, ne s’exhaleraient plus vers le vôtre ; projets chimériques, mon ami, les plaisirs dont je voulais me priver ne sont venus s’offrir qu’avec plus d’ardeur à mon esprit, et j’ai vu que quand on était, comme moi, née pour le libertinage, il devenait inutile de songer à s’imposer des freins, de fougueux désirs les brisent bientôt. Enfin, mon cher, je suis un animal amphibie ; j’aime tout, je m’amuse de tout, je veux réunir tous les genres ; mais, avoue-le, mon frère, n’est-ce pas une extravagance complète à moi, que de vouloir connaître ce singulier Dolmancé qui de ses jours, dis-tu, n’a pu voir une femme comme l’usage le prescrit, qui, sodomite par principe, non seulement est idolâtre de son sexe, mais ne cède même pas au nôtre que sous la clause spéciale de lui livrer les attraits chéris dont il est accoutumé de se servir chez les hommes ? Vois, mon frère, quelle est ma bizarre fantaisie ! je veux être le Ganymède de ce nouveau Jupiter, je veux jouir de ses goûts, de ses débauches, je veux être la victime de ses erreurs : jusqu’à présent tu le sais, mon cher, je ne me suis livrée ainsi qu’à toi, par complaisance, ou qu’à quelqu’un de mes gens qui, payé pour me traiter de cette façon, ne s’y prêtait que par intérêt ; aujourd’hui ce n’est plus ni la complaisance ni le caprice, c’est le goût seul qui me détermine… Je vois, entre les procédés qui m’ont asservie, et ceux qui vont m’asservir à cette manie bizarre, une inconcevable différence, et je veux la connaître. Peins-moi ton Dolmancé, je t’en conjure, afin que je l’aie bien dans la tête avant que de le voir arriver ; car tu sais que je ne le connais que pour l’avoir rencontré l’autre jour dans une maison où je ne fus que quelques minutes avec lui.\par
LE CHEVALIER : Dolmancé, ma sœur, vient d’atteindre sa trente-sixième année ; il est grand, d’une fort belle figure, des yeux très vifs et très spirituels, mais quelque chose d’un peu dur et d’un peu méchant se peint malgré lui dans ses traits ; il a les plus belles dents du monde, un peu de mollesse dans la taille et dans la tournure, par l’habitude, sans doute, qu’il a de prendre si souvent des airs féminins ; il est d’une élégance extrême, une jolie voix, des talents, et principalement beaucoup de philosophie dans l’esprit.\par
MME DE SAINT-ANGE : Il ne croit pas en Dieu, j’espère ?\par
LE CHEVALIER : Ah ! que dis-tu là ? c’est le plus célèbre athée, l’homme le plus immoral… Oh ! c’est bien la corruption la plus complète et la plus entière, l’individu le plus méchant et le plus scélérat qui puisse exister au monde.\par
MME DE SAINT-ANGE : Comme tout cela m’échauffe, je vais raffoler de cet homme, et ses goûts, mon frère ?\par
LE CHEVALIER : Tu les sais ; les délices de Sodome lui sont aussi chers comme agent que comme patient ; il n’aime que les hommes dans ses plaisirs, et si quelquefois néanmoins il consent à essayer les femmes, ce n’est qu’aux conditions qu’elles seront assez complaisantes pour changer de sexe avec lui. Je lui ai parlé de toi, je l’ai prévenu de tes intentions ; il accepte et t’avertit à son tour des clauses du marché. Je t’en préviens, ma sœur, il te refusera tout net, si tu prétends l’engager à autre chose : ce que je consens à faire avec votre sœur, est, prétend-il, une licence… une incartade dont on ne se souille que rarement et avec beaucoup de précautions.\par
MME DE SAINT-ANGE : {\itshape Se souiller !… des précautions} ! J’aime à la folie le langage de ces aimables gens ; entre nous autres femmes, nous avons aussi de ces mots exclusifs qui prouvent comme ceux-là, l’horreur profonde dont elles sont pénétrées pour tout ce qui ne tient pas au culte admis… Eh, dis-moi, mon cher… il t’a eu ? Avec ta délicieuse figure et tes vingt ans, on peut, je crois, captiver un tel homme !\par
LE CHEVALIER : Je ne te cacherai point mes extravagances avec lui, tu as trop d’esprit pour les blâmer. Dans le fait, j’aime les femmes moi, et je ne me livre à ces goûts bizarres que quand un homme aimable m’en presse. Il n’y a rien que je ne fasse alors ; je suis loin de cette morgue ridicule qui fait croire à nos jeunes freluquets qu’il faut répondre par des coups de canne à de semblables propositions ; l’homme est-il le maître de ses goûts ? Il faut plaindre ceux qui en ont de singuliers, mais ne les insulter jamais, leur tort est celui de la nature, ils n’étaient pas plus les maîtres d’arriver au monde avec des goûts différents que nous ne le sommes de naître ou bancal ou bien fait. Un homme vous dit-il d’ailleurs une chose désagréable en vous témoignant le désir qu’il a de jouir de vous ? non, sans doute, c’est un compliment qu’il vous fait ; pourquoi donc y répondre par des injures ou des insultes ? Il n’y a que les sots qui puissent penser ainsi, jamais un homme raisonnable ne parlera de cette matière différemment que je ne fais ; mais c’est que le monde est peuplé de plats imbéciles qui croient que c’est leur manquer que de leur avouer qu’on les trouve propres à des plaisirs, et qui, gâtés par les femmes, toujours jalouses de ce qui a l’air d’attenter à leurs droits, s’imaginent être les Don Quichotte de ces droits ordinaires, en brutalisant ceux qui n’en reconnaissent pas toute l’étendue.\par
MME DE SAINT-ANGE : Ah ! mon ami, baise-moi, tu ne serais pas mon frère si tu pensais différemment ; mais un peu de détails, je t’en conjure, et sur le physique de cet homme et sur ses plaisirs avec toi.\par
LE CHEVALIER : M. Dolmancé était instruit par un de mes amis, du superbe membre dont tu sais que je suis pourvu, il engagea le marquis de V*** à me donner à souper avec lui. Une fois là, il fallut bien exhiber ce que je portais ; la curiosité parut d’abord être le seul motif, un très beau cul qu’on me tourna, et dont on me supplia de jouir, me fit bientôt voir que le goût seul avait eu part à cet examen. Je prévins Dolmancé de toutes les difficultés de l’entreprise, rien ne l’effaroucha. Je suis à l’épreuve du bélier, me dit-il, et vous n’aurez même pas la gloire d’être le plus redoutable des hommes qui perforèrent le cul que je vous offre. Le marquis était là, il nous encourageait en tripotant, maniant, baisant tout ce que nous mettions au jour l’un et l’autre. Je me présente… je veux au moins quelques apprêts : « Gardez-vous-en bien, me dit le marquis, vous ôteriez la moitié des sensations que Dolmancé attend de vous ; il veut qu’on le pourfende… il veut qu’on le déchire. – Il sera satisfait », dis-je en me plongeant aveuglément dans le gouffre… et tu crois peut-être, ma sœur, que j’eus beaucoup de peine…, pas un mot ; mon vit, tout énorme qu’il est, disparut sans que je m’en doutasse, et je touchai le fond de ses entrailles sans que le bougre eût l’air de le sentir. Je traitai Dolmancé en ami, l’excessive volupté qu’il goûtait, ses frétillements, ses propos délicieux, tout me rendit bientôt heureux moi-même, et je l’inondai. À peine fus-je dehors que Dolmancé, se retournant vers moi, échevelé, rouge comme une bacchante : « Tu vois l’état où tu m’as mis, cher Chevalier, me dit-il, en m’offrant un vit sec et mutin, fort long et d’au moins six pouces de tour, daigne, je t’en conjure, ô mon amour ! me servir de femme après avoir été mon amant, et que je puisse dire que j’ai goûté dans tes bras divins tous les plaisirs du goût que je chéris avec tant d’empire. » Trouvant aussi peu de difficultés à l’un qu’à l’autre, je me prêtai ; le marquis se déculottant à mes yeux, me conjura de vouloir bien être encore un peu homme avec lui pendant que j’allais être la femme de son ami ; je le traitai comme Dolmancé, qui me rendant au centuple toutes les secousses dont j’accablais notre tiers, exhala bientôt au fond de mon cul, cette liqueur enchanteresse dont j’arrosais presque en même temps celui de V***.\par
MME DE SAINT-ANGE : Tu dois avoir eu le plus grand plaisir, mon frère, à te trouver ainsi entre deux, on dit que c’est charmant.\par
LE CHEVALIER : Il est bien certain, mon ange, que c’est la meilleure place ; mais quoi qu’on en puisse dire, tout cela sont des extravagances que je ne préférerai jamais au plaisir des femmes.\par
MME DE SAINT-ANGE : Eh bien ! mon cher amour, pour récompenser aujourd’hui ta délicate complaisance, je vais livrer à tes ardeurs une jeune fille vierge, et plus belle que l’amour.\par
LE CHEVALIER : Comment, avec Dolmancé… tu fais venir une femme chez toi ?\par
MME DE SAINT-ANGE : Il s’agit d’une éducation, c’est une petite fille que j’ai connue au couvent l’automne dernier, pendant que mon mari était aux eaux. Là nous ne pûmes rien, nous n’osâmes rien, trop d’yeux étaient fixés sur nous, mais nous nous promîmes de nous réunir dès que cela serait possible ; uniquement occupée de ce désir j’ai, pour y satisfaire, fait connaissance avec sa famille. Son père est un libertin… que j’ai captivé. Enfin la belle vient, je l’attends, nous passerons deux jours ensemble… deux jours délicieux, la meilleure partie de ce temps, je l’emploie à éduquer cette jeune personne. Dolmancé et moi nous placerons dans cette jolie petite tête tous les principes du libertinage le plus effréné, nous l’embraserons de nos feux, nous l’alimenterons de notre philosophie, nous lui inspirerons nos désirs, et comme je veux joindre un peu de pratique à la théorie, comme je veux qu’on démontre à mesure qu’on dissertera, je t’ai destiné, mon frère, à la moisson des myrtes de Cythère, Dolmancé à celle des roses de Sodome. J’aurai deux plaisirs à la fois, celui de jouir moi-même de ces voluptés criminelles et celui d’en donner des leçons, d’en inspirer les goûts à l’aimable innocente que j’attire dans nos filets. Eh bien Chevalier, ce projet est-il digne de mon imagination ?\par
LE CHEVALIER : Il ne peut être conçu que par elle, il est divin, ma sœur, et je te promets d’y remplir à merveille le rôle charmant que tu m’y destines. Ah ! friponne, comme tu vas jouir du plaisir d’éduquer cette enfant ; quelles délices pour toi de la corrompre, d’étouffer dans ce jeune cœur toutes les semences de vertu et de religion qu’y placèrent ses institutrices ! En vérité, cela est trop {\itshape roué} pour moi.\par
MME DE SAINT-ANGE : Il est bien sûr que je n’épargnerai rien pour la pervertir, pour dégrader, pour culbuter dans elle tous les faux principes de morale dont on aurait pu l’étourdir ; je veux, en deux leçons, la rendre aussi scélérate que moi… aussi impie… aussi débauchée. Préviens Dolmancé, mets-le au fait dès qu’il arrivera, pour que le venin de ses immoralités, circulant dans ce jeune cœur avec celui que j’y lancerai, parvienne à déraciner dans peu d’instants toutes les semences de vertu qui pourraient y germer sans nous.\par
LE CHEVALIER : Il était impossible de mieux trouver l’homme qu’il te fallait, l’irréligion, l’impiété, l’inhumanité, le libertinage découlent des lèvres de Dolmancé, comme autrefois l’onction mystique, de celles du célèbre archevêque de Cambrai ; c’est le plus profond séducteur, l’homme le plus corrompu, le plus dangereux… Ah ! ma chère amie, que ton élève réponde aux soins de l’instituteur, et je te la garantis bientôt perdue.\par
MME DE SAINT-ANGE : Cela ne sera sûrement pas long avec les dispositions que je lui connais…\par
LE CHEVALIER : Mais dis-moi, chère sœur, ne redoutes-tu rien des parents ? Si cette petite fille venait à jaser quand elle retournera chez elle.\par
MME DE SAINT-ANGE : Ne crains rien, j’ai séduit le père… il est à moi, faut-il enfin te l’avouer, je me suis livrée à lui pour qu’il fermât les yeux, il ignore mes desseins, mais il n’osera jamais les approfondir… Je le tiens.\par
LE CHEVALIER : Tes moyens sont affreux.\par
MME DE SAINT-ANGE : Voilà comme il les faut pour qu’ils soient sûrs.\par
LE CHEVALIER : Eh ! dis-moi, je te prie, quelle est cette jeune personne ?\par
MME DE SAINT-ANGE : On la nomme {\itshape Eugénie}, elle est la fille d’un certain Mistival, l’un des plus riches traitants de la capitale, âgé d’environ trente-six ans ; la mère en a tout au plus trente-deux, et la petite fille quinze. Mistival est aussi libertin que sa femme est dévote. Pour Eugénie, ce serait en vain, mon ami, que j’essaierais de te la peindre : elle est au-dessus de mes pinceaux, qu’il te suffise d’être convaincu que ni toi, ni moi n’avons certainement jamais vu rien d’aussi délicieux au monde.\par
LE CHEVALIER : Mais esquisse au moins, si tu ne peux peindre, afin que sachant à peu près à qui je vais avoir affaire, je me remplisse mieux l’imagination de l’idole où je dois sacrifier.\par
MME DE SAINT-ANGE : Eh bien ! mon ami, ses cheveux châtains qu’à peine on peut empoigner, lui descendent au bas des fesses, son teint est d’une blancheur éblouissante, son nez un peu aquilin, ses yeux d’un noir d’ébène, et d’une ardeur… Oh ! mon ami, il n’est pas possible de tenir à ces yeux-là… Tu n’imagines point toutes les sottises qu’ils m’ont fait faire… Si tu voyais les jolis sourcils qui les couronnent… les intéressantes paupières qui les bordent, sa bouche est très petite, ses dents superbes, et tout cela d’une fraîcheur… Une de ses beautés est la manière élégante dont sa belle tête est attachée sur ses épaules, l’air de noblesse qu’elle a quand elle la tourne… Eugénie est grande pour son âge, on lui donnerait dix-sept ans, sa taille est un modèle d’élégance et de finesse, sa gorge délicieuse… ; ce sont bien les deux plus jolis tétons… à peine y a-t-il de quoi remplir la main, mais si doux… si frais… si blancs ; vingt fois j’ai perdu la tête en les baisant, et si tu avais vu comme elle s’animait sous mes caresses… comme ses deux grands yeux me peignaient l’état de son âme… ; mon ami, je ne sais pas comme est le reste. Ah ! s’il faut en juger par ce que je connais, jamais l’Olympe n’eut une divinité qui la valût… Mais je l’entends… Laisse-nous, sors par le jardin pour ne la point rencontrer, et sois exact au rendez-vous.\par
LE CHEVALIER : Le tableau que tu viens de me faire te répond de mon exactitude… Oh ciel ! sortir… te quitter dans l’état où je suis… Adieu… un baiser… un seul baiser, ma sœur, pour me satisfaire au moins jusque-là.\par
{\itshape Elle le baise, touche son vit au travers de sa culotte, et le jeune homme sort avec précipitation.}
\section[{Second dialogue}]{Second dialogue}\phantomsection
\label{d2}\renewcommand{\leftmark}{Second dialogue}

\textit{MME DE SAINT-ANGE, EUGÉNIE}\par
\noindent MME DE SAINT-ANGE : Eh ! bonjour, ma belle, je t’attendais avec une impatience que tu devines bien aisément si tu lis dans mon cœur.\par
EUGÉNIE : Oh ! ma toute bonne, j’ai cru que je n’arriverais jamais, tant j’avais d’empressement d’être dans tes bras ; une heure avant que de partir j’ai frémi que tout ne changeât ; ma mère s’opposait absolument à cette délicieuse partie, elle prétendait qu’il n’était pas convenable qu’une jeune fille de mon âge allât seule ; mais mon père l’avait si mal traitée avant-hier qu’un seul de ses regards a fait rentrer M\textsuperscript{me} de Mistival dans le néant ; elle a fini par consentir à ce qu’accordait mon père, et je suis accourue. On me donne deux jours, il faut absolument que ta voiture et l’une de tes femmes me ramène après-demain.\par
MME DE SAINT-ANGE : Que cet intervalle est court, mon cher ange, à peine pourrai-je, en si peu de temps, t’exprimer tout ce que tu m’inspires…, et d’ailleurs nous avons à causer ; ne sais-tu pas que c’est dans cette entrevue que je dois t’initier dans les plus secrets mystères de Vénus ; aurons-nous le temps en deux jours ?\par
EUGÉNIE : Ah ! si je ne savais pas tout je resterais… je suis venue ici pour m’instruire et je ne m’en irai pas que je ne sois savante…\par
MME DE SAINT-ANGE, {\itshape la baisant} : Oh ! cher amour, que de choses nous allons faire et dire réciproquement ; mais à propos veux-tu déjeuner, ma reine, il serait possible que la leçon fût longue ?\par
EUGÉNIE : Je n’ai, chère amie, d’autre besoin que celui de t’entendre, nous avons déjeuné à une lieue d’ici, j’attendrais maintenant jusqu’à huit heures du soir sans éprouver le moindre besoin.\par
MME DE SAINT-ANGE : Passons donc dans mon boudoir, nous y serons plus à l’aise ; j’ai déjà prévenu mes gens ; sois assurée qu’on ne s’avisera pas de nous interrompre.\par
{\itshape Elles y passent dans les bras l’une de l’autre.}
\section[{Troisième dialogue}]{Troisième dialogue}\phantomsection
\label{d3}\renewcommand{\leftmark}{Troisième dialogue}

\textit{La scène est dans un boudoir délicieux.}\par
\textit{MME DE SAINT-ANGE, EUGÉNIE, DOLMANCÉ}\par
\noindent EUGÉNIE, {\itshape très surprise de voir dans ce cabinet un homme qu’elle n’attendait pas} : Oh dieu, ma chère amie, c’est une trahison !\par
MME DE SAINT-ANGE, {\itshape également surprise} : Par quel hasard ici, monsieur, vous ne deviez ce me semble arriver qu’à quatre heures ?\par
DOLMANCÉ : On devance toujours le plus qu’on peut le bonheur de vous voir, madame ; j’ai rencontré monsieur votre frère, il a senti le besoin dont serait ma présence aux leçons que vous devez donner à mademoiselle, il savait que ce serait ici le lycée où se ferait le cours, il m’y a secrètement introduit, n’imaginant pas que vous le désapprouvassiez, et pour lui, comme il sait que ses démonstrations ne seront nécessaires qu’après les dissertations théoriques, il ne paraîtra que tantôt.\par
MME DE SAINT-ANGE : En vérité, Dolmancé, voilà un tour…\par
EUGÉNIE : Dont je ne suis pas la dupe, ma bonne amie, tout cela est ton ouvrage…, au moins fallait-il me consulter…, me voilà d’une honte à présent qui, certainement, s’opposera à tous nos projets.\par
MME DE SAINT-ANGE : Je te proteste, Eugénie, que l’idée de cette surprise n’appartient qu’à mon frère ; mais qu’elle ne t’effraie pas, Dolmancé que je connais pour un homme fort aimable, et précisément du degré de philosophie qu’il nous faut pour ton instruction, ne peut qu’être très utile à nos projets ; à l’égard de sa discrétion, je te réponds de lui comme de moi. Familiarise-toi donc, ma chère, avec l’homme du monde le plus en état de te former, et de te conduire dans la carrière du bonheur et les plaisirs que nous voulons parcourir ensemble.\par
EUGÉNIE, {\itshape rougissant} : Oh ! je n’en suis pas moins d’une confusion…\par
DOLMANCÉ : Allons, belle Eugénie, mettez-vous à votre aise… la pudeur est une vieille vertu dont vous devez, avec autant de charmes, savoir vous passer à merveille.\par
EUGÉNIE : Mais la décence…\par
DOLMANCÉ : Autre usage gothique, dont on fait bien peu cas aujourd’hui. Il contrarie si fort la nature.\par
{\itshape Dolmancé saisit Eugénie, la presse entre ses bras et la baise.}\par
EUGÉNIE, {\itshape se défendant} : Finissez donc, monsieur… ; en vérité, vous me ménagez bien peu.\par
MME DE SAINT-ANGE : Eugénie, crois-moi, cessons l’une et l’autre d’être prudes avec cet homme charmant ; je ne le connais pas plus que toi, regarde pourtant comme je me livre à lui {\itshape (elle le baise lubriquement sur la bouche)} ; imite-moi.\par
EUGÉNIE : Oh ! je le veux bien ; de qui prendrais-je de meilleurs exemples !\par
{\itshape Elle se livre à Dolmancé qui la baise ardemment langue en bouche.}\par
DOLMANCÉ : Ah ! l’aimable et délicieuse créature.\par
MME DE SAINT-ANGE, {\itshape la baisant de même} : Crois-tu donc, petite friponne, que je n’aurai pas également mon tour ?\par
{\itshape Ici Dolmancé les tenant l’une et l’autre dans ses bras, les langote un quart d’heure toutes deux, et toutes deux se le rendent et le lui rendent. }\par
DOLMANCÉ : Ah ! voilà des préliminaires qui m’enivrent de volupté ! Mesdames, voulez-vous m’en croire, il fait extraordinairement chaud, mettons-nous à notre aise, nous jaserons infiniment mieux.\par
MME DE SAINT-ANGE : J’y consens ; revêtons-nous de ces simarres de gaze ; elles ne voileront de nos attraits que ce qu’il faut cacher au désir.\par
EUGÉNIE : En vérité, ma bonne, vous me faites faire des choses…\par
MME DE SAINT-ANGE, {\itshape l’aidant à se déshabiller} : Tout à fait ridicules, n’est-ce pas ?\par
EUGÉNIE : Au moins bien indécentes, en vérité… eh ! comme tu me baises !\par
MME DE SAINT-ANGE : La jolie gorge… c’est une rose à peine épanouie.\par
DOLMANCÉ, {\itshape considérant les tétons d’Eugénie sans les toucher} : Et qui promet d’autres appas… infiniment plus estimables.\par
MME DE SAINT-ANGE : Plus estimables ?\par
DOLMANCÉ : Oh ! oui, d’honneur !\par
{\itshape En disant cela, Dolmancé fait mine de retourner Eugénie pour l’examiner par-derrière.}\par
EUGÉNIE : Oh ! non, non, je vous en conjure.\par
MME DE SAINT-ANGE : Non, Dolmancé…, je ne veux pas que vous voyiez encore… un objet dont l’empire est trop grand sur vous, pour que l’ayant une fois dans la tête, vous puissiez ensuite raisonner de sens-froid\footnote{ Sic. {\itshape (Note du correcteur - ELG.)}}. Nous avons besoin de vos leçons, donnez-nous-les, et les myrtes que vous voulez cueillir formeront ensuite votre couronne.\par
DOLMANCÉ : Soit, mais pour démontrer, pour donner à ce bel enfant les premières leçons du libertinage, il faut bien au moins vous, madame, que vous ayez la complaisance de vous prêter.\par
MME DE SAINT-ANGE : À la bonne heure… Eh bien ! tenez, me voilà toute nue, dissertez sur moi autant que vous voudrez.\par
DOLMANCÉ : Ah ! le beau corps… C’est Vénus, elle-même, embellie par les grâces !\par
EUGÉNIE : Oh ! ma chère amie, que d’attraits, laissez-moi les parcourir à mon aise, laissez-moi les couvrir de baisers.\par

\labelblock{Elle exécute.}

\noindent DOLMANCÉ : Quelles excellentes dispositions ! Un peu moins d’ardeur, belle Eugénie, ce n’est que de l’attention que je vous demande pour ce moment-ci.\par
EUGÉNIE : Allons, j’écoute, j’écoute… C’est qu’elle est si belle… si potelée, si fraîche : ah ! comme elle est charmante, ma bonne amie, n’est-ce pas, monsieur ?\par
DOLMANCÉ : Elle est belle, assurément… parfaitement belle ; mais je suis persuadé que vous ne le lui cédez en rien… Allons, écoutez-moi, jolie petite élève, ou craignez que, si vous n’êtes pas docile, je n’use sur vous des droits que me donne amplement le titre de votre instituteur.\par
MME DE SAINT-ANGE : Oh ! oui, oui, Dolmancé, je vous la livre, il faut la gronder d’importance si elle n’est pas sage.\par
DOLMANCÉ : Je pourrais bien ne pas m’en tenir aux remontrances.\par
EUGÉNIE : Oh, juste ciel ! vous m’effrayez… et qu’entreprendriez-vous donc, monsieur ?\par
DOLMANCÉ, {\itshape balbutiant et baisant Eugénie sur la bouche} : Des châtiments… des corrections, et ce joli petit cul pourrait bien me répondre des fautes de la tête.\par
{\itshape Il le lui frappe au travers de la simarre de gaze dont est maintenant vêtue Eugénie.}\par
MME DE SAINT-ANGE : Oui, j’approuve le projet, mais non pas le geste. Commençons notre leçon, ou le peu de temps que nous avons à jouir d’Eugénie va se passer ainsi en préliminaires, et l’instruction ne se fera point.\par
DOLMANCÉ (il touche à mesure, sur M\textsuperscript{me} de Saint-Ange, toutes les parties qu’il démontre) : Je commence.\par
Je ne parlerai point de ces globes de chair, vous savez aussi bien que moi, Eugénie, que l’on les nomme indifféremment {\itshape gorge, seins, tétons} ; leur usage est d’une grande vertu dans le plaisir, un amant les a sous les yeux en jouissant, il les caresse, il les manie, quelques-uns en forment même le siège de la jouissance, et leur membre se nichant entre les deux monts de Vénus, que la femme serre et comprime sur ce membre, au bout de quelques mouvements, certains hommes parviennent à répandre là le baume délicieux de la vie, dont l’écoulement fait tout le bonheur des libertins… Mais ce membre sur lequel il faudra disserter sans cesse, ne serait-il pas à propos, madame, d’en donner une dissertation à notre écolière ?\par
MME DE SAINT-ANGE : Je le crois de même.\par
DOLMANCÉ : Eh bien ! madame, je vais m’étendre sur ce canapé, vous vous placerez près de moi, vous vous emparerez du sujet, et vous en expliquerez vous-même les propriétés à notre jeune élève.\par
{\itshape Dolmancé se place et M\textsuperscript{me} de Saint-Ange démontre. }\par
MME DE SAINT-ANGE : Ce sceptre de Vénus, que tu vois sous tes yeux, Eugénie, est le premier agent des plaisirs de l’amour, on le nomme membre par excellence : il n’est pas une seule partie du corps humain dans lequel il ne s’introduise ; toujours docile aux passions de celui qui le meut, tantôt il se niche là {\itshape (elle touche le con d’Eugénie)}, c’est sa route ordinaire…, la plus usitée, mais non pas la plus agréable ; recherchant un temple plus mystérieux, c’est souvent ici {\itshape (elle écarte ses fesses et montre le trou de son cul)} que le libertin cherche à jouir : nous reviendrons sur cette jouissance la plus délicieuse de toutes ; la bouche, le sein, les aisselles lui présentent souvent encore des autels où brûle son encens ; et quel que soit enfin celui de tous les endroits qu’il préfère, on le voit, après s’être agité quelques instants, lancer une liqueur blanche et visqueuse dont l’écoulement plonge l’homme dans un délire assez vif pour lui procurer les plaisirs les plus doux qu’il puisse espérer de sa vie.\par
EUGÉNIE : Oh ! que je voudrais voir couler cette liqueur !\par
MME DE SAINT-ANGE : Cela se pourrait par la simple vibration de ma main ; vois comme il s’irrite à mesure que je le secoue, ces mouvements se nomment {\itshape pollution} et, en terme de libertinage, cette action s’appelle {\itshape branler}.\par
EUGÉNIE : Oh ! ma chère amie, laisse-moi branler ce beau membre.\par
DOLMANCÉ : Je n’y tiens pas ! laissons-la faire, madame, cette ingénuité me fait horriblement bander.\par
MME DE SAINT-ANGE : Je m’oppose à cette effervescence, Dolmancé, soyez sage, l’écoulement de cette semence, en diminuant l’activité de vos esprits animaux ralentirait la chaleur de vos dissertations.\par
EUGÉNIE, {\itshape maniant les testicules de Dolmancé} : Oh ! que je suis fâchée, ma bonne amie, de la résistance que tu mets à mes désirs… Et ces boules, quel est leur usage, et comment les nomme-t-on ?\par
MME DE SAINT-ANGE : Le mot technique est {\itshape couilles},… testicules est celui de l’art. Ces boules renferment le réservoir de cette semence prolifique dont je viens de te parler, et dont l’éjaculation dans la matrice de la femme, produit l’espèce humaine ; mais nous appuierons peu sur ces détails, Eugénie, plus dépendants de la médecine que du libertinage. Une jolie fille ne doit s’occuper que de {\itshape foutre} et jamais d’{\itshape engendrer}. Nous glisserons sur tout ce qui tient au plat mécanisme de la population, pour nous attacher principalement et uniquement aux voluptés libertines dont l’esprit n’est nullement populateur.\par
EUGÉNIE : Mais, ma chère amie, lorsque ce membre énorme, qui peut à peine tenir dans ma main, pénètre, ainsi que tu m’assures que cela se peut, dans un trou aussi petit que celui de ton derrière, cela doit faire une bien grande douleur à la femme.\par
MME DE SAINT-ANGE : Soit que cette introduction se fasse par-devant, soit qu’elle se fasse par-derrière, lorsqu’une femme n’y est pas encore accoutumée, elle y éprouve toujours de la douleur. Il a plu à la Nature de ne nous faire arriver au bonheur que par des peines ; mais, une fois vaincue, rien ne peut rendre les plaisirs que l’on goûte, et celui qu’on éprouve à l’introduction de ce membre dans nos culs, est incontestablement préférable à tous ceux que peut procurer cette même introduction par-devant ; que de dangers, d’ailleurs, n’évite pas une femme alors ! moins de risques pour sa santé, et plus aucuns pour la grossesse. Je ne m’étends pas davantage à présent sur cette volupté : notre maître à toutes deux, Eugénie, l’analysera bientôt amplement, et joignant la pratique à la théorie, te convaincra, j’espère, ma toute bonne, que de tous les plaisirs de la jouissance, c’est le seul que tu doives préférer.\par
DOLMANCÉ : Dépêchez vos démonstrations, madame, je vous en conjure, je n’y puis plus tenir, je déchargerai malgré moi, et ce redoutable membre réduit à rien, ne pourrait plus servir à vos leçons.\par
EUGÉNIE : Comment ! il s’anéantirait, ma bonne, s’il perdait cette semence dont tu parles… Oh ! laisse-moi la lui faire perdre, pour que je voie comme il deviendrait… et puis j’aurais tant de plaisir à voir couler cela.\par
MME DE SAINT-ANGE : Non, non, Dolmancé, levez-vous, songez que c’est là le prix de vos travaux, et que je ne puis vous le livrer qu’après que vous l’aurez mérité.\par
DOLMANCÉ : Soit ; mais pour mieux convaincre Eugénie de tout ce que nous allons lui débiter sur le plaisir, quel inconvénient y aurait-il que vous la branliez devant moi, par exemple ?\par
MME DE SAINT-ANGE : Aucun, sans doute, et j’y vais procéder avec d’autant plus de joie, que cette épisode lubrique ne pourra qu’aider nos leçons. Place-toi sur ce canapé, ma toute bonne.\par
EUGÉNIE : Oh dieu ! la délicieuse niche ! Mais pourquoi toutes ces glaces ?\par
MME DE SAINT-ANGE : C’est pour que, répétant les attitudes en mille sens divers, elles multiplient à l’infini les mêmes jouissances aux yeux de ceux qui les goûtent sur cette ottomane ; aucune des parties de l’un ou l’autre corps ne peut être cachée par ce moyen, il faut que tout soit en vue, ce sont autant de groupes rassemblés autour de ceux que l’amour enchaîne, autant d’imitateurs de leurs plaisirs, autant de tableaux délicieux dont leur lubricité s’enivre, et qui servent bientôt à la compléter elle-même.\par
EUGÉNIE : Que cette invention est délicieuse !\par
MME DE SAINT-ANGE : Dolmancé, déshabillez vous-même la victime.\par
DOLMANCÉ : Cela ne sera pas difficile, puisqu’il ne s’agit que d’enlever cette gaze pour distinguer à nu les plus touchants attraits. {\itshape (Il la met nue, et ses premiers regards se portent aussitôt sur le derrière.)} Je vais donc le voir ce cul divin et précieux que j’ambitionne avec tant d’ardeur… Sacredieu ! que d’embonpoint et de fraîcheur, que d’éclat et d’élégance !… Je n’en vis jamais un plus beau.\par
MME DE SAINT-ANGE : Ah ! fripon, comme tes premiers hommages prouvent tes plaisirs et tes goûts !\par
DOLMANCÉ : Mais peut-il être au monde rien qui vaille cela ? Où l’Amour aurait-il de plus divins autels ?… Eugénie… sublime Eugénie, que j’accable ce cul des plus douces caresses.\par
{\itshape Il le manie et le baise avec transport.}\par
MME DE SAINT-ANGE : Arrêtez, libertin, vous oubliez qu’à moi seule appartient Eugénie, unique prix des leçons qu’elle attend de vous ; ce n’est qu’après les avoir reçues qu’elle deviendra votre récompense : suspendez cette ardeur, ou je me fâche.\par
DOLMANCÉ : Ah ! friponne ; c’est de la jalousie… Eh bien, livrez-moi le vôtre, je vais l’accabler des mêmes hommages. {\itshape (Il enlève la simarre de M\textsuperscript{me} de Saint-Ange et lui caresse le derrière.)} Ah ! qu’il est beau, mon ange… qu’il est délicieux aussi, que je les compare… que je les admire l’un près de l’autre, c’est Ganymède à côté de Vénus. {\itshape (Il les accable de baisers tous deux.)} Afin de laisser toujours sous mes yeux le spectacle enchanteur de tant de beautés, ne pourriez-vous pas, madame, en vous enchaînant l’une à l’autre, offrir sans cesse à mes regards ces culs charmants que j’idolâtre ?\par
MME DE SAINT-ANGE : À merveille… Tenez, êtes-vous satisfait ?\par
{\itshape Elles s’enlacent l’une dans l’autre, de manière à ce que leurs deux culs soient en face de Dolmancé.}\par
DOLMANCÉ : On ne saurait davantage : voilà précisément ce que je demandais ; agitez maintenant ces beaux culs de tout le feu de la lubricité ; qu’ils se baissent et se relèvent en cadence, qu’ils suivent les impressions dont le plaisir va les mouvoir… Bien, bien, c’est délicieux !\par
EUGÉNIE : Ah ! ma bonne, que tu me fais de plaisir… Comment appelle-t-on ce que nous faisons là ?\par
MME DE SAINT-ANGE : Se {\itshape branler}, ma mie,… se donner du plaisir ; mais, tiens, changeons de posture, examine mon {\itshape con}… c’est ainsi que se nomme le temple de Vénus ; cet antre que ta main couvre, examine-le bien, je vais l’entrouvrir ; cette élévation dont tu vois qu’il est couronné s’appelle la {\itshape motte} ; elle se garnit de poils communément à quatorze ou quinze ans, quand une fille commence à être réglée. Cette languette qu’on trouve au-dessous se nomme le {\itshape clitoris}. Là gît toute la sensibilité des femmes, c’est le foyer de toute la mienne ; on ne saurait me chatouiller cette partie sans me voir pâmer de plaisir… Essaie-le… Ah ! petite friponne, comme tu y vas… On dirait que tu n’as fait que cela toute ta vie…, arrête… arrête… Non, te dis-je, je ne veux pas me livrer… Ah contenez-moi, Dolmancé, sous les doigts enchanteurs de cette jolie fille, je suis prête à perdre la tête.\par
DOLMANCÉ : Eh bien ! pour attiédir, s’il se peut, vos idées en les variant, branlez-la vous-même ; contenez-vous, et qu’elle seule se livre… Là, oui, dans cette attitude ; son joli cul, de cette manière, va se trouver sous mes mains ; je vais le {\itshape polluer} légèrement d’un doigt… Livrez-vous, Eugénie, abandonnez tous vos sens au plaisir, qu’il soit le seul dieu de votre existence ; c’est à lui seul qu’une jeune fille doit tout sacrifier, et rien à ses yeux ne doit être aussi sacré que le plaisir.\par
EUGÉNIE : Ah ! rien au moins n’est aussi délicieux, je l’éprouve… Je suis hors de moi… Je ne sais plus ce que je dis, ni ce que je fais… quelle ivresse s’empare de mes sens !\par
DOLMANCÉ : Comme la petite friponne décharge… Son anus se resserre à me couper le doigt… Qu’elle serait délicieuse à enculer dans cet instant !\par
{\itshape Il se lève et présente son vit au trou du cul de la jeune fille.}\par
MME DE SAINT-ANGE : Encore un moment de patience. Que l’éducation de cette chère fille nous occupe seule… Il est si doux de la former.\par
DOLMANCÉ : Eh bien ! Tu le vois, Eugénie, après une pollution plus ou moins longue, les glandes séminales se gonflent et finissent par exhaler une liqueur dont l’écoulement plonge la femme dans le transport le plus délicieux. Cela s’appelle {\itshape décharger}, quand ta bonne amie le voudra, je te ferai voir de quelle manière plus énergique et plus impérieuse cette même opération se fait dans les hommes.\par
MME DE SAINT-ANGE : Attends, Eugénie, je vais maintenant t’apprendre une nouvelle manière de plonger une femme dans la plus extrême volupté, écarte bien tes cuisses… Dolmancé, vous voyez que de la façon dont je la place son cul vous reste, gamahuchez-le-lui pendant que son con va l’être par ma langue, et faisons-la pâmer entre nous, ainsi, trois ou quatre fois de suite, s’il se peut. Ta motte est charmante, Eugénie, que j’aime à baiser ce petit poil follet… Ton clitoris, que je vois mieux maintenant, est peu formé, mais bien sensible… Comme tu frétilles… Laisse-moi t’écarter… Ah ! tu es bien sûrement vierge, dis-moi l’effet que tu vas éprouver dès que nos langues vont s’introduire, à la fois, dans tes deux ouvertures.\par
{\itshape On exécute.}\par
EUGÉNIE : Ah ! ma chère. C’est délicieux, c’est une sensation impossible à peindre ; il me serait bien difficile de dire laquelle de vos deux langues me plonge mieux dans le délire.\par
DOLMANCÉ : Par l’attitude où je me place, mon vit est très près de vos mains, madame ; daignez le branler, je vous prie, pendant que je suce ce cul divin. Enfoncez davantage votre langue, madame, ne vous en tenez pas à lui sucer le clitoris, faites pénétrer cette langue voluptueuse jusque dans la matrice, c’est la meilleure façon de hâter l’éjaculation de son foutre.\par
EUGÉNIE, {\itshape se roidissant} : Ah ! je n’en peux plus, je me meurs, ne m’abandonnez pas, mes amis, je suis prête à m’évanouir.\par
{\itshape Elle décharge au milieu de ses deux instituteurs.}\par
MME DE SAINT-ANGE : Eh bien ! ma mie, comment te trouves-tu du plaisir que nous t’avons donné ?\par
EUGÉNIE : Je suis morte, je suis brisée…, je suis anéantie… Mais expliquez-moi, je vous prie, deux mots que vous avez prononcés et que je n’entends pas ; d’abord que signifie {\itshape matrice} ?\par
MME DE SAINT-ANGE : C’est une espèce de vase ressemblant à une bouteille dont le cou embrasse le membre de l’homme, et qui reçoit le foutre produit chez la femme par le suintement des glandes, et, dans l’homme, par l’éjaculation que nous te ferons voir ; et du mélange de ces liqueurs naît le germe qui produit tour à tour des garçons ou des filles.\par
EUGÉNIE : Ah ! j’entends ; cette définition m’explique en même temps le mot {\itshape foutre} que je n’avais pas d’abord bien compris. Et l’union des semences est-elle nécessaire à la formation du fœtus ?\par
MME DE SAINT-ANGE : Assurément, quoiqu’il soit néanmoins prouvé que ce fœtus ne doive son existence qu’au foutre de l’homme ; élancé seul, sans mélange avec celui de la femme, il ne réussirait cependant pas ; mais celui que nous fournissons ne fait qu’élaborer, il ne crée point, il aide à la création, sans en être la cause ; plusieurs naturalistes modernes prétendent même qu’il est inutile, d’où les moralistes, toujours guidés par la découverte de ceux-ci, ont conclu, avec assez de vraisemblance, qu’en ce cas l’enfant formé du sang du père ne devait de tendresse qu’à lui. Cette assertion n’est point sans apparence, et, quoique femme, je ne m’aviserais pas de la combattre.\par
EUGÉNIE : Je trouve dans mon cœur la preuve de ce que tu me dis, ma bonne, car j’aime mon père à la folie, et je sens que je déteste ma mère.\par
DOLMANCÉ : Cette prédilection n’a rien d’étonnant ; j’ai pensé tout de même ; je ne suis pas encore consolé de la mort de mon père, et lorsque je perdis ma mère, je fis un feu de joie… je la détestais cordialement. Adoptez, sans crainte, ces mêmes sentiments, Eugénie, ils sont dans la nature. Uniquement formés du sang de nos pères, nous ne devons absolument rien à nos mères, elles n’ont fait d’ailleurs que se prêter dans l’acte, au lieu que le père l’a sollicité ; le père a donc voulu notre naissance pendant que la mère n’a fait qu’y consentir ; quelle différence pour les sentiments !\par
MME DE SAINT-ANGE : Mille raisons de plus sont en ta faveur, Eugénie ; s’il est une mère au monde qui doive être détestée, c’est assurément la tienne, acariâtre, superstitieuse, dévote, grondeuse… et d’une pruderie révoltante ; je gagerais que cette bégueule n’a pas fait un faux pas dans sa vie ; ah ! ma chère, que je déteste les femmes vertueuses… mais nous y reviendrons.\par
DOLMANCÉ : Ne serait-il pas nécessaire, à présent, qu’Eugénie, dirigée par moi, apprît à rendre ce que vous venez de lui prêter, et qu’elle vous branlât sous mes yeux ?\par
MME DE SAINT-ANGE : J’y consens, je le crois même utile, et sans doute que, pendant l’opération, vous voulez aussi voir mon cul, Dolmancé ?\par
DOLMANCÉ : Pouvez-vous douter, madame, du plaisir avec lequel je lui rendrai mes plus doux hommages ?\par
MME DE SAINT-ANGE, {\itshape lui présentant les fesses} : Eh bien ! me trouvez-vous comme il faut ainsi ?\par
DOLMANCÉ : À merveille, je puis au mieux vous rendre, de cette manière, les mêmes services dont Eugénie s’est si bien trouvée. Placez-vous à présent, petite folle, la tête bien entre les jambes de votre amie, et rendez-lui, avec votre jolie langue, les mêmes soins que vous venez d’en obtenir. Comment donc ! mais par l’attitude je pourrai posséder vos deux culs, je manierai délicieusement celui d’Eugénie, en suçant celui de sa belle amie… Là, bien… Voyez comme nous sommes ensemble.\par
MME DE SAINT-ANGE, {\itshape se pâmant} : Je me meurs, sacredieu !… Dolmancé, que j’aime à toucher ton beau vit, pendant que je décharge… Je voudrais qu’il m’inondât de foutre… Branlez… sucez-moi, foutredieu ! Ah ! que j’aime à faire la {\itshape putain} quand mon sperme éjacule ainsi… C’est fini, je n’en puis plus… vous m’avez accablée tous les deux, je crois que de mes jours je n’eus tant de plaisir.\par
EUGÉNIE : Que je suis aise d’en être la cause ; mais un mot, chère amie, un mot vient de t’échapper encore, et je ne l’entends pas. Qu’entends-tu par cette expression de {\itshape putain} ? Pardon, mais tu sais que je suis ici pour m’instruire.\par
MME DE SAINT-ANGE : On appelle de cette manière, ma toute belle, ces victimes publiques de la débauche des hommes, toujours prêtes à se livrer à leur tempérament ou à leur intérêt ; heureuses et respectables créatures, que l’opinion flétrit, mais que la volupté couronne, et qui, bien plus nécessaires à la société que les prudes, ont le courage de sacrifier pour la servir, la considération que cette société ose leur enlever injustement. Vivent celles que ce titre honore à leurs yeux ! Voilà les femmes vraiment aimables, les seules véritablement philosophes ! Quant à moi, ma chère, qui depuis douze ans travaille à le mériter, je t’assure que loin de m’en formaliser, je m’en amuse ; il y a mieux, j’aime qu’on me nomme ainsi quand on me fout, cette injure m’échauffe la tête.\par
EUGÉNIE : Oh ! je le conçois, ma bonne, je ne serais pas fâchée non plus que l’on me l’adressât, encore bien moins d’en mériter le titre ; mais la vertu ne s’oppose-t-elle pas à une telle inconduite, et ne l’offensons-nous pas en nous comportant comme nous le faisons ?\par
DOLMANCÉ : Ah ! renonce aux vertus, Eugénie, est-il un seul des sacrifices qu’on puisse faire à ces fausses divinités, qui vaille une minute des plaisirs que l’on goûte en les outrageant ? Va, la vertu n’est qu’une chimère dont le culte ne consiste qu’à des immolations perpétuelles, qu’à des révoltes sans nombre contre les inspirations du tempérament ; de tels mouvements peuvent-ils être naturels ? la Nature conseille-t-elle ce qui l’outrage ? Ne sois pas la dupe, Eugénie, de ces femmes que tu entends nommer vertueuses, ce ne sont pas, si tu veux, les mêmes passions que nous qu’elles servent, mais elles en ont d’autres, et souvent bien plus méprisables… C’est l’ambition, c’est l’orgueil, ce sont des intérêts particuliers, souvent encore la froideur seule d’un tempérament qui ne leur conseille rien ; devons-nous quelque chose à de pareils êtres, je le demande ? n’ont-elles pas suivi les uniques impressions de l’amour de soi ? Est-il donc meilleur, plus sage, plus à propos de sacrifier à l’égoïsme qu’aux passions ? Pour moi, je crois que l’un vaut bien l’autre, et qui n’écoute que cette dernière voix, a bien plus de raison sans doute, puisqu’elle est seule l’organe de la Nature, tandis que l’autre n’est que celle de la sottise et du préjugé. Une seule goutte de foutre éjaculée de ce membre, Eugénie, m’est plus précieuse que les actes les plus sublimes d’une vertu que je méprise.\par
EUGÉNIE {\itshape (Le calme s’étant un peu rétabli pendant ces dissertations, les femmes revêtues de leurs simarres, sont à demi couchées sur le canapé, et Dolmancé auprès d’elles dans un grand fauteuil)} : Mais il est des vertus de plus d’une espèce que pensez-vous, par exemple, de la piété ?\par
DOLMANCÉ : Que peut être cette vertu pour qui ne croit pas à la religion ? et qui peut croire à la religion ? Voyons, raisonnons avec ordre, Eugénie, n’appelez-vous pas religion le pacte qui lie l’homme à son Créateur, et qui l’engage à lui témoigner, par un culte, la reconnaissance qu’il a de l’existence reçue de ce sublime auteur ?\par
EUGÉNIE : On ne peut mieux le définir.\par
DOLMANCÉ : Eh bien ! s’il est démontré que l’homme ne doit son existence qu’aux plans irrésistibles de la Nature ; s’il est prouvé qu’aussi ancien sur ce globe que le globe même, il n’est, comme le chêne, comme le lion, comme les minéraux qui se trouvent dans les entrailles de ce globe, qu’une production nécessitée par l’existence du globe, et qui ne doit la sienne à qui que ce soit ; s’il est démontré que ce Dieu, que les sots regardent comme auteur et fabricateur unique de tout ce que nous voyons, n’est que le {\itshape nec plus ultra} de la raison humaine, que le fantôme créé à l’instant où cette raison ne voit plus rien, afin d’aider à ses opérations ; s’il est prouvé que l’existence de ce Dieu est impossible, et que la Nature, toujours en action, toujours en mouvement, tient d’elle-même ce qu’il plaît aux sots de lui donner gratuitement ; s’il est certain qu’à supposer que cet être inerte existât, ce serait assurément le plus ridicule de tous les êtres, puisqu’il n’aurait servi qu’un seul jour, et que depuis des millions de siècles il serait dans une inaction méprisable ; qu’à supposer qu’il existât, comme les religions nous le peignent, ce serait assurément le plus détestable des êtres, puisqu’il permettrait le mal sur la terre, tandis que sa toute-puissance pourrait l’empêcher ; si, dis-je, tout cela se trouvait prouvé, comme cela l’est incontestablement, croyez-vous alors, Eugénie, que la piété qui lierait l’homme à ce Créateur imbécile, insuffisant, féroce et méprisable, fût une vertu bien nécessaire ?\par
EUGÉNIE, {\itshape à M\textsuperscript{me} de Saint-Ange} : Quoi ! réellement, mon aimable amie, l’existence de Dieu serait une chimère ?\par
MME DE SAINT-ANGE : Et des plus méprisables, sans doute.\par
DOLMANCÉ : Il faut avoir perdu le sens pour y croire ; fruit de la frayeur des uns et de la faiblesse des autres, cet abominable fantôme, Eugénie, est inutile au système de la terre, il y nuirait infailliblement, puisque ses volontés, qui devraient être justes, ne pourraient jamais s’allier avec les injustices essentielles aux lois de la nature, qu’il devrait constamment vouloir le bien, et que la nature ne doit le désirer qu’en compensation du mal qui sert à ses lois, qu’il faudrait qu’il agît toujours, et que la nature, dont cette action perpétuelle est une des lois, ne pourrait que se trouver en concurrence et en opposition perpétuelle avec lui. Mais dira-t-on à cela que dieu et la nature sont la même chose, ne serait-ce pas une absurdité ? La chose créée ne peut être égale à l’être créant ; est-il possible que la montre soit l’horloger ? Eh bien, continuera-t-on, la nature n’est rien, c’est dieu qui est tout, autre bêtise ; il y a nécessairement deux choses dans l’univers, l’agent créateur, et l’individu créé ; or, quel est cet agent créateur, voilà la seule difficulté qu’il faut résoudre, c’est la seule question à laquelle il faille répondre. Si la matière agit, se meut, par des combinaisons qui nous sont inconnues, si le mouvement est inhérent à la matière, si elle seule enfin peut, en raison de son énergie, créer, produire, conserver, maintenir, balancer dans les plaines immenses de l’espace tous les globes dont la vue nous surprend et dont la marche uniforme, invariable, nous remplit de respect et d’admiration, quel sera le besoin de chercher alors un agent étranger à tout cela, puisque cette faculté active se trouve essentiellement dans la nature elle-même, qui n’est autre chose que la matière en action, votre chimère déifique éclaircira-t-elle quelque chose ? Je défie qu’on puisse me le prouver ; à supposer que je me trompe sur les facultés internes de la matière, je n’ai du moins devant moi qu’une difficulté ; que faites-vous en m’offrant votre Dieu ? vous m’en donnez une de plus, et comment voulez-vous que j’admette pour cause de ce que je ne comprends pas quelque chose que je comprends encore moins ? Sera-ce au moyen des dogmes de la religion chrétienne que j’examinerai… que je me représenterai votre effroyable dieu, voyons un peu comme elle me le peint, que vois-je dans le dieu de ce culte infâme, si ce n’est pas un être inconséquent et barbare, créant aujourd’hui un monde, de la construction duquel il se repent demain ; qu’y vois-je, qu’un être faible qui ne peut jamais faire prendre à l’homme le pli qu’il voudrait. Cette créature quoique émanée de lui le domine, elle peut l’offenser et mériter par là des supplices éternels, quel être faible que ce dieu-là ! Comment, il a pu créer tout ce que nous voyons, et il lui est impossible de former un homme à sa guise ! Mais, me répondez-vous à cela, s’il l’eût créé tel, l’homme n’eût pas eu de mérite, quelle platitude ! et quelle nécessité y a-t-il à ce que l’homme mérite de son Dieu ? En le formant tout à fait bon il n’aurait jamais pu faire le mal, et de ce moment seul l’ouvrage était digne d’un Dieu, c’est tenter l’homme que de lui laisser un choix ; or Dieu par sa prescience infinie savait bien ce qu’il en résulterait ; de ce moment c’est donc à plaisir qu’il perd la créature que lui-même a formée, quel horrible dieu que ce dieu-là, quel monstre ! quel scélérat plus digne de notre haine et de notre implacable vengeance ? Cependant, peu content d’une aussi sublime besogne, il noie l’homme pour le convertir, il le brûle, il le maudit, rien de tout cela ne le change, un être plus puissant que ce vilain dieu, le {\itshape Diable}, conservant toujours son empire, pouvant toujours braver son auteur, parvient sans cesse par ses séductions à débaucher le troupeau que s’était réservé l’Éternel, rien ne peut vaincre l’énergie de ce démon sur nous ; qu’imagine alors, selon vous, l’horrible dieu que vous prêchez, il n’a qu’un fils, un fils unique qu’il possède de je ne sais quel commerce, car comme l’homme {\itshape fout}, il a voulu que son dieu {\itshape foutît} également ; il détache du ciel cette respectable portion de lui-même ; on s’imagine peut-être que c’est sur des rayons célestes, au milieu du cortège des anges, à la vue de l’univers entier que celle sublime créature va paraître… Pas un mot ; c’est dans le sein d’une putain juive ; c’est au milieu d’une étable à cochons que s’annonce le dieu qui vient sauver la terre ; voilà la digne extraction qu’on lui prête ; mais son honorable mission nous dédommagera-t-elle ? Suivons un instant le personnage, que dit-il ? que fait-il ? quelle sublime mission recevons-nous de lui ? quel mystère va-t-il révéler ? quel dogme va-t-il nous prescrire ? dans quels actes enfin sa grandeur va-t-elle éclater ? Je vois d’abord une enfance ignorée, quelques services, très libertins sans doute, rendus par ce polisson, aux prêtres du temple de Jérusalem ; ensuite une disparution\footnote{ Sic. {\itshape (Note du correcteur - ELG.)}} de quinze ans, pendant laquelle le fripon va s’empoisonner de toutes les rêveries de l’école égyptienne qu’il rapporte enfin en Judée ; à peine y reparaît-il que sa démence débute par lui faire dire qu’il est le fils de dieu, égal à son père, il associe à cette alliance un autre fantôme qu’il appelle l’esprit saint, et ces trois personnes assure-t-il, ne doivent en faire qu’une ; plus ce ridicule mystère étonne la raison, plus le faquin assure qu’il y a du mérite à l’adopter… de dangers à l’anéantir. C’est pour nous sauver tous, assure l’imbécile, qu’il a pris chair, quoique dieu, dans le sein d’un enfant des hommes ; et les miracles éclatants qu’on va lui voir opérer en convaincront bientôt l’univers ; dans un souper d’ivrognes, en effet, le fourbe change, à ce qu’on dit, l’eau en vin ; dans un désert il nourrit quelques scélérats avec des provisions cachées que ses sectateurs préparèrent. Un de ses camarades fait le mort, notre imposteur le ressuscite. Il se transporte sur une montagne, et là, seulement devant deux ou trois de ses amis, il fait un tour de passe-passe dont rougirait le plus mauvais bateleur de nos jours. Maudissant d’ailleurs avec enthousiasme tous ceux qui ne croient pas en lui, le coquin promet les cieux à tous les sots qui l’écouteront ; il n’écrit rien vu son ignorance, parle fort peu vu sa bêtise, fait encore moins vu sa faiblesse, et lassant à la fin les magistrats, impatientés de ses discours séditieux, quoique fort rares, le charlatan se fait mettre en croix après avoir assuré les gredins qui le suivent que, chaque fois qu’ils l’invoqueront, il descendra vers eux pour s’en faire manger ; on le supplicie, il se laisse faire ; monsieur son Papa, ce Dieu sublime, dont il ose dire qu’il descend, ne lui donne pas le moindre secours, et voilà le coquin traité comme le dernier des scélérats, dont il était si digne d’être le chef. Ses satellites s’assemblent ; « Nous voilà perdus, disent-ils, et toutes nos espérances évanouies, si nous ne nous sauvons par un coup d’éclat. Enivrons la garde qui entoure Jésus, dérobons son corps, publions qu’il est ressuscité, le moyen est sûr ; si nous parvenons à faire croire cette friponnerie, notre nouvelle religion s’étaie, se propage, elle séduit le monde entier… Travaillons » : le coup s’entreprend, il réussit ; à combien de fripons la hardiesse n’a-t-elle pas tenu lieu de mérite Le corps est enlevé, les sots, les femmes, les enfants crient, tant qu’ils le peuvent, au miracle, et cependant dans cette ville où de si grandes merveilles viennent de s’opérer, dans cette ville, teinte du sang d’un Dieu, personne ne veut croire à ce Dieu ; pas une conversion ne s’y opère, il y a mieux : le fait est si peu digne d’être transmis, qu’aucun historien n’en parle. Les seuls disciples de cet imposteur pensent à tirer parti de la fraude, mais non pas dans le moment, cette considération est encore bien essentielle ; ils laissent écouler plusieurs années avant de faire usage de leur fourberie ; ils érigent enfin sur elle l’édifice chancelant de leur dégoûtante doctrine ; tout changement plait aux hommes. Las du despotisme des empereurs, une révolution devenait nécessaire : on écoute ces fourbes, leur progrès devient très rapide, c’est l’histoire de toutes les erreurs. Bientôt les autels de Vénus et de Mars sont changés en ceux de Jésus et de Marie, on publie la vie de l’imposteur, ce plat roman trouve des dupes, on lui fait dire cent choses auxquelles il n’a jamais pensé ; quelques-uns de ses propos saugrenus deviennent aussitôt la base de sa morale, et comme cette nouveauté se prêchait à des pauvres, la charité en devient la première vertu, des rites bizarres s’instituent sous le nom de {\itshape sacrements}, dont le plus indigne et le plus abominable de tous est celui par lequel un prêtre, couvert de crimes, a néanmoins, par la vertu de quelques paroles magiques, le pouvoir de faire arriver Dieu dans un morceau de pain. N’en doutons pas, dès sa naissance même ce culte indigne eût été détruit sans ressource, si l’on n’eût employé contre lui que les armes du mépris qu’il méritait ; mais on s’avisa de le persécuter, il s’accrut, le moyen était inévitable. Qu’on essaie encore aujourd’hui de le couvrir de ridicule, il tombera : l’adroit Voltaire n’employait jamais d’autres armes, et c’est de tous les écrivains celui qui peut se flatter d’avoir fait le plus de prosélytes. En un mot, Eugénie, telle est l’histoire de Dieu et de la religion ; voyez le cas que ces fables méritent, et déterminez-vous sur leur compte.\par
EUGÉNIE : Mon choix n’est pas embarrassant, je méprise toutes ces rêveries dégoûtantes, et ce Dieu même auquel je tenais encore par faiblesse ou par ignorance, n’est plus pour moi qu’un objet d’horreur.\par
MME DE SAINT-ANGE : Jure-moi bien de n’y plus penser, de ne t’en occuper jamais, de ne l’invoquer en aucun instant de ta vie, et de n’y revenir de tes jours.\par
EUGÉNIE, {\itshape se précipitant sur le sein de M\textsuperscript{me} de Saint-Ange} : Ah ! j’en fais le serment dans tes bras, ne m’est-il pas facile de voir que ce que tu exiges est pour mon bien, et que tu ne veux pas que de pareilles réminiscences puissent jamais troubler ma tranquillité.\par
MME DE SAINT-ANGE : Pourrais-je avoir d’autre motif ?\par
EUGÉNIE : Mais, Dolmancé, c’est, ce me semble, l’analyse des vertus qui nous a conduits à l’examen des religions ; revenons-y. N’existerait-il pas dans cette religion, toute ridicule qu’elle est, quelques vertus prescrites par elle, et dont le culte pût contribuer à notre bonheur ?\par
DOLMANCÉ : Eh bien ! examinons. Sera-ce la chasteté, Eugénie, cette vertu que vos yeux détruisent, quoique votre ensemble en soit l’image ? Révérerez-vous l’obligation de combattre tous les mouvements de la nature, les sacrifierez-vous tous au vain et ridicule honneur de n’avoir jamais une faiblesse ? Soyez juste, et répondez, belle amie : croyez-vous trouver dans cette absurde et dangereuse pureté d’âme tous les plaisirs du vice contraire ?\par
EUGÉNIE : Non, d’honneur, je ne veux point de celle-là, je ne me sens pas le moindre penchant à être chaste, et la plus grande disposition au vice contraire ; mais, Dolmancé, la {\itshape charité}, la {\itshape bienfaisance}, ne pourraient-elles pas faire le bonheur de quelques âmes sensibles ?\par
DOLMANCÉ : Loin de nous, Eugénie, les vertus qui ne font que des ingrats ; mais ne t’y trompe point d’ailleurs, ma charmante amie ; la bienfaisance est bien plutôt un vice de l’orgueil, qu’une véritable vertu de l’âme ; c’est par ostentation qu’on soulage ses semblables, jamais dans la seule vue de faire une bonne action ; on serait bien fâché que l’aumône qu’on vient de faire n’eût pas toute la publicité possible ; ne t’imagine pas non plus, Eugénie, que cette action ait d’aussi bon effets qu’on se l’imagine ; je ne l’envisage, moi, que comme la plus grande de toutes les duperies ; elle accoutume le pauvre à des secours qui détériorent son énergie, il ne travaille plus quand il s’attend à vos charités, et devient, dès qu’elles lui manquent, un voleur ou un assassin. J’entends de toutes parts demander les moyens de supprimer la mendicité, et l’on fait pendant ce temps-là tout ce qu’on peut pour la multiplier. Voulez-vous ne pas avoir de mouches dans une chambre, n’y répandez pas de sucre pour les attirer. Voulez-vous ne pas avoir de pauvres en France, ne distribuez aucune aumône, et supprimez surtout vos maisons de charité : l’individu né dans l’infortune, se voyant alors privé de ces ressources dangereuses, emploiera tout le courage, tous les moyens qu’il aura reçus de la nature, pour se tirer de l’état où il est né, il ne vous importunera plus ; détruisez, renversez sans aucune pitié ces détestables maisons où vous avez l’effronterie de receler les fruits du libertinage de ce pauvre, cloaques épouvantables vomissant chaque jour dans la société un essaim dégoûtant de ces nouvelles créatures qui n’ont d’espoir que dans votre bourse ; à quoi sert-il, je le demande, que l’on conserve de tels individus avec tant de soin ? A-t-on peur que la France ne se dépeuple ? Ah ! n’ayons jamais cette crainte ! Un des premiers vices de ce gouvernement consiste dans une population beaucoup trop nombreuse, et il s’en faut bien que de tels superflus soient des richesses pour l’État. Ces êtres surnuméraires sont comme des branches parasites qui, ne vivant qu’aux dépens du tronc, finissent toujours par l’exténuer. Souvenez-vous que toutes les fois que, dans un gouvernement quelconque, la population sera supérieure aux moyens de l’existence, ce gouvernement languira ; examinez bien la France, vous verrez que c’est ce qu’elle vous offre, qu’en résulte-t-il ? On le voit. Le Chinois, plus sage que nous, se garde bien de se laisser dominer ainsi par une population trop abondante ; point d’asile pour les fruits honteux de sa débauche, on abandonne ces affreux résultats comme les suites d’une digestion. Point de maisons pour la pauvreté, on ne la connaît point en Chine. Là, tout le monde travaille, là, tout le monde est heureux, rien n’altère l’énergie du pauvre, et chacun y peut dire comme Néron : {\itshape Quid est pauper} ?\par
EUGÉNIE, {\itshape à M\textsuperscript{me} de Saint-Ange} : Chère amie, mon père pense absolument comme Monsieur, de ses jours il ne fit une bonne œuvre, il ne cesse de gronder ma mère des sommes qu’elle dépense à de telles pratiques : elle était de la Société {\itshape Maternelle}, de la Société {\itshape Philanthropique}, je ne sais de quelle association elle n’était point ; il l’a contrainte à quitter tout cela, en l’assurant qu’il la réduirait à la plus modique pension si elle s’avisait de retomber encore dans de pareilles sottises.\par
MME DE SAINT-ANGE : Il n’y a rien de plus ridicule, et en même temps de plus dangereux, Eugénie, que toutes ces associations ; c’est à elles, aux écoles gratuites et aux maisons de charité que nous devons le bouleversement horrible dans lequel nous voici maintenant. Ne fais jamais d’aumône, ma chère, je t’en supplie.\par
EUGÉNIE : Ne crains rien, il y a longtemps que mon père a exigé de moi la même chose, et la bienfaisance me tente trop peu pour enfreindre sur cela ses ordres… les mouvements de mon cœur, et tes désirs.\par
DOLMANCÉ : Ne divisons pas cette portion de sensibilité que nous avons reçue de la nature, c’est l’anéantir que de l’étendre ; que me font à moi les maux des autres, n’ai-je donc point assez des miens, sans aller m’affliger de ceux qui me sont étrangers, que le foyer de cette sensibilité n’allume jamais que nos plaisirs ; soyons sensibles à tout ce qui les flatte, absolument inflexibles sur tout le reste, il résulte de cet état de l’âme une sorte de cruauté qui n’est quelquefois pas sans délices, on ne peut pas toujours faire le mal, privés du plaisir qu’il donne, équivalons au moins cette sensation par la méchanceté piquante de ne jamais faire le bien.\par
EUGÉNIE : Ah ! Dieu, comme vos leçons m’enflamment, je crois qu’on me tuerait plutôt maintenant que de me faire faire une bonne action.\par
MME DE SAINT-ANGE : Et s’il s’en présentait une mauvaise, serais-tu de même prête à la commettre ?\par
EUGÉNIE : Tais-toi, séductrice, je ne répondrai sur cela que lorsque tu auras fini de m’instruire ; il me paraît que, d’après tout ce que vous me dites, Dolmancé, rien n’est aussi indifférent sur la terre que d’y commettre le bien ou le mal ; nos goûts, notre tempérament doivent seuls être respectés.\par
DOLMANCÉ : Ah ! n’en doutez pas, Eugénie, ces mots de {\itshape vice} et de {\itshape vertu} ne nous donnent que des idées purement locales, il n’y a aucune action, quelque singulière que vous puissiez la supposer, qui soit vraiment criminelle, aucune qui puisse réellement s’appeler vertueuse : tout est en raison de nos mœurs, et du climat que nous habitons, ce qui fait crime ici, est souvent vertu quelque cent lieues plus bas, et les vertus d’un autre hémisphère pourraient bien réversiblement être des crimes pour nous ; il n’y a pas d’horreur qui n’ait été divinisée, pas une vertu qui n’ait été flétrie. De ces différences, purement géographiques, naît le peu de cas que nous devons faire de l’estime ou du mépris des hommes, sentiments ridicules et frivoles au-dessus desquels nous devons nous mettre au point même de préférer sans crainte leur mépris pour peu que les actions qui nous le méritent soient de quelque volupté pour nous.\par
EUGÉNIE : Mais il me semble pourtant qu’il doit y avoir des actions assez dangereuses, assez mauvaises en elles-mêmes pour avoir été généralement considérées comme criminelles, et punies comme telles d’un bout de l’univers à l’autre.\par
MME DE SAINT-ANGE : Aucune, mon amour, aucune, pas même le viol, ni l’inceste, pas même le meurtre ni le parricide.\par
EUGÉNIE : Quoi ! ces horreurs ont pu s’excuser quelque part ?\par
DOLMANCÉ : Elles y ont été honorées, couronnées, considérées comme d’excellentes actions, tandis qu’en d’autres lieux, l’humanité, la candeur, la bienfaisance, la chasteté, toutes nos vertus, enfin, étaient regardées comme des monstruosités.\par
EUGÉNIE : Je vous prie de m’expliquer tout cela ; j’exige une courte analyse de chacun de ces crimes, en vous priant de commencer par m’expliquer d’abord votre opinion sur le libertinage des filles, ensuite sur l’adultère des femmes.\par
MME DE SAINT-ANGE : Écoute-moi donc, Eugénie, il est absurde de dire qu’aussitôt qu’une fille est hors du sein de sa mère, elle doive, de ce moment, devenir la victime de la volonté de ses parents, pour rester telle jusqu’à son dernier soupir. Ce n’est pas dans un siècle où l’étendue et les droits de l’homme viennent d’être approfondis avec tant de soins, que de jeunes filles doivent continuer à se croire l’esclave de leurs familles, quand il est constant que les pouvoirs de ces familles sur elle sont absolument chimériques ; écoutons la nature sur un objet aussi intéressant, et que les lois des animaux, bien plus rapprochées d’elle, nous servent un moment d’exemples ; les devoirs paternels s’étendent-ils chez eux au-delà des premiers besoins physiques, les fruits de la jouissance du mâle et de la femelle ne possèdent-ils pas toute leur liberté, tous leurs droits ? Sitôt qu’ils peuvent marcher et se nourrir seuls, dès cet instant les auteurs de leurs jours les connaissent-ils ? et eux croient-ils devoir quelque chose à ceux qui leur ont donné la vie ? Non, sans doute. De quel droit les enfants des hommes sont-ils donc astreints à d’autres devoirs ? et qui les fondent ces devoirs, si ce n’est l’avarice ou l’ambition des pères ? Or je demande s’il est juste qu’une jeune fille qui commence à sentir et à raisonner, se soumette à de tels freins ? N’est-ce donc pas le préjugé tout seul qui prolonge ces chaînes ? Et y a-t-il rien de plus ridicule que de voir une jeune fille de quinze ou seize ans, brûlée par des désirs qu’elle est obligée de vaincre, attendre dans des tourments, pires que ceux des Enfers, qu’il plaise à ses parents, après avoir rendu sa jeunesse malheureuse, de sacrifier encore son âge mûr, en l’immolant à leur perfide cupidité, en l’associant, malgré elle, à un époux, ou qui n’a rien pour se faire aimer, ou qui a tout pour se faire haïr ! Eh ! non, non, Eugénie, de tels liens s’anéantiront bientôt ; il faut que, dégageant dès l’âge de raison la jeune fille de la maison paternelle, après lui avoir donné une éducation nationale, on la laisse maîtresse, à quinze ans, de devenir ce qu’elle voudra : donnera-t-elle dans le vice ? Eh qu’importe ! Les services que rend une jeune fille, en consentant à faire le bonheur de tous ceux qui s’adressent à elle, ne sont-ils pas infiniment plus importants que ceux qu’en s’isolant elle offre à son époux ? La destinée de la femme est d’être comme la chienne, comme la louve ; elle doit appartenir à tous ceux qui veulent d’elle ; c’est visiblement outrager la destination que la nature impose aux femmes, que de les enchaîner par le lien absurde d’un hymen solitaire. Espérons qu’on ouvrira les yeux, et qu’en assurant la liberté de tous les individus, on n’oubliera pas le sort des malheureuses filles ; mais si elles sont assez à plaindre pour qu’on les oublie, que, se plaçant d’elles-mêmes au-dessus de l’usage et du préjugé, elles foulent hardiment aux pieds les fers honteux dont on prétend les asservir, elles triompheront bientôt alors de la coutume et de l’opinion ; l’homme devenu plus sage, parce qu’il sera plus libre, sentira l’injustice qu’il y aurait à mépriser celles qui agiront ainsi et que l’action de céder aux impulsions de la nature, regardée comme un crime chez un peuple captif, ne peut plus l’être chez un peuple libre. Pars donc de la légitimité de ces principes, Eugénie, et brise tes fers à quelque prix que ce puisse être ; méprise les vaines remontrances d’une mère imbécile à qui tu ne dois légitimement que de la haine et que du mépris ; si ton père qui est un libertin te désire, à la bonne heure, qu’il jouisse de toi, mais sans t’enchaîner ; brise le joug s’il veut t’asservir, plus d’une fille ont agi de même avec leur père. Fous, en un mot, fous, c’est pour cela que tu es mise au monde ; aucunes bornes à tes plaisirs, que celles de tes forces ou de tes volontés ; aucune exception de lieux, de temps et de personnes ; toutes les heures, tous les endroits, tous les hommes doivent servir à tes voluptés ; la continence est une vertu impossible dont la nature, violée dans ses droits, nous punit aussitôt par mille malheurs. Tant que les lois seront telles qu’elles sont encore aujourd’hui, usons de quelques voiles, l’opinion nous y contraint ; mais dédommageons-nous en silence de cette chasteté cruelle que nous sommes obligées d’avoir en public. Qu’une jeune fille travaille à se procurer une bonne amie qui, libre et dans le monde, puisse lui en faire secrètement goûter les plaisirs ; qu’elle tâche, au défaut de cela, de séduire les argus dont elle est entourée, qu’elle les supplie de la prostituer, et leur promettant tout l’argent qu’ils pourront retirer de sa vente, ou ces argus par eux-mêmes, ou des femmes qu’ils trouveront, et qu’on nomme {\itshape Maquerelles}, rempliront bientôt les vues de la jeune fille ; qu’elle jette alors de la poudre aux yeux de tout ce qui l’entoure, frères, cousins, amis, parents, qu’elle se livre à tous, si cela est nécessaire pour cacher sa conduite ; qu’elle fasse même, si cela est exigé, le sacrifice de ses goûts et de ses affections ; une intrigue qui lui aura déplu, et dans laquelle elle ne se sera livrée que par politique, la mènera bientôt dans une plus agréable, et la voilà {\itshape lancée}.\par
Mais qu’elle ne revienne plus sur les préjugés de son enfance, menaces, exhortations, devoirs, vertus, religion, conseils, qu’elle foule tout aux pieds, qu’elle rejette et méprise opiniâtrement tout ce qui ne tend qu’à la renchaîner, tout ce qui ne vise point, en un mot, à la livrer au sein de l’impudicité. C’est une extravagance de nos parents, que ces prédictions de malheurs dans la voie du libertinage ; il y a des épines partout, mais les roses se trouvent au-dessus d’elles dans la carrière du vice ; il n’y a que dans les sentiers bourbeux de la vertu que la nature n’en fait jamais naître. Le seul écueil à redouter dans la première de ces routes, c’est l’opinion des hommes ; mais quelle est la fille d’esprit qui, avec un peu de réflexion, ne se rendra pas supérieure à cette méprisable opinion ? Les plaisirs reçus par l’estime, Eugénie, ne sont que des plaisirs moraux, uniquement convenables à certaines têtes ; ceux de la {\itshape fouterie} plaisent à tous, et ces attraits séducteurs dédommagent bientôt de ce mépris illusoire auquel il est difficile d’échapper en bravant l’opinion publique, mais dont plusieurs femmes sensées se sont moquées au point de s’en composer un plaisir de plus. Fous, Eugénie, fous donc, mon cher ange, ton corps est à toi, à toi seule, il n’y a que toi seule au monde qui aies le droit d’en jouir et d’en faire jouir qui bon te semble ; profite du plus heureux temps de ta vie ; elles ne sont que trop courtes ces heureuses années de nos plaisirs ! Si nous sommes assez heureux pour en avoir joui, de délicieux souvenirs nous consolent, et nous amusent encore dans notre vieillesse ; les avons-nous perdus ?… Des regrets amers, d’affreux remords nous déchirent, et se joignent aux tourments de l’âge pour entourer de larmes et de ronces les funestes approches du cercueil… Aurais-tu la folie de l’immortalité ? Eh bien c’est en foutant, ma chère, que tu resteras dans la mémoire des hommes ; on a bientôt oublié les Lucrèce, tandis que les Théodore et les Messaline font les plus doux entretiens et les plus fréquents de la vie. Comment donc, Eugénie, ne pas préférer un parti qui, nous couronnant de fleurs ici-bas, nous laisse encore l’espoir d’un culte bien au-delà du tombeau ? Comment, dis-je, ne pas préférer ce parti à celui qui, nous faisant végéter imbécilement sur la terre, ne nous promet après notre existence que du mépris et de l’oubli ?\par
EUGÉNIE, {\itshape à M\textsuperscript{me} de Saint-Ange} : Ah ! cher amour, comme ces discours séducteurs enflamment ma tête et séduisent mon âme, je suis dans un état difficile à peindre… Et, dis-moi, pourras-tu me faire connaître quelques-unes de ces femmes… {\itshape (troublée)} qui me prostitueront, si je leur dis ?\par
MME DE SAINT-ANGE : D’ici à ce que tu aies plus d’expérience, cela ne regarde que moi seule, Eugénie, rapporte-t’en à moi de ce soin, et plus encore à toutes les précautions que je prendrai pour couvrir tes égarements ; mon frère, et cet ami solide qui t’instruit, seront les premiers auxquels je veux que tu te livres ; nous en trouverons d’autres après ; ne t’inquiète pas, chère amie, je te ferai voler de plaisirs en plaisirs, je te plongerai dans une mer de délices, je t’en comblerai, mon ange, je t’en rassasierai.\par
EUGÉNIE, {\itshape se précipitant dans les bras de M\textsuperscript{me} de Saint-Ange} : Oh ! ma bonne, je t’adore ; va, tu n’auras jamais une écolière plus soumise que moi ; mais il me semble que tu m’as fait entendre dans nos anciennes conversations, qu’il était difficile qu’une jeune épouse se jette dans le libertinage, sans que l’époux qu’elle doit prendre après ne s’en aperçoive ?\par
MME DE SAINT-ANGE : Cela est vrai, ma chère, mais il y a des secrets qui raccommodent toutes ces brèches. Je te promets de t’en donner connaissance, et alors eusses-tu foutu comme Antoinette, je me charge de te rendre aussi vierge que le jour que tu vins au monde.\par
EUGÉNIE : Ah ! tu es délicieuse, allons, continue de m’instruire, presse-toi donc en ce cas de m’apprendre quelle doit être la conduite d’une femme dans le mariage ?\par
MME DE SAINT-ANGE : Dans quelque état que se trouve une femme, ma chère, soit fille, soit femme, soit veuve, elle ne doit jamais avoir d’autre but, d’autre occupation, d’autre désir, que de se faire foutre du matin au soir. C’est pour cette unique fin que l’a créée la nature ; mais si, pour remplir cette intention, j’exige d’elle de fouler aux pieds tous les préjugés de son enfance, si je lui prescris la désobéissance la plus formelle aux ordres de sa famille, le mépris le plus constaté de tous les conseils de ses parents ; tu conviendras, Eugénie, que de tous les freins à rompre, celui dont je lui conseillerai le plus tôt l’anéantissement, sera bien sûrement celui du mariage. Considère, en effet, Eugénie, une jeune fille à peine sortie de la maison paternelle ou de sa pension, ne connaissant rien, n’ayant nulle expérience, obligée de passer subitement de là dans les bras d’un homme qu’elle n’a jamais vu, obligée de jurer à cet homme aux pieds des autels, une obéissance, une fidélité d’autant plus injuste, qu’elle n’a souvent au fond de son cœur que le plus grand désir de lui manquer de parole. Est-il au monde, Eugénie, un sort plus affreux que celui-là ? Cependant la voilà liée, que son mari lui plaise ou non, qu’il ait, ou non, pour elle de la tendresse ou des procédés, son honneur tient à ses serments, il est flétri si elle les enfreint ; il faut qu’elle se perde ou qu’elle traîne le joug, dût-elle en mourir de douleur. Eh ! non, Eugénie, non, ce n’est point pour cette fin que nous sommes nées, ces lois absurdes sont l’ouvrage des hommes, et nous ne devons pas nous y soumettre. Le divorce même est-il capable de nous satisfaire, non, sans doute ; qui nous répond de trouver plus sûrement dans de seconds liens le bonheur qui nous a fuies dans les premiers ; dédommageons-nous donc en secret de toute la contrainte de nœuds si absurdes, bien certaines que nos désordres en ce genre, à quelque excès que nous puissions les porter, loin d’outrager la nature, ne sont qu’un hommage sincère que nous lui rendons : c’est obéir à ses lois, que de céder aux désirs qu’elle seule a placés dans nous, ce n’est qu’en lui résistant que nous l’outragerions ; l’adultère que les hommes regardent comme un crime…, qu’ils ont osé punir comme tel en nous arrachant la vie, l’adultère, Eugénie, n’est donc que l’acquit d’un droit à la nature, auquel les fantaisies de ces tyrans ne sauraient jamais nous soustraire. Mais n’est-il pas horrible, disent nos époux, de nous exposer à chérir comme nos enfants, à embrasser, comme tels, les fruits de vos désordres ? C’est l’objection de Rousseau, c’est, j’en conviens, la seule un peu spécieuse dont on puisse combattre l’adultère ; eh ! n’est-il pas extrêmement aisé de se livrer au libertinage sans redouter la grossesse ? n’est-il pas encore plus facile de la détruire, si par imprudence elle a lieu ? Mais, comme nous reviendrons sur cet objet, ne traitons maintenant que le fond de la question, nous verrons que l’argument, tout spécieux qu’il paraît d’abord, n’est cependant que chimérique.\par
Premièrement, tant que je couche avec mon mari, tant que sa semence coule au fond de ma matrice, verrais-je dix hommes en même temps que lui, rien ne pourra jamais lui prouver que l’enfant qui naîtra ne lui appartienne pas ; il peut être à lui comme n’y pas être, et dans le cas de l’incertitude, il ne peut ni ne doit jamais (puisqu’il a coopéré à l’existence de cette créature) se faire aucun scrupule d’avouer cette existence. Dès qu’elle peut lui appartenir, elle lui appartient, et tout homme qui se rendra malheureux par des soupçons sur cet objet, le serait de même quand sa femme serait une vestale ; parce qu’il est impossible de répondre d’une femme, et que celle qui a été sage dix ans, peut cesser de l’être un jour : donc, si cet époux est soupçonneux, il le sera dans tous les cas, jamais alors il ne sera sûr que l’enfant qu’il embrasse est véritablement le sien. Or, s’il peut être soupçonneux dans tous les cas, il n’y a aucun inconvénient à légitimer quelquefois des soupçons ; il n’en serait, pour son état de bonheur ou de malheur moral, ni plus ni moins ; donc il vaut tout autant que cela soit ainsi ; le voilà donc, je le suppose, dans une complète erreur, le voilà caressant le fruit du libertinage de sa femme, où donc est le crime à cela ? Nos biens ne sont-ils pas communs ; en ce cas, quel mal fais-je en plaçant dans le ménage un enfant qui doit avoir une portion de ces biens ? Ce sera la mienne qu’il aura, il ne volera rien à mon tendre époux ; cette portion dont il va jouir, je la regarde comme prise sur ma dot ; donc ni cet enfant, ni moi, ne prenons rien à mon mari : à quel titre, si cet enfant eût été de lui, aurait-il eu part dans mes biens ? N’est-ce point en raison de ce qu’il serait émané de moi ? Eh bien ! il va jouir de cette part en vertu de cette même raison d’alliance intime. C’est parce que cet enfant m’appartient, que je lui dois une portion de mes richesses.\par
Quel reproche avez-vous à me faire ? il en jouit. Mais vous trompez votre mari, cette fausseté est atroce ; non, c’est un rendu, voilà tout ; je suis dupe la première des liens qu’il m’a forcée de prendre, je m’en venge, quoi de plus simple ! Mais il y a un outrage réel fait à l’honneur de votre mari : préjugé que cela, mon libertinage ne touche mon mari en rien, mes fautes sont personnelles, ce prétendu déshonneur était bon il y a un siècle, on est revenu de cette chimère aujourd’hui, et mon mari n’est pas plus flétri de mes débauches, que je ne saurais l’être des siennes ; je foutrais avec toute la terre sans lui faire une égratignure ; cette prétendue lésion n’est donc qu’une fable dont l’existence est impossible : de deux choses l’une ; ou mon mari est un brutal, un jaloux, ou c’est un homme délicat ; dans la première hypothèse, ce que je puis faire de mieux est de me venger de sa conduite ; dans la seconde, je ne saurais l’affliger ; puisque je goûte des plaisirs, il en sera heureux s’il est honnête ; il n’y a point d’homme délicat qui ne jouisse au spectacle du bonheur de la personne qu’il adore. Mais si vous l’aimiez, voudriez-vous qu’il en fît autant ? Ah ! malheur à la femme qui s’avisera d’être jalouse de son mari, qu’elle se contente de ce qu’il lui donne si elle l’aime ; mais qu’elle n’essaie pas de le contraindre, non seulement elle n’y réussirait pas, mais elle s’en ferait bientôt détester. Si je suis raisonnable, je ne m’affligerai donc jamais des débauches de mon mari, qu’il en fasse de même avec moi, et la paix régnera dans le ménage.\par
{\itshape Résumons} : Quels que soient les effets de l’adultère, dût-il même introduire dans la maison des enfants qui n’appartinssent pas à l’époux ; dès qu’ils sont à la femme ils ont des droits certains à une partie de la dot de cette femme ; l’époux, s’il est instruit, doit les regarder comme des enfants que sa femme aurait eus d’un premier mariage ; s’il ne sait rien, il ne saurait être malheureux, car on ne saurait l’être d’un mal qu’on ignore ; si l’adultère n’a point de suite, et qu’il soit inconnu du mari, aucun jurisconsulte ne saurait prouver, en ce cas, qu’il pourrait être un crime ; l’adultère n’est plus de ce moment qu’une action parfaitement indifférente pour le mari qui ne le sait pas, parfaitement bonne pour la femme qu’elle délecte ; si le mari découvre l’adultère, ce n’est plus l’adultère qui est un mal alors, car il ne l’était pas tout à l’heure, et il ne saurait avoir changé de nature ; il n’y a plus d’autre mal que la découverte qu’en a faite le mari ; or, ce tort-là n’appartient qu’à lui seul, il ne saurait regarder la femme ; ceux qui jadis ont puni l’adultère étaient donc des bourreaux, des tyrans, des jaloux qui, rapportant tout à eux, s’imaginaient injustement qu’il suffisait de les offenser pour être criminelle, comme si une injure personnelle devait jamais se considérer comme un crime, et comme si l’on pouvait justement appeler crime une action qui, loin d’outrager la nature et la société, sert évidemment l’une et l’autre ; il est cependant des cas où l’adultère, facile à prouver, devient plus embarrassant pour la femme, sans être pour cela plus criminel ; c’est, par exemple, celui où l’époux se trouve, ou dans l’impuissance, ou sujet à des goûts contraires à la population. Comme elle jouit, et que son mari ne jouit jamais, sans doute alors ses déportements deviennent plus ostensibles, mais doit-elle se gêner pour cela ? non sans doute. La seule précaution qu’elle doive employer est de ne point faire d’enfants, ou de se faire avorter si ces précautions viennent à le tromper ; si c’est par raison de goûts antiphysiques qu’elle est contrainte à se dédommager des négligences de son mari, il faut d’abord qu’elle le satisfasse sans répugnance dans ses goûts de quelque nature qu’ils puissent être, qu’ensuite elle lui fasse entendre que de pareilles complaisances méritent bien quelques égards, qu’elle demande une liberté entière en raison de ce qu’elle accorde ; alors le mari refuse ou consent ; s’il consent, comme a fait le mien, on s’en donne à l’aise en redoublant de soins et de condescendances à ses caprices ; s’il refuse on épaissit les voiles et l’on fout tranquillement à leur ombre. Est-il impuissant, on se sépare, mais dans tous les cas on s’en donne, on fout dans tous les cas, cher amour, parce que nous sommes nées pour foutre, que nous accomplissons les lois de la nature en foutant, et que toute loi humaine qui contrarierait celles de la nature ne serait faite que pour le mépris ; elle est bien dupe la femme que des nœuds aussi absurdes que ceux de l’hymen empêchent de se livrer à ses penchants, qui craint ou la grossesse, ou les outrages à son époux, ou les taches plus vaines encore à sa réputation. Tu viens de le voir, Eugénie, oui, tu viens de sentir comme elle est dupe… comme elle immole bassement aux plus ridicules préjugés, et son bonheur, et tous les délices de la vie. Ah ! qu’elle foute, qu’elle foute impunément, un peu de fausse gloire, quelques frivoles espérances religieuses la dédommageront-elles de ses sacrifices ? Non, non, et la vertu, le vice, tout se confond dans le cercueil ; le public, au bout de quelques années, exalte-t-il plus les uns qu’il ne condamne les autres ? Eh ! non, encore une fois, non, non, et la malheureuse ayant vécu sans plaisir, expire hélas sans dédommagement.\par
EUGÉNIE : Comme tu me persuades, mon ange, comme tu triomphes de mes préjugés ! comme tu détruis tous les faux principes que ma mère avait mis en moi ! Ah ! je voudrais être mariée demain pour mettre aussitôt tes maximes en usage. Qu’elles sont séduisantes ! qu’elles sont vraies, et combien je les aime ! Une chose seulement m’inquiète, chère amie, dans ce que tu viens de me dire, et comme je ne l’entends point, je te supplie de me l’expliquer. Ton mari, prétends-tu, ne s’y prend pas, dans la jouissance, de manière à avoir des enfants, que te fait-il donc je t’en prie ?\par
MME DE SAINT-ANGE : Mon mari était déjà vieux quand je l’épousai ; dès la première nuit de ses noces, il me prévint de ses fantaisies, en m’assurant que de son côté, jamais il ne gênerait les miennes ; je lui jurai de lui obéir, et nous avons toujours, depuis ce temps-là, vécu tous deux dans la plus délicieuse liberté ; le goût de mon mari consiste à se faire sucer, et voici le très singulier épisode qu’il y joint ; pendant que, courbée sur lui mes fesses à plomb sur son visage, je pompe avec ardeur le foutre de ses couilles, il faut que je lui chie dans la bouche… Il avale.\par
EUGÉNIE : Voilà une fantaisie bien extraordinaire.\par
DOLMANCÉ : Aucune ne peut se qualifier ainsi, ma chère, toutes sont dans la nature, elle s’est plu, en créant les hommes, à différencier leurs goûts comme leurs figures, et nous ne devons pas plus nous étonner de la diversité qu’elle a mise dans nos traits, que [de] celle qu’elle a placée dans nos affections. La fantaisie dont vient de vous parler votre amie est on ne saurait plus à la mode ; une infinité d’hommes, et principalement ceux d’un certain âge, y sont prodigieusement adonnés ; vous y refuseriez-vous, Eugénie, si quelqu’un l’exigeait de vous ?\par
EUGÉNIE, {\itshape rougissant} : D’après les maximes qui me sont inculquées ici, puis-je donc refuser quelque chose ? je ne demande grâce que pour ma surprise, c’est la première fois que j’entends toutes ces lubricités, il faut d’abord que je les conçoive ; mais de la solution du problème à l’exécution du procédé, je crois que mes instituteurs doivent être sûrs qu’il n’y aurait jamais que la distance qu’ils exigeront eux-mêmes. Quoi qu’il en soit, ma chère, tu gagnas donc ta liberté par l’acquiescement à cette complaisance ?\par
MME DE SAINT-ANGE : La plus entière, Eugénie ; je fis de mon côté tout ce que je voulus, sans qu’il y mît d’obstacles, mais je ne pris point d’amant ; j’aimais trop le plaisir pour cela, malheur à la femme qui s’attache, il ne faut qu’un amant pour la perdre, tandis que dix scènes de libertinage, répétées chaque jour, si elle le veut, s’évanouiront dans la nuit du silence aussitôt qu’elles seront consommées. J’étais riche, je payais des jeunes gens qui me foutaient sans me connaître ; je m’entourais de valets charmants, sûrs de goûter les plus doux plaisirs avec moi s’ils étaient discrets, certains d’être renvoyés s’ils disaient un mot. Tu n’as pas d’idée, cher ange, du torrent de délices dans lequel je me suis plongée de cette manière. Voilà la conduite que je prescrirai toujours à toutes les femmes qui voudront m’imiter ; depuis douze ans que je suis mariée, j’ai peut-être été foutue par plus de dix ou douze mille individus… et on me croit sage dans mes sociétés ; une autre aurait eu des amants, elle se serait perdue au second.\par
EUGÉNIE : Cette maxime est la plus sûre, ce sera bien décidément la mienne ; il faut que j’épouse, comme toi, un homme riche, et surtout un homme à fantaisies… mais, ma chère, ton mari, strictement lié à ses goûts, n’exigea jamais autre chose de toi ?\par
MME DE SAINT-ANGE : Jamais, depuis douze ans, il ne s’est pas démenti un seul jour, excepté lorsque j’ai mes règles. Une très jolie fille, qu’il a voulu que je prenne avec moi me remplace alors, et les choses vont le mieux du monde.\par
EUGÉNIE : Mais il ne s’en tient pas là, sans doute, d’autres objets concourent extérieurement à diversifier ses plaisirs ?\par
DOLMANCÉ : N’en doutez pas, Eugénie, le mari de madame est un des plus grands libertins de son siècle ; il dépense plus de cent mille écus par an aux goûts obscènes que votre amie vient de vous peindre tout à l’heure.\par
MME DE SAINT-ANGE : À vous dire le vrai je m’en doute, mais que me font ses déportements, puisque leur multiplicité autorise et voile les miens.\par
EUGÉNIE : Suivons, je t’en conjure, le détail des manières par lesquelles une jeune personne, mariée ou non, peut se préserver de la grossesse, car je t’avoue que cette crainte m’effarouche beaucoup, soit avec l’époux que je dois prendre, soit dans la carrière du libertinage ; tu viens de m’en indiquer une en me parlant des goûts de ton époux ; mais cette manière de jouir, qui peut être fort agréable pour l’homme, ne me semble pas l’être autant pour la femme, et ce sont nos jouissances, exemptes des risques que j’y crains, dont je désire que tu m’entretiennes.\par
MME DE SAINT-ANGE : Une fille ne s’expose jamais à faire d’enfants qu’autant qu’elle se le laisse mettre dans le con, qu’elle évite avec soin cette manière de jouir ; qu’elle offre à la place indistinctement sa main, sa bouche, ses tétons ou le trou de son cul ; par cette dernière voie elle prendra tout autant de plaisir, et même beaucoup davantage qu’ailleurs ; par les autres manières elle en donnera ; on procède à la première de ces façons, je veux dire celle de la main, ainsi que tu l’as vu tout à l’heure, Eugénie ; on secoue comme si l’on pompait le membre de son ami, au bout de quelques mouvements le sperme s’élance, l’homme vous baise, vous caresse pendant ce temps-là, et couvre de cette liqueur la partie de votre corps qui lui plaît le mieux. Veut-on le faire mettre entre les seins, on s’étend sur le lit, on place le membre viril au milieu des deux mamelles, on l’y presse, et au bout de quelques secousses l’homme décharge de manière à vous inonder les tétons et quelquefois le visage. Cette manière est la moins voluptueuse de toutes, et ne peut convenir d’ailleurs qu’à des femmes dont la gorge, à force de service, a déjà acquis assez de flexibilité pour serrer le membre de l’homme en se comprimant sur lui. La jouissance de la bouche est infiniment plus agréable tant pour l’homme que pour la femme ; la meilleure façon de la goûter est que la femme s’étende à contresens sur le corps de son fouteur, il vous met le vit dans la bouche, et, sa tête se trouvant entre vos cuisses, il vous rend ce que vous lui faites en vous introduisant sa langue dans le con ou sur le clitoris ; il faut, lorsqu’on emploie cette attitude, se prendre, s’empoigner les fesses, et se chatouiller réciproquement le trou du cul, épisode toujours nécessaire au complément de la volupté. Des amants chauds et pleins d’imagination avalent alors le foutre qui s’exhale dans leur bouche, et jouissent délicatement ainsi du plaisir voluptueux de faire mutuellement passer dans leurs entrailles cette précieuse liqueur méchamment dérobée à sa destination d’usage.\par
DOLMANCÉ : Cette façon est délicieuse, Eugénie, je vous en recommande l’exécution. Faire perdre ainsi les droits de la propagation, et contrarier de cette manière ce que les sots appellent les lois de la nature, est vraiment plein d’appas, les cuisses, les aisselles servent quelquefois aussi d’asiles au membre de l’homme, et lui offrent des réduits où sa semence peut se perdre sans risque de grossesse.\par
MME DE SAINT-ANGE : Quelques femmes s’introduisent des éponges dans l’intérieur du vagin qui, recevant le sperme, l’empêchent de s’élancer dans le vase qui le propagerait, d’autres obligent leurs fouteurs de se servir d’un petit sac de peau de Venise, vulgairement nommé condom, dans lequel leur semence coule sans risquer d’atteindre le but ; mais de toutes ces manières, celle du cul est la plus délicieuse, sans doute. Dolmancé, je vous en laisse la dissertation, qui doit mieux peindre que vous un goût pour lequel vous donneriez vos jours, si on les exigeait pour sa défense ?\par
DOLMANCÉ : J’avoue mon faible, il n’est, j’en conviens, aucune jouissance au monde qui soit préférable à celle-là, je l’adore dans l’un et l’autre sexe ; mais le cul d’un jeune garçon, il en faut convenir, me donne encore plus de volupté que celui d’une fille. On appelle {\itshape Bougres} ceux qui se livrent à cette passion ; or, quand on fait tant que d’être bougre, Eugénie, il faut l’être tout à fait. Foutre des femmes en cul, n’est l’être qu’à moitié ; c’est dans l’homme que la nature veut que l’homme serve cette fantaisie, et c’est spécialement pour l’homme qu’elle nous en a donné ce goût. Il est absurde de dire que cette manie l’outrage, cela se peut-il, dès qu’elle nous l’inspire ? Peut-elle dicter ce qui la dégrade ? Non, Eugénie, non, on la sert aussi bien là qu’ailleurs, et peut-être plus saintement encore ; la propagation n’est qu’une tolérance de sa part. Comment pourrait-elle avoir prescrit pour loi un acte qui la prive des droits de sa toute-puissance ? puisque la propagation n’est qu’une suite de ses premières intentions, et que de nouvelles constructions refaites par sa main, si notre espèce était absolument détruite, redeviendraient des intentions primordiales, dont l’acte serait bien plus flatteur pour son orgueil et sa puissance.\par
MME DE SAINT-ANGE : Savez-vous, Dolmancé, qu’au moyen de ce système vous allez jusqu’à prouver que l’extinction totale de la race humaine ne serait qu’un service rendu à la nature ?\par
DOLMANCÉ : Qui en doute, madame ?\par
MME DE SAINT-ANGE : Oh, juste ciel ! les guerres, les pestes, les famines, les meurtres, ne seraient plus que des accidents nécessaires des lois de la nature, et l’homme agent ou patient de ces effets ne serait donc pas plus criminel dans l’un des cas, qu’il ne serait victime dans l’autre ?\par
DOLMANCÉ : Victime, il l’est, sans doute, quand il fléchit sous les coups du malheur ; mais criminel, jamais. Nous reviendrons sur toutes ces choses, analysons, en attendant, pour la belle Eugénie, la jouissance sodomite qui fait maintenant l’objet de notre entretien. La posture la plus en usage pour la femme dans cette jouissance, est de se coucher à plat ventre sur le bord du lit, les fesses bien écartées, la tête la plus basse possible, le paillard, après s’être un instant amusé de la perspective du beau cul que l’on présente, après l’avoir claqué, manié, quelquefois même fouetté, pincé, mordu, humecte de sa bouche le trou mignon qu’il va perforer, et prépare l’introduction avec le bout de sa langue, il mouille de même son engin avec de la salive ou de la pommade, et le présente doucement au trou qu’il veut percer, il le conduit d’une main, de l’autre il écarte les fesses de sa jouissance ; dès qu’il sent son membre pénétrer, il faut qu’il pousse avec ardeur, en prenant bien garde de perdre du terrain ; quelquefois la femme souffre alors, si elle est neuve et jeune ; mais sans aucun égard pour des douleurs qui vont bientôt se changer en plaisirs, le fouteur doit pousser vivement son vit par gradations, jusqu’à ce qu’il ait enfin atteint le but, c’est-à-dire jusqu’à ce que le poil de son engin frotte exactement les bords de l’anus de l’objet qu’il encule. Qu’il poursuive alors sa route avec rapidité, toutes les épines sont cueillies ; il ne reste plus que des roses. Pour achever de métamorphoser en plaisirs les restes de douleur que son objet éprouve encore, si c’est un jeune garçon, qu’il lui saisisse le vit et le branle ; qu’il chatouille le clitoris, si c’est une fille, les titillations du plaisir qu’il fera naître, en rétrécissant prodigieusement l’anus du patient, doubleront les plaisirs de l’agent, qui, comblé d’aise et de volupté, dardera bientôt au fond du cul de sa jouissance, un sperme aussi abondant qu’épais, qu’auront déterminé tant de lubriques détails. Il en est d’autres qui ne veulent pas que le patient jouisse, c’est ce que nous expliquerons bientôt.\par
MME DE SAINT-ANGE : Permettez qu’un moment je sois écolière à mon tour, et que je vous demande, Dolmancé, dans quel état il faut, pour le complément des plaisirs de l’agent, que se trouve le cul du patient ?\par
DOLMANCÉ : Plein, très assurément ; il est essentiel que l’objet qui sert, ait alors la plus complète envie de chier, afin que le bout du vit du fouteur, atteignant l’étron, s’y enfonce et y dépose plus chaudement et plus mollement le foutre qui l’irrite et qui le met en feu.\par
MME DE SAINT-ANGE : Je craindrais que le patient y prît moins de plaisir.\par
DOLMANCÉ : Erreur ! Cette jouissance est telle qu’il est impossible que rien lui nuise, et que l’objet qui la sert ne soit transporté au troisième ciel en la goûtant : aucune ne vaut celle-là, aucune ne peut aussi complètement satisfaire l’un et l’autre des individus qui s’y livrent ; il est difficile que ceux qui l’ont goûtée puissent revenir à autre chose : telles sont, Eugénie, les meilleures façons de goûter le plaisir avec un homme, sans courir les risques de la grossesse ; car on jouit, soyez-en bien sûre, non seulement à prêter le cul à un homme, ainsi que je viens de vous l’expliquer, mais aussi à le sucer, à le branler, etc., etc., etc. et j’ai connu des femmes libertines qui mettaient souvent plus de charmes à ces épisodes qu’aux jouissances réelles, l’imagination est l’aiguillon des plaisirs ; dans ceux de cette espèce elle règle tout, elle est le mobile de tout ; or n’est-ce pas par elle que l’on jouit, n’est-ce pas d’elle que viennent les voluptés les plus piquantes ?\par
MME DE SAINT-ANGE : Soit : mais qu’Eugénie y prenne garde, l’imagination ne nous sert que quand notre esprit est absolument dégagé de préjugés ; un seul suffit à la refroidir ; cette capricieuse portion de notre esprit est d’un libertinage que rien ne peut contenir ; son plus grand triomphe, ses délices les plus éminents consistent à briser tous les freins qu’on lui oppose, elle est ennemie de la règle, idolâtre du désordre et de tout ce qui porte les couleurs du crime ; voilà d’où vient la singulière réponse d’une femme à imagination, qui foutait froidement avec son mari. « Pourquoi tant de glace, lui disait celui-ci ? – Eh vraiment, lui répondit cette singulière créature, {\itshape c’est que ce que vous me faites est tout simple}. »\par
EUGÉNIE : J’aime à la folie cette réponse… Ah ! ma chère, quelles dispositions je me sens à connaître ces élans divins d’une imagination déréglée ! Tu n’imaginerais pas, depuis que nous sommes ensemble,… seulement depuis cet instant, non, non, ma chère bonne, tu ne concevrais pas toutes les idées voluptueuses que mon esprit a caressées… Oh ! comme le mal est maintenant compris par moi ! combien il est désiré de mon cœur !\par
MME DE SAINT-ANGE : Que les atrocités, les horreurs, que les crimes les plus odieux ne t’étonnent pas davantage, Eugénie, ce qu’il y a de plus sale, de plus infâme et de plus défendu est ce qui irrite le mieux la tête ;… c’est toujours ce qui nous fait le plus délicieusement décharger.\par
EUGÉNIE : À combien d’écarts incroyables vous avez dû vous livrer l’un et l’autre ! que j’en voudrais connaître les détails.\par
DOLMANCÉ, {\itshape baisant et maniant la jeune personne} : Belle Eugénie, j’aimerais cent fois mieux vous voir éprouver tout ce que je voudrais faire, que de vous raconter ce que j’ai fait.\par
EUGÉNIE : Je ne sais s’il ferait trop bon pour moi de me prêter à tout.\par
MME DE SAINT-ANGE : Je ne te le conseillerais pas, Eugénie.\par
EUGÉNIE : Eh bien ! je fais grâce à Dolmancé de ses détails, mais toi, ma bonne amie, dis-moi, je t’en conjure, ce que tu as fait de plus extraordinaire en ta vie.\par
MME DE SAINT-ANGE : J’ai fait la chouette à quinze hommes ; je fus foutue quatre-vingt-dix fois en vingt-quatre heures, tant par-devant que par-derrière.\par
EUGÉNIE : Ce ne sont que des débauches cela, des tours de force ; je gage que tu as fait des choses plus singulières ?\par
MME DE SAINT-ANGE : J’ai été au bordel.\par
EUGÉNIE : Que veut dire ce mot ?\par
DOLMANCÉ : On appelle ainsi des maisons publiques où, moyennant un prix convenu, chaque homme trouve de jeunes et jolies filles toutes prêtes à satisfaire ses passions.\par
EUGÉNIE : Et tu t’es livrée là, ma bonne ?\par
MME DE SAINT-ANGE : Oui, j’y ai été comme une putain, j’y ai satisfait pendant une semaine entière les fantaisies de plusieurs paillards, et j’ai vu là des goûts bien singuliers ; par un égal principe de libertinage, comme la célèbre impératrice Théodora, femme de Justinien\footnote{ Voyez les {\itshape Anecdotes} de Procope.}, j’ai raccroché au coin des rues… dans les promenades publiques, et j’ai mis à la loterie l’argent venu de ces prostitutions.\par
EUGÉNIE : Ma bonne, je connais ta tête, tu as été beaucoup plus loin encore.\par
MME DE SAINT-ANGE : Cela se peut-il ?\par
EUGÉNIE : Oh ! oui, oui, et voici comme je le conçois, ne m’as-tu pas dit que nos sensations morales les plus délicieuses nous venaient de l’imagination ?\par
MME DE SAINT-ANGE : Je l’ai dit.\par
EUGÉNIE : Eh bien ! en laissant errer cette imagination, en lui donnant la liberté de franchir les dernières bornes que voudraient lui prescrire la religion, la décence, l’humanité, la vertu, tous nos prétendus devoirs ; enfin, n’est-il pas vrai que ces écarts seraient prodigieux ?\par
MME DE SAINT-ANGE : Sans doute.\par
EUGÉNIE : Or, n’est-ce pas en raison de l’immensité de ses écarts qu’elle nous irritera davantage ?\par
MME DE SAINT-ANGE : Rien de plus vrai.\par
EUGÉNIE : Si cela est, plus nous voudrons être agitées, plus nous désirerons nous émouvoir avec violence, plus il faudra donner carrière à notre imagination sur les choses les plus inconcevables ; notre jouissance alors s’améliorera en raison du chemin qu’aura fait la tête, et…\par
DOLMANCÉ, {\itshape baisant Eugénie} : Délicieuse.\par
MME DE SAINT-ANGE : Que de progrès la friponne a faits en peu de temps ! mais sais-tu, ma charmante, qu’on peut aller loin par la carrière que tu nous traces ?\par
EUGÉNIE : Je l’entends bien de cette manière, et puisque je ne me prescris aucun frein, tu vois où je suppose que l’on peut aller.\par
MME DE SAINT-ANGE : Aux crimes, scélérate, aux crimes les plus noirs et les plus affreux.\par
EUGÉNIE, {\itshape d’une voix basse…, et entrecoupée} : Mais tu dis qu’il n’en existe pas… et puis ce n’est que pour embraser sa tête : on n’exécute point.\par
DOLMANCÉ : Il est pourtant si doux d’exécuter ce qu’on a conçu.\par
EUGÉNIE, {\itshape rougissant} : Eh bien ! on exécute… Ne voudriez-vous pas me persuader, mes chers instituteurs, que vous n’avez jamais fait ce que vous avez conçu ?\par
MME DE SAINT-ANGE : Il m’est quelquefois arrivé de le faire.\par
EUGÉNIE : Nous y voilà.\par
DOLMANCÉ : Quelle tête !\par
EUGÉNIE, {\itshape poursuivant} : Ce que je te demande, c’est ce que tu as conçu, et ce que tu as fait après avoir conçu ?\par
MME DE SAINT-ANGE, {\itshape balbutiant} : Eugénie, je te raconterai ma vie quelque jour ; poursuivons notre instruction…, car tu me ferais dire des choses…\par
EUGÉNIE : Allons, je vois que tu ne m’aimes pas assez pour m’ouvrir à ce point ton âme, j’attendrai le délai que tu me prescris ; reprenons nos détails : dis-moi, ma chère, quel est l’heureux mortel que tu rendis le maître de tes prémices ?\par
MME DE SAINT-ANGE : Mon frère : il m’adorait depuis l’enfance, dès nos plus jeunes ans, nous nous étions souvent amusés sans atteindre le but, je lui avais promis de me livrer à lui dès que je serais mariée ; je lui tins parole ; heureusement que mon mari n’avait rien endommagé, il cueillit tout. Nous continuons de nous livrer à cette intrigue, mais sans nous gêner ni l’un ni l’autre, nous ne nous en plongeons pas moins tous les deux, chacun de notre côté, dans les plus divins excès du libertinage, nous nous servons même mutuellement, je lui procure des femmes, il me fait connaître des hommes.\par
EUGÉNIE : Le délicieux arrangement ; mais l’inceste n’est-il pas un crime ?\par
DOLMANCÉ : Pourrait-on regarder comme tel les plus douces unions de la nature ? celles qu’elle nous prescrit, et nous conseille le mieux ? Raisonnez un moment, Eugénie, comment l’espèce humaine, après les grands malheurs qu’éprouva notre globe, put-elle autrement se reproduire que par l’inceste ? n’en trouvons-nous pas l’exemple et la preuve, même dans les livres respectés par le christianisme, les familles d’Adam\footnote{ Adam ne fut comme Noé qu’un restaurateur du genre humain. Un affreux bouleversement laissa Adam seul sur la terre ; comme un pareil événement y laissa Noé ; mais la tradition d’Adam se perdit, celle de Noé se conserva.} et de Noé purent-elles autrement se perpétuer que par ce moyen ? Fouillez, compulsez les mœurs de l’univers, partout vous y verrez l’inceste autorisé, regardé comme une loi sage et faite pour cimenter les liens de la famille. Si l’amour, en un mot, naît de la ressemblance, où peut-elle être plus parfaite qu’entre frère et sœur, qu’entre père et fille ? Une politique mal entendue, produite par la crainte de rendre certaines familles trop puissantes, interdisit\footnote{ Sic. {\itshape (Note du correcteur - ELG.)}} l’inceste dans nos mœurs ; mais ne nous abusons pas au point de prendre pour loi de la nature ce qui n’est dicté que par l’intérêt ou par l’ambition ; sondons nos cœurs, c’est toujours là où je renvoie nos pédants moralistes ; interrogeons cet organe sacré, et nous reconnaîtrons qu’il n’est rien de plus délicat que l’union charnelle des familles ; cessons de nous aveugler sur les sentiments d’un frère pour sa sœur, d’un père pour sa fille. En vain l’un et l’autre les déguisent-ils sous le voile d’une légitime tendresse, le plus violent amour est l’unique sentiment qui les enflamme, tel est le seul que la nature ait mis dans leurs cœurs. Doublons, triplons donc sans rien craindre ces délicieux incestes, et croyons que plus l’objet de nos désirs nous appartiendra de près, plus nous aurons de charmes à en jouir. Un de mes amis vit habituellement avec la fille qu’il a eue de sa propre mère, il n’y a pas huit jours qu’il dépucela un garçon de treize ans, fruit de son commerce avec cette fille ; dans quelques années ce même jeune homme épousera sa mère, ce sont les vœux de mon ami, il leur fait un sort analogue à ces projets, et ses intentions, je le sais, sont de jouir encore des fruits qui naîtront de cet hymen ; il est jeune et peut l’espérer. Voyez, tendre Eugénie, de quelle quantité d’incestes et de crimes se serait souillé cet honnête ami, s’il y avait quelque chose de vrai dans le préjugé qui nous fait admettre du mal à ces liaisons. En un mot, sur toutes ces choses, je pars, moi, toujours d’un principe ; si la nature défendait les jouissances sodomites, les jouissances incestueuses, les pollutions, etc., permettrait-elle que nous y trouvassions autant de plaisirs ? Il est impossible qu’elle puisse tolérer qui l’outrage véritablement.\par
EUGÉNIE : Oh mes divins instituteurs, je vois bien que, d’après vos principes, il est très peu de crimes sur la terre, et que nous pouvons nous livrer en paix à tous nos désirs, quelque singuliers qu’ils puissent paraître aux sots qui s’offensant et s’alarmant de tout prennent imbécilement les institutions sociales pour les divines lois de la nature ; mais cependant, mes amis, n’admettez-vous pas au moins qu’il existe de certaines actions absolument révoltantes, et décidément criminelles, quoique dictées par la nature ? Je veux bien convenir avec vous que cette nature, aussi singulière dans les productions qu’elle crée, que variée dans les penchants qu’elle nous donne, nous porte quelquefois à des actions cruelles ; mais si, livrés à cette dépravation, nous cédions aux inspirations de cette bizarre nature, au point d’attenter, je le suppose, à la vie de nos semblables, vous m’accorderez bien, au moins je l’espère, que cette action serait un crime.\par
DOLMANCÉ : Il s’en faut bien, Eugénie, que nous puissions nous accorder une telle chose. La destruction étant une des premières lois de la nature, rien de ce qui détruit ne saurait être un crime. Comment une action qui sert aussi bien la nature pourrait-elle jamais l’outrager ? Cette destruction, dont l’homme se flatte, n’est d’ailleurs qu’une chimère ; le meurtre n’est point une destruction, celui qui le commet ne fait que varier les formes, il rend à la nature des éléments dont la main de cette nature habile se sert aussitôt pour récompenser d’autres êtres ; or, comme les créations ne peuvent être que des jouissances pour celui qui s’y livre, le meurtrier en prépare donc une à la nature, il lui fournit des matériaux qu’elle emploie sur-le-champ, et l’action que des sots ont eu la folie de blâmer, ne devient plus qu’un mérite aux yeux de cette agence universelle. C’est notre orgueil qui s’avise d’ériger le meurtre en crime, nous estimant les premières créatures de l’univers nous avons sottement imaginé que toute lésion qu’endurerait cette sublime créature devrait nécessairement être un crime énorme ; nous avons cru que la nature périrait si notre merveilleuse espèce venait à s’anéantir sur ce globe, tandis que l’entière destruction de cette espèce, en rendant à la nature la faculté créatrice qu’elle nous cède, lui redonnerait une énergie que nous lui enlevons en propageant ; mais quelle inconséquence, Eugénie ! Eh quoi ! un souverain ambitieux pourra détruire à son aise et sans le moindre scrupule les ennemis qui nuisent à ses projets de grandeur… ? Des lois cruelles, arbitraires, impérieuses pourront de même assassiner chaque siècle des millions d’individus, et nous, faibles et malheureux particuliers, nous ne pourrons pas sacrifier un seul être à nos vengeances ou à nos caprices ? Est-il rien de si barbare, de si ridiculement étrange, et ne devons-nous pas, sous le voile du plus profond mystère, nous venger amplement de cette ineptie\footnote{ Cet article se trouvant traité plus loin avec étendue, on s’est contenté de jeter seulement ici quelques bases du système que l’on développera bientôt.} ?\par
EUGÉNIE : Assurément… Oh ! comme votre morale est séduisante, et comme je la goûte !… Mais, dites-moi… Dolmancé… là, bien en conscience, ne vous seriez-vous pas quelquefois satisfait en ce genre ?\par
DOLMANCÉ : Ne me forcez pas à vous dévoiler mes fautes, leur nombre et leur espèce me contraindraient trop à rougir. Je vous les avouerai peut-être un jour.\par
MME DE SAINT-ANGE : Dirigeant le glaive des lois, le scélérat s’en est souvent servi pour satisfaire à ses passions.\par
DOLMANCÉ : Puissé-je n’avoir pas d’autres reproches à me faire !\par
MME DE SAINT-ANGE, {\itshape lui sautant au col} : Homme divin…, je vous adore, qu’il faut avoir d’esprit et de courage pour avoir, comme vous, goûté tous les plaisirs ; c’est à l’homme de génie seul qu’est réservé l’honneur de briser tous les freins de l’ignorance et de la stupidité ; baisez-moi, vous êtes charmant.\par
DOLMANCÉ : Soyez franche, Eugénie, n’avez-vous jamais souhaité la mort à personne ?\par
EUGÉNIE : Oh ! oui, oui, et j’ai sous mes yeux chaque jour une abominable créature que je voudrais voir depuis longtemps au tombeau.\par
MME DE SAINT-ANGE : Je gage que je devine.\par
EUGÉNIE : Qui soupçonnes-tu ?\par
MME DE SAINT-ANGE : Ta mère.\par
EUGÉNIE : Ah ! laisse-moi cacher ma rougeur dans ton sein !\par
DOLMANCÉ : Voluptueuse créature ! je veux t’accabler à mon tour des caresses qui doivent être le prix de l’énergie de ton cœur et de ta délicieuse tête. {\itshape (Dolmancé la baise sur tout le corps, et lui donne de légères claques sur les fesses, il bande ; M\textsuperscript{me} de Saint-Ange empoigne et secoue son vit ; ses mains, de temps en temps, s’égarent aussi sur le derrière de M\textsuperscript{me} de Saint-Ange qui le lui prête avec lubricité ; un peu revenu à lui, Dolmancé continue :)} Mais cette idée sublime, pourquoi ne l’exécuterions-nous pas ?\par
MME DE SAINT-ANGE : Eugénie, j’ai détesté ma mère tout autant que tu hais la tienne, et je n’ai pas balancé.\par
EUGÉNIE : Les moyens m’ont manqué.\par
MME DE SAINT-ANGE : Dis le courage.\par
EUGÉNIE : Hélas, si jeune encore !\par
DOLMANCÉ : Mais à présent, Eugénie, que feriez-vous ?\par
EUGÉNIE : Tout… Qu’on me donne les moyens, et l’on verra.\par
DOLMANCÉ : Vous les aurez, Eugénie, je vous le promets, mais j’y mets une condition.\par
EUGÉNIE : Quelle est-elle, ou plutôt quelle est celle que je ne sois prête à accepter ?\par
DOLMANCÉ : Viens, scélérate, viens dans mes bras, je n’y puis plus tenir ; il faut que ton charmant derrière soit le prix du don que je te promets, il faut qu’un crime paie l’autre, viens… ou plutôt accourez toutes deux éteindre par des flots de foutre le feu divin qui nous enflamme.\par
MME DE SAINT-ANGE : Mettons, s’il vous plaît, un peu d’ordre à ces orgies, il en faut même au sein du délire et de l’infamie.\par
DOLMANCÉ : Rien de si simple, l’objet majeur, ce me semble, est que je décharge, en donnant à cette charmante petite fille le plus de plaisir que je pourrai ; je vais lui mettre mon vit dans le cul, pendant que courbée dans vos bras vous la branlerez de votre mieux ; au moyen de l’attitude où je vous place, elle pourra vous le rendre, vous vous baiserez l’une et l’autre ; après quelques courses dans le cul de cette enfant, nous varierons le tableau. Je vous enculerai, madame, Eugénie au-dessus de vous, votre tête entre ses jambes m’offrira son clitoris à sucer, je lui ferai perdre ainsi du foutre une seconde fois ; je me replacerai ensuite dans son anus, vous me présenterez votre cul au lieu du con qu’elle m’offrait, c’est-à-dire que vous prendrez, comme elle viendra de le faire, sa tête entre vos jambes, je sucerai le trou de votre cul ; comme je viendrai de lui sucer le con, vous déchargerez, j’en ferai autant, pendant que ma main, embrassant le joli petit corps de cette charmante novice, ira lui chatouiller le clitoris pour la faire pâmer également.\par
MME DE SAINT-ANGE : Bien, mon cher Dolmancé, mais il vous manquera quelque chose ?\par
DOLMANCÉ : Un vit dans le cul ; vous avez raison, madame.\par
MME DE SAINT-ANGE : Passons-nous-en pour ce matin, nous l’aurons ce soir, mon frère viendra nous aider, et nos plaisirs seront au comble ; mettons-nous à l’œuvre.\par
DOLMANCÉ : Je voudrais qu’Eugénie me branlât un moment. {\itshape (Elle le fait.)} Oui, c’est cela… un peu plus vite, mon cœur ; tenez toujours bien à nu cette tête vermeille, ne la recouvrez jamais ; plus vous faites tendre le filet, mieux vous décidez l’érection…, il ne faut jamais recalotter le vit qu’on branle… Bon !… préparez ainsi vous-même l’état du membre qui va vous perforer ; voyez-vous comme il se décide. Donnez-moi votre langue, petite friponne… que vos fesses posent sur ma main droite, pendant que ma main gauche va vous chatouiller le clitoris.\par
MME DE SAINT-ANGE : Eugénie, veux-tu lui faire goûter de plus grands plaisirs ?\par
EUGÉNIE : Assurément… je veux tout faire pour lui en donner.\par
MME DE SAINT-ANGE : Eh bien ! prends son vit dans ta bouche, et suce-le quelques instants.\par
EUGÉNIE {\itshape le fait} : Est-ce ainsi ?\par
DOLMANCÉ : Ah ! bouche délicieuse ! quelle chaleur ! elle vaut pour moi le plus joli des culs… Femmes voluptueuses et adroites, ne refusez jamais ce plaisir à vos amants, il vous les enchaînera pour jamais ; ah {\itshape sacredieu ! foutredieu} !\par
MME DE SAINT-ANGE : Comme tu blasphèmes, mon ami !\par
DOLMANCÉ : Donnez-moi votre cul, madame… Oui, donnez-le-moi, que je le baise pendant qu’on me suce, et ne vous étonnez point de mes blasphèmes ; un de mes plus grands plaisirs est de jurer Dieu quand je bande ; il me semble que mon esprit, alors mille fois plus exalté, abhorre et méprise bien mieux cette dégoûtante chimère : je voudrais trouver une façon ou de la mieux invectiver, ou de l’outrager davantage, et quand mes maudites réflexions m’amènent à la conviction de la nullité de ce dégoûtant objet de ma haine, je m’irrite et voudrais pouvoir aussitôt réédifier le fantôme, pour que ma rage au moins portât sur quelque chose. Imitez-moi, femme charmante, et vous verrez l’accroissement que de tels discours porteront infailliblement à vos sens. Mais, doubledieu !… je le vois, il faut, quel que soit mon plaisir, que je me retire absolument de cette bouche divine… j’y laisserais mon foutre… Allons, Eugénie, placez-vous, exécutons le tableau que j’ai tracé, et plongeons-nous tous trois dans la plus voluptueuse ivresse.\par
{\itshape L’attitude s’arrange.}\par
EUGÉNIE : Que je crains, mon cher, l’impuissance de vos efforts, la disproportion est trop forte.\par
DOLMANCÉ : J’en sodomise tous les jours de plus jeunes ; hier encore un petit garçon de sept ans fut dépucelé par ce vit en moins de trois minutes… Courage, Eugénie, courage.\par
EUGÉNIE : Ah ! vous me déchirez.\par
MME DE SAINT-ANGE : Ménagez-la, Dolmancé, songez que j’en réponds.\par
DOLMANCÉ : Branlez-la bien, madame, elle sentira moins la douleur ; au reste, tout est dit maintenant, m’y voilà jusqu’au poil.\par
EUGÉNIE : Oh ciel ! ce n’est pas sans peine… Vois la sueur qui couvre mon front, cher ami… Ah, Dieu ! jamais je n’éprouvai d’aussi vives douleurs !\par
MME DE SAINT-ANGE : Te voilà à moitié dépucelée, ma bonne, te voilà au rang des femmes ; on peut bien acheter cette gloire par un peu de tourment ; mes doigts, d’ailleurs, ne te calment-ils donc point ?\par
EUGÉNIE : Pourrais-je y résister sans eux ?… Chatouille-moi, mon ange, je sens qu’imperceptiblement la douleur se métamorphose en plaisir. Poussez, poussez, Dolmancé, je me meurs.\par
DOLMANCÉ : Ah, foutredieu ! sacredieu ! tripledieu ! changeons, je n’y résisterais pas ; votre derrière, madame, je vous en conjure, et placez-vous sur-le-champ comme je vous l’ai dit. {\itshape (On s’arrange, et Dolmancé continue.)} J’ai moins de peine ici… Comme mon vit pénètre… Mais ce beau cul n’en est pas moins délicieux, madame.\par
EUGÉNIE : Suis-je bien ainsi, Dolmancé ?\par
DOLMANCÉ : À merveille ! ce joli petit con vierge s’offre délicieusement à moi ; je suis un coupable, un infractaire, je le sais ; de tels attraits sont peu faits pour nos yeux ; mais le désir de donner à cette enfant les premières leçons de la volupté l’emporte sur toute autre considération, je veux faire couler son foutre… je veux l’épuiser, s’il est possible.\par
{\itshape Il la gamahuche.}\par
EUGÉNIE : Ah ! vous me faites mourir de plaisir, je n’y puis résister !\par
MME DE SAINT-ANGE : Pour moi, je pars Ah ! fous… fous, Dolmancé, je décharge !\par
EUGÉNIE : J’en fais autant, ma bonne… Ah ! mon Dieu, comme il me suce !\par
MME DE SAINT-ANGE : Jure donc, petite putain, jure donc.\par
EUGÉNIE : Eh bien ! sacredieu ! je décharge… Je suis dans la plus douce ivresse.\par
DOLMANCÉ : Au poste, au poste, Eugénie, je serai ta dupe de tous ces changements de main. {\itshape (Eugénie se replace.)} Ah, bien ! me revoici dans mon premier gîte ; montrez-moi le trou de votre cul, madame, que je le gamahuche à mon aise… Que j’aime à baiser un cul que je viens de foutre. Ah ! faites-le-moi bien lécher pendant que je vais lancer mon sperme au fond de celui de votre amie. Le croiriez-vous, madame, il y est entré cette fois-ci sans peine ; ah ! foutre, foutre, vous n’imaginez pas comme elle le serre, comme elle le comprime ! Sacréfoutudieu, comme j’ai du plaisir ! Ah ! c’en est fait ! je n’y résiste plus, mon foutre coule… et je suis mort.\par
EUGÉNIE : Il me fait aussi mourir, ma chère bonne, je te le jure.\par
MME DE SAINT-ANGE : La friponne ! comme elle s’y habituera promptement !\par
DOLMANCÉ : Je connais une infinité de jeunes filles de son âge que rien au monde ne pourrait engager à jouir différemment ; il n’y a que la première fois qui coûte ; une femme n’a pas plutôt tâté de cette manière qu’elle ne veut plus faire autre chose… Oh, ciel ! je suis épuisé, laissez-moi reprendre haleine au moins quelques instants.\par
MME DE SAINT-ANGE : Voilà les hommes, ma chère, à peine nous regardent-ils quand leurs désirs sont satisfaits, cet anéantissement les mène au dégoût, et le dégoût bientôt au mépris.\par
DOLMANCÉ, {\itshape froidement} : Ah ! quelle injure, beauté divine ! {\itshape (Il les embrasse toutes deux.)} Vous n’êtes faites l’une et l’autre que pour des hommages, quel que soit l’état où l’on se trouve.\par
MME DE SAINT-ANGE : Au reste, console-toi, mon Eugénie, s’ils acquièrent le droit de nous négliger, parce qu’ils sont satisfaits, n’avons-nous pas de même celui de les mépriser quand leur procédé nous y force ? Si Tibère sacrifiait à Caprée les objets qui venaient de servir ses passions\footnote{ Voyez Suétone et Dion Cassius de Nicée.}, Zingua, reine d’Afrique, immolait aussi ses amants\footnote{ Voyez l’{\itshape Histoire de Zingua, reine d’Angola}.}.\par
DOLMANCÉ : Ces excès parfaitement simples et très conçus de moi, sans doute, ne doivent pourtant jamais s’exécuter entre nous. « Jamais entre eux ne se mangent les loups », dit le proverbe et, tel trivial qu’il soit, il est juste. Ne redoutez jamais rien de moi, mes amies, je vous ferai peut-être faire beaucoup de mal, mais je ne vous en ferai jamais.\par
EUGÉNIE : Oh ! non, non, ma chère, j’ose en répondre ; jamais Dolmancé n’abusera des droits que nous lui donnons sur nous : je lui crois la probité des roués, c’est la meilleure ; mais ramenons notre instituteur à ses principes, et revenons, je vous supplie, au grand dessein qui nous enflammait avant que nous ne nous calmassions.\par
MME DE SAINT-ANGE : Quoi ! friponne, tu y penses encore, j’avais cru que ce n’était l’histoire que de l’effervescence de ta tête.\par
EUGÉNIE : C’est le mouvement le plus certain de mon cœur, et je ne serai contente qu’après la consommation de ce crime.\par
MME DE SAINT-ANGE : Oh ! bon, bon, fais-lui grâce, songe qu’elle est ta mère.\par
EUGÉNIE : Le beau titre !\par
DOLMANCÉ : Elle a raison ; cette mère a-t-elle pensé à Eugénie en la mettant au monde, la coquine se laissait foutre, parce qu’elle y trouvait du plaisir, mais elle était bien loin d’avoir cette fille en vue ; qu’elle agisse comme elle voudra à cet égard ; laissons-lui la liberté tout entière, et contentons-nous de lui certifier qu’à quelque excès qu’elle arrive en ce genre, elle ne se rendra jamais coupable d’aucun mal.\par
EUGÉNIE : Je l’abhorre, je la déteste, mille raisons légitiment ma haine, il faut que j’aie sa vie, à quelque prix que ce puisse être.\par
DOLMANCÉ : Eh bien ! puisque tes résolutions sont inébranlables, tu seras satisfaite, Eugénie, je te le jure ; mais permets-moi quelques conseils qui deviennent, avant que d’agir, de la première nécessité pour toi ; que jamais ton secret ne t’échappe, ma chère, et surtout agis seule ; rien n’est plus dangereux que les complices ; méfions-nous toujours de ceux mêmes que nous croyons nous être le plus attachés : {\itshape il faut}, disait Machiavel, {\itshape ou n’avoir jamais de complices, ou s’en défaire dès qu’ils nous ont servi}. Ce n’est pas tout : la feinte est indispensable, Eugénie, aux projets que tu formes. Rapproche-toi plus que jamais de ta victime avant que de l’immoler, aie l’air de la plaindre ou de la consoler, cajole-la, partage ses peines, jure-lui que tu l’adores, fais plus encore, persuade-le-lui, la fausseté, dans de tels cas, ne saurait être portée trop loin ; Néron caressait Agrippine sur la barque même qui devait l’engloutir ; imite cet exemple, use de toute la fourberie, de toutes les impostures que pourra te suggérer ton esprit. Si le mensonge est toujours nécessaire aux femmes, c’est surtout lorsqu’elles veulent tromper, qu’il leur devient plus indispensable.\par
EUGÉNIE : Ces leçons seront retenues et mises en action, sans doute ; mais approfondissons, je vous prie, cette fausseté que vous conseillez aux femmes de mettre en usage ; croyez-vous donc cette manière d’être, absolument essentielle dans le monde ?\par
DOLMANCÉ : Je n’en connais pas, sans doute, de plus nécessaire dans la vie ; une vérité certaine va vous en prouver l’indispensabilité, tout le monde l’emploie, je vous demande, d’après cela, comment un individu sincère n’échouera pas toujours au milieu d’une société de gens faux ! Or s’il est vrai, comme on le prétend, que les vertus soient de quelque utilité dans la vie civile, comment voulez-vous que celui qui n’a ni la volonté, ni le pouvoir, ni le don d’aucune vertu, ce qui arrive à beaucoup de gens ; comment voulez-vous, dis-je, qu’un tel être ne soit pas essentiellement obligé de feindre pour obtenir à son tour un peu de la portion de bonheur que ses concurrents lui ravissent ? Et dans le fait, est-ce bien sûrement la vertu, ou son apparence, qui devient réellement nécessaire à l’homme social ? Ne doutons pas que l’apparence seule lui suffise ; il a tout ce qu’il faut en la possédant. Dès qu’on ne fait qu’effleurer les hommes dans le monde, ne leur suffit-il pas de nous montrer l’écorce ? Persuadons-nous bien, au surplus, que la pratique des vertus n’est guère utile qu’à celui qui la possède, les autres en retirent si peu que, pourvu que celui qui doit vivre avec nous paraisse vertueux, il devient parfaitement égal qu’il le soit en effet ou non ; la fausseté, d’ailleurs, est presque toujours un moyen assuré de réussir, celui qui la possède acquiert nécessairement une sorte de priorité sur celui qui commerce ou qui correspond avec lui ; en l’éblouissant par de faux dehors, il le persuade, de ce moment il réussit : m’aperçois-je que l’on m’a trompé, je ne m’en prends qu’à moi, et mon suborneur a d’autant plus beau jeu encore, que je ne me plaindrai pas par orgueil ; son ascendant sur moi sera toujours prononcé ; il aura raison quand j’aurai tort ; il s’avancera quand je ne serai rien ; il s’enrichira quand je me ruinerai ; toujours enfin au-dessus de moi, il captivera bientôt l’opinion publique ; une fois là, j’aurai beau l’inculper, on ne m’écoutera seulement pas. Livrons-nous donc hardiment et sans cesse à la plus insigne fausseté ; regardons-la comme la clé de toutes les grâces, de toutes les faveurs, de toutes les réputations, de toutes les richesses, et calmons à loisir le petit chagrin d’avoir fait des dupes par le piquant plaisir d’être fripon.\par
MME DE SAINT-ANGE : En voilà, je le pense, infiniment plus qu’il n’en faut sur cette matière ; Eugénie, convaincue, doit être apaisée, encouragée, elle agira quand elle voudra ; j’imagine qu’il est nécessaire de continuer maintenant nos dissertations sur les différents caprices des hommes dans le libertinage ; ce champ doit être vaste, parcourons-le ; nous venons d’initier notre élève dans quelques mystères de la pratique, ne négligeons pas la théorie.\par
DOLMANCÉ : Les détails libertins des passions de l’homme sont peu susceptibles, madame, de motifs d’instruction pour une jeune fille qui, comme Eugénie surtout, n’est pas destinée à faire le métier de femme publique ; elle se mariera, et dans cette hypothèse, il y a à parier dix contre un que son mari n’aura point ces goûts-là ; si cela était cependant, la conduite est facile : beaucoup de douceur et de complaisance avec lui ; d’autre part, beaucoup de fausseté et de dédommagement en secret, ce peu de mots renferme tout. Si votre Eugénie pourtant désire quelques analyses des goûts de l’homme dans l’acte du libertinage : pour les examiner plus sommairement, nous les réduirons à trois : {\itshape la sodomie, les fantaisies sacrilèges et les goûts cruels}. La première passion est universelle aujourd’hui, nous allons joindre quelques réflexions à ce que nous en avons déjà dit ; on la divise en deux classes, l’active et la passive : l’homme qui encule, soit un garçon, soit une femme, commet la sodomie active ; il est sodomite passif quand il se fait foutre. On a souvent mis en question laquelle de ces deux façons de commettre la sodomie était la plus voluptueuse ; c’est assurément la passive, puisqu’on jouit à la fois de la sensation du devant et de celle du derrière ; il est si doux de changer de sexe, si délicieux de contrefaire la putain, de se livrer à un homme qui nous traite comme une femme, d’appeler cet homme son amant, de s’avouer sa maîtresse : ah ! mes amies, quelle volupté ! Mais, Eugénie, bornons-nous ici à quelques conseils de détails, uniquement relatifs aux femmes qui, se métamorphosant en hommes, veulent jouir à notre exemple de ce plaisir délicieux. Je viens de vous familiariser avec ces attaques, Eugénie, et j’en ai assez vu pour être persuadé que vous ferez un jour bien des progrès dans cette carrière. Je vous exhorte à la parcourir comme une des plus délicieuses de l’île de Cythère, parfaitement sûr que vous accomplirez ce conseil ; je vais me borner à deux ou trois avis essentiels à toute personne décidée à ne plus connaître que ce genre de plaisirs, ou ceux qui leur sont analogues. Observez d’abord de vous faire toujours branler le clitoris quand on vous sodomise ; rien ne se marie comme ces deux plaisirs ; évitez le bidet ou le frottement de linge quand vous venez d’être foutue de cette manière ; il est bon que la brèche soit toujours ouverte, il en résulte des désirs, des titillations qu’éteignent aussitôt les soins de propreté ; on n’a pas d’idée du point auquel les sensations se prolongent. Ainsi, quand vous serez dans le train de vous amuser de cette manière, Eugénie, évitez les acides, ils enflamment les hémorroïdes et rendent alors les introductions douloureuses ; opposez-vous à ce que plusieurs hommes vous déchargent de suite dans le cul, ce mélange de sperme, quoique voluptueux pour l’imagination, est souvent dangereux pour la santé ; rejetez toujours au-dehors ces différentes émissions à mesure qu’elles se font.\par
EUGÉNIE : Mais si elles étaient faites par-devant, ne serait-ce pas un crime ?\par
MME DE SAINT-ANGE : N’imagine donc pas, pauvre folle, qu’il y ait le moindre mal à se prêter de telle manière que ce puisse être à détourner du grand chemin la semence de l’homme, parce que la propagation n’est nullement le but de la nature, elle n’en est qu’une tolérance ; et lorsque nous n’en profitons pas, ses intentions sont bien mieux remplies : Eugénie, sois l’ennemie jurée de cette fastidieuse propagation, et détourne sans cesse, même en mariage, cette perfide liqueur dont la végétation ne sert qu’à gâter nos tailles, qu’à émousser dans nous les sensations voluptueuses, nous flétrir, nous vieillir et déranger notre santé ; engage ton mari à s’accoutumer à ces pertes, offre-lui toutes les routes qui peuvent éloigner l’hommage du temple, dis-lui que tu détestes les enfants, que tu le supplies de ne point t’en faire. Observe-toi sur cet article, ma bonne, car, je te le déclare, j’ai la propagation dans une telle horreur que je cesserais d’être ton amie à l’instant où tu deviendrais grosse ; si pourtant ce malheur t’arrive, sans qu’il y ait de ta faute, préviens-moi dans les sept ou huit premières semaines, et je te ferai couler cela tout doucement ; ne crains point l’infanticide, ce crime est imaginaire, nous sommes toujours les maîtresses de ce que nous portons dans notre sein, et nous ne faisons pas plus de mal à détruire cette espèce de matière, qu’à purger l’autre, par des médicaments, quand nous en éprouvons le besoin.\par
EUGÉNIE : Mais si l’enfant était à terme ?\par
MME DE SAINT-ANGE : Fût-il au monde, nous serions toujours les maîtresses de le détruire. Il n’y a sur la terre aucun droit plus certain que celui des mères sur leurs enfants. Il n’est aucun peuple qui n’ait reconnu cette vérité, elle est fondée en raison, en principes.\par
DOLMANCÉ : Ce droit est dans la nature… il est incontestable. L’extravagance du système déifique fut la source de toutes ces erreurs grossières, les imbéciles qui croyaient un dieu, persuadés que nous ne tenions l’existence que de lui, et qu’aussitôt qu’un embryon était en maturité, une petite âme, émanée de dieu, venait l’animer aussitôt ; ces sots, dis-je, durent assurément considérer comme un crime capital la destruction de cette petite créature, parce que, d’après eux, elle n’appartenait plus aux hommes ; c’était l’ouvrage de dieu ; elle était à dieu, en pouvait-on disposer sans crime ! Mais depuis que le flambeau de la philosophie a dissipé toutes ces impostures, depuis que la chimère divine est foulée aux pieds, depuis que mieux instruits des lois et des secrets de la physique nous avons développé le principe de la génération, et que ce mécanisme matériel n’offre aux yeux rien de plus étonnant que la végétation du grain de blé, nous en avons appelé à la nature de l’erreur des hommes ; étendant la mesure de nos droits, nous avons enfin reconnu que nous étions parfaitement libres de reprendre ce que nous n’avions donné qu’à contre-cœur ou par hasard, et qu’il était impossible d’exiger d’un individu quelconque de devenir père ou mère s’il n’en a pas envie, que cette créature de plus ou de moins sur la terre n’était pas d’ailleurs d’une bien grande conséquence, et que nous devenions en un mot aussi certainement les maîtres de ce morceau de chair quelque animé qu’il fût, que nous le sommes des ongles que nous retranchons de nos doigts, des excroissances de chair que nous extirpons de nos corps, ou des digestions que nous supprimons de nos entrailles. Parce que l’un et l’autre sont de nous, parce que l’un et l’autre sont à nous, et que nous sommes absolument possesseurs de ce qui émane de nous. En vous développant, Eugénie, la très médiocre importance dont l’action du meurtre était sur terre, vous avez dû voir de quelle petite conséquence doit être également tout ce qui tient à l’infanticide commis sur une créature déjà même en âge de raison ; il est donc inutile d’y revenir, l’excellence de votre esprit ajoute à mes preuves, la lecture de l’histoire des mœurs de tous les peuples de la terre, en vous faisant voir que cet usage est universel, achèvera de vous convaincre qu’il n’y aurait que de l’imbécillité à admettre du mal à cette très indifférente action.\par
EUGÉNIE, {\itshape d’abord à Dolmancé} : Je ne puis vous dire à quel point vous me persuadez ; {\itshape (s’adressant ensuite à M\textsuperscript{me} de Saint-Ange :)} mais dis-moi, ma toute bonne, t’es-tu quelquefois servie du remède que tu m’offres pour détruire intérieurement le fœtus ?\par
MME DE SAINT-ANGE : Deux fois, et toujours avec le plus grand succès, mais je dois t’avouer que je n’en ai fait l’épreuve que dans les premiers temps ; cependant deux femmes de ma connaissance ont employé ce même remède à mi-terme, et elles m’ont assuré qu’il leur avait également réussi. Compte donc sur moi dans l’occasion, ma chère, mais je t’exhorte à ne te jamais mettre dans le cas d’en avoir besoin, c’est le plus sûr. Reprenons maintenant la suite des détails lubriques que nous avons promis à cette jeune fille. Poursuivez, Dolmancé, nous en sommes aux fantaisies sacrilèges.\par
DOLMANCÉ : Je suppose qu’Eugénie est trop revenue des erreurs religieuses pour ne pas être intimement persuadée que tout ce qui tient à se jouer des objets de la piété des sots, ne peut avoir aucune sorte de conséquence, ces fantaisies en ont si peu qu’elles ne doivent, dans le fait, échauffer que de très jeunes têtes, pour qui toute rupture de frein devient une jouissance ; c’est une espèce de petite vindicte qui enflamme l’imagination, et qui, sans doute, peut amuser quelques instants ; mais ces voluptés, ce me semble, doivent devenir insipides et froides quand on a eu le temps de s’instruire et de se convaincre de la nullité des objets dont les idoles que nous bafouons ne sont que la chétive représentation ; profaner les reliques, les images de saints, l’hostie, le crucifix, tout cela ne doit être, aux yeux du philosophe, que ce que serait la dégradation d’une statue païenne ; une fois qu’on a dévoué ces exécrables babioles au mépris, il faut les y laisser sans s’en occuper davantage, il n’est bon de conserver de tout cela que le blasphème, non qu’il ait plus de réalité, car dès l’instant où il n’y a plus de dieu, à quoi sert-il d’insulter son nom ? Mais c’est qu’il est essentiel de prononcer des mots forts, ou sales, dans l’ivresse du plaisir, et que ceux du blasphème servent assez bien l’imagination ; il n’y faut rien épargner, il faut orner ces mots du plus grand luxe d’expressions, il faut qu’ils scandalisent le plus possible ; car il est très doux de scandaliser, il existe là un petit triomphe pour l’orgueil qui n’est nullement à dédaigner, je vous l’avoue, mesdames, c’est une de mes voluptés secrètes, il est peu de plaisirs moraux plus actifs sur mon imagination ; essayez-le, Eugénie, et vous verrez ce qu’il en résulte ; étalez surtout une prodigieuse impiété lorsque vous vous trouvez avec des personnes de votre âge qui végètent encore dans les ténèbres de la superstition. Affichez la débauche et le libertinage, affectez de vous mettre en fille, de leur laisser voir votre gorge ; si vous allez avec elles dans les lieux secrets, troussez-vous avec indécence, laissez-leur voir avec affectation les plus secrètes parties de votre corps, exigez la même chose d’elles, séduisez-les, sermonnez-les, faites-leur voir le ridicule de leurs préjugés, mettez-les ce qui s’appelle à mal, jurez comme un homme avec elles, si elles sont plus jeunes que vous, prenez-les de force, amusez-vous-en et corrompez-les soit par des exemples, soit par des conseils, soit par tout ce que vous pourrez croire, en un mot, de plus capable de les pervertir ; soyez de même extrêmement libre avec les hommes, affichez avec eux l’irréligion et l’impudence ; loin de vous effrayer des libertés qu’ils prendront, accordez-leur mystérieusement tout ce qui peut les amuser sans vous compromettre, laissez-vous manier par eux, branlez-les, faites-vous branler, allez même jusqu’à leur prêter le cul ; mais puisque l’honneur chimérique des femmes tient à leurs prémices antérieures, rendez-vous plus difficile sur cela, une fois mariée, prenez des laquais, point d’amant, ou payez quelques gens sûrs ; de ce moment tout est à couvert, plus d’atteinte à votre réputation, et sans qu’on ait jamais pu vous suspecter, vous avez trouvé l’art de faire tout ce qui vous a plu.\par
Poursuivons :\par
Les plaisirs de la cruauté sont les troisièmes que nous nous sommes promis d’analyser.\par
Ces sortes de plaisirs sont aujourd’hui très communs parmi les hommes, et voici l’argument dont ils se servent pour les légitimer. Nous voulons être émus, disent-ils, c’est le but de tout homme qui se livre à la volupté, et nous voulons l’être par les moyens les plus actifs ; en partant de ce point, il ne s’agit pas de savoir si nos procédés plairont ou déplairont à l’objet qui nous sert, il s’agit seulement d’ébranler la masse de nos nerfs par le choc le plus violent possible ; or il n’est pas douteux que la douleur affectant bien plus vivement que le plaisir, les chocs résultatifs sur nous de cette sensation produite sur les autres, seront essentiellement d’une vibration plus vigoureuse, retentiront plus énergiquement dans nous, mettront dans une circulation plus violente les esprits animaux qui, se déterminant sur les basses régions par le mouvement de rétrogradation qui leur est essentiel alors, embraseront aussitôt les organes de la volupté, et les disposeront au plaisir ; les effets du plaisir sont toujours trompeurs dans les femmes ; il est d’ailleurs très difficile qu’un homme laid ou vieux les produise, y parvient-il ? ils sont faibles, et les chocs beaucoup moins nerveux, il faut donc préférer la douleur, dont les effets ne peuvent tromper, et dont les vibrations sont plus actives ; mais, objecte-t-on aux hommes entichés de cette manie, cette douleur afflige le prochain, est-il charitable de faire du mal aux autres pour se délecter soi-même ? les coquins vous répondent à cela, qu’accoutumés dans l’acte du plaisir à se compter pour tout, et les autres pour rien, ils sont persuadés qu’il est tout simple, d’après les impulsions de la nature, de préférer ce qu’ils sentent, à ce qu’ils ne sentent point ; que nous font, osent-ils dire, les douleurs occasionnées sur le prochain, les ressentons-nous ? non, au contraire, nous venons de démontrer que de leur production résulte une sensation délicieuse pour nous ; à quel titre ménagerions-nous donc un individu qui ne nous touche en rien, à quel titre lui éviterions-nous une douleur qui ne nous coûtera jamais une larme, quand il est certain que de cette douleur va naître un très grand plaisir pour nous ; avons-nous jamais éprouvé une seule impulsion de la nature qui nous conseille de préférer les autres à nous ? et chacun n’est-il pas pour soi dans le monde ? Vous nous parlez d’une voix chimérique de cette nature qui nous dit de ne pas faire aux autres ce que nous ne voudrions pas qui nous fût fait ; mais cet absurde conseil ne nous est jamais venu que des hommes, et des hommes faibles, l’homme puissant ne s’avisa jamais de parler un tel langage. Ce furent les premiers chrétiens, qui journellement persécutés pour leur imbécile système, criaient à qui voulait l’entendre : « Ne nous brûlez pas, ne nous écorchez pas, {\itshape la nature dit qu’il ne faut pas faire aux autres ce que nous ne voudrions pas qu’il nous fût fait}. » Imbéciles, comment la nature qui nous conseille toujours de nous délecter, qui n’imprime jamais dans nous d’autres mouvements, d’autres inspirations, pourrait-elle le moment d’après, par une inconséquence sans exemple, nous assurer qu’il ne faut pourtant pas nous aviser de nous délecter si cela peut faire de la peine aux autres ? ah ! croyons-le, croyons-le, Eugénie, la nature, notre mère à tous, ne nous parle jamais que de nous, rien n’est égoïste comme sa voix, et ce que nous y reconnaissons de plus clair est l’immuable et saint conseil qu’elle nous donne de nous délecter, n’importe aux dépens de qui. Mais les autres, vous dit-on à cela, peuvent se venger… À la bonne heure, le plus fort seul aura raison. Eh bien ! voilà l’état primitif de guerre et de destruction perpétuelles pour lequel sa main nous créa, et dans lequel seul il lui est avantageux que nous soyons.\par
Voilà, ma chère Eugénie, comme raisonnent ces gens-là, et moi, j’y ajoute, d’après mon expérience et mes études, que la cruauté, bien loin d’être un vice, est le premier sentiment qu’imprime en nous la nature : l’enfant brise son hochet, mord le téton de sa nourrice, étrangle son oiseau bien avant que d’avoir l’âge de raison ; la cruauté est empreinte dans les animaux chez lesquels, ainsi que je crois vous l’avoir dit, les lois de la nature se lisent bien plus énergiquement que chez nous. Elle est chez les sauvages bien plus rapprochée de la nature que chez l’homme civilisé ; il serait donc absurde d’établir qu’elle fût une suite de la dépravation ; ce système est faux, je le répète, la cruauté est dans la nature, nous naissons tous avec une dose de cruauté que la seule éducation modifie ; mais l’éducation n’est pas dans la nature, elle nuit autant aux effets sacrés de la nature que la culture nuit aux arbres. Comparez dans vos vergers l’arbre abandonné aux soins de la nature, avec celui que votre art soigne en le contraignant, et vous verrez lequel est le plus beau, vous éprouverez lequel vous donnera de meilleurs fruits ; la cruauté n’est autre chose que l’énergie de l’homme que la civilisation n’a point encore corrompue, elle est donc une vertu et non pas un vice ; retranchez vos lois, vos punitions, vos usages, et la cruauté n’aura plus d’effets dangereux, puisqu’elle n’agira jamais sans pouvoir être aussitôt repoussée par les mêmes voies ; c’est dans l’état de civilisation qu’elle est dangereuse, parce que l’être lésé manque presque toujours, ou de la force, ou des moyens de repousser l’injure ; mais dans l’état d’incivilisation, si elle agit sur le fort, elle sera repoussée par lui, et si elle agit sur le faible, ne lésant qu’un être qui cède au fort par les lois de la nature, elle n’a pas le moindre inconvénient.\par
Nous n’analyserons point la cruauté dans les plaisirs lubriques chez les hommes ; vous voyez à peu près, Eugénie, les différents excès où ils doivent porter, et votre ardente imagination doit vous faire aisément comprendre que, dans une âme ferme et stoïque, ils ne doivent point avoir de bornes. Néron, Tibère, Héliogabale immolaient des enfants pour se faire bander ; le maréchal de Retz, Charolais l’oncle de Condé, commirent aussi des meurtres de débauche : le premier avoua dans son interrogatoire qu’il ne connaissait pas de volupté plus puissante que celle qu’il retirait du supplice infligé par son aumônier et lui sur de jeunes enfants des deux sexes. On en trouva sept ou huit cents d’immolés dans un de ses châteaux de Bretagne. Tout cela se conçoit, je viens de vous le prouver. Notre constitution, nos organes, le cours des liqueurs, l’énergie des esprits animaux, voilà les causes physiques qui font, dans la même heure, ou des {\itshape Titus} ou des {\itshape Néron}, des {\itshape Messaline} ou des {\itshape Chantal} ; il ne faut pas plus s’enorgueillir de la vertu, que se repentir du vice, pas plus accuser la nature de nous avoir fait naître bon, que de nous avoir créé scélérat ; elle a agi d’après ses vues, ses plans et ses besoins, soumettons-nous. Je n’examinerai donc ici que la cruauté des femmes, toujours bien plus active chez elles que chez les hommes, par la puissante raison de l’excessive sensibilité de leurs organes. Nous distinguons en général deux sortes de cruauté ; celle qui naît de la stupidité, qui jamais raisonnée, jamais analysée, assimile l’individu né tel, à la bête féroce : celle-là ne donne aucun plaisir, parce que celui qui y est enclin n’est susceptible d’aucune recherche, les brutalités d’un tel être sont rarement dangereuses, il est toujours facile de s’en mettre à l’abri ; l’autre espèce de cruauté, fruit de l’extrême sensibilité des organes, n’est connue que des êtres extrêmement délicats, et les excès où elle les porte ne sont que des raffinements de leur délicatesse ; c’est cette délicatesse trop promptement émoussée à cause de son excessive finesse qui, pour se réveiller, met en usage toutes les ressources de la cruauté, qu’il est peu de gens qui conçoivent ces différences… Comme il en est peu qui les sentent, elles existent pourtant, elles sont indubitables ; or, c’est ce second genre de cruauté dont les femmes sont le plus souvent affectées. Étudiez-les bien, vous verrez si ce n’est pas l’excès de leur sensibilité qui les a conduites là. Vous verrez si ce n’est pas l’extrême activité de leur imagination, la force de leur esprit qui les rend scélérates et féroces ; aussi celles-là sont-elles toutes charmantes, aussi n’en est-il pas une seule de cette espèce qui ne fassent tourner des têtes quand elles l’entreprennent ; malheureusement la rigidité, ou plutôt l’absurdité de nos mœurs laisse peu d’aliment à leur cruauté ; elles sont obligées de se cacher, de dissimuler, de couvrir leur inclination par des actes de bienfaisance ostensibles qu’elles détestent au fond de leur cœur ; ce ne peut plus être que sous le voile le plus obscur, avec les précautions les plus grandes, aidées de quelques amies sûres qu’elles peuvent se livrer à leurs inclinations ; et comme il en est beaucoup de ce genre, il en est par conséquent beaucoup de malheureuses ; voulez-vous les connaître ? annoncez-leur un spectacle cruel, celui d’un duel, d’un incendie, d’une bataille, d’un combat de gladiateurs, vous verrez comme elles accourront mais ces occasions ne sont pas assez nombreuses pour alimenter leur fureur, elles se contiennent, et elles souffrent. Jetons un coup d’œil rapide sur les femmes de ce genre ; Zingua, reine d’Angola, la plus cruelle des femmes, immolait ses amants dès qu’ils avaient joui d’elle ; souvent elle faisait battre des guerriers sous ses yeux et devenait le prix du vainqueur ; pour flatter son âme féroce, elle se divertissait à faire piler dans un mortier toutes les femmes devenues enceintes avant l’âge de trente ans\footnote{ Voyez l’{\itshape Histoire de Zingua, reine d’Angola}, par un missionnaire.}. Zoé, femme d’un empereur chinois, n’avait pas de plus grand plaisir que de voir exécuter des criminels sous ses yeux ; à leur défaut, elle faisait immoler des esclaves pendant qu’elle foutait avec son mari, et proportionnait les élans de sa décharge à la cruauté des angoisses qu’elle faisait supporter à ces malheureux. Ce fut elle qui, raffinant sur le genre de supplice à imposer à ces victimes, inventa cette fameuse colonne d’airain creuse que l’on faisait rougir après y avoir enfermé le patient. Théodora, la femme de Justinien, s’amusait à voir faire des eunuques ; et Messaline se branlait pendant que, par le procédé de la masturbation, on exténuait des hommes devant elle. Les Floridiennes faisaient grossir le membre de leurs époux et plaçaient de petits insectes sur le gland, ce qui leur faisait endurer des douleurs horribles, elles les attachaient pour cette opération, et se réunissaient plusieurs autour d’un seul homme pour en venir plus sûrement à bout ; dès qu’elles aperçurent les Espagnols, elles tinrent elles-mêmes leurs époux pendant que ces barbares Européens les assassinaient ; la Voisin, la Brinvilliers empoisonnaient pour leur seul plaisir de commettre un crime. L’histoire en un mot nous fournit mille et mille traits de la cruauté des femmes, et c’est en raison du penchant naturel qu’elles éprouvent à ces mouvements que je voudrais qu’elles s’accoutumassent à faire usage de la flagellation active, moyen par lequel les hommes cruels apaisent leur férocité ; quelques-unes d’entre elles en usent, je le sais, mais il n’est pas encore en usage, parmi ce sexe, au point où je le désirerais, au moyen de cette issue donnée à la barbarie des femmes, la société y gagnerait ; car ne pouvant être méchantes de cette manière, elles le sont d’une autre, et, répandant ainsi leur venin dans le monde, elles font le désespoir de leurs époux et de leur famille. Le refus de faire une bonne action, lorsque l’occasion s’en présente, celui de secourir l’infortune, donnent bien, si l’on veut, de l’essor à cette férocité où certaines femmes sont naturellement entraînées ; mais cela est faible et souvent beaucoup trop loin du besoin qu’elles ont de faire pis. Il y aurait, sans doute, d’autres moyens par lesquels une femme, à la fois sensible et féroce, pourrait calmer ses fougueuses passions ; mais ils sont dangereux, Eugénie, et je n’oserais jamais te les conseiller… Oh ciel ! qu’avez-vous donc, cher ange ?… Madame, dans quel état voilà votre élève ?\par
EUGÉNIE, {\itshape se branlant} : Ah ! sacredieu, vous me tournez la tête… Voilà l’effet de vos foutus propos.\par
DOLMANCÉ : Au secours, madame, au secours… laisserons-nous donc décharger cette belle enfant sans l’aider ?\par
MME DE SAINT-ANGE : Oh ! ce serait injuste ; {\itshape (la prenant dans ses bras :)} adorable créature, je n’ai jamais vu une sensibilité comme la tienne, jamais une tête si délicieuse !\par
DOLMANCÉ : Soignez le devant, madame, je vais avec ma langue effleurer le joli petit trou de son cul, en lui donnant de légères claques sur ses fesses il faut qu’elle décharge entre nos mains au moins sept ou huit fois de cette manière.\par
EUGÉNIE, {\itshape égarée} : Ah, foutre ! ce ne sera pas difficile.\par
DOLMANCÉ : Par l’attitude où nous voilà, mesdames, je remarque que vous pourriez me sucer le vit tour à tour, excité de cette manière, je procéderais avec bien plus d’énergie aux plaisirs de notre charmante élève.\par
EUGÉNIE : Ma bonne, je te dispute l’honneur de sucer ce beau vit.\par
{\itshape Elle le prend.}\par
DOLMANCÉ : Ah ! quelles délices, quelle chaleur voluptueuse !… Mais, Eugénie, vous comporterez-vous bien à l’instant de la crise ?\par
MME DE SAINT-ANGE : Elle avalera… elle avalera, je réponds d’elle, et d’ailleurs si, par enfantillage… par je ne sais quelle cause, enfin… elle négligeait les devoirs que lui impose ici la lubricité…\par
DOLMANCÉ, {\itshape très animé} : Je ne lui pardonnerais pas, madame, je ne lui pardonnerais pas, une punition exemplaire… Je vous jure qu’elle serait fouettée… qu’elle le serait jusqu’au sang… Ah, sacredieu ! je décharge, mon foutre coule… avale… avale, Eugénie, qu’il n’y en ait pas une goutte de perdue… Et vous, madame, soignez donc mon cul, il s’offre à vous… Ne voyez-vous donc pas comme il bâille, mon foutu cul ?… ne voyez-vous donc pas comme il appelle vos doigts ?… Foutredieu, mon extase est complète, vous les y enfoncez jusqu’au poignet… Ah ! remettons-nous ; je n’en puis plus… cette charmante fille m’a sucé comme un ange.\par
EUGÉNIE : Mon cher et adorable instituteur, je n’en ai pas perdu une goutte ; baise-moi, cher amour, ton foutre est maintenant au fond de mes entrailles.\par
DOLMANCÉ : Elle est délicieuse…, et comme la petite friponne a déchargé !\par
MME DE SAINT-ANGE : Elle est inondée… Oh ciel ! qu’entends-je… on frappe, qui peut venir ainsi nous troubler ?… c’est mon frère… imprudent !\par
EUGÉNIE : Mais, ma chère, ceci est une trahison !\par
DOLMANCÉ : Sans exemple, n’est-ce pas ; ne craignez rien, Eugénie, nous ne travaillons que pour vos plaisirs.\par
MME DE SAINT-ANGE : Ah ! nous allons bientôt l’en convaincre. Approche, mon frère, et ris de cette petite fille qui se cache pour n’être pas vue de toi.
\section[{Quatrième dialogue}]{Quatrième dialogue}\phantomsection
\label{d4}\renewcommand{\leftmark}{Quatrième dialogue}

\textit{MME DE SAINT-ANGE, EUGÉNIE, DOLMANCÉ, LE CHEVALIER DE MIRVEL}\par
\noindent LE CHEVALIER : Ne redoutez rien, je vous en conjure, de ma discrétion, belle Eugénie, elle est entière, voilà ma sœur, voilà mon ami qui peuvent tous les deux vous répondre de moi.\par
DOLMANCÉ : Je ne vois qu’une chose pour terminer tout d’un coup ce ridicule cérémonial ; tiens, Chevalier, nous éduquons cette jolie fille, nous lui apprenons tout ce qu’il faut que sache une demoiselle de son âge, et pour la mieux instruire nous joignons toujours un peu de pratique à la théorie, il lui faut le tableau d’un vit qui décharge, c’est où nous en sommes, veux-tu nous donner le modèle ?\par
LE CHEVALIER : Cette proposition est assurément trop flatteuse pour que je m’y refuse, et mademoiselle a des attraits qui décideront bien vite les effets de la leçon désirée.\par
MME DE SAINT-ANGE : Eh bien ! allons ; à l’œuvre à l’instant.\par
EUGÉNIE : Oh ! en vérité, c’est trop fort ; vous abusez de ma jeunesse à un point… mais pour qui monsieur va-t-il me prendre ?\par
LE CHEVALIER : Pour une fille charmante, Eugénie… pour la plus adorable créature que j’aie vue de mes jours. ({\itshape Il la baise et laisse promener ses mains sur ses charmes}.) Oh, dieu quels appas frais et mignons… quels charmes enchanteurs !\par
DOLMANCÉ : Parlons moins, Chevalier, et agissons beaucoup davantage ; je vais diriger la scène, c’est mon droit ; l’objet de celle-ci est de faire voir à Eugénie le mécanisme de l’éjaculation ; mais comme il est difficile qu’elle puisse observer un tel phénomène de sang-froid, nous allons nous placer tous quatre bien en face, et très près les uns des autres, vous branlerez votre amie, madame, je me chargerai du Chevalier ; quand il s’agit de pollution, un homme s’y entend, pour un homme, infiniment mieux qu’une femme, comme il sait ce qui lui convient, il sait ce qu’il faut faire aux autres… Allons, plaçons-nous.\par
{\itshape On s’arrange.}\par
MME DE SAINT-ANGE : Ne sommes-nous pas trop près ?\par
DOLMANCÉ, {\itshape s’emparant déjà du Chevalier} : Nous ne saurions l’être trop, madame ; il faut que le sein et le visage de votre amie soient inondés des preuves de la virilité de votre frère ; il faut qu’il lui décharge ce qui s’appelle au nez : maître de la pompe, j’en dirigerai les flots, de manière à ce qu’elle s’en trouve absolument couverte ; branlez-la soigneusement pendant ce temps sur toutes les parties lubriques de son corps ; Eugénie, livrez votre imagination tout entière aux derniers écarts du libertinage ; songez que vous allez en voir les plus beaux mystères s’opérer sous vos yeux, foulez toute retenue aux pieds ; la pudeur ne fut jamais une vertu ; si la nature eût voulu que nous cachassions quelques parties de nos corps, elle eût pris ce soin elle-même ; mais elle nous a créés nus, donc elle veut que nous allions nus, et tout procédé contraire outrage absolument ses lois. Les enfants qui n’ont encore aucune idée du plaisir, et par conséquent de la nécessité de le rendre plus vif par la modestie, montrent tout ce qu’ils portent ; on rencontre aussi quelquefois une singularité plus grande ; il est des pays où la pudeur des vêtements est d’usage, sans que la modestie des mœurs s’y rencontre. À Otaïti les filles sont vêtues, et elles se troussent dès qu’on l’exige.\par
MME DE SAINT-ANGE : Ce que j’aime de Dolmancé, c’est qu’il ne perd pas son temps, tout en discourant, voyez comme il agit, comme il examine avec complaisance le superbe cul de mon frère, comme il branle voluptueusement le beau vit de ce jeune homme… Allons, Eugénie, mettons-nous à l’ouvrage, voilà le tuyau de la pompe en l’air, il va bientôt nous inonder.\par
EUGÉNIE : Ah ! ma chère amie, quel monstrueux membre, à peine puis-je l’empoigner… Oh ! mon dieu, sont-ils tous aussi gros que cela ?\par
DOLMANCÉ : Vous savez, Eugénie, que le mien est bien inférieur ; de tels engins sont redoutables pour une jeune fille ; vous sentez bien que celui-là ne vous perforerait pas sans danger.\par
EUGÉNIE, {\itshape déjà branlée par M\textsuperscript{me} de Saint-Ange} : Ah ! je les braverais tous pour en jouir.\par
DOLMANCÉ : Et vous auriez raison ; une jeune fille ne doit jamais s’effrayer d’une telle chose ; la nature se prête, et les torrents de plaisirs dont elle vous comble vous dédommagent bientôt des petites douleurs qui les précèdent. J’ai vu des filles, plus jeunes que vous, soutenir de plus gros vits encore. Avec du courage et de la patience on surmonte les plus grands obstacles. C’est une folie que d’imaginer qu’il faille, autant qu’il est possible, ne faire dépuceler une jeune fille que par de très petits vits, je suis d’avis qu’une vierge doit se livrer au contraire aux plus gros engins qu’elle pourra rencontrer, afin que les ligaments de l’hymen plus tôt brisés, les sensations du plaisir puissent aussi se décider plus promptement dans elle ; il est vrai qu’une fois à ce régime, elle aura bien de la peine à en revenir au médiocre, mais si elle est riche, jeune et belle, elle en trouvera de cette taille tant qu’elle voudra, qu’elle s’y tienne ; s’en présente-t-il à elle de moins gros, et qu’elle ait pourtant envie d’employer, qu’elle les place alors dans son cul.\par
MME DE SAINT-ANGE : Sans doute, et pour être encore plus heureuse, qu’elle se serve de l’un et de l’autre à la fois, que les secousses voluptueuses dont elle agitera celui qui l’enconne servent à précipiter l’extase de celui qui l’encule ; et, qu’inondée du foutre de tous deux, elle élance le sien en mourant de plaisir.\par
DOLMANCÉ ({\itshape il faut observer que les pollutions vont toujours pendant le dialogue}) : Il me semble qu’il devrait entrer deux ou trois vits de plus dans le tableau que vous arrangez, madame ; la femme que vous placez, comme vous venez de le dire, ne pourrait-elle pas avoir un vit dans la bouche et un dans chaque main ?\par
MME DE SAINT-ANGE : Elle en pourrait avoir sous les aisselles et dans les cheveux, elle devrait en avoir trente autour d’elle s’il était possible ; il faudrait, dans ces moments-là, n’avoir, ne toucher, ne dévorer que des vits autour de soi, être inondée par tous au même instant où l’on déchargerait soi-même. Ah ! Dolmancé, quelque putain que vous soyez, je vous défie de m’avoir égalée dans ces délicieux combats de la luxure… J’ai fait tout ce qu’il est possible en ce genre.\par
EUGÉNIE, {\itshape toujours branlée par son amie, comme le Chevalier l’est par Dolmancé} : Ah ! ma bonne… tu me fais tourner la tête… quoi ! je pourrai aussi me procurer de tels plaisirs… je pourrai me livrer… à tout plein d’hommes ; ah ! quelles délices… comme tu me branles, chère amie… tu es la déesse même du plaisir… Et ce beau vit, comme il se gonfle… comme sa tête majestueuse s’enfle et devient vermeille !\par
DOLMANCÉ : Il est bien près du dénouement.\par
LE CHEVALIER : Eugénie… ma sœur… approchez-vous… ah ! quelles gorges divines… quelles cuisses douces et potelées… déchargez… déchargez toutes deux, mon foutre va s’y joindre… il coule… ah sacredieu !\par
{\itshape Dolmancé, pendant cette crise, a soin de diriger les flots de sperme de son ami sur les deux femmes, et principalement sur Eugénie, qui s’en trouve inondée. }\par
EUGÉNIE : Quel beau spectacle !… comme il est noble et majestueux. M’en voilà tout à fait couverte ; il m’en est sauté jusque dans les yeux.\par
MME DE SAINT-ANGE : Attends, ma mie, laisse-moi recueillir ces perles précieuses, je vais en frotter ton clitoris pour provoquer plus vite ta décharge.\par
EUGÉNIE : Ah ! oui, ma bonne, ah ! oui, cette idée est délicieuse… exécute, et je pars dans tes bras.\par
MME DE SAINT-ANGE : Divin enfant, baise-moi mille et mille fois… laisse-moi sucer ta langue… que je respire ta voluptueuse haleine quand elle est embrasée par le feu du plaisir… ah ! foutre, je décharge moi-même, mon frère, finis-moi, je t’en conjure.\par
DOLMANCÉ : Oui, Chevalier… oui, branlez votre sœur.\par
LE CHEVALIER : J’aime mieux la foutre, je bande encore.\par
DOLMANCÉ : Eh bien ! mettez-lui, en me présentant votre cul, je vous foutrai pendant ce voluptueux inceste, Eugénie armée de ce godemiché m’enculera. Destinée à jouer un jour tous les différents rôles de la luxure, il faut qu’elle s’exerce dans les leçons que nous lui donnons ici à les remplir tous également.\par
EUGÉNIE, {\itshape s’affublant du godemiché} : Oh ! volontiers, vous ne me trouverez jamais en défaut quand il s’agira de libertinage, il est maintenant mon seul dieu, l’unique règle de ma conduite, la seule base de toutes mes actions. ({\itshape Elle encule Dolmancé}.) Est-ce ainsi, mon cher maître, fais-je bien ?\par
DOLMANCÉ : À merveille… En vérité la petite friponne m’encule comme un homme ; bon, il me semble que nous voilà parfaitement liés tous les quatre, il ne s’agit plus que d’aller.\par
MME DE SAINT-ANGE : Ah ! je me meurs, Chevalier, il m’est impossible de m’accoutumer aux délicieuses secousses de ton beau vit !\par
DOLMANCÉ : Sacredieu, que ce cul charmant me donne de plaisir, ah ! foutre, foutre, déchargeons tous les quatre à la fois… Double dieu, je me meurs… j’expire… Ah ! de ma vie je ne déchargeai plus voluptueusement ! As-tu perdu ton sperme, Chevalier ?\par
LE CHEVALIER : Vois ce con, comme il en est barbouillé.\par
DOLMANCÉ : Ah ! mon ami, que n’en ai-je autant dans le cul !\par
MME DE SAINT-ANGE : Reposons-nous, je me meurs.\par
DOLMANCÉ, {\itshape baisant Eugénie} : Cette charmante fille m’a foutu comme un dieu.\par
EUGÉNIE : En vérité, j’y ai ressenti du plaisir.\par
DOLMANCÉ : Tous les excès en donnent quand on est libertine, et ce qu’une femme a de mieux à faire est de les multiplier au-delà même du possible.\par
MME DE SAINT-ANGE : J’ai placé cinq cents louis chez un notaire pour l’individu quelconque qui m’apprendra une passion que je ne connaisse pas, et qui puisse plonger mes sens dans une volupté dont je n’aie pas encore joui.\par
DOLMANCÉ {\itshape (Ici les interlocuteurs, rajustés, ne s’occupent plus que de causer)} : Cette idée est bizarre, et je la saisirai, mais je doute, madame, que cette envie singulière, après laquelle vous courez, ressemble aux minces plaisirs que vous venez de goûter.\par
MME DE SAINT-ANGE : Comment donc ?\par
DOLMANCÉ : C’est qu’en honneur je ne connais rien de si fastidieux que la jouissance du con, et quand une fois comme vous, madame, on a goûté les plaisirs du cul, je ne conçois pas comment on revient aux autres.\par
MME DE SAINT-ANGE : Ce sont de vieilles habitudes ; quand on pense comme moi on veut être foutue partout, et quelle que soit la partie qu’un engin perfore on est heureuse quand on l’y sent. Je suis pourtant bien de votre avis, et j’atteste ici à toutes les femmes voluptueuses que le plaisir qu’elles éprouveront à foutre en cul, surpassera toujours de beaucoup celui qu’elles éprouveront à le faire en con ; qu’elles s’en rapportent sur cela à la femme de l’Europe qui l’a le plus fait de l’une et de l’autre manière ; je leur certifie qu’il n’y a pas la moindre comparaison, et qu’elles reviendront bien difficilement au devant, quand elles auront fait l’expérience du derrière.\par
LE CHEVALIER : Je ne pense pas tout à fait de même, je me prête à tout ce qu’on veut, mais, par goût, je n’aime vraiment dans les femmes que l’autel qu’indiqua la nature pour leur rendre hommage.\par
DOLMANCÉ : Eh bien ! mais c’est le cul, jamais la nature, mon cher Chevalier, si tu scrutes avec soin ses lois, n’indiqua d’autres autels à notre hommage que le trou du derrière ; elle permet le reste, mais elle ordonne celui-ci ; ah ! sacredieu, si son intention n’était pas que nous foutions des culs, aurait-elle aussi justement proportionné leur orifice à nos membres ; cet orifice n’est-il pas rond comme eux, quel être assez ennemi du bon sens peut imaginer qu’un trou ovale puisse avoir été créé par la nature pour des membres ronds ; ses intentions se lisent dans cette difformité, elle nous fait voir clairement par là que des sacrifices trop réitérés dans cette partie, en multipliant une propagation dont elle ne fait que nous accorder la tolérance, lui déplairaient infailliblement. Mais poursuivons notre éducation. Eugénie vient de considérer, tout à l’aise, le sublime mystère d’une décharge, je voudrais maintenant qu’elle apprît à en diriger les flots.\par
MME DE SAINT-ANGE : Dans l’épuisement où vous voilà tous deux, c’est lui préparer bien de la peine.\par
DOLMANCÉ : J’en conviens, aussi voilà pourquoi je désirerais que nous puissions avoir, dans votre maison, ou dans votre campagne, quelque jeune garçon bien robuste, qui nous servirait de mannequin, et sur lequel nous pourrions donner des leçons.\par
MME DE SAINT-ANGE : J’ai précisément votre affaire.\par
DOLMANCÉ : Ne serait-ce point par hasard un jeune jardinier, d’une figure délicieuse, d’environ dix-huit ou vingt ans, que j’ai vu tout à l’heure travaillant à votre potager ?\par
MME DE SAINT-ANGE : Augustin, oui précisément, Augustin, et dont le membre a treize pouces de long sur huit et demi de circonférence.\par
DOLMANCÉ : Ah ! juste ciel, quel monstre… et cela décharge ?…\par
MME DE SAINT-ANGE : Oh ! comme un torrent ; je vais le chercher.
\section[{Cinquième dialogue}]{Cinquième dialogue}\phantomsection
\label{d5}\renewcommand{\leftmark}{Cinquième dialogue}

\textit{DOLMANCÉ, LE CHEVALIER, AUGUSTIN, EUGÉNIE, MME DE SAINT-ANGE}\par
\noindent MME DE SAINT-ANGE, amenant Augustin : Voilà l’homme dont je vous ai parlé ; allons mes amis, amusons-nous : que serait la vie sans le plaisir… Approche, benêt… Oh ! le sot ; croyez-vous qu’il y a six mois, que je travaille à débourrer ce gros cochon, sans pouvoir en venir à bout ?\par
AUGUSTIN : Ma fig, Madame, vous dites pourtant quelquefois comme ça que je commence à ne pas si mal aller à présent, et quand y a du terrain en friche, c’est toujours à moi que vous le donnez.\par
DOLMANCÉ, {\itshape riant} : Ah ! charmant… charmant… Le cher ami, il est aussi franc qu’il est frais… {\itshape (Montrant Eugénie :)} Augustin, voilà une banquette de fleurs en friche, veux-tu l’entreprendre ?\par
AUGUSTIN : Ah ! tatiguai, Monsieur, de si gentils morceaux ne sont pas faits pour nous.\par
DOLMANCÉ : Allons, mademoiselle.\par
EUGÉNIE, {\itshape rougissant} : Oh ciel ! je suis d’une honte !\par
DOLMANCÉ : Éloignez de vous ce sentiment pusillanime ; toutes nos actions, et surtout celles du libertinage, nous étant inspirées par la nature, il n’en est aucune, de quelque espèce que vous puissiez la supposer, dont nous devions concevoir de la honte ; allons, Eugénie, faites acte de putanisme avec ce jeune homme ; songez que toute provocation, faite par une fille à un garçon est une offrande à la nature, et que votre sexe ne la sert jamais mieux, que quand il se prostitue au nôtre ; que c’est en un mot, pour être foutue que vous êtes née et que celle qui se refuse à cette intention de la nature sur elle, ne mérite pas de voir le jour. Rabaissez vous-même la culotte de ce jeune homme jusqu’au bas de ses belles cuisses ; roulez sa chemise sous sa veste ; que le devant… et le derrière, qu’il a, par parenthèse, fort beau, se trouvent à votre disposition… Qu’une de vos mains s’empare maintenant de cet ample morceau de chair qui bientôt, je le vois, va vous effrayer par sa forme, et que l’autre se promène sur les fesses, et chatouille, ainsi, l’orifice du cul… Oui, de cette manière. {\itshape (Pour faire voir à Eugénie ce dont il s’agit, il socratise Augustin lui-même.)} Décalottez bien cette tête rubiconde ; ne la recouvrez jamais en polluant, tenez-la nue… tendez le filet, au point de le rompre… Eh bien ! voyez-vous déjà l’effet de mes leçons… Et toi, mon enfant, je t’en conjure, ne reste pas ainsi les mains jointes, n’y a-t-il donc pas là de quoi les occuper ; promène-les sur ce beau sein, sur ces belles fesses.\par
AUGUSTIN : Monsieur, est-ce que je ne pourrions pas baiser cette demoiselle qui me fait tant de plaisir ?\par
MME DE SAINT-ANGE : Eh ! baise-la, imbécile, baise-la tant que tu voudras ; ne me baises-tu pas, moi, quand je couche avec toi ?\par
AUGUSTIN : Ah ! tatiguai, la belle bouche, comme ça vous est frais ; il me semble avoir le nez sur les roses de not jardin. {\itshape (Montrant son vit bandant :)} Aussi, voyez-vous, Monsieur, v’là l’effet que ça produit.\par
EUGÉNIE : Oh ciel ! comme il s’allonge.\par
DOLMANCÉ : Que vos mouvements deviennent, à présent, plus réglés, plus énergiques… Cédez-moi la place un instant, et regardez bien comme je fais. {\itshape (Il branle Augustin.)} Voyez-vous comme ces mouvements-là sont plus fermes et en même temps plus moelleux… Là, reprenez, et surtout ne recalottez pas… Bon, le voilà dans toute son énergie ; examinons maintenant s’il est vrai qu’il l’ait plus gros que le Chevalier.\par
EUGÉNIE : N’en doutons pas, vous voyez bien que je ne puis l’empoigner.\par
DOLMANCÉ {\itshape mesure} : Oui, vous avez raison, treize de longueur sur huit et demi de circonférence ; je n’en ai jamais vu de plus gros ; voilà ce qu’on appelle un superbe vit ; et vous vous en servez, madame ?\par
MME DE SAINT-ANGE : Régulièrement toutes les nuits quand je suis à cette campagne.\par
DOLMANCÉ : Mais pas dans le cul, j’espère ?\par
MME DE SAINT-ANGE : Un peu plus souvent que dans le con.\par
DOLMANCÉ : Ah ! sacredieu, quel libertinage… Eh bien ! en honneur, je ne sais pas si je le soutiendrais.\par
MME DE SAINT-ANGE : Ne faites donc pas l’étroit, Dolmancé, il entrera dans votre cul comme dans le mien.\par
DOLMANCÉ : Nous verrons cela ; je me flatte que mon Augustin me fera l’honneur de me lancer un peu de foutre dans le derrière, je le lui rendrai ; mais continuons notre leçon… Allons, Eugénie, le serpent va vomir son venin, préparez-vous ; que vos yeux se fixent sur la tête de ce sublime membre ; et quand, pour preuve de sa prompte éjaculation, vous allez le voir se gonfler, se nuancer du plus beau pourpre, que vos mouvements alors acquièrent toute l’énergie dont ils sont susceptibles ; que les doigts qui chatouillent l’anus, s’y enfoncent le plus avant que faire se pourra ; livrez-vous tout entière au libertin qui s’amuse de vous ; cherchez sa bouche, afin de la sucer : que vos attraits volent, pour ainsi dire, au-devant de ses mains… il décharge, Eugénie, voilà l’instant de votre triomphe.\par
AUGUSTIN : Ahe, ahe, ahe, Mameselle, je me meurs… je ne puis plus, allez donc plus fort, je vous en conjure… Ah, sacrédié, je n’y vois plus clair !\par
DOLMANCÉ : Redoublez, redoublez, Eugénie, ne le ménagez plus, il est dans l’ivresse, ah, quelle abondance de sperme, avec quelle vigueur il s’est élancé, voyez les traces du premier jet, il a sauté à plus de dix pieds… Foutredieu, la chambre en est pleine, je n’ai jamais vu décharger comme cela, et il vous a, dites-vous, foutue, cette nuit, madame ?\par
MME DE SAINT-ANGE : Neuf ou dix coups, je crois, il y a longtemps que nous ne comptons plus.\par
LE CHEVALIER : Belle Eugénie, vous en êtes couverte.\par
EUGÉNIE : Je voudrais en être inondée. {\itshape (À Dolmancé :)} Eh bien ! mon maître, es-tu content ?\par
DOLMANCÉ : Fort bien pour un début ; mais il est encore quelques épisodes que vous avez négligés.\par
MME DE SAINT-ANGE : Attendons, ils ne peuvent être en elle que le fruit de l’expérience ; pour moi, je l’avoue, je suis fort contente de mon Eugénie ; elle annonce les plus heureuses dispositions, et je crois que nous devons maintenant la faire jouir d’un autre spectacle, faisons-lui voir les effets d’un vit dans le cul ; Dolmancé, je vais vous offrir le mien, je serai dans les bras de mon frère ; il m’enconnera ; vous m’enculerez, et c’est Eugénie qui préparera votre vit ; qui le placera dans mon cul, qui en réglera tous les mouvements ; qui les étudiera afin de se rendre familière à cette opération, que nous lui ferons ensuite subir à elle-même par l’énorme vit de cet hercule.\par
DOLMANCÉ : Je m’en flatte, et ce joli petit derrière sera bientôt déchiré sous nos yeux par les secousses violentes du brave Augustin, j’approuve en attendant ce que vous proposez, madame ; mais si vous voulez que je vous traite bien, permettez-moi d’y mettre une clause ; Augustin, que je vais faire rebander en deux tours de poignet, m’enculera, pendant que je vous sodomiserai.\par
MME DE SAINT-ANGE : J’approuve fort cet arrangement, j’y gagnerai, et ce sera pour mon écolière, deux excellentes leçons au lieu d’une.\par
DOLMANCÉ, {\itshape s’emparant d’Augustin} : Viens, mon gros garçon, que je te ranime… comme il est beau… Baise-moi, cher ami, tu es encore tout mouillé de foutre, et c’est du foutre que je te demande… Ah ! sacredieu ! il faut que je lui gamahuche le cul, tout en le branlant !\par
LE CHEVALIER : Approche, ma sœur, afin de répondre aux vues de Dolmancé et aux tiennes, je vais m’étendre sur ce lit, tu te coucheras dans mes bras, en lui exposant tes belles fesses dans le plus grand écartement possible… oui, c’est cela : nous pourrions toujours commencer.\par
DOLMANCÉ : Non pas vraiment, attendez-moi, il faut d’abord que j’encule ta sœur, puisque Augustin me l’insinue ; ensuite je vous marierai : ce sont mes doigts qui doivent vous lier ; ne manquons à aucun des principes, songeons qu’une écolière nous regarde, et que nous lui devons des leçons exactes ; Eugénie, venez me branler pendant que je détermine l’énorme engin de ce mauvais sujet ; soutenez l’érection de mon vit, en le polluant avec légèreté sur vos fesses…\par
{\itshape Elle exécute.}\par
EUGÉNIE : Fais-je bien ?\par
DOLMANCÉ : Il y a toujours trop de mollesse dans vos mouvements, serrez beaucoup plus le vit que vous branlez, Eugénie ; si la masturbation n’est agréable qu’en ce qu’elle comprime davantage que la jouissance, il faut donc que la main qui coopère, devienne pour l’engin qu’elle travaille, un local infiniment plus étroit qu’aucune autre partie du corps… Mieux, c’est mieux, cela, écartez le derrière un peu plus, afin qu’à chaque secousse la tête de mon vit touche au trou de votre cul ; oui, c’est cela, branle ta sœur en attendant ; Chevalier, nous sommes à toi dans la minute… Ah bon ! voilà mon homme qui bande… allons, préparez-vous, madame, ouvrez ce cul sublime à mon ardeur impure ; guide le dard Eugénie ; il faut que ce soit ta main qui le conduise sur la brèche ; il faut que ce soit elle qui le fasse pénétrer, dès qu’il sera dedans, tu t’empareras de celui d’Augustin, dont tu rempliras mes entrailles ; tout cela sont là des devoirs de novice, il y a de l’instruction à recevoir à tout cela ; voilà pourquoi je te le fais faire.\par
MME DE SAINT-ANGE : Mes fesses sont-elles bien à toi, Dolmancé ? Ah mon ange, si tu savais combien je te désire, combien il y a de temps que je veux être enculée par un bougre !\par
DOLMANCÉ : Vos vœux vont être exaucés, madame, mais souffrez que je m’arrête un instant aux pieds de l’idole : je veux la fêter avant que de m’introduire au fond de son sanctuaire… Quel cul divin !… que je le baise, que je le lèche mille et mille fois. Tiens, le voilà, ce vit que tu désires, le sens-tu coquine ? dis, dis ; sens-tu comme il pénètre ?\par
MME DE SAINT-ANGE : Ah ! mets-le-moi jusqu’au fond des entrailles… douce volupté, quel est donc ton empire ?\par
DOLMANCÉ : Voilà un cul comme je n’en foutis de mes jours ; il est digne de Ganymède lui-même ; allons, Eugénie, par vos soins qu’Augustin m’encule à l’instant.\par
EUGÉNIE : Le voilà, je vous l’apporte. {\itshape (À Augustin :)} Tiens, bel ange, vois-tu le trou qu’il te faut perforer ?\par
AUGUSTIN : Je le voyons bien… dame, y a de la place là, j’entrerai mieux là-dedans que chez vous, au moins, Mam’selle ; baisez-moi donc un peu pour qu’il entre mieux.\par
EUGÉNIE, {\itshape l’embrassant} : Oh ! tant que tu voudras, tu es si frais ; mais pousse donc… Comme la tête s’y est engloutie, tout de suite… Ah ! il me paraît que le reste ne tardera pas.\par
DOLMANCÉ : Pousse, pousse, mon ami ; déchire-moi, s’il le faut… Tiens, vois mon cul, comme il se prête… Ah ! sacredieu, quelle massue ! je n’en reçus jamais de pareil… combien reste-t-il de pouces au-dehors, Eugénie ?\par
EUGÉNIE : À peine deux.\par
DOLMANCÉ : J’en ai donc onze dans le cul… quelles délices… Il me crève, je n’en puis plus… Allons, Chevalier, es-tu prêt ?\par
LE CHEVALIER : Tâte, et dis ce que tu en penses.\par
DOLMANCÉ : Venez mes enfants, que je vous marie… que je coopère de mon mieux à ce divin inceste.\par
{\itshape Il introduit le vit du Chevalier dans le con de sa sœur.}\par
MME DE SAINT-ANGE : Ah ! mes amis, me voilà donc foutue des deux côtés… sacredieu, quel divin plaisir ! non, il n’en est pas de semblable au monde… Ah ! foutre, que je plains la femme qui ne l’a pas goûté ; secoue-moi, Dolmancé, secoue-moi ; force-moi, par la violence de tes mouvements à me précipiter sur le glaive de mon frère ; et toi, Eugénie, contemple-moi, viens me regarder dans le vice ; viens apprendre à mon exemple, à le goûter, avec transport, à le savourer avec délices… Vois mon amour, vois tout ce que je fais à la fois, scandale, séduction, mauvais exemple, inceste, adultère, sodomie… Ô Lucifer ! seul et unique dieu de mon âme, inspire-moi quelque chose de plus, offre à mon cœur de nouveaux écarts, et tu verras comme je m’y plongerai !\par
DOLMANCÉ : Voluptueuse créature comme tu détermines mon foutre, comme tu en presses la décharge par tes propos et l’extrême chaleur de ton cul… tout va me faire partir à l’instant. Eugénie, échauffe le courage de mon fouteur ; presse ses flancs, entrouvre ses fesses ; tu connais maintenant l’art de ranimer des désirs vacillants… Ta seule approche donne de l’énergie au vit qui me fout… Je le sens, ses secousses sont plus vives… Friponne, il faut que je te cède ce que je n’aurais voulu devoir qu’à mon cul. Chevalier, tu t’emportes, je le sens… attends-moi… attends-nous. Ô mes amis, ne déchargeons qu’ensemble, c’est le seul bonheur de la vie.\par
MME DE SAINT-ANGE : Ah ! foutre… foutre, partez quand vous voudrez… pour moi, je n’y tiens plus ! Double nom d’un dieu, dont je me fous… sacré bougre de dieu ! je décharge… inondez-moi, mes amis… inondez votre putain, lancez les flots de votre foutre écumeux, jusqu’au fond de son âme embrasée, elle n’existe que pour les recevoir… ahe, ahe, ahe, foutre… foutre, quel incroyable excès de volupté, je me meurs ; Eugénie, que je te baise, que je te mange… que je dévore ton foutre, en perdant le mien.\par
{\itshape Augustin, Dolmancé et le Chevalier font chorus, la crainte d’être monotone nous empêche de rendre des expressions qui, dans de tels instants, se ressemblent toutes.}\par
DOLMANCÉ : Voilà une des bonnes jouissances que j’aie eues de ma vie. {\itshape (Montrant Augustin :)} Ce bougre-là m’a rempli de sperme… mais je vous l’ai bien rendu, madame.\par
MME DE SAINT-ANGE : Ah ! ne m’en parlez pas, j’en suis inondée.\par
EUGÉNIE : Je n’en peux pas dire autant, moi. {\itshape (Se jetant en folâtrant dans les bras de son amie :)} Tu dis que tu as fait bien des péchés, ma bonne, mais pour moi, dieu merci, pas un seul ; ah ! si je mange longtemps mon pain à la fumée, comme cela, je n’aurai pas d’indigestion.\par
MME DE SAINT-ANGE, {\itshape éclatant de rire} : La drôle de créature !\par
DOLMANCÉ : Elle est charmante, venez ici, petite fille, que je vous fouette. {\itshape (Il lui claque le cul.)} Baisez-moi, vous aurez bientôt votre tour.\par
MME DE SAINT-ANGE : Il ne faut à l’avenir s’occuper que d’elle seule, mon frère, considère-la, c’est ta proie… examine ce charmant pucelage, il va bientôt t’appartenir.\par
EUGÉNIE : Oh ! non, pas par-devant, cela me ferait trop de mal, par-derrière tant que vous voudrez, comme Dolmancé me l’a fait tout à l’heure.\par
MME DE SAINT-ANGE : La naïve et délicieuse fille… Elle vous demande précisément ce qu’on a tant de peine à obtenir des autres.\par
EUGÉNIE : Oh ! ce n’est pas sans un peu de remords ; car vous ne m’avez point rassurée sur le crime énorme que j’ai toujours entendu dire qu’il y avait à cela, et surtout à le faire d’homme à homme, comme cela vient d’arriver à Dolmancé et à Augustin ; voyons, voyons, monsieur, comment votre philosophie explique cette sorte de délit. Il est affreux, n’est-ce pas ?\par
DOLMANCÉ : Commencez à partir d’un point, Eugénie, c’est que rien n’est affreux en libertinage, parce que tout ce que le libertinage inspire l’est également par la nature ; les actions les plus extraordinaires, les plus bizarres, celles qui paraissent choquer le plus évidemment toutes les lois, toutes les institutions humaines (car pour du ciel, je n’en parle pas), eh bien ! Eugénie, celles-là même ne sont point affreuses, et il n’en est pas une d’elles qui ne puisse se démontrer dans la nature ; il est certain que celle dont vous me parlez, belle Eugénie, est la même relativement à laquelle on trouve une fable si singulière dans le plat roman de l’Écriture sainte, fastidieuse compilation d’un juif ignorant, pendant la captivité de Babylone, mais il est faux, hors de toute vraisemblance, que ce soit en punition de ces écarts que ces villes, ou plutôt ces bourgades aient péri par le feu ; placées sur le cratère de quelques anciens volcans, Sodome, Gomorrhe, périrent comme ces villes de l’Italie qu’engloutirent les laves du Vésuve ; voilà tout le miracle, et ce fut pourtant de cet événement tout simple que l’on partit pour inventer barbarement le supplice du feu, contre les malheureux humains qui se livreraient dans une partie de l’Europe à cette naturelle fantaisie.\par
EUGÉNIE : Oh, naturelle !\par
DOLMANCÉ : Oui, naturelle, je le soutiens, la nature n’a pas deux voix, dont l’une fasse journellement le métier de condamner ce que l’autre inspire, et il est bien certain que ce n’est que par son organe, que les hommes entichés de cette manie reçoivent les impressions qui les y portent. Ceux qui veulent proscrire ou condamner ce goût, prétendent qu’il nuit à la population ; qu’ils sont plats, ces imbéciles qui n’ont jamais que cette idée de population dans la tête, et qui ne voient jamais que du crime à tout ce qui s’éloigne de là ; est-il donc démontré que la nature ait de cette population un aussi grand besoin qu’ils voudraient nous le faire croire ? est-il bien certain qu’on l’outrage chaque fois qu’on s’écarte de cette stupide propagation ? Scrutons un instant, pour nous en convaincre, et sa marche et ses lois. Si la nature ne faisait que créer, et qu’elle ne détruisît jamais, je pourrais croire avec ces fastidieux sophistes que le plus sublime de tous les actes serait de travailler sans cesse à celui qui produit, et je leur accorderais à la suite de cela que le refus de produire devrait nécessairement être un crime, mais le plus léger coup d’œil sur les opérations de la nature ne prouve-t-il pas que les destructions sont aussi nécessaires à ses plans que les créations ; que l’une et l’autre de ces opérations se lient et s’enchaînent même si intimement qu’il devient impossible que l’une puisse agir sans l’autre ; que rien ne naîtrait, rien ne se régénérerait sans des destructions ? La destruction est donc une des lois de la nature comme la création ; ce principe admis, comment puis-je offenser cette nature, en refusant de créer ; ce qui, à supposer un mal à cette action, en deviendrait un infiniment moins grand, sans doute, que celui de détruire qui, pourtant se trouve dans ses lois, ainsi que je viens de le prouver ; si d’un côté j’admets donc le penchant que la nature me donne à cette perte, que j’examine de l’autre qu’il lui est nécessaire, et que je ne fais qu’entrer dans ses vues en m’y livrant ; où sera le crime alors, je vous le demande ? Mais, vous objectent encore les sots et les populateurs, ce qui est synonyme, ce sperme productif ne peut être placé dans vos reins à aucun autre usage que pour celui de la propagation, l’en détourner est une offense, je viens d’abord de prouver que non, puisque cette perte n’équivaudrait même pas à une destruction et que la destruction bien plus importante que la perte ne serait pas elle-même un crime ; secondement il est faux que la nature veuille que cette liqueur spermatique soit absolument et entièrement destinée à produire, si cela était, non seulement, elle ne permettrait pas que cet écoulement eût lieu dans tout autre cas, comme nous le prouve l’expérience, puisque nous la perdons, et quand nous voulons et où nous voulons, et ensuite elle s’opposerait à ce que ces pertes eussent lieu sans coït, comme il arrive et dans nos rêves et dans nos souvenirs ; avare d’une liqueur aussi précieuse, ce ne serait jamais que dans le vase de la propagation qu’elle en permettrait l’écoulement ; elle ne voudrait assurément pas que cette volupté dont elle nous couronne alors, pût être ressentie, quand nous détournerions l’hommage ; car il ne serait pas raisonnable de supposer qu’elle consentît à nous donner du plaisir même au moment où nous l’accablerions d’outrages ; allons plus loin ; si les femmes n’étaient nées que pour produire, ce qui serait assurément, si cette production était si chère à la nature, arriverait-il que, sur la plus longue vie d’une femme, il ne se trouve cependant que sept ans, toute déduction faite, où elle soit en état de donner la vie à son semblable ? Quoi, la nature est avide de propagations, tout ce qui ne tend pas à ce but l’offense ; et sur cent ans de vie, le sexe destiné à produire ne le pourra que pendant sept ans ? La nature ne veut que des propagations et la semence qu’elle prête à l’homme pour servir ces propagations, se perd tant qu’il plaît à l’homme ; il trouve le même plaisir à cette perte qu’à l’emploi utile, et jamais le moindre inconvénient ?… Cessons, mes amis, cessons de croire à de telles absurdités ; elles font frémir le bon sens ; ah ! loin d’outrager la nature, persuadons-nous bien au contraire que le sodomite et la tribade la servent, en se refusant opiniâtrement à une conjonction, dont il ne résulte qu’une progéniture fastidieuse pour elle. Cette propagation, ne nous trompons point, ne fut jamais une de ses lois, mais une tolérance tout au plus, je vous l’ai dit ; et que lui importe que la race des hommes s’éteigne ou s’anéantisse sur la terre ; elle rit de notre orgueil à nous persuader que tout finirait si ce malheur avait lieu ; mais elle ne s’en apercevrait seulement pas. S’imagine-t-on qu’il n’y ait pas déjà des races éteintes ; Buffon en compte plusieurs, et la nature muette à une perte aussi précieuse, ne s’en aperçoit seulement pas, l’espèce entière s’anéantirait que ni l’air n’en serait moins pur, ni l’astre moins brillant, la marche de l’univers moins exacte. Qu’il fallait d’imbécillité, cependant, pour croire que notre espèce est tellement utile au monde, que celui qui ne travaillerait pas à la propager ou celui qui troublerait cette propagation, devînt nécessairement un criminel. Cessons de nous aveugler à ce point ; et que l’exemple des peuples plus raisonnables que nous, serve à nous persuader de nos erreurs ; il n’y a pas un seul coin sur la terre où ce prétendu crime de sodomie n’ait eu des temples et des sectateurs, les Grecs, qui en faisaient pour ainsi dire une vertu, lui érigèrent une statue sous le nom de Vénus Callipyge ; Rome envoya chercher des lois à Athènes, et elle en rapporta ce goût divin. Quel progrès ne lui voyons-nous pas faire sous les empereurs, à l’abri des aigles romaines, il s’étend d’un bout de la terre à l’autre, à la destruction de l’empire, il se réfugie près de la tiare, il suit les arts en Italie, il nous parvient quand nous nous poliçons. Découvrons-nous un hémisphère, nous y trouvons la sodomie. Cook mouille dans un nouveau monde, elle y règne ; si nos ballons eussent été dans la lune, elle s’y serait trouvée tout de même. Goût délicieux, enfant de la nature et du plaisir, vous devez être partout où se trouveront des hommes, et partout où l’on vous aura connu, l’on vous érigera des autels ; ô mes amis, peut-il être une extravagance pareille à celle d’imaginer qu’un homme doit être un monstre digne de perdre la vie, parce qu’il a préféré dans sa jouissance le trou d’un cul à celui d’un con, parce qu’un jeune homme avec lequel il trouve deux plaisirs, celui d’être à la fois amant et maîtresse, lui a paru préférable à une fille qui ne lui promet qu’une jouissance ; il sera un scélérat, un monstre ; pour avoir voulu jouer le rôle d’un sexe qui n’est pas le sien, et pourquoi la nature l’a-t-elle créé sensible à ce plaisir ? Examinez sa conformation ; vous y observerez des différences totales avec celle des hommes qui n’ont pas reçu ce goût en partage ; ses fesses seront plus blanches, plus potelées ; pas un poil n’ombragera l’autel du plaisir dont l’intérieur tapissé d’une membrane plus délicate, plus sensuelle, plus chatouilleuse, se trouvera positivement du même genre que l’intérieur du vagin d’une femme ; le caractère de cet homme encore différent de celui des autres, aura plus de mollesse, plus de flexibilité ; vous lui trouverez presque tous les vices et toutes les vertus d’une femme. Vous y reconnaîtrez jusqu’à leur faiblesse ; tous auront leur manie et quelques-uns de leurs traits. Serait-il donc possible que la nature, en les assimilant de cette manière à des femmes, pût s’irriter de ce qu’ils ont leurs goûts ? n’est-il pas clair que c’est une classe d’hommes différente de l’autre, et que la nature créa ainsi pour diminuer cette propagation dont la trop grande étendue lui nuirait infailliblement… ah ma chère Eugénie, si vous saviez comme on jouit délicieusement, quand un gros vit nous remplit le derrière, lorsque enfoncé jusqu’aux couillons, il s’y trémousse avec ardeur ; que ramené jusqu’au prépuce, il s’y renfonce jusqu’au poil ; non, non, il n’est point dans le monde entier une jouissance qui vaille celle-là : c’est celle des philosophes, c’est celle des héros, ce serait celle des dieux, si les parties de cette divine jouissance n’étaient pas elles-mêmes les seuls dieux que nous devions adorer sur la terre\footnote{ La suite de cet ouvrage nous promettant une dissertation bien plus étendue sur celle matière, on s’est borné ici à la plus légère analyse.} !\par
EUGÉNIE, {\itshape très animée} : Oh ! mes amis, que l’on m’encule… Tenez, voilà mes fesses… je vous les offre… Foutez-moi, je décharge !\par
{\itshape Elle tombe en prononçant ces mots, dans les bras de M\textsuperscript{me} de Saint-Ange qui la serre, l’embrasse et offre les reins élevés de cette jeune fille à Dolmancé. }\par
MME DE SAINT-ANGE : Divin instituteur, résisterez-vous à cette proposition ? Ce sublime derrière ne vous tentera-t-il pas ; voyez comme il bâille, et comme il s’entrouvre !\par
DOLMANCÉ : Je vous demande pardon, belle Eugénie ; ce ne sera pas moi, si vous le voulez bien, qui me chargerai d’éteindre les feux que j’allume. Chère enfant, vous avez à mes yeux le grand tort d’être femme. J’ai bien voulu oublier toute prévention pour cueillir vos prémices, trouvez bon que j’en reste là ; le Chevalier va se charger de la besogne ; sa sœur, armée de ce godemiché, portera au cul de son frère les coups les plus redoutables, tout en présentant son beau derrière à Augustin, qui l’enculera et que je foutrai pendant ce temps-là ; car, je ne vous le cache pas, le cul de ce beau garçon me tente depuis une heure et je veux absolument lui rendre ce qu’il m’a fait.\par
EUGÉNIE : J’adopte le change, mais en vérité, Dolmancé, la franchise de votre aveu n’en soustrait pas l’impolitesse.\par
DOLMANCÉ : Mille pardons, mademoiselle ; mais nous autres bougres, nous ne nous piquons que de franchise et d’exactitude dans nos principes.\par
MME DE SAINT-ANGE : La réputation de franchise n’est pourtant pas celle que l’on donne à ceux qui, comme vous, sont accoutumés à ne prendre les gens que par-derrière.\par
DOLMANCÉ : Un peu traître… oui, un peu faux ; vous croyez. Eh bien ! madame, je vous ai démontré que ce caractère était indispensable dans la société, condamnés à vivre avec des gens qui ont le plus grand intérêt à se cacher à nos yeux, à nous déguiser les vices qu’ils ont, pour ne nous offrir que les vertus qu’ils n’encensèrent jamais ; il y aurait à nous, le plus grand danger à ne leur montrer que de la franchise car alors il est clair que vous leur donneriez sur vous tous les avantages qu’ils vous refusent, et la duperie serait manifeste ; la dissimulation et l’hypocrisie sont des besoins que la société nous a faits : cédons-y. Permettez-moi de m’offrir à vous un instant pour exemple ; madame, il n’est assurément dans le monde aucun être plus corrompu, eh bien ! mes contemporains s’y trompent : demandez-leur ce qu’ils pensent de moi, tous vous diront que je suis un honnête homme, tandis qu’il n’est pas un seul crime dont je n’aie fait mes plus chères délices.\par
MME DE SAINT-ANGE : Oh ! vous ne me persuaderez pas que vous en ayez commis d’atroces.\par
DOLMANCÉ : D’atroces…, en vérité, madame, j’ai fait des horreurs.\par
MME DE SAINT-ANGE : Eh bien ! oui, vous êtes comme celui qui disait à son confesseur : « Le détail est inutile, monsieur, excepté le meurtre et le vol, vous pouvez être sûr que j’ai tout fait. »\par
DOLMANCÉ : Oui, madame, je dirai la même chose, mais à l’exception près.\par
MME DE SAINT-ANGE : Quoi, libertin, vous vous êtes permis…\par
DOLMANCÉ : Tout, madame, tout ; se refuse-t-on quelque chose avec mon tempérament et mes principes ?\par
MME DE SAINT-ANGE : Ah ! foutons, foutons ; je ne puis plus tenir à ces propos ; nous y reviendrons, Dolmancé ; mais, pour ajouter plus de foi à vos aveux, je ne veux les entendre qu’à {\itshape tête fraîche} ; quand vous bandez, vous aimez à dire des horreurs, et peut-être nous donneriez-vous ici pour des vérités, les libertins prestiges de votre imagination enflammée.\par
{\itshape On s’arrange.}\par
DOLMANCÉ : Attends, Chevalier, attends ; c’est moi-même qui vais l’introduire ; mais il faut préalablement, j’en demande pardon à la belle Eugénie, il faut qu’elle me permette de la fouetter pour la mettre en train.\par
{\itshape Il la fouette.}\par
EUGÉNIE : Je vous réponds que cette cérémonie était inutile… Dites, Dolmancé, qu’elle satisfait votre luxure ; mais, en y procédant, n’ayez pas l’air, je vous prie, de rien faire pour moi.\par
DOLMANCÉ, {\itshape toujours fouettant} : Ah ! tout à l’heure, vous m’en direz des nouvelles ; vous ne connaissez pas l’empire de ce préliminaire… allons, allons, petite coquine, vous serez fustigée.\par
EUGÉNIE : Oh, ciel ! comme il y va ; mes fesses sont en feu ; mais vous me faites mal, en vérité.\par
MME DE SAINT-ANGE : Je vais te venger, ma mie ; je vais le lui rendre.\par
{\itshape Elle fouette Dolmancé.}\par
DOLMANCÉ : Oh ! de tout mon cœur ; je ne demande qu’une grâce à Eugénie, c’est de trouver bon que je la fouette aussi fort que je désire l’être moi-même ; vous voyez comme me voilà dans la loi de la nature ; mais attendez, arrangeons cela, qu’Eugénie monte sur vos reins, madame ; elle s’accrochera à votre col, comme ces mères qui portent leurs enfants sur leur dos ; là j’aurai deux culs sous ma main ; je les étrillerai ensemble ; le Chevalier et Augustin me le rendront, en frappant à la fois tous deux sur mes fesses… Oui, c’est aussi, ah ! nous y voilà !… Quelles délices !\par
MME DE SAINT-ANGE : N’épargnez pas cette petite coquine, je vous en conjure, et comme je ne vous demande point de grâce, je ne veux pas que vous lui en fassiez aucune.\par
EUGÉNIE : Ahe ! ahe ! ahe ! en vérité, je crois que mon sang coule.\par
MME DE SAINT-ANGE : Il embellira mes fesses en les colorant…, courage, mon ange, courage ; souviens-toi que c’est par les peines qu’on arrive toujours aux plaisirs.\par
EUGÉNIE : En vérité, je n’en puis plus.\par
DOLMANCÉ {\itshape suspend une minute pour contempler son ouvrage ; puis reprenant} : Encore une soixantaine, Eugénie, oui, oui, soixante encore sur chaque cul… Oh ! coquines comme vous allez avoir du plaisir à foutre maintenant !\par
{\itshape La posture se défait.}\par
MME DE SAINT-ANGE, {\itshape examinant les fesses d’Eugénie} : Ah ! la pauvre petite, son derrière est en sang ! scélérat, comme tu as du plaisir à baiser ainsi les vestiges de ta cruauté !\par
DOLMANCÉ, {\itshape se polluant} : Oui, je ne le cache pas, et mes baisers seraient plus ardents, si les vestiges étaient plus cruels.\par
EUGÉNIE : Ah ! vous êtes un monstre !\par
DOLMANCÉ : J’en conviens.\par
LE CHEVALIER : Il y a de la bonne foi au moins.\par
DOLMANCÉ : Allons, sodomise-la, Chevalier…\par
LE CHEVALIER : Contiens ses reins, et dans trois secousses, il y est.\par
EUGÉNIE : Oh ciel, vous l’avez plus gros que Dolmancé ; Chevalier, vous me déchirez… ménagez-moi, je vous en conjure.\par
LE CHEVALIER : Cela est impossible, mon ange, je dois atteindre le but… Songez que je suis ici sous les yeux de mon maître ; il faut que je me rende digne de ses leçons.\par
DOLMANCÉ : Il y est, j’aime prodigieusement à voir le poil d’un vit frotter les parois d’un anus… Allons, madame, enculez votre frère… Voilà le vit d’Augustin tout prêt à s’introduire en vous et moi, je vous réponds de ne pas ménager votre fouteur… Ah ! bon, il me semble que voilà le chapelet forme ; ne pensons plus qu’à décharger maintenant.\par
MME DE SAINT-ANGE : Examinez donc cette petite gueuse, comme elle frétille.\par
EUGÉNIE : Est-ce ma faute ; je meurs de plaisirs… Cette fustigation… ce vit immense… et cet aimable Chevalier qui me branle encore pendant ce temps-là… ma bonne, ma bonne, je n’en puis plus !\par
MME DE SAINT-ANGE : Sacredieu, je t’en livre autant, je décharge !\par
DOLMANCÉ : Un peu d’ensemble, mes amis ; si vous vouliez seulement m’accorder deux minutes, je vous aurais bientôt atteints, et nous partirions tous à la fois.\par
LE CHEVALIER : Il n’est plus temps, mon foutre coule dans le cul de la belle Eugénie… je me meurs… ah ! sacré nom d’un Dieu, que de plaisirs !\par
DOLMANCÉ : Je vous suis, mes amis… je vous suis ; le foutre m’aveugle également…\par
AUGUSTIN : Et moi donc !… Et moi donc !\par
MME DE SAINT-ANGE : Quelle scène !… Ce bougre-là m’a rempli le cul.\par
LE CHEVALIER : Au bidet, mesdames, au bidet !\par
MME DE SAINT-ANGE : Non, en vérité, j’aime cela moi, j’aime à me sentir du foutre dans le cul, je ne le rends jamais quand j’en ai.\par
EUGÉNIE : En vérité, je n’en puis plus… dites-moi maintenant, mes amis, si une femme doit toujours accepter la proposition d’être ainsi foutue quand on la lui fait ?\par
MME DE SAINT-ANGE : Toujours, ma chère, toujours elle doit faire plus ; même comme cette manière de foutre est délicieuse, elle doit l’exiger de ceux dont elle se sert, mais si elle dépend de celui avec lequel elle s’amuse, si elle espère en obtenir des faveurs, des présents ou des grâces, qu’elle se fasse valoir, qu’elle se fasse presser ; il n’y a pas d’homme de ce goût, qui dans pareil cas, ne se ruine avec une femme assez adroite pour ne lui faire de refus qu’avec le dessein de l’enflammer davantage ; elle en tirera tout ce qu’elle voudra, si elle possède bien l’art de n’accorder qu’à propos ce qu’on lui demande.\par
DOLMANCÉ : Eh bien ! petit ange, es-tu convertie ; cesses-tu de croire que la sodomie soit un crime ?\par
EUGÉNIE : Et quand elle en serait un, que m’importe ? Ne m’avez-vous pas démontré le néant des crimes ? Il est bien peu d’actions maintenant qui soient criminelles à mes yeux.\par
DOLMANCÉ : Il n’est de crime à rien, chère fille, à quoi que ce soit au monde, la plus monstrueuse des actions n’a-t-elle pas un côté par lequel elle nous est propice ?\par
EUGÉNIE : Qui en doute ?\par
DOLMANCÉ : Eh bien ! de ce moment elle cesse d’être un crime ; car pour que ce qui sert l’un, en nuisant à l’autre, fût un crime, il faudrait démontrer que l’être lésé est plus précieux à la nature que l’être servi : or tous les individus étant égaux aux yeux de la nature, cette prédilection est impossible ; donc l’action qui sert à l’un en nuisant à l’autre est d’une indifférence parfaite à la nature.\par
EUGÉNIE : Mais si l’action nuisait à une très grande quantité d’individus, et qu’elle ne nous rapportât à nous, qu’une très légère dose de plaisir, ne serait-il pas affreux de s’y livrer alors ?\par
DOLMANCÉ : Pas davantage, parce qu’il n’y a aucune comparaison entre ce qu’éprouvent les autres et ce que nous ressentons : la plus forte dose de douleur chez les autres doit assurément être nulle pour nous, et le plus léger chatouillement de plaisir, éprouvé par nous, nous touche ; donc nous devons à tel prix que ce soit, préférer ce léger chatouillement qui nous délecte, à cette somme immense des malheurs d’autrui, qui ne saurait nous atteindre ; mais s’il arrive au contraire que la singularité de nos organes, une construction bizarre, nous rendent agréables les douleurs du prochain, ainsi que cela arrive souvent, qui doute alors que nous ne devions incontestablement préférer cette douleur d’autrui qui nous amuse à l’absence de cette douleur qui deviendrait une privation pour nous ? La source de toutes nos erreurs en morale vient de l’admission ridicule de ce fil de fraternité qu’inventèrent les chrétiens, dans leur siècle d’infortune et de détresse ; contraints à mendier la pitié des autres, il n’était pas maladroit d’établir qu’ils étaient tous frères ; comment refuser des secours d’après une telle hypothèse ; mais il est impossible d’admettre cette doctrine ! Ne naissons-nous pas tous isolés ; je dis plus, tous ennemis les uns des autres, tous dans un état de guerre perpétuelle et réciproque ? Or je vous demande si cela serait, dans la supposition que les vertus exigées par ce prétendu fil de fraternité fussent réellement dans la nature ; si sa voix les inspirait aux hommes, ils les éprouveraient dès en naissant, dès lors, la pitié, la bienfaisance, l’humanité seraient des vertus naturelles dont il serait impossible de se défendre, et qui rendraient cet état primitif de l’homme sauvage totalement contraire à ce que nous le voyons.\par
EUGÉNIE : Mais si, comme vous le dites, la nature fait naître les hommes isolés, tous indépendants les uns des autres, au moins m’accorderez-vous que les besoins, en les rapprochant, ont dû nécessairement établir quelques liens entre eux ; de là, ceux du sang nés de leur alliance réciproque, ceux de l’amour, de l’amitié, de la reconnaissance ; vous respecterez au moins ceux-là, j’espère ?\par
DOLMANCÉ : Pas plus que les autres, en vérité ; mais analysons-les, je le veux, un coup d’œil rapide, Eugénie, sur chacun en particulier ; direz-vous, par exemple, que le besoin de me marier ou pour voir prolonger ma race, ou pour arranger ma fortune, doit établir des liens indissolubles ou sacrés avec l’objet auquel je m’allie ; ne serait-ce pas, je vous le demande, une absurdité que de soutenir cela ; tant que dure l’acte du coït, je peux, sans doute, avoir besoin de cet objet pour y participer ; mais sitôt qu’il est satisfait, que reste-t-il, je vous prie, entre lui et moi ? et quelle obligation réelle enchaînera à lui ou à moi les résultats de ce coït ? ces derniers liens furent les fruits de la frayeur qu’eurent les parents d’être abandonnés dans leur vieillesse, et les soins intéressés qu’ils ont de nous dans notre enfance, ne sont que pour mériter ensuite les mêmes attentions dans leur dernier âge ; cessons d’être la dupe de tout cela, nous ne devons rien à nos parents… Pas la moindre chose, Eugénie, et comme c’est bien moins pour nous que pour eux qu’ils ont travaillé, il nous est permis de les détester, et de nous en défaire même, si leur procédé nous irrite, nous ne devons les aimer que s’ils agissent bien avec nous, et cette tendresse, alors, ne doit pas avoir un degré de plus que celle que nous aurions pour d’autres amis, parce que les droits de la naissance n’établissent rien, ne fondent rien, et qu’en les scrutant avec sagesse et réflexion, nous n’y trouverons sûrement que des raisons de haine pour ceux qui ne songeant qu’à leurs plaisirs, ne nous ont donné souvent qu’une existence malheureuse ou malsaine ; vous me parlez des liens de l’amour, Eugénie, puissiez-vous ne les jamais connaître, ah ! qu’un tel sentiment, pour le bonheur que je vous souhaite, n’approche jamais de votre cour ; qu’est-ce que l’amour ? On ne peut le considérer, ce me semble, que comme l’effet résultatif des qualités d’un bel objet sur nous : ces effets nous transportent ; ils nous enflamment, si nous possédons cet objet, nous voilà contents, s’il nous est impossible de l’avoir, nous nous désespérons ; mais quelle est la base de ce sentiment ?… Le désir : quelles sont les suites de ce sentiment ? La folie ; tenons-nous-en donc au motif, et garantissons-nous des effets ; le motif est de posséder l’objet ; eh bien ! tâchons de réussir, mais avec sagesse ; jouissons-en, dès que nous l’avons ; consolons-nous : dans le cas contraire, mille autres objets semblables, et souvent bien meilleurs, nous consoleront de la perte de celui-là ; tous les hommes, toutes les femmes se ressemblent, il n’y a point d’amour qui résiste aux effets d’une réflexion saine : oh ! quelle duperie que cette ivresse qui, absorbant en nous le résultat des sens, nous met dans un tel état que nous ne voyons plus, que nous n’existons plus que par cet objet follement adoré ; est-ce donc là vivre ; n’est-ce pas bien plutôt se priver volontairement de toutes les douceurs de la vie ? N’est-ce pas vouloir rester dans une fièvre brûlante qui nous absorbe et qui nous dévore, sans nous laisser d’autre bonheur que des jouissances métaphysiques si ressemblantes aux effets de la folie : si nous devions toujours l’aimer cet objet adorable, s’il était certain que nous ne dussions jamais l’abandonner, ce serait encore une extravagance, sans doute, mais excusable au moins : cela arrive-t-il ? a-t-on beaucoup d’exemples de ces liaisons éternelles qui ne se sont jamais démenties ? Quelques mois de jouissance remettant bientôt l’objet à sa véritable place, nous font rougir de l’encens que nous avons brûlé sur ses autels, et nous arrivons souvent à ne pas même concevoir qu’il ait pu nous séduire à ce point. Ô filles voluptueuses, livrez-nous donc vos corps tant que vous le pourrez ! foutez, divertissez-vous, voilà l’essentiel mais fuyez avec soin l’amour, il n’y a de bon que son physique, disait le naturaliste {\itshape Buffon}, et ce n’était pas sur cela seul qu’il raisonnait en bon philosophe. Je le répète, amusez-vous : mais n’aimez point, ne vous embarrassez pas davantage de l’être : ce n’est pas de s’exténuer en lamentations, en soupirs, en œillades, en billets doux qu’il faut, c’est de foutre, c’est de multiplier et de changer souvent ses fouteurs, c’est de s’opposer fortement surtout à ce qu’un seul veuille vous captiver parce que le but de ce constant amour serait en vous liant à lui, de vous empêcher de vous livrer à un autre, égoïsme cruel qui deviendrait bientôt fatal à vos plaisirs. Les femmes ne sont pas faites pour un seul homme, c’est pour tous que les a créées la nature, n’écoutant que cette voix sacrée, qu’elles se livrent indifféremment à tous ceux qui veulent d’elles, toujours putains, jamais amantes, fuyant l’amour, adorant le plaisir, ce ne seront plus que des roses qu’elles trouveront dans la carrière de la vie : ce ne seront plus que des fleurs qu’elles nous prodigueront ; demandez, Eugénie, demandez à la femme charmante qui veut bien se charger de votre éducation, le cas qu’il faut faire d’un homme quand on en a joui. {\itshape (Assez bas pour n’être pas entendu d’Augustin :)} Demandez-lui si elle ferait un pas pour conserver cet Augustin qui fait aujourd’hui ses délices, dans l’hypothèse où l’on voudrait le lui enlever : elle en prendrait un autre, ne penserait plus à celui-ci, et bientôt lasse du nouveau, elle l’immolerait elle-même dans deux mois si de nouvelles jouissances devaient naître de ce sacrifice.\par
MME DE SAINT-ANGE : Que ma chère Eugénie soit bien sûre que Dolmancé lui explique ici mon cœur ainsi que celui de toutes les femmes, comme si nous lui en ouvrions les replis.\par
DOLMANCÉ : La dernière partie de mon analyse porte donc sur les liens de l’amitié et sur ceux de la reconnaissance : respectons les premiers, j’y consens tant qu’ils nous sont utiles ; gardons nos amis tant qu’ils nous servent ; oublions-les dès que nous n’en tirons plus rien ; ce n’est jamais que pour soi qu’il faut aimer les gens : les aimer pour eux-mêmes n’est qu’une duperie, jamais il n’est dans la nature d’inspirer aux hommes d’autres mouvements, d’autres sentiments que ceux qui doivent leur être bons à quelque chose ; rien n’est égoïste comme la nature, soyons-le donc aussi, si nous voulons accomplir ses lois.\par
Quant à la reconnaissance, Eugénie, c’est le plus faible de tous les liens sans doute. Est-ce donc pour nous que les hommes nous obligent ; n’en croyons rien, ma chère ; c’est par ostentation, par orgueil ; n’est-il donc pas humiliant dès lors, de devenir ainsi le jouet de l’amour-propre des autres ? Ne l’est-il pas encore davantage d’être obligé ? Rien de plus à charge qu’un bienfait reçu ; point de milieu, il faut le rendre, ou en être avili : les âmes fières se font mal au poids du bienfait ; il pèse sur elles avec tant de violence que le seul sentiment qu’elles exhalent est de la haine pour le bienfaiteur.\par
Quels sont donc maintenant, à votre avis, les liens qui suppléent à l’isolement où nous a créés la nature ; quels sont ceux qui doivent établir des rapports entre les hommes, à quels titres les aimerons-nous : les chérirons-nous, les préférerons-nous à nous-mêmes ; de quel droit soulagerons-nous leur infortune ? Où sera maintenant dans nos âmes le berceau de vos belles et inutiles vertus de bienfaisance, d’humanité, de charité, indiquées dans le code absurde de quelques religions imbéciles, qui, prêchées par des imposteurs ou par des mendiants, durent nécessairement conseiller ce qui pouvait les soutenir ou les tolérer ?\par
Eh bien ! Eugénie, admettez-vous encore quelque chose de sacré parmi les hommes ? Concevez-vous quelques raisons de ne pas toujours nous préférer à eux ?\par
EUGÉNIE : Ces leçons que mon cœur devance, me flattent trop pour que mon esprit les récuse.\par
MME DE SAINT-ANGE : Elles sont dans la nature, Eugénie ; la seule approbation que tu leur donnes, le prouve ; à peine éclose de son sein, comment ce que tu sens, pourrait-il être le fruit de la corruption ?\par
EUGÉNIE : Mais si toutes les erreurs que vous préconisez, sont dans la nature, pourquoi les lois s’y opposent-elles ?\par
DOLMANCÉ : Parce que les lois ne sont pas faites pour le particulier, mais pour le général, ce qui les met dans une perpétuelle contradiction avec l’intérêt personnel, attendu que l’intérêt personnel l’est toujours avec l’intérêt général. Mais les lois bonnes pour la société, sont très mauvaises pour l’individu qui la compose : car pour une fois qu’elles le protègent ou le garantissent, elles le gênent et le captivent les trois quarts de sa vie ; aussi l’homme sage et plein de mépris pour elles les tolère-t-il, comme il fait des serpents et des vipères qui bien qu’elles blessent ou qu’elles empoisonnent, servent pourtant quelquefois dans la médecine ; il se garantira des lois comme il fera de ces bêtes venimeuses ; il s’en mettra à l’abri par des précautions, par des mystères, toutes choses faciles à la richesse et à la prudence. Que la fantaisie de quelques crimes vienne enflammer votre âme, Eugénie, et soyez bien certaine de les commettre en paix entre votre amie et moi.\par
EUGÉNIE : Ah ! cette fantaisie est déjà dans mon cœur.\par
MME DE SAINT-ANGE : Quel caprice t’agite, Eugénie ? dis-nous-le avec confiance !\par
EUGÉNIE, {\itshape égarée} : Je voudrais une victime.\par
MME DE SAINT-ANGE : Et de quel sexe la désires-tu ?\par
EUGÉNIE : Du mien.\par
DOLMANCÉ : Eh bien ! madame, êtes-vous contente de votre élève ; ses progrès sont-ils assez rapides ?\par
EUGÉNIE {\itshape (comme ci-dessus)} : Une victime, ma bonne, une victime ; oh dieux ! cela ferait le bonheur de ma vie !\par
MME DE SAINT-ANGE : Et que lui ferais-tu ?\par
EUGÉNIE : Tout… tout… tout ce qui pourrait la rendre la plus malheureuse des créatures ; oh ! ma bonne, ma bonne, aie pitié de moi ; je n’en puis plus.\par
DOLMANCÉ : Sacredieu, quelle imagination… viens, Eugénie, tu es délicieuse ; viens que je te baise mille et mille fois. {\itshape (Il la reprend dans ses bras.)} Tenez, madame, tenez ; regardez cette libertine, comme elle décharge {\itshape de tête}, sans qu’on la touche… Il faut absolument que je l’encule encore une fois.\par
EUGÉNIE : Aurai-je après ce que je demande ?\par
DOLMANCÉ : Oui, folle, oui, l’on t’en répond.\par
EUGÉNIE : Oh ! mon ami, voilà mon cul, faites-en ce que vous voudrez.\par
DOLMANCÉ : Attendez que je dispose cette jouissance d’une manière un peu luxurieuse. {\itshape (Tout s’exécute à mesure que Dolmancé indique :)} Augustin, étends-toi, sur le bord de ce lit ; qu’Eugénie se couche dans tes bras, pendant que je la sodomiserai ; je branlerai son clitoris avec la superbe tête du vit d’Augustin qui, pour ménager son foutre, aura soin de ne pas décharger ; le cher Chevalier qui, sans dire un mot, se branle tout doucement en nous écoutant, voudra bien s’étendre sur les épaules d’Eugénie, en exposant ses belles fesses à mes baisers, je le branlerai en dessous ; ce qui fait qu’ayant mon engin dans un cul, je polluerai un vit de chaque main ; et vous, madame, après avoir été votre mari, je veux que vous deveniez le mien ; revêtissez-vous du plus énorme de vos godemichés. {\itshape (M\textsuperscript{me} de Saint-Ange ouvre une cassette qui en est remplie, et notre héros choisit le plus redoutable.)} Bon, celui-ci, dit le numéro, a quatorze pouces de long sur dix de tour ; arrangez-vous cela autour des reins, madame, et portez-moi maintenant les plus terribles coups.\par
MME DE SAINT-ANGE : En vérité, Dolmancé, vous êtes fou, et je vais vous estropier avec cela.\par
DOLMANCÉ : Ne craignez rien ; poussez, pénétrez, mon ange, je n’enculerai votre chère Eugénie que quand votre membre énorme sera bien avant dans mon cul… Il y est ; il y est, sacredieu, ah ! tu me mets aux nues ; point de pitié, ma belle, je vais, je te le déclare, foutre ton cul sans préparation… Ah ! sacredieu, le beau derrière !\par
EUGÉNIE : Oh ! mon ami, tu me déchires… Prépare au moins les voies.\par
DOLMANCÉ : Je m’en garderai pardieu bien ; on perd la moitié du plaisir avec ces sottes attentions ; songe à nos principes, Eugénie, je travaille pour moi, maintenant victime un moment, mon bel ange, et tout à l’heure persécutrice… Ah ! sacredieu, il entre !\par
EUGÉNIE : Tu me fais mourir !\par
DOLMANCÉ : Oh ! foutredieu, je touche au but.\par
EUGÉNIE : Ah ! fais ce que tu voudras à présent, il y est, je ne sens que du plaisir !\par
DOLMANCÉ : Que j’aime à branler ce gros vit sur le clitoris d’une vierge… toi, Chevalier, fais-moi beau cul… te branlé-je bien, libertin ?… Et vous madame, foutez-moi, foutez votre garce : oui, je la suis, et je veux l’être… Eugénie, décharge, mon ange, oui décharge ; Augustin, malgré lui, me remplit de foutre… Je reçois celui du Chevalier, le mien s’y joint… Je n’y résiste plus ; Eugénie, agite tes fesses ; que ton anus presse mon vit : fais élancer au fond de tes entrailles le foutre brûlant qui s’exhale… Ah foutu bougre de dieu ! je me meurs ! {\itshape (Il se retire ; l’attitude se rompt.)} Tenez, madame, voilà votre petite libertine encore pleine de foutre ; l’entrée de son con en est inondée ; branlez-la, secouez vigoureusement son clitoris tout mouillé de sperme, c’est une des plus délicieuses choses qui puissent se faire.\par
EUGÉNIE, palpitant : Oh ! ma mie, que de plaisir tu me feras… ah ! cher amour, je brûle de lubricité.\par
{\itshape Cette posture s’arrange.}\par
DOLMANCÉ : Chevalier, comme c’est toi qui vas dépuceler ce bel enfant ; joins tes secours à ceux de ta sœur pour la faire pâmer dans tes bras et par ton attitude, présente-moi les fesses : je vais te foutre pendant qu’Augustin m’enculera.\par
{\itshape Tout se dispose.}\par
LE CHEVALIER : Me trouves-tu bien de cette manière ?\par
DOLMANCÉ : Le cul tant soit peu plus haut, mon amour : là, bien… sans préparation, Chevalier.\par
LE CHEVALIER : Ma foi ! comme tu voudras ; puis-je sentir autre chose que du plaisir au sein de cette délicieuse fille ?\par
{\itshape Il la baise et la branle en lui enfonçant légèrement un doigt dans le con pendant que M\textsuperscript{me} de Saint-Ange chatouille le clitoris d’Eugénie. }\par
DOLMANCÉ : Pour quant à moi, mon cher, j’en prends, sois-en assuré, beaucoup davantage avec toi, que je n’en pris avec Eugénie ; il y a tant de différence entre le cul d’un garçon et celui d’une fille ; encule-moi donc Augustin ! que de peine tu as à te décider !\par
AUGUSTIN : Dame, monseu, c’est que ça venoit de couler tout près du chose d’cette gentille tourterelle ; et vous voulez que ça dresse tout d’suite pour vot cul qui n’est vraiment pas si joli, dâ.\par
DOLMANCÉ : L’imbécile ! mais pourquoi se plaindre ! voilà la nature, chacun prêche pour son saint ; allons, allons, pénètre toujours, véridique Augustin, et quand tu auras un peu plus d’expérience, tu me diras si les culs ne valent pas mieux que les cons… Eugénie, rends donc au Chevalier, ce qu’il te fait ; tu ne t’occupes que de toi, tu as raison, libertine ; mais pour l’intérêt de tes plaisirs mêmes, branle-le, puisqu’il va cueillir tes prémices.\par
EUGÉNIE : Eh bien ! je le branle, je le baise, je perds la tête… ahe ! ahe ! ahe ! mes amis, je n’en puis plus, ayez pitié de mon état ; je me meurs ; je décharge… sacredieu, je suis hors de moi.\par
DOLMANCÉ : Pour moi je serai sage, je ne voulais que me remettre en train dans ce beau cul, je garde pour M\textsuperscript{me} de Saint-Ange le foutre qui s’y est allumé ; rien ne m’amuse comme de commencer dans un cul, l’opération que je veux terminer dans un autre ; eh bien, Chevalier, te voilà bien en train… Dépucelons-nous ?\par
EUGÉNIE : Oh, Ciel ! non je ne veux pas l’être par lui, j’en mourrais, le vôtre est plus petit, Dolmancé que ce soit à vous que je doive cette opération, je vous en conjure !\par
DOLMANCÉ : Cela n’est pas possible, mon ange ; je n’ai jamais foutu de con de ma vie ; vous me permettrez de ne pas commencer à mon âge. Vos prémices appartiennent au Chevalier, lui seul ici est digne de les cueillir, ne lui ravissons pas ses droits.\par
MME DE SAINT-ANGE : Refuser un pucelage… aussi frais, aussi joli que celui-là, car je défie qu’on puisse dire que mon Eugénie n’est pas la plus belle fille de Paris ! Oh ! monsieur… monsieur, en vérité, voilà ce qui s’appelle tenir un peu trop à ses principes.\par
DOLMANCÉ : Pas autant que je le devrais, madame ; car il est tout plein de mes confrères qui ne vous enculeraient assurément pas… Moi, je l’ai fait et je vais le refaire ; ce n’est donc point comme vous m’en soupçonnez porter mon culte jusqu’au fanatisme.\par
MME DE SAINT-ANGE : Allons donc, Chevalier, mais ménage-la, regarde la petitesse du détroit que tu vas enfiler ; est-il quelque proportion entre le contenu et le contenant ?\par
EUGÉNIE : Oh ! j’en mourrai, cela est inévitable… Mais le désir ardent que j’ai d’être foutue, me fait tout hasarder sans rien craindre… Va, pénètre, mon cher, je m’abandonne à toi.\par
LE CHEVALIER, {\itshape tenant à pleine main son vit bandant} : Oui, foutre, il faut qu’il y pénètre… Ma sœur… Dolmancé, tenez-lui chacun une jambe… Ah ! sacredieu ! quelle entreprise !… Oui, oui, dût-elle en être pourfendue, déchirée, il faut double Dieu, qu’elle y passe.\par
EUGÉNIE : Doucement, doucement, je n’y puis tenir… {\itshape (Elle crie, les pleurs coulent sur ses joues…)} À mon secours, ma bonne amie… {\itshape (Elle se débat.)} Non, je ne veux pas qu’il entre ; je crie au meurtre, si vous persistez !\par
LE CHEVALIER : Crie tant que tu voudras, petite coquine, je te dis qu’il faut qu’il entre, en dusses-tu crever mille fois.\par
EUGÉNIE : Quelle barbarie !\par
DOLMANCÉ : Ah ! foutre ! est-on délicat, quand on bande ?\par
LE CHEVALIER : Tenez-la, il y est…, il y est, sacredieu… foutre, voilà le pucelage au diable ; regardez son sang, comme il coule !\par
EUGÉNIE : Va, tigre… va, déchire-moi si tu veux maintenant, je m’en moque, baise-moi, bourreau, baise-moi, je t’adore… ah ! ce n’est plus rien, quand il est dedans ; toutes les douleurs sont oubliées… Malheur aux jeunes filles qui s’effaroucheraient d’une telle attaque… Que de grands plaisirs elles refuseraient pour une bien petite peine… pousse, pousse, Chevalier, je décharge ; arrose de ton foutre les plaies dont tu m’as couverte ; pousse-le donc au fond de ma matrice ; ah ! la douleur cède au plaisir ; je suis prête à m’évanouir.\par
{\itshape Le Chevalier décharge, pendant qu’il a foutu, Dolmancé lui a branlé le cul et les couilles et M\textsuperscript{me} de Saint-Ange a chatouillé le clitoris d’Eugénie, la posture se rompt. }\par
DOLMANCÉ : Mon avis serait que, pendant que les voies sont ouvertes, la petite friponne fût à l’instant foutue par Augustin.\par
EUGÉNIE : Par Augustin… un vit de cette taille… ah ! tout de suite… Quand je saigne encore… avez-vous donc envie de me tuer !\par
MME DE SAINT-ANGE : Cher amour… baise-moi, je te plains… mais la sentence est prononcée ; elle est sans appel, mon cœur, il faut que tu la subisses.\par
AUGUSTIN : Ah ! jerdinieu, me voilà prêt, dès qu’il s’agit d’enfiler c’te petite fille, je vinrois pardieu de Rome à pied.\par
LE CHEVALIER, {\itshape empoignant le vit énorme d’Augustin} : Tiens, Eugénie, vois comme il bande… comme il est digne de me remplacer.\par
EUGÉNIE : Ah ! juste ciel ! quel arrêt… Oh ! vous voulez me tuer, cela est clair.\par
AUGUSTIN, {\itshape s’emparant d’Eugénie} : Oh ! que non, mameselle : ça n’a jamais fait mourir personne.\par
DOLMANCÉ : Un moment, beau fils, un moment ; il faut qu’elle me présente le cul, pendant que tu vas foutre… Oui, ainsi, approchez-vous, madame de Saint-Ange ; je vous ai promis de vous enculer ; je tiendrai parole ; mais placez-vous de manière qu’en vous foutant, je puisse être à portée de fouetter Eugénie ; que le Chevalier me fouette pendant ce temps-là.\par
{\itshape Tout s’arrange.}\par
EUGÉNIE : Ah ! foutre, il me crève… Va donc doucement, gros butor… Ah ! le bougre… il enfonce… l’y voilà, le jean-foutre, il est tout au fond, je me meurs !… Oh ! Dolmancé, comme vous frappez ; c’est m’allumer des deux côtés : vous me mettez les fesses en feu.\par
DOLMANCÉ, {\itshape fouettant à tour de bras} : Tu en auras… tu en auras, petite coquine ; tu n’en déchargeras que plus délicieusement ; comme vous la branlez, Saint-Ange, comme ce doigt léger doit adoucir les maux qu’Augustin et moi lui faisons… mais votre anus se resserre ; je le vois, madame, nous allons décharger ensemble ; ah ! comme il est divin d’être ainsi entre le frère et la sœur.\par
MME DE SAINT-ANGE, {\itshape à Dolmancé} : Fous, mon astre, fous ; jamais, je crois je n’eus tant de plaisirs !\par
LE CHEVALIER : Dolmancé, changeons de main ; passe lestement du cul de ma sœur dans celui d’Eugénie, pour lui faire connaître les plaisirs de l’entre-deux, et moi j’enculerai ma sœur, qui pendant ce temps rendra sur tes fesses les coups de verges dont tu viens d’ensanglanter celles d’Eugénie.\par
DOLMANCÉ, {\itshape exécutant} : J’accepte… tiens, mon ami, se peut-il faire un changement plus leste que celui-là ?\par
EUGÉNIE : Quoi, tous les deux sur moi, juste Ciel ! je ne sais plus auquel entendre ; j’avais bien assez de ce butor ! Ah ! que de foutre va me coûter cette double jouissance : il coule déjà ; sans cette sensuelle éjaculation, je serais, je crois, déjà morte… Eh ! quoi, ma bonne, tu m’imites ?… oh, comme elle jute, la coquine… Dolmancé décharge… décharge, mon amour, ce gros paysan m’inonde : il me l’élance au fond de mes entrailles… Ah ! mes fouteurs, quoi tous deux à la fois… Sacredieu… mes amis, recevez mon foutre, il se joint au vôtre… je suis anéantie… {\itshape (Les attitudes se rompent.)} Eh bien ! ma bonne, es-tu contente de ton écolière ; suis-je assez putain, maintenant… Mais vous m’avez mise dans un état… dans une agitation… Oh ! oui, je jure que dans l’ivresse où me voilà, j’irais s’il le fallait, me faire foutre au milieu des rues.\par
DOLMANCÉ : Comme elle est belle ainsi !\par
EUGÉNIE : Je vous déteste, vous m’avez refusée.\par
DOLMANCÉ : Pouvais-je contrarier mes dogmes ?\par
EUGÉNIE : Allons, je vous pardonne, et je dois respecter des principes qui conduisent à des égarements. Comment ne les adopterais-je pas, moi qui ne veux plus vivre que dans le crime ; asseyons-nous et jasons un instant. Je n’en puis plus. Continuez mon instruction, Dolmancé, et dites-moi quelque chose qui me console des excès où me voilà livrée ; éteignez mes remords ; encouragez-moi.\par
MME DE SAINT-ANGE : Cela est juste, il faut qu’un peu de théorie succède à la pratique ; c’est le moyen d’en faire une écolière parfaite.\par
DOLMANCÉ : Eh bien ! quel est l’objet, Eugénie, sur lequel vous voulez qu’on vous entretienne ?\par
EUGÉNIE : Je voudrais savoir si les mœurs sont vraiment nécessaires dans un gouvernement, si leur influence est de quelque poids sur le génie d’une nation ?\par
DOLMANCÉ : Ah ! parbleu, en partant ce matin, j’ai acheté au palais de l’Égalité une brochure qui, s’il faut en croire le titre, doit nécessairement répondre à votre question… À peine sort-elle de la presse.\par
MME DE SAINT-ANGE : Voyons {\itshape (elle lit) Français, encore un effort si vous voulez être républicains}. Voilà, sur ma parole un singulier titre, il promet ; Chevalier, toi qui possèdes un bel organe, lis-nous cela.\par
DOLMANCÉ : Ou je me trompe, ou cela doit parfaitement répondre à la question d’Eugénie.\par
EUGÉNIE : Assurément.\par
MME DE SAINT-ANGE : Sors, Augustin, ceci n’est pas fait pour toi ; mais ne t’éloigne pas, nous sonnerons dès qu’il faudra que tu reparaisses.\par
LE CHEVALIER : Je commence.\par
\phantomsection
\label{d5\_francais}FRANÇAIS,\par
{\itshape Encore un effort si vous voulez être républicains}\par

\labelblock{LA RELIGION}

\noindent Je viens offrir de grandes idées, on les écoutera, elles seront réfléchies ; si toutes ne plaisent pas, au moins en restera-t-il quelques-unes ; j’aurai contribué en quelque chose, au progrès des lumières, et j’en serai content.\par
Je ne le cache point, c’est avec peine que je vois la lenteur avec laquelle nous tâchons d’arriver au but, c’est avec inquiétude que je sens que nous sommes à la veille de le manquer encore une fois ; croit-on que ce but sera atteint quand on nous aura donné des lois ? Qu’on ne l’imagine pas ; que ferions-nous de lois sans religion ; il nous faut un culte et un culte fait pour le caractère d’un républicain, bien éloigné de jamais pouvoir reprendre celui de Rome ; dans un siècle où nous sommes aussi convaincus que la religion doit être appuyée sur la morale, et non pas la morale sur la religion, il faut une religion qui aille aux mœurs, qui en soit comme le développement, comme la suite nécessaire et qui puisse, en élevant l’âme, la tenir perpétuellement à la hauteur de cette liberté précieuse dont elle fait aujourd’hui son unique idole ; or je demande si l’on peut supposer que celle d’un esclave de Titus, que celle d’un vil histrion de Judée, puisse convenir à une nation libre et guerrière, qui vient de se régénérer ; non, mes compatriotes, non, vous ne le croyez pas : et malheureusement pour lui le Français s’ensevelissait encore dans les ténèbres du christianisme, d’un côté l’orgueil, la tyrannie, le despotisme des prêtres, vices toujours renaissants dans cette horde impure, de l’autre la bassesse, les petites vues, les platitudes des dogmes et des mystères de cette indigne et fabuleuse religion, en émoussant la fierté de l’âme républicaine l’auraient bientôt ramenée sous le joug que son énergie vient de briser, ne perdons pas de vue que cette puérile religion était une des meilleures armes des mains de nos tyrans, un de ses premiers dogmes était de {\itshape rendre à César ce qui appartient à César} ; mais nous avons détrôné César et nous ne voulons plus rien lui rendre ; Français, ce serait en vain que vous vous flatteriez que l’esprit d’un clergé assermenté ne doit plus être celui d’un clergé réfractaire, il est des vices d’état dont on ne se corrige jamais, avant dix ans, au moyen de la religion chrétienne, de sa superstition, de ses préjugés, vos prêtres, malgré leur serment, malgré leur pauvreté, ils reprendraient sur les âmes l’empire qu’ils avaient envahi, ils vous renchaîneraient à des rois, parce que la puissance de ceux-ci étaya toujours celle de l’autre, et votre édifice républicain s’écroulerait faute de bases. Ô vous qui avez la faux à la main, portez le dernier coup à l’arbre de la superstition, ne vous contentez pas d’élaguer les branches, déracinez tout à fait une plante dont les effets sont si contagieux, soyez parfaitement convaincus que votre système de liberté et d’égalité contrarie trop ouvertement les ministres des autels du Christ, pour qu’il en soit jamais un seul, ou qui l’adopte de bonne foi, ou qui ne cherche pas à l’ébranler s’il parvient à reprendre quelque empire sur les consciences. Quel sera le prêtre qui comparant l’état où l’on vient de le réduire avec celui dont il jouissait autrefois, ne fera pas tout ce qui dépendra de lui pour recouvrer et la confiance, et l’autorité qu’on lui a fait perdre ? Et que d’êtres faibles et pusillanimes redeviendront bientôt les esclaves de cet ambitieux tonsuré ; pourquoi n’imagine-t-on pas que les inconvénients qui ont existé peuvent encore renaître ? Dans l’enfance de l’Église chrétienne, les prêtres n’étaient-ils pas ce qu’ils sont aujourd’hui ? Vous voyez où ils étaient parvenus, qui pourtant les avaient conduits là : n’étaient-ce pas les moyens que leur fournissait la religion ? Or si vous ne la défendez pas absolument, cette religion, ceux qui la prêchent ayant toujours les mêmes moyens, arriveront bientôt au même but.\par
Anéantissez donc à jamais tout ce qui peut détruire un jour votre ouvrage ; songez que le fruit de vos travaux n’étant réservés qu’à vos neveux, il est de votre devoir, de votre probité, de ne leur laisser aucun de ces germes dangereux qui pourraient les replonger dans le chaos dont nous avons tant de peine à sortir ; déjà nos préjugés se dissipent, déjà le peuple abjure les absurdités catholiques, il a déjà supprimé les temples, il a culbuté les idoles, il est convenu que le mariage n’est plus qu’un acte civil. Les confessionnaux brisés servent aux foyers publics : les prétendus fidèles, désertant le banquet apostolique, laissent les dieux de farine aux souris. Français, ne vous arrêtez point, l’Europe entière, une main déjà sur le bandeau qui fascine ses yeux, attend de vous l’effort qui doit l’arracher de son front ; hâtez-vous, ne laissez pas à Rome la sainte, s’agitant en tous sens pour réprimer votre énergie, le temps de se conserver peut-être encore quelques prosélytes. Frappez sans ménagement sa tête altière et frémissante, et qu’avant deux mois l’arbre de la liberté, ombrageant les débris de la chaire de Saint Pierre, couvre du poids de ses rameaux victorieux, toutes ces méprisables idoles du christianisme effrontément élevées sur les cendres des Catons et des Brutus. Français, je vous le répète, l’Europe attend de vous d’être à la fois délivrée du sceptre et de l’encensoir ; songez qu’il vous est impossible de l’affranchir de la tyrannie royale, sans lui faire briser en même temps les freins de la superstition religieuse ; les liens de l’une sont trop intimement unis à l’autre, pour qu’en en laissant subsister un des deux, vous ne retombiez pas bientôt sous l’empire de celui que vous aurez négligé de dissoudre ; ce n’est plus ni aux genoux d’un être imaginaire, ni à ceux d’un vil imposteur, qu’un républicain doit fléchir ; ses uniques dieux doivent être maintenant le courage et la liberté. Rome disparut dès que le christianisme s’y prêcha, et la France est perdue s’il s’y révère encore. Qu’on examine avec attention les dogmes absurdes, les mystères effrayants, les cérémonies monstrueuses, la morale impossible de cette dégoûtante religion, et l’on verra si elle peut convenir à une république ; croyez-vous de bonne foi que je me laisserais dominer par l’opinion d’un homme que je viendrais de voir aux pieds de l’imbécile prêtre de Jésus ? non, ou certes, cet homme toujours vil tiendra toujours par la bassesse de ses vues aux atrocités de l’ancien régime ; dès lors qu’il peut se soumettre aux stupidités d’une religion aussi plate que celle que nous avions la folie d’admettre, il ne peut plus ni me dicter des lois, ni me transmettre des lumières, je ne le vois plus que comme un esclave des préjugés et de la superstition ; jetons les yeux, pour nous convaincre de cette vérité, sur le peu d’individus qui restent attachés au culte insensé de nos pères, nous verrons si ce ne sont pas tous des ennemis irréconciliables du système actuel, nous verrons si ce n’est pas dans leur nombre qu’est entièrement comprise cette caste si justement méprisée de {\itshape royalistes} et d’{\itshape aristocrates}. Que l’esclave d’un brigand couronné fléchisse s’il le veut aux pieds d’une idole de pâte, un tel objet est fait pour son âme de boue, qui peut servir des rois doit adorer des dieux ; mais nous, Français, mais nous mes compatriotes, nous ramper encore humblement sous des freins aussi méprisables, plutôt mourir mille fois que de nous y asservir de nouveau ; puisque nous croyons un culte nécessaire, imitons celui des Romains : les actions, les passions, les héros, voilà quels en étaient les respectables objets ; de telles idoles élevaient l’âme, elles l’électrisaient, elles faisaient plus, elles lui communiquaient les vertus de l’être respecté ; l’adorateur de Minerve voulait être prudent. Le courage était dans le cœur de celui qu’on voyait aux pieds de Mars, pas un seul dieu de ces grands hommes n’était privé d’énergie, tous faisaient passer le feu dont ils étaient eux-mêmes embrasés dans l’âme de celui qui les vénérait ; et, comme on avait l’espoir d’être adoré soi-même un jour, on aspirait à devenir au moins aussi grand que celui qu’on prenait pour modèle. Mais que trouvons-nous au contraire dans les vains dieux du christianisme, que vous offre, je le demande, cette imbécile religion\footnote{ Si quelqu’un examine attentivement cette religion, il trouvera que les impiétés dont elle est remplie, viennent en partie de la férocité et de l’innocence des juifs, et en partie de l’indifférence et de la confusion des gentils ; au lieu de s’approprier ce que les peuples de l’Antiquité pouvaient avoir de bon, les chrétiens paraissent n’avoir formé leur religion que du mélange des vices qu’ils ont rencontré partout.} ? Le plat imposteur de Nazareth vous fait-il naître quelques grandes idées ? Sa sale et dégoûtante mère, l’impudique Marie, vous inspire-t-elle quelques vertus ? et trouvez-vous dans les saints dont est garni son Élysée quelque modèle de grandeur, ou d’héroïsme ou de vertus ? Il est si vrai que cette stupide religion ne prête rien aux grandes idées, qu’aucun artiste n’en peut employer les attributs dans les monuments qu’il élève ; à Rome même, la plupart des embellissements ou des ornements du palais des papes ont leurs modèles dans le paganisme, et tant que le monde subsistera, lui seul échauffera la verve des grands hommes.\par
Sera-ce dans le théisme pur que nous trouverons plus de motifs de grandeur et d’élévation ? Sera-ce l’adoption d’une chimère, qui donnant à notre âme ce degré d’énergie essentiel aux vertus républicaines, portera l’homme à les chérir, ou à les pratiquer ? ne l’imaginons pas, on est revenu de ce fantôme, et l’athéisme est à présent le seul système de tous les gens qui savent raisonner ; à mesure que l’on s’est éclairé, on a senti que le mouvement étant inhérent à la matière, l’agent nécessaire à imprimer ce mouvement devenait un être illusoire et que tout ce qui existait devant être en mouvement par essence, le moteur était inutile ; on a senti que ce dieu chimérique prudemment inventé par les premiers législateurs, n’était entre leurs mains qu’un moyen de plus pour nous enchaîner, et que se réservant le droit de faire parler seul ce fantôme, ils sauraient bien ne lui faire dire que ce qui viendrait à l’appui des lois ridicules par lesquelles ils prétendaient nous asservir. Lycurgue, Numa, Moïse, Jésus-Christ, Mahomet, tous ces grands fripons, tous ces grands despotes de nos idées, surent associer les divinités qu’ils fabriquaient à leur ambition démesurée, et certains de captiver les peuples avec la sanction de ces dieux, ils avaient, comme on sait, toujours soin ou de ne les interroger qu’à-propos, ou de ne leur faire répondre que ce qu’ils croyaient pouvoir les servir. Tenons donc aujourd’hui dans le même mépris, et le dieu vain que des imposteurs ont prêché, et toutes les subtilités religieuses qui découlent de sa ridicule adoption, ce n’est plus avec ce hochet qu’on peut amuser des hommes libres ; que l’extinction totale des cultes entre donc dans les principes que nous propageons dans l’Europe entière, ne nous contentons pas de briser les sceptres, pulvérisons à jamais les idoles ; il n’y eut jamais qu’un pas de la superstition au royalisme\footnote{ Suivez l’histoire de tous les peuples, vous ne les verrez jamais changer le gouvernement qu’ils avaient pour un gouvernement monarchique, qu’en raison de l’abrutissement où la superstition les tient. Vous verrez toujours les rois étayer la religion, et la religion sacrer des rois, on sait l’histoire de l’intendant et du cuisinier, {\itshape passez-moi le poivre, je vous passerai le beurre} ; malheureux humains, êtes-vous donc toujours destinés à ressembler au maître de ces deux fripons !}, il faut bien que cela soit sans doute, puisqu’un des premiers articles du sacre des rois, était toujours le maintien de la religion dominante, comme une des bases politiques qui devaient le mieux soutenir leur trône, mais dès qu’il est abattu ce trône, dès qu’il l’est heureusement pour jamais, ne redoutons point d’extirper de même ce qui en formait les appuis ; oui, citoyens, la religion est incohérente au système de la liberté ; vous l’avez senti, jamais l’homme libre ne se courbera près des dieux du christianisme, jamais ses dogmes, jamais ses rites, ses mystères ou sa morale ne conviendront à un républicain ; encore un effort, puisque vous travaillez à détruire tous les préjugés, n’en laissez subsister aucun, s’il n’en faut qu’un seul pour les ramener tous ; combien devons-nous être plus certains de leur retour, si celui que vous laissez vivre est positivement le berceau de tous les autres ?\par
Cessons de croire que la religion puisse être utile à l’homme, ayons de bonnes lois, et nous saurons nous passer de religion. Mais il en faut une au peuple, assure-t-on, elle l’amuse, elle le contient, à la bonne heure ; donnez-nous donc, en ce cas, celle qui convient à des hommes libres. Rendez-nous les dieux du paganisme. Nous adorerons volontiers Jupiter, Hercule ou Pallas, mais nous ne voulons plus du fabuleux auteur d’un univers qui se meut lui-même, nous ne voulons plus d’un dieu sans étendue et qui pourtant remplit tout de son immensité, d’un dieu tout-puissant, et qui n’exécute jamais ce qu’il désire, d’un être souverainement bon, et qui ne fait que des mécontents, d’un être ami de l’ordre, et dans le gouvernement duquel tout est en désordre. Non, nous ne voulons plus d’un dieu qui dérange la nature, qui est le père de la confusion, qui meut l’homme au moment où l’homme se livre à des horreurs ; un tel dieu nous fait frémir d’indignation, et nous le reléguons pour jamais dans l’oubli, d’où l’infâme Robespierre a voulu le sortir\footnote{ Toutes les religions s’accordent à nous exalter la sagesse et la puissance intime de la divinité, mais dès qu’elles nous exposent sa conduite, nous n’y trouvons qu’imprudence, que faiblesse et que folie. Dieu, dit-on, a créé le monde pour lui-même, et jusqu’ici il n’a pu parvenir à s’y faire convenablement honorer, dieu nous a créés pour l’adorer, et nous passons nos jours à nous moquer de lui ; quel pauvre dieu que ce dieu-là !}.\par
Français, à cet indigne fantôme, substituons les simulacres imposants qui rendaient Rome maîtresse de l’univers, traitons toutes les idoles chrétiennes comme nous avons traité celles de nos rois ; nous avons replacé les emblèmes de la liberté sur les bases qui soutenaient autrefois des tyrans, réédifions de même l’effigie des grands hommes sur les piédestaux de ces polissons adorés par le christianisme\footnote{ Il ne s’agit ici que de ceux dont la réputation est faite depuis longtemps.}, cessons de redouter, pour nos campagnes, l’effet de l’athéisme ; les paysans n’ont-ils pas senti la nécessité de l’anéantissement du culte catholique si contradictoire aux vrais principes de la liberté ? N’ont-ils pas vu sans effroi, comme sans douleur, culbuter leurs autels et leurs presbytères ? Ah croyez qu’ils renonceront de même à leur ridicule dieu ; les statues de Mars, de Minerve et de la Liberté seront mises aux endroits les plus remarquables de leurs habitations, une fête annuelle s’y célébrera tous les ans, la couronne civique y sera décernée au citoyen qui aura le mieux mérité de la patrie ; à l’entrée d’un bois solitaire, Vénus, l’Hymen et l’Amour érigés sous un temple agreste, recevront l’hommage des amants ; là ce sera par la main des grâces que la beauté couronnera la constance, il ne s’agira pas seulement d’aimer pour être digne de cette couronne, il faudra encore avoir mérité de l’être ; l’héroïsme, les talents, l’humanité, la grandeur d’âme, un civisme à l’épreuve ; voilà les titres qu’aux pieds de sa maîtresse sera forcé d’établir l’amant ; et ceux-là vaudront bien ceux de la naissance et de la richesse, qu’un sot orgueil exigeait autrefois. Quelques vertus au moins écloront de ce culte, tandis qu’il ne naît que des crimes de celui que nous avons eu la faiblesse de professer. Ce culte s’alliera avec la liberté que nous servons, il l’animera, l’entretiendra, l’embrassera, au lieu que le théisme est par son essence et par sa nature la plus mortelle ennemie de la liberté que nous servons.\par
En coûta-t-il une goutte de sang, quand les idoles païennes furent détruites sous le Bas-Empire ? La révolution préparée par la stupidité d’un peuple redevenu esclave, s’opéra sans le moindre obstacle ; comment pouvons-nous redouter que l’ouvrage de la philosophie soit plus pénible que celui du despotisme ? Ce sont les prêtres seuls qui captivent encore aux pieds de leur dieu chimérique ce peuple que vous craignez tant d’éclairer, éloignez-les de lui et le voile tombera naturellement ; croyez que ce peuple bien plus sage que vous ne l’imaginez, dégagé des fers de la tyrannie, le sera bientôt de ceux de la superstition ; vous le redoutez, s’il n’a pas ce frein, quelle extravagance ! ah ! croyez-le, citoyens, celui que le glaive matériel des lois n’arrête point, ne le sera pas davantage par la crainte morale des supplices de l’enfer dont il se moque depuis son enfance ; votre théisme, en un mot, a fait commettre beaucoup de forfaits, mais il n’en arrêta jamais un seul ; s’il est vrai que les passions aveuglent, que leur effet soit d’élever sur nos yeux un nuage qui nous déguise les dangers dont elles sont environnées, comment pouvons-nous supposer que ceux qui, loin de nous, comme le sont les punitions annoncées par votre dieu, puissent parvenir à dissiper ce nuage que ne peut dissoudre le glaive même des lois toujours suspendu sur les passions ? S’il est donc prouvé que ce supplément de freins imposé par l’idée d’un dieu, devienne inutile, s’il est démontré qu’il est dangereux par ses autres effets, je demande à quel usage il peut donc servir, et de quels motifs nous pourrions nous appuyer pour en prolonger l’existence ? Me dira-t-on que nous ne sommes pas assez mûrs pour consolider encore notre révolution d’une manière aussi éclatante ? Ah ! mes concitoyens, le chemin que nous avons fait depuis 89 est bien autrement difficile que celui qui nous reste à faire, et nous avons bien moins à travailler l’opinion dans ce que je vous propose, que nous ne l’avons tourmentée en tout sens, depuis l’époque du renversement de la Bastille ; croyons qu’un peuple assez sage, assez courageux, pour conduire un monarque impudent du faîte des grandeurs aux pieds de l’échafaud, qui dans ce peu d’années sut vaincre autant de préjugés, sut briser tant de freins ridicules, le sera suffisamment pour immoler au bien de la chose, à la prospérité de la république un fantôme bien plus illusoire encore que ne pouvait l’être celui d’un roi. Français, vous frapperez les premiers coups, votre éducation nationale fera le reste ; mais travaillez promptement à cette besogne, qu’elle devienne un de vos soins le plus important ; qu’elle ait surtout pour base cette morale essentielle, si négligée dans l’éducation religieuse ; remplacez les sottises déifiques, dont vous fatiguiez les jeunes organes de vos enfants, par d’excellents principes sociaux ; qu’au lieu d’apprendre à réciter de futiles prières qu’il fera gloire d’oublier dès qu’il aura seize ans, il soit instruit de ses devoirs dans la société ; apprenez-lui à chérir des vertus dont vous lui parliez à peine autrefois, et qui, sans vos fables religieuses suffisent à son bonheur individuel ; faites-leur sentir que ce bonheur consiste à rendre les autres aussi fortunés que nous désirons l’être nous-mêmes, si vous asseyez ces vérités sur des chimères chrétiennes comme vous aviez la folie de le faire autrefois : à peine vos élèves auront-ils reconnu la futilité des bases, qu’ils feront crouler l’édifice, et ils deviendront scélérats seulement, parce qu’ils croiront que la religion qu’ils ont culbutée, leur défendait de l’être. En leur faisant sentir au contraire la nécessité de la vertu uniquement parce que leur propre bonheur en dépend, ils seront honnêtes gens par égoïsme, et cette loi qui régit tous les hommes sera toujours la plus sûre de toutes ; que l’on évite donc avec le plus grand soin de mêler aucune fable religieuse dans cette éducation nationale, ne perdons jamais de vue que ce sont des hommes libres que nous voulons former, et non de vils adorateurs d’un dieu ; qu’un philosophe simple instruise ces nouveaux élèves des sublimités incompréhensibles de la nature, qu’il leur prouve que la connaissance d’un dieu, souvent très dangereuse aux hommes, ne servit jamais à leur bonheur, et qu’ils ne seront pas plus heureux en admettant comme cause de ce qu’ils ne comprennent pas quelque chose qu’ils comprendront encore moins ; qu’il est bien moins essentiel d’entendre la nature que d’en jouir, et d’en respecter les lois ; que ces lois sont aussi sages que simples, qu’elles sont écrites dans le cœur de tous les hommes, et qu’il ne faut qu’interroger ce cœur, pour en démêler l’impulsion ; s’ils veulent qu’absolument vous leur parliez d’un créateur, répondez que les choses ayant toujours été ce qu’elles sont, n’ayant jamais eu de commencement et ne devant jamais avoir de fin, il devient aussi inutile qu’impossible à l’homme de pouvoir remonter à une origine imaginaire qui n’expliquerait rien et n’avancerait à rien, dites-leur qu’il est impossible aux hommes d’avoir des idées vraies d’un être qui n’agit sur aucun de nos sens ; toutes nos idées sont des représentations des objets qui nous frappent ; qu’est-ce qui peut nous représenter l’idée de dieu qui est évidemment une idée sans objet, une telle idée, leur ajouterez-vous, n’est-elle pas aussi impossible que des effets sans cause ? Une idée sans prototype, est-elle autre chose qu’une chimère ? Quelques docteurs, poursuivrez-vous, assurent que l’idée de dieu est innée, et que les hommes [ont] cette idée dès le ventre de leur mère ; mais cela est faux, leur ajouterez-vous, tout principe est un jugement ; tout jugement est l’effet de l’expérience, et l’expérience ne s’acquiert que par l’exercice des sens, d’où suit que les principes religieux ne portent évidemment sur rien et ne sont point innés ; comment, poursuivrez-vous, a-t-on pu persuader à des êtres raisonnables que la chose la plus difficile à comprendre était la plus essentielle pour eux, c’est qu’on les a grandement effrayés, c’est que quand on a peur, on cesse de raisonner, c’est qu’on leur a surtout recommandé de se défier de leur raison, et que quand la cervelle est troublée, on croit tout et n’examine rien ; l’ignorance et la peur, leur direz-vous encore, voilà les deux bases de toutes les religions, l’incertitude où l’homme se trouve par rapport à son dieu, est précisément le motif qui l’attache à sa religion ; l’homme a peur dans les ténèbres tant au physique qu’au moral, sa peur devient habituelle en lui et se change en besoin ; il croirait qu’il lui manquerait quelque chose, s’il n’avait plus rien à espérer ou à craindre. Revenez ensuite à l’utilité de la morale, donnez-leur sur ce grand objet beaucoup plus d’exemples que de leçons, beaucoup plus de preuves que de livres, et vous en ferez de bons citoyens, vous en ferez de bons guerriers, de bons pères, de bons époux ; vous en ferez des hommes d’autant plus attachés à la liberté de leur pays, qu’aucune idée de servitude ne pourra plus se présenter à leur esprit, qu’aucune terreur religieuse ne viendra troubler leur génie ; alors le véritable patriotisme éclatera dans toutes les âmes, il y régnera dans toute sa force et dans toute sa pureté, parce qu’il y deviendra le seul sentiment dominant, et qu’aucune idée étrangère n’en attiédira l’énergie. Alors votre seconde génération est sûre et votre ouvrage consolidé par elle va devenir la loi de l’univers ; mais si par crainte ou pusillanimité, ces conseils ne sont pas suivis, si l’on laisse subsister les bases de l’édifice que l’on avait cru détruire, qu’arrivera-t-il ? on rebâtira sur ces bases, et l’on y placera les mêmes colosses, à la cruelle différence qu’ils y seront cette fois cimentés d’une telle force, que ni votre génération ni celles qui la suivront ne réussiront à les culbuter. Qu’on ne doute pas que les religions ne soient le berceau du despotisme, le premier de tous les despotes fut un prêtre ; le premier roi et le premier empereur de Rome, Numa et Auguste, s’associèrent l’un et l’autre au sacerdoce ; Constantin et Clovis furent plutôt des abbés que des souverains ; Héliogabale fut prêtre du soleil. De tous les temps, dans tous les siècles il y eut, dans le despotisme et dans la religion, une telle connexité, qu’il reste plus que démontré qu’en détruisant l’un, l’on doit saper l’autre, par la grande raison que le premier servira toujours de loi au second ; je ne propose cependant ni massacres ni exportations ; toutes ces horreurs sont trop loin de mon âme pour oser seulement les concevoir une minute ; non, n’assassinez point ; n’exportez point, ces atrocités sont celles des rois, ou des scélérats qui les imitèrent, ce n’est point en faisant comme eux que vous forcerez de prendre en horreur ceux qui les exerçaient ; n’employons la force que pour les idoles, il ne faut que des ridicules pour ceux qui les servent ; les sarcasmes de Julien nuisirent plus à la religion chrétienne, que tous les supplices de Néron ; oui, détruisons à jamais toute l’idée de dieu, et faisons des soldats de ses prêtres, quelques-uns le sont déjà, qu’ils s’en tiennent à ce métier si noble pour un républicain, mais qu’ils ne nous parlent plus ni de leur être chimérique, ni de sa religion fabuleuse, unique objet de nos mépris ; condamnons à être bafoué, ridiculisé, couvert de boue dans tous les carrefours des plus grandes villes de France, le premier de ces charlatans bénis qui viendra nous parler encore ou de dieu ou de religion ; une éternelle prison sera la peine de celui qui retombera deux fois dans les mêmes fautes ; que les blasphèmes les plus insultants, les ouvrages les plus athées soient ensuite autorisés pleinement, afin d’achever d’extirper dans le cœur et la mémoire des hommes ces effrayants jouets de notre enfance ; que l’on mette au concours l’ouvrage le plus capable d’éclairer enfin les Européens sur une matière aussi importante, et qu’un prix considérable, et décerné par la nation, soit la récompense de celui qui ayant tout dit, tout démontré sur cette matière, ne laissera plus à ses compatriotes qu’une faux pour culbuter tous ces fantômes, et qu’un cœur droit pour les haïr. Dans six mois tout sera fini : votre infâme dieu sera dans le néant et cela sans cesser d’être juste, jaloux de l’estime des autres, sans cesser de redouter le glaive des lois, et d’être honnête homme, parce qu’on aura senti que le véritable ami de la patrie ne doit point, comme l’esclave des rois, être mené par des chimères, que ce n’est en un mot, ni l’espoir frivole d’un monde meilleur ni la crainte de plus grands maux que ceux que nous envoya la nature, qui doivent conduire un républicain dont le seul guide est la vertu, comme l’unique frein le remords.\par

\labelblock{LES MŒURS}

\noindent Après avoir démontré que le théisme ne convient nullement à un gouvernement républicain, il me paraît nécessaire de prouver que les mœurs françaises ne lui conviennent pas davantage. Cet article est d’autant plus essentiel, que ce sont les mœurs qui vont servir de motifs aux lois qu’on va promulguer.\par
Français vous êtes trop éclairés pour ne pas sentir qu’un nouveau gouvernement va nécessiter de nouvelles mœurs, il est impossible que le citoyen d’un État libre se conduise comme l’esclave d’un roi despote, ces différences de leurs intérêts, de leurs devoirs, de leurs relations entre eux, déterminent essentiellement une manière tout autre de se comporter dans le monde ; une foule de petites erreurs, de petits délits sociaux considérés comme très essentiels sous le gouvernement des rois, qui devaient exiger d’autant plus, qu’ils avaient plus besoin d’imposer des freins pour se rendre respectables ou inabordables à leurs sujets, vont devenir nuls ici ; d’autres forfaits connus sous les noms de régicide ou de sacrilège, sous un gouvernement qui ne connaît plus ni rois ni religion, doivent s’anéantir de même dans un État républicain. En accordant la liberté de conscience et celle de la presse, songez, citoyens, qu’à bien peu de chose près, on doit accorder celle d’agir, et qu’excepté ce qui choque directement les bases du gouvernement, il vous reste on ne saurait moins de crimes à punir, parce que dans le fait, il est fort peu d’actions criminelles dans une société dont la liberté et l’égalité font les bases, et qu’à bien peser et bien examiner les choses, il n’y a vraiment de criminel que ce que réprouve la loi, car la nature nous dictant également des vices et des vertus, en raison de notre organisation, ou plus philosophiquement encore en raison du besoin qu’elle a de l’un ou de l’autre, ce qu’elle nous inspire deviendrait une mesure très incertaine pour régler avec précision ce qui est bien ou ce qui est mal. Mais pour mieux développer mes idées sur un objet aussi essentiel, nous allons classer les différentes actions de la vie [de] l’homme, que l’on était convenu jusqu’à présent de nommer criminelles, et nous les {\itshape toiserons} ensuite aux vrais devoirs d’un républicain.\par
On a considéré de tout temps les devoirs de l’homme sous les trois différents rapports suivants :\par
1° ceux que sa conscience et sa crédulité lui imposent envers l’Être suprême ;\par
2° ceux qu’il est obligé de remplir avec ses frères ;\par
3° enfin ceux qui n’ont de relation qu’avec lui.\par
La certitude où nous devons être qu’aucun dieu ne s’est mêlé de nous, et que créatures nécessitées de la nature comme les plantes et les animaux, nous sommes ici parce qu’il était impossible que nous n’y fussions pas, cette certitude sans doute anéantit comme on le voit tout d’un coup la première partie de ces devoirs, je veux dire ceux dont nous nous croyons faussement responsables envers la divinité ; avec eux disparaissent tous les délits religieux, tous ceux connus sous les noms vagues et indéfinis d’{\itshape impiété}, de {\itshape sacrilège}, de {\itshape blasphème}, d’{\itshape athéisme} etc., tous ceux en un mot qu’Athènes punit avec tant d’injustice dans Alcibiade et la France dans l’infortuné {\itshape Labarre}. S’il y a quelque chose d’extravagant dans le monde, c’est de voir des hommes qui ne connaissent leur Dieu et ce que peut exiger ce Dieu, que d’après leurs idées bornées ; vouloir néanmoins décider sur la nature de ce qui contente ou de ce qui fâche ce ridicule fantôme de leur imagination, ce ne serait donc point à permettre indifféremment tous les cultes que je voudrais qu’on se bornât, je désirerais qu’on fût libre de se rire ou de se moquer de tous, que des hommes réunis dans un temple quelconque pour invoquer l’éternel à leur guise, fussent vus comme des comédiens sur un théâtre, au jeu desquels il est permis à chacun d’aller rire ; si vous ne voyez pas les religions sous ce rapport, elles reprendront le sérieux qui les rend importantes, elles protégeront bientôt les opinions, et l’on ne se sera pas plus tôt disputé sur les religions, que l’on se rebattra pour les religions\footnote{ Chaque peuple prétend que sa religion est la meilleure, et s’appuie, pour le persuader, sur une infinité de preuves non seulement discordantes entre elles, mais presque toutes contradictoires, dans la profonde ignorance où nous sommes, quelle est celle qui peut plaire à Dieu, à supposer qu’il y ait un Dieu ; nous devons, si nous sommes sages, ou les protéger toutes également, ou les proscrire toutes de même ; or les proscrire est assurément le plus sûr, puisque nous avons la certitude morale que toutes sont des mômeries dont aucune ne peut plaire plus que l’autre à un Dieu qui n’existe pas.}, l’égalité détruite par la préférence ou la protection accordée à l’une d’elles disparaîtra bientôt du gouvernement, et de la théocratie réédifiée, renaîtra bientôt l’aristocratie. Je ne saurais donc trop le répéter, plus de Dieux, Français, plus de Dieux, si vous ne voulez pas que leur funeste empire vous replonge bientôt dans toutes les horreurs du despotisme, mais ce n’est qu’en vous en moquant que vous les détruirez, tous les dangers qu’ils traînent à leur suite renaîtront aussitôt en foule si vous y menez de l’humeur ou de l’importance. Ne renversez point leurs idoles en colère pulvérisez-les en jouant, et l’opinion tombera d’elle-même.\par
En voilà suffisamment, je l’espère, pour démontrer qu’il ne doit être promulgué aucune loi contre les délits religieux, parce que qui offense une chimère n’offense rien, et qu’il serait de la dernière inconséquence de punir ceux qui outragent ou qui méprisent un culte dont rien ne vous démontre avec évidence la priorité sur les autres ; ce serait nécessairement adopter un parti, et influencer dès lors la balance de l’égalité, première loi de votre nouveau gouvernement.\par
Passons aux seconds devoirs de l’homme, ceux qui le lient avec ses semblables ; cette classe est la plus étendue sans doute.\par
La morale chrétienne trop vague sur les rapports de l’homme avec ses semblables, pose des bases si pleines de sophismes, qu’il nous est impossible de les admettre ; parce que, lorsqu’on veut édifier des principes, il faut bien se garder de leur donner des sophismes pour bases. Elle nous dit, cette absurde morale, d’aimer notre prochain comme nous-même ; rien ne serait assurément plus sublime, s’il était possible que ce qui est faux, pût jamais porter les caractères de la beauté ; il ne s’agit pas d’aimer ses semblables comme soi-même, puisque cela est contre toutes les lois de la nature, et que son seul organe doit diriger toutes les actions de notre vie ; il n’est question que d’aimer nos semblables comme des frères, comme des amis que la nature nous donne, et avec lesquels nous devons vivre d’autant mieux dans un État républicain, que la disparution\footnote{ Sic. {\itshape (Note du correcteur - ELG.)}} des distances doit nécessairement resserrer les liens.\par
Que l’humanité, la fraternité, la bienfaisance nous prescrivent d’après cela nos devoirs réciproques, et remplissons-les individuellement dans le simple degré d’énergie que nous a sur ce point donné la nature, sans blâmer et surtout sans punir ceux qui, plus froids ou plus atrabilaires, n’éprouvent pas dans ces liens néanmoins si touchants toutes les douceurs que d’autres y rencontrent ; car on en conviendra, ce serait ici une absurdité palpable que de vouloir prescrire des lois universelles ; ce procédé serait aussi ridicule que celui d’un général d’armée qui voudrait que tous ses soldats fussent vêtus d’un habit fait sur la même mesure ; c’est une injustice effrayante que d’exiger que des hommes de caractères inégaux se plient à des lois égales ; ce qui va à l’un ne va point à l’autre, je conviens que l’on ne peut pas faire autant de lois qu’il y a d’hommes ; mais les lois peuvent être si douces, en si petit nombre, que tous les hommes de quelque caractère qu’ils soient, puissent facilement s’y plier, encore exigerais-je que ce petit nombre de lois fût d’espèce à pouvoir s’adapter facilement à tous les différents caractères ; l’esprit de celui qui les dirigerait, serait de frapper plus ou moins, en raison de l’individu qu’il faudrait atteindre ; il est démontré qu’il y a telle vertu dont la pratique est impossible à certains hommes, comme il y a tel remède qui ne saurait convenir à tel tempérament ; or quel sera le comble de votre injustice, si vous frappez de la loi celui auquel il est impossible de se plier à la loi ; l’iniquité que vous commettriez en cela, ne serait-elle pas égale à celle dont vous vous rendriez coupable, si vous vouliez forcer un aveugle à discerner les couleurs ? de ces premiers principes il découle, on le sent, la nécessité de faire des lois douces, et surtout d’anéantir pour jamais l’atrocité de la peine de mort, parce que la loi qui attente à la vie d’un homme, est impraticable, injuste, inadmissible ; ce n’est pas, ainsi que je le dirai tout à l’heure, qu’il n’y ait une infinité de cas où, sans outrager la nature (et c’est ce que je démontrerai), les hommes n’aient reçu de cette mère commune l’entière liberté d’attenter à la vie les uns des autres, mais c’est qu’il est impossible que la loi puisse obtenir le même privilège, parce que la loi froide par elle-même, ne saurait être accessible aux passions qui peuvent légitimer dans l’homme la cruelle action du meurtre ; l’homme reçoit de la nature les impressions qui peuvent lui faire pardonner cette action, et la loi au contraire, toujours en opposition à la nature et ne recevant rien d’elle, ne peut être autorisée à se permettre les mêmes écarts ; n’ayant pas les mêmes motifs, il est impossible qu’elle ait les mêmes droits, voilà de ces distinctions savantes et délicates qui échappent à beaucoup de gens, parce que fort peu de gens réfléchissent ; mais elles seront accueillies des gens instruits à qui je les adresse, et elles influeront, je l’espère, sur le nouveau Code que l’on nous prépare.\par
La seconde raison pour laquelle on doit anéantir la peine de mort, c’est qu’elle n’a jamais réprimé le crime, puisqu’on le commet chaque jour aux pieds de l’échafaud.\par
On doit supprimer cette peine, en un mot, parce qu’il n’y a point de plus mauvais calcul que celui de faire mourir un homme pour en avoir tué un autre, puisqu’il résulte évidemment de ce procédé, qu’au lieu d’un homme de moins, en voilà tout d’un coup deux et qu’il n’y a que des bourreaux ou des imbéciles auxquels une telle arithmétique puisse être familière.\par
Quoi qu’il en soit enfin, les forfaits que nous pouvons commettre envers nos frères se réduisent à quatre principaux : la {\itshape calomnie}, le {\itshape vol}, les délits qui, causés par l’{\itshape impureté}, peuvent atteindre désagréablement les autres, et le {\itshape meurtre}.\par
Toutes ces actions considérées comme capitales dans un gouvernement monarchique, sont-elles aussi graves dans un État républicain ? C’est ce que nous allons analyser avec le flambeau de la philosophie, car c’est à sa seule lumière qu’un tel examen doit s’entreprendre ; qu’on ne me taxe point d’être un novateur dangereux, qu’on ne dise pas qu’il y a du risque à émousser, comme le feront peut-être ces écrits, le remords dans l’âme des malfaiteurs, qu’il y a le plus grand mal à augmenter par la douceur de ma morale le penchant que ces mêmes malfaiteurs ont aux crimes ; j’atteste ici formellement n’avoir aucune de ces vues perverses ; j’expose les idées qui depuis l’âge de raison se sont identifiées avec moi et au jet desquelles l’infâme despotisme des tyrans s’était opposé tant de siècles. Tant pis pour ceux que ces grandes idées corrompraient, tant pis pour ceux qui ne savent saisir que le mal dans des opinions philosophiques, susceptibles de se corrompre à tout ; qui sait s’ils ne se gangrèneraient peut-être pas aux lectures de {\itshape Sénèque} et de {\itshape Charron}, ce n’est point à eux que je parle : je ne m’adresse qu’à des génies capables de m’entendre, et ceux-là me liront sans danger.\par
J’avoue avec la plus extrême franchise, que je n’ai jamais cru que la calomnie fût un mal, et surtout dans un gouvernement comme le nôtre, où tous les hommes plus liés, plus rapprochés, ont évidemment un plus grand intérêt à se bien connaître ; de deux choses l’une, ou la calomnie porte sur un homme véritablement pervers, ou elle tombe sur un être vertueux. On conviendra que dans le premier cas, il devient à peu près indifférent que l’on dise un peu plus de mal d’un homme connu pour en faire beaucoup, peut-être même alors le mal qui n’existe pas, éclairera-t-il sur celui qui est, et voilà le malfaiteur mieux connu.\par
S’il règne, je le suppose, une influence malsaine à Hanovre, mais que je ne doive courir d’autres risques, en m’exposant à cette inclémence de l’air, que de gagner un accès de fièvre, pourrai-je savoir mauvais gré à l’homme qui, pour m’empêcher d’y aller, m’aurait dit qu’on y mourait dès en arrivant ? non sans doute, car en m’effrayant par un grand mal, il m’a empêché d’en éprouver un petit.\par
La calomnie porte-t-elle au contraire sur un homme vertueux, qu’il ne s’en alarme pas, qu’il se montre, et tout le venin du calomniateur retombera bientôt sur lui-même. La calomnie, pour de telles gens, n’est qu’un scrutin épuratoire dont leur vertu ne sortira que plus brillante, il y a même ici du profit pour la masse des vertus de la république ; car cet homme vertueux et sensible, piqué de l’injustice qu’il vient d’éprouver, s’appliquera à mieux faire encore ; il voudra surmonter cette calomnie dont il se croyait à l’abri, et ses belles actions n’acquerront qu’un degré d’énergie de plus. Ainsi, dans le premier cas, le calomniateur aura produit d’assez bons effets, en grossissant les vices de l’homme dangereux ; dans le second, il en aura produit d’excellents, en contraignant la vertu à s’offrir à nous tout entière. Or, je demande maintenant sous quel rapport le calomniateur pourra vous paraître à craindre, dans un gouvernement surtout où il est si essentiel de connaître les méchants et d’augmenter l’énergie des bons ? Que l’on se garde donc bien de prononcer aucune peine contre la calomnie, considérons-la sous le double rapport d’un fanal et d’un stimulant, et dans tous les cas comme quelque chose de très utile ; le législateur, dont toutes les idées doivent être grandes comme l’ouvrage auquel il s’applique, ne doit jamais étudier l’effet du délit qui ne frappe qu’individuellement ; c’est son effet en masse qu’il doit examiner, et quand il observera de cette manière les effets qui résultent de la calomnie, je le défie d’y trouver rien de punissable, je défie qu’il puisse placer quelque ombre de justice à la loi qui la punirait, il devient au contraire l’homme le plus juste et le plus intègre, s’il la favorise ou la récompense.\par
Le vol est le second des délits moraux dont nous nous sommes proposé l’examen.\par
Si nous parcourons l’Antiquité, nous verrons le vol permis, récompensé dans toutes les républiques de la Grèce ; Sparte ou Lacédémone le favorisait ouvertement ; quelques autres peuples l’ont regardé comme une vertu guerrière ; il est certain qu’il entretient le courage, la force, l’adresse, toutes les vertus, en un mot, utiles à un gouvernement républicain, et par conséquent au nôtre ; j’oserai demander, sans partialité maintenant, si le vol, dont l’effet est d’égaliser les richesses, est un grand mal dans un gouvernement dont le but est l’égalité : non sans doute, car s’il entretient l’égalité d’un côté, de l’autre il rend plus exact à conserver son bien. Il y avait un peuple qui punissait, non pas le voleur, mais celui qui s’était laissé voler, afin de lui apprendre à soigner ses propriétés : ceci nous amène à des réflexions plus étendues.\par
À dieu ne plaise que je veuille attaquer ou détruire ici le serment du respect des propriétés que vient de prononcer la nation ; mais me permettra-t-on quelques idées sur l’injustice de ce serment ? Quel est l’esprit d’un serment prononcé par tous les individus d’une nation ? N’est-il pas de maintenir une parfaite égalité parmi les citoyens, de les soumettre tous également à la loi protectrice des propriétés de tous ? Or je vous demande maintenant si elle est bien juste, la loi qui ordonne à celui qui n’a rien de respecter celui qui a tout ? Quels sont les éléments du pacte social ? Ne consiste-t-il pas à céder un peu de sa liberté et de ses propriétés, pour assurer et maintenir ce que l’on conserve de l’un et de l’autre ? Toutes les lois sont assises sur ces bases, elles sont les motifs des punitions infligées à celui qui abuse de sa liberté, elles autorisent de même les impositions ; ce qui fait qu’un citoyen ne se récrie pas lorsqu’on les exige de lui, c’est qu’il sait qu’au moyen de ce qu’il donne, on lui conserve ce qui lui reste ; mais, encore une fois, de quel droit celui qui n’a rien s’enchaînera-t-il sous un pacte qui ne protège que celui qui a tout ? Si vous faites un acte d’équité en conservant, par votre serment, les propriétés du riche, ne faites-vous pas une injustice en exigeant ce serment du conservateur qui n’a rien ? Quel intérêt celui-ci a-t-il à votre serment ? Et pourquoi voulez-vous qu’il promette une chose uniquement favorable à celui qui diffère autant de lui par ses richesses ? Il n’est assurément rien de plus injuste, un serment doit avoir un effet égal sur tous les individus qui le prononcent ; il est impossible qu’il puisse enchaîner celui qui n’a aucun intérêt à son maintien, parce qu’il ne serait plus alors le pacte d’un peuple libre, il serait l’arme du fort sur le faible, contre lequel celui-ci devrait se révolter sans cesse ; or c’est ce qui arrive dans le serment du respect des propriétés que vient d’exiger la nation, le riche seul y enchaîne le pauvre, le riche seul a intérêt au serment que prononce le pauvre avec tant d’inconsidération, qu’il ne voit pas qu’au moyen de ce serment extorqué à sa bonne foi, il s’engage à faire une chose qu’on ne peut pas faire vis-à-vis de lui. Convaincus ainsi que vous devez l’être, de cette barbare inégalité, n’aggravez donc pas votre injustice en punissant celui qui n’a rien, d’avoir osé dérober quelque chose à celui qui a tout, votre inéquitable serment lui en donne plus de droit que jamais ; en le contraignant au parjure par ce serment absurde pour lui, vous légitimez tous les crimes où le portera ce parjure, il ne vous appartient donc plus de punir ce dont vous avez été la cause ; je n’en dirai pas davantage pour faire sentir la cruauté horrible qu’il y a à punir les voleurs. Imitez la loi sage du peuple dont je viens de parler, punissez l’homme assez négligent pour se laisser voler, mais ne prononcez aucune espèce de peine contre celui qui vole, songez que votre serment l’autorise à cette action, et qu’il n’a fait en s’y livrant, que suivre le premier et le plus sage des mouvements de la nature, celui de conserver sa propre existence, n’importe aux dépens de qui.\par
Les délits que nous devons examiner dans cette seconde classe des devoirs de l’homme envers ses semblables, consistent dans les actions que peut faire entreprendre le libertinage, parmi lesquelles se distinguent particulièrement, comme plus attentatoires à ce que chacun doit aux autres, la {\itshape prostitution}, l’{\itshape adultère}, l’{\itshape inceste}, le {\itshape viol} et la {\itshape sodomie}. Nous ne devons certainement pas douter un moment, que tout ce qui s’appelle crimes moraux, c’est-à-dire toutes les actions de l’espèce de celles que nous venons de citer, ne soient parfaitement indifférentes dans un gouvernement, dont le seul devoir consiste à conserver, par tel moyen que ce puisse être, la forme essentielle à son maintien : voilà l’unique morale d’un gouvernement républicain ; or, puisqu’il est toujours contrarié par les despotes qui l’environnent, on ne saurait imaginer raisonnablement que ses moyens conservateurs puissent être des {\itshape moyens moraux} ; car il ne se conservera que par la guerre, et rien n’est moins moral que la guerre ; maintenant je demande comment on parviendra à démontrer que, dans un état {\itshape immoral} par ses obligations, il soit essentiel que les individus soient {\itshape moraux}, je dis plus, il est bon qu’ils ne le soient pas, les législateurs de la Grèce avaient parfaitement senti l’importante nécessité de gangrener les membres pour que, leur {\itshape dissolution morale} influant sur celle utile à la machine, il en résultât l’insurrection toujours indispensable dans un gouvernement qui, parfaitement heureux comme le gouvernement républicain, doit nécessairement exciter la haine et la jalousie de tout ce qui l’entoure. L’insurrection, pensaient ces sages législateurs, n’est point un état {\itshape moral} ; il doit être pourtant l’état permanent d’une république ; il serait donc aussi absurde que dangereux d’exiger que ceux qui doivent maintenir le perpétuel ébranlement {\itshape immoral} de la machine, fussent eux-mêmes des êtres très {\itshape moraux}, parce que l’état {\itshape moral} d’un homme est un état de paix et de tranquillité, au lieu que son état {\itshape immoral} est un état de mouvement perpétuel qui le rapproche de l’insurrection nécessaire dans laquelle il faut que le républicain tienne toujours le gouvernement dont il est membre.\par
Détaillons maintenant, et commençons par analyser la pudeur, ce mouvement pusillanime, contradictoire aux affections impures. S’il était dans les intentions de la nature que l’homme fût pudique, assurément elle ne l’aurait pas fait naître nu ; une infinité de peuples, moins dégradés que nous par la civilisation, vont nus et n’en éprouvent aucune honte ; il ne faut pas douter que l’usage de se vêtir n’ait eu pour unique base et l’inclémence de l’air et la coquetterie des femmes ; elles sentirent qu’elles perdraient bientôt tous les effets du désir, si elles les prévenaient, au lieu de les laisser naître, elles conçurent que la nature d’ailleurs ne les ayant pas créées sans défauts, elles s’assureraient bien mieux tous les moyens de plaire, en déguisant ces défauts par des parures ; ainsi la pudeur, loin d’être une vertu, ne fut donc plus qu’un des premiers effets de la corruption, qu’un des premiers moyens de la coquetterie des femmes. Lycurgue et Solon, bien pénétrés que les résultats de l’impudeur tiennent le citoyen dans l’état {\itshape immoral} essentiel aux lois du gouvernement républicain, obligèrent les jeunes filles à se montrer nues aux théâtres\footnote{ On a dit que l’intention de ces législateurs était, en émoussant la passion que les hommes éprouvent pour une fille nue, de rendre plus active celle que les hommes éprouvent quelquefois pour leur sexe ; ces sages faisaient montrer ce dont ils voulaient que l’on se dégoûtât, et cacher de qu’ils croyaient fait pour inspirer de plus doux désirs ; dans tous les cas, ne travaillaient-ils pas au but que nous venons de dire ? Ils sentaient, on le voit, le besoin de l’immoralité dans les mœurs républicaines.}. Rome imita bientôt cet exemple, on dansait nu aux jeux de Flore, la plus grande partie des mystères païens se célébraient ainsi, la nudité passa même pour vertu chez quelques peuples. Quoi qu’il en soit, de l’impudeur naissent des penchants luxurieux, ce qui résulte de ces penchants compose les prétendus crimes que nous analysons, dont la prostitution est le premier effet. Maintenant que nous sommes revenus sur tout cela de la foule d’erreurs religieuses qui nous captivaient et que, plus rapprochés de la nature par la quantité de préjugés que nous venons d’anéantir, nous n’écoutons que sa voix, bien assurés que s’il y avait du crime à quelque chose, ce serait bien plutôt à résister aux penchants qu’elle nous inspire, qu’à les combattre, persuadés que la luxure étant une suite de ces penchants, il s’agit bien moins d’éteindre cette passion dans nous, que de régler les moyens d’y satisfaire en paix ; nous devons donc nous attacher à mettre de l’ordre dans cette partie, à y établir toute la sûreté nécessaire à ce que le citoyen, que le besoin rapproche des objets de luxure, puisse se livrer avec ces objets à tout ce que ses passions lui prescrivent, sans jamais être enchaîné par rien, parce qu’il n’est aucune passion dans l’homme qui ait plus besoin de toute l’extension de la liberté, que celle-là. Différents emplacements sains, vastes, proprement meublés, et sûrs dans tous les points, seront érigés dans les villes ; là, tous les sexes, tous les âges, toutes les créatures seront offertes aux caprices des libertins qui viendront jouir, et la plus entière subordination sera la règle des individus présentés ; le plus léger refus sera puni aussitôt arbitrairement par celui qui l’aura éprouvé, je dois encore expliquer ceci, le mesurer aux mœurs républicaines ; j’ai promis partout la même logique, je tiendrai parole. Si, comme je viens de le dire tout à l’heure, aucune passion n’a plus besoin de toute l’extension de la liberté que celle-là, aucune sans doute n’est aussi despotique ; c’est là que l’homme aime à commander, à être obéi, à s’entourer d’esclaves contraints à le satisfaire ; or, toutes les fois que vous ne donnerez pas à l’homme le moyen secret d’exhaler la dose de despotisme que la nature mit au fond de son cœur, il se rejettera, pour l’exercer, sur les objets qui l’entoureront, il troublera le gouvernement. Permettez, si vous voulez éviter ce danger, un libre essor à ces désirs tyranniques qui, malgré lui, le tourmentent sans cesse ; content d’avoir pu exercer sa petite souveraineté au milieu du harem d’icoglans ou de sultanes que vos soins et son argent lui soumettent, il sortira satisfait, et sans aucun désir de troubler un gouvernement qui lui assure aussi complaisamment tous les moyens de sa concupiscence ; exercez, au contraire, des procédés différents, imposez sur ces objets de la luxure publique, les ridicules entraves jadis inventées par la tyrannie ministérielle et par la lubricité de nos Sardanapales\footnote{ On sait que l’infâme et scélérat Sartine composait à Louis XV des moyens de luxure, en lui faisant lire trois fois la semaine, par la Dubaril, le détail privé et enrichi par lui de tout ce qui se passait dans les mauvais lieux de Paris ; cette branche de libertinage du Néron français coûtait trois millions à l’État !}. L’homme, bientôt aigri contre votre gouvernement, bientôt jaloux du despotisme qu’il vous voit exercer tout seul, secouera le joug que vous lui imposez et las de votre manière de le régir, en changera comme il vient de le faire. Voyez comme les législateurs grecs, bien pénétrés de ces idées, traitaient la débauche à Lacédémone, à Athènes, ils en enivraient le citoyen, bien loin de la lui interdire ; aucun genre de lubricité ne lui était défendu, et Socrate, déclaré par l’oracle le plus sage des philosophes de la terre, passant indifféremment des bras d’{\itshape Aspasie} dans ceux d’{\itshape Alcibiade}, n’en était pas moins la gloire de la Grèce. Je vais aller plus loin, et quelque contraires que soient mes idées à nos coutumes actuelles, comme mon objet est de prouver que nous devons nous presser de changer ces coutumes, si nous voulons conserver le gouvernement adopté, je vais essayer de vous convaincre que la prostitution des femmes connues sous le nom d’honnêtes, n’est pas plus dangereuse que celle des hommes, et que non seulement nous devons les associer aux luxures exercées dans les maisons que j’établis, mais que nous devons même en ériger pour elles, où leurs caprices et les besoins de leur tempérament, bien autrement ardent que le nôtre, puissent de même se satisfaire avec tous les sexes.\par
De quel droit prétendez-vous d’abord que les femmes doivent être exceptées de l’aveugle soumission que la nature leur prescrit aux caprices des hommes, et ensuite par quel autre droit prétendez-vous les asservir à une continence impossible à leur physique, et absolument inutile à leur honneur ?\par
Je vais traiter séparément l’une et l’autre de ces questions. Il est certain que, dans l’état de nature, les femmes naissent {\itshape vulgivagues}, c’est-à-dire jouissant des avantages des autres animaux femelles et appartenant, comme elles et sans aucune exception, à tous les mâles ; telles furent sans aucun doute, et les premières lois de la nature, et les seules institutions des premiers rassemblements que les hommes firent. L’{\itshape intérêt}, l’{\itshape égoïsme} et l’{\itshape amour} dégradèrent ces premières vues si simples et si naturelles ; on crut s’enrichir en prenant une femme, et avec elle le bien de sa famille ; voilà les deux premiers sentiments que je viens d’indiquer satisfaits, plus souvent encore on enleva cette femme, et on s’y attacha ; voilà le second motif en action et, dans tous les cas, de l’injustice. Jamais un acte de possession ne peut être exercé sur un être libre ; il est aussi injuste de posséder exclusivement une femme, qu’il l’est de posséder des esclaves ; tous les hommes sont nés libres, tous sont égaux en droit, ne perdons jamais de vue ces principes ; il ne peut donc être jamais donné, d’après cela, de droit légitime à un sexe de s’emparer exclusivement de l’autre, et jamais l’un de ces sexes, ou l’une de ces classes, ne peut posséder l’autre arbitrairement. Une femme même, dans la pureté des lois de la nature, ne peut alléguer pour motif du refus qu’elle fait à celui qui la désire, l’amour qu’elle a pour un autre, parce que ce motif en devient un d’exclusion, et qu’aucun homme ne peut être exclu de la possession d’une femme, du moment qu’il est clair qu’elle appartient décidément à tous les hommes. L’acte de possession ne peut être exercé que sur un immeuble ou sur un animal, jamais il ne peut l’être sur un individu qui nous ressemble, et tous les liens qui peuvent enchaîner une femme à un homme, de telle espèce que vous puissiez les supposer, sont aussi injustes que chimériques. S’il devient donc incontestable que nous avons reçu de la nature le droit d’exprimer nos vœux indifféremment à toutes les femmes, il le devient de même que nous avons celui de l’obliger de se soumettre à nos vœux, non pas exclusivement, je me contrarierais, mais momentanément\footnote{ Qu’on ne dise pas ici que je me contrarie, et qu’après avoir établi plus haut que nous n’avions aucun droit de lier une femme à nous, je détruis ces principes en disant maintenant que nous avons le droit de la contraindre ; je répète qu’il ne s’agit ici que de la jouissance, et non de la propriété ; je n’ai nul droit sur la propriété de cette fontaine que je rencontre dans mon chemin, mais j’ai des droits certains sur sa jouissance ; j’ai le droit de profiter de l’eau limpide qu’elle offre à ma soif ; je n’ai de même aucun droit réel sur la propriété de telle ou telle femme, mais j’en ai d’incontestables à sa jouissance, j’en ai de la contraindre à cette jouissance, si elle me la refuse par tel motif que ce puisse être.}. Il est incontestable que nous avons le droit d’établir des lois qui la contraignent de céder aux feux de celui qui la désire ; la violence même étant un des effets de ce droit, nous pouvons l’employer légalement. Eh ! la nature n’a-t-elle pas prouvé que nous avions ce droit, en nous départissant la force nécessaire à les soumettre à nos désirs ?\par
En vain les femmes doivent-elles faire parler pour leur défense, ou la pudeur ou leur attachement à d’autres hommes ; ces moyens chimériques sont nuls ; nous avons vu plus haut combien la pudeur était un sentiment factice et méprisable ; l’amour, qu’on peut appeler la {\itshape folie de l’âme}, n’a pas plus de titres pour légitimer leur constance, ne satisfaisant que deux individus, l’être aimé et l’être aimant ; il ne peut servir au bonheur des autres, et c’est pour le bonheur de tous, et non pour un bonheur égoïste et privilégié, que nous ont été données les femmes. Tous les hommes ont donc un droit de jouissance égal sur toutes les femmes ; il n’est donc aucun homme qui, d’après les lois de la nature, puisse s’ériger sur une femme un droit unique et personnel ; la loi qui les obligera de se prostituer, tant que nous le voudrons, aux maisons de débauche dont il vient d’être question, et qui les y contraindra si elles s’y refusent, qui les punira si elles y manquent, est donc une loi des plus équitables, et contre laquelle aucun motif légitime ou juste ne saurait réclamer. Un homme qui voudra jouir d’une femme ou d’une fille quelconque, pourra donc, si les lois que vous promulguez sont justes, la faire sommer de se trouver dans l’une des maisons dont j’ai parlé et là, sous la sauvegarde des matrones de ce temple de Vénus, elle lui sera livrée pour satisfaire, avec autant d’humilité que de soumission, tous les caprices qu’il lui plaira de se passer avec elle, de quelque bizarrerie ou de quelque irrégularité qu’ils puissent être, parce qu’il n’en est aucun qui ne soit dans la nature, aucun qui ne soit avoué par elle. Il ne s’agirait plus ici que de fixer l’âge ; or, je prétends qu’on ne le peut, sans gêner la liberté de celui qui désire la jouissance d’une fille de tel ou tel âge. Celui qui a le droit de manger le fruit d’un arbre, peut assurément le cueillir mûr ou vert, suivant les inspirations de son goût ; mais, objectera-t-on, il est un âge où les procédés de l’homme nuiront décidément à la santé de la fille ; cette considération est sans aucune valeur, dès que vous m’accordez le droit de propriété sur la jouissance, ce droit est indépendant des effets produits par la jouissance, de ce moment il devient égal que cette jouissance soit avantageuse ou nuisible à l’objet qui doit s’y soumettre. N’ai-je pas déjà prouvé qu’il était égal de contraindre la volonté d’une femme sur cet objet, et qu’aussitôt qu’elle inspirait le désir de la jouissance, elle devait se soumettre à cette jouissance, abstraction faite de tout sentiment égoïste ; il en est de même de sa santé, dès que les égards qu’on aurait pour cette considération détruiraient ou affaibliraient la jouissance de celui qui la désire, et qui a le droit de se l’approprier, cette considération d’âge devient nulle, parce qu’il ne s’agit nullement ici de ce que peut éprouver l’objet condamné par la nature et par la loi à l’assouvissement momentané des désirs de l’autre, il n’est question, dans cet examen, que de ce qui convient à celui qui désire ; nous rétablirons la balance.\par
Oui, nous la rétablirons, nous le devons sans doute ; ces femmes que nous venons d’asservir si cruellement, nous devons incontestablement les dédommager, et c’est ce qui va former la réponse à la seconde question que je me suis proposée.\par
Si nous admettons, comme nous venons de le faire, que toutes les femmes doivent être soumises à nos désirs, assurément nous pouvons leur permettre de même de satisfaire amplement tous les leurs ; nos lois doivent favoriser sur cet objet leur tempérament de feu, et il est absurde d’avoir placé et leur honneur et leur vertu dans la force antinaturelle qu’elles mettent à résister aux penchants qu’elles ont reçus avec bien plus de profusion que nous ; cette injustice de nos mœurs est d’autant plus criante, que nous consentons à la fois à les rendre faibles à force de séduction, et à les punir ensuite de ce qu’elles cèdent à tous les efforts que nous avons faits pour les provoquer à la chute. Toute l’absurdité de nos mœurs est gravée, ce me semble, dans cette inéquitable atrocité, et ce seul exposé devrait nous faire sentir l’extrême besoin que nous avons de les changer pour de plus pures.\par
Je dis donc que les femmes, ayant reçu des penchants bien plus violents que nous aux plaisirs de la luxure, pourront s’y livrer tant qu’elles le voudront, absolument dégagées de tous les liens de l’hymen, de tous les faux préjugés de la pudeur, absolument rendues à l’état de nature ; je veux que les lois leur permettent de se livrer à autant d’hommes que bon leur semblera ; je veux que la jouissance de tous les sexes et de toutes les parties de leur corps leur soit permise comme aux hommes, et sous la clause spéciale de se livrer de même à tous ceux qui le désireront, il faut qu’elles aient la liberté de jouir également de tous ceux qu’elles croiront dignes de les satisfaire. Quels sont, je le demande, les dangers de cette licence ? Des enfants qui n’auront point de pères ? et qu’importe dans une république où tous les individus ne doivent avoir d’autre mère que la patrie, où tous ceux qui naissent, sont tous enfants de la patrie ? Ah ! combien l’aimeraient mieux ceux qui, n’ayant jamais connu qu’elle, sauront dès en naissant que ce n’est que d’elle qu’ils doivent tout attendre ; n’imaginez pas de faire de bons républicains tant que vous isolerez dans leurs familles les enfants qui ne doivent appartenir qu’à la république, en donnant là seulement à quelques individus, la dose d’affection qu’ils doivent répartir sur tous leurs frères, ils adoptent inévitablement les préjugés souvent dangereux de ces individus, leurs opinions, leurs idées s’isolent, se particularisent, et toutes les vertus d’un homme d’État leur deviennent absolument impossibles ; abandonnant enfin leur cœur tout entier à ceux qui les ont fait naître ; ils ne trouvent plus dans ce cœur aucune affection pour celle qui doit les faire vivre, les faire connaître et les illustrer. Comme si ces seconds bienfaits n’étaient pas plus importants que le premier ; s’il y a le plus grand inconvénient à laisser des enfants sucer ainsi dans leurs familles des intérêts souvent bien différents de ceux de la patrie, il y a donc le plus grand avantage à les en séparer ; ne le sont-ils pas naturellement par les moyens que je propose, puisqu’en détruisant absolument tous les liens de l’hymen il ne naît plus d’autres fruits des plaisirs de la femme que des enfants auxquels la connaissance de leur père est absolument interdite, et avec cela les moyens de ne plus appartenir qu’à une même famille, au lieu d’être ainsi qu’ils le doivent uniquement les enfants de la patrie ?\par
Il y aura donc des maisons destinées au libertinage des femmes, et, comme celles des hommes, sous la protection du gouvernement ; là, leur seront fournis tous les individus de l’un et l’autre sexe qu’elles pourront désirer, et plus elles fréquenteront ces maisons, plus elles seront estimées ; il n’y a rien de si barbare et de si ridicule que d’avoir attaché l’honneur et la vertu des femmes à la résistance qu’elles mettent à des désirs qu’elles ont reçus de la nature, et qu’échauffent sans cesse ceux qui ont la barbarie de les blâmer ; dès l’âge le plus tendre\footnote{ Les Babyloniennes n’attendaient pas sept ans pour porter leurs prémices au temple de Vénus ; le premier mouvement de concupiscence qu’éprouve une jeune fille, est l’époque que la nature lui indique pour se prostituer, et sans aucune autre espèce de considération, elle doit céder dès que sa nature parle ; elle en outrage les lois si elle résiste.}, une fille dégagée des liens paternels, n’ayant plus rien à conserver pour l’hymen (absolument aboli par les sages lois que je désire), au-dessus du préjugé enchaînant autrefois son sexe, pourra donc se livrer à tout ce que lui dictera son tempérament, dans les maisons établies à ce sujet. Elle y sera reçue avec respect, satisfaite avec profusion, et de retour dans la société, elle y pourra parler aussi publiquement des plaisirs qu’elle aura goûtés, qu’elle le fait aujourd’hui d’un bal ou d’une promenade ; sexe charmant, vous serez libre ; vous jouirez comme les hommes de tous les plaisirs dont la nature vous fait un devoir ; vous ne vous contraindrez sur aucun, la plus divine partie de l’humanité doit-elle donc recevoir des fers de l’autre ? Ah ! brisez-les, la nature le veut ; n’ayez plus d’autres freins que celui de vos penchants, d’autres lois que vos seuls désirs, d’autre morale que celle de la nature ; ne languissez pas plus longtemps dans des préjugés barbares qui flétrissaient vos charmes, et captivaient les élans divins de vos cœurs\footnote{ Les femmes ne savent pas à quel point leurs lascivités les embellissent ; que l’on compare deux femmes d’âge et de beauté à peu près semblables, dont l’une vit dans le célibat et l’autre dans le libertinage ; on verra combien cette dernière l’emportera d’éclat et de fraîcheur ; toute violence faite à la nature use bien plus que l’abus des plaisirs ; il n’y a personne qui ne sache que les couches embellissent une femme.} ; vous êtes libres comme nous, et la carrière des combats de Vénus vous est ouverte comme à nous ; ne redoutez plus d’absurdes reproches ; le pédantisme et la superstition sont anéantis ; on ne vous verra plus rougir de vos charmants écarts. Couronnées de myrtes et de roses, l’estime que nous concevrons pour vous, ne sera plus qu’en raison de la plus grande étendue que vous vous serez permis de leur donner.\par
Ce qui vient d’être dit, devrait nous dispenser sans doute d’examiner l’adultère ; jetons-y néanmoins un coup d’œil, quelque nul qu’il soit après les lois que j’établis ; à quel point il était ridicule de le considérer comme criminel dans nos anciennes institutions ; s’il y avait quelque chose d’absurde dans le monde, c’était bien sûrement l’éternité des liens conjugaux ; il ne fallait, ce me semble, qu’examiner ou que sentir toute la lourdeur de ces liens pour cesser de voir comme un crime l’action qui les allégeait ; la nature, comme nous l’avons dit tout à l’heure, ayant doué les femmes d’un tempérament plus ardent, d’une sensibilité plus profonde qu’elle n’a fait des individus de l’autre sexe, c’était pour elles sans doute que le joug d’un hymen éternel était plus pesant ; femmes tendres et embrasées du feu de l’amour, dédommagez-vous maintenant sans crainte ; persuadez-vous qu’il ne peut exister aucun mal à suivre les impulsions de la nature, que ce n’est pas pour un seul homme qu’elle vous a créées, mais pour plaire indifféremment à tous, qu’aucun frein ne vous arrête ; imitez les républicaines de la Grèce ; jamais les législateurs qui leur donnèrent des lois, n’imaginèrent de leur faire un crime de l’adultère et presque tous autorisèrent le désordre des femmes. {\itshape Thomas Morus} prouve, dans son {\itshape Utopie}, qu’il est avantageux aux femmes de se livrer à la débauche, et les idées de ce grand homme n’étaient pas toujours des rêves\footnote{ Le même voulait que les fiancés se vissent tout nus avant de s’épouser ; que de mariages manqueraient si cette loi s’exécutait ; on avouera que le contraire est bien ce qu’on appelle acheter de la marchandise sans la voir.} ; chez les Tartares, plus une femme se prostituait, plus elle était honorée ; elle portait publiquement au col les marques de son impudicité, et l’on n’estimait point celles qui n’en étaient point décorées ; au Pégu, les familles elles-mêmes livrent leurs femmes ou leurs filles aux étrangers qui y voyagent ; on les loue à tant par jour comme des chevaux et des voitures ; des volumes enfin ne suffiraient pas à démontrer que jamais la luxure ne fut considérée comme criminelle chez aucun des peuples sages de la terre, tous les philosophes savent bien que ce n’est qu’aux imposteurs chrétiens que nous devons de l’avoir érigé[e] en crime ; les prêtres avaient bien leur motif, en nous interdisant la luxure ; cette recommandation en leur réservant la connaissance et l’absolution de ces péchés secrets, leur donnait un incroyable empire sur les femmes, et leur ouvrait une carrière de lubricité dont l’étendue n’avait point de bornes. On sait comment ils en profitèrent, et comme ils en abuseraient encore si leur crédit n’était pas perdu sans ressource.\par
L’inceste est-il plus dangereux ? Non, sans doute, il étend les liens des familles, et rend par conséquent plus actif l’amour des citoyens pour la patrie, il nous est dicté par les premières lois de la nature, nous l’éprouvons, et la jouissance des objets qui nous appartiennent, nous sembla toujours plus délicieuse ; les premières institutions favorisent l’inceste ; on le trouve dans l’origine des sociétés ; il est consacré dans toutes les religions ; toutes les lois l’ont favorisé ; si nous parcourons l’univers, nous trouverons l’inceste établi partout ; les nègres de la Côte-du-Poivre et de Rio-Gabon prostituent leurs femmes à leurs propres enfants ; l’aîné des fils au royaume de Juda, doit épouser la femme de son père ; les peuples du Chili couchent indifféremment avec leurs sœurs, leurs filles, et épousent souvent à la fois et la mère et la fille ; j’ose assurer en un mot que l’inceste devrait être la loi de tout gouvernement dont la fraternité fait la base ; comment des hommes raisonnables purent-ils porter l’absurdité au point de croire que la jouissance de sa mère, de sa sœur, ou de sa fille pourrait jamais devenir criminelle, n’est-ce pas, je vous le demande, un abominable préjugé que celui qui paraît faire un crime à un homme d’estimer plus pour sa jouissance, l’objet dont le sentiment de la nature le rapproche davantage, il vaudrait autant dire qu’il nous est défendu d’aimer trop les individus que la nature nous enjoint d’aimer le mieux, et que plus elle nous donne de penchants pour un objet, plus elle nous ordonne en même temps de nous en éloigner ; ces contrariétés sont absurdes ; il n’y a que des peuples abrutis par la superstition, qui puissent les croire ou les adopter ; la communauté des femmes que j’établis, entraînant nécessairement l’inceste, il reste peu de chose à dire sur un prétendu délit dont la nullité est trop démontrée pour s’y appesantir davantage, et nous allons passer au viol, qui semble être au premier coup d’œil de tous les écarts du libertinage, celui dont la lésion est la mieux établie, en raison de l’outrage qu’il paraît faire. Il est pourtant certain que le viol, action si rare et si difficile à prouver, fait moins de tort au prochain que le vol, puisque celui-ci envahit la propriété que l’autre se contente de détériorer ; qu’aurez-vous d’ailleurs à objecter au violateur, s’il vous répond qu’au fait le mal qu’il a commis est bien médiocre, puisqu’il n’a fait que placer un peu plus tôt l’objet dont il a abusé, au même état où l’aurait bientôt mis l’hymen ou l’amour ?\par
Mais la sodomie, mais ce prétendu crime qui attira le feu du ciel sur les villes qui y étaient adonnées, n’est-il point un égarement monstrueux, dont le châtiment ne saurait être assez fort ? Il est sans doute bien douloureux pour nous d’avoir à reprocher à nos ancêtres les meurtres judiciaires qu’ils ont osé se permettre à ce sujet ; est-il possible d’être assez barbare pour oser condamner à mort un malheureux individu dont tout le crime est de ne pas avoir les mêmes goûts que vous ? On frémit lorsqu’on pense qu’il n’y a pas encore quarante ans que l’absurdité des législateurs en était encore là. Consolez-vous, citoyens, de telles absurdités n’arriveront plus, la sagesse de vos législateurs vous en répond. Entièrement éclairci sur cette faiblesse de quelques hommes, on sent bien aujourd’hui qu’une telle erreur ne peut être criminelle, et que la nature ne saurait avoir mis au fluide qui coule dans nos reins une assez grande importance, pour se courroucer sur le chemin qu’il nous plaît de faire prendre à cette liqueur. Quel est le seul crime qui puisse exister ici ? Assurément ce n’est pas de se placer dans tel ou tel lieu, à moins qu’on ne voulût soutenir que toutes les parties du corps ne se ressemblent point, et qu’il en est de pures et de souillées ; mais comme il est impossible d’avancer de telles absurdités, le seul prétendu délit ne saurait consister ici que dans la perte de la semence ; or, je demande s’il est vraisemblable que cette semence soit tellement précieuse aux yeux de la nature, qu’il devienne impossible de la perdre sans crime, procéderait-elle tous les jours à ces pertes si cela était ? et n’est-ce pas les autoriser que de les permettre dans les rêves, dans l’acte de la jouissance d’une femme grosse ? Est-il possible d’imaginer que la nature nous donnât la possibilité d’un crime qui l’outragerait ? est-il possible qu’elle consente à ce que les hommes détruisent ses plaisirs, et deviennent par là plus forts qu’elle ? Il est inouï dans quel gouffre d’absurdités l’on se jette, quand on abandonne, pour raisonner, les secours du flambeau de la raison. Tenons-nous donc pour bien assurés qu’il est aussi simple de jouir d’une femme d’une manière que de l’autre, qu’il est absolument indifférent de jouir d’une fille ou d’un garçon, et qu’aussitôt qu’il est constant qu’il ne peut exister en nous d’autres penchants que ceux que nous tenons de la nature, elle est trop sage et trop conséquente pour en avoir mis dans nous qui puissent jamais l’offenser.\par
Celui de la sodomie est le résultat de l’organisation, et nous ne contribuons pour rien à cette organisation ; des enfants de l’âge le plus tendre annoncent ce goût, et ne s’en corrigent jamais, quelquefois il est le fruit de la satiété ; mais, dans ce cas même, en appartient-il moins à la nature ? Sous tous les rapports il est son ouvrage, et, dans tous les cas, ce qu’elle inspire doit être respecté par les hommes. Si, par un recensement exact, on venait à prouver que ce goût affecte infiniment plus que l’autre, que les plaisirs qui en résultent sont beaucoup plus vifs, et qu’en raison de cela ses sectateurs sont mille fois plus nombreux que ses ennemis, ne serait-il pas possible de conclure alors que, loin d’outrager la nature, ce vice servirait ses vues, et qu’elle tient bien moins à la progéniture que nous n’avons la folie de le croire ; or, en parcourant l’univers, que de peuples ne voyons-nous pas mépriser les femmes ; il en est qui ne s’en servent absolument que pour avoir l’enfant nécessaire à les remplacer. L’habitude que les hommes ont de vivre ensemble dans les républiques, y rendra toujours ce vice plus fréquent, mais il n’est certainement pas dangereux. Les législateurs de la Grèce l’auraient-ils introduit dans leur République, s’ils l’avaient cru tel ? Bien loin de là, ils le croyaient nécessaire à un peuple guerrier. Plutarque nous parle avec enthousiasme du bataillon des {\itshape amants} et des {\itshape aimés}, eux seuls défendirent longtemps la liberté de la Grèce. Ce vice régna dans l’association des frères d’armes, il la cimenta, les plus grands hommes y furent enclins. L’Amérique entière, lorsqu’on la découvrit, se trouva peuplée de gens de ce goût ; à la Louisiane, chez les Illinois, des Indiens vêtus en femmes se prostituaient comme des courtisanes ; les nègres de Bengale entretiennent publiquement des hommes, presque tous les sérails d’Alger ne sont plus aujourd’hui peuplés que de jeunes garçons. On ne se contentait pas de tolérer, on ordonnait à Thèbes l’amour des garçons ; le philosophe de {\itshape Chéronée} le prescrivit pour adoucir les mœurs des jeunes gens ; nous savons à quel point il régna dans Rome : on y trouvait des lieux publics où de jeunes garçons se prostituaient sous l’habit de filles, et de jeunes filles sous celui de garçons. Martial, Catulle, Tibulle, Horace et Virgile écrivaient à des hommes comme à leurs maîtresses, et nous lisons enfin dans Plutarque\footnote{{\itshape Œuvres morales}, « Traité de l’amour ».} que les femmes ne doivent avoir aucune part à l’amour des hommes. Les Amasiens de l’île de Crète enlevaient autrefois des jeunes garçons avec les plus singulières cérémonies. Quand ils en aimaient un, ils en faisaient part aux parents le jour où le ravisseur voulait l’enlever ; le jeune homme faisait quelque résistance si son amant ne lui plaisait pas ; dans le cas contraire, il partait avec lui, et le séducteur le renvoyait à sa famille sitôt qu’il s’en était servi ; car dans cette passion, comme dans celle des femmes, on en a toujours trop dès qu’on en a assez. Strabon nous dit que dans cette même île, ce n’était qu’avec des garçons que l’on remplissait les sérails, on les prostituait publiquement. Veut-on une dernière autorité faite pour prouver combien ce vice est utile dans une république ? Écoutons {\itshape Jérôme le péripatéticien} ; l’amour des garçons, nous dit-il, se répandit dans toute la Grèce, parce qu’il donnait du courage et de la force, et qu’il servait à chasser les tyrans ; les conspirations se formaient entre les amants, et ils se laissaient plutôt torturer, que de révéler leurs complices ; le patriotisme sacrifiait ainsi tout à la prospérité de l’État, on était certain que ces liaisons affermissaient la République, on déclamait contre les femmes, et c’était une faiblesse réservée au despotisme que de s’attacher à de telles créatures. Toujours la pédérastie fut le vice des peuples guerriers ; César nous apprend que les Gaulois y étaient extraordinairement adonnés : les guerres qu’avaient à soutenir les Républiques, en séparant les deux sexes, propagèrent ce vice, et quand on y reconnut des suites si utiles à l’État, la religion le consacra bientôt ; on sait que les Romains sanctifièrent les amours de Jupiter et de Ganymède ; {\itshape Sextus Empiricus} nous assure que cette fantaisie était ordonnée chez les Perses ; enfin les femmes, jalouses et méprisées, offrirent à leurs maris de leur rendre le même service qu’ils recevaient des jeunes garçons, quelques-uns l’essayèrent, et revinrent à leurs anciennes habitudes, ne trouvant pas l’illusion possible. Les Turcs, fort enclins à cette dépravation que Mahomet consacra dans son Alcoran, assurent néanmoins qu’une très jeune vierge peut assez bien remplacer un garçon, et rarement les leurs deviennent femmes avant que d’avoir passé par cette épreuve. Sixte-Quint et Sanchez permirent cette débauche, ce dernier entreprit même de prouver qu’elle était utile à la propagation, et qu’un enfant créé après cette course préalable en devenait infiniment mieux constitué ; enfin les femmes se dédommagèrent entre elles, cette fantaisie sans doute n’a pas plus d’inconvénients que l’autre, parce que le résultat n’en est que le refus de créer, et que les moyens de ceux qui ont le goût de la population sont assez puissants pour que les adversaires n’y puissent jamais nuire ; les Grecs appuyaient de même cet égarement des femmes, sur des raisons d’État ; il en résultait que se suffisant entre elles, leurs communications avec les hommes étaient moins fréquentes, et qu’elles ne nuisaient point ainsi aux affaires de la république, Lucien nous apprend quel progrès fit cette licence, et ce n’est pas sans intérêt que nous la voyons dans Sapho. Il n’est, en un mot, aucune sorte de danger dans toutes ces manies, se portassent-elles même plus loin, allassent-elles jusqu’à caresser des monstres et des animaux, ainsi que nous l’apprend l’exemple de plusieurs peuples ; il n’y aurait pas dans toutes ces fadaises le plus petit inconvénient, parce que la corruption des mœurs souvent très utile dans un gouvernement, ne saurait y nuire sous aucun rapport, et nous devons attendre de nos législateurs assez de sagesse, assez de prudence pour être bien sûrs qu’aucune loi n’émanera d’eux pour la répression de ces misères, qui tenant absolument à l’organisation, ne sauraient jamais rendre plus coupable celui qui y est enclin, que ne l’est l’individu que la nature créa contrefait.\par
Il ne nous reste plus que le meurtre à examiner dans la seconde classe des délits de l’homme envers son semblable, et nous passerons ensuite à ses devoirs envers lui-même. De toutes les offenses que l’homme peut faire à son semblable, le meurtre est, sans contredit, la plus cruelle de toutes, puisqu’il lui enlève le seul bien qu’il ait reçu de la nature, le seul dont la perte soit irréparable. Plusieurs questions néanmoins se présentent ici, abstraction faite du tort que le meurtre cause à celui qui en devient la victime.\par
1° Cette action, eu égard aux seules lois de la nature, est-elle vraiment criminelle ?\par
2° L’est-elle relativement aux lois de la politique ?\par
3° Est-elle nuisible à la société ?\par
4° Comment doit-elle être considérée dans un gouvernement républicain ?\par
5° Enfin le meurtre doit-il être réprimé par le meurtre ?\par
Nous allons examiner séparément chacune de ces questions, l’objet est assez essentiel pour qu’on nous permette de nous y arrêter ; on trouvera peut-être nos idées un peu fortes : qu’est-ce que cela fait ? N’avons-nous pas acquis le droit de tout dire ? Développons aux hommes de grandes vérités, ils les attendent de nous, il est temps que l’erreur disparaisse, il faut que son bandeau tombe à côté de celui des rois.\par
Le meurtre est-il un crime aux yeux de la nature ? Telle est la première question proposée.\par
Nous allons sans doute humilier ici l’orgueil de l’homme, en le rabaissant au rang de toutes les autres productions de la nature, mais le philosophe ne caresse point les petites vanités humaines ; toujours ardent à poursuivre la vérité, il la démêle sous les sots préjugés de l’amour-propre, l’atteint, la développe et la montre hardiment à la terre étonnée.\par
Qu’est-ce que l’homme, et quelle différence y a-t-il entre lui et les autres plantes, entre lui et tous les autres animaux de la nature ? Aucune assurément. Fortuitement placé, comme elles, sur ce globe, il est né comme elles, il se propage, croît et décroît comme elles ; il arrive comme elles à la vieillesse et tombe comme elles dans le néant, après le terme que la nature assigne à chaque espèce d’animaux, en raison de la construction de ses organes. Si les rapprochements sont tellement exacts, qu’il devienne absolument impossible à l’œil examinateur du philosophe d’apercevoir aucune dissemblance, il y aura donc alors tout autant de mal à tuer un animal qu’un homme, ou tout aussi peu à l’un qu’à l’autre, et dans les préjugés de notre orgueil se trouvera seulement la distance, mais rien n’est malheureusement absurde comme les préjugés de l’orgueil ; pressons néanmoins la question. Vous ne pouvez disconvenir qu’il ne soit égal de détruire un homme ou une bête ; mais la destruction de tout animal qui a vie n’est-elle pas décidément un mal, comme le croyaient les pythagoriciens et comme le croient encore quelques habitants des bords du Gange ? Avant de répondre à ceci, rappelons d’abord aux lecteurs que nous n’examinons la question que relativement à la nature ; nous l’envisagerons ensuite par rapport aux hommes.\par
Or, je demande de quel prix peuvent être à la nature des individus qui ne lui coûtent ni la moindre peine ni le moindre soin ? L’ouvrier n’estime son ouvrage qu’en raison du travail qu’il lui coûte, du temps qu’il emploie à le créer Or, l’homme coûte-t-il à la nature ? et en supposant qu’il lui coûte, lui coûte-t-il plus qu’un singe ou qu’un éléphant ? Je vais plus loin ; quelles sont les matières génératrices de la nature ? de quoi se composent les êtres qui viennent à la vie ? les trois éléments qui les forment ne résultent-ils pas de la primitive destruction des autres corps ? si tous les individus étaient éternels, ne deviendrait-il pas impossible à la nature d’en créer de nouveaux ? Si l’éternité des êtres est impossible à la nature, leur destruction devient donc une de ses lois ; or, si les destructions lui sont tellement utiles qu’elle ne puisse absolument s’en passer, et si elle ne peut parvenir à ses créations sans puiser dans ces masses de destruction que lui prépare la mort, de ce moment l’idée d’anéantissement que nous attachons à la mort ne sera donc plus réelle, il n’y aura plus d’anéantissement constaté ; ce que nous appelons la fin de l’animal qui a vie, ne sera plus une fin réelle, mais une simple transmutation dont est la base le mouvement perpétuel, véritable essence de la matière, et que tous les philosophes modernes admettent comme une de ses premières lois ; la mort, d’après ces principes irréfutables, n’est donc plus qu’un changement de forme, qu’un passage imperceptible d’une existence à une autre, et voilà ce que Pythagore appelait la métempsycose.\par
Ces vérités une fois admises, je demande si l’on pourra jamais avancer que la destruction soit un crime. À dessein de conserver vos absurdes préjugés, oserez-vous me dire que la transmutation est une destruction ? Non, sans doute ; car il faudrait pour cela prouver un instant d’inaction dans la matière, un moment de repos. Or, vous ne découvrirez jamais ce moment ; de petits animaux se forment à l’instant que le grand animal a perdu le souffle, et la vie de ces petits animaux n’est qu’un des effets nécessaires et déterminés par le sommeil momentané du grand. Oserez-vous dire à présent que l’un plaît mieux à la nature que l’autre ? Il faudrait prouver pour cela une chose impossible : c’est que la forme longue ou carrée est plus utile, plus agréable à la nature que la forme oblongue ou triangulaire ; il faudrait prouver que, eu égard aux plans sublimes de la nature, un fainéant qui s’engraisse dans l’inaction et dans l’indolence, est plus utile que le cheval dont le service est si essentiel, ou que le bœuf dont le corps est si précieux, qu’il n’en est aucune partie qui ne serve ; il faudrait dire que le serpent venimeux est plus nécessaire que le chien fidèle. Or, comme tous ces systèmes sont insoutenables, il faut donc absolument consentir à admettre que, vu l’impossibilité où nous sommes d’anéantir les ouvrages de la nature, qu’attendu la certitude que la seule chose que nous faisons, en nous livrant à la destruction, n’est que d’opérer une variation dans les formes, mais qui ne peut éteindre la vie, il devient alors au-dessus des forces humaines de prouver qu’il puisse exister aucun crime dans la prétendue destruction d’une créature de quelque âge, de quelque sexe, de quelque espèce que vous la supposiez. Conduits plus avant encore par la série de nos conséquences, qui naissent toutes les unes des autres, il faudra convenir enfin que, loin de nuire à la nature, l’action que vous commettez en variant les formes de ses différents ouvrages, est avantageuse pour elle, puisque vous lui fournissez par cette action la matière première de ses reconstructions, dont le travail lui deviendrait impraticable, si vous n’anéantissiez pas. Eh laissez-la faire, vous dit-on, assurément il faut la laisser faire, mais ce sont ses impulsions que suit l’homme quand il se livre à l’homicide, c’est la nature qui le lui conseille, et l’homme qui détruit son semblable, est à la nature ce que lui est la peste ou la famine, également envoyées par sa main, laquelle se sert de tous les moyens possibles pour obtenir plus tôt cette matière première de destruction, absolument essentielle à ses ouvrages, daignons éclairer un instant notre âme du saint flambeau de la philosophie ; quelle autre voix que celle de la nature nous suggère les haines personnelles, les vengeances, les guerres, en un mot tous ces motifs de meurtres perpétuels ? or, si elle nous les conseille, elle en a donc besoin. Comment donc pouvons-nous, d’après cela, nous supposer coupables envers elle, dès que nous ne faisons que suivre ses vues ?\par
Mais en voilà plus qu’il ne faut pour convaincre tout lecteur éclairé qu’il est impossible que le meurtre puisse jamais outrager la nature.\par
Est-il un crime en politique ? Osons avouer au contraire qu’il n’est malheureusement qu’un des plus grands ressorts de la politique. N’est-ce pas à force de meurtres que Rome est devenue la maîtresse du monde ? n’est-ce pas à force de meurtres que la France est libre aujourd’hui ? Il est inutile d’avertir ici qu’on ne parle que des meurtres occasionnés par la guerre, et non des atrocités commises par les factieux et les désorganisateurs ; ceux-là, voués à l’exécration publique, n’ont besoin que d’être rappelés, pour exciter à jamais l’horreur et l’indignation générale. Quelle science humaine a plus besoin de se soutenir par le meurtre, que celle qui ne tend qu’à tromper ? qui n’a pour but que l’accroissement d’une nation aux dépens d’une autre ? Les guerres, uniques fruits de cette barbare politique, sont-elles autre chose que les moyens dont elle se nourrit, dont elle se fortifie, dont elle s’étaie ? et qu’est-ce que la guerre, sinon la science de détruire ? Étrange aveuglement de l’homme, qui enseigne publiquement l’art de tuer, qui récompense celui qui y réussit le mieux, et qui punit celui qui, pour une cause particulière, s’est défait de son ennemi ! N’est-il pas temps de revenir sur des erreurs si barbares ?\par
Enfin, le meurtre est-il un crime contre la société ? Qui put jamais l’imaginer raisonnablement ? Ah ! qu’importe à cette nombreuse société qu’il y ait parmi elle un membre de plus ou de moins ? Ses lois, ses mœurs, ses coutumes en seront-elles viciées ? Jamais la mort d’un individu influa-t-elle sur la masse générale ? Et après la perte de la plus grande bataille, que dis-je, après l’extinction de la moitié du monde, de sa totalité, si l’on veut, le petit nombre d’êtres qui pourrait survivre éprouverait-il la moindre altération matérielle ? Hélas ! non. La nature entière n’en éprouverait pas davantage, et le sot orgueil de l’homme qui croit que tout est fait pour lui, serait bien étonné après la destruction totale de l’espèce humaine, s’il voyait que rien ne varie dans la nature et que le cours des astres n’en est seulement pas retardé. Poursuivons.\par
Comment le meurtre doit-il être vu dans un État guerrier et républicain ?\par
Il serait assurément du plus grand danger, ou de jeter de la défaveur sur cette action, ou de la punir, la fierté du républicain demande un peu de férocité ; s’il s’amollit, son énergie se perd, il sera bientôt subjugué. Une très singulière réflexion se présente ici, mais comme elle est vraie malgré sa hardiesse, je la dirai. Une nation qui commence à se gouverner en république, ne se soutiendra que par des vertus, parce que, pour arriver au plus, il faut toujours débuter par le moins ; mais une nation déjà vieille et corrompue, qui courageusement secouera le joug de son gouvernement monarchique pour en adopter un républicain, ne se maintiendra que par beaucoup de crimes ; car elle est déjà dans le crime ; et si elle voulait passer du crime à la vertu, c’est-à-dire d’un état violent dans un état doux, elle tomberait dans une inertie dont sa ruine certaine serait bientôt le résultat. Que deviendrait l’arbre que vous transplanteriez d’un terrain plein de vigueur, dans une plaine sablonneuse et sèche ? Toutes les idées intellectuelles sont tellement subordonnées à la physique de la nature, que les comparaisons fournies par l’agriculture ne nous tromperont jamais en morale.\par
Les plus indépendants des hommes, les plus rapprochés de la nature, les sauvages, se livrent avec impunité journellement au meurtre. À Sparte, à Lacédémone, on allait à la chasse des ilotes, comme nous allons en France à celle des perdrix ; les peuples les plus libres sont ceux qui l’accueillent davantage. À Mindanao, celui qui veut commettre un meurtre est élevé au rang des braves, on le décore aussitôt d’un turban ; chez les Caraguos, il faut avoir tué sept hommes pour obtenir les honneurs de cette coiffure ; les habitants de Bornéo croient que tous ceux qu’ils mettent à mort les serviront quand ils ne seront plus ; les dévots espagnols même faisaient vœu à Saint-Jacques de Galice de tuer douze Américains par jour ; dans le royaume de Tangut, on choisit un jeune homme fort et vigoureux, auquel il est permis, dans certains jours de l’année, de tuer tout ce qu’il rencontre. Était-il un peuple plus ami du meurtre que les Juifs ? On le voit sous toutes les formes, à toutes les pages de leur histoire. L’empereur et les mandarins de la Chine prennent de temps en temps des mesures pour faire révolter le peuple, afin d’obtenir de ses manœuvres le droit d’en faire un horrible carnage ; que ce peuple mou et efféminé s’affranchisse du joug de ses tyrans, il les assommera à son tour avec beaucoup plus de raison, et le meurtre, toujours adopté, toujours nécessaire, n’aura fait que changer de victimes ; il était le bonheur des uns, il deviendra la félicité des autres ; une infinité de nations tolèrent les assassinats publics, ils sont entièrement permis à Gênes, à Venise, à Naples, et dans toute l’Albanie ; à Kachao, sur la rivière de San Domingo, les meurtriers, sous un costume connu et avoué, égorgent à vos ordres et sous vos yeux l’individu que vous leur indiquez ; les Indiens prennent de l’opium pour s’encourager au meurtre ; se précipitant ensuite au milieu des rues, ils massacrent tout ce qu’ils rencontrent ; des voyageurs anglais ont retrouvé cette manie à Batavia.\par
Quel peuple fut à la fois plus grand et plus cruel que les Romains, et quelle nation conserva plus longtemps sa splendeur et sa liberté ? Le spectacle des gladiateurs soutint son courage, elle devenait guerrière par l’habitude de se faire un jeu du meurtre, douze ou quinze cents victimes journalières remplissaient l’arène du cirque, et là les femmes, plus cruelles que les hommes, osaient exiger que les mourants tombassent avec grâce et se dessinassent encore sous les convulsions de la mort. Les Romains passèrent de là aux plaisirs de voir des nains s’égorger devant eux ; et quand le culte chrétien, en infectant la terre, vint persuader aux hommes qu’il y avait du mal à se tuer, des tyrans aussitôt enchaînèrent ce peuple, et les héros du monde en devinrent bientôt les jouets. Partout enfin on crut, avec raison, que le meurtrier, c’est-à-dire l’homme qui étouffait sa sensibilité au point de tuer son semblable, et de braver la vengeance publique ou particulière ; partout, dis-je, on crut qu’un tel homme ne pouvait être que très courageux, et par conséquent très précieux dans un gouvernement guerrier et républicain. Parcourrons-nous des nations qui, plus féroces encore, ne se satisfirent qu’en immolant des enfants, et bien souvent les leurs ? Nous verrons ces actions universellement adoptées, faire même quelquefois partie des lois ; plusieurs peuplades sauvages tuent leurs enfants aussitôt qu’ils naissent ; les mères sur les bords du fleuve Orénoque, dans la persuasion où elles étaient que leurs filles ne naissaient que pour être malheureuses, puisque leur destination était de devenir les épouses des sauvages de cette contrée, qui ne pouvaient souffrir les femmes, les immolaient aussitôt qu’elles leur avaient donné le jour. Dans la Trapobane et dans le royaume de Sopit, tous les enfants difformes étaient immolés par les parents mêmes ; les femmes de Madagascar exposent aux bêtes sauvages ceux de leurs enfants nés certains jours de la semaine ; dans les républiques de la Grèce, on examinait soigneusement tous les enfants qui arrivaient au monde, et si l’on ne les trouvait pas conformés de manière à pouvoir défendre un jour la République, ils étaient aussitôt immolés : là l’on ne jugeait pas qu’il fût essentiel d’ériger des maisons richement dotées, pour conserver cette vile écume de la nature humaine \footnote{ Il faut espérer que la nation réformera cette dépense, la plus inutile de toutes ; tout individu qui naît sans les qualités nécessaires pour devenir un jour utile à la république, n’a nul droit à conserver la vie, et ce qu’on peut faire de mieux, est de la lui ôter au moment où il la reçoit.}. Jusqu’à la translation du siège de l’Empire, tous les Romains qui ne voulaient pas nourrir leurs enfants, les jetaient à la voirie ; les anciens législateurs n’avaient aucun scrupule de dévouer les enfants à la mort, et jamais aucun de leur code ne réprima les droits qu’un père se crut toujours sur sa famille. Aristote conseillait l’avortement ; et ces antiques républicains remplis d’enthousiasme, d’ardeur pour la patrie, méconnaissaient cette commisération individuelle qu’on retrouve parmi les nations modernes ; on aimait moins ses enfants, mais on aimait mieux son pays. Dans toutes les villes de la Chine, on trouve chaque matin une incroyable quantité d’enfants abandonnés dans les rues, un tombereau les enlève à la pointe du jour, et on les jette dans une fosse ; souvent les accoucheuses elles-mêmes en débarrassent les mères, en étouffant aussitôt leurs fruits dans des cuves d’eau bouillante ou les jetant dans la rivière ; à Pékin, on les met dans de petites corbeilles de joncs que l’on abandonne sur les canaux, on écume chaque jour ces canaux, et le célèbre voyageur {\itshape Du Halde} évalue à plus de trente mille le nombre journalier qui s’enlève à chaque recherche ; on ne peut nier qu’il ne soit extraordinairement nécessaire, extrêmement politique de mettre une digue à la population dans un gouvernement républicain ; par des vues absolument contraires, il faut l’encourager dans une monarchie ; là les tyrans n’étant riches qu’en raison du nombre de leurs esclaves, assurément il leur faut des hommes ; mais l’abondance de cette population, n’en doutons, pas, est un vice réel dans un gouvernement républicain ; il ne faut pourtant pas l’égorger pour l’amoindrir, comme le disaient nos modernes décemvirs, il ne s’agit que de ne pas lui laisser les moyens de s’étendre au-delà des bornes que sa félicité lui prescrit. Gardez-vous de multiplier trop un peuple dont chaque être est souverain, et soyez bien sûrs que les révolutions ne sont jamais les effets que d’une population trop nombreuse. Si, pour la splendeur de l’État, vous accordez à vos guerriers le droit de détruire des hommes, pour la conservation de ce même État, accordez de même à chaque individu de se livrer tant qu’il le voudra, puisqu’il le peut sans outrager la nature, au droit de se défaire des enfants qu’il ne peut nourrir, ou desquels le gouvernement ne peut tirer aucun secours ; accordez-lui de même de se défaire, à ses risques et périls, de tous les ennemis qui peuvent lui nuire, parce que le résultat de toutes ces actions, absolument nulles en elles-mêmes, sera de tenir votre population dans un état modéré, et jamais assez nombreux pour bouleverser votre gouvernement ; laissez dire aux monarchistes qu’un État n’est grand qu’en raison de son extrême population, cet État sera toujours pauvre si sa population excède ses moyens de vivre, et il sera toujours florissant, si, contenu dans de justes bornes, il peut trafiquer de son superflu ; n’élaguez-vous pas l’arbre quand il a trop de branches ? et pour conserver le tronc, ne taillez-vous pas les rameaux ? Tout système qui s’écarte de ces principes, est une extravagance dont les abus nous conduiraient bientôt au renversement total de l’édifice que nous venons d’élever avec tant de peine ; mais ce n’est pas quand l’homme est fait, qu’il faut le détruire afin de diminuer la population, il est injuste d’abréger les jours d’un individu bien conformé, il ne l’est pas, je le dis, d’empêcher d’arriver à la vie un être qui certainement sera inutile au monde. L’espèce humaine doit être épurée dès le berceau ; c’est ce que vous prévoyez ne pouvoir jamais être utile à la société qu’il faut retrancher de son sein ; voilà les seuls moyens raisonnables d’amoindrir une population dont la trop grande étendue est, ainsi que nous venons de le prouver, le plus dangereux des abus.\par
Il est temps de se résumer.\par
Le meurtre doit-il être réprimé par le meurtre ? Non, sans doute ; n’imposons jamais au meurtrier d’autre peine que celle qu’il peut encourir par la vengeance des amis ou de la famille de celui qu’il a tué ; {\itshape je vous accorde votre grâce}, disait Louis XV à Charolais, qui venait de tuer un homme pour se divertir, {\itshape mais je la donne aussi à celui qui vous tuera}. Toutes les bases de la loi contre les meurtriers se trouvent dans ce mot sublime\footnote{ La loi salique ne punissait le meurtre que d’une simple amende, et comme le coupable trouvait facilement les moyens de s’y soustraire, Childebert, roi d’Austrasie, décerna, par un règlement fait à Cologne, la peine de mort non contre le meurtrier, mais contre celui qui se soustrairait à l’amende décernée contre le meurtrier ; la loi ripuaire n’ordonnait de même contre cette action qu’une amende proportionnée à l’individu qu’il avait tué, il en coûtait fort cher pour un prêtre ; on faisait à l’assassin une tunique de plomb de sa taille , et il devait équivaloir en or le poids de cette tunique, à défaut de quoi le coupable et sa famille demeuraient esclaves de l’Église.}.\par
En un mot, le meurtre est une horreur, mais une horreur souvent nécessaire, jamais criminelle, essentielle à tolérer dans un État républicain ; j’ai fait voir que l’univers entier en avait donné l’exemple ; mais faut-il le considérer comme une action faite pour être punie de mort ? Ceux qui répondront au dilemme suivant auront satisfait à la question.\par
Le meurtre est-il un crime ou ne l’est-il pas ? S’il n’en est pas un, pourquoi faire des lois qui le punissent ? Et s’il en est un, par quelle barbare et stupide inconséquence le punirez-vous par un crime semblable ?\par
Il nous reste à parler des devoirs de l’homme envers lui-même. Comme le philosophe n’adopte ces devoirs qu’autant qu’ils tendent à son plaisir ou à sa conservation, il est fort inutile de lui en recommander la pratique, plus inutile encore de lui imposer des peines s’il y manque. Le seul délit que l’homme puisse commettre en ce genre est le suicide ; je ne m’amuserai point ici à prouver l’imbécillité des gens qui érigent cette action en crime, je renvoie à la fameuse lettre de Rousseau ceux qui pourraient avoir encore quelques doutes sur cela ; presque tous les anciens gouvernements autorisaient le suicide, par la politique et par la religion ; les Athéniens exposaient à l’Aréopage les raisons qu’ils avaient de se tuer, ils se poignardaient ensuite ; toutes les républiques de la Grèce tolérèrent le suicide, il entrait dans le plan des anciens législateurs, on se tuait en public, et l’on faisait de sa mort un spectacle d’appareil ; la république de Rome encouragea le suicide, les dévouements si célèbres pour la patrie n’étaient que des suicides. Quand Rome fut prise par les Gaulois, les plus illustres sénateurs se dévouèrent à la mort ; en reprenant ce même esprit, nous adoptons les mêmes vertus. Un soldat s’est tué pendant la campagne de 92, du chagrin de ne pouvoir suivre ses camarades à l’affaire de Jemmapes. Incessamment placés à la hauteur de ces fiers républicains, nous surpasserons bientôt leurs vertus ; c’est le gouvernement qui fait l’homme, une si longue habitude du despotisme avait totalement énervé notre courage, il avait dépravé nos mœurs, nous renaissons ; on va bientôt voir de quelles actions sublimes est capable le génie, le caractère français, quand il est libre ; soutenons, au prix de nos fortunes et de nos vies, cette liberté qui nous coûte déjà tant de victimes, n’en regrettons aucune si nous parvenons au but, elles-mêmes se sont toutes dévouées volontairement, ne rendons pas leur sang inutile ; mais de l’union… de l’union, ou nous perdrons le fruit de toutes nos peines ; asseyons d’excellentes lois sur les victoires que nous venons de remporter ; nos premiers législateurs, encore esclaves du despote qu’enfin nous avons abattu, ne nous avaient donné que des lois dignes de ce tyran, qu’ils encensaient encore, refaisons leur ouvrage, songeons que c’est pour des républicains et pour des philosophes que nous allons enfin travailler ; que nos lois soient douces comme le peuple qu’elles doivent régir ; en offrant ici, comme je viens de le faire, le néant, l’indifférence d’une infinité d’actions que nos ancêtres, séduits par une fausse religion, regardaient comme criminelles, je réduis notre travail à bien peu de chose ; faisons peu de lois, mais qu’elles soient bonnes ; il ne s’agit pas de multiplier les freins, il n’est question que de donner à celui qu’on emploie une qualité indestructible. Que les lois que nous promulguons n’aient pour but que la tranquillité du citoyen, son bonheur et l’éclat de la république ; mais après avoir chassé l’ennemi de vos terres, Français, je ne voudrais pas que l’ardeur de propager vos principes vous entraînât plus loin ; ce n’est qu’avec le fer et le feu que vous pourrez les porter au bout de l’univers. Avant que d’accomplir ces résolutions, rappelez-vous le malheureux succès des Croisades ; quand l’ennemi sera de l’autre côté du Rhin, croyez-moi, gardez vos frontières et restez chez vous ; ranimez votre commerce, redonnez de l’énergie et des débouchés à vos manufactures, faites refleurir vos arts, encouragez l’agriculture, si nécessaire dans un gouvernement tel que le vôtre, et dont l’esprit doit être de pouvoir fournir à tout le monde sans avoir besoin de personne, laissez les trônes de l’Europe s’écrouler d’eux-mêmes ; votre exemple, votre prospérité les culbuteront bientôt, sans que vous ayez besoin de vous en mêler. Invincibles dans votre intérieur, et modèles de tous les peuples par votre police et vos bonnes lois, il ne sera pas un gouvernement dans le monde qui ne travaille à vous imiter, pas un seul qui ne s’honore de votre alliance ; mais si, pour le vain honneur de porter vos principes au loin, vous abandonnez le soin de votre propre félicité, le despotisme qui n’est qu’endormi renaîtra, des dissensions intestines vous déchireront, vous aurez épuisé vos finances et vos soldats, et tout cela pour revenir baiser les fers que vous imposeront les tyrans qui vous auront subjugués pendant votre absence ; tout ce que vous désirez peut se faire, sans qu’il soit besoin de quitter vos foyers ; que les autres peuples vous voient heureux, et ils courront au bonheur par la même route que vous leur aurez tracée\footnote{ Qu’on se souvienne que la guerre extérieure ne fut jamais proposée que par l’infâme Dumouriez.}.\par
EUGÉNIE, {\itshape à Dolmancé} : Voilà ce qui s’appelle un écrit très sage, et tellement dans vos principes, au moins sur beaucoup d’objets, que je serais tentée de vous en croire l’auteur.\par
DOLMANCÉ : Il est bien certain que je pense une partie de ces réflexions, et mes discours qui vous l’ont prouvé, donnent même à la lecture que nous venons de faire, l’apparence d’une répétition…\par
EUGÉNIE, {\itshape coupant} : Je ne m’en suis pas aperçue, on ne saurait trop dire les bonnes choses. Je trouve cependant quelques-uns de ces principes un peu dangereux.\par
DOLMANCÉ : Il n’y a de dangereux dans le monde que la pitié et la bienfaisance, la bonté n’est jamais qu’une faiblesse dont l’ingratitude et l’impertinence des faibles forcent toujours les honnêtes gens à se repentir. Qu’un bon observateur s’avise de calculer tous les dangers de la pitié, et qu’il les mette en parallèle avec ceux d’une fermeté soutenue, il verra si les premiers ne l’emportent pas.\par
Mais nous allons trop loin, Eugénie, résumons pour votre éducation l’unique conseil qu’on puisse tirer de tout ce qui vient d’être dit, n’écoutez jamais votre cœur, mon enfant ; c’est le guide le plus faux que nous ayons reçu de la nature, fermez-le avec grand soin aux accents fallacieux de l’infortune ; il vaut beaucoup mieux que vous refusiez à celui qui vraiment serait fait pour vous intéresser, que de risquer de donner au scélérat, à l’intrigant et au cabaleur : l’un est d’une très légère conséquence, l’autre du plus grand inconvénient.\par
LE CHEVALIER : Qu’il me soit permis, je vous en conjure, de reprendre en sous-œuvre et d’anéantir, si je peux, les principes de Dolmancé. Ah ! qu’ils seraient différents, homme cruel, si, privé de cette fortune immense où tu trouves sans cesse les moyens de satisfaire tes passions, tu pouvais languir quelques années dans cette accablante infortune dont ton esprit féroce ose composer des torts aux misérables ; jette un coup d’œil de pitié sur eux, et n’éteins pas ton âme au point de l’endurcir sans retour aux cris déchirants du besoin ! Quand ton corps, uniquement las de voluptés, repose languissamment sur des lits de duvet, vois le leur affaissé des travaux qui te font vivre, recueillir à peine un peu de paille pour se préserver de la fraîcheur de la terre, dont ils n’ont, comme les bêtes, que la froide superficie pour s’étendre ; jette un regard sur eux, lorsque entouré de mets succulents dont vingt élèves de Comus réveillent chaque jour ta sensualité, ces malheureux disputent aux loups, dans les bois, la racine amère d’un sol desséché ; quand les jeux, les grâces et les ris conduisent à ta couche impure les plus touchants objets du temple de Cythère, vois ce misérable étendu près de sa triste épouse, satisfait des plaisirs qu’il cueille au sein des larmes, ne pas même en soupçonner d’autres ; regarde-le, quand tu ne te refuses rien, quand tu nages au milieu du superflu ; regarde-le, te dis-je, manquer même opiniâtrement des premiers besoins de la vie ; jette les yeux sur sa famille désolée, vois son épouse tremblante se partager avec tendresse entre les soins qu’elle doit à son mari languissant auprès d’elle, et ceux que la nature commande pour les rejetons de son amour ; privée de la possibilité de remplir aucun de ces devoirs si sacrés pour son âme sensible, entends-la sans frémir, si tu peux, réclamer près de toi ce superflu que ta cruauté lui refuse ! Barbare, ne sont-ce donc pas des hommes comme toi ; et s’ils te ressemblent, pourquoi dois-tu jouir quand ils languissent ? Eugénie, Eugénie, n’éteignez jamais dans votre âme la voix sacrée de la nature, c’est à la bienfaisance qu’elle vous conduira malgré vous, quand vous séparerez son organe du feu des passions qui l’absorbe ; laissons là les principes religieux, j’y consens, mais n’abandonnons pas les vertus que la sensibilité nous inspire ; ce ne sera jamais qu’en les pratiquant, que nous goûterons les jouissances de l’âme les plus douces et les plus délicieuses ; tous les égarements de votre esprit seront rachetés par une bonne œuvre, elle éteindra dans vous les remords que votre inconduite y fera naître, et formant dans le fond de votre conscience un asile sacré, où vous vous replierez quelquefois sur vous-même, vous y trouverez la consolation des écarts où vos erreurs vous auront entraînée. Ma sœur, je suis jeune, je suis libertin, impie, je suis capable de toutes les débauches de l’esprit, mais mon cœur me reste, il est pur, et c’est avec lui, mes amis, que je me console de tous les travers de mon âge.\par
DOLMANCÉ : Oui, Chevalier, vous êtes jeune, vous le prouvez par vos discours, l’expérience vous manque, je vous attends ; quand elle vous aura mûri, alors, mon cher, vous ne parlerez plus si bien des hommes, parce que vous les aurez connus ; ce fut leur ingratitude qui sécha mon cœur, leur perfidie qui détruisit dans moi ces vertus funestes pour lesquelles j’étais peut-être né comme vous ; or, si les vices des uns rendent dans les autres ces vertus dangereuses, n’est-ce donc pas un service à rendre à la jeunesse, que de les étouffer de bonne heure en elle ? que me parles-tu de remords, mon ami, peuvent-ils exister dans l’âme de celui qui ne connaît de crime à rien ? que vos principes les étouffent ; si vous en craignez l’aiguillon, vous sera-t-il possible de vous repentir d’une action de l’indifférence de laquelle vous serez profondément pénétré ? Dès que vous ne croirez plus de mal à rien, de quel mal pourrez-vous vous repentir ?\par
LE CHEVALIER : Ce n’est pas de l’esprit que viennent les remords, ils ne sont les fruits que du cœur, et jamais les sophismes de la tête n’éteignirent les mouvements de l’âme.\par
DOLMANCÉ : Mais le cœur trompe, parce qu’il n’est jamais que l’expression des faux calculs de l’esprit ; mûrissez celui-ci, l’autre cédera bientôt, toujours de fausses définitions nous égarent lorsque nous voulons raisonner ; je ne sais ce que c’est que le cœur, moi, je n’appelle ainsi que les faiblesses de l’esprit, un seul et unique flambeau luit en moi ; quand je suis sain et ferme, il ne me fourvoie jamais ; suis-je vieux, hypocondre ou pusillanime, il me trompe, alors je me dis sensible, tandis qu’au fond je ne suis que faible et timide ; encore une fois, Eugénie, que cette perfide sensibilité ne vous abuse pas ; elle n’est, soyez-en bien sûre, que la faiblesse de l’âme, on ne pleure que parce que l’on craint, et voilà pourquoi les rois sont des tyrans ; rejetez, détestez donc les perfides conseils du Chevalier ; en vous disant d’ouvrir votre cœur à tous les maux imaginaires de l’infortune, il cherche à vous composer une somme de peines qui n’étant pas les vôtres, vous déchireraient bientôt en pure perte. Ah ! croyez, Eugénie, croyez que les plaisirs qui naissent de l’apathie, valent bien ceux que la sensibilité vous donne, celle-ci ne sait qu’atteindre dans un sens le cœur que l’autre chatouille, et bouleverse de toutes parts ; les jouissances permises, en un mot, peuvent-elles donc se comparer aux jouissances qui réunissent à des attraits bien plus piquants, ceux inappréciables de la rupture des freins sociaux et du renversement de toutes les lois ?\par
EUGÉNIE : Tu triomphes, Dolmancé, tu l’emportes, les discours du Chevalier n’ont fait qu’effleurer mon âme, les tiens la séduisent et l’entraînent. Ah ! croyez-moi, Chevalier, adressez-vous plutôt aux passions qu’aux vertus, quand vous voudrez persuader une femme.\par
MME DE SAINT-ANGE, {\itshape au Chevalier} : Oui, mon ami, fous-nous bien, mais ne nous sermonne pas : tu ne nous convertirais point, et tu pourrais troubler les leçons dont nous voulons abreuver l’âme et l’esprit de cette charmante fille.\par
EUGÉNIE : Troubler, oh ! non, non, votre ouvrage est fini ; ce que les sots appellent la corruption, est maintenant assez établi dans moi, pour ne laisser même aucun espoir de retour, et vos principes sont trop bien étayés dans mon cœur, pour que les sophismes du Chevalier parviennent jamais à les détruire.\par
DOLMANCÉ : Elle a raison, ne parlons plus de cela, Chevalier, vous auriez des torts, et nous ne voulons vous trouver que des procédés.\par
LE CHEVALIER : Soit, nous sommes ici pour un but très différent, je le sais, que celui où je voulais atteindre ; marchons droit à ce but, j’y consens, je garderai ma morale pour ceux qui, moins ivres que vous, seront plus en état de l’entendre.\par
MME DE SAINT-ANGE : Oui, mon frère, oui, oui, ne nous donne ici que ton foutre ; nous te faisons grâce de ta morale, elle est trop douce pour des {\itshape roués} de notre espèce.\par
EUGÉNIE : Je crains bien, Dolmancé, que cette cruauté que vous préconisez avec chaleur, n’influence un peu vos plaisirs ; j’ai déjà cru le remarquer, vous êtes dur en jouissant ; je me sentirais bien aussi quelques dispositions à ce vice. Pour débrouiller mes idées sur tout cela, dites-moi, je vous prie, de quel œil vous voyez l’objet qui sert vos plaisirs.\par
DOLMANCÉ : Comme absolument nul, ma chère ; qu’il partage ou non mes jouissances, qu’il éprouve du contentement, de l’apathie ou même de la douleur, pourvu que je sois heureux, le reste m’est absolument égal.\par
EUGÉNIE : Il vaut même mieux que cet objet éprouve de la douleur, n’est-ce pas ?\par
DOLMANCÉ : Assurément cela vaut beaucoup mieux ; je vous l’ai déjà dit, la répercussion plus active sur nous, détermine bien plus énergiquement, et bien plus promptement alors les esprits animaux, à la direction qui leur est nécessaire pour la volupté. Ouvrez les sérails de l’Afrique, ceux de l’Asie, ceux de votre Europe méridionale, et voyez si les chefs de ces harems célèbres s’embarrassent beaucoup, quand ils bandent, de donner du plaisir aux individus qui leur servent ; ils commandent, on leur obéit ; ils jouissent, on n’ose leur répondre ; sont-ils satisfaits, on s’éloigne. Il en est parmi eux qui puniraient, comme un manque de respect, l’audace de partager leur jouissance ; le roi d’Achem fait impitoyablement trancher la tête à la femme qui a osé s’oublier en sa présence au point de jouir, et très souvent il la lui coupe lui-même ; ce despote, un des plus singuliers de l’Asie, n’est absolument gardé que par des femmes ; ce n’est jamais que par signes qu’il leur donne ses ordres ; la mort la plus cruelle est la punition de celles qui ne l’entendent pas, et les supplices s’exécutent toujours ou par sa main, ou sous ses yeux. Tout cela, ma chère Eugénie, est absolument fondé sur des principes que je vous ai déjà développés. Que désire-t-on quand on jouit ? Que tout ce qui nous entoure ne s’occupe que de nous, ne pense qu’à nous, ne soigne que nous ; si les objets qui nous servent jouissent, les voilà dès lors bien plus sûrement occupés d’eux que de nous, et notre jouissance conséquemment dérangée ; il n’est point d’homme qui ne veuille être despote quand il bande, il semble qu’il a moins de plaisir si les autres paraissent en prendre autant que lui ; par un mouvement d’orgueil bien naturel en ce moment, il voudrait être le seul au monde qui fût susceptible d’éprouver ce qu’ils sentent ; l’idée de voir un autre jouir comme lui le ramène à une sorte d’égalité qui nuit aux attraits indicibles que fait éprouver le despotisme alors\footnote{ La pauvreté de la langue française nous contraint à employer des mots que notre heureux gouvernement réprouve aujourd’hui avec tant de raison ; nous espérons que nos lecteurs éclairés nous entendront, et ne confondront point l’absurde despotisme politique, avec le très luxurieux despotisme des passions de libertinage.} ; il est faux d’ailleurs qu’il y ait du plaisir à en donner aux autres, c’est les servir cela, et l’homme qui bande est loin du désir d’être utile aux autres ; en faisant du mal, au contraire, il éprouve tous les charmes que goûte un individu nerveux à faire usage de ses forces, il domine alors, il est {\itshape tyran}, et quelle différence pour l’amour-propre ? Ne croyons point qu’il se taise en ce cas ; l’acte de la jouissance est une passion qui, j’en conviens, subordonne à elle toutes les autres, mais qui les réunit en même temps. Cette envie de dominer dans ce moment est si forte dans la nature, qu’on la reconnaît même dans les animaux ; voyez si ceux qui sont en esclavage procréent comme ceux qui sont libres ; le dromadaire va plus loin, il n’engendre plus s’il ne se croit pas seul ; essayez de le surprendre, et par conséquent de lui montrer un maître, il fuira et se séparera sur-le-champ de sa compagne. Si l’intention de la nature n’était pas que l’homme eût cette supériorité, elle n’aurait pas créé plus faibles que lui les êtres qu’elle lui destine dans ce moment-là ; cette débilité où la nature condamna les femmes, prouve incontestablement que son intention est que l’homme qui jouit plus que jamais alors de sa puissance, l’exerce par toutes les violences que bon lui semblera, par des supplices même, s’il le veut ; la crise de la volupté serait-elle une espèce de rage, si l’intention de cette mère du genre humain n’était pas que le traitement du coït fût le même que celui de la colère ? Quel est l’homme bien constitué, en un mot, l’homme doué d’organes vigoureux, qui ne désirera pas, soit d’une façon, soit d’une autre, de molester sa jouissance alors ? Je sais bien qu’une infinité de sots qui ne se rendent jamais compte de leurs sensations, comprendront mal les systèmes que j’établis ; mais que m’importent ces imbéciles, ce n’est pas à eux que je parle. Plats adorateurs de femmes, je les laisse aux pieds de leur insolente dulcinée attendre le soupir qui doit les rendre heureux, et bassement esclaves du sexe qu’ils devraient dominer, je les abandonne aux vils charmes de porter des fers, dont la nature leur donne le droit d’accabler les autres ; que ces animaux végètent dans la bassesse qui les avilit, ce serait en vain que nous les prêcherions, mais qu’ils ne dénigrent pas ce qu’ils ne peuvent entendre, et qu’ils se persuadent que ceux qui ne veulent établir leurs principes en ces sortes de matières que sur les élans d’une âme vigoureuse et d’une imagination sans frein, comme nous le faisons vous et moi, madame, seront toujours les seuls qui mériteront d’être écoutés, les seuls qui seront faits pour leur prescrire des lois et pour leur donner des leçons… Foutre, je bande ; rappelez Augustin, je vous prie {\itshape (On sonne ; il rentre)} ; il est inouï comme le superbe cul de ce beau garçon m’occupe la tête depuis que je parle, toutes mes idées semblaient involontairement se rapporter à lui… montre à mes yeux ce chef-d’œuvre, Augustin… que je le baise et caresse un quart d’heure ; viens, bel amour, viens que je me rende digne, dans ton beau cul, des flammes dont Sodome m’embrase ; il a les plus belles fesses… les plus blanches ; je voudrais qu’Eugénie, à genoux, lui suçât le vit pendant ce temps-là ; par l’attitude, elle exposerait son derrière au Chevalier qui l’enculerait, et M\textsuperscript{me} de Saint-Ange, à cheval sur les reins d’Augustin, me présenterait ses fesses à baiser ; armée d’une poignée de verges, elle pourrait au mieux, ça me semble, en se courbant un peu, fouetter le Chevalier, que cette stimulante cérémonie engagerait à ne pas épargner notre écolière. {\itshape (La posture s’arrange.)} Oui, c’est cela tout au mieux, mes amis, en vérité, c’est un plaisir que de vous commander des tableaux ; il n’est pas un artiste au monde en état de les exécuter comme vous… ce coquin a le cul d’un étroit… C’est tout ce que je peux faire que de m’y loger… Voulez-vous bien me permettre, madame, de mordre et pincer vos belles chairs pendant que je fous ?\par
MME DE SAINT-ANGE : Tant que tu voudras, mon ami, mais ma vengeance est prête, je t’en avertis ; je jure qu’à chaque vexation je te lâche un pet dans la bouche.\par
DOLMANCÉ : Ah ! sacredieu, quelle menace c’est me presser de t’offenser, ma chère {\itshape (il la mord)} ; Voyons si tu tiendras parole {\itshape (il reçoit un pet).} Ah ! foutre délicieux… délicieux {\itshape (Il la claque et reçoit sur-le-champ un autre pet).} Oh c’est divin, mon ange ! garde-m’en quelques-uns pour l’instant de la crise… et sois sûre que je te traiterai alors avec toute la cruauté… toute la barbarie… Foutre… je n’en puis plus… je décharge… {\itshape (il la mord, la claque, et elle ne cesse de péter.)} Vois-tu comme je te traite, coquine… comme je te maîtrise… encore celle-ci… et celle-là… et que la dernière insulte soit à l’idole même où j’ai sacrifié. {\itshape (Il lui mord le trou du cul, l’attitude se rompt.)} Et vous autres, qu’avez-vous fait, mes amis ?\par
EUGÉNIE, {\itshape rendant le foutre qu’elle a dans le cul et dans la bouche} : Hélas ! mon maître… vous voyez comme vos élèves m’ont accommodée ; j’ai le derrière et la bouche pleins de foutre, je ne dégorge que du foutre de tous les côtés.\par
DOLMANCÉ, {\itshape vivement} : Attendez, je veux que vous me rendiez dans la bouche celui que le Chevalier vous a mis dans le cul.\par
EUGÉNIE, {\itshape se plaçant} : Quelle extravagance !\par
DOLMANCÉ : Ah ! rien n’est bon comme le foutre qui sort du fond d’un beau derrière… C’est un mets digne des dieux {\itshape (il l’avale)} ; voyez le cas que j’en fais {\itshape (se reportant au cul d’Augustin qu’il baise).} Je vais vous demander, mesdames, la permission de passer un instant dans un cabinet voisin avec ce jeune homme.\par
MME DE SAINT-ANGE : Ne pouvez-vous donc pas faire ici tout ce qu’il vous plaît avec lui ?\par
DOLMANCÉ, {\itshape bas et mystérieusement} : Non, il est de certaines choses qui demandent absolument des voiles.\par
EUGÉNIE : Ah ! parbleu, mettez-nous au fait au moins.\par
MME DE SAINT-ANGE : Je ne le laisse pas sortir sans cela.\par
DOLMANCÉ : Vous voulez le savoir ?\par
EUGÉNIE : Absolument.\par
DOLMANCÉ, {\itshape entraînant Augustin} : Eh bien ! mesdames, je vais… mais, en vérité, cela ne peut pas se dire.\par
MME DE SAINT-ANGE : Est-il donc une infamie dans le monde que nous ne soyons dignes d’entendre et d’exécuter ?\par
LE CHEVALIER : Tenez, ma sœur, je vais vous le dire.\par
{\itshape Il parle bas aux deux femmes.}\par
EUGÉNIE, {\itshape avec l’air de la répugnance} : Vous avez raison, cela est horrible.\par
MME DE SAINT-ANGE : Oh ! je m’en doutais.\par
DOLMANCÉ : Vous voyez bien que je devais vous taire cette fantaisie, et vous concevez à présent qu’il faut être seul et dans l’ombre pour se livrer à de pareilles turpitudes.\par
EUGÉNIE : Voulez-vous que j’aille avec vous, je vous branlerai pendant que vous vous amuserez d’Augustin ?\par
DOLMANCÉ : Non, non, ceci est une affaire d’honneur, et qui doit se passer entre hommes, une femme nous dérangerait… À vous dans l’instant, mesdames.\par
{\itshape Il sort en entraînant Augustin.}
\section[{Sixième dialogue}]{Sixième dialogue}\phantomsection
\label{d6}\renewcommand{\leftmark}{Sixième dialogue}

\textit{MME DE SAINT-ANGE, EUGÉNIE, LE CHEVALIER}\par
\noindent MME DE SAINT-ANGE : En vérité, mon frère, ton ami est bien libertin.\par
LE CHEVALIER : Je ne t’ai donc pas trompée en te le donnant pour tel.\par
EUGÉNIE : Je suis persuadée qu’il n’a pas son égal au monde… Oh ! ma bonne, il est charmant ; voyons-le souvent, je t’en prie.\par
MME DE SAINT-ANGE : On frappe… qui cela peut-il être… j’avais défendu ma porte… il faut que cela soit bien pressé. Vois ce que c’est, Chevalier, je t’en prie.\par
LE CHEVALIER : Une lettre qu’apportait Lafleur ; il s’est retiré bien vite, en disant qu’il se souvenait des ordres que vous lui aviez donnés, mais que la chose lui avait paru aussi importante que pressée.\par
MME DE SAINT-ANGE : Ah ! ah ! qu’est-ce que c’est que ceci… c’est votre père, Eugénie.\par
EUGÉNIE : Mon père !… Ah ! nous sommes perdues.\par
MME DE SAINT-ANGE : Lisons avant que de nous décourager. (Elle lit :) {\itshape Croiriez-vous, ma belle dame, que mon insoutenable épouse, alarmée du voyage de ma fille chez vous, part à l’instant pour aller la rechercher ; elle s’imagine tout plein de choses… qui, à supposer même qu’elles fussent, ne seraient en vérité que fort simples. Je vous prie de la punir rigoureusement de cette impertinence ; je la corrigeai hier pour une semblable, la leçon n’a pas suffi ; mystifiez-la donc d’importance, je vous le demande, en grâce, et croyez qu’à quelque point que vous portiez les choses, je ne m’en plaindrai pas… Il y a si longtemps que cette catin me pèse… qu’en vérité… vous m’entendez, ce que vous ferez sera bienfait, c’est tout ce que je puis vous dire ; elle va suivre ma lettre de très près, tenez-vous donc sur vos gardes. Adieu, je voudrais bien être des vôtres. Ne me renvoyez Eugénie qu’instruite, je vous en conjure ; je veux bien vous laisser faire les premières récoltes, mais soyez assurée cependant que vous aurez un peu travaillé pour moi.} Eh bien ! Eugénie, tu vois qu’il n’y a point trop de quoi s’effrayer ; il faut convenir que voilà une petite femme bien insolente.\par
EUGÉNIE : La putain !… Ah ! ma chère, puisque mon papa nous donne carte blanche, il faut, je t’en conjure, recevoir cette coquine-là comme elle le mérite.\par
MME DE SAINT-ANGE : Baise-moi, mon cœur ; que je suis aise de te voir dans de telles dispositions… Va, tranquillise-toi je te réponds que nous ne l’épargnerons pas. Tu voulais une victime, Eugénie, en voilà une que te donnent à la fois la nature et le sort.\par
EUGÉNIE : Nous en jouirons, ma chère, nous en jouirons, je te le jure.\par
MME DE SAINT-ANGE : Ah ! qu’il me tarde de savoir comment Dolmancé va prendre cette nouvelle.\par
DOLMANCÉ, {\itshape rentrant avec Augustin} : Le mieux du monde, mesdames, je n’étais pas assez loin de vous pour ne pas vous entendre, je sais tout… M\textsuperscript{me} de Mistival arrive on ne saurait plus à propos… Vous êtes bien décidée, j’espère, à remplir les vues de son mari.\par
EUGÉNIE, {\itshape à Dolmancé} : Les remplir !… les outrepasser, mon cher… Ah ! que la terre s’effondre sous moi, si vous me voyez faiblir, quelles que soient les horreurs où vous condamniez cette gueuse… Cher ami, charge-toi de diriger tout cela, je t’en prie.\par
DOLMANCÉ : Laissez faire votre amie et moi, obéissez seulement vous autres, c’est tout ce que nous vous demandons… Ah ! l’insolente créature, je n’ai jamais rien vu de semblable.\par
MME DE SAINT-ANGE : C’est d’un maladroit !… Eh bien ! nous remettons-nous un peu décemment pour la recevoir ?\par
DOLMANCÉ : Au contraire, il faut que rien, dès qu’elle entrera, ne puisse l’empêcher d’être sûre de la manière dont nous faisons passer le temps à sa fille ; soyons tous dans le plus grand désordre.\par
MME DE SAINT-ANGE : J’entends du bruit, c’est elle ; allons, courage, Eugénie, rappelle-toi bien nos principes. Ah ! sacredieu, la délicieuse scène !
\section[{Septième et dernier dialogue}]{Septième et dernier dialogue}\phantomsection
\label{d7}\renewcommand{\leftmark}{Septième et dernier dialogue}

\textit{MME DE SAINT-ANGE, EUGÉNIE, LE CHEVALIER, AUGUSTIN, DOLMANCÉ, MME DE MISTIVAL}\par
\noindent MME DE MISTIVAL, {\itshape à M\textsuperscript{me} de Saint-Ange} : Je vous prie de m’excuser, madame, si j’arrive chez vous sans vous prévenir ; mais on dit que ma fille y est, et comme son âge ne permet pas encore qu’elle aille seule, je vous prie, madame, de vouloir bien me la rendre, et de ne pas désapprouver ma démarche.\par
MME DE SAINT-ANGE : Cette démarche est des plus impolies, madame ; on dirait, à vous entendre, que votre fille est en mauvaises mains.\par
MME DE MISTIVAL : Ma foi, s’il faut en juger par l’état où je la trouve, elle, vous et votre compagnie, madame, je crois que je n’ai pas grand tort de la juger fort mal ici.\par
DOLMANCÉ : Ce début est impertinent, madame, et sans connaître précisément les degrés de liaison qui existent entre M\textsuperscript{me} de Saint-Ange et vous, je ne vous cache pas qu’à sa place je vous aurais déjà fait jeter par les fenêtres.\par
MME DE MISTIVAL : Qu’appelez-vous jeter par les fenêtres ? Apprenez, monsieur, qu’on n’y jette pas une femme comme moi ; j’ignore qui vous êtes, mais aux propos que vous tenez, à l’état dans lequel vous voilà, il est aisé de juger vos mœurs. Eugénie, suivez-moi.\par
EUGÉNIE : Je vous demande pardon, madame, mais je ne puis avoir cet honneur.\par
MME DE MISTIVAL : Quoi ! ma fille me résiste !\par
DOLMANCÉ : Elle vous désobéit formellement même, comme vous le voyez, madame. Croyez-moi, ne souffrez point cela. Voulez-vous que j’envoie chercher des verges pour corriger cet enfant indocile ?\par
EUGÉNIE : J’aurais bien peur, s’il en venait, qu’elles ne servissent plutôt à madame qu’à moi.\par
MME DE MISTIVAL : L’impertinente créature !\par
DOLMANCÉ, {\itshape s’approchant de M\textsuperscript{me} de Mistival} : Doucement, mon cœur, point d’invectives ici ; nous protégeons tous Eugénie, et vous pourriez vous repentir de vos vivacités avec elle.\par
MME DE MISTIVAL : Quoi ! ma fille me désobéira, et je ne pourrai pas lui faire sentir les droits que j’ai sur elle ?\par
DOLMANCÉ : Et quels sont-ils, ces droits, je vous prie, madame ? Vous flattez-vous de leur légitimité ? Quand M. de Mistival, ou je ne sais qui, vous lança dans le vagin les gouttes de foutre qui firent éclore Eugénie, l’aviez-vous en vue pour lors ? Non, n’est-ce pas ? Eh bien ! quel gré voulez-vous qu’elle vous sache aujourd’hui pour avoir déchargé quand on foutait votre vilain con ? Apprenez, madame, qu’il n’est rien de plus illusoire que les sentiments du père ou de la mère pour les enfants, et de ceux-ci pour les auteurs de leurs jours ; rien ne fonde, rien n’établit de pareils sentiments en usage ici ; détestez-la, puisqu’il est des pays où les parents tuent leurs enfants, d’autres où ceux-ci égorgent ceux de qui ils tiennent la vie. Si les mouvements d’amour réciproque étaient dans la nature, la force du sang ne serait plus chimérique, et sans s’être vus, sans s’être connus mutuellement, les parents distingueraient, adoreraient leurs fils, et réversiblement ceux-ci au milieu de la plus grande assemblée, discerneraient leurs pères inconnus, voleraient dans leurs bras, et les adoreraient. Que voyons-nous au lieu de tout cela ? Des haines réciproques et invétérées, des enfants qui, même avant l’âge de raison, n’ont jamais pu souffrir la vue de leurs pères, des pères éloignant leurs enfants d’eux, parce que jamais ils ne purent en soutenir l’approche. Ces prétendus mouvements sont donc illusoires, absurdes, l’intérêt seul les imagina, l’usage les prescrivit, l’habitude les soutint, mais la nature jamais ne les imprima dans nos cœurs. Voyez si les animaux les connaissent : non, sans doute ; c’est pourtant toujours eux qu’il faut consulter, quand on veut connaître la nature. Ô pères ! soyez donc bien en repos sur les prétendues injustices que vos passions ou vos intérêts vous conduisent à faire à ces êtres nuls pour vous, auxquels quelques gouttes de votre sperme ont donné le jour ; vous ne leur devez rien, vous êtes au monde pour vous et non pour eux, vous seriez bien fous de vous gêner, ne vous occupez que de vous, ce n’est que pour vous que vous devez vivre ; et vous, enfants, bien plus dégagés, s’il se peut encore, de cette piété filiale dont la base est une vraie chimère, persuadez-vous de même que vous ne devez rien non plus à ces individus dont le sang vous a mis au jour. Pitié, reconnaissance, amour, aucun de ces sentiments ne leur est dû, ceux qui vous ont donné l’être n’ont pas un seul titre pour les exiger de vous ; ils ne travaillaient que pour eux, qu’ils s’arrangent ; mais la plus grande de toutes les duperies, serait de leur donner ou des soins ou des secours que vous ne leur devez sous aucuns rapports, rien ne vous en prescrit la loi, et si par hasard vous vous imaginiez en démêler l’organe, soit dans les inspirations de l’usage, soit dans celles des effets moraux du caractère, étouffez sans remords des sentiments absurdes… des sentiments locaux, fruits des mœurs climatérales que la nature réprouve, et que désavoua toujours la raison.\par
MME DE MISTIVAL : Eh quoi ! les soins que j’ai eus d’elle, l’éducation que je lui ai donnée…\par
DOLMANCÉ : Oh ! pour les soins, ils ne sont jamais les fruits que de l’usage ou de l’orgueil ; n’ayant rien fait de plus pour elle que ce que prescrivent les mœurs du pays que vous habitez, assurément Eugénie ne vous doit rien ; quant à l’éducation, il faut qu’elle ait été bien mauvaise, car nous sommes obligés de refondre ici tous les principes que vous lui avez inculqués ; il n’y en a pas un seul qui tienne à son bonheur, pas un qui ne soit absurde ou chimérique ; vous lui avez parlé de dieu, comme s’il y en avait un ; de vertu, comme si elle était nécessaire ; de religion, comme si tous les cultes religieux étaient autre chose que le résultat de l’imposture du plus fort, et de l’imbécillité du plus faible ; de jésus-christ, comme si ce coquin-là était autre chose qu’un fourbe et qu’un scélérat ; vous lui avez dit que {\itshape foutre} était un péché, tandis que {\itshape foutre} est la plus délicieuse action de la vie ; vous avez voulu lui donner des mœurs, comme si le bonheur d’une jeune fille n’était pas dans la débauche et l’immoralité, comme si la plus heureuse de toutes les femmes ne devait pas être incontestablement celle qui est la plus vautrée dans l’ordure et le libertinage, celle qui brave le mieux tous les préjugés et qui se moque le plus de la réputation. Ah ! détrompez-vous, détrompez-vous, madame, vous n’avez rien fait pour votre fille, vous n’avez rempli à son égard aucune obligation dictée par la nature, Eugénie ne vous doit donc que de la haine.\par
MME DE MISTIVAL : Juste ciel ! mon Eugénie est perdue, cela est clair… Eugénie, ma chère Eugénie, entends pour la dernière fois les supplications de celle qui t’a donné la vie ; ce ne sont plus des ordres, mon enfant, ce sont des prières ; il n’est malheureusement que trop vrai que tu es ici avec des monstres, arrache-toi de ce commerce dangereux, et suis-moi, je te le demande à genoux.\par
{\itshape Elle s’y jette.}\par
DOLMANCÉ : Ah ! bon, voilà une scène de larmes… Allons, Eugénie, attendrissez-vous.\par
EUGÉNIE, {\itshape à moitié nue, comme on doit s’en souvenir} : Tenez, ma petite maman, je vous apporte mes fesses… les voilà positivement au niveau de votre bouche ; baisez-les, mon cœur, sucez-les, c’est tout ce qu’Eugénie peut faire pour vous… Souviens-toi, Dolmancé, que je me montrerai toujours digne d’être ton élève.\par
MME DE MISTIVAL, {\itshape repoussant Eugénie avec horreur} : Ah ! monstre ! Va, je te renie à jamais pour ma fille.\par
EUGÉNIE : Joignez-y même votre malédiction, ma très chère mère, si vous le voulez, afin de rendre la chose plus touchante, et vous me verrez toujours du même flegme.\par
DOLMANCÉ : Oh ! doucement, doucement, madame ; il y a une insulte ici ; vous venez à nos yeux de repousser un peu trop durement Eugénie ; je vous ai dit qu’elle était sous notre sauvegarde, il faut une punition à ce crime. Ayez la bonté de vous déshabiller toute nue, pour recevoir celle que mérite votre brutalité.\par
MME DE MISTIVAL : Me déshabiller !…\par
DOLMANCÉ : Augustin, sers de femme de chambre à madame, puisqu’elle résiste.\par
{\itshape Augustin se met brutalement à l’ouvrage, M\textsuperscript{me} de Mistival se défend. }\par
MME DE MISTIVAL, à M\textsuperscript{me} de Saint-Ange : Oh ciel ! où suis-je ? Mais, madame, songez-vous donc à ce que vous permettez qu’on me fasse chez vous ? imaginez-vous donc que je ne me plaindrai pas de pareils procédés ?\par
MME DE SAINT-ANGE : Il n’est pas bien certain que vous le puissiez.\par
MME DE MISTIVAL : Oh ! grand dieu ! l’on va donc me tuer ici.\par
DOLMANCÉ : Pourquoi pas ?\par
MME DE SAINT-ANGE : Un moment, messieurs. Avant que d’exposer à vos yeux le corps de cette charmante beauté, il est bon que je vous prévienne de l’état dans lequel vous allez le trouver. Eugénie vient de me tout dire à l’oreille ; hier son mari lui donna le fouet à tour de bras, pour quelques petites fautes de ménage… et vous allez, m’assure Eugénie, trouver ses fesses comme du taffetas chiné.\par
DOLMANCÉ, {\itshape dès que M\textsuperscript{me} de Mistival est nue} : Ah ! parbleu, rien n’est plus véritable ; je ne vis, je crois, jamais un corps plus maltraité que celui-là… Comment, morbleu, mais elle en a autant par-devant que par-derrière !… Voilà pourtant un fort beau cul.\par
{\itshape Il le baise et le manie.}\par
MME DE MISTIVAL : Laissez-moi, laissez-moi, ou je vais crier au secours.\par
MME DE SAINT-ANGE, {\itshape s’approchant d’elle et la saisissant par le bras} : Écoute, putain, je vais à la fin t’instruire… Tu es pour nous une victime envoyée par ton mari même, il faut que tu subisses ton sort, rien ne saurait t’en garantir… quel sera-t-il ? je n’en sais rien ; peut-être seras-tu pendue, rouée, écartelée, tenaillée, brûlée vive, le choix de ton supplice dépend de ta fille, c’est elle qui prononcera ton arrêt, mais tu souffriras, catin, oh oui, tu ne seras immolée qu’après avoir subi une infinité de tourments préalables ; quant à tes cris, je t’en préviens, ils seraient inutiles : on égorgerait un bœuf dans ce cabinet, que ses beuglements ne seraient pas entendus ; tes chevaux, tes gens, tout est déjà parti ; encore une fois, ma belle, ton mari nous autorise à ce que nous faisons, et la démarche que tu fais n’est qu’un piège tendu à ta simplicité, et dans lequel tu vois qu’il est impossible de mieux tomber.\par
DOLMANCÉ : J’espère que voilà madame parfaitement tranquillisée maintenant.\par
EUGÉNIE : La prévenir à ce point, est assurément ce qui s’appelle avoir des égards.\par
DOLMANCÉ, lui palpant et lui claquant toujours les fesses : En vérité, madame, on voit que vous avez une amie chaude dans M\textsuperscript{me} de Saint-Ange… Où en trouver maintenant de cette franchise ? C’est qu’elle vous parle avec une vérité… Eugénie, venez mettre vos fesses à côté de celles de votre mère… que je compare vos deux culs {\itshape (Eugénie obéit)} ; ma foi le tien est beau, ma chère, mais pardieu celui de la maman n’est pas mal encore… il faut qu’un instant je m’amuse à les foutre tous les deux… Augustin, contenez madame.\par
MME DE MISTIVAL : Ah ! juste ciel, quel outrage !\par
DOLMANCÉ, {\itshape allant toujours son train, et commençant par enculer la mère} : Et point du tout, rien de plus simple… tenez, à peine l’avez-vous senti… Ah ! comme on voit que votre mari s’est souvent servi de cette route ? À ton tour, Eugénie… quelle différence… là, me voilà content, je ne voulais que peloter, pour me mettre en train. Un peu d’ordre maintenant. Premièrement, mesdames, vous Saint-Ange, et vous, Eugénie, ayez la bonté de vous armer de godemichés, afin de porter tour à tour à cette respectable dame, soit en con, soit en cul, les plus redoutables coups. Le Chevalier, Augustin et moi, agissant de nos propres membres, nous vous relaierons avec exactitude. Je vais commencer, et comme vous croyez bien, c’est encore une fois son cul qui va recevoir mon hommage ; pendant la jouissance, chacun sera maître de la condamner à tel supplice que bon lui semblera, en observant d’aller par gradation, afin de ne la point crever tout d’un coup… Augustin, console-moi, je t’en prie, en m’enculant, de l’obligation où je suis de sodomiser cette vieille vache. Eugénie, fais-moi baiser ton beau derrière, pendant que je fous celui de ta maman, et vous, madame, approchez le vôtre, que je le manie… que je le socratise… Il faut être entouré de culs, quand c’est un cul qu’on fout.\par
EUGÉNIE : Que vas-tu faire, mon ami, que vas-tu faire à cette garce ? à quoi vas-tu la condamner, en perdant ton sperme ?\par
DOLMANCÉ, {\itshape toujours foutant} : La chose du monde la plus naturelle, je vais l’épiler, et lui meurtrir les cuisses à force de pinçures.\par
MME DE MISTIVAL, {\itshape recevant cette vexation} : Ah le monstre ! le scélérat ! il m’estropie… juste ciel !\par
DOLMANCÉ : Ne l’implore pas, ma mie, il sera sourd à ta voix, comme il l’est à celle de tous les hommes ; jamais ce ciel puissant ne s’est mêlé d’un cul.\par
MME DE MISTIVAL : Ah ! comme vous me faites mal !\par
DOLMANCÉ : Incroyables effets des bizarreries de l’esprit humain !… Tu souffres, ma chère, tu pleures, et moi je décharge… Ah ! double gueuse ! je t’étranglerais, si je n’en voulais laisser le plaisir aux autres. À toi, Saint-Ange. {\itshape (M\textsuperscript{me} de Saint-Ange l’encule et l’enconne avec son godemiché, elle lui donne quelques coups de poing ; le Chevalier succède, il parcourt de même les deux routes, et la soufflette en déchargeant. Augustin vient ensuite, il agit de même, et termine par quelques chiquenaudes, quelques nasardes. Dolmancé, pendant ces différentes attaques, a parcouru de son engin les culs de tous les agents, en les excitant de ses propos.)} Allons, belle Eugénie, foutez votre mère ; enconnez-la d’abord.\par
EUGÉNIE : Venez, belle maman, venez, que je vous serve de mari ; il est un peu plus gros que celui de votre époux, n’est-ce pas, ma chère, n’importe, il entrera… Ah ! tu cries, ma mère, tu cries, quand ta fille te fout ; et toi, Dolmancé, tu m’encules ; me voilà donc à la fois incestueuse, adultère, sodomite, et tout cela pour une fille qui n’est dépucelée que d’aujourd’hui… que de progrès, mes amis… avec quelle rapidité je parcours la route épineuse du vice… oh ! je suis une fille perdue… je crois que tu décharges, ma douce mère… Dolmancé, vois ses yeux… n’est-il pas certain qu’elle décharge… Ah ! garce, je vais t’apprendre à être libertine… Tiens, gueuse, tiens. {\itshape (Elle lui presse et flétrit la gorge.)} Ah ! fous, Dolmancé… fous, mon doux ami, je me meurs.\par
{\itshape Eugénie donne, en déchargeant, dix ou douze coups de poing sur le sein et dans les flancs de sa mère.}\par
MME DE MISTIVAL, {\itshape perdant connaissance} : Ayez pitié de moi, je vous en conjure… je me trouve mal, je m’évanouis.\par
{\itshape M\textsuperscript{me} de Saint-Ange veut la secourir, Dolmancé s’y oppose. }\par
DOLMANCÉ : Eh ! non, non, laissez-la dans cette syncope, il n’a rien de si lubrique à voir qu’une femme évanouie, nous la fouetterons pour la rendre à la lumière… Eugénie, venez vous étendre sur le corps de la victime… C’est ici où je vais reconnaître si vous êtes ferme, Chevalier ; foutez-la sur le sein de sa mère en défaillance, et qu’elle nous branle, Augustin et moi, de chacune de ses mains… Vous, Saint-Ange, branlez-la pendant qu’on la fout.\par
LE CHEVALIER : En vérité, Dolmancé, ce que vous nous faites faire est horrible ; c’est outrager à la fois la nature, le ciel et les plus saintes lois de l’humanité.\par
DOLMANCÉ : Rien ne me divertit comme les solides élans de la vertu du Chevalier. Où diable voit-il dans tout ce que nous faisons le moindre outrage à la nature, au ciel et à l’humanité ? Mon ami, c’est de la nature que les roués tiennent les principes qu’ils mettent en action ; je t’ai déjà dit mille fois que la nature, qui, pour le parfait maintien des lois de son équilibre, a tantôt besoin de vices, et tantôt besoin de vertus, nous inspire tour à tour le mouvement qui lui est nécessaire ; nous ne faisons donc aucune espèce de mal en nous livrant à ces mouvements, de telle sorte que l’on puisse les supposer ; à l’égard du ciel, mon cher Chevalier, cesse donc, je te prie, d’en craindre les effets, un seul moteur agit dans l’univers, et ce moteur est la nature ; les miracles, ou plutôt les effets physiques de cette mère du genre humain, différemment interprétés par les hommes, ont été déifiés par eux sous mille formes plus extraordinaires les unes que les autres ; des fourbes ou des intrigants, abusant de la crédulité de leurs semblables, ont propagé leurs ridicules rêveries, et voilà ce que le Chevalier appelle le ciel, voilà ce qu’il craint d’outrager… Les lois de l’humanité, ajoute-t-il, sont violées par les fadaises que nous nous permettons ; retiens donc une fois pour toutes, homme simple et pusillanime, que ce que les sots appellent l’humanité n’est qu’une faiblesse née de la crainte et de l’égoïsme ; que cette chimérique vertu, n’enchaînant que les hommes faibles, est inconnue de ceux dont le stoïcisme, le courage et la philosophie forment le caractère. Agis donc, Chevalier, agis donc sans rien craindre ; nous pulvériserions cette catin qu’il n’y aurait pas encore le soupçon d’un crime, les crimes sont impossibles à l’homme ; la nature, en leur inculquant l’irrésistible désir d’en commettre, sut prudemment éloigner d’eux les actions qui pouvaient déranger ses lois ; va, sois sûr, mon ami, que tout le reste est absolument permis, et qu’elle n’a pas été absurde, au point de nous donner le pouvoir de la troubler ou de la déranger dans sa marche ; aveugles instruments de ses inspirations, nous dict[ât] -elle d’embraser l’univers ? le seul crime serait d’y résister, et tous les scélérats de la terre ne sont que les agents de ses caprices… Allons, Eugénie, placez-vous… mais, que vois-je… elle pâlit.\par
EUGÉNIE, {\itshape s’étendant sur sa mère} : Moi, pâlir ! Sacredieu, vous allez bientôt voir que non !\par
{\itshape L’attitude s’exécute ; M\textsuperscript{me} de Mistival est toujours en syncope. Quand le Chevalier a déchargé, le groupe se rompt. }\par
DOLMANCÉ : Quoi la garce n’est pas encore revenue ? des verges, des verges !… Augustin, va vite me cueillir une poignée d’épines dans le jardin. {\itshape (En attendant, il la soufflette et lui donne des camouflets.)} Oh ! par ma foi, je crains qu’elle ne soit morte, rien ne réussit.\par
EUGÉNIE, {\itshape avec humeur} : Morte ! morte ! Quoi ! il faudrait que je portasse le deuil cet été, moi qui ai fait faire de si jolies robes !\par
MME DE SAINT-ANGE, {\itshape éclatant de rire} : Ah ! le petit monstre.\par
DOLMANCÉ, {\itshape prenant les épines de la main d’Augustin, qui rentre} : Nous allons voir l’effet de ce dernier remède. Eugénie, sucez mon vit, pendant que je travaille à vous rendre une mère, et qu’Augustin me rende les coups que je vais porter ; je ne serais point fâché, Chevalier, de te voir enculer ta sœur, tu te placeras de manière à ce que je puisse te baiser les fesses pendant l’opération.\par
LE CHEVALIER : Obéissons, puisqu’il n’est aucun moyen de persuader à ce scélérat que tout ce qu’il nous fait faire est affreux.\par
{\itshape Le tableau s’arrange ; à mesure que M\textsuperscript{me} de Mistival est fouettée, elle revient à la vie. }\par
DOLMANCÉ : Eh bien ! voyez-vous l’effet de mon remède ; je vous avais bien dit qu’il était sûr.\par
MME DE MISTIVAL, {\itshape ouvrant les yeux} : Oh ciel ! pourquoi me rappelle-t-on du sein des tombeaux ? pourquoi me rendre aux horreurs de la vie ?\par
DOLMANCÉ, {\itshape toujours flagellant} : Eh ! vraiment ma petite mère, c’est que tout n’est pas dit. Ne faut-il pas que vous entendiez votre arrêt ? ne faut-il pas qu’il s’exécute ?… Allons, réunissons-nous autour de la victime, qu’elle se tienne à genoux au milieu du cercle, et qu’elle écoute en tremblant ce qui va lui être annoncé. Commencez, madame de Saint-Ange.\par
{\itshape Les prononcés suivants se font pendant que les acteurs sont toujours en action.}\par
MME DE SAINT-ANGE : Je la condamne à être pendue.\par
LE CHEVALIER : Coupée, comme chez les Chinois, en vingt-quatre mille morceaux.\par
AUGUSTIN : Tenez, moi je la tiens quitte pour être rompue vive.\par
EUGÉNIE : Ma belle petite maman sera lardée avec des mèches de soufre, auxquelles je me chargerai de mettre le feu en détail.\par
{\itshape Ici l’attitude se rompt.}\par
DOLMANCÉ, {\itshape de sang-froid} : Eh bien ! mes amis, en ma qualité de votre instituteur, moi j’adoucis l’arrêt ; mais la différence qui va se trouver entre mon prononcé et le vôtre, c’est que vos sentences n’étaient que les effets d’une mystification mordante, au lieu que la mienne va s’exécuter. J’ai là-bas un valet muni d’un des plus beaux membres qui soient peut-être dans la nature, mais malheureusement distillant le virus, et rongé d’une des plus terribles véroles qu’on ait encore vues dans le monde ; je vais le faire monter, il lancera son venin dans les deux conduits de la nature de cette chère et aimable dame, afin qu’aussi longtemps que dureront les impressions de cette cruelle maladie, la putain se souvienne de ne pas déranger sa fille quand elle se fera foutre.\par
{\itshape Tout le monde applaudit, on fait monter le valet.}\par
DOLMANCÉ, {\itshape au valet} : Lapierre, foutez cette femme-là, elle est extraordinairement saine, cette jouissance peut vous guérir, le remède n’est pas sans exemple.\par
LAPIERRE : Devant tout le monde, Monsieur ?\par
DOLMANCÉ : As-tu peur de nous montrer ton vit ?\par
LAPIERRE : Non, ma foi, car il est fort beau… Allons, Madame, ayez la bonté de vous tenir, s’il vous plaît.\par
MME DE MISTIVAL : Oh ! juste ciel ! quelle horrible condamnation !\par
EUGÉNIE : Cela vaut mieux que de mourir, maman, au moins je porterai mes jolies robes cet été.\par
DOLMANCÉ : Amusons-nous pendant ce temps-là ; mon avis serait de nous flageller tous : M\textsuperscript{me} de Saint-Ange étrillera Lapierre, pour qu’il enconne fermement M\textsuperscript{me} de Mistival, j’étrillerai M\textsuperscript{me} de Saint-Ange, Augustin m’étrillera, Eugénie étrillera Augustin, et sera fouettée elle-même très vigoureusement par le Chevalier. {\itshape (Tout s’arrange. Quand Lapierre a foutu le con, son maître lui ordonne de foutre le cul, et il le fait. Quand tout est fini :)} Bon ! sors Lapierre. Tiens, voilà dix louis… Oh ! parbleu voilà une inoculation comme Tronchin n’en fit de ses jours.\par
MME DE SAINT-ANGE : Je crois qu’il est maintenant très essentiel que le venin qui circule dans les veines de madame ne puisse s’exhaler ; en conséquence, il faut qu’Eugénie vous couse avec soin et le con et le cul, pour que l’humeur virulente, plus concentrée, moins sujette à s’évaporer, vous calcine les os plus promptement.\par
EUGÉNIE : L’excellente chose ! allons, allons, des aiguilles, du fil ; écartez vos cuisses, maman, que je vous couse, afin que vous ne me donniez plus ni frères ni sœurs.\par
{\itshape M\textsuperscript{me} de Saint-Ange donne à Eugénie une grande aiguille, où tient un gros fil rouge ciré ; Eugénie coud. }\par
MME DE MISTIVAL : Oh ciel ! quelle douleur !\par
DOLMANCÉ, {\itshape riant comme un fou} : Parbleu, l’idée est excellente ; elle te fait honneur, ma chère ; je ne l’aurais jamais trouvée.\par
EUGÉNIE, piquant de temps en temps les lèvres du con, dans l’intérieur, et quelquefois le ventre et la motte : Ce n’est rien que cela, maman, c’est pour essayer mon aiguille.\par
LE CHEVALIER : La petite putain va la mettre en sang.\par
DOLMANCÉ, {\itshape se faisant branler par M\textsuperscript{me} de Saint-Ange, en face de l’opération} : Ah ! sacredieu, comme cet écart-là me fait bander. Eugénie, multipliez vos points, pour que cela tienne mieux.\par
EUGÉNIE : J’en ferai plus de deux cents, s’il le faut… Chevalier, branlez-moi pendant que j’opère.\par
LE CHEVALIER, {\itshape obéissant} : Jamais on ne vit une petite fille aussi coquine que cela.\par
EUGÉNIE, {\itshape très enflammée} : Point d’invectives, Chevalier, ou je vous pique, contentez-vous de me chatouiller comme il faut, un peu de cul, mon ange, je t’en prie ; n’as-tu donc qu’une main ? Je n’y vois plus, je vais faire des points tout de travers. Tenez, voyez jusqu’où mon aiguille s’égare… jusque sur les cuisses, les tétons… Ah ! foutre ! quel plaisir !\par
MME DE MISTIVAL : Tu me déchires, scélérate… Que je rougis de t’avoir donné l’être !\par
EUGÉNIE : Allons, la paix, la paix, petite maman, voilà qui est fini.\par
DOLMANCÉ, {\itshape sortant bandant des mains de M\textsuperscript{me} de Saint-Ange} : Eugénie, cède-moi le cul, c’est ma partie.\par
MME DE SAINT-ANGE : Tu bandes trop, Dolmancé, tu vas la martyriser.\par
DOLMANCÉ : Qu’importe ! n’en avons-nous pas la permission par écrit ?\par
{\itshape Il la couche sur le ventre, prend une aiguille et commence à lui coudre le trou du cul.}\par
MME DE MISTIVAL, criant comme un diable : Ahe ! ahe ! ahe !\par
DOLMANCÉ, {\itshape lui plantant l’aiguille très avant dans les chairs} : Tais-toi donc, garce, ou je te mets les fesses en marmelade… Eugénie, branle-moi.\par
EUGÉNIE : Oui, mais à condition que vous piquerez plus fort, car vous conviendrez que c’est la ménager beaucoup trop.\par
{\itshape Elle le branle.}\par
MME DE SAINT-ANGE : Travaillez-moi donc un peu ces deux grosses fesses-là.\par
DOLMANCÉ : Patience, je vais bientôt la larder comme une culotte de bœuf ; tu oublies tes leçons, Eugénie, tu recalottes mon vit.\par
EUGÉNIE : C’est que les douleurs de cette gueuse-là enflamment mon imagination, au point que je ne sais exactement plus ce que je fais.\par
DOLMANCÉ : Sacré foutredieu, je commence à perdre la tête. Saint-Ange, qu’Augustin t’encule devant moi, je t’en prie, pendant que ton frère t’enconnera, et que je voie des culs surtout ; ce tableau-là va m’achever. {\itshape (Il pique les fesses, pendant que l’attitude qu’il a demandée s’arrange :)} Tiens, chère maman, reçois celle-ci, et encore celle-là.\par
{\itshape Il la pique en plus de vingt endroits.}\par
MME DE MISTIVAL : Ah ! pardon, monsieur, mille et mille pardons, vous me faites mourir.\par
DOLMANCÉ, {\itshape égaré par le plaisir} : Je le voudrais… Il y a longtemps que je n’ai si bien bandé ; je ne l’aurais pas cru après tant de décharges.\par
MME DE SAINT-ANGE, {\itshape exécutant l’attitude demandée} : Sommes-nous bien ainsi, Dolmancé ?\par
DOLMANCÉ : Qu’Augustin tourne un peu à droite, je ne vois pas assez le cul ; qu’il se penche je veux voir le trou.\par
EUGÉNIE : Ah ! foutre, voilà la bougresse en sang.\par
DOLMANCÉ : Il n’y a pas de mal. Allons, êtes-vous prêts, vous autres ? Pour moi, dans un instant, j’arrose du baume de la vie les plaies que je viens de faire.\par
MME DE SAINT-ANGE : Oui, oui, mon cœur, je décharge, nous arrivons au but en même temps que toi.\par
DOLMANCÉ, {\itshape qui a fini son opération, ne fait que multiplier ses piqûres sur les fesses de la victime, en déchargeant} : Ah triple foutredieu, mon sperme coule, il se perd, sacredieu. Eugénie, dirige-le donc sur les fesses que je martyrise. Ah ! foutre, foutre ! c’est fini, je n’en puis plus. Pourquoi faut-il que la faiblesse succède à des passions si vives ?\par
MME DE SAINT-ANGE : Fous, fous-moi, mon frère, je décharge. {\itshape (À Augustin :)} Remue-toi donc, jean-foutre ; ne sais-tu donc pas que c’est quand je décharge qu’il faut entrer le plus avant dans mon cul ? Ah ! sacré nom d’un dieu, qu’il est doux d’être ainsi foutue par deux hommes !\par
Le groupe se rompt.\par
DOLMANCÉ : Tout est dit. {\itshape (À M\textsuperscript{me} de Mistival :)} Putain, tu peux te rhabiller, et partir maintenant quand tu le voudras. Apprends que nous étions autorisés, par ton époux même, à tout ce que nous venons de faire, nous te l’avons dit, tu ne l’as pas cru, lis-en la preuve {\itshape (il lui montre la lettre)} : que cet exemple serve à te rappeler que ta fille est en âge de faire ce qu’elle veut, qu’elle aime à foutre, qu’elle est née pour foutre, et que si tu ne veux pas être foutue toi-même, le plus court est de la laisser faire ; sors, le Chevalier va te ramener ; salue la compagnie, putain, mets-toi à genoux devant ta fille, et demande-lui pardon de ton abominable conduite envers elle… Vous, Eugénie, appliquez deux bons soufflets à madame votre mère, et sitôt qu’elle sera sur le seuil de la porte, faites-le-lui passer à grands coups de pied dans le cul. {\itshape (Tout s’exécute.)} Adieu, Chevalier, ne va pas foutre madame en chemin, souviens-toi qu’elle est cousue et qu’elle a la vérole. {\itshape (Quand tout est sorti :)} Pour nous, mes amis, allons nous mettre à table, et de là, tous quatre dans le même lit. Voilà une bonne journée ; je ne mange jamais mieux, je ne dors jamais plus en paix, que quand je me suis suffisamment souillé dans le jour de ce que les sots appellent des crimes.
 


% at least one empty page at end (for booklet couv)
\ifbooklet
  \pagestyle{empty}
  \clearpage
  % 2 empty pages maybe needed for 4e cover
  \ifnum\modulo{\value{page}}{4}=0 \hbox{}\newpage\hbox{}\newpage\fi
  \ifnum\modulo{\value{page}}{4}=1 \hbox{}\newpage\hbox{}\newpage\fi


  \hbox{}\newpage
  \ifodd\value{page}\hbox{}\newpage\fi
  {\centering\color{rubric}\bfseries\noindent\large
    Hurlus ? Qu’est-ce.\par
    \bigskip
  }
  \noindent Des bouquinistes électroniques, pour du texte libre à participation libre,
  téléchargeable gratuitement sur \href{https://hurlus.fr}{\dotuline{hurlus.fr}}.\par
  \bigskip
  \noindent Cette brochure a été produite par des éditeurs bénévoles.
  Elle n’est pas faîte pour être possédée, mais pour être lue, et puis donnée.
  Que circule le texte !
  En page de garde, on peut ajouter une date, un lieu, un nom ; pour suivre le voyage des idées.
  \par

  Ce texte a été choisi parce qu’une personne l’a aimé,
  ou haï, elle a en tous cas pensé qu’il partipait à la formation de notre présent ;
  sans le souci de plaire, vendre, ou militer pour une cause.
  \par

  L’édition électronique est soigneuse, tant sur la technique
  que sur l’établissement du texte ; mais sans aucune prétention scolaire, au contraire.
  Le but est de s’adresser à tous, sans distinction de science ou de diplôme.
  Au plus direct ! (possible)
  \par

  Cet exemplaire en papier a été tiré sur une imprimante personnelle
   ou une photocopieuse. Tout le monde peut le faire.
  Il suffit de
  télécharger un fichier sur \href{https://hurlus.fr}{\dotuline{hurlus.fr}},
  d’imprimer, et agrafer ; puis de lire et donner.\par

  \bigskip

  \noindent PS : Les hurlus furent aussi des rebelles protestants qui cassaient les statues dans les églises catholiques. En 1566 démarra la révolte des gueux dans le pays de Lille. L’insurrection enflamma la région jusqu’à Anvers où les gueux de mer bloquèrent les bateaux espagnols.
  Ce fut une rare guerre de libération dont naquit un pays toujours libre : les Pays-Bas.
  En plat pays francophone, par contre, restèrent des bandes de huguenots, les hurlus, progressivement réprimés par la très catholique Espagne.
  Cette mémoire d’une défaite est éteinte, rallumons-la. Sortons les livres du culte universitaire, cherchons les idoles de l’époque, pour les briser.
\fi

\ifdev % autotext in dev mode
\fontname\font — \textsc{Les règles du jeu}\par
(\hyperref[utopie]{\underline{Lien}})\par
\noindent \initialiv{A}{lors là}\blindtext\par
\noindent \initialiv{À}{ la bonheur des dames}\blindtext\par
\noindent \initialiv{É}{tonnez-le}\blindtext\par
\noindent \initialiv{Q}{ualitativement}\blindtext\par
\noindent \initialiv{V}{aloriser}\blindtext\par
\Blindtext
\phantomsection
\label{utopie}
\Blinddocument
\fi
\end{document}
