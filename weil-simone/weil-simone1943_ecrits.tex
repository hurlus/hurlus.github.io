%%%%%%%%%%%%%%%%%%%%%%%%%%%%%%%%%
% LaTeX model https://hurlus.fr %
%%%%%%%%%%%%%%%%%%%%%%%%%%%%%%%%%

% Needed before document class
\RequirePackage{pdftexcmds} % needed for tests expressions
\RequirePackage{fix-cm} % correct units

% Define mode
\def\mode{a4}

\newif\ifaiv % a4
\newif\ifav % a5
\newif\ifbooklet % booklet
\newif\ifcover % cover for booklet

\ifnum \strcmp{\mode}{cover}=0
  \covertrue
\else\ifnum \strcmp{\mode}{booklet}=0
  \booklettrue
\else\ifnum \strcmp{\mode}{a5}=0
  \avtrue
\else
  \aivtrue
\fi\fi\fi

\ifbooklet % do not enclose with {}
  \documentclass[french,twoside]{book} % ,notitlepage
  \usepackage[%
    papersize={105mm, 297mm},
    inner=12mm,
    outer=12mm,
    top=20mm,
    bottom=15mm,
    marginparsep=0pt,
  ]{geometry}
  \usepackage[fontsize=9.5pt]{scrextend} % for Roboto
\else\ifav
  \documentclass[french,twoside]{book} % ,notitlepage
  \usepackage[%
    a5paper,
    inner=25mm,
    outer=15mm,
    top=15mm,
    bottom=15mm,
    marginparsep=0pt,
  ]{geometry}
  \usepackage[fontsize=12pt]{scrextend}
\else% A4 2 cols
  \documentclass[twocolumn]{report}
  \usepackage[%
    a4paper,
    inner=15mm,
    outer=10mm,
    top=25mm,
    bottom=18mm,
    marginparsep=0pt,
  ]{geometry}
  \setlength{\columnsep}{20mm}
  \usepackage[fontsize=9.5pt]{scrextend}
\fi\fi

%%%%%%%%%%%%%%
% Alignments %
%%%%%%%%%%%%%%
% before teinte macros

\setlength{\arrayrulewidth}{0.2pt}
\setlength{\columnseprule}{\arrayrulewidth} % twocol
\setlength{\parskip}{0pt} % classical para with no margin
\setlength{\parindent}{1.5em}

%%%%%%%%%%
% Colors %
%%%%%%%%%%
% before Teinte macros

\usepackage[dvipsnames]{xcolor}
\definecolor{rubric}{HTML}{0c71c3} % the tonic
\def\columnseprulecolor{\color{rubric}}
\colorlet{borderline}{rubric!30!} % definecolor need exact code
\definecolor{shadecolor}{gray}{0.95}
\definecolor{bghi}{gray}{0.5}

%%%%%%%%%%%%%%%%%
% Teinte macros %
%%%%%%%%%%%%%%%%%
%%%%%%%%%%%%%%%%%%%%%%%%%%%%%%%%%%%%%%%%%%%%%%%%%%%
% <TEI> generic (LaTeX names generated by Teinte) %
%%%%%%%%%%%%%%%%%%%%%%%%%%%%%%%%%%%%%%%%%%%%%%%%%%%
% This template is inserted in a specific design
% It is XeLaTeX and otf fonts

\makeatletter % <@@@


\usepackage{blindtext} % generate text for testing
\usepackage{contour} % rounding words
\usepackage[nodayofweek]{datetime}
\usepackage{DejaVuSans} % font for symbols
\usepackage{enumitem} % <list>
\usepackage{etoolbox} % patch commands
\usepackage{fancyvrb}
\usepackage{fancyhdr}
\usepackage{fontspec} % XeLaTeX mandatory for fonts
\usepackage{footnote} % used to capture notes in minipage (ex: quote)
\usepackage{framed} % bordering correct with footnote hack
\usepackage{graphicx}
\usepackage{lettrine} % drop caps
\usepackage{lipsum} % generate text for testing
\usepackage[framemethod=tikz,]{mdframed} % maybe used for frame with footnotes inside
\usepackage{pdftexcmds} % needed for tests expressions
\usepackage{polyglossia} % non-break space french punct, bug Warning: "Failed to patch part"
\usepackage[%
  indentfirst=false,
  vskip=1em,
  noorphanfirst=true,
  noorphanafter=true,
  leftmargin=\parindent,
  rightmargin=0pt,
]{quoting}
\usepackage{ragged2e}
\usepackage{setspace}
\usepackage{tabularx} % <table>
\usepackage[explicit]{titlesec} % wear titles, !NO implicit
\usepackage{tikz} % ornaments
\usepackage{tocloft} % styling tocs
\usepackage[fit]{truncate} % used im runing titles
\usepackage{unicode-math}
\usepackage[normalem]{ulem} % breakable \uline, normalem is absolutely necessary to keep \emph
\usepackage{verse} % <l>
\usepackage{xcolor} % named colors
\usepackage{xparse} % @ifundefined
\XeTeXdefaultencoding "iso-8859-1" % bad encoding of xstring
\usepackage{xstring} % string tests
\XeTeXdefaultencoding "utf-8"
\PassOptionsToPackage{hyphens}{url} % before hyperref, which load url package
\usepackage{hyperref} % supposed to be the last one, :o) except for the ones to follow
\urlstyle{same} % after hyperref

% TOTEST
% \usepackage{hypcap} % links in caption ?
% \usepackage{marginnote}
% TESTED
% \usepackage{background} % doesn’t work with xetek
% \usepackage{bookmark} % prefers the hyperref hack \phantomsection
% \usepackage[color, leftbars]{changebar} % 2 cols doc, impossible to keep bar left
% \usepackage[utf8x]{inputenc} % inputenc package ignored with utf8 based engines
% \usepackage[sfdefault,medium]{inter} % no small caps
% \usepackage{firamath} % choose firasans instead, firamath unavailable in Ubuntu 21-04
% \usepackage{flushend} % bad for last notes, supposed flush end of columns
% \usepackage[stable]{footmisc} % BAD for complex notes https://texfaq.org/FAQ-ftnsect
% \usepackage{helvet} % not for XeLaTeX
% \usepackage{multicol} % not compatible with too much packages (longtable, framed, memoir…)
% \usepackage[default,oldstyle,scale=0.95]{opensans} % no small caps
% \usepackage{sectsty} % \chapterfont OBSOLETE
% \usepackage{soul} % \ul for underline, OBSOLETE with XeTeX
% \usepackage[breakable]{tcolorbox} % text styling gone, footnote hack not kept with breakable



% Metadata inserted by a program, from the TEI source, for title page and runing heads
\title{\textbf{ Écrits historiques et politiques }}
\date{1932–1943}
\author{Simone Weil}
\def\elbibl{Simone Weil. 1932–1943. \emph{Écrits historiques et politiques}}
\def\elsource{Simone Weil. \emph{Écrits politiques et historiques}}

% Default metas
\newcommand{\colorprovide}[2]{\@ifundefinedcolor{#1}{\colorlet{#1}{#2}}{}}
\colorprovide{rubric}{red}
\colorprovide{silver}{Gray}
\@ifundefined{syms}{\newfontfamily\syms{DejaVu Sans}}{}
\newif\ifdev
\@ifundefined{elbibl}{% No meta defined, maybe dev mode
  \newcommand{\elbibl}{Titre court ?}
  \newcommand{\elbook}{Titre du livre source ?}
  \newcommand{\elabstract}{Résumé\par}
  \newcommand{\elurl}{http://oeuvres.github.io/elbook/2}
  \author{Éric Lœchien}
  \title{Un titre de test assez long pour vérifier le comportement d’une maquette}
  \date{1566}
  \devtrue
}{}
\let\eltitle\@title
\let\elauthor\@author
\let\eldate\@date


\defaultfontfeatures{
  % Mapping=tex-text, % no effect seen
  Scale=MatchLowercase,
  Ligatures={TeX,Common},
}

\@ifundefined{\columnseprulecolor}{%
    \patchcmd\@outputdblcol{% find
      \normalcolor\vrule
    }{% and replace by
      \columnseprulecolor\vrule
    }{% success
    }{% failure
      \@latex@warning{Patching \string\@outputdblcol\space failed}%
    }
}{}

\hypersetup{
  % pdftex, % no effect
  pdftitle={\elbibl},
  % pdfauthor={Your name here},
  % pdfsubject={Your subject here},
  % pdfkeywords={keyword1, keyword2},
  bookmarksnumbered=true,
  bookmarksopen=true,
  bookmarksopenlevel=1,
  pdfstartview=Fit,
  breaklinks=true, % avoid long links
  pdfpagemode=UseOutlines,    % pdf toc
  hyperfootnotes=true,
  colorlinks=false,
  pdfborder=0 0 0,
  % pdfpagelayout=TwoPageRight,
  % linktocpage=true, % NO, toc, link only on page no
}


% generic typo commands
\newcommand{\astermono}{\medskip\centerline{\color{rubric}\large\selectfont{\syms ✻}}\medskip\par}%
\newcommand{\astertri}{\medskip\par\centerline{\color{rubric}\large\selectfont{\syms ✻\,✻\,✻}}\medskip\par}%
\newcommand{\asterism}{\bigskip\par\noindent\parbox{\linewidth}{\centering\color{rubric}\large{\syms ✻}\\{\syms ✻}\hskip 0.75em{\syms ✻}}\bigskip\par}%

% lists
\newlength{\listmod}
\setlength{\listmod}{\parindent}
\setlist{
  itemindent=!,
  listparindent=\listmod,
  labelsep=0.2\listmod,
  parsep=0pt,
  % topsep=0.2em, % default topsep is best
}
\setlist[itemize]{
  label=—,
  leftmargin=0pt,
  labelindent=1.2em,
  labelwidth=0pt,
}
\setlist[enumerate]{
  label={\bf\color{rubric}\arabic*.},
  labelindent=0.8\listmod,
  leftmargin=\listmod,
  labelwidth=0pt,
}
\newlist{listalpha}{enumerate}{1}
\setlist[listalpha]{
  label={\bf\color{rubric}\alph*.},
  leftmargin=0pt,
  labelindent=0.8\listmod,
  labelwidth=0pt,
}
\newcommand{\listhead}[1]{\hspace{-1\listmod}\emph{#1}}

\renewcommand{\hrulefill}{%
  \leavevmode\leaders\hrule height 0.2pt\hfill\kern\z@}

% General typo
\DeclareTextFontCommand{\textlarge}{\large}
\DeclareTextFontCommand{\textsmall}{\small}


% commands, inlines
\newcommand{\anchor}[1]{\Hy@raisedlink{\hypertarget{#1}{}}} % link to top of an anchor (not baseline)
\newcommand\abbr[1]{#1}
\newcommand{\autour}[1]{\tikz[baseline=(X.base)]\node [draw=rubric,thin,rectangle,inner sep=1.5pt, rounded corners=3pt] (X) {\color{rubric}#1};}
\newcommand\corr[1]{#1}
\newcommand{\ed}[1]{ {\color{silver}\sffamily\footnotesize (#1)} } % <milestone ed="1688"/>
\newcommand\expan[1]{#1}
\newcommand\foreign[1]{\emph{#1}}
\newcommand\gap[1]{#1}
\renewcommand{\LettrineFontHook}{\color{rubric}}
\newcommand{\initial}[2]{\lettrine[lines=2, loversize=0.3, lhang=0.3]{#1}{#2}}
\newcommand{\initialiv}[2]{%
  \let\oldLFH\LettrineFontHook
  % \renewcommand{\LettrineFontHook}{\color{rubric}\ttfamily}
  \IfSubStr{QJ’}{#1}{
    \lettrine[lines=4, lhang=0.2, loversize=-0.1, lraise=0.2]{\smash{#1}}{#2}
  }{\IfSubStr{É}{#1}{
    \lettrine[lines=4, lhang=0.2, loversize=-0, lraise=0]{\smash{#1}}{#2}
  }{\IfSubStr{ÀÂ}{#1}{
    \lettrine[lines=4, lhang=0.2, loversize=-0, lraise=0, slope=0.6em]{\smash{#1}}{#2}
  }{\IfSubStr{A}{#1}{
    \lettrine[lines=4, lhang=0.2, loversize=0.2, slope=0.6em]{\smash{#1}}{#2}
  }{\IfSubStr{V}{#1}{
    \lettrine[lines=4, lhang=0.2, loversize=0.2, slope=-0.5em]{\smash{#1}}{#2}
  }{
    \lettrine[lines=4, lhang=0.2, loversize=0.2]{\smash{#1}}{#2}
  }}}}}
  \let\LettrineFontHook\oldLFH
}
\newcommand{\labelchar}[1]{\textbf{\color{rubric} #1}}
\newcommand{\milestone}[1]{\autour{\footnotesize\color{rubric} #1}} % <milestone n="4"/>
\newcommand\name[1]{#1}
\newcommand\orig[1]{#1}
\newcommand\orgName[1]{#1}
\newcommand\persName[1]{#1}
\newcommand\placeName[1]{#1}
\newcommand{\pn}[1]{\IfSubStr{-—–¶}{#1}% <p n="3"/>
  {\noindent{\bfseries\color{rubric}   ¶  }}
  {{\footnotesize\autour{ #1}  }}}
\newcommand\reg{}
% \newcommand\ref{} % already defined
\newcommand\sic[1]{#1}
\newcommand\surname[1]{\textsc{#1}}
\newcommand\term[1]{\textbf{#1}}

\def\mednobreak{\ifdim\lastskip<\medskipamount
  \removelastskip\nopagebreak\medskip\fi}
\def\bignobreak{\ifdim\lastskip<\bigskipamount
  \removelastskip\nopagebreak\bigskip\fi}

% commands, blocks
\newcommand{\byline}[1]{\bigskip{\RaggedLeft{#1}\par}\bigskip}
\newcommand{\bibl}[1]{{\RaggedLeft{#1}\par\bigskip}}
\newcommand{\biblitem}[1]{{\noindent\hangindent=\parindent   #1\par}}
\newcommand{\dateline}[1]{\medskip{\RaggedLeft{#1}\par}\bigskip}
\newcommand{\labelblock}[1]{\medbreak{\noindent\color{rubric}\bfseries #1}\par\mednobreak}
\newcommand{\salute}[1]{\bigbreak{#1}\par\medbreak}
\newcommand{\signed}[1]{\bigbreak\filbreak{\raggedleft #1\par}\medskip}

% environments for blocks (some may become commands)
\newenvironment{borderbox}{}{} % framing content
\newenvironment{citbibl}{\ifvmode\hfill\fi}{\ifvmode\par\fi }
\newenvironment{docAuthor}{\ifvmode\vskip4pt\fontsize{16pt}{18pt}\selectfont\fi\itshape}{\ifvmode\par\fi }
\newenvironment{docDate}{}{\ifvmode\par\fi }
\newenvironment{docImprint}{\vskip6pt}{\ifvmode\par\fi }
\newenvironment{docTitle}{\vskip6pt\bfseries\fontsize{18pt}{22pt}\selectfont}{\par }
\newenvironment{msHead}{\vskip6pt}{\par}
\newenvironment{msItem}{\vskip6pt}{\par}
\newenvironment{titlePart}{}{\par }


% environments for block containers
\newenvironment{argument}{\itshape\parindent0pt}{\vskip1.5em}
\newenvironment{biblfree}{}{\ifvmode\par\fi }
\newenvironment{bibitemlist}[1]{%
  \list{\@biblabel{\@arabic\c@enumiv}}%
  {%
    \settowidth\labelwidth{\@biblabel{#1}}%
    \leftmargin\labelwidth
    \advance\leftmargin\labelsep
    \@openbib@code
    \usecounter{enumiv}%
    \let\p@enumiv\@empty
    \renewcommand\theenumiv{\@arabic\c@enumiv}%
  }
  \sloppy
  \clubpenalty4000
  \@clubpenalty \clubpenalty
  \widowpenalty4000%
  \sfcode`\.\@m
}%
{\def\@noitemerr
  {\@latex@warning{Empty `bibitemlist' environment}}%
\endlist}
\newenvironment{quoteblock}% may be used for ornaments
  {\begin{quoting}}
  {\end{quoting}}

% table () is preceded and finished by custom command
\newcommand{\tableopen}[1]{%
  \ifnum\strcmp{#1}{wide}=0{%
    \begin{center}
  }
  \else\ifnum\strcmp{#1}{long}=0{%
    \begin{center}
  }
  \else{%
    \begin{center}
  }
  \fi\fi
}
\newcommand{\tableclose}[1]{%
  \ifnum\strcmp{#1}{wide}=0{%
    \end{center}
  }
  \else\ifnum\strcmp{#1}{long}=0{%
    \end{center}
  }
  \else{%
    \end{center}
  }
  \fi\fi
}


% text structure
\newcommand\chapteropen{} % before chapter title
\newcommand\chaptercont{} % after title, argument, epigraph…
\newcommand\chapterclose{} % maybe useful for multicol settings
\setcounter{secnumdepth}{-2} % no counters for hierarchy titles
\setcounter{tocdepth}{5} % deep toc
\markright{\@title} % ???
\markboth{\@title}{\@author} % ???
\renewcommand\tableofcontents{\@starttoc{toc}}
% toclof format
% \renewcommand{\@tocrmarg}{0.1em} % Useless command?
% \renewcommand{\@pnumwidth}{0.5em} % {1.75em}
\renewcommand{\@cftmaketoctitle}{}
\setlength{\cftbeforesecskip}{\z@ \@plus.2\p@}
\renewcommand{\cftchapfont}{}
\renewcommand{\cftchapdotsep}{\cftdotsep}
\renewcommand{\cftchapleader}{\normalfont\cftdotfill{\cftchapdotsep}}
\renewcommand{\cftchappagefont}{\bfseries}
\setlength{\cftbeforechapskip}{0em \@plus\p@}
% \renewcommand{\cftsecfont}{\small\relax}
\renewcommand{\cftsecpagefont}{\normalfont}
% \renewcommand{\cftsubsecfont}{\small\relax}
\renewcommand{\cftsecdotsep}{\cftdotsep}
\renewcommand{\cftsecpagefont}{\normalfont}
\renewcommand{\cftsecleader}{\normalfont\cftdotfill{\cftsecdotsep}}
\setlength{\cftsecindent}{1em}
\setlength{\cftsubsecindent}{2em}
\setlength{\cftsubsubsecindent}{3em}
\setlength{\cftchapnumwidth}{1em}
\setlength{\cftsecnumwidth}{1em}
\setlength{\cftsubsecnumwidth}{1em}
\setlength{\cftsubsubsecnumwidth}{1em}

% footnotes
\newif\ifheading
\newcommand*{\fnmarkscale}{\ifheading 0.70 \else 1 \fi}
\renewcommand\footnoterule{\vspace*{0.3cm}\hrule height \arrayrulewidth width 3cm \vspace*{0.3cm}}
\setlength\footnotesep{1.5\footnotesep} % footnote separator
\renewcommand\@makefntext[1]{\parindent 1.5em \noindent \hb@xt@1.8em{\hss{\normalfont\@thefnmark . }}#1} % no superscipt in foot


% orphans and widows
\clubpenalty=9996
\widowpenalty=9999
\brokenpenalty=4991
\predisplaypenalty=10000
\postdisplaypenalty=1549
\displaywidowpenalty=1602
\hyphenpenalty=400
% Copied from Rahtz but not understood
\def\@pnumwidth{1.55em}
\def\@tocrmarg {2.55em}
\def\@dotsep{4.5}
\emergencystretch 3em
\hbadness=4000
\pretolerance=750
\tolerance=2000
\vbadness=4000
\def\Gin@extensions{.pdf,.png,.jpg,.mps,.tif}
% \renewcommand{\@cite}[1]{#1} % biblio

\makeatother % /@@@>
%%%%%%%%%%%%%%
% </TEI> end %
%%%%%%%%%%%%%%


%%%%%%%%%%%%%
% footnotes %
%%%%%%%%%%%%%
\renewcommand{\thefootnote}{\bfseries\textcolor{rubric}{\arabic{footnote}}} % color for footnote marks

%%%%%%%%%
% Fonts %
%%%%%%%%%
\usepackage[]{roboto} % SmallCaps, Regular is a bit bold
% \linespread{0.90} % too compact, keep font natural
\newfontfamily\fontrun[]{Roboto Condensed Light} % condensed runing heads
\ifav
  \setmainfont[
    ItalicFont={Roboto Light Italic},
  ]{Roboto}
\else\ifbooklet
  \setmainfont[
    ItalicFont={Roboto Light Italic},
  ]{Roboto}
\else
\setmainfont[
  ItalicFont={Roboto Italic},
]{Roboto Light}
\fi\fi
\renewcommand{\LettrineFontHook}{\bfseries\color{rubric}}
% \renewenvironment{labelblock}{\begin{center}\bfseries\color{rubric}}{\end{center}}

%%%%%%%%
% MISC %
%%%%%%%%

\setdefaultlanguage[frenchpart=false]{french} % bug on part


\newenvironment{quotebar}{%
    \def\FrameCommand{{\color{rubric!10!}\vrule width 0.5em} \hspace{0.9em}}%
    \def\OuterFrameSep{\itemsep} % séparateur vertical
    \MakeFramed {\advance\hsize-\width \FrameRestore}
  }%
  {%
    \endMakeFramed
  }
\renewenvironment{quoteblock}% may be used for ornaments
  {%
    \savenotes
    \setstretch{0.9}
    \normalfont
    \begin{quotebar}
  }
  {%
    \end{quotebar}
    \spewnotes
  }


\renewcommand{\headrulewidth}{\arrayrulewidth}
\renewcommand{\headrule}{{\color{rubric}\hrule}}

% delicate tuning, image has produce line-height problems in title on 2 lines
\titleformat{name=\chapter} % command
  [display] % shape
  {\vspace{1.5em}\centering} % format
  {} % label
  {0pt} % separator between n
  {}
[{\color{rubric}\huge\textbf{#1}}\bigskip] % after code
% \titlespacing{command}{left spacing}{before spacing}{after spacing}[right]
\titlespacing*{\chapter}{0pt}{-2em}{0pt}[0pt]

\titleformat{name=\section}
  [block]{}{}{}{}
  [\vbox{\color{rubric}\large\raggedleft\textbf{#1}}]
\titlespacing{\section}{0pt}{0pt plus 4pt minus 2pt}{\baselineskip}

\titleformat{name=\subsection}
  [block]
  {}
  {} % \thesection
  {} % separator \arrayrulewidth
  {}
[\vbox{\large\textbf{#1}}]
% \titlespacing{\subsection}{0pt}{0pt plus 4pt minus 2pt}{\baselineskip}

\ifaiv
  \fancypagestyle{main}{%
    \fancyhf{}
    \setlength{\headheight}{1.5em}
    \fancyhead{} % reset head
    \fancyfoot{} % reset foot
    \fancyhead[L]{\truncate{0.45\headwidth}{\fontrun\elbibl}} % book ref
    \fancyhead[R]{\truncate{0.45\headwidth}{ \fontrun\nouppercase\leftmark}} % Chapter title
    \fancyhead[C]{\thepage}
  }
  \fancypagestyle{plain}{% apply to chapter
    \fancyhf{}% clear all header and footer fields
    \setlength{\headheight}{1.5em}
    \fancyhead[L]{\truncate{0.9\headwidth}{\fontrun\elbibl}}
    \fancyhead[R]{\thepage}
  }
\else
  \fancypagestyle{main}{%
    \fancyhf{}
    \setlength{\headheight}{1.5em}
    \fancyhead{} % reset head
    \fancyfoot{} % reset foot
    \fancyhead[RE]{\truncate{0.9\headwidth}{\fontrun\elbibl}} % book ref
    \fancyhead[LO]{\truncate{0.9\headwidth}{\fontrun\nouppercase\leftmark}} % Chapter title, \nouppercase needed
    \fancyhead[RO,LE]{\thepage}
  }
  \fancypagestyle{plain}{% apply to chapter
    \fancyhf{}% clear all header and footer fields
    \setlength{\headheight}{1.5em}
    \fancyhead[L]{\truncate{0.9\headwidth}{\fontrun\elbibl}}
    \fancyhead[R]{\thepage}
  }
\fi

\ifav % a5 only
  \titleclass{\section}{top}
\fi

\newcommand\chapo{{%
  \vspace*{-3em}
  \centering % no vskip ()
  {\Large\addfontfeature{LetterSpace=25}\bfseries{\elauthor}}\par
  \smallskip
  {\large\eldate}\par
  \bigskip
  {\Large\selectfont{\eltitle}}\par
  \bigskip
  {\color{rubric}\hline\par}
  \bigskip
  {\Large LIVRE LIBRE À PRIX LIBRE, DEMANDEZ AU COMPTOIR\par}
  \centerline{\small\color{rubric} {hurlus.fr, tiré le \today}}\par
  \bigskip
}}


\begin{document}
\pagestyle{empty}
\ifbooklet{
  \thispagestyle{empty}
  \centering
  {\LARGE\bfseries{\elauthor}}\par
  \bigskip
  {\Large\eldate}\par
  \bigskip
  \bigskip
  {\LARGE\selectfont{\eltitle}}\par
  \vfill\null
  {\color{rubric}\setlength{\arrayrulewidth}{2pt}\hline\par}
  \vfill\null
  {\Large LIVRE LIBRE À PRIX LIBRE, DEMANDEZ AU COMPTOIR\par}
  \centerline{\small{hurlus.fr, tiré le \today}}\par
  \newpage\null\thispagestyle{empty}\newpage
  \addtocounter{page}{-2}
}\fi

\thispagestyle{empty}
\ifaiv
  \twocolumn[\chapo]
\else
  \chapo
\fi
{\it\elabstract}
\bigskip
\makeatletter\@starttoc{toc}\makeatother % toc without new page
\bigskip

\pagestyle{main} % after style

  \frontmatter 
\begin{titlepage}
 
\byline{Simone Weil (1909-1943)}
1932–1943Écrits historiques et politiques
\end{titlepage}

\begin{argument}Note de l’éditeur\noindent Bien qu'on ne puisse pas faire de distinction nette entre les écrits histo­riques et les écrits politiques contenus dans ce volume (car même dans les écrits politiques il y a des considérations historiques, et dans les écrits historiques, une intention politique), on a cherché à réunir dans la première partie les écrits de caractère plutôt historique et dans la seconde, au contraire, les écrits de caractère plutôt politique.\par
La première partie contient, entre autres, le grand article sur les origines de l'hitlérisme, article qui concerne bien moins Hitler que Rome et l'Empire Romain. Elle se termine par des textes concernant l'histoire contemporaine de l'Allemagne et de l'Espagne. Ces textes ont pour origine deux expériences personnelles de Simone Weil : le voyage qu'elle fit en Allemagne en août et septembre 1932, et sa participation à la guerre d'Espagne en 1936.\par
Les textes contenus dans la seconde partie ont pour objet la politique française et expriment plus directement le désir de voir cette politique inflé­chie dans un sens déterminé. Ils ont été groupés en trois sections. La première concerne la politique extérieure et la menace de guerre ; la seconde, les difficultés intérieures auxquelles se heurta le gouvernement de Front Popu­laire ; la troisième, la politique à l'égard des colonies. Dans chaque section, on a cherché à classer les différents écrits selon l'ordre chronologique. Pour les fragments qui n'étaient pas datés, on a proposé, entre parenthèses, une date approximative.\par

\begin{center}
*\end{center}
\noindent On doit à la vérité de noter que Simone Weil n'approuvait plus dans les dernières années de sa vie le pacifisme extrême qu'elle avait soutenu, comme on le verra dans ce livre, jusqu'au printemps de 1939 (invasion de la Tchécoslovaquie par l'Allemagne hitlérienne).
\end{argument}

\mainmatter \section[Première partie, Histoire]{Première partie, \\
Histoire}\renewcommand{\leftmark}{Première partie, \\
Histoire}

\noindent \par

\begin{center}
\end{center}
\subsection[1. Quelques réflexions sur les origines de l’hitlérisme, (1939-1940) ]{1. \\
Quelques réflexions sur les origines de l’hitlérisme \\
(1939-1940) \protect\footnotemark }
\footnotetext{ La 2\textsuperscript{e} partie de cette étude : {\itshape Hitler et la politique extérieure de la Rome} antique, a été publiée dans les {\itshape Nouveaux Cahiers} (n° 53, 1\textsuperscript{er} janvier 1940). La 3\textsuperscript{e} déjà imprimée sur épreuves, fut refusée par la censure. (Note de l'éditeur.)}
\noindent \par
\subsubsection[I. - Permanence et changements, des caractères nationaux]{I. - Permanence et changements \\
des caractères nationaux}
\noindent \par
À la faveur des événements, de vieilles expressions reparaissent ; on parle de nouveau de « la France éternelle » et de « l'éternelle Allemagne », la place de l'adjectif suffisant à en indiquer la portée. Il faut examiner une bonne fois ces formules et savoir si elles ont un sens. Car ni la guerre ni la paix ne peuvent être conçues de la même manière, selon qu'elles ont ou non un sens. Si une nation nuisible aux autres est telle de toute éternité, le seul but qu'on puisse assigner aux négociations comme aux combats est de l'anéantir, ou du moins de l'enchaîner de chaînes capables de durer plusieurs siècles ; si une nation amoureuse de paix et de liberté pour elle-même et autrui est telle de toute éternité, on ne peut jamais lui accorder trop de puissance. Si au contraire l'esprit des nations change, le but de la politique, en guerre comme en paix, doit être de créer, du moins dans toute la mesure des possibilités humaines, des conditions de vie internationale telles que les nations qui sont paisibles le restent, et que celles qui ne le sont pas le deviennent. Il y a là deux politiques possibles, qui diffèrent presque sur tous les points. Il faut choisir. Un choix erroné serait fatal ; ne pas choisir serait pire. En 1918 on n'a pas choisi ; nous en souffrons les conséquences.\par
Que parfois certains caractères nationaux durent des siècles, ou même des millénaires, on ne peut en douter après examen. Don Quichotte vit toujours en Espagne ; bien plus, la grandiloquence qui y enfle les paroles des hommes politiques se retrouve non seulement dans les tragiques espagnols des XVI\textsuperscript{e} et XVII\textsuperscript{e} siècles, inspirateurs et modèles de notre Corneille, mais encore dans les poètes latins d'origine espagnole, Lucain et Sénèque. Qui croirait en revanche aujourd'hui qu'au XVI\textsuperscript{e} siècle l'Espagne ait pu menacer par son ambition et sa puissance les libertés du monde ? Un siècle plus tard c'était déjà incroyable. Si l'on examine l'Italie, deux peuples peuvent-ils différer plus complètement que les Romains antiques et les Italiens du moyen âge ? Les Romains n'avaient de supériorité que dans les armes et l'organisation d'un État centralisé. Les Italiens du moyen âge et de la Renaissance étaient incapables d'unité, d'ordre et d'administration ; ils ne se battaient guère que par procuration ; au moyen de mercenaires, et la guerre était conçue de manière telle que Machiavel cite une campagne d'été, au cours d'une guerre menée par Florence, pendant laquelle il n'y eut ni un mort ni un blessé d'un côté ni de l'autre. En revanche les Italiens parurent à cette époque, ce que les Romains n'avaient jamais été avec tous leurs efforts d'imitation servile, les héritiers directs des Grecs pour toutes les grâces et les pouvoirs de l'esprit. L'histoire des nations offre ainsi des exemples, également surprenants de permanence et de transformations. Mais ce sont deux nations seulement qui nous intéressent ici, la France et l'Allemagne ; et les caractères nationaux n'importent pas tous, mais seulement ceux qui font qu'une nation constitue ou ne constitue pas un danger grave pour la civilisation, la paix et la liberté des peuples. La question est de savoir si ces caractères, en ce qui concerne la France et l'Allemagne, sont durables ou chan­geants. La réponse à cette question ne peut être cherchée que dans le passé ; car l'avenir nous demeure caché.\par
Un fait d'abord éclate aux yeux : jusqu'au XX\textsuperscript{e} siècle il n'y a jamais eu de danger de domination universelle de la part des Germains ou des Allemands ; car les prétentions de la maison de Habsbourg à l'empire du monde sont restées aussi vides de portée effective que les prédictions de Merlin, jusqu'au jour où cette maison est devenue espagnole. Ce fait indéniable n'a de signifi­cation que si l'on veut bien se rappeler, ce qu'on oublie aujourd'hui si facile­ment, que le danger de domination universelle n'a rien de nouveau ou d'inouï. Rome la première n'a pas seulement menacé, mais anéanti les libertés du monde, si du moins l'on veut se servir de l'expression exagérée des écrivains latins, et nommer monde une large étendue autour de la Méditerranée. Soit dit en passant, ceux qui, comme Péguy et tant d'autres, accordent une part égale de leur admiration à l'Empire romain et aux guerres pour l'indépendance des patries commettent une contradiction sans excuse. Au moyen âge, après la brève résurrection de l'Empire romain tentée par Charlemagne, les deux héri­tiers de cet Empire, le Saint Empire Romain Germanique et la Papauté, se sont livré pour la domination temporelle de la chrétienté une lutte qui d'ailleurs ne comportait aucun danger pour les libertés locales, grâce au désordre de l'époque et à la nature intrinsèquement faible de ces deux pouvoirs. Mais depuis quatre siècles l'Occident a subi trois menaces graves de domination universelle ; la première est venue d'Espagne, sous Charles Quint et Philippe II, la seconde de France, sous Louis XIV, la troisième de France encore, sous le Directoire et Napoléon. Les trois menaces ont été écartées après un sacrifice effroyable de vies humaines, et dans les trois cas l'Angleterre a joué le rôle principal. Aujourd'hui, cette vieille histoire se reproduit, sans différences considérables. Le danger n'est peut-être pas plus grave ; la lutte n'est pas plus atroce. Le massacre des non-combattants, des femmes et des enfants n'est pas une nouveauté, comme certains hommes d'État l'ont affirmé naïvement. Ni l'Espagne, ni la France n'ont été, à la suite de leurs défaites, anéanties, démem­brées, ou même désarmées ; nulle contrainte ne leur a été imposée. Le danger s'est déplacé par suite du changement des circonstances. Qui peut dire de quelle manière il pourra encore se déplacer ? Nul ne peut rien en savoir.\par

\begin{center}
*\end{center}
\paragraph[« la France éternelle »]{« la France éternelle »}
\noindent Ce bref rappel de faits universellement connus résout déjà la question posée en ce qui concerne la France. Il n'y a pas de « France éternelle », tout au moins en ce qui concerne la paix et la liberté. Napoléon n'a pas inspiré au monde moins de terreur et d'horreur qu'Hitler, ni moins justement. Quiconque parcourt, par exemple, le Tyrol, y trouve à chaque pas des inscriptions rappe­lant les cruautés commises alors par les soldats français contre un peuple pauvre, laborieux et heureux pour autant qu'il est libre. Oublie-t-on ce que la France a fait subir à la Hollande, à la Suisse, à l'Espagne ? On prétend que Napoléon a propagé, les armes à la main, les idées de liberté et d'égalité de la Révolution française ; mais ce qu'il a principalement propagé, c'est l'idée de l'État centralisé, l'État comme source unique d'autorité et objet exclusif de dévouement ; l'État ainsi conçu, inventé pour ainsi dire par Richelieu, conduit à un point plus haut de perfection par Louis XIV, à un point plus haut encore par la Révolution, puis par Napoléon, a trouvé aujourd'hui sa forme suprême en Allemagne. Il nous fait à présent horreur, et cette horreur est juste ; n'oublions pas pourtant qu'il est venu de chez nous.\par
Sous la Restauration, plus encore sous Louis-Philippe, la France était de­venue la plus pacifique des nations. Pourtant à l'étranger, le souvenir du passé faisait qu'on continuait à la craindre, comme nous avons craint l'Allemagne après 1918, et à regretter que ses vainqueurs ne l'eussent pas anéantie en 1814 ou 1815. Parmi les Français eux-mêmes beaucoup désiraient ouvertement la guerre et la conquête, et se croyaient un droit héréditaire à l'empire du monde. Que penserait-on aujourd'hui, par exemple, de ces vers écrits en 1831 par Barthélemy, poète alors populaire :\par

\begin{quoteblock}
 \noindent ... Berlin est le domaine \\
Que la France a pour but lorsqu'elle se promène.
 \end{quoteblock}

\noindent Et que penser d'ailleurs de tant de vers de Hugo à l'éloge des conquêtes françaises, où l'habitude ne nous laisse plus voir qu'un exercice littéraire ? Par bonheur, ce courant ne l'emporta pas ; pour des raisons mystérieuses la France avait cessé d'être une nation conquérante, du moins en Europe. Le second Empire même ne put par ses folies en faire une nation conquérante, mais seulement une nation conquise. La victoire de 1918 l'a rendue, si possible, moins conquérante qu'avant, de sorte qu'elle croit ne l'avoir jamais été et ne plus pouvoir le redevenir. Ainsi changent les peuples.\par
Si l'on remonte plus haut dans le passé, il y a analogie entre Hitler et Louis XIV, non certes quant à leur personne, mais quant à leur rôle. Louis XIV était un roi légitime, mais il n'en avait pas l'esprit ; les misères de son enfance, environnée des terreurs de la Fronde, lui avaient donné pour une part l'état d'esprit des dictateurs modernes qui, partis de rien, humiliés dans leur jeu­nesse, n'ont cru pouvoir commander leur peuple qu'en le matant. Le régime établi par lui méritait déjà, pour la première fois en Europe depuis Rome, le nom moderne de totalitaire. L'abaissement des esprits et des cœurs pendant la seconde partie de son règne, celle où a écrit Saint-Simon, est quelque chose d'aussi douloureux que tout ce qu'on a pu voir par la suite de plus triste. Aucune classe de la nation n'y a échappé. La propagande intérieure, malgré l'absence des moyens techniques actuels, atteignait une perfection difficile à dépasser ; Liselotte, la seconde Madame, n'écrivait-elle pas qu'on ne pouvait publier aucun livre sans y insérer les louanges du roi ? Et pour trouver aujour­d'hui quelque chose de comparable au ton presque idolâtre de ces louanges, ce n'est pas même à Hitler, c'est presque à Staline qu'il faut penser. Nous avons aujourd'hui l'habitude de voir dans ces basses flatteries une simple clause de style, liée à l'institution monarchique ; mais c'est une erreur ; ce ton était tout nouveau en France, où jusque-là, sinon dans une certaine mesure sous Richelieu, on n'avait pas coutume d'être servile. Quant aux cruautés des persé­cutions et au silence établi autour d'elles, la comparaison se soutient facile­ment. L'emprise du pouvoir central sur la vie des particuliers n'était peut-être pas moindre, quoiqu'il soit difficile d'en juger.\par
La politique extérieure procédait du même esprit d'orgueil impitoyable, du même art savant d'humilier, de la même mauvaise foi que la politique d'Hitler. La première action de Louis XIV fut de contraindre l'Espagne, à qui il venait de s'allier par mariage, à s'humilier publiquement devant lui sous menace de guerre. Il humilia de la même manière le pape ; il contraignit le doge de Gênes à venir lui demander pardon ; il prit Strasbourg exactement comme Hitler a pris Prague, en pleine paix, parmi les larmes des habitants impuissants à résister, au mépris d'un traité tout récemment conclu et qui avait fixé des frontières théoriquement définitives. La dévastation atroce du Palatinat n'eut pas non plus l'excuse des nécessités de la guerre. L'agression non motivée contre la Hollande faillit anéantir un peuple libre et fier de l'être, et dont la civilisation à ce moment était plus brillante encore que celle de la France, comme les noms de Rembrandt, Spinoza, Huyghens le montrent assez. On aurait peine à trouver dans la littérature allemande contemporaine quelque chose de plus bassement cruel qu'un petit poème gai composé à cette occasion par La Fontaine pour prédire la destruction des cités hollandaises \footnote{\noindent Voici quelques-uns des vers de La Fontaine :\par
À vous, marchands de fromage,\par
Salut, révérence, hommage.\par
………………………………..\par
Votre lâche paysan…\par
Tournera bientôt visage.\par
………………………………..Mandez lettres et messages\par
Chez le Goth et l'Allemand,\par
Et dans tout le voisinage\par
Criez au meurtre, à l'outrage\par
On me pille, on me saccage,\par
Proposez un arbitrage,\par
Offrez des places d'otage\par
Malgré votre tripotage\par
Et votre patelinage,\par
Notre roi vaillant et sage\par
Ruinera ville et pacage,\par
Mettra votre or au pillage,\par
Vos personnes au carcan\par
Et vos meubles d l'encan.\par
(Note de S. W.)\par
............................………..\par
Je ne suis sorcier ni mage,\par
Mais je prédis et je gage\par
Qu'on verra naître l'herbage\par
Dans les places d'Amsterdam,\par
Que Dordrecht et Rotterdam\par
Ne seront qu'un hermitage,\par
Qu'un lieu désert et sauvage.\par
............................………..
}. Que La Fontaine soit un grand poète rend seulement la chose plus triste. Louis XIV devint enfin l'ennemi public en Europe, l'homme par qui tout homme libre, toute cité libre se sentaient menacés. Cette terreur et cette haine, on les voit dans les textes anglais de l'époque, par exemple le journal de Pepys ; et Winston Churchill, dans sa biographie de son illustre ancêtre Marlborough, témoigne rétrospectivement à Louis XIV les mêmes sentiments qui l'animent contre Hitler.\par
Mais le véritable, le premier précurseur d'Hitler depuis l'antiquité est sans doute Richelieu. Il a inventé l'État. Avant lui, des rois, comme Louis XI, avaient pu établir un pouvoir fort ; mais ils défendaient leur couronne. Des sujets avaient pu se montrer citoyens dans le maniement des affaires ; ils se dévouaient au bien public. L'État auquel Richelieu s'est donné corps et âme, au point de n'avoir plus conscience d'aucune ambition personnelle, n'était pas la couronne, encore moins le bien public ; c'était la machine anonyme, aveu­gle, productrice d'ordre et de puissance, que nous connaissons aujourd'hui sous ce nom et que certains pays adorent. Cette adoration implique un mépris avoué de toute morale, et en même temps le sacrifice de soi-même qui accompagne d'ordinaire la vertu ; ce mélange se trouve chez Richelieu, qui, disait, avec la merveilleuse clarté d'esprit des Français de cette époque, que le salut de l'État ne se procure pas par les mêmes règles que le salut de l'âme, parce que le salut de l'âme se fait dans, l'autre monde, au lieu que les États ne peuvent se sauver que dans ce monde-ci. Sans recourir aux pamphlets de ses adversaires, ses propres mémoires montrent comment il a appliqué ce princi­pe, par des violations de traités, des intrigues destinées à prolonger indéfini­ment les guerres les plus atroces, et le sacrifice de toute autre considération, sans exception aucune, à la réputation de l'État, c'est-à-dire, dans le mauvais langage d'aujourd'hui, à son prestige. Le cardinal-infant, dont on a pu voir récemment à Genève le visage courageux, lucide et triste, peint avec amour par Vélasquez, fit précéder ses armes en France d'un manifeste qu'il suffirait aujourd’hui de traduire, sans changer aucun mot que les noms de Français et de Richelieu, pour en faire une excellente proclamation au peuple allemand. Car c'est une erreur grave de croire que la morale de cette époque, même en matière internationale, différât de la nôtre ; on y trouve, et même dans des discours de ministres, des textes qui ressembleraient aux meilleurs textes d'aujourd'hui à l'éloge d'une politique de paix s'ils n'étaient mieux raisonnés et infiniment mieux écrits. Une partie des ennemis de Richelieu, de son propre aveu, étaient animés par une horreur sincère de la guerre. On avait la même morale qu'aujourd'hui ; on la pratiquait aussi peu ; et comme aujourd'hui tous ceux qui faisaient la guerre disaient, à tort ou à raison, qu'ils la faisaient pour mieux l'éviter.\par
Si on remonte plus haut dans l'histoire de la France, on voit notamment que sous Charles VI les Flamands se disaient entre eux, pour s'encourager à maintenir leurs droits les armes à la main : « Voulons-nous devenir esclaves comme les Français ? » À vrai dire, depuis la mort de Charles V jusqu'à la Révolution, la France a eu en Europe la réputation d'être la terre d'élection non pas de la liberté, mais bien plutôt de l'esclavage, du fait que les impôts n'étaient soumis à aucune règle et dépendaient exclusivement de la volonté du roi. L'Allemagne, pendant la même période, fut regardée comme étant émi­nemment une terre de liberté ; qu'on lise plutôt les notes de Machiavel sur la France et l'Allemagne. Il en est de même de l'Angleterre, bien entendu, quelques moments pénibles mis à part. L'Espagne n'a perdu toutes ses libertés que lorsque le petit-fils de Louis XIV en eut occupé le trône. Les Français eux-mêmes, depuis Charles VI jusqu'à l'écrasement de la Fronde, ne perdirent jamais le sentiment qu'ils étaient privés de leurs droits naturels et légaux ; le XVIII\textsuperscript{e} siècle n'a pas fait autre chose que reprendre une longue tradition anéantie pendant plus d'un demi-siècle par Louis XIV. C'est au XIX\textsuperscript{e} siècle seulement que la France s'est regardée elle-même et a été regardée comme étant par excellence un pays de lumière et de liberté ; les hommes du XVIII\textsuperscript{e} siècle, dont la gloire a tant contribué à donner à la France cette réputation, pensaient cela, eux, de l'Angleterre. Au reste, jusqu'au siècle, la culture occidentale formait un tout ; nul, avant le règne de Louis XIV, n'eût songé à la découper par nations. La « France éternelle » est de fabrication très récente.
\paragraph[L'éternelle Allemagne]{L'éternelle Allemagne}
\noindent \par
« L'éternelle Allemagne » est de fabrication à peu prés aussi récente. Avant Frédéric II de Prusse on ne peut trouver dans aucune période de l'histoire de l'Allemagne aucun des caractères qui, en 1939, la font haïr et redouter ; ni l'inclination à commander et à obéir d'une manière absolue, ni l'inclination à dominer le monde par la terreur de ses armes. Rien de tout cela n'existait avant que la Prusse n'eût une armée permanente. L'Allemagne qui nous gêne tant aujourd'hui, on peut dire que c'est la France qui l'a faite étape par étape. Richelieu et Louis XIV furent les modèles copiés par Frédéric II. Napoléon, après avoir si follement aboli les survivances du Saint Empire Ro­main germanique, suscita par ses conquêtes et son oppression le nationalisme allemand. Napoléon III, par sa malheureuse agression de 1870, éleva l'Alle­magne au premier rang des grandes puissances. Le romantisme allemand, dont l'adoration de la force est un aspect, et dont Hitler, ce wagnérien, est pour une part l'héritier, date de l'époque où toutes les forces du pays furent tendues à se rompre contre Napoléon ; au XVIII\textsuperscript{e} siècle il n'y en avait pas trace. S'il y a eu au monde un penseur inspiré par l'esprit dont la France se réclame, c'est bien Kant. Au reste, faute de pouvoir trouver dans les siècles récents ou dans le moyen âge des précédents à l'Allemagne hitlérienne, ceux qui croient à « l'éternelle Allemagne » sautent généralement deux mille ans, comme Hitler lui-même, et citent César et Tacite.
\paragraph[L'hitlérisme et les germains]{L'hitlérisme et les germains}
\noindent Le préjugé raciste, qu'ils n'avouent pourtant pas, leur fait ainsi fermer les yeux à une vérité bien claire ; c'est que ce qui, il y a deux mille ans, ressem­blait à l'Allemagne hitlérienne, ce ne sont pas les Germains, c'est Rome. Entre les peuplades décrites par César et Tacite et les Allemands actuels, il n'y a aucune ressemblance. On veut en trouver une dans le fait que ces peuplades aimaient guerroyer. C'est un goût assez général chez les peuples primitifs, libres et plus au moins nomades. Hitler et les siens n'aiment pas la guerre ; ils aiment la domination et ne rêvent que de paix, une paix, bien entendu, soumise à leurs volontés ; c'était aussi le cas de l'ancienne Rome.\par
On cite beaucoup les lignes de César : « La plus grande gloire pour ces peuples est d'avoir autour d'eux le plus loin possible des territoires dévastés et des solitudes. Ils regardent comme la marque du courage que leurs voisins. soient chassés de leurs champs et s'en aillent, et que personne n'ose s'établir près d'eux ; ils croient aussi devoir être ainsi plus en sécurité, ayant supprimé le danger d'une agression brusque. » C'est là une coutume de peuplades amou­reuses d'indépendance et de renommée militaire, nullement de domination. Ils chassaient les voisins dangereux ; Rome les désarmait et les asservissait ; qui n'eût préféré être voisin des Germains ? D'autant plus que dans ces forêts cultivées de place en place, la migration n'était une tragédie que par accident.\par
On voit très bien, dans Tacite, d'où venait aux Germains leur goût de la guerre ; il venait de ce qu'ils méprisaient le travail. Ce trait les apparente aux Espagnols, non aux Allemands, dont l'aptitude à la guerre industrielle de notre époque vient de leur amour pour le travail obstiné, méthodique et conscien­cieux. « Leurs corps, dit Tacite, sont propres seulement à l'assaut, mais ne supportent pas le travail et la souffrance... On ne leur persuade pas si facile­ment de labourer la terre ou d'attendre la récolte que d'appeler les ennemis au combat et de gagner des blessures ; il leur semble paresseux et lâche d'acquérir par la sueur ce qu'on peut obtenir par le sang. Quand ils ne font pas la guerre, ils ne consacrent guère de temps à la chasse, mais plutôt au repos, livrés au sommeil ou aux repas. Ceux qui sont forts et belliqueux ne font rien ; les soins de la maison, du foyer, des champs, sont laissés aux femmes, aux vieillards et aux faibles. Eux restent dans la torpeur. Étrange diversité de nature, qui fait que les mêmes hommes ont tant d'amour pour l'inaction et de haine pour la tranquillité. »\par
Dans leur goût des combats, certains se souciaient pourtant de la justice, s'il faut croire Tacite ; témoins les Chauques, qui étaient, il est vrai, alliés des Romains. « Ce territoire immense, les Chauques ne l'occupent pas seulement, ils le remplissent ; peuple illustre entre les Germains, et qui choisit de défen­dre sa grandeur par la justice. Dépourvus d'avidité et de cruauté, tranquilles et se tenant à part, ils ne provoquent aucune guerre, ils ne dévastent rien par les pillages ou les larcins. Le plus grand témoignage de leur vaillance et de leur puissance est que, quand ils sont les plus forts, ils n'en profitent pas pour faire des injustices. »\par
Ce qui frappe le plus chez les Germains de Tacite, ce qui les met le plus loin et des sujets ou citoyens romains de leur temps et des Allemands de 1939, c'est qu'ils étaient libres. « Les rois mêmes n'ont pas une puissance absolue ou arbitraire ; quant aux chefs, c'est l'exemple plutôt que le commandement, s'ils sont résolus, s'ils se distinguent, s'ils combattent devant les lignes, c'est l'admiration qui fait leur autorité. Au reste, il n'est licite à personne de sévir, de mettre en prison ou même de frapper, sinon aux prêtres ; et ils ne le font pas en guise de châtiment, ni sur l'ordre du chef, mais comme par le comman­dement du dieu qu'ils croient présent aux batailles... Les princes délibèrent des affaires de faible importance, le peuple tout entier des grandes affaires, mais de telle manière que les affaires mêmes qui dépendent du peuple sont traitées par les princes... Un roi, un prince, selon l'âge, la noblesse, la réputation militaire, l'éloquence de chacun sont entendus ; ils ont de l'autorité pour persuader plutôt qu'un pouvoir de commandement. » Il n'y a pas d'impôts prélevés par contrainte. « Les peuples ont coutume d'apporter d'eux-mêmes aux princes, chaque homme pour son compte, du bétail ou du blé ; dons reçus comme un honneur et qui subviennent en même temps aux besoins. » Les esclaves mêmes sont presque libres. « Ils n'en usent pas à notre mode, pour des services déterminés dans la maison. Chaque esclave est souverain dans son domaine et son foyer. Le maître exige de lui, comme d'un colon, une cer­taine quantité de froment, de bétail ou de vêtements ; c'est dans cette mesure que l'esclave obéit... Frapper un esclave, le punir par des chaînes ou des travaux est chose rare ; il arrive qu'on en tue, non pour la discipline et comme châtiment, mais dans l'emportement de la colère, comme on tue un ennemi particulier, sauf qu'on le peut impunément... Le maître et l'esclave ne se distinguent par aucun raffinement dans l'éducation ; ils passent leur enfance au milieu du même bétail, sur le même sol, jusqu'à ce que l'âge sépare ceux qui sont libres et que le courage les fasse reconnaître. »\par
Il serait trop long de reproduire les vives louanges que fait Tacite des mœurs des Germains, de leur chasteté, de leur hospitalité, de leur générosité. « Écarter de son toit n'importe quel mortel est regardé comme un crime... Nul ne fait de différence entre les gens qu'on connaît et ceux qu'on ne connaît pas, quant au droit de l'hospitalité. Si l'hôte qui part demande quelque chose, la coutume est de le lui donner ; on lui demande avec la même facilité. Ils aiment les présents, mais sans demander de reconnaissance pour ceux qu'ils donnent, sans en accorder pour ceux qu'ils reçoivent. »\par
Le trait le plus singulier, trait qui empêche tout à fait de reconnaître en ces Germains les ancêtres d'Hitler, mais qu'on a généralement attribué aux Allemands jusqu'en 1870, c'est la simplicité d'âme, l'absence de ruse. « Pour réconcilier des ennemis, contracter des alliances, choisir des chefs, enfin pour traiter de la paix ou de la guerre, ils délibèrent le plus souvent au milieu des banquets, parce que nulle autre occasion n'ouvre mieux l'âme à des pensées sincères et ne l'échauffe mieux pour les grandes pensées. Peuple sans ruse, sans artifice, la licence du divertissement dévoile encore mieux les secrets de leurs cœurs. Le lendemain, on traite les mêmes affaires ; ainsi l'opportunité de chaque moment est préservée ; ils délibèrent quand ils sont incapables de feindre, ils décident quand ils ne sont pas susceptibles de se tromper. »\par
César, il est vrai, accuse parfois les Germains de perfidie ; mais son propre témoignage, même en admettant qu'il ait tout dit, montre assez de quel côté était la perfidie. Un peuple germain chassé de son territoire ayant traversé le Rhin en quête d'une terre nouvelle, César marcha contre lui. Des députés vinrent lui dire qu'ils s'établiraient où il voudrait ; il leur indiqua un peuple germain, ami des Romains, susceptible de les recevoir. Il ne voulait pas leur permettre de s'établir en Gaule, parce que, comme il le reconnaît lui-même, les Gaulois, trouvant la tyrannie des Germains préférable à celle des Romains, n'auraient été que trop disposés à les recevoir et à en appeler d'autres. Les députés des Germains demandèrent vainement à César de s'arrêter pendant que leur peuple délibérerait. Il était arrivé tout près quand les envoyés ger­mains le supplièrent encore de s'arrêter, ou du moins d'accorder trois jours de trêve, pendant lesquels ils pourraient vérifier que le peuple désigné par César consentait effectivement à les recevoir ; ils s'engageaient alors à y aller. César accorda la trêve, quoique à contrecœur ; car, s'il faut l'en croire, il pensait, à tort ou à raison, qu'ils la demandaient seulement pour donner à une partie de leur cavalerie, à ce moment de l'autre côté de la Meuse, le temps de revenir. Il leur dit de venir le trouver le lendemain aussi nombreux que possible. Quand la cavalerie romaine, que César laissa poursuivre son avance, arriva aux yeux des cavaliers germains, ceux-ci se jetèrent sur elle et, au nombre de huit cents, mirent en fuite les cinq mille cavaliers de César.\par
Il est évident, puisque les Germains avaient tant imploré la trêve et en avaient tant besoin, que c'était là un accident, dû ou à un retard dans la trans­mission des ordres, car les envoyés germains n'avaient quitté César que peu de temps auparavant, ou à l'indiscipline des Germains et à l'approche provoca­trice de la cavalerie romaine. César décida de ne pas observer la trêve ; mais il attendit le lendemain et il reçut la députation convenue, qui était formée de tous les chefs et anciens venus s'excuser de l'incident. César les mit sous bonne garde et marcha contre les Germains, qu'il surprit privés de leurs chefs et ne s'attendant à rien de tel. Les femmes et les enfants se mirent à fuir ; César envoya sa cavalerie, vaincue la veille, les massacrer ; la vue de ce mas­sacre affola les Germains. Finalement ils périrent tous, tous jusqu'au dernier, hommes, femmes et enfants, soit sous le fer, soit dans le Rhin, au nombre de quatre cent trente mille. Les Romains n'eurent pas un mort et peu de blessés. Le Sénat décerna à César beaucoup d'honneurs pour cet exploit, mais Caton demanda qu'en expiation de la foi violée on le livrât enchaîné aux Germains, conformément à une ancienne coutume. César profita de ce massacre pour aller porter la terreur de l'autre côté du Rhin, usant de cet art propre aux Romains, et retrouvé de nos jours, de tout soumettre par la terreur et le prestige plus que par la force effective. Si quelqu'un fait penser à Hitler, par la barbarie, la perfidie préméditée, l'art de la provocation, l'efficacité de la ruse, c'est bien César ; ces malheureux Germains ressemblent seulement, par leur incohérence naïve, à tous les peuples ignorants de la discipline, de l'organi­sation et de la méthode.\par
Il n'y a rien, exactement rien, dans les seuls textes par lesquels nous connaissions les Germains d'il y a deux mille ans, qui puisse faire croire à un esprit mauvais ou dangereux qui subsisterait dans la race germanique à travers les siècles. À quoi aura-t-on recours pour défendre encore cette thèse ? Cherchera-t-on des exemples de cruautés et de dévastations commises pendant la période dite des grandes invasions ? On pourrait en trouver sans doute, mais on rencontrerait aussi une des plus pures figures de l'histoire dans la personne du premier roi Goth qui gouverna l'Italie, Théodoric. Bien qu'on se soit demandé si les Goths étaient d'origine purement germanique, on n'a jamais hésité à les ranger au nombre des populations teutonnes, leur langue étant la forme la plus antique de l'allemand connue aujourd'hui. On ne pourrait pas citer de souverain légitime qui ait régné d'une manière plus juste et plus hu­maine sur son propre peuple que Théodoric sur le pays dont il s'était emparé ; il n'y porta atteinte ni à la langue, ni aux institutions, ni à la religion, ni aux biens, ni à quoi que ce fût. Ce ne sont pas seulement les écrits de ses serviteurs qui en témoignent, mais aussi ceux de Procope, serviteur du général qui détruisit peu de temps après les Goths d'Italie ; il raconte que Théodoric ne commit au cours de son règne qu'un seul acte de violence injuste, et qu'il en mourut de chagrin presque aussitôt.\par
Si l'on passe au moyen âge, où trouvera-t-on des marques de l'esprit mauvais qu'on dénonce ? Serait-ce dans l'institution des villes libres ? Les libertés communales fleurissaient en Allemagne au moment où elles étaient écrasées en France et en Flandre et commençaient à périr en Italie. Serait-ce dans l'institution du Saint Empire Romain Germanique, ce lien fédéral entre villes et principautés indépendantes dont on ne saurait trop regretter la disparition ? Qui pourrait croire que cet Empire ait été violent, dangereux, animé d'impulsions de domination et d'expansion, alors qu'il ne put même pas conquérir une Italie alors faible, divisée et insoucieuse des armes, et cela bien qu'il fût appuyé par les vœux sincères d'un certain nombre d'Italiens dont Dante est le plus illustre ? C'est seulement quand la dynastie impériale fut devenue espagnole que l'Italie tomba sous le joug de l'étranger. Dans la période qui suit, on pourrait citer les horreurs des guerres religieuses ; mais elles ne dépassèrent pas celles des guerres de religion en France. Au temps de Louis XIV, l'Empire se trouva être du côté de la liberté, dans la coalition glorieusement dirigée par Marlborough. Les rois de Prusse firent apparaître en Allemagne quelque chose d'entièrement nouveau dont nous maudissons aujourd'hui les effets. Mais ignore-t-on que Frédéric II de Prusse dut sa gloire, non pas seulement à ses talents d'organisation, d'administration et de domi­nation, à ses succès militaires, à ses violences sans scrupules, mais aussi et dans une mesure au moins égale aux louanges démesurées de tous les écrivains français les plus illustres de l'époque ?\par

\subsubsection[II. - Hitler et la politique extérieure de la Rome antique]{II. - Hitler et la politique extérieure de la Rome antique}
\noindent \par
L'analogie entre le système hitlérien et l'ancienne Rome est si frappante qu'on pourrait croire que seul depuis deux mille ans Hitler a su copier correctement les Romains. Si elle n'est pas tout d'abord évidente à nos yeux, c'est que nous avons presque appris à lire dans Corneille et dans le {\itshape De Viris} ; nous sommes habitués\par
24\par
à nous mettre à la place des Romains; même quand c'est la Gaule dont ils s'emparent ; aujourd'hui que nous sommes pris dans une situation analogue, mais où c'est notre ennemi qui joue le rôle de Rome, nous ne reconnaissons pas l'analogie. Car ce sont les conquêtes qu'on est menacé de subir qui font horreur ; celles qu'on accomplit sont toujours bonnes et belles. De, plus, nous ne connaissons l'histoire romaine que par les Romains eux-mêmes et par leurs sujets grecs, contraints, les malheureux, à flatter leurs maîtres ; il faut ainsi un effort de critique continuel pour apprécier équitablement la politique de Rome. Nous ne possédons pas la version qu'auraient pu en donner les Cartha­ginois, les Espagnols, les Gaulois, les Germains, les Bretons.\par
Les Romains ont conquis le monde par le sérieux, la discipline, l'organisa­tion, la continuité des vues et de la méthode ; par la conviction qu'ils étaient une race supérieure et née pour commander ; par l'emploi médité, calculé, méthodique de la plus impitoyable cruauté, de la perfidie froide, de la propagande la plus hypocrite, employées simultanément ou tour à tour ; par une résolution inébranlable de toujours tout sacrifier au prestige, sans être jamais sensibles ni au péril, ni à la pitié, ni à aucun respect humain ; par l'art de décomposer sous la terreur l'âme même de leurs adversaires, ou de les endormir par l'espérance, avant de les asservir avec les armes ; enfin par un maniement si habile du plus grossier mensonge qu'ils ont trompé même la postérité et nous trompent encore. Qui ne reconnaîtrait ces traits ?\par
Les Romains ont su manier à leur gré les sentiments des hommes. C'est ainsi qu'on devient maître du monde. Tout pouvoir qui s'accroît suscite autour de lui des sentiments divers ; si, par science ou bonne fortune, il suscite ceux qui lui donnent le moyen de s'accroître encore, il ira loin. Les peuples et les hommes placés autour du territoire soumis à Rome ont éprouvé, comme tous les mortels, tour à tour la crainte, la terreur, la colère, l'indignation, l'espé­rance, la tranquillité, la torpeur ; mais ce qu'ils éprouvaient à chaque moment était précisément ce qui était utile à Rome, et cela par l'art des Romains. Il faut pour un tel art une espèce de génie, mais aussi une brutalité sans limites et qui n'a d'égard à rien.\par
On ne peut dépasser les Romains dans l'art d'être perfide. La perfidie a deux inconvénients : elle suscite l'indignation et empêche qu'on ne soit cru par la suite. Les Romains ont su éviter l'un et l'autre, parce qu'ils étaient perfides seulement quand ils pouvaient à ce prix anéantir leurs victimes. Ainsi aucune d'elles n'était en état de leur reprocher leur mauvaise foi. D'autre part, les spectateurs étaient frappés de terreur ; comme la terreur rend l'âme crédule, la perfidie même des Romains avait pour effet d'augmenter au lieu de diminuer autour d'eux l'inclination à les croire ; on croit volontiers ce qu'on désire vivement être vrai. En même temps les Romains louaient leur propre bonne foi avec une conviction contagieuse, et mettaient un soin extrême à sembler se défendre et non attaquer, à paraître respecter les traités et les conventions, sauf lorsqu'ils pouvaient frapper impunément, et parfois même en ce cas. Une de leurs coutumes était, quand un traité conclu par un de leurs consuls leur semblait trop modéré, de recommencer la guerre et de livrer ce consul nu et enchaîné aux ennemis en expiation du traité rompu ; ceux-ci, qui ignoraient cet usage et croyaient la paix établie, ne trouvaient dans ce corps nu qu'une faible consolation. Les exemples de perfidie et de foi violée sont si nombreux dans l'histoire romaine qu'ils seraient trop longs à tous citer ; ils ont tous un caractère commun, c'est qu'ils sont calculés et prémédités. C'est, ainsi que les Romains purent se donner une réputation de loyauté. Retz dit que lorsqu'on a froidement résolu de faire le mal, on peut garder les apparences, au lieu que si on ne veut pas le faire, et si néanmoins on s'y laisse aller, on provoque toujours un scandale. Les Romains ont appliqué ce principe à la violation de la parole donnée ; au lieu que les autres peuples qui, comme les Carthaginois, manquèrent à leur parole par besoin, par fureur, par désespoir, eurent une réputation de perfidie même auprès de la postérité, qui n'écoute jamais les vaincus.\par
Le plus bel exemple est celui de Paul-Émile, quand il fit en une seule heure le sac de soixante-dix villes et emmena leurs habitants comme esclaves, après avoir promis à ces villes le salut et avoir introduit dans chacune un détachement armé à la faveur de cette promesse. Le Sénat était l'auteur de cette ruse. L'histoire des guerres espagnoles, dans Appien, est pleine de géné­raux romains qui violent leur parole, massacrent des peuples entiers après les avoir désarmés en leur promettant la liberté et la vie, attaquent par surprise après avoir conclu la paix. Mais le plus affreux effet de la perfidie romaine fut le malheur de Carthage, où une civilisation qui, grâce à l'influence de l'Orient et de la Grèce, était sans doute pour le moins aussi brillante que la civilisation latine, fut anéantie pour toujours et sans laisser aucune trace.\par
Carthage avait eu d'abord la bonne fortune, si l'on peut dire, d'être vaincue par un des très rares Romains qui aient donné des marques de modération, à savoir le premier Africain. Elle put ainsi survivre à la perte de toute sa puis­sance, mais dut contracter une alliance avec Rome et promettre de ne jamais engager la guerre sans sa permission. Au cours du demi-siècle qui suivit, les Numides ne cessèrent pas d'envahir et de piller le territoire de Carthage, qui n'osait se défendre ; pendant la même période de temps elle aida les Romains dans trois guerres. Les envoyés carthaginois, prosternés sur le sol de la Curie, tenant des rameaux de suppliants, imploraient avec des larmes la protection de Rome, à laquelle le traité leur donnait droit ; le Sénat se gardait bien de la leur accorder. Enfin, poussée à bout par une incursion numide plus menaçante que les autres, Carthage prit les armes, fut vaincue, vit son armée entièrement détruite. Ce fut le moment que Rome choisit pour déclarer la guerre, alléguant que les Carthaginois avaient combattu sans sa permission. Des délégués de Carthage vinrent à Rome implorer la paix, et, ne pouvant l'obtenir autrement, se remirent à la discrétion du Sénat. Celui-ci accorda aux Carthaginois la liberté, leurs lois, leur territoire, la jouissance de tous leurs biens privés et publics, à condition pour eux de livrer en otages trois cents enfants nobles dans un délai d'un mois et d'obéir aux consuls. Les enfants furent livrés aussi­tôt. Les consuls arrivèrent devant Carthage avec flotte de guerre et armée, et ordonnèrent qu'on leur remît toutes les armes et tous les instruments de guerre sans exception. L'ordre fut exécuté immédiatement. Les sénateurs, les anciens et les prêtres de Carthage vinrent alors se présenter aux consuls devant l'armée romaine. La scène qui suit, telle que la raconte Appien, est d'un tragique digne de Shakespeare, et rappelle en bien plus atroce ce qu'on a dit de la nuit passée par Hacha chez Hitler.\par
Un des consuls annonça aux Carthaginois présents devant lui que tous leurs concitoyens devaient quitter la proximité de la mer et abandonner la ville, et que celle-ci serait complètement rasée. « Il parlait encore qu'ils levè­rent les mains vers le ciel avec des cris, attestant les dieux qu'on les avait trompés ; ils se répandirent en injures et en imprécations contre les Romains... Ils se jetèrent à terre, frappèrent le sol de leurs mains et de leurs fronts ; certains déchiraient leurs vêtements, se meurtrissaient le corps, comme pour punir la déraison qui les avait fait tomber dans ce piège. Quand leur fureur eut pris fin, il y eut un grand et morne silence, comme s'ils avaient été des cada­vres gisants... Les consuls... savaient bien que les malheurs extrêmes poussent tout d'abord à la hardiesse, mais qu'avec le temps l'audace est subjuguée par la nécessité. C'est ce qui arriva aux Carthaginois. Car dans le silence, leur mal­heur les touchant plus vivement, ils cessèrent de s'indigner ; ils se mirent à pleurer sur eux-mêmes, sur leurs enfants et leurs femmes qu'ils appelaient par leurs noms, sur leur patrie à qui, comme si elle eût été un être humain capable de les entendre, ils disaient toutes sortes de choses pitoyables... Les consuls, bien que pris de pitié devant les vicissitudes des choses humaines, restaient impassibles, attendant le moment où les Carthaginois en auraient assez de pleurer. Quand les plaintes eurent pris fin, il y eut de nouveau un silence. Ils se rendirent compte alors que leur ville était sans armes, abandonnée, sans un vaisseau, sans une catapulte, sans un javelot, sans une épée, sans citoyens capables de combattre, après qu'il en était péri cinquante mille, sans merce­naires, sans un ami, sans un allié... Ils renoncèrent alors au tumulte et à l'indignation, comme inutiles dans le malheur, et eurent recours aux paroles. »\par
Suit un discours où l'orateur invoque le traité conclu avec Scipion et les promesses récentes et formelles du Sénat. « Il n'y a rien qui ait plus de pouvoir dans les supplications que l'invocation des traités ; et nous n'avons aucun autre refuge que les paroles, puisque tout ce que nous possédions de puissance, nous vous l'avons livré... Si pourtant vous ne supportez même pas ces arguments, nous allons laisser de côté tout cela, et, seul droit qui reste aux malheureux, pleurer et supplier... Nous vous donnons un autre choix, bien préférable pour nous, bien plus honorable pour vous ; notre ville, laissez-la intacte, elle qui ne vous a fait aucun mal ; et nous à qui vous ordonnez de la quitter, si vous voulez, faites-nous tous mourir... Ne souillez pas votre réputation par un acte dont l'exécution sera atroce, le récit atroce, et que vous seriez les premiers à avoir commis dans toute l'histoire ; car les Grecs et les Barbares ont fait bien des guerres, et vous en avez fait beaucoup, Romains, aux autres peuples ; mais jamais personne n'a anéanti une ville qui avait tendu les mains avant le combat, livré ses enfants et ses armes, et qui, s'il existe encore quelque autre châtiment parmi les hommes, consentait aussi à le souffrir. »\par
\par
Les consuls refusèrent jusqu'à l'autorisation d'aller une fois encore supplier le Sénat, et ils expliquèrent que l'ordre de raser la ville avait été donné dans l'intérêt des Carthaginois eux-mêmes. Ce genre de raffinement dans l'injure, tout à fait ignoré des Grecs, n'a peut-être été retrouvé pleinement depuis qu'après 1933 Le peuple de Carthage fut pris de désespoir, et l'armée romaine, trop confiante en ses forces, livrée au désordre et aux plaisirs, dut faire un siège de trois ans avant de pouvoir détruire la ville et les habitants sous les ordres du second Africain. On peut voir dans Polybe, ce que pensèrent les Grecs de cette agression, à commencer par Polybe lui-même, malgré la réti­cence imposée à un sujet de Rome et à un humble client des Scipion.\par
La cruauté la plus horrible apparaît dans cette histoire autant que la perfidie, et s'y combine. Nul n'a jamais égalé les Romains dans l'habile usage de la cruauté. Quand la cruauté est l'effet d'un caprice, d'une sensibilité mala­de, d'une colère, d'une haine, elle a souvent des conséquences fatales à qui y cède ; la cruauté froide, calculée et qui constitue une méthode, la cruauté qu'aucune instabilité d'humeur, aucune considération de prudence, de respect ou de pitié ne peut tempérer, à laquelle on ne peut espérer échapper ni par le courage, la dignité et l'énergie, ni par la soumission, les supplications et les larmes, une telle cruauté est un instrument incomparable de domination. Car étant aveugle et sourde comme les forces de la nature, et pourtant clairvoyante et prévoyante comme l'intelligence humaine, par cet alliage monstrueux elle paralyse les esprits sous le sentiment d'une fatalité. On y résiste avec fureur, avec désespoir, avec le pressentiment du malheur, ou on y cède lâchement, ou on fait l'un et l'autre tour à tour ; de toutes manières l'esprit est aveuglé, incapable de calcul, de sang-froid et de prévision. Cet aveuglement apparaît chez tous les adversaires des Romains. De plus une cruauté de cette espèce fait naître des sentiments qui sembleraient n'être dus qu'à la clémence. Elle suscite la confiance, comme on le voit par l'histoire de Carthage, dans toutes les circonstances où il serait trop affreux de se défier ; car l'âme humaine répugne à regarder en face l'extrême malheur. Elle suscite la reconnaissance chez tous ceux qui auraient pu être anéantis et qui ne l'ont pas été ; car ils s'attendaient à l'être. Quant à ceux qui ont été anéantis, c'est-à-dire tués ou vendus comme esclaves, leur sentiment n'importe pas, puisqu'ils se taisent.\par
On trouve dans Polybe, à propos de la prise de Carthagène par le premier Africain, un exemple de cette cruauté romaine, qui, comme le texte l'implique, allait plus loin que la cruauté habituelle à cette époque. « Publius envoya la plupart des soldats contre les habitants de la ville, conformément à la coutume romaine, avec ordre de tuer tout ce qu'ils rencontreraient, sans épargner per­sonne ni se jeter sur le butin jusqu'à ce qu'il donnât le signal convenu. À mon avis, ils agissent ainsi pour frapper de terreur ; c'est pourquoi on peut voir souvent aussi, dans les villes prises par les Romains, non seulement des êtres humains massacrés, mais même des chiens coupés en quartiers, et des membres épars d'animaux. Ce jour-là, le massacre fut extrêmement considé­rable, à cause de la multitude de ceux qui tombèrent aux mains des soldats. » Après la prise de la citadelle, Scipion fit réunir tous les survivants du massacre ordonné par lui ; chacun d'eux sans doute avait perdu des êtres chers. « Il mit à part les citoyens, hommes et femmes, ainsi que leurs enfants ; puis, les ayant exhortés à être reconnaissants aux Romains et à se souvenir du bienfait reçu, il les renvoya tous dans leurs maisons. Eux, versant des larmes et joyeux d'un salut inespéré, remercièrent le général à genoux et s'en allèrent... Ainsi il les rendit extrêmement dévoués et fidèles à lui-même et à Rome. » Cette ville n'avait commis aucune faute à l'égard des Romains, sinon qu'étant soumise à une garnison carthaginoise, elle se trouvait automatiquement par rapport à eux dans le camp ennemi.\par
La destruction de Carthage montre assez que l'extrême soumission ne pouvait garantir de la cruauté des Romains. Les populations libres des nations alliées servaient à Rome de réservoir d'esclaves non moins que celles des pays ennemis. La Bithynie, pays de culture grecque, allié de Rome, dont le roi Prusias s'était déclaré l'affranchi du peuple romain et avait baisé à genoux le seuil de la Curie, fut incapable sous le règne suivant d'aider militairement Marius, parce que la plupart des Bithyniens avaient été saisis par les publi­cains et servaient comme esclaves dans les provinces romaines. Apprenant cela, le Sénat, anxieux de conserver des alliés militairement utilisables, décida la libération de tous les hommes de naissance libre, sujets de nations alliées, qui se trouveraient comme esclaves dans les provinces. Un préteur de Sicile se mit à appliquer ce sénatus-consulte, et en quelques jours, dans cette seule petite île, huit cents de ces malheureux furent ainsi libérés ; mais les proprié­taires d'esclaves firent aussitôt arrêter la procédure. Tout cela nous est raconté par Diodore de Sicile.\par
D'après Appien, en Espagne, pendant le siège de Numance, la jeunesse d'une ville voisine fut prise un jour du désir d'aider cette malheureuse cité ; les vieillards, qui craignaient la colère de Rome et espéraient l'apaiser par une prompte soumission, dévoilèrent d'eux-mêmes ce projet aux Romains ; ceux-ci, investissant la ville par surprise et menaçant d'en faire immédiatement le sac, se firent livrer quatre cents jeunes gens nobles et leur coupèrent les mains à tous.\par
Après la défaite de Philippe, roi de Macédoine, les Romains eux-mêmes avaient solennellement proclamé la liberté de la Grèce au milieu de l'enthou­siasme délirant des Grecs. Rome n'en intervint pas moins à partir de ce moment dans les affaires intérieures des cités grecques, et cela non pas par des conseils, mais par des ordres. Les cités réunies dans la ligue achéenne dépu­tèrent à Rome pour expliquer respectueusement que ces ordres étaient contrai­res aux lois, aux serments, aux conventions publiques. Mais un membre de la députation, nommé Callicrate, précurseur de Seyss Inquart, se mit à exciter le Sénat à l'énergie, disant que ceux qui voulaient persuader de mettre l'obéis­sance à Rome au-dessus des lois étaient mal accueillis par le peuple dans toutes les cités grecques, et l'emporteraient seulement s'ils étaient appuyés par la force romaine. Le Sénat fit savoir en Grèce que les cités devaient être gouvernées par des hommes tels que Callicrate ; la ligue achéenne, apeurée, prit aussitôt Callicrate pour chef.\par
Dans la période qui suivit, les deux hommes qui exerçaient alors la plus grande autorité morale en Grèce, Aristaenos et Philopoemen, furent en désac­cord seulement pour savoir s'il fallait exécuter n'importe quel ordre donné par les Romains, sans jamais même oser une remontrance, ou si l'on pouvait parfois se permettre de protester, avec la résolution d'obéir si la protestation restait vaine. Aristaenos disait que, si l'on ne peut pas faire ce qui est honorable, il faut se résigner à faire ce qui est expédient ; et que si l'on ne peut se montrer capable de ne pas obéir, il n faut même pas oser en parler. Philopoemen répondait qu'alors la condition des Achéens ne différerait pas de celle des gens de Sicile et de Capoue, dont la servitude était ouvertement reconnue ; qu'il savait bien qu'à un moment donné les Grecs seraient con­traints d'exécuter tout ce que Rome leur ordonnerait, mais qu'il s'agissait de savoir s'il fallait hâter ou retarder ce moment.\par
Lors de la guerre contre Persée, l'intérêt de la Grèce à ce que la Macédoine ne fût pas vaincue était évident, et Polybe laisse bien entendre, à travers des réticences d'esclave, qu'à peu près tous les Grecs le sentaient. Mais presque aucune cité n'osa agir à cet effet. Rhodes offrit sa flotte à Rome et refusa d'aider Persée, même au cas d'une agression non provoquée des Romains contre lui ; mais, une fois la guerre engagée, elle fit des tentatives d médiation auprès des deux adversaires. Cela suffit pour qu'après la victoire sur Persée le Sénat, accusant Rhodes de ne pas avoir souhaité cette victoire, délibérât pour savoir s'il fallait lui déclarer la guerre. Les délégués de Rhodes, voyant leurs protestations inutiles, s'abaissèrent au dernier degré de la supplication et de la soumission ; Tite-Live leur fait dire que, si Rome leur déclarait la guerre, ils abandonneraient tous leurs biens intacts et viendraient tous à Rome, hommes, femmes et enfants, livrer leurs corps à l'esclavage. En fin de compte, grâce à un discours de Caton, ils s'en tirèrent avec un châtiment moindre.\par
Les Achéens avaient offert une armée à Rome et envoyé à cet effet une députation dont Polybe fit partie ; néanmoins, une fois la guerre finie, le Sénat fit venir plusieurs milliers de citoyens grecs réputés, parmi lesquels Polybe lui-même ; leur crime était de n'avoir désiré la victoire ni de Rome ni de la Macédoine et d'avoir attendu passivement l'issue de la lutte. Le Sénat refusa également de les laisser repartir et de les juger, preuve qu'aucune accusation ne pouvait être portée contre eux ; ils furent dispersés à travers l'Italie, sauf Polybe qui fut recueilli à Rome par les Scipion. Les cités grecques ne se lassèrent pas d'envoyer en leur faveur des députations qui venaient avec des rameaux de suppliants ; mais ce fut en vain. La Grèce resta livrée à Callicrate et à ses amis, tellement haïs que les enfants les traitaient de traîtres en pleine rue. La plupart des malheureux astreints à la résidence forcée en Italie y moururent ; les autres ne furent libérés qu'après quinze ans et ne recouvrèrent pas leurs droits civiques.\par
Quand Rome déclara la guerre à Carthage, les Grecs tentèrent de secouer le joug des traîtres vendus à Rome ; sans doute ils étaient poussés par l'indi­gnation devant une agression si cruelle, et aussi par l'espoir que Rome, occupée en Afrique, ne pourrait exercer aucune répression contre eux. Comme il est naturel, les chefs qui s'imposèrent alors ne se trouvèrent pas à la hauteur d'une pareille tâche. Rome se contenta d'agir par voie diplomatique jusqu'à la destruction de Carthage. La terreur où tomba à ce moment la Grèce fait invinciblement penser, mais en bien plus atroce, aux pays sur lesquels s'étend l'ombre de la domination hitlérienne. « Les uns s'enlevaient eux-mêmes la vie sans raison ; d'autres fuyaient des villes par des endroits impraticables, sans savoir le moins du monde où aller, frappés de terreur par ce qui se passait dans les cités. Certains se dénonçaient les uns les autres à leurs ennemis comme ayant été hostiles aux Romains ; d'autres livraient et accusaient leurs proches, alors que personne ne leur demandait encore aucun compte ; d'autres allaient avec des rameaux de suppliants avouer qu'ils avaient agi contre les traités et demander quel serait leur châtiment, bien que personne ne se fût informé de leurs actions. Tout était plein d'une extraordinaire puanteur, tant il s'était jeté de gens dans les précipices et les puits de sorte que même un ennemi, comme on dit, aurait eu pitié de l'état où était tombée la Grèce... Les Thébains quittèrent tous leur ville et la laissèrent absolument déserte. » Corinthe fut alors anéantie ; Polybe en vit détruire sous ses yeux les merveilles. La Grèce fut réduite au régime colonial, et les Grecs tombèrent dans un avilissement dont témoignent les écrivains latins de l'époque impériale. Le génie grec, qui, malgré la décadence, fleurissait encore au III\textsuperscript{e} siècle dans tous les domaines, périt alors sans retour, exception faite pour les traces qui en demeurèrent en Syrie et en Palestine. Quant à Rome, elle ne fit qu'en corrompre la pureté par une imitation servile qui le dérobe encore aujourd'hui à nos yeux ; elle en retint à peine quelques éclairs de poésie. En ce qui concerne les autres arts, la philosophie, la science, on peut presque considérer que la civilisation antique s'est éteinte avec la Grèce.\par
Le courage, la fierté, une énergie inlassable, une résolution héroïque, ne préservaient pas de la cruauté des Romains mieux que la soumission. Les Espagnols en firent l'épreuve, et surtout la malheureuse Numance, cité où des fouilles pratiquées récemment ont montré que l'influence grecque et les arts avaient pénétré. Viriathe, échappé par bonne fortune au massacre de tout son peuple, qui s'était rendu après que les Romains lui eurent promis la vie, la liberté et des terres, incita plusieurs peuples espagnols, parmi lesquels les Numantins, à devenir ennemis de Rome. Trois ans plus tard, ayant acculé l'armée romaine dans un lieu où elle ne pouvait que périr, il offrit généreuse­ment la paix, fut déclaré ami du peuple romain et reconnu possesseur des terres qu'il occupait. Mais le Sénat donna secrètement l'ordre de le harceler par tous les moyens, puis, presque aussitôt, lui déclara la guerre au mépris du traité. Viriathe, pris par surprise, dut fuir avec les siens et périt peu après par trahison.\par
Numance éprouva un sort semblable. Rome, maîtresse du monde, qui avait détruit Carthage et complètement subjugué la Grèce et la Macédoine, dut envoyer, à cette ville défendue par huit mille guerriers des armées de quarante mille, puis soixante mille soldats, et mit dix ans à la réduire. La quatrième année du siège, le général de l'armée romaine proposa la paix, s'engageant à obtenir du Sénat des conditions favorables si Numance se rendait à discrétion. Le Sénat ne voulut pas accorder ces conditions et la guerre continua. L'année suivante, les Romains se trouvèrent encerclés à la suite d'une panique. N'ayant aucun autre moyen d’éviter d'être tous massacrés, ils acceptèrent la paix avec Numance à des conditions d'égalité ; le général romain s'y lia par serment. D'après Plutarque, dont le récit concorde avec celui d'Appien, cet accord sauva la vie de vingt mille soldats romains. Le Sénat ne le ratifia pourtant pas, et livra le général nu et enchaîné aux Numantins, qui ne voulurent pas le rece­voir. Enfin Scipion, le destructeur de Carthage, fut envoyé contre Numance. À la tête de soixante mille hommes, il investit complètement la ville de fortifications pour la réduire par la famine ; il refusa constamment le combat que les Numantins ne se lassaient pas d'offrir. Il ne voulut pas non plus leur permettre de se rendre autrement qu'à discrétion. Enfin, contraints par la faim à l'anthropophagie, ils se soumirent. Beaucoup se tuèrent plutôt que de se livrer à Scipion. Les autres furent vendus comme esclaves, et la ville fut rasée. Ni les lois de la guerre, qui ont toujours prescrit de traiter une cité moins dure­ment si elle se rend que si elle est prise d'assaut, ni la vaillance et l'endurance héroïques dont ces hommes avaient fait preuve ne purent conseiller à Scipion un peu de clémence. D'après Appien, certains crurent qu'il avait agi ainsi dans la pensée que les grandes renommées se fondent sur les grandes catastrophes. Il avait raison, puisque les Romains l'ont honoré du surnom de {\itshape Numantinus} et que sa gloire est venue jusqu'à nous.\par
Toutes ces cruautés constituaient des moyens d'élever le prestige. Le premier principe de la politique romaine, à partir de la seconde victoire sur Carthage, et même auparavant, fut de maintenir le plus haut degré de prestige en toutes circonstances et à n'importe quel prix. Il est impossible d'aller autre­ment d'une certaine quantité de puissance à la domination universelle ; car un seul peuple ne peut pas en dominer beaucoup d'autres par les forces dont il dispose réellement. Ce principe, poussé jusqu'au bout, prescrit que nul ne puisse se croire en mesure d'exercer une pression quelconque sur la volonté du peuple qui prétend à la domination ; l'impuissance des attitudes énergiques, des armes, des traités, des services passés, de la soumission, des prières doit être tour à tour éprouvée. Voilà pourquoi les Romains s'épuisèrent en une guerre interminable contre une petite cité dont l'existence ne les menaçait en rien, dont la destruction ne pouvait leur être utile à rien ; mais ils ne pouvaient souffrir qu'elle restât libre. Voilà aussi pourquoi ils n'acceptèrent presque jamais de parler de paix sinon après une victoire écrasante. Les traités ne furent jamais pour eux un obstacle aux desseins politiques ; il ne fallait que trouver la meilleure manière de passer outre. Les services rendus, comme il arrive souvent auprès des souverains et des maîtres, n'amenaient généralement comme salaire que des humiliations, afin que nul ne se crût de droits sur Rome ; celui qui se révoltait contre un tel traitement se trouvait isolé des ennemis de Rome par les services mêmes qu'il lui avait précédemment rendus, et était contraint par la défaite à la soumission, sans avoir même le droit d'invoquer ces services pour obtenir des conditions favorables. Ceux qui commençaient par l'alliance et la soumission et s'y maintenaient, n'étant guère mieux traités que des ennemis subjugués par la force, éprouvaient tous les jours leur propre impuissance par le fait même qu'ils obéissaient à contre-cœur. Les prières sont le moyen d'action suprême là où il n'en existe aucun autre, et par elles les hommes tentent de faire fléchir la volonté même des dieux ; mais d'affreux malheurs, de pitoyables supplications amenaient rare­ment les Romains à la clémence. Ainsi tous se sentaient livrés absolument et sans recours à la volonté de Rome quelle qu'elle fût, et ils en arrivaient à voir dans cette volonté le destin.\par
Une nation ne peut puiser la force d'agir ainsi que dans la conviction qu'elle a été choisie de toute éternité pour être la maîtresse souveraine des autres. Beaucoup de peuples, même misérables, se bercent de mythes où ils sont les maîtres de tout ; mais les Assyriens peut-être exceptés, les Romains les premiers, autant qu'on puisse savoir, formèrent sérieusement l'idée d'un peuple destiné à une telle mission ; c'est même la seule idée originale qu'ils aient su former. La meilleure formule s'en trouve dans Virgile : « Toi, Romain, occupe-toi de régir souverainement les nations. » Un peuple étant le maître par nature, tous ceux qui ne veulent pas lui obéir sont des esclaves rebelles et doivent être regardés et traités comme tels ; c'est ainsi qu'il faut comprends le vers « Épargner ceux qui sont soumis et abattre les orgueil­leux. » On peut estimer que ce vers exprime réellement la politique de Rome, en ce sens qu'un maître épargne ses esclaves pour autant qu'il ne leur fait pas tout le mal qu'il pourrait leur faire ; car ils n'ont aucun droit. Ils sont coupables d'orgueil quand ils croient qu'ils en ont. Car il ressort du vers de Virgile qu'aucun intermédiaire n'est admis entre la soumission et l'orgueil.\par
Le « Vae victis » des Gaulois voulait dire seulement que la défaite est un malheur qui expose aux mauvais traitements ; mais pour les Romains un ennemi vaincu était un coupable à châtier. Les paroles, authentiques ou non, que Tite-Live met dans la bouche de Paul-Émile sont significatives à cet égard. Persée, roi de Macédoine, dont le père Philippe avait été humilié de propos délibéré par les Romains même après qu'ils en eurent reçu des services essentiels, avait suscité des inquiétudes par ses préparatifs militaires. Le Sénat donna audience aux accusations de ses ennemis et refusa de leur confronter les ambassadeurs macédoniens ; puis, quand ceux-ci eurent enfin été admis à lui présenter leur défense, sans faire aucune réponse, sans formuler aucune exigence, aucun ultimatum, il déclara immédiatement la guerre. Persée, après avoir remporté une victoire, offrit aux Romains les conditions de paix les plus favorables, mais en vain. Enfin, vaincu par Paul-Émile et fait captif avec sa femme et ses fils tout petits, il fut amené devant le vainqueur et tomba à ses genoux. Paul-Émile le releva, et d'après Tite-Live lui dit à peu près : « Explique-moi donc, Persée, pourquoi, après avoir éprouvé du temps de ton père combien les Romains sont de bons alliés et des ennemis terribles, tu as choisi de les avoir pour ennemis ? » Persée baissa la tête et pleura en silence. Après le triomphe il fut jeté nu, ainsi que ses enfants, dans une fosse remplie de condamnés à mort, où il serait mort de faim au milieu des ordures, comme plus tard Jugurtha, si au bout d'une semaine l'intervention d'un des Scipion ne l'en avait fait tirer ; il périt deux ans plus tard torturé par ses gardiens, un siècle et demi environ après les victoires d'Alexandre.\par
\par
Les Romains eurent presque toujours à l'égard des chefs vaincus ces manières de maître légitime qui punit la rébellion. La cérémonie du triomphe, cette institution horrible, propre à Rome, à laquelle Cicéron trouvait tant de douceur, contribuait à créer cette illusion. Il semblait toujours d'après ses actes et ses paroles que Rome punît ses ennemis non par intérêt ou par plaisir, mais par devoir. Elle arrivait ainsi, par contagion, à faire naître dans une certaine mesure chez ses adversaires mêmes le sentiment qu'ils étaient des rebelles, ce qui constituait un avantage sans prix ; car, comme Richelieu l'a remarqué par expérience, toutes choses égales d'ailleurs, des rebelles sont toujours de beaucoup les plus faibles. On trouve dans Hérodote une histoire selon laquelle des Scythes, combattant une troupe de bâtards issus de leurs femmes et de leurs esclaves, auraient soudain laissé leurs armes pour saisir leurs fouets et mis ainsi leurs adversaires en fuite ; tel est, dans la guerre et la politique, le pouvoir de l'opinion.\par
Le maître doit toujours avoir raison, et ceux qu'il punit toujours tort. Une habileté considérable est nécessaire à cet effet. Deux opinions sur la force et le droit, l'une et l'autre erronée, l'une et l'autre fatale à qui s'y fie, abusent les esprits médiocres ; les uns croient que la cause juste continue toujours à apparaître comme juste même après avoir été vaincue, les autres que la force toute seule suffit pour avoir raison. En réalité, la brutalité muette a presque toujours tort si la victime invoque son droit, et la force a besoin de se couvrir de prétextes plausibles ; en revanche des prétextes entachés de contradiction et de mensonge sont néanmoins assez plausibles quand ils sont ceux du plus fort. Quand même ils seraient trop grossiers, trop transparents pour tromper per­sonne, ce serait une erreur de croire qu'ils sont de ce fait inutiles ; ils suffisent pour fournir une excuse aux adulations des lâches, au silence et à la soumis­sion des malheureux, à l'inertie des spectateurs, et permettre au vainqueur d'oublier qu'il commet des crimes ; mais rien de tout cela ne se produirait en l'absence de tout prétexte, et le vainqueur risquerait d'aller alors à sa perte. Le loup de la fable le savait ; l'Allemagne l'a oublié en 1914 et a payé cher cet oubli, au lieu qu'elle le sait maintenant ; les Romains le savaient fort bien. C'est pour cela que, selon Polybe, ils prenaient presque toujours un très grand soin ou de sembler observer les traités ou de trouver un prétexte pour les rompre, et de paraître mener partout des guerres défensives. Bien entendu, leurs desseins étaient seulement voilés par ces précautions et n'y étaient jamais subordonnés.\par
Cet art de conserver les apparences supprime ou diminue chez autrui l'élan que l'indignation donnerait, et permet de n'être pas soi-même affaibli par l'hésitation. Mais, pour que cet effet se produise pleinement, il faut être réellement convaincu qu'on a toujours raison, qu'on possède non seulement le droit du plus fort, mais aussi le droit pur et simple, et cela même quand il n'en est rien. Les Grecs n'ont jamais su être ainsi ; on voit dans Thucydide avec quelle netteté les Athéniens, quand ils commettaient de cruels abus de pouvoir, reconnaissaient qu'ils les commettaient. On ne bâtit pas un empire quand on a l'esprit si lucide. Les Romains ont pu parfois reconnaître que des sujets suppliants avaient été soumis à des cruautés trop grandes, mais ils le reconnaissaient alors en gens qui, loin d'éprouver des remords, s'applaudis­saient de condescendre à avoir pitié ; quant à admettre que des sujets révoltés ou des ennemis pussent avoir quelque bon droit, ils n'y songeaient même pas. Certains ont pu être autrement ; il ne nous en reste guère de traces. D'une manière générale les Romains jouissaient de cette satisfaction collective de soi-même, opaque, imperméable, impossible à percer, qui permet de garder au milieu des crimes une conscience parfaitement tranquille. Une conscience aussi impénétrable à la vérité implique un avilissement du cœur et de l'esprit qui entrave la pensée ; aussi les Romains n'ont-ils apporté d'autre contribution à l'histoire de la science que le meurtre d'Archimède. En revanche une telle satisfaction de soi, appuyée par la force et la conquête, est contagieuse, et nous en subissons encore la contagion.\par
Rien n'est plus essentiel à une politique de prestige que la propagande ; rien ne fut l'objet de plus de soin de la part des Romains. Chaque Romain d'abord était un propagandiste naturel au service de Rome ; car, sauf quelques exceptions, au nombre desquelles Lucrèce, cet unique disciple véritable des Grecs, chaque Romain mettait en son âme Rome au-dessus de tout. La vie spirituelle n'était guère à Rome qu'une expression de la volonté de puissance. La mythologie grecque fut un jeu de l'esprit qui laissait la pensée entièrement libre ; mais Cicéron réclamait le respect pour la religion romaine en raison de ses liens avec la grandeur de Rome. La masse des monuments, les routes, les aqueducs rendaient cette grandeur sensible aux yeux. La littérature est en grande partie souillée du souci de la rendre sensible à l'esprit ; au lieu que chez les Grecs, hors les discours purement politiques, aucune œuvre parvenue jusqu'à nous n'est entachée d'un souci de propagande en faveur de la grandeur grecque ou athénienne, Ennius, Virgile, Horace, Cicéron, César, Tite-Live, Tacite même, ont toujours écrit avec une arrière-pensée politique, et leur poli­tique, quelle qu'elle fût, était toujours impériale. Ils en ont été punis, comme il arrive toujours en pareil cas, car, hormis peut-être Tacite, leur infériorité par rapport aux Grecs est accablante. Quant aux formes de création spirituelle qui ne pouvaient pas être mises au service de la grandeur nationale, on pourrait presque dire qu'elles n'ont pas existé à Rome.\par
La propagande orale tenait bien entendu la plus grande place ; Rome savait amener à y prendre part ceux mêmes qui avaient les meilleures raisons de la haïr. Polybe, retenu quinze ans à Rome par force, rappelé impérative­ment après quelques mois de liberté, joua ce rôle. Partout où il y avait deux partis, l'un des deux était pro-romain ; dans les familles royales il y avait souvent un partisan et protégé de Rome, parfois élevé à Rome comme otage. Être ennemi de Rome était un crime punissable des plus impitoyables châti­ments ; même n'en être pas partisan était un crime. Les sujets, les vaincus, étaient contraints de louer Rome de toutes leurs forces, Nul n'était admis à protester contre un abus de pouvoir des Romains, sinon sous forme de supplication, nul n'était reçu à supplier sans louer les vertus de Rome, et avant tout celles qu'elle ne possédait pas, la générosité, la justice, la modération, la clémence. Tite-Live en fournit bien des exemples, certains peut-être exagérés ; mais l'exagération même serait alors significative. Tout ce que la lâcheté, la prudence, l'espoir des faveurs, le désir de participer, même au second plan, à l'éclat de la puissance, tout ce que l'admiration sincère peut donner de parti­sans influents et efficaces fut utilisé pleinement par Rome de son agrandisse­ment. On voit dans les différentes étapes Polybe pour les affaires grecques, dans César pour les affaires gauloises, avec quel soin cette action était menée parallèlement aux démarches politiques ou militaires. La propagande et la force se soutenaient mutuellement ; la force rendait la propagande à peu prés irrésistible en empêchant dans une large mesure qu'on osât y résister ; la propagande faisait pénétrer partout la réputation de la force.\par
Mais rien de tout cela n'aurait suffi sans un art que ni Richelieu, ni Louis XIV, ni Napoléon n'ont bien possédé, et où les Romains ont été prodi­gieux ;c'est celui d'observer, dans les actions exercées sur les autres pays, un rythme propre tantôt à les bercer dans une apparente sécurité, tantôt à les paralyser par l'angoisse et la stupeur, sans jamais souffrir chez eux d'état intermédiaire. On parle souvent de la maxime «diviser pour régner » comme si elle renfermait un secret de domination ; mais il n'en est rien, car le difficile est de l'appliquer. Un tel art en fournit seul le moyen. Il suffit, pour avoir la possibilité de l'exercer, d'avoir su une fois faire peur. Car les hommes, et surtout les peuples, qui ont moins de vertu que les individus, ne se dressent contre quiconque leur a fait peur que s'ils sont mus par une impulsion plus forte que la peur ; il dépend de celui qui manie la force qu'une semblable impulsion existe ou non à tel ou tel moment. Ainsi, lorsqu'un vainqueur tombe victime d'une coalition, c'est toujours parce qu'il a mal manœuvré. Si l'on manœuvre bien, on peut presque à son gré obtenir l'inaction ou le secours d'un pays, en lui faisant espérer qu'il échappera ainsi à tout mal et aura part aux rayons dorés de la victoire ; on peut l'humilier et en même temps l'encourager à préparer sa vengeance par une apparente torpeur ; on peut le contraindre à une soumission totale et sans combat, ou paralyser l'efficacité de ses armes, en fondant sur lui assez soudainement pour lui glacer l'âme de stupeur. Ces procédés peuvent être recommencés presque indéfiniment et toujours réussir, parce que dès qu'une peur diffuse a été suscitée, ce sont les passions et non l'intelligence qui déterminent l'attitude des peuples ; lorsque l'âme est abattue par la rigueur du destin, c'est le pouvoir de prévision qui disparaît le premier. Bien entendu, chaque succès obtenu augmente la capacité de manœuvre en augmentant la crainte ; mais aussi, le danger auquel on s'exposerait en négli­geant de manœuvrer ou en manœuvrant mal augmente à mesure, jusqu'au jour où la puissance progressivement acquise possède un prestige si écrasant que nul n'ose plus s'y attaquer et qu'elle s'écroule lentement d'elle-même.\par
On pourrait analyser en détail chaque action des Romains, à partir de la victoire de Zama, pour y retrouver l'application de cet art. Ils ont su surtout en observer le point le plus essentiel, la rapidité fulgurante de l'attaque. Hitler, dans une conversation, a parfaitement bien formulé à cet égard la règle à suivre, en disant qu'il ne faut jamais traiter quelqu'un en ennemi jusqu'au moment précis où on est en état de l'écraser. Cette dissimulation a les plus grandes chances de réussir dès qu'on s'est rendu redoutable ; car ceux qui craignent une éventualité espèrent toujours qu'elle ne se produira pas, jusqu'au moment où ils ne peuvent plus douter qu'elle s'est produite. Rome a souffert les conquêtes d'Antiochus, les préparatifs de guerre de Persée, le manque d'ardeur de Rhodes et des Achéens en sa faveur, plus tard la révolte des mêmes Achéens, la guerre de Carthage contre les Numides, et autres choses analogues, à peu prés sans protestation, jusqu'au moment marqué pour fondre sur le coupable en vue d'un châtiment impitoyable et immédiat. En chaque occasion la victime de ce traitement, qu'elle fût forte ou faible, avait l'âme décomposée par un désarroi qui l'empêchait d'user efficacement d'aucun moyen de défense. Mais tout était mis en œuvre pour que les spectateurs de telles exécutions ne sentissent aucune inquiétude immédiate concernant leur propre sort. À chaque nouveau succès les peuples et les princes s'accoutu­maient de plus en plus à voir dans les Romains leurs maîtres, et Rome était ainsi en mesure d'obtenir des victoires diplomatiques aussi brutales que la contrainte des armes. Témoin ce sénateur qui, arrivant près d'un roi, au lieu de répondre à son salut, traça un cercle autour de lui avec son bâton et lui ordonna de dire ses intentions concernant les exigences de Rome avant de sortir du cercle ; le roi s'engagea aussitôt à obéir en tous points. Ne prendrait-on pas ce sénateur pour un ministre des Affaires étrangères du Troisième Reich ?\par
Certes Hitler apparaît comme beaucoup moins redoutable à ses voisins que Rome. C'est heureux, sans quoi nous serions perdus, c'est-à-dire que notre pays serait écrasé, avec d'autres, sous une {\itshape pax germanica} dont nos descen­dants, dans deux mille ans, célébreraient éperdument les bienfaits. La princi­pale cause de faiblesse d'Hitler est qu'il applique les procédés qui ont infailliblement réussi à Rome après la victoire de Zama, alors que lui n'a pas vaincu Carthage, c'est-à-dire l'Angleterre ; ainsi ces procédés peuvent le perdre au lieu de le porter à la domination suprême. Il semble aussi que son habileté dans l'application des méthodes romaines reste inférieure, parfois peut-être très inférieure, au modèle original. Cependant les Romains n'ont certainement encore jamais eu un aussi remarquable imitateur, si toutefois il a imité et non inventé à nouveau. En tout cas tout ce qui nous indigne et aussi tout ce qui nous frappe d'étonnement dans ses procédés lui est commun avec Rome. Ni l'objet de la politique, à savoir imposer aux peuples la paix au moyen de la servitude et les soumettre par contrainte à une forme d'organi­sation et de civilisation prétendue supérieure, ni les méthodes de la politique ne diffèrent. Ce qu'Hitler a ajouté de proprement germanique aux traditions romaines n'est que pure littérature et mythologie fabriquée de toutes pièces. Nous serions singulièrement dupes, plus dupes encore que les jeunes hitlé­riens, en prenant au sérieux, si peu que ce soit, le culte de Wotan, le romantisme néo-wagnérien, la religion du sang et de la terre, et en croyant que le racisme est autre chose qu'un nom un peu plus romantique du nationalisme.
\subsubsection[III. - Hitler et le régime intérieur de l'empire romain]{III. - Hitler et le régime intérieur de l'empire romain}
\noindent \par
Le parallèle entre le système hitlérien et la Rome antique serait incomplet s'il se limitait aux méthodes de la politique extérieure. Il peut s'étendre au-delà ; il peut s'étendre à l'esprit des deux nations. Tout d'abord, la vertu propre de Rome était la même qui d'un certain point de vue met l'Allemagne du XX\textsuperscript{e} siècle au-dessus des autres nations, à savoir l'ordre, la méthode, la discipline et l'endurance, l'obstination, la conscience apportées au travail. La supériorité des armes romaines était due avant tout à l'aptitude exceptionnelle des soldats romains aux travaux ennuyeux et pénibles ; on peut dire qu'à cette époque comme aujourd'hui la victoire était obtenue par le travail plus encore que par le courage. Nul n'ignore le talent des Romains pour les grands travaux de dimensions colossales, destinés comme aujourd'hui à faire spectacle plus qu'à toute autre chose. Leur capacité de commandement, d'organisation et d'admi­nistration est assez prouvée par la durée d'une domination que les guerres civiles ont à peine ébranlée et qui ne s'est écroulée que par l'effet d'une lente décomposition interne. Tant que la machine de l'Empire resta intacte, aucune fantaisie de la part des empereurs ne put en compromettre le fonctionnement efficace. Dans cet ordre de vertu, Rome a mérité les louanges ; mais elles doivent se borner là.\par
L'inhumanité était générale dans les esprits et dans les mœurs. Dans la littérature latine on trouve peu de paroles qui rendent un véritable son d'huma­nité, tandis qu'on en trouve tant dans Homère, Eschyle, Sophocle et les prosateurs grecs ; on peut excepter un vers de Térence, qui était d'ailleurs carthaginois, et quelques vers de Lucrèce et de Juvénat. En revanche, le pieux héros du doux Virgile est représenté plus d'une fois tuant un ennemi désarmé qui implore la vie, et sans prononcer les paroles qui rendent même une scène de ce genre admirable dans l'{\itshape Iliade.} Quand ils ne s'attachaient pas à glorifier la force, les poètes latins, Lucrèce et Juvénat toujours exceptés, s'occupaient surtout à chanter le plaisir et l'amour, parfois d'une manière délicieuse ; mais la bassesse surprenante de la conception de l'amour chez les élégiaques n'est pas sans rapports étroits, selon toute vraisemblance, avec l'adoration de la force, et contribue à l'impression générale de brutalité. D’une manière géné­rale le mot de pureté, qu'on a si souvent le droit d'employer pour louer la Grèce dans tous les domaines de la création spirituelle, ne convient presque jamais quand il s'agit de Rome.\par
Les jeux de gladiateurs étaient une institution spécifiquement romaine, apparue quelque temps après la victoire sur Hannibal, et qui avait pour objet de provoquer la férocité ; elle y parvenait fort bien. L'horreur et la bassesse d'une telle institution, qui nous sont voilées par l'habitude d'en lire des descriptions dès l'enfance, ne peuvent se comparer à rien d'autre ; car le sang humain coulait non pas pour gagner la faveur des dieux, ni pour punir, ni pour terrifier par l'exemple d'un châtiment, mais uniquement pour procurer du plaisir. Sous l'Empire ce plaisir devint l'objet de passions aussi obsédantes, aussi irrésistibles, que la passion du jeu ou des stupéfiants ; saint Augustin en donne un exemple poignant. Les Romains et Romaines riches, rentrant chez eux encore ivres d'un tel plaisir, y trouvaient une foule d'esclaves absolument soumis à leurs caprices ; il serait étonnant dans ces conditions que l'esclavage à Rome n'eût pas été d'une extrême cruauté, ainsi que certains ont voulu le soutenir. Pour le croire, il faut n'avoir pas lu les textes, ou les avoir lus avec une aveugle indifférence. Les plus significatifs peut-être se placent aux environs de la victoire décisive sur Carthage ; ce sont les comédies de Plaute. L'œuvre de Plaute, l'une des plus sombres de la littérature universelle, bien qu'elle n'en ait pas la réputation, ne montre que trop bien combien les cruautés et les mépris soufferts par les esclaves, les menaces illimitées perpétuellement suspendues sur eux, abaissaient à la fois l'âme des esclaves et celle des maî­tres. Il savait de quoi il parlait, car il fut, dit-on, réduit par la misère presque au niveau d'un esclave. Son témoignage est dans l'ensemble confirmé par celui de Térence, autant qu'il est possible dans des comédies si gracieuses, ailées et poétiques, où le mal et la douleur tiennent peu de place.\par
L'histoire des proscriptions, notamment dans Appien, est instructive pour connaître les sentiments des esclaves envers leurs maîtres. Sénèque, Martial, Juvénal montrent par toutes sortes de traits affreux et poignants que les choses n'allaient pas mieux sous l'Empire. Sans doute Sénèque, qui d'ailleurs était espagnol, semble-t-il avoir été équitable et généreux envers ses esclaves, mais il ressort de la manière dont il en parle qu'il avait peu d'imitateurs ou point. Pline le Jeune aussi parle de sa propre humanité en ce domaine comme de quelque chose d'exceptionnel. On cite souvent le haut degré de faveur et de puissance où quelques esclaves pouvaient parvenir ; mais ce prix accordé aux plus vils arts de courtisan confirme que la manière dont les Romains usaient d'une telle institution entraînait un abaissement général des âmes. Le mépris attaché chez eux à la condition servile est surtout surprenant du fait qu'il frappait, outre ceux qui y étaient nés, un si grand nombre d'hommes et de femmes libres qui y étaient tombés par l'effet des armes victorieuses, de l'avidité ou de la perfidie des Romains. Sans doute on ne sait pas d'une manière précise comment les autres peuples de l'antiquité considéraient leurs esclaves, ce qui rend la comparaison difficile ; on sait du moins que les malheurs de l'esclavage tiennent une place dans la tragédie grecque, et que des penseurs grecs du V\textsuperscript{e} siècle ont regardé l'esclavage comme étant absolument et dans tous les cas contraire à la nature, ce qui voulait dire pour eux contraire à la justice et à la raison.\par
Comme les Romains des grandes familles se formaient au gouvernement des peuples par le spectacle des jeux de gladiateurs et le commandement de milliers ou de dizaines de milliers d'esclaves, il aurait fallu un miracle pour procurer aux provinces un traitement un peu humain. Rien, absolument rien ne peut donner à croire que ce miracle se soit produit. Nous n'avons à ce sujet, bien entendu, d'autre documentation que celle fournie par les Romains eux-mêmes ; il faudrait être bien crédule pour croire sur parole les louanges vagues qu'il pouvait leur arriver de se décerner. Les récits de cruautés qu'ils nous ont transmis sont au contraire fort précis. Il est vrai qu'elles étaient généralement racontées en vue d'en obtenir la punition ; mais il va de soi que celles, si grandes et si nombreuses fussent-elles, qu'on n'a pas cherché à punir n'ont pu, sauf exceptions, parvenir jusqu'à nous. Les {\itshape Verrines} de Cicéron montrent jusqu'à quel degré d'horreur non surpassée pouvaient aller ces cruau­tés envers une population pitoyablement soumise, combien de temps elles se prolongeaient impunément, combien le châtiment était difficile à obtenir et combien il était léger. Si elles avaient été exceptionnelles, Cicéron aurait-il dit, dans un de ses rares élans de véritable indignation : « Toutes les provinces pleurent, tous les peuples libres se plaignent... Le peuple romain ne peut plus désormais supporter de voir contre lui, dans toutes les nations, non plus la violence, les armes, la guerre, mais le deuil, les larmes et les gémissements. » Le peuple romain semble en fait avoir supporté tout cela fort bien ; le reste est certes vrai. Les lettres de Cicéron témoignent, pour qui sait lire, du malheur des provinces, même là où il était proconsul ; on y voit l'impitoyable avidité d'un homme aussi renommé pour sa vertu que Brutus. Les impôts accablants, multipliés par les prêts usuraires, allant jusqu'à contraindre les parents à vendre leurs enfants comme esclaves, étaient un trait permanent du régime des provinces ; de même les levées de soldats arrachant de force les jeunes gens à leurs foyers et les envoyant peiner jusqu'à la vieillesse dans des terres loin­taines ; de même le pouvoir des chefs imposés par Rome, absolu pour le mal, limité pour le bien, à peine corrigé par la crainte d'un châtiment lointain et improbable ; de même l'humiliation des provinciaux, qui osaient à peine quelquefois protester avec les plus humbles supplications et élevaient des statues à leurs pires oppresseurs.\par
On a dit que l'horreur du sort des provinces s'est atténuée sous l'Empire. Les charges financières furent sans doute en effet moins lourdes pendant certaines périodes ; la classe des sénateurs était devenue facile à punir. Pour­tant on voit dans Juvénal que même sous Trajan il y avait des imitateurs de Verrès. Les levées de soldats, bien entendu, subsistaient. Il est parfois ques­tion dans Tacite de populations indociles déportées tout entières d'un territoire dans un autre. Un récit de Tacite, dont le témoignage est ici celui d'un contem­porain, montre trop bien à quel avilissement la « paix romaine » réduisait les cœurs ; l'armée de Gaule, commandée par Vitellius, ayant massacré des milliers de personnes dans une ville, sans raison, par une sorte de délire, elle fut accueillie, dans la suite de ses déplacements, par des cités tout entières qui sortaient hors des murs à sa rencontre, avec des rameaux de suppliants, magistrats en tête, femmes et enfants prosternés à terre le long de la route. Cela se passait en pleine paix ; un seul siècle de régime colonial avait à ce point abaissé un peuple fier. Enfin, en raison même de l'admiration de Tacite pour la grandeur romaine, on peut croire qu'il n'accomplissait pas un simple exercice de rhétorique quand il faisait dire à un chef breton, ennemi de son beau-père Agricola : « On ne peut éviter leur insolence par la soumission et la modération. Ces voleurs du globe terrestre... seuls ils ont une même passion pour s'emparer soit de la richesse, soit de la pauvreté de tous les hommes... Celles des épouses et des sœurs qui ont échappé à leurs violences dans la guerre, c'est sous le nom d'hôtes et d'amis qu'ils les souillent... Quand ils ont fait le désert, ils appellent cela la paix. »\par
Il serait singulier que la civilisation pût se transporter d'un pays à un autre par de pareilles méthodes. Mais quel pays au juste Rome a-t-elle civilisé ? Non pas certes ceux de la Méditerranée orientale, qui l'étaient depuis long­temps. Carthage, au moment ou elle a disparu, possédait très probablement une civilisation plus brillante de beaucoup que n'était celle de Rome à cette époque, car c'était une cité phénicienne que le commerce et la navigation mettaient en rapport avec la Grèce et tout l'Orient méditerranéen. Rome n'a donc pas civilisé l'Afrique ; elle n'a pas non plus civilisé l'Espagne, que Carthage avait déjà colonisée d'une manière sans doute assez dure, mais infiniment moins dure que Rome ne fit plus tard. Pendant tant de siècles de domination romaine, l'Afrique ne produisit de grand homme que saint Augus­tin, l'Espagne que Sénèque, Lucain, et dans un autre domaine Trajan. Qu'est-ce que la Gaule a fait qui vaille la peine d'être cité, pendant les siècles où elle fut romaine ? On ne peut guère soutenir qu'elle n'ait pas su auparavant créer dans le domaine de l'esprit, puisque les druides étudiaient pendant vingt ans, apprenaient par cœur des poèmes entiers concernant l'âme, la divinité, l'uni­vers ; bien plus, ceux des Grecs qui croyaient que la philosophie avait été empruntée par la Grèce à l'étranger la disaient venue, d'après Diogène Laërce, de Perse, de Babylone, d'Égypte, de l'Inde et des druides de Gaule. Tout a disparu sans laisser de traces, et le pays n'a repris une vie originale et créatrice que lorsqu'il n'a plus été romain. Sauf en Syrie, en Palestine, en Perse, où pendant longtemps l'emprise de la force romaine s'est fait assez peu sentir, les provinces et les pays soumis à Rome ont servilement imité Rome, qui elle-même imitait. Sans doute les arts, les sciences, les lettres, la pensée ne sont pas tout ; mais quels biens les provinces eurent-elles à la place ? Non pas certainement la liberté, la fierté des caractères ; non pas sans doute, sauf exception, la justice ou l'humanité. Les routes et les ponts, et même le bien-être matériel, en admettant qu'il y en ait eu peut-être à certaines périodes, ne sont pas la civilisation. Mais si l'Allemagne, grâce à Hitler et à ses succes­seurs, asservissait les nations européennes et y abolissait la plupart des trésors du passé, l'histoire dirait certainement qu'elle a civilisé l'Europe.\par
La servitude à laquelle étaient soumis les sujets des Romains ne tarda pas à s'étendre aux Romains eux-mêmes. Cela se fit facilement. Depuis la mort des Gracques, Caton si l'on veut mis à part, on ne trouve plus un caractère ferme à Rome, ni aucune fierté. La fierté romaine subsistait seulement envers les étrangers, parce qu'on pouvait voir en eux des vaincus au moins en puis­sance ; encore des consuls, des préteurs proscrits par Octave et Antoine ne dédaignèrent-ils pas, d'après Appien, de tomber aux genoux de leurs propres esclaves en les nommant leurs sauveurs et leurs maîtres. Soixante ans après la destruction de Carthage, Rome subissait de la part de Marius et de Sylla, de leurs soldats et de leurs esclaves, tous les outrages infligés à une ville con­quise, et elle se taisait et se soumettait. Dès lors Cicéron pouvait à son aise jouer au citoyen, Brutus pouvait se prétendre le libérateur du globe terrestre pour avoir libéré par un crime quelques centaines de milliers d'hommes avides et cruels comme lui-même. Il n'y avait plus moyen d'éviter la servitude, et ceux qu'on nommait citoyens étaient prêts à se mettre à genoux même avant d'avoir un maître. Ce que produisirent enfin l'idée fixe de la domination, la cruauté, la bassesse d'âme, ce fut ce que nous nommons aujourd'hui un État totalitaire.\par
Nous ne songeons guère à reconnaître dans tels et tels phénomènes de notre temps l'image de ce qui se passait dans l'Empire romain. Les déporta­tions massives de paysans dans le Sud-Tyrol et l'Europe orientale nous font justement horreur ; elles ne nous rappellent pas cette première églogue de Virgile sur laquelle nous avons rêvé dès l'enfance, et ceux qui disent : « Nous, nous quittons la terre de la patrie et nos champs bien-aimés... Nous allons vers l'Afrique pleine de soif. » Il s'agissait pourtant d'une mesure tout à fait pareille ; les masses humaines furent aussi brutalement brassées, les fibres qui attachent l'homme à sa propre existence aussi impitoyablement arrachées. Mais comme Virgile fut excepté de cette mesure par grâce particulière, elle ne nous fait pas horreur à distance. La ressemblance entre le titre d'{\itshape imperator} et celui de chef dans les États totalitaires modernes ne nous fait pas ouvrir les yeux sur l'analogie des fonctions. Nous ne croyons surtout pas que le degré de la tyrannie soit le même. Pourtant jamais les esprits ne se sont plus complète­ment qu'alors courbés devant la puissance d'un homme et n'ont senti plus durement la froide étreinte de la force. Le malheur d'Ovide, malheur dont la cause n'a été connue ni de son temps ni depuis, rend sensible la condition des hommes de son époque ; il faut seulement lire d'un trait, dans leur affreuse monotonie, la suite des termes d'abjecte supplication et d'adoration prosternée qu'il ne s'est pas lassé de répéter une année après l'autre jusqu'à sa mort. Il implorait ainsi non pas sa grâce, mais un lieu de déportation un peu moins rigoureux ; il n'obtint rien. Tant d'arbitraire et de dureté d'une part, tant de bassesse de l'autre, ne seraient pas possibles sans une disposition générale des esprits qui les rende tels.\par
Dira-t-on qu'Auguste réservait ses sévérités aux hommes trop confiants dans leur fortune, mais protégeait les faibles ? Il fut responsable dans sa vieil­lesse d'une disposition législative condamnant à la torture et à la mort tous les esclaves habitant sous le toit d'un maître assassiné par un inconnu. En réalité, on peut sans crainte conclure du sort d'un homme défendu par une brillante renommée et des amis influents aux malheurs suspendus sur tous ceux qui ne possédaient aucun de ces deux avantages. Le concert unanime d'adulations en prose et en vers qui entourait Auguste nous étourdit ; comme si un maître absolu et impitoyable n'était pas toujours en état d'obtenir l'unanimité ! Et s'il se trouve parmi ses sujets des hommes de génie, pour qu'ils prennent part à un tel concert il suffit le plus souvent qu'on les y invite. Un prince n'a besoin, pour paraître à jamais admirable aux yeux de la postérité, que de savoir choisir des écrivains suffisamment bien doués et d'en faire ses serviteurs ; pourtant le discernement dans l'appréciation des talents n'a aucun rapport avec les capa­cités et les vertus qui conviennent à un souverain.\par
En revanche, on a coutume maintenant d'accuser Tacite d'hostilité systé­matique contre les successeurs d'Auguste. Il est probable à vrai dire qu'il noircit leurs portraits et qu'il exagère considérablement leurs responsabilités personnelles dans les malheurs de l'époque ; ses propres écrits suffisent pour s'en convaincre. On ne peut pas non plus avoir de sympathie pour ses nostal­gies républicaines, quand on sait ce qu'avait été la République, ni de compassion envers ce Sénat qui avait été pour les nations un maître si arrogant et si cruel, et tomba dans une bassesse sans limites dès qu'il eut des maîtres à son tour. Mais si l'on peut révoquer en doute le témoignage de Tacite quant à la personne des empereurs, il n'y a aucune bonne raison d'en faire autant quant à l'état de l'Empire. Car ces sénateurs, à qui leurs maîtres pouvaient infliger tous les outrages sans exception et qui toujours les louaient et les remerciaient, gardaient encore dans une large mesure le privilège des honneurs et des hautes fonctions ; peut-on dès lors supposer que plus bas il y ait eu plus d'équité et plus de fierté ? Ne doit-on pas penser que l'arbitraire, l'insolence et la cruauté, la servilité et l'obéissance passive se retrouvaient de haut en bas dans tout l'Empire ? Il est vrai que les empereurs prenaient grand soin du bas peuple de Rome ; mais ce soin consistait à le nourrir d'aumônes et à le soûler continuel­lement du sang des gladiateurs. Ils prenaient grand soin aussi de l'armée, qui jouait dans une certaine mesure le rôle du parti d'État dans les États totalitaires modernes ; on peut voir dans Juvénal jusqu'où allait en pleine rue la licence toujours impunie des soldats. Si les provinces craignaient peut-être moins qu'avant les magistrats, elles craignaient davantage l'armée. Et la « paix ro­maine » n'empêchait pas, quand il fallait donner des encouragements aux soldats, de franchir une frontière par surprise, sans aucun incident préalable, et de tout massacrer dans un certain territoire sans épargner ni le sexe, ni l'âge, ni les lieux sacrés ; Germanicus fit ainsi sous Tibère en Germanie.\par
Mais l'écrasement des nations sujettes, la cruauté diffuse, quotidienne, publiquement encouragée, la bassesse et la soumission sans limites à l'égard d'une autorité capable de manier les masses et les individus comme des objets sans valeur, tous ces traits ne sont pas encore ceux qui rappellent de la manière la plus frappante les dictatures totalitaires modernes. Des structures sociales très différentes peuvent comporter le pouvoir absolu d'un homme. Par exemple, dans l'Espagne de la Renaissance et aussi, semble-t-il, dans la Perse antique, c'était la personne du souverain légitime, c'est-à-dire déterminé par les lois, qui était l'objet d'une obéissance et d'un dévouement illimités ; si loin qu'aille en pareil cas la soumission, elle peut comporter une véritable gran­deur, car elle peut être causée par la fidélité aux lois et à la foi jurée et non par la bassesse d'âme. Mais à Rome ce n'était pas l'empereur en tant qu'homme, c'était l'Empire devant quoi tout pliait ; et la force de l'Empire était constituée par le mécanisme d'une administration très centralisée, parfaitement bien organisée, par une armée permanente nombreuse et généralement disciplinée, par un système de contrôle qui s'étendait partout. En d'autres termes l'État était la source du pouvoir, non le souverain. Celui qui était parvenu à la tête de l'État obtenait la même obéissance, par quelque procédé qu'il y fût parvenu. Les luttes civiles, quand elles se produisaient, avaient pour objet de changer la personne placée à la tête de l'État, mais non pas les rapports entre l'État et ses sujets ; l'autorité absolue de l'État ne pouvait être mise en question, car elle reposait non pas sur une convention, sur une conception de la fidélité, mais sur le pouvoir que possède la force de glacer les âmes des hommes.\par
Cet État centralisé produisait l'effet qu'il produit aussi de nos jours, même sous sa forme démocratique, de drainer la vie du pays vers la capitale et de ne laisser dans le reste du territoire qu'une existence morte, monotone et stérile. Malgré l'insolence et le luxe effréné des riches, la mendicité servile à laquelle étaient contraints la plupart de ceux qui ne l'étaient pas, Rome exerçait un attrait invincible. Toute vie locale et régionale avait péri dans cet immense territoire ; la disparition des langages de la plupart des pays conquis en est la meilleure preuve. Mais, trait qu'on retrouve aujourd'hui seulement dans les dictatures totalitaires d'Allemagne et de Russie, l'État était également l'unique objet des aspirations spirituelles, l'unique objet d'adoration. Théoriquement l'empereur devenait un dieu seulement après sa mort, mais la flatterie en faisait déjà un dieu sur terre, et il était en fait le seul dieu qui comptât. D'ailleurs, que l'objet des sentiments religieux fût les empereurs morts ou l'empereur vivant, c'était toujours l'État qui était adoré. Ce culte était protégé comme aujourd'hui par un contrôle minutieux et impitoyable et par un encou­ragement systématique de la délation ; la {\itshape lex majestatis} permettait de punir non seulement les offenses à la religion officielle, mais même au besoin l'insuffisance du zèle. Les statues, les temples, les cérémonies étendaient cette religion à tout le territoire, et tous les hommes de marque en étaient obligatoi­rement les instruments de propagande. La tolérance bien connue des Romains de cette époque en matière de dieux s'appliquait seulement aux dieux suscepti­bles de servir de satellites à l'Empire ; elle n'empêcha pas par exemple qu'on ne détruisit impitoyablement le clergé des Druides. En réalité, seules des sectes clandestines, comme Carcopino a montré que ce fut le cas des pytha­goriciens, pouvaient adorer autre chose que l'État ; et l'Église ne fait que retrouver aujourd'hui son ennemi des premiers temps. On pourrait regarder les efforts de ces sectes, à commencer par la secte chrétienne, comme exprimant la lutte de l'esprit grec contre l'esprit romain. Si notre lutte d'aujourd'hui a un sens, elle a le même sens.\par
Il nous est certes difficile de nous résoudre à admettre une espèce d'identité entre notre ennemi et la nation dont la littérature et l'histoire nous fournissent presque exclusivement la matière de ce que nous nommons les humanités. L'esprit antijuridique, antiphilosophique, antireligieux qui est inhé­rent au système hitlérien fait regarder notre ennemi comme un danger pour la civilisation ; les Romains n'ont-ils pas au contraire la réputation d'avoir été religieux, curieux de philosophie, et d'avoir inventé l'esprit juridique ? Mais l'opposition n'est qu'apparente. Les Romains n'eurent jamais d'autre religion, du moins à partir de leurs grandes victoires, que celle de leur propre nation en tant que maîtresse d'un empire. Leurs dieux n'étaient utiles qu'à maintenir et accroître leur grandeur ; nulle religion ne fut jamais plus étrangère à toute notion du bien et du salut de l'âme ; l'amour de la nature non plus n'y avait aucune part. Pendant un temps la mode et le snobisme les rendirent curieux de philosophie grecque ; mais, Lucrèce mis à part, il n'y a aucun signe qu'ils l'aient jamais comprise, et il vaudrait mieux ne rien savoir de la pensée grec­que que d'être renseigné sur elle seulement par des textes latins. Sous l'Empire, l'autorité de l'État découragea cette curiosité. Pendant cette période l'œuvre de l'esclave phrygien Épictète et celle de Marc-Aurèle sont seules précieuses en ce domaine, et toutes deux appartiennent à la littérature grecque. Marc-Aurèle écrivait sans doute en secret. Plusieurs empereurs persécutèrent systématiquement la philosophie. Quant au droit, il est faux d'abord que les Romains aient créé l'esprit juridique ; dans la limite des temps historiquement connus l'esprit juridique est né en Mésopotamie, et il a atteint son plus haut degré de développement il y a quarante siècles. Il est incontestable pourtant que les Romains furent juristes. Mais quand an reproche aux Allemands hitlé­riens de détruire l'essence même du droit en le subordonnant à la souveraineté et à l'intérêt de l'État, en le faisant émaner de la nation, on ne prend pas garde qu'en ce point ils sont les héritiers fidèles des Romains. Il serait extrêmement difficile de soutenir en s'appuyant sur des textes que les Romains aient conçu le droit comme émanant des individus et apportant une limite à la souveraineté de l'État dans ses rapports avec eux. Si la souveraineté de la cité eut chez eux une limite, cette limite lui fut imposée par la souveraineté du groupe familial ; à mesure que la cité se transforma en État et que la famille se décomposa, cette limite perdit sa force. L'empereur eut en fait le pouvoir de contraindre un homme marié au divorce et d'annuler un testament ; c'est pour cette raison que tant de riches Romains léguèrent à l'empereur une grande part de leur fortune, afin que leur famille pût conserver le reste ; ce fait n'indique-t-il pas suffisam­ment la subordination du droit privé à l'autorité souveraine ? Quant aux contrats internationaux, jamais les Romains ne se crurent obligés d'observer un traité quand il leur était avantageux de le violer ou de le tourner. Quand on leur attribue l'esprit juridique, on commet une équivoque ; la compilation de vastes recueils de lois n'a aucun rapport avec la sainteté des contrats.\par
Au reste, qu'on compte les siècles pendant lesquels a duré l'Empire romain, les territoires auxquels il s'est étendu, que l'on compare ces siècles à ceux qui précédèrent Rome et à ceux qui suivirent l'invasion des barbares, et l'on verra à quel point l'État totalitaire a frappé de stérilité spirituelle le bassin méditerranéen. Sans doute cette période désertique fut coupée au cours de tant d'années par des moments de fécondité. Le talent des écrivains contemporains d'Auguste, talent formé d'ailleurs au cours des luttes civiles, n'est pas contes­table, bien que servile. L'étonnante dynastie des Antonins suscita un renou­veau des lettres dont les œuvres de Tacite et de Juvénat sont les nobles fruits ; et le stoïcisme grec prit place sur le trône. Plus tard Julien fut une figure bien attirante. La tyrannie de l'État n'empêcha pas l'Évangile d'être rédigé dans quelque coin de la Méditerranée orientale ; saint Augustin et les Pères de l'Église grecque témoignent que le christianisme réussit à répandre une certai­ne tendresse d'âme tout à fait étrangère à la tradition latine, bien qu'il ne soit devenu le principe d'une civilisation originale qu'après l'heureuse invasion de ceux qu'on nommait les barbares.\par
Les quelques moments lumineux de l'Empire romain ne doivent pas retenir l'attention au point d'empêcher de sentir l'analogie entre ce système et celui d'Hitler ; car il y a analogie. Hitler n'écrase pas plus la Bohême que Rome n'écrasait ses provinces. Les camps de concentration ne sont pas un moyen plus efficace d'éteindre la vertu d'humanité que ne le furent les jeux de gladiateurs et les souffrances infligées aux esclaves. Le pouvoir d'un homme n'est pas exercé d'une manière plus absolue, plus arbitraire et plus brutale à Berlin qu'il ne fut à Rome ; la vie spirituelle n'est pas traquée avec plus de soin et de cruauté en Allemagne qu'elle ne le fut dans l'Empire romain. Supposons Hitler vainqueur en Europe ; le talent manquerait peut-être alors aux poètes pour le célébrer en beaux vers, mais non pas certainement la bonne volonté. D'autre part, dans la série de ses successeurs, certains seraient peut-être relativement des hommes de bien. Si le régime durait des siècles, la vie spirituelle se ferait certainement jour par moments. De telles considérations ne peuvent certes pas nous empêcher d'aspirer à la perte d'Hitler. La principale différence entre lui et les Romains est qu'il exerce une dictature totalitaire avant d'être devenu le maître du monde, ce qui l'empêchera vraisembla­blement de le devenir ; car il semble qu'un État totalitaire soit propre à écraser ses sujets plutôt qu'à en conquérir beaucoup d'autres. Mais l'esprit des deux systèmes, à l'intérieur comme à l'extérieur, n'en apparaît pas moins comme étant à peu près identique et méritant des termes identiques, soit de louange, soit d'exécration.
\subsubsection[Conclusion]{Conclusion}
\noindent \par
Vainement tenterait-on de diminuer la portée de telles analogies en soutenant que la morale change, et que des actes qui étaient pardonnables ou admirables autrefois sont devenus inadmissibles. Ce lieu commun ne soutient pas l'examen. Rien ne permet de croire que la morale ait jamais changé. Tout porte à penser que les hommes des temps les plus reculés ont conçu le bien, quand ils l'ont conçu, d'une manière aussi pure et aussi parfaite que nous, bien qu'ils aient pratiqué le mal et l'aient célébré quand il était victorieux, exacte­ment comme nous faisons. En tout cas la plus ancienne conception de la vertu qui soit parvenue jusqu'à nous, celle élaborée par l'Égypte, est pure et complète au point de ne pas laisser place au progrès. Il y a quarante siècles, dans cette terre comblée de grâce, l'être humain le plus misérable avait un prix infini, parce qu'il devait être jugé et pouvait être sauvé. On faisait alors dire à Dieu : « J'ai créé les quatre vents pour que tout homme puisse respirer comme son frère... J'ai créé tout homme pareil à son frère. Et j'ai défendu qu'ils commettent l'iniquité, mais leurs cœurs ont défait ce que ma parole avait prescrit. » On n'a jamais rien écrit de si touchant pour définir la vertu que les paroles prononcées dans le Livre des Morts par l'âme qui va être sauvée : « Seigneur de la vérité... Je t'apporte la vérité. J'ai détruit le mal pour toi... Je n'ai pas méprisé Dieu... Je n'ai pas mis en avant mon nom pour les honneurs... Je n'ai pas été cause qu'un maître ait fait souffrir son serviteur... Je n'ai fait pleurer personne... Je n'ai causé de peur à personne... Je n'ai pas rendu ma voix hautaine... Je ne me suis pas rendu sourd à des paroles justes et vraies. » Pendant quelques centaines d'années l'Égypte a donné l'exemple de ce que peut être une civilisation non souillée d'impérialisme ou de brutalité systéma­tique ; et nous, trente-cinq siècles plus tard, à peine imaginons-nous parfois en rêve que pareille chose est peut-être possible.\par
L'idée que se faisaient de la vertu les Grecs d'il y a vingt-cinq siècles est mieux connue encore ; il serait insensé de croire qu'à cet égard il ait pu y avoir progrès par rapport à eux. Il se trouvait alors des penseurs pour condamner sans réserve l'institution de l'esclavage ; Eschyle, dans son {\itshape Agamemnon}, con­damnait implicitement la violence et la guerre ; l'Antigone de Sophocle repoussait toute haine quelle qu'elle fût. Sans doute Athènes eut des velléités d'impérialisme qui firent d'ailleurs sa perte ; et il s'en faut de beaucoup que la perfidie et la cruauté aient été absentes de sa politique extérieure. Mais per­sonne, pas plus alors qu'aujourd'hui, ne regardait de telles pratiques comme étant, du point de vue de la morale, louables ou indifférentes ; quand on en faisait l'apologie, c'était en développant sous une forme ou sous une autre la maxime « Politique d'abord », exactement comme à présent. D'ailleurs l'impé­rialisme et les méthodes qu'il comporte avaient des adversaires irréductibles, dont Socrate est le plus connu ; il fut, il est vrai, mis à mort, mais on l'avait laissé vivre jusqu'à soixante-dix ans, et ses disciples eurent toute licence pour le célébrer dans leurs écrits et leurs discours. Quant aux Romains, si on lit leur histoire dans les textes de l'antiquité, notamment dans Polybe, on a l'impres­sion que de leur temps au nôtre la sensibilité morale n'a guère changé. Sans doute toutes les nations de cette époque pratiquaient-elles plus ou moins, par intervalles, la perfidie et la cruauté ; on ne saurait affirmer qu'il en ait été autrement depuis à aucune époque, sans exclure la nôtre. Mais comme aujourd'hui la perfidie et la cruauté, bien que pratiquées, étaient généralement réprouvées ; comme aujourd'hui une seule nation en faisait froidement et systématiquement le principe même de sa politique pour un but de domination impériale. Une telle politique, de la part de notre ennemi, nous parait mons­trueuse ; elle ne devait pas paraître moins monstrueuse aux contemporains des Romains ; la meilleure preuve en est le voile épais d'hypocrisie dont les Ro­mains l'on recouverte, hypocrisie si singulièrement semblable à celle pratiquée de nos jours, notamment quant au camouflage de l'agression en légitime défense. Puisqu'on était hypocrite alors de la même manière qu'aujourd'hui, c'est que l'on concevait le bien de la même manière.\par
Mais quand même la morale aurait changé, cela ne diminuerait en rien la portée du fait que de nos jours on parle partout de la Rome antique avec admiration. Car un homme ne peut pas juger une action, quelle qu'en soit la date, par rapport à une conception de la vertu autre que celle qui sert de critérium pour ses propres actions. Si j'admire ou si même j'excuse aujourd'hui un acte de brutalité commis il y a deux mille ans, je manque aujourd'hui, dans ma pensée, à la vertu d'humanité. L'homme n'est pas fait de compartiments, et il est impossible d'admirer certaines méthodes employées autrefois sans faire naître en soi-même une disposition à les imiter dès que l'occasion rendra une telle imitation facile.\par
Rome a aboli par la force les différentes cultures du bassin méditerranéen, sauf la culture grecque, qu'elle a reléguée au second plan, et elle a imposé à la place une culture presque entièrement subordonnée aux besoins de la propa­gande et à la volonté de domination. Par là le sens de la vérité et de la justice a été et est demeuré presque irrémédiablement faussé ; car, pendant tout le moyen âge, la culture romaine a été à peu prés la seule connue des gens instruits dans tout l'Occident. L'influence aurait pu en être suffisamment contre-balancée par celle du christianisme, si la seconde avait pu être séparée de la première. Par malheur Rome, ayant adopté le christianisme après quel­ques siècles et l'ayant officiellement établi chez les nations sujettes, a ainsi contracté avec lui une alliance qui l'a souillé. Par un second malheur, le lieu d'origine du christianisme lui a imposé l'héritage de textes où s'expriment souvent une cruauté, une volonté de domination, un mépris inhumain des ennemis vaincus ou destinés à l'être, un respect de la force qui s'accordent extraordinairement bien avec l'esprit de Rome. Ainsi, par l'effet d'un double accident historique, la double tradition hébraïque et romaine étouffe dans une large mesure depuis deux mille ans l'inspiration divine du christianisme. Aussi l'Occident n'a-t-il jamais retrouvé l'accent d'incomparable humanité qui fait de l'{\itshape Iliade} et des tragédies grecques des œuvres sans égales.\par
La France a eu beaucoup d'esprits de premier ordre qui n'ont été ni les serviteurs ni les adorateurs de la force. Du XV\textsuperscript{e} au XVII\textsuperscript{e} siècle, Villon, Rabe­lais, La Boétie, Montaigne, Maurice Scève, Agrippa d'Aubigné, Théophile, Retz, Descartes, Pascal, si divers et de renommée inégale, ont eu cela en commun outre le génie. Mais ceux qui furent l'un et l'autre contribuent à former chaque génération successive. La seule chanson de geste connue dans les lycées célèbre Charlemagne, c'est-à-dire une entreprise de domination universelle. Les héros des tragédies non religieuses de Corneille mettent au-dessus de tout leur gloire, qui consiste à vaincre, à conquérir, à dominer, et ne songeraient jamais à subordonner cette gloire à la justice ou au bien public ; en eux, la démesure est proposée à l'admiration. Plusieurs des héros de Racine ont la même idée fixe quand ils s'occupent à autre chose qu'à l'amour ; aussi est-ce seulement à propos de l'amour que Racine a retrouvé une fois, dans {\itshape Phèdre}, quelque chose de l'accent de la tragédie grecque. L'évocation de la mort et de Dieu n'empêche pas les grandeurs humaines d'apparaître avec une souveraine majesté dans Bossuet. Les despotes n'ont pas cessé, au XVIII\textsuperscript{e} siècle, de trouver en France des adulateurs illustres ; il fallait seulement alors qu'ils fussent étrangers. Combien n'a-t-on pas plus tard chanté Napoléon ! L'idée du héros méprisé et humilié, si commune chez les Grecs, et qui forme le sujet même des Évangiles, est presque étrangère à notre tradition ; le culte de la grandeur conçue selon le modèle romain nous a été transmis par une chaîne presque ininterrompue d'écrivains célèbres.\par
Ce culte a toujours inspiré chez nous des actes aussi bien que des paroles. De notre temps même, il serait difficile sans doute d'affirmer que nous n'ayons pas usé et n'usions pas, pour la conquête et la domination de notre empire colonial, de méthodes semblables à celles de Rome ; beaucoup de Français seraient plutôt enclins à s'en vanter qu'à le nier. Les hommes de la Révolution ne se seraient pas laissé si facilement tenter par la guerre de conquête s'ils n'avaient pas été nourris des écrivains latins et de Plutarque, ce sujet servile des Romains, et s'ils n'avaient pas songé à Rome toutes les fois qu'ils parlaient de République. Napoléon et Louis XIV ont visiblement été obsédés par le souvenir d'Auguste, et tous les procédés de Rome leur ont paru bons à imiter. Si leurs efforts n'ont pas été couronnés par un succès durable, un certain défaut d'habileté en est cause, mais non pas certes un excès de scrupule. L'agression non provoquée de la Hollande, l'annexion de plusieurs villes, en pleine paix, au cours des années qui suivirent un traité par lequel les frontières avaient été solennellement fixées, et contre le gré des habitants, enfin la dévastation du Palatinat, qui n'avait pas non plus l'excuse de la guerre, ce sont là des incidents singulièrement semblables aux faits les plus caractéristiques de l'histoire romaine. On peut en dire autant, par exemple, du piège tendu par Napoléon à la famille royale d'Espagne, des manœuvres qui le préparèrent et du sort infligé par la suite à ce malheureux pays. Enfin, sous Napoléon et surtout sous Louis XIV la servilité des sujets, la soumission aveugle, la flatterie extrême, l'absence de toute liberté spirituelle mirent quelque temps la France au niveau de la Rome impériale et de ses provinces.\par
Quel objet Hitler et ceux qui acceptent en lui un maître poursuivent-ils en ce moment, sinon la grandeur conçue selon le modèle romain ? Comment le poursuivent-ils, sinon par les méthodes que tous les émules des Romains avaient déjà imitées plus ou moins bien ? Y a-t-il rien de plus connu, de plus familier ? Il ne devrait pas suffire qu'un phénomène familier apparaisse ailleurs que chez nous et nous menace pour nous sembler soudain incom­préhensible. Il est infiniment triste que notre plus grand poète ait participé à cette abdication de l'intelligence en affirmant que nous ne pouvons rien comprendre à l'Allemagne. L'Allemagne est aujourd'hui la nation qui com­promet continuellement la paix et les libertés de l'Europe ; la France était dans le même cas en 1815. Elle constituait pour l'Europe la menace principale depuis Richelieu, c'est-à-dire depuis près de deux siècles, avec un intervalle de faiblesse sous Louis XV et Louis XVI. En 1814 et 1815, on détruisit Napoléon ; on ne fit pas de mal à la France vaincue ; l'Europe fut en paix pour un demi-siècle. Si l'on veut faire remonter la menace allemande à Frédéric II de Prusse, et on ne saurait d'aucune manière la faire remonter plus loin, on obtient aussi deux siècles. Au nom de quoi pourrait-on soutenir qu'il est plus nécessaire de subjuguer l'Allemagne à présent si l'on en triomphe, qu'il n'a été nécessaire de subjuguer la France en 1815 ?\par
On dit que l'Allemagne est devenue conquérante et menaçante depuis qu'elle est devenue une nation une et centralisée. Cela est incontestable. Mais il en fut exactement de même pour la France ; l'unité française est seulement plus ancienne de deux siècles que l'unité allemande. Tout peuple qui devient une nation en se soumettant à un État centralisé, bureaucratique et militaire devient aussitôt et reste longtemps un fléau pour ses voisins et pour le monde. Il y a là un phénomène inhérent, non pas au sang germanique, mais à la structure de l'État moderne, semblable à tant d'égards à la structure politique élaborée par Rome. Pourquoi il en est ainsi, c'est peut-être difficile à claire­ment concevoir ; qu'il en soit ainsi, c'est tout à fait clair. En même temps qu'il naît quelque part une nation dominée par un État, il naît un nouveau facteur d'agression, et le développement ultérieur de la nation reste longtemps agres­sif. Quelques petits pays européens, formés dans le respect des libertés locales, ont échappé à cette fatalité, et aussi dans une certaine mesure un grand pays, l'Angleterre, qui même aujourd'hui est loin de présenter tous les caractères de l'État moderne. Mais un phénomène qui ne date pas d'hier devrait inquiéter tous les esprits réfléchis sur le danger que courent à notre époque non seulement la paix et la liberté, mais toutes les valeurs humaines sans exception. Tous les changements accomplis depuis trois siècles appro­chent les hommes d'une situation où il n'y aurait absolument plus aucune source d'obéissance dans le monde entier excepté l'autorité de l'État. La plupart des hommes en Europe n'obéissent à rien d'autre. L'emprise familiale est faible sur les mineurs, nulle sur les autres ; les autorités locales et régionales, dans la plupart des pays, exercent seulement la part de pouvoir qui leur est déléguée par l'autorité centrale. Dans le domaine de la production l'obéissance n'est pas accordée aux chefs, mais leur est vendue pour de l'argent ; ainsi leur autorité prend sa source non dans une tradition, non dans un consentement mutuel plus ou moins tacite, mais dans un marchandage qui exclut toute dignité et lui ôterait toute efficacité sans la protection de l'État. Dans le domaine même de l'intelligence, l'État, avec les diplômes qu'il confère, est devenu presque la seule source d'autorité effective.\par
D'autre part le pouvoir de l'État n'est arrêté dans aucune direction par aucune limite légitime. Il n'existe pas une telle limite pour lui au-dehors, car chaque nation est souveraine. Quoi qu'elle fasse, il n'est aucune autorité supé­rieure qui ait le droit de juger ses actions ; les traités même qu'elle signe ne l'engagent que d'après l'interprétation qu'elle leur accorde elle-même, sans qu'aucune autre interprétation lui puisse être légitimement imposée du dehors ; son pouvoir n'est en fait limité que par la force des autres nations souveraines, c'est-à-dire ou par la guerre ou par la menace explicite ou implicite de guerre. C'est là une limite de fait et non de droit, une limite subie et non pas acceptée. À l'intérieur, les États démocratiques voient seuls leur autorité bornée par les droits des individus ; mais si des ambitieux veulent et savent choisir un moment favorable, le mécanisme même de la démocratie peut être utilisé pour supprimer une partie ou la totalité de ces droits ; et une fois ces droits supprimés il n'existe plus aucun moyen légitime de les rétablir, mais seulement la rébellion. Si les hommes qui ont en main à des titres divers l'autorité de l'État cessent de vouloir la démocratie, la crainte de la rébellion peut parfois les forcer, mais aucune loi ne peut les obliger à y rester fidèles. Hors d'Europe, de larges territoires, soumis par la conquête, subissent les volontés de tel ou tel État européen ; et dans le reste du monde, il y a de plus en plus tendance à élaborer des structures politiques plus ou moins copiées sur celle de l'État occidental. Le terme d'une telle évolution, terme heureusement théorique, serait une situation telle que dans tout le globe terrestre chaque être humain obéisse continuellement et exclusivement à l'État dont il est le sujet, sans qu'aucun État obéisse à autre chose qu'à ses propres caprices. Quelle stabilité, quel équilibre, quelle harmonie peut-on espérer trouver dans cette direction, de quelque point de vue qu'on se place ?\par
Il est clair, et M. Scelle notamment a très bien montré, que la notion juridique de la nation souveraine est incompatible avec l'idée d'un ordre inter­national. C'est parce que les individus, même là où ils sont libres, ne disposent pas souverainement d'eux-mêmes et de leurs biens qu'il y a un ordre civil. Tout ordre implique une autorité légitime dont les décisions soient obliga­toires pour ceux qui sont soumis à cet ordre. Mais la souveraineté de chaque nation par rapport à toutes les autres est liée au pouvoir souverain qu'exerce chaque État sur ses sujets. Car, tant que ce pouvoir est intact, toute action prise contre une nation nuisible doit s'exercer sur tous les membres et sujets de cette nation ; elle doit donc prendre la forme de la guerre, de la menace de guerre, ou d'une pression économique qui au-delà d'un certain degré mène inévitablement à la guerre. Comme la guerre est une opération qui ne peut être décidée et menée que par une ou plusieurs nations, et comme nul ne peut du dehors ordonner à une nation de faire la guerre, il ne peut y avoir là un moyen de police pour une autorité qui dominerait les nations. Une telle autorité ne peut s'exercer que si elle possède le pouvoir légitime, c'est-à-dire publique­ment et généralement reconnu, de dispenser en certains cas les citoyens et sujets d'un État du devoir d'obéissance envers cet État. L'État atteint par une telle mesure serait alors frappé du sentiment d'infériorité propre aux rebelles, et non pas ses sujets révoltés. Mais il est clair qu'une telle mesure n'est possible que si au-dessous de l'État et au-dessus des individus il existe des pouvoirs légitimes capables d'exécuter une décision de ce genre. L'ordre international suppose qu'un certain fédéralisme soit établi non seulement entre les nations, mais à l'intérieur de chaque grande nation. À plus forte raison, le lien entre les colonies et leur métropole devrait-il devenir un lien fédéral au lieu d'être un rapport de simple subordination.\par
Les vainqueurs de 1918 ont voulu constituer un ordre international. À cet effet, on a tenté de faire accepter certaines interdictions et certaines obliga­tions communes à toutes les nations ; et sur quelques-unes on a voulu faire peser des obligations particulières, comme celle du désarmement pour l'Alle­magne et celles concernant le traitement des minorités pour certains pays d'Europe centrale. Par une contradiction singulière, on a laissé subsister le dogme de la souveraineté nationale, et l'autorité de chaque État sur ses sujets est restée intacte. Une telle tentative ne pouvait aboutir, et elle n'a pas eu même un commencement de succès. Un pouvoir qui n'est arrêté par aucune borne légitimement imposée tend nécessairement à s'accroître au-dedans et au-dehors ; tout État centralisé et souverain est conquérant et dictatorial en puissance, et devient tel en effet pour autant qu'il croit en avoir la force.\par
On peut imaginer à cette guerre, en admettant que l'Allemagne doive y subir une plus ou moins grande défaite, plusieurs issues possibles. L'Europe peut être ramenée à une situation peu différente de celle où elle se trouvait par exemple en 1930. Les peuples respireraient alors, mais non pas sans doute pour très longtemps. Elle peut tomber par épuisement dans un désordre diffus et sanglant dont la Russie tirerait peut-être profit ; la suite serait alors impossible à prévoir. Le rapport des forces peut changer de telle manière que la puissance de l'Allemagne soit écrasée pour très longtemps par celle des nations victorieuses ; une telle éventualité n'est peut-être pas très probable, mais il faut la supposer telle, puisque beaucoup de personnes réclament dès maintenant à grand bruit des conditions de paix qui impliquent un écrasement durable de l'Allemagne. Comme il est heureusement impossible d'exterminer la totalité ou même une partie considérable du peuple allemand, un tel écrase­ment suppose une contrainte imposée au moment de la victoire et maintenue pendant une longue période de temps. Une coalition ne peut exercer un effort long et continu ; une seule nation devrait s'en charger ; ce serait inévitable­ment une nation continentale, c'est-à-dire la France.\par
Si la France entreprenait un tel effort sans être matériellement et morale­ment capable de le soutenir, une nouvelle guerre, qui peut-être l'anéantirait, serait pour elle le châtiment d'avoir osé au-delà de ses forces. Mais admettons qu'elle en soit capable. Ce qu'elle a de sage, de libre et d'humain ne périrait-il pas inévitablement dans la tension extrême des forces nationales nécessaire à une si grande tâche ? D'autre part, ne lui faudrait-il pas posséder un pouvoir de contrainte qui en ferait la maîtresse de l'Europe ? Certainement, les traditions héritées des Romains, de Richelieu, de Louis XIV et de Napoléon triomphe­raient alors en France. Autrement dit le système hitlérien ne disparaîtrait pas ; il se transporterait chez nous avec ses fins et ses méthodes. Ceux qui en feraient usage ne le reconnaîtraient pas, une fois naturalisé, mais ceux qui en subiraient les effets douloureux ne le reconnaîtraient que trop. Pour l'avenir de l'humanité, pour la civilisation, pour la liberté, une telle victoire ne serait pas beaucoup meilleure que la défaite. La victoire de ceux qui défendent par les armes une cause juste n'est pas nécessairement une victoire juste ; une victoire est plus ou moins juste non pas en fonction de la cause qui a fait prendre les armes, mais en fonction de l'ordre qui s'établit une fois les armes déposées. L'écrasement du vaincu est non seulement toujours injuste, mais aussi toujours funeste à tous, vaincus, vainqueurs et spectateurs, et d'autant plus funeste que le peuple vaincu était plus puissant ; car le déséquilibre qui en résulte est d'autant plus grave.\par
Il est pourtant infiniment souhaitable que la nation allemande soit démem­brée, mais à condition que ce démembrement, quand même il serait accompli par la force, ne doive pas être maintenu par la force. Cela n'est possible que dans un seul cas, à savoir si les vainqueurs, en admettant que nous soyons destinés à l'être, acceptent pour eux-mêmes la transformation qu'ils auraient imposée au vaincu. Aussi longtemps que les hommes continueront à n'avoir entre eux d'autres liens que ceux qui passent par l'État, les États continueront à organiser systématiquement et périodiquement le massacre mutuel de leurs sujets, sans qu'aucune pression de l'opinion, aucun effort de bonne volonté, aucune combinaison internationale puisse éviter un tel destin. La série des massacres doit aboutir ou au triomphe d'un seul État qui réussirait à écraser une multitude de peuples sous une nouvelle paix romaine et se décomposerait lentement par la suite, ou à la destruction mutuelle des États qui finiraient par se briser sous l'effet d'une trop grande tension. De toutes manières, il est inévitable que la transformation accomplie par l'humanité depuis quelques siècles dans le sens de la centralisation soit un jour suivie par une transfor­mation en sens contraire ; car toute chose dans la nature trouve une fois sa limite. Il y a deux espèces d'organisation dispersée que l'histoire nous permet de connaître assez bien, à savoir les petites cités et les liens féodaux. Aucune n'exclut la tyrannie ni la guerre, mais l'une et l'autre est plus féconde et plus favorable aux meilleures formes de la vie humaine que la centralisation subie par les hommes de l'époque romaine et par ceux de notre époque. L'humanité arrivera peut-être à quelque chose d'analogue soit à l'une de ces deux organisations, soit à un mélange des deux, ou peut-être elle en trouvera une troisième. Il se peut que ce temps ne soit pas très lointain et que nous assis­tions en ce moment à l'agonie des États. Malheureusement les États ne peuvent agoniser eux-mêmes sans faire agoniser en même temps beaucoup de choses précieuses et beaucoup d'hommes. Mais la somme des misères et des destructions irréparables dans le domaine de la matière et dans celui de l'esprit serait diminuée, si un nombre suffisant d'hommes responsables pouvaient être assez lucides et assez résolus pour préparer et favoriser méthodiquement la transformation que l'humanité, par bonheur, ne peut éviter en aucun cas.\par

\begin{center}
\end{center}
\subsection[2. Rome et l’Albanie, 1er juillet 1939]{2. \\
Rome et l’Albanie \\
1\textsuperscript{er} juillet 1939}
\noindent \par
Ciano, dans son discours au sujet de l'expédition d'Albanie, déclarait : « Ce qui vient de se passer est le résultat de relations entre l'Italie et l'Albanie qui datent de vingt-deux siècles. » Il se trompe d'un siècle ; c'est il y a presque exactement vingt et un siècles que Rome entra, si l'on peut dire, en rapports avec l'Albanie, où elle ne possédait jusque-là que quelques comptoirs sur la côte ; bien qu'on ne puisse employer le mot d'Albanie que par extension, le hasard fait qu'à cette époque un certain roi Gentius avait sous son autorité le sud de l'Illyrie et le nord de l'Épire, ce qui correspond sensiblement à l'Albanie actuelle. Ce qui est tout à fait juste, c'est le lien qu'établit Ciano entre la manière dont les « relations » commencèrent et les « résultats » que nous avons sous les yeux. On peut en juger d'après Polybe, Tite-Live, Plutarque et Appien. Voici le récit d'Appien ;\par
« Les Romains étaient en guerre contre les Macédoniens, et Persée était alors roi de Macédoine après Philippe. Gentius, roi d'une partie des Illyriens, s'était allié à Persée pour de l'argent, et il avait pénétré dans le territoire de l'Illyrie soumis aux Romains. Les Romains lui ayant envoyé des ambassadeurs, il les fit enchaîner en disant qu'ils ne venaient pas comme ambassadeurs, mais comme espions. Anicius, général romain, prit avec sa flotte quelques bateaux de Gentius, puis le rencontra sur terre, gagna la bataille, et le cerna dans un espace étroit. Gentius implora la paix ; il reçut l'ordre de se rendre aux Romains à discrétion ; il demanda trois jours pour délibérer et les obtint. Mais avant que le délai ne fût écoule, il envoya ses serviteurs prier Anicius de le recevoir, et, tombant à ses genoux, il lui adressa les plus basses supplications. Anicius lui dit d'avoir bon courage, le releva de son attitude prosternée et l'invita à sa table, puis, au sortir du repas, ordonna à ses serviteurs de le mettre sous bonne garde. Par la suite Anicius traîna Gentius en triomphe, à Rome, avec sa femme et ses enfants ; toute cette guerre n'avait duré que vingt jours. Il y eut soixante-dix villes soumises à Gentius auprès desquelles Paul-Émile, le vainqueur de Persée, passa à dessein alors qu'il revenait vers Rome, et cela en vertu d'un ordre secret du Sénat. À ces villes terrifiées il promit de pardonner le passé, si elles lui livraient tout ce qu'elles avaient d'or et d'argent. Elles s'y engagèrent ; il envoya dans chacune d'elles un détachement de son armée, et, ayant fixé un jour avec les chefs de ces détachements, le même pour tous, il leur ordonna de proclamer à l'aube, chacun dans sa ville respective, qu'il fallait apporter l'argent sur la place publique dans l'espace de trois heures ; puis, l'argent une fois rassemblé, de saccager la ville de fond en comble. C'est ainsi que Paul-Émile fit le sac de soixante-dix villes en une heure. »\par
Polybe raconte qu'au cours de ce sac « cent cinquante mille personnes furent vendues comme esclaves ».\par
Ce qui s'est passé, récemment, de toute évidence, est bien le « résultat » de relations ainsi commencées.\par
({\itshape Nouveaux Cahiers}, 1\textsuperscript{er} juillet 1939.)\par

\begin{center}
\end{center}
\subsection[3. Réflexions sur la barbarie, (fragments) 1939(?)]{3. \\
Réflexions sur la barbarie \\
(fragments) \\
1939(?)}
\noindent \par
Bien des gens aujourd'hui, émus par les horreurs de toute espèce que notre époque apporte avec une profusion accablante pour les tempéraments un peu sensibles, croient que, par l'effet d'une trop grande puissance technique, ou d'une espèce de décadence morale, ou pour toute autre cause, nous entrons dans une période de plus grande barbarie que les siècles traversés par l'huma­nité au cours de son histoire. Il n'en est rien. Il suffit, pour s'en convaincre, d'ouvrir n'importe quel texte antique, la Bible, Homère, César, Plutarque. Dans la Bible, les massacres se chiffrent généralement par dizaines de milliers. L'extermination totale, en une journée, sans acception de sexe ni d'âge, d'une ville de quarante mille habitants n'est pas, dans les récits de César, quelque chose d'extraordinaire. D'après Plutarque, Marius se promenait dans les rues de Rome suivi d'une troupe d'esclaves qui abattaient sur-le-champ quiconque le saluait sans qu'il daignât répondre. Sylla, imploré en plein Sénat de bien vouloir au moins déclarer qui il voulait faire mourir, dit qu'il n'avait pas tous les noms présents à l'esprit, mais qu'il les publierait, jour par jour, à mesure qu'ils lui viendraient à la mémoire. Aucun des siècles passés historiquement connus n'est pauvre en événements atroces. La puissance des armements, à cet égard, est sans importance. Pour les massacres massifs, la simple épée, même de bronze, est un instrument plus efficace que l'avion.\par
La croyance contraire, si commune à la fin du XIX\textsuperscript{e} siècle et jusqu'en 1914, c'est-à-dire la croyance en une diminution progressive de la barbarie dans l'humanité dite civilisée, n'est, me semble-il, pas moins erronée. Et l'illusion en pareille matière est dangereuse, car on ne cherche pas à conjurer ce qu'on croit être en voie d'extinction. L'acceptation de la guerre, en 1914, a été ainsi rendue bien plus aisée ; on ne croyait pas qu'elle pût être sauvage, faite par des hommes que l'on croyait exempts de sauvagerie. Comme les personnes qui répètent sans cesse qu'elles sont trop bonnes sont celles dont il faut attendre, à l'occasion, la plus froide et la plus tranquille cruauté, de même, lorsqu'un groupement humain se croit porteur de civilisation, cette croyance même le fera succomber à la première occasion qui pourra se présenter à lui d'agir en barbare. À cet égard, rien n'est plus dangereux que la foi en une race, en une nation, en une classe sociale, en un parti. Aujourd'hui, nous ne pouvons plus avoir dans le progrès la même confiance naïve qu'ont eue nos pères et nos grands-pères ; mais à la barbarie qui ensanglante le monde nous cherchons tous des causes hors du milieu où nous vivons, dans des groupements humains qui nous sont ou que nous affirmons nous être étrangers. Je voudrais proposer de considérer la barbarie comme un caractère permanent et universel de la nature humaine, qui se développe plus ou moins selon que les circonstances lui donnent plus ou moins de jeu.\par
Une telle vue s'accorde parfaitement avec le matérialisme dont les marxistes se réclament ; mais elle ne s'accorde pas avec le marxisme lui-mê­me, qui, dans sa foi messianique, croit qu'une certaine classe sociale est, par une sorte de prédestination, porteuse et unique porteuse de civilisation. Il a cru trouver dans la notion de classe la clef de l'histoire, mais il n'a jamais même commencé à utiliser effectivement cette clef ; aussi bien n'est-elle pas utili­sable. Je ne crois pas que l'on puisse former des pensées claires sur les rapports humains tant qu'on n'aura pas mis au centre la notion de force, comme la notion de rapport est au centre des mathématiques. Mais la première a besoin, comme en a eu besoin la seconde, d'être élucidée. Ce n'est pas aisé.\par
Je proposerais volontiers ce postulat : on est toujours barbare envers les faibles. Ou du moins, pour ne pas nier tout pouvoir à la vertu, on pourrait affirmer que, sauf au prix d'un effort de générosité aussi rare que le génie, on est toujours barbare envers les faibles. Le plus ou moins de barbarie diffuse dans une société dépendrait ainsi de la distribution des forces. Cette vue, si on pouvait l'étudier assez sérieusement pour lui donner un contenu clair, permet­trait au moins en principe de situer toute structure sociale, soit stable, soit passagère, dans une échelle de valeurs, à condition que l'on considère la barbarie comme un mal et son absence comme un bien. Cette restriction est nécessaire ; car il ne manque pas d'hommes qui, soit par une estime exclusive et aristocratique de la culture intellectuelle, soit par ambition, soit par une sorte d'idolâtrie de l'Histoire et d'un avenir rêvé, soit parce qu'ils confondent la fermeté d'âme avec l'insensibilité, soit, enfin, qu'ils manquent d'imagination, s'accommodent fort bien de la barbarie et la considèrent ou comme un détail indifférent ou comme un instrument utile. Ce n'est pas là mon cas ; ce n'est pas non plus, je suppose, le cas de ceux qui lisent cette revue.\par
\par
Pour entrevoir une telle relation entre la carte des forces dans un système social et le degré de la barbarie, il faut considérer cette dernière nation un peu autrement que ne le fait la foule. La sensibilité publique ne s'émeut...\par

\begin{center}
*\end{center}
\noindent Hitler n'est pas un barbare, plût au ciel qu'il en fût un ! Les barbares, dans leurs ravages, n'ont jamais fait que des maux limités. Comme les calamités naturelles, en détruisant, ils réveillent l'esprit rappelé à l'insécurité des choses humaines ; leurs cruautés, leurs perfidies, mêlées d'actes de loyauté et de générosité, tempérées par l'inconstance et le caprice, ne mettent en péril rien de vital chez ceux qui survivent à leurs armes. Seul un État extrêmement civilisé, mais bassement civilisé, si l'on peut s'exprimer ainsi, comme fut Rome, peut amener chez ceux qu'il menace et chez ceux qu'il soumet cette décomposition morale qui non seulement brise d'avance tout espoir de résistance effective, mais rompt brutalement et définitivement la continuité dans la vie spirituelle, lui substituant une mauvaise imitation de médiocres vainqueurs. Car seul un État parvenu à un mode savant d'organisation peut paralyser chez ses adversaires la faculté même de réagir, par l'empire qu'exerce sur l'imagination un mécanisme impitoyable, que ni les faiblesses humaines ni les vertus humaines ne peuvent arrêter dès qu'il s'agit de saisir un avantage, et qui utilise indifféremment à cette fin le mensonge ou la vérité, le respect simulé ou le mépris avoué des conventions. Nous ne sommes pas en Europe dans la situation de civilisés qui luttent contre un barbare, mais dans la position bien plus difficile et plus périlleuse de pays indépendants menacés de colonisation ; et nous ne ferons pas utilement face à ce danger si nous n'inventons pas des méthodes qui y correspondent.\par

\begin{center}
\end{center}
\subsection[4. L’agonie d’une civilisation vue à travers un poème épique, février 1943]{4. \\
L’agonie d’une civilisation vue à travers un poème épique \\
février 1943}
\noindent \par
Quand on compare à l'{\itshape Iliade} les épopées composées au moyen âge en langue française, on sent vivement que les exploits, les souffrances et la mort de quelques guerriers semblent, dans le cadre épique, choses petites et froides. Une civilisation tout entière, naguère en plein essor, frappée soudain d'un coup mortel par la violence des armes, destinée à disparaître sans retour, et représentée dans les dernières palpitations de l'agonie, tel est peut-être le seul thème assez grand pour l'épopée. C'est celui de l'{\itshape Iliade} ; c'est aussi celui d'un fragment d'épopée composé au moyen âge en langue d'oc, et qui constitue la deuxième partie du texte connu sous le nom de {\itshape Chanson de la Croisade contre les Albigeois}. Toulouse en est le centre, comme Troie est le centre de l'{\itshape Iliade.} Certes, on ne peut même pas songer à comparer les deux poèmes pour la langue, la versification, le style, le génie poétique ; pourtant, dans le poème de Toulouse, le véritable accent épique se fait sentir, et les traits poignants n'y sont pas rares. Composé pendant le combat, avant que l'issue n'en fût connue, par un partisan de la ville menacée, ces circonstances le privent de la merveil­leuse poésie qui enveloppe l'{\itshape Iliade}, mais en font un document de grande valeur. L'authenticité du témoignage, que confirme la comparaison avec d'au­tres récits contemporains, est garantie par l'abondance et la minutie des détails, mais surtout par l'accent, par ce mélange de passion et d'impartialité qui fait le ton propre aux grandes œuvres.\par
La civilisation qui constitue le sujet du poème n'a pas laissé d'autres traces que ce poème même, quelques chants de troubadours, de rares textes concer­nant les cathares, et quelques merveilleuses églises. Le reste a disparu ; nous pouvons seulement tenter de deviner ce que fut cette civilisation que les armes ont tuées, dont les armes ont détruit les œuvres. Avec si peu de données, on ne peut espérer qu'en retrouver l'esprit ; c'est pourquoi, si le poème en donne un tableau embelli, il n'en est pas par là un moins bon guide ; car c'est l'esprit même d'une civilisation qui s'exprime dans les tableaux qu'en donnent ses poètes. Ainsi le vers de Virgile : « Toi, Romain, occupe-toi de dominer souve­rainement les peuples » permettrait à lui seul de concevoir l'esprit de la civilisation romaine aussi bien qu'une vaste documentation. Il suffit qu'en lisant le poème de Toulouse, et en évoquant ce que l'on sait d'autre part con­cernant ce temps et ce pays, on fasse un effort d'imagination ; on verra apparaître la ressemblance de ce qui fut.\par
Ce qui frappe tout d'abord dans ce récit d'une guerre religieuse, c'est qu'il n'y est pour ainsi dire pas question de religion. Sans doute Simon de Montfort et ses évêques y parlent trois ou quatre fois des hérétiques ; des évêques, en présence du pape, accusent les comtes de Toulouse et de Foix de les favoriser, et le comte de Foix s'en défend ; les partisans de Toulouse et le poète lui-même, à chaque victoire, se félicitent d'être soutenus par Dieu, le Christ, le Fils de la Vierge, la Trinité. Mais on chercherait vainement quelque autre allusion à des controverses religieuses ; on ne peut guère expliquer ce silence, dans un poème aussi vivant, où palpite toute une ville, qu'en admettant qu'il n'y avait à peu près pas de dissensions religieuses dans la cité et parmi ses défenseurs. Les désastres qui s'abattirent sur ce pays auraient pu porter la population soit à s'en prendre aux cathares comme cause de son malheur et à les persécuter, soit à adopter leur doctrine par haine de l'envahisseur et à regarder les catholiques comme des traîtres. Apparemment ni l'une ni l'autre réaction ne se produisit. Cela est extraordinaire.\par
Soit que l'on veuille louer, blâmer ou excuser les hommes du moyen âge, on croit volontiers aujourd'hui que l'intolérance était une fatalité de leur époque ; comme s'il y avait des fatalités pour les temps et les lieux. Chaque civilisation, comme chaque homme, a la totalité des notions morales à sa disposition, et choisit. Si le père de saint Louis, comme le raconte le poème, crut servir Dieu en autorisant froidement le massacre d'une ville entière après qu'elle se fut rendue, c'est qu'il avait choisi ainsi ; son petit-fils devait plus tard choisir de même, et saint Louis lui-même aussi, lui qui regardait le fer comme un bon moyen, pour des laïques, de régler les controverses religieuses. Ils auraient pu choisir autrement, et la preuve en est que les villes du Midi, au XII\textsuperscript{e} siècle, choisirent autrement. Si l'intolérance l'emporta, c'est seulement parce que les épées de ceux qui avaient choisi l'intolérance furent victorieuses. Ce fut une décision purement militaire. Contrairement à un préjugé très répandu, une décision purement militaire peut influer sur le cours des pensées pendant de longs siècles, sur de vastes espaces ; car l'empire de la force est grand.\par
\par
L'Europe n'a plus jamais retrouvé au même degré la liberté spirituelle perdue par l'effet de cette guerre. Car au XVIII\textsuperscript{e} et au XIX\textsuperscript{e} siècle on élimina seulement de la lutte des idées les formes les plus grossières de la force ; la tolérance alors en faveur contribua même à la constitution de partis cristallisés et substitua aux contraintes matérielles les barrières spirituelles. Mais le poème de Toulouse nous montre, par le silence même qu'il observe à ce sujet, combien le pays d'oc, au XII\textsuperscript{e} siècle, était éloigné de toute lutte d'idées. Les idées ne s'y heurtaient pas, elles y circulaient dans un milieu en quelque sorte continu. Telle est l'atmosphère qui convient à l'intelligence ; les idées ne sont pas faites pour lutter. La violence même du malheur ne put susciter une lutte d'idées dans ce pays ; catholiques et cathares, loin de constituer des groupes distincts, étaient si bien mélangés que le choc d'une terreur inouïe ne put les dissocier. Mais les armes étrangères imposèrent la contrainte, et la conception de la liberté spirituelle qui périt alors ne ressuscita plus.\par
S'il y a un lieu du globe terrestre où un tel degré de liberté puisse être précieux et fécond, c'est le pourtour de la Méditerranée. À qui regarde la carte, la Méditerranée semble destinée à constituer un creuset pour la fusion des traditions venues des pays nordiques et de l'Orient ; ce rôle, elle l'a joué peut-être avant les temps historiques, mais elle ne l'a joué pleinement qu'une fois dans l'histoire, et il en résulta une civilisation dont l'éclat constitue encore aujourd'hui, ou peu s'en faut, notre seule clarté, à savoir la civilisation grec­que. Ce miracle dura quelques siècles et ne se reproduisit plus. Il y a vingt-deux siècles les armes romaines tuèrent la Grèce, et leur domination frappa de stérilité le bassin méditerranéen ; la vie spirituelle se réfugia en Syrie, en Judée, puis en Perse. Après la chute de l'Empire romain, les invasions du Nord et de l'Orient, tout en apportant une vie nouvelle, empêchèrent quelque temps la formation d'une civilisation. Ensuite le souci dominant de l'orthodoxie religieuse mit obstacle aux relations spirituelles entre l'Occident et l'Orient. Quand ce souci disparut, la Méditerranée devint simplement la route par où les armes et les machines de l'Europe allèrent détruire les civilisations et les traditions de l'Orient. L'avenir de la Méditerranée repose sur les genoux des dieux. Mais une fois au cours de ces vingt-deux siècles une civilisation méditerranéenne a surgi qui peut-être aurait avec le temps constitué un second miracle, qui peut-être aurait atteint un degré de liberté spirituelle et de fécondité aussi élevé que la Grèce antique, si on ne l'avait pas tuée.\par
Après le X\textsuperscript{e} siècle, la sécurité et la stabilité étaient devenues suffisantes pour le développement d'une civilisation ; l'extraordinaire brassage accompli depuis la chute de l'Empire romain pouvait dès lors porter ses fruits. Il ne le pouvait nulle part au même degré que dans ce pays d'oc où le génie méditer­ranéen semble s'être alors concentré. Les facteurs d'intolérance constitués en Italie par la présence du pape, en Espagne par la guerre ininterrompue contre les Maures, n'y avaient pas d'équivalent ; les richesses spirituelles y affluaient de toutes parts sans obstacle. La marque nordique est assez visible dans une société avant tout chevaleresque ; l'influence arabe pénétrait facilement dans des pays étroitement liés à l'Aragon ; un prodige incompréhensible fit que le génie de la Perse prit racine dans cette terre et y fleurit, au temps même où il semble avoir pénétré jusqu'en Chine. Ce n'est pas tout peut-être ; ne voit-on pas à Saint-Sernin, à Toulouse, des têtes sculptées qui évoquent l'Égypte ? Les attaches de cette civilisation étaient aussi lointaines dans le temps que dans l'espace. Ces hommes furent les derniers peut-être pour qui l'antiquité était encore chose vivante. Si peu qu'on sache des cathares, il semble clair qu'ils furent de quelque manière les héritiers de la pensée platonicienne, des doctrines initiatiques et des Mystères de cette civilisation pré-romaine qui embrassait la Méditerranée et le Proche-Orient ; et, par hasard ou autrement, leur doctrine rappelle par certains points, en même temps que le bouddhisme, en même temps que Pythagore et Platon, la doctrine des druides qui autrefois avait imprégné la même terre. Quand ils eurent été tués, tout cela devint simple matière d'érudition. Quels fruits une civilisation si riche d'éléments divers a-t-elle portés, aurait-elle portés ? Nous l'ignorons ; on a coupé l'arbre. Mais quelques sculptures peuvent évoquer un monde de merveilles, et rien ne dépasse ce que suggèrent celles des églises romanes du Midi de la France.\par
\par
Le poète de Toulouse sent très vivement la valeur spirituelle de la civilisa­tion attaquée ; il l'évoque continuellement ; mais il semble impuissant à l'exprimer, et emploie toujours les mêmes mots, Prix et Parage, parfois Parage et Merci. Ces mots, sans équivalents aujourd'hui, désignent des valeurs cheva­leresques. Et pourtant c'est une cité, c'est Toulouse qui vit dans le poème, et elle y palpite tout entière, sans aucune distinction de classes. Le comte ne fait rien sans consulter toute la cité, « li cavalier el borgez e la cuminaltatz » , et il ne lui donne pas d'ordres, il lui demande son appui ; cet appui, tous l'accor­dent, artisans, marchands, chevaliers, avec le même dévouement joyeux et complet. C'est un membre du Capitole qui harangue devant Muret l'armée opposée aux croisés ; et ce que ces artisans, ces marchands, ces citoyens d'une ville - on ne saurait leur appliquer le terme de bourgeois -voulaient sauver au prix de leur vie, c'était Joie et Parage, c'était une civilisation chevaleresque.\par
Ce pays qui a accueilli une doctrine si souvent accusée d'être antisociale fut un exemple incomparable d'ordre, de liberté et d'union des classes. L'apti­tude à combiner des milieux, des traditions différentes y a produit des fruits uniques et précieux à l'égard de la société comme de la pensée. Il s'y trouvait ce sentiment civique intense qui a animé l'Italie du moyen âge ; il s'y trouvait aussi une conception de la subordination semblable à celle que T.-E. Lawrence a trouvée vivante en Arabie en 1917, à celle qui, apportée peut-être par les Maures, a imprégné pendant des siècles la vie espagnole. Cette conception, qui rend le serviteur égal au maître par une fidélité volontaire et lui permet de s'agenouiller, d'obéir, de souffrir les châtiments sans rien perdre de sa fierté, apparaît au XIII\textsuperscript{e} siècle dans le {\itshape Poème du Cid}, comme aux XVI\textsuperscript{e} et XVII\textsuperscript{e} siècles dans le théâtre espagnol ;elle entoura la royauté, en Espagne, d'une poésie qui n'eut jamais d'équivalent en France ; étendue même à la subordination imposée par violence, elle ennoblit jusqu'à l'esclavage, et permettait à des Espagnols nobles, pris et vendus comme esclaves en Afrique, de baiser à genoux les mains de leurs maîtres, sans s'abaisser, par devoir et non par lâcheté. L'union d'un tel esprit avec le sentiment civique, un attache­ment également intense à la liberté et aux seigneurs légitimes, voilà ce qu'on n'a peut-être pas vu ailleurs que dans le pays d'oc au XII\textsuperscript{e} siècle. C'est une civilisation de la cité qui se préparait sur cette terre, mais sans le germe funeste des dissensions qui désolèrent l'Italie ; l'esprit chevaleresque fournis­sait le facteur de cohésion que l'esprit civique ne contient pas. De même, malgré certains conflits entre seigneurs, et en l'absence de toute centralisation, un sentiment commun unissait ces contrées ; On vit Marseille, Beaucaire, Avignon, Toulouse, la Gascogne, l'Aragon, la Catalogne, s'unir spontanément contre Simon de Montfort. Plus de deux siècles avant Jeanne d'Arc, le sentiment de la patrie, une patrie qui, bien entendu, n'était pas la France, fut le principal mobile de ces hommes ; et ils avaient mémo un mot pour désigner la patrie ; ils l'appelaient le langage.\par
Rien n'est si touchant dans le poème que l'endroit où la cité libre d'Avignon se soumet volontairement au comte de Toulouse vaincu, dépouillé de ses terres, dépourvu de toute ressource, à peu près réduit à la mendicité. Le comte, averti des intentions d'Avignon, s'y rend ; il trouve les habitants à genoux, qui lui disent : « Tout Avignon se met en votre seigneurie - chacun vous livre son corps et son avoir. » Avec des larmes ils demandent au Christ le pouvoir et la force de le remettre dans son héritage. Ils énumèrent les droits seigneuriaux qu'ils s'engagent désormais à acquitter ; et, après avoir tous prêté serment, ils disent au comte : « Seigneur légitime et aimé - N'ayez aucune crainte de donner et de dépenser - Nous donnerons nos biens et sacrifierons nos corps - Pour que vous recouvriez votre terre ou que nous mourions avec vous. » Le comte, en les remerciant, leur dit que leur langage leur saura gré de cette action. Peut-on imaginer, pour des hommes libres, une manière plus généreuse de se donner un maître ? Cette générosité fait voir à quel point l'esprit chevaleresque avait imprégné toute la population des villes.\par
Il en était tout autrement dans les pays d'où provenaient les vainqueurs de cette guerre ; là, il y avait non pas union, mais lutte entre l'esprit féodal et l'esprit des villes. Une barrière morale y séparait nobles et roturiers. Il devait en résulter, une fois le pouvoir des nobles épuisé, ce qui se produisit en effet, à savoir l'avènement d'une classe absolument ignorante des valeurs chevale­resques ; un régime où l'obéissance devenait chose achetée et vendue ; les conflits de classes aigus qui accompagnent nécessairement une obéissance dépouillée de tout sentiment de devoir, obtenue uniquement par les mobiles les plus bas. Il ne peut y avoir d'ordre que là où le sentiment d'une autorité légitime permet d'obéir sans s'abaisser ; c'est peut-être là ce que les hommes d'oc nommaient Parage. S'ils avaient été vainqueurs, qui sait si le destin de l'Europe n'aurait pas été bien différent ? La noblesse aurait pu alors disparaître sans entraîner l'esprit chevaleresque dans son désastre, puisqu'en pays d'oc les artisans et les marchands y avaient part. Ainsi à notre époque encore nous souffrons tous et tous les jours des conséquences de cette défaite.\par
L'impression dominante que laisse le tableau de ces populations, tel qu'on le trouve dans la {\itshape Chanson de la Croisade}, c'est l'impression de bonheur. Quel coup dut être pour elles le premier choc de la terreur, quand, dès la première bataille, la cité entière de Béziers fut massacrée froidement ! Ce coup les fit plier ; il avait été infligé à cet effet. Il ne leur fut pas permis de s'en relever ; les atrocités se succédèrent. Il se produisit des effets de panique très favora­bles aux agresseurs. La terreur est une arme à un seul tranchant. Elle a toujours bien plus de prise sur ceux qui songent à conserver leur liberté et leur bonheur que sur ceux qui songent à détruire et à écraser ; l'imagination des premiers est bien plus vulnérable, et c'est pourquoi, la guerre étant, avant tout, affaire d'imagination, il y a presque toujours quelque chose de désespéré dans les luttes que livrent des hommes libres contre des agresseurs. Les gens d'oc subirent défaite après défaite : tout le pays fut soumis. S'il faut croire le poète, Toulouse, ayant prêté serment à Simon de Montfort, sur le conseil du comte de Toulouse lui-même, après la défaite de Muret, ne songea pas à violer sa parole ; et sans doute les vainqueurs auraient pu s'appuyer sur l'esprit de fidélité qui dans ces pays accompagnait toujours l'obéissance. Mais ils traitèrent les populations conquises en ennemies, et ces hommes, accoutumés à obéir par devoir et noblement, furent contraints d'obéir par crainte et dans l'humiliation.\par
Quand Simon de Montfort eut fait sentir aux habitants de Toulouse qu'il les regardait en ennemis malgré leur soumission, ils prirent les armes ; mais ils les déposèrent aussitôt et se mirent à sa merci, poussés par leur évêque qui promettait de les protéger. C'était un piège ; les principaux habitants furent enchaînés, frappés et chassés avec une brutalité telle que plusieurs en moururent ; la ville fut entièrement désarmée, dépouillée de tous ses biens, argent, étoffes et vivres, et en partie démolie. Mais, tout lien de fidélité étant dès lors rompu, il suffit que le seigneur légitime pénétrât dans Toulouse avec quelques chevaliers pour que cette population écrasée et sans armes se soulevât. Elle remporta des victoires répétées sur un ennemi puissamment armé et enflé par ses triomphes ; tant le courage, lorsqu'il procède du déses­poir, est parfois efficace contre un armement supérieur. Selon le mot de Simon de Montfort, les lièvres se retournèrent alors contre les lévriers. Au cours d'un de ces combats, une pierre lancée par la main d'une femme tua Simon de Montfort ; puis la ville osa se mettre en défense contre le fils du roi de France, arrivé avec une nombreuse armée. Le poème s'achève là, et sur un cri d'espoir. Mais cet espoir ne devait être réalisé qu'en partie. Toulouse échappa à l'anéantissement ; mais le pays ne devait pas échapper à la conquête ; Prix et Parage devaient disparaître. Par la suite, le destin de ce pays eut longtemps encore quelque chose de tragique. Un siècle et demi plus tard, un oncle de Charles VI le traitait en pays conquis, avec tant de cruauté que quarante mille hommes s'enfuirent en Aragon. Il eut encore des frémissements à l'occasion des guerres religieuses, des luttes contre Richelieu, et fut maintes fois ravagé ; l'exécution du duc de Montmorency, mis à mort à Toulouse parmi la vive douleur de la population, en marque la soumission définitive. Mais à ce moment, ce pays, depuis longtemps déjà, n'avait plus d'existence véritable ; la langue d'oc avait disparu comme langue de civilisation, et le génie de ces lieux, bien qu'il ait influé sur le développement de la culture française, n'a jamais trouvé d'expression propre après le XIII\textsuperscript{e} siècle.\par
En ce cas comme en plusieurs autres, l'esprit reste frappé de stupeur en comparant la richesse, la complexité, la valeur de ce qui a péri avec les mobiles et le mécanisme de la destruction. L'Église cherchait à obtenir l'unité religieuse ; elle mit en action le ressort le plus simple, en promettant le pardon des péchés aux combattants et le salut inconditionnel à ceux qui tomberaient. La licence constitue le grand attrait de toutes les luttes armées ; quelle puis­sante ivresse doit être la licence poussée à ce degré, l'impunité et même l'approbation assurées dans ce monde et dans l'autre à n'importe quel degré de cruauté et de perfidie ! On voit, il est vrai, dans le poème, certains croisés refuser de croire au salut automatique qui leur est promis ; mais ces éclairs de lucidité étaient trop rares pour être dangereux. La nature du stimulant employé par les hommes d'église les obligeait à exercer une pression continuelle dans le sens de la plus grande cruauté ; cette pression excitait le courage des croisés et abattait celui des populations. La perfidie autorisée par l'Église était aussi une arme précieuse. Mais cette guerre ne pouvait se prolonger qu'en devenant une guerre de conquête. On eut du mal d'abord à trouver quelqu'un qui consentît à prendre en charge Carcassonne ; enfin Simon de Montfort, homme alors relativement obscur et pauvre, accepta cette responsabilité, mais il entendit naturellement être payé de ses peines par un gain tangible. Ainsi le chantage au salut et l'esprit d'acquisition d'un homme assez ordinaire, il n'en fallut pas plus pour détruire un monde. Car une conception du monde qui vivait en ces lieux fut alors anéantie pour toujours.\par
Rien qu'en regardant cette terre, et quand même on n'en connaîtrait pas le passa on y voit la marque d'une blessure. Les fortifications de Carcassonne, si visiblement faites pour la contrainte, les églises dont une moitié est romane, et l'autre d'une architecture gothique si visiblement importée, ce sont des specta­cles qui parlent. Ce pays a souffert la force. Ce qui a été tué ne peut jamais ressusciter ; mais la piété conservée à travers les âges permet un jour d'en faire surgir l'équivalent, quand se présentent des circonstances favorables. Rien n'est plus cruel envers le passé que le lieu commun selon lequel la force est impuissante à détruire les valeurs spirituelles ; en vertu de cette opinion, on nie que les civilisations effacées par la violence des armes aient jamais existé ; on le peut sans craindre le démenti des morts. On tue ainsi une seconde fois ce qui a péri, et on s'associe à la cruauté des armes. La piété commande de s'attacher aux traces, même rares, des civilisations détruites, pour essayer d'en concevoir l'esprit. L'esprit de la civilisation d'oc au XII\textsuperscript{e} siècle, tel que nous pouvons l'entrevoir, répond à des aspirations qui n'ont pas disparu et que nous ne devons pas laisser disparaître, même si nous ne pouvons pas espérer les satisfaire.\par
{\itshape (Le Génie d'Oc}, février 1943.)\par

\begin{center}
\end{center}
\subsection[5. En quoi consiste l’inspiration occitanienne ? février 1943]{5. \\
En quoi consiste l’inspiration occitanienne ? \\
février 1943}
\noindent \par
Pourquoi s'attarder au passé, et non s'orienter vers l'avenir ? De nos jours, pour la première fois depuis des siècles, on se porte à la contemplation du passé. Est-ce parce que nous sommes fatigués et proches du désespoir ? Nous le sommes ; mais la contemplation du passé a un meilleur fondement.\par
Depuis plusieurs siècles, nous avions vécu sur l'idée de progrès. Aujour­d'hui, la souffrance a presque arraché cette idée hors de notre sensibilité. Ainsi nul voile n'empêche de reconnaître qu'elle n'est pas fondée en raison. On l'a crue liée à la conception scientifique du monde, alors que la science lui est contraire tout comme la philosophie authentique. Celle-ci enseigne, avec Platon, que l'imparfait ne peut pas produire du parfait ni le moins bon du meilleur. L'idée de progrès, c'est l'idée d'un enfantement par degrés, au cours du temps, du meilleur par le moins bon. La science montre qu'un accroisse­ment d'énergie ne peut venir que d'une source extérieure d'énergie ; qu'une transformation d'énergie inférieure en énergie supérieure ne se produit que comme contre-partie d'une transformation au moins équivalente d'une énergie supérieure en énergie inférieure. Toujours le mouvement descendant est la condition du mouvement montant. Une loi analogue régit les choses spiri­tuelles. Nous ne pouvons pas être rendus meilleurs, sinon par l'influence sur nous de ce qui est meilleur que nous.\par
Ce qui est meilleur que nous, nous ne pouvons pas le trouver dans l'avenir. L'avenir est vide et notre imagination le remplit. La perfection que nous ima­ginons est à notre mesure ; elle est exactement aussi imparfaite que nous-mêmes ; elle n'est pas d'un cheveu meilleure que nous. Nous pouvons la trouver dans le présent, mais confondue avec le médiocre et le mauvais ; et notre faculté de discrimination est imparfaite comme nous-mêmes. Le passé nous offre une discrimination déjà en partie opérée. Car de même que ce qui est éternel est seul invulnérable au temps, de même aussi le simple écoule­ment du temps opère une certaine séparation entre ce qui est éternel et ce qui ne l'est pas. Nos attachements et nos passions opposent à la faculté de discri­miner l'éternel des ténèbres moins épaisses pour le passé que pour le présent. Il en est ainsi surtout du passé temporellement mort et qui ne fournit aucune sève aux passions.\par
Rien ne vaut la piété envers les patries mortes. Personne ne peut avoir l'espoir de ressusciter ce pays d'Oc. On l'a, par malheur, trop bien tué. Cette piété ne menace en rien l'unité de la France, comme certains en ont exprimé la crainte. Quand même on admettrait qu'il est permis de voiler la vérité quand elle est dangereuse pour la patrie, ce qui est au moins douteux, il n'y a pas ici de telle nécessité. Ce pays, qui est mort et qui mérite d'être pleuré, n'était pas la France. Mais l'inspiration que nous pouvons y trouver ne concerne pas le découpage territorial de l'Europe. Elle concerne notre destinée d'hommes.\par
Hors d'Europe, il est des traditions millénaires qui nous offrent des riches­ses spirituelles inépuisables. Mais le contact avec ces richesses doit moins nous engager à essayer de les assimiler telles quelles, sinon pour ceux qui en ont particulièrement la vocation, que nous éveiller à la recherche de la source de spiritualité qui nous est propre ; la vocation spirituelle de la Grèce antique est la vocation même de l'Europe, et c'est elle qui, au XII\textsuperscript{e} siècle, a produit des fleurs et des fruits sur ce coin de terre où nous nous trouvons.\par
Chaque pays de l'antiquité pré-romaine a eu sa vocation, sa révélation orientée non pas exclusivement, mais principalement vers un aspect de la vérité surnaturelle. Pour Israël ce fut l'unité de Dieu, obsédante jusqu'à l'idée fixe. Nous ne pouvons plus savoir ce que ce fut pour la Mésopotamie. Pour la Perse, ce fut l'opposition et la lutte du bien et du mal. Pour l'Inde, l'identifi­cation, grâce à l'union mystique, de Dieu et de l'âme arrivée à l'état de perfection. Pour la Chine, l'opération propre de Dieu, la non action divine qui est plénitude de l'action, l'absence divine qui est plénitude de la présence. Pour l'Égypte, ce fut la charité du prochain, exprimée avec une pureté qui n'a jamais été dépassée ; ce fut surtout la félicité immortelle des âmes sauvées après une vie juste, et le salut par l'assimilation à un Dieu qui avait vécu, avait souffert, avait péri de mort violente, était devenu dans l'autre monde le juge et le sauveur des âmes. La Grèce reçut le message de l'Égypte, et elle eut aussi sa révélation propre : ce fut la révélation de la misère humaine, de la transcen­dance de Dieu, de la distance infinie entre Dieu et l'homme.\par
Hantée par cette distance, la Grèce n'a travaillé qu'à construire des ponts. Toute sa civilisation en est faite. Sa religion des Mystères, sa philosophie, son art merveilleux, cette science qui est son invention propre et toutes les bran­ches de la science, tout cela, ce furent des ponts entre Dieu et l'homme. Sauf le premier, nous avons hérité de tous ces ponts. Nous en avons beaucoup surélevé l'architecture. Mais nous croyons maintenant qu'ils sont faits pour y habiter. Nous ne savons pas qu'ils sont là pour qu'on y passe ; nous ignorons, si l'on y passait, qui l'on trouverait de l'autre côté.\par
Les meilleurs parmi les Grecs ont été habités par l'idée de médiation entre Dieu et l'homme, de médiation dans le mouvement descendant par lequel Dieu va chercher l'homme. C'est cette idée qui s'exprimait dans leur notion d'harmonie, de proportion, laquelle est au centre de toute leur pensée, de tout leur art, de toute leur science, de toute leur conception de la vie. Quand Rome se mit à brandir son glaive, la Grèce avait seulement commencé d'accomplir sa vocation de bâtisseuse de ponts.\par
Rome détruisit tout vestige de vie spirituelle en Grèce, comme dans tous les pays qu'elle soumit et réduisit à la condition de provinces. Tous sauf un seul. Contrairement à celle des autres pays, la révélation d'Israël avait été essentiellement collective, et par là même beaucoup plus grossière, mais aussi beaucoup plus solide ; seule elle pouvait résister à la pression de la terreur romaine. Protégé par cette carapace, couva un peu d'esprit grec qui avait survécu sur le bord oriental de la Méditerranée. Ainsi, après trois siècles désertiques, parmi la soif ardente de tant de peuples, jaillit la source parfaite­ment pure. L'idée de médiation reçut la plénitude de la réalité, le pont parfait apparut, la Sagesse divine, comme Platon l'avait souhaité, devint visible aux yeux. La vocation grecque trouva ainsi sa perfection en devenant la vocation chrétienne.\par
Cette filiation, et par suite aussi la mission authentique du christianisme, fut longtemps empêchée d'apparaître. D'abord par le milieu d'Israël et par la croyance à la fin imminente du monde, croyance d'ailleurs indispensable à la diffusion du message. Bien plus encore ensuite par le statut de religion officielle de l'Empire romain. La Bête était baptisée, mais le baptême en fut souillé. Les Barbares vinrent heureusement détruire la Bête et apporter un sang jeune et frais avec des traditions lointaines. À la fin du X\textsuperscript{e} siècle la stabilité, la sécurité furent retrouvées, les influences de Byzance et de l'Orient circulèrent librement. Alors apparut la civilisation romane. Les églises, les sculptures, les mélodies grégoriennes de cette époque, les quelques fresques qui nous restent du X\textsuperscript{e} et du XI\textsuperscript{e} siècle, sont seules à être presque équivalentes à l'art grec en majesté et en pureté. Ce fut la véritable Renaissance. L'esprit grec renaquit sous la forme chrétienne qui est sa vérité.\par
Quelques siècles plus tard eut lieu l'autre Renaissance, la fausse, celle que nous nommons aujourd'hui de ce nom. Elle eut un point d'équilibre où l'unité des deux esprits fut pressentie. Mais très vite elle produisit l'humanisme, qui consiste à prendre les ponts que la Grèce nous a légués comme habitations permanentes. On crut pouvoir se détourner du christianisme pour se tourner vers l'esprit grec, alors qu'ils sont au même lieu. Depuis lors la part du spiri­tuel dans la vie de l'Europe n'a fait que diminuer pour arriver presque au néant. Aujourd'hui la morsure du malheur nous fait prendre en dégoût l'évolution dont la situation présente est le terme. Nous injurions et voulons rejeter cet humanisme qu'ont élaboré la Renaissance, le XVIII\textsuperscript{e} siècle et la Révolution. Mais par là, loin de nous élever, nous abandonnons la dernière pâle et confuse image que nous possédions de la vocation surnaturelle de l'homme.\par
Notre détresse présente a sa racine dans cette fausse Renaissance. Entre la vraie et la fausse, que s'était-il passé ?\par
Beaucoup de crimes et d'erreurs. Le crime décisif a peut-être été le meurtre de ce pays occitanien sur la terre duquel nous vivons. Nous savons qu'il fut à plusieurs égards le centre de la civilisation romane. Le moment où il a péri est aussi celui où la civilisation romane a pris fin.\par
Il y avait encore alors un lien vivant avec les traditions millénaires que de nouveau aujourd'hui nous essayons de découvrir avec peine, celles de l'Inde, de la Perse, de l'Égypte, de la Grèce, d'autres encore peut-être. Le XIII\textsuperscript{e} siècle coupa le lien. Il y avait ouverture à tous les courants spirituels du dehors. Si déplorables qu'aient été les croisades, du moins elles s'accompagnèrent réellement d'un échange mutuel d'influences entre les combattants, échange où même la part des Arabes fut plus grande que celle de la chrétienté. Elles ont été ainsi infiniment supérieures à nos guerres colonisatrices modernes. À partir du XIII\textsuperscript{e} siècle l'Europe se replia sur elle-même et bientôt ne sortit plus du territoire de son continent que pour détruire. Enfin, il y avait les germes de ce que nous nommons aujourd'hui notre civilisation. Ces germes furent ensui­te enfouis jusqu'à la Renaissance. Et tout cela, le passé, l'extérieur, l'avenir, était tout enveloppé de la lumière surnaturelle du christianisme. Le surnaturel ne se mélangeait pas au profane, ne l'écrasait pas, ne cherchait pas à le supprimer. Il le laissait intact et par là même demeurait pur. Il en était l'origine et la destination.\par
Le moyen âge gothique, qui apparut après la destruction de la patrie occitanienne, fut un essai de spiritualité totalitaire. Le profane comme tel n'avait pas droit de cité. Ce manque de proportion n'est ni beau ni juste ; une spiritualité totalitaire est par là même dégradée. Ce n'est pas là la civilisation chrétienne. La civilisation chrétienne, c'est la civilisation romane, prématuré­ment disparue après un assassinat. Il est infiniment douloureux de penser que les armes de ce meurtre étaient maniées par l'Église. Mais ce qui est doulou­reux est parfois vrai. Peut-être en ce début du XIII\textsuperscript{e} siècle la chrétienté a-t-elle eu un choix à faire. Elle a mal choisi. Elle a choisi le mal. Ce mal a porté des fruits, et nous sommes dans le mal. Le repentir est le retour à l'instant qui a précédé le mauvais choix.\par
L'essence de l'inspiration occitanienne est identique à celle de l'inspiration grecque. Elle est constituée par la connaissance de la force. Cette connais­sance n'appartient qu'au courage surnaturel. Le courage surnaturel enferme tout ce que nous nommons courage et, en plus, quelque chose d'infiniment plus précieux. Mais les lâches prennent le courage surnaturel pour de la faiblesse d'âme. Connaître la force, c'est, la reconnaissant pour presque abso­lument souveraine en ce monde, la refuser avec dégoût et mépris. Ce mépris est l'autre face de la compassion pour tout ce qui est exposé aux blessures de la force.\par
Ce refus de la force a sa plénitude dans la conception de l'amour. L'amour courtois du pays d'oc est la même chose que l'amour grec, quoique le rôle si différent joué par la femme cache cette identité. Mais le mépris de la femme n'était pas ce qui portait les Grecs à honorer l'amour entre hommes, aujour­d'hui chose basse et vile. Ils honoraient pareillement l'amour entre femmes, comme on voit dans le {\itshape Banquet} de Pluton et par l'exemple de Sapho. Ce qu'ils honoraient ainsi, ce n'était pas autre chose que l'amour impossible. Par suite, ce n'était pas autre chose que la chasteté. Par la trop grande facilité des mœurs, il n'y avait presque aucun obstacle à la jouissance dans le commerce entre hommes et femmes, au lieu que la honte empêchait toute âme bien orientée de songer à une jouissance que les Grecs eux-mêmes nommaient contre nature. Quand le christianisme et la grande pureté de mœurs importée par les peuplades germaniques eurent mis entre l'homme et la femme la barrière qui manquait en Grèce, ils devinrent l'un pour l'autre objet d'amour platonique. Le lien sacré du mariage tint lieu de l'identité des sexes. Les troubadours authen­tiques n'avaient pas plus de goût pour l'adultère que Sapho et Socrate pour le vice ; il leur fallait l'amour impossible. Aujourd'hui nous ne pouvons penser l'amour platonique que sous la forme de l'amour courtois, mais c'est bien le même amour.\par
L'essence de cet amour est exprimée par quelques lignes merveilleuses du {\itshape Banquet} : « Le principal, c'est que l'Amour ne fait ni ne subit aucune injustice, ni parmi les dieux, ni parmi les hommes. Car il ne souffre pas par force, quoi qu'il ait à souffrir, car la force n'atteint pas l'Amour. Et quand il agit, il n'agit pas par force ; car chacun volontiers obéit en tout à l'Amour. Un accord consenti de part et d'autre est juste, disent les lois de la cité royale. »\par
\par
Tout ce qui est soumis au contact de la force est avili, quel que soit le contact. Frapper ou être frappé, c'est une seule et même souillure. Le froid de l'acier est pareillement mortel à la poignée et à la pointe. Tout ce qui est exposé au contact de la force est susceptible de dégradation. Toutes choses en ce monde sont exposées au contact de la force, sans aucune exception, sinon celle de l'amour. Il ne s'agit pas de l'amour naturel, comme celui de Phèdre et d'Arnolphe, qui est esclavage et tend à la contrainte. C'est l'amour surnaturel, celui qui dans sa vérité va tout droit vers Dieu, qui en redescend tout droit, uni à l'amour que Dieu porte à sa création, qui directement ou indirectement s'adresse toujours au divin.\par
L'amour courtois avait pour objet un être humain ; mais il n'est pas une convoitise. Il n'est qu'une attente dirigée vers l'être aimé et qui en appelle le consentement. Le mot de merci par lequel les troubadours désignaient ce consentement est tout proche de la notion de grâce. Un tel amour dans sa plénitude est amour de Dieu à travers l'être aimé. Dans ce pays comme en Grèce, l'amour humain fut un des ponts entre l'homme et Dieu.\par
La même inspiration resplendit dans l'art roman. L'architecture, quoique ayant emprunté une forme à Rome, n'a aucun souci de la puissance ni de la force, mais uniquement de l'équilibre ; au lieu qu'il y a quelque souillure de force et d'orgueil dans l'élan des flèches gothiques et la hauteur des voûtes ogivales. L'église romane est suspendue comme une balance autour de son point d'équilibre, un point d'équilibre qui ne repose que sur le vide et qui est sensible sans que rien en marque l'emplacement. C'est ce qu'il faut pour enclore cette croix qui fut une balance où le corps du Christ fut le contrepoids de l'univers. Les êtres sculptés ne sont jamais des personnages ; ils ne sem­blent jamais représenter ; ils ne savent pas qu'on les voit. Ils se tiennent d'une manière dictée seulement par le sentiment et par la proportion architecturale. Leur gaucherie est une nudité. Le chant grégorien monte lentement, et au moment qu'on croit qu'il va prendre de l'assurance, le mouvement montant est brisé et abaissé ; le mouvement montant est continuellement soumis au mouvement descendant. La grâce est la source de tout cet art.\par
La poésie occitanienne, dans ses quelques réussites sans défaut, a une pureté comparable à celle de la poésie grecque. La poésie grecque exprimait la douleur avec une pureté telle qu'au fond de l'amertume sans mélange resplen­dissait la parfaite sérénité. Quelques vers des troubadours ont su exprimer la joie d'une manière si pure qu'à travers elle transparaît la douleur poignante, la douleur inconsolable de la créature finie.\par

\begin{quoteblock}
 \noindent Quand je vois l’alouette mouvoir \\
De joie ses ailes contre le rayon, \\
Comme elle ne se connaît plus et se laisse tomber \\
Par la douceur qui au cœur lui va...
 \end{quoteblock}

\noindent Quand ce pays eut été détruit, la poésie anglaise reprit la même note, et rien dans les langues modernes d'Europe n'a l'équivalent des délices qu'elle enferme.\par
Les Pythagoriciens disaient que l'harmonie ou la proportion est l'unité des contraires en tant que contraires. Il n'y a pas harmonie là où l'on fait violence aux contraires pour les rapprocher ; non plus là où on les mélange ; il faut trouver le point de leur unité. Ne jamais faire de violence à sa propre âme ; ne jamais chercher ni consolation ni tourment ; contempler la chose, quelle qu'elle soit, qui suscite une émotion, jusqu'à ce que l'on parvienne au point secret où douleur et joie, à force d'être pures, sont une seule et même chose ; c'est la vertu même de la poésie.\par
Dans ce pays la vie publique procédait aussi du même esprit. [Il aimait la liberté \footnote{ Cette phrase manque dans le texte imprimé du {\itshape Génie d'oc.} Il a paru nécessaire de l'ajouter. (Note de l'éditeur.)}.] Il n'aimait pas moins l'obéissance. L'unité de ces deux contraires, c'est l'harmonie pythagoricienne dans la société. Mais il ne peut y avoir d'harmonie qu'entre choses pures.\par
La pureté dans la vie publique, c'est l'élimination poussée le plus loin possible de tout ce qui est force, c'est-à-dire de tout ce qui est collectif, de tout ce qui procède de la bête sociale, comme Platon l'appelait. La Bête sociale a seule la force. Elle l'exerce comme foule ou la dépose dans des hommes ou un homme. Mais la loi comme telle n'a pas de force ; elle n'est qu'un texte écrit, elle qui est l'unique rempart de la liberté. L'esprit civique conforme à cet idéal grec dont Socrate fut un martyr est parfaitement pur. Un homme, quel qu'il soit, considéré simplement comme un homme, est aussi tout à fait dépourvu de force. Si on lui obéit en cette qualité, l'obéissance est parfaitement pure. Tel est le sens de la fidélité personnelle dans les rapports de subordination ; elle laisse la fierté tout à fait intacte. Mais quand on exécute les ordres d'un homme en tant que dépositaire d'une puissance collective, que ce soit avec ou sans amour, on se dégrade. Théophile de Viau encore, grand poète et à plu­sieurs égards héritier authentique de la tradition occitanienne, comprenait comme elle le dévouement à un roi ou à un maître. Mais quand Richelieu, dans son travail d'unification, eut tué en France tout ce qui n'était pas Paris, cet esprit disparut complètement. Louis XIV imposait à ses sujets une soumis­sion qui ne mérite pas le beau nom d'obéissance.\par
Dans la Toulouse du début du XIII\textsuperscript{e} siècle la vie sociale était sans doute souillée, comme partout et toujours. Mais du moins l'inspiration, faite unique­ment d'esprit civique et d'obéissance, était pure. Chez ceux qui l'attaquèrent victorieusement, l'inspiration même était souillée.\par
Nous ne pouvons pas savoir s'il y aurait eu une science romane. En ce cas sans doute elle aurait été à la nôtre ce qu'est le chant grégorien à Wagner. Les Grecs, chez qui ce que nous appelons notre science est né, la regardaient comme issue d'une révélation divine et destinée à conduire l'âme vers la contemplation de Dieu. Elle s'est écartée de cette destination, non par excès, mais par insuffisance d'esprit scientifique, d'exactitude et de rigueur. La science est une exploration de tout ce qu'il apparaît d'ordre dans le monde à l'échelle de notre organisme physique et mental. À cette échelle seulement, car ni les télescopes, ni les microscopes, ni les notations mathématiques les plus vertigineuses, ni aucun procédé quel qu'il soit ne permet d'en sortir. La science n'a donc pas d'autre objet que l'action du Verbe, ou, comme disaient les Grecs, de l'Amour ordonnateur. Elle seule, et seulement dans sa plus pure rigueur, peut donner un contenu précis à la notion de providence, et dans le domaine de la connaissance elle ne peut rien d'autre. Comme l'art elle a pour objet la beauté. La beauté romane aurait pu resplendir aussi dans la science.\par
Le besoin de pureté du pays occitanien trouva son expression extrême dans la religion cathare, occasion de son malheur. Comme les cathares sem­blent avoir pratiqué la liberté spirituelle jusqu'à l'absence de dogmes, ce qui n'est pas sans inconvénients, il fallait sans aucun doute qu'hors de chez eux le dogme chrétien fût conservé par l'Église, dans son intégrité, comme un dia­mant, avec une rigueur incorruptible. Mais avec un peu plus de foi, on n'aurait pas cru que pour cela leur extermination à tous fût nécessaire.\par
Ils poussèrent l'horreur de la force jusqu'à la pratique de la non-violence et jusqu'à la doctrine qui fait procéder du mal tout ce qui est du domaine de la force, c'est-à-dire tout ce qui est charnel et tout ce qui est social. C'était aller loin, mais non pas plus loin que l'Évangile. Car il est deux paroles de l'Évangile qui vont aussi loin qu'il soit possible d'aller. L'une concerne les eunuques qui se sont faits eunuques eux-mêmes à cause du royaume des cieux. L'autre est celle que le diable adresse au Christ en lui montrant les royaumes de la terre : « Je te donnerai toute cette puissance et la gloire qui y est attachée, car elle m'a été abandonnée, à moi et à quiconque il me plaît d'en faire part. »\par
\par
L'esprit de cette époque a reparu et s'est développé depuis la Renaissance jusqu'à nos jours, mais avec le surnaturel en moins ; privé de la lumière qui nourrit, il s'est développé comme peut le faire une plante sans chlorophylle. Aujourd'hui cet égarement que la {\itshape Bhagavat-Gîta} nommait l'égarement des contraires nous pousse à chercher le contraire de l'humanisme. Certains cher­chent ce contraire dans l'adoration de la force, du collectif, de la Bête sociale ; d'autres dans un retour au moyen âge gothique. L'un est possible et même facile, mais c'est le mal ; l'autre n'est pas non plus désirable, et d'ailleurs est tout à fait chimérique, car nous ne pouvons pas faire que nous n'ayons été élevés dans un milieu constitué presque exclusivement de valeurs profanes. Le salut serait d'aller au lieu pur où les contraires sont un.\par
Si le XIII\textsuperscript{e} siècle avait lu Platon, il n'aurait pas nommé lumières des connaissances et des facultés simplement naturelles. L'image de la caverne fait manifestement apercevoir que l'homme a pour condition naturelle les ténèbres, qu'il y naît, qu'il y vit et qu'il y meurt s'il ne se tourne pas vers une lumière qui descend d'un lieu situé de l'autre côté du ciel. L'humanisme n'a pas eu tort de penser que la vérité, la beauté, la liberté, l'égalité sont d'un prix infini, mais de croire que l'homme peut se les procurer sans la grâce.\par
Le mouvement qui détruisit la civilisation romane amena plus tard comme réaction l'humanisme. Arrivés au terme de ce second mouvement, allons-nous continuer cette oscillation monotone et où nous descendons à chaque fois beaucoup plus bas ? N'allons-nous pas tourner nos regards vers le point d'équilibre ? En remontant le cours de l'histoire, nous ne rencontrons pas le point d'équilibre avant le XII\textsuperscript{e} siècle.\par
Nous n'avons pas à nous demander comment appliquer à nos conditions actuelles d'existence l'inspiration d'un temps si lointain. Dans la mesure où nous contemplerons la beauté de cette époque avec attention et amour, dans cette mesure son inspiration descendra en nous et rendra peu à peu impossible une partie au moins des bassesses qui constituent l'air que nous respirons.\par
({\itshape Le Génie d'Oc}, février 1943.)\par

\begin{center}
Simone Weil : Écrits historiques et politiques.\end{center}

\begin{center}
Première partie : Histoire\end{center}
\subsection[6. Un soulèvement prolétarien à Florence au XIVe siècle, 1934]{6. \\
Un soulèvement prolétarien à Florence au XIVe siècle \\
1934}
\noindent La fin du XIV\textsuperscript{e} siècle fut, d'une manière générale, en Europe, une période de troubles sociaux et de soulèvements populaires. Les pays où le mouvement fut le plus violent furent ceux qui se trouvaient être économiquement les plus avancés, c'est-à-dire la Flandre et l'Italie ; à Florence, ville des gros marchands drapiers et des manufactures de laine, il prit la forme d'une véritable insurrec­tion prolétarienne, qui fut un moment victorieuse. Cette insurrection, connue sous le nom de soulèvement des {\itshape Ciompi}, est sans doute l'aînée des insurrec­tions prolétariennes. Elle mérite d'autant plus d'être étudiée à ce titre qu'elle présente déjà, avec une pureté remarquable, les traits spécifiques que l'on retrouve plus tard dans les grands mouvements de la classe ouvrière, alors à peine constituée, et qui apparaît ainsi comme ayant été un facteur révolution­naire dès son apparition.\par
La Florence du XIV\textsuperscript{e} siècle est en apparence un État corporatif. Depuis les {\itshape ordinamenti di giustizia} de 1293, le pouvoir est aux mains des arts, c'est-à-dire des corporations. Un art est soit une corporation, soit, plus fréquemment, une union de corporations qui forme un petit État dans l'État, avec des chefs élus dont les pouvoirs s'étendent à la juridiction civile sur les membres de l'art, une caisse, des statuts ; et Florence est gouvernée par les {\itshape prieurs des arts}, magis­trats désignés par les {\itshape arts}, et par un {\itshape gonfalonier de justice}, désigné par les prieurs, et qui a sous ses ordres mille mercenaires armés. Quant aux nobles, les {\itshape ordinamenti di giustizia} les ont exclus de toute fonction publique et soumis à des mesures d'exception très sévères. Si l'on ajoute que tous les magistrats sont élus pour des délais fort courts et doivent rendre compte de leur gestion, il semble que Florence soit une république d'artisans.\par
Mais en réalité les {\itshape arts} florentins sont tout autre chose que les corporations médiévales. Tout d'abord leur nombre est fixé à vingt et un, et ne peut être modifié ; il est interdit de former un {\itshape art} nouveau. Ceux qui sont en dehors des vingt et un sont donc privés de droits politiques. Puis, si les {\itshape arts} d'artisans et de petits commerçants ressemblent aux corporations ordinaires du moyen âge, ces {\itshape arts}, nommés {\itshape arts mineurs}, sont maintenus au second plan dans la vie politique. Le pouvoir réel appartient aux {\itshape arts majeurs}, qui comprennent seule­ment, si l'on met à part juges, notaires et médecins, les banquiers, les gros commerçants, les fabricants de drap et les fabricants de soieries. Quant à ceux qui travaillent la laine ou la soie, certains sont « membres mineurs » de l'{\itshape art} correspondant à leur métier, avec des droits très restreints ; mais la plu­part sont simplement subordonnés à l'art, c'est-à-dire soumis à sa juridiction sans y posséder aucun droit ; et il leur est la plupart du temps sévèrement interdit non seulement de s'organiser, mais même de se réunir entre eux. L'{\itshape arte di Por Santa Maria} - celui des fabricants de soieries - et surtout L’{\itshape arte della lana} sont donc non des corporations, mais des syndicats patronaux. Loin d'être une démocratie, l'État florentin est directement aux mains du capital bancaire, commercial et industriel.\par
Au cours du XIV\textsuperscript{e} siècle, L’{\itshape arte della lana} prit peu à peu l'influence prépondérante, à mesure que la fabrication du drap devenait la principale ressource de la cité, et que toutes les grandes familles des autres corporations y engageaient des capitaux. Par sa structure, il constitue un petit État, qui organise ses services publics, touche des impôts, émet des emprunts, construit des locaux, aménage des entrepôts, se charge des arrangements qui dépassent les possibilités de chaque entrepreneur ; c'est aussi un cartel, qui impose à ses membres un maximum de production qu'il leur est interdit de dépasser ; c'est surtout une organisation de classe, qui a pour principal objet de défendre en toute occasion les intérêts des fabricants de drap contre les travailleurs. Ceux-ci, au contraire, privés de toute espèce d'organisation, se trouvent désarmés. Telle est la raison principale de l'insurrection des {\itshape Ciompi.}\par
Ces travailleurs de la laine se partageaient en catégories très différentes quant à la situation technique, économique et sociale, et qui, en conséquence, ont joué un rôle différent dans l'insurrection. La plus nombreuse était celle des ouvriers salariés des ateliers. Chaque marchand drapier avait, auprès de sa boutique, un grand atelier, ou plutôt, si l'on tient compte de la division et de la coordination des travaux, une manufacture où l'on préparait la laine avant de la confier aux fileurs. Les travaux exécutés dans ces ateliers - lavage, nettoya­ge, battage, peignage, cordage - étaient en partie des travaux de manœuvres, mais en partie aussi relativement qualifiés. L'organisation de l'atelier était celle d'une fabrique moderne, le machinisme excepté. La division et la spécia­lisation étaient poussées à l'extrême ; une équipe de contremaître assurait la surveillance ; la discipline était une discipline de caserne. Les ouvriers, sala­riés, payés à la journée, sans tarifs ni contrats, dépendaient entièrement du patron. Ce prolétariat de la laine était à Florence la partie la plus méprisée de la population. C'était lui aussi qui, de toutes les couches révoltées de la popu­lation, faisait preuve de l'esprit le plus radical. On surnommait ces ouvriers les {\itshape Ciompi} et le fait qu'ils ont donné leur nom à l'insurrection montre assez quelle part ils y ont prise.\par
Les fileurs et les tisserands étaient, eux aussi, réduits en fait à la condition d'ouvriers salariés ; mais c'étaient des ouvriers à domicile. Isolés par leur travail même, privés du droit de s'organiser, ils ne semblent avoir fait preuve à aucun moment d'esprit combatif. Le tissage était à vrai dire un travail haute­ment qualifié ; mais l'avantage que les tisserands auraient pu retirer de ce fait fut annulé, au XIV\textsuperscript{e} siècle, par l'afflux à Florence de tisserands étrangers, et surtout allemands. Les teinturiers au contraire, ouvriers très hautement quali­fiés, impossibles à remplacer par des étrangers, parce qu'il n'y avait de bons teinturiers qu'à Florence, entrèrent les premiers de tous dans la lutte revendi­cative. À vrai dire, les teinturiers étaient privilégiés par rapport aux autres travailleurs de la laine. La teinturerie demandait l'investissement d'un capital considérable, et cet investissement comportait de gros risques ; aussi les fabri­cants ne cherchèrent-ils pas à avoir leurs propres teintureries. Ce fut l’{\itshape arte della lana} qui créa, pour la teinture, de grands locaux renfermant une partie de l'outillage, et les mit à la disposition de tous les industriels qui voulaient s'en servir ; ainsi les teinturiers ne dépendirent jamais d'un industriel particulier, comme c'était le cas pour les {\itshape Ciompi}, et aussi pour les tisserands, dont les métiers appartenaient en général aux fabricants. Les fouleurs et tondeurs de drap se trouvaient à cet égard dans la même situation que les teinturiers. Enfin, les teinturiers n'étaient pas entièrement privés de droits politiques. Ils avaient une organisation, purement religieuse, il est vrai, mais qui leur permet­tait de se réunir. Ils n'étaient pas simplement subordonnés à l’{\itshape arte della lana}, comme les ouvriers des ateliers, les fileurs et les tisserands ; ils en étaient membres, bien que « membres mineurs », et avaient ainsi une certaine part au gouvernement. Aussi leurs intérêts étaient-ils loin de coïncider avec ceux des {\itshape Ciompi}, et leur attitude au cours de l'insurrection le fit voir. Cependant, les raisons de se révolter ne leur manquaient pas. Privés du droit de s'organiser pour défendre leurs conditions de travail, subordonnés à leurs employeurs qui, de par le droit corporatif, devenaient leurs juges dès qu'il y avait un différend, ils auraient été rapidement réduits à la situation des autres ouvriers s'ils n'avaient su profiter des crises économiques et politiques.\par
Les premières luttes sociales sérieuses eurent lieu en 1342, sous la tyran­nie du duc d'Athènes. C'était un aventurier français à qui Florence, épuisée par les querelles qui avaient sans cesse lieu entre les familles les plus puissantes, remit le pouvoir à vie afin qu'il rétablît l'ordre.\par
Cette élection avait été appuyée surtout par les mécontents, c'est-à-dire d'une part par les nobles, à qui on avait rendu l'accès aux fonctions publiques, mais qui n'en désiraient pas moins voir finir l'État corporatif, et d'autre part par le peuple. Le duc d'Athènes s'appuya principalement, pendant les quelques mois qu'il régna, sur les ouvriers, grâce auxquels il espérait pouvoir résister à l'hostilité de la haute bourgeoisie. Il donna satisfaction aux teinturiers, qui se plaignaient d'être payés avec des années de retard et d'être sans recours légal, et qui demandaient à constituer un vingt-deuxième {\itshape art} ; il organisa les ouvriers des ateliers de laine, non pas en une corporation, mais en une asso­ciation armée. Peu après, il fut renversé par une émeute à laquelle presque toute la population prit part, et où il n'eut pour défenseurs que des bouchers et quelques ouvriers ; l'art des teinturiers ne fut pas créé, mais les prolétaires de la laine gardèrent leurs armes et s'en servirent dans les années suivantes. À la démagogie du duc d'Athènes qui, au mépris du droit corporatif, donnait satisfaction à toutes les revendications des ouvriers de la laine, succédait la plus brutale dictature capitaliste. Aussi les révoltes éclatèrent-elles bientôt. En 1343, 1 300 ouvriers se soulèvent ; en 1345, nouveau soulèvement, dirigé par un cardeur, et ayant pour objectif l'organisation des ouvriers de la laine. La grande peste de Florence, qui décime la classe ouvrière, raréfie la main-d'oeuvre et provoque ainsi une hausse des salaires telle que l’{\itshape arte della lana} doit établir des taxes, rend la lutte de classes plus aiguë encore. Après une crise provoquée par la guerre contre Pise, et qui arrête momentanément les conflits, le retour de la prospérité, par un phénomène qui s'est fréquemment reproduit depuis lors, amène une grève des teinturiers qui dure deux ans et se termine par une défaite, en 1372 ; mais cette défaite ne met pas fin à la fermentation des couches laborieuses.\par
Cette fermentation coïncide avec un conflit entre la petite bourgeoisie d'une part, et la grande bourgeoisie, unie dans une certaine mesure à la nobles­se, de l'autre. Les nobles, en tant que classe, ont été définitivement battus quand, après la chute du duc d'Athènes, ils ont tenté de s'emparer du pouvoir ; mais la plupart des familles nobles sont alliées à la haute bourgeoisie à l'intérieur du « parti guelfe ». Ce parti guelfe s'est formé dans la lutte, depuis longtemps terminée, des Guelfes et des Gibelins ; la confiscation des biens des Gibelins lui a donné richesse et puissance. Devenu l'organisation politique de la haute bourgeoisie, il domine la cité depuis la chute du duc d'Athènes, fausse les scrutins, profite d'une mesure d'exception prise autrefois contre les Gibelins et demeurée en vigueur pour écarter ses adversaires des fonctions publiques. Quand, malgré les manœuvres du parti guelfe, Salvestro de Medici, un des chefs de la petite bourgeoisie, devient, en juin 1378, gonfalonier de justice, et quand il propose des mesures contre les nobles et contre le parti guelfe, le conflit devient aigu. Les compagnies des {\itshape arts} descendent dans la rue en armes ; les ouvriers les soutiennent et mettent le feu à quelques riches demeures et aux prisons, qui sont pleines de prisonniers pour dettes. Finale­ment, Salvestro de Medici a satisfaction. Mais, comme dit Machiavel, « qu'on se garde d'exciter la sédition dans une cité en se flattant qu'on l'arrêtera ou qu'on la dirigera à sa guise ».\par
De la direction de la petite et moyenne bourgeoisie, le mouvement tombe sous celle du prolétariat. Les ouvriers restent dans la rue ; les {\itshape arts mineurs} les appuient ou laissent faire. Et déjà apparaît le trait qui se reproduira sponta­nément dans les insurrections prolétariennes françaises et russes : la peine de mort est décrétée par les insurgés contre les pillards. Autre trait propre aux soulèvements de la classe ouvrière, le mouvement n'est nullement sangui­naire ; il n'y a aucune effusion de sang, exception faite pour un nommé Nuto, policier particulièrement haï. La liste des revendications des insurgés, liste portée aux prieurs le 20 juillet, a, elle aussi, un caractère de classe. On deman­de la transformation des impôts, qui pèsent lourdement sur les ouvriers ; la suppression des « officiers étrangers » de l{\itshape 'arte della lana}, qui constituent des instruments de répression contre les travailleurs, et jouent un rôle analogue à celui de la police privée que possèdent de nos jours les compagnies minières d'Amérique. Surtout on demande la création de trois nouveaux {\itshape arts} : un vingt-deuxième {\itshape art} pour les teinturiers, fouleurs et tondeurs de drap, c'est-à-dire pour les travailleurs de la laine non encore réduits à la condition de prolé­taires ; un vingt-troisième {\itshape art} pour les tailleurs et autres petits artisans non encore organisés ; enfin et surtout un vingt-quatrième {\itshape art} pour le « menu peuple » , c'est-à-dire en fait pour le prolétariat, qui est alors principalement constitué par les ouvriers des ateliers de laine. De même que l'{\itshape arte della lana} n'était en réalité qu'un syndicat patronal, cet {\itshape art du .menu peuple} aurait fonctionné comme un syndicat ouvrier ; et il devait avoir la même part au pouvoir d'État que le syndicat patronal, car les insurgés réclamaient le tiers des fonctions publiques pour les trois {\itshape arts} nouveaux, et le tiers pour les {\itshape arts mineurs.} Ces revendications tardant à être acceptées, les ouvriers envahissent le Palais le 21 juillet, conduits par un cardeur de laine devenu contremaître, Michele de Lando, qui est aussitôt nommé gonfalonier de justice, et qui forme un gouvernement provisoire avec les chefs du mouvement des {\itshape arts mineurs.} Le 8 août, la nouvelle forme de gouvernement, conforme aux revendications des ouvriers, est organisée et pourvue d'une force armée composée non plus de mercenaires, mais de citoyens. La grande bourgeoisie, se sentant momen­tanément la plus faible, ne fait pas d'opposition ouverte ; mais elle ferme ses ateliers et ses boutiques. Quant au prolétariat, il s'aperçoit rapidement que ce qu'il a obtenu ne lui donne pas la sécurité, et qu'un partage égal du pouvoir entre lui, les artisans et les patrons est une utopie. Il fait dissoudre l'organi­sation politique que s'étaient donnée les {\itshape arts mineurs} ; il élabore pétitions sur pétitions ; il se retire à Santa Maria Novella, s'organise comme avait fait autrefois le parti guelfe, nomme huit officiers et seize conseillers, et invite les autres arts à venir conférer sur la constitution à donner à la cité. Dès lors la cité possède deux gouvernements, l'un au Palais, conforme à la légalité nou­velle, l'autre non légal, à Santa Maria Novella. Ce gouvernement extra-légal ressemble singulièrement à un soviet ; et nous voyons apparaître pour quel­ques jours, à ce premier éveil d'un prolétariat à peine formé, le phéno­mène essentiel des grandes insurrections ouvrières, la dualité du pouvoir. Le prolétariat, en août 1378, oppose déjà, comme il devait faire après février 1917, à la nouvelle légalité démocratique qu'il a lui-même fait instituer, l'organe de sa propre dictature.\par
Michele de Lando fait ce qu'aurait fait à sa place n'importe quel bon chef d'État social-démocrate : il se retourne contre ses anciens compagnons de travail. Les prolétaires, ayant contre eux le gouvernement, la grande bour­geoisie, les {\itshape arts mineurs}, et sans doute aussi les deux nouveaux {\itshape arts} non prolétariens, sont vaincus après une sanglante bataille et férocement extermi­nés au début de septembre. On dissout le vingt-quatrième {\itshape art} et la force armée organisée en août ; on désarme les ouvriers ; on fait venir des compagnies de la campagne, comme à Paris après juin 1848. Quelques tentatives de soulè­vement sont faites au cours des mois suivants, avec comme mot d'ordre : pour le vingt-quatrième {\itshape art} ! Elles sont férocement réprimées. Les {\itshape arts mineurs} gardent encore quelques mois la majorité dans les fonctions publiques ; puis le pouvoir est partagé également entre eux et les {\itshape arts majeurs}. Les teinturiers, qui ont conservé leur {\itshape art}, peuvent encore l'utiliser pour une action revendi­cative et imposent un tarif minimum. Mais une fois privés, par leur faute, de l'appui de ce prolétariat dont l'énergie et la résolution les avaient poussés au pouvoir, les artisans, les petits patrons, les petits commerçants sont incapables de maintenir leur domination. La bourgeoisie, comme le remarque Machiavel, ne leur laisse le champ libre que dans la mesure ou elle craint encore le prolétariat ; dès qu'elle le juge définitivement écrasé, elle se débarrasse de ses alliés d'un jour. Au reste, eux-mêmes se désagrègent de l'intérieur sous l'influence de la démoralisation, elle aussi bien caractéristique, qui pénètre leurs rangs. Ils laissèrent exécuter un des chefs principaux des classes moyen­nes, Scali ; et cette exécution ouvrit la voie à une brutale réaction, qui amena l'exil de Michele de Lando, de Benedetto Alberti lui-même et de bien d'autres, la suppression du vingt-deuxième et du vingt-troisième {\itshape art}, la domination des {\itshape arts majeurs}, le rétablissement des prérogatives du parti guelfe. En janvier 1382, le {\itshape statu quo} d'avant l'insurrection était rétabli. La puissance des entre­preneurs était désormais absolue ; et le prolétariat, privé d'organisation, ne pouvant se réunir, même pour un enterrement, sans permission spéciale, devait attendre longtemps avant de pouvoir la mettre même en question.\par
Machiavel, écrivant un siècle et demi après l'événement, en une période de calme social complet, trois siècles avant que ne fût élaborée la doctrine du matérialisme historique, a su néanmoins, avec la merveilleuse pénétration qui lui est propre, discerner les causes de l'insurrection et analyser les rapports de classe qui en ont déterminé le cours. Son récit de l'insurrection, que nous donnons ici, est, en dépit d'une hostilité apparente à l'égard des insurgés, qu'il prend à tort pour des pillards, plus remarquable encore par une étonnante précision dans tout ce qui répond à nos préoccupations actuelles que par le caractère captivant de la narration et la beauté du style.\par

\begin{quoteblock}
 \noindent  \footnote{ Le texte qui suit est tiré des {\itshape Istorie fiorentine} de Machiavel, livre III, ch. XII-XVII. La traduction est de S. W. (Note de l'éditeur.)} À peine ce premier soulèvement apaisé, il s'en produisit un autre qui fit plus de tort que le premier à la république. La plupart des incendies et des vols qui avaient eu lieu le jour précédent avaient été commis par la plus basse plèbe ; et ceux qui s'étaient montrés les plus audacieux craignaient de recevoir, une fois les différends plus graves apaisés et réglés, le châtiment de leurs fautes, et d'être, comme il arrive toujours, abandonnés par les insti­gateurs de leurs mauvaises actions. À cela s'ajoutait la haine que le petit peuple portait aux citoyens riches et aux chefs des arts, qui ne lui accordaient pas de salaires suffisants à proportion de ce qu'il croyait mériter. Lorsque la cité, sous Charles 1\textsuperscript{er}, s'était divisée en arts, chacun s'était donné des chefs et une forme de gouvernement ; et l'on statua que les chefs de chaque art jugeraient, en matière civile, tous ceux qui s'y rattachaient. Ces arts, comme j'ai dit furent d'abord douze ; et, avec le temps, ils s'accrurent au point de parvenir au nombre de vingt et un, et devinrent si puissants qu'au bout de peu de temps ils s'emparèrent de tout le gouvernement de la cité. Et comme parmi eux il s'en trouvait qui étaient plus honorés les uns que les autres, on les divisa en majeurs et en mineurs. Il y en eut sept majeurs et quatorze mineurs... Mais lorsqu'on organisa les arts, beaucoup de métiers auxquels s'adonnaient le menu peuple et la basse plèbe restèrent sans art propre ; et ceux qui les exer­çaient furent subordonnés aux arts avec lesquels ils se trouvaient en rapports. Il en résulta que, lorsqu'ils étaient mécontents de leurs salaires, ou, d'une manière générale, opprimés par leurs maîtres, ils n'avaient d'autre recours que les magistrats des arts auxquels ils étaient soumis ; et il ne leur semblait jamais que ces magistrats leur rendissent justice comme il convenait. De tous les arts, celui auquel le plus grand nombre d'ouvriers se trouvait subordonné de la sorte était l'art de la laine ; il était le plus puissant de tous et le premier en autorité, et nourrissait de son industrie, comme il fait encore, la plus grande partie de la plèbe et du menu peuple. \par
 Ainsi ces hommes de la plèbe, ceux qui étaient soumis à l'art de la laine comme ceux qui dépendaient des autres arts, étant pleins de ressentiment, et aussi de peur à cause des incendies et des vols qu'ils avaient commis, se réunirent à plusieurs reprises, la nuit, en secret, pour parler des événements passés et examiner les dangers qui les menaçaient. Là l'un d'entre eux, plus ardent et plus expérimenté que les autres, parla de la sorte pour animer ses compagnons : « Si nous devions en ce moment délibérer pour savoir s'il faut prendre les armes, brûler et piller les maisons des citoyens, dépouiller l'Église, je serais de ceux qui jugeraient que cela mérite réflexion ; et peut-être serais-je d'avis de préférer une pauvreté tranquille à un gain périlleux. Mais puisque les armes sont prises et qu'il y a déjà beaucoup de mal de fait, il me semble que nous devons chercher par quel moyen conserver les armes et parer au danger où nous mettent les délits commis par nous... Vous voyez que toute la ville est pleine de rancune et de haine contre nous; les citoyens se réunissent, les prieurs se joignent aux autres magistrats. Croyez que l'on prépare des pièges contre nous et que de nouveaux périls menacent nos têtes. Nous devons donc chercher à obtenir deux choses et assigner à nos délibérations un double but : à savoir d'une part ne pas être châtiés pour ce que nous avons fait les jours précédents, d'autre part pouvoir vivre avec plus de liberté et plus de bien-être que par le passé. Il convient à cet effet, à ce qu'il me semble, si nous voulons nous faire pardonner les fautes anciennes, d'en commettre de nouvelles, de redoubler les excès, de multiplier vols et incendies et de chercher à entraîner un grand nombre de compagnons. Car là où il y a beaucoup de coupables, personne n'est châtié ; les petites fautes sont punies, celles qui sont impor­tantes et graves sont récompensées. Et quand un grand nombre de gens souf­frent, la plupart ne cherchent pas à se venger, parce que les injures générales sont supportées plus patiemment que les particulières. Ainsi, en multipliant le mal, nous trouverons plus facilement le pardon et nous verrons s'ouvrir devant nous la voie qui nous mènera vers les buts que nous désirons atteindre pour être libres. Et nous allons, me semble-t-il, à une conquête certaine ; car ceux qui pourraient nous faire obstacle sont désunis et riches ; leur désunion nous donnera la victoire, et leurs richesses, une fois devenues nôtres, nous permet­tront de la maintenir \footnote{ D'après cette formule, Machiavel concevait l'insurrection des {\itshape Ciompi} comme essentiellement dirigée vers l'expropriation des riches. (Note de S. W.)}. Ne vous laissez pas effrayer par cette ancienneté du sang dont ils se targuent ; car tous les hommes, ayant eu une même origine, sont également anciens et la nature nous a tous faits sur un même modèle. Déshabillés et nus, vous seriez tous semblables ; revêtons leurs habits, qu'ils mettent les nôtres, nous paraîtrons sans aucun doute nobles et eux gens du commun ; car seules la pauvreté et la richesse font l'inégalité. J'ai peine à voir que beaucoup d'entre vous regrettent ce qu'ils ont fait et veulent s'abstenir d'actions nouvelles. Et certes, s'il en est ainsi, vous n'êtes pas ceux que j'ai cru ; ni le remords ni la honte ne doivent vous effrayer ; car pour des vain­queurs, de quelque manière qu'ils aient vaincu, il n'y a jamais de honte. Et nous ne devons pas tenir compte des remords de conscience ; car là où se trouve, comme en nous, la crainte de la faim et de la prison, l'enfer ne peut ni ne doit effrayer. Mais si vous remarquez comment se conduisent les hommes, vous verrez que tous ceux qui sont parvenus à une grande richesse et une grande puissance y sont parvenus ou par la ruse ou par la force ; et ce qu'ils ont usurpé par fourberie ou par violence, pour dissimuler le caractère brutal de leur acquisition ils le décorent ensuite de faux titre de gain... Dieu et la nature ont placé tous les biens devant l'homme ; mais ces biens sont plutôt le prix de la rapine que de l'industrie, et des procédés malhonnêtes que des procédés honnêtes. De là vient que les hommes se mangent l'un l'autre et que le plus faible est toujours victime ; nous devons donc employer la force quand l'occasion s'en présente... Je confesse que ce parti est audacieux et périlleux ; mais quand la nécessité commande, l'audace devient prudence et les hommes courageux ne tiennent jamais compte des périls dans les grandes entreprises... Au reste je crois que, quand on voit préparer la prison, la torture et la mort, il est plus téméraire d'attendre que de chercher à se mettre en sûreté ; dans le premier cas le mal est certain, dans le second il est douteux... Vous voyez les préparatifs de vos adversaires ; prévenons leurs projets. La victoire est assurée à celui de nous qui prendra le premier les armes, et en même temps la ruine des ennemis et sa propre élévation ; beaucoup parmi nous tireront honneur de cette victoire, tous y trouveront la sécurité. »\par
 Violemment enflammés par cette éloquence et ayant déjà par eux-mêmes l'esprit échauffé pour le mal, ils résolurent de prendre les armes quand ils auraient associé à leurs projets un plus grand nombre de compagnons. Et ils s'obligèrent par serment à se secourir les uns les autres, s'il arrivait que l'un d'eux fût opprimé par les magistrats.\par
 Pendant qu'ils se préparaient à s'emparer de l'État, leur dessein parvint à la connaissance des prieurs ; et ceux-ci firent arrêter sur la place un certain Simon qui leur dévoila toute la conjuration et que l'on voulait commencer le soulèvement le lendemain. Voyant le danger, ils réunirent les Collèges et les citoyens qui travaillaient avec les syndics des arts à rétablir l'union dans la cité. La nuit était venue avant que tout le monde lut réuni ; l'assemblée con­seilla aux prieurs de faire venir les consuls des arts. Tous furent d'avis de faire venir toutes les troupes à Florence et de faire descendre le lendemain matin sur la place les gonfaloniers du peuple avec leurs compagnies armées. Un certain Nicolo de San Friano était en train de réparer l'horloge du palais au moment où on mettait Simon à la torture et où les citoyens se réunissaient ; s'étant aperçu de ce qui se passait, il rentra chez lui et souleva tout son quar­tier. Un instant plus tard, plus de mille hommes armés se trouvaient rassem­blés sur la place du Saint-Esprit. Le bruit en parvint aux autres conjurés, et Saint-Pierre Majeur et Saint-Laurent, lieux désignés par eux, se remplirent d'hommes en armes.\par
 Le lendemain matin, qui était le 21 juillet, il ne se trouvait pas réuni sur la place plus de quatre-vingts hommes armés pour la défense des prieurs. Aucun gonfalonier ne vint, parce qu'ils avaient appris que toute la ville était en armes et avaient peur de quitter leurs maisons. Les premiers parmi les insurgés qui se trouvèrent sur la place furent ceux qui s'étaient réunis à Saint-Pierre Majeur ; à leur arrivée la troupe ne fit pas un mouvement. Le reste de la multitude appa­rut ensuite, et n'ayant pas rencontré d'obstacles réclama les prisonniers aux prieurs avec des cris terribles ; puis, voulant les obtenir par la force, puisque les menaces n'avaient pas suffi, ils mirent le feu à la maison de Luigi Guicciardini ; aussi les prieurs, crainte de pire, leur rendirent leurs hommes. Dès qu'ils eurent repris les prisonniers, les insurgés arrachèrent l'étendard de la justice à celui qui le portait ; et, marchant sous ce drapeau, ils allèrent brûler les maisons de nombreux citoyens, en s'attaquant à ceux qui leur étaient odieux pour des raisons publiques ou privées. Beaucoup de citoyens, pour venger leurs injures particulières, les menaient aux maisons de leurs ennemis ; il suffisait pour cela qu'une voix criât parmi la multitude : « À la maison d'un tel! » ou que celui qui portait en main le gonfalon se tournât de ce côté. En même temps qu'ils faisaient tant de mal, pour y joindre quelque action louable ils nommèrent chevaliers Salvestro de Medici et jusqu'à soixante-quatre autres citoyens, parmi lesquels Benedetto et Antonio Albenti, Tommaso Strozzi et d'autres de leurs amis; beaucoup reçurent ce titre malgré eux. Ce qui est le plus digne de remarque dans cette affaire, c'est le fait que l'on brûla les maisons de beaucoup de citoyens qui reçurent ensuite dans la même journée, et des mêmes gens ( tant le bienfait chez eux était proche de l'injure), le titre de chevalier ; c'est ce qui arriva notamment à Luigi Guicciardini, gonfalonier de justice. Les prieurs, au milieu d'un tel tumulte, se voyant abandonnés des troupes, des chefs des arts et de leurs gonfaloniers, se décourageaient ; car personne n'était venu à leur secours selon les ordres donnés. Des seize gonfalons apparurent seulement l'étendard du Lion d'or et celui de la Belette, avec Giovenco della Stufa et Giovanni Cambi. Encore ceux-là ne restèrent-ils que peu de temps sur la place ; ne se voyant pas suivis par les autres, ils se retirèrent. D'autre part les citoyens, voyant la fureur de cette foule effrénée et le palais abandonné, allèrent les uns se renfermer dans leurs maisons, les autres suivre la multitude armée, pour mieux pouvoir, en se trouvant au milieu d'elle, défendre leurs maisons et celles de leurs amis. Ainsi la puissance de cette foule allait croissant, celle des prieurs décroissant. Ce désordre dura tout le jour ; et, la nuit venue, les insurgés s'arrêtèrent au palais de Stefano, derrière l'église de Saint-Barnabé. Ils étaient plus de six mille ; et avant la levée du jour, ils contraignirent les arts, par des menaces, à leur envoyer leurs éten­dards. Lorsque le jour fut venu, ils allèrent avec le gonfalon de la justice et les étendards devant le palais du podestat, et comme celui-ci refusait de leur livrer le palais, ils l'attaquèrent et triomphèrent.\par
 Les prieurs, voulant essayer de composer avec eux puisqu'ils ne voyaient aucun moyen de les arrêter par la force, firent venir quatre membres de leurs collèges et les envoyèrent au palais du podestat pour connaître la volonté des insurgés. Ces envoyés virent en arrivant que les chefs de la plèbe, de concert avec les syndics des arts, avaient déjà préparé les demandes qu'ils voulaient adresser aux prieurs. Ils revinrent donc chez les prieurs, accompagnés de quatre députés de la plèbe chargés de ces demandes. On devait interdire à l'art de la laine de conserver ses juges étrangers ; former trois nouveaux arts, l'un pour les tondeurs de drap et les teinturiers, l'autre pour les barbiers, faiseurs de pourpoints, tailleurs et autres arts mécaniques, et le troisième pour le menu peuple ; l'on devait toujours prendre deux prieurs dans ces trois nouveaux arts, et trois dans les arts mineurs ; les prieurs devaient assigner à ces nouveaux arts un lieu où ils pourraient se réunir ; aucun membre de ces arts ne devait pouvoir être contraint, avant un délai de deux ans, à payer une dette de plus de cinquante ducats; le mont-de-piété devait cesser de réclamer les intérêts et ne plus exiger que le capital; il fallait absoudre les emprisonnés et les condamnés et rendre à tous les admonestés  \footnote{Citoyens privés de leurs droits en tant que « gibelins », c'est-à-dire sur la pression du parti « guelfe ». (Note de S. W.)} leurs droits civiques. Les insurgés deman­daient aussi beaucoup de faveurs pour ceux qui les avaient soutenus ; en revanche, ils voulaient faire admonester et bannir leurs ennemis. Ces deman­des, bien que déshonorantes et dangereuses pour la république, furent accor­dées immédiatement; crainte de pire, par les prieurs, les collèges et le conseil du peuple. Mais pour qu'elles eussent force de loi, il fallait encore l'approba­tion du conseil de la commune ; et comme on ne pouvait réunir deux conseils le même jour, on dut remettre la chose au lendemain. Néanmoins les arts et la plèbe parurent satisfaits et promirent qu'une fois la loi achevée, il n'y aurait plus aucun désordre. \par
 Le matin venu, pendant que le conseil de la commune délibérait, la multi­tude impatiente et versatile vint sur la place avec ses étendards accoutumés et avec des cris si perçants et si effroyables que le conseil et les prieurs en furent épouvantés. Aussi Guerriante Marignoli, un des prieurs, sur qui la peur agissait plus violemment qu'aucune autre émotion, sortit sous prétexte de garder la porte d'en bas et s'enfuit chez lui. Il ne put si bien se cacher en sor­tant qu'il ne fût reconnu par la foule; on ne lui fit aucun mal, mais la multitude, en le voyant, se mit à crier que tous les prieurs devaient abandonner le palais, sans quoi leurs fils seraient égorgés et leurs maisons brûlées. Cependant la loi avait été adoptée et les prieurs s'étaient retirés dans leurs chambres ; les membres du conseil étaient descendus et, sans sortir, restaient dans la galerie et la cour, désespérant du salut de la cité en voyant si peu de sentiments d'honneur dans la multitude et tant de malignité ou tant de frayeur chez ceux qui auraient pu soit la contenir, soit l'écraser. Le trouble régnait aussi parmi les prieurs, incertains du salut de la patrie, abandonnés par un des leurs, et à qui aucun citoyen ne donnait de secours ni même de conseils. Tandis qu'ils se tenaient ainsi, sans savoir ce qu'ils pouvaient et ce qu'ils devaient faire, Tommaso Strozzi et Benedetto Alberti, soit qu'ils fussent poussés par l'ambition personnelle et pour rester les maîtres du palais, soit qu'ils crussent que c'était le meilleur parti à prendre, les engagèrent à céder à la poussée populaire et à rentrer dans leurs maisons comme de simples particuliers. Devant un pareil conseil, donné par ceux qui avaient été les chefs de l'insurrection, Alamanno Acciaioli et Niccolo del Bene, deux prieurs, s'indignèrent, bien que les autres prieurs fussent disposés à s'y conformer ; et, reprenant un peu de vigueur, ils dirent que, si les autres voulaient partir, ils ne pouvaient s'y opposer, mais que, pour eux, ils ne voulaient pas renoncer à leur autorité avant que le moment en fût venu, à moins de perdre en même temps la vie. Ces désaccords redoublèrent la frayeur des prieurs et la colère du peuple ; enfin le gonfalonier, préférant terminer sa magistrature honteusement plutôt qu'avec péril, se mit sous la protection de Tommaso Strozzi qui le fit sortir du palais et le conduisit chez lui. Les autres prieurs s'en allèrent ensuite de la même manière l'un après l'autre ; et Alamanno et Niccolo, pour ne pas être regardés comme étant plutôt courageux que sages, se retirèrent à leur tour quand ils se virent seuls. Le palais restait donc entre les mains de la plèbe et aussi des huit de la guerre  \footnote{Le Conseil des huit de la guerre était une organisation dont tous les membres se trouvaient alors être du parti de la petite bourgeoisie. (Note de S. W.)}, qui n'avaient pas encore déposé leurs pouvoirs. \par
 Au moment où la plèbe entra dans le palais, celui qui tenait en main l'éten­dard du gonfalonier de justice était un certain Michele di Lando, peigneur de laine. Celui-ci, nu-pieds et à peine vêtu, monta l'escalier avec toute la foule ; et quand il fut à la salle d'audience des prieurs, il s'arrêta, se tourna vers la multitude et dit : « Vous voyez que ce palais est à vous et que la cité est entre vos mains. Que voulez-vous faire à présent ? » À quoi tous répondirent qu'il fût gonfalonier et seigneur, et les gouvernât, eux et la cité, comme bon lui semblerait. Michele accepta la seigneurie ; et comme c'était un homme sage et prudent, qui devait plus à la nature qu'à la fortune, il résolut d'apaiser la cité et de mettre fin aux désordres. Pour occuper le peuple et se donner le temps de prendre ses mesures, il donna ordre de chercher un certain Nuto, qui avait été désigné comme chef de la police par Lapo da Castiglionchio. La plupart de ceux qui l'entouraient allèrent exécuter cet ordre. Et, pour inaugurer par un acte de justice le pouvoir qu'il avait acquis par la faveur, il fit interdire publiquement vols et incendies. Pour épouvanter tout le monde, il fit élever un gibet sur la place. Puis, pour commencer la réforme de l'État, il destitue les syndics des arts, en nomme de nouveaux, prive les prieurs et les collèges de leurs charges... Cependant Nuto était porté sur la place par la multitude ; on le pend au gibet par un pied ; et chacun de ceux qui l'entourent ayant arraché un lambeau de son corps, il n'en reste bientôt plus que le pied. Les huit de la guerre, d'autre part, croyant que le départ des prieurs les faisait maîtres de la cité, avaient déjà nommé de nouveaux prieurs. Michele, se doutant de la chose, leur fit dire de quitter immédiatement le palais, ajoutant qu'il voulait montrer à tout le monde qu'il était capable de gouverner Florence sans leur conseil. Il fit ensuite réunir les syndics des arts et nomma les prieurs, dont il prit quatre dans la basse plèbe, deux dans les arts majeurs et deux dans les arts mineurs. Il organisa en outre un nouveau scrutin et divisa le pouvoir d'État en trois parts, dont l'une devait échoir aux arts nouveaux, la seconde aux arts mineurs, la troisième aux arts majeurs. Il accorda à Salvestro de Medici le revenu des boutiques du Vieux-Pont, prit pour lui le podestat d'Empoli et fit bien d'autres faveurs à bien d'autres citoyens amis de la plèbe, non pas tant pour les récompenser de leurs services que pour s'en faire des défenseurs contre les envieux.\par
 Il parut à la plèbe que Michele, en réformant l'État, avait trop bien partagé la haute bourgeoisie ; elle ne pensa pas avoir une part du pouvoir assez grande pour être en mesure de la conserver et de se défendre ; si bien qu'avec son audace accoutumée elle reprit soudain les armes, descendit en tumulte sur la place derrière ses étendards et demanda que les prieurs se rendissent à la salle des audiences pour délibérer à nouveau sur les mesures à prendre pour la sécurité et le bien de la plèbe. Michele, voyant leur arrogance et ne voulant pas augmenter leur colère, ne fit guère attention à ce qu'ils réclamaient, et, blâmant simplement la manière dont ils présentaient leurs demandes, les engagea à déposer les armes ; ils obtiendraient en ce cas ce que la dignité des prieurs ne permettait point que l’on accordât à la violence. Cette réponse fit que la multitude, indignée contre le palais, se retira à Santa Maria Novella ; et là ils nommèrent huit chefs pris parmi les leurs, ainsi que des ministres, et créèrent encore d'autres dignités qui leur semblaient propres à attirer la considération et le respect. Ainsi l'État avait deux sièges et la cité deux gou­vernements distincts. Ces chefs décidèrent que huit délégués, choisis par les arts qui participaient au mouvement, habiteraient au palais avec les prieurs et que toutes les décisions prises par les prieurs devraient être confirmées par eux. Ils enlevèrent à Salvestro de Medici et à Michele di Lando tout ce que la plèbe, par ses décisions précédentes, leur avait accordé. Ils donnèrent à un grand nombre des leurs des fonctions et des pensions, pour leur permettre de soutenir leur rang avec dignité. Une fois ces décisions prises, ils voulurent leur faire donner force de loi, et ils envoyèrent deux d'entre eux aux prieurs pour leur demander de faire confirmer ces décisions par les conseils, avec le dessein d'arracher cette confirmation par la force s'ils ne pouvaient l'obtenir de bon gré. Ces envoyés remplirent leur mandat en présence des prieurs avec une grande audace et une présomption plus grande encore, en reprochant au gonfa­lonier la dignité que la plèbe lui avait donnée, les honneurs qu'elle lui avait accordés, et l'ingratitude et le manque de respect dont il faisait preuve à son égard. Et comme ils en étaient venus, à la fin de leur discours, à des menaces, Michele ne put supporter plus longtemps tant d'arrogance ; se souvenant plutôt du rang qu'il occupait que de la bassesse de sa condition, il jugea bon de réprimer par des moyens extraordinaires une insolence extraordinaire et, tirant l'épée qu'il portait au côté, il les blessa grièvement ; après quoi il les fit lier et mettre en prison.\par
 Cette action, quand elle fut connue, souleva de fureur la multitude. Celle-ci, croyant qu'elle pourrait conquérir par les armes ce qu'elle n'avait pu obtenir désarmée, prit les armes avec rage et en tumulte et se mit en marche pour arracher l'assentiment des prieurs de vive force. Michele, de son côté, se dou­tant qu'ils allaient arriver, décida de les prévenir, pensant qu'il y aurait plus de gloire pour lui à attaquer le premier qu'à demeurer chez lui pour attendre l'ennemi, et se voir ensuite forcé, comme ses prédécesseurs, de déshonorer le palais et de se couvrir lui-même de honte en prenant la fuite. Il rassembla donc un grand nombre de citoyens, à savoir ceux qui avaient déjà commencé à se repentir de leur erreur, monta à cheval et, suivi de beaucoup d'hommes armés, marcha sur Santa Maria Novella pour combattre la plèbe. Celle-ci, qui avait, comme nous l'avons dit, pris une décision semblable, était partie elle aussi, presque au même moment que Michele, pour se rendre sur la place. Le hasard fit qu'ils ne prirent pas le même chemin et ne se rencontrèrent pas en route. Alors Michèle, revenant sur ses pas, trouva la place occupée par la plèbe qui était en traite d'attaquer le palais. Il engagea le combat contre eux, les vainquit, en chassa une partie de la ville et contraignit les autres à déposer les armes et à se cacher. Cette victoire fut obtenue, et l'ordre rétabli, par le seul mérite du gonfalonier... S'il avait eu un caractère pervers ou ambitieux, la République aurait entièrement perdu la liberté et serait tombée sous une tyrannie pire que celle du duc d'Athènes. Mais la prudence de Michele, et sa vertu qui ne laissait venir à son esprit aucune pensée qui fût contraire au bien général, firent qu'il conduisit l'affaire de manière à se faire suivre d'un grand nombre des gens de son parti et à pouvoir dompter les autres par les armes. Il inspira ainsi de la frayeur à la plèbe et du repentir aux meilleurs artisans; ceux-ci comprirent quelle honte il y aurait pour eux, qui avaient dompté l'orgueil des grands, à supporter le joug de la plèbe.
 \end{quoteblock}

\noindent ({\itshape La Critique sociale}, n°11 mars 1934.)
\subsection[7. Ébauches de lettres  (1938 ?-1939 ?)]{7. \\
Ébauches de lettres \protect\footnotemark  \\
(1938 ?-1939 ?)}
\footnotetext{Le destinataire de ces lettres est inconnu. (Note de l'éditeur.)}
\noindent \par
\subsubsection[I.]{I.}
\noindent Cher ami,\par
J'ai bien reçu votre carte et votre lettre sur Machiavel. J'avoue que je ne comprends pas vos distinctions sur les diverses espèces de manque de foi chez les grands personnages de l'histoire. J'avoue que c'est un des points les plus faibles de Machiavel - que, d'ailleurs, j'admire beaucoup, mais pour l'ensemble de son œuvre plutôt que pour le {\itshape Prince} en particulier ; c'est un livre très obscur, qu'on ne peut comprendre, à mon avis, qu'en tenant compte du conflit entre les sentiments vraiment civiques et républicains de Machiavel, et la servilité de parole et même peut-être de pensée imposée par la situation où il se trouvait ; sans compter l'ardent désir qu'il avait, voyant la liberté de toutes manières perdue (comme elle l'était en effet), de voir s'établir un despotisme national plutôt que la domination étrangère. Compte tenu de ce dernier point, qui explique l'accent de certains passages, je pense que Rousseau avait raison de voir dans ce livre le manuel du citoyen et non du prince ; ou peut-être plutôt une description froide de la mécanique du gouvernement despotique. Quoi qu'il en soit, pour en revenir au manque de parole, on pourrait sans doute établir d'une manière bien plus précise les conditions nécessaires pour que ce procédé soit utile à l'établissement d'une domination solide. J'en vois trois, à première vue, que voici. Ou bien le manque de parole doit être assez exceptionnel, entouré de circonstances assez obscures et dirigé contre des adversaires assez faibles pour que la réputation de fidélité à la parole donnée reste intacte auprès de ceux qui ne regardent pas les choses de près, lesquels forment la plus grande masse des hommes. Ou bien le manque de parole doit avoir pour effet le massacre à peu près complet de ceux qui en sont les victimes, de manière qu'ils ne puissent se plaindre. Ou bien celui qui manque de parole doit être assez puissant pour que personne ou presque n'ose le lui reprocher, ni même s'en apercevoir. À quoi il faut ajouter, bien entendu, comme quatrième cas, la condition que vous posez, c'est-à-dire la nécessité issue d'un changement essentiel des circonstances. L'exemple dont vous êtes parti ne rentre dans aucun de ces cas ; aussi est-ce à mon avis une faute et même une erreur grave (à moins cependant que la 3\textsuperscript{e} condition ne soit bien prés d'être réalisée, ce qui est possible). Mais les trois premières conditions sont tout à fait susceptibles de s'appliquer à ce que vous nommez brigandage. Et quand vous alléguez César, Auguste, Richelieu, comme exemples d'hom­mes qui ont construit quelque chose de solide parce qu'ils n'ont pas commis de manquements de foi vulgaires, vous m'étonnez beaucoup. César a remporté sa victoire essentielle, celle qui a arrêté presque définitivement le flot d'émigra­tion germaine en Gaule, et en même temps frappé tous les Gaulois d'une terreur paralysante, par la plus basse des trahisons. Les Germains lui ayant envoyé des délégués pour connaître ses exigences, et ceux-ci ayant demandé trois jours de trêve pour les faire accepter par le peuple, César les leur accor­da, bien qu'il fût persuadé (du moins il l'affirme) qu'ils avaient besoin de ces trois jours seulement pour être plus prêts à livrer bataille. Au moment même où les délégués revenaient - donc avant qu'ils n'aient pu annoncer cette trêve à leur armée - une poignée de cavaliers germains, attaquant inopinément, mit en déroute toute la cavalerie des Romains. César avait ordonné aux délégués germains (avant cette escarmouche) de faire venir le lendemain à son camp tous les chefs et les personnages les plus considérables pour parlementer. Quand ils arrivèrent au moment prescrit, il les fit enchaîner sous prétexte que la trêve avait été rompue par l'engagement de cavalerie, puis se jeta avec son infanterie sur les Germains sans défiance, en confiant à la cavalerie, démora­lisée par sa déroute de la veille, le soin d'égorger les femmes et les enfants. Ils périrent tous sans combat, ou sous le glaive, ou dans le Rhin, au nombre de 430 000. Après quoi César eut la bonté d'accorder la liberté aux chefs qu'il avait enchaînés ; mais ils furent réduits à le supplier de les garder dans son camp, craignant de périr dans les tortures entre les mains des Gaulois. Le Sénat accorda à César, pour cette victoire, des honneurs extraordinaires ; mais Caton proposa de le livrer, selon la vieille tradition romaine, nu et enchaîné aux Germains. Il fut seul de cet avis. Comme ces Germains avaient la réputa­tion d'être invincibles, cette victoire si facilement obtenue et sans un mort du côté romain a donné aux armes romaines un prestige qui explique seul la soumission si rapide des Gaules, et plus durable encore que rapide.\par
Au reste, à partir des guerres puniques, à partir exactement de la victoire sur Hannibal, l'histoire de Rome est pleine d'actes de brigandage. Un procédé fort efficace était celui qui consistait à gagner une alliance par des promesses afin d'écraser un ennemi ; l'ennemi une fois écrasé, l'allié était méthodique­ment humilié, ce qui en faisait un ennemi, car les injures ne sont jamais plus difficiles à accepter que quand on croit avoir des droits à la reconnaissance ; il était alors écrasé à son tour, l'ennemi précédemment vaincu se trouvant neutralisé par d'autres promesses fallacieuses ; après quoi on l'écrasait une seconde fois s'il n'était pas content. De tels procédés ruineraient un État manœuvrant parmi d'autres États égaux ; mais à partir d'un certain degré de puissance et surtout de réputation, la coutume d'observer la parole donnée devient à peu près inutile, parce que la crainte et l'espérance sont assez violen­tes pour produire d'elles-mêmes et sans le secours du jugement, dans l'âme des hommes, la croyance aux promesses et aux menaces. Afin de mettre la même bassesse de procédés dans les petites choses que dans les grandes, le général romain vainqueur se levait pour recevoir le chef ou le roi captif, ne souffrait pas qu'il lui embrassât les genoux, le faisait asseoir, lui ordonnait d'avoir bon courage, lui vantait la clémence du peuple romain, l'admettait à sa table ; tout cela pour l'empêcher de se tuer et l'amener vivant jusqu'au triom­phe ; après quoi on le laissait périr misérablement de faim, de froid ou de manque de sommeil. La destruction de Carthage, cette ville splendide, siège d'une civili­sation sans doute égale à celle des Romains (mais aussi, semble-t-il, non moins cruelle) est un des événements les plus atroces de l'histoire, et si cela ne mérite par le nom de brigandage, je ne sais pas ce qui peut le mériter. Elle a frappé une génération de Carthaginois qui n'avait jamais rien fait contre Rome, leurs pères avant été écrasés par le premier Scipion ; ils étaient réduits à venir implorer devant le Sénat, pleurant et prosternés à terre, la permission de se défendre contre les incursions continuelles des Numides. Rome choisit le moment où ils avaient été vaincus à plate couture par les Numides pour leur ordonner d'abord d'envoyer trois cents enfants nobles comme otages, ce qu'ils firent immédiatement ; puis de livrer toutes leurs armes, sans aucune excep­tion, tous les vaisseaux de guerre, tout ce qui avait trait à la navigation, promettant en échange le salut et la liberté de la cité. Cela une fois exécuté, les consuls, présents devant Carthage avec une flotte et une armée, annoncèrent aux délégués des Carthaginois qu'ils devaient quitter leur ville, et s'établir à dix kilomètres de la mer, et que Carthage serait entièrement rasée. Les délé­gués se jetèrent à terre, se frappèrent la tête contre le sol, supplièrent qu'on tuât plutôt tous les habitants sans exception et qu'on laissât subsister la ville (je comprends ce sentiment), tout cela en vain ; et enfin, épuisés de cris, de pleurs et de supplications, se soumirent. Mais ils ne purent persuader le peuple, qui, mû par le désespoir, soutint encore un siège de trois ans ; après quoi la nation et la cité disparurent de la surface de la terre, ce qui provoqua une joie délirante à Rome ; bien à tort, car peu de temps devait s'écouler jusqu'au moment où des esclaves, sous la protection de Marius, purent se permettre d'entrer dans les maisons des citoyens romains, d'y violer femmes et enfants, d'y massacrer à volonté, sans que nul osât résister. Quant à Auguste, l'adulation incroyablement basse qui était de règle sous son règne ne permet guère de se rendre compte de sa politique, et d'ailleurs, comme il pouvait tout sans exception, il n'avait pas lieu de contracter des engagements avec ou sans fidélité ; mais pour le temps où il ne se nommait encore qu'Octave, c'est autre chose ; s'il était mort avant la défaite d'Antoine, ou immédiatement après, je ne vois rien dans le misérable tissu de fourberies, de cruautés et de débauches effrénées qui forme la première partie de sa carrière politique qui pût l'élever au-dessus du niveau d'un César Borgia. Il n'a rien fait que de facile, et en s'emparant de l'Empire, et en le gouvernant ; la seule difficulté qu'il eût à vaincre était d'empêcher qu'on l'assassinât. Est-ce la faute de César Borgia s'il n'y avait pas à son époque un empire tout prêt pour quiconque pourrait mettre la main dessus ?\par
L'Empire romain est à mon avis le phénomène le plus funeste pour le développement de l'humanité que l'on puisse trouver dans l'histoire, puisqu'il a tué jusqu'à presque en supprimer les traces plusieurs civilisations, et tout l'échange prodigieux d'idées dans le bassin méditerranéen qui avait fait la grandeur de ce qu'on nomme l’antiquité ; mais c'est aussi le phénomène le plus vaste qu'on ait connu, et un des plus durables. D'où je conclus qu'il n'est pas exact, malgré ce que vous dites, que le brigandage soit impuissant à établir des choses solides. Et, réciproquement, qu'il est imprudent de compter sur le manque de solidité des choses établies par le brigandage. J'ai l'impression, au contraire, d'une assez grande solidité, du moins si on prend pour mesure de temps une fraction importante d'une vie humaine : et peut-être au-delà.\par
Tout cela me rappelle à l'esprit deux fort beaux vers de Racine dans {\itshape Bajazet} :\par

\begin{quoteblock}
 \noindent D'un empire si saint la moitié n'est fondée \\
Que sur la foi promise et rarement gardée.
 \end{quoteblock}

\noindent Ce « rarement gardée » n'est-il pas magnifique ?\par
En fait de belles formules du genre cynique, j'en ai relevé deux admirables l'autre jour dans Thucydide. L'une dans la bouche des Athéniens ordonnant à la petite île de Mélos, toujours libre jusqu'alors, de se soumettre à leur empire. (Ces Méliens refusèrent, se battirent, bien que n'étant qu'une poignée, et furent vaincus ; les hommes adultes furent tous exécutés, les femmes et les enfants vendus comme esclaves.)\par
« Vous le savez comme nous, tel qu'est fait l'esprit humain, on n'examine ce qui est juste que si la nécessité est égale de part et d'autre ; mais s'il y a un fort et un faible, tout ce qui est possible est fait par le premier et accepté par le second. »\par
Je trouve cela admirable, non seulement de franchise et de clarté, mais aussi de profondeur.\par
La seconde formule est dans la bouche d'Alcibiade, répondant à ceux qui l'accusent d'orgueil et d'insolence :\par
« Il est juste que celui qui a de grandes ambitions ne veuille pas de l'égalité. Car quand un homme est dans l'infortune, il ne trouve personne qui se fasse l'égal d'un malheureux. Ainsi, de même qu'on ne nous adresse pas la parole quand nous échouons, de même chacun doit supporter d'être traité en inférieur par ceux qui réussissent. Ou sans cela, qu'on accorde une considé­ration égale dans les deux cas et qu'on reçoive en retour le même traitement. »\par
Cela me fait penser à ceux qui sont autour de vous et que vous méprisez parce qu'ils sont aussi bas dans la mauvaise fortune qu'ils étaient insolents dans la bonne ; ce qui est presque toujours le cas. Somme toute, il est bien juste qu'il en soit ainsi : comme la prospérité leur donnait à leurs propres yeux le droit d'accabler de leur mépris les malheureux, de même, malheureux à leur tour, ils se soumettent au mépris que ceux qui ont aujourd'hui la prospérité leur infligent. Malheureusement ce n'est certainement pas cette pensée qui inspire leur attitude, mais plutôt, n'ayant jamais distingué leur fortune de leur existence même, ne s'étant jamais crus nés comme d'autres hommes pour souffrir des indignités, ils ne se reconnaissent plus eux-mêmes dans ce nouvel état et n'ont rien à quoi s'accrocher pour imposer une limite à leur propre abaissement. Il faut avouer qu'à moins d'être soutenu par un fanatisme quel­conque, qui rend tout facile, mais sans valeur (notamment ce singulier fanatis­me de l'Histoire qui est aujourd'hui la seule foi vivante), supporter décemment le changement de fortune dans un sens ou dans l'autre est le chef-d'œuvre de la plus haute vertu. Et, d'une manière générale, il n'est pas facile de bien supporter le malheur ; vu qu'une attitude de défi ne convient pas à la condition de l'homme, qui l'oblige à se plier à la nécessité, et que la soumission ne doit jamais pénétrer jusqu'à l'âme sous peine d'y effacer tout ce qui est humain. Il n'est pas étonnant que ceux qui ne se sont pas exercés toute leur vie à sentir qu'étant hommes aucune fortune, si extrême soit-elle, ne peut être trop haute ou trop basse pour leur convenir, ne sachent pas comment se conduire. J'ai bien souvent amèrement regretté qu'on n'ait pas de renseignements précis sur la vie d'Épictète ; car à ma connaissance c'est le seul exemple d'un homme réduit par le sort à l'extrême malheur - il ne peut exister de condition plus basse et plus douloureuse que celle d'un esclave à Rome - et dont on ait des raisons sérieuses de supposer, d'après l'accent de ses propres écrits, qu'il s'y est toujours parfaitement bien comporté.\par
La réflexion que vous faites dans votre avant-dernière lettre sur l'enfer, je l'ai faite presque identique quand je travaillais en usine. Mais comme il y a des gens qui meurent avant d'avoir jamais connu cet aspect de la condition humaine, il faut bien imaginer quelque disposition particulière pour eux, afin qu'ils complètent leur expérience. Je crois cependant que sur le globe terrestre la grande masse de l'espèce humaine n'a pas besoin de cela et n'en a jamais eu besoin. Cette pensée est presque intolérable.\par
Quant à ce qui se passe au Sud-Ouest de la France, je n'ai jamais été et ne suis encore pas de votre avis. D'abord, vos prévisions s'appuieraient sur des raisons très solides s'il s'agissait de phénomènes intérieurs ; mais il s'agit d'une entreprise de colonisation sur un territoire que tout (et d'abord son extrême richesse) destine à être colonisé, puisqu'au XVII\textsuperscript{e} siècle déjà c'étaient des étrangers qui y faisaient les guerres civiles. D'autre part, je crois comme vous que la décentralisation succédera un jour à une centralisation excessive ; mais je ne crois pas que cela doive se produire bientôt, ni peut-être même dans un avenir qui soit à l'échelle d'une vie humaine ; et surtout je suis tout à fait convaincue que la centralisation, une fois établie quelque part, ne disparaîtra pas avant d'avoir tué - non pas paralysé momentanément, mais tué - toutes sortes de choses précieuses et dont la conservation serait indispensable pour que le régime suivant soit un échange vivant entre milieux divers et mutuelle­ment indépendants, plutôt qu'un triste désordre. Pourquoi répète-t-on toujours ce lieu commun, selon lequel la force brutale serait impuissante à anéantir les valeurs spirituelles ? Elle les anéantit très vite et très facilement. On cite toujours les nationalités qui ont survécu à une oppression séculaire ; mais celles qui n'ont pas survécu sont bien plus nombreuses ; et les autres n'ont le plus souvent survécu que comme un fanatisme vidé de chaleur humaine. Combien de religions aussi ont été anéanties par la force, et presque jusqu'à leur souvenir ! Il n'y a rien au monde de si précieux, mais rien non plus de si fragile, qui périsse si facilement, ni qui soit si difficile ou même impossible à ressusciter, que la chaleur vitale d'un milieu humain, cette atmosphère qui baigne et nourrit les pensées et les vertus...
\subsubsection[II. (Variante de la lettre précédente)]{II. \\
(Variante de la lettre précédente)}
\noindent Je ne m'étonne pas de ce que vous m'écrivez sur ces gens qui étaient orgueilleux dans la prospérité et sont vils dans le malheur. C'est assez l'ordi­naire parmi les hommes, et l'on peut considérer que n'être ni l'un ni l'autre est le suprême effort de la plus haute vertu. Mais ceux dont il est question, surtout, sur quoi pourraient-ils, la plupart d'entre eux, s'appuyer contre la mauvaise fortune ? Ceux qui ne se sont jamais crus nés pour subir un sort semblable, qui n'ont jamais fait la réflexion - pourtant si simple - que ce que certains hommes supportent, tout homme peut le supporter ; qui n'ont jamais séparé l'idée qu'ils se faisaient d'eux-mêmes de la considération dont ils jouissaient, ils croient s'être perdus eux-mêmes et ne peuvent se retrouver quand ils sont précipités dans une catégorie d'hommes exposés aux insultes ; et il ne peut pas alors y avoir de limite à l'abaissement. Je n'ai pas d'admiration pour ceux qu'une croyance indéracinable ou en leur propre étoile, ou dans la victoire fatale et prochaine de la cause qu'ils représentent ou croient représen­ter, ou dans la gloire de leur nom auprès de la postérité (comme Napoléon à Sainte-Hélène) aident à rester fermes dans l'adversité ; ceux-là ne connaissent pas la véritable adversité. Ceux qui, sans aucun de ces secours, ne s'abaissent que dans des moments de défaillance, après une longue résistance, et savent toujours se reprendre, peuvent être considérés comme d'une vertu presque divine. C'est à peine si le Christ même va plus loin dans l'Évangile. J'ai bien souvent regretté qu'on ne sache rien de la vie d'Épictète : c'est presque le seul exemple que je connaisse d'une condition vraiment malheureuse - on ne peut guère imaginer quelque chose qui approche en malheur la situation d'un esclave des Romains - dont on ait des raisons sérieuses de supposer qu'elle a été réellement supportée en tous points comme il convient à un homme. Car comme il ne convient à l'homme ni de défier la force - lui qui est né et mis au monde pour se plier à la nécessité - ni d'y soumettre son âme, l'attitude juste n'est certes pas facile à trouver, encore moins à tenir. Et comment des gens qui ont été précipités au dernier degré du malheur avant d'avoir jamais pensé même à chercher cette attitude pourraient-ils la trouver ? Pour moi, qui tout enfant, dans tout ce que je lisais ou entendais raconter, me mettais toujours instinctivement, par indignation plutôt que par pitié, à la place de tous ceux qui souffraient une contrainte, je me regarde pour ainsi dire depuis ma nais­sance comme destinée, si le hasard...
\subsubsection[III.]{III.}
\noindent ... [La] violence est souvent nécessaire, mais il n'y a à mes yeux de grandeur que dans la douceur (je n'entends par ce mot rien de fade, vous le supposez bien).\par
Sans doute la contrainte est indispensable à la vie sociale, comme vous l'avez montré. Mais il y en a bien des formes. Certaines formes laissent subsister une atmosphère où les valeurs spirituelles (j'emploie ces mots faute d'autres) peuvent se développer. Ce sont les bonnes. D'autres les tuent. Quels idiots ont répandu le bruit que les idées ne peuvent pas être tuées par la force brutale ? Il n'y a pas d'opération plus facile. Quand elles sont tuées, il ne reste que des cadavres. D'autres, équivalentes, même identiques, peuvent surgir beaucoup plus tard ; mais la continuité dans la tradition spirituelle est un bien infiniment précieux, dont la perte est vraiment une perte. C'est par une espèce de miracle que surgissent sur une terre donnée, à certains moments, des for­mes de vie sociale où la contrainte ne détruit pas cette chose délicate et fragile qu'est un milieu favorable au développement de l'âme. Il y faut une vie sociale peu centralisée, des lois qui limitent l'arbitraire, et, dans la mesure où l'autorité s'exerce arbitrairement, une volonté d'obéissance qui permette de se soumettre sans s'abaisser.\par
Je crois qu'une conquête - et surtout une conquête colonisatrice - détruit généralement cela sur le territoire conquis, sauf quand elle s'opère par l'immi­gration et l'installation massive des conquérants (comme pour les barbares, ou les Normands en Angleterre). Il en est de même pour l'installation d'un pouvoir fort, méthodique, centralisé, et qui se fait adorer, comme sous Louis XIV (et grâce à votre Richelieu). Remarquez que la soumission totale à un roi n'a pas abaissé les Espagnols au XVI\textsuperscript{e} siècle et au début du XVII\textsuperscript{e} comme elle a abaissé les Français sous Louis XIV, parce que ce qu'ils adoraient, c'était leur propre serment et la vertu de loyauté ; ils pouvaient (conformément à l'étiquette) baiser les pieds du roi ou de n'importe quel supérieur sans rien perdre de leur fierté. Au lieu que sous la personne de Louis XIV c'est le pouvoir d'État qu'on adorait ; il en est résulté un abaissement effrayant.\par
\par
Entre parenthèses (ceci se rapporte à une de vos lettres déjà lointaine), c'est parce que je crois que les valeurs spirituelles meurent vraiment quand on les a tuées que je ne partage absolument pas vos espérances concernant le pays où j'ai fait ce voyage que vous désapprouviez.\par
À mes yeux, tout progrès dans le sens de la centralisation du pouvoir implique des pertes irréparables à l'égard de tout ce qui est vraiment précieux. Que l'excès de la centralisation doive inévitablement produire un jour (mais, je crois, après un délai beaucoup plus long que vous ne supposez) la décen­tralisation, je le crois comme vous. Mais ni les conditions matérielles d'une vie dispersée n'existeront avant longtemps, ni les valeurs spirituelles tuées ne ressusciteront. Sans doute, il en surgira d'autres, mais pas avant longtemps ; pas avant une longue période de désordre morne et ténébreux. Et rien ne remplacera ce qui a été perdu, comme rien ne remplace un être humain précieux prématurément tué.\par
– À ce propos, j'en ai découvert un, un grand poète du XVII\textsuperscript{e} siècle à peu près oublié, mort à trente-six ans, un an après être sorti d'un affreux cachot, plein de vermine et sans clarté sinon une demi-heure par jour, où il avait passé deux ans. Il se nomme Théophile. Si Richelieu n'a pas provoqué son supplice, il l'a permis, car il a eu tout le pouvoir quelques mois après que Théophile avait été mis en prison. Il a ainsi privé la France d'un poète de tout premier ordre, car, comme presque tous les poètes français (contrairement aux Anglais), Théophile était apparemment de développement tardif. Il manquait à un degré singulier de bassesse d'âme ; c'est pourquoi il n'a pu vivre jusqu'à l'âge où il aurait pu faire des poèmes assez continûment parfaits pour être retenus par la postérité. Ceux qui avaient dans le caractère un peu plus ou beaucoup plus de bassesse que lui ont vécu et ont atteint la gloire et l'immor­talité. Il était accusé d'impiété et de vers licencieux, et n'a échappé au bûcher que de justesse. Voici quelques-uns de ses vers pour vous divertir.\par
D'abord la fin d'un poème à la louange de Guillaume d'Orange :\par

\begin{quoteblock}
 \noindent Les astres dont la bienveillance \\
Se sent forcer de ta vaillance \\
Sont apprêtés pour t'accueillir ; \\
Déjà leur splendeur t'environne ; \\
Dieu comme fleurs les vient cueillir \\
Pour t'en donner une couronne \\
Qui ne pourra jamais vieillir.
 \end{quoteblock}

\noindent Ceci est écrit en prison :\par

\begin{quoteblock}
 \noindent Dieux, que c'est un contentement \\
Bien doux à la raison humaine \\
Que d'exhaler si doucement \\
La douleur que nous fait la haine ! \\
…………………………………… \\
Puisque l'horreur de la prison \\
Nous laisse encore la raison, \\
Muses, laissons passer l'orage ; \\
Donnons plutôt notre entretien \\
À louer qui nous fait du bien \\
Qu'à maudire qui nous outrage.\par
 Et mon esprit voluptueux \\
Souvent pardonne par faiblesse, \\
Et comme font les vertueux \\
Ne s'aigrit que quand on le blesse. \\
Encore, dans ces lieux d'horreur, \\
Je ne sais quelle molle erreur \\
Parmi tous ces objets funèbres \\
Me tire toujours au plaisir, \\
Et mon œil, qui suit mon désir, \\
Voit Chantilly dans ces ténèbres.
 \end{quoteblock}

\noindent Chantilly était la résidence du duc de Montmorency, son maître, selon le langage de l'époque, et qui ne l'avait pas abandonné dans son malheur. Il périt sur l'échafaud quelques années plus tard pour crime de rébellion.\par
Voici des vers à Montmorency, de la prison :\par

\begin{quoteblock}
 \noindent O ciel, que me faut-il choisir \\
Pour louer mon dieu tutélaire ? \\
Que ferai-je en l'ardent désir \\
Que mon esprit a de vous plaire ? \\
Je dirai partout mon bonheur, \\
Je peindrai si bien votre honneur \\
Que la mer qui voit les deux pôles \\
Dont se mesure l'univers \\
Gardera sur ses ondes molles \\
Le caractère de mes vers.
 \end{quoteblock}

\noindent Une fin de poème sur la guerre civile de 1620 (bien avant la prison) :\par

\begin{quoteblock}
 \noindent Quelque *……………* [J'ai oublié.] \\
Dont l'honneur flatte nos exploits, \\
Il n'est rien de tel que de vivre \\
Sous un roi tranquille, et de suivre \\
La sainte majesté des lois.
 \end{quoteblock}

\noindent Ceci encore en prison :\par

\begin{quoteblock}
 \noindent Quelle vengeance n'a point pris \\
Le plus fier de tous ces esprits \\
Qui s'irritent de ma constance ! \\
Ils m'ont vu, lâchement soumis, \\
Contrefaire une repentance \\
De ce que je n'ai point commis.\par
 Ha ! que les cris d'un innocent, \\
Quelques longs maux qui les exercent, \\
Trouvent malaisément l'accent \\
Dont ces âmes de fer se percent ! \\
Leur rage dure un an sur moi \\
Sans trouver ni raison ni loi \\
Qui l'apaise ou qui lui résiste ; \\
Le plus juste et le plus chrétien \\
Croit que sa charité m'assiste \\
Si sa haine ne me fait rien. \\
L'énorme suite de malheurs \\
Dois-je donc aux races meurtrières \\
Tant de fièvres et tant de pleurs, \\
Tant de respects, tant de prières, \\
Pour passer mes nuits sans sommeil, \\
Sans feu, sans air et sans soleil \\
Et pour mordre ici les murailles ! \\
N'ai-je encore souffert qu'en vain ? \\
Me dois-je arracher les entrailles \\
Pour soûler leur dernière faim ?
 \end{quoteblock}

\noindent Après la prison (donc moins d'un an avant sa mort)\par

\begin{quoteblock}
 \noindent Ôte-toi, laisse-moi rêver ; \\
Je sens un feu me soulever \\
Dont mon âme est toute embrasée. \\
Ô beaux prés, beaux rivages verts, \\
Ô grand flambeau de l'univers, \\
Que je trouve ma veine aisée ! \\
Belle aurore, douce rosée, \\
Que vous m'allez donner de vers !\par
 ……………………………………….\par
 Mon Dieu, que le soleil est beau ! \\
Que les froides nuits du tombeau \\
Font d'outrages à la nature ! \\
La mort, grosse de déplaisirs, \\
De ténèbres et de soupirs, \\
D'os, de vers et de pourriture \\
Étouffe dans la sépulture \\
Et nos forces et nos désirs.
 \end{quoteblock}

\noindent …………………………………\par
\par
J'espère que vous aimez Théophile \footnote{Voir la deuxième note de l'éditeur.}. Il est devenu un de mes amis les plus chers.\par
Je n'ai pas la place de parler de choses personnelles. Je pense que cette lettre, au moins pour la longueur, paiera des dettes passées... La mesure me manque, vous le voyez.\par
Amitiés.
\subsubsection[IV.]{IV.}
\noindent Cher ami,\par
Je continue à vous parler de Richelieu, puisque cela vous passionne com­me moi. Il ne livre pas le secret de sa domination sur le roi ; il ne livre d'ailleurs aucun secret, mais il permet d'en deviner à qui est attentif à lire entre les lignes. Je ne crois pas que cette domination ait été si difficile que la postérité l'a cru. Luynes, qui était un homme ordinaire, et qu'il dépeint comme au-dessous du médiocre - il est vrai qu'il le haïssait - l'a exercée sans la moindre difficulté jusqu'à sa mort, et si son pouvoir, qui était absolu, n'a duré que quatre ans, nul ne peut dire combien il aurait duré si la mort n'était venue l'interrompre. Un ressort secret jouait infailliblement entre les mains de celui qui était maître du gouvernement, et par suite de la police : la peur de l'assas­sinat. Toute cette famille royale, qui avait vu mourir Henri IV, était obsédée, au sens le plus fort du mot, par cette peur, aussi bien Marie de Médicis et Gaston d'Orléans, qui n'avait pourtant alors que deux ans, que Louis XIII. Gaston d'Orléans, quand il n'avait encore que dix-huit ans, paraissait tellement plus capable que son frère, âgé alors de vingt-cinq ans, de soutenir le fardeau de l'État, qu'on parlait couramment de substituer l'un à l'autre ; Louis XIII le savait très bien. Chacun des deux frères, à des périodes différentes, a regardé sa mère comme tout à fait capable de le faire tuer. Luynes le premier avait semé et entretenu quatre ans dans l'esprit du roi cette mortelle méfiance à l'égard de sa mère et de son frère. Les deux ministres qui l'ont suivi dans la faveur du roi n'ont certainement pas su appliquer ce procédé ; aussi ont-ils passé rapidement. Richelieu a commencé par ruiner l'ancien parti de Luynes, déjà très abaissé avant lui, qui s'était rassemblé autour de Gaston d'Orléans ; leur manie des conjurations en chambre et la haine du roi à l'égard de son frère le lui a permis. Pour ruiner ensuite la reine mère et tout son parti, dans lequel Richelieu lui-même avait, du temps de Luynes, pris part à une guerre civile contre l'État, il lui a suffi de réveiller les terreurs et les méfiances que Luynes avait profondément implantées dans un esprit de seize ans. Pour empêcher le roi de songer à le disgracier, Riche lieu lui montrait, non pas l'État livré au désordre, mais, mobile autrement puissant, sa personne livrée aux assassins. Cette crainte n'était pas chimérique, puisqu'il a longtemps existé en France, dans ces profondeurs dont l'histoire ne parle pas, une mystique du régicide. Tout au début de la Fronde, Retz a trouvé sur un homme du peuple un mé­daillon où il y avait : Saint Jacques Clément. Aussi je ne pense pas qu'il y ait jamais eu péril de disgrâce ; les périls apparents semblent avoir été des effets de mise en scène. Le vrai péril, et qu'il fallait une âme bien forte pour surmon­ter, était celui de la mort du roi avant que le pouvoir de Richelieu fût par lui-même assez solide. Il aurait été livré, soit aux misères de l'exil parmi les peuples étrangers qui tous, ennemis ou alliés, le haïssaient, soit à celles de quelque affreux cachot de la Bastille. Sa grandeur d'âme a consisté à ne pas se chercher de ressources pour un tel événement, à mettre tout son jeu sur le tableau de la vie du roi, et d'un roi si chancelant que tout le monde attendait continuellement sa mort. Il a su conserver tout son calme et toute sa froide application aux affaires de l'État dans les moments mêmes où le roi semblait presque à l'agonie. Cela est certes admirable, et frappait de terreur ses adver­saires. Au reste chacune de ces maladies donnait ensuite de nouvelles ressour­ces à Richelieu, par des rapports sur la joie que tels et tels avaient manifestée. Quant à lui-même, la haine de la reine mère, de la Reine régnante, de Mon­sieur, des grands, des corps constitués à son égard était pour le roi la meilleure raison de se fier à lui comme ne pouvant souhaiter sa mort ; sa brouille avec la reine mère le délivra des seules raisons que pût avoir le roi de le soupçonner.\par
Sa domination sur la France livrée à l'anarchie ne me paraît pas non plus si difficile, et pour la même raison, c'est que Luynes y avait également bien réussi. L'anarchie était précisément la meilleure ressource. Le souvenir des guerres civiles qui avaient désolé le XVI\textsuperscript{e} siècle, l'impitoyable cruauté des soldats, qui tous, quel que fût leur maître, pillaient, tuaient, torturaient, ran­çonnaient, violaient les jeunes filles jusque dans les églises, faisait que le peuple, quelle que fût sa haine du pouvoir, était toujours passionnément hostile à quiconque s'opposait au pouvoir par les armes. Luynes n'a eu aucun mal à maintenir la paix en France et écraser en un combat des fauteurs de guerre civile ; il est trop facile d'insinuer qu'il était proche de la chute quand il est mort, car on n'en sait rien, et on a sans cesse dit la même chose de Richelieu pendant sa vie. Richelieu n'a eu aucun mal à ne pas craindre les conjurations et les guerres intérieures contre l'État, en ayant éprouvé lui-même la faiblesse pour y avoir pris part. Il vit le parti de la reine mère, combattant un pouvoir exécré, hors d'état de trouver un appui dans les villes et décomposé par la terreur au seul aspect des troupes du roi. Le parti huguenot était seul capable de plus de fermeté ; aussi après un siège qui a exterminé presque toute une ville par la faim, la France était à lui.\par
Ce qui était difficile, c'était de rendre efficace ce pouvoir qu'il possédait sur le roi et le pays, d'obtenir des actions effectives et coordonnées à une époque où, excepté l'élite de la bourgeoisie, personne n'avait le sens de l'obéis­sance ou d'un devoir quelconque envers l'État. Que diraient ceux qui raillent le régime parlementaire s'ils savaient assez d'histoire pour se souvenir de ce qui se passait aux moments mêmes les plus glorieux du gouvernement de Richelieu ? Mais, en sa présence, les choses se faisaient ; et avec le temps, il est arrivé à ce qu'une fois sur deux environ les choses se fassent même en son absence. C'était un chef. Il possédait à un haut degré ce magnétisme qui est, je crois, quelque chose de purement physique, indépendant des qualités du cœur ou de l'esprit, mais qui, lorsqu'il s'y joint, accomplit des prodiges.\par
Il a eu certes raison d'abaisser une noblesse dont le jeu préféré était la guerre civile agrémentée de haute trahison, et d'ailleurs incapable. Sous lui la guerre, soit civile, soit étrangère, n'a que rarement désolé le territoire de la France. Mais il résulte de ses propres Mémoires que, bien qu'il le nie, il a délibérément et impitoyablement entretenu les guerres en Europe, soit qu'il y fit ou non participer la France. Dans son désir passionné d'abaisser la maison d'Autriche, il a empêché toute paix en Allemagne ; et il est impossible d'exa­gérer les horreurs qu'a souffertes pendant cette guerre interminable ce malheureux pays, où les armées brûlaient ou coupaient tous les blés sur leur passage, détruisaient les moulins, et passaient des villes entières au fil de l'épée. Sans doute le premier devoir d'un chef d'État est envers son propre pays ; pourtant, pour juger Richelieu, il me semble qu'il faut tenir compte, autant que le peut la conjecture, de tous les maux qu'il a épargnés et de tous ceux qu'il a causés aux hommes. Beaucoup diraient qu'il ne faut pas le juger sur les idées de notre temps, parce qu'on n'avait pas alors les mêmes opinions qu'aujourd'hui sur la paix et la guerre ; mais c'est faux, on avait exactement les mêmes. Le contraire serait étonnant, puisqu'il n'y a peut-être pas une idée morale parmi les hommes qui ne remonte au-delà des temps historiques, bien que notre ignorance et notre orgueil nous persuadent qu'il y en a. Richelieu dut même, pour apaiser les scrupules du roi, lui faire garantir par des théolo­giens que les misères de la guerre ne lui seraient pas imputées, puisqu'il l'entreprenait pour une juste cause (la défense du duc de Mantoue, qui se trouvait sujet français). D'autre part, à quoi servait aux paysans français que la guerre ne fût pas sur leur territoire, alors qu'un grand nombre d'entre eux étaient réduits par la misère à manger de l'herbe ? Ce n'est pas là de la litté­rature, car Richelieu, analysant un manifeste de Gaston d'Orléans où cet excès de misère lui est reproché, répond, non pas qu'il y a exagération, mais seulement que Gaston d'Orléans et ses gens en sont eux-mêmes pour une part responsables par leurs folles dépenses.\par

\begin{center}
Simone Weil : Écrits historiques et politiques.\end{center}

\begin{center}
Première partie : Histoire\end{center}
\subsection[8. Conditions d'une révolution allemande, (25 juillet 1932)]{8. \\
Conditions d'une révolution allemande \\
(25 juillet 1932)}
\noindent \par

\begin{quoteblock}
 \noindent (« Et maintenant ? » par Léon Trotsky)
 \end{quoteblock}

\noindent C'est une étude de Trotsky, écrite en janvier dernier, au sujet de la situa­tion en Allemagne. Elle n'a, aujourd'hui encore, rien perdu de son intérêt. Au milieu du désarroi, du découragement général, Trotsky, exilé, isolé, calomnié en tous pays par tous les partis, les quelques amis qui lui sont restés fidèles en Russie presque tous morts, déportés ou en prison, a su seul garder intacts son courage, son espérance, et cette lucidité héroïque qui est sa marque propre. En cette étude se manifeste une fois de plus cette faculté, propre à l'homme d'action véritable, de passer froidement en revue tous les éléments de n'im­porte quelle situation, et de maintenir pourtant cette analyse, menée avec une probité théorique sans reproche, orientée tout entière vers l'action immédiate.\par
Il s'agit, à vrai dire, moins de l'Allemagne que de la situation mondiale étudiée à travers l'Allemagne. La question qui se pose en ce moment dans le monde, la question que pose la crise actuelle de l'économie capitaliste, est, aux yeux de Trotsky, la suivante : fascisme ou révolution. L'étape actuelle du régi­me capitaliste - la plupart des études des économistes bourgeois tendent à cette conclusion - n'est plus compatible, ni avec le libéralisme économique, ni par suite avec la démocratie bourgeoise. L'économie et l'État doivent être dirigés par les ouvriers pour les ouvriers, ou par le capitalisme des banques et des monopoles contre les ouvriers. La crise pose la question d'une manière aiguë. Les moyens policiers ordinaires ne suffisent plus à maintenir la société capitaliste en équilibre. C'est l'heure de la tactique fasciste. Cette tactique consiste à mettre en mouvement, à exciter au combat contre les ouvriers « les masses de la petite bourgeoisie enragée, les bandes de déclassés, les lumpen-prolétaires démoralisés, toutes ces innombrables existences humaines que le capital financier lui-même poussa au désespoir et à la rage ». - Quant à la victoire du fascisme, elle « aboutit à l'accaparement direct et immédiat, par le capital financier, de tous les organes et institutions de domination, de direction et d'éducation : l'appareil d'État et l'armée, les municipalités, les universités, les écoles, la presse, les syndicats, les coopératives. » La fascisation de l'État signifie... « avant tout et surtout : détruire les organisations ouvrières, réduire le prolétariat à un état amorphe, créer un système d'organismes pénétrant profondément dans les masses et qui sont destinés à empêcher la cristallisation indépendante du prolétariat. C'est précisément en cela que consiste le régime fasciste. » Ainsi, tout se trouve menacé à la fois ; non pas simplement la classe ouvrière, mais aussi toutes les conquêtes de la bourgeoisie libérale, et, d'une manière générale, toute la culture.\par
L'Allemagne, encerclée par les nations qui l'ont vaincue en 1918, privée de colonies, son économie désorganisée, d'une part, par les réparations, accablée, d'autre part, par l'outillage industriel monstrueux qu'elle s'est donné et dont une grande partie n'a jamais fonctionné, est le pays où la crise est la plus aiguë. Il n'est donc pas étonnant que le problème : fascisme ou révolution, ne soit encore qu'un problème propre à l'Allemagne. Mais, à en croire Trotsky, l'Allemagne est appelée à résoudre ce problème pour le monde entier. Il est clair, en tout cas, que la révolution allemande, se joignant, par-dessus la Pologne, à la révolution russe et lui imprimant un nouvel essor, constituerait une force révolutionnaire formidable. D'autre part, le fascisme allemand, fai­sant bloc avec le fascisme italien et les pays de terreur blanche qui entourent le Danube, appuyant les courants fascistes qui se manifestent déjà dans divers pays d'Europe et aux États-Unis, menacerait le monde entier et jusqu'à l'U.R.S.S.\par

\begin{center}
*\end{center}
\noindent Devant cette situation tragique, ceux qui ont coutume de parler au nom des ouvriers allemands ferment les yeux. Il n'est nécessaire d'examiner ici en détail que les deux grands partis, le parti social-démocrate et le parti commu­niste ; dans la poussière d'organisations dissidentes qui les entourent, il n'en est aucune qui semble avoir jusqu'ici une importance politique, exception faite, peut-être, pour le « parti socialiste ouvrier » ; mais cette aile gauche de la social-démocratie n'a pas de programme politique précis, en dehors du front unique. Les deux grands partis, sous des prétextes différents, restent dans une inaction également criminelle. Tous deux sont menés par des bureaucrates, qui aiment toujours mieux attendre on ne sait quelle autre occasion d'agir, plutôt que de saisir l'occasion actuelle, qui ne reviendra jamais. Les social-démo­crates ne pensent qu'à reculer le moment où, pour les ouvriers allemands, se posera la question de la prise du pouvoir. Mais, dit Trotsky, ce moment ne peut plus être reculé. « La putréfaction du capitalisme signifie que la question du pouvoir doit se résoudre sur la base des forces productives actuelles. En prolongeant l'agonie du régime capitaliste, la social-démocratie n'aboutit qu'à la décadence continue de la culture économique, au morcellement du prolé­tariat, à la gangrène sociale. Elle n'a plus aucune autre perspective devant elle ; demain, ce sera pire qu'aujourd'hui ; après-demain - pire que demain. » La social-démocratie, fermant les yeux, comme dit encore Trotsky, devant l'avenir, rêvant d'on ne sait quel progrès dans le cadre du régime, s'est trouvée acculée à la politique misérable qui consistait à s'appuyer, contre la menace hitlérienne, uniquement sur l'État et sur la police. Les sarcasmes de Trotsky au sujet de cette politique sont encore bien moins cruels que le démenti cinglant donné aussitôt par les faits eux-mêmes. Il n'importe. La grande masse des ouvriers allemands reste encore menée par des bureaucrates incapables de se détacher de l'État capitaliste auquel ils ont été si longtemps entièrement soumis.\par
Le parti communiste, lui, n'est pas passé de la sorte à l'ennemi. Il ne s'est pas mis sous la dépendance de l'État allemand. Mais en revanche, étant complètement gouverné par la bureaucratie d'État russe, il est, tout comme le parti social-démocrate, pris de ce vertige qui saisit tout bureaucrate placé devant la nécessité d'agir. Lui aussi préfère fermer les yeux et attendre. Pendant longtemps la théorie, ouvertement avouée, des bureaucrates placés à la tête du parti communiste allemand a été qu'on pouvait très bien laisser Hitler arriver au pouvoir ; qu'il s'y userait très vite, et ouvrirait la voie pour la révolution. Pourtant l'expérience italienne a montré trop clairement que la conquête de l'État par les bandes fascistes signifie dissolution des organisa­tions ouvrières et extermination des militants. Certes le parti communiste allemand a abandonné bien qu'un peu tard, cette théorie plus lâche encore que sotte ; actuellement ses membres se font tuer tous les jours en combattant les bandes hitlériennes. Mais le parti comme tel a gardé, au fond, à peu près la même attitude. Il attend. Au lieu de profiter du conflit aigu qui oppose la social-démocratie et même le centre catholique au fascisme, il attend, pour agir, que la classe ouvrière allemande ait fait le front unique sous sa direction ; autrement dit, il attend avant d'agir d'avoir tous les ouvriers allemands derrière lui. Mais en attendant, comme il ne fait rien, il n'est pas capable d'attirer même quelques milliers d'ouvriers social-démocrates. Il se contente de nommer superbement fasciste tout ce qui n'est pas communiste. Les bolchéviks avaient aidé Kerensky à écraser Kornilov, avec l'intention d'écraser ensuite Kerensky lui-même ; le parti communiste allemand, à un moment où sa seule chance de salut réside dans le conflit qui met aux prises ses divers ennemis, fait ce qu'il peut pour les souder en un seul bloc. Ceux qui, comme Trotsky, recomman­dent une autre tactique, il les traite de « conciliateurs » qui voudraient « sépa­rer la lutte contre le fascisme de la lutte contre le social-fascisme ». Ainsi les bureaucrates de la social-démocratie et les bureaucrates du parti communiste s'aident mutuellement, sans en avoir conscience, à demeurer de part et d'autre dans leur quiétude bureaucratique.\par
Il n'est pas très étonnant que, parmi les ouvriers allemands, les mieux adaptés au régime suivent aveuglément de simples bureaucrates ; il est plus étonnant qu'il en soit de même pour les plus révolutionnaires. La cause en est la subordination étroite de l'{\itshape Internationale communiste} à un appareil d'État. Cet état, à en croire Trotsky, se définit comme « une dégénérescence bureau­cratique de la dictature », comme « une dictature personnelle s'appuyant sur un appareil impersonnel, qui saisit à la gorge la classe dominante du pays ». On objectera que les gouvernants russes actuels ont, en Russie même, accompli de grandes choses. Mais une bureaucratie d'État, dans le pays qu'elle gouverne, peut, employant d'un côté la contrainte, s'aidant d'autre part de l'élan merveilleux imprimé par la Révolution d'Octobre, accomplir des progrès industriels bien supérieurs à ce que peut faire la bureaucratie des sociétés anonymes dans les pays capitalistes, où l'économie est anarchique et où n'existe aucun élan. Ce qu'une bureaucratie est incapable de faire, c'est une révolution. Les deux caractères d'une bureaucratie d'État sont la peur devant l'action décisive, et ce que Trotsky nomme « l'ultimatisme bureaucratique ». « L'appareil stalinien ne fait que commander. Le langage du commandement, c'est le langage de l'ultimatum. Chaque ouvrier doit reconnaître par avance que toutes les décisions précédentes, actuelles et futures, du Comité central, sont infaillibles. » Ces deux caractères n'affaiblissent pas une bureaucratie d'État comme telle, et ne l'empêchent même pas de faire des merveilles dans le domaine industriel, bien qu'ils empêchent tout progrès vers un régime socialiste. Mais l'État russe a imprimé ces, deux caractères à toute l'Interna­tionale. Or « dans l'Union soviétique la révolution victorieuse créa tout au moins les prémices matérielles pour l'ultimatisme bureaucratique sous forme d'appareil de coercition. Dans les pays capitalistes, y compris l'Allemagne, l'ultimatisme se transforme en une caricature impuissante et entrave la marche du parti communiste vers le pouvoir. » Tous ceux qui connaissent les excom­munications lancées, par exemple, par Semard contre tous ceux « qui n'approuvent pas entièrement la ligne du Parti » reconnaîtront la justesse de cette appréciation.\par
Ainsi la dictature bureaucratique qui pèse sur la classe ouvrière russe étouffe aussi la révolution allemande. Si les Russes la secouaient, ce serait un puissant secours pour les ouvriers allemands. Inversement la révolution allemande imprimerait à la révolution russe un nouvel essor qui balayerait sans doute l'appareil bureaucratique. À en croire Trotsky, la bureaucratie stali­nienne s'est appuyée jusqu'ici principalement sur le chômage, chaque ouvrier russe aimant mieux se taire que de perdre sa place ; ainsi les succès écono­miques dont Staline et ses amis sont si fiers ébranleront leur pouvoir ; car il n'y a plus de chômage en Russie. Mais, même en admettant qu'elle se trouve au bord de l'abîme, la bureaucratie russe a encore le temps de faire échouer la révolution allemande.\par

\begin{center}
*\end{center}
\noindent \par
Que faire pour que la révolution allemande triomphe ? Trotsky, pour résoudre cette question, se place du point de vue d'un parti communiste digne de ce nom ; et, à ce parti imaginaire, il donne un programme simple et gran­diose. Avant tout, former le front unique ; non pas, selon la formule communiste orthodoxe : « faire le front unique par la base », c'est-à-dire inviter les ouvriers social-démocrates à abandonner leurs propres organisa­tions pour se ranger sous la bannière du parti communiste ; mais inviter au front unique les organisations social-démocrates telles qu'elles sont, chefs et ouvriers ensemble, en vue d'objectifs déterminés, à savoir la préparation d'une lutte à main armée contre les bandes hitlériennes. Favoriser partout la forma­tion de Comités comprenant les délégués des ouvriers, sans distinction de partis, autrement dit de Soviets, ceux-ci étant considérés comme « les organes suprêmes du front unique ». Donner aux ouvriers, comme objectif immédiat, le contrôle ouvrier sur la production, mot d'ordre propre au début d'une période révolutionnaire ; à ce mot d'ordre, joindre celui d'une large collabo­ration industrielle entre l'Allemagne et la Russie en vue du second plan quinquennal, collaboration qui serait assurée par une entente entre les organi­sations ouvrières allemandes et l'État soviétique, et qui permettrait de faire fonctionner à plein rendement des centaines de grandes usines actuellement arrêtées.\par
Mais quel moyen de faire appliquer un tel programme ? Trotsky n'en voit qu'un seul : le « redressement » du parti communiste allemand, opéré grâce à la pression de l' « opposition de gauche ». Tous les partis social-démocrates ou communistes ont pourtant jusqu'ici joué dans l'histoire le rôle le plus pitoyable ; seul fait exception le parti bolchevik russe, qui, par son caractère d'organisation illégale, différait essentiellement des partis ordinaires. Mais Trotsky, gardant pour le parti communiste un attachement qu'on ne peut s'em­pêcher de juger superstitieux, se refuse à admettre qu'une révolution puisse triompher sinon sous sa direction.\par
Six mois après la date où Trotsky terminait cette étude, le seul symptôme de « redressement » du parti communiste allemand consistait dans une décla­ration de Thaelmann, selon laquelle le Parti consentirait à proposer le front unique même aux chefs social-démocrates, si un fort courant de masses l'y entraînait. À quoi est bon un parti qui ne sait que suivre les masses avec quelque retard ? Quelques efforts qu'on fasse pour la « redresser », on ne peut rendre une bureaucratie capable de diriger une guerre civile. Si le mouvement révolutionnaire allemand, qui commence à s'esquisser en quelques villes où se constituent des comités de front unique, se développe victorieusement, ce sera contre les bureaucraties de partis, en brisant les bureaucraties de partis. Les syndicats eux-mêmes, qui seraient les organes naturels d'un mouvement de front unique, semblent avoir été mis par leur longue et étroite subordination à la bureaucratie social-démocrate dans l'impossibilité de remplir une pareille tâche.\par
C'est bien là ce qui constitue le caractère tragique de la situation actuelle en Allemagne. Chargés de défendre, contre les bandes fascistes, l'héritage du passé, les espérances de l'avenir, les ouvriers allemands ont contre eux tout ce qui est pouvoir constitué, tout ce qui est installé dans une place. État allemand et État russe, partis bourgeois et représentants attitrés des ouvriers,, tout contribue, soit comme obstacle à vaincre, soit comme poids mort à soulever, à paralyser les ouvriers allemands. Or victoire et défaite, dans une bataille, dépendent d'une question d'heures.\par
25 juillet 1932.\par
({\itshape Libres Propos}, nouvelle série, 6\textsuperscript{e} année, n° 8, août 1932)\par

\subsection[9. Premières impressions d'Allemagne  (25 août 1932)]{9. \\
Premières impressions d'Allemagne \protect\footnotemark  \\
(25 août 1932)}
\footnotetext{ Titre donné par la revue {\itshape La Révolution prolétarienne} à cet extrait d'une lettre de S. W., que la revue publia en le faisant précéder de ces mots : « Une camarade française, partie récemment en Allemagne, nous fait part de ses premières impressions. » (Note de l'éditeur.)}
\noindent \par
... Politiquement, tout est toujours tranquille. On est moins fiévreux concernant les événements allemands ici qu'à Paris. À peine si on voit quelques nazis en uniforme dans la rue, et ils se conduisent comme tout le monde. On lit le matin dans les journaux qu'il y a eu ici et là quelques attentats, à peu près dans le même état d'esprit qu'on lit qu'il y a eu tant d'accidents d'automobile. Les journaux ennemis ne s'affrontent pas dans les métros. On ne discute pas politique.\par
Pour les travailleurs, la question qui est en suspens, c'est l' « Arbeitsdienst », ces camps de concentration pour chômeurs qui existent actuellement sous forme de camps où l'on peut aller volontairement (10 pfennigs par semaine), mais qui deviendraient obligatoires sous un gouverne­ment hitlérien. En ce moment n'y vont que les plus désespérés. On n'imagine pas cette magnifique jeunesse ouvrière allemande qui fait du sport, du cam­ping, chante, lit, fait faire du sport aux enfants, réduite à ce régime militaire.\par
Une autre question, c'est celle, non seulement de l'interdiction du parti communiste, mais d'un massacre systématique des meilleurs éléments. Les journaux nazis sont pleins d'appels au meurtre et disent ouvertement : « Il ne faut pas nous énerver maintenant (c'est-à-dire ne pas faire d'attentats), attendons d'avoir le pouvoir. » Les ouvriers attendent simplement l'heure où tout cela s'abattra sur eux. La lenteur même du processus augmente la démora­lisation. Ce n'est pas le courage qui manque, mais les occasions de lutter ne se présentent pas.\par
L'idéologie nazie est étonnamment contagieuse, notamment chez le parti communiste. Dernièrement, les nazis tonnaient contre le fait qu'une « femme juive marxiste » (Clara Zetkin) allait présider la séance de rentrée du Reichstag. À quoi la Welt am Abend (journal officieux du parti) répondit : D'abord Clara Zetkin n'est pas juive. Et puis, si elle l'était, ça ne ferait rien. Rosa Luxembourg, {\itshape bien que juive}, était une tout à fait « ehrliche Person » (une personne honorable)... ! Quant au nationalisme, le parti communiste en est (paraît-il) incroyablement imprégné, il appelle les social-démocrates « Landesverräter » (traîtres à la patrie), etc.\par
Mon impression jusqu'ici est que les ouvriers allemands ne sont nullement disposés à capituler, mais qu'ils sont incapables de lutter. Les communistes et les social-démocrates accusent chacun (et très justement) le parti adverse de ne mériter aucune confiance, et cela même parmi les plus honnêtes militants de la base (exemple : l'ouvrier communiste chez qui j'habitais et qui est contre le front unique). Division d'autant plus grave que les communistes sont des chômeurs, alors que les social-démocrates travaillent. À cela s'ajoute que ceux qui chôment depuis deux, trois, quatre, cinq ans ne sont plus capables de l'énergie que demande une révolution... Des jeunes gens qui n'ont jamais tra­vaillé, las des reproches de leurs parents, se tuent ou s'en vont vagabonder, ou se démoralisent complètement. On voit des enfants d'une maigreur effrayante, des gens qui chantent lamentablement dans les cours, etc. D'autre part cette question terrible des camps de concentration pour chômeurs ne touche pas les ouvriers qui travaillent - et chez les chômeurs mêmes, sans doute ce régime d'esclavage militaire est-il le seul que puissent supporter les plus démora­lisés... Au contraire ceux qui font des sports, de la propagande politique, etc., ne pourront pas le supporter. Mais il est à craindre qu'ils luttent seuls et soient exterminés...\par
({\itshape La Révolution prolétarienne},\par
8° année, n° 134, 25 août 1932.)\par

\subsection[10. L'Allemagne en attente, (Impressions d'août et septembre) (1932)]{10. \\
L'Allemagne en attente \\
(Impressions d'août et septembre) \\
(1932)}
\noindent \par
Celui qui, ces temps-ci, venant de France, arrive en Allemagne, a le sentiment que le train l'a amené d'un monde à un autre, ou plutôt d'une retraite séparée du monde dans le monde véritable. Non pas que Berlin soit en fait moins calme que Paris ; mais le calme même a là-bas quelque chose de tragi­que. Tout est en attente. Les problèmes concernant la structure de la société humaine se posent. Ils ne se posent pas comme en France, où ils appartiennent à un domaine à part, le domaine de la politique, comme on dit, c'est-à-dire, en somme, le domaine des journaux, des élections, des réunions publiques, des discussions dans les cafés, et où les problèmes réels sont ailleurs pour chacun. En Allemagne, en ce moment, le problème politique est pour chacun le problème qui le touche de plus près. Pour mieux dire, aucun problème concer­nant ce qu'il y a de plus intime dans la vie de chaque homme n'est formulable, sinon en fonction du problème de la structure sociale. Les révolutionnaires enseignent depuis longtemps que l'individu dépend étroitement, et sous tous les rapports, de la société, laquelle est elle-même constituée essentiellement par des relations économiques ; mais ce n'est là, en période normale, qu'une théorie. En Allemagne cette dépendance est un fait auquel presque chacun se heurte sans cesse plus ou moins fort, mais toujours douloureusement.\par
La crise a brisé tout ce qui empêche chaque homme de se poser complète­ment le problème de sa propre destinée, à savoir les habitudes, les traditions, les cadres sociaux stables, la sécurité ; surtout la crise, dans la mesure où on ne la considère pas, en général, comme une interruption passagère dans le développement économique, a fermé toute perspective d'avenir pour chaque homme considéré isolément. En ce moment, cinq millions et demi d'hommes vivent et font vivre leurs enfants grâce aux secours précaires de l'État et de la commune ; plus de deux millions sont à la charge de leur famille, ou mendient, ou volent ; des vieillards en faux col et chapeau melon, qui ont exercé toute leur vie une profession libérale, mendient aux portes des métros et chantent misérablement dans les rues. Mais le tragique de la situation réside moins dans cette misère elle-même que dans le fait qu'aucun homme, si énergique soit-il, ne peut former le moindre espoir d'y échapper par lui-même. Les jeunes surtout, qu'ils appartiennent à la classe ouvrière ou à la petite bourgeoisie, - eux pour qui la crise constitue l'état de chose normal, le seul qu'ils aient connu, - ne peuvent même pas former une pensée d'avenir quel­conque se rapportant à chacun d'eux personnellement. Ils ne peuvent pas, la politique mise à part, former même des projets d'action ; ils sont ou peuvent être, d'un moment à l'autre, réduits à l'oisiveté, ou plutôt à l'agitation harassante et dégradante qui consiste à courir d'une administration à l'autre pour obtenir des secours. Nul n'espère pouvoir, grâce à sa valeur profession­nelle, garder ou trouver une place. Cherchent-ils une consolation dans la vie de famille ? Tous les rapports de famille sont aigris par la dépendance absolue dans laquelle se trouve le chômeur par rapport au membre de sa famille qui travaille. Les chômeurs qui ont dans leur famille quelqu'un qui travaille, et les jeunes de moins de vingt ans sans exception, ne touchent aucun secours. Cette dépendance, dont l'amertume est encore accrue par les reproches des parents affolés par la misère, chasse souvent les jeunes chômeurs de la maison paternelle, les pousse au vagabondage et à la mendicité. Quant à fonder soi-même une famille, à se marier, à avoir des enfants, les jeunes Allemands ne peuvent en général même pas en avoir la pensée. La pensée des années à venir n'est remplie pour eux d'aucun contenu.\par
L'avenir immédiat n'est pas plus sûr que l'avenir lointain. Dans la vie au grand air et au soleil, dans les lacs et les fleuves, dans la gymnastique, la musique, la lecture, dans la responsabilité de la vie publique, enfin dans une fraternelle camaraderie, la meilleure partie de la jeunesse allemande ne trouve qu'une précaire consolation. Chaque chômeur, à mesure que le temps s'écoule, voit les secours qu'il reçoit diminuer, et s'approcher le moment où, chômant depuis trop longtemps, il ne touchera plus rien. Des camps de concentration pour jeunes chômeurs, où l'on travaille sous une discipline militaire pour une solde de soldat (Arbeitsdienst), reçoivent ceux qui, étant sans ressources, aiment mieux aller là que de vagabonder misérablement ou d'aller s'engager dans la Légion étrangère française ; ces camps ne recrutent encore que des volontaires ; mais tous les partis réactionnaires parlent d'y envoyer les jeunes chômeurs de force, contraignant ainsi les meilleurs, ceux qui ont su se faire malgré tout une vie humaine à tout abandonner. En somme, le jeune Allemand, ouvrier ou le petit bourgeois, n'a pas un coin de sa vie privée qui soit hors d'atteinte de la crise. Pour lui les perspectives bonnes ou mauvaise, concernant les aspects même les plus intimes de son existence propre se formulent immédiatement comme des perspectives concernant la structure même de la société. Il ne peut même rêver d'un effort à faire pour reprendre son propre sort en main qui n'ait la forme d'une action politique. La somme d'énergie dont la plus grande part est d'ordinaire absorbée par la défense des intérêts privés se trouve ainsi, dans l'Allemagne actuelle, portée presque tout entière sur les rapports économiques et politiques qui constituent l'ossature de la société elle-même.\par

\begin{center}
*\end{center}
\noindent Cette énergie reste latente. Dans une situation semblable, qui semble répondre parfaitement à la définition d'une situation révolutionnaire, tout demeure passif. L'observateur, frappé par la convergence de toutes les pensées sur le problème politique, est aussitôt frappé, et plus vivement encore, par l'absence d'agitation, de discussions passionnées dans les rues ou les métros, de lecteurs se jetant anxieusement sur leur journal, d'action ébauchée ou seulement concertée. Cette contradiction apparente constitue le caractère essentiel de la situation. Le peuple allemand n'est ni découragé, ni endormi ; il ne se détourne pas de l'action ; et pourtant il n'agit pas ; il attend. La tâche à remplir peut bien faire hésiter. Car le problème qui se pose aux ouvriers allemands n'est pas de l'ordre de ceux qui se posaient, en 1917, aux ouvriers russes, paix à conclure et terre à partager ; non, il s'agit ici de reconstruire toute l'économie sur des fondements nouveaux. Seule peut donner la force de se résoudre à une telle tâche la conscience aiguë qu'il n'y a pas d'autre issue possible. C'est à quoi les jeunes sont amenés tour à tour par une crise qui semble leur ôter toute perspective d'avenir dans le cadre du régime ; mais cette même crise leur ôte aussi, peu à peu, la force de chercher une issue quelcon­que. Cette vie d'oisiveté et de misère, qui prive les ouvriers de leur dignité de producteurs, qui ôte aux ouvriers qualifiés leur habileté et aux autres toute chance de devenir habiles à quoi que ce soit, cette vie, à l'égard de laquelle il se produit, après deux, trois, quatre ans, une douloureuse accoutumance, ne prépare pas à assumer toutes les responsabilités d'une économie nouvelle.\par
Les employés de bureau, qui sont peu enclins à se considérer comme ne solidaires des ouvriers, sont bien moins capables encore que les ouvriers les plus découragés de chercher le salut en eux-mêmes ; et ils forment une partie considérable des salariés et des chômeurs allemands ; la folle prodigalité déployée par le capitalisme allemand en période de haute conjoncture, et qui a produit comme une course à l'accroissement des frais généraux, s'est manifes­tée aussi dans ce domaine, au point qu'il y a, dit-on, en certaines usines, plus d'employés de bureau que d'ouvriers.\par
Quant aux ouvriers des entreprises, ils existent encore, si pénible que soit leur vie, dans les cadres du régime ; ils y vivent mieux que d'autres ; ils ont quelque chose à perdre. Eux aussi, comme les chômeurs, sont de simples fétus dans le remous de la crise capitaliste ; mais ils peuvent, eux, n'y pas penser à tout instant. Une séparation s'établit ainsi entre les chômeurs et eux, qui prive les chômeurs de toute prise sur l'économie, mais qui en même temps les affaiblit eux-mêmes, menacés qu'ils sont par une réserve de travailleurs disponibles presque aussi nombreux que les travailleurs effectifs. Ainsi la crise n'a d'autre effet que de pousser à des sentiments révolutionnaires, mais de ramener ensuite, comme des vagues, des couches toujours nouvelles de la population. Si elle force presque chaque ouvrier ou petit bourgeois allemand à sentir, un moment ou l'autre, toutes ses espérances se briser contre la structure même du système social, elle ne groupe pas le peuple allemand autour des ouvriers résolus à transformer ce système.\par
Une organisation pourrait, dans une certaine mesure, y suppléer ; et le peuple allemand est le peuple du monde qui s'organise le plus. Les trois seuls partis allemands qui soient, actuellement, des partis de masse, se réclament tous trois d'une révolution qu'ils nomment tous trois socialiste. Comment se fait-il donc que les organisations restent, elles aussi, inertes ? Pour le comprendre, il faut les examiner dans leur vie intérieure et dans leurs rapports mutuels. Il faut les examiner surtout dans leur rapport avec les forces con­scientes et inconscientes dont le jeu détermine la situation politique ; c'est-à-dire, d'une part, avec les courants que produit la crise elle-même dans la masse de la population, à savoir ceux qui s'accrochent malgré tout au régime, ceux qui désirent aveuglément autre chose, ceux qui veulent tout transformer, ceux qui se laissent vivre sans espoir au jour le jour ; - d'autre part, avec les deux seuls éléments susceptibles d'agir d'une manière méthodique : la fraction révolutionnaire du prolétariat et la grande bourgeoisie.\par

\begin{center}
*\end{center}
\noindent Une révolution ne peut être menée que par des hommes conscients et responsables : on pourrait donc formuler la contradiction essentielle au parti national-socialiste en disant que c'est le parti des révolutionnaires inconscients et irresponsables. Toute crise grave soulève des masses de gens qui étouffent dans le régime qu'ils subissent sans avoir la force de vouloir eux-mêmes le transformer ; ces masses, derrière les révolutionnaires véritables, pourraient constituer une force ; la signification essentielle du mouvement hitlérien con­siste en ceci, qu'il en a groupé une grande partie à part, la faisant ainsi nécessairement tomber sous le contrôle du grand capital. Le mouvement national-socialiste - car les chefs considèrent, avec raison, le terme de mouve­ment populaire comme préférable à celui de parti - est composé, comme il résulte de son essence même, des intellectuels, d'une large masse de petits bourgeois, d'employés de bureau et de paysans, et d'une partie des chômeurs ; mais, parmi ces derniers, beaucoup sont attirés surtout par le logement, la nourriture et l'argent qu'ils trouvent dans les troupes d'assaut. Le lien entre ces éléments si divers est constitué moins par un système d'idées que par un ensemble de sentiments confus, appuyés par une propagande incohérente. On promet aux campagnes de hauts prix de vente, aux villes la vie à bon marché. Les jeunes gens romanesques sont attirés par des perspectives de lutte, de dévouement, de sacrifice ; les brutes, par la certitude de pouvoir un jour massacrer à volonté. Une certaine unité est néanmoins assurée en apparence par le fanatisme nationaliste, que nourrit, chez les petits bourgeois, un vif regret à l'égard de l'union sacrée d'autrefois, baptisée « socialisme du front » ; ce fanatisme, qu'exaspère une savante démagogie, va parfois, chez les fem­mes, jusqu'à une fureur presque hystérique contre les ouvriers conscients. Mais, dans l'ensemble du mouvement hitlérien, la propagande nationaliste s'appuie avant tout sur le sentiment que les Allemands éprouvent, à tort ou à raison, d'être écrasés moins par leur propre capitalisme que par le capitalisme des pays victorieux ; il en résulte quelque chose de fort différent du natio­nalisme sot et cocardier que l'on connaît en France, une propagande qui, essayant en outre de persuader que la plupart des capitalistes d'Allemagne sont juifs, s'efforce de poser les termes de capitaliste et d'Allemand comme deux termes antagonistes. On peut mesurer la puissance de rayonnement que possède en ce moment la classe ouvrière allemande par le fait que le parti hitlérien doit présenter le patriotisme lui-même comme une forme de la lutte contre le capital.\par
Même sous cette forme, la propagande nationaliste touche assez peu les ouvriers allemands, et les ouvriers hitlériens eux-mêmes. Dans leurs discus­sions avec les communistes, la question nationale reste le plus souvent au second plan ; au premier plan se posent les questions de classe ; tout au plus se demande-t-on dans quelle mesure il est sage de compter sur les ouvriers des autres pays. Dans l'ensemble, les ouvriers hitlériens sont corrompus par leur participation à un tel mouvement beaucoup moins qu'on ne pourrait s'y attendre. Leur sentiment dominant est une haine violente à l'égard du « systè­me », comme ils disent, haine qui s'étend aussi aux social-démocrates, consi­dérés comme les soutiens du régime, et même aux communistes, accusés de collusion avec la social-démocratie ; car les ouvriers hitlériens, qui se croient engagés dans un mouvement révolutionnaire, s'étonnent sincèrement que les communistes veuillent s'unir aux réformistes contre eux. De plus, le régime russe leur semble avoir bien des points communs avec le régime capitaliste. « Vous voulez une nation de prolétaires, disent-ils aux communistes ; Hitler veut supprimer le prolétariat. » Que désirent-ils donc ? Un régime idyllique, où les ouvriers, assurés d'une certaine indépendance par la possession d'un lopin de terre, seraient en outre défendus contre les patrons trop rapaces par un État tout-puissant et plein de soins paternels. Quant au programme économi­que, ils ne s'en inquiètent guère ; il a pu être modifié considérablement à leur insu. Ils se reposent de tous les soucis de réalisation pratique sur celui qu'on nomme « le chef », bien qu'il ne dirige pas grand-chose, c'est-à-dire Hitler. En réalité, ce qui les attire au mouvement national-socialiste, c'est, tout comme les intellectuels et les petits bourgeois, qu'ils y sentent une force. Ils ne se rendent pas compte que cette force n'apparaît si puissante que parce qu'elle n'est pas leur force, parce qu'elle est la force de la classe dominante, leur ennemi capital ; et ils comptent sur cette force pour suppléer à leur propre faiblesse, et réaliser, ils ne savent comment, leur rêve confus.\par

\begin{center}
*\end{center}
\noindent Les social-démocrates sont, au contraire, des gens raisonnables, que la situation n'a pas encore réduits au désespoir, et qui refusent de se lancer dans des aventures. C'est dire que la social-démocratie, bien qu'elle compte dans ses rangs des petits bourgeois et des chômeurs, s'appuie surtout sur des ouvriers qui travaillent. Elle a établi son emprise au cours des années de prospérité, et principalement par l'intermédiaire des syndicats dont elle n'a fait, en somme, au Parlement, que seconder l'action. Les syndicats réformistes, qui comptent quatre millions de membres, qui ont en main le personnel des services publics, des cheminots, des industries-clef, se sont, pendant la période de haute conjoncture, admirablement acquittés de leur tâche, à savoir aména­ger le mieux possible la vie des ouvriers dans le cadre du régime. Caisses de secours, bibliothèques, écoles, tout a été réalisé dans des proportions gran­dioses, installé dans des locaux témoignant de la même folle prodigalité dont les capitalistes ont été saisis au même moment. Des organisations ainsi modelées sur le développement de l'économie capitaliste dans ses périodes de stabilité apparente, se sont naturellement attachées à la force qui fait la stabi­lité du régime, au pouvoir d'État. Aussi, se sont-elles, d'une part, liées à un parti parlementaire, et à un parti qui est allé jusqu'aux plus extrêmes conces­sions pour rester dans la majorité gouvernementale ; et, d'autre part, elles se sont abritées derrière la loi, acceptant le principe du « tarif », c'est-à-dire les contrats de travail ayant force de loi et l'arbitrage obligatoire. La crise est venue. Les capitalistes se sont abrités eux-mêmes derrière le principe des tarifs pour attaquer les salaires. Mais plus l'économie capitaliste a été secouée par la crise, plus les organisations syndicales, qui, comme il arrive toujours, voient le but suprême dans leur propre développement, et non dans les servi­ces qu'elles peuvent rendre à la classe ouvrière, se sont réfugiées peureuse­ment derrière le seul élément de stabilité, le pouvoir d'État. Elles sont restées à peu prés inertes : les syndiqués qui participaient aux grèves dites « sauvages », c'est-à-dire non autorisées par les organisations, étaient exclus.\par
Vint le 20 juillet, le coup d'État qui ôta brutalement à la social-démocratie ce qui lui restait de pouvoir politique ; toujours même inertie. « C'est que, disaient ouvertement les fonctionnaires syndicaux, nous songeons avant tout au salut des organisations ; or, la réaction politique ne les met pas en péril. Le capitalisme lui-même, à l'état actuel de l'économie, a besoin des syndicats. Le péril hitlérien non plus n'existe pas ; Hitler ne pourrait prendre tout le pouvoir que par un coup d'État, qui ne se heurterait pas seulement à notre résistance, mais aussi à celle de l'appareil gouvernemental. Le seul péril serait d'engager les syndicats dans une lutte politique où l'État les briserait. » Il s'agit en som­me, avant tout, d'éviter que s'engage une lutte qui poserait la question : révolu­tion ou fascisme, - lutte qui aboutirait de toute manière à la destruction des organisations réformistes. Pour éviter qu'une telle lutte ne s'engage, et, si elle s'engage, pour la briser, on peut s'attendre que les fonctionnaires de la social-démocratie et des syndicats ne reculeront devant rien. Pour la même raison, ils ne veulent à aucun prix du front unique ; ils ont compris la leçon de 1917 et l'imprudence de Kerensky. En fin de compte, le fascisme semble être moins redoutable à leurs yeux que la révolution.\par
Les ouvriers qui composent les syndicats réformistes n'ont pas, avec le régime et l'État, les mêmes attaches indissolubles que leurs organisations. Quelques-uns, et surtout les vieux, suivent les syndicats réformistes et s'accro­chent au régime ; mais, d'une manière générale, la crise, qui menace à chaque instant les ouvriers qu'elle n'a pas encore réduits au chômage, fait que les ouvriers ne peuvent plus avoir l'illusion d'être chez eux dans le régime. Ainsi, à mesure que les organisations réformistes, sous l'action de la crise, se ratta­chaient de plus en plus peureusement au régime, les ouvriers, sous l'influence de la même cause, s'en détachaient de plus en plus. Le divorce entre les organisations et leurs membres est donc allé en s'accentuant. Depuis le 20 juillet, surtout, on se met, chose inusitée jusque-là, à discuter dans les réunions intérieures de la social-démocratie ; les jeunes y attaquent violemment la direction, proclament qu'ils ne veulent plus rester passifs sous prétexte qu'il faut éviter la guerre civile, qu'ils veulent s'entendre avec les ouvriers commu­nistes, et lutter. Mais lutter pour quoi ? Pour la république de Weimar ? La force de la position des chefs réformistes réside en ceci, qu'une lutte peut difficilement s'engager en ce moment sans mettre en question l'existence même du régime. Or la question du régime, les ouvriers social-démocrates n'osent guère la regarder en face. Aussi leur opposition demeure-t-elle sourde, incertaine, dispersée. Certes, quelques-uns d'entre eux s'en vont au mouve­ment national-socialiste ou au communisme ; mais la plupart restent membres disciplinés, bien que mécontents, de leurs organisations. Qu'ils préfèrent les organisations réformistes au mouvement hitlérien, cela fait leur éloge ; mais qu'est-ce qui les tient éloignés du parti communiste. Où en est le parti communiste allemand ?\par
Quatre-vingts à quatre-vingt-dix pour cent des membres du parti communiste allemand sont chômeurs. Plus de la moitié des membres a adhéré au parti depuis moins d'un an, plus des quatre cinquièmes depuis moins de deux ans. Ces seuls chiffres permettent d'apprécier la faiblesse du parti par rapport aux tâches qu'il se propose. La crise a pour effet naturel de rendre prudents même les ouvriers hautement qualifiés qui, en période de prospérité, ne craignant pas de perdre leur place, sont les plus disposés à mener une action révolutionnaire ; et elle amène au contraire à des opinions radicales ceux qui n'ont plus rien à perdre : les chômeurs. De même la crise use et remplace très vite des couches successives de révolutionnaires. Mais ces phénomènes produits par la crise dans la classe ouvrière, et qui sont pour elle une cause de faiblesse, se reflètent dans le parti communiste, non pas atténués comme il faudrait pour qu'il constitue un instrument aux mains des ouvriers, mais au contraire grossis. Ce grossissement ne peut être dû qu'à la politique du parti.\par
Sa politique syndicale, menée selon les deux mots d'ordre contradictoires : « Renforcez les syndicats rouges », et « Travaillez dans les syndicats social-démocrates » , a abouti à des syndicats rouges très faibles, et à une influence communiste à peu près nulle dans les syndicats réformistes. Le régime intérieur, régime de dictature bureaucratique sans contrôle de la base, a permis à la direction de mener une politique d'aventures qui a ôté au parti tout crédit dans les entreprises, les ouvriers des entreprises ayant beaucoup plus que les chômeurs la crainte des aventures. Ce même régime intérieur, en rendant la vie impossible aux éléments les plus conscients s'ils ne taisent pas au moins une partie de ce qu'ils pensent, en empêchant que les membres nouveaux, souvent recrutés au hasard, reçoivent une éducation sérieuse, condamne le parti à n'avoir presque que des membres fraîchement acquis. Ainsi le prolé­tariat allemand n'a comme avant-garde, pour faire la révolution, que des chômeurs, des hommes privés de toute fonction productrice, rejetés hors du système économique, condamnés à vivre en parasites malgré eux, et qui sont de plus entièrement dépourvus aussi bien d'expérience que de culture politi­que. Un tel parti peut propager des sentiments de révolte, non se proposer la révolution comme tâche.\par
Si l'on ajoute que les organisations de sympathisants groupent, elles aussi, surtout des chômeurs, et seulement au nombre d'une ou deux centaines de mille - que le parti n'a même pas construit de solides organisations de chô­meurs - qu'il a laissé dissoudre, il y a deux ans, une excellente organisation militaire (R.F.K.), qu'il n'a pu faire vivre illégalement, et dont les membres se trouvent aujourd'hui en partie dans les troupes d'assaut hitlériennes - on reconnaîtra qu'il est difficile d'imaginer une organisation plus faible à l'égard des problèmes que pose toute action.\par

\begin{center}
*\end{center}
\noindent Cette faiblesse intérieure lui rend à la fois indispensable et difficile d'ac­quérir une influence sur les ouvriers des autres partis. Cependant, la situation intérieure des partis national-socialiste et social-démocrate lui est favorable.\par
Dans le mouvement hitlérien se trouvent des ouvriers qu'on ne peut nom­mer conscients, mais qui ont, sinon des conceptions, du moins des sentiments révolutionnaires, qui croient sincèrement dans ce parti nationaliste, servir la révolution. En exposant clairement les contradictions intérieures du parti hitlérien, en dénonçant surtout, avec une vigueur implacable, le caractère contre-révolutionnaire de toute propagande nationaliste, on pourrait, dans une certaine mesure, isoler le parti hitlérien de la classe ouvrière, en détacher même certains éléments petits bourgeois.\par
Au contraire, les ouvriers social-démocrates, sourdement mécontents de la politique réformiste, n'osent pas s'engager dans la lutte révolutionnaire par une crainte légitime de l'aventure. La polémique ne peut mordre sur eux ; on ne peut les entraîner que par des accords pratiques permettant aux ouvriers social-démocrates et communistes, impuissants séparément, d'accomplir ensemble des actions bien déterminées ; actions qui contribueraient aussi à attirer ceux qui vont au parti hitlérien simplement parce qu'il est le seul à donner l'impression qu'il existe.\par
Or, par une perversité qui semble diabolique, le parti communiste mène une politique exactement contraire.\par
Il n'emploie d'autre moyen d'action auprès des social-démocrates que la polémique contre leurs chefs, polémique menée dans le langage le plus violent ; les offres de front unique, faites « à la base », par-dessus la tête des organisations, et dont chacun sait d'avance qu'elles seront rejetées, constituent simplement un aspect de cette polémique. En juillet, sous la pression des ouvriers de la base, et devant la menace des bandes fascistes, on a plusieurs fois réalisé le front unique entre organisations locales ; mais, depuis, si le terme de « social-fasciste » est devenu de moins en moins usité, tout en faisant toujours partie du vocabulaire officiel, le front unique a été pratiquement abandonné. Les propositions d'organisation à organisation ne se sont pas renouvelées.\par
Avec les hitlériens, au contraire, le parti a longtemps pratiqué une sorte de front unique contre la social-démocratie. Les ouvriers social-démocrates n'ont pas oublié le fameux plébiscite hitlérien, que la bureaucratie communiste s'est soudain avisée de transformer en « plébiscite rouge ». En se donnant ainsi l'apparence de prendre au sérieux les phrases révolutionnaires du parti hitlé­rien, elle a considérablement encouragé les ouvriers hitlériens dans leur erreur ; mais elle a fait pire ; elle a suivi le mouvement hitlérien sur le terrain national. Le parti a publié comme brochure de propagande, et sans commen­taires, le recueil des lettres où l'officier Scheringer expliquait qu'il était passé du national-socialisme au communisme parce que le communisme, par une alliance militaire avec la Russie, était bien mieux capable de servir les fins nationales de l'Allemagne. Sur cette plate-forme, Scheringer a formé un grou­pe, composé de gens du meilleur monde, et officiellement contrôlé par le parti. Le mot d'ordre de libération nationale (Volksbefreiung) tient, dans la propagande du parti, une place souvent aussi importante, parfois plus impor­tante, que les mots d'ordre de lutte sociale. Il faut remarquer que les ouvriers communistes eux-mêmes ne sont pas le moins du monde nationalistes. Mais cette politique les désarme dans leurs discussions avec les ouvriers hitlériens, au cours desquelles on a l'impression qu'ils n'arrivent pas à trouver le point de désaccord. On dirait que le parti communiste fait tout ce qu'il peut pour ne pas laisser apparaître, aux yeux des ouvriers peu cultivés, de caractère qui le distingue du mouvement hitlérien, en dehors d'une extrême faiblesse. Le résultat de toute cette politique est, pour le parti communiste allemand, un isolement complet au sein de la classe ouvrière.\par

\begin{center}
*\end{center}
\noindent Cette situation impose au parti communiste une attitude passive qui donne à ses mots d'ordre révolutionnaires le caractère de la plus creuse phraséologie. À moitié illégal, sa presse muselée, ses manifestations le plus souvent inter­dites, il ne peut réagir, de peur d'être réduit à l'illégalité complète. Dans son désarroi, il essaie, dans une situation qui ne laisse place Qu'à des luttes de caractère politique, de reprendre contact avec les ouvriers des entreprises sur le terrain des revendications ; ce qui est comique, si l'on songe qu'en France et en Belgique, il tente de donner artificiellement un caractère politique aux grèves revendicatives. Il essaie de cacher son impuissance par des mensonges, des vantardises, des mots d'ordre lancés à vide, tels que le mot d'ordre de grève générale lancé sans préparation le 21 juillet, qu'aucun ouvrier n'a pris au sérieux, et qui n'a fait que rendre le parti ridicule.\par
Tout cela répand, dans les rangs du parti, un profond découragement. Le succès remporté aux élections ne leur a rendu quelque confiance que grâce aux illusions les plus dangereuses concernant la valeur d'un succès électoral. Malgré ce succès, les ouvriers communistes sont en proie à un vague malaise ; ils se rendent compte que quelque chose ne va pas ; dans les cellules, où on essaie de les absorber dans des tâches de petite envergure, ils élèvent la voix, ils discutent, chose nouvelle depuis quelques années. Mais ils discutent encore timidement ; ils ne posent pas les questions essentielles. Des arguments d'ordre purement sentimental, tels que : « on ne peut pas faire le front unique avec Noske et Grzesinsky » , ont facilement prise sur les communistes de fraîche date, sans expérience ni culture historique.\par
De plus, les communistes de la base n'ont pas, en général, conscience de traverser un moment décisif de l'histoire ; ils ont le sentiment d'avoir beau­coup de temps devant eux, sentiment qui s'explique par la lenteur de l'évolu­tion politique en Allemagne. Ceux qui ont gardé quelque espoir de victoire, s'attendent vaguement qu'un jour une trahison des chefs réformistes, plus scandaleuse que les autres, amènera au parti communiste les masses social-démocrates.\par
Les petites oppositions communistes essaient en vain de transformer ce sourd malaise en quelque chose d'articulé ; elles-mêmes, d'ailleurs, gardent une attitude quelque peu craintive et plus ou moins ambiguë à l'égard du parti officiel. En général, leurs chefs n'ont d'espoir qu'en un renouveau spontané d'un mouvement révolutionnaire, après une catastrophe où périront les cadres officiels ; il faut faire exception pour le petit groupe trotskyste, qui n'arrive guère à faire plus qu'à répandre la littérature de Trotsky, et pour le parti socialiste ouvrier (S.A.P.). Ce parti, bien que constitué comme opposition social-démocrate, s'est, en fait, orienté vers le mouvement communiste, grâce à l'impulsion de la base, formée surtout de jeunes ouvriers remarquablement conscients, et sous l'influence de militants de valeur, anciens brandlériens sortis de l'opposition brandlérienne parce que celle-ci, dans les questions russes, se solidarise avec Staline. Mais un vice essentiel, qui tient à sa forma­tion même, frappe ce petit parti d'impuissance ; à la faiblesse numérique d'une secte, il joint l'intolérance d'une organisation de masse. D'une manière générale, les oppositions n'arrivent ni à agir par elles-mêmes, ni à mordre sur le parti officiel. Et celui-ci reste réduit à prêcher la révolution sans pouvoir la préparer.\par
\par

\begin{center}
*\end{center}
\noindent Cette impuissance du parti, qui dit constituer l'avant-garde du prolétariat allemand, pourrait faire conclure, légitimement, en apparence, à l'impuissance du prolétariat allemand lui-même. Mais le parti communiste allemand n'est pas l'organisation des ouvriers allemands résolus à préparer la transformation du régime, bien que ceux-ci en soient ou en aient été membres pour la plu­part ; il constitue une organisation de propagande aux mains de la bureaucratie d'État russe, et ses faiblesses sont par là facilement explicables. On comprend sans peine que le parti communiste allemand, armé, par les soins de la bureaucratie russe, de la théorie du « socialisme dans un seul pays », soit en mauvaise posture pour lutter contre le parti qui s'intitule « parti de la révolu­tion allemande » . Il est clair, d'une manière plus générale, que les intérêts de la bureaucratie d'État russe ne coïncident pas avec les intérêts des ouvriers allemands. Ce qui est d'intérêt vital pour ceux-ci, c'est d'arrêter la réaction fasciste ou militaire ; pour l'État russe, c'est simplement d'empêcher que l'Allemagne, quel que soit son régime intérieur, ne se tourne contre la Russie en formant bloc avec la France. De même une révolution ouvrirait des pers­pectives d'avenir aux ouvriers allemands ; mais elle ne pourrait que troubler la construction de la grande industrie en Russie ; et, de plus, un mouvement révolutionnaire sérieux apporterait nécessairement un secours considérable à l'opposition russe dans sa lutte contre la dictature bureaucratique. Il est donc naturel que la bureaucratie russe, même en cet instant tragique, subordonne tout au souci de conserver sa mainmise sur le mouvement révolutionnaire allemand.\par

\begin{center}
*\end{center}
\noindent Ainsi, les trois partis qui attirent les ouvriers allemands en déployant le drapeau du socialisme sont entre les mains, l'un, du grand capital, qui a pour seul but d'arrêter, au besoin par une extermination systématique, le mouve­ment révolutionnaire ; l'autre, avec les syndicats qui l'entourent, de bureau­crates étroitement liés à l'appareil d'État de la classe possédante ; le troisième, d'une bureaucratie d'État étrangère, qui défend ses intérêts de caste et ses intérêts nationaux. Devant les périls qui la menacent, la classe ouvrière allemande se trouve les mains nues. Ou plutôt, on est tenté de se demander s'il ne vaudrait pas mieux pour elle se trouver les mains nues ; les instruments qu'elle croit saisir sont maniés par d'autres, dont les intérêts sont ou contraires, ou tout au moins étrangers aux siens.\par
Il n'est pas étonnant, dans ces conditions, que la lutte entre les fractions de la bourgeoisie occupe le premier plan dans la politique intérieure allemande. L'extrême obscurité que présentent ces luttes, vient de la complexité des rapports entre le parti national-socialiste et la bourgeoise. Quand le grand capital groupe sous son contrôle les révoltés inconscients pour les pousser contre les révolutionnaires, il peut avoir pour objet soit d'exterminer ceux-ci, soit, simplement, de les paralyser. On pouvait ainsi, fin juillet, déterminer deux perspectives.\par
L'une était celle d'un gouvernement fasciste. C'est là, pour la bourgeoisie, la dernière ressource ; l'avènement au pouvoir des bandes hitlériennes présente le double danger de dresser côte à côte les ouvriers social-démocrates et communistes, et de lancer dans l'action à main armée les ouvriers hitlériens, qui prennent au sérieux la propagande démagogique de leur parti. Le fascisme ne peut être nécessaire à la bourgeoisie allemande qu'au cas où les ouvriers, malgré l'absence d'organisations qui leur appartiennent réellement, menace­raient de l'empêcher de réaliser les mesures économiques qu'elle juge être d'importance vitale dans la crise présente. Il lui faudrait alors engager le combat suprême.\par
L'autre perspective était celle d'un « gouvernement présidentiel », comme on dit en Allemagne, appuyé sur une union nationale s'étendant des hitlériens aux social-démocrates. Une telle union est possible sur la base du capitalisme d'État. En opposition avec la théorie communiste, les social-démocrates et les hitlériens s'accordent pour affirmer que la première étape vers le socialisme est la nationalisation des banques et des industries-clefs, sans transformation de l'appareil d'État ni organisation du contrôle ouvrier. Or la crise actuelle amène les capitalistes, non certes à accepter un tel programme, mais à cher­cher à se servir de l'appareil d'État en en faisant jusqu'à un certain point, d'une manière encore obscure pour eux-mêmes, un rouage de l'économie. Dans tous les pays, des économistes bourgeois ont écrit dans ce sens. En Allemagne, où, plus qu'en aucun autre pays, les gouvernements sont intervenus dans la vie économique, sans en excepter von Papen qui se dit le défenseur de l'économie libérale, cette tendance a trouvé son expression économique la plus achevée dans la revue {\itshape Tat.} La revue {\itshape Tat} est l'organe des jeunes économistes brillants, représentants du capital financier, qui voient les éléments du régime à venir dans les syndicats et le parti national-socialiste. Les social-démocrates ne cachent pas qu'ils considèrent tout accroissement du pouvoir économique de l'État comme « un morceau de socialisme », et qu'ils sont prêts, pour réaliser ce qu'ils nomment le socialisme, à accepter le concours des hitlériens eux-mêmes. La bourgeoisie semble avoir ainsi un moyen d'établir une sorte de régime fasciste sans massacres ni destruction des organisations syndicales, qui deviendraient simplement une pièce de l'appareil d'État.\par
Aucune de ces perspectives ne s'est réalisée. Hitler n'a pas le pouvoir. L'industrie lourde, qui le soutenait contre von Papen, l'homme des hobereaux, l'a jusqu'à un certain point abandonné ; elle a diminué les subventions qu'elle lui accorde ; elle a mis son organe, la {\itshape Deutsche Allgemeine Zeitung}, au service du gouvernement ; elle est intervenue auprès de Hindenburg pour l'empêcher de donner le pouvoir à Hitler.\par
D'autre part, si l'Allemagne a toujours un « gouvernement présidentiel » , ce gouvernement est bien loin de s'appuyer sur une coalition nationale ; au contraire, la grande bourgeoisie exceptée, il a toute la nation contre lui. Von Papen a fait comme si le parti hitlérien était un régiment de soldats de plomb qu'on peut à volonté sortir et remettre dans sa boîte ; mais, malheureusement pour la grande bourgeoisie allemande, les hitlériens ne sont pas des soldats de plomb ; ce sont des hommes révoltés et désespérés. Les ouvriers social-démocrates, eux aussi, ne peuvent être entraînés au-delà d'une certaine limite. Aussi assiste-t-on, en ce moment, à ce spectacle étrange d'un gouvernement qui reste au pouvoir malgré l'opposition violente des trois seuls partis de masse, lesquels, tous trois, hitlérien et social-démocrate aussi bien que com­muniste, soutiennent la vague de grèves que von Papen vient de décréter illégale. Bien que ces grèves soient de petite envergure, les organisations syndicales, si attachées à la légalité, avouent naïvement que la pression des masses les empêche de s'incliner devant ce décret.\par
Cette situation, exceptionnellement favorable pour les ouvriers révolution­naires, s'ils sont capables d'en profiter, ne peut pas durer longtemps. L'alter­native qui se posait au début d'août se pose encore. Le mouvement hitlérien a perdu, il est vrai, une bonne part de son prestige en cessant d'apparaître comme la force suprême ; mais il pourrait la regagner s'il avait de nouveau le grand capital derrière lui. Si le tournant annoncé par von Papen dans la conjoncture économique ne se produit pas, si la masse grandissante des chô­meurs continue à menacer la bourgeoisie d'une sorte de jacquerie, si les négociations avec la France n'apportent pas de satisfaction sérieuse aux petits bourgeois nationalistes, la grande bourgeoisie se verra sans doute forcée d'avoir de nouveau recours à Hitler. Or, Hitler signifie le massacre organisé, la suppression de toute liberté et de toute culture.\par
Il y a encore un élément inconnu, en dehors de la conjoncture économique et de la diplomatie ; c'est l'attitude que prendront les ouvriers allemands. Quand on considère abstraitement l'histoire des dernières années, on est tenté de croire que la classe ouvrière allemande, qui a subi passivement toutes les défaites, n'a plus aucune ressource en elle-même. Mais il est impossible de désespérer des ouvriers allemands lorsqu'on les approche. Les jeunes ouvriers aux yeux fiévreux, aux joues creuses, que l'on voit arpenter les rues de Berlin, ne sont pas restés passifs parce qu'ils sont lâches ou inconscients. Qu'après des années de chômage et de misère il n'y ait parmi eux qu'un nombre relative­ment faible de voleurs et de criminels ; qu'ils soient restés pour la plupart hors du mouvement hitlérien ; que la propagande nationaliste ait à peine pu mordre sur eux, cela ne peut qu'exciter l'admiration. Dans cette situation désespérée, ils ont résisté à toutes les formes de désespoir. Dans leurs moments de tristesse, comme dans leurs moments de gaieté en apparence insouciante, leur maintien, leur langage restent empreints d'une gravité, ou, plutôt, d'un sérieux, qui les fait apparaître, non pas comme accablés par le poids de la misère, mais comme continuellement conscients du sort tragique qui est le leur. Ils n'aperçoivent pas d'issue, mais ils ont conservé, ils conservent, dans la condi­tion inhumaine où ils sont placés, leur dignité d'êtres humains, par une vie saine et une haute culture. Beaucoup, qui ne mangent pas à leur faim, trouvent encore quelques sous pour les organisations sportives, grâce auxquelles ils peuvent s'en aller en bandes joyeuses, hommes et femmes, garçons et filles, vers les lacs et les forêts, marcher, nager, jouir de l'air et du soleil. D'autres se privent de pain pour acheter des livres ; le commerce des livres est un de ceux qui ont le moins souffert de la crise. Le niveau de culture des ouvriers allemands est quelque chose de surprenant pour un Français. En dehors des organisations politiques, il se forme spontanément, parmi les jeunes ouvriers, quelques cercles d'études où on lit les ouvrages classiques du mouvement révolutionnaire, où on écrit, où on discute. Ainsi, supportant une misère écrasante sans se plaindre ni chercher à s'étourdir, la meilleure partie de la classe ouvrière allemande échappe à la déchéance que constitue la condition de chômeur. La passivité même des ouvriers allemands devant les attaques de la réaction politique ne provient que de leur répugnance à se jeter dans l'aventure ; elle est signe de courage et non de désespoir. Vienne le moment où tous ensemble, ouvriers des entreprises et chômeurs, voudront se soulever, la classe ouvrière apparaîtra dans sa force, avec bien plus d'éclat qu'à Paris en 1871, ou à Saint-Pétersbourg en 1905. Mais qui peut dire si une telle lutte ne se terminerait pas par la défaite qui a écrasé jusqu'ici tous les mouvements spontanés ?\par
({\itshape La Révolution prolétarienne}, n° 138, 25 octobre 1932 ;\par
{\itshape Libres Propos}, nouvelle série, n\textsuperscript{os} 10 et 11, 25 octobre\par
et 25 novembre 1932.)\par

\subsection[11. La grève des transports à Berlin, (25 novembre 1932)]{11. \\
La grève des transports à Berlin \\
(25 novembre 1932)}
\noindent \par
La grève de Berlin, décidée, sur l'appel des communistes et des hitlériens, et malgré l'opposition des cadres syndicaux, par 78 \% des ouvriers, faillit constituer un événement décisif.\par
Dans un pays où il y a presque huit millions de chômeurs, qui souffrent la plupart de la faim, une grève non soutenue par le syndicat a pu supprimer complètement, et pendant plusieurs jours, tout transport dans la capitale (le chemin de ceinture excepté). Ce résultat a été obtenu grâce à l'appui de la population ouvrière, que la nature même du métier touché par la grève amenait à prendre part à l'action : les ouvriers, les ouvrières de Berlin appor­taient à manger aux membres des piquets de grève, et se joignaient à eux pour empêcher le départ des tramways et autobus conduits par des jaunes.\par
Un semblable mouvement, réalisant ainsi l'union spontanée des forces ouvrières dans la capitale, aurait facilement pu prendre le caractère d'une lutte contre le régime. Aussi, affolée, la {\itshape Deutsche Allgemeine Zeitung}, organe de la grande industrie, jeta-t-elle un cri d'alarme et réclamait-elle une action vigou­reuse de la police, en demandant que celle-ci fût « couverte par ses chefs même en cas d'agression ». En même temps le Vorwärts présentait tout le mouvement comme une provocation des communistes et des hitlériens unis, provocation qui pouvait, disait-il, servir de prétexte pour un ajournement des élections.\par
La grève, qui durait encore au moment des élections, apporta au Parti communiste, à Berlin, un succès électoral foudroyant. Un gain de 138 596 voix lui permit de dépasser les hitlériens de plus de 141 000 voix, les social-démocrates de plus de 214 300 voix, alors qu'en juillet il se plaçait au troisième rang. Les hitlériens eux-mêmes, grâce à leur participation active à la grève, perdirent beaucoup moins que dans l'ensemble du pays.\par
Mais le lendemain, les hitlériens, comme il était à prévoir, donnèrent le mot d'ordre de la reprise du travail, pendant que les cadres syndicaux accen­tuaient leur pression. Et, immédiatement, le travail reprit.\par
Ainsi, même à Berlin, même dans les circonstances les plus favorables, le Parti communiste allemand ne remporte qu'un succès d'ordre électoral. Au moment même d'une victoire éblouissante dans les élections, les faits ont montré que la puissance du Parti communiste, quand il est réduit à ses propres forces, est, dès qu'il s'agit d'une action réelle, exactement nulle.\par
Cela permet de mesurer la bonne foi ou la perspicacité de {\itshape L'Humanité}, selon laquelle les six millions de bulletins communistes représentent : « six millions de combattants pour les luttes extra-parlementaires, six millions de futurs grévistes ».\par
LES ÉLECTIONS\par
L'échec de la grève des transports est d'une importance bien plus grande que les élections.\par
Cependant les élections sont significatives en ce sens que le « gouverne­ment des barons » avait en quelque sorte posé la question de confiance au peuple allemand.\par
La réponse est écrasante. Plus de 83 \% des voix sont allées aux partis d'opposition (Centre et Partis communiste, social-démocrate et national-socialiste). Plus de 70 \% des voix sont allées aux trois partis dont toute la propagande s'était faite, cette fois, sur ce thème commun : « Contre le gouver­nement des barons ! Pour le socialisme ! »\par
Les deux partis de gouvernement (Nationaux-Allemands et Populistes), soutenus par l'appareil d'État, n'ont pourtant gagné ensemble qu'un million deux mille voix, évidemment venues du courant grand-bourgeois qui existait, à côté d'autres courants bien différents, dans le parti hitlérien.\par
En dehors de cette perte prévue et normale, les hitlériens ont perdu près d'un million de voix. Preuve que leur prestige, diminué par le fait essentiel qu'ils n'ont pas le pouvoir, n'a pu être rétabli par leur démagogie révolution­naire. Cependant le parti hitlérien ne se désagrège pas, il s'en faut de beaucoup ; il est encore de loin le plus fort.\par
Le « bloc marxiste », comme disent les hitlériens, n'a perdu que 17 300 voix, ce qui, vu le nombre des abstentions, accroît légèrement son importance relative. Sa composition intérieure a changé. Comme en juillet, le Parti communiste gagne et la Social-Démocratie perd. Comme en juillet les gains de l'un (604 511 voix) équivalent presque exactement aux pertes de l'autre (721 818 voix). Les chiffres de ces gains et pertes atteignent presque ceux de juillet. Le rythme s'accélère donc beaucoup. Cependant la Social-Démocratie, elle aussi, est loin de se désagréger ; elle dépasse encore le Parti communiste de plus de un million deux cent mille voix.\par
Il semble probable que ceux des électeurs perdus par Hitler qui ne se sont pas ralliés au gouvernement n'ont en général pas voté, et qu'au contraire les électeurs perdus par la Social-Démocratie ont voté communiste.\par
L'échec des « barons » , le succès des communistes rendent le Parti hitlé­rien indispensable à la grande bourgeoisie. La {\itshape Deutsche Allgemeine Zeitung} s'en aperçoit de plus en plus, et non sans angoisse. Les « barons » devront disparaître, ou s'entendre avec les hitlériens. Si cette entente se fait, comment se fera-t-elle ? Par un « bloc des droites » allant du centre à Hitler ? Par un « gouvernement syndical » allant du chef syndical socialiste Leipart au national-socialiste Gregor Strasser ? La {\itshape Deutsche Allgemeine Zeitung}, c'est-à-dire l'industrie lourde, préfère la première solution. De toutes manières le danger fasciste, bien qu'il ne soit peut-être pas immédiat, est aussi menaçant que jamais. Tout mouvement avorté qui, comme la grève des transports, effraie la bourgeoisie sans l'affaiblir, le rend plus aigu.\par
P.-S. - Nous connûmes trop tard la démission de von Papen pour pouvoir la commenter dans ce numéro.\par
\par
({\itshape La Révolution prolétarienne}, 8• année, n° 140, 25 novembre 1932.)\par

\subsection[12. La situation en Allemagne  (1932-1933)]{12. \\
La situation en Allemagne \protect\footnotemark  \\
(1932-1933)}
\footnotetext{ Nous avons conservé la division de cette étude en dix articles qui ont paru successivement dans {\itshape L'École émancipée.} (Note de l'éditeur.)}
\noindent \par
\subsubsection[I.]{I.}
\noindent Tous ceux qui ont mis toute leur espérance dans la victoire de la classe ouvrière, tous ceux mêmes qui tiennent à conserver les anciennes conquêtes de la bourgeoisie libérale, doivent avoir en ce moment les yeux tournés vers l'Allemagne. L'Allemagne est le pays où le problème du régime social se pose. Pour nous, en France, et même pour les militants, le problème du régime social est un objet de discours dans les réunions, d'articles dans les journaux, de discussions dans les cafés, tout au plus d'étude théorique ; et, tout le long du jour, il est oublié en faveur des occupations courantes, des petits événe­ments, des passions, des intérêts. Pour la plus grande partie de la population allemande, il n'y a pas de problème plus pressant, plus aigu dans la vie quotidienne.\par
On voit, en Allemagne, d'anciens ingénieurs qui arrivent à prendre un repas froid par jour en louant des chaises dans les jardins publics ; on voit des vieillards en faux col et en chapeau melon tendre la main à la sortie des métros ou chanter d'une voix cassée par les rues. Des étudiants quittent leurs études et vendent dans la rue des cacahuètes, des allumettes, des lacets ; leurs camarades jusqu'ici plus heureux, mais qui n'ont pour la plupart aucune chance d'obtenir une situation à la fin de leurs études, savent qu'ils peuvent, d'un jour à l'autre, en venir là. Les paysans sont ruinés par les bas prix et les impôts. Les ouvriers des entreprises reçoivent un salaire précaire et miséra­blement réduit ; chacun s'attend à être un jour ou l'autre rejeté à cette oisiveté forcée qui est le lot de près de la moitié de la classe ouvrière allemande ; ou, pour mieux dire, à l'agitation harassante et dégradante qui consiste à courir d'une administration à l'autre pour faire pointer sa carte et obtenir des secours ({\itshape stempeln}). Une fois chômeur, les secours, qui sont proportionnels au salaire touché avant le renvoi, diminuent et diminuent encore jusqu'à devenir à peu près nuls à mesure que s'éloigne le jour où l'on a cessé d'avoir part à la production. Un chômeur, une chômeuse habitant avec un père ou une mère, un mari ou une femme qui travaille, ne reçoit rien. Un chômeur de moins de vingt ans ne reçoit rien. Cette dépendance complète où est mis le chômeur par l'impossibilité de vivre, sinon aux dépens des siens, aigrit tous les rapports de famille ; souvent cette dépendance, quand elle est rendue insupportable par les reproches de parents qui comprennent mal la situation et que la misère affole, chasse les jeunes chômeurs du logis paternel, les pousse au vagabondage, à la mendicité, parfois au suicide. Quant à fonder soi-même une famille, à se marier, à avoir des enfants, la plupart des jeunes Allemands ne peuvent même pas en avoir la pensée. Que reste-t-il au jeune chômeur qui soit à lui ? Un peu de liberté. Mais cette liberté même est menacée par l'institution de l'{\itshape Arbeitsdienst}, travail accompli sous une discipline militaire, pour une simple solde, dans des sortes de camps de concentration pour jeunes chômeurs. Facultatif jusqu'à présent, ce travail peut d'un jour à l'autre devenir obligatoire sous la pression des hitlériens. L'ouvrier, le petit bourgeois allemand, n'a pas un coin de sa vie privée, surtout s'il est jeune, où il ne soit touché ou menacé par les conséquences économiques et politiques de la crise. Les jeunes, pour qui la crise est l'état normal, le seul qu'ils aient connu, ne peuvent même pas y échapper dans leurs rêves. Ils sont privés de tout dans le présent, et ils n'ont pas d'avenir.\par
C'est en cela que réside le caractère décisif de la situation, et non pas dans la misère elle-même. Le caractère décisif réside en ceci, que d'une part, à tort ou à raison, on croit de moins en moins en Allemagne, et surtout parmi les jeunes au caractère passager de la crise ; et que d'autre part, à la misère générale amenée par la crise, aucun homme, si énergique, si intelligent soit-il, ne peut avoir le moindre espoir d'échapper par ses propres ressources. Une crise faible, en ne chassant guère de l'entreprise que les moins bons ouvriers, employés ou ingénieurs, laisse subsister le sentiment que le sort de chaque individu dépend en grande partie de ses efforts pour se tirer individuellement d'affaire. Une crise intense est essentiellement différente. Ici aussi « la quan­tité se change en qualité ». En Allemagne, aujourd'hui, presque personne, dans aucune profession, ne peut compter sur sa valeur professionnelle pour trouver ou garder une place. Ainsi chacun se sent sans cesse entièrement au pouvoir du régime et de ses fluctuations ; et inversement, nul ne peut même imaginer un effort à faire pour reprendre son propre sort en main qui n'ait la forme d'une action sur la structure même de la société. Pour presque chaque Allemand, du moins dans la petite bourgeoisie et la classe ouvrière, les pers­pectives bonnes ou mauvaises concernant les aspects même les plus intimes de sa vie propre se formulent immédiatement, surtout s'il est jeune, comme des perspectives concernant l'avenir du régime. Ainsi la somme d'énergie qui est d'ordinaire, dans un peuple, absorbée presque tout entière par diverses passions et par la défense des intérêts privés, se trouve, en ce moment, en Allemagne, porter sur les rapports économiques et politiques qui constituent l'ossature même de la Société.\par
La situation, en Allemagne, peut donc être dite révolutionnaire. Le signe le plus apparent en est que les pensées et les conversations de chacun, y compris les enfants de onze ans, se portent constamment et naturellement sur le pro­blème du régime social, et avec le sérieux et la sincérité propres aux Alle­mands. Mais on ne voit pas de signe précurseur de la révolution dans les actes. La vague de grèves qui vient de parcourir l'Allemagne, au lieu d'embraser le pays, s'est éteinte grève après grève, y compris cette grève de transports qui avait semblé devoir soulever Berlin. Cependant la situation actuelle dure depuis déjà longtemps. Il faut comprendre que la crise pose le problème d'un nouveau régime de la production, non pas comme pour les Russes en 1917, voilé par d'autres problèmes en apparence plus faciles, mais brutalement, directement, et devant une classe ouvrière non homogène. Les chômeurs, les jeunes surtout, sont presque tous amenés, un moment ou l'autre, par cette crise qui leur ôte toute perspective, à sentir que la seule issue est la transformation du régime de production, mais, à mesure que pour chacun d'eux, le chômage se prolonge, cette même crise finit trop souvent par lui ôter la force de chercher en général une issue. Cette vie d'oisiveté et de misère, qui ôte à l'ouvrier qualifié son habileté, aux jeunes toute possibilité d'apprendre un métier, qui prive les ouvriers de leur dignité de producteurs, qui amène enfin - et c'est ce qu'elle a de pire - après deux, trois, quatre ans, une sorte de doulou­reuse accoutumance, cette vie ne prépare pas à assumer les responsabilités de tout le système de production. Ainsi la crise amène sans cesse de nouvelles couches ouvrières à la conscience de classe, mais sans cesse aussi les retire, comme la mer amène et retire ses vagues. Le prolétariat allemand est affaibli aussi par le nombre des employés de bureau, nombre qui a été accru par le capitalisme allemand, en période de prospérité, avec la même prodigalité folle qu'il a mise à bâtir ses usines et à renouveler son outillage. Car les employés de bureau, qui forment ainsi une partie considérable des salariés et des chômeurs allemands, sont peu enclins à se serrer autour des ouvriers, et incapables, par leur métier même, de vouloir prendre leur sort en leurs propres mains. Enfin i1 y a une coupure entre les ouvriers des entreprises et les chômeurs. Les ouvriers des entreprises peuvent malgré tout vivre à la rigueur dans le régime, ils ont quelque chose à perdre et s'y raccrochent ; ils sont, eux aussi, à la merci des remous de la crise, mais peuvent, contrairement aux chômeurs, ne pas en avoir conscience à tout instant. Ce défaut de solidarité ôte aux chômeurs toute prise sur l'économie, en même temps qu'il prive en partie les ouvriers des entreprises de la sécurité nécessaire aux luttes. Ainsi la crise, si elle force presque chaque ouvrier ou petit bourgeois allemand à sentir, un moment ou l'autre, toutes ses espérances se briser contre la structure même du système social, ne groupe pas par elle-même le peuple allemand autour des ouvriers résolus à transformer ce système. Seule une organisation peut remé­dier à cette faiblesse. Parmi les organisations qui groupent en si grand nombre les ouvriers allemands, y en a-t-il une qui y remédie en effet ?\par
C'est là une question de vie et de mort, au sens le plus littéral, pour bien des ouvriers allemands. Lénine, en octobre 1917, remarquait que les périodes révolutionnaires sont celles où les masses inconscientes, tant qu'elles ne sont pas entraînées par l'action dans le sillage des ouvriers conscients, absorbent le plus avidement les poisons contre-révolutionnaires. Le mouvement hitlérien en est un nouvel exemple. Et, en dépit des défaites électorales, tant que la crise durera et qu'un mouvement révolutionnaire n'aura pas triomphé, les trou­pes d'assaut hitlériennes derrière lesquelles peut se trouver d'un jour à l'autre l'appareil d'État, constituent une menace permanente d'extermination pour les meilleurs ouvriers. Mais, même en dehors de l'éventualité d'une extermination systématique, la crise elle-même, pour peu qu'elle dure encore quelque temps, détruira des générations d'ouvriers allemands, et plus particulièrement les jeunes générations. Déjà, parmi ceux qui ont pu survivre à trois ou quatre ans de chômage, les moins résistants sont amoindris au moral et au physique par la misère et l'oisiveté. Il va falloir traverser un hiver peut-être rigoureux, sans feu, sans repas chauds ; et après cet hiver peut-être un autre encore. Ceux qui n'y mourront pas y laisseront leur santé et leur force. Et la vie des ouvriers allemands est d'importance vitale aussi pour nous. Car, dans cette décompo­sition de l'économie capitaliste qui menace de détruire, sous une vague de réaction, les conquêtes des ouvriers dans les pays démocratiques et peut-être même en U. R. S. S., notre plus grand espoir réside dans cette classe ouvrière allemande, la plus mûre, la plus disciplinée, la plus cultivée du monde ; et plus particulièrement dans la jeunesse ouvrière d'Allemagne.\par
Rien n'est plus écrasant que la vie de dépendance, d'oisiveté et de privations qui est faite aux jeunes ouvriers allemands ; et l'on ne peut rien imaginer de plus courageux, de plus lucide, de plus fraternel que les meilleurs d'entre eux, en dépit de cette vie. Le vol, le crime, ont, malgré la misère, peu de prise en somme, l'agitation fasciste de son côté a relativement peu d'in­fluence sur cette jeunesse. Ils ne cherchent pas à s'étourdir ; ils ne se plaignent pas ; ils résistent, dans cette situation sans espoir, à toutes les formes de désespoir. Ils cherchent en général avec plus ou moins d'énergie, et les meilleurs y arrivent pleinement, à se faire, dans la condition inhumaine où ils sont placés, une vie humaine. Ils n'ont pas de quoi manger à leur faim ; mais beaucoup se privent de ce qui est nécessaire à la vie pour se procurer ce qui la rend digne d'être vécue. Ils trouvent quelques sous pour rester dans les organisations sportives qui les emmènent, garçons et filles, en bandes, malgré tout joyeuses, aux forêts, aux lacs, se livrer aux joies saines et gratuites que procurent l'eau, l'air, le soleil. Ils rognent sur la nourriture pour acheter des livres ; certains forment des cercles d'études où on lit les classiques du mouvement révolutionnaire, où on écrit, où on discute. Il n'est pas rare de trouver parmi eux des esprits plus cultivés que certains bourgeois soi-disant instruits de chez nous. Mais ce qui est plus frappant encore, c'est le degré auquel cette jeunesse est consciente d'elle-même. Il n'y a en France que des jeunes et des vieux ; là-bas, il y a une jeunesse. Chez ces jeunes ouvrières au teint bronzé, chez ces jeunes ouvriers aux yeux fiévreux, aux joues creuses, que l'on voit arpenter les rues de Berlin, l'on sent à tous moments, sous la tristesse comme sous l'insouciance apparente, un sérieux qui est le contraire du désespoir, une pleine et continuelle conscience du sort tragique qui leur est fait ; une continuelle conscience du poids dont pèse sur eux, de manière à écraser toutes leurs aspirations, ce vieux régime qu'ils n'ont pas accepté. Le fait que le régime, en cette période de crise, les prive complètement de ces perspectives d'avenir qui sont le privilège naturel de la jeunesse rend, par contraste, plus aiguë la conscience qu'ils ont de renfermer en eux un avenir. Et ils renferment en eux un avenir. Si notre régime en décomposition contient des hommes capables de nous donner quelque chose de nouveau, c'est cette génération de jeunes ouvriers allemands. À condition toutefois que les bandes fascistes, ou plus simplement le froid et la faim, ne leur ôtent pas soit la vie, soit du moins cette énergie qui est le ressort de la vie.\par
Nous ne sommes, nous ne pouvons guère être que spectateurs dans ce drame. Portons-y du moins l'attention qu'il mérite. Et tout d'abord faisons le bilan de la situation, établissons le rapport des forces.\par
({\itshape L'École émancipée}, 23\textsuperscript{e} année, n° 10, 4 décembre 1932.)
\subsubsection[II.]{II.}
\noindent Le 6 novembre, alors que la grande bourgeoisie massait toutes ses forces derrière le gouvernement von Papen, 70 \% des votants se sont prononcés pour les mots d'ordre : « Contre le gouvernement des barons ! Contre les exploi­teurs ! Vers le socialisme ! » Et la grande bourgeoisie continue à régner sur l'Allemagne. Pourtant, sept dixièmes de la population, c'est une force pour le socialisme ! Mais ces sept dixièmes se partagent entre trois partis.\par

\begin{center}
LE MOUVEMENT HITLÉRIEN\end{center}
\noindent De ces trois partis, le plus fort de beaucoup est le parti national-socialiste. Bien que, du 31 juillet au 6 novembre, il ait perdu des voix, il groupait encore derrière lui, à cette dernière date, le tiers des votants. Le caractère fondamen­tal du mouvement national-socialiste, et qui le rend presque incompréhensible pour un Français, c'est l'incohérence ; une incohérence inouïe, qui n'est qu'un reflet de l'incohérence essentielle au peuple allemand dans sa situation présente. Incohérence, d'abord, dans la composition sociale du mouvement. Toute crise grave soulève, dans toutes les couches d'une population, les plus hautes exceptées, des masses en révolte ; parmi ces masses, il se trouve des hommes qui sont capables d'être les artisans conscients et responsables d'un régime nouveau. Mais il y a, en plus grand nombre, des hommes inconscients et irresponsables, qui ne savent que désirer aveuglément la fin du régime qui les écrase. Groupés derrière les premiers, ils constituent une force révolution­naire ; mais si un homme arrive, comme c'est le cas pour Hitler, à en grouper à part la plus grande partie, ils tombent nécessairement sous le contrôle du grand capital, au service duquel ils forment des bandes armées pour la pire réaction, pour la dictature, pour les pogroms. Telle étant la nature du mouvement hitlérien, il groupe ceux qui sentent le poids du régime sans pou­voir compter sur eux-mêmes pour le transformer ; la plupart des intellectuels, de larges masses dans la petite bourgeoisie de la ville et des champs, presque tous les ouvriers agricoles ; enfin un certain nombre d'ouvriers des villes, presque tous chômeurs. Parmi ces derniers, qui sont au moins en partie dans les troupes d'assauts en qualité de simples mercenaires, on trouve beaucoup d'adolescents de quinze à dix-huit ans, qui appartiennent à peine à la classe ouvrière ; car ils ont trouvé la crise au sortir de l'école, et, pour eux, il n'a jamais été même question de travailler. Si on ajoute de grands bourgeois, la plupart dans la coulisse, mais quelques-uns membres du parti, et un ou deux princes, on aura un tableau complet du mouvement hitlérien.\par
La propagande n'est pas moins incohérente. On attire les jeunes garçons romanesques, par des perspectives de luttes héroïques, de dévouement, et les brutes par la promesse implicite qu'ils pourront un jour frapper et massacrer à tort et à travers. On promet aux campagnes de hauts prix de vente, aux villes la vie à bon marché. Mais l'incohérence de la politique hitlérienne apparaît surtout dans les rapports entre le parti national-socialiste et les autres partis. Le parti avec lequel les hitlériens ont un lien essentiel, c'est le parti national-allemand, celui de la grande bourgeoisie, celui qui soutient les « barons »; comme les « barons » , les hitlériens ont pour but fondamental la lutte à mort contre le mouvement communiste, l'écrasement de toute résistance ouvrière ; ils se proclament défenseurs de la propriété privée, de la famille, de la religion, et adversaires irréductibles de la lutte des classes. Mais ils se trouvent séparés des partis de la grande bourgeoisie par la composition sociale du mouvement, par la démagogie qui en résulte, et par les ambitions personnelles des chefs. Et, d'autre part, il se trouve, si surprenant que cela puisse sembler, entre le mouvement hitlérien et le mouvement communiste, des ressemblances si frappantes qu'après les élections la presse hitlérienne a dû consacrer un long article à démentir le bruit de pourparlers entre hitlériens et communistes en vue d'un gouvernement de coalition. C'est que, du mois d'août au 6 novembre, les mots d'ordre des deux partis ont été presque identiques. Les hitlériens, eux aussi, déclament contre l'exploitation, les bas salaires, la misère des chômeurs. Leur mot d'ordre principal, c'est « contre le système » ; la transformation du système, eux aussi l'appellent révolution ; le système à venir, eux aussi l'appellent socialisme. Bien que le parti hitlérien nie la lutte des classes, et qu'il emploie souvent ses troupes d'assaut à briser les grèves, il peut fort bien aussi, comme on l'a vu lors de la grève des transports de Berlin, publier, en faveur d'une grève, des articles de la dernière violence, lancer des mots d'ordre impliquant une lutte acharnée des classes, traiter les réformistes de traîtres. Quant aux social-démocrates, que les hitlériens accusent de trahir à la fois l'Allemagne, comme internationalistes, et le prolétariat, comme réformistes, il y a entre eux et le national-socialisme un point commun, qui est d'importan­ce ; c'est le programme économique. Pour le parti national-socialiste comme pour la social-démocratie, le socialisme n'est que la direction d'une partie plus ou moins considérable de l'économie par l'État, sans transformation préalable de l'appareil d'État, sans organisation d'un contrôle ouvrier effectif ; c'est, par suite, un simple capitalisme d'État. Sur cette communauté de vues se fonde une tendance vers un gouvernement qui se transformerait, d'une manière encore indéterminée, en un rouage essentiel de l'économie, en s'appuyant à la fois sur les syndicats social-démocrates et sur le mouvement national-socia­liste. Leipart, dans la bureaucratie syndicale ; Gregor Strasser, chez les hitlériens, soutiennent cette tendance, que défendent, au nom du capital financier, quelques jeunes et brillants économistes groupés autour de la revue {\itshape die Tat}, et dont le principal représentant est, dit-on, von Schleicher lui-même.\par
Ce mouvement si disparate semble, à première vue, trouver une sorte d'unité dans le fanatisme nationaliste, qui va jusqu'à l'hystérie chez certaines petites bourgeoises, et au moyen duquel on essaie de ressusciter l'union sacrée d'autrefois, baptisée « socialisme du front » Mais on n'y réussit guère. La propagande nationaliste ne se suffit pas à elle-même. Les hitlériens doivent profiter du sentiment commun à tous les Allemands, que leur peuple n'est pas seulement écrasé par l'oppression du capitalisme allemand, mais aussi par le poids supplémentaire dont pèse, sur toute l'économie allemande, l'oppression des nations victorieuses ; et ils s'efforcent de faire croire, d'une part que ce dernier poids est de beaucoup le plus écrasant, d'autre part que le caractère oppressif du capitalisme allemand est dû uniquement aux juifs. Il en résulte un patriotisme bien différent du nationalisme sot et cocardier que nous connais­sons en France ; un patriotisme fondé sur le sentiment que les nations victo­rieuses, et surtout la France, représentent le système actuel, et l'Allemagne, toutes les valeurs humaines écrasées par le régime ; sur le sentiment, en somme, d'une opposition radicale entre les termes d'Allemand et de capita­liste. On ne peut qu'admirer la puissance de rayonnement que possède en ce moment, en Allemagne, le prolétariat, quand on voit que le parti hitlérien, qui est aux mains de la grande bourgeoisie et recrute surtout des petits bourgeois, doit présenter le patriotisme même comme une simple forme de la lutte contre le capital. Ce qui est plus beau encore, c'est que, dans une semblable situation, cette propagande ait en somme peu de prise sur les ouvriers, y compris les ouvriers hitlériens. Ceux-ci, dans leurs discussions fréquentes, et parfois presque amicales, avec les communistes, les raillent bien pour leurs illusions concernant une soi-disant solidarité internationale qui n'existe pas ; ils ne les accusent pas de trahir la patrie. Même quand ils combattent les communistes, les ouvriers hitlériens laissent en général la question nationale au second plan, et restent sur le terrain des intérêts ouvriers. Ils accusent le gouvernement russe de rendre les ouvriers russes malheureux, alors que Hitler rendrait les ouvriers allemands heureux, en donnant à chacun un lopin de terre, et en les protégeant paternellement contre les exigences exagérées du patronat ; et ils reprochent amèrement aux communistes allemands de trahir la révolution, quand ils se rangent aux côtés des social-démocrates, ces soutiens du régime, contre eux, ouvriers hitlériens, qui se croient sincèrement de bons révolution­naires.\par
En somme la propagande hitlérienne, tout en donnant aux ouvriers qu'elle touche les idées les plus confuses, et en les empêchant d'atteindre à une véri­table conscience de classe, leur laisse leur esprit ouvrier. Un ouvrier allemand, même hitlérien, reste avant tout un ouvrier. Et surtout les jeunes ouvriers gardent intact, dans le mouvement hitlérien, ce sentiment qui est au cœur de toute la jeunesse ouvrière allemande, ce sentiment impérieux d'un avenir qui leur appartient, auquel ils ont droit, dont les coupe impitoyablement le système social, et pour lequel il leur faut briser le système. Les petits bou­rgeois hitlériens, de leur côté, restent des petits bourgeois, ballottés entre l'influence de la grande bourgeoisie, et l'influence, nettement plus forte en ce moment, du prolétariat. Les grands bourgeois restent, dans le mouvement hitlérien, de grands bourgeois, les aristocrates des aristocrates ; quant aux simples brutes qui, quelle que soit leur classe sociale, y sont facilement attirées, elles restent de simples brutes ; Hitler est arrivé à réunir, dans son mouvement, toutes les classes ; il n'est nullement arrivé à les fondre. Mais, plus ce parti est disparate, plus sa politique comporte de contradictions essen­tielles, plus il faut que quelque chose maintienne ces éléments si divers en un seul bloc. Mais quoi ?\par
Ce qui unit les membres du mouvement hitlérien, c'est tout d'abord l'avenir que celui-ci leur promet. Quel avenir ? Un avenir qui n'est pas décrit, ou l'est de plusieurs manières contradictoires, et peut être ainsi pour chacun de la couleur de ses rêves. Mais, ce dont on est sûr, c'est que ce sera un système neuf, un « troisième reich », quelque chose qui ne ressemblera ni au passé, ni surtout au présent. Et ce qui attire, vers cet avenir confus, intellectuels, petits bourgeois, employés, chômeurs, c'est qu'ils sentent, dans le parti qui le leur promet, une force. Cette farce éclate partout, dans les défilés en uniforme, dans les attentats, dans les avions employés pour la propagande ; et tous ces faibles vont vers cette force comme des mouches vers la flamme. Ils ne savent pas que, si cette force apparaît comme si puissante, c'est qu'elle est la farce, non de ceux qui préparent l'avenir, mais de ceux qui règnent sur le présent. La perspective d'un avenir indéterminé, le sentiment d'une force inconnue, en voilà plus qu'il ne faut pour conduire, en bandes disciplinées, ces désespérés, qui ont soif d'une transformation sociale, au massacre de tous ceux qui préparent cette transformation.\par
Mais ce danger n'est-il pas écarté ? Le mouvement hitlérien n'est-il pas en pleine décadence ? Et les ouvriers hitlériens ne sont-ils pas convertis à la pratique de la lutte des classes ?\par
Sans aucun doute le mouvement hitlérien est affaibli. Cet affaiblissement est moins dû à la résistance ouvrière qu'aux luttes des fractions bourgeoises. En juillet les hobereaux ont placé en face de Hitler, au lieu du faible gouvernement de Brüning, le gouvernement autoritaire de von Papen. En août, l'industrie lourde, sans tout à fait abandonner Hitler, est venue, elle aussi, se ranger derrière von Papen. Dès que Hitler a cessé d'apparaître comme la plus grande force, il a perdu une bonne partie de son prestige ; c'est ce qui explique son recul le 6 novembre. Mais, vis-à-vis de la grande bourgeoisie, cet affai­blissement rapide constitue un moyen de chantage qui est une force. La grande bourgeoisie voudrait se servir de Hitler sans lui livrer le gouvernement. Mais, si elle continuait à le laisser à l'écart, comme elle a fait d'août à décembre, le mouvement hitlérien se décomposerait, et elle se trouverait isolée, coupée des masses, devant la vague grandissante du mécontentement ouvrier. En dépit de ses mitrailleuses, elle serait sans doute alors perdue, à moins d'un retour rapide de la prospérité industrielle, fort improbable en Allemagne. Il faut d'autant moins espérer une tactique aussi folle, de la part de la grande bourgeoisie, que la {\itshape Deutsche Allgemeine Zeitung}, organe de l'indus­trie lourde, ne cesse, depuis le début de novembre, de mettre en garde contre ce péril, et de demander une concentration nationale. Von Schleicher va essayer d'avoir l'appui plus ou moins avoué, soit des hitlériens seuls, en vue d'un bloc national, soit des hitlériens et des syndicats ensemble, en vue d'un « gouvernement syndical » . Si les hitlériens, sous l'influence des sentiments révolutionnaires de la base et des ambitions personnelles des chefs, refusent toute négociation, on ne voit pas d'autre issue, pour la grande bourgeoisie, qu'un gouvernement hitlérien ; c'est-à-dire la suppression des organisations ouvrières et le massacre organisé.\par
Quant aux actions révolutionnaires des hitlériens, toute illusion à ce sujet serait dangereuse. Elles constituent, elles aussi, un moyen de chantage vis-à-vis de la grande bourgeoisie. Le parti hitlérien a soutenu, avec les commu­nistes, la vague de grèves qui a répondu aux décrets-lois de von Papen, et en particulier cette grève des transports qui a bouleversé Berlin au début de novembre, à laquelle toute la population a pris part et qui a fait pousser à la {\itshape Deutsche Allgemeine Zeitung} des cris d'alarme. C'était là un moyen pour le parti national-socialiste, à la fois d'acquérir des voix ouvrières et de rappeler à la grande bourgeoisie qu'il lui était indispensable ; et, si l'on juge par les commentaires de la {\itshape Deutsche Allgemeine Zeitung}, cela réussit parfaitement Au lendemain des élections, la grève, combattue déjà depuis plusieurs jours par la bureaucratie syndicale, devenait, pour Hitler aussi, embarrassante et dangereuse ; elle se termina aussitôt, et par une défaite. La veille encore, la presse hitlérienne lançait des mots d'ordre identiques à ceux des commu­nistes ; le lendemain elle commençait une série d'articles retentissants, desti­nés à prouver que Hitler seul pouvait anéantir le communisme en Allemagne. Quant aux grévistes eux-mêmes, le fait que des ouvriers hitlériens sont en grève ne garantit nullement, tant que la crève ne constitue pas une infraction à la discipline du parti, que ces mêmes ouvriers ne mettront pas la même violence, quelques mois plus tard, à réprimer des grèves ; il suffirait que le parti hitlérien ait, entre-temps, pris part au pouvoir. Et ces ouvriers croiraient continuer à servir leur classe, tout comme l'ont cru, à meilleur titre, les ouvriers bolcheviks qui, après Octobre, ont réprimé des mouvements anar­chistes. Ainsi, pour les communistes, le fait d'avoir entraîné des ouvriers hitlériens à leurs côtés dans une lutte ne constitue pas un succès, tant qu'ils ne les ont pas détachés de leur parti. On peut même dire que le front unique entre communistes et ouvriers hitlériens, tant qu'il ne s'étend pas aux ouvriers social-démocrates, augmente, aux yeux des masses ouvrières, le prestige de Hitler ; les élections du 6 novembre l'ont montré à Berlin.\par
Il n'y a qu'un moyen, pour les ouvriers conscients, de vaincre le mouve­ment hitlérien ; c'est d'une part de faire comprendre aux masses qu'un mouvement nationaliste, fondé sur l'union des classes, ne peut apporter aucun système nouveau ; d'autre part de leur faire sentir l'existence, en face de la force hitlérienne, d'une autre force, celle du prolétariat groupé dans ses organisations propres.\par
Mais, dès qu'on a énoncé ce programme, on s'étonne qu'il ne soit pas déjà rempli, et le mouvement hitlérien décomposé. Nul peuple n'est plus accessible à la bonne propagande que le peuple allemand, qui lit et réfléchit tant. Quant à la question de force, les syndicats allemands groupent quatre millions d'ouvriers, et il y a eu, le 6 novembre, six millions de voix communistes.\par
Que manque-t-il au mouvement ouvrier allemand ? Pour le comprendre il faut l'examiner sous son double aspect, réformiste et révolutionnaire.\par
({\itshape L'École émancipée}, 23\textsuperscript{e} année, n° 12, 18 décembre 1932.)
\subsubsection[III. Le réformisme allemand]{III. \\
Le réformisme allemand}
\noindent À la révolution, écrivait Marx en 1848, « les prolétaires n'ont rien à perdre, que leurs chaînes. Et c'est un monde qu'ils ont à y gagner ». Le réformisme repose sur la négation de cette formule. La force du réformisme allemand repose sur le fait que le mouvement ouvrier allemand est le mouve­ment d'un prolétariat pour qui, longtemps, cette formule ne s'est pas vérifiée ; qui, longtemps, a eu à l'intérieur du régime quelque chose à conserver.\par
\par
Le mouvement ouvrier, qui, de 1792 à 1871, s'était développé surtout en France, et sous une forme aventureuse, violente, presque toujours inégale, et d'ailleurs anarchique, a trouvé, après 1871, une sorte de patrie dans l'Allema­gne ; mais, là, il a pris une forme prudente, méthodique, soigneuse­ment organisée et presque toujours légale. Les ouvriers allemands se sont organisés au début de la période impérialiste ; ils ont profité de ce nouvel essor de l'économie capitaliste pour conquérir, dans les cadres du régime, une condi­tion plus humaine. Interrompu par la misère de la guerre et de l'après-guerre, et l'essor révolutionnaire qui en a résulté, ce mouvement a repris après octobre 1923, une fois que cet essor révolutionnaire se fut brisé ; et il trouva, de 1924 à 1929, un excellent terrain de développement dans une Allemagne à qui la prospérité économique donnait en quelque sorte le vertige. On peut dire qu'en Allemagne l'organisation ouvrière dans les limites de la légalité capitaliste a donné sa pleine mesure. Les résultats ne sont pas à dédaigner. En ce moment, malgré plus de quatre ans de crise, la Confédération générale du travail compte, là-bas, plus de quatre millions de membres ; les entreprises qui n'em­ploient que des syndiqués ne sont pas rares ; les industries-clefs, les chemins de fer, les entreprises d'État sont, par la proportion des ouvriers syndiqués qui y travaillent, aux mains de la Confédération. Cette organisation formidable a assuré aux ouvriers allemands, pendant la période de bonne conjoncture, des salaires assez élevés. Mais le groupement des ouvriers dans l'entreprise n'est qu'une partie de l'activité déployée par les syndicats allemands pour augmen­ter le bien-être des ouvriers. Les cotisations servent à alimenter des caisses de secours nombreuses et très riches. D'autre part des bibliothèques syndicales, des écoles syndicales où des ouvriers, choisis par les syndicats, peuvent séjourner aux frais des caisses syndicales, des associations de libres penseurs et des groupements sportifs qui sont organiquement liés aux syndicats permettent aux ouvriers de consacrer leurs loisirs à la culture de l'esprit et du corps. Tout cela est organisé de manière à exciter l'admiration, et installé dans de somptueux bâtiments, où l'on retrouve la même prodigalité presque folle que les capitalistes ont déployée de leur côté au moment de la bonne conjonc­ture. Il faut reconnaître que le réformisme allemand a merveilleusement accompli sa tâche, qui consiste à aménager la vie des ouvriers aussi humai­nement qu'il est possible de le faire à l'intérieur du régime capitaliste. Il n'a pas délivré les ouvriers allemands de leurs chaînes, mais il leur a procuré des biens précieux ; un peu de bien-être, un peu de loisir, des possibilités de culture.\par
Mais les organisations syndicales allemandes n'ont pas seulement dû s'adapter aux conditions créées par le régime ; par la force des choses, elles se sont liées au régime par des liens qu'elles ne peuvent briser. Elles se sont développées avec le régime, elles en ont pour ainsi dire épousé les formes ; elles ne peuvent exister qu'à l'intérieur du système capitaliste, et à l'ombre du pouvoir qui est le gardien de l'ordre actuel, du pouvoir d'État. Il va de soi que pour ces bureaux, ces écoles, ces bibliothèques qui s'étalent dans de si beaux bâtiments, une existence illégale n'est même pas concevable. Quant aux caisses syndicales, leur richesse même les met sous la dépendance de l'État, gardien des capitaux.\par
« Les biens des syndicats », écrit l'organe syndical die {\itshape Arbeit}, « sont, pour la plupart, placés en bons hypothécaires. Pour se procurer des disponibilités, les syndicats sont obligés d'engager ces bons à la Reichsbank, ou de les vendre en Bourse. Le gouvernement peut fermer ces deux voies si les syndi­cats dépensent cet argent, non pas pour des secours sociaux, mais pour financer de grandes luttes. »\par
Ainsi les syndicats sont enchaînés à l'appareil d'État par des chaînes d'or ; et par ces mêmes chaînes d'or, qu'ils ont eux-mêmes forgées, les ouvriers sont, à leur tour, enchaînés à l'appareil syndical, et, par cet intermédiaire, à l'appareil d'État. Car l'exclusion, arme fréquemment employée par l'appareil syndical contre les ouvriers qui voudraient l'orienter vers la lutte contre l'État, constitue une véritable peine ; l'ouvrier exclu perd ses droits à l'assistance de ces caisses de secours pour lesquelles il a versé des cotisations si lourdes. Quant aux organisations qui sont liées à la confédération syndicale, elles adoptent nécessairement, à l'égard de l'État, une attitude analogue. C'est le cas du parti social-démocrate. Les syndicalistes purs essayent parfois de rejeter la responsabilité de la ligne politique suivie par les syndicats allemands sur la social-démocratie, dont sont membres tous les militants du mouvement syndical. Mais, dès avant la guerre, les syndicats allemands étaient bien plus éloignés encore que la social-démocratie d'une attitude de combat à l'égard du régime ; et aujourd'hui, il en est encore de même. Loin que le parti social-démocrate dirige le mouvement syndical, on peut dire qu'il est plutôt l'expres­sion parlementaire des relations qui existent entre l'appareil syndical et l'appareil d'État ; l'on verra que les événements des 20 et 27 juillet 1932 en fournissent l'exemple. C'est la structure même des syndicats allemands qui leur interdit de se détacher du système social actuel, sous peine de se briser. Vienne une étape de l'économie capitaliste ou la formule de Marx se trouve vérifiée, où le régime prive les ouvriers de tout, sauf de leurs chaînes, il est impossible que les syndicats allemands changent de destination, et deviennent des instruments propres à briser le régime ; pas plus qu'une lime ne peut, en cas de besoin, se transformer en marteau.\par
Or, ce moment semble venu. Ou du moins il est incontestable que, pour la durée de la crise, et, d'une manière générale, dans la mesure ou l'économie capitaliste est destinée, même après la fin de la crise actuelle, à demeurer, comme la plupart des gens le pensent, dans un état de crise latent, la formule citée plus haut est vérifiée. Et, à mesure que la crise secouait plus durement le régime capitaliste, les syndicats allemands, loin de se détacher du régime, se sont en effet accrochés de plus en plus peureusement au seul élément de stabilité, au pouvoir d'État.\par
Il y a plusieurs années déjà que la confédération syndicale allemande a ouvertement subordonné son action à l'État en acceptant ce que les Allemands appellent le « principe des tarifs ». Selon ce principe, tout contrat de travail a force de loi, tout conflit est obligatoirement porté devant un tribunal d'arbi­trage dont la décision a également force de loi. Si les ouvriers veulent faire grève sans que les patrons aient violé les conditions imposées par le contrat de travail ou la décision arbitrale, les syndicats, liés par la « Friedenspflicht » (littéralement : devoir de paix), doivent s'y opposer, sous peine de tomber dans l'illégalité. Bien entendu, l'appareil syndical respecte scrupuleusement la « Friedenspflicht », non seulement en privant de secours toutes les grèves dites sauvages, c'est-à-dire non approuvées par le syndicat, mais encore en excluant, de temps en temps, les syndiqués qui y participent. Cette loi des tarifs, qui devait, disait-on, protéger les ouvriers contre l'arbitraire patronal, a, en fait, servi de bouclier aux entrepreneurs dans leurs attaques contre les salaires. Quant au parti social-démocrate, on sait comment, de son côté, sous le gouvernement Brüning, il a constamment capitulé ; au moment du renou­vellement des contrats de travail, il a laissé Brüning diminuer tous les salaires par décret-loi. Le résultat de cette politique, c'est qu'au moment où Brüning perdait le pouvoir, les ouvriers des entreprises avaient passé, presque sans résistance, d'un niveau de vie assez élevé à une condition misérable.\par
Vint le gouvernement von Papen et le coup d'État du 20 juillet, qui, en chassant brutalement la social-démocratie du gouvernement de Prusse, lui ôta ce qui lui restait de pouvoir politique. On attendait une vigoureuse résistance. En 1920, la social-démocratie allemande avait montré, lors du coup d'État de Kapp, combien elle était capable de vigueur. En juillet 1932, elle resta inerte. Pourquoi ? Dans des conversations particulières, les militants des syndicats s'en expliquaient ainsi, dévoilant du même coup les rapports véritables entre la social-démocratie, les organisations syndicales et l'État : « Le salut de nos organisations, c'est la considération qui prime toutes les autres à nos yeux. La réaction politique, en elle-même, ne les met aucunement en péril. Dans l'étape présente de son développement, c'est l'économie capitaliste elle-même qui a besoin des syndicats ouvriers. Quant à Hitler, nous ne le craignons pas ; s'il tentait un coup d'État, il trouverait devant lui nous-mêmes d'un côté, et l'appareil gouvernemental de l'autre. »\par
La conclusion était que la seule chose à craindre, pour les organisations syndicales, c'est une lutte entre elles et l'État, lutte ou elles seraient infaillible­ment brisées.\par
Les syndicats durent pourtant sortir de leur passivité quand les décrets-lois des 4 et 5 octobre autorisèrent les entrepreneurs, en certains cas, à abaisser les salaires au-dessous du tarif indiqué par le contrat de travail. L'une après l'autre, les entreprises touchées par le décret firent grève. Les syndicats approu­vèrent ces grèves dans la mesure où elles s'appuyaient sur le principe des tarifs, et, par suite, n'étaient pas contraires à la « Friedenspflicht » ; mais, toujours pour respecter la « Friedenspflicht », les syndicats firent ouvertement tout ce qu'ils purent pour empêcher chaque grève, soit de s'étendre à d'autres objectifs que la lutte contre le décret-loi, soit de sortir des limites de l'entre­prise en se liant avec les mouvements parallèles. Ils y réussirent, et d'autant plus facilement que les patrons, alléchés par les primes à la production que venait d'établir von Papen, cédaient en général tout de suite. Certains entrepre­neurs eurent recours au tribunal d'arbitrage, qui, bien entendu, leur donnait raison ; le syndicat, toujours en vertu de la « Friedenspflicht », brisait alors la grève. La grève ne fut difficile à briser que dans un cas, celui du fameux mouvement des transports de Berlin ; communistes et hitlériens unis réussirent alors à arrêter complètement, et pendant plusieurs jours, tramways, autobus et métros ; mais, le 8 novembre, les grévistes des transports reprenaient à leur tour le travail, et sans avoir rien obtenu.\par
La crise ministérielle déterminée par les élections du 6 novembre marque le début d'une nouvelle étape dans l'histoire des rapports entre le réformisme allemand et l’État. La social-démocratie annonçait qu'elle combattrait von Papen par tous les moyens, s'il revenait au pouvoir ; elle laissait entendre que son opposition contre von Schleicher serait plus modérée. Quand van Schleicher fut chancelier, Leipart, un chef de la bureaucratie syndicale, dit l'envoyé spécial d'Excelsior : « Nous n'avons rien à reprocher au chancelier de son passé politique ; la question sociale est au premier plan de ses préoccupations \footnote{ Leipart a, par la suite, démenti cette interview, le journaliste maintient ses dires. Tout ce qu'on peut affirmer à ce sujet, c'est que les paroles rapportées par l'{\itshape Excelsior} sont tout à fait conformes à la réputation de Leipart en Allemagne. (Note de S. W.)}. » Pour apprécier ces paroles, il faut savoir que c'est von Schleicher qui a fait rendre aux troupes d'assaut hitlériennes le droit de porter l'uniforme, et qui a organisé le coup d'État du 20 juillet. Leipart annonça que les syndicats étaient prêts à accorder une trêve à von Schleicher s'il organisait des travaux pour les chômeurs, abolissait les décrets-lois par lesquels von Papen avait diminué les salaires et les secours de chômage, renonçait à modifier la loi électorale et la Constitution. Cependant, jusqu'ici, la social-démocratie et von Schleicher n'ont rien fait pour se rapprocher ; chacun reste sur ses positions, l'une avec ses revendications sociales, l'autre avec ses déclarations d'attachement à l'égard de l'économie traditionnelle et de von Papen. Peut-être y a-t-il des négociations souterraines. En tout cas, de part et d'autre, on attend.\par
Comment s'orientera le réformisme allemand ? La crise lui fait perdre en grande partie son efficacité, non seulement du point de vue du prolétariat, mais aussi du point de vue des capitalistes. Quand la bourgeoisie est en mesure de laisser aux ouvriers quelque bien-être, elle peut compter sur ces biens qu'ils possèdent à l'intérieur du régime pour les attacher au régime ; quand il ne leur reste plus que des chaînes, la bourgeoisie ne peut compter sur le souvenir des biens qu'ils ont possédés pour les maintenir dans le calme. Dans le premier cas, la bourgeoisie se contente d'organisations corporatives indirectement placées sous le contrôle de l'État ; dans le second, elle peut avoir besoin d'organisations à l'intérieur desquelles elle puisse exercer une contrainte directe sur les ouvriers, c'est-à-dire d'organisations fascistes. Car, malgré les déclamations des partis communistes, les syndicats allemands ne sont pas encore fascistes. Dans un syndicat fasciste, les entrepreneurs et les ouvriers sont organisés ensemble, l'adhésion des ouvriers est obligatoire, le gouvernement intervient directement ; aucun de ces caractères ne se trouve dans les syndicats allemands. La complicité sourde de l'appareil syndical suffira-t-elle à la bourgeoisie allemande, ou devra-t-elle établir un régime fasciste ? Tout ce qu'on peut dire de certain à ce sujet, c'est que l'équilibre actuel est instable. L'équilibre le plus instable peut durer, mais le moindre choc suffit à le détruire ; et, en Allemagne, des manifestations de désespoir de la part des chômeurs peuvent produire ce choc. Mais, si la bourgeoisie allemande a recours au fascisme, une autre question se pose ; lancera-t-elle les bandes hitlériennes, sous la direction de von Schleicher ou de Hitler lui-même, centre les organisations ouvrières, qu'elles soient révolutionnaires ou réformistes ? Ou aura-t-elle recours à l'appareil syndical lui-même, pour opé­rer sans douleur la transformation des syndicats en organisations fascistes ?\par
Les ouvriers syndiqués laisseraient-ils leurs propres organisations trahir aussi cyniquement leur cause ? Avant de poser cette question, il faut se demander comment il se fait qu'ils aient accepté tout ce qu'ils ont accepté jusqu'ici.\par
({\itshape L'École émancipée}, 23\textsuperscript{e} année, n° 15, 8 janvier 1933.)
\subsubsection[IV.]{IV.}
\noindent Les cadres de la social-démocratie allemande, ce sont les êtres qui se trouvent tout naturellement logés dans les somptueux bureaux des organi­sations réformistes, comme le coquillage loge son habitant ; êtres satisfaits, importants, frères de ceux que renferment les plus luxueux bureaux des ministères ou de l'industrie, bien étrangers au prolétariat :nos camarades allemands nomment cela des bonzes. Mais la base, malgré la présence de quelques petits-bourgeois, est essentiellement ouvrière ; et, qui plus est, composée d'ouvriers tout à fait conscients d'appartenir, avant tout et par-dessus tout, au prolétariat.\par
On le niera peut-être. Certes les ouvriers allemands, trouvant dans le régime même un peu de bien-être et la possibilité de goûter, de temps à autre, les plaisirs qui comptent seuls pour eux, musique, culture intellectuelle, exer­cice physique, vie en pleine nature, se sont attachés aux organisations qui leur procuraient ces biens, et se sont détournés des périlleuses tentatives destinées à briser le système de production pour en construire un autre. Mais nul n'a le droit de le leur reprocher. Il est naturel qu'un ouvrier qui ne cherche pas dans la révolution une aventure, et n'y voit pas non plus un simple mythe, ne s'engage dans la voie révolutionnaire qu'autant qu'il ne voit pas d'autre issue, et que celle-là lui parait praticable. L'ouvrier le plus conscient est celui qui se rend le mieux compte, non pas seulement des vices essentiels du régime, mais aussi des tâches immenses, des responsabilités écrasantes que comporte une révolution. Pour la génération qui a maintenant atteint la maturité, la guerre, il est vrai, a suffisamment fait apparaître le caractère essentiellement inhumain du régime, et surtout dans l'Allemagne affamée et vaincue. Mais le mouve­ment révolutionnaire qui l'a suivie s'est brisé deux fois, une fois par l'écrase­ment sanglant des spartakistes, une autre fois par la défaite sans bataille d'octobre 1923. Comment s'étonner qu'ensuite les ouvriers allemands se soient de nouveau laissé vive dans ce monde capitaliste que la prospérité rendait de nouveau habitable ? Et si certains d'entre eux se sont laissé étourdir par cette prospérité qui a donné le vertige à Ford lui-même, il faut se souvenir que tout le monde autour d'eux annonçait, avec cette éloquence que donne la convic­tion, une période de progrès continus de l'économie capitaliste et, pour les ouvriers, un accroissement continu du bien-être, des possibilités de culture et des libertés politiques nouvellement conquises.\par
À présent la crise est là. Plus clairement encore que la guerre, parce qu'elle est un phénomène purement économique, elle montre le caractère inhumain du régime. Les ouvriers, tous plus ou moins brutalement frappés, compren­nent, mieux que par aucune propagande, qu'ils n'ont pas place dans le système en tant qu'êtres humains, mais sont de simples outils, outils qu'on laisse user par la rouille quand on n'a pas avantage à les user au travail. En même temps le régime perd sa parure de démocratie. Et voilà qu'en cette quatrième année de crise le prolétariat allemand semble être impuissant et déconcerté par la catastrophe comme pendant les premières années de la guerre. Impuissant, comme alors, à se détacher d'organisations qui l'ont servi en un temps plus heureux, mais ne savent maintenant que le livrer à la cruauté mécanique du système.\par
Cependant la social-démocratie a, sans aucun doute, perdu une partie assez considérable de son influence, depuis la crise. Aux élections du 31 juillet, elle a perdu 700 000 voix ; à celles du 6 novembre, de nouveau 600 000. Mais l'essentiel est qu'à l'intérieur même des organisations réformistes se manifes­tent des tiraillements, parfois d'une rare violence, entre ceux qui représentent les organisations, et sont, comme elles, indissolublement liés au régime, et les ouvriers qui composent les organisations, et que le régime repousse en quel­que sorte lui-même. Déjà les social-démocrates n'ont pas accepté sans peine de voter pour Hindenbourg. La passivité de la social-démocratie, le 20 juillet, a indigné toute la base, et même la partie de l'appareil syndical qui se trouve en contact avec la base. À partir du 20 juillet, on s'est mis à discuter dans les sections social-démocrates, chose inconnue jusque-là. Les vieux soutiennent les chefs ; les jeunes réclament de l'action, le front unique avec les commu­nistes, la lutte contre toutes les mesures de terreur que la bureaucratie essaye de faire accepter en agitant l'épouvantail de la guerre civile. Mais ils ne modifient point l'attitude des organisations. Et celles-ci maintiennent, sinon leurs effectifs, du moins, ce qui est plus important, leurs positions straté­giques. Les syndicats continuent à régner dans les entreprises ; et, jusqu'ici, ils réussissent à briser tous les mouvements de quelque envergure auxquels ils s'opposent. La bureaucratie réformiste continue à tenir en main les rouages de la production. Les ouvriers regimbent, mais elle fait ce qu’elle veut.Que veut-elle ? Conserver ses bureaux. Conserver les organisations, sans se demander à quoi elles servent. Dans cette tâche, les bonzes sont aidés par des militants sincères, qui, à force de s'être dévoués pour les organisations, les considèrent comme des fins en soi. Or, quel danger menace ces organisations, dans le capitalisme, comme disait ce fonctionnaire syndical, ne peut se passer ? Un seul : la guerre civile. Une insurrection ouvrière commencerait par les balayer, une attaque à main armée des bandes hitlériennes les briserait. Il s'agit donc d'éviter les formes aiguës de la lutte des classes, de conserver la paix à tout prix, c'est-à-dire au prix de n'importe quelle capitulation. Accepter un régime fasciste ne ferait sans doute pas peur aux bonzes ; d'autant moins peur que les mesures de capitalisme d'État que le fascisme comporte apparaîtraient facile­ment comme « un morceau de socialisme » à des gens pour qui le socialisme n'est pas autre chose que le capitalisme d'État. En revanche le front unique leur fait peur ; ils savent, comme tout le monde, que Kerensky aurait mieux fait de s'allier à Kornilov qu'à Lénine. Rien n'égale le ton de haine avec lequel ils parlent du parti communiste. Au reste ils répètent leurs belles paroles d'autrefois, mais sans conviction ; par moments leurs contradictions prouvent qu'ils mentent, et n'ont pas d'illusions. « On ne peut aller au socialisme qu'en passant par la démocratie », disent-ils. Mais, en août, ils ont affirmé que, seuls, les moments de crise aiguë sont favorables à la proposition de mesures socialistes. Comme ces moments sont toujours ceux ou la démocratie est suspendue, et où la social-démocratie n'a nulle part au pouvoir, c'est là un aveu d'impuissance. Ils disent aussi : « la question n'est pas : Allemagne fas­ciste ou Allemagne soviétique, mais Allemagne fasciste ou république de Weimar ». Ce n'est pas par ces formules creuses que les bonzes peuvent entraîner les ouvriers. Mais comment les entraînent-ils ?\par
Quand les ouvriers allemands, sur qui l'on comptait pour assurer la paix du monde, se sont trouvés devant la guerre, leur désarroi a eu pour cause, non seulement les mensonges nationalistes, mais aussi le fait qu'ils n'avaient qu'une organisation, et que cette organisation les envoyait à la guerre. Les ouvriers allemands ont une peine infinie à se résoudre à une action non orga­nisée. Aussi, s'ils étaient dans la même situation à présent, comprendrait-on qu'ils se laissent mener par la bureaucratie réformiste. Car la tactique réformiste est la seule raisonnable à l'intérieur du régime, et, par suite, pour quiconque ne veut pas ou n'ose pas briser le régime. Cette tactique, c'est, par définition, la tactique du moindre mal. Le régime est mauvais ; si on ne le brise pas, il faut s'arranger pour y être le moins mal possible. Quand la bourgeoisie est forcée par la crise de resserrer son étreinte, la tactique du moindre mal est nécessairement une tactique de capitulation. Les vieux, qui, en période de crise, continuent à penser comme ils pensaient avant la crise, s'y plient tout naturellement. Les jeunes crient qu'ils veulent lutter ; mais comme ils n'osent pas engager la lutte pour briser tout le système de production, et que chacun sent bien, tout en se gardant de le dire, qu'il n'y a pas d'autre objectif de lutte, ils acceptent finalement, eux aussi, de capituler. Encore une fois, tout cela se comprendrait fort bien dans un prolétariat désarmé, dont les organisations seraient aux mains de ses ennemis.\par
Le parti hitlérien est en effet aux mains du grand capital, et la social-démocratie aux mains de l'État allemand. Mais le prolétariat allemand possède à présent ce qu'il n'avait pas en 1914, un parti communiste. Certes, depuis le début de la crise, le parti communiste a attiré à lui de larges masses de chô­meurs et quelques petits bourgeois ; de même que le parti hitlérien a attiré de larges masses de petits bourgeois ruinés et un nombre assez important de chômeurs. Il est naturel que ceux qui, ayant, du fait de la crise, tout perdu, sont prêts à tout essayer, se jettent sur les deux partis qui promettent du nouveau ; et ceux qui se sentent solidaires de la classe ouvrière vont à celui des deux partis qui annonce la dictature du prolétariat. Mais ce courant n'entraîne pas les ouvriers des entreprises. Ceux-ci, c'est-à-dire ceux qui ont une responsabilité dans la production, ceux qui, par leur travail quotidien, conservent la société et peuvent la transformer, ou bien restent indifférents, ou bien suivent les réformistes. Il y a, bien entendu, des exceptions ; mais elles sont relativement assez peu nombreuses. Aussi la social-démocratie se soucie-t-elle peu de perdre les chômeurs et les petits bourgeois, que d'ailleurs elle est loin de perdre tous. Sa force stratégique est dans les entreprises ; et, quoi que puisse réserver l'avenir, cette force est jusqu'ici presque intacte. La social-démocratie ne sera vaincue qu'au moment où les ouvriers qui travaillent engageront une action assez cohérente et assez étendue pour mettre le système social en péril.\par
Que les ouvriers des entreprises résistent à la démagogie hitlérienne, cela prouve que, malgré la situation misérable à laquelle ils sont réduits, leurs bas salaires, l'absence de sécurité, le chômage qui réduit leurs heures de travail et qui frappe en général leur famille, ils ne cèdent pas au désespoir. On ne peut assez l'admirer. Mais comment comprendre qu'ils se laissent écraser par le régime, alors qu'il existe un parti communiste, tout propre, semble-t-il, à constituer entre leurs mains un outil puissant pour briser et reconstruire le système de production ? Qu'est-ce qui retient les masses ouvrières qu'occupent encore les entreprises, et particulièrement les éléments encore jeunes, de faire confiance au parti communiste ?\par
({\itshape L'École émancipée}, 23\textsuperscript{e} année, n° 16, 15 janvier 1933)
\subsubsection[V. Le mouvement communiste]{V. \\
Le mouvement communiste}
\noindent En période de prospérité, le mouvement révolutionnaire s'appuie en général surtout sur ce qu'il y a de plus fort dans le prolétariat, sur ces ouvriers hautement qualifiés qui se sentent l'élément essentiel de la production, se savent indispensables et n'ont peur de rien. La crise pousse les chômeurs vers les positions politiques les plus radicales ; mais elle permet au patronat de chasser de la production les ouvriers révolutionnaires, et contraint ceux qui sont restés dans les entreprises, et qui tous, même les plus habiles, craignent de perdre leur place, à une attitude de soumission. Dès lors le mouvement révolutionnaire s'appuie au contraire sur ce que la classe ouvrière a de plus faible. Ce déplacement de l'axe du mouvement révolutionnaire permet seul à la bourgeoisie de traverser une crise sans y sombrer ; et inversement, seul un soulèvement des masses demeurées dans les entreprises peut véritablement mettre la bourgeoisie en péril. L'existence d'une forte organisation révolution­naire constitue dès lors un facteur à peu près décisif. Mais pour qu'une organisation révolutionnaire puisse être dite forte, il faut que le phénomène qui, en période de crise, réduit le prolétariat à l'impuissance, ne s'y reflète pas ou ne s'y reflète que très atténué.\par
Le parti communiste allemand a une force apparente considérable. Les souvenirs encore vivants de 1919, 1920, même 1923, lui donnent un rayonne­ment, un prestige que nous, Français, dont la tradition révolutionnaire s'est interrompue en somme en mai 1871, pouvons difficilement imaginer. Il a entraîné en novembre six millions d'électeurs ; lui-même comptait alors 330 000 membres. Les groupements tels que la section du Secours Ouvrier, la section du Secours Rouge, l'association des Libres-penseurs prolétariens, et surtout les associations sportives rouges sont, eux aussi, nombreux, influents, pleins de vie. Cependant la plus importante de ces organisations, une organi­sation militaire destinée à la guerre des rues ({\itshape Rote Front Kämpferbund}) a été interdite il y a deux ans, et n'a pu vivre dans l'illégalité ; une partie des adhérents est même passée depuis aux sections d'assaut hitlériennes. Quant au parti lui-même, il est nombreux, mais les nouveaux venus y sont en majorité ; certains disent qu'un cinquième seulement des adhérents est au parti depuis plus de trois ans. Ce phénomène, qui se retrouve dans plusieurs autres sections de l'Internationale, et qui est évidemment dû, au moins pour une part, au mode de recrutement et au régime intérieur, est particulièrement dangereux en période révolutionnaire. Enfin et surtout le parti communiste allemand est pratiquement un parti de chômeurs ; en mai dernier, les chômeurs y étaient déjà dans une proportion de quatre-vingt-quatre pour cent (cf. I.S.R., n° 1920, p. 916). Quant aux organisations syndicales placées sous le contrôle du parti, leurs effectifs sont inférieurs à ceux du parti lui-même, et elles sont com­posées de chômeurs pour une moitié environ. Ainsi à cette séparation entre chômeurs et ouvriers des entreprises que la crise économique produit par elle-même, le mouvement communiste allemand ne porte pas remède ; il la reflète, et la reflète accentuée.\par
C'est donc que le parti communiste n'était pas arrivé à s'implanter solide­ment dans les entreprises. La cause principale en est sans doute la tactique syndicale. Devant les mesures d'exclusion qui frappèrent maintes fois les adhérents révolutionnaires des syndicats réformistes, on pouvait choisir entre deux méthodes : appeler tous les ouvriers conscients dans les syndicats rouges, comme fait notre C.G.T.U., ou constituer, avec les seuls exclus, des syndicats rouges qui ne recrutent pas, et cherchent à augmenter, non leurs effectifs, mais leur influence. Cette dernière tactique a été celle des « che­valiers du travail » belges ; la grève du Borinage en a montré la valeur, et elle semble la mieux appropriée quand il ne se produit que des exclusions indivi­duelles. Mais en Allemagne, on n’a pas choisi ; on a organisé parallèlement les syndicats rouges et les organisations d'opposition, et on a toujours main­tenu côte à côte les deux mots d'ordre contradictoires : « Renforcez les syndicats rouges », et : « Travaillez dans les syndicats réformistes. » En conséquence, les syndicats rouges sont restés squelettiques en face des quatre millions d'adhérents des syndicats réformistes; mais leur existence a suffi, d'une part, pour permettre aux réformistes de présenter les communistes comme étant, au même titre que les hitlériens, des ennemis des organisations syndicales, d'autre part, pour faire négliger aux militants la propagande dans les syndicats social-démocrates. Bien plus, à un moment donné on a lancé ouvertement le mot d'ordre : « Brisez les syndicats. » Bien que ce mot d'ordre ait été abandonné, la crainte d'être accusés de revenir au mot d'ordre condam­né par les brandlériens (forcer les bonzes à lutter) a longtemps paralysé les militants. Ce n'est que tout récemment qu'a pris fin cette négligence inouïe à l'égard de la propagande dans les syndicats réformistes. D'autre part, aussi bien les groupements d'opposition que les syndicats rouges souffrent d'un régime étouffant de dictature bureaucratique. Quant à l'organisation dans les entreprises, il a longtemps régné à cet égard une indifférence inconcevable.\par
En face de la social-démocratie, si puissamment implantée dans les entre­prises par son influence sur les ouvriers, en face du mouvement hitlérien qui, à côté de ses adhérents ouvriers, bénéficie de l'appui secret ou avoué du patronat, le parti communiste allemand se trouve au contraire sans liens avec la production. Il n'est pas étonnant que l'existence d'un tel parti ne suffise pas, par elle-même, à tourner les masses opprimées vers l'action révolutionnaire. Dès lors tout dépend de l'influence que le parti communiste allemand peut arriver à acquérir à l'intérieur des deux autres partis ; autrement dit il faut, dans la période actuelle, juger le parti communiste allemand d'après ses rela­tions avec le mouvement hitlérien d'une part, la social-démocratie de l'autre.\par
Comment le parti communiste allemand peut-il acquérir de l'influence au sein du mouvement hitlérien ? Toute tentative en ce sens doit être réglée d'après cette vue que le parti communiste, même s'il amenait à lui tout ce qui, dans le mouvement hitlérien, se rattache au prolétariat, ne deviendrait pas de ce fait capable de remplir ses tâches révolutionnaires. D'une part l'adhésion d'ouvriers qui ont pu se laisser tromper par la démagogie fasciste ne serait pas précisément de nature à relever le niveau politique du parti ; d'autre part, dans le mouvement hitlérien, il n'y a, en dehors des grands et petits bourgeois, des employés de bureau, des chômeurs, que quelques ouvriers des entreprises, la plupart placés, il est vrai, aux points stratégiques de l'économie, mais trop peu nombreux pour constituer par eux-mêmes une force considérable. Au contraire, si le parti communiste pouvait attirer la confiance des masses réfor­mistes groupées dans les entreprises, il acquerrait de ce fait une force presque invincible. Ainsi un accord partiel entre hitlériens et communistes, même s'il peut faciliter la propagande communiste auprès des ouvriers hitlériens, est éminemment dangereux s'il est susceptible d'éveiller la défiance des ouvriers social-démocrates. Au reste la tactique des accords partiels ne constitue pas une tactique efficace dans la lutte contre l'influence de Hitler. Car les chefs hitlériens sont avant tout des démagogues et des aventuriers, et ne reculent pas, au besoin, devant l'action illégale ; ainsi, dans telles circonstances déter­minées, ils peuvent, à la différence des chefs réformistes, non seulement lancer les mêmes mots d'ordre que le parti communiste, mais encore aller au moins aussi loin que lui dans l'application. Au reste il va de soi que, dans une grève par exemple, on ne peut refuser l'appui des grévistes hitlériens ; mais, dans un cas semblable, il faut faire grande attention à ne rien faire qui puisse donner à cette unité d'action un caractère suspect aux yeux des ouvriers social-démocrates ; et il faut s'efforcer par tous les moyens d'entraîner ceux-ci dans la lutte. En somme le parti communiste n'a pas d'autre moyen, pour conquérir les ouvriers hitlériens sans éloigner les ouvriers social-démocrates, que de dénoncer impitoyablement aussi bien les petites trahisons quotidiennes des chefs hitlériens que le caractère essentiellement réactionnaire de toute leur politique. Ou plutôt il y a à cet égard, pour le parti communiste, un autre moyen encore plus efficace, c'est de devenir véritablement fort. Mais c'est à quoi il ne peut arriver qu'en gagnant la confiance des ouvriers réformistes.\par
À l'égard de ceux-ci, la simple propagande ne sert pas à grand-chose. Ils sentent bien, les jeunes surtout, que leurs chefs capitulent ; mais qu'il y ait une autre issue, ils n'en sont point persuadés. Il est clair que sur ce point de simples paroles sont sans efficacité. On ne peut les convaincre qu'en leur faisant sentir par l'expérience, dans l'action, la force que possède la classe ouvrière organisée. Et à cet effet, c'est eux-mêmes qu'il faut amener à agir ; car le parti communiste est trop faible pour mener seul, avec succès, une action de quelque envergure. Il faut donc qu'il organise le front unique en vue d'objectifs limités, et mène la lutte assez bien pour inspirer aux ouvriers plus de confiance qu'ils n'en ont en eux-mêmes et en lui. Les chefs réformistes, il est vrai, manœuvrent toujours pour empêcher un tel front unique ; et leurs manœuvres sont facilitées par l'isolement où se trouve le parti communiste. Celui-ci n'a qu'un moyen de déjouer ces manœuvres, c'est de faire des concessions. Il y a une concession qu'il ne peut jamais faire, à savoir renoncer à son droit de critique. Mais il peut modérer le ton des critiques, et renoncer à une violence peut-être légitime, mais en fait nuisible. Il peut, pour montrer qu'en dépit des mots d'ordre qui lui ont été funestes, il n'est pas l'ennemi des organisations syndicales, suspendre le recrutement des minuscules syndicats rouges ou même demander la réintégration des exclus. Il peut surtout adresser toutes les propositions de front unique, non seulement à la base, mais à tous les degrés de la hiérarchie réformiste. Une telle tactique pourrait sans doute, en certains cas, contraindre les chefs réformistes à accepter le front unique ; en tout cas elle leur rendrait beaucoup plus difficile de le refuser sans se discréditer auprès des ouvriers qui les suivent.\par
({\itshape L'École émancipée}, 23\textsuperscript{e} année n° 18, 29 janvier 1933.)
\subsubsection[VI.]{VI.}
\noindent La politique du parti communiste allemand est en fait, jusqu'ici, exacte­ment contraire. Le seul moyen employé pour conquérir les ouvriers réfor­mistes, c'est une propagande purement verbale ; on prêche la révolution à des gens qui se demandent, non pas si elle est désirable, mais si elle est possible. On accuse les chefs réformistes de trahison ; or le parti communiste n'aurait le droit d'apprécier ainsi la politique de capitulation des social-démocrates que s'il montrait qu'il est capable, lui, de diriger victorieusement le prolétariat dans la voie opposée, celle de la lutte. C'est ce qu'il ne peut montrer par de simples paroles ; et, jusqu'ici, il ne l'a pas montré autrement. Aussi les formules, si souvent répétées, concernant la social-démocratie comme « ennemi princi­pal », le « social-fascisme », etc., n'ont-elles pour effet que de couper le parti communiste des ouvriers social-démocrates ; et d'autant plus qu'on va jusqu’à appeler la social-fasciste » le petit « parti socialiste, ouvrier », composé en partie d'ouvriers révolutionnaires, en partie d'éléments réformistes, mais que révoltent les capitulations des social-démocrates. Sur le terrain syndical, on s'obstine à conserver les deux mots d'ordre contradictoires. Quant au front unique, on refuse de faire des propositions autrement qu'à la base ; et les manœuvres des chefs réformistes en sont facilitées d'autant.\par
Avec les hitlériens, au contraire, le parti communiste allemand a plusieurs fois pratiqué une sorte de front unique ; il lui est même arrivé, notamment pendant la grève des transports de Berlin, de suspendre, en faveur des national-socialistes, ce fameux « droit de critique » qu'il refuse, avec raison, de sacrifier au front unique avec les social-démocrates. Le plus grave, c'est que le front unique entre hitlériens et communistes a paru parfois dirigé contre la social-démocratie et l'a effectivement été en certains cas. Au reste, l'on peut comprendre que, dans la lutte sociale, et par l'effet de la démagogie hitlé­rienne, le parti national-socialiste et le parti communiste puissent parfois paraître mener une action à peu près semblable ; mais le parti communiste allemand est allé jusqu'à se mettre à la remorque du mouvement hitlérien sur le terrain de la propagande nationaliste. Il a accordé une large place, et parfois, notamment en période électorale, la première place, aux revendications nationalistes, à la lutte contre le système de Versailles, au mot d'ordre de « libération nationale ». Au moment des élections du 31 juillet, il a accusé les social-démocrates de trahison, non seulement envers la classe ouvrière, mais envers le pays (Landesverräter). Bien plus, il a publié comme brochure de propagande, et sans commentaires, le recueil des lettres où l'officier Scheringer expliquait qu'il avait abandonné le national-socialisme pour le communisme, parce que le communisme, par les perspectives d'alliance mili­taire avec la Russie qu'il comporte, est plus propre à servir les intérêts nationaux de l'Allemagne ; et ce même Scheringer a formé sur cette base un groupement, auquel ont adhéré des gens du meilleur monde, et qui est officiellement placé sous le contrôle du parti. Il faut remarquer que cette orientation ne correspond pas du tout aux sentiments des ouvriers communis­tes allemands ; car ils sont sincèrement et profondément internationalistes. Les seuls arguments qu'on donne en faveur de cette politique, c'est qu'elle procure des voix aux élections, et qu'elle facilite le passage au communisme des ouvriers séduits par la démagogie hitlérienne. Au reste elle facilite aussi bien le passage inverse. Et, chez les ouvriers social-démocrates, elle ne provoque qu'indignation et moqueries. Que penser d'un parti révolutionnaire dont le Comité Central a dit, dans un appel lancé en vue des élections du 6 novem­bre : « Les chaînes de Versailles pèsent de plus en plus lourdement sur les ouvriers allemands » ? Ce sont donc les chaînes de Versailles que le prolé­tariat allemand aurait à briser, et non les chaînes du capitalisme ? Le Comité Central a beau ajouter que seule une Allemagne socialiste peut briser les chaînes de Versailles, sa formule n'en constitue pas moins une aide précieuse pour les démagogues hitlériens, qui cherchent avant tout à persuader aux ouvriers que leur misère est due au traité de Versailles, et non pas au fait que toute la production est subordonnée à la recherche du profit capitaliste. D'une manière générale, la bourgeoisie de chaque pays s'efforce toujours, dans les moments de crise, de tourner le mécontentement des masses ouvrières et paysannes contre les pays étrangers, évitant ainsi qu'on ne lui demande à elle-même des comptes. En Allemagne, le parti communiste s'est publiquement associé à cette manœuvre au mépris de la grande parole de Liebknecht : « L'ennemi principal est chez nous. »\par
En fin de compte, le parti communiste allemand reste isolé et livré à ses propres forces, c'est-à-dire à sa propre faiblesse. Bien qu'il ne cesse de recruter, le rythme de ses progrès ne correspond aucunement aux conditions réelles de l'action ; il existe une disproportion monstrueuse entre les forces dont il dispose et les tâches auxquelles il ne peut renoncer sans perdre sa raison d'être. Cette disproportion est plus frappante encore si l'on tient compte de l'apparence de force que donnent au parti les succès électoraux. Le prolétariat allemand n'a en somme pour avant-garde que des hommes à vrai dire dévoués et courageux, mais qui sont dépourvus pour la plupart d'expé­rience et de culture politique, et qui ont été presque tous rejetés hors de la production, hors du système économique, condamnés à une vie de parasites. Un tel parti peut propager des sentiments de révolte, non se proposer la révolution comme tâche.\par
Le parti communiste allemand fait ce qu'il peut pour dissimuler cet état de choses, aussi bien aux adhérents qu'aux non-adhérents. À l'égard des adhé­rents, il use de méthodes dictatoriales qui, en empêchant la libre discussion, suppriment du même coup toute possibilité d'éducation véritable à l'intérieur du parti. À l'égard des non-adhérents, le parti essaie de cacher l'inaction par le bavardage. Ce n'est pas que le parti reste tout à fait inactif ; il essaie, malgré tout, de faire quelque chose dans les entreprises ; il a fait quelques tentatives pour organiser les chômeurs ; il dirige des grèves de locataires, auxquelles il n'arrive pas d'ailleurs à faire dépasser le cadre d'une rue ou d'un fragment de quartier ; il lutte contre les terroristes hitlériens. Tout cela ne va pas bien loin. Le parti y supplée par le verbiage, la vantardise, les mots d'ordre lancés à vide. Quand les hitlériens veulent se donner l'apparence d'un parti ouvrier, leur propagande et celle du parti communiste rend presque le même son. Cette attitude démagogique du parti communiste ne fait d'ailleurs qu'augmenter la défiance des ouvriers des entreprises à son égard. Et d'autre part, par l'orga­nisation des réunions, les paroles rituelles, les gestes rituels, la propagande communiste ressemble de plus en plus à une propagande religieuse ; comme si la révolution tendait à devenir un mythe, qui aurait simplement pour effet, comme les autres mythes, de faire supporter une situation intolérable.\par
Beaucoup de camarades seront tentés de croire que ce tableau a été chargé à plaisir ; ils en trouveront pourtant la confirmation au moins partielle dans la presse communiste, notamment dans le n° 19-20 de l'{\itshape Internationale Syndicale Rouge} et dans le discours de Piatnitsky au XII\textsuperscript{e} Plenum du Comité exécutif de l'Internationale Communiste. À vrai dire, on pourrait citer quelques faits qui sembleraient démentir ce tableau ; et ces dernières semaines, en particulier, il semble y avoir eu au moins un léger progrès. Pour permettre à nos camarades de juger si, dans l'ensemble, ce tableau est juste, il faut mettre sous leurs yeux un exposé aussi objectif que possible de la récente histoire du mouvement communiste allemand.\par
({\itshape L'École émancipée}, 23\textsuperscript{e} année, n° 19, 5 février 1933.)
\subsubsection[VII.]{VII.}
\noindent « Le Parti Communiste allemand », a dit Piatnitsky au XII\textsuperscript{e} Plenum du Comité exécutif de l'Internationale, « s'est révélé capable de manœuvrer et de réorganiser son travail quand le besoin s'en est fait sentir. Vous savez, par exemple, que la direction du parti communiste allemand s'est prononcée contre la participation au référendum sur la dissolution du Landtag prussien. Des articles leaders contre la participation à ce référendum furent publiés dans quelques journaux du parti. Mais quand le Comité Central, de concert avec l'Internationale Communiste, conclut à la nécessité de prendre une part active au référendum, les camarades allemands mirent en quelques jours en action tout le Parti. Sauf le parti communiste de l'U.R.S.S., aucun parti n'aurait pu en faire autant. » Ce dont Piatnitsky félicite en ces termes le parti communiste allemand, c'est quelque chose d'analogue à ces revirements dont nous sommes parfois témoins aux Congrès fédéraux. Qu'on se souvienne du plus caractéris­tique de ces tournants soudains ; et qu'on imagine l'effet qu'il aurait produit, s'il avait concerné une question de première importance pour le pays tout entier, et d'actualité brûlante. C'est exactement ce qui se passa alors en Allemagne, il y a un an et demi, lors du plébiscite organisé par les hitlériens contre le Landtag social-démocrate de Prusse. Les ouvriers social-démocrates rient encore aujourd'hui en racontant comment leurs camarades communistes se mirent tout à coup à faire de la propagande pour ce plébiscite, alors que la veille encore ils juraient qu'il ne pouvait même être question d'y prendre part. Les conséquences de cette participation se font encore sentir. Ce plébiscite avait été entrepris par les hitlériens dans le but avoué de conquérir le gouver­nement de Prusse ; il s'agissait donc d'une lutte entre la social-démocratie, parti qui, par essence, cherche toujours à établir un certain équilibre entre le prolétariat et la bourgeoisie, et le parti hitlérien, qui a pour but l'écrasement complet du prolétariat et l'extermination de son avant-garde. Dans cette lutte, c'est aux côtés du parti hitlérien que s'est rangé le parti communiste allemand, et cela sur l'ordre formel du Komintern. C'est là ce qu'on appela officiellement le « plébiscite rouge ». Ce tournant brusque fut, d'après la {\itshape Rote Fahne}, accepté « sans discussion » dans le parti. C'est alors qu'on se mit à parler de « libé­ration nationale », et qu'on publia dans la presse du parti ({\itshape Fanfare} du 1\textsuperscript{er} août 1931) des paroles de Scheringer concernant la « guerre libératrice » et « les morts de la guerre mondiale qui ont donné leur vie pour une Allemagne libre » . L'impression que produisit tout cela sur les masses réformistes est facile à concevoir.\par
Les conséquences logiques de cette orientation se sont développées pendant le reste de l'année 1931 et les premiers mois de 1932. Pendant cette période d'ascension verticale, d'ascension vertigineuse du national-socialisme, tous les efforts du parti communiste allemand furent dirigés contre la social-démocratie. C'est ce que reconnaît Piatnitsky. « Après le XI\textsuperscript{e} Plenum », dit-il (c'est-à-dire à partir de juin 1931), « dans les documents du parti communiste allemand, la social-démocratie est justement caractérisée comme appui social principal de la bourgeoisie, mais alors on a oublié les fascistes ». Il commença même à cette époque à se développer dans le parti l'idée d'une stratégie étran­ge, consistant à tendre à Hitler « le piège du pouvoir ». Toute cette politique trouva sa sanction au cours de ces élections présidentielles où la social-démocratie arriva à faire marcher toutes ses troupes pour Hindenburg ; où Thaelmann, seul candidat ouvrier, ne fut pas soutenu pour ainsi dire par la voix d'un seul ouvrier social-démocrate.\par
Dans les mois qui suivirent, il se produisit un redressement très net ; mais le mérite n'en revient pas au parti communiste allemand, il revient aux sections d'assaut hitlériennes. Les hitlériens se mirent en effet à assassiner les communistes chez eux ou dans la rue ; il y eut toute une série d'escarmouches sanglantes entre hitlériens et communistes ; dès lors le parti communiste fut contraint, en dépit des résolutions du XI\textsuperscript{e} Plenum, de se tourner principalement contre le fascisme. Mais les hitlériens rendirent un service plus grand encore au prolétariat allemand quand ils se mirent à attaquer indistinctement communistes et social-démocrates ; ils provoquèrent ainsi la réalisation d'un front unique spontané, et faillirent réaliser ce « bloc marxiste » qui n'existait malheureusement que dans les déclarations de leurs chefs. Le parti commu­niste aurait pu organiser ce front unique spontané, l'élargir, le transporter dans les entreprises. Il n'en fit rien. C'est ce que constate Piatnitsky :\par
« Comment le front unique a-t-il été réalisé ? Dans la rue. Grâce au fait que les nazis ne distinguaient pas les social-démocrates des communistes, mais qu'ils frappaient aussi bien les ouvriers social-démocrates, les réfor­mistes et les sans-parti que les communistes ; grâce à ce fait on a réussi à réaliser le front unique. C'est là un très bon front unique contre lequel je n'ai rien à dire. C'est le front unique de la lutte de classes. Mais il nous est arrivé de côté, pas à l'entreprise, pas aux Bourses du Travail, pas dans les syndicats. Et si ce front unique était transféré dans les entreprises, alors, dans la majorité des cas, ce n'était pas grâce à notre travail, mais parce que les ouvriers étaient indignés par les assassinats, qu'ils organisaient des grèves le jour de l'enterrement des victimes. »\par
Au mois de juillet cependant, le front unique commença à prendre une forme organisée. Il se créa en beaucoup d'endroits des comités de lutte anti­fasciste : contrairement à la doctrine du front unique « seulement à la base » , il y eut des propositions de front unique d'organisation à organisation. Bien plus, le front unique se transporta même dans les parlements ; aux Landtag de Prusse et de Hesse, les communistes donnèrent leurs voix à un social-démo­crate pour éviter l'élection d'un hitlérien. Néanmoins on continuait à parler de « social-fascisme », et les revendications nationales étaient placées en tête du programme électoral. L'incohérence de cette politique, les fautes des mois antérieurs, le peu de contact avec les masses, tout cela trouva sa sanction dans l'échec des 20 et 21 juillet 1932.\par
Le 20 juillet, c'est le coup d'État par lequel von Papen chassa du gouvernement prussien la social-démocratie. Le parti communiste allemand décida de prendre contre von Papen la défense de ce même gouvernement prussien qu'il avait, un an auparavant, attaqué aux côtés de Hitler. Cette déci­sion aurait pu accroître rapidement son influence sur les masses réformistes. Car le divorce entre les cadres et la base de la social-démocratie, divorce que n 'avait pu réaliser même le soutien accordé par la social-démocratie à la candidature de Hindenburg, devint un fait accompli grâce à la passivité de la social-démocratie en cette journée historique. Si le parti communiste avait pu opposer à cette passivité sa propre capacité d'action, il aurait acquis de ce fait une autorité considérable.\par
Que fit-il ? Il ne fit absolument rien, sinon lancer brusquement et sans préparation un appel en faveur de la grève générale. Cet appel ne rencontra que le vide, et cela au sein même du parti. « Les organisations du Parti n'ont pas répondu à l'appel de la grève », dit Piatnitsky ; et il ajoute que ce n'était nullement inattendu. Pas un ouvrier social-démocrate, bien plus, pas un ouvrier communiste n'a pris au sérieux un appel qui, lancé dans de telles conditions, ne pouvait aboutir qu'à discréditer le mot d'ordre de grève géné­rale. Le parti communiste allemand a donné, ce jour-là, l'impression qu'il jouait avec un mot d'ordre aussi grave, qu'il lui donnait en quelque sorte la valeur d'un alibi pour dissimuler son inaction. Chez les ouvriers social-démocrates, le souvenir du 20 juillet éveille à la fois l'indignation et le rire ; l'indignation quand ils songent à la social-démocratie, le rire quand ils songent au parti communiste. Quant aux militants communistes, les événements de cette journée provoquèrent, dans leurs rangs, une dépression très profonde. Pour la première fois, ils avaient clairement aperçu à quel point leur parti était isolé des masses, dans quelle impuissance radicale il se trouvait par rapport à une action effective.\par
Le succès électoral du 31 juillet leur donna l'impression qu'ils pouvaient enfin respirer. Mais ce succès empêcha aussi de se produire les conséquences salutaires qu'aurait pu avoir cette dépression ; à savoir un examen sérieux des fautes du parti. Bien plus, le Comité Central du parti décida de blâmer les fractions parlementaires des Landtags de Prusse et de Hesse, et de revenir à la pratique du front unique « seulement à la base ». Ou du moins c'est ce qu'affirmèrent les trotskystes ; à la base du parti, on n'eut pas connaissance de ces décisions. Mais ce qui est sûr, c'est qu'après le 31 juillet, il n'y eut plus de propositions de front unique d'organisation à organisation ; quant aux comités de front unique surgis spontanément en juillet, ils disparurent, et d'autant plus rapidement que les attentats hitlériens se firent plus rares, et bientôt s'interrompirent.\par
({\itshape L'École émancipée}, 23\textsuperscript{e} année, n° 20, 12 février 1933)
\subsubsection[VIII.]{VIII.}
\noindent Pendant tout ce mois d'août 1932, le parti communiste fut pratiquement réduit à l'illégalité. Sa presse fut muselée ; ses réunions, même parfois ses réunions privées, interdites ; ses militants frappés de peines incroyables lorsqu'il se produisait des luttes, sanglantes ou non, entre eux et les hitlériens ; et, bien entendu, ses partisans traqués dans les entreprises. Isolé comme il l'était, et sans expérience de l'action illégale, il ne pouvait résister. Il craignait d'être réduit, s'il agissait, à l'illégalité complète. Pendant tout ce mois d'août, on eut à Berlin, la plupart du temps, l'impression pénible que le parti commu­niste allemand n'existait pas. Cependant le parti s'efforçait fiévreusement de reprendre pied dans les entreprises ; et il dirigeait ses efforts selon la théorie en faveur, depuis quelque temps, dans l'Internationale : « Défendre les revendications immédiates, même les plus minimes. » Ainsi cette même Internationale communiste qui a si souvent voulu donner artificiellement un caractère politique aux luttes revendicatives, voulait entraîner sur le terrain des revendications un prolétariat bien convaincu que la situation ne laissait place qu'à une lutte de caractère politique. Heureusement pour le parti, les décrets-lois des 5 et 6 septembre vinrent eux-mêmes donner à toute lutte revendicative le caractère d'une lutte directe contre le pouvoir d'État.\par
Ces décrets-lois firent enfin éclater une vague spontanée de grèves. En maintes entreprises, les ouvriers, grâce au front unique des trois partis, grâce surtout aux primes à la production, récemment instituées, qui rendaient le patronat désireux d'éviter les conflits, eurent gain de cause après un ou deux jours de luttes, ou même sur de simples menaces. Grèves et succès, le parti communiste, sans autre explication, porta tout à son propre actif ; de sorte qu'en lisant sa presse, on était tenté de supposer qu'un miracle avait transfor­mé, en quelques jours, en une force écrasante l'extrême faiblesse du prolétariat et du parti. Le parti manquait ainsi à son devoir le plus élémentaire, celui d'informer les ouvriers et de leur faire apercevoir clairement le rapport des forces. Il est d'ailleurs probable que l'influence des organisations syndicales rouges s'accrut pendant cette période ; mais elles ne purent ni coordonner ni étendre le mouvement gréviste.\par
Deux semaines après que ce mouvement se fut en apparence terminé, une semaine avant les élections, un conflit éclata dans les transports de Berlin. Quatorze mille syndiqués réformistes sur vingt et un mille demandèrent la grève ; la majorité statutaire n'était pas atteinte. Le syndicat rouge entraîna les ouvriers à la grève, malgré la bureaucratie réformiste, et avec l'appui des hitlériens. La social-démocratie présenta les rouges et les hitlériens comme voulant de concert détruire les organisations ; elle les accusa de buts purement électoraux ; elle dénonça la grève comme constituant une manœuvre hitlé­rienne imprudemment soutenue par les communistes, et destinée peut-être à préparer un putsch ; elle voulait faire croire qu'en essayant de briser la grève elle ne trahissait pas les ouvriers, mais luttait contre le fascisme. L'attitude du parti communiste fut celle qui était la plus propre à donner à ces formules l'apparence de la vérité. La {\itshape Rote Fahne} raconta avec enthousiasme comment des communistes et des hitlériens unis démolirent un camion du Vorwaerts, journal social-démocrate ; elle célébra comme une victoire prolétarienne cette action à laquelle avaient participé des fascistes. Le parti communiste ne fit rien pour parer au danger d'un progrès de l'influence hitlérienne sur les ouvriers à la faveur de la grève ; pendant la semaine que dura la grève, c'est-à-dire jusqu'aux élections inclusivement, il dirigea toutes ses attaques contre la social-démocratie, et suspendit à peu près sa lutte contre les conceptions fascistes. Et il semble bien qu'à Berlin le parti national-socialiste ait alors gagné des voix ouvrières, alors qu'il en perdait partout ailleurs. Or, l'aspect politique de la grève était d'autant plus important que, par la nature même de la corporation en lutte, tout se passait dans la rue. Les hitlériens lancèrent leurs sections d'assaut contre les briseurs de grève, et se rendirent ainsi maîtres de la rue dans certains quartiers. Les ouvriers aussi, il est vrai, et surtout les chômeurs descendirent dans la rue, en masse et spontanément, pour aider les piquets de grève ; si l'on en juge par l'affolement des journaux bourgeois et leurs appels à la répression, le prolétariat dut à ce moment montrer sa force. Le parti communiste ne fit rien pour organiser cette solidarité spontanée. Il avait pourtant assez de militants à Berlin pour pouvoir le faire ; mais ces militants étaient à peu prés entièrement absorbés par la propagande électorale. Le dimanche, jour des élections, la bureaucratie du syndicat réformiste redoubla ses efforts ; les hitlériens se mirent à troubler la grève en lançant de faux bruits. Mais les militants communistes avaient donné toutes leurs forces, sans réserve, à la propagande électorale, sans songer qu'une autre tâche plus importante les attendait. Berlioz, dans {\itshape L}'{\itshape Humanité}, a raconté comment ils s'endormirent ce soir-là dans un épuisement total et sans arrière-pensée. Le lendemain, la {\itshape Rote Fahne} annonçait en lettres géantes que la grève se poursui­vrait jusqu'à la victoire ; mais des lettres de renvoi vinrent démoraliser les grévistes qui n'étaient plus soutenus par les masses de la rue. Le parti repoussa avec indignation la proposition faite par quelques oppositionnels de ne plus lutter que pour le retrait des licenciements. Néanmoins, le soir même, la circulation reprenait. Et, le lendemain, les grévistes retournèrent au travail en acceptant, non seulement la réduction de salaire qui faisait l'objet du conflit, mais encore le licenciement de 2 500 de leurs camarades qui se trouvèrent ainsi jetés sur le pavé sans même la maigre ressource du secours de chômage. C'est là ce que la Rote Fahne osa annoncer sous le titre : « Trahis, mais non vaincus. »\par
Ainsi, dans la ville qui est, en Allemagne, la citadelle du communisme, au moment même où, ayant gagné prés de 140 000 voix, il dépassait de loin, à Berlin, tous les autres partis, le parti communiste allemand a dû terminer par une capitulation complète une grève déclenchée sous sa responsabilité, dont la portée politique était considérable, et où l'action des masses constituait un facteur aussi important que celle des grévistes eux-mêmes. Le lendemain même du jour où la victoire électorale avait donné au parti communiste une si grande apparence de force, l'impuissance réelle du parti apparut aussi claire­ment et aussi tragiquement que le 20 juillet. Au reste il va de soi que toute la presse officielle de l'Internationale Communiste a célébré cette grève comme une victoire.\par
Si cette grève n'a pas accru les forces du prolétariat, elle a accru indirec­tement, et d'une manière décisive, celles de la bourgeoisie. Elle constituait, pour les hitlériens, une manœuvre de chantage à l'égard de la grande bour­geoisie, destinée à faire comprendre à celle-ci qu'elle devait à tout prix, pour son propre salut, empêcher. le parti hitlérien soit de s'affaiblir, soit de se tourner, même momentanément, contre elle. Il n'y a pas d'ailleurs d'autres relations, entre le parti hitlérien et la grande bourgeoisie, qu'un chantage réciproque et permanent ; chacun voudrait se subordonner l'autre ; chacun périrait sans l'aide de l'autre ; mais un parti peut courir des risques que ne peut courir une classe. C'est pourquoi la manœuvre tentée par les hitlériens au moyen de la grève des transports vient finalement d'aboutir, après la tentative infructueuse de von Schleicher pour domestiquer le mouvement hitlérien, à la nomination de Hitler au poste de chancelier.\par
Pendant les trois mois qui se sont écoulés dans l'intervalle, le parti communiste allemand n'a à peu près rien fait. Quelques jours avant la grève, une conférence extraordinaire du parti avait exclu Neumann, en lui reprochant notamment d'avoir trop vivement critiqué les fautes commises le 20 juillet, fautes reconnues par la conférence ; et elle avait renouvelé la confiance du parti à Thaelmann, qui ne risque pas de critiquer trop vivement les fautes, puisque c'est lui qui les commet. Elle avait aussi abandonné, il est vrai, en en rejetant à tort toute la responsabilité sur Neumann, certaines déviations natio­nalistes (le mot d'ordre de « révolution populaire » substitué à celui de « révolution prolétarienne ») ; mais, en fait, la politique nationaliste continua, à peine moins accentuée. On insista beaucoup plus sur la solidarité des prolétariats français et allemand dans la lutte contre Versailles, ce qui semble excellent ; mais il n'y avait là que de la poudre jetée aux yeux du prolétariat allemand, le parti communiste français ne faisant pas grand-chose à cet égard en dehors de quelque titres tapageurs dans {\itshape L'Humanité} \footnote{Nous ne devons pas nous dissimuler notre responsabilité dans la situation allemande. L'argument favori des hitlériens contre les communistes est « Un ouvrier français est d'abord français, ensuite seulement ouvrier ». Nous aurions le droit d'organiser immé­diatement une vaste campagne, par articles, tracts brochures ou réunions, ou par tous ces moyens, pour manifester au prolétariat allemand notre solidarité agissante, faire comprendre au peuple français que France, par son impérialisme agressif et son attache­ment au système de Versailles est directement responsable du mouvement hitlérien, enfin préparer un accueil fraternel aux camarades allemands que la terreur fasciste forcera peut-être bientôt à passer la frontière. Rester inactifs sur ce terrain serait une grave faute ; et, au cas où le parti communiste n'entreprendrait rien de sérieux en ce sens, elle ne ferait que rendre ce devoir plus impérieux pour les organisations indépendantes comme notre Fédération. (Note de S. W.)}. Il est vrai aussi que la conférence avait décidé d'orienter la propagande syndicale surtout vers la conquête d'une position solide à l'intérieur des syndicats réformistes ; on ne peut dire jusqu'à quel point ce mot d'ordre si tardif a été appliqué. Pour le front unique, la tactique est restée la même ; et, bien qu'en décembre et janvier un renouveau de terreur hitlérienne ait créé dans les masses un vif courant en faveur du front unique, on n'a rien tenté pour rendre ces velléités efficaces en les faisant passer dans le domaine de l'organisation.\par
Après les journées du 20 juillet et du 7 novembre, celle du 22 janvier manifesta une fois de plus, et plus tragiquement encore, l'impuissance du parti communiste. Une semaine auparavant, les hitlériens avaient annoncé qu'ils organiseraient ce jour-là une manifestation en plein nord de Berlin, c'est-à-dire dans la partie rouge de la ville, et devant la maison même du parti communiste (maison Karl Liebknecht). La contre-manifestation annoncée par le parti communiste fut interdite. Le parti ne vit là qu'une provocation préparant une mesure d'interdiction contre lui ; et il resta inerte, avant, pendant et après la manifestation hitlérienne, en dehors de quelques appels dans sa presse et d'une contre-manifestation paisible organisée le 24 au même endroit, c'est-à-dire dans le quartier où il est maître. Péri, spécialement chargé, dans {\itshape L'Humanité}, de transformer en succès les défaites du parti allemand, chanta victoire, aussi bien à propos de quelques manifestations spontanées qui eurent lieu, le 22 janvier, sur le passage des hitlériens, qu'à propos de l'inertie où demeura le parti, inertie présentée comme un triomphe de la « discipline communiste ».\par
En fait, l'appréciation donnée par le parti était entièrement fausse. Il ne s'agissait pas de préluder à l'interdiction du parti communiste ; une telle mise en scène aurait été inutile à cet effet. Il s'agissait, pour le fascisme, d'une sorte de répétition générale destinée à éprouver la force de résistance du prolétariat berlinois ; et, à l'expérience, cette force se révéla nulle. Certes le parti eut raison de ne pas envoyer ses militants au massacre ; mais le prolétariat a d'autres moyens de lutter. Ses forteresses à lui, ce sont les entreprises. Or les entreprises protestèrent, mais elles ne bougèrent pas. Et cependant, d'après la presse communiste, qui a dû dire la vérité sur ce point, les masses réformistes étaient soulevées par une vive indignation. Malgré cette indignation, et bien qu'averti plusieurs jours à l'avance, le parti ne fit aucune proposition de front unique d'organisation à organisation ; il s'en tint aux appels vagues et ineffi­caces à l'action spontanée des ouvriers. Comme de tels appels ne constituent pas une action, ni même une tentative d'action, on est forcé de dire qu'en cette journée décisive le parti communiste allemand a purement et simplement capitulé. Quelques jours après cette capitulation eut lieu l'événement prévu, redouté depuis si longtemps ; Hitler fut nommé chancelier.\par
Cette nomination n'a pourtant pas décidé encore de l'issue de la lutte. Il est extrêmement grave que les bandes d'assassins fascistes aient à présent derrière elles le pouvoir d'État avec ses caisses, son appareil judiciaire, son appareil policier ; mais la bourgeoisie n'a pas encore livré l'Allemagne au fascisme ; elle a écarté les hitlériens de tous les ministères importants, et notamment du Ministère de la Reichswehr. Hitler est forcé de continuer, à l'intérieur du ministère, la lutte qu'il mène depuis huit mois contre les partis de la grande bourgeoisie. Mais n'oublions pas qu'il garde entre les mains le même atout grâce auquel il a obtenu, malgré la répugnance de Hindenbourg, le poste de chancelier ; à savoir le fait que le mouvement hitlérien est indispensable et risquerait de se décomposer si les partis purement bourgeois l'écartaient du pouvoir ou le domestiquaient.\par
Toute la question est donc de savoir qui, du prolétariat ou de la bour­geoisie, réalisera en premier lieu l'unité d'action dans ses propres rangs. La bourgeoisie est de beaucoup la plus avancée sur ce terrain. Certes, dès l'arrivée de Hitler au pouvoir, le parti communiste et la social-démocratie se sont lancé mutuellement des appels à l'unité d'action, mais ils ont eu soin de les formuler de la manière la plus vague. Il y a d'autre part, ces derniers jours, un assez grand nombre de manifestations de masse et même de grèves dans lesquelles le front unique a été réalisé spontanément. Mais, même maintenant, on ne fait rien pour organiser ce front unique. De la social-démocratie on ne peut évidemment attendre aucune initiative à cet égard ; mais il lui serait assez difficile de refuser des propositions précises de front unique formulées par le parti communiste. Mais le parti communiste ne se décide pas à abandonner la tactique du front unique « seulement par en bas ». S'il va aux manifestations de masse organisées par la social-démocratie contre le fascisme, il n'en profite pas pour mettre les chefs social-démocrates au pied du mur ; il ne fait qu'y lancer, comme il a fait récemment au Lustgarten, des appels vagues adressés seulement aux ouvriers, appels qui laissent bonzes réformateurs et terroristes hitlériens dans une quiétude parfaite.\par
Quant à l'avenir du parti, il faut avant tout mettre les militants en garde contre les illusions. Il va sans doute y avoir des mouvements spontanés, grèves, batailles de rue, où les communistes seront en tête, parce que ceux qui vont se ranger sous le drapeau du communisme sont aussi d'ordinaire ceux qui sont les plus ardents et les plus résolus. Mais le parti comme tel n'aura le droit de porter à son actif que les actions dues, non à l'héroïsme individuel de ses membres, mais à la force qu'il possède en tant qu'organisation. L'héroïsme des meilleurs ouvriers allemands, qu'ils portent ou non la carte du parti, n'atténue pas, mais aggrave les responsabilités du parti communiste.\par
D'autre part, si le parti change sa ligne politique, il faut se demander non pas seulement si le changement est bon, mais s'il est assez complet et assez rapide. Ce n'est pas là une simple différence de degré, la situation est une de celles où « la quantité se change en qualité ». Entre Rigoulot et un avorton il n'y a qu'une différence de degré si on les met devant un poids de dix kilos ; si le poids est de cent kilos, il y a différence de nature. Si le parti corrige ses erreurs trop lentement ou trop faiblement, c'est comme s'il ne corrigeait rien.\par
Et même on peut douter de l'efficacité d'un changement radical au cas où ce changement ne serait qu'une volte-face bureaucratique, et n'aurait pas été précédé d'un examen honnête et libre de la situation à tous les échelons du parti. Car une volte-face bureaucratique n'offrirait aucune garantie, risquerait de dérouter les militants et de discréditer encore un peu plus le parti aux yeux des masses. La question décisive est donc celle de la démocratie à l'intérieur du parti ; or la démocratie ne se rétablit pas du jour au lendemain, ni surtout dans la semi-illégalité où se trouve réduit le parti en ce moment, et qui menace de s'aggraver de jour en jour.\par
Cependant, pour résoudre complètement la question de savoir si l'on peut garder quelque espoir dans la direction du parti communiste allemand, il faut examiner la structure intérieure du parti.\par
({\itshape L'École émancipée}, 23\textsuperscript{e} année, n° 21, 19 février 1933.)
\subsubsection[IX.]{IX.}
\noindent Le parti communiste, surtout en Allemagne, où les autres tendances révolutionnaires n'existent à peu près pas, et où l'on aime tant à s'organiser, attire la fraction du prolétariat la plus consciente et la plus résolue. Quel vice caché dans l'organisation empêche les prolétaires dévoués et héroïques qui sont groupés dans le parti communiste allemand de servir effectivement leur classe ? Les simples adhérents au parti se posent parfois la question ; et ils se donnent à eux-mêmes comme réponse, avec une touchante bonne foi, la formule consacrée : « La base est responsable. »\par
La base, il est vrai, est assez mêlée. À côté de jeunes ouvriers qui, par leur maturité, leur résolution et leur courage, donnent l'impression d'être en mesure de bâtir une société nouvelle, elle en contient d'autres qui, poussés au com­munisme par le désespoir ou par le goût de l'aventure, auraient aussi bien pu aller au fascisme, et parfois en viennent ; on y trouve surtout beaucoup de nou­veaux venus, pleins de bonne volonté, mais aussi d'ignorance. Dans l'ensemble il y a certainement, parmi eux, une capacité trop faible de décision et d'initiative. D'autre part les fautes politiques du parti sont, sans aucun doute, appuyées à la base par des sentiments vifs. Beaucoup de communistes, notamment, se laissent aveugler par le dégoût légitime que leur inspire la social-démocratie ; et surtout les militants déjà âgés, qui n'ont pas oublié les jours tragiques où le ministre social-démocrate Noske se qualifiait lui-même de « chien sanglant ». Comme leur brusquerie fait dégénérer en disputes toutes leurs conversations avec les ouvriers social-démocrates, ils ignorent que ceux-ci éprouvent souvent, à l'égard de Noske, un dégoût semblable, et le proclament bien haut dans leur propre parti. En revanche, on ne peut nier qu'il existe parmi les communistes un certain courant de sympathie à l'égard des hitlériens, dont parfois, notamment dans les grèves, l'énergie apparente con­traste avantageusement avec les capitulations social-démocrates. On a souvent l'impression qu'ouvriers communistes et ouvriers hitlériens, dans leurs discussions, cherchent en vain à trouver le point de désaccord, et frappent à vide. En pleine terreur hitlérienne, on pouvait entendre des hitlériens et des communistes regretter ensemble le temps où ils luttaient, comme ils disaient, « côte à côte », c'est-à-dire le temps du « plébiscite rouge » ; on pouvait enten­dre un communiste s'écrier : « Mieux vaut être nazi que social-démocrate. » Cependant de semblables paroles excitaient des protestations ; car il se trouve aussi, dans le parti communiste, un fort courant en faveur du front unique avec la social-démocratie. Beaucoup de communistes se rendent clairement compte qu'ils ne peuvent rien faire sans l'appui des ouvriers réformistes. La politique suivie en juillet les a rendus heureux ; ils n'ont pas vu ce qu'elle avait d'insuffisant, et ne se sont pas rendu compte qu'elle était abandonnée dès le début d'août. D'une manière générale, il y a, surtout depuis le 20 juillet, un malaise, qui s'est traduit par un renouveau de vie dans les cellules ; mais un malaise sourd. Longtemps les jeunes communistes n'ont pas eu conscience du caractère aigu de la situation, de la valeur inappréciable de chaque instant ; à cet égard, ils ont fait des progrès dans le courant du mois d'octobre, si l'on en croit le reportage de Berlioz. Néanmoins leur inquiétude, autant qu'on peut savoir, ne prend pas encore une forme articulée.\par
C'est l'appareil du parti qui empêche cette inquiétude de prendre une forme articulée, en laissant la menace d'exclusion pour déviation « trotskyste » ou « brandlérienne » suspendue sur la tête de chaque communiste. C'est l'appareil qui, en supprimant toute liberté d'expression à l'intérieur du parti, et en accablant les militants de tâches épuisantes et de petite envergure, anéantit tout esprit de décision et d'initiative, et empêche toute éducation véritable des nouveaux venus par les militants expérimentés. D'ailleurs cette masse de nouveaux venus aussi ignorants qu'enthousiastes est le véritable appui de l'appareil à l'intérieur du parti ; sans cette masse docile, comment réaliser ces changements complets d'orientation, accomplis sans discussion et en quelques jours, dont Piatnitsky félicite naïvement le parti allemand ? D'autre part ces sentiments fort compréhensibles, mais si dangereux, de haine à l'égard des social-démocrates, et d'indulgence ou même de sympathie à l'égard des hitlé­riens, ont été, dans bien des occasions, encouragés par la politique imposée au parti par l'appareil. D'une manière générale, c'est l'appareil qui a mis la confusion dans l'esprit de chaque communiste allemand, en parant sa propre politique de tout le prestige de la révolution d'octobre ; exactement comme les prêtres ôtent toute faculté d'examen aux fidèles enthousiastes, en couvrant les pires absurdités par l'autorité de l'Église.\par
Si la responsabilité de la faiblesse du parti communiste allemand incombe à la direction, quel est le caractère d'une telle responsabilité ? Les trotskystes, à ce sujet, ont souvent répété : « Les fautes de la social-démocratie sont des trahisons, celles du parti communiste ne sont que des erreurs. » Certes l'Inter­nationale Communiste n'a encore rien fait de comparable au vote des crédits militaires le 4 août 1914, ou au massacre des spartakistes par la social-démocratie en 1919. Mais, si l'on s'en tient à la période présente, il est difficile de saisir le sens d'une telle formule. Est-ce à dire que la social-démocratie détourne les ouvriers de toute action révolutionnaire, et que le parti commu­niste, s'il s'y prend mal pour préparer la révolution, engage du moins les ouvriers à chercher une issue dans cette voie ? Mais la révolution, ce n'est pas une religion par rapport à laquelle un mauvais croyant vaille mieux qu'un incrédule ; c'est une tâche pratique. On ne peut pas plus être révolutionnaire par de simples paroles que maçon ou forgeron. Seule est révolutionnaire l'action qui prépare une transformation du régime ; ou encore les analyses et les mots d'ordre qui ne prêchent pas simplement, mais qui préparent une telle action. En ce sens il serait hasardeux d'affirmer que le parti communiste allemand, en tant qu'organisation, et indépendamment des sentiments de la base, est plus révolutionnaire que la social-démocratie ; encore celle-ci peut-elle se vanter d'avoir rendu des services beaucoup plus sérieux que lui, autre­fois, sur le terrain des réformes. Ou bien la formule des trotskystes signifie-t-elle que les intentions des bureaucrates communistes sont plus pures que celles des bureaucrates social-démocrates ? La question est insoluble et d'ailleurs sans intérêt. Pratiquement on peut, pour un parti ouvrier, nommer erreurs les fautes dues à une mauvaise appréciation des intérêts du prolétariat, trahisons les fautes dues à une liaison, d'ailleurs consciente ou inconsciente, mais organique, avec des intérêts opposés à ceux du prolétariat.\par
P.-S. - Où en est la question du front unique ?\par
Le 5 février le bruit courait, selon Trotsky, que Moscou avait donné ordre de le réaliser.\par
Néanmoins, vers le milieu du mois, le parti communiste allemand a répon­du par un refus brutal à la proposition, faite par le parti social-démocrate, d'un « pacte de non-agression ». La social-démocratie expliquait que chaque parti conserverait son point de vue et son indépendance, mais qu'on mettrait un terme aux attaques haineuses. Cette proposition constituait une base de négo­ciations très acceptable, et, dans ces négociations, le parti communiste, qui aurait naturellement proposé une unité d'action effective avec pleine liberté de critique des deux côtés, aurait trouvé un précieux appui auprès de la base même de la social-démocratie. Il s'est privé de cet appui en refusant même de négocier.\par
({\itshape L'École émancipée}, 23\textsuperscript{e} année, n° 22, 26 février 1933.)
\subsubsection[X.]{X.}
\noindent Le parti communiste allemand, comme toutes les sections de l'Internatio­nale, est organiquement subordonné à l'appareil d'État russe. Les bolcheviks, contraints par la nécessité, ont fait ce qu'avaient fait avant eux, comme l'a vu Marx, tous les révolutionnaires à l'exception des communards ; à savoir per­fectionner l'appareil d'État au lieu de le briser. « Briser la machine bureau­cratique et militaire », c'est, comme l'écrivait Marx en avril 1871, « la condition de toute révolution populaire sur le Continent » ; et Lénine, après Marx, a montré, dans {\itshape L'État et la Révolution}, qu'un appareil d'État distinct de la population, composé d'une bureaucratie, d'une police, d'une armée perma­nente, a des intérêts distincts des intérêts de la population, et notamment du prolétariat. La politique intérieure russe n'a pas à être examinée ici. Mais l'appareil d'État de la nation russe, bien qu'issu d'une révolution, a, en tant qu'appareil permanent, des intérêts distincts de ceux du prolétariat mondial ; distincts, c'est-à-dire en partie identiques, en partie opposés. Peut-être pourrait-on considérer la célèbre doctrine du « socialisme dans un seul pays » comme constituant l'expression théorique de cette divergence.\par
Jusqu'où cette divergence va-t-elle ? Dans quelle mesure explique-t-elle l'orientation politique de l'Internationale Communiste ? Nous n'en savons rien. Ce que nous savons, c'est que les principales fautes du parti communiste allemand, à savoir la lutte sectaire contre la social-démocratie considérée comme « l'ennemi principal », le sabotage du front unique, la participation au soi-disant « plébiscite rouge », la honteuse démagogie nationaliste, tout cela a été imposé au parti par l'Internationale (cf. notamment le n° 21-22 de l{\itshape 'Inter­nationale Communiste}). Et, quand on parle de l'Internationale Communiste, il ne faut pas oublier qu'il s'agit d'un appareil sans mandat, puisque le Congrès n'a pas été réuni depuis cinq ans ; un appareil irresponsable, qui se trouve entièrement entre les mains du Comité Central russe. On l'a bien vu quand le parti russe a exclu Zinovief qui était à la tête de l'Internationale, sans consulter les autres sections, ni même leur donner des renseignements précis. Et, en ce qui concerne l'orientation nationaliste du parti allemand, nous savons qu'elle correspondait parfaitement bien à la politique extérieure de l'U.R.S.S., alors soucieuse avant tout d'empêcher que la France n'attire l'Allemagne dans un bloc antisoviétique. D'autre part, si les conversations entre Leipart et von Schleicher nous indignent, nous avons le droit de nous demander aussi ce que Litvinof, lors de son récent passage à Berlin, a pu dire à von Schleicher au cours de leur entretien particulier. Enfin, d'une manière générale, on remarque dans les fautes du parti allemand une continuité, une persévérance assez rares quand il y a une erreur fortuite ; et le Komintern emploie, contre ceux qui critiquent ces fautes, des méthodes assez semblables à celles qu'emploie un appareil d'État contre ceux qui menacent ses intérêts. Thorez et Sémard l'ont montré le jour où, à propos de l'Allemagne, ils ont fait assommer les trotskystes en pleine réunion publique.\par
Tout cela ne suffirait pas peut-être à justifier une accusation de trahison. Mais ce qui est certain, c'est qu'une transformation dans le régime intérieur du parti, transformation qui constituerait la condition première d'un redressement véritable, est directement contraire aux intérêts de l'appareil d'État russe. L'appareil d'État russe, comme tous les appareils d'État du monde, déporte, exile, tue directement ou indirectement ceux qui essayent de diminuer son pouvoir ; mais, comme État issu d'une révolution, il a besoin de pouvoir se dire approuvé par l'avant-garde du prolétariat mondial. C'est pourquoi, non seulement dans le parti russe, mais dans l'Internationale, la chasse aux oppo­sitionnels prime toute autre considération. Étant donné la gravité de la situation en Allemagne, on peut dire que l'appareil du parti allemand, toutes les fois qu'il exclut un communiste pour désaccord avec la ligne du parti, trahit les ouvriers allemands pour sauver la bureaucratie d'État russe\par
Contre le triple appareil du parti, du Komintern et de l'État russe, seuls luttent quelques minuscules groupements d'opposition. Leur opposition est d'ailleurs plut ou moins catégorique. Le « Leninbund », avec Urbahns, va jusqu'à refuser entièrement à l'U.R.S.S. le titre d'État ouvrier ; les trotskystes, au contraire, d'ailleurs actifs et courageux, ont à l'égard de l' « État ouvrier », du « parti de la classe ouvrière », un perpétuel souci de fidélité qui nuit souvent à la netteté de leur jugement \footnote{ Le groupe trotskyste allemand n'est pas « liquidé », contrairement à ce que {\itshape L'Humanité} a annoncé bruyamment ; il a seulement perdu une fraction importante de ses membres (un quart, dit-on), qui a non seulement renié Trotsky, mais condamné toutes les oppositions. Un mois auparavant, les mêmes se vantaient d'être les meilleurs trotskystes. Le départ de tels éléments est un événement heureux pour le groupement ; mais la manière dont il s'est opéré semble indiquer un régime intérieur peu différent de celui du parti. (Note de S. W.)}. Brandler, dont les partisans sont les seuls oppositionnels à avoir de l'influence dans les syndicats réformistes, va bien plus loin encore, puisqu'il approuve tout ce qui se fait en U.R.S.S., y compris les déportations. Quelques brandlériens, groupés autour de militants de valeur, tels que Fröhlich, ont fini par juger cette attitude comme une capitulation ; et, désespérant d'autre part de redresser le parti communiste, ils sont allés au petit « parti socialiste ouvrier » où les attirait la présence d'une jeunesse d'esprit vraiment révolutionnaire. Mais ils se sont ainsi trouvés pris dans une organisation qui joint à la faiblesse numérique d'une secte l'incohé­rence d'un mouvement de masse ; situation qui paralyse un militant. Au reste la lutte que mènent ces groupements est à peu près désespérée. Comment discréditer à l'intérieur du parti communiste une direction qui a tant de moyens d'y conserver son crédit ? Quant à tenter de discréditer le parti lui-mê­me auprès du prolétariat, aucun oppositionnel ne l'oserait, de peur de favoriser ainsi le réformisme ou le fascisme. Et, à la base, c'est à peine si, parmi de jeunes ouvriers communistes et social-démocrates, un mouvement en faveur d'un nouveau parti commence à se dessiner.\par
\par

\begin{center}
*\end{center}
\noindent Ainsi voilà où en sont ces sept dixièmes de la population allemande qui aspirent au socialisme. Les inconscients, les désespérés, ceux qui sont prêts à toutes les aventures, sont, grâce à la démagogie hitlérienne, enrôlés comme troupes de guerre civile au service du capital financier ; les travailleurs pru­dents et pondérés sont livrés par la social-démocratie, pieds et poings liés, à l'appareil d'État allemand ; les prolétaires les plus ardents et les plus résolus sont maintenus dans l'impuissance par les représentants de l'appareil d'État russe.\par
Faut-il essayer de formuler des perspectives ? Les diverses combinaisons auxquelles peut aboutir le jeu des luttes et des alliances entre classes et frac­tions de classes fournissent autant de perspectives qu'on ne peut complètement repousser ; non pas même celle d'un soulèvement spontané du prolétariat, et d'une nouvelle Commune à l'échelle de l'Allemagne. Peser les probabilités serai sans intérêt. D'ailleurs la signification des divers éléments de la situation allemande dépend entièrement des perspectives mondiales ; notamment du moment où se produira ce « tournant de la conjoncture » dont on parle à Berlin depuis plus de six mois ; du rythme selon lequel la prospérité se réta­blira ; de l'ordre selon lequel elle touchera les différents pays. Nous ne pouvons rien dire de vraiment précis à ce sujet. La faiblesse du mouvement révolutionnaire est due sans doute en grande partie à cette ignorance où nous sommes des phénomènes économiques ; ignorance au reste honteuse pour des matérialistes, et qui se réclament de Marx.\par
Du moins apercevons-nous clairement la décadence du régime, et même en dehors de la crise actuelle. De plus en plus les dépenses consacrées à la guerre économique, dont la guerre militaire peut être considérée comme un cas particulier, l'emportent sur les dépenses productives ; et parallèlement, le système s'accroche de plus en plus aux appareils d'État, ces appareils qui, au cours de l'histoire, ont été toujours perfectionnés et jamais brisés. La limite idéale de cette évolution pourrait être définie comme le fascisme absolu, c'est-à-dire une emprise étouffante, sur toutes les formes de la vie sociale, d'un pouvoir d'État servant lui-même d'instrument au capital financier. Un retour de la prospérité peut interrompre en apparence cette évolution ; le danger restera le même tant que le régime continuera à se décomposer. Déjà la crise générale du régime semble s'étendre, non seulement aux restes de la véritable culture bourgeoise, mais au peu que nous possédons comme ébauches d'une culture nouvelle ; le mouvement ouvrier, le prolétariat, sont atteints matériel­lement et moralement, et surtout au moment actuel, par la décadence de l'économie capitaliste. Et l'on aurait tort d'être sûr que cette décadence ne peut arriver à étouffer aussi, et peut-être même sans agression militaire, les survivances de l'esprit d'Octobre en U.R.S.S. C'est en Allemagne qu'a lieu présentement l'attaque que mène la classe dominante, contraint par la crise à aggraver sans cesse l'oppression qu'elle exerce ; et nous ne pouvons mesurer l'importance de la bataille qui peut-être se livrera là-bas demain.\par
Au seuil d'une telle bataille, devant les formidables organisations écono­miques et politiques du capital, devant les bandes de terroristes hitlériens, devant les mitrailleuses de la Reichswehr, la classe ouvrière d'Allemagne se trouve seule et les mains nues. Ou plutôt on est tenté de se demander s'il ne vaudrait pas mieux encore pour elle se trouver les mains nues. Les instruments qu'elle a forgés, et qu'elle croit saisir, sont en réalité maniés par d'autres, pour la défense d'intérêts qui ne sont pas les siens.\par
{\itshape P.-S. –} En ce moment, dans le camp capitaliste, les contradictions s'affir­ment. Von Papen vient de lancer un appel en faveur d'un front « noir-blanc-rouge » comprenant les « partis bourgeois », c'est-à-dire les partis de droite à l'exclusion des national-socialistes. La {\itshape Deutsche Allgemeine Zeitung} fait de la propagande contre l'intervention de l'État dans l'économie. La social-démo­cratie voit dans ces combats intérieurs de la bourgeoisie un signe de la fai­blesse de Hitler, mais rien n'est moins sûr. Le ton des articles de la {\itshape Deutsche Allgemeine Zeitung} semble exprimer une sorte de panique, comme si la grande bourgeoisie se sentait débordée par le mouvement hitlérien. Et quant à croire qu'elle se tournera jamais catégoriquement contre Hitler, dans les conditions actuelles, il ne serait pas raisonnable d'y compter ; ce serait rendre la voie libre pour un soulèvement du prolétariat, et la bourgeoisie le sait.\par
En fait, jusqu'ici, la position de Hitler ne cesse de se renforcer, et la terreur hitlérienne de s'accroître (ordres formels à la police d'attaquer systéma­tiquement les partis de gauche, et assurance que les meurtres qu'elle commet­tra seront toujours couverts ; suppression de toute la presse communiste ; fermeture de la maison du parti (maison Karl Liebknecht) ; fermeture de l'école Karl Marx, école célèbre à Berlin, comprenant des cours primaires et secondaires, et des cours secondaires spéciaux pour jeunes ouvriers ; suppres­sion de toutes les écoles laïques prévue pour Pâques ; création d'une police auxiliaire composée de membres des sections d'assaut hitlériennes et du Casque d'Acier ; etc., etc.\par
La résistance prend la forme de grèves locales d'un jour et de manifesta­tions de masse, au cours desquelles le front unique est généralement réalisé entre organisations locales. Cette résistance reste incohérente et dispersée, à la manière des mouvements spontanés, et en somme, jusqu'ici, inefficace. Le front unique de même n'est réalisé que localement, partiellement, et d'une manière également incohérente, ce qui montre que les cadres se laissent entraîner par le courant spontané des masses en faveur de l'unité d'action au lieu de l'organiser. Ce courant est tous les jours plus fort ; un des signes en est les résolutions, d'ailleurs aussi vagues qu'ardentes, que prennent à ce sujet toutes les assemblées générales de syndicats réformistes qui ont lieu en ce moment.\par
Il y a eu propositions précises de front unique de la part des fédérations rouges des métaux et des bâtiments et de l'organisation syndicale rouge de Berlin aux organisations réformistes correspondantes ; et bien entendu, sans résultats. Les organisations syndicales rouges d'Allemagne, qui, par leur activité et leur organisation, ne font que « reproduire le parti » ({\itshape I.S.R.}, n° 19-20, P. 914), à qui elles restent inférieures quant à leurs effectifs et à leur influence, sont trop faibles pour avoir des chances sérieuses de réaliser l'unité d'action avec les organisations réformistes correspondantes, sinon en allant jusqu'à l'extrême limite des concessions, c'est-à-dire en demandant à rentrer dans la Centrale réformiste. Il est difficile de dire de loin si cette tactique, recommandée par les trotskystes, serait ou non la meilleure.\par
En tout cas l'on ne pourra croire que le parti communiste veut véritable­ment faire ce qui dépend de lui pour réaliser l'unité d'action du prolétariat que le jour où il fera lui-même, en tant que parti, des propositions assez précises et assez conciliantes pour mettre les dirigeants réformistes au pied du mur.\par
Or jusqu'ici, malgré l'état d'esprit des masses et le danger pressant, le Comité Central du parti n'a rien dit, depuis l'appel vague qu'il avait lancé au lendemain de la nomination de Hitler comme Chancelier. On pourrait croire qu'il n'existe pas.\par
({\itshape L'École émancipée}, 23\textsuperscript{e} année, n° 23, 5 mars 1933.\par

\begin{center}
\end{center}
\subsection[13. Sur la situation en Allemagne, (Tribune de discussion) (9 avril 1933)]{13. \\
Sur la situation en Allemagne \\
(Tribune de discussion) \\
(9 avril 1933)}
\noindent \par
Je saisis avec plaisir l'occasion que me donne l'article du camarade Naville pour signaler aux lecteurs de l' {\itshape E. E.} que Trotsky, non seulement a pleinement compris dès 1930 la signification de ce qui se passait en Allemagne, mais encore n'a pas cessé, depuis ce temps, de suivre de jour en jour le dévelop­pement de la situation allemande, par des analyses d'une clarté, d'une lucidité qu'il est impossible de ne pas admirer. Il n'est que juste aussi de signaler que, sous le gouvernement von Schleicher, il n'y a eu en France, dans la presse bourgeoise, social-démocrate ou révolutionnaire, que {\itshape La Vérité} pour poser des perspectives justes.\par
Cette constatation laisse entière la question de l'attitude qu'ont observée les trotskystes à l'égard de l'Internationale communiste. Le moment actuel n'est sans doute pas propice pour une discussion concernant la tactique d'une période qui est à jamais passée. Pour le présent, le camarade Naville demande avec raison à la majorité fédérale de prendre position. Mais on peut lui adresser à lui-même, ou plutôt à son organisation, et notamment à Trotsky, la même demande. Car le camarade Naville écrit : « Le fait est que l'Interna­tionale communiste a renoncé à son rôle de direction du prolétariat révolu­tionnaire {\itshape en Allemagne}. » On aimerait savoir {\itshape dans quel pays} l'Internationale communiste a conservé ce rôle. Est-ce en France, où le parti communiste joint à toutes les faiblesses dont souffrait le parti allemand celle d'être squelettique, où la C.G.T.U. s'effrite de jour en jour ? Est-ce en Espagne, en Angleterre, en Amérique, etc., où les sections de l'I.C. sont pratiquement à peu près inexis­tantes ? Ou serait-ce par hasard en U.R.S.S., où, d'après Trotsky, le prolétariat étouffe sous une dictature bureaucratique qui risque de le mener à sa perte ? Il faut savoir si la manière étrange dont s'exprime le camarade Naville est l'effet d'une maladresse de langage, ou le signe que les organisations trotskystes désirent éviter, même à présent, de prendre nettement position à l'égard de l'Internationale communiste. La situation est trop grave pour que les militants, et surtout un militant comme Trotsky, aient le droit d'adopter une attitude ambiguë.\par
({\itshape L'École émancipée}, 23\textsuperscript{e} année, n° 28, avril 1933.)\par

\subsection[14. Quelques remarques, sur la réponse de la M.O.R.  (Tribune de discussion) (7 mai 1933)]{14. \\
Quelques remarques \\
sur la réponse de la M.O.R. \protect\footnotemark  \\
(Tribune de discussion) \\
(7 mai 1933)}
\footnotetext{La M. O. R. (Minorité oppositionnelle révolutionnaire) était, dans la Fédération de l'enseignement, le groupe des partisans d'une alliance étroite avec le parti communiste. Bien que la Fédération fût affiliée à la C. G T. U., le majorité fédérale était en faveur d'une indépendance réelle du syndicalisme. (Note de l'éditeur.)}
\noindent \par
La M. O. R. se décide donc enfin, par la plume de ses adhérents de l'Hérault, à prendre la défense du parti communiste allemand. Elle n'a pas la priorité, et même son silence s'était prolongé assez longtemps pour engager {\itshape L'École émancipée} à publier l'article du camarade Ranc, qui n'est ni de l'Ensei­gnement ni du parti communiste. Cet article n'apporte d'ailleurs rien, ni fait, ni raisonnement qui soit susceptible de modifier les données du problème ; il nous apprend seulement que les fautes récentes ont leur racine dans un passé plus lointain. Nous pouvions espérer que la M. O. R., qui a des liaisons avec le parti communiste français et l'Internationale, nous apporterait autre chose ; et cela d'autant plus facilement que nos camarades communistes pouvaient entrer en rapports avec des camarades allemands compétents dans la question. Or cette réponse de la M. O. R. nous apporte en effet autre chose ; à savoir des procédés de polémique qui nous reportent aux plus beaux jours d'avant Bordeaux. Mais, bien que la M. O. R. parle d'une « accumulation d'erreurs et de mensonges », elle ne relève pas la plus petite inexactitude dans les articles qu'elle voudrait réfuter. Elle voudrait ruiner des affirmations étayées par des analyses et une documentation ; et elle ne trouve à leur opposer que des affirmations contraires dont elle ne tente même pas d'établir l'exactitude. Bien plus, ces affirmations se trouvent démenties par des faits que nos camarades de la M. O. R. doivent connaître, pour peu qu'ils se soient documentés, pour peu qu'ils aient seulement lu les articles auxquels ils répondent. Leur réponse fera plaisir, malgré tout, à tous ceux qui désirent conserver d'agréables illusions. Quant aux camarades qui veulent connaître la vérité, les seuls dont l'approbation nous soit précieuse, ils pourront trouver, dans cette vaine tentative de réponse, la preuve claire que l'exposé publié par {\itshape L'École émanci­pée} était rigoureusement exact.\par
Espérons que ces camarades comprendront pleinement le péril mortel que court le mouvement révolutionnaire du fait qu'il se trouve entre les mains d'hommes dont les uns aiment mieux régler leurs pensées et leurs actions d'après des mythes que d'après une vue claire de la réalité, et dont les autres pensent que toutes les vérités ne sont pas bonnes à dire. Un tel état d'esprit est dangereux même quand il s'agit de questions secondaires ; or, aujourd'hui, dans toutes les questions importantes, le mouvement ouvrier est entièrement livré à l'illusion et au mensonge. Nous sommes perdus si nous ne remettons pas en vigueur le grand principe que Lénine rappelait sans cesse, à savoir que la vérité, quelle qu'elle soit, est toujours salutaire pour le mouvement ouvrier, l'erreur, l'illusion et le mensonge toujours funestes.\par
Quant à la soi-disant réponse de la M. O. R., il n'y a pas lieu de la discuter. On ne discute pas des affirmations sans preuves. Mais elle fournit l'occasion de faire quelques remarques qui peuvent n'être pas inutiles.\par
La M. O. R. ne tente pas de défendre le mot d'ordre du « front unique seulement à la base ». Elle voudrait faire croire que le parti communiste allemand a tenté par deux fois d'appliquer la tactique du front unique au sommet. Ainsi lancer, sans la moindre préparation préalable, un mot d'ordre aussi grave que celui de grève générale, alors qu'on est tout à fait incapable de l'appliquer, et inviter ceux qui peut-être en seraient capables à se charger d'appliquer ce mot d'ordre, ce serait là la tactique du front unique ? Ce n'en est même pas une mauvaise caricature. Chercher à établir le front unique, c'est chercher à former à tous les degrés de la hiérarchie, de la base au sommet, des organisations mixtes établies pour un objectif déterminé et grâce auxquelles chaque tendance peut proposer sa tactique à l'ensemble des ouvriers organisés.\par
Ce que la M. O. R. se garde bien de dire, c'est ce qui s'est passé en février et mars. Au début de février, la social-démocratie allemande proposa un « pacte de non-agression », c'est-à-dire la suspension des attaques entre socia­listes et communistes. Le parti communiste non seulement refusa, ce qu'on ne peut blâmer, mais refusa brutalement et en fermant la porte à toute négocia­tion ultérieure. Le 19 février, l'Internationale Socialiste elle-même écrivit une lettre ouverte à l'Internationale Communiste, proposant le front unique contre le capitalisme et le fascisme. À cette proposition de front unique, lancée dans un moment si grave, l'Internationale Communiste ne répondit rien. Pendant près de trois semaines elle garda le silence. Survint l'incendie du Reichstag ; brutalement la répression s'abattit sur socialistes et communistes à la fois, les réduisant en fait à l'illégalité, empêchant toute organisation du front unique à l'échelle nationale. C'est alors que l'Internationale Communiste prit la parole, demandant à ses sections nationales de proposer le front unique aux partis socialistes correspondants, et d'accepter même cette fameuse « non-agres­sion » si bruyamment repoussée en Allemagne un mois plus tôt. Peut-on imaginer incohérence plus pitoyable ?\par
« Le Parti communiste allemand n'a jamais cessé de dénoncer Hitler et ses alliés. » C'est faux. Je ne reviendrai pas sur la question du « plébiscite rouge » et de la grève des transports ; les faits ont été exposés, et la M. O. R. ne tente même pas de montrer que l'exposé était inexact.\par
Il vaut mieux inviter les camarades honnêtes et réfléchis à se demander quel rôle joue en ce moment le parti communiste français à l'égard de la menace fasciste. Car il y a en France des mouvements fascistes ; ce sont tous les mouvements de révolte qui ne se placent pas sur un terrain de classe : anciens combattants, contribuables, commerçants, paysans conduits par la Ligue agraire. Tous ces mouvements, notre parti communiste les appuie. Il dit que c'est pour amener à lui les masses conduites par les démagogues fascistes. Il oublie seulement de nous dire de quels moyens il dispose à cet effet. Ce que nous voyons, c'est qu'il apporte son autorité morale, qui, grâce à la révolution d'Octobre et au plan quinquennal est encore considérable, à l'appui de courants fascistes. Ce front unique au sommet, qu'on refuse de pratiquer entre organisations ouvrières, la C. G. P. T. le pratique, dans le cadre départemental, avec toutes les organisations paysannes, y compris celles qui sont nettement fascistes, comme la Ligue agraire ; et ce front unique ne se limite pas à organiser en commun des réunions ; les réunions se terminent par des ordres du jour rédigés et déposés en commun. Et quand paysans et commerçants saccagent les coopératives ouvrières, {\itshape L'Humanité} fait leur éloge.\par
« Le Parti communiste allemand a été discipliné dans l'Internationale communiste. » C'est absurde. On ne peut parler de discipline à l'égard d'un appareil sans mandat ; et c'est le cas puisque le congrès de l'Internationale ne s'est pas réuni depuis près de cinq ans. Et cet appareil sans mandat n'a pas craint, de l'aveu de Piatnitsky lui-même, de s'immiscer directement dans les affaires allemandes. Il n'y a pas eu de « directives » de l'Internationale que le parti allemand ait eu à « transposer à la situation allemande ». L'appareil de l'Internationale a purement et simplement conduit le parti allemand, comme le prouve notamment l'affaire du « plébiscite rouge », et l'a conduit à sa perte. Thaelmann paye ses fautes par des souffrances inouïes ; il les paiera de sa vie peut-être. Sans oublier que ces fautes ont fait tomber également entre les mains des hitlériens des militants aussi courageux et cent fois plus clair­voyants que lui, nous n'attaquerons aucun de nos camarades allemands. Nous savons que les vrais responsables ne sont pas dans la forteresse de Spandau. Ils sont à Moscou, et en complète sécurité.\par
« L'heure n'est pas aux reportages défaitistes », dit la M. O. R. pour conclure, « mais à l'action ». Le jeune Daudin et ses amis de la rue d'Ulm disaient déjà être aux premiers rangs de l'action antifasciste. On peut en conclure que, tandis que nous cherchons avec anxiété quelle action efficace il nous est encore possible de mener, nos camarades de la M. O. R. ont déjà résolu le problème. Nous les supplions de nous faire profiter de leur savoir. Mais si agir, pour eux, c'est envoyer des protestations à l'ambassade alle­mande, ou encore brûler Hitler en effigie, comme on l'a fait le 9 avril à Bagnolet, nous avouons que nous préférons les analyses exactes de la situation, fussent-elles « défaitistes ».\par
C'est pourquoi il n'est pas inutile de dire quelques mots sur la situation actuelle. Hitler, qui ignorait évidemment la vue géniale de Staline, selon laquelle « le fascisme et le social-fascisme sont des frères jumeaux », n'a pas hésité à saccager toutes les organisations réformistes, à emprisonner et à torturer sauvagement quantité de militants social-démocrates. Il en résulte la capitulation cynique et honteuse que tous nos camarades connaissent, et que tous ceux qui savent raisonner avaient prévue. Les communistes au contraire ont eu en mainte occasion, et surtout à la base, une attitude héroïque. Mais ils sont et restent complètement impuissants. Les mensonges conscients de {\itshape L’Humanité} à ce sujet, comparables seulement à ceux du gouvernement français pendant la guerre, n'y peuvent rien changer. Il y a eu et il y a, certes, des tracts et des journaux illégaux, qui n'ont d'ailleurs pour contenu que des mots d'ordre irréalisables ; mais la distribution, malgré l'héroïsme des mili­tants, est très restreinte. Tous les camarades comprendront pourquoi on ne peut donner des détails plus précis. Mais on peut signaler que la catastrophe a éclaté comme un coup de foudre dans un ciel serein, et qu'en dehors des imprimeries clandestines, utilisées depuis longtemps, il n'y avait pas eu de préparation en vue de l'action illégale. Quant au pays qu'on nous a appris à considérer comme le rempart du socialisme mondial, l'U. R. S.S., tout s'est passé comme s'il n'existait pas. La bureaucratie réformiste allemande, Leipart en tête, collabore avec Hitler. Mais la « patrie internationale des travailleurs » ne voit pas d'autre obstacle à sa collaboration avec le gouvernement hitlérien que les perquisitions opérées au siège de ses représentations commerciales. Une « patrie internationale » considérerait les communistes allemands comme ses membres au même titre que ceux qui sont soumis aux lois de son État ; et l'U. R. S. S. possède de puissants moyens de pression. Les commandes de l'U.R.S.S. ne sont-elles pas, dans cette période de crise aiguë, indispensables à l'industrie allemande ? Or la menace de boycott n'a été utilisée que contre les provocations d'Hitler à l'égard de l'État russe, et nullement en faveur des militants assassinés et torturés. Même les simples protestations émises par l'U.R.S.S. ne portent que sur les mauvais traitements infligés aux citoyens soviétiques. Nous sommes forcés d'enregistrer que l'U.R.S.S. n'a plus aucun titre à être considérée comme la « patrie socialiste ».\par
Quant à Hitler, il continue à faire le nécessaire pour canaliser à son profit les aspirations socialistes de la masse. L'antisémitisme, avec tout ce qu'il comporte, et notamment le pillage des magasins, est une arme puissante à cet effet. Hitler est allé jusqu'à souiller la plus pure tradition du mouvement ouvrier, la plus imprégnée d'internationalisme prolétarien, en faisant du 1\textsuperscript{er} mai une fête national-socialiste. Certes il se heurtera à des difficultés insolu­bles, et notamment sur le terrain économique ; mais il serait puéril de croire que ces difficultés suffiront à changer le régime.\par
Il est inutile et déshonorant de fermer les yeux. Pour la deuxième fois en moins de vingt ans, le prolétariat le mieux organisé, le plus puissant, le plus avancé du monde, celui d'Allemagne, a capitulé sans résistance. Il n'y a pas eu défaite ; une défaite suppose une lutte préalable. Il y a eu effondrement. La portée de cet effondrement dépasse de beaucoup la limite des frontières allemandes. Le drame qui s'est joué en Allemagne était un drame d'une portée mondiale. Et le coup subi par le mouvement ouvrier en mars 1933 est plus grave peut-être que celui qui avait été subi le 4 août 1914.\par
({\itshape L'École émancipée}, 23\textsuperscript{e} année, n° 31, 7 mai 1933)\par

\subsection[15. Le rôle de l’U.R.S.S. dans la politique mondiale, (Tribune de discussion) (23 juillet 1933)]{15. \\
Le rôle de l’U.R.S.S. \\
dans la politique mondiale \\
(Tribune de discussion) \\
(23 juillet 1933)}
\noindent \par
La terreur blanche et le fascisme couvrent en ce moment presque tout le territoire de l'Europe ; on remarque des signes précurseurs du fascisme aux États-Unis et dans les rares pays européens demeurés démocratiques, plus particulièrement en Suisse, en Belgique, en Espagne, en France. De plus en plus, toute l'espérance des militants réside exclusivement en ce sixième du globe où se trouve, nous dit-on, un État ouvrier, qui représente les intérêts généraux du prolétariat mondial, qui construit le socialisme, et qui, quand il aura réalisé le socialisme sur son territoire, l'étendra aux cinq autres sixièmes du globe.\par
Jamais il n'a été plus urgent pour nous de savoir clairement si cet État ouvrier est une réalité ou une illusion. Dans le premier cas, nous devons tout subordonner à sa défense. Nous devons, s'il est menacé ou attaqué, le défen­dre, non seulement contre notre propre bourgeoisie, si celle-ci lui est hostile, mais même, le cas échéant, unis à notre propre bourgeoisie, si jamais celle-ci, par le jeu des rivalités impérialistes, se trouvait combattre un pays qui mena­cerait l'existence de l'U.R.S.S. Nous devons subordonner toutes nos critiques au souci de ne pas discréditer l'U.R.S.S. auprès du prolétariat, non plus que les organisations qui établissent la liaison entre l'U.R.S.S. et l'avant-garde de la classe ouvrière. Si, au contraire, aucun État au monde ne peut en ce moment se dire le représentant des intérêts historiques du prolétariat mondial, nous devons ne compter que sur nous-mêmes, si faibles que nous soyons ; nous devons, en toutes circonstances, considérer notre propre bourgeoisie comme l'ennemi principal et saboter toute guerre menée par elle, quand même elle serait alliée à l'U.R.S.S. ; nous devons proclamer partout tout ce que nous pensons être la vérité ; et nous devons nous orienter vers un regroupement des forces ouvrières, dans les pays capitalistes, sur une base indépendante par rapport au Parti communiste russe.\par
Notre camarade Prader a envoyé à {\itshape L'École émancipée} une série d'articles concernant le régime intérieur de l'U.R.S.S. Pour moi, je voudrais simplement ici donner quelques documents permettant d'examiner si, dans ses rapports avec les États capitalistes, l'État russe joue le rôle d'un représentant des inté­rêts de classe du prolétariat mondial, si, autrement dit, les rapports entre l'U.R.S.S. et le monde capitaliste doivent être regardés comme une des formes de la lutte des classes.\par
J'attends que l'on me montre, ces deux ou trois dernières années, un acte de solidarité de l'U.R.S.S. avec les prolétariats opprimés. Ce que je sais en revanche, c'est qu'à la conférence économique, Litvinof a proposé des moyens propres à rétablir l'économie capitaliste, comme aurait pu faire un vulgaire social-démocrate ; qu'à Genève il a formulé une définition de l'agresseur, per­mettant ainsi, dans une guerre éventuelle entre États impérialistes, à l'un des blocs impérialistes de se dire en état de légitime défense et d'appeler les communistes à participer à l'union sacrée ; que l'U.R.S.S., dans les clauses du pacte de non-agression franco-russe, a renoncé à protéger les organisations indigènes en lutte contre notre impérialisme et à les abriter sur son territoire ; que Litvinof s'est exprimé, dans de récentes lettres à Paul-Boncour et à Herriot, comme aucun social-démocrate peut-être n'a osé le faire. J'espère en particulier qu'aucun communiste n'a lu sans honte, dans {\itshape L'Humanité}, les lignes où Litvinof félicite Herriot de « sa lutte infatigable pour la paix ».\par
Mais plus instructifs encore sont les textes où la bourgeoisie elle-même examine la nature de ses relations avec l'État soviétique et leur rapport avec la lutte des classes. Je choisirai comme exemple la bourgeoisie du pays même qui semble s'orienter vers la guerre antisoviétique, la bourgeoisie allemande.\par
{\itshape Die Tat}, revue de jeunes économistes qui sympathisent depuis longtemps avec le national-socialisme, ont soutenu von Schleicher, et sont à présent favorables au mouvement hitlérien, sans y être incorporés, écrit :\par
« L'Allemagne et les États-Unis ne représentent pour la Russie, à l'heure actuelle, aucun danger politique, au contraire... Depuis que, même sous Staline, là Russie devient de plus en plus un État national, depuis que le reste du monde liquide de plus en plus le libéralisme international, et que chaque puissance tend à revenir dans les limites de son territoire national, la tension provoquée par le danger du bolchevisme international a disparu. {\itshape La Russie est redevenue une puissance parmi les puissances} \footnote{Ici et plus loin les italiques sont de nous. (Note de S. W.)}... Une collaboration germano-russe est la base de toute la politique extérieure allemande. » (Reproduit par {\itshape Lu} le 9 juin 1933.)\par
La {\itshape Deutsche Allgemeine Zeitung}, journal de l'industrie lourde, qui a toujours demandé le bloc des partis nationaux et représente jusqu’à un certain point l'opposition bourgeoise contre Hitler, a consacré, le 27 mai dernier, un long article au problème des relations germano-russes. Cet article expliquait qu'au lendemain de Versailles le rapprochement germano-russe était pour l'Allemagne un moyen de pression sur les puissances victorieuses ; et que ce moyen est resté inefficace parce qu'un tel rapprochement ne pouvait être pris au sérieux par rapport à une guerre éventuelle. En effet :\par
« L'éventualité d'une lutte sur le même front que les soldats de l'armée rouge devait apparaître comme absurde, parce que c'eût été abandonner nos propres troupes au danger d'un bouleversement bolchevik, danger contre lequel le système de Weimar n'avait pas d'arme efficace.\par
« Ce point de vue, décisif dans la politique des dix dernières années, a à présent perdu de son poids grâce à la révolution nationale en Allemagne. Si absurde que cela puisse sembler au premier abord, précisément le gouverne­ment du relèvement national, qui est en train d'extirper complètement le communisme en Allemagne, mais s'appuie sur de très larges couches de la population, et espère organiser solidement le peuple allemand, pourra peut-être, mieux que les gouvernements précédents, regarder en face l'éventualité d'une alliance germano-russe.\par
« Ainsi seulement peut-on expliquer que le jugement porté par la presse de Moscou sur la situation allemande après la première tempête d'indignation contre les persécutions à l'égard des communistes, s'oriente vers une position relativement moins pure et plus réaliste. »\par
L'article explique ensuite que le souci qui s'exprime à présent dans les documents soviétiques est moins l'effondrement du communisme allemand que le danger d'une guerre d'intervention où l'Allemagne hitlérienne jouerait le rôle d'avant-garde ; que, en face de ce danger, le gouvernement soviétique espère que les conflits d'intérêts entre l'Allemagne et la France sont trop consi­dérables et ont un retentissement trop profond dans la conscience nationale pour permettre un rapprochement ; mais que surtout les Russes comptent sur une issue révolutionnaire de la crise allemande. À vrai dire, au Kremlin, « on a complètement abandonné l'espoir que les Thaelmann et les Neumann pourraient jamais jouer à nouveau un rôle dans la politique allemande », mais on suit d'autant plus attentivement les éléments révolutionnaires qui se trouvent au sein du parti national.\par
Et voici les conclusions de l'article :\par
« On peut comprendre, d'après ces espérances de Moscou, quelle signifi­cation a la consolidation de la situation intérieure pour la liberté des mouvements à l'extérieur. Il est évident que la possibilité pratique d'une collaboration germano-russe, {\itshape bien que, grâce au mouvement hitlérien, elle entre tout à fait sérieusement en ligne de compte}, ne peut être encore réalisée aussi longtemps que Moscou n'aura pas dû complètement abandonner ses espérances d'une bolchévisation de l'Allemagne. Un véritable rapprochement de l'Allemagne et de la Russie, {\itshape allant jusqu'à une alliance défensive et offen­sive effective}, ne sera possible que quand l'Allemagne aura accompli à sa manière {\itshape une forme de vie politique aussi solide et aussi efficace du point de vue de la propagande que ce que la Russie a accompli dans son système}. Alors seulement... la politique allemande pourra, en cas de besoin, utiliser ses bonnes relations avec la Russie comme un facteur décisif de la politique mon­diale. »\par
En face de cet article bourgeois, voici quelques extraits d'un article écrit par un communiste allemand dans la revue d'extrême gauche, {\itshape Die neue Weltbühne}:\par
« Des milliers d'ouvriers et de révolutionnaires allemands sont en prison ou parqués dans les camps de concentration ; beaucoup ont été tués, plusieurs acculés au suicide par des tortures intolérables...\par
« Et l'Union soviétique ?\par
« Les travailleurs du monde entier le savent : ...la situation de l'Union soviétique est difficile... Cependant… avec la patience exercée trop souvent et trop longtemps, qui pèse sur le prolétariat comme une maladie, les ouvriers et intellectuels révolutionnaires du monde entier attendent le geste si naturel de solidarité de la part de la Russie, qui doit apporter un asile aux opprimés et la liberté aux emprisonnés.\par
« Cent vingt jours sont écoulés.\par
« On ne dira pas qu'il est devenu impossible, dans les cadres des négocia­tions diplomatiques germano-russes, de revendiquer la vie et la liberté des militants ouvriers ! Et si même c'était devenu impossible - l'a-t-on seulement essayé ? Et en ce cas pourquoi pas tout haut ? À la face du monde entier ?\par
« Dans les frontières de l'Union soviétique vivent cent quatre-vingts millions d'êtres humains... Et n'y aurait-il pas de place, là, pour quelques milliers à qui on a négligé d'arracher la vie et la liberté au profit du fascisme ? La Russie a à sa disposition des méthodes et des moyens dont ne dispose aucun pays au monde.\par
« À une époque... où la guerre civile battait son plein..., en 1919, la Russie soviétique a demandé et obtenu de la contre-révolution hongroise des centai­nes d'emprisonnés. Et aujourd'hui, après quatorze ans de développement et de construction, ce serait impossible ?...\par
« Que nul n'ose dire que c'est de la sentimentalité, des sentiments pour lesquels il n'y a pas de place parmi les gigantesques échafaudages, machines et blocs de pierres !\par
\par
« Est-ce que les ouvriers révolutionnaires d'Allemagne ont versé leur sang pour des chiffres d'exportation ? Pour des statistiques ?\par
« Dans tous les pays de la terre, les juifs ont reçu leurs coreligionnaires... La France impérialiste a donné asile aux immigrants, la Pologne fasciste les a autorisés à pénétrer chez elle... Est-ce donc que, devant les portes du capitalis­me occidental, devant les palais des millionnaires... le réfugié sans abri aura plus de raison d'espérer que devant les poteaux de frontière rouges de l'Union soviétique ?\par
« Nous ne voulons pas l'admettre... »\par
Si l'on rassemble ces documents, si l'on y ajoute le fait qu'au Congrès antifasciste de Pleyel aucune organisation soviétique n'est intervenue, même par l'envoi d'une déclaration écrite, la conclusion est claire. L'U.R.S.S. n'est plus « la patrie internationale des travailleurs », mais, comme le dit {\itshape die Tat}, « une puissance parmi les puissances ». Les prolétariats opprimés se tournent encore vers elle ; mais c'est en vain. Les capitalistes en revanche ont cessé d'en avoir peur et horreur. Ils voudraient bien au contraire pouvoir « accomplir une forme de vie politique aussi solide en son genre », autrement dit pouvoir réaliser, sans expropriation, bien entendu, une concentration analogue des pouvoirs économiques et politiques. Si une fraction de la bourgeoisie alleman­de veut attaquer l'U.R.S.S., c'est pour assouvir ses appétits impérialistes, et non pas, comme le croient les staliniens et même les trotskystes, pour anéantir un ennemi de classe ; sans quoi il serait inconcevable qu'une autre fraction, politiquement fort proche de la première, songe à une alliance défensive et offensive.\par
La diplomatie de l'État russe doit donc nous inspirer de la défiance, en cas de guerre comme en cas de paix, tout comme la diplomatie des États capitalistes, sinon au même degré. Nous ne devons soutenir l'État russe que dans la mesure où nous le pouvons sans violer les principes généraux de la lutte prolétarienne. Et nous ne devons jamais oublier qu'au nombre des cartes dont dispose la diplomatie secrète de l'U.R.S.S. se trouvent l'Inter nationale Communiste et toutes les organisations qui en dépendent. Les staliniens le nient. Mais il leur reste à prouver que l'Exécutif de l'Internationale, dont le congrès ne s'est pas réuni depuis cinq ans, est soumis à un autre contrôle qu'à celui du Comité Central du parti russe. Ou bien encore que le titre de secré­taire du parti russe - c'est le seul titre que possède Staline -n'équivaut pas en fait, en U.R.S.S., au pouvoir suprême. Ou encore que les partis nationaux sont indépendants du Komintern.\par
Ne fermons pas les yeux. Préparons-nous à ne compter que sur nous-mêmes. Notre puissance est bien petite ; du moins n’abandonnons pas ce peu que nous pouvons faire aux mains de ceux dont les intérêts sont étrangers à l'idéal que nous défendons. Songeons à préserver tout au moins notre honneur.\par
({\itshape L'École émancipée}, 23\textsuperscript{e} année, n° 42, 23 juillet 1933 \footnote{{\itshape L'École émancipée} met en note à la fin de cet article : {\itshape Reçu le 24 juin.}})\par

\subsection[16. Journal d’Espagne, (août 1936)]{16. \\
Journal d’Espagne \\
(août 1936)}
\noindent \par
Port-Bou.\par
{\itshape Barcelone}.\par
Premières impressions de la guerre civile.\par
On croirait difficilement que Barcelone est la capitale d'une région en pleine guerre civile. Quand on a connu Barcelone en temps de paix, et qu'on débarque à la gare, on n'a pas l'impression d'un changement. Les formalités ont eu lieu à Port-Bou ; on sort de la gare de Barcelone comme un touriste quelconque, on déambule le long de ces rues heureuses. Les cafés sont ouverts, quoique moins fréquentés que d'habitude ; les magasins aussi. La monnaie joue toujours le même rôle. S'il n'y avait pas si peu de police et tant de gamins avec des fusils, on ne remarquerait rien du tout. Il faut un certain temps pour se rendre compte que c'est bien la Révolution, et que ces périodes historiques sur lesquelles on lit des livres, qui ont fait rêver depuis l'enfance, 1792, 1871, 1917, on est en train d'en vivre une, ici. Puisse-t-elle avoir des effets plus heureux.\par
Rien n'est changé, effectivement, sauf une petite chose : le pouvoir est au peuple. Les hommes en bleu commandent. C'est à présent une de ces périodes extraordinaires, qui jusqu'ici n'ont pas duré, où ceux qui ont toujours obéi prennent les responsabilités. Cela ne va pas sans inconvénients, c'est sûr. Quand on donne à des gamins de dix-sept ans des fusils chargés au milieu d'une population désarmée...\par
Lérida.\par
Mil. com, reg. C.N.T. - 5 ouvr. bâtim. - com. lib. « pas tout de suite, dans un ou deux mois ».\par
Columna Durruti.\par
Vendredi 14.\par
Samedi 15\par
Conversation avec les paysans de Pina :\par
S'ils sont d'accord pour tout cultiver ensemble ?\par
1\textsuperscript{re} réponse (à plusieurs reprises) : on fera ce que dira le comité.\par
Vieux : oui - à condition qu'on lui donne tout ce qu'il lui faut - qu'il ne soit pas tout le temps embêté, comme maintenant, pour payer charpentier, médecin...\par
Un autre : il faut voir comment ça marchera...\par
S'ils aiment mieux cultiver ensemble que partager ? - Oui (pas très catégorique).\par
Comment ils vivaient ? - Travailler jour et nuit, et manger très mal. La plupart ne savent pas lire. Les enfants vont en place. Une petite de quatorze ans qui travaille depuis deux ans, fait la lessive (ils ont un bon rire en racontant tout ça). Gagnent 20 pes. par mois (une fille de vingt ans), 17, 16... Vont pieds nus.\par
Riches propriétaires de Saragosse.\par
Le curé. - On n'avait rien pour faire l'aumône, mais on donnait des volailles au curé. - Aimé ? - Oui, par beaucoup. - Pourquoi ? Pas de réponse claire.\par
Ceux qui nous parlaient n'avaient jamais été à la messe. (Tout âge...) S'il y avait beaucoup de haine contre les riches ? - Oui, mais encore plus entre pauvres.\par
Si cet état de choses ne peut pas gêner le travail en commun ? - Non, puisqu'il n'y aura plus d'inégalité.\par
Si on travaillera tous pareil ? - Celui qui ne travaillera pas assez devra être forcé. Seuls ceux qui travailleront mangeront.\par
Si la vie des villes vaut mieux que celle des champs ? - Deux fois mieux. Travaillent moins. Mieux habillés, distractions, etc. Ouvriers des villes plus au courant des choses... Un des leurs parti travailler en ville est revenu après trois mois avec des habits neufs.\par
\par
Si on jalouse la ville ? - On ne s'occupe pas...\par
Service militaire : un an. Ils ne pensent qu'à retourner chez eux. - Pour­quoi ? - Mangent mal. Fatigue. Discipline. Coups (si on ripostait, fusillé). Coups avec la main, crosse de fusil, etc. Les riches le font dans d'autres conditions.\par
S'il faut le supprimer ? - Oui, ça vaudrait autant.\par
Ceux qui étaient pour le curé n'ont pas changé d'avis, mais se taisent.\par
Régime : payent rente à propriétaire.\par
Beaucoup chassés parce qu'incapables de payer rente. Doivent se faire ouvriers agricoles à deux pesetas par jour.\par
Sentiment d'infériorité assez vif.\par
Dimanche 16.\par
Durruti à Pina.\par
(Garde civile - gardes d'assaut - paysans.) Sevillan.\par
Discours de Durruti aux paysans : Suis un travailleur. Quand tout sera fini, j'irai travailler à l'usine.\par
Durruti à Osera.\par
Ordres : ne pas manger ni coucher chez les paysans. Obéir au « technicien militaire ». Discussion violente.\par
Organisation : délégués élus. Sans compétence. Sans autorité. Ne font pas respecter l'autorité du technicien militaire.\par
Paysan se plaint au type d'Oran (Marquet) que les sentinelles s'endorment.\par
Retour au Q. G.\par
Camarade échappé de Saragosse. Prop. d'expédition. Sevillan. Celui qui veut rester avec son ami. Celui qui veut rendre ses armes.\par
300 hommes non armés envoyés de Lérida. Cinq canons « prêtés » à la colonne de Huesca (i. e. envoyés de Lérida avec consentement de Durruti). Garcia Oliver parti en avion à Valence. Officier disparu. Coordination télé­graphistes-téléphonistes.\par
S.  \footnote{Secours.} annoncés : 2 000 h. armés, esc. de cavalerie, 2 [ ? ] batteries de 15, a tanks de montagne.\par
Conversation téléph. Durruti-Santillan. Prise Quinto coûterait 1 200 h. (?) sans canons. Avec canons on peut aller aux portes de Saragosse.\par
Très énergique : On peut bombarder Saragosse.\par
[Vx type : « Si, Señor... »]\par
Lundi 17.\par
On déménage le Q. G. à la maison de paysans en face de laquelle il y a tant de blé (un drôle de déménagement !). Dans la matinée, voiture pour Pina. Les deux petits fiancés qui se bécotent au volant. Trouvé le groupe installé dans l'école. Magnifique. (Manuels patriotiques...) (L'hôpital est là aussi.) On mange chez les mêmes paysans (au 18). On me donne un fusil : beau petit mousqueton. Dans l'après-midi, on bombarde vaguement. J'écris à Boris : « Pas encore entendu un coup de fusil. » (Vrai, sauf exercice de tir...) Aussitôt boum !... fracas terrible. « L'aviation bombarde. » On sort avec les fusils. Ordre : dans le maïs. Couchés. Je me couche en pleine boue pour tirer en l'air. Au bout de quelques minutes on se lève. Avions bien trop haut pour tirer. Salve de balles de la moitié des Espagnols. Un tire horizontalement vers le fleuve. (Quelques-uns tirent au revolver ?) On va trouver bombe. Minuscule. Dégâts dans ½ m de rayon. N'ai pas été émue du tout.\par
Encore des paysans oisifs sur la place, mais beaucoup moins.\par
Louis Berthomieux (délégué) : « On passe le fleuve. » Il s'agit d'aller brûler trois cadavres ennemis. On passe en barque (un quart d'h. de discus­sions...) On cherche. - Un cadavre en bleu, dévoré, horrible. On le brûle. Les autres cherchent ce qui reste. Nous, on se repose. On parle de coup de main. On laisse le gros de la troupe retraverser. Puis on décide (?) de remettre le coup de main au lendemain. On revient vers le fleuve, sans se cacher beau­coup. On voit une maison. Pascual (du comité de guerre) : « On va chercher des melons. » (Très sérieusement !) On va par la brousse. Chaleur, un peu d'angoisse. Je trouve ça idiot. Tout à coup, je comprends qu'on va en expédi­tion (sur la maison). Là, suis {\itshape très} émue (j'ignore l'utilité de la chose, et je sais que si on est pris on est fusillé). On se partage en deux groupes. Délégué, Ridel et trois Allemands vont à plat ventre jusqu'à la maison. Nous, dans les fossés (après coup le délégué nous engueule : on aurait dû aller jusqu'à la maison). On attend. On entend parler... Tension épuisante. On voit les copains revenir sans se cacher, on les rejoint, on repasse le fleuve tranquillement. La fausse manœuvre aurait pu leur coûter la vie. Pascual est le responsable. (Carpentier, Giral avec nous.)\par
On couche dans la paille (deux bottes dans un coin, et bonne couverture). L'infirmier qui veut faire éteindre la lumière se fait engueuler.\par
Cette expédition est la première et la seule fois que j'aie eu peur pendant ce séjour à Pina.\par
Mardi 18.\par
Des tas de projets pour l'autre côté du fleuve. Vers la fin de la matinée, on décide d'y passer au milieu de la nuit, nous le « groupe », pour tenir quelques jours jusqu'à l'arrivée de la colonne de Sastano. La journée se passe en démarches. Question angoissante : celle des fusils mitrailleurs. Le comité de guerre de Pina les refuse. En fin de compte, grâce au colonel italien chef de la « Banda Negra », on se débrouille pour en avoir un - puis deux. On ne les essaie pas.\par
C'est le colonel qui nous a proposé le premier d'aller là-bas, mais en fin de compte mission officielle du comité de guerre de Pina.\par
Volontaires, bien sûr. La veille au soir, Berthomieux nous a réunis au 18, demandé notre avis. Silence complet. Il insiste pour qu'on dise ce qu'on pense. Encore un silence. Puis Ridel : « Ben quoi, on est tous d'accord. » Et c'est tout.\par
On se couche. Infirmier qui veut éteindre... Je couche habillée. Ne dors guère. Lever à 2 h. Mon sac est fait. Émotion : lunettes. Partage des charges (moi : carte, une bassine). Ordre. On chemine sans parler. Un peu émue quand même. Traversée en deux fois. Pour nous Louis s'énerve, crie (s'ils sont là ...). On débarque. On attend. Le jour apparaît un peu. L'Allemand va faire le jus. Louis découvre la butte, y fait porter les affaires, m'y envoie. J'y reste un peu, puis vais prendre le jus à mon tour. Louis a installé les gardes. On travaille tout de suite à aménager la cuisine et la butte, barricader pour pas être vus. Pendant ce temps, les autres vont à la maison. Y trouvent une famille, et un petit gars de 17 ans (beau !). Renseignements : on nous a vus, à l'autre recon­naissance. Ils avaient gardé la rive. Retiré les gardes à notre arrivée. 112 h  \footnote{Hommes.}. Le lieutenant a juré de nous avoir. Reviennent. Je traduis ces renseigne­ments aux Allemands. Ils demandent : « On repasse le fleuve ? - Non, on reste, bien sûr. » (On va à Pina téléphoner à Durruti ?) Ordre : retourner rame­ner la famille de paysans. (Pendant tout ça, le copain allemand promu cuisinier râle parce qu'il n'y a ni sel, ni huile, ni légumes.) Berthomieux, furieux (c'est dangereux de retourner encore une fois à la maison), rassemble expédition. Me dit : « Toi, à la cuisine ! » Je n'ose pas protester. D'ailleurs, cette expédi­tion ne me va qu'à moitié... Je les regarde partir avec angoisse... (au fond, d'ailleurs, je suis presque autant en danger). On prend nos fusils, on attend. Bientôt l'Allemand propose d'aller au petit retranchement sous l'arbre occupé par Ridel et Carpentier (ils sont de l'expédition, bien entendu). On s'y couche, à l'ombre, avec les fusils (non armés). On attend. De temps à autre, l'Allemand laisse échapper un soupir. Il a peur, visiblement. Moi pas. Mais comme tout, autour de moi, existe intensément ! Guerre sans prisonniers. Si on est pris, on est fusillé. Les copains reviennent. Un paysan, son fils et le petit gars... Fontana lève le poing en regardant les garçons. Le fils répond visiblement à contrecœur. Contrainte cruelle... Le paysan retourne chercher sa famille. On revient à ses places respectives. Reconnaissance aérienne. Se planquer. Louis gueule contre les imprudences. Je m'étends sur le dos, je regarde les feuilles, le ciel bleu. Jour très beau. S'ils me prennent, ils me tueront... Mais c'est mérité. Les nôtres ont versé assez de sang. Suis moralement complice. Calme complet. On se regroupe - puis ça recommence. Me planque dans la butte. On bombarde. Sors pour aller vers fusil mitrailleur. Louis dit : « Faut pas avoir peur (!). » Me fait aller avec l'Allemand dans la cuisine, nos fusils à l'épaule. On attend. Enfin vient la famille du paysan (trois filles, un garçon de huit ans), tous épouvantés (on bombarde pas mal). S'apprivoisent un peu. Très craintifs. Préoccupés du bétail laissé à la ferme (on finira par le leur ramener à Pina). Évidemment pas sympathisants.\par
[Sitges.]\par
5 sept.\par
Retour brusque des miliciens de Mayorque. Rien que pour Sitges, dix morts. (On ne le savait pas.) Expédition punitive, la nuit, en auto, pour tuer dix « fascistes ». On en fait autant la nuit suivante. Des gens s'enfuient (le boulanger qui fournit l'hôtel...).\par
Histoires de C. : Lérida. Colonne de Garcia Oliver, {\itshape malgré} la C.N.T. de Lérida, brûle la cathédrale (pleine de valeurs, d'or, de trésors artistiques) et massacre vingt personnes dans la prison, où ils pénètrent de force.\par
Infirmier de la colonne du P.O.U.M. (étudiant en médecine). Ramène en auto à Lérida un blessé atteint de gangrène à la jambe. Prétend (faussement) qu'il n'y a pas de place à Lérida, et donne ordre au chauffeur de continuer sur [?]. À six kilomètres de Lérida, panne. L'infirmier retourne à Lérida en emportant la « documentation », abandonnant l'auto sur la route. Chauffeur italien, ne sait pas l'espagnol. [Espagnols] sur le point de lui faire un mauvais parti, quand par hasard un camion du P.O.U.M. passe. Infirmier, huit jours de prison.\par
Avion de bombardement abandonné par avion de chasse qui l'accom­pagnait (mitrailleuses enrayées)...\par
Villafranca (près de Sitges).\par
Berthollet m'avait dit qu'il y régnait le communisme libertaire. En fait, on n'a pas supprimé la monnaie, même un jour. Ni collectivisé les champs. Les paysans (rabassaires) ne paient pas la rente (...), un point c'est tout. On collectivisera d'ici l'an prochain (?). Un grand magasin dont le patron a été fusillé. Collectivisé ? « On est en train. » Des tas de petites usines (huit à dix ouvriers), mécanique, etc. Patrons y travaillent comme ouvriers. Collectivisées ou coopératives (différence ?). Le Comité du Front populaire (C.N.T., P.O.U.M., Esquerra) leur a commandé et payé un camion blindé. Ressources : impôt de guerre, comptes en banque des réactionnaires. « On n'a pas tué les réactionnaires, on les fait payer. » La Esquerra et la Lliga avaient presque la même force. « Qu'est-ce qu'on a fait aux militants de la Lliga ? - Rien, ils ont adhéré à la C.N.T. ( ! ! !) (C'étaient les petits patrons devenus ouvriers.) On a fait une trentaine d'exécutions : le curé et de grands propriétaires. « Fascistes ? - Non, fascistes de fait », i.e. vaches.\par
Carpentier, Ridel (Siétamo).\par
Roanna. C'est lui qui a tué B. (bon travail !). 50 h. à Lérida (le premier jour) (?). À Siétamo, chauffeur de tank arrive avec douze heures de retard, qui ne voulait pas avancer et par la faute de qui un copain a été blessé.\par
Santillan voulait tuer les soldats prisonniers. Louis lui dit que s'il les fusille, on le fusillera après. Il se tient tranquille.\par
Avant, encore à Pina - des Espagnols du groupe international, ont participé à une exécution à Pina (le notaire, revenu). On parle de les expulser du grou­pe. Louis furieux. On décide que le groupe ne participera pas à des expéditions.\par
La « Maritima ».\par
9 délégués. 4 permanents. 5 font demi-journée d'ouvriers. Salaires d'ouvriers.\par
17 à 19 pesetas. 40 heures + 16 heures gratuites. Contribution volontaire de 12 pesetas. C.N.T. à 98 \%\par
Bombes, etc. - Locomotives.\par
Capital espagnol et allemand. Directeur a emporté le fric (12 millions).\par
Dessins d'art, trouvés dans les archives. Ouvriers ayant travaillé dans fabriques de munitions en France.\par
Primes supprimées. « On travaille plus. »\par
Hispano.\par
(Fusillé directeur, 4 ouvriers.)\par
Conditions morales très mauvaises.\par
Comité exécutif de 8 membres (6 ouvriers, 2 des bureaux) plus un prési­dent (bureaux). Ces 8 se sont emparés de l'usine, ont fait venir les ouvriers, se sont nommés eux-mêmes. Se sont fait plébisciter.\par
Chefs subalternes conservés. Certains changés cette semaine seulement (incapacité).\par
Comité de techniciens. Au début, 3 chefs d'ateliers. Depuis plus nom­breux. Suggestions reçues par la voie hiérarchique.\par
Cars blindés improvisés. Depuis, perfectionnés peu à peu.\par
(Heures de travail : de 9 à 12 h., de 2 à 5 h.)\par
Discipline - renvois de mauvais éléments (mauvais camarades).\par
Admonestations aux ouvriers indisciplinés. Amendes pour retards. Pas de malfaçons.\par
Primes supprimées. « On travaille plus. »\par
Mines potasse.\par
\par
Travaillent pas, mais payés. Pourquoi ne travaillent pas ? À cause du trust de la potasse, par lequel il faut passer.\par

\begin{center}
\end{center}
\subsection[17. Fragment, (1936 ?)]{17. \\
Fragment \\
(1936 ?)}
\noindent \par
Que se passe-t-il en Espagne ? Chacun, là-dessus, a son mot à dire, ses histoires à raconter, un jugement à prononcer. C'est la mode, actuellement, d'aller faire un tour là-bas, voir un bout de révolution et de guerre civile, et revenir avec des articles plein sa plume. On ne peut plus ouvrir un journal ou une revue sans y trouver des récits d'événements d'Espagne. Comment tout cela ne serait-il pas superficiel ? Tout d'abord une transformation sociale ne peut être correctement appréciée qu'en fonction de ce qu'elle apporte à la vie quotidienne de chacun de ceux qui composent le peuple. Il n'est pas facile de pénétrer dans cette vie quotidienne. D'ailleurs chaque jour amène du nouveau. Et puis la contrainte et la spontanéité, la nécessité et l'idéal se mêlent de manière à apporter une confusion inextricable non seulement dans les faits, mais encore dans la conscience même des acteurs et spectateurs du drame. C'est même là le caractère essentiel et peut-être le plus grand mal de la guerre civile. C'est aussi la première conclusion qu'on peut tirer d'un examen rapide des événements espagnols, et ce qu'on sait de la révolution russe ne le confirme que trop. Il n'est pas vrai que la révolution corresponde automati­quement à une conscience plus haute, plus intense et plus claire du problème social. C'est le contraire qui est vrai, du moins quand la révolution prend la forme de la guerre civile. Dans la tourmente de la guerre civile, les principes perdent toute commune mesure avec les réalités, toute espèce de critérium en fonction duquel on puisse juger les actes et les institutions disparaît, et la transformation sociale est livrée au hasard. Comment pouvoir rapporter quelque chose de cohérent, après un court séjour et des observations frag­mentaires ? Tout au plus si on peut exprimer quelques impressions, tirer au clair quelques leçons.\par

\begin{center}
\end{center}
\subsection[18. Réflexions pour déplaire, (1936 ?)]{18. \\
Réflexions pour déplaire \\
(1936 ?)}
\noindent \par
Je vais, je le sais, choquer, scandaliser beaucoup de bons camarades. Mais quand on se réclame de la liberté, on doit avoir le courage de dire ce qu'on pense, même si on doit déplaire.\par
Tous nous suivons jour par jour, anxieusement, avec angoisse, la lutte qui se déroule de l'autre côté des Pyrénées. Nous tâchons d'aider les nôtres. Mais cela n'empêche pas, ne dispense pas de tirer les leçons d'une expérience que tant d'ouvriers, de paysans paient là-bas de leur sang.\par
On a déjà eu en Europe une expérience de ce genre, payée de beaucoup de sang elle aussi. C'est l'expérience russe. Lénine, là-bas, avait publiquement revendiqué un État où il n'y aurait ni armée, ni police, ni bureaucratie distinc­tes de la population. Une fois au pouvoir, lui et les siens se sont mis, à travers une longue et douloureuse guerre civile, à construire la machine bureau­cratique, militaire et policière la plus lourde qui ait jamais pesé sur un malheureux peuple.\par
Lénine était le chef d'un parti politique, d'une machine à prendre et à exercer le pouvoir. On a pu mettre en doute sa bonne foi et celle de ses compagnons ; on a pu du moins penser qu'il y avait contradiction entre les buts définis par Lénine et la nature d'un parti politique. Mais on ne saurait mettre en doute la bonne foi de nos camarades libertaires de Catalogne. Cependant que voyons-nous là-bas ? Là aussi, hélas, nous voyons se produire des formes de contrainte, des cas d'inhumanité directement contraires à l'idéal libertaire et humanitaire des anarchistes. Les nécessités, l'atmosphère de la guerre civile l'emportent sur les aspirations que l'on cherche à défendre au moyen de la guerre civile.\par
Nous haïssons, ici, la contrainte militaire, la contrainte policière, la con­trainte dans le travail, le mensonge répandu par la presse, la T.S.F., tous les moyens de diffusion. Nous haïssons les différenciations sociales, l'arbitraire, la cruauté.\par
Eh bien ! il y a, là-bas, contrainte militaire. Malgré l'afflux de volontaires, on a décrété la mobilisation. Le conseil de défense de la Généralité, où nos camarades de la F.A.I. ont quelques-uns des postes dirigeants, vient de décré­ter l'application aux milices de l'ancien code militaire.\par
Il y a contrainte dans le travail. Le conseil de la Généralité, où nos camarades détiennent les ministères économiques, vient de décréter l'obli­gation, pour les ouvriers, d'effectuer autant d'heures supplémentaires non payées qu'il serait jugé nécessaire. Un autre décret prévoit que les ouvriers qui ne produiraient pas à une cadence suffisante seront considérés comme factieux et traités comme tels ; ce qui signifie, tout simplement, l'application de la peine de mort dans la production industrielle.\par
Quant à la contrainte policière, la police d'avant le 19 juillet a perdu presque tout son pouvoir. En revanche, pendant les trois premiers mois de la guerre civile, les comités d'investigation, les militants responsables, et, trop souvent, des individus irresponsables ont fusillé sans le moindre simulacre de jugement, et par suite sans aucune possibilité de contrôle syndical ou autre. C'est seulement il y a quelques jours qu'on a institué des tribunaux populaires destinés à juger les factieux ou présumés factieux. Il est trop tôt encore pour savoir quel effet aura cette réforme.\par
Le mensonge organisé existe, lui aussi, depuis le 19 juillet...\par

\begin{center}
\end{center}
\subsection[19. Lettre à Georges Bernanos, (1938 ?)]{19. \\
Lettre à Georges Bernanos \\
(1938 ?)}
\noindent \par
Monsieur,\par
Quelque ridicule qu'il y ait à écrire à un écrivain, qui est toujours, par la nature de son métier, inondé de lettres, je ne puis m'empêcher de le faire après avoir lu {\itshape Les Grands cimetières sous la lune}. Non que ce soit la première fois qu'un livre de vous me touche ; le {\itshape Journal d'un curé de campagne} est à mes yeux le plus beau, du moins de ceux que j'ai lus, et véritablement un grand livre. Mais si j'ai pu aimer d'autres de vos livres, je n'avais aucune raison de vous importuner en vous l'écrivant. Pour le dernier, c'est autre chose ; j'ai eu une expérience qui répond à la vôtre, quoique bien plus brève, moins pro­fonde, située ailleurs et éprouvée, en apparence - en apparence seulement-, dans un tout autre esprit.\par
Je ne suis pas catholique, bien que - ce que je vais dire doit sans doute sembler présomptueux à tout catholique, de la part d'un non-catholique, mais je ne puis m'exprimer autrement - bien que rien de catholique, rien de chrétien ne m'ait jamais paru étranger. Je me suis dit parfois que si seulement on affichait aux portes des églises que l'entrée est interdite à quiconque jouit d'un revenu supérieur à telle ou telle somme, peu élevée, je me convertirais aussi­tôt. Depuis l'enfance, mes sympathies se sont tournées vers les groupements qui se réclamaient des couches méprisées de la hiérarchie sociale, jusqu'à ce que j'aie pris conscience que ces groupements sont de nature à décourager toutes les sympathies. Le dernier qui m'ait inspiré quelque confiance, c'était la C.N.T. espagnole. J'avais un peu voyagé en Espagne - assez peu - avant la guerre civile, mais assez pour ressentir l'amour qu'il est difficile de ne pas éprouver envers ce peuple ; j'avais vu dans le mouvement anarchiste l'expres­sion naturelle de ses grandeurs et de ses tares, de ses aspirations les plus et les moins légitimes. La C.N.T., la F.A.I. étaient un mélange étonnant, où on admettait n'importe qui, et où, par suite, se coudoyaient l'immoralité, le cynisme, le fanatisme, la cruauté, mais aussi l'amour, l'esprit de fraternité, et surtout la revendication de l'honneur si belle chez des hommes humiliés ; il me semblait que ceux qui venaient là animés par un idéal l'emportaient sur ceux que poussait le goût de la violence et du désordre. En juillet 1936, j'étais à Paris. Je n'aime pas la guerre ; mais ce qui m'a toujours fait le plus horreur dans la guerre, c'est la situation de ceux qui se trouvent à l'arrière. Quand j'ai compris que, malgré mes efforts, je ne pouvais m'empêcher de participer moralement à cette guerre, c'est-à-dire de souhaiter tous les jours, toutes les heures, la victoire des uns, la défaite des autres, je me suis dit que Paris était pour moi l'arrière, et j'ai pris le train pour Barcelone dans l'intention de m'engager. C'était au début d'août 1936.\par
Un accident m'a fait abréger par force mon séjour en Espagne. J'ai été quelques jours à Barcelone ; puis en pleine campagne aragonaise, au bord de l'Èbre, à une quinzaine de kilomètres de Saragosse, à l'endroit même où récemment les troupes de Yaguë ont passé l'Èbre ; puis dans le palace de Sitgès transformé en hôpital ; puis de nouveau à Barcelone ; en tout à peu près deux mois. J'ai quitté l'Espagne malgré moi et avec l'intention d'y retourner ; par la suite, c'est volontairement que je n'en ai rien fait. Je ne sentais plus aucune nécessité intérieure de participer à une guerre qui n'était plus, comme elle m'avait paru être au début, une guerre de paysans affamés contre les propriétaires terriens et un clergé complice des propriétaires, mais une guerre entre la Russie, l'Allemagne et l'Italie.\par
J'ai reconnu cette odeur de guerre civile, de sang et de terreur que dégage votre livre ; je l'avais respirée. Je n'ai rien vu ni entendu, je dois le dire, qui atteigne tout à fait l'ignominie de certaines des histoires que vous racontez, ces meurtres de vieux paysans, ces ballilas faisant courir des vieillards à coups de matraques. Ce que j'ai entendu suffisait pourtant. J'ai failli assister à l'exécu­tion d'un prêtre ; pendant les minutes d'attente, je me demandais si j'allais regarder simplement, ou me faire fusiller moi-même en essayant d'intervenir ; je ne sais pas encore ce que j'aurais fait si un hasard heureux n'avait empêché l'exécution.\par
Combien d'histoires se pressent sous ma plume... Mais ce serait trop long ; et à quoi bon ? Une seule suffira. J'étais à Sitgès quand sont revenus, vaincus, les miliciens de l'expédition de Majorque. Ils avaient été décimés. Sur quarante jeunes garçons partis de Sitgès, neuf étaient morts. On ne le sut qu'au retour des trente et un autres. La nuit même qui suivit, on fit neuf expéditions punitives, on tua neuf fascistes ou soi-disant tels, dans cette petite ville où, en juillet, il ne s'était rien passé. Parmi ces neuf, un boulanger d'une trentaine d'années, dont le crime était, m'a-t-on dit, d'avoir appartenu à la milice des « somaten » ; son vieux père, dont il était le seul enfant et le seul soutien, devint fou. Une autre encore : en Aragon, un petit groupe international de vingt-deux miliciens de tous pays prit, après un léger engagement, un jeune garçon de quinze ans, qui combattait comme phalangiste. Aussitôt pris, tout tremblant d'avoir vu tuer ses camarades à ses côtés, il dit qu'on l'avait enrôlé de force. On le fouilla, on trouva sur lui une médaille de la Vierge et une carte de phalangiste ; on l'envoya à Durruti, chef de la colonne, qui, après lui avoir exposé pendant une heure les beautés de l'idéal anarchiste, lui donna le choix entre mourir et s'enrôler immédiatement dans les rangs de ceux qui l'avaient fait prisonnier, contre ses camarades de la veille. Durruti donna à l'enfant vingt-quatre heures de réflexion ; au bout de vingt-quatre heures, l'enfant dit non et fut fusillé. Durruti était pourtant à certains égards un homme admirable. La mort de ce petit héros n'a jamais cessé de me peser sur la conscience, bien que je ne l'aie apprise qu'après coup. Ceci encore : dans un village que rouges et blancs avaient pris, perdu, repris, reperdu je ne sais combien de fois, les miliciens rouges, l'ayant repris définitivement, trouvèrent dans les caves une poignée d'êtres hagards, terrifiés et affamés, parmi lesquels trois ou quatre jeunes hommes. Ils raisonnèrent ainsi : si ces jeunes hommes, au lieu d'aller avec nous la dernière fois que nous nous sommes retirés, sont restés et ont attendu les fascistes, c'est qu'ils sont fascistes. Ils les fusillèrent donc immé­diatement, puis donnèrent à manger aux autres et se crurent très humains. Une dernière histoire, celle-ci de l'arrière : deux anarchistes me racontèrent une fois comment, avec des camarades, ils avaient pris deux prêtres ; on tua l'un sur place, en présence de l'autre, d'un coup de revolver, puis on dit à l'autre qu'il pouvait s'en aller. Quand il fut à vingt pas, on l'abattit. Celui qui me racontait l'histoire était très étonné de ne pas me voir rire.\par
À Barcelone, on tuait en moyenne, sous forme d'expéditions punitives, une cinquantaine d'hommes par nuit. C'était proportionnellement beaucoup moins qu'à Majorque, puisque Barcelone est une ville de près d'un million d'habi­tants ; d'ailleurs il s'y était déroulé pendant trois jours une bataille de rues meurtrière. Mais les chiffres ne sont peut-être pas l'essentiel en pareille matière. L'essentiel, c'est l'attitude à l'égard du meurtre. Je n'ai jamais vu, ni parmi les Espagnols, ni même parmi les Français venus soit pour se battre, soit pour se promener - ces derniers le plus souvent des intellectuels ternes et inoffensifs - je n'ai jamais vu personne exprimer même dans l'intimité de la répulsion, du dégoût ou seulement de la désapprobation à l'égard du sang inutilement versé. Vous parlez de la peur. Oui, la peur a eu une part dans ces tueries ; mais là où j'étais, je ne lui ai pas vu la part que vous lui attribuez. Des hommes apparemment courageux - il en est un au moins dont j'ai de mes yeux constaté le courage - au milieu d'un repas plein de camaraderie, racontaient avec un bon sourire fraternel combien ils avaient tué de prêtres ou de « fascis­tes » - terme très large. J'ai eu le sentiment, pour moi, que lorsque les autorités temporelles et spirituelles ont mis une catégorie d'êtres humains en dehors de ceux dont la vie a un prix, il n'est rien de plus naturel à l'homme que de tuer. Quand on sait qu'il est possible de tuer sans risquer ni châtiment ni blâme, on tue ; ou du moins on entoure de sourires encourageants ceux qui tuent. Si par hasard on éprouve d'abord un peu de dégoût, on le tait et bientôt on l'étouffe de peur de paraître manquer de virilité. Il y a là un entraînement, une ivresse à laquelle il est impossible de résister sans une force d'âme qu'il me faut bien croire exceptionnelle, puisque je ne l'ai rencontrée nulle part. J'ai rencontré en revanche des Français paisibles, que jusque-là je ne méprisais pas, qui n'auraient pas eu l'idée d'aller eux-mêmes tuer, mais qui baignaient dans cette atmosphère imprégnée de sang avec un visible plaisir. Pour ceux-là je ne pourrai jamais avoir à l'avenir aucune estime.\par
Une telle atmosphère efface aussitôt le but même de la lutte. Car on ne peut formuler le but qu'en le ramenant au bien public, au bien des hommes - et les hommes sont de nulle valeur. Dans un pays où les pauvres sont, en très grande majorité, des paysans, le mieux-être des paysans doit être un but essentiel pour tout groupement d'extrême-gauche ; et cette guerre fut peut-être avant tout, au début, une guerre pour et contre le partage des terres. Eh bien, ces misérables et magnifiques paysans d'Aragon, restés si fiers sous les humiliations, n'étaient même pas pour les miliciens un « objet de curiosité. Sans insolences, sans injures, sans brutalité - du moins je n'ai rien vu de tel, et je sais que vol et viol, dans les colonnes anarchistes, étaient passibles de la peine de mort - un abîme séparait les hommes armés de la population désarmée, un abîme tout à fait semblable à celui qui sépare les pauvres et les riches. Cela se sentait à l'attitude toujours un peu humble, soumise, craintive des uns, à l'aisance, la désinvolture, la condescendance des autres.\par
On part en volontaire, avec des idées de sacrifice, et on tombe dans une guerre qui ressemble à une guerre de mercenaires, avec beaucoup de cruautés en plus et le sens des égards dus à l'ennemi en moins.\par
Je pourrais prolonger indéfiniment de telles réflexions, mais il faut se limiter. Depuis que j'ai été en Espagne, que j'entends, que je lis toutes sortes de considérations sur l'Espagne, je ne puis citer personne, hors vous seul, qui, à ma connaissance, ait baigné dans l'atmosphère de la guerre espagnole et y ait résisté. Vous êtes royaliste, disciple de Drumont - que m'importe ? Vous m'êtes plus proche, sans comparaison, que mes camarades des milices d'Aragon - ces camarades que, pourtant, j'aimais.\par
Ce que vous dites du nationalisme, de la guerre, de la politique extérieure française après la guerre m'est également allé au cœur. J'avais dix ans lors du traité de Versailles. Jusque-là j'avais été patriote avec toute l'exaltation des enfants en période de guerre. La volonté d'humilier l'ennemi vaincu, qui déborda partout à ce moment (et dans les années qui suivirent) d'une manière si répugnante, me guérit une fois pour toutes de ce patriotisme naïf. Les humiliations infligées par mon pays me sont plus douloureuses que celles qu'il peut subir.\par
Je crains de vous avoir importuné par une lettre aussi longue. Il ne me reste qu'à vous exprimer ma vive admiration.\par
S. WEIL.\par
\par
M\textsuperscript{lle} Simone Weil, 3, rue Auguste-Comte, Paris (VI\textsuperscript{e}).\par
P. -S. - C'est machinalement que je vous ai mis mon adresse. Car, d'abord, je pense que vous devez avoir mieux à faire que de répondre aux lettres. Et puis je vais passer un ou deux mois en Italie, où une lettre de vous ne me suivrait peut-être pas sans être arrêtée au passage.\par
Fin de la première partie.
\section[Deuxième partie, Politique]{Deuxième partie, \\
Politique}\renewcommand{\leftmark}{Deuxième partie, \\
Politique}

\noindent \par
\par
\subsection[I. Guerre et paix]{I \\
Guerre et paix}
\subsubsection[1. Réflexion sur la guerre, (novembre 1933)]{1. \\
Réflexion sur la guerre \\
(novembre 1933)}
\noindent \par
La situation actuelle et l'état d'esprit qu'elle suscite ramènent une fois de plus à l'ordre du jour le problème de la guerre. On vit présentement dans l'attente perpétuelle d'une guerre ; le danger est peut-être imaginaire, mais le sentiment du danger existe, et en constitue un facteur non négligeable. Or, on ne peut constater aucune réaction si ce n'est la panique, moins panique des courages devant la menace du massacre que panique des esprits devant les problèmes qu'elle pose. Nulle part le désarroi n'est plus sensible que dans le mouvement ouvrier. Nous risquons, si nous ne faisons pas un sérieux effort d'analyse, qu'un jour proche ou lointain la guerre nous trouve impuissants, non seulement à agir, mais même à juger. Et tout d'abord il faut faire le bilan des traditions sur lesquelles nous avons jusqu'ici vécu plus ou moins con­sciemment.\par
Jusqu'à la période qui a suivi la dernière guerre, le mouvement révolution­naire, sous ses diverses formes, n'avait rien de commun avec le pacifisme. Les idées révolutionnaires sur la guerre et la paix se sont toujours inspirées des souvenirs de ces années 1792-93-94 qui furent le berceau de tout le mouvement révolutionnaire du XIX\textsuperscript{e} siècle. La guerre de 1792 apparaissait, en contradiction absolue avec la vérité historique, comme un élan victorieux qui, tout en dressant le peuple français contre les tyrans étrangers, aurait du même coup brisé la domination de la Cour et de la grande bourgeoisie pour porter au pouvoir les représentants des masses laborieuses. De ce souvenir légendaire, perpétué par le chant de la {\itshape Marseillaise}, naquit la conception de la guerre révolutionnaire, défensive et offensive, comme étant non seulement une forme légitime, mais une des formes les plus glorieuses de la lutte des masses tra­vailleuses dressées contre les oppresseurs. Ce fut là une conception commune à tous les marxistes et à presque tous les révolutionnaires jusqu'à ces quinze dernières années. En revanche, sur l'appréciation des autres guerres, la tradi­tion socialiste nous fournit non pas une conception, mais plusieurs, contra­dictoires, et qui n'ont pourtant jamais été opposées clairement les unes aux autres.\par
Dans la première moitié du XIX\textsuperscript{e} siècle, la guerre semble avoir eu par elle-même un certain prestige aux yeux des révolutionnaires qui, en France par exemple, reprochaient vivement à Louis-Philippe sa politique de paix ; Proud­hon écrivait alors un éloge éloquent de la guerre ; et l'on rêvait de guerres libératrices pour les peuples opprimés tout autant que d'insurrections. La guerre, de 1870 força pour la première fois les organisations proléta­riennes, c'est-à-dire, en l'occurrence, l'Internationale, à prendre position d'une manière concrète sur la question de la guerre ; et l'Internationale, par la plume de Marx, invita les ouvriers des deux pays en lutte à s'opposer à toute tentative de conquête, mais à prendre part résolument à la défense de leur pays contre l'attaque de l'adversaire.\par
C'est au nom d'une autre conception qu'Engels, en 1892, évoquant avec éloquence les souvenirs de la guerre qui avait éclaté cent ans auparavant, invitait les social-démocrates allemands à prendre part de toutes leurs forces, le cas échéant, à une guerre qui eût dressé contre l'Allemagne la France alliée à la Russie. Il ne s'agissait plus de défense ou d'attaque, mais de préserver, par l'offensive ou la défensive, le pays où le mouvement ouvrier se trouve être le plus puissant et d'écraser le pays le plus réactionnaire. Autrement dit, selon cette conception qui fut également celle de Plekhanov, de Mehring et d'autres, il faut, pour juger un conflit, chercher quelle issue serait la plus favorable au prolétariat international et prendre parti en conséquence.\par
À cette conception s'en oppose directement une autre, qui fut celle des bolcheviks et de Spartacus, et selon laquelle, dans toute guerre, à l'exception des guerres nationales ou révolutionnaires selon Lénine, à l'exception des guerres révolutionnaires seulement selon Rosa Luxembourg, le prolétariat doit souhaiter que son propre pays soit vaincu et en saboter la lutte. Cette concep­tion, fondée sur la notion du caractère impérialiste par lequel toute guerre, sauf les exceptions rappelées ci-dessus, peut être comparée à une querelle de brigands se disputent un butin, ne va pas sans de sérieuses difficultés ; car elle semble briser l'unité d'action du prolétariat international en engageant les ouvriers de chaque pays, qui doivent travailler à la défaite de leur propre pays, à favoriser par là même la victoire de l'impérialisme ennemi, victoire que d'autres ouvriers doivent s'efforcer d'empêcher. La célèbre formule de Liebknecht : « Notre principal ennemi est dans notre propre pays » fait claire­ment apparaître cette difficulté en assignant aux diverses fractions nationales du prolétariat un ennemi différent, et en les opposant ainsi, du moins en apparence, les unes aux autres.\par
On voit que la tradition marxiste ne présente, en ce qui concerne la guerre, ni unité ni clarté. Un point du moins était commun à toutes les théories, à savoir le refus catégorique de condamner la guerre comme telle. Les marxis­tes, et notamment Kautsky et Lénine, paraphrasaient volontiers la formule de Clausewitz selon laquelle la guerre ne fait que continuer la politique du temps de paix, mais par d'autres moyens, la conclusion étant qu'il faut juger une guerre non par le caractère violent des procédés employés, mais par les objectifs poursuivis au travers de ces procédés.\par
L'après-guerre a introduit dans le mouvement ouvrier non pas une autre conception, car on ne saurait accuser les organisations ouvrières ou soi-disant telles de notre époque d'avoir des conceptions sur quelque sujet que ce soit, mais une autre atmosphère morale. Déjà en 1918, le parti bolchevik, qui désirait ardemment la guerre révolutionnaire, dut se résigner à la paix, non pour des raisons de doctrine ; mais sous la pression directe des soldats russes à qui l'exemple de 1793 n'inspirait pas plus d'émulation évoquée par les bolcheviks que par Kérenski. De même dans les autres pays, sur le plan de la simple propagande, les masses meurtries par la guerre contraignirent les partis qui se réclamaient du prolétariat à adopter un langage purement pacifiste, langage qui n'empêchait pas d'ailleurs les uns de célébrer l'armée rouge, les autres de voter les crédits de guerre de leur propre pays. Jamais, bien entendu, ce langage nouveau ne fut justifié par des analyses théoriques, jamais même on ne sembla remarquer qu'il était nouveau. Mais le fait est qu'au lieu de flétrir la guerre en tant qu'impérialiste, on se met à flétrir l'impérialisme en tant que fauteur de guerres. Le soi-disant mouvement d'Amsterdam, théoriquement dirigé contre la guerre impérialiste, dut, pour se faire écouter, se présenter comme dirigé contre la guerre en général. Les dispositions pacifiques de l'U.R.S.S. furent mises en relief, dans la propagande, plus encore que son caractère prolétarien ou soi-disant tel. Quant aux formules des grands théori­ciens du socialisme sur l'impossibilité de condamner la guerre comme telle, elles étaient complètement oubliées.\par
Le triomphe de Hitler en Allemagne a pour ainsi dire fait remonter à la surface toutes les anciennes conceptions, inextricablement mélangées. La paix apparaît comme moins précieuse du moment qu'elle peut comporter les horreurs indicibles sous le poids desquelles gémissent des milliers de travail­leurs dans les camps de concentration d'Allemagne. La conception exprimée par Engels dans son article de 1892 reparaît. L'ennemi principal du prolétariat international n'est-il pas le fascisme allemand, comme il était alors le tsarisme russe ? Ce fascisme, qui fait tache d'huile, ne peut être écrasé que par la force ; et, puisque le prolétariat allemand est désarmé, seules les nations restées démocratiques peuvent s'acquitter, semble-t-il, de cette tâche.\par
Peu importe au reste qu'il s’agisse d'une une guerre de défense ou d’une « guerre préventive » mieux vaudrait même une guerre préventive ; Marx et Engels n'ont-ils pas essayé, à un moment donné, de pousser l'Angleterre à attaquer la Russie ? Une semblable guerre n'apparaîtrait plus, pense-t-on, com­me une lutte entre deux impérialismes concurrents, mais entre deux régimes politiques. Et, tout comme faisait le vieil Engels en 1892, en se souvenant de ce qui s'était passé cent ans plus tôt, on se dit qu'une guerre forcerait l'État à faire des concessions sérieuses au prolétariat ; et cela d'autant plus que, dans la guerre qui menace, il y aurait nécessairement conflit entre l'État et la classe capitaliste, et sans doute des mesures de socialisation poussées assez loin. Qui sait si la guerre ne porterait pas ainsi automatiquement les représentants du prolétariat au pouvoir ? Toutes ces considérations créent dès maintenant, dans les milieux politiques qui se réclament du prolétariat, un courant d'opinion plus ou moins explicite en faveur d'une participation active du prolétariat à une guerre contre l'Allemagne ; courant encore assez faible, mais qui peut aisément s'étendre. D'autres s'en tiennent à la distinction entre agression et défense nationale ; d'autres à la conception de Lénine ; d'autres enfin, encore nombreux, restent pacifistes, mais, pour la plupart, plutôt par la force de l'habitude que pour toute autre raison. On ne saurait imaginer confusion pire.\par
Tant d'incertitude et d'obscurité peut surprendre et doit faire honte, si l'on songe qu'il s'agit d'un phénomène qui, avec son cortège de préparatifs, de réparations, de nouveaux préparatifs, semble, eu égard à toutes les consé­quences morales et matérielles qu'il entraîne, dominer notre époque et en constituer le fait caractéristique. Le surprenant serait pourtant qu'on fût arrivé à mieux en partant d'une tradition absolument légendaire et illusoire, celle de 1793, et en employant la méthode la plus défectueuse possible, celle qui prétend apprécier chaque guerre par les fins poursuivies et non par le caractère des moyens employés. Ce n'est pas qu'il vaille mieux blâmer en général l'usage de la violence, comme font les purs pacifistes ; la guerre constitue, à chaque époque, une espèce bien déterminée de violence, et dont il faut étudier le mécanisme avant de porter un jugement quelconque. La méthode matéria­liste consiste avant tout à examiner n'importe quel fait humain en tenant compte bien moins des fins poursuivies que des conséquences nécessairement impliquées par le jeu même des moyens mis en usage. On ne peut résoudre ni même poser un problème relatif à la guerre sans avoir démonté au préalable le mécanisme de la lutte militaire, c'est-à-dire analysé les rapports sociaux qu'elle implique dans des conditions techniques, économiques et sociales données.\par
On ne peut parler de guerre en général que par abstraction ; la guerre moderne diffère absolument de tout ce que l'on désignait par ce nom sous les régimes antérieurs. D'une part la guerre ne fait que prolonger cette autre guer­re qui a nom concurrence, et qui fait de la production elle-même une simple forme de la lutte pour la domination ; d'autre part toute la vie économique est présentement orientée vers une guerre à venir. Dans ce mélange inextricable du militaire et de l'économique, où les armes sont mises au service de la concurrence et la production au service de la guerre, la guerre ne fait que reproduire les rapports sociaux qui constituent la structure même du régime, mais à un degré beaucoup plus aigu. Marx a montré avec force que le mode moderne de la production se définit par la subordination des travailleurs aux instruments du travail, instruments dont disposent ceux qui ne travaillent pas ; et comment la concurrence, ne connaissant d'autre arme que l'exploitation des ouvriers, se transforme en une lutte de chaque patron contre ses propres ouvriers, et, en dernière analyse, de l'ensemble des patrons contre l'ensemble des ouvriers. De même la guerre, de nos jours, se définit par la subordination des combattants aux instruments de combat ; et les armements, véritables héros des guerres modernes, sont, ;ainsi que les hommes voués à leur service. dirigés par ceux qui ne combattent pas. Comme cet appareil de direction n’a pas d'autre moyen de battre l'ennemi que d'envoyer par contrainte ses propres soldats à la mort, la guerre d'un État contre un autre État se transforme aussitôt en guerre de l'appareil étatique et militaire contre sa propre armée ; et la guerre apparaît finalement comme une guerre menée par l'ensemble des appareils d'État et des états-majors contre l'ensemble des hommes valides en âge de porter les armes. Seulement, alors que les machines n'arrachent aux travailleurs que leur force de travail, alors que les patrons n'ont d'autre moyen de contrainte que le renvoi, moyen émoussé par la possibilité, pour le travailleur, de choisir entre les différents patrons, chaque soldat est contraint de sacrifier sa vie elle-même aux exigences de l'outillage militaire, et il y est contraint par la menace d'exécution sans jugement que le pouvoir d'État suspend sans cesse sur sa tête. Dès lors il importe bien peu que la guerre soit défensive ou offensive, impérialiste ou nationale ; tout État en guerre est contraint d'employer cette méthode, du moment que l'ennemi l'emploie. La grande erreur de presque toutes les études concernant la guerre, erreur dans laquelle sont tombés notamment tous les socialistes, est de considérer la guerre comme un épisode de la politique extérieure, alors qu'elle constitue avant tout un fait de politique intérieure, et le plus atroce de tous. Il ne s'agit pas ici de considérations sentimentales, ou d'un respect superstitieux de la vie humaine ; il s'agit d'une remarque bien simple, à savoir que le massacre est la forme la plus radicale de l'oppression ; et les soldats ne s'exposent pas à la mort, ils sont envoyés au massacre. Comme un appareil oppressif, une fois constitué, demeure jusqu'à ce qu'on le brise, toute guerre qui fait peser un appareil chargé de diriger les manœuvres stratégiques sur les masses que l'on contraint à servir de masses de manœuvres doit être considérée, même si elle est menée par des révolutionnaires, comme un facteur de réaction. Quant à la portée extérieure d'une telle guerre, elle est déterminée par les rapports politi­ques établis à l'intérieur ; des armes maniées par un appareil d'État souverain ne peuvent apporter la liberté à personne.\par
C'est ce qu'avait compris Robespierre et ce qu'a vérifié avec éclat cette guerre même de 1792 qui a donné naissance à la nation de guerre révolu­tionnaire. La technique militaire était loin encore à ce moment d'avoir atteint le même degré de centralisation que de nos jours ; cependant, depuis Frédéric II, la subordination des soldats chargés d'exécuter les opérations au haut commandement chargé de les coordonner était fort stricte. Au moment de la Révolution, une guerre devait transformer toute la France, comme le dira Barère, en un vaste camp, et donner par suite à l'appareil l'État ce pouvoir sans appel qui est le propre de l'autorité militaire. C'est le calcul que firent en 1792 la Cour et les Girondins ; car cette guerre, qu'une légende trop facilement acceptée par les socialistes a fait apparaître comme un élan spontané du peuple dressé à la fois contre ses propres oppresseurs et contre les tyrans étrangers qui le menaçaient, constitua en fait une provocation de la part de la Cour et de la haute bourgeoisie complotant de concert contre la liberté du peuple. En apparence elles se trompèrent, puisque la guerre, au lieu d'amener l'union sacrée qu'elles espéraient, exaspéra tous les conflits, mena le roi, puis les Girondins à l'échafaud et mit aux mains de la Montagne un pouvoir dictatorial. Mais cela n'empêche pas que le 20 avril 1792, jour de la décla­ration de guerre, tout espoir de démocratie sombra sans retour ; et le 2 juin ne fut suivi que de trop prés par le 9 thermidor, dont les conséquences, à leur tour, devaient bientôt amener le 18 brumaire. À quoi servit d'ailleurs à Robes­pierre et à ses amis le pouvoir qu'ils exercèrent avant le 9 thermidor ? Le but de leur existence n'était pas de s'emparer du pouvoir, mais d'établir une démocratie effective, à la fois démocratique et sociale ; c'est par une sanglante ironie de l'histoire que la guerre les contraignit à laisser sur le papier la Constitution de 1793, à forger un appareil centralisé, à exercer une terreur sanglante qu'ils ne purent même pas tourner contre les riches, à anéantir toute liberté, et à se faire en somme les fourriers du despotisme militaire, bureaucra­tique et bourgeois de Napoléon. Du moins restèrent-ils toujours lucides. L'avant-veille de sa mort, Saint-Just écrivait cette formule profonde : « Il n'y a que ceux qui sont dans les batailles qui les gagnent, et il n'y a que ceux qui sont puissants qui en profitent. » Quant à Robespierre, dès que la question se posa, il comprit qu'une guerre, sans pouvoir délivrer aucun peuple étranger (« on n'apporte pas la liberté à la pointe des baïonnettes »), livrerait le peuple français aux draines du pouvoir d'État, pouvoir qu'on ne pouvait plus chercher à affaiblir du moment qu'il fallait lutter contre l'ennemi extérieur. « La guerre est bonne pour les officiers militaires, pour les ambitieux, pour les agioteurs, ... pour le pouvoir exécutif... Ce parti dispense de tout autre soin, on est quitte envers le peuple quand on lui donne la guerre. » Il prévoyait dès lors le despotisme militaire, et ne cessa de le prédire par la suite, malgré les succès apparents de la Révolution ; il le prédisait encore l'avant-veille de sa mort, dans son dernier discours, et laissa cette prédiction après lui comme un testament dont ceux qui depuis se sont réclamés de lui n'ont malheureusement pas tenu compte.\par
L'histoire de la Révolution russe fournit exactement les mêmes enseigne­ments, et avec une analogie frappante. La Constitution soviétique a eu identi­quement le même sort que la Constitution de 1793 : Lénine a abandonné ses doctrines démocratiques pour établir le despotisme d'un appareil d'État centra­lisé, tout comme Robespierre, et a été en fait le précurseur de Staline, comme Robespierre celui de Bonaparte. La différence est que Lénine, qui avait d'ailleurs depuis longtemps préparé cette domination de l'appareil d'État en se forgeant un parti fortement centralisé, déforma par la suite ses propres doctri­nes pour les adapter aux nécessités de l'heure ; aussi ne fut-il pas guillotiné, et sert-il d'idole à une nouvelle religion d'État. L'histoire de la Révolution russe est d'autant plus frappante que la guerre y constitue constamment le problème central. La révolution fut faite contre la guerre, par des soldats qui, sentant l'appareil gouvernemental et militaire se décomposer au-dessus d'eux, se hâtè­rent de secouer un joug intolérable. Kérenski, invoquant avec une sincérité involontaire, due à son ignorance, les souvenirs de 1792, appela à la guerre exactement pour les mêmes motifs qu'autrefois les Girondins ; Trotsky a admirablement montré comment la bourgeoisie, comptant sur la guerre pour ajourner les problèmes de politique intérieure et ramener le peuple sous le joug du pouvoir d'État, voulait transformer « la guerre jusqu'à épuisement de l'ennemi en une guerre pour l'épuisement de la Révolution ». Les bolcheviks appelaient alors à lutter contre l'impérialisme ; mais c'était la guerre elle-même, non l'impérialisme, qui était en question, et ils le virent bien quand, une fois au pouvoir, ils se virent contraints de signer la paix de Brest-Litovsk. L'ancienne armée était alors décomposée et Lénine avait répété après Marx que la dictature du prolétariat ne peut comporter ni armée, ni police, ni bureaucratie permanentes. Mais les armées blanches et la crainte d'interven­tions étrangères ne tardèrent pas à mettre la Russie tout entière en état de siège. L'armée fut alors reconstituée, l'élection des officiers supprimée, trente mille officiers de l'ancien régime réintégrés dans les cadres, la peine de mort, l'ancienne discipline, la centralisation rétablies ; parallèlement se reconsti­tuaient la bureaucratie et la police. On sait assez ce que cet appareil militaire, bureaucratique et policier a fait du peuple russe par la suite.\par
La guerre révolutionnaire est le tombeau de la révolution et le restera tant qu’on n’aura pas donné aux soldats eux-mêmes, ou plutôt aux citoyens armés, le moyen de faire la guerre sans appareil dirigeant, sans pression policière, sans juridiction d'exception, sans peines pour les déserteurs. Une fois dans l'histoire moderne la guerre s'est faite ainsi, à savoir sous la Commune ; et l'on n'ignore pas comment cela s'est terminé. Il semble qu'une révolution engagée dans une guerre n'ait le choix qu'entre succomber sous les coups meurtriers de la contre-révolution, ou se transformer elle-même en contre-révolution par le mécanisme même de la lutte militaire. Les perspectives de révolution sem­blent dès lors bien restreintes ; car une révolution peut-elle éviter la guerre ? C'est pourtant sur cette faible chance qu'il faut miser, ou abandonner tout espoir. L'exemple russe est là pour nous instruire. Un pays avancé ne rencon­trerait pas, en cas de révolution, les difficultés qui, dans la Russie arriérée, servent de base au régime barbare de Staline ; mais une guerre de quelque envergure lui en susciterait d'autres pour le moins équivalentes.\par
À plus forte raison une guerre entreprise par un État bourgeois ne peut-elle que transformer le pouvoir en despotisme, et l'asservissement en assassinat. Si la guerre apparaît parfois comme un facteur révolutionnaire, c'est seulement en ce sens qu'elle constitue une épreuve incomparable pour le fonctionnement de l'appareil d'État. À son contact, un appareil mal organisé se décompose ; mais si la guerre ne se termine pas aussitôt et sans retour, ou si la décompo­sition n'est pas allée assez loin, il s'ensuit seulement une de ces révolutions qui, selon la formule de Marx, perfectionnent l'appareil d'État au lieu de le briser. C'est ce qui s'est toujours produit jusqu'ici. De nos jours la difficulté que la guerre porte à un degré aigu est celle qui résulte d'une opposition toujours croissante entre l'appareil d'État et le système capitaliste ; l'affaire de Briey pendant la dernière guerre \footnote{Cet article a paru en 1933.} en constitue un exemple frappant. La dernière guerre a apporté aux divers appareils d'État une certaine autorité sur l'économie, ce qui a donné lieu au terme tout à fait erroné de « socialisme de guerre » ; par la suite le système capitaliste s'est remis à fonctionner d'une manière à peu prés normale, en dépit des barrières douanières, du contingen­tement et des monnaies nationales. Dans une prochaine guerre les choses iraient sans doute beaucoup plus loin, et l'on sait que la quantité est suscepti­ble de se transformer en qualité. En ce sens, la guerre peut constituer de nos jours un facteur révolutionnaire, mais seulement si l'on veut comprendre le terme de révolution dans l'acception dans laquelle l'emploient les national-socialistes ; comme la crise, la guerre provoquerait une vive hostilité contre les capitalistes, et cette hostilité, à la faveur de l'union sacrée, tournerait au profit de l'appareil d'État et non des travailleurs. Au reste, pour reconnaître la parenté profonde qui lie le phénomène de la guerre et celui du fascisme, il suffit de se reporter aux textes fascistes qui évoquent « l'esprit guerrier » et le « socialisme du front ». Dans les deux cas, il s'agit essentiellement d'un effacement total de l'individu devant la bureaucratie d'État à la faveur d'un fanatisme exaspéré. Si le système capitaliste se trouve plus ou moins endom­magé dans l'affaire, ce ne peut être qu'aux dépens et non au profit des valeurs humaines et du prolétariat, si loin que puisse peut-être aller en certains cas la démagogie.\par
L'absurdité d'une lutte antifasciste qui prendrait la guerre comme moyen d'action apparaît ainsi assez clairement. Non seulement ce serait combattre une oppression barbare en écrasant les peuples sous le poids d'un massacre plus barbare encore, mais encore ce serait étendre sous une autre forme le régime qu'on veut supprimer. Il est puéril de supposer qu'un appareil d'État rendu puissant par une guerre victorieuse viendrait alléger l'oppression qu'ex­erce sur son propre peuple l'appareil d'État ennemi, plus puéril encore de croire qu'il laisserait une révolution prolétarienne éclater chez ce peuple à la faveur de la défaite sans la noyer aussitôt dans le sang. Quant à la démocratie bourgeoise anéantie par le fascisme, une guerre n'abolirait pas, mais renfor­cerait et étendrait les causes qui la rendent présentement impossible. Il semble, d'une manière générale, que l'histoire contraigne de plus en plus toute action politique à choisir entre l'aggravation de l'oppression intolérable qu'ex­ercent les appareils d'État et une lutte sans merci dirigée directement contre eux pour les briser. Certes les difficultés peut-être insolubles qui apparaissent de nos jours peuvent justifier l'abandon pur et simple de la lutte. Mais si l'on ne veut pas renoncer à agir, il faut comprendre qu'on ne peut lutter contre un appareil d'État que de l'intérieur. Et en cas de guerre notamment il faut choisir entre entraver le fonctionnement de la machine militaire dont on constitue soi-même un rouage, ou bien aider cette machine à broyer aveuglément les vies humaines. La parole célèbre de Liebknecht : « L'ennemi principal est dans notre propre pays » prend ainsi tout son sens, et se révèle applicable à toute guerre ou les soldats sont réduits à l'état de matière passive entre les mains d'un appareil militaire et bureaucratique ; c'est-à-dire, tant que la technique actuelle persistera, à toute guerre, absolument parlant. Et l'on ne peut entrevoir de nos jours l'avènement d'une autre technique. Dans la production comme dans la guerre, la manière de plus en plus collective dont s'opère la dépense des forces n'a pas modifié le caractère essentiellement individuel des fonctions de décision et de direction ; elle n'a fait que mettre de plus en plus les bras ou les vies des masses à la disposition des appareils de commandement.\par
Tant que nous n'apercevrons pas comment il est possible d'éviter, dans l'acte même de produire ou de combattre, cette emprise des appareils sur les masses, toute tentative révolutionnaire aura quelque chose de désespéré ; car si nous savons quel système de production et de combat nous aspirons de toute notre âme à détruire, on ignore quel système acceptable pourrait le remplacer. Et d'autre part toute tentative de réforme apparaît comme puérile au regard des nécessités aveugles impliquées par le jeu de ce monstrueux engrenage. La société actuelle ressemble à une immense machine qui happe­rait sans cesse des hommes, et dont personne ne connaîtrait les commandes ; et ceux qui se sacrifient pour le progrès social ressemblent à des gens qui s'agripperaient aux rouages et aux courroies de transmission pour essayer d'arrêter la machine, et se feraient broyer à leur tour. Mais l'impuissance où l'on se trouve à un moment donné, impuissance qui ne doit jamais être regardée comme définitive, ne peut dispenser de rester fidèle à soi-même, ni excuser la capitulation devant l'ennemi, quelque masque qu'il prenne. Et, sous tous les noms dont il peut se parer, fascisme, démocratie ou dictature du prolétariat, l'ennemi capital reste l'appareil administratif, policier et militaire ; non pas celui d'en face, qui n'est notre ennemi qu'autant qu'il est celui de nos frères, mais celui qui se dit notre défenseur et fait de nous ses esclaves. Dans n'importe quelle circonstance, la pire trahison possible consiste toujours à accepter de se subordonner à cet appareil et de fouler aux pieds pour le servir, en soi--même et chez autrui, toutes les valeurs humaines.\par
({\itshape La Critique sociale}, n° 10, novembre 1933.)\par

\subsubsection[2. Fragment sur la guerre révolutionnaire, (fin 1933)]{2. \\
Fragment sur la guerre révolutionnaire \\
(fin 1933)}
\noindent \par
[Ces questions] se ramènent toutes à la question de la valeur révolution­naire de la guerre. La légende de 1793 a créé sur ce point, dans tout le mouvement ouvrier, une équivoque dangereuse et qui dure encore.\par
La guerre de 1792 n'a pas été une guerre révolutionnaire. Elle n'a pas été une défense à main armée de la république française contre les rois, mais, du moins, à l'origine, une manœuvre de la cour et des Girondins pour briser la révolution, manœuvre à laquelle Robespierre, dans son magnifique discours contre la déclaration de la guerre, tenta en vain de s'opposer. Il est vrai que la guerre elle-même, par ses exigences propres, chassa ensuite les Girondins du gouvernement et y porta les Montagnards ; néanmoins la manœuvre des Girondins, dans ce qu'elle avait d'essentiel, fut un succès. Car Robespierre et ses amis, bien que placés aux postes responsables de l'État, ne purent rien réaliser, ni de la démocratie politique ni des transformations sociales qu'ils avaient à leurs propres yeux pour unique raison d'être de donner au peuple français. Ils ne purent même pas s'opposer à la corruption qui finit par les faire périr. Ils ne firent en fait, par la centralisation brutale et la terreur insensée que la guerre rendait indispensables, qu'ouvrir la voie à la dictature militaire. Robespierre s'en rendait compte avec cette étonnante lucidité qui faisait sa grandeur, et il l'a dit, non sans amertume, dans le fameux discours qui a immédiatement précédé sa mort. Quant aux conséquences de cette guerre à l'étranger, elle contribua évidemment à détruire la vieille structure féodale de quelques pays, mais par contre, dès que, par un développement inéluctable, elle s'orienta vers la conquête, elle affaiblit singulièrement la force de propagande des idées révolutionnaires françaises, conformément à la célèbre parole de Robespierre : « On n'aime pas les missionnaires armés. » Ce n'est pas sans cause que Robespierre a été accusé de voir sans plaisir les victoires des armées françaises. C'est la guerre qui, pour reprendre l'expression de Marx, à Liberté, Égalité, Fraternité, a substitué Infanterie, Cavalerie, Artillerie.\par
Au reste, même la guerre d'intervention, en Russie, guerre véritablement défensive, et dont les combattants méritent notre admiration, a été un obstacle infranchissable pour le développement de la révolution russe. C'est cette guer­re qui a imposé à une révolution dont le programme était l'abolition de l'armée, de la police et de la bureaucratie permanentes une armée rouge dont les cadres furent constitués par les officiers tsaristes, une police qui ne devait pas tarder à frapper les communistes plus durement que les contre-révolution­naires, un appareil bureaucratique sans équivalent dans le reste du monde. Tous ces appareils devaient répondre à des nécessités passagères ; mais ils survécurent fatalement à ces nécessités. D'une manière générale la guerre renforce toujours le pouvoir central aux dépens du peuple ; comme l'a écrit Saint-Just : « Il n'y a que ceux qui sont dans les batailles qui les gagnent, et il n'y a que ceux qui sont puissants qui en profitent. » La Commune de Paris a fait exception ; mais aussi a-t-elle été vaincue. La guerre est inconcevable sans une organisation oppressive, sans un pouvoir absolu de ceux qui dirigent, constitués en un appareil distinct, sur ceux qui exécutent. En ce sens, si l'on admet, avec Marx et Lénine, que la révolution, de nos jours, consiste avant tout à briser immédiatement et définitivement l'appareil d'État, la guerre, même faite par des révolutionnaires pour défendre la révolution qu'ils ont faite, constitue un facteur contre-révolutionnaire. À plus forte raison, quand la guerre est dirigée par une classe oppressive, l'adhésion des opprimés à la guerre constitue-t-elle une abdication complète entre les mains de l'appareil d'État qui les écrase. C'est ce qui s'est produit en 1914 ; et dans cette honteuse trahison il faut bien reconnaître qu'Engels porte sa part de responsabilité.\par

\subsubsection[3. Encore quelques mots sur le boycottage (fragment) (Fin 1933 ? Début 1934 ?)]{3. \\
Encore quelques mots sur le boycottage (fragment) \\
(Fin 1933 ? Début 1934 ?)}
\noindent \par
La question du boycottage économique de l'Allemagne hitlérienne a soulevé et soulève bien des discussions entre camarades qui sont tous d'égale bonne foi. Les uns sont poussés par le désir de lutter contre l'odieuse terreur hitlérienne ; les autres retenus par la crainte d'éveiller les passions nationales. Les deux Internationales réformistes, politique et syndicale, ont pris pour le boycottage des résolutions non appliquées encore ; des secrétaires d'organisa­tions confédérées se sont élevés contre ces décisions. Le plus clair de l'affaire est que voici bientôt un an écoulé sans qu'il y ait eu le moindre geste de solidarité internationale contre les immondes tortures que l'on inflige à la fleur du mouvement ouvrier d'Allemagne. Cette constatation serre le cœur. Il me semble que, de part et d'autre, le problème de l'action antifasciste a été mal posé.\par
Il faut faire, en faveur de nos camarades allemands, une action dont les masses populaires d'Allemagne aient connaissance. Car une des bases psycho­logiques du national-socialisme est l'amer sentiment d'isolement où se sont trouvées les masses laborieuses d'Allemagne accablées par le double poids de la crise et du « diktat » de Versailles. De cet isolement nous sommes pleine­ment responsables, nous tous qui en France nous disons internationa­listes et ne savons l'être que du bout des lèvres. Le seul moyen efficace pour nous de lutter contre Hitler est de montrer aux ouvriers allemands que leurs camarades français sont prêts à faire des efforts et des sacrifices pour eux. D'autre part il ne faut à aucun prix attiser les passions nationalistes, et cela parce que de ce fait l'action antifasciste deviendrait non seulement dangereuse par rapport à la France, mais encore vaine par rapport à l'Allemagne ; le peuple allemand croirait les ouvriers français dressés non pas contre le despotisme, mais contre la nation allemande, et cela de concert avec leur propre bourgeoisie et l'impé­rialisme de leur propre pays. Peut-être pourrions-nous négliger ce risque si nous avions, nous tous qui prenons part au mouvement ouvrier français, su montrer avant l'avènement de Hitler que nous n'étions solidaires ni de l'impérialisme français ni du « système de Versailles ». Ce n'est, hélas ! point le cas, et nous ne pourrons jamais nous le pardonner. Mais toujours est-il que nous devons à présent tenir compte des difficultés suscitées par notre propre lâcheté de naguère.\par
La solution se trouve dans une action purement ouvrière. Il y a des actions pour lesquelles le prolétariat a avantage à se joindre à la petite bourgeoisie libérale ; ce fut le cas par exemple lors de l'Affaire Dreyfus. Mais ce n'est jamais le cas lorsque le nationalisme peut entrer en jeu ; car les petits bour­geois sont toujours prompts à se révéler comme des chauvins enragés, et rien ne peut jamais être plus dangereux pour le prolétariat que les passions nationales, qui toujours aboutissent à une sorte d'union sacrée, et font le jeu de l'État bourgeois. Les ouvriers allemands doivent être secourus par les ouvriers français, et par eux seuls. Ils ne peuvent rien avoir de commun avec la petite bourgeoisie française, qui a toujours été le pilier le plus solide du système de Versailles, et porte par suite une lourde part de responsabilité dans la victoire du national-socialisme. On dira que c'est là une question de pur sentiment ; mais précisément la répercussion d'une action anti-hitlérienne venue de France sur la classe ouvrière allemande serait d'ordre principalement sentimental, et n'en serait pas moins importante pour cela.\par
À vrai dire une union des classes dans une action menée contre Hitler serait beaucoup moins à craindre aujourd'hui qu'il y a quelques mois...\par

\subsubsection[4. Réponse à une question d’Alain  (1936 ?)]{4. \\
Réponse à une question d’Alain \protect\footnotemark  \\
(1936 ?)}
\footnotetext{ Il s'agit du {\itshape Questionnaire d'Alain}, publié dans le n° 34 de {\itshape Vigilance}, le 20 mars 1936, pp. 10-11. (Note de l'éditeur.)}
\noindent \par
Je ne répondrai qu'à la dernière des questions d'Alain. Elle me paraît d'une grande importance. Mais je crois qu'il faut la poser plus largement. Les mots de dignité et d'honneur sont peut-être aujourd'hui les plus meurtriers du vocabulaire. Il est bien difficile de savoir au juste comment le peuple français a réellement réagi aux derniers événements. Mais j'ai trop souvent remarqué que dans toutes sortes de milieux l'appel à la dignité et à l'honneur en matière internationale continue à émouvoir. La formule « la paix dans la dignité » ou « la paix dans l'honneur », formule de sinistre mémoire qui, sous la plume de Poincaré, a immédiatement préludé au massacre, est encore employée cou­ramment. Il n'est pas sûr que les orateurs qui préconiseraient la paix même sans honneur rencontreraient où que ce soit un accueil favorable. Cela est très grave.\par
Le mot de dignité est ambigu. Il peut signifier l'estime de soi-même ; nul n'osera alors nier que la dignité ne soit préférable à la vie, car préférer la vie serait « pour vivre, perdre les raisons de vivre » . Mais l'estime de soi dépend exclusivement des actions que l'on exécute soi-même après les avoir librement décidées. Un homme outragé peut avoir besoin de se battre pour retrouver sa propre estime ; ce sera le cas seulement s'il lui est impossible de subir passi­vement l'outrage sans se trouver convaincu de lâcheté à ses propres yeux. Il est clair qu'en pareille matière chacun est juge et seul juge. On ne peut imaginer qu'aucun homme puisse déléguer à un autre le soin de juger si oui ou non la conservation de sa propre estime exige qu'il mette sa vie en jeu. Il est plus clair encore que la défense de la dignité ainsi comprise ne peut être imposée par contrainte ; dès que la contrainte entre en jeu, l'estime de soi cesse d'être en cause. D'autre part ce qui délivre de la honte, ce n'est pas la vengeance, mais le péril. Par exemple, tuer un offenseur par ruse et sans risque n'est jamais un moyen de préserver sa propre estime.\par
Il faut en conclure que jamais la guerre n'est une ressource pour éviter d'avoir à se mépriser soi-même. Elle ne peut être une ressource pour les non-combattants, parce qu'ils n'ont pas part au péril, ou relativement peu ; la guerre ne peut rien changer à l'opinion qu'ils se font de leur propre courage. Elle ne peut pas non plus être une ressource pour les combattants, parce qu'ils sont forcés. La plupart partent par contrainte, et ceux-mêmes qui partent volontai­rement restent par contrainte. La puissance d'ouvrir et de fermer les hostilités est exclusivement entre les mains de ceux qui ne se battent pas. La libre résolution de mettre sa vie en jeu est l'âme même de l'honneur ; l'honneur n'est pas en cause là où les uns décident sans risques, et les autres meurent pour exécuter. Et si la guerre ne peut constituer pour personne une sauvegarde de l'honneur, il faut en conclure aussi qu'aucune paix n'est honteuse, quelles qu'en soient les clauses.\par
En réalité, le terme de dignité, appliqué aux rapports internationaux, ne désigne pas l'estime de soi-même, laquelle ne peut être en cause ; il ne s'oppose pas au mépris de soi, mais à l'humiliation. Ce sont choses distinctes ; il y a bien de la différence entre perdre le respect de soi-même et être traité sans respect par autrui. Épictète manié comme un jouet par son maître, Jésus souffleté et couronné d'épines n'étaient en rien amoindris à leurs propres yeux. Préférer la mort au mépris de soi, c'est le fondement de n'importe quelle mora­le ; préférer la mort à l'humiliation, c'est bien autre chose, c'est simplement le point d'honneur féodal. On peut admirer le point d'honneur féodal ; on peut aussi, et non sans de bonnes raisons, refuser d'en faire une règle de vie. Mais la question n'est pas là. Il faut voir qui l'on envoie mourir pour défendre ce point d'honneur dans les conflits internationaux.\par
On envoie les masses populaires, ceux-mêmes qui, n'ayant aucune ri­chesse, n'ont en règle générale droit à aucun égard, ou peu s'en faut. Nous sommes en République, il est vrai ; mais cela n'empêche pas que l'humiliation ne soit en fait le pain quotidien de tous les faibles. Ils vivent néanmoins et laissent vivre. Qu'un subordonné subisse une réprimande méprisante sans pouvoir discuter ; qu'un ouvrier soit mis à la porte sans explications, et, s'il en demande à son chef, s'entende répondre « je n'ai pas de comptes à vous rendre » ; que des chômeurs convoqués devant un bureau d'embauche apprennent au bout d'une heure d'attente qu'il n'y a rien pour eux ; qu'une châtelaine de village donne des ordres à un paysan pauvre et lui octroie cinq sous pour un dérangement de deux heures ; qu'un gardien de prison frappe et injurie un prisonnier ; qu'un magistrat fasse de l'esprit en plein tribunal aux dépens d'un prévenu ou même d'une victime ; néanmoins le sang ne coulera pas. Mais cet ouvrier, ces chômeurs et les autres sont perpétuellement exposés à devoir un jour tuer et mourir parce qu'un pays étranger n'aura pas traité leur pays ou ses représentants avec tous les égards désirables. S'ils voulaient se mettre à laver l'humiliation dans le sang pour leur propre compte comme on les invite à le faire pour le compte de leur pays, que d'hécatombes quotidien­nes en pleine paix ! Parmi tous ceux qui possèdent une puissance grande ou petite, bien peu peut-être survivraient ; il périrait à coup sûr beaucoup de chefs militaires.\par
Car le plus fort paradoxe de la vie moderne, c'est que non seulement on foule aux pieds dans la vie civile la dignité personnelle de ceux que l'on enverra un jour mourir pour la dignité nationale ; mais au moment même où leur vie se trouve ainsi sacrifiée pour sauvegarder l'honneur commun, ils se trouvent exposés à des humiliations bien plus dures encore qu'auparavant. Que sont les outrages considérés de pays à pays comme des motifs de guerre auprès de ceux qu'un officier peut impunément infliger à un soldat ? Il peut l'insulter, et sans qu'aucune réponse soit permise ; il peut lui donner des coups de pied - un auteur de souvenirs de guerre ne s'est-il pas vanté de l'avoir fait ? Il peut lui donner n'importe quel ordre sous la menace du revolver, y compris celui de tirer sur un camarade. Il peut lui infliger à titre de punition les brimades les plus mesquines. Il peut à peu près tout, et toute désobéissance est punie de mort ou peut l'être. Ceux qu'à l’arrière on célèbre hypocritement comme des héros, on les traite effectivement comme des esclaves. Et ceux des soldats survivants qui sont pauvres, délivrés de l'esclavage militaire retombent à l'esclavage civil, où plus d'un est contraint de subir les insolences de ceux qui se sont enrichis sans risques.\par
L'humiliation perpétuelle et presque méthodique est un facteur essentiel de notre organisation sociale, en paix comme en guerre, mais en guerre à un degré plus élevé. Le principe selon lequel il faudrait repousser l'humiliation au prix même de la vie, s'il était appliqué à l'intérieur du pays, serait subversif de tout ordre social, et notamment de la discipline indispensable à la conduite de la guerre. Qu'on ose, dans ces conditions, faire de ce principe une règle de politique internationale, c'est véritablement le comble de l'inconscience. Une formule célèbre dit qu'on peut à la rigueur avoir des esclaves, mais qu'il n'est pas tolérable qu'on les traite de citoyens. Il est moins tolérable encore qu'on en fasse des soldats. Certes il y a toujours eu des guerres ; mais que les guerres soient faites par les esclaves, c'est le propre de notre époque. Et qui plus est, ces guerres où les esclaves sont invités à mourir au nom d'une dignité qu'on ne leur a jamais accordée, ces guerres constituent le rouage essentiel dans le mécanisme de l'oppression. Toutes les fois qu'on examine de près et d'une manière concrète les moyens de diminuer effectivement l'oppression et l'iné­galité, c'est toujours à la guerre qu'on se heurte, aux suites de la guerre, aux nécessités imposées par la préparation à la guerre. On ne dénouera pas ce nœud, il faut le couper, si toutefois on le peut.\par

\subsubsection[5. Faut-il graisser les godillots ? (27 octobre 1936)]{5. \\
Faut-il graisser les godillots ? \\
(27 octobre 1936)}
\noindent \par
On commençait à s'accoutumer à entendre certains de nos camarades chanter la Marseillaise ; mais depuis la guerre d'Espagne, c'est de tous côtés qu'on entend des paroles qui nous rajeunissent, hélas ! de vingt-deux ans. Il paraîtrait que cette fois-ci, on mettrait sac au dos pour le droit, la liberté et la civilisation, sans compter que ce serait, bien entendu, la dernière des guerres. Il est question aussi de détruire le militarisme allemand, et de défendre la démocratie aux côtés d'une Russie dont le moins qu'on puisse dire est qu'elle n'est pas un État démocratique. À croire qu'on a inventé la machine à parcou­rir le temps...\par
Seulement cette fois-ci il y a l'Espagne, il y a une guerre civile. Il ne s'agit plus pour certains camarades de transformer la guerre internationale en guerre civile, mais la guerre civile en guerre internationale. On entend même parler de « guerre civile internationale ». Il paraît qu'en s'efforçant d'éviter cet élar­gissement de la guerre, on fait preuve d'une honteuse lâcheté. Une revue qui se réclame de Marx a pu parler de la « politique de la fesse tendue ».\par
De quoi s'agit-il ? De prouver à soi-même qu'on n'est pas un lâche ? Camarades, on engage pour l'Espagne. La place est libre. On vous trouvera bien quelques fusils là-bas... Ou de défendre un idéal ? Alors, camarades, posez-vous cette question : est-ce qu'aucune guerre peut amener dans le monde plus de justice, plus de liberté, plus de bien-être ? l'expérience a-t-elle été faite, ou non ? Chaque génération va-t-elle la recommencer ? Combien de fois ?\par
Mais, dira-t-on, il n'est pas question de faire la guerre. Qu'on parle ferme, et les puissances fascistes reculeront. Singulier manque de logique ! Le fascisme, dit-on, c'est la guerre. Qu'est-ce à dire, sinon que les États fascistes ne reculeront pas devant les désastres indicibles que provoquerait une guerre ? Au lieu que nous, nous reculons. Oui, nous reculons et nous reculerons devant la guerre. Non pas parce que nous sommes des lâches. Encore une fois, libre à tous ceux qui craignent de passer pour lâches à leurs propres yeux d'aller se faire tuer en Espagne. S'ils allaient sur le front d'Aragon, par exemple, ils y rencontreraient peut-être, le fusil à la main, quelques Français pacifistes et qui sont restés pacifistes. Il ne s'agit pas de courage ou de lâcheté, il s'agit de peser ses responsabilités et de ne pas prendre celle d'un désastre auquel rien ne saurait se comparer.\par
Il faut en prendre son parti. Entre un gouvernement qui ne recule pas devant la guerre et un gouvernement qui recule devant elle, le second sera ordinairement désavantagé dans les négociations internationales. Il faut choisir entre le prestige et la paix. Et qu'on se réclame de la patrie, de la démocratie ou de la révolution, la politique de prestige, c'est la guerre. Alors ? Alors il serait temps de se décider : ou fleurir la tombe de Poincaré, ou cesser de nous exhorter à faire les matamores. Et si le malheur des temps veut que la guerre civile devienne aujourd'hui une guerre comme une autre, et presque inévitablement liée à la guerre internationale, on n'en peut tirer qu'une con­clusion : éviter aussi la guerre civile.\par
Nous sommes quelques-uns qui jamais, en aucun cas, n'iront fleurir la tombe de Poincaré.\par
({\itshape Vigilance}, n° 44/45, 27 octobre 1936.)\par

\subsubsection[6. La politique de neutralité et l’assistance mutuelle, (1936)]{6. \\
La politique de neutralité et l’assistance mutuelle \\
(1936)}
\noindent \par
La politique de neutralité à l'égard de l'Espagne suscite des polémiques si passionnées qu'on néglige de remarquer quel précédent formidable elle constitue en matière de politique internationale.\par
Dans l'ensemble, la classe ouvrière française semble avoir approuvé les efforts accomplis par Léon Blum pour sauvegarder la paix. Mais le moins qu'on puisse lui demander, c'est de ne les approuver que conditionnellement. Il faut savoir si ces efforts auront la suite logique qu'ils comportent. Et, à parler franc, cette suite logique serait en contradiction directe avec le programme du Front Populaire. Pour poser nettement la question, neutralité ou assistance mutuelle, il faut choisir.\par
Assistance mutuelle, c'est le mot d'ordre que le Front Populaire a fait résonner à nos oreilles jusqu'à l'obsession, avant, pendant et après la période électorale. Ce mot d'ordre nous était familier ; les politiciens de droite nous y avaient accoutumés. Il constitue à présent toute la doctrine des partis de gauche. Le grand discours de Blum à Genève n'a fait que le développer, l'exposer sous tous ses aspects. Et voici qu'à présent Blum lui-même, non par ses paroles, mais par ses actes, en proclame l'absurdité.\par
Qu'est-ce qui s'est produit de l'autre côté des Pyrénées, au mois de juillet ? Une agression caractérisée, qui ne peut faire de doute pour personne. Bien sûr, ce n'est pas une nation qui a attaqué une nation. C'est une caste militaire qui a attaqué un grand peuple. Mais nous n'en sommes que plus directement intéres­sés à l'issue du conflit. Les libertés du peuple français sont étroitement liées aux libertés du peuple espagnol. Si la doctrine de l'assistance mutuelle était raisonnable, ce serait là l'occasion ou jamais d'intervenir par la force armée, de courir au secours des victimes de l'agression.\par
On ne l'a pas fait, de peur de mettre en feu l'Europe entière. On a proclamé la neutralité. On a mis l'embargo sur les armes. Nous laissons des camarades bien chers exposer seuls leur vie pour une cause qui est la nôtre aussi bien que la leur. Nous les laissons tomber, le fusil ou la grenade à la main, parce qu'ils doivent remplacer avec leur chair vivante les canons qui leur manquent. Tout cela pour éviter la guerre européenne.\par
Mais si, le cœur serré, nous avons accepté une pareille situation, qu'on ne s'avise pas par la suite de nous envoyer aux armes quand il s'agira d'un conflit entre nations. Ce que nous n'avons pas fait pour nos chers camarades d'Espagne, nous ne le ferons ni pour la Tchécoslovaquie, ni pour la Russie, ni pour aucun État. En présence du conflit le plus poignant pour nous, nous avons laissé le gouvernement proclamer la neutralité. Qu'il ne s'avise plus par la suite de nous parler d'assistance mutuelle. Devant tous les conflits, quels qu'ils soient, qui pourront éclater sur la surface du globe, nous crierons à notre tour, de toutes nos forces : Neutralité ! Neutralité ! Nous ne pourrons nous pardonner d'avoir accepté la neutralité à l'égard de la tuerie espagnole que si nous faisons tout pour transformer cette attitude en un précédent qui règle à l'avenir toute la politique extérieure française.\par
Pourrait-il en être autrement ? Nous regardons presque passivement couler le plus beau sang du peuple espagnol, et nous partirions en guerre pour un quelconque État de l'Europe centrale ! Nous exposons à la défaite, à l'exter­mination une révolution toute jeune, toute neuve, débordante de vie, riche d'un avenir illimité, et nous partirions en guerre pour ce cadavre de révolution qui a nom U.R.S.S.\par
La politique actuelle de neutralité constituerait la pire trahison de la part des organisations ouvrières françaises si elle n'était pas dirigée contre la guerre. Et elle ne peut être efficacement dirigée contre la guerre que si elle est élargie, si le principe de la neutralité se substitue entièrement au principe meurtrier de l'assistance mutuelle. Nous n'avons le droit d'approuver Léon Blum que sous cette condition.\par

\begin{center}
\end{center}
\subsubsection[7. Non-intervention généralisée, (1936 ? 1937 ?)]{7. \\
Non-intervention généralisée \\
(1936 ? 1937 ?)}
\noindent \par
Depuis le début de la politique de non-intervention, une préoccupation me pèse sur le cœur. Beaucoup d'autres, certainement, la partagent.\par
Mon intention n'est pas de me joindre aux violentes attaques, quelques-unes sincères, la plupart perfides, qui se sont abattues sur notre camarade Léon Blum. Je reconnais les nécessités qui déterminent son action. Si dures, si amères qu'elles soient, j'admire le courage moral qui lui a permis de s'y soumettre malgré toutes les déclamations. Même quand j'étais en Aragon, en Catalogne, au milieu d'une atmosphère de combat, parmi des militants qui n'avaient pas de terme assez sévère pour qualifier la politique de Blum, j'approuvais cette politique. C'est que je me refuse pour mon compte person­nel à sacrifier délibérément la paix, même lorsqu'il s'agit de sauver un peuple révolutionnaire menacé d'extermination.\par
Mais dans presque tous les discours que notre camarade Léon Blum a prononcés depuis le début de la guerre espagnole, je trouve, à côté de formu­les profondément émouvantes sur la guerre et la paix, d'autres formules qui rendent un son inquiétant. J'ai attendu avec anxiété que des militants respon­sables réagissent, discutent, posent certaines questions. Je constate que l'atmosphère trouble qui existe à l'intérieur du Front Populaire réduit bien des camarades au silence ou à une expression enveloppée de leur pensée.\par
Léon Blum ne manque pas une occasion, au milieu des phrases les plus émouvantes, d'exposer en substance ceci : nous voulons la paix, nous la maintiendrons à tout prix, sauf si une agression contre notre territoire ou les territoires garantis par nous nous contraint à la guerre.\par
Autrement dit, nous ne ferons pas la guerre pour empêcher les ouvriers, les paysans espagnols d'être exterminés par une clique de sauvages plus ou moins galonnés. Mais, le cas échéant, nous ferions la guerre pour l'Alsace-Lorraine, pour le Maroc, pour la Russie, pour la Tchécoslovaquie, et, si un Tardieu quelconque avait signé un pacte d'alliance avec Honolulu, nous ferions la guerre pour Honolulu.\par
En raison de la sympathie que j'éprouve pour Léon Blum, et surtout à cause des menaces qui pèsent sur tout notre avenir, je donnerais beaucoup pour pouvoir interpréter autrement les formules auxquelles je pense. Mais il n'y a pas d'autre interprétation possible. Les paroles de Blum ne sont que trop claires.\par
Est-ce que les militants des organisations de gauche et de la C.G.T., est-ce que les ouvriers et les paysans de notre pays acceptent cette position ? Je n'en sais rien. Chacun doit prendre ses responsabilités. En ce qui me concerne, je ne l'accepte pas.\par
Les ouvriers, les paysans qui, de l'autre côté des Pyrénées, se battent pour défendre leur vie, leur liberté, pour soulever le poids de l'oppression sociale qui les a écrasés si longtemps, pour arriver à prendre en main leur destinée, ne sont liés à la France par aucun traité écrit. Mais tous, C.G.T., parti socialiste, classe ouvrière, nous nous sentons liés à eux par un pacte de fraternité non écrit, par des liens de chair et de sang plus forts que tous les traités. Que pèsent, au regard de cette fraternité unanimement ressentie, les signatures apposées par des Poincaré, des Tardieu, des Laval quelconques sur des papiers qui n'ont jamais été soumis à notre approbation ? Si jamais la somme de souffrances, de sang et de larmes que représente une guerre pouvait se justi­fier, ce serait lorsqu'un peuple lutte et meurt pour une cause qu'il a le désir de défendre, non pour un morceau de papier dont il n'a jamais eu à connaître.\par
Léon Blum partage sans doute, sur la question espagnole, les sentiments des masses populaires. On dit que lorsqu'il a parlé de l'Espagne devant les secrétaires de fédérations socialistes, il a pleuré. Très probablement, s'il était dans l'opposition, il prendrait à son compte le mot d'ordre : « des canons pour l'Espagne » . Ce qui a retenu son élan de solidarité, c'est un sentiment lié à la possession du pouvoir : le sentiment de responsabilité d'un homme qui tient entre ses mains le sort d'un peuple, et qui se voit sur le point de le précipiter dans une guerre. Mais si au lieu des ouvriers et des paysans espagnols une quelconque Tchécoslovaquie était en jeu, serait-il saisi du même sentiment de responsabilité ? Ou bien un certain esprit juridique lui ferait-il croire qu’en pareil cas toute la responsabilité appartient à un morceau de papier ? Cette question est pour chacun de nous une question de vie ou de mort.\par
La sécurité collective est au programme du Front Populaire. À mon avis, quand les communistes accusent Léon Blum d'abandonner, dans l'affaire espagnole, le programme du Front Populaire, ils ont raison. Il est vrai que les pactes et autres textes se rapportant à la sécurité collective ne prévoient rien de semblable au conflit espagnol ; c'est qu'on ne s'est jamais attendu à rien de semblable. Mais enfin les faits sont assez clairs. Il y a eu agression, agression militaire caractérisée, quoique sous forme de guerre civile. Des pays étrangers ont soutenu cette agression. Il semblerait normal d'étendre à un cas pareil le principe de la sécurité collective, d'intervenir militairement pour écraser l'armée coupable d'agression. Au lieu de s'orienter dans cette voie, Léon Blum a essayé de limiter le conflit. Pourquoi ? Parce que l'intervention, au lieu de rétablir l'ordre en Espagne, aurait mis le feu à toute l'Europe. Mais il en a toujours été, il en sera toujours de même toutes les fois qu'une guerre locale pose la question de la sécurité collective. Je défie n'importe qui, y compris Léon Blum, d'expliquer pourquoi les raisons qui détournent d'intervenir en Espagne auraient moins de force s'il s'agissait de la Tchécoslovaquie envahie par les Allemands.\par
Beaucoup de gens ont demandé à Léon Blum de « reconsidérer » sa politique à l'égard de l'Espagne. C'est une position qui se défend. Mais si on ne l'adopte pas, alors, pour être conséquent envers soi-même, il faut demander à Léon Blum d'une part, aux masses populaires de l'autre, de « reconsidérer » le principe de la sécurité collective. Si la non-intervention en Espagne est raisonnable, la sécurité collective est une absurdité, et réciproquement.\par
Le jour où Léon Blum a décidé de ne pas intervenir en Espagne, il a assumé une lourde responsabilité. Il a décidé alors d'aller, le cas échéant, jusqu'à abandonner nos camarades d'Espagne à une extermination massive. Nous tous qui l'avons soutenu, nous partageons cette responsabilité. Eh bien ! si nous avons accepté de sacrifier les mineurs des Asturies, les paysans affamés d'Aragon et de Castille, les ouvriers libertaires de Barcelone, plutôt que d'allumer une guerre mondiale, rien d'autre au monde ne doit nous amener à allumer la guerre. Rien, ni l'Alsace-Lorraine, ni les colonies, ni les pactes. Il ne sera pas dit que rien au monde nous est plus cher que la vie du peuple espagnol. Ou bien si nous les abandonnons, si nous les laissons massacrer, et si ensuite nous faisons quand même la guerre pour un autre motif, qu'est-ce qui pourra nous justifier à nos propres yeux ?\par
Est-ce qu'on va se décider, oui ou non, à regarder ces questions en face, à poser dans son ensemble le problème de la guerre et de la paix ? Si nous continuons à éluder le problème, à fermer volontairement les yeux, à répéter des mots d'ordre qui ne résolvent rien, que vienne donc alors la catastrophe mondiale. Tous nous l'aurons méritée par notre lâcheté d'esprit.\par

\subsubsection[8. Ne recommençons pas la guerre de Troie, (Pouvoir des mots) (1er au 15 avril 1937)]{8. \\
Ne recommençons pas la guerre de Troie \\
(Pouvoir des mots) (1\textsuperscript{er} au 15 avril 1937)}
\noindent \par
Nous vivons à une époque où la sécurité relative qu'apporte aux hommes une certaine domination technique sur la nature est largement compensée par les dangers de ruines et de massacres que suscitent les conflits entre grou­pements humains. Si le péril est si grave, c'est sans doute en partie à cause de la puissance des instruments de destruction que la technique a mis entre nos mains ; mais ces instruments ne partent pas tout seuls, et il n'est pas honnête de vouloir faire retomber sur la matière inerte une situation dont nous portons la pleine responsabilité. Les conflits les plus menaçants ont un caractère commun qui pourrait rassurer des esprits superficiels, mais qui, malgré l'apparence, en constitue le véritable danger ; {\itshape c'est qu'ils n'ont pas d'objectif définissable}. Tout au long de l'histoire humaine, on peut vérifier que les conflits sans comparaison les plus acharnés, sont ceux qui n'ont pas d'objectif. Ce paradoxe, une fois qu'on l'a aperçu clairement, est peut-être une des clefs de l'histoire ; il est sans doute la clef de notre époque.\par
Quand il y a lutte autour d'un enjeu bien défini, chacun peut peser ensemble la valeur de cet enjeu et les frais probables de la lutte, décider jusqu'où cela vaudra la peine de pousser l'effort ; il n'est même pas difficile en général de trouver un compromis qui vaille mieux, pour chacune des parties adverses, qu'une bataille même victorieuse. Mais quand une lutte n'a pas d'objectif, il n'y a plus de commune mesure, il n'y a plus de balance, plus de proportion, plus de comparaison possible ; un compromis n'est même pas concevable ; l'importance de la bataille se mesure alors uniquement aux sacri­fices qu'elle exige, et comme, de ce fait même, les sacrifices déjà accomplis appellent perpétuellement des sacrifices nouveaux, il n'y aurait aucune raison de s'arrêter de tuer et de mourir, si par bonheur les forces humaines ne finissaient par trouver leur limite. Ce paradoxe est si violent qu'il échappe à l'analyse. Pourtant, tous les hommes dits cultivés en connaissent l'exemple le plus parfait ; mais une sorte de fatalité nous fait lire sans comprendre.\par
Les Grecs et les Troyens s'entre-massacrèrent autrefois pendant dix ans à cause d'Hélène. Aucun d'entre eux, sauf le guerrier amateur Pâris, ne tenait si peu que ce fût à Hélène ; tous s'accordaient pour déplorer qu'elle fût jamais née. Sa personne était si évidemment hors de proportion avec cette gigan­tesque bataille qu'aux yeux de tous elle constituait simplement le symbole du véritable enjeu ; mais le véritable enjeu, personne ne le définissait et il ne pouvait être défini, car il n'existait pas. Aussi ne pouvait-on pas le mesurer. On en imaginait simplement l'importance par les morts accomplies et les massacres attendus. Dès lors cette importance dépassait toute limite assi­gnable. Hector pressentait que sa ville allait être détruite, son père et ses frères massacrés, sa femme dégradée par un esclavage pire que la mort ; Achille savait qu'il livrait son père aux misères et aux humiliations d'une vieillesse sans défense ; la masse des gens savait que leurs foyers seraient détruits par une absence si longue ; aucun n'estimait que c'était payer trop cher, parce que tous poursuivaient un néant dont la valeur se mesurait uniquement au prix qu'il fallait payer. Pour faire honte aux Grecs qui proposaient de retourner chacun chez soi, Minerve et Ulysse croyaient trouver un argument suffisant dans l'évocation des souffrances de leurs camarades morts. À trois mille ans de distance, on retrouve dans leur bouche et dans la bouche de Poincaré exactement la même argumentation pour flétrir les propositions de paix blanche. De nos jours, pour expliquer ce sombre acharnement à accumuler les ruines inutiles, l'imagination populaire a parfois recours aux intrigues supposées des congrégations économiques. Mais il n'y a pas lieu de chercher si loin. Les Grecs du temps d'Homère n'avaient pas de marchands d'airain organisés, ni de Comité de Forgerons. À vrai dire, dans l'esprit des contem­porains d'Homère, le rôle que nous attribuons aux mystérieuses oligarchies économiques était tenu par les dieux de la mythologie grecque. Mais pour acculer les hommes aux catastrophes les plus absurdes, il n'est besoin ni de dieux ni de conjurations secrètes. La nature humaine suffit.\par
Pour qui sait voir, il n'y a pas aujourd'hui de symptôme plus angoissant que le caractère irréel de la plupart des conflits qui se font jour. Ils ont encore moins de réalité que le conflit entre les Grecs et les Troyens. Au centre de la guerre de Troie, il y avait du moins une femme, et qui plus est une femme parfaitement belle. Pour nos contemporains, ce sont des mots ornés de majuscules qui jouent le rôle d'Hélène. Si nous saisissons, pour essayer de le serrer, un de ces mots tout gonflés de sang et de larmes, nous le trouvons sans contenu. Les mots qui ont un contenu et un sens ne sont pas meurtriers. Si parfois l'un d'eux est mêlé à une effusion de sang, c'est plutôt par accident que par fatalité, et il s'agit alors en général d'une action limitée et efficace. Mais qu'on donne des majuscules à des mots vides de signification, pour peu que les circonstances y poussent, les hommes verseront des flots de sang, amon­celleront ruines sur ruines en répétant ces mots, sans pouvoir jamais obtenir effectivement quelque chose qui leur corresponde ; rien de réel ne peut jamais leur correspondre, puisqu'ils ne veulent rien dire. Le succès se définit alors exclusivement par l'écrasement des groupes d'hommes qui se réclament de mots ennemis ; car c'est encore là un caractère de ces mots, qu'ils vivent par couples antagonistes. Bien entendu, ce n'est pas toujours par eux-mêmes que de tels mots sont vides de sens ; certains d'entre eux en auraient un, si on prenait la peine de les définir convenablement. Mais un mot ainsi défini perd sa majuscule, il ne peut plus servir de drapeau ni tenir sa place dans les cliquetis des mots d'ordre ennemis ; il n'est plus qu'une référence pour aider à saisir une réalité concrète, au un objectif concret, ou une méthode d'action. Éclaircir les notions, discréditer les mots congénitalement vides, définir l'usage des autres par des analyses précises, c'est là, si étrange que cela puisse paraître, un travail qui pourrait préserver des existences humaines.\par
Ce travail, notre époque y semble à peu près inapte. Notre civilisation couvre de son éclat une véritable décadence intellectuelle. Nous n'accordons à la superstition, dans notre esprit, aucune place réservée, analogue à la mytho­logie grecque, et la superstition se venge en envahissant sous le couvert d'un vocabulaire abstrait tout le domaine de la pensée. Notre science contient comme dans un magasin les mécanismes intellectuels les plus raffinés pour résoudre les problèmes les plus complexes, mais nous sommes presque incapables d'appliquer les méthodes élémentaires de la pensée raisonnable. En tout domaine nous semblons avoir perdu les notions essentielles de l'intelli­gence, les notions de limite, de mesure, de degré, de proportion, de relation, de rapport, de condition, de liaison nécessaire, de connexion entre moyens et résultats. Pour s'en tenir aux affaires humaines, notre univers politique est exclusivement peuplé de mythes et de monstres ; nous n'y connaissons que des entités, que des absolus. Tous les mots du vocabulaire politique et social pourraient servir d'exemple. Nation, sécurité, capitalisme, communisme, fas­cisme, ordre, autorité, propriété, démocratie, on pourrait les prendre tous les uns après les autres. Jamais nous ne les plaçons dans des formules telles que : Il y a démocratie dans la mesure où..., ou encore : Il y a capitalisme pour autant que... L'usage d'expressions du type « dans la mesure où » dépasse notre puissance intellectuelle. Chacun de ces mots semble représenter une réalité absolue, indépendante de toutes les conditions, ou un but absolu, indépendant de tous les modes d'action, ou encore un mal absolu ; et en même temps, sous chacun de ces mots nous mettons tour à tour ou même simulta­nément n'importe quoi. Nous vivons au milieu de réalités changeantes, diverses, déterminées par le jeu mouvant des nécessités extérieures, se transformant en fonction de certaines conditions et dans certaines limites ; mais nous agissons, nous luttons, nous sacrifions nous-mêmes et autrui en vertu d'abstractions cristallisées, isolées, impossibles à mettre en rapport entre elles ou avec les choses concrètes. Notre époque soi-disant technicienne ne sait que se battre contre les moulins à vent.\par
Aussi n'y a-t-il qu'à regarder autour de soi pour trouver des exemples d'absurdités meurtrières. L'exemple de choix, ce sont les antagonismes entre nations. On croit souvent les expliquer en disant qu'ils dissimulent simplement des antagonismes capitalistes ; mais on oublie un fait qui pourtant crève les yeux, c'est que le réseau de rivalités et de complexités, de luttes et d'alliances capitalistes qui s'étend sur le monde, ne correspond nullement à la division du monde en nations. Le jeu des intérêts peut opposer entre eux deux groupe­ments français, et unir chacun d'eux à un groupement allemand. L'industrie allemande de transformation peut être considérée avec hostilité par les entreprises françaises de mécanique ; mais il est à peu près indifférent aux compagnies minières que le fer de Lorraine soit transformé en France ou en Allemagne, et les vignerons, les fabricants d'articles de Paris et autres sont intéressés à la prospérité de l'industrie allemande. Ces vérités élémentaires rendent inintelligible l'explication courante des rivalités entre nations. Si l'on dit que le nationalisme recouvre toujours des appétits capitalistes, on devrait dire les appétits de qui. Des Houillères ? De la grosse métallurgie ? De la construction mécanique ? De l'Électricité ? Du Textile ? Des Banques ? Ce ne peut être tout cela ensemble, car les intérêts ne concordent pas ; et si on a en vue une fraction du capitalisme, encore faudrait-il expliquer pourquoi cette fraction s'est emparée de l'État. Il est vrai que la politique d'un État coïncide toujours à un moment donné avec les intérêts d'un secteur capitaliste quelconque ; on a ainsi une explication passe-partout qui du fait même de son insuffisance s'applique à n'importe quoi. Étant donné la circulation interna­tionale du capital, on ne voit pas non plus pourquoi un capitaliste recher­cherait plutôt la protection de son propre État que d'un État étranger, ou exercerait plus difficilement les moyens de pression et de séduction dont il dispose sur les hommes d'État étrangers que sur ses compatriotes. La structure de l'Économie mondiale ne correspond à la structure politique du monde que pour autant que les États exercent leur autorité en matière économique ; mais aussi le sens dans lequel s'exerce cette autorité ne peut pas s'expliquer par le simple jeu des intérêts économiques. Quand on examine le contenu du mot : intérêt national, on n'y trouve même pas l'intérêt des entreprises capitalistes. « On croit mourir pour la patrie, disait Anatole France ; on meurt pour des industriels. » Ce serait encore trop beau. On ne meurt même pas pour quelque chose d'aussi substantiel, d'aussi tangible qu'un industriel.\par
L'intérêt national ne peut se définir ni par un intérêt commun des grandes entreprises industrielles, commerciales ou bancaires d'un pays, car cet intérêt commun n'existe pas, ni par la vie, la liberté et le bien-être des citoyens, car on les adjure continuellement de sacrifier leur bien-être, leur liberté et leur vie à l'intérêt national. En fin de compte, si on examine l'histoire moderne, on arrive à la conclusion que l'intérêt national, c'est pour chaque État la capacité de faire la guerre. En 1911 la France a failli faire la guerre pour le Maroc ; mais pourquoi le Maroc était-il si important ? À cause de la réserve de chair à canon que devait constituer l'Afrique du Nord, à cause de l'intérêt qu'il y a pour un pays, du point de vue de la guerre, à rendre son économie aussi indépendante que possible par la possession de matières premières et de débouchés. Ce qu'un pays appelle intérêt économique vital, ce n'est pas ce qui permet à ses citoyens de vivre, c'est ce qui lui permet de faire la guerre ; le pétrole est bien plus propre à susciter les conflits internationaux que le blé. Ainsi, quand on fait la guerre, c'est pour conserver ou pour accroître les moyens de la faire. Toute la politique internationale roule autour de ce cercle vicieux. Ce qu'on nomme prestige national consiste à agir de manière à toujours donner l'impression aux autres pays qu'éventuellement on est sûr de les vaincre, afin de les démoraliser. Ce qu'on nomme sécurité nationale, c'est un état de choses chimérique où l'on conserverait la possibilité de faire la guerre en en privant tous les autres pays. Somme toute, une nation qui se respecte est prête à tout, y compris la guerre, plutôt que de renoncer à faire éventuellement la guerre. Mais pourquoi faut-il pouvoir faire la guerre ? On ne le sait pas plus que les Troyens ne savaient pourquoi ils devaient garder Hélène. C'est pour cela que la bonne volonté des hommes d'État amis de la paix est si peu efficace. Si les pays étaient divisés par des oppositions réelles d'intérêts, an pourrait trouver des compromis satisfaisants. Mais quand les intérêts économiques et politiques n'ont de sens qu'en vue de la guerre, comment les concilier d'une manière pacifique ? C'est la notion même de nation qu'il faudrait supprimer. Ou plutôt c'est l'usage de ce mot : car le mot national et les expressions dont il fait partie sont vides de toute signification, ils n'ont pour contenu que les millions de cadavres, les orphelins, les mutilés, le désespoir, les larmes.\par
Un autre exemple admirable d'absurdité sanglante, c'est l'opposition entre fascisme et communisme. Le fait que cette opposition détermine aujourd'hui pour nous une double menace de guerre civile et de guerre mondiale est peut-être le symptôme de carence intellectuelle le plus grave parmi tous ceux que nous pouvons constater autour de nous. Car si on examine le sens qu'ont aujourd'hui ces deux termes, on trouve deux conceptions politiques et sociales presque identiques. De part et d'autre, c'est la même mainmise de l'État sur presque toutes les formes de vie individuelle et sociale ; la même militari­sation forcenée ; la même unanimité artificielle, obtenue par la contrainte, au profit d'un parti unique qui se confond avec l'État et se définit par cette confusion ; le même régime de servage imposé par l'État aux masses labo­rieuses à la place du salariat classique. Il n'y a pas deux nations dont la structure soit plus semblable que l'Allemagne et la Russie, qui se menacent mutuellement d'une croisade internationale et feignent chacune de prendre l'autre pour la Bête de l'Apocalypse. C'est pourquoi on peut affirmer sans crainte que l'opposition entre fascisme et communisme n'a rigoureusement aucun sens. Aussi la victoire du fascisme ne peut-elle se définir que par l'extermination des communistes, et la victoire du communisme que par l'extermination des fascistes. Il va de soi que dans ces conditions, l'anti­fascisme et l'anticommunisme sont eux aussi dépourvus de sens. La position des antifascistes, c'est : Tout plutôt que le fascisme ; tout, y compris le fascisme sous le nom de communisme. La position des anticommunistes, c'est : Tout plutôt que le communisme ; tout, y compris le communisme sous le nom de fascisme. Pour cette belle cause, chacun, dans les deux camps, est résigné d'avance à mourir, et surtout à tuer. Pendant l'été de 1932, à Berlin, il se formait fréquemment dans la rue un petit attroupement autour de deux ouvriers ou petits bourgeois, l'un communiste, l'autre nazi, qui discutaient ensemble ; ils constataient toujours au bout d'un temps donné qu'ils défen­daient rigoureusement le même programme, et cette constatation leur donnait le vertige, mais augmentait encore chez chacun d'eux la haine contre un adversaire si essentiellement ennemi qu'il restait ennemi en exposant les mêmes idées. Depuis, quatre années et demie se sont écoulées ; les commu­nistes allemands sont encore torturés par les nazis dans les camps de concentration, et il n'est pas sûr que la France ne soit pas menacée d'une guerre d'extermination entre antifascistes et anticommunistes. Si une telle guerre avait lieu, la guerre de Troie serait un modèle de bon sens en compa­raison ; car même si on admet avec un poète grec qu'il y avait seulement à Troie le fantôme d'Hélène, le fantôme d'Hélène est encore une réalité substantielle à côté de l'opposition entre fascisme et communisme.\par
L'opposition entre dictature et démocratie, qui s'apparente à celle entre ordre et liberté, est, elle au moins, une opposition véritable. Cependant elle perd son sens si on en considère chaque terme comme une entité, ce qu'on fait le plus souvent de nos jours, au lieu de le prendre comme une référence permettant de mesurer les caractéristiques d'une structure sociale. Il est clair qu'il n'y a nulle part ni dictature absolue, ni démocratie absolue, mais que l'organisme social est toujours et partout un composé de démocratie et de dictature, avec des degrés différents ; il est clair aussi que le degré de la démocratie se définit par les rapports qui lient les différents rouages de la machine sociale, et dépend des conditions qui déterminent le fonctionnement de cette machine ; c'est donc sur ces rapports et sur ces conditions, qu'il faut essayer d'agir. Au lieu de quoi on considère en général qu'il v a des groupe­ments humains, nations ou partis, qui incarnent intrinsèquement la dictature ou la démocratie, de sorte que selon qu'on est porté par tempérament à tenir surtout à l'ordre ou surtout à la liberté, on est obsédé du désir d'écraser les uns ou les autres de ces groupements. Beaucoup de Français croient de bonne foi par exemple qu'une victoire militaire de la France sur l'Allemagne serait une victoire de la démocratie. À leurs yeux, la liberté réside dans la nation française et la tyrannie dans la nation allemande, à peu près comme pour les contemporains de Molière une vertu dormitive résidait dans l'opium. Si un jour les nécessités dites « de la défense nationale » font de la France un camp retranché où toute la nation soit entièrement soumise à l'autorité militaire, et si la France ainsi transformée entre en guerre avec l'Allemagne, ces Français se feront tuer, non sans avoir tué le plus possible d'Allemands, avec l'illusion touchante de verser leur sang pour la démocratie. Il ne leur vient pas à l'esprit que la dictature a pu s'installer en Allemagne à la faveur d'une situation déterminée ; et que susciter pour l'Allemagne une autre situation qui rende possible un certain relâchement de l'autorité étatique serait peut-être plus efficace que de tuer les petits gars de Berlin et de Hambourg.\par
Pour prendre un autre exemple, si on ose exposer devant un homme de parti l'idée d'un armistice en Espagne, il répondra avec indignation, si c'est un homme de droite, qu'il faut lutter jusqu'au bout pour la victoire de l'ordre et l'écrasement des fauteurs d'anarchie ; il répondra avec non moins d'indi­gnation, si c'est un homme de gauche, qu'il faut lutter jusqu'au bout pour la liberté du peuple, pour le bien-être des masses laborieuses, pour l'écrasement des oppresseurs et des exploiteurs. Le premier oublie qu'aucun régime politique, quel qu'il soit, ne comporte de désordres qui puissent égaler de loin ceux de la guerre civile, avec les destructions systématiques, les massacres en série sur la ligne de feu, le relâchement de la production, les centaines de crimes individuels commis quotidiennement dans les deux camps du fait que n'importe quel voyou a un fusil en main. L'homme de gauche oublie de son côté que, même dans le camp des siens, les nécessités de la guerre civile, l'état de siège, la militarisation du front et de l'arrière, la terreur policière, la suppression de toute limitation à l'arbitraire, de toute garantie individuelle, suppriment la liberté bien plus radicalement que ne ferait l'accession au pouvoir d'un parti d'extrême droite ; il oublie que les dépenses de guerre, les ruines, le ralentissement de la production condamnent le peuple, et pour longtemps, à des privations bien plus cruelles que ne feraient ses exploiteurs. L'homme de droite et l'homme de gauche oublient tous deux que de longs mois de guerre civile ont peu à peu amené dans les deux camps un régime presque identique. Chacun des deux a perdu son idéal sans s'en apercevoir, en lui substituant une entité vide ; pour chacun des deux, la victoire de ce qu'il nomme encore son idée ne peut plus se définir que par l'extermination de l'adversaire ; et chacun des deux, si on lui parle de paix, répondra avec mépris par l'argument massue, l'argument de Minerve dans Homère, l'argument de Poincaré en 1917 : « Les morts ne le veulent pas. »\par

\begin{center}
*\end{center}
\noindent Ce qu'on nomme de nos jours, d'un terme qui demanderait des précisions, la {\itshape lutte des classes}, c'est de tous les conflits qui opposent des groupements humains le mieux fondé, le plus sérieux, on pourrait peut-être dire le seul sérieux ; mais seulement dans la mesure où n'interviennent pas là des entités imaginaires qui empêchent toute action dirigée, font porter les efforts dans le vide, et entraînent le danger de haines inexpiables, de folles destructions, de tueries insensées. Ce qui est légitime, vital, essentiel, c'est la lutte éternelle de ceux qui obéissent contre ceux qui commandent, lorsque le mécanisme du pouvoir social entraîne l'écrasement de la dignité humaine chez ceux d'en bas. Cette lutte est éternelle parce que ceux qui commandent tendent toujours, qu'ils le sachent ou non, à fouler aux pieds la dignité humaine au-dessous d'eux. La fonction de commandement, pour autant qu'elle s'exerce, ne peut pas, sauf cas particuliers, respecter l'humanité dans la personne des agents d'exécution. Si elle s'exerce comme si les hommes étaient des choses, et encore sans aucune résistance, elle s'exerce inévitablement sur des choses exceptionnellement ductiles ; car l'homme soumis à la menace de mort, qui est en dernière analyse la sanction suprême de toute autorité, peut devenir plus maniable que la matière inerte. Aussi longtemps qu'il y aura une hiérarchie sociale stable, quelle qu'en puisse être la forme, ceux d'en bas devront lutter pour ne pas perdre tous les droits d'un être humain. D'autre part la résistance de ceux d'en haut, si elle apparaît d'ordinaire comme contraire à la justice, repose elle aussi sur des motifs concrets. D'abord des motifs personnels ; sauf le cas d'une générosité assez rare, les privilégiés répugnent à perdre une part de leurs privilèges matériels ou moraux. Mais aussi des motifs plus élevés. Ceux qui sont investis des fonctions de commandement se sentent la mission de défendre l'ordre indispensable à toute vie sociale, et ils ne conçoivent pas d'autre ordre passible que celui qui existe. Ils n'ont pas entièrement tort, car jusqu'à ce qu'un autre ordre ait été en fait établi, on ne peut affirmer avec certitude qu'il sera possible ; c'est justement pourquoi il ne peut y avoir progrès social que si la pression d'en bas est suffisante pour changer effectivement les rapports de force, et contraindre ainsi à établir en fait des relations sociales nouvelles. La rencontre entre la pression d'en bas et la résistance d'en haut suscite ainsi continuellement un équilibre instable, qui définit à chaque instant la structure d'une société. Cette rencontre est une lutte, mais elle n'est pas une guerre ; elle peut se transformer en guerre dans certaines circonstances, mais il n'y a là aucune fatalité. L'Antiquité ne nous a pas seulement légué l'histoire des massacres interminables et inutiles autour de Troie, elle nous a laissé également l'histoire de l'action énergique et unanime par laquelle les plébéiens de Rome, sans verser une goutte de sang, sont sortis d'une condition qui touchait à l'esclavage et ont obtenu comme garantie de leurs droits nouveaux l'institution des tribuns. C'est exactement de la même manière que les ouvriers français, par l'occupation des usines, mais sans violences, ont imposé la reconnaissance de quelques droits élémentaires, et comme garantie de ces droits l'institution de délégués élus.\par

\begin{center}
*\end{center}
\noindent La Rome primitive avait pourtant sur la France moderne un sérieux avantage. Elle ne connaissait en matière sociale ni abstractions, ni entités, ni mots à majuscule, ni mots en isme ; rien de ce qui risque chez nous d'annuler les efforts les plus soutenus, ou de faire dégénérer la lutte sociale en une guerre aussi ruineuse, aussi sanglante, aussi absurde de n'importe quel point de vue que la guerre entre nations. On peut prendre presque tous les termes, toutes les expressions de notre vocabulaire politique, et les ouvrir ; au centre on trouvera le vide. Que peut bien vouloir dire par exemple le mot d'ordre, si populaire pendant les élections, de « lutte contre les trusts » ? Un trust, c'est un monopole économique placé aux mains de puissances d'argent, et dont elles usent non pas au mieux de l'intérêt public, mais de manière à accroître leur pouvoir. Qu'est-ce qu'il y a de mauvais là-dedans ? C'est le fait qu'un monopole sert d'instrument à une volonté de puissance étrangère au bien public. Or, ce n'est pas ce fait qu'on cherche à supprimer, mais le fait, indiffé­rent en lui-même, que cette volonté de puissance est celle d'une oligarchie économique. On propose de substituer à ces oligarchies l'État, qui a lui aussi sa volonté de puissance tout aussi étrangère au bien public ; encore s'agit-il pour l'État de puissance non plus économique mais militaire, et par suite bien plus dangereuse pour les braves gens qui aiment à vivre. Réciproquement, du côté bourgeois, que peut-on bien entendre par l'hostilité à l'étatisme écono­mique, alors qu'on admet les monopoles privés, qui comportent tous les inconvénients économiques et techniques des monopoles d'État, et peut-être d'autres encore ? On pourrait faire une longue liste de mots d'ordre ainsi groupés deux par deux, et également illusoires. Ceux-là sont relativement inoffensifs, mais ce n'est pas le cas pour tous.\par

\begin{center}
*\end{center}
\noindent Ainsi que peuvent bien avoir dans l'esprit ceux pour qui le mot « capitalisme » représente le mal absolu ? Nous vivons dans une société qui comporte des formes de contrainte et d'oppression trop souvent écrasantes pour les masses d'êtres humains qui les subissent, des inégalités très doulou­reuses, quantité de tortures inutiles. D'autre part, cette société se caractérise, du point de vue économique, par certains modes de production, de consom­mation, d'échange, qui sont d'ailleurs en transformation perpétuelle et qui dépendent de quelques rapports fondamentaux entre la production et la circulation des marchandises, entre la circulation des marchandises et la monnaie, entre la monnaie et la production, entre la monnaie et la consom­mation. Cet ensemble de phénomènes économiques divers et changeants, on le cristallise arbitrairement en une abstraction impossible à définir, et on rapporte à cette abstraction, sous le nom de capitalisme, toutes les souffrances qu'on subit ou qu'on constate autour de soi. À partir de là, il suffit qu'un homme ait du caractère pour qu'il dévoue sa vie à la destruction du capita­lisme, ou, ce qui revient au même, à la révolution ; car ce mot de révolution n'a aujourd'hui que cette signification purement négative.\par
Comme la destruction du capitalisme n'a aucun sens, du fait que le capitalisme est une abstraction, comme elle n'implique pas un certain nombre de modifications précises apportées au régime - de telles modifications sont traitées dédaigneusement de « réformes » - elle peut seulement signifier l'écra­sement des capitalistes et plus généralement de tous ceux qui ne se déclarent pas contre le capitalisme. Il est apparemment plus facile de tuer, et même de mourir, que de se poser quelques questions bien simples, telles que celles-ci : les lois, les conventions, qui régissent actuellement la vie économique, forment-elles un système ? Dans quelle mesure y a-t-il connexion nécessaire entre tel ou tel phénomène économique et les autres ? Jusqu'à quel point la modification de telle ou telle de ces lois économiques se répercuterait-elle sur les autres ? Dans quelle mesure les souffrances imposées par les rapports sociaux de notre époque dépendent-elles de telle ou telle convention de notre vie économique ; dans quelle mesure de l'ensemble de toutes ces conven­tions ? Dans quelle mesure ont-elles pour causes d'autres facteurs, soit des facteurs durables qui persisteraient après la transformation de notre organi­sation économique, soit au contraire des facteurs qu'on pourrait supprimer sans mettre fin à ce qu'on nomme le régime ? Quelles souffrances nouvelles, soit passagères, soit permanentes, impliquerait nécessairement la méthode à mettre en œuvre pour une telle transformation ? Quelles souffrances nouvelles risquerait d'apporter la nouvelle organisation sociale que l'on instituerait ? Si l'on étudiait sérieusement ces problèmes, on pourrait peut-être arriver à avoir quelque chose dans l'esprit quand on dit que le capitalisme est un mal ; mais il ne s'agirait que d'un mal relatif, et une transformation du régime social ne pourrait être proposée qu'en vue de parvenir à un moindre mal. Encore ne devrait-il s'agir que d'une transformation déterminée.\par

\begin{center}
*\end{center}
\noindent Toute cette critique pourrait tout aussi bien s'appliquer à l'autre camp, en remplaçant la préoccupation des souffrances infligées aux couches sociales d'en bas par le souci de l'ordre à sauvegarder, et le désir de transformation par le désir de conservation. Les bourgeois assimilent volontiers à des facteurs de désordre tous ceux qui envisagent la fin du capitalisme, et même parfois ceux qui désirent le réformer, parce qu'ils ignorent dans quelle mesure et en fonction de quelles circonstances les divers rapports économiques dont l'ensemble forme ce qu'on appelle actuellement capitalisme peuvent être con­sidérés comme des conditions de l'ordre. Beaucoup d'entre eux, ne sachant pas quelle modification peut être ou non dangereuse, préfèrent tout conserver, sans se rendre compte que la conservation parmi des circonstances changean­tes constitue elle-même une modification dont les conséquences peuvent être des désordres. La plupart invoquent les lois économiques aussi religieusement que s'il s'agissait des lois non écrites d'Antigone, alors qu'ils les voient quotidiennement changer sous leurs yeux. Pour eux aussi, la conservation du régime capitaliste est une expression vide de sens, puisqu'ils ignorent ce qu'il faut conserver, sous quelles conditions, dans quelle mesure ; elle ne peut signifier pratiquement que l'écrasement de tous ceux qui parlent de la fin du régime. La lutte entre adversaires et défenseurs du capitalisme, cette lutte entre novateurs qui ne savent pas quoi innover et conservateurs qui ne savent pas quoi conserver, est une lutte aveugle d'aveugles, une lutte dans le vide, et qui pour cette raison même risque de tourner en extermination. On peut faire les mêmes remarques pour la lutte qui se déroule dans le cadre plus restreint des entreprises industrielles. Un ouvrier, en général rapporte instinctivement au patron toutes les souffrances qu'il subit dans l'usine ; il ne se demande pas si dans tout autre système de propriété la direction de l'entreprise ne lui infligerait pas encore une partie des mêmes souffrances, ou bien peut-être des souffrances identiques, ou peut-être même des souffrances accrues ; il ne se demande pas non plus quelle part de ces souffrances on pourrait supprimer, en en faisant disparaître les causes, sans toucher au système de propriété actuel. Pour lui, la lutte « contre le patron » se confond avec la protestation irrépres­sible de l'être humain écrasé par une vie trop dure. Le patron, de son côté, se préoccupe avec raison de son autorité. Seulement le rôle de l'autorité patronale consiste exclusivement à indiquer les fabrications, coordonner au mieux les travaux partiels, contrôler, en recourant à une certaine contrainte, la bonne exécution du travail ; tout régime de l'entreprise, quel qu'il soit, ou cette coordination et ce contrôle peuvent être convenablement assurés, accorde une part suffisante à l'autorité patronale. Pour le patron, cependant, le sentiment qu'il a de son autorité dépend avant tout d'une certaine atmosphère de soumis­sion et de respect qui n'a pas nécessairement de rapport avec la bonne exécution du travail ; et surtout, quand il s'aperçoit d'une révolte latente ou ouverte parmi son personnel, il l'attribue toujours à quelques individus, alors qu'en réalité la révolte, soit bruyante, soit silencieuse, agressive ou refoulée par le désespoir, est inséparable de toute existence physiquement ou morale­ment accablante. Si, pour l'ouvrier, la lutte « contre le patron » se confond avec le sentiment de la dignité, pour le patron la lutte contre les « meneurs » se confond avec le souci de sa fonction et la conscience professionnelle ; dans les deux cas il s'agit d'efforts à vide, et qui par suite ne sont pas susceptibles d'être renfermés dans une limite raisonnable. Alors qu’on constate que les grèves qui se déroulent autour de revendications déterminées aboutissent sans trop de mal à un arrangement, on a pu voir des grèves qui ressemblaient à des guerres en ce sens que ni d'un côté ni de l'autre la lutte n'avait d'objectif ; des grèves où l'on ne pouvait apercevoir rien de réel ni de tangible, rien, excepté l'arrêt de la production, la détérioration des machines, la misère, la faim, les larmes des femmes, la sous-alimentation des enfants ; et l'acharnement de part et d'autre était tel qu'elles donnaient l'impression de ne jamais devoir finir. Dans de pareils événements, la guerre civile existe déjà en germe.\par

\begin{center}
*\end{center}
\noindent Si l'on analysait de cette manière tous les mots, toutes les formules qui ont ainsi suscité, au long de l'histoire humaine, l'esprit de sacrifice et la cruauté tout ensemble, on les trouverait tous sans doute pareillement vides. Pourtant, toutes ces entités avides de sang humain doivent bien avoir un rapport quelconque avec la vie réelle. Elles en ont un en effet. Il n'y avait peut-être à Troie que le fantôme d'Hélène, mais l'armée grecque et l'armée troyenne n'étaient pas des fantômes ; de même, si le mot de nation et les expressions dont il fait partie sont vides de sens, les différents États, avec leurs bureaux, leurs prisons, leurs arsenaux, leurs casernes, leurs douanes, sont bien réels. La distinction théorique entre les deux formes de régime totalitaire, fascisme et communisme, est imaginaire, mais en Allemagne, en 1932, il existait bien effectivement deux organisations politiques dont chacune aspirait au pouvoir total et par suite à l'élimination de l'autre. Un parti démocratique peut devenir peu à peu un parti de dictature, mais il n'en reste pas moins distinct du parti dictatorial qu'il s'efforce d'écraser ; la France peut, en vue de se défendre contre l'Allemagne, se soumettre à son tour à un régime totalitaire, l'État français et l'État allemand resteront néanmoins deux États distincts. Destruc­tion et conservation du capitalisme sont des mots d'ordre sans contenu, mais des organisations sont groupées derrière ces mots d'ordre. À chaque abstrac­tion vide correspond un groupement humain. Les abstractions dont ce n'est pas le cas restent inoffensives ; réciproquement les groupements qui n'ont pas sécrété d'entités ont des chances de n'être pas dangereux. Jules Romains a magnifiquement représenté cette espèce particulière de sécrétion quand il a mis dans la bouche de Knock la formule : « Au-dessus de l'intérêt du malade et de l'intérêt du médecin, il y a l'intérêt de la médecine. » C'est là un mot de comédie, simplement parce qu'il n'est pas sorti encore des syndicats des médecins une entité de ce genre ; de pareilles entités procèdent toujours d'organismes qui ont pour caractère commun de détenir un pouvoir ou de viser au pouvoir. Toutes les absurdités qui font ressembler l'histoire à un long délire ont leur racine dans une absurdité essentielle, la nature du pouvoir. La nécessité qu'il y ait un pouvoir est tangible, palpable, parce que l'ordre est indispensable à l'existence ; mais l'attribution du pouvoir est arbitraire, parce que les hommes sont semblables ou peu s'en faut ; or elle ne doit pas apparaître comme arbitraire, sans quoi il n'y a plus de pouvoir. Le prestige, c'est-à-dire l'illusion, est ainsi au cœur même du pouvoir. Tout pouvoir repose sur des rapports entre les activités humaines ; mais un pouvoir, pour être stable, doit apparaître comme quelque chose d'absolu, d'intangible, à ceux qui le détiennent, à ceux qui le subissent, aux pouvoirs extérieurs. Les conditions de l'ordre sont essentiellement contradictoires, et les hommes semblent avoir le choix entre l'anarchie qui accompagne les pouvoirs faibles et les guerres de toutes sortes suscitées par le souci du prestige.\par
Traduites dans le langage du pouvoir, les absurdités énumérées ici cessent d'apparaître comme telles. N'est-il pas naturel que chaque État définisse l'intérêt national par la capacité de faire la guerre, puisqu'il est entouré d'autres États capables, s'ils le voient faible, de le subjuguer par les armes ? On ne voit pas de milieu entre tenir sa place dans la course à la préparation de la guerre, ou être prêts à subir n'importe quoi de la part d'autres États armés. Le désarmement général ne supprimerait cette difficulté que s'il était complet, ce qui est à peine concevable. D'autre part un État ne peut pas paraître faible devant l'étranger sans risquer de donner aussi à ceux qui lui obéissent la tentation de secouer un peu son autorité. Si Priam et Hector avaient rendu Hélène aux Grecs, ils auraient risqué de leur inspirer d'autant plus le désir de saccager une ville apparemment si mal préparée à se défendre ; ils auraient risqué aussi un soulèvement général à Troie ; non pas parce que la restitution d'Hélène aurait indigné les Troyens, mais parce qu'elle leur aurait donné à penser que les hommes auxquels ils obéissaient n'étaient pas tellement puissants. Si en Espagne l'un des deux camps donnait l'impression de désirer la paix, d'abord il encouragerait les ennemis, il en augmenterait la valeur offensive ; et puis il risquerait des soulèvements parmi les siens. De même, pour un homme qui n'est engagé ni dans le bloc anticommuniste ni dans le bloc antifasciste, le heurt de deux idéologies presque identiques peut paraître ridicule ; mais dès lors que ces blocs existent, ceux qui se trouvent dans l'un des deux considèrent nécessairement l'autre comme le mal absolu, parce qu'il les écrasera s'ils ne sont pas les plus forts ; les chefs doivent de part et d'autre paraître prêts à écraser l'ennemi pour conserver leur autorité sur leurs troupes ; et quand ces blocs ont atteint une certaine puissance, la neutralité devient une position pratiquement presque intenable. De même encore lorsque dans une hiérarchie sociale quelconque ceux d'en bas craignent d'être totalement écrasés s'ils ne dépossèdent pas leurs supérieurs, et si les uns ou les autres deviennent alors assez forts pour n'avoir plus à craindre, ils ne résistent pas à l'ivresse de la puissance stimulée par la rancune. D'une manière générale, tout pouvoir est essentiellement fragile ; il doit donc se défendre, sans quoi comment y aurait-il dans la vie sociale un minimum de stabilité ? Mais l'offensive apparaît pres­que toujours, à tort ou à raison, comme l'unique tactique défensive, et cela de tous côtés. Il est naturel d'ailleurs que ce soient surtout les différends imagi­naires qui suscitent des conflits inexpiables, parce qu'ils se posent uniquement sur le plan du pouvoir et du prestige. Il est peut-être plus facile à la France d'accorder à l'Allemagne des matières premières que quelques arpents de terre baptisés colonie, plus facile à l'Allemagne de se passer de matières premières que du mot de colonie. La contradiction essentielle à la société humaine, c'est que toute situation sociale repose sur un équilibre de forces, un équilibre de pressions analogue à l'équilibre des fluides ; mais les prestiges, eux, ne s'équilibrent pas, le prestige ne comporte pas de limites, toute satisfaction de prestige est une atteinte au prestige ou à la dignité d'autrui. Or le prestige est inséparable du pouvoir. Il semble qu'il y ait là une impasse dont l'humanité ne puisse sortir que par miracle. Mais la vie humaine est faite de miracles. Qui croirait qu'une cathédrale gothique pût tenir debout, si on ne le constatait tous les jours ? Puisque en fait il n'y a pas toujours guerre, il n'y a pas impossibilité à ce qu'il y ait indéfiniment la paix. Un problème posé avec toutes ses données réelles est bien près d'être résolu. On n'a encore jamais posé ainsi le problème de la paix internationale et civile.\par

\begin{center}
*\end{center}
\noindent C'est le nuage des entités vides qui empêche non seulement d'apercevoir les données du problème, mais même de sentir qu'il y a un problème à résoudre et non une fatalité à subir. Elles stupéfient les esprits ; non seulement elles font mourir, mais, ce qui est infiniment plus grave, elles font oublier la valeur de la vie. La chasse aux entités dans tous les domaines de la vie politique et sociale est une œuvre urgente de salubrité publique. Ce n'est pas une chasse facile ; toute l'atmosphère intellectuelle de notre époque favorise la floraison et la multiplication des entités. On peut se demander si en réformant les méthodes d'enseignement et de vulgarisation scientifique, et en chassant la superstition grossière qui s'y est installée à la faveur d'un vocabulaire artifi­ciel, en rendant aux esprits le bon usage des locutions du type {\itshape dans la mesure où, pour autant que, à condition que, par rapport à}, en discréditant tous les raisonnements vicieux qui reviennent à faire admettre qu'il y a dans l'opium une vertu dormitive, on ne rendrait pas â. nos contemporains un service prati­que de premier ordre. Une élévation générale du niveau intellectuel favorise­rait singulièrement tout effort d'éclaircissement pour dégonfler les causes imaginaires de conflit. Certes nous ne manquons pas de gens pour prêcher l'apaisement dans tous les domaines ; mais en général ces sermons ont pour objet non d'éveiller les intelligences et d'éliminer les faux conflits, mais d'endormir et d'étouffer les conflits réels. Les beaux parleurs qui, en décla­mant sur la paix internationale, comprennent par cette expression le maintien indéfini du {\itshape statu qu}o au profit exclusif de l'État français, ceux qui, en recommandant la paix sociale, entendent conserver les privilèges intacts ou du moins subordonner toute modification au bon vouloir des privilégiés, ceux-là sont les ennemis les plus dangereux de la paix internationale et civile. Il ne s'agit pas d'immobiliser artificiellement des rapports de force essentiellement variables, et que ceux qui souffrent chercheront toujours à faire varier ; il s'agit de discriminer l'imaginaire et le réel pour diminuer les risques de guerre sans renoncer à la lutte, dont Héraclite disait qu'elle est la condition de la vie.\par
({\itshape Nouveaux Cahiers}, 1\textsuperscript{re} année, n° 2-3, 1\textsuperscript{er}-15 avril 1937)\par

\begin{center}
\end{center}
\subsubsection[9. L’Europe en guerre pour la Tchécoslovaquie ? (25 mai 1938)]{9. \\
L’Europe en guerre pour la Tchécoslovaquie ? \\
(25 mai 1938)}
\noindent \par
À l'égard du problème tchécoslovaque, beaucoup commettent la faute de ne pas regarder en face comment il se posera s'il prend la forme la plus aiguë. Pour ne céder à aucun affolement, il est nécessaire d'élaborer aussi lucidement que possible un mode d'action pour le pire comme pour le meilleur des cas. Ce qui suit se rapporte au pire des cas, c'est-à-dire au cas où Hitler, pour des raisons intérieures et extérieures, serait résolu à obtenir un succès frappant et décisif en Europe Centrale.\par
Toute question internationale peut être considérée sous quatre aspects, d'ailleurs souvent liés ; le droit pris comme tel ; le rapport des forces et leur équilibre ; les engagements pris par la France ; les chances de guerre et de paix. À aucun de ces points de vue, le maintien de l'État Tchécoslovaque tel qu'il existe actuellement, ne paraît avoir l'importance qu'on lui attribue.\par

\begin{center}
*\end{center}
\noindent Au point de vue du droit, la Tchécoslovaquie a bien reçu des morceaux de territoire allemand, et il ne semble pas contestable que la population alleman­de y soit brimée à quelque degré. On peut discuter à quel degré. Il est difficile de faire de ces territoires disséminés une province séparée jouissant d'une pleine autonomie dans le cadre de l'État Tchécoslovaque ; en revanche, com­me ils forment une frange aux confins de la frontière allemande et de l'ancienne frontière autrichienne, il semble facile à l'Allemagne nouvellement agrandie de les annexer purement et simplement pas une rectification de frontières.\par
On peut se demander si l'Allemagne veut s'emparer aussi de territoires tchèques. Il est vraisemblable que la rectification de frontières lui suffirait, surtout si une démarche simultanée de la France et de l'Angleterre se faisait à Berlin et à Prague, acceptant une telle rectification et interdisant toute entre­prise plus ambi­tieuse. Car, tout d'abord, Hitler a toujours proclamé qu'il veut, en Europe, les territoires allemands et rien d'autre. De plus, les territoires de population allemande contiennent, d'une part, une bonne partie des ressources industrielles de la Tchécoslovaquie, d'autre part les massifs montagneux qui la défendent. L'annexion de ces territoires par l'Allemagne mettrait la Tchécos­lovaquie à sa merci ; de sorte que l'Allemagne n'aurait nul besoin d'attenter à son indépendance pour réaliser, en ce qui la concerne, tous ses objectifs diplomatiques, économiques et militaires. Une espèce de protectorat répon­drait bien mieux à la politique générale d'Hitler que l'annexion du territoire tchèque. Bien plus, il est probable qu'un simple changement d'orientation diplomatique, de la part de la Tchécoslovaquie, suffirait à éliminer tout pro­blème de minorité. L'essentiel, pour Hitler, c'est que la Tchécoslovaquie devienne, démembrée ou non, un État satellite de l'Allemagne.\par
Quels seraient les inconvénients de cette situation ? On peut considérer que cette dépendance où serait jetée la Tchécoslovaquie à l'égard de l'Allema­gne est quelque chose d'injuste. Sans doute ; mais le {\itshape statu quo}, d'autre part, est une injustice infligée aux Sudètes ; cela prouve simplement que le droit des peuples à disposer d'eux-mêmes rencontre un obstacle dans la nature des choses, du fait que les trois cartes physique, économique et ethnographique de l'Europe ne coïncident pas.\par
La Tchécoslovaquie peut fort bien, soit parce qu'affaiblie par l'ablation de ses territoires allemands, soit pour éviter une telle ablation, devenir un satellite de l'Allemagne sans devoir sacrifier sa culture, sa langue ou ses caractéris­tiques nationales ; ce qui limite l'injustice. L'idéologie national-socialiste est purement raciste ; elle n'a d'universel que l'anticommunisme et l'antisémi­tisme. Les Tchèques peuvent interdire le parti communiste et exclure les juifs des fonctions quelque peu importantes, sans perdre quoi que ce soit de leur vie nationale. Bref, injustice pour injustice, puisqu'il doit y en avoir une de toutes manières, choisissons celle qui risque le moins d'amener une guerre.\par
Au reste, l'injustice devrait-elle être plus grande, n'y a-t-il pas une amère ironie à ce que la France revête son armure comme redresseur de torts ? En empêchant l'{\itshape Anschluss} pendant vingt ans, elle a attenté elle-même, de la manière la plus flagrante, au fameux droit des peuples à disposer d'eux-mê­mes. Et Dieu sait qu'elle ne manque pas, en Afrique et en Asie, de peuples à émanciper sans risques de guerre, si les droits des peuples l'intéressent.\par
Il est vrai que la satisfaction des revendications de l'Allemagne en Tchécoslovaquie ferait tomber toute l'Europe centrale sous son influence. Ceci nous amène à un autre point de vue, celui du rapport des forces. Il n'est plus question de droit.\par
\par

\begin{center}
*\end{center}
\noindent Il est possible que la volonté d'Hitler soit tendue vers un but : établir une hégémonie allemande en Europe, par la guerre s'il le faut, sans guerre si possible. La France, par tradition, n'admet pas d'hégémonie en Europe, sinon la sienne propre quand elle peut l'établir. Aujourd'hui, entre la France victo­rieuse il y a vingt ans et l'Allemagne en pleine convalescence, si l'on peut dire, il existe un espèce d'équilibre instable. Faut-il tendre à maintenir cet équilibre ? À rétablir l'hégémonie française ?\par
Il est évident, si l'on y réfléchit, que le grand principe de « l'équilibre européen » est un principe de guerre. En vertu de ce principe, un pays se sent privé de sécurité, placé dans une situation intolérable, dès qu'il est le plus faible par rapport à un adversaire possible. Or, comme il n'existe pas de balance pour mesurer la force des États, un pays ou un bloc de pays n'a qu'un moyen de ne pas être le plus faible : c'est d'être le plus fort. Quand deux pays ou deux blocs ressentent chacun le besoin impérieux d'être le plus fort, on peut prédire la guerre sans risquer de se tromper.\par
S'il doit y avoir une hégémonie au centre de l'Europe, il est dans la nature des choses que ce soit une hégémonie allemande. La force est du côté de l'Allemagne. En 1918, elle n'a été vaincue que tout juste, et par une coalition formidable. Au reste, pourquoi une hégémonie allemande est-elle une éven­tualité pire qu'une hégémonie française ? L'Allemagne est « totalitaire », il est vrai. Mais les régimes politiques sont instables ; dans trente ans, de la France et de l'Allemagne, qui peut dire laquelle sera une dictature, laquelle une démocratie ? En ce moment, une hégémonie allemande serait étouffante. Mais pourrait-elle l'être plus, je ne dis même pas qu'une guerre, mais que la paix, avec la tension nerveuse, affolante, l'esprit d'état de siège, l'appauvrissement matériel et moral que nous y subissons de plus en plus ? En admettant que la France possède encore une culture, des traditions, un esprit qui lui soit propre, un idéal généreux, son rayonnement spirituel pourra être bien plus grand si elle abandonne l'Europe centrale à l'influence allemande que si elle s'obstine à lutter contre une évolution difficilement évitable. D'ailleurs, il est sans exemple que l'hégémonie n'affaiblisse pas en fin de compte le pays qui l'a obtenue. Seulement, jusqu'ici, l'acquisition de l'hégémonie, puis l'affaiblisse­ment qui en résulte se sont toujours accomplis, sauf erreur, au moyen de guerres. Si le même processus pouvait, cette fois, avoir lieu sans guerre, ne serait-ce pas le vrai progrès ?\par

\begin{center}
*\end{center}
\noindent On discute beaucoup des engagements de la France à l'égard de la Tchécoslovaquie. Mais un engagement même formel ne constitue pourtant pas, en matière internationale, une raison suffisante d'agir. Les hommes d'État de tous les pays le savent, bien qu'ils le taisent. Quand ils ne le sauraient pas, peut-on admettre que toute une jeunesse meure pour un pacte qu'elle n'a pas ratifié ? Le pacte de la S.D.N. constitue un engagement formel ; il n'a pourtant jamais déterminé, autant dire, aucune action, et on l'a tacitement reconnu nul toutes les fois qu'on lui a ajouté des conventions particulières. Il y a un peu plus d'un quart de siècle, la France a violé sa signature quand elle s'est emparée du Maroc, risquant par là une guerre européenne ; elle pourrait bien, aujourd'hui, en faire autant pour éviter une guerre. Son prestige, il est vrai, serait alors ruiné auprès des petites nations ; et, depuis Talleyrand, la France a pour tradition de s'appuyer sur elles. Cette politique est une application de l'équilibre des forces en Europe ; la France essaie, au moyen des petits pays, de remédier à son infériorité propre ; en même temps, elle se donne une espèce d'auréole d'idéalisme, auréole bien imméritée, car quelles atroces misères n'a pas créés le morcellement de l'Europe centrale depuis vingt ans ! Quoi qu'il en soit, jamais politique n'a subi plus sanglant échec que celle-là, puisque c'est la « petite Serbie » qui a jeté l'Europe dans le massacre dont nous subissons les suites. Quand on y réfléchit, il ne semble pas qu'il y ait eu là accident, mais conséquence nécessaire. Les petits pays sont une tentation irrésistible pour la volonté de puissance, que celle-ci prenne pour forme la conquête ou, ce qui est préférable à tous les points de vue, la création de zones d'influence ; entre deux grandes nations, il est naturel qu'ils tombent sous la domination plus ou moins déguisée de la plus forte, et si l'autre tente de s'y opposer, le recours aux armes est presque inévitable.\par

\begin{center}
*\end{center}
\noindent C'est là le centre même de la question. Les chances de paix seront-elles augmentées si la France et l'Angleterre - en les supposant d'accord - garantissent de nouveau solennellement l'intégrité de la Tchécoslovaquie, ou si elles se résignent, avec les formes convenables, à l'abandonner à son sort ? On dit que, dans le premier cas, Hitler reculerait. Peut-être. Mais c'est une chance terrible à courir. Il est emporté, dans son action, par un double dynamisme, le sien propre et celui qu'il a su communiquer à son peuple et qu'il doit maintenir à la température du fer incandescent pour garder son pouvoir. Il est vrai que jusqu'ici il ne s'est jamais exposé au risque d'une guerre ; mais jusqu'ici il n'en a jamais eu besoin. On ne peut nullement en conclure qu'il soit résolu à toujours éviter ce risque. Ce serait folie de sa part, dit-on, de risquer la guerre générale pour attaquer la Tchécoslovaquie ; oui, mais folie toute aussi grande de la part de l'Angleterre et de la France de courir le même risque pour la défendre. Si elles se résolvent à ce risque, pourquoi pas lui ? Il apparaît de plus en plus nettement qu'une attitude ferme sur la question tchécoslovaque, même jointe à des propositions de négociation générale, ne détendrait par l'Europe actuellement. Matériellement, des négo­ciations, des compromis, des arrangements économiques seraient fort avantageux pour l'Allemagne, même nécessaires ; mais moralement - et les considérations morales priment de beaucoup, pour une pareille dictature - moralement, ce qu'il faut à Hitler, ce n'est rien de tout cela, ce sont des affir­mations périodiques et brutales de l'existence et de la force de son pays. Il n'est pas vraisemblable qu'on puisse l'arrêter sur cette voie autrement que par les armes.\par
S'il ne s'agissait que de le faire reculer par un bluff, qui ne le désirerait ? Mais s'il doit s'agir, comme il est au moins possible, d'une action effective par les armes, je me demande combien on trouverait de jeunes hommes mobilisa­bles, de pères, de mères, de femmes de mobilisables, pour estimer raisonnable et juste que le sang français coule à propos de la Tchécoslovaquie. Il y en aurait peu, je crois, si toutefois il y en a. Une guerre provoquée par des événements d'Europe centrale serait une vérification nouvelle des paroles amères, mais fortes de Mussolini dans sa préface à Machiavel : « Même dans les pays où « les mécanismes (de la démocratie) datent d'une tradition séculaire, il vient des heures solennelles « où on ne demande plus rien au peuple parce qu'on sait que la réponse serait funeste. On lui enlève « la couronne de carton de la souveraineté, bonne pour les temps normaux, et on lui ordonne « purement et simplement d'accepter une révolution, ou une paix, ou de marcher vers l'inconnu « d'une guerre. Au peuple, il ne reste qu'un monosyllabe pour consentir et obéir. Nous voyons que « la souveraineté généreusement accordée au peuple lui est retirée dans les moments où il pourrait « en sentir le besoin... Imagine-t-on une guerre proclamée par réfé­rendum ? Le référendum est une « très bonne chose quand il s'agit de choisir l'endroit le plus convenable pour y placer la fontaine du village, mais quand les intérêts suprêmes d'un peuple sont en jeu, même les gouvernements ultra démocratiques se gardent bien de les remettre au jugement du peuple lui-même... »\par
Pour en revenir à la Tchécoslovaquie, il n'y a que deux partis clairs et défendables : ou la France et l'Angleterre se déclarent décidées à la guerre pour en maintenir l'intégrité, ou elles acceptent publiquement une transforma­tion de l'État Tchécoslovaque propre à satisfaire les principales visées allemandes. En dehors de ces deux partis, il ne peut y avoir que des humilia­tions terribles, ou la guerre, ou probablement les deux. Que le second soit infiniment préférable, c'est ce qui est à mes yeux évident. Toute une partie de l'opinion anglaise est prête accueillir une telle solution et pas seulement à droite.\par
({\itshape Feuilles libres}, 4\textsuperscript{e} année, n° 58, 25 mai 1938.)\par

\begin{center}
\end{center}
\subsubsection[10. Réflexions sur la conférence de Bouché, (1938)]{10. \\
Réflexions sur la conférence de Bouché \\
(1938)}
\noindent \par
Je me place, pour commenter en moi-même la conférence de Bouché, sur le terrain qu'il a choisi, c'est-à-dire en prenant pour point de départ, par hypothèse, l'idée d'une défense nationale armée. À l'heure présente, la non-violence est tout à fait défendable comme attitude individuelle, mais n'est pas concevable comme politique d'un gouvernement.\par
Le système actuel de défense nationale comporte, comme Bouché l'a montré admirablement, des malheurs prochains effrayants, des risques presque illimités, à peu près aucun espoir. Donc tout système moins lourd, comportant moins de risque et plus d'espoir, doit être préféré. On ne peut pas demander que tout projet d'un système nouveau élimine, pour la France, la possibilité de perdre son indépendance nationale ; car le système actuel, si loin qu'on le pousse, ne l'élimine pas, puisqu'une défaite écrasante est toujours possible, sinon même probable.\par
La France, en Europe, n'est pas, et de loin, la plus forte. Elle doit donc renoncer à imprimer à l'Europe un avenir, même prochain, conforme à ses vues ou à ses traditions. Le problème de la défense nationale doit être pour elle celui d'une défense de son territoire contre une invasion, non celui d'une défense du système de traités et de pactes établi par elle au temps où elle pouvait se croire la plus forte.\par
La défense contre l'invasion apparaît, à la réflexion, comme plus diploma­tique que militaire. À l'exception des expéditions coloniales, les guerres de ces derniers siècles n'ont jamais eu, sauf erreur, comme objet ou occasion immédiate, bien qu'elles aient eu parfois pour résultat l'annexion par un pays d'un territoire étranger. Elles ont toujours été provoquées par des conflits ayant pour objet la conservation ou la conquête d'une certaine position diplo­matique. Une diplomatie raisonnable et modérée peut éviter à la France d'être prise en un pareil conflit.\par
Une telle diplomatie doit pourtant être couverte par un système militaire qui évite qu'une invasion de la France apparaisse, aux yeux des Français et aux yeux de l'étranger, comme assimilable à une expédition coloniale. Mais ce système n'étant plus que l'auxiliaire d'une diplomatie destinée à sauvegarder la paix pour la France, le problème à résoudre ne doit pas être : comment assurer encas d'invasion la défaite de l'ennemi, mais doit être : comment rendre une invasion éventuelle assez difficile pour que l'idée d'une telle invasion ne constitue pas une tentation aux yeux des États voisins.\par
Si ce problème était résolu, la sécurité ne serait pas de ce fait absolue ; mais elle serait plus grande que dans notre système, quand même nous aurions deux fois plus d'avions et de tanks.\par
Cette formule nouvelle du problème de la sécurité impliquerait une transformation complète de la méthode militaire, qui devrait dès lors, au point de vue technique, constituer une sorte de compromis entre la méthode de la guerre et celle de l'insurrection. Bouché préconise, comme procédé de défense passive contre avions, la décentralisation ; il me semble qu'on pourrait élargir cette idée, l'étendre à toute la conception de la défense du territoire. Décentra­lisation de la vie politique, économique et sociale en France, dispersion des agglomérations, union de la vie urbaine et de la vie rurale ; mais aussi décentralisation d'une résistance armée éventuelle, dont on devra toujours considérer que dans le cours naturel des choses elle ne doit pas avoir à se produire. Une certaine décentralisation étant supposée, la technique moderne rend, il me semble, possible, notamment par la rapidité des communications, une certaine forme de résistance qui tiendrait plus de la guérilla que de la guerre. Ne pas constituer de fronts, ne pas assiéger de villes ; harceler l'ennemi, entraver ses communications, l'attaquer toujours là où il ne s'y attend pas, le démoraliser et stimuler la résistance par une série d'actions infimes, mais victorieuses. Si les républicains espagnols avaient appliqué pareille méthode, surtout au début - ils ne l'ont jamais tenté -ils ne seraient peut-être pas dans la situation déplorable où nous les voyons. Mais, encore une fois, le véritable but d'un pareil système ne devrait pas être de forcer l'ennemi, une fois entré sur notre territoire, à en sortir ; il devrait être de donner à réfléchir à ceux que l'idée d'entrer en armes chez nous pourrait tenter.\par
Une semblable conception de la défense suppose un véritable esprit public, une vive conscience, chez tous les Français, des bienfaits de la liberté. Nous n'en sommes pas là. Présentement les dictatures, russe, italienne ou allemande, exercent un grand prestige sur une partie non négligeable de la population, à commencer par les ouvriers et les intellectuels, soutiens tradi­tionnels de la démocratie ; les paysans n'ont jamais encore trouvé le moyen de faire entendre leur voix dans le pays ; quant à la bourgeoisie urbaine grande et petite, absorbée par des intérêts mesquins ou livrée au fanatisme, on ne peut guère compter sur elle pour aucune forme d'enthousiasme civique. L'esprit idéaliste et généreux qu'on attribue par tradition à la France peut-il encore ressusciter ? Il est permis d'en douter. Mais dans le cas contraire, si la liberté doit périr lentement dans les âmes avant même d'être politiquement anéantie, la défense nationale perd tout objet réel ; car ce n'est pas un mot ou une tache sur la carte dont la défense peut valoir des sacrifices, mais un certain esprit lié à un milieu humain déterminé. Faute d'un tel esprit la France risque, et nous en voyons déjà des signes, d'être sans invasion la proie de l'étranger.\par
Tout cela ne concerne que la France. Dès qu'on considère l'Empire fran­çais, le problème se transforme totalement. Bouché n'a guère fait que toucher cet aspect de la question, sans doute fauté de temps. Il semble évident que l'Empire français dans sa forme actuelle exige que la France conserve et développe un armement offensif, et que le système militaire actuel se main­tienne ; les colons français sont de très loin trop peu nombreux pour la défensive, et les indigènes y seraient à juste titre peu disposés. Il semble non moins évident, étant donné le rapport des forces, que si la forme actuelle de l'Empire français est maintenue, la France est à peu prés vouée à perdre un jour ou l'autre ses colonies, en partie ou en totalité, avec ou sans guerre mondiale. Il serait infiniment préférable que ce fût sans guerre ; mais même sans guerre une pareille éventualité n'est désirable à aucun point de vue.\par
L'application de la méthode défensive préconisée par Bouché exige, comme condition essentielle et première, que l'Empire français évolue très rapidement dans le sens où a évolué l'Empire anglais, c'est-à-dire que la plu­part de nos colonies aient très vite une indépendance considérable, suffisante pour les satisfaire. Une pareille transformation doit être préparée d'assez longue main pour s'accomplir sans heurts ; la France, dans son aveuglement, ne l'ayant pas préparée, elle ne s'accomplira pas sans heurts. Mais il y a nécessité urgente. Les colonies françaises constitueront des proies tentantes et une cause perpétuelle de risques immédiats dans le monde aussi longtemps qu'il n'y aura pas pour les indigènes aussi une défense nationale, ce qui suppose une nation.\par
En somme, la conception de la défense nationale présentée par Bouché suppose la liberté ; elle ne convient qu'à des citoyens. Non seulement elle exige que les citoyens français aient l'esprit et possèdent les droits effectifs du citoyen ; mais surtout elle ne peut être qu'une chimère aussi longtemps que l'ensemble des territoires placés sous l'autorité de l'État français contiendra beaucoup moins de citoyens que d'esclaves. De sorte qu'on a présentement le choix : faire de ces esclaves des citoyens, ou devenir nous-mêmes des esclaves.\par

\begin{center}
\end{center}
\subsubsection[11. Lettre à G. Bergery, (1938)]{11. \\
Lettre à G. Bergery \\
(1938)}
\noindent \par
Cher camarade Bergery,\par
J'ai été heureuse de voir dans {\itshape La Flèche} la question tchécoslovaque nettement posée par vous. Excusez-moi d'y revenir encore : ce sujet est trop important et angoissant pour ne pas sans cesse occuper l'esprit. Je remarque d'abord que des deux conditions que vous posez pour le soutien de la Tchécoslovaquie, l'une au moins peut très probablement être considérée comme absente, à savoir la cohésion du pays à défendre. Mais quelle que soit l'importance pratique et immédiate de ce point, il laisse subsister un problème bien plus large, puisque votre article ramène la question de la Tchécoslo­vaquie à celle de l'emprise de l'Allemagne sur l'Europe centrale et son hégémonie en Europe. À mon avis, cette dernière question mériterait d'être traitée directement et dans toute son ampleur. Depuis trois quarts de siècle, et plus que jamais à l'instant présent, elle est essentielle ; elle commande toute la politique extérieure et intérieure.\par
Votre pensée est que l'hégémonie de l'Allemagne en Europe, avant comme corollaire la faiblesse de la France, donnerait à la première la tentation d'une entreprise armée sur la seconde. Personne ne peut effectivement écarter une telle crainte ; personne ne peut la prendre à la légère. Pourtant une attitude ferme de la France, si habile soit-elle, peut, elle aussi, aboutir à la guerre, et la guerre à la défaite, à l'invasion, et à leurs conséquences extrêmes. On peut donc considérer que l'une et l'autre position peuvent, dans la pire éventualité, amener un résultat final équivalent (avec pourtant, il me semble, un cortège de massacres et de désastres, dans l'Europe et le monde, moindre dans le premier cas). Il s'agit de savoir si la pire éventualité est plus probable avec la première position qu'avec la seconde. Il s'agit de savoir aussi laquelle des deux positions amènerait, dans la meilleure éventualité, un résultat préférable.\par
Examinons d'abord le second point. Que peut-on concevoir de plus heureux comme résultat d'une politique fondée sur la préservation de l'équilibre européen ? Que la France, unie à l'Angleterre, arrête l'Allemagne dans son élan vers l'hégémonie, sans que celle-ci ose recourir à la guerre. Le dynamisme étant l'essence même du système politique allemand - ce qu'il ne faut jamais oublier - la France, pour tenir en respect la volonté d'expansion de l'Allemagne, doit rester forte, vigilante, tendue tout entière vers l'extérieur, constamment chargée pour la guerre, comme disait Péguy, en alerte perpétuelle. L'union avec l'Angleterre doit être étroite et continuelle, ce qui, soit dit en passant, n'est pas propre à faciliter la lutte contre les trusts, tant que la City sera ce qu'elle est. Le budget de guerre doit être maintenu, accru, doublé. Inutile d'insister sur la misère matérielle et morale qui en résultera, la tension des nerfs, l'enrégimentement des esprits, l'extinction des libertés, l'angoisse individuelle et collective. Une telle situation devrait durer aussi longtemps que la menace allemande ; or il faudrait savoir si moralement, politiquement, économiquement, elle peut durer. Rien qu'économiquement, elle peut durer. Rien qu'économiquement et techniquement, la France peut-elle supporter longtemps - peut-elle supporter aussi longtemps que l'Allemagne - un effort d'armement qu'il faut sans cesse renouveler, quand même elle y sacrifierait tout ce qui lui reste de liberté et de démocratie ? Si elle ne le peut pas, pourquoi s'engager dans une politique susceptible seulement de retarder une échéance qui serait celle ou de la guerre, eu de l'abdication ?\par
Si même elle le peut, qu'est-il permis d'espérer ? Un changement de régime en Allemagne ? Certes, une sérieuse défaite de prestige suffirait à faire tomber le régime. Mais cela, sans aucun doute, Hitler aussi le sait, et il préférerait à une telle défaite la guerre engagée dans les pires conditions. Le simple usage de la menace pour ralentir, détourner, même arrêter son élan n'est pas propre à le faire tomber ; l'état d'alerte ainsi établi des deux côtés du Rhin amènerait plutôt la France à un national-socialisme français.\par
Nous tenons toujours, il est vrai, en réserve, le grand, le beau projet de pacification européenne par une négociation générale inspirée d'un esprit de justice. Certes ce projet contenait la seule voie de salut ; mais je crains qu'à force de le tenir en réserve il ne soit mort. Avant Hitler, la France aurait pu l'appliquer avec une facilité dérisoire ; elle s'en est bien gardée ; cela se paie, et nous le payons. En mai ou juin 1936, Blum, profitant d'un grand mouve­ment de masse, du vent d'aventure qui soufflait sur la France, du pouvoir très étendu que lui laissaient les circonstances pendant quelques semaines, pouvait rompre solennellement avec la politique extérieure de la France depuis 1918, et faire le geste « spectaculaire » que vous avez toujours réclamé ; il n'en a rien fait, et c'est là le secret de sa faillite. À présent, je crois qu'il est trop tard, si dur, si amer qu'il puisse être de parler ainsi. Au point de vue intérieur, d'abord, parce qu'après le grand élan de 1936 et son enlisement progressif, je ne crois pas qu'on ait d'ici assez longtemps un vaste mouvement populaire qui permette une coupure éclatante et solennelle dans la politique française. Et surtout Hitler a dit et répété à plusieurs reprises depuis un an ou deux que ce qu'il réclame, il veut le prendre ou l'obtenir sans contrepartie, sans conditions, sans marchandages, sans compromis. Ce qu'il dit ainsi, en général il le fait, et je crois qu'en l'occurrence il le peut, matériellement, politiquement et moralement. Matériellement, je pense qu'il a dès maintenant suffisamment modifié le rapport des forces en sa faveur pour pouvoir espérer obtenir ce qu'il veut au moment opportun sans contrepartie. Politiquement, au point de vue intérieur qui prime toujours pour les dictatures, il considère sans doute ce procédé comme plus dynamique, plus frappant pour l'imagination populaire, plus enivrant pour un peuple qui si longtemps a été humilié sans défense et a demandé sans obtenir. Moralement, si justes, si généreuses que puissent être les propositions de la France, la position d'Hitler sera encore la meilleure. Car il peut toujours dire : aussi longtemps que nous n'avons invoqué que la justice, on nous a maintenus écrasés sous le fardeau d'un traité oppressif ; à présent que nous sommes assez forts pour prendre ce à quoi nous avons droit, on nous offre des négociations qu'on nous avait toujours refusées ; mais nous n'en avons plus besoin et nous n'en voulons pas. Il me semble clair que la logique de son mouvement l'amène nécessairement à une telle attitude.\par
Dans une opposition de la France à l'hégémonie allemande, je ne vois pas d'autre avenir que le cercle vicieux inclus dans la notion même de l'équilibre européen. Si aucun des deux peuples ne peut, sans sacrifier sa sécurité, permettre l'hégémonie de l'autre en Europe, il n'a pas d'autre moyen sûr de l'en empêcher que d'exercer une certaine hégémonie lui-même, ce qui oblige l'autre à s'efforcer de la lui prendre, et ainsi de suite. Il y a une contradiction interne dans l'idée de sécurité ; car sur le plan de la force, qui est celui sur lequel la question de la sécurité se pose, il n'y a d'autre sécurité que d'être un peu plus fort que le peuple d'en face, qui, lui, en est alors privé ; ainsi subordonner l'organisation de la paix à une sécurité générale, comme la France l'a fait si longtemps, c'est proclamer la paix impossible. Si même le cercle vicieux enfermé dans la doctrine de l'équilibre européen n'entraîne pas nécessairement la guerre, il entraîne en tout cas la militarisation toujours croissante de la vie civile. La France sera-t-elle moins asservie à l'Allemagne si l'asservissement prend la forme de l'état de siège prolongé à perte de vue que s'il prend la forme de subordination politique ?\par
Si, à présent, nous supposons que la France laisse l'Allemagne établir son hégémonie en Europe centrale, et sans doute par suite en Europe, que peut-on espérer dans le meilleur des cas ? Rien, non plus, de bien séduisant. Tout ce qu'on peut espérer, c'est que la France une fois repliée derrière ses frontières, ayant réduit son système militaire à quelque chose de plus modeste et d'essentiellement défensif, n'opposant plus d'obstacle aux visées diplomati­ques de l'Allemagne, forcée d'être pour le moins très conciliante sur le terrain économique, l'Allemagne ne se donnerait pas la peine de l'envahir. Une telle possibilité n'est certainement pas à exclure. Il est possible aussi qu'en pareil cas la France accomplirait à l'intérieur de ses frontières, à condition qu'elle s'en donne la peine, un effort de culture, de civilisation, de rénovation sociale, sans que l'Allemagne y fasse obstacle. Sans doute la supériorité des forces allemandes amènerait-elle la France à adopter certaines exclusives, surtout contre les communistes, contre les juifs : cela est, à mes yeux et probablement aux yeux de la plupart des Français, à peu près indifférent en soi. On peut fort bien concevoir que rien d'essentiel ne serait touché, et que tout ce qui, dans notre pays, est encore soucieux de bien public, pourrait enfin s'occuper un peu d'une manière effective des logements, des écoles, de la conciliation néces­saire des entre les nécessités de la production et la dignité des travailleurs, de diffuser largement parmi le peuple les merveilles de l'art, de la science et de la pensée, et autres tâches paisibles.\par
Si on compare ces deux hypothèses, qui représentent, encore une fois, la meilleure éventualité par rapport à deux politiques, la seconde, quelque rési­gnation qu'elle implique pour une nation jadis de premier plan, me paraît très clairement et de très loin préférable. Elle comporte un avenir précaire, mais un avenir ; la première n'en comporte aucun, elle ne comporte que la continuation indéfinie et sans doute l'aggravation d'un présent à peine tolérable.\par
Il faut se demander aussi quelles sont les probabilités respectives de la meilleure et de la pire éventualité pour l'une et l'autre politique. L'Allemagne résisterait-elle à la tentation d'englober une France relativement faible dans son système totalitaire, soit par occupation militaire, soit par une espèce de vassalité politique et économique très rigoureuse ? Peut-être oui, peut-être non. Cela dépendra non seulement du rapport des forces, mais de ce que la France possédera encore de vivant en fait de ressources morales et spiri­tuelles ; cela dépendra aussi de la durée du dynamisme allemand. Des élans de cette espèce, s'ils sont brisés par la défaite, finissent aussi par s'affaiblir à force de succès. Un tel affaiblissement peut, même une fois la France réduite pour un temps à l'état de vassale, faire évoluer le régime politique allemand d'une manière qui changerait tout à fait le problème de l'hégémonie allemande en Europe. N'oublions pas que les régimes politiques sont instables ; il n'est pas sage d'en faire des données essentielles dans l'orientation de la politique extérieure qui détermine l'avenir à longue échéance.\par
Dois-je dire toute ma pensée ? Une guerre en Europe serait un malheur certain, dans tous les cas, pour tous, à tous les points de vue. Une hégémonie de l'Allemagne en Europe, si amère qu'en soit la perspective, peut en fin de compte n'être pas un malheur pour l'Europe. Si on tient compte que le national-socialisme, sous sa forme actuelle d'extrême tension, n'est peut-être pas durable, on peut concevoir à une semblable hégémonie, dans le cours prochain de l'histoire, plusieurs conséquences possibles qui ne sont pas toutes funestes.\par
Au reste, si la France veut arrêter l'accroissement continu de la puissance allemande, le peut-elle ? N'est-il pas dans la nature des choses que l'Europe centrale tombe sous la domination de l'Allemagne ? Le maintien du {\itshape statu quo} en Tchécoslovaquie est inconcevable ; il n'est défendable ni en fait ni en droit. Or comme le territoire peuplé par les Sudètes renferme les défenses naturelles en même temps qu'une partie importante des ressources industrielles de la Tchécoslovaquie, je ne conçois aucune réforme même intérieure qui ne mette pas celle-ci, pratiquement, à genoux devant l'Allemagne. Une telle réforme pouvait à la rigueur être conçue en 1930, non pas maintenant, étant donné les rancœurs qui animent les Sudètes, la puissance de l'Allemagne, militaire et économique - puisque avec la complicité de la Hongrie et de la Pologne elle encerclerait complètement la Tchécoslovaquie - l'intelligence politique indé­niable de Hitler. Seule peut varier, je crois, selon les circonstances, la forme plus ou moins brutale, le rythme plus ou moins rapide qui sera imprimé à cette opération.\par
Si même la France et l'Angleterre pouvaient opposer un barrage efficace à la coulée de l'Allemagne vers l'Europe centrale, on ne peut pas affirmer qu'Hitler hésiterait à faire la guerre pour rompre le barrage. Le contraire me paraît probable. Peut-être Hitler préférerait-il ne faire la guerre en aucun cas, même s'il arrive à posséder les ressources nécessaires à une guerre longue ; mais il veut certainement posséder ces ressources, pour avoir la possibilité de parler à l'Europe sur le ton qu'il lui faut arriver à prendre afin de continuer à parler en maître aux Allemands. Pour autant qu'on puisse conjecturer en pareille matière, je pense qu'au besoin, pour acquérir ces ressources, et s'il n'avait pas d'autre possibilité, il risquerait la guerre. Les moyens de la guerre ne constituent-ils pas, aujourd'hui, le véritable but de guerre ?\par
Dans une pareille guerre, la France, appuyée par l'Angleterre seule - car quant à la Russie, mieux vaut ne pas en parler -aurait bien des chances d'être vaincue, et ne pourrait être victorieuse qu'en s'épuisant, en se détruisant plus que ne pourrait le faire un ennemi victorieux. Et quelle serait ensuite la situa­tion de l'Europe devant les autres continents ?\par
Qu'est-ce qui constituerait pour l'Allemagne une tentation de guerre plus grande, la faiblesse relative de la France, ou le barrage opposé par une France encore assez forte à ses ambitions d'hégémonie ? Il est bien difficile de le dire. Peut-être peut-on considérer les chances comme à peu près égales, ou que, s'il y a une différence, elle est en faveur de la politique de repli derrière nos frontières. S'il est vrai que c'est aussi cette politique qui, en cas de succès relatif, présente l'issue la plus favorable, j'en conclus qu'elle est la meilleure, étant bien entendu que la France devrait profiter de la position qu'elle occupe encore pour tenter une fois sérieusement, même avec peu d'espoir, le grand règlement européen.\par
Le plus grand obstacle à cette politique de repli, c'est que la France est un empire. Mais c'est là un empêchement déshonorant, car il ne s'agit plus pour elle de conserver son indépendance, mais la dépendance où elle tient des millions d'hommes. Si la France veut adopter cette politique sans se voir pure­ment et simplement ravir son empire colonial, cette politique devra s'accom­pagner d'une évolution rapide de ses colonies vers une autonomie assez large, avec des modalités diverses. Pour moi, ce serait là un motif suffisant, si je n'en avais pas d'autre, d'aspirer à un tel changement d'orientation ; car, je l'avoue, selon mon sentiment, il y aurait moins de honte pour la France même à perdre une partie de son indépendance qu'à continuer à fouler aux pieds les Arabes, les Indochinois et autres.\par
Je crois aussi que l'atmosphère morale se trouverait éclaircie par la disparition de tous les mensonges, de toute la démagogie, de toute l'hypocrisie liée à l'effort que fait la France depuis vingt ans pour jouer un rôle dispro­portionné à ses forces. Bref, cette politique, si précaire soit-elle et si pénible à certains égards, est la seule à mes yeux qui comporte des possibilités, même faibles, de progrès humain et de tentatives neuves. Et je crois urgent, si on doit l'adopter, qu'on s'y détermine au plus tôt.\par
C'est pourquoi je déplore qu'elle n'ait pas derrière elle un homme comme vous, sympathique à tous ceux qui aiment l'indépendance, l'intelligence et l'honnêteté, et qui, n'ayant pas été compromis dans les fautes ou les crimes du passé, peut un jour avoir une grande autorité auprès d'une large partie de la population, plutôt qu'un Flandin en qui personne ne peut avoir confiance à aucun point de vue.\par
J'ai sans doute retenu trop longtemps votre attention. Mais ayant com­mencé d'aborder un pareil problème, il m'a paru préférable de l'examiner tout de suite sous tous les aspects que je lui vois. J'espère qu'un numéro prochain de {\itshape la Flèche} apportera à vos lecteurs l'expression motivée de votre pensée sur ce sujet. Ils l'attendent certainement ; car, au milieu des politiciens et de tous ceux qui s'expriment simplement en tant que citoyens, il n'y a que vous, somme toute, qui, bien qu'actuellement sans responsabilités gouvernemen­tales, parliez en homme d'État.\par
Bien cordialement\par
S. WEIL.\par

\subsubsection[12. Désarroi de notre temps, (1938 ?)]{12. \\
Désarroi de notre temps \\
(1938 ?)}
\noindent \par
Notre époque n'est pas la première dans l'histoire où le sentiment dominant soit le désarroi, l'anxiété, l'attente d'on ne sait quoi, et où les hommes se croient le douloureux privilège d'être une génération promise à un destin exceptionnel. Comme l'histoire est passée, qu'elle ne se trouve plus que sur le papier, on a facilement l'illusion que toutes les périodes antérieures ont été paisibles à côté de celle qu'on est en train de vivre ; tout comme les adoles­cents de vingt ans se croient toujours les premiers qui aient jamais éprouvé les inquiétudes de la jeunesse. Cependant on peut dire, sans crainte d'exagérer, que l'humanité dans notre petit coin d'Europe qui depuis si longtemps domine le monde, traverse une crise profonde et grave. Les grandes espérances héri­tées des trois siècles précédents, et surtout du dernier, espoir d'une diffusion progressive des lumières, espoir d'un bien-être général, espoir de démocratie, espoir de paix, sont en train de s'effriter à une cadence rapide. Ce ne serait pas là chose tellement grave s'il s'agissait simplement d'une désillusion atteignant certains cercles intellectuels, ou certains milieux particulièrement préoccupés de problèmes politiques et sociaux. Mais nous sommes placés dans des conditions de vie telles que le désarroi touche et corrompt tous les aspects de la vie des hommes, toutes les sources d'activité, d'espérance et de bonheur. La vie privée, dans son cours quotidien, se détache de moins en moins de la vie publique, et cela dans tous les milieux. Il y a déjà eu des moments de l'histoire où de grands élans collectifs ont passagèrement réduit la vie privée à peu de chose ; mais aujourd'hui, ce sont les conditions durables de notre existence qui nous empêchent de trouver dans la vie quotidienne des ressources morales indépendantes de la situation politique et sociale.\par
Le sentiment de la sécurité est profondément atteint. Ce n'est pas absolument un mal, d'ailleurs ; il ne peut y avoir de sécurité pour l'homme sur cette terre, et le sentiment de la sécurité, au-delà d'un certain degré, est une illusion dangereuse qui fausse tout, qui rend les esprits étroits, bornés, superficiels, sottement satisfaits ; on l'a assez vu pendant la période dite de prospérité, et on le voit encore dans quelques catégories sociales, de plus en plus rares, qui se croient à l'abri. Mais l'absence totale de sécurité, surtout quand les catastrophes à craindre sont sans commune mesure avec les ressources que pourraient procurer l'intelligence, l'activité, le courage, n'est pas non plus favorable à la santé de l'âme. On a vu une crise économique ôter dans plusieurs grands pays à toute une jeune génération toute espérance de pouvoir jamais entrer dans les cadres de la société, gagner de quoi vivre, nourrir une famille. On a bien des chances de voir dans quelque temps une nouvelle jeunesse dans la même impasse. On a vu, on voit les conditions actuelles de la production faire commencer la vieillesse, et une vieillesse sans soutien, à l'âge de quarante ans pour certaines catégories sociales. La crainte de la guerre, d'une guerre qui ne laisserait plus rien intact, a cessé d'être un sujet de conférences ou de brochures pour se changer en une préoccupation générale, et qui devient de plus en plus quotidienne à mesure que la vie civile se subordonne partout à la préparation militaire. Les moyens modernes de diffusion, presse, radio, cinéma, sont assez puissants aujourd'hui pour secouer les nerfs de tout un peuple. Certes la vie se défend toujours, protégée par l'instinct, par une certaine couche d'inconscience ; pourtant la crainte des grandes catastrophes collectives, attendues aussi passivement que des raz de marée ou des tremblements de terre, imprègne de plus en plus le sentiment que chacun peut avoir de son avenir personnel.\par

\subsubsection[13. Fragment, (1939 ?)]{13. \\
Fragment \\
(1939 ?)}
\noindent \par
Ces derniers mois ont vu s'esquisser en France une transformation profonde des pensées, dont on ne peut encore prévoir les conséquences, mais qui est par elle-même bien digne qu'on s'y arrête. Des hommes, il y a peu de temps encore démocrates, socialistes, syndicalistes, les uns connus et pourvus d'autorité, d'autres obscurs et sans pouvoir, témoignent plus ou moins clairement qu'ils ne sont pas loin de souhaiter pour la France une dictature totalitaire semblable à celle qui permet à l'Allemagne de triompher en Europe. Peut-être certains font-ils plus que souhaiter, peut-être pensent-ils déjà à préparer. Et si la Russie était une plus forte alliée, on verrait sans doute symétriquement tous ceux qu'on nomme bourgeois passer à son égard de l'horreur à l'amour, comme on sait qu'un certain nombre a fait. L'Italie, elle, a tout d'un coup perdu l'estime de ceux qui naguère en louaient le régime presque jusqu'à l'adoration. Croirait-on qu'il y a à peu près deux ans, on pouvait à peine imaginer une guerre où la France ne se séparât pas en deux camps qui auraient l'un et l'autre oublié leur pays pour leur doctrine ? Les passions qui portaient les uns vers la Russie et l'Espagne républicaine, les autres vers l'Italie et même l'Allemagne, étaient si vives qu'on pouvait croire qu'elles effaceraient, dans un pays mobilisé, le souci de défendre le territoire. Que de changements en deux ans ! Aujourd'hui il n'y a presque plus, dans les esprits, que la Nation. Ceux qui s'attachent à d'autres pensées, c'est qu'ils se font violence ; et encore s'y attachent-ils moins fermement qu'ils ne le croient.\par
C'est ainsi que, chez ceux qui ne résistent pas à eux-mêmes, la pensée de maintenir et d'accroître les loisirs, le bien-être et la liberté du peuple aux dépens des privilèges, ou la pensée de conserver les privilèges et l'orgueil qu'y puisent les privilégiés, tout cela disparaît devant le désir d'agrandir la Nation. Non pas, certes, qu'il y ait le moindre élan civique. On continue, bien entendu, tout comme naguère, à puiser dans les affaires d'aviation, au préjudice de l'État, qui ses millions, qui, modestement, ses centaines ou dizaines de milliers de francs. Mais ce n'est pas là une pensée, c'est une pratique. De même les ouvriers de l'aviation, à leur niveau par force bien plus bas, ne souhaitent pas, quelque haine qu'ils aient pour la politique de Munich, travailler soixante heures par semaine ni pour de maigres salaires. Ce ne sont pas les intérêts qu'on sacrifie à la Nation ; il est rare qu'on sacrifie des intérêts sans y être aidé par quelque contrainte. Ce qu'on sacrifie, ce sont les pensées au nom des­quelles on défendait ses intérêts et qui les ennoblissaient en leur donnant une portée universelle. Ce sacrifice entraîne d'ailleurs par la suite celui des intérêts, car il entraîne la soumission à la contrainte qui les anéantira.\par
Ceux qui approuvent la politique de Munich ont coutume de se moquer quand leurs adversaires parlent d'humiliation. En quoi ils se trompent. Il y a eu, en France, en septembre, une humiliation générale ; qu'est-ce qui peut mieux en témoigner que l'espèce de sommeil léthargique où, depuis lors, nous sommes tous plongés, fruit ordinaire d'une humiliation récente ? Ce qu'on a raison de nier, c'est qu'il s'agisse d'une humiliation nationale. Nous avons été humiliés bien plus profondément que dans notre attachement au prestige national ; nous avons subi chacun au centre de nous-mêmes ce qui est, à vrai dire, l'essence de n'importe quelle humiliation, l'abaissement de la pensée devant la puissance du fait. Se chercher soi-même tel qu'on était hier encore et ne pouvoir se retrouver, non pas parce qu'on s'est renouvelé par quelque effort de pensée ou d'action, mais parce qu'entre hier et aujourd'hui il s'est produit, au-dehors, sans qu'on l'ait voulu, un fait, voilà ce que c'est que d'être humilié. Quand il s'agit d'un fait qui tient uniquement aux mouvements de la matière inerte, inondation, tremblement de terre, maladie, on trouve en soi-même des ressources pour se relever. Quand il s'agit d'un fait humain, on ne peut se consoler. On éprouve que les hommes ont le pouvoir, s'ils le veulent, d'arra­cher nos pensées aux objets auxquels nous les appliquions, et de les amener, non pas quelques-unes, mais toutes, non pas par intervalles, mais continuelle­ment, à quelque obsession que nous n'avons pas choisie. La puissance du fait est telle ; elle n'est pas moindre. Elle se soumet toutes nos pensées, et quand elle n'en change pas le soutenu, elle en change la couleur.\par
Que nous est-il donc arrivé en septembre ? C'est fort simple ; il nous est arrivé que la guerre nous est apparue comme un fait, bien qu'elle ne se soit pas produite. Et du même coup la paix, bien qu'elle ait subsisté, a cessé de sembler un fait. Pendant ces quelques semaines, les uns prévoyaient un événement, les autres un autre, et chez le même homme les prévisions variaient plusieurs fois par jour ; mais je ne crois pas qu'il y ait personne qui n'ait senti à quelque moment la présence de la guerre. Et maintenant la paix, bien qu'elle soit encore là et peut-être, si le sort le veut, doive se prolonger longtemps encore, nous est à peine présente. Ainsi, quoique nous parlions encore de guerre et de paix comme autrefois, et que certains s'appliquent à en dire les mêmes paroles qu'autrefois, ce n'est plus la même guerre ni la même paix. La guerre que nous pensions autrefois, et que nous pensions comme une chimère absurde, même quand nous la disions inévitable, ressemblait plus à la paix que ce que nous pensons quand nous parlons de paix, aujourd'hui que nous avons frôlé la guerre. Quoi d'étonnant si le mot de Nation, si longtemps relégué dans les froideurs du vocabulaire officiel, renferme aujourd'hui une richesse inépui­sable d'arguments sans réplique, et si le nom de la France revient sans cesse sous la plume et sur les lèvres ? Un pays devient nation quand il prend les armes contre un autre ou s'apprête à les prendre. Quoi d'étonnant si nous penchons à n'imaginer l'avenir de notre pays que sous l'aspect d'un camp retranché, sans loisirs ni libertés, pourvu de peines de mort et de torture pour châtier les déserteurs ? Chacun de nous, soit avant 1914, soit depuis, a lu dans les livres d'histoire ou dans les vieilles chroniques des récits affreux que nous savions authentiques, mais que nous ne pouvions pourtant prendre pour autre chose que des contes. Nous avions tort, sans doute, puisque ces horreurs avaient été. Aujourd'hui nous penchons, sans y atteindre encore, vers un état où le respect de la vie et de la liberté des hommes, le souci d'accroître les loisirs, le bien-être, les lumières et les joies de toutes sortes dans la multitude du peuple, les égards rendus à la justice et à l'humanité nous sembleront à leur tour des contes. Nous n'aurons pas moins tort ; nous aurons tort de la même manière.\par
Peut-on s'étonner qu'un syndicaliste, par exemple, abandonne non pas son nom, mais l'idéal que ce nom représentait et dont il faisait profession, pour n'appliquer ses pensées qu'à la défense de la Nation et à l'organisation de l'État totalitaire ? C'est comme si on s'étonnait qu'un homme, après une offense, ait de la haine pour celui qui l'a offensé, alors qu'il ne le haïssait pas auparavant ; ou soit pris de peur devant un danger qu'il bravait lorsqu'il ne faisait que le prévoir. Sans doute, la vertu consiste à n'éprouver pas plus de haine après qu'avant l'offense, pas plus de trouble devant qu'avant le danger. Mais la vertu est difficile et rare. Aujourd'hui, à l'égard des affaires d'État, la vertu exige, non pas qu'on pense les mêmes choses qu'autrefois, mais qu'on garde présent à l'esprit tout ce qu'on pensait autrefois. La raison, qui est la même chose que la vertu, consiste à garder dans l'esprit, aussi bien que le présent, un passé et un avenir qui ne lui sont pas semblables.
\subsubsection[14. Réflexions en vue d’un bilan, (1939 ?)]{14. \\
Réflexions en vue d’un bilan \\
(1939 ?)}
\noindent \par
Que l'Europe se trouve présentement dans un moment tragique, chacun ne le sent que trop. Depuis des années déjà ceux qui répugnent à perdre leur sang-froid et à être leurs propres dupes se répètent à eux-mêmes avec effort que chaque génération, au cours des siècles, a toujours exagéré la grandeur et la misère de son propre destin et a cru décisifs de petits épisodes de l'histoire. Mais plus on fait effort pour juger notre temps avec mesure, plus on reconnaît le poids exceptionnel dont pèse aujourd'hui sur les esprits du monde entier le jeu des rapports de force. Pour trouver une époque où le développement d'une situation politique a au même degré accaparé l'attention des esprits, dans tous les milieux, à travers d'immenses étendues de territoires, il faut remonter jusqu'au moment où Rome anéantissait Carthage et étouffait la Grèce ; moment si décisif que nous en subissons encore les conséquences, sans savoir d'ailleurs en apprécier la gravité. Depuis lors il ne s'est jamais rien produit de semblable. Les invasions des barbares ont beaucoup détruit et beaucoup apporté, mais l'un et l'autre d'une manière sporadique, disséminée, locale, sans jamais créer dans le monde d'obsession collective ; la guerre de Cent ans n'a eu aucune influence matérielle ni morale sur le développement du petit miracle florentin ; les guerres de religion accompagnaient une floraison d'idées extrêmement féconde ; aux époques de Charles Quint, de Louis XIV, de Napoléon, on peut trouver en Europe des milieux et des hommes qui développaient et exerçaient librement leurs propres facultés sans guère se soucier de ces personnages bruyants. Aujourd'hui, non seulement en Europe, mais peut-être dans la plus grande partie du monde, ni les grands esprits, ni les médiocres, ni les milieux populaires ne se défendent d'être obsédés par le jeu des forces politiques. C'est là, semble-t-il, le signe certain qu'une grande tragédie est en cours d'accomplissement. Les moments tragiques paralysent généralement les intelligences ; pourtant ils imposent plus que les autres, et pour le salut et pour l'honneur, l'obligation d'évaluer clairement l'ensemble d'une situation. Voici quelques réflexions parmi celles qui peuvent permettre d'évaluer la situation actuelle de la France, puisque nous sommes forcés aujourd'hui de réfléchir dans le cadre d'une nation.\par
Le sentiment dominant de tous aujourd'hui est celui d'un danger. Mais sur la nature du danger on n'est pas d'accord ; on ne cherche guère, à vrai dire, à poser clairement cette question. Les plus inquiets ont un vague pressentiment que pour la France et ses alliés actuels ou éventuels la continuation même d'une existence nationale est devenue douteuse. Ceux-là désirent, pour parer à ce danger, ou la guerre, ou, de préférence, une préparation militaire capable d'imposer à l'adversaire sinon un recul, du moins un arrêt. Certains sont prêts à subir même la perte de l'existence nationale plutôt que d'avoir recours soit à la guerre, soit à la militarisation complète du pays ; et leur opinion peut être défendue ou condamnée par des arguments également sans réplique, car s'ils montrent facilement que la militarisation de la vie civile et la guerre compor­tent des maux égaux à ceux de l'asservissement à l'étranger, on peut alléguer tout aussi facilement qu'un pays asservi peut être soumis à un régime militaire et contraint de participer aux guerres de ses maîtres. Mais ceux-là sont de toutes manières trop rares et trop peu écoutés pour compter parmi les facteurs politiques ; de nos jours une nation ne renonce pas à défendre son indépen­dance par l'effet d'une idéologie, mais seulement parce qu'à tort ou à raison elle se croit militairement impuissante. Les thèses du pacifisme intégral peuvent être éliminées d'une étude de la situation présente.\par
D'autres disent que le danger d'une disparition de plusieurs grandes nations libres est un danger fictif inventé par les fauteurs de guerre. L'habitude conseille de les croire ; car il y a des siècles, six siècles peut-être, que la France n'a pas craint pour son existence, ni l'Angleterre. On est accoutumé de voir échouer les tentatives de domination universelle ; Charles Quint, Louis XIV, Napoléon n'ont causé qu'une terreur vaine et fugitive. Leur exemple empêche de croire qu'un homme puisse concevoir l'ambition de réussir là où ils ont misérablement échoué. On ne connaît qu'un exemple de ce qu'on appelle - très improprement, car l'univers ne peut être soumis, ni même le globe terrestre - domination universelle ; c'est l'empire romain. L'ironie des choses veut que tout le monde ou presque glorifie cet exemple et considère Rome comme la civilisatrice du genre humain, et que tout le monde ou presque n'évoque la possibilité d'un phénomène semblable à notre époque qu'avec horreur. Ou l'admiration ou l'horreur, apparemment, est injustifiée, à moins que les éléments d'appréciation ne soient bien différents ; ce point prêterait à de longues discussions ; pour moi qui crois que les conquêtes romaines, avec leur manière atroce d'anéantir matériellement ou spirituelle­ment des populations entières, ont été la grande catastrophe de l'histoire, j'admets sans peine, conformément à l'opinion générale, qu'une domination universelle exercée par l'Allemagne serait une catastrophe. Le précédent de Rome suffit au moins pour se demander si pareil danger est ou non fictif.\par
Ce problème doit être examiné ; car à l'instant présent il domine tous les problèmes. La période des guerres limitées est aujourd'hui au moins momen­tanément close. On le dit souvent ; c'est même devenu un lieu commun ; mais quand une vérité est devenue un lieu commun on oublie généralement d'en tirer les conséquences. La guerre illimitée - ce terme vaut mieux, je crois, que ceux de guerre absolue ou de guerre totale employés par les spécialistes - est en Europe un phénomène nouveau ; ou si peut-être il s'est produit déjà sous la Révolution et l'Empire et en 1914, il est nouveau que la notion en soit passée dans la pensée commune. Là aussi, pour trouver un précédent, il faudrait sans doute remonter jusqu'à Rome ; du moins si on considère les guerres politiques, car les guerres de religion sont à classer à part. Ce n'est pas que les guerres limitées des siècles passés aient toujours été moins destructrices, moins atroces ; elles pouvaient fort bien comporter le massacre de toute une ville, l'anéantissement d'une province ; mais les massacres et les ravages étaient seulement des accidents, d'ailleurs fréquents, causés par la cruauté. Ces guer­res étaient limitées en ce sens qu'elles ne comportaient dans leur principe que des efforts et des objectifs limités.\par
Aujourd'hui, la pensée commune considère une guerre de grande enver­gure comme une catastrophe totale, qui exigera de tous les derniers efforts, les suprêmes sacrifices, et risque de ne se terminer qu'après épuisement complet du vaincu et épuisement presque égal du vainqueur. Peut-être après tout cette opinion est-elle erronée ; mais le fait est qu'elle est incontestée. Il en résulte qu'il ne peut plus y avoir d'objectifs de guerre. En 1914, on avait déjà un sentiment confus de cette situation ; mais on était encore dominé par une tradition vieille de tant de siècles, et c'est pourquoi les Alliés avaient encore ou disaient avoir des buts de guerre. Aujourd'hui cette notion a disparu. C'est aujourd'hui le peuple tout entier, sans aucune exception, qui fait la guerre - et même dans la mesure où il ne la fait pas, il croit la faire -et le peuple tout entier ne saurait avoir de buts de guerre ; car n'importe quel but est mesquin à ses yeux à côté de son propre sacrifice. Il ne peut plus y avoir aujourd'hui d'autre objectif de guerre pour une nation que sa propre existence, du moins dans le cas d'une guerre entre grandes nations.\par
Il en résulte que la guerre non seulement est une catastrophe, mais ne peut être suivie que par une paix qui constitue par elle-même une catastrophe nouvelle. Car si une nation prend sa propre existence comme objectif de la victoire, elle ne peut vouloir retirer comme fruit de la victoire que la garantie de sa sécurité ; ce qui semble innocent, mais signifie en réalité la suppression du danger qui l'a contrainte à la guerre ; or ce danger est une autre nation ou plusieurs autres nations. Si la guerre mondiale éclate, l'Allemagne, une fois engagée dans cette guerre, aura nécessairement comme objectif la domination universelle. Les puissances démocratiques et leurs alliés auront nécessaire­ment comme objectif l'anéantissement de l'Allemagne. L'anéantissement d'un pareil pays implique, ou qu'un autre pays acquiert la « domination univer­selle » - qui ne serait pas meilleure entre ses mains - ou plus probablement, car aucun pays ne semble de taille à jouer un pareil rôle, la ruine complète de l'Europe, vouée sans doute dès lors à devenir à son tour un territoire colonial. Si le hasard amène une paix relativement modérée, comme ce fut le cas en 1919, on ne voit pas ce qui peut empêcher la situation qui a amené la guerre de se reproduire au bout d'une génération, et les mêmes problèmes se posent. Aux yeux de certains, tout cela se résoudra au bout de quelques mois ou quelques années de guerre par la chute d'Hitler et de Mussolini, suivie d'une fraternisation universelle des peuples ; il ne me paraît pas utile de discuter cette opinion.\par
Une autre conséquence est que la guerre ne peut plus être remplacée par des négociations. « Faire la paix avant d'avoir fait la guerre » est un mot d'ordre excellent, récemment inventé ; il est malheureux pour l'humanité qu'il ait été inventé seulement au moment où il n'a plus aucun sens. Dès lors qu'il n'y a plus d'objectifs de guerre, aucun problème international, si épineux soit-il, ne peut impliquer de danger de guerre tant que l'existence des grandes nations n'est pas en cause ; au reste cette sécurité, instinctivement sentie, fait souvent alors qu'on néglige de négocier. Quand l'existence des grandes nations est en cause, il n'y a plus de problème si facile qu'il ne comporte un grave danger de guerre, parce que la négociation est alors regardée elle-même comme une phase de la guerre, et la moindre concession comporte une perte de prestige qui diminue, pour la nation qui l'a consentie, les chances de défendre sa propre indépendance. Dès qu'une pareille situation a lieu, on ne saurait reprocher aux gouvernements leur souci de prestige ; car le prestige est vraiment une force, il est même peut-être en dernière analyse l'essence de la force ; et une grande nation qui aurait fait toutes les concessions possibles, au point de n'avoir plus que sa propre existence à défendre, serait probablement devenue de ce fait même incapable de la défendre. Ainsi du fait même que « rien ne vaut la guerre », n'importe quoi peut valoir la guerre.\par
Dans cette situation une nation qui veut faire la guerre pour sa propre existence et ne veut la faire pour rien d'autre se trouve devant le problème de savoir si tel ou tel objet de conflit est ou non, compte tenu de tous les facteurs et notamment du prestige, d'importance absolument vitale. Problème inso­luble ; la limite entre les concessions qu'on peut faire - et par conséquent qu'on doit faire - et celles qu'on ne peut pas faire n'existe pas. Ainsi s'explique actuellement l'hésitation perpétuelle des démocraties ; et on comprend que les gouvernements pour qui s'ajoute à toutes ces considérations un besoin intérieur de prestige aient constamment l'initiative. Mais eux non plus, même en faisant abstraction de ce besoin, ne peuvent pas éliminer le danger de guerre, une fois ce danger apparu, en acceptant de négocier. Car dès qu'une nation est regardée comme constituant un danger pour l'existence d'autres nations, ce fait même met en péril sa propre existence. Ainsi elle a un besoin de prestige aussi vital, aussi impérieux que ceux qu'elle menace ; et comme elle a la position offensive, la conservation du prestige consiste pour elle à acquérir, comme elle consiste pour les autres à ne pas perdre. Même acquérir ne peut lui suffire, si elle n'acquiert d'une manière telle qu'elle semble avoir imposé sa volonté, et non avoir profité de la bonne volonté d'autrui. Ainsi dès qu'il y a danger de guerre, les négociations ne peuvent plus y remédier, parce que les objets mêmes des négociations perdent absolument toute valeur intrinsèque ; ils ne valent plus que comme signes, et accessoirement comme avantages stratégiques. Les négociations peuvent, ainsi que le temps qui s'écoule pendant qu'elles ont lieu, modifier les chances respectives de victoire, mais non pas apaiser. Nous le sentons instinctivement ; de là notre angoisse.\par
Il semble ainsi qu'il y ait, dans les conditions actuelles de la guerre, deux états de choses discontinus, l'état de paix et - pour employer le terme allemand - l'état de danger de guerre. Dans l'état de paix, le problème consiste à réformer les rapports internationaux de manière que cet état soit stable. Dans l'état de danger de guerre, s'il est vrai que l'issue ne peut être qu'une forme de « domination universelle » établie sans guerre, ou une guerre d'extermination suivie soit d'une telle domination, soit de quelque malheur équivalent, on peut dire qu'il n'y a plus de problème. L'habitude entraîne encore à obéir à de vieilles loyautés, pour la plupart des gens à la loyauté à l'égard de la nation ; mais le seul problème est alors individuel et non politique, et consiste à trouver une manière de souffrir avec constance tout ce que le destin peut apporter. À une telle situation s'applique bien la singulière parole d'un Perse à un ami d'Hérodote avant Platées : « La plus haïssable des douleurs humaines est de beaucoup comprendre et de ne rien pouvoir. » Mais il est clair qu'entre l'état de paix et l'état de danger de guerre, bien qu'ils soient discontinus, il y a un état intermédiaire, instable par nature, mais que le hasard peut peut-être parfois prolonger assez longtemps pour que les causes qui ont mis fin à l'état de paix disparaissent. Dans cet espoir, il faut toujours, quand on se trouve dans cet état intermédiaire, chercher à le prolonger, car les avantages possibles dépassent infiniment les risques. Les négociations, déjà impuissantes, pour les causes ci-dessus indiquées, à rétablir l'état de paix, ne doivent plus tendre qu'à la prolongation de cet état instable, jusqu'au jour où le jeu d'autres facteurs aura de nouveau amené des possibilités d'apaisement. Cette politique implique évidemment qu'on a reconnu l'existence de tels facteurs.\par
Dans lequel des trois états ainsi définis nous trouvons-nous ? C'est là une question d'appréciation en partie intuitive. On peut encore à la rigueur soutenir que nous sommes en état de paix, que la domination universelle est un mythe et les fantômes de guerre des épouvantails à usage externe ou interne. Cette opinion est pourtant devenue à peu près insoutenable depuis le changement de la politique anglaise. Nous assistons à cet égard à un événement dont nous sommes loin de mesurer la portée ; si nous la mesurions, il nous arracherait des larmes au lieu de nous réjouir. En nous réjouissant de la conscription anglaise, nous cédons au sentiment des élèves de seconde année de Polytech­nique, qui trouvent bon que les nouveaux soient brimés puisqu'ils l'ont été. L'Angleterre était pour l'Europe quelque chose d'infiniment précieux, le seul pays où la liberté a poussé comme une plante, à peine arrêtée dans sa crois­sance par les dominations étrangères et les tentatives d'absolutisme, où le libre citoyen d'aujourd'hui se rattache, par une succession à peu près ininter­rompue d'hommes avant tout soucieux de liberté, au plus lointain moyen âge. La seule existence d'un tel pays ajoutait à la valeur du monde. Aujourd'hui « cette heureuse race d'hommes, ce petit monde, ... terre d'âmes précieuses, cette précieuse terre », comme disait Shakespeare, se dégrade à notre niveau et entre le système de la guerre illimitée auquel elle avait échappé même en 1914. Dans ce pays seul, l’uniforme de soldat, jusqu'aujourd'hui, a été regardé comme une livrée qui déclassait l'homme qui le portait et auquel se rési­gnaient seulement les déchets de la vie civile. Maintenant les jeunes Anglais connaîtront l'obéissance passive et la promiscuité de la caserne, et y perdront les traits qui les distinguaient de toutes les autres jeunesses. Pour que ce pays ait accepté quelque chose qui lui répugne aussi profondément, il faut qu'il croie véritablement que son existence est en question. Dès lors le problème de la domination universelle est posé dans l'opinion publique, quand même il ne le serait pas dans les faits, et l'état de paix n'est plus.\par
L'état de danger de guerre est-il déjà là ? Ou, pour formuler clairement la question, n'y a-t-il aucune chance d'échapper à la fois à la domination universelle et à la guerre ? On ne peut pas ici donner de réponse assurée. Le gouvernement anglais continue à se demander si la liberté de l'Europe est menacée par l'appétit de domination d'une nation. Tant qu'il laisse un point d'interrogation, nous pouvons le laisser aussi. Le jour où il affirmerait au lieu d'interroger, quand même il affirmerait à tort, la question serait résolue, car son affirmation équivaudrait à une déclaration de guerre à brève ou longue échéance. Admettons donc que nous nous trouvons dans l'état intermédiaire. Il s'agit de chercher quels facteurs nous permettront éventuellement de sortir de cet état du bon côté, c'est-à-dire du côté de la paix.\par
Parmi ces facteurs on aurait tort de compter aujourd'hui la personne même d'Hitler. Beaucoup de gens perdent leur temps à se demander si Hitler veut absolument dominer le monde ou si l'on peut s'entendre avec lui en lui proposant des choses raisonnables et en le menaçant au besoin pour le cas où il ne s'en contenterait pas. La question est mal posée. Nous n'avons aucune raison de considérer Hitler comme un maniaque atteint de la folie des grandeurs ; nous serions fous nous-mêmes de le considérer comme un homme modéré. L'appétit de domination, même universelle, n'est une folie que si les possibilités de domination sont absentes ; celui qui voit des chemins vers la domination s'ouvrir devant lui ne s'abstient de s'y avancer, même s'il doit y jouer son existence et celle de son pays, que s'il est ou un saint ou un homme de petite envergure. S'il s'y avance, quand même il foulerait aux pieds, pour passer, la morale, les engagements pris par lui, et tout ce qui mérite le respect, on n'a pas le droit d'en conclure que c'est un barbare, un fou ou un monstre. Les Athéniens, ces créateurs de toute notre civilisation occidentale, disaient aux gens d'une malheureuse petite île qui invoquait l'aide des dieux contre leur injuste agression : « Nous avons à l'égard des dieux la croyance, à l'égard des hommes la connaissance certaine, que toujours, par une nécessité absolue de la nature, chacun commande partout où il en a le pouvoir. » Ils leur disaient aussi ; « Vous le savez comme nous, tel qu'est fait l'esprit humain, ce qui est juste n'est examiné que s'il y a nécessité égale de part et d'autre ; s'il y a un fort et un faible, ce qui est possible est accompli par l'un et accepté par l'autre. » Ces formules, vraiment admirables de netteté, étaient si peu des plaisanteries que les Athéniens firent mourir dans la petite île tous les hommes d'âge militaire et vendirent tout le reste comme esclaves. Hitler n'est évidemment pas une nouvelle édition de Socrate, Marc-Aurèle ou saint François d'Assise ; il est tout aussi loin d'être un homme médiocre. Il commande un pays tendu au maximum ; il a une volonté brûlante, inlassable, impitoyable, qu'aucun respect humain ne peut arrêter ; il est doué d'une imagination qui fabrique de l'histoire dans des proportions grandioses, selon une esthétique wagnérienne, bien loin au-delà du présent ; et il est né joueur. Il est donc clair que ce n'est pas lui qui restera, si peu que ce soit, en deçà des possibilités qui s'ouvriront devant lui ; ni les propositions ni les menaces n'y peuvent plus rien. Il y a peu de temps encore, l'idée de domination universelle ne pouvait être pour lui qu'une idée abstraite, de sorte qu'il aurait pu peut-être être détourné vers d'autres chemins ; mais aujourd'hui les actes et les paroles montrent que l'opinion internationale considère la domination universelle comme une possibilité effective, et par suite c'est là nécessairement aux yeux d'Hitler aussi une possibilité effective. Tant qu'il en sera ainsi, la personne d'Hitler ne doit plus compter comme une donnée distincte dans le problème international. La seule question, c'est de savoir quels effets le temps peut développer, susceptibles de faire disparaître de l'esprit des hommes la possibilité d'une telle domination comme danger réel et proche.\par
Si cette possibilité est illusoire, l'illusion même est un fait capital ; mais on pourrait alors espérer de faire disparaître ce fait dans un délai court, par la diffusion de vues plus raisonnables appuyée au besoin d'actes habilement choisis. Il est à craindre que cette possibilité ne soit pas illusoire. La force de l'Allemagne dans l'Europe contemporaine est incontestable et date déjà de loin. Si le sens de l'organisation, du travail efficace et de l'État, possédé à un degré supérieur, implique un droit surnaturel à coloniser autrui - et a-t-on jamais justifié autrement la colonisation ? - une grande partie du territoire européen peut être regardé comme surnaturellement destiné à une colonisation allemande ; notamment l'Italie, l'Espagne, l'Europe centrale, la Russie ; le cas de la France est différent, mais moins que nous n'aimerions le croire. De ces territoires, l'Italie et l'Espagne semblent bien déjà être à peu près réduites à cette situation, restituant ainsi à Hitler, la Flandre et l'Amérique exceptées, l'empire de Charles Quint. Un pays si méthodique et si dévoué, une fois pris par une exaltation mystique de la volonté de puissance, conduit par un chef qui joint les avantages d'une demi-hystérie à tous ceux d'une intelligence politique au plus haut point lucide et audacieuse, peut aller loin. Sans doute, si on compare Mein Kampf au Mémorial de Sainte-Hélène, on ne peut imaginer que l'auteur d'un de ces livres puisse réussir où l'autre a échoué. Mais en réalité Charles Quint, Louis XIV et même Napoléon ne possédaient pas l'instrument essentiel d'une vaste domination, l'État, sous sa forme achevée ; et ces hommes ignoraient, sinon dans ses rudiments, l'art de dominer. L'art de dominer, le seul où aient excellé les Romains (« Toi, Romain, songe à dominer les peuples »), et heureusement perdu depuis, a été retrouvé par l'Allemagne contemporaine. On trouve véritablement, entre les circonstances qui ont accompagné l'accroissement de la domination romaine, notamment au II\textsuperscript{e} siècle avant notre ère, et celles qui accompagnent l'accroissement de la domination hitlérienne, certaines analogies étonnantes, même à l'égard de ce qui nous paraît le plus nouveau. Les Romains semblent avoir atteint dans leurs méthodes le dernier degré à la fois de l'horreur et de l'efficacité ; et ils n'ont jamais eu de meilleurs élèves que les Allemands d'aujourd'hui, si toutefois il s'agit d'imitation et non d'une invention nouvelle de procédés déjà employés.\par
Les analogies sont trompeuses ; employées avec soin, elles sont pourtant le seul guide. Celle-là contient de quoi faire craindre, mais aussi de quoi rassurer. Car la ressemblance entre le Troisième Reich et la Rome répu­blicaine du II\textsuperscript{e} siècle ne s'étend pas au régime intérieur. Pour le régime intérieur, c'est l'Empire romain qui pourrait fournir des ressemblances ; il en fournit même beaucoup ; mais il a conservé les conquêtes, il a fort peu conquis. Il ne constituait plus une pépinière de maîtres du monde ; il perpé­tuait un mécanisme au moyen duquel n'importe quel fou, avant d'être assas­siné, pouvait fort convenablement jouer le rôle de maître du monde. Nous ignorons quelles auraient pu être ses facultés d'expansion. Mais à l'égard du régime intérieur une autre analogie toute proche de nous se présente à l'esprit, celle-là tout à fait rassurante ; c'est celle de la Russie soviétique. Nous som­mes oublieux, et nous avons oublié, parmi beaucoup d'autres choses, quelle terreur entourait il n'y a pas tant d'années, dans ce qu'on nomme les milieux bourgeois, le nom de la Russie. À ce moment c'est la Russie qui devait conquérir le monde. On ne parlait pas de conquête ; on parlait de révolution mondiale ; mais comme c'était un État qui devait diriger souverainement cette révolution, par la propagande, la diplomatie, les intrigues, l'argent, et au besoin avec son armée, il s'agissait bien d'un danger de domination universelle. Les richesses prodigieuses de la Russie en matières premières, ses ressources en hommes, les complicités qu'elle possédait parmi les ouvriers et les intellectuels de tous les pays et qui pouvaient aller jusqu'à la haute trahison, tout rendait ce danger en apparence redoutable. Peut-être l'était-il. Cependant, aujourd'hui, qui craint la Russie, excepté peut-être quelques États limitrophes ? Elle ne compte plus dans la politique internationale, sinon comme force d'appoint, et la plupart des Français voudraient bien maintenant qu'elle fût plus redoutable. Les tares essentielles au régime ont avec le temps produit ce changement. Sans doute, l'Allemagne constitue intrinsèquement une force bien plus grande que la Russie, parce qu'un Allemand est dans le travail, dans la guerre et dans la politique un homme d'action incompara­blement plus efficace qu'un Russe. Mais les deux régimes sont tellement semblables que les tares sont sans doute susceptibles des mêmes effets dans les deux cas. S'il en est ainsi, on peut espérer que d'ici dix ou quinze ans, si la France et l'Angleterre savent conserver jusque-là, sans guerre, leur indépendance nationale et un minimum de prestige, l'Allemagne cessera évidemment d'être dangereuse pour qui que ce soit. À moins qu'un autre danger n'ait surgi dans l'intervalle, ce qui n'est pas probable, l'état de paix sera alors rendu à l'Europe, et on aura de nouveau licence de rechercher et d'appliquer les principes d'une organisation durable de la paix. Mais ces dix ou quinze ans seront durs à passer.\par
Cet affaiblissement progressif de l'Allemagne sous l'effet de son propre régime, nous le voyons commencer sous nos yeux, en même temps que sa puissance s'accroît. Les tares profondes qui le produisent procèdent toutes d'une même cause. Ce qui rend ces régimes terrifiants est aussi ce qui les affaiblit avec les années, c'est-à-dire leur prodigieux dynamisme. Dans ces régimes, tout ce qui assure la permanence de la force est sacrifié à ce qui en procure le progrès ; ainsi, quand le progrès a atteint une certaine limite, la paralysie survient.\par
Par exemple, un tel régime a pour premier caractère de se dévorer continuellement lui-même. On sait assez, depuis Machiavel, qu'une conquête ou une révolution doit s'appuyer, après le succès, sur les éléments qui l'ont combattue et éliminer ceux qui l'ont favorisée. Les révolutions totalitaires, si on peut employer cette expression, observent dans une certaine mesure cette loi ; mais comme le régime qu'elles établissent entretient une atmosphère permanente de révolution, le même processus s'y répète continuellement sous une forme sourde. D'une part les éléments solides de la nation, les citoyens sérieux, actifs et disposés à servir quel que soit le régime sont sans cesse humiliés, brisés ou même supprimés comme pendant la période aiguë d'une révolution ; d'autre part les éléments exaltés, qui sont pour une part ce qu'il y a de plus sincère dans un mouvement et pour une part ce qu'il y a de plus méprisable, sont sans cesse éliminés par la disgrâce, la prison ou la mort comme pendant la réaction qui suit une révolution. Bien des êtres insi­gnifiants, pris entre ces deux classes d'hommes, partagent leur sort. Ceux qu'on a perdus sont remplacés ; mais le processus se perpétue. Chacun sait qu'il en est ainsi en Russie ; nous pouvons reconnaître par quelques faits parvenus à la connaissance du public qu'il en est de même en Allemagne, bien que non dans les mêmes proportions jusqu'ici. L'exemple de la Russie amène à croire qu'il s'agit d'un mécanisme qui joue de plus en plus avec le temps. Il aboutit à l'impuissance, parce que la possibilité de grandes actions repose toujours et partout sur l'existence et la solidité d'une équipe.\par
Dans le domaine de la technique la faiblesse essentielle de ces régimes est particulièrement éclatante. Ils manient impitoyablement la matière humaine, mais la revanche est que les hommes perdent le stimulant qu'ils puisent dans la conscience de la supériorité par rapport à la matière. Sous un tel régime il n'y a d'autre stimulant pour la besogne quotidienne que la peur et l'appétit de pouvoir ; mais le pouvoir n'est accordé qu'aux spécialistes de la politique, et la peur n'est un stimulant suffisant que pour les formes basses du travail. Déjà l'Allemagne commence à perdre l'avantage que lui procuraient la conscience et le dévouement exceptionnels de ses ouvriers ; une revue économique nationale-socialiste se plaint qu'ils « tirent au flanc » sous l'effet de la fatigue physique et morale. S'il s'agissait de la Russie, on parlerait de l' « âme russe » ; à l'égard d'ouvriers allemands c'est un fait inouï et presque incroya­ble. Mais ce qui concerne les techniciens est encore plus grave. Le régime, en Allemagne comme en Russie, leur ôte la considération sociale, malgré le besoin évident qu'il a d'eux ; il faut croire qu'il y a là une nécessité bien profonde. Même la qualité du travail de ceux qui sont déjà formés et qualifiés ne peut pas ne pas en être diminuée ; mais le problème du recrutement surtout devient presque insoluble. On sait dans quelles proportions les effectifs des écoles d'ingénieurs ont déjà baissé depuis 1929. Enfin même en ce qui concerne les objets ce qui assure la continuité du mécanisme économique est négligé. Il semble qu'en Allemagne comme en Russie les transports soient le secteur le plus faible, malgré les autostrades et le développement de l'industrie automobile, s'il est vrai que le matériel ferroviaire est moindre qu'en 1929 pour un trafic double. Ce dernier point semblerait secondaire s'il n'était un signe de plus que ces régimes s'usent par leur propre tension.\par
Mais c'est surtout à l'égard des hommes qui les subissent qu'il en est ainsi. Du point de vue purement politique, les régimes totalitaires ont pour caractère essentiel qu'ils maintiennent année après année une situation qui n'est naturelle que dans l'enthousiasme. Tous les peuples sont susceptibles d'avoir, si l'on peut dire, des moments totalitaires. Alors les foules unanimes accla­ment, la partie passive de la population, y compris ceux qui étaient auparavant hostiles à ce qu'on acclame, est vaguement admirative et se sent heureuse, quelques-uns des adversaires actifs sont déchirés et mis à mort sans que personne songe à s'en indigner, les autres sont, ils ne savent comment, réduits à l'impuissance. Ces moments sont fort agréables. Les régimes totalitaires commencent plus ou moins dans une atmosphère de ce genre, et c'est pourquoi l'étranger en reconnaît le caractère tyrannique plus tôt que ceux qui les subissent. Puis les années s'écoulent, et tout doit se passer, tous les jours, dans tout le pays, comme si l'atmosphère d'enthousiasme était permanente. Le véritable écueil du régime ne réside pas dans le besoin spirituel qu'éprouvent les hommes à penser d'une manière indépendante, mais dans leur impuissance physique et nerveuse à se maintenir dans un état durable d'enthousiasme, sinon pendant quelques années de jeunesse. C'est en fonction de cette impuis­sance physique que le problème de la liberté se pose, car on se sent libre, sous un pareil régime, exactement dans la mesure où on est enthousiaste. Quand on a ce bonheur, on n'a pas de raison de changer, car, le manque de liberté mis à part, de tels régimes seraient à bien des égards admirables. Mais l'enthou­siasme s'use mécaniquement ; alors la contrainte est sentie, et le sentiment de la contrainte suffit pour susciter ce mélange de docilité et de rancœur qui est l'état d'âme propre aux esclaves. Il s'y ajoute ce léger dégoût qu'éprouve, au milieu de gens un peu ivres, un homme qui n'a bu que de l'eau. L'étouffante nécessité de dissimuler amène enfin une haine sourde, et dès lors tout ce que chacun subit de misère, de privations ou d'humiliations, même si le régime n'en est pas directement responsable, alimente la haine. Un moment arrive enfin où la grande masse de la population, excepté la jeunesse, souhaite non pas la victoire, non pas même la paix, mais la guerre et la défaite pour se débarrasser de ses maîtres. Le pays qui en arrive là cesse jusqu'à nouvel ordre de compter comme facteur indépendant dans les combinaisons internationales. La Russie semble avoir atteint ce point vers 1932 ; l'Italie, depuis quelque temps, y est à peu près arrivée ; l'Allemagne est actuellement dans une situation bien différente, mais le régime n'y dure que depuis six ans.\par
Que produiraient ces facteurs de décomposition en cas de guerre proche ? Que produiraient-ils si l'Allemagne arrivait à dominer l'Europe avant d'avoir atteint le point de paralysie ? On ne peut le prévoir. Peut-être leur influence affaiblissante serait-elle suspendue ou annulée. Peut-être non ; mais le risque est trop grand pour pouvoir le courir. On pourrait, en comptant sur la faiblesse secrète de l'Allemagne, prendre deux partis opposés. L'un serait de risquer la guerre, dans l'espoir qu'elle serait brève, et se terminerait au moyen d'un changement de régime chez l'adversaire. Mais les conditions stratégiques, techniques et politiques de la guerre actuelle sont inconnues à un degré tel, y compris même la valeur et l'utilisation des différentes armes, qu'aucune prédiction en cette matière ne peut avoir de valeur, soit concernant la durée, soit concernant l'issue. On sait seulement par expérience que pour qu'un système politique craque au cours d'une guerre, avant l'épuisement du pays, il faut qu'il soit déjà faible quand la guerre éclate. Une autre tentation serait de laisser l'expansion de la puissance allemande se poursuivre jusqu'à sa limite naturelle, quelle qu'elle puisse être, sans jamais courir un risque de guerre pour la ralentir ou l'arrêter, dans l'espoir qu'ensuite les facteurs internes de décomposition du régime amèneront un reflux avec le minimum de dégâts. La tentation est grande, car il serait beau que pour la première fois dans l'histoire une aventure à la Napoléon commence, réussisse, échoue et se termine au point de départ, le tout sans guerre. Mais le risque aussi est grand, si le reflux ne se produisait pas. Il est plus grand encore qu'il y a vingt et un siècles. Sans doute, à cause de Rome, l'ennui, l'uniformité et la monotonie de l'existence ont tué toute source de fraîcheur, d'originalité et de vie sur une grande partie du globe. Mais il y avait encore pourtant des civilisations indépendantes ; et il y avait, grâce au ciel, les barbares, qui au bout de quelques siècles ont rudement introduit dans le monde la diversité et la vie, sources d'une civilisation nouvelle. Nous n'avons rien à espérer des barbares ; nous les avons colonisés. Nous avons colonisé aussi toutes les civilisations différentes de la nôtre. Le monde entier aujourd'hui, à peu de choses près, est ou assimilé ou soumis à l'Europe. Si l'Europe tombait pour plusieurs générations, avec les territoires qu'elle possède, sous une même et aveugle tyrannie, on ne peut mesurer ce que l'humanité y perdrait. Car, contrairement à ce qu'on affirme souvent, la force tue très bien les valeurs spirituelles, et peut en abolir jusqu'aux traces. Sans cela qui donc, sauf les âmes basses, s'inquiéterait beaucoup de politique ?\par
L'obligation s'impose ainsi impérieusement de tenir, pendant le délai où le système allemand comporte encore une force interne d'expansion. Sur ce point d'ailleurs tout le monde est pratiquement d'accord. Tenir ne veut pas dire ne pas reculer ; une détermination tout à fait rigide de ne plus reculer au cours des dix prochaines années entraînerait probablement la guerre. Tenir signifie tantôt rester sur place, tantôt reculer, tantôt même légèrement avancer, de manière à ne donner au partenaire ni le sentiment d'un obstacle qu'il ne pour­rait renverser que par une violence désespérée, ni le sentiment d'une faiblesse à l'égard de laquelle il pourrait beaucoup se permettre. Jusqu'à ces derniers temps, le premier inconvénient surtout était ou semblait être à éviter ; maintenant c'est surtout le second, de peur que ceux d'en face ne soient tentés d'en arriver à un point tel que nous n'ayons le choix qu'entre l'acceptation de la servitude et la guerre. Une telle politique est essentiellement intuitive ; elle ressemble beaucoup à la navigation entre deux rangées d'écueils. Si jamais la politique a constitué un art, c'est à présent. Cette situation rend la critique de la politique gouvernementale presque impossible, sauf cas d'erreurs grossiè­res, car il est difficile de critiquer des intuitions, surtout quand on ignore, comme c'est nécessairement le cas, de quelles informations les intuitions procèdent. Mais surtout toute critique faite au nom de tel ou tel principe devient absurde. Jusqu'à nouvel ordre les principes n'ont plus cours.\par
Nous avons été quelques-uns à beaucoup réfléchir concernant les principes d'une politique internationale, en nous efforçant de les trouver ailleurs que dans la ruse, la violence et l'hypocrisie, dont nous étions fatigués. Nous ne pouvons renoncer sans peine aux résultats de nos réflexions ; mais ils n'ont plus, actuellement, aucun sens. Cela ne signifie pas que nous ayons mal réfléchi ; sans doute au contraire. La vertu est en soi chose intemporelle, mais elle doit être exercée dans le cours du temps ; et quand, ayant le pouvoir d'agir à l'égard d'une situation donnée d'une manière sage et juste, on s'abstient de l'exercer, on en est puni souvent par la perte même de ce pouvoir. C'est ce qui arrive à la France. Il y a dix ans encore, elle pouvait agir en Europe d'une manière généreuse ; il y a trois ans encore, elle pouvait au moins se montrer raisonnable ; elle ne peut plus ni l'un ni l'autre, parce qu'elle n'est plus assez forte. Pour donner à l'Europe le sentiment que ce qu'elle consent est un signe de générosité ou du moins de modération, il faudrait que l'Europe crût qu'il était en son pouvoir de ne pas le consentir ; c'est ce qui est jusqu'à nouvel ordre impossible. Tant qu'il en est ainsi, tout espoir de grandes actions, de règlement général des difficultés, de création d'un nouvel ordre est momen­tanément aboli. Les négociations ne peuvent plus avoir d'autre règle que l'opportunité, et elles seront bonnes dans la mesure où elles donneront aux partenaires et aux spectateurs une impression de force et d'élasticité à la fois. La politique au jour le jour, qui était jusqu'ici une faute et un crime, devient une nécessité provisoire. Ceux qui tiennent aujourd'hui, à l'égard de la poli­tique extérieure, le même langage qu'il y a un an oublient en l'occurrence qu'il convient à l'homme de se conformer au temps.\par
La politique intérieure dépend pour une part de la politique extérieure ; et, dans cette mesure, là aussi l'opportunité devient provisoirement la règle. On le sent instinctivement, et c'est ce qui fait que la France est depuis plusieurs mois plongée dans une sorte de sommeil. Ceux qui continuent par tradition à exprimer leur opposition à l'égard du gouvernement le font avec un manque secret de conviction que leur ton trahit. Nous sommes tout étonnés de nous trouver, sans savoir comment nous y sommes entrés, dans l'union nationale, dans la militarisation progressive de la vie civile, dans une sorte de dictature sourde et modérée ; nous n'aimons guère cela ; nous nous mettons en règle avec notre conscience en disant que nous ne l'aimons pas, ce qui est trop facile pour être courageux. Au fond de nous-mêmes, plus ou moins secrètement, nous sommes contents de n'avoir pas le pouvoir d'en sortir. Bien peu de gens aujourd'hui prendraient la responsabilité, si elle pouvait leur incomber, de faire quoi que ce fût qui pût diminuer pour la France les possibilités de donner encore l'impression d'une certaine force.\par
Pour mon compte, je préfère le reconnaître, quoique sans plaisir, les vœux que nous pouvons former en matière politique ou sociale, internationale ou intérieure, doivent désormais, pour être raisonnables, être limités par la politique extérieure louvoyante qui nous est imposée en châtiment de nos fautes. Mais il ne s'ensuit nullement que cette politique, avec les conditions intérieures qu'elle implique, doive être regardée comme suffisante. Loin de là, elle ne peut pas se suffire à elle-même. Elle constitue une tactique défensive, et on sait que la simple défensive, si elle se prolonge, est peu efficace et détruit le courage. Elle a donc, par elle seule, bien peu de chances de réussir. Quand même, d'ailleurs, elle réussirait par miracle, quel fruit pourrions-nous en tirer ? Nous serions contraints à un mimétisme progressif à l'égard de l'Allemagne, et, dans le meilleur des cas, celui où le danger allemand s'étein­drait sans guerre, nous aurions, dans dix ou quinze ans, une Europe matérielle­ment presque aussi épuisée que par une guerre, spirituellement presque aussi vide que sous une domination allemande. Mais même un si lamentable succès ne se produirait probablement pas. Une certaine forme d'offensive nous est indispensable ; il nous faut, nous aussi, posséder une force d'expansion. Mais non pas sur le terrain de la violence et de l'appétit de pouvoir ; sur ce terrain nous sommes battus d'avance. On a raison de réclamer de la France une politique généreuse. Seulement ce n'est plus vis-à-vis de l'Allemagne ni de ses clients qu'elle peut être généreuse ; on n'est pas généreux vis-à-vis de plus fort que soi. On est généreux, si on l'est, vis-à-vis de ceux que l'on a à sa merci.\par
On a raison, de même, de vouloir que la France sait publiquement le champion de la liberté. En cas de guerre, la prétention de représenter la liberté, le droit et la civilisation apparaît depuis 1914 comme une hypocrisie déplai­sante, et à juste titre, car il ne s'agit plus alors pour chacun des adversaires que d'éviter d'être écrasé en écrasant. Au contraire pendant la paix, et s'il ne s'agit pas de préparer la guerre, mais au contraire de chercher à l'éviter, ce n'est qu'en faisant appel à l'amour naturel des hommes pour la liberté qu'on peut suffisamment ralentir, et finalement arrêter, une marche à la domination comme celle à laquelle nous assistons présentement. Mais pour qu'un tel appel ait un sens, il faut qu'une atmosphère nouvelle surgisse. On ne créera pas une telle atmosphère en feignant de prendre pour des démocraties les tyrannies dont des raisons stratégiques, d'ailleurs légitimes, amènent à rechercher l'appui. On ne la créera pas non plus en vantant continuellement les libertés dont jouit la France. La simple conservation de ce qui existe est une forme de défensive, et comporte les mêmes faiblesses. Bien plus, dans le moment présent, si le seul objectif était la conservation des libertés françaises, il y aurait par là même recul ; car ces libertés dépérissent de jour en jour sous l'effet des nécessités militaires. Si on ne lutte pas avec tout son courage pour conserver simplement ce qui est, à plus forte raison lutte-t-on mal pour ce qu'on voit s'effriter sous ses yeux. Il ne suffit pas que la France soit considérée comme un pays qui jouit des restes d'une liberté depuis longtemps acquise ; si elle doit encore compter dans le monde - et si elle ne le doit plus, elle peut périr - il faut qu'elle apparaisse à ses propres citoyens et au monde comme une source perpétuellement jaillissante de liberté. Il ne faudrait pas que dans le monde un seul homme sincèrement amoureux de liberté pût se croire des raisons légitimes de haïr la France ; il faudrait que tous les hommes sérieux qui aiment la liberté soient heureux que la France existe. Nous croyons qu'il en est ainsi, mais c'est une erreur ; il dépend de nous qu'il en soit désormais ainsi.\par

\begin{center}
\end{center}
\subsubsection[15. Fragment, (1939 ?)]{15. \\
Fragment \\
(1939 ?)}
\noindent \par
Pour tenir dans la lutte qui oppose les deux seuls grands pays d'Europe restés démocratiques à un régime de domination totale, quelques formes que le temps puisse donner à cette lutte, il faut avant tout avoir bonne conscience. Ne croyons pas que parce que nous sommes moins brutaux, moins violents, moins inhumains que ceux d'en face nous devons l'emporter. La brutalité, la violence, l'inhumanité ont un prestige immense, que les livres d'école cachent aux enfants, que les hommes faits ne s'avouent pas, mais que tous subissent. Les vertus contraires, pour avoir un prestige équivalent, doivent s'exercer d'une manière constante et effective. Quiconque est seulement incapable d'être aussi brutal, aussi violent, aussi inhumain qu'un autre, sans pourtant exercer les vertus contraires, est inférieur à cet autre et en force intérieure et en prestige ; et il ne tiendra pas devant lui.\par
Certes les Français, du point de vue national, ont presque tous fort bonne conscience. Mais il y a plusieurs manières d'avoir bonne conscience. Un bourgeois satisfait et ignorant des réalités de la vie a fort bonne conscience ; un homme juste a bonne conscience d'une tout autre façon ; il a moins bonne conscience généralement ; mais il possède une puissance de rayonnement, une force d'attraction que le premier n'a pas. Les Français sont presque tous persuadés que, d'une manière générale, ce que la France a fait, ce qu'elle fait, ce qu'elle fera, est, sauf de rares exceptions sans portée, juste et bon. Mais cette persuasion est abstraite, car elle est presque toujours accompagnée de beaucoup d'ignorance ; elle ne constitue pas une source intérieure d'énergie. De même, aux yeux des peuples étrangers, le nom de la France est associé aux grands principes de justice et d'humanité dont elle s'est si souvent réclamée ; mais cette association n'est guère qu'une habitude, un lieu commun, et non, comme nous en aurions besoin, le principe d'une attraction irrésistible. Quiconque, au cours des dernières années, passait du territoire d'un pays totalitaire sur le sol français constatait qu'il avait étouffé, et qu'il n'étouffait plus ; mais on ne saurait dire pourtant qu'on ait respiré chez nous une atmosphère effectivement imprégnée et comme chargée de l'idéal au nom duquel nous luttons. Il ne suffit pas, pour bien lutter, de défendre une absence de tyrannie. Il faut être pris dans un milieu où toute l'activité soit dirigée, d'une manière effective, dans un sens contraire à la tyrannie. Notre propa­gande à nous ne peut être faite de mots ; pour être efficace, il faudrait qu'elle fût constituée par des réalités éclatantes.\par

\subsubsection[16. Fragment, (Après juin 1940)]{16. \\
Fragment \\
(Après juin 1940)}
\noindent \par
Il n'y a pas besoin d'un tank ou d'un avion pour tuer un homme. Il suffit d'un couteau de cuisine. Quand tous ceux qui en ont assez des bourreaux nazis se lèveront ensemble, en même temps que les forces armées frapperont le coup décisif, la délivrance sera rapide. Il faut seulement se garder, d'ici-là, à la fois de gaspiller des vies humaines inutilement et de tomber dans l'inertie, de croire que la libération sera accomplie par d'autres. Il faut que chacun sache qu'un jour il lui incombera d'y prendre part et se tienne prêt.\par
Cette période d'attente douloureuse est la plus importante pour la destinée de la France. L'avenir de la France sera celui qu'auront forgé ces années d'apparente passivité.\par
Garder la pensée fixée, au-dessus des douleurs personnelles de chaque journée, sur le drame immense qui se joue dans le monde ; empêcher la souffrance d'être une cause de désunion entre Français à cause de la mauvaise humeur, de la jalousie, des efforts mesquins pour avoir un peu plus que le voisin ; en faire au contraire un lien indissoluble par la générosité et l'entraide ; penser aux biens précieux que nous avons laissé perdre parce que nous ne savions pas les apprécier, qu'il nous faut reconquérir, qu'il nous faudra conserver, dont maintenant nous savons le prix...\par

\begin{center}
\end{center}
\subsection[II. Front populaire]{II. \\
Front populaire}
\subsubsection[1. Quelques méditations concernant l’économie, (Esquisse d'une apologie de la banqueroute) (1937 ?)]{1. \\
Quelques méditations concernant l’économie \\
(Esquisse d'une apologie de la banqueroute) \\
(1937 ?)}
\noindent \par
L'économie est chose singulière. Combien de fois, depuis un certain nombre d'années, ne parle-t-on pas, soit à propos de tel ou tel pays, soit à propos du monde capitaliste dans son ensemble, d'effondrement économique ? On a ainsi l'impression, excitante et romantique, de vivre dans une maison qui, d'un jour à l'autre, peut s'écrouler. Pourtant, qu'on s'arrête un instant pour réfléchir au sens des mots, et qu'on se demande s'il y a jamais eu effondrement économique. Comme toutes les questions extrêmement simples, si simples qu'on ne songe jamais à les poser, celle-ci est propre à jeter dans un abîme de réflexions.\par
Il y a eu, du moins selon la première apparence, des effondrements dans l'histoire ; l'exemple qui vient le premier à l'esprit, c'est celui de l'Empire romain. Mais le déclin du monde romain fut administratif, militaire, politique, intellectuel, autant qu'économique, et sauf examen plus approfondi il ne semble pas y avoir de raison de donner à l'économie le premier rôle dans ce drame. De nos jours, tous les effondrements économiques prédits à satiété depuis des années, Russie, Italie, Allemagne, capitalisme, se rapprochent selon toute apparence aussi peu que la fin du monde ; car tous les jours on les prédit pour le lendemain.\par
On nous cite, il est vrai, des exemples convaincants. L'ancien régime, en 1789, n'est-il pas tombé par impossibilité économique et financière de subsister ? Plus près de nous, la République de Weimar n'a-t-elle pas succom­bé à des difficultés économiques qu'elle n'a pas pu ou n'a pas su résoudre ? On pourrait trouver plusieurs exemples analogues. On n'a certes pas tort de les alléguer. On omet pourtant d'ordinaire, à leur sujet, une remarque pourtant bien frappante. Ces difficultés économiques, si graves qu'elles brisent les régimes, sont toujours reçues en héritage par les régimes qui suivent, et sous une forme d'ordinaire encore aggravée ; pourtant elles deviennent alors bien moins nocives. La situation économique et financière de 1789 était loin d'être brillante ; mais les manuels d'histoire qui expliquent ainsi la chute de la royauté oublient que la Révolution a apporté, au lieu de remède, une guerre ruineuse, et a survécu pourtant à la terrible aventure des assignats. Les diffi­cultés qui ont fait sombrer la République de Weimar n'ont pas disparu, sauf erreur, à l'avènement du Troisième Reich, et pourtant elles l'ont laissé subsister. Et les antifascistes qui jugent économiquement impossible que le Troisième Reich se prolonge oublient que le régime démocratique, socialiste, communiste ou autre qui s'établirait en Allemagne souffrirait très probable­ment des mêmes maux au moins pendant un assez long espace de temps, et devrait s'en accommoder.\par
Ces observations amèneraient à penser qu'il n'y a pas d'effondrement économique, mais qu'il y a dans certains cas crise politique provoquée ou aggravée par une mauvaise situation économique ; ce qui est bien différent. Une analogie permettra d'y voir plus clair. La liaison de cause à effet entre les défaites militaires et les changements de gouvernement ou de régime politique est un fait d'expérience courante. Ce n'est pourtant pas, en ce cas non plus, parce que les conditions nouvelles créées par la défaite militaire rendent impossible au régime existant de subsister ; le régime nouveau s'accommode de ces conditions sans être mieux armé pour les supporter. C'est que la défaite amoindrit ou efface ce prestige du pouvoir qui, beaucoup plus que la force proprement dite, maintient les peuples dans l'obéissance. Dans beaucoup de cas, il serait matériellement aussi facile, peut-être plus facile, de se révolter contre un État vainqueur que contre un État vaincu ; mais la victoire étouffe les velléités de révolte même chez les plus mécontents, et la défaite les excite chez tous. Les répercussions politiques des faits économiques ne procéderaient-elles pas d'un mécanisme analogue ?\par
Les difficultés économiques ne sont pas toujours analogues à des défaites militaires ; elles ne le sont que dans certaines circonstances.\par

\begin{center}
*\end{center}
\noindent L'économie n'est pas comparable à une architecture ni les malheurs de l'économie à des effondrements.\par
Dans tous les domaines auxquels s'applique la pensée et l'activité humaine, la clef est constituée par une certaine notion de l'équilibre, sans laquelle il n'y a que misérables tâtonnements ; équilibre dont la proportion, chère aux Pythagoriciens, constitue le symbole mathématique. Les Grecs, et après eux les Florentins du XIV\textsuperscript{e} siècle, ont inventé la sculpture quand ils ont conçu un certain équilibre propre au marbre et au bronze à forme humaine. Florence a découvert la peinture quand elle a formé la notion de composition. Bach est le plus pur des musiciens parce qu'il semble s'être donné pour tâche d'étudier tous les modes d'équilibre sonore. Archimède a créé la physique quand il a construit mathématiquement les différentes formes de levier. Hippocrate est parti de la conception pythagoricienne assimilant la santé à un équilibre dans le jeu des divers organes. Le miracle grec, dû principalement aux pythago­riciens, consiste essentiellement à avoir reconnu la vertu de la conception et du sentiment de l'équilibre.\par
Le miracle grec ne s'est pas encore étendu à la vie économique. La notion de l'équilibre propre à l'économie, nous ne la possédons pas. Les hommes ne l'ont jamais formée ; mais aussi n'y a-t-il pas deux siècles qu'on s'est mis à étudier l'économie. On ne dirait sans doute que la stricte vérité en affirmant que ce siècle et demi d'études économiques a été vain. Il n'y a pas eu encore de Thalès, d'Archimède, de Lavoisier de l'économie. L'apparition, il y a un peu plus d'un siècle, de doctrines révolutionnaires est probablement pour beaucoup dans cet échec. Les révolutionnaires, anxieux de démontrer que la société bourgeoise est devenue impossible, n'ont naturellement jamais cherché à définir l'équilibre économique à partir des conditions qui leur étaient données ; et pour l'avenir ils ont admis comme évident que la révolution, en matière économique, apporterait automatiquement toutes les solutions en supprimant tous les problèmes. Aucun révolutionnaire n'a jamais tenté sérieu­sement de définir les conditions de l'équilibre économique dans le régime social qu'il attendait. Quant aux non-révolutionnaires, la polémique en a fait des contre-révolutionnaires soucieux non pas d'étudier la réalité qu'ils avaient sous les yeux, mais d'en chanter les louanges. Nous subissons aujourd'hui, dans tous les camps, les conséquences funestes de cette improbité intellec­tuelle que d'ailleurs, plus ou moins, nous partageons. Nous possédons, il est vrai, une sorte d'équivalent à bon marché de cette notion d'équilibre écono­mique. C'est l'idée, si on peut ici employer un tel mot, de l'équilibre financier. Elle est d'une ingénuité désarmante. Elle se définit par le signe égal placé entre les ressources et les dépenses, évaluées les unes et les autres en termes comptables. Appliqué à l'État, aux entreprises industrielles et commerciales, aux simples particuliers, ce critérium semblait naguère suffire à tout. Il constituait en même temps un critérium de vertu. Payer ses dettes, cet idéal de vertu bourgeoise, comme tout autre idéal, a eu ses martyrs, dont César Birotteau restera toujours le meilleur représentant. Déjà au V\textsuperscript{e} siècle avant notre ère le vieillard Céphalès, pour faire comprendre à Socrate qu'il avait toujours vécu selon la justice, disait : « J'ai dit la vérité et j'ai payé mes dettes. » Socrate doutait que ce fût là une définition satisfaisante de la justice. Mais Socrate était un mauvais esprit.\par
Aujourd'hui ce critérium a beaucoup perdu de son prestige, aussi bien du point de vue économique que du point de vue moral ; il n'a pourtant pas disparu. On a toujours tendance à appliquer à l'État la formule de Céphalès, ou du moins la moitié de cette formule ; personne ne demande à l'État de dire la vérité, mais on juge abominable qu'il ne paye pas ses dettes.\par
On n'a pas encore compris que l'idéal du bon Céphalès est rendu inapplicable par deux phénomènes liés et presque aussi vieux que la monnaie elle-même ; ce sont le crédit, et la rétribution du capital. Proudhon, dans son lumineux petit livre {\itshape Qu'est-ce que la propriété} ? prouvait que la propriété était, non pas injuste, non pas immorale, mais impossible ; il entendait par propriété non le droit d'user exclusivement d'un bien, mais le droit de le prêter à intérêt, quelque forme que prenne cet intérêt : loyer, fermage, rente, dividende. C'est en effet le droit fondamental dans une société où on calcule d'ordinaire la fortune d'après le revenu.\par
Dès lors que le capital foncier ou mobilier est rétribué, dès lors que cette rétribution figure dans un grand nombre de comptabilités publiques ou privées, la recherche de l'équilibre financier est un principe permanent de déséquilibre. C'est une évidence qui saute aux yeux. Un intérêt à 4 \% quin­tuple un capital en un siècle ; mais si le revenu est réinvesti, on a une progression géométrique si rapide, comme toutes les progressions géomé­triques, qu'avec un intérêt de 3 \% un capital est centuplé en deux siècles.\par
Sans doute il n'y a jamais qu'une part assez petite des biens meubles et immeubles qui soit louée au placée à intérêt ; sans doute aussi, les revenus ne sont pas tous réinvestis. Ces chiffres indiquent néanmoins qu'il est mathé­matiquement impossible que dans une satiété fondée sur l'argent et le prêt à intérêt la probité se maintienne pendant deux siècles. Si elle se maintenait, la fructification du capital ferait automatiquement passer toutes les ressources entre les mains de quelques-uns.\par
Un coup d'œil rapide sur l'histoire montre quel rôle perpétuellement subversif y a joué, depuis que la monnaie existe, le phénomène de l’endette­ment. Les réformes de Solon, de Lycurgue, ont consisté avant tout dans l'abolition des dettes. Par la suite, les petites cités grecques ont été plus d'une fois déchirées par des mouvements en faveur d'une nouvelle abolition. La révolte à la suite de laquelle les plébéiens de Rome ont obtenu l'institution des tribuns avait pour cause un endettement qui réduisait à la condition d'esclaves un nombre croissant de débiteurs insolvables ; même sans révolte, une abolition partielle des dettes était devenue une nécessité, car à chaque plébéien devenu esclave Rome perdait un soldat.\par
Le paiement des dettes est nécessaire à l'ordre social. Le non-paiement des dettes est tout aussi nécessaire à l'ordre social. Entre ces deux nécessités con­tradictoires, l'humanité oscille depuis des siècles avec une belle inconscience. Par malheur, la seconde lèse bien des intérêts en apparence légitimes, et ne se fait guère respecter sans trouble et sans quelque violence.\par

\subsubsection[2. Méditations sur un cadavre, (1937)]{2. \\
Méditations sur un cadavre \\
(1937)}
\noindent \par
Le gouvernement de juin 1936 n'est plus. Libérés les uns et les autres de nos obligations de partisans ou d'adversaires envers cette chose à présent défunte, soustraite à l'actualité, devenue aussi étrangère à nos préoccupations d'avenir que la constitution d'Athènes, tirons du moins des leçons de cette brève histoire, qui a été un beau rêve pour beaucoup, un cauchemar pour quelques-uns.\par
Rêve ou cauchemar, il y a eu quelque chose d'irréel dans l'année qui vient de s'écouler. Tout y a reposé sur l'imagination. Qu'on se rappelle avec un peu de sang-froid cette histoire prodigieuse, si proche encore, et déjà, hélas, si lointaine. Entre le mois de juillet 1936 et, par exemple, le mois de février de la même année, quelle différence y avait-il dans les données réelles de la vie sociale ? Presque aucune ; mais une transformation totale dans les sentiments, comme pour ce crucifix de bois qui exprime la sérénité ou l'agonie selon qu'on le regarde d'un point ou d'un autre. Le pouvoir semblait avoir changé de camp, simplement parce que ceux qui, en février, ne parlaient que pour commander, se croyaient encore trop heureux, en juillet, qu'on leur reconnût le droit de parler pour négocier ; et ceux qui, au début de l'année, se croyaient parqués pour la vie dans la catégorie des hommes qui n'ont que le droit de se taire, se figurèrent quelques mois plus tard que le cours des astres dépendait de leurs cris.\par
L'imagination est toujours le tissu de la vie sociale et le moteur de l'histoire. Les vraies nécessités, les vrais besoins, les vraies ressources, les vrais intérêts n'agissent que d'une manière indirecte, parce qu'ils ne parvien­nent pas à la conscience des foules. Il faut de l'attention pour prendre conscience des réalités même les plus simples, et les foules humaines ne font pas attention. La culture, l'éducation, la place dans la hiérarchie sociale ne font à cet égard qu'une faible différence. Cent ou deux cents chefs d'industrie assemblés dans une salle font un troupeau à peu près aussi inconscient qu'un meeting d'ouvriers ou de petits commerçants. Celui qui inventerait une méthode permettant aux hommes de s'assembler sans que la pensée s'éteigne en chacun d'eux produirait dans l'histoire humaine une révolution comparable à celle apportée par la découverte du feu, de la roue, des , premiers outils. En attendant, l'imagination est et restera dans les affaires des hommes un facteur dont l'importance réelle est presque impossible à exagérer. Mais les effets qui peuvent en résulter sont bien différents selon qu'on manie ce facteur de telle ou telle manière, ou bien qu'on néglige même de le manier. L'état des imaginations à tel moment donne les limites à l'intérieur desquelles l'action du pouvoir peut s'exercer efficacement à ce moment et mordre sur la réalité. Au moment suivant, les limites se sont déjà déplacées. Il peut arriver que l'état des esprits permette à un gouvernement de prendre une certaine mesure trois mois avant qu'elle ne devienne nécessaire, alors qu'au moment ou elle s’impose l'état des esprits ne lui laisse plus passage. Il fallait la prendre trois mois plus tôt. Sentir, percevoir perpétuellement ces choses, c'est savoir gouverner.\par
Le cours du temps est l'instrument, la matière, l'obstacle de presque tous les arts. Qu'entre deux notes de musique une pause se prolonge un instant de plus qu'il ne faut, que le chef d'orchestre ordonne un crescendo à tel moment et non une minute plus tard, et l'émotion musicale ne se produit pas. Qu'on mette dans une tragédie à tel moment une brève réplique au lieu d'un long discours, à tel autre un long discours au lieu d'une brève réplique, qu'on place le coup de théâtre au troisième acte au lieu du quatrième, et il n'y a plus de tragédie. Le remède, l'intervention chirurgicale qui sauve un malade à telle étape de sa maladie aurait pu le perdre quelques jours plus tôt. Et l'art de gouverner serait seul soustrait à cette condition de l'opportunité ? Non, il y est astreint plus qu'aucun autre. Le gouvernement aujourd'hui défunt ne l'a jamais compris. Sans même parler de la sincérité, de la sensibilité, de l'élévation morale qui rendent Léon Blum cher, à juste titre, à ceux que n'aveugle pas le partis-pris, où trouverait-on, dans les sphères politiques françaises, un homme d'une pareille intelligence ? Et pourtant l'intelligence politique lui fait défaut. Il est comme ces auteurs dramatiques qui ne conçoivent leurs ouvrages que sous la forme du livre imprimé ; leurs pièces de théâtre ne passent jamais la rampe, parce que les choses qu'il faut ne sont jamais dites au moment qu'il faut. Ou comme ces architectes qui savent faire sur le papier de beaux dessins, mais non conformes aux lois des matériaux de construction. On croit d'ordinaire définir convenablement les gens de ce caractère en les traitant de purs théoriciens. C'est inexact. Ils pèchent non par excès, mais par insuffi­sance de théorie. Ils ont omis d'étudier la matière propre de leur art.\par
La matière propre de l'art politique, c'est la double perspective, toujours instable, des conditions réelles d'équilibre social et des mouvements d'imagi­nation collective. Jamais l'imagination collective, ni celle des foules populai­res, ni celle des dîners en smoking, ne porte sur les facteurs réellement décisifs de la situation sociale donnée ; toujours ou elle s'égare, ou retarde, ou avance. Un homme politique doit avant tout se soustraire à son influence, et la considérer froidement du dehors comme un courant à employer en qualité de force motrice. Si des scrupules légitimes lui défendent de provoquer des mouvements d'opinion artificiellement et à coups de mensonges, comme on fait dans les États totalitaires et même dans les autres, aucun scrupule ne peut l'empêcher d'utiliser des mouvements d'opinion qu'il est impuissant à rectifier. Il ne peut les utiliser qu'en les transposant. Un torrent ne fait rien, sinon creuser un lit, charrier de la terre, parfois inonder ; qu'on y place une turbine, qu'on relie la turbine à un tour automatique, et le torrent fera tomber des petites vis d'une précision miraculeuse. Mais la vis ne ressemble nullement au torrent. Elle peut sembler un résultat insignifiant au regard de ce formidable fracas ; mais quelques-unes de ces petites vis placées dans une grosse machine pourront permettre de soulever des rochers qui résistaient à l'élan du torrent. Il peut arriver qu'un grand mouvement d'opinion permette d'accomplir une réforme en apparence sans rapport avec lui, et toute petite, mais qui serait impossible sans lui. Réciproquement il peut arriver que faute d'une toute petite réforme un grand mouvement d'opinion se brise et passe comme un rêve.\par
Pour prendre un exemple parmi bien d'autres, au mois de juin 1936, parce que les usines étaient occupées et que les bourgeois tremblaient au seul mot de soviet, il était facile d'établir la carte d'identité fiscale et toutes les mesures propres à réprimer les fraudes et l'évasion des capitaux, bref d'imposer jusqu'à un certain point le civisme en matière financière. Mais ce n'était pas encore indispensable, et l'occupation des usines accaparait l'attention du gouverne­ment comme celle des multitudes ouvrières et bourgeoises. Quand ces mesures sont apparues comme la dernière ressource, le moment de les imposer était passé. Il fallait prévoir. Il fallait profiter du moment où le champ d'action du gouvernement était plus large qu'il ne pouvait jamais l'être par la suite pour faire passer au moins toutes les mesures sur lesquelles avaient trébuché les gouvernements de gauche précédents, et quelques autres encore. C'est là que se reconnaît la différence entre l'homme politique et l'amateur de politique. L'action méthodique, dans tous les domaines, consiste à prendre une mesure non au moment où elle doit être efficace, mais au moment où elle est possible en vue de celui où elle sera efficace. Ceux qui ne savent pas ruser ainsi avec le temps, leurs bonnes intentions sont de la nature de celles qui pavent l'enfer.\par
Parmi tous les phénomènes singuliers de notre époque, il en est un digne d'étonnement et de méditation ; c'est la social-démocratie. Quelles différences n'y a-t-il pas entre les divers pays européens, entre les divers moments critiques de l'histoire récente, entre les diverses situations ! Cependant, pres­que partout, la social-démocratie s'est montrée identique à elle-même, parée des mêmes vertus, rongée des mêmes faiblesses. Toujours les mêmes excel­lentes intentions qui pavent si bien l'enfer, l'enfer des camps de concentration. Léon Blum est un homme d'une intelligence raffinée, d'une grande culture ; il aime Stendhal, il a sans doute lu et relu la Chartreuse de Parme ; il lui manque cependant cette pointe de cynisme indispensable à la clairvoyance. On peut tout trouver dans les rangs de la social-démocratie, sauf des esprits véritable­ment libres. La doctrine est cependant souple, sujette à autant d'interprétations et modifications qu'on voudra ; mais il n'est jamais bon d'avoir derrière soi une doctrine, surtout quand elle enferme le dogme du progrès, la confiance inébranlable dans l'histoire et dans les masses. Marx n'est pas un bon auteur pour former le jugement ; Machiavel vaut infiniment mieux.\par

\begin{center}
\end{center}
\subsection[III. Colonies]{III. \\
Colonies}
\subsubsection[1. Le Maroc ou de la prescription en matière de vol, (10 février 1937)]{1. \\
Le Maroc ou de la prescription en matière de vol \\
(10 février 1937)}
\noindent \par
Le début de l'année 1937 nous a apporté une chaude alerte. Le territoire de la patrie était menacé. Toute la presse quotidienne, sans aucune exception, unanime comme en ces quatre années si belles, trop vite écoulées, où le cœur de tous les Français battait à l'unisson, toute la presse s'est dressée fièrement pour la défense de ce sol sacré. Les dissensions civiles se sont effacées devant ce magnifique élan.\par
Oui, le territoire de la patrie était menacé. Quelle portion du territoire, à propos ? L'Alsace-Lorraine ? Oui, précisément. Ou plutôt non, ce n'était pas exactement l'Alsace-Lorraine, mais quelque chose d'équivalent. C'était le Maroc. Oui, le Maroc, cette province si essentiellement française. Chose à peine croyable, l'Allemagne semblait manifester des velléités de mettre la main sur la population marocaine, de l'arracher aux traditions héritées de ses ancêtres, les Gaulois, aux cheveux blonds, aux yeux bleus. Prétention absurde ! Le Maroc a toujours fait partie de la France. Ou sinon toujours, du moins depuis un temps presque immémorial. Oui, exactement depuis décembre 1911. Pour tout esprit impartial, il est évident qu'un territoire qui est à la France depuis 1911 est français de droit pour l'éternité.\par
C'est ce qui apparaît d'ailleurs encore plus clairement si on se reporte à l'histoire du Maroc. Cette histoire doit faire sentir au plus indifférents que le Maroc est pour la France en quelque sorte une seconde Lorraine.\par

\begin{center}
*\end{center}
\noindent Jusqu'en 1904, l'indépendance du Maroc n'avait jamais été mise en question, du moins dans des textes diplomatiques. Il était seulement convenu par le traité de Madrid (1880) que toute, les puissances y avaient droit, pour leur commerce, au traitement de la nation la plus favorisée.\par
En 1904, la France et l'Angleterre éprouvèrent le besoin de régler leurs comptes, à la suite de l'échec infligé à la France à Fachoda. La France, jusque-là, avait noblement défendu, au nom des droits de l'homme, l'indépendance du peuple égyptien. En 1904, elle autorisa l'Angleterre à fouler aux pieds cette indépendance. En échange, l'Angleterre lui abandonna le Maroc.\par
Un traité fut signé, comportant la mainmise immédiate de l'Angleterre sur l'Égypte, et le partage éventuel du Maroc entre la France et l'Espagne. Comme la France est toujours loyale, ce partage ne fut inscrit que dans les clauses secrètes du traité. Les clauses publiques, elles, garantissaient solennellement l'indépendance du Maroc.\par
L'Allemagne eut-elle vent de quelque chose ? En tout cas ce traité franco-anglais ne lui disait rien de bon. Elle voulait avoir sa part au Maroc. Prétention insoutenable ! Dès ce moment, le Maroc appartenait de droit à la France. Ne l'avait-elle pas payé ? Elle l'avait payé de la liberté des Égyptiens.\par
Guillaume II fit un discours retentissant à Tanger. L'Allemagne réclama une conférence internationale pour résoudre la question marocaine. Delcassé, ministre des Affaires étrangères, tint tête. On était exactement au bord de la guerre quand Delcassé fut écarté. Il était, on peut le dire, moins cinq. Le successeur de Delcassé céda.\par
L'Acte d'Algésiras (1906), signé de toutes les puissances européennes, n'accordait à la France aucun privilège, sauf celui de fournir au Sultan, pour cinq ans, quelques dizaines d'instructeurs pour sa police indigène. Il ne devait y avoir au Maroc aucune force militaire européenne, et les diverses puissances devaient y jouir de droits économiques égaux.\par
Dès lors, la question qui se posait était : comment violer l'Acte d'Algésiras ? En effet, cet Acte était nul de plein droit, puisqu'il n'accordait pas le Maroc à la France. Ce point doit être clair pour toute intelligence moyenne.\par
Seuls des esprits primaires pourraient rapprocher la violation de l'Acte d'Algésiras et la violation du Traité de Versailles. Ces deux cas sont sans aucun rapport. L'Acte d'Algésiras défavorisait la France, il était donc caduc dès son apparition. Le Traité de Versailles devait être éternel pour la raison contraire.\par
Après 1906, on essaya diverses combinaisons avec l'Allemagne, elle aussi désireuse de violer l'Acte d'Algésiras, mais - avidité monstrueuse ! - à condi­tion d'y trouver un profit. On alla jusqu'à lui offrir un port au Maroc avec « Hinterland ». On essaya de partager avec elle le pouvoir économique au Maroc, mais comme en même temps la France tenait à se réserver tout le pouvoir politique, cette solution s'avéra impraticable.\par
Enfin, en 1917, la France sentit qu'il était temps d'agir. Elle envoya purement et simplement des troupes à Fez, capitale du Maroc. Elle allégua qu'il y avait des commencements de troubles qui mettaient en danger la vie des Européens, et promit de retirer les troupes dès que la sécurité serait rétablie. On n'a jamais su s'il y avait eu effectivement danger. En tout cas l'occupation militaire de Fez, accomplie sans consultation formelle des puissances signataires de l'Acte d'Algésiras, déchirait enfin cet Acte ridicule.\par
Une fois installée à Fez, il va de soi que la France ne s'en retira plus. Le souci du prestige, bien plus important - quand il s'agit de la France - que le droit international, le lui interdisait.\par
Au bout de quelques mois, l'Allemagne, voyant que les troupes françaises étaient toujours à Fez, envoya un navire de guerre sur la côte marocaine, à Agadir. Elle s'obstinait à réclamer sa part.\par
Caillaux, qui venait d'arriver au pouvoir, entama les négociations. Elles se terminèrent fin 1911. Dans l'intervalle, la guerre avait été plusieurs fois sur le point d'éclater. Enfin un traité franco-allemand reconnut le protectorat français au Maroc, contre la cession d'une petite partie du Congo français au Cameroun allemand.\par
Le gouvernement allemand s'était laissé jouer. L'Allemagne le sentit. L'explosion d'août 1914 fut sans doute pour une part une suite de l'expédition militaire à Fez. Du moins c'est l'opinion exprimée par Jaurès dans son dernier discours (à Vaise, le 28 juillet 1914).\par
Le plus beau, c'est qu'après la victoire on a repris le morceau du Congo cédé en 1911, et on a pris le Cameroun, et on a gardé le Maroc.\par

\begin{center}
*\end{center}
\noindent \par
À présent, l'Allemagne prétend mettre en cause les clauses coloniales du Traité de Versailles. Elle peut le faire de deux manières. Elle peut réclamer le Cameroun tel qu'elle l'avait en 1914, ou elle peut considérer le traité de 1911 comme annulé par Versailles, et réclamer les droits sur le Maroc qu'elle avait échangés contre l'agrandissement du Cameroun.\par
La question ne se pose pas, heureusement. Chacun sait que le Traité de Versailles est intangible. Et puis le Maroc est devenu la chair même de la France, du fait des sacrifices accomplis pour lui. Sacrifices non seulement en hommes et en argent, mais d'un ordre bien plus grave. En vue du Maroc, la France s'est comportée en vraie « puissance coloniale » : - elle a vendu les libertés égyptiennes, signé un traité dont les clauses secrètes contredisaient les clauses publiques, violé ouvertement un autre traité. De pareils sacrifices moraux, pour la nation la plus loyale du monde, confèrent des droits sacrés.\par
Aussi, que l'Allemagne le sache bien, le moindre débarquement de troupes allemandes au Maroc nous trouverait tous résolus à tuer et à mourir !\par
Il est vrai qu'aux dernières nouvelles il semble qu'il n'y ait pas eu de troupes allemandes au Maroc. Qu'importe ? La présence d'ingénieurs allemands au Maroc espagnol est incontestable ; l'envoi en Allemagne de minerai de fer marocain aussi. Il est évident que toute mainmise économique de l'Allemagne sur une portion du Maroc serait intolérable. Aucun traité ne l'interdit, mais cette interdiction est sous-entendue.\par
L'Allemagne manque du sens le plus élémentaire des convenances. À preuve cette histoire de concessions économiques dans les colonies portu­gaises. Bien sûr, aucun traité n'interdit au Portugal et à l'Allemagne des arrangements de cet ordre. Mais une interdiction devrait-elle être nécessaire ?\par
Puisque l'Allemagne a besoin qu'on mette les points sur les {\itshape i}, nous le ferons. Nous avions voulu, par politesse, lui épargner certaines vérités désagréables, espérant qu'elle saurait se tenir à sa place.\par
Puisqu'elle ne le sait pas, que notre gouvernement convoque une confé­rence internationale pour compléter le Traité de Versailles par deux additifs : Un additif dans le préambule, comportant la définition suivante :\par
« Toute situation internationale où l'Allemagne est économiquement, militairement et politiquement inférieure à la France constitue un état de paix. Tout ce qui ferait tendre les forces de l'Allemagne à égaler ou à dépasser celles de la France constituerait une provocation à la guerre. »\par
Et une clause nouvelle, dont la légitimité crève les yeux :\par
« Toute expansion économique de l'Allemagne, soit par rapport aux débouchés, soit par rapport aux matières premières, est contraire au droit international. Des dérogations ne seront possibles qu'avec l'autorisation for­melle de la France. »\par
Si le gouvernement de Front populaire, si les partis du Front populaire n'ont pas encore compris que le devoir est là, le Comité de Vigilance saura le leur rappeler.\par
Et autour d'une politique si juste se réalisera, enfin, l'union de la nation française !\par
({\itshape Vigilance}, n° 48/49, 10 février 1937.)\par

\subsubsection[2. Le sang coule en Tunisie, (mars 1937)]{2. \\
Le sang coule en Tunisie \\
(mars 1937)}
\noindent \par
« Du sang à la une » dans les journaux ouvriers. Le sang coule en Tunisie. Qui sait ? On va peut-être se souvenir que la France est un petit coin d'un grand Empire, et que dans cet Empire des millions et des millions de travailleurs souffrent ?\par
Il y a huit mois que le Front populaire est au pouvoir, mais ou n'a pas encore eu le temps de penser à eux. Quand des métallos de Billancourt sont en difficulté, Léon Blum reçoit une délégation ; il se dérange pour aller à l'Exposition parler aux gars du bâtiment ; quand il lui semble que les fonctionnaires grognent, il leur adresse un beau discours par radio tout exprès pour eux. Mais les millions de prolétaires des colonies, nous tous, nous les avions oubliés.\par
D'abord ils sont loin. Chacun sait que la souffrance diminue en raison de l'éloignement. Un homme qui peine sous les coups, épuisé par la faim, tremblant devant ses chefs, là-bas en Indochine, cela représente une souf­france et une injustice bien moindres qu'un métallo de la région parisienne qui n'obtient pas ses 15 \% d'augmentation, ou un fonctionnaire victime des décrets-lois. Il doit y avoir là une loi de physique qui se rapporte à l'inverse du carré de la distance. La distance a le même effet sur l'indignation et la sympa­thie que sur la pesanteur.\par
D'ailleurs ces gens-là - jaunes, noirs, « bicots » - sont habitués à souffrir. C'est bien connu. Depuis le temps qu'ils crèvent de faim et qu'ils sont soumis à un arbitraire total, ça ne leur fait plus rien. La meilleure preuve, c'est qu'ils ne se plaignent pas. Ils ne disent rien. Ils se taisent. Au fond, ils ont un caractère servile. Ils sont faits pour la servitude. Sans quoi ils résisteraient.\par
Il y en a bien quelques-uns qui résistent, mais ceux-là, ce sont des « meneurs », des « agitateurs », probablement payés par Franco et Hitler ; on ne peut employer vis-à-vis d'eux que des mesures de répression, comme la dissolution de l'Étoile Nord-Africaine.\par
Et puis il n'y a rien de spectaculaire dans le drame de ces gens-là. Du moins jusqu'au dernier incident. Des fusillades, des massacres, voilà qui parle à l'imagination ; cela fait sensation, cela fait du bruit. Mais les larmes versées en silence, les désespoirs muets, les révoltes refoulées, la résignation, l'épuisement, la mort lente - qui donc songerait à se préoccuper de pareilles choses ? Les gosses tués à Madrid par des bombes d'avion, cela cause un frisson d'indignation et de pitié. Mais tous les petits gars de dix ou douze ans, affamés et surmenés, qui ont péri d'épuisement dans les mines indochinoises, nous n'y avons jamais pensé. Ils sont morts sans que leur sang coule. Des morts pareilles, cela ne compte pas. Ce ne sont pas de vraies morts.\par
\par
Au fond, nous - et quand je dis nous, j'entends tous ceux qui adhérent à une organisation du Rassemblement Populaire - nous sommes exactement semblables aux bourgeois. Un patron est capable de condamner ses ouvriers à la plus atroce misère, et de s'émouvoir d'un mendiant rencontré sur son chemin ; et nous, qui nous unissons au nom de la lutte contre la misère et l'oppression, nous sommes indifférents au sort inhumain que subissent au loin des millions d'hommes qui dépendent du gouvernement de notre pays. Aux yeux des bourgeois, les souffrances physiques et morales des ouvriers n'existent pas tant qu'ils se taisent, et les patrons les contraignent à se taire par des moyens de force. Nous aussi, Français « de gauche », nous continuons à faire peser sur les indigènes des colonies la même contrainte impitoyable, et comme la terreur les rend muets, nous avons vaguement l'impression que les choses ne vont pas si mal là-bas, qu'on ne souffre pas trop, qu'on est accoutumé aux privations et à la servitude. La bourgeoisie s'intéresse à un crime, à un suicide, à un accident de chemin de fer, et ne pense jamais à ceux dont la vie est lentement écrasée, broyée et détruite par le jeu quotidien de la machine sociale. Et nous aussi, avides de nouvelles sensationnelles, nous n'avons pas accordé une pensée aux millions d'êtres humains qui n'espéraient qu'en nous, qui du fond d'un abîme d'esclavage et de malheur tournaient les yeux vers nous, et qui depuis huit mois, sans fracas, sans bruit, passent progressivement de l'espérance au désespoir.\par
À présent, le sang a coulé. La tragédie coloniale a fini par prendre la forme de fait divers, seule accessible à notre sensibilité et à notre intelligence rudimentaires. À partir de maintenant, nous ne pouvons plus nous vanter que la fameuse « expérience » s'accomplit sans effusion de sang. Du sang l'a souillée.\par
Il est facile de parler de responsabilités, de sabotage. Sans enquête, nous savons où sont les responsabilités. Que chacun de nous se regarde dans la glace, il verra l'un des responsables. Le gouvernement actuel ne gouverne-t-il pas au nom du Rassemblement Populaire ? Ses membres ne sont guère en cause ; surmenés, accablés comme ils le sont, il est forcé que leur activité dépende pour une grande part des préoccupations qu'on leur impose. Si, par exemple, Léon Blum avait eu l'impression que nous sommes plus préoccupés de l'esclavage colonial que des traitements des fonctionnaires, il aurait sûrement consacré aux colonies le temps passé à préparer aux fonctionnaires un beau discours.\par
Quoi qu'il en soit, on doit bien avouer que, jusqu'ici, l'œuvre coloniale du gouvernement se ramène à peu prés à la dissolution de l'Étoile Nord-Africaine. On dira que des réformes coloniales n'étaient pas prévues au programme du Rassemblement Populaire. La dissolution non motivée de l'Étoile Nord-Africaine n'y était pas non plus prévue. Les morts de Tunisie non plus, d'ailleurs. Ce sont des morts hors-programme.\par
Quand je songe à une guerre éventuelle, il se mêle, je l'avoue, à l'effroi et à l'horreur que me cause une pareille perspective, une pensée quelque peu réconfortante. C'est qu'une guerre européenne pourrait servir de signal à la grande revanche des peuples coloniaux pour punir notre insouciance, notre indifférence et notre cruauté.\par
({\itshape Feuilles libres}, mars 1937 .)\par

\begin{center}
\end{center}
\subsubsection[3. Qui est coupable de menées anti-françaises ? (1938 ?)]{3. \\
Qui est coupable de menées anti-françaises ? \\
(1938 ?)}
\noindent \par
En condamnant Messali à deux années de prison, le tribunal a écarté l'inculpation de menées antifrançaises. Que peut-on en conclure, sinon qu'on n'a pas pu trouver de menées antifrançaises du Parti du Peuple Africain ? Et sans doute, si on n'a pas pu en trouver, c'est qu'il n'y en avait pas.\par
Il n'en est pas moins certain que l'amour de la France n'est pas très vif en ce moment au cœur des populations nord-africaines. Il y a apparemment, sur ce territoire, des menées antifrançaises. Mais qui se livre à ces menées ? Qui est coupable de faire le jeu des ambitions fascistes en discréditant la France et le régime démocratique ?\par
Pour moi, je suis Française. Je n'ai jamais été en Afrique du Nord. J'ignore tout des intrigues compliquées auxquelles peuvent se livrer l'Allemagne et l'Italie dans la population musulmane. Je crois pourtant en savoir assez pour porter une accusation. Une accusation qu'aucun tribunal ne confirmera, bien sûr.\par
J'accuse l'État français et les gouvernements successifs qui l'ont représenté jusqu'à ce jour, y compris les deux gouvernements de Front Populaire ; j'accuse les administrations d'Algérie, de Tunisie, du Maroc ; j'accuse le général Noguès, j'accuse une grande partie des colons et des fonctionnaires français de menées antifrançaises en Afrique du Nord. Tous ceux à qui il est arrivé de traiter un Arabe avec mépris ; ceux qui font verser le sang arabe par la police ; ceux qui ont opéré et opèrent l'expropriation progressive des cultivateurs indigènes ; ceux qui, colons, industriels, traitent leurs ouvriers comme des bêtes de somme ; ceux qui, fonctionnaires, acceptent, réclament qu'on leur verse pour le même travail un tiers de plus qu'à leurs collègues arabes ; voilà quels sont ceux qui sèment en territoire africain la haine de la France.\par
Lors des occupations d'usines, en juin 1936, la France s'est divisée en deux camps. Les uns ont accusé les militants ouvriers, ces « meneurs », ces « agitateurs », d'avoir excité les troubles. Les autres - et ces autres, c'étaient notamment les membres et les partisans du Front Populaire - ont répondu : Non, ceux qui ont mis au cœur des ouvriers tant de révolte, tant d'amertume, qui les ont amenés à recourir enfin à la force, ce sont les patrons eux-mêmes, à cause de la contrainte, de la terreur, de la misère qu'ils avaient fait peser pendant des années sur les travailleurs des usines.\par
À ce moment, en juin 1936, les hommes « de gauche » avaient compris comment, en France, se posait le problème. Aujourd'hui, c'est de l'Afrique du Nord qu'il s'agit ; et ces mêmes hommes ne comprennent plus. C'est pourtant le même problème qui se pose ; mais ils ne s'en sont pas aperçus. C'est toujours, partout, le même problème qui se pose. Toujours, partout où il y a des opprimés.\par
Il s'agit toujours de savoir, là où il y a oppression, qui met au cœur des opprimés l'amertume, la rancune, la révolte, le désespoir. Est-ce que ce sont ceux des opprimés qui, les premiers, osent dire qu'ils souffrent, et qu'ils souffrent injustement ? Ou est-ce que ce sont les oppresseurs eux-mêmes, du seul fait qu'ils oppriment ?\par
Des hommes qui, étant brimés, offensés, humiliés, réduits à la misère, auraient besoin de « meneurs » pour avoir le cœur plein d'amertume, de tels hommes seraient nés esclaves. Pour quiconque a un peu de fierté, il suffit d'avoir été humilié pour avoir la révolte au cœur. Aucun « meneur » n'est nécessaire. Ceux qu'on appelle les « meneurs », c'est-à-dire les militants, ne créent pas les sentiments de révolte, ils les expriment simplement. Ceux qui créent les sentiments de révolte, ce sont les hommes qui osent humilier leurs semblables.\par
Y a-t-il quelque part une race d'hommes si naturellement serviles qu'on puisse les traiter avec mépris sans exciter en eux, tout au moins, une protestation muette, une rancune impuissante ? Ce n'est certainement pas le cas de la race arabe, si fière lorsqu'elle n'est pas brisée par une force impitoya­ble. Mais ce n'est le cas d'aucune race d'hommes. Tous les hommes, quels que soient leur origine, leur milieu social, leur race, la couleur de leur peau, sont des êtres naturellement fiers. Partout où on opprime des hommes, on excite la révolte aussi inévitablement que la compression d'un ressort en amène la détente.\par
Cette vérité, les hommes qui sont aujourd'hui au pouvoir la comprennent un peu lorsque les opprimés sont des ouvriers français, et les oppresseurs, les patrons. Ils ne la comprennent plus du tout lorsque les opprimés sont les indigènes des colonies, et les oppresseurs, entre autres, eux-mêmes, hommes au pouvoir. Pourquoi ? Croient-ils que le fait d'avoir la peau de couleur un peu foncée rend l'humiliation plus facile à supporter ? S'ils le croient, j'appelle de tous mes vœux le jour où les faits les forceront de reconnaître qu'ils se sont trompés. Le jour où les populations indigènes des colonies françaises auront enfin l'équivalent de ce qu'ont été, pour les ouvriers français, les journées de juin 1936.\par
Je n'oublierai jamais le moment où, pour la première fois, j'ai senti et compris la tragédie de la colonisation. C'était pendant l'Exposition Coloniale, peu après la révolte de Yen-Bay en Indochine. Un jour, par hasard, j'avais acheté {\itshape le Petit Parisien} ; j'y vis, en première page, le début de la belle enquête de Louis Roubaud sur les conditions de vie des Annamites, leur misère, leur esclavage, l'insolence toujours impunie des blancs. Parfois, le cœur plein de ces articles, j'allais à l'Exposition Coloniale ; j'y trouvais une foule béate, inconsciente, admirative. Pourtant beaucoup de ces gens avaient certainement lu, le matin même, un article poignant de Louis Roubaud.\par
Il y a sept ans de cela. Je n'eus pas de peine, peu de temps après, à me convaincre que l'Indochine n'avait pas le privilège de la souffrance parmi les colonies françaises. Depuis ce jour, j'ai honte de mon pays.\par
Depuis ce jour, je ne peux pas rencontrer un Indochinois, un Algérien, un Marocain, sans avoir envie de lui demander pardon. Pardon pour toutes les douleurs, toutes les humiliations qu'on lui a fait souffrir, qu'on a fait souffrir à leur peuple. Car leur oppresseur, c'est l'État français, il le fait au nom de tous les Français, donc aussi, pour une petite part, en mon nom. C'est pourquoi, en présence de ceux que l'État français opprime, je ne peux pas ne pas rougir, je ne peux pas ne pas sentir que j'ai des fautes à racheter.\par
Mais si j'ai honte de mon pays depuis sept ans, j'éprouve, depuis un an et demi, un sentiment encore plus douloureux. J'ai honte de ceux dont je me suis toujours sentie le plus proche. J'ai honte des démocrates français, des socia­listes français, de la classe ouvrière française.\par
Que les ouvriers français, mal informés, harassés par le travail d'usine, ne se préoccupent pas beaucoup de ce qui se passe dans des territoires lointains, c'est assez excusable. Mais depuis des années ils voient leurs compagnons de travail nord-africains souffrir à leurs côtés plus de souffrances qu'eux-mêmes, subir plus de privations, plus de fatigues, un esclavage plus brutal. Ils savent que ces malheureux sont encore des privilégiés par rapport aux autres malheureux qui, poussés par la faim, ont vainement essayé de venir en France. Le contact a pu s'établir entre travailleurs français et arabes au cours des longues journées d'occupation des usines. Les ouvriers français ont constaté à ce moment-là comment l{\itshape 'Étoile Nord-Africaine} les a soutenus ; ils l'ont vue défiler avec eux le 14 juillet 1936. Pourtant ils l'ont laissé dissoudre sans protester. Ils sont restés indifférents à la condamnation de Messali. Ils voient, semble-t-il, avec indifférence leurs malheureux camarades privés d'allocations familiales.\par
Quant aux organisations antifascistes, elles se chargent, par leur attitude à l'égard des colonies, d'une honte ineffaçable. Y a-t-il beaucoup d'hommes, parmi les militants ou les simples membres de la S.F.I.O. et de la C.G.T., qui ne s'intéressent pas beaucoup plus au traitement d'un instituteur français, au salaire d'un ajusteur français, qu'à la misère atroce qui fait périr de mort lente les populations d'Afrique du Nord ?\par

\begin{center}
*\end{center}
\noindent Les outrages déshonorent ceux qui les infligent bien plus que ceux qui les subissent. Toutes les fois qu'un Arabe ou un Indochinois est insulté sans pouvoir répondre, frappé sans pouvoir rendre les coups, affamé sans pouvoir protester, tué impunément, c'est la France qui est déshonorée. Et elle est, hélas, déshonorée de cette manière tous les jours.\par
Mais l'outrage le plus sanglant, c'est quand elle envoie de force ceux qu'elle prive de leur dignité, de leur liberté, de leur pays, mourir pour la digni­té, la liberté, la patrie de leurs maîtres. Dans l'antiquité, il y avait des esclaves, mais les citoyens seuls combattaient. Aujourd'hui on a trouvé mieux ; on réduit d'abord des populations entières à l'esclavage, et ensuite on s'en sert comme de chair à canon.\par
Pourtant les opprimés des colonies peuvent trouver une amère consolation dans la pensée que leurs vainqueurs subissent parfois à cause d'eux une misère égale à celle qu'ils leur infligent. Quand on étudie l'histoire de l'avant-guerre, on voit que c'est le conflit concernant le Maroc qui a envenimé les rapports franco-allemands au point de faire tourner, en 1914, l'attentat de Sarajevo en catastrophe mondiale. La France a vaincu et soumis les Marocains, mais c'est à cause de ces Marocains vaincus et soumis que tant de Français ont croupi pendant quatre ans dans les tranchées. Ce fut leur punition, et elle était méritée. Aujourd'hui, si un nouveau conflit éclate, la question coloniale en sera encore l'origine. Une fois de plus les Français souffriront, mourront, et une fois de plus ils l'auront mérité.\par
Quant à l'Afrique du Nord, j'aime à croire qu'elle perd de plus en plus l'envie d'être un réservoir de chair à canon. Il n'est pas besoin, pour lui faire perdre cette envie un peu plus tous les jours, que Berlin, Rome ou Moscou exercent leur influence. La France s'en charge.\par
De même il n'est pas besoin de Rome ni de Berlin pour que l'Afrique du Nord se détache un peu plus tous les jours de la cause antifasciste. Le Front Populaire, parvenu au pouvoir, s'en charge, en continuant à laisser subir aux populations d'Afrique du Nord plus de douleurs et plus d'outrages que n'en subissent les peuples soumis aux régimes fascistes.\par
Le principal auteur des menées antifrançaises en Afrique du Nord, c'est la France. Les principaux auteurs de menées fascistes en Afrique du Nord, ce sont, sauf exceptions, les organisations antifascistes.\par

\subsubsection[4. “ Ces membres palpitantsde la patrie ”, (10 mars 1938)]{4. \\
“ Ces membres palpitantsde la patrie ” \\
(10 mars 1938)}
\noindent \par
Il y a quelques semaines, un article paru dans notre grande presse d'information, se réclamant pour une fois de Jaurès, et voulant écraser d'un coup tous les raisonnements possibles en faveur des revendications alleman­des, appelait les colonies « ces membres palpitants de la patrie » . On ne peut refuser à cette expression un singulier bonheur, une grande valeur d'actualité. Palpitants, oui. Sous la faim, les coups, les menaces, les peines d'emprisonne­ment ou de déportation ; devant l'aspect redoutable des mitrailleuses ou des avions de bombardement. Une population domptée, désarmée serait palpitante à moins.\par
Si les colonies sont palpitantes, la mère-patrie ne palpite guère avec elles. La tragédie de l'Afrique du Nord se poursuit au milieu d'une indifférence presque complète. {\itshape Le Populaire} du moins avait publié, sur le Maroc, une série émouvante d'articles de Magdeleine Paz. Les autres journaux, ou bien ne se sont pas aperçus qu'il y a une crise nord-africaine, ou bien y ont vu exclu­sivement une crise de l'autorité française.\par
En vérité, il semble que les Français aient été bien plus remués par les événements de Chine que par les événements d'Afrique du Nord. Sans doute en Chine, on tue beaucoup plus de gens, on y tue même des enfants - à ce propos, comment vivront donc les enfants de ceux qui sont tombés récemment sous les balles françaises au Maroc ? Mais enfin, ce qui se passe en Chine, nous n'y pouvons pas grand-chose ; et il n'est pas sûr qu'une action dans ce domaine ne mettrait pas le feu à l'Europe et au monde. Tandis qu'en Afrique du Nord on pourrait être un peu humain, on pourrait préserver des vies d'enfants - car les enfants ne meurent pas seulement sous les bombes d'avion, la faim les tue très bien - sans courir des risques si effroyables. Il suffirait de le vouloir.\par
En voyant aujourd'hui tant de bons bourgeois, d'un impérialisme naïf, s'émouvoir pour la Chine, exécrer les Japonais, on se demande malgré soi si les sympathies qu'excite en France la Chine ne sont pas du même ordre que celles éprouvées par les riches en faveur des « bons pauvres » , des pauvres qui « savent rester à leur place » . La Chine, jusqu'ici, a su rester à sa place, sa place de peuple inférieur, humblement respectueux des blancs. Les Japonais sont des jaunes intolérablement présomptueux : ils veulent civiliser en massacrant - ils veulent faire comme les blancs !Quant aux Nord-Africains, quelques-uns d'entre eux - de simples « meneurs » , heureusement - sont peut-être encore pires : ils ne veulent pas être massacrés, ni même brimés et humiliés. Prétention d'autant plus exorbitante que, le jour où la France, en la personne de son gouvernement ou d'un ambassadeur, aura subi une humilia­tion, on les autorisera à tuer et à mourir pour venger cette humiliation. Que leur faut-il de plus, en fait de dignité ?\par
\par

\begin{center}
*\end{center}
\noindent Parmi tous les événements qui se sont passés récemment en Afrique du Nord, le plus caractéristique peut-être, bien qu'il y en ait eu de plus tragiques, est l'histoire de l{\itshape 'Étoile Nord-Africaine}.\par
L'{\itshape Étoile Nord-Africaine} fut autrefois tenue sur les fonts de baptême par le parti communiste premier style. Au bout d'un certain temps, elle a su conquérir son indépendance d'organisation adulte ; c'est ce qui lui a permis, ces dernières années, de ne pas se retourner contre les revendications vitales des peuples colonisés. Elle est composée exclusivement de Nord-Africains, ou plus exactement d'Algériens, et exclusivement de travailleurs, au sens le moins large du terme ; elle ne compte dans ses rangs ni un blanc, ni un intellectuel. Son influence, sans être insignifiante en Algérie, s'exerce surtout en France, où elle a su grouper la très grande majorité des travailleurs algériens.\par
La plupart des Français ignorent dans quelles conditions vivent et ont vécu, surtout avant juin 1936, les ouvriers algériens qui travaillent chez nous. Privés de la plupart des droits dont jouissent leurs camarades français, tou­jours passibles d'un renvoi brutal dans leur pays d'origine qu'ils ont quitté chassés par la faim, voués aux tâches les plus malpropres et les plus épuisantes, misérablement payés, traités avec mépris même par ceux de leurs compagnons de travail qui ont une peau d'autre couleur, il est difficile d'imaginer plus complète humiliation. L'{\itshape Étoile Nord-Africaine} a su donner à ces hommes une dignité, un but, une organisation à eux, un idéal à eux ; cet idéal ne les rattachait pas seulement à l'ensemble du monde musulman, il les rattachait d'une manière bien plus étroite à l'ensemble de leurs frères de classe, y compris ceux qui méconnaissaient cette fraternité en les traitant en inférieurs. C'est grâce à l'{\itshape Étoile Nord-Africaine} que les patrons n'ont pas trouvé en eux une masse de jeunes manœuvrables à merci ; c'est grâce à elle, notamment, qu'ils ont participé à l'occupation des usines en juin 1936, assurant ainsi la victoire, au lieu du désastre, dans un certain nombre d'usines importantes où ils constituaient une large part du personnel. L{\itshape 'Étoile Nord-Africaine} a défilé en rangs pressés dans le cortège du 14 juillet 1936, fournissant le spectacle le plus poignant peut-être dans cette journée si riche en émotions. Aujourd'hui, les trois ou quatre hommes dont le travail, le courage, l'intelligence ont rendu cette grande chose possible, sont en prison dans une prison française et pour deux ans.\par
Bien sûr l'{\itshape Étoile Nord-Africaine} faisait partie de ce qu'on appelle le nationalisme Nord-Africain. Son rêve lointain était la constitution progressive d'un État de l'Afrique du Nord, dont les rapports avec la France auraient pu être, par exemple, ceux d'un Dominion anglais avec l'Angleterre. Ses reven­dications immédiates étaient l'extension des libertés démocratiques aux indigènes, la suppression du Code de l'indigénat, cet ensemble de contraintes à côté de quoi les régimes totalitaires apparaissent, par comparaison, presque libéraux, et, en France, l'égalité des travailleurs algériens et des travailleurs français. Comme toutes les organisations qui groupent des opprimés, comme, par exemple, les organisations du prolétariat français, elle hésitait entre une opposition radicale, violente, et le réformisme, penchant vers l'un ou vers l'autre selon qu'il apparaissait ou non des possibilités de réformes. Le {\itshape Rassem­blement populaire} lui donna l'espérance de progrès importants et paisibles ; elle y adhéra avec enthousiasme. Quand Viénot conclut le traité franco-syrien, sa grande revendication fut l'élaboration progressive d'un statut analogue pour l'Afrique du Nord. Certains affirmeront-ils que ces dispositions pacifiques étaient feintes, que {\itshape l'Étoile Nord-Africain}e ne rêvait que de violences ? Encore faudrait-il le prouver. Ce qui est incontestable, c'est que l'{\itshape Étoile} n'a pas changé de politique entre le moment où elle a été reçue au {\itshape Rassemblement populaire}, où elle a pris part au défilé du 14 juillet, et le moment où soudain, brutalement, le gouvernement Blum l'a dissoute.\par

\begin{center}
*\end{center}
\noindent \par
On n'a jamais donné les motifs de cette dissolution. On s'est contenté de prendre des airs mystérieux, en insinuant « Ah ! si vous saviez ce que nous savons ! » Nous connaissons ces airs-là. Bien naïfs ceux sur qui ils feraient impression. Mais le plus intéressant, c'est ce qui a suivi. Quelques organi­sations adhérant au {\itshape Rassemblement populaire} ont proposé à ce dernier d'exclure l'{\itshape Étoile} en raison du décret de dissolution porté contre elle. On considérait donc, notons-le, que, bien que dissoute, elle était toujours membre du {\itshape Rassemblement populaire}, puisqu'on proposait de l'exclure. Le représentant de la C.G.T. et celui du C.V.I.A. demandèrent et obtinrent qu'elle ne fût pas exclue sans que son chef, Messali, fût entendu. Messali constitua un dossier, le communiqua à quelques membres du {\itshape Comité de Rassemblement populaire.} Cependant il ne fut pas convoqué officiellement pour être entendu, et la ques­tion de l'exclusion ne fut plus posée. L{\itshape 'Étoile Nord-Africaine} bien que dissoute depuis des mois, est donc toujours membre du Rassemblement populaire !\par
Messali, ayant sous les yeux l'exemple des ligues fascistes, pouvait à bon droit considérer la dissolution comme une invitation à reconstituer une organisation semblable sous un autre nom. Il est vrai qu'en y réfléchissant bien, il y a quelque chose comme une action judiciaire intentée contre les ligues fascistes ; mais elle ressemble singulièrement à une inaction judiciaire. Au reste, cette action, si action il y a, repose sur une définition des ligues caractérisées comme des organisations para-militaires. Tel n'a jamais été le caractère de l'{\itshape Étoile}, et, à ma connaissance, on ne l'en a même jamais accusée. S'il en avait été autrement, aurait-elle été admise au {\itshape Rassemblement popu­laire} ? Cependant c'est pour avoir reconstitué cette organisation qui n'est pas une ligue, qui est toujours membre du {\itshape Rassemblement populaire}, que, sous un gouvernement qui émane du {\itshape Rassemblement populaire}, Messali et trois de ses camarades ont été condamnés à deux ans de prison. Pour ce seul délit ; car l'inculpation de menées antifrançaises a été écartée par le tribunal, qui a retenu seulement celle de reconstitution de ligue dissoute.\par
\par
Peut-on se permettre de demander ce que doivent faire les hommes, les militants qui ont appartenu à l{\itshape 'Étoile Nord-Africaine} ? S'ils veulent se grouper, on pourra toujours les accuser d'avoir reconstitué l'{\itshape Étoile.} C'est à vrai dire une pure et simple interdiction de s'organiser, et sous peine de prison, qui a été portée contre eux sans aucune explication. Ce ne sont pas seulement les quatre militants frappés par la condamnation qui en subissent durement l'atteinte, c'est bien plus encore tant de milliers d'hommes malheureux, opprimés, qui n'avaient à eux que leur organisation, et qui en sont privés. Croit-on sérieu­sement qu'ils se résigneront à cet état de chose, et qu'ils n'iront pas du seul côté où apparemment il soit permis de s'organiser, c'est-à-dire à droite ? On nous dit qu'il y avait des Algériens parmi les « cagoulards ». S'il n'y avait pas des milliers, des milliers et des milliers d'Algériens, ce n'est pas la faute de notre gouvernement. Et si un jour comme en Espagne, l'Afrique du Nord déverse chez nous des flots d'indigènes armés sous la conduite de généraux factieux, la « justice immanente » ne serait-elle pas sans doute satisfaite au moment où tels grands personnages périraient de la main d'un Arabe ?\par
On colporte, bien entendu, contre l'{\itshape Étoile Nord-Africaine}, les mêmes bruits de collusion avec le fascisme espagnol ou italien qu'on a colportés lors­qu'on voulait l'exclure du {\itshape Rassemblement populaire} ; à ce moment, Messali les a complètement réfutés. Ce qui était faux alors serait-il devenu vrai depuis ? Comme on a pris soin de mettre Messali et ses camarades en prison, il leur est difficile de prouver le contraire ; qui sait d'ailleurs ce que peut devenir une organisation composée d'hommes malheureux, en général igno­rants, quand on la prive brutalement des chefs en qui elle a mis sa confiance ?\par

\begin{center}
*\end{center}
\noindent Au reste, ces collusions avec le fascisme, si - comme je le crois pour ma part - elles n'existent pas, existeront indubitablement pour peu que la même politique se poursuive. Ceux qui préconisent cette politique triompheront alors d'avoir vu si clair. Ils ne comprendront pas que les vrais auteurs de ces collusions, ce sont eux, et je parle pour les membres du gouvernement respon­sables de cette politique comme pour ceux qui les ont conseillés.\par
Ce sont eux qui sont coupables de menées antifrançaises en Afrique du Nord, en achevant d'y rendre la France odieuse. Eux qui, dès mars 1937, trouvaient presque naturel que la police tire sur les grévistes, dès lors que ces grévistes étaient simplement des mineurs indigènes de Tunisie, contraints de travailler douze heures, à un rythme épuisant, pour des salaires infimes ; Blum, qui a pleuré après Clichy, n'a pas jugé les dix-neuf morts arabes de Metlaoui dignes de ses larmes. Eux qui ont laissé le général Noguès terminer cette même année 1937 au Maroc par la provocation, la terreur et les tueries. Eux qui ont fait si peu que rien pour donner aux milliers de milliers d'hommes qui subissent la faim et l'esclavage en Afrique du Nord plus de pain et de liberté, pour aménager la culture, alléger le budget, réformer le cade de l'Indigénat. Eux qui refusent aux Nord-Africains venus en France le bénéfice des allocations familiales pour les enfants demeurés en Afrique du Nord, les contraignant à des privations inhumaines pour envoyer de maigres mandats. Eux qui ont condamné Messali à la privation des droits civiques, au moment même où les élections cantonales lui donnaient une victoire éclatante dès le premier tour. Et ce ne sont là que quelques faits cités au hasard.\par
Ils sont plaisants vraiment, ceux qui parlent avec scandale et comme d'un crime, de collusions possibles entre les indigènes Nord-Africains et le fascisme. Et pourquoi donc, ayant tâté de tout le reste et toujours vu leurs espoirs déçus, ne tâteraient-ils pas aussi du fascisme avant de sombrer dans un complet désespoir ? Sans doute savons-nous bien qu'avec le fascisme les malheureux ne tomberont pas mieux. Du moins peuvent-ils se dire qu'ils ne risquent guère de tomber plus mal. On croirait vraiment, à entendre la plupart de nos camarades, que le {\itshape Front populaire} possède un droit absolu, un droit divin au soutien, à la fidélité des opprimés, y compris ceux qu'il foule aux pieds. Ne leur fait-il pas « en les croquant, beaucoup d'honneur » ? N'est-on pas plus libre, mis en prison par un gouvernement de gauche, qu'en liberté sous un gouvernement de droite ?\par
Je ne terminerai pas en disant qu'il est scandaleux de voir une telle politi­que menée par un gouvernement de {\itshape Front populaire.} Non. Pourquoi feindre de croire à une fiction qu'on connaît pour telle ? Un pareil gouvernement, héritier du Cartel, est bien dans la ligne de celui qui, en 1924-1925, fit la guerre au Maroc. Que dire pourtant du rôle des socialistes ? Sans doute, le parti socialiste, en tant que parti, s'est-il ému ces derniers temps du drame Nord-Africain. Mais qu'ont fait ses ministres au pouvoir ? On sait que Dormoy s'est déchargé de l'Algérie sur Raoul Aubaud, mais celui-ci n'était qu'un sous-secrétaire d'État ; qui croira que le ministre de l'Intérieur n'avait pas le pouvoir de faire mettre Messali et ses camarades en liberté ? Sans doute aussi l'Afrique du Nord ne se trouvait-elle pas placée sous l'autorité de Marius Moutet ; mais le Gabon s'y trouvait placé ; qui, dès lors, est responsable de la déportation meurtrière du professeur marocain El Fassi au Gabon, dans un climat fatal pour un malade comme lui ?\par
Quand on récapitule les événements de ces derniers mois en Afrique du Nord, et qu'on songe ensuite aux problèmes brûlants de la politique extérieure, on ne peut que rire amèrement. Ce sont ces colonies infortunées qui pourraient nous valoir une guerre européenne ! Quel juste retour si, à cause de ces hommes de peau diversement colorée que nous abandonnons si froidement à leur misère, chaque Français devait être voué aux misères non moins atroces du P.C.D.F. ! Nous les laissons périr, et nous périrons pour pouvoir continuer à les laisser périr ! Et c'est cette France que beaucoup voudraient lancer dans une croisade libératrice pour l'Espagne ou pour la Chine. Sans doute alors les Indochinois, les Nord-Africains seraient-ils admis parmi les premiers à l'honneur de mourir pour la liberté des peuples ?\par
({\itshape Vigilance}, n° 63, 10 mars 1938.)\par
\par

\subsubsection[5. Les nouvelles données du problème colonial dans l’empire français, (décembre 1938)]{5. \\
Les nouvelles données du problème colonial dans l’empire français \\
(décembre 1938)}
\noindent \par
Les problèmes de la colonisation se posent avant tout en termes de force. La colonisation commence presque toujours par l'exercice de la force sous sa forme pure, c'est-à-dire par la conquête. Un peuple, soumis par les armes, subit soudain le commandement d'étrangers d'une autre couleur, d'une autre langue, d'une tout autre culture, et convaincus de leur propre supériorité. Par la suite, comme il faut vivre, et vivre ensemble, une certaine stabilité s'établit, fondée sur un compromis entre la contrainte et la collaboration. Toute vie sociale, il est vrai, est fondée sur un tel compromis, mais les proportions de la contrainte et de la collaboration diffèrent, et dans les colonies la part de la contrainte est généralement beaucoup plus grande qu'ailleurs. Il ne serait pas difficile de trouver une colonie appartenant à un État démocratique où la contrainte soit à bien des égards pire que dans le pire État totalitaire d'Europe.\par
La coexistence de deux races, même si l'une dirige, n'implique pas par elle-même une si grande contrainte. Des bases d'une collaboration suffisante pour réduire la contrainte au minimum pourraient être trouvées. Les Européens qui vont dans d'autres continents pourraient tout d'abord ne pas se sentir dépaysés parmi des êtres crus inférieurs s'ils connaissaient mieux leur propre culture et son histoire ; ils ne croiraient pas alors que les leurs ont tout inventé.\par
Réciproquement, la culture européenne, parée de ses propres prestiges et de tous ceux de la victoire, arrive toujours à attirer une partie de la jeunesse dans les pays colonisés. La technique, après avoir choqué beaucoup d'habi­tudes, étonne et séduit par sa puissance. Les populations conquises ne deman­dent, au moins en partie, qu'à s'assimiler cette culture et cette technique ; si ce désir n'apparaît pas aussitôt, le temps l'amène presque infailliblement. Une collaboration cordiale serait de ce fait possible, malgré la subordination d'une race à l'autre, si chaque étape dans le sens de l'assimilation apparaissait à la population soumise comme une étape dans la voie de l'indépendance écono­mique et politique. Dans le cas contraire, l'assimilation aiguise au contraire les conflits. Une jeunesse élevée dans la culture du vainqueur ne supporte que par force d'être traitée avec dédain par des hommes à qui elle se sent semblable et égale. La technique, si la misère des masses augmente, ou simplement se maintient, ou même diminue, mais non pas à un rythme qui corresponde à la mise en valeur du pays, la technique apparaît comme un bien monopolisé par des étrangers et dont on souhaite s'emparer. Si la population de la colonie a le sentiment que le vainqueur compte prolonger indéfiniment le rapport de conquérant à conquis, il s'établit une paix qui diffère de la guerre uniquement par le fait que l'un des camps est privé d'armes.\par
\par
C'est vers une telle situation que tend automatiquement, par une sorte d'inertie, toute colonisation. Il va de soi que c'est là une situation intolérable. Si on la suppose donnée, de quelle manière est-il possible qu'elle s'améliore ?\par

\begin{center}
*\end{center}
\noindent Un des moyens que l'on peut concevoir est la naissance d'un mouvement d'opinion dans la nation colonisatrice contre les injustices effrayantes imposées aux colonies. Un tel mouvement d'opinion semblerait devoir être facile à susciter dans un pays qui se réclame d'un idéal de liberté et d'humanité. L'expérience montre qu'il n'en est rien. En 1931, Louis Roubaud a publié en première page du {\itshape Petit Parisien} une série d'articles sur l'Indochine pleins de révélations terribles qui ne furent pas démenties ; ils n'ont produit aucune impression, et aujourd'hui encore beaucoup de gens cultivés, de ceux qu'on considère bien informés, ignorent tout de l'atroce répression de 1931. Au cours du grand mouvement qui a soulevé, en 1936, les ouvriers français, ils ne se sont pour ainsi dire pas souvenus qu'il existât des colonies. Les organisations qui les représentaient ne s'en sont bien entendu guère mieux souvenues. D'une manière générale, les Français sont tellement persuadés de leur propre générosité qu'ils ne s'enquièrent pas des maux que souffrent par eux des populations lointaines ; et la contrainte prive ces populations de la faculté de se plaindre. La générosité ne va guère chez aucun peuple jusqu'à faire effort pour découvrir les injustices qu'on commet en son nom ; en tout cas elle ne va certes pas jusque-là en France. La propagande de quelques-uns ne peut y apporter qu'un faible remède.\par
Un autre moyen, celui qui se présente le glus naturellement à l'esprit, est une révolte victorieuse. Mais il est difficile qu'une révolte coloniale soit victorieuse. Le nombre serait du côté des révoltés ; mais le monopole de la technique et des armes les plus modernes pèse plus lourd dans la balance des forces. Une guerre qui absorberait les forces armées de la nation colonisatrice peut présenter peut-être des possibilités d'émancipation violente ; mais même en pareil cas une révolte n'arriverait que difficilement à réussir, et surtout elle serait singulièrement menacée par les ambitions des autres nations en armes. D'une manière générale, en supposant qu'une révolte armée soit heureuse, l'acquisition et le maintien de l'indépendance dans de telles conditions, la nécessité d'assurer la défense à la fois contre la nation qui commandait naguè­re et contre les autres convoitises, exigeraient une telle tension morale, une mise en jeu si intensive de toutes les ressources matérielles que la population risquerait de n'y gagner ni bien-être ni liberté. Sans doute l'indépendance nationale est un bien ; mais quand elle suppose une telle soumission à l'État que la contrainte, l'épuisement et la faim soient aussi grands que sous une domination étrangère, elle est vaine. Nous ne voulons pas, nous, Français, mettre un tel prix à la défense de notre indépendance nationale ; pourquoi serait-il désirable que les populations des colonies mettent un tel prix à l'acquisition de la leur ?\par

\begin{center}
*\end{center}
\noindent Il semble qu'il n'y ait pas d'issue, et pourtant il y en a une. Il existe une troisième possibilité. C'est que la nation colonisatrice ait intérêt elle-même à émanciper progressivement ses propres colonies, et qu'elle comprenne cet intérêt. Or, les conditions d'une telle solution existent. Le jeu des forces internationales fait que la France a intérêt, un intérêt urgent, évident, à trans­former ses sujets en collaborateurs. Il faut qu'elle comprenne cet intérêt ; ici, la propagande peut s'employer.\par
Pour qui regarde seulement l'Europe, il est regrettable à bien des égards que la paix n'ait pu se maintenir qu'au prix des concessions de Munich. C'est affreux pour ceux des Allemands des Sudètes que ne séduit pas le régime hitlérien ; c'est fort douloureux pour la Tchécoslovaquie, qui n'a plus qu'une ombre d'indépendance nationale ; c'est amer pour les États démocratiques, dont le prestige et par suite la sécurité apparaissent amoindris. Mais si on regarde l'Asie et l'Afrique, les accords de Munich ouvrent des espérances jusque-là chimériques. La France, dont la position en Europe a subi une si grave atteinte, ne se maintient au rang des grandes puissances que par son Empire. Mais ce qui lui reste de force et de prestige ne peut plus lui suffire à conserver cet Empire si ceux qui le composent ne désirent pas eux-mêmes y demeurer.\par
Les revendications de l'Allemagne concernant ses anciennes colonies ne touchent qu'un aspect partiel et secondaire de ce problème. Nul ne sait quand elle posera officiellement ces revendications, ni quelles revendications plus étendues pourront suivre. Mais dès aujourd’hui l'Empire français est l'objet des convoitises de l'Allemagne et de ses alliés. L'Allemagne a toujours considéré comme abusif - non sans motifs - le protectorat français sur le Maroc ; l'Italie a depuis longtemps les yeux sur la Tunisie ; le Japon désire l'Indochine. La France ne possède pas la puissance nécessaire pour défendre de si vastes territoires si les populations intéressées lui sont dans le for intérieur hostiles, ou même si elles assistent au conflit d'ambitions en simples spectatrices.\par
Certaine fable de La Fontaine sur l'Âne et son maître est utile à relire en l'occurrence. Tout le monde en France la connaît ; il n'est que de songer à l'appliquer. Quand même tous les Français des colonies adopteraient soudain les procédés les plus humains, les plus bienveillants, les plus désintéressés, cela ne suffirait pas à susciter dans l'Empire les sentiments nécessaires à la sécurité de la France. Il est indispensable que les sujets de la France aient quelque chose à eux qu'une autre domination risquerait de leur faire perdre ; et à cet effet il est indispensable qu'ils cessent d'être des sujets, autrement dit des êtres passifs, bien ou mal traités, mais entièrement soumis au traitement qu'on leur accorde. Il faut qu'ils entrent effectivement, et bientôt, et assez rapide­ment, dans le chemin qui mène de la situation de sujet à celle de citoyen.\par
\par

\begin{center}
*\end{center}
\noindent Il n'est pas question de faire des colonies, tout d'un coup, des États indépendants. Une telle métamorphose serait sans doute sans lendemain ; mais de toute façon, aucun gouvernement français, de quelque parti qu'il se récla­me, n'y songerait. Il y aurait à examiner des modalités d'autonomie adminis­trative, de collaboration au pouvoir politique et militaire, de défense économique. Ces modalités différeraient nécessairement selon les colonies. Les mêmes solutions ne sont pas applicables sans doute aux Annamites, qui n'ont pas attendu l'invasion française pour être un peuple hautement civilisé, et à tels territoires du centre de l'Afrique. Le passé, les mœurs, les croyances doivent entrer en ligne de compte. Mais quelles que soient les modalités, le succès n'est possible que si elles s'inspirent de la même nécessité urgente : les populations des colonies doivent participer activement et en vue de leur propre intérêt à la vie politique et économique de leurs pays.\par
En ce qui concerne la France, il n'est pas sûr qu'une telle politique, même rapidement et intelligemment appliquée, puisse être efficace. Il est peut-être trop tard. S'il est vrai, par exemple, que sur les millions d'habitants du Nord Annam et du Tonkin neuf familles sur dix environ avaient perdu au moins un de leurs membres à cause de la répression de 1931, ces millions d'hommes ne pardonneront peut-être pas facilement. Mais ce qui est à peu près certain, c'est que cette politique offre à la France l'unique chance de conserver son rang de grande puissance que presque tous les hommes politiques jugent indispensable à sa sécurité.\par
En revanche, en ce qui concerne les colonies, une telle politique, si elle est effectivement suivie, sera efficace dans tous les cas. Soit que les populations colonisées, à la suite d'une émancipation partielle, forment ou non des senti­ments favorables au maintien de l'Empire français ; soit qu'elles demeurent, dans l'avenir prochain, sous la domination française ou passent sous une autre domination ; dans tous les cas, les libertés acquises leur donneront des possi­bilités de se défendre contre n'importe quelle oppression et des possibilités d'aller vers une émancipation complète qu'elles ne possèdent pas actuellement. Elles sont présentement désarmées et à la merci de quiconque demeure parmi elles avec des armes. Il n'y a aucun doute, par exemple, que si le Japon s'emparait présentement de l'Indochine, il profiterait de l'état d'impuissance et de passivité où il trouverait les Annamites. S'il les trouvait en possession de certaines libertés, il lui serait difficile de ne pas au moins les maintenir. Ainsi du point de vue français, une telle politique est nécessaire ; du point de vue humain - qui, soit dit en passant, est naturellement le mien - quelles que puissent en être les conséquences pour la France, elle serait heureuse. Ceux qui sont habitués à tout considérer sous la double catégorie « révolutionnaire » et « réformiste » - la première épithète, dans ce système manichéen, désignant le bien et la seconde le mal - trouveront sans doute qu'une telle solution du problème colonial est atteinte de la tare indélébile du réformisme. Pour moi, sans hésitation, je la juge infiniment préférable, si elle se réalise, à une éman­cipation qui résulterait d'un soulèvement victorieux. Car elle permettrait aux populations soumises aujourd'hui à tant d'intolérables contraintes d'accéder au moins à une liberté partielle sans être forcées de tomber dans un nationalisme forcené - à son tour impérialiste et conquérant - ,dans une industrialisation à outrance fondée sur la misère indéfiniment prolongée des masses populaires, dans un militarisme aigu, dans une étatisation de toute la vie sociale analogue à celle des pays totalitaires. Telles seraient presque infailliblement les suites d'un soulèvement victorieux ; quant aux suites d'un soulèvement non victo­rieux, elles seraient trop atroces pour qu'on ait envie de les évoquer. L'autre voie, moins glorieuse sans doute, ne coûterait pas de sang ; et comme disait Lawrence d'Arabie, ceux qui ont pour objet la liberté désirent vivre pour en jouir plutôt que mourir pour elle.\par
Ce qui risque d'empêcher qu'une solution si désirable au problème colonial devienne une réalité, c'est l'ignorance où on est en France des données du problème. On ignore que la France n'est pas, aux yeux de la plupart de ses sujets, la nation démocratique, juste et généreuse qu'elle est aux yeux de tant de Français moyens et autres. On ignore que les Annamites, notamment, n'ont aucune raison de la préférer au Japon, et en fait, à ce qu'on entend dire de plusieurs côtés, ne la préfèrent pas. Ici le rôle des informations peut être très important. Tant que les informations concernant le régime colonial ne mettaient en cause que la générosité de la France, elles risquaient de tomber dans l'indifférence et surtout dans l'incrédulité générale. C'est effectivement ce qui s'est produit. Dès lors qu'il est question de sécurité, elles ont chance d'être prises plus au sérieux. Si pénible et si humiliant qu'il soit de l'admettre, l'opinion d'un pays, sans aucune distinction de classes sociales, est beaucoup plus sensible à ce qui menace sa sécurité qu'à ce qui offense la justice.\par
({\itshape Essais et combats}, décembre 1938.)\par

\subsubsection[6. Fragment, (1938-1939 ?)]{6. \\
Fragment \\
(1938-1939 ?)}
\noindent \par
La position et le prestige de la France en Europe ont considérablement diminué depuis Munich. Le gouvernement a préféré cet amoindrissement à la guerre, et il a eu mille fois raison. L'amoindrissement n'en est pas moins un fait. Pourtant la possession d'un vaste empire colonial lui permet de faire encore figure de grande puissance. Mais l'affaiblissement de sa position en Europe lui rend bien plus difficile la conservation de cet Empire. Je ne pense pas simplement aux anciennes colonies allemandes, dont on se préoccupe beaucoup trop. Il est possible que les prétentions allemandes à ce sujet soient un trompe-l’œil destiné à couvrir l'accomplissement d'autres visées politi­ques ; en tout cas, si l'Allemagne n'est pas provisoirement absorbée par l'expansion vers l'Est, ce qui est possible, ses ambitions coloniales sont certai­nement beaucoup plus étendues. D'une manière générale, l'Empire français est exposé aux convoitises. L'Italie tend les mains vers la Tunisie, le Japon vers l'Indochine ; il serait surprenant que l'Allemagne ne se souvînt pas un jour que la France a occupé le Maroc malgré elle et malgré le traité signé sous sa pression. Par malheur aussi l'Italie, l'Allemagne et le Japon sont précisément alliés. Personne, j'imagine, ne compte sur la Russie pour aider la France à conserver ses territoires d'Asie et d'Afrique. Malgré l'Angleterre, on peut dire que dans la situation internationale actuelle, l'Empire français est dès aujour­d'hui en danger. Le problème pour la France est donc de chercher, en plus de l'alliance anglaise et de sa propre farce militaire, quels moyens elle peut mettre en œuvre contre ce danger.\par
À la force française et à la force anglaise, un troisième facteur pourrait seul servir d'appoint ; à savoir la volonté des populations colonisées de rester dans le cadre de l'Empire français. À cet égard, il ne se pose que deux questions, des questions bien précises. D'abord, cette volonté existe-t-elle dès maintenant, et est-elle portée au degré le plus haut où elle puisse parvenir ? Et si non, quelles mesures ont chance soit de la susciter, soit de la développer ?\par
La première question surtout est susceptible de donner lieu à des contro­verses. Pour beaucoup, il est criminel même de mettre en doute l'attachement des populations colonisées à la France. D'autres, à vrai dire peu nombreux, affirment qu'il n'y a pas d'attachement, mais une haine sourde et impuissante. L'une ou l'autre affirmation peut être vraie ; la vérité peut aussi être entre les deux, et plus proche de l'une ou de l'autre. La vérification directe est impos­sible, car les seuls qualifiés pour répondre, qui sont les intéressés eux-mêmes, n'ont pas le droit de s'exprimer, ou plutôt n'ont le droit de s'exprimer que dans un sens, ce qui ôte toute valeur à leur témoignage. Cette difficulté est la même que celle où l'on se trouve pour apprécier l'attachement au régime dans les États totalitaires ; là aussi on se trouve en présence d'affirmations contraires, émanant les unes et les autres de gens sérieux. Pourtant, en ce qui concerne les colonies, comme il s'agit d'une donnée essentielle d'un problème pratique et urgent, on ne peut se contenter d'un point d'interrogation. Le raisonnement, l'investigation des faits doivent permettre de former une opinion ferme qui serve de principe d'action.\par
Tout d'abord, ce que nul ne contestera, c'est que les populations des colonies sont des populations sujettes. Admettons pour l'instant qu'elles soient parfaitement bien traitées. Elles n'en subissent pas moins passivement le traitement que d'autres décident. Dans les pays dictatoriaux la population, si peu qu'elle ait à dire, est poussée par le patriotisme, par les organisations politiques, les groupements de jeunesse, par une technique de l'enthousiasme collectif, à imaginer avec plus ou moins de conviction une sorte de partici­pation mystique à la dictature. Dans les colonies il n'existe et il ne peut exister aucun facteur de cet ordre...\par

\subsubsection[7. Fragment, (1938-1939 ?)]{7. \\
Fragment \\
(1938-1939 ?)}
\noindent \par
Les considérations qui suivent ne se rapporteront qu'à l'intérêt de la France dans le problème colonial et non à sa responsabilité morale. Ce n'est pas que la seconde considération ne soit à mes yeux infiniment plus importante que la première ; je ne vois pas de raison d'établir à cet égard une autre échelle pour les problèmes nationaux que pour les problèmes individuels. Mais nous nous trouvons à un moment où la considération de l'intérêt, un intérêt clair et urgent, conseille à peu près les mêmes mesures que conseillerait la seule considération de la responsabilité morale. Dès lors les arguments tirés de l'intérêt ont beaucoup plus de chances d'être efficaces. Un homme est parfois sensible à la justice, même quand elle exige qu'il aille contre son propre intérêt ; une collectivité, qu'elle soit nation, classe, parti, groupement, n'y est pour ainsi dire jamais sensible hors les cas où elle est elle-même lésée.\par
Il est certain en tout cas que la France, tant qu'elle s'est sentie en sécurité dans son territoire et dans son Empire, tant qu'elle a occupé en Europe une position dominante, est restée indifférente à ses propres responsabilités mora­les en matière coloniale. Quelque jugement qu'on porte sur le régime colonial, et quand même on l'estimerait parfait, nul ne peut nier que tout ce qu'il contient de bon ou de mauvais est dû entièrement à ceux qui vivent dans les colonies ou s'en occupent professionnellement. Les Français de France ont été jusqu'ici complètement indifférents aux colonies ; ils sont presque tous complètement ignorants de ce qui s'y passe, ne cherchent guère à s'en infor­mer, ne discutent presque jamais entre eux de ce qu'il serait possible ou désirable de faire à leur égard. Même ceux qui, avant ou depuis 1936, sont sérieusement préoccupés de la justice en matière sociale pensent bien plus souvent et bien plus longtemps à une différence d'un franc dans le salaire d'un métallurgiste de la région parisienne qu'à la vie et à la mort d'Arabes et d'Annamites. Ces gens sont trop loin, dit-on. Non ! ils ne sont pas loin. Des peuples qui se trouvent sous la domination de la France, dont la misère ou le bien-être, la honte ou la dignité, et parfois la vie même dépendent entièrement de la politique française sont aussi proches de nous que les lieux mêmes où s'élabore cette politique.\par
Au reste, ces territoires lointains se rapprochent ; ils se rapprocheront plus encore par la suite. Ils se rapprochent, par rapport à l'imagination des Français, dans la mesure où ceux-ci sentent que la sécurité de la France s'y trouve menacée. Encore une fois, ce qui menace la sécurité parle tout autrement à l'imagination, surtout collective, que ce qui menace simplement la pureté de la conscience. L'humanité, en politique, consiste non à invoquer sans cesse des principes moraux, ce qui reste généralement vain, mais à s'efforcer de mettre au premier plan tous les mobiles d'ordre plus bas qui sont susceptibles, dans une situation donnée, d'agir dans le sens des principes moraux. Je suis de ceux qui pensent que tous les problèmes coloniaux doivent être posés avant tout dans leur rapport avec les aspirations, les libertés, le bien-être des populations colonisées, et seulement d'une manière secondaire dans leur rapport avec les intérêts de la nation colonisatrice. Mais comme celle-ci détient la force, lorsque par suite des circonstances les deux points de vue amènent à des conclusions semblables, c'est le second qu'il est utile de mettre en lumière. Un telle coïncidence ne peut se produire que par un jeu particulier de circonstan­ces, du moins si on considère l'intérêt prochain d'une nation ; car la considération de l'intérêt lointain reste malheureusement presque aussi vaine que celle de la justice. Cette coïncidence est un bonheur, quand même elle tiendrait à des circonstances malheureuses. Tel est précisément, sauf erreur, le cas de la France en ce moment.\par

\subsubsection[8. Lettre à Jean Giroudoux, (Fin 1939 ? 1940 ?)]{8. \\
Lettre à Jean Giroudoux \\
(Fin 1939 ? 1940 ?)}
\noindent \par
Monsieur et cher archicube\par
Vos fonctions constituent une excuse à la liberté que je prends de vous écrire ; car puisque vous parlez au public, le public doit pouvoir vous parler. L'admiration et la sympathie qu'ont provoquées en moi vos livres et surtout votre théâtre m'ont donné plusieurs fois le désir, si naturel à des lecteurs, d'entrer en rapport avec vous à la faveur de la camaraderie traditionnelle de la rue d'Ulm et de quelques relations communes ; mais on doit résister à ce genre de désir, car la sympathie entre un auteur et un lecteur est nécessairement unilatérale ; quant aux expressions d'admiration, rien n'est plus ennuyeux à entendre. Mais aujourd'hui c'est différent ; puisque vous vous adressez aux femmes de France et que j'en suis une, j'ai pour ma part, qui doit être, je suppose, une sur vingt millions, le droit de me faire entendre de vous. Et bien que ce soit mon admiration pour vous qui m'oblige à vous écrire, ce n'est pas de l'admiration que je viens vous exprimer. Je n'ai pas entendu votre allocu­tion ; je l'ai lue dans {\itshape le Temps.}\par
Elle contient un passage qui m'a fait une vive peine. Car j'ai toujours été fière de vous comme d'un de ceux dont on peut prononcer le nom quand on veut trouver des raisons d'aimer la France actuelle. C'est pourquoi je voudrais que vous disiez toujours la vérité, même à la radio. Sûrement vous croyez la dire ; mais je voudrais de tout mon cœur pouvoir vous amener à vous demander si vous la dites, quand vous affirmez que la France dispose d'un domaine colonial attaché à sa métropole par des liens autres que subordination et l'exploitation.\par
Je donnerais ma vie et plus s'il est possible pour pouvoir penser qu'il en est ainsi ; car il est douloureux de se sentir coupable par complicité involontaire. Mais dès qu'on s'informe et qu'on étudie la question, il est clair comme le jour qu'il n'en est pas ainsi. Combien d'hommes sont privés par notre fait de toute patrie, que nous contraignons maintenant à mourir pour nous conserver la nôtre ! La France n'a-t-elle pas pris l'Annam par conquête ? N'était-ce pas un pays paisible, un, organisé, de vieille culture, pénétré d'influences chinoises, hindoues, bouddhiques ? Ils nomment notamment du nom de kharma une notion populaire chez eux, exactement identique à celle, malheureusement oubliée par nous, de la Némésis grecque comme châtiment automatique de la démesure. Nous avons tué leur culture ; nous leur interdisons l'accès des manuscrits de leur langue ; nous avons imposé à une petite partie d'entre eux notre culture, qui n'a pas de racines chez eux et ne peut leur faire aucun bien. Les populations du Nord, chez eux, meurent chroniquement de faim, pendant que le Sud regorge de riz qu'il exporte. Chacun est soumis à un impôt annuel égal pour les riches et les pauvres. Des parents vendent leurs enfants, comme jadis dans les provinces romaines ; des familles vendent l'autel des ancêtres, le bien pour eux le plus précieux, non pas même pour ne plus souffrir la faim, mais pour payer l'impôt. Jamais je n'oublierai d'avoir entendu un ingénieur agronome, fonctionnaire du ministère des Colonies, me dire froidement qu'on a raison là-bas de frapper les coolies dans les plantations, parce que, comme ils sont réduits à l'extrême limite de la fatigue et des privations, on ne saurait les punir autrement sans plus d'inhumanité. Ignorez-vous qu'on a massacré à la mitrailleuse des paysans qui sont venus sans armes dire qu'ils ne pouvaient pas payer les impôts ? A-t-on jamais même osé démentir les atrocités commises après les troubles de Yen-Bay ? On a détruit des villages avec des avions ; on a lâché la Légion sur le Tonkin pour y tuer au hasard ; des jeunes gens employés dans les prisons y ont entendu à longueur de journée les cris des malheureux torturés. Il y aurait malheureusement bien plus encore à raconter. Et pour l'Afrique, ignorez-vous les expropriations massives dont ont été victimes, encore après l'autre guerre, les Arabes et les noirs ? Peut-on dire que nous avons apporté la culture aux Arabes, eux qui ont conservé pour nous les traditions grecques pendant le moyen âge ? Pourtant j'ai lu des journaux rédigés par des Arabes à Paris en français, parce qu'eux ni leur public ne savaient lire l'arabe. N'avez-vous pas lu dans les journaux, il v a environ un an, qu'une grève avait éclaté dans une mine de Tunisie parce q'on voulait y contraindre les ouvriers musulmans à fournir pendant le Ramadan, par conséquent sans manger, le même effort que d'habitude ? Comment des musulmans accepteraient-ils ces choses et d'autres analogues, s'ils n'étaient soumis par la force ?\par
Je n'ignore pas que cette lettre me met sous le coup du décret du 24 mai 1938, prévoyant des peines de un à cinq ans de prison. Je n'ai pas d'inquiétude à cet égard ; mais quand j'aurais lieu d'en avoir que m'importe ? La prison perpétuelle ne me ferait pas plus de mal que l'impossibilité où je suis, à cause des calomnies, de penser que la cause de la France est juste.\par

\begin{center}
\end{center}
\subsubsection[9. À propos de la question coloniale dans ses rapports avec le destin du peuple français  (1943)]{9. \\
À propos de la question coloniale dans ses rapports avec le destin du peuple français \protect\footnotemark  \\
(1943)}
\footnotetext{ Écrit à Londres pour les services de la {\itshape France Libre}. (Note de l'éditeur.)}
\noindent \par
Le problème d'une doctrine ou d'une foi pour l'inspiration du peuple français en France, dans sa résistance actuelle et dans la construction future, ne peut pas se séparer du problème de la colonisation. Une doctrine ne s'enferme pas à l'intérieur d'un territoire. Le même esprit s'exprime dans les relations d'un peuple avec ceux qui l'ont maîtrisé par la force, dans les relations intérieures d'un peuple avec lui-même, et dans ses relations avec ceux qui dépendent de lui.\par
Pour la politique intérieure de la France, personne n'a la folie de proclamer que la Troisième République, telle qu'elle était le 3 septembre 1939, va ressusciter de toutes pièces. On parle seulement d'un régime conforme aux traditions de la France, c'est-à-dire, principalement, à l'inspiration qui a fait jouer à la France du moyen âge un si grand rôle en Europe, et à l'inspiration de la Révolution française. C'est d'ailleurs la même en gros, traduite du langage catholique en langage laïque.\par
Si vraiment ce critérium est valable pour la France, s'il est réel, il ne doit pas y en avoir un autre pour les colonies.\par
Cela suppose non un maintien, mais une suspension du {\itshape statu quo} jusqu'à ce que le problème colonial ait été repensé, ou plutôt pensé. Car il n'y a jamais eu en France de doctrine coloniale. Il ne pouvait pas y en avoir. Il y a eu des pratiques coloniales.\par
Pour penser ce problème, il y a trois tentations à surmonter. La première est le patriotisme, qui incline à préférer son pays à la justice, ou à admettre qu'il n'y a jamais lieu en aucun cas de choisir entre l'un et l'autre. S'il y a dans la patrie quelque chose de sacré, nous devons reconnaître qu'il y a des peuples que nous avons privés de leur patrie. S'il n'y a rien de tel, nous ne devons pas tenir compte de notre pays quand il se pose un problème de justice.\par
La seconde tentation, c'est le recours aux compétences. Les compétences, en cette matière, ce sont les coloniaux. Ils sont partie dans le problème. Même, si le problème était posé à fond, ils pourraient devenir accusés. Leur jugement n'est pas impartial. D'ailleurs, s'ils ont quitté la France pour les colonies, c'est dans beaucoup de cas que d'avance le système colonial les attirait. Une fois là-bas surtout, leur situation leur a fait subir une trans­formation. Le langage des indigènes même les plus révoltés est un document moins accablant pour la colonisation que celui de beaucoup de coloniaux.\par
Les indigènes qui viennent en France aiment bien mieux avoir affaire, toutes les fois qu'ils peuvent, à des Français de France qu'à des coloniaux. Cette compétence n'est pas appréciée par eux. Mais en fait on les renvoie toujours à des coloniaux. Le prestige des compétences est tel en France que lorsque des indigènes hasardent une plainte contre un acte d'oppression, souvent cette plainte, de bureau en bureau, retourne à celui même contre qui elle était portée, et il en tire vengeance. On a tendance à faire la même opération à une grande échelle.\par
Non seulement cette compétence est viciée, mais elle est très fragmentaire. Elle l'est souvent dans l'espace, en ce sens que beaucoup connaissent un coin de l'Empire et généralisent. Elle l'est surtout dans le temps. Excepté au Maroc, où certains Français sont devenus réellement amoureux de la culture arabe - et ce milieu, soit dit en passant, commence à constituer une source de renou­vellement pour la culture française - les Français coloniaux ne sont généra­lement pas curieux de l'histoire des pays où ils se trouvent. Le seraient-ils que l'administration française ne fait rien pour rendre une telle étude possible.\par
Comment prétendre qu'on comprend si peu que ce soit à un peuple quand on oublie qu'il a un passé ? Nous, Français, ne cherchons-nous pas notre inspi­ration dans le passé de la France ? Croit-on qu'elle est seule à en avoir un ?\par
La troisième tentation est la tentation chrétienne. La colonisation consti­tuant un milieu favorable pour les missions, les chrétiens sont tentés de l'aimer pour cette raison, même quand ils en reconnaissent les tares.\par
Mais, sans discuter la question - qui pourtant mériterait l'examen - de savoir si un Hindou, un Bouddhiste, un musulman ou un de ceux qu'on nomme païens n'a pas dans sa propre tradition un chemin vers la spiritualité que lui proposent les Églises chrétiennes, en tout cas le Christ n'a jamais dit que les bateaux de guerre doivent accompagner même de loin ceux qui annoncent la bonne nouvelle. Leur présence change le caractère du message. Le sang des martyrs peut difficilement conserver l'efficacité surnaturelle qu'on lui attribue quand il est vengé par les armes. On veut avoir plus d'atouts dans son jeu qu'il n'est permis à l'homme quand on veut avoir à la fois César et la Croix.\par
Les plus fervents des laïques, des francs-maçons, des athées, aiment la colonisation pour une raison diamétralement contraire, mais mieux fondée dans les faits. Ils l'aiment comme une extirpeuse de religions, ce qu'elle est effectivement ; le nombre des gens à qui elle fait perdre leur religion l'emporte de loin sur le nombre des gens à qui elle en apporte une nouvelle. Mais ceux qui comptent sur elle pour répandre ce qu'on appelle la foi laïque se trompent aussi. La colonisation française entraîne bien, d'une part une influence chré­tienne, d'autre part une influence des idées de 1789. Mais les deux influences sont relativement faibles et passagères. Il ne peut pas en être autrement, étant donné le mode de propagation de ces influences, et la distance exagérée entre la théorie et la pratique. L'influence forte et durable est dans le sens de l'incrédulité, ou plus exactement du scepticisme.\par
Le plus grave est que, comme l'alcoolisme, la tuberculose et quelques autres maladies, le poison du scepticisme est bien plus virulent dans un terrain naguère indemne. Nous ne croyons malheureusement pas à grand-chose. Nous fabriquons à notre contact une espèce d'hommes qui ne croit à rien. Si cela continue, nous en subirons un jour le contre-coup, avec une brutalité dont le Japon nous donne seulement un avant-goût.\par
On ne peut pas dire que la colonisation fasse partie de la tradition fran­çaise. C'est un processus qui s'est accompli en dehors de la vie du peuple français. L'expédition d'Algérie a été d'un côté une affaire de prestige dynastique ; de l'autre une mesure de police méditerranéenne ; comme il arrive souvent, la défense s'est transformée en conquête. Plus tard l'acquisition de la Tunisie et du Maroc ont été, comme disait un de ceux qui ont pris une grande part à la seconde, surtout un réflexe de paysan qui agrandit son lopin de terre. La conquête de l'Indochine a été une réaction de revanche contre l'humiliation de 1870. N'ayant pas su résister aux Allemands, nous sommes allés en compensation priver de sa patrie, en profitant de troubles passagers, un peuple de civilisation millénaire, paisible et bien organisé. Mais le gouvernement de jules Ferry a accompli cet acte en abusant de ses pouvoirs et en bravant ouvertement l'opinion publique française ; d'autres parties de la conquête ont été exécutées par des officiers ambitieux et dilettantes qui désobéissaient aux ordres formels de leurs chefs.\par
Les îles d'Océanie ont été prises au hasard de la navigation, sur l'initiative de tel ou tel officier, et livrées à une poignée de gendarmes, de missionnaires et de commerçants, sans que le pays s'y soit jamais intéressé.\par
Ce n'est guère que la colonisation en Afrique noire qui a provoqué l'intérêt public. C'était aussi la plus justifiable, étant donné l'état de ce malheureux continent, dont on ignore presque entièrement quelle fut l'histoire, mais où les blancs avaient en tout cas causé tous les ravages possibles depuis quatre siècles, avec leurs armes à feu et leur commerce d'esclaves. Cela n'empêche pas qu'il y ait un problème non résolu de l'Afrique noire.\par
On ne peut pas dire que le {\itshape statu quo} soit une réponse aux problèmes de l'Empire français. Et il y a une autre chose encore qu'on ne peut ni dire ni penser. C'est que ce problème concerne seulement le peuple français. Ce serait exactement aussi légitime que la prétention analogue de Hitler sur l'Europe centrale. Ce problème concerne, en dehors du peuple français, le monde entier, et avant tout les populations sujettes.\par
La force sur laquelle repose un empire colonial, c'est une flotte de guerre. La France a perdu presque toute la sienne. On ne peut pas dire qu'elle l'ait sacrifiée ; elle l'a perdue du fait de l'ennemi, qui s'en serait emparé si elle n'avait été détruite. Dès lors la France dépendra après la victoire, pour ses relations avec l'Empire, des pays qui ont une flotte. Comment ces pays n'auraient-ils pas voix au chapitre dans tout grand problème concernant l'Empire ? Si c'est la force qui décide, la France a perdu la sienne ; si c'est le droit, la France n'a jamais eu celui de disposer du destin de populations non françaises. En aucun sens, ni en droit ni en fait, on ne peut dire que les territoires habités par ces populations sont la propriété de la France.\par
La plus grande faute que pourrait commettre actuellement la France libre serait de vouloir, le cas échéant, maintenir cette prétention comme un absolu devant l'Amérique. Il ne peut rien y avoir de pire qu'une attitude radicalement opposée à la fois à l'idéal et à la réalité. Une attitude opposée à l'un des deux et conforme à l'autre a déjà de grands inconvénients ; mais l'autre les a tous.\par
Il faut regarder le problème colonial comme un problème nouveau. Deux idées essentielles peuvent y jeter quelque lumière.\par
La première idée, c'est que l'hitlérisme consiste dans l'application par l'Allemagne au continent européen, et plus généralement aux pays de race blanche, des méthodes de la conquête et de la domination coloniales. Les Tchèques les premiers ont signalé cette analogie quand, protestant contre le protectorat de Bohême, ils ont dit : « Aucun peuple européen n'a jamais été soumis à un tel régime. » Si on examine en détail les procédés des conquêtes coloniales, l'analogie avec les procédés hitlériens est évidente. On peut en trouver un exemple dans les lettres écrites par Lyautey de Madagascar. L'excès d'horreur qui depuis quelque temps semble distinguer la domination hitlérienne de toutes les autres s'explique peut-être par la crainte de la défaite. Il ne doit pas faire oublier l'analogie essentielle des procédés, d'ailleurs venus les uns et les autres du modèle romain.\par
Cette analogie fournit une réponse toute faite à tous les arguments en faveur du système colonial. Car tous ces arguments, les bons, les moins bons et les mauvais, sont employés par l'Allemagne, avec le même degré de légi­timité, dans sa propagande concernant l'unification de l'Europe.\par
Le mal que l'Allemagne aurait fait à l'Europe si l'Angleterre n'avait pas empêché la victoire allemande, c'est le mal que fait la colonisation, c'est le déracinement. Elle aurait privé les pays conquis de leur passé. La perte du passé, c'est la chute dans la servitude coloniale.\par
Ce mal que l'Allemagne a vainement essayé de nous faire, nous l'avons fait à d'autres. Par notre faute, de petits Polynésiens récitent à l'école : « Nos ancêtres les Gaulois avaient les cheveux blonds, les yeux bleus... » Alain Gerbault a décrit dans des livres qui ont été très lus, mais n'ont eu aucune influence, comment nous faisons littéralement mourir de tristesse ces popula­tions en leur interdisant leurs coutumes, leurs traditions, leurs fêtes, toute leur joie de vivre.\par
Par notre faute, les étudiants et les intellectuels annamites ne peuvent pas, sauf de rares exceptions, pénétrer dans la bibliothèque qui contient tous les documents relatifs à l'histoire de leur pays. L'idée qu'ils se font de leur patrie avant la conquête, ils la tiennent de leurs pères. Cette idée est, à tort ou à raison, celle d'un État paisible, sagement administré, où le surplus de riz était conservé dans des entrepôts pour être distribué en temps de famine, contrairement à la pratique plus récente d'exporter le riz du sud pendant que la famine ravage les populations du nord. La machine de l'État reposait entière­ment sur les concours, auxquels toutes les classes sociales pouvaient prendre part. Il suffisait d'avoir étudié, et on le pouvait même sans fortune et dans un lointain village. Les concours avaient lieu tous les trois ans. Les candidats s'assemblaient dans une prairie, et pendant trois jours composaient un essai sur un thème donné, généralement tiré de la philosophie chinoise classique. Les concours avaient des degrés de difficulté différents, et on passait de degré en degré. Chaque concours fournissait le milieu dans lequel étaient choisis les fonctionnaires de la dignité correspondante, et au concours le plus élevé correspondait la dignité de premier ministre ; l'empereur n'était pas libre de prendre un premier ministre ailleurs. Il y avait un très haut degré de décen­tralisation dans l'administration et dans la culture ; il y en a des traces même maintenant dans certains villages du nord du Tonkin, où les paysans connaissent les caractères chinois et improvisent de la poésie au cours des grandes fêtes.\par
Ce tableau est peut-être embelli, mais il faut avouer qu'il correspond à l'impression que donnent certaines lettres de missionnaires au XVII\textsuperscript{e} siècle. En tout cas, quelle qu'y soit la part de légende, ce passé est le passé de ce peuple, et il ne saurait trouver d'inspiration ailleurs. Il en est déjà presque entièrement déraciné, mais non pas entièrement. Si, une fois les japonais chassés, il retombe sous la domination européenne, le mal sera sans remède.\par
Quelque soulagement que doive probablement causer le départ des japo­nais, une continuation de la domination française ne serait sans doute pas subie sans horreur, à cause des atrocités qui, d'après des témoignages concor­dants, ont été commises par les Français pour réprimer une rébellion au moment de l'accord franco-japonais. D'après l'un de ces témoignages, des villages auraient été anéantis par des bombardements aériens, et des milliers de personnes, accusées d'être les familles des rebelles, mises sur des pontons et coulées. Quoique ces atrocités, si elles sont exactes, aient été commises par les hommes de Vichy, la population annamite ne fera pas la distinction.\par
En privant les peuples de leur tradition, de leur passé, par suite de leur âme, la colonisation les réduit à l'état de matière humaine. Les populations des pays occupés ne sont pas autre chose aux yeux des Allemands. Mais on ne peut pas nier que la plupart des coloniaux n'aient la même attitude envers les indigènes. Le travail forcé a été extrêmement meurtrier dans l'Afrique noire française, et la méthode des déportations massives y a été pratiquée pour peupler la boucle du Niger. En Indochine, le travail forcé existe dans les plantations sous des déguisements transparents ; les fuyards sont ramenés par la police et parfois, comme châtiment, exposés aux fourmis rouges ; un Français, ingénieur dans une de ces plantations, disait au sujet des coups qui y sont la punition la plus ordinaire : « Même si on se place du point de vue de la bonté, c'est le meilleur procédé, car, comme ils sont à l'extrême limite de la fatigue et de la faim, toute autre punition serait plus cruelle. » Un Cambod­gien, domestique d'un gendarme français, disait : « Je voudrais être le chien du gendarme ; on lui donne à manger et il n'est pas battu. »\par
Dans notre lutte contre l'Allemagne, nous pouvons avoir deux attitudes. Quelle que soit la nécessité de l'union, il faut absolument choisir, rendre le choix public et l'exprimer dans les actes. Nous pouvons regretter que l'Alle­magne ait accompli ce que nous aurions désiré voir accomplir par la France. C'est ainsi que quelques jeunes Français disent qu'ils sont derrière le général de Gaulle pour les mêmes motifs qui les rangeraient derrière Hitler s'ils étaient Allemands. Ou bien nous pouvons avoir horreur non de la personne ou de la nationalité, mais de l'esprit, des méthodes, des ambitions de l'ennemi. Nous ne pouvons guère faire que le second choix. Autrement il est inutile de parler de la Révolution française ou du christianisme. Si nous faisons ce choix, il faut le montrer par toutes nos attitudes.\par
Lutter contre les Allemands, ce n'est pas une preuve suffisante que nous aimons la liberté. Car les Allemands ne nous ont pas seulement enlevé notre liberté. Ils nous ont enlevé aussi notre puissance, notre prestige, notre tabac, notre vin et notre pain. Des mobiles mélangés soutiennent notre lutte. La preuve décisive serait de favoriser tout arrangement assurant une liberté au moins partielle à ceux à qui nous l'avons enlevée. Nous pourrions ainsi persuader non seulement aux autres, mais à nous-mêmes, que nous sommes vraiment inspirés par un idéal.\par
L'analogie entre l'hitlérisme et l'expansion coloniale, en nous dictant du point de vue moral l'attitude à prendre, fournit aussi la solution pratique la moins mauvaise. L'expérience des dernières années montre qu'une Europe formée de nations grandes et petites, toutes souveraines, est impossible. La nationalité est un phénomène indécis sur une grande partie du territoire européen. Même dans un pays comme la France, l'unité nationale a subi un choc assez rude ; Bretons, Lorrains, Parisiens, Provençaux ont une conscience bien plus aiguë qu'avant la guerre d'être différents les uns des autres. Malgré plusieurs inconvénients, cela est loin d'être un mal, En Allemagne, les vainqueurs s'efforceront d'affaiblir le plus possible le sentiment d'unité nationale. Très probablement une partie de la vie sociale en Europe sera morcelée à une échelle beaucoup plus petite que l'échelle nationale ; une autre partie sera unifiée à une échelle beaucoup plus grande ; la nation ne sera qu'un des cadres de la vie collective, au lieu d'être pratiquement tout, comme au cours des vingt dernières années. Pour les pays faibles, mais à longue tradition accompagnée d'une conscience aiguë, comme la Bohême, la Hollande, les pays scandinaves, il sera nécessaire d'élaborer un système d'indépendance combinée avec une protection militaire extérieure. Ce système peut être appliqué tel quel dans d'autres continents. Il va de soi qu'en ce cas l'Indochine serait, comme elle a toujours été, dans l'orbite de la Chine. La partie arabe de l'Afrique pourrait retrouver une vie propre sans perdre toute espèce de lien avec la France. Quant à l'Afrique noire, il semble raisonnable que pour les problèmes d'ensemble elle dépende tout entière de l'Europe tout entière, et que pour tout le reste elle reprenne une vie heureuse village par village.\par
La seconde idée qui peut éclairer le problème colonial, c'est que l'Europe est située comme une sorte de moyenne proportionnelle entre l'Amérique et l'Orient. Nous savons très bien qu'après la guerre l'américanisation de l'Europe est un danger très grave, et nous savons très bien ce que nous perdrions si elle se produisait. Or ce que nous perdrions, c'est la partie de nous-mêmes qui est toute proche de l'Orient.\par
Nous regardons les Orientaux, bien à tort, comme des primitifs et des sauvages, et nous le leur disons. Les Orientaux nous regardent, non sans quelques motifs, comme des barbares, mais ne le disent pas. De même nous avons tendance à regarder l'Amérique comme n'ayant pas une vraie civilisa­tion, et les Américains à croire que nous sommes des primitifs.\par
Si un Américain, un Anglais et un Hindou sont ensemble, les deux pre­miers ont en commun ce que nous nommons la culture occidentale, c'est-à-dire une certaine participation à une atmosphère intellectuelle composée par la science, la technique et les principes démocratiques. À tout cela l'Hindou est étranger. En revanche l'Anglais et lui ont en commun quelque chose dont l'Américain est absolument privé. Ce quelque chose, c'est un passé. Leurs passés sont différents, certes. Mais beaucoup moins qu'on ne le croit. Le passé de l'Angleterre, c'est le christianisme, et auparavant un système de croyances probablement proche de l'hellénisme. La pensée hindoue est très proche de l'un et de l'autre.\par
Nous autres Européens en lutte contre l'Allemagne, nous parlons beaucoup aujourd'hui de notre passé. C'est que nous avons l'angoisse de le perdre. L'Allemagne a voulu nous l'arracher ; l'influence américaine le menace. Nous n'y tenons plus que par quelques fils. Nous ne voulons pas que ces fils soient coupés. Nous voulons nous y réenraciner. Or ce dont nous avons trop peu conscience, c'est que notre passé nous vient en grande partie d'Orient.\par
C'est devenu un lieu commun de dire que notre civilisation, étant d'origine gréco-latine, s'oppose à l'Orient. Comme beaucoup de lieux communs, c'est là une erreur. Le terme gréco-latin ne veut rien dire de précis. L'origine de notre civilisation est grecque. Nous n'avons reçu des Latins que la notion d'État, et l'usage que nous en faisons donne à penser que c'est un mauvais héritage. On dit qu'ils ont inventé l'esprit juridique ; mais la seule chose certaine là-dessus, c'est que leur système juridique est le seul qui se soit conservé. Depuis qu'on connaît un code babylonien vieux de quatre mille ans, on ne peut plus croire qu'ils aient eu un monopole. Dans tout autre domaine, leur apport créateur a été nul.\par
Quant aux Grecs, source authentique de notre culture, ils avaient reçu ce qu'ils nous ont transmis. Jusqu'à ce que l'orgueil des succès militaires les ait rendus impérialistes, ils l'ont avoué ouvertement. Hérodote est on ne peut plus clair à ce sujet. Il y avait, avant les temps historiques, une civilisation méditerranéenne dont l'inspiration venait avant tout d'Égypte, en second lieu des Phéniciens. Les Hellènes sont arrivés sur les bords de la Méditerranée comme une population de conquérants nomades presque sans culture propre. Ils ont imposé leur langue, mais reçu la culture du pays conquis. La culture grecque a été le fait soit de cette assimilation des Hellènes, soit de la persis­tance des populations antérieures, non helléniques. La guerre de Troie a été une guerre où l'un des deux camps représentait la civilisation, et ce camp, c'était Troie. On sent par l'accent de l'Iliade que le poète le savait. La Grèce dans son ensemble a toujours eu envers l'Égypte une attitude de respect filial.\par
L'origine orientale du christianisme est évidente. Qu'on ait à l'égard du christianisme une attitude croyante ou agnostique, dans les deux cas il est certain que comme fait historique il a été préparé par les siècles antérieurs. En dehors de la Judée, qui est un pays d'Orient, les courants de pensée qui y ont contribué venaient d'Égypte, de Perse, peut-être de l'Inde, et surtout de Grèce, mais de la partie de la pensée grecque directement inspirée par l'Égypte et la Phénicie.\par
Quant au moyen âge, les moments brillants du moyen âge ont été ceux ou la culture orientale est venue de nouveau féconder l'Europe, par l'intermédiaire des Arabes et aussi par d'autres voies mystérieuses, puisqu'il y a eu des infiltrations de traditions persanes. La Renaissance aussi a été en partie causée par le stimulant des contacts avec Byzance.\par
À d'autres moments de l'histoire, certaines influences orientales ont pu être des facteurs de décomposition. C'était le cas à Rome ; c'est le cas de nos jours. Mais, dans les deux cas, il s'agit d'un pseudo-orientalisme fabriqué par et pour des snobs, et non pas de contact avec les civilisations d'Orient authentiques.\par
En résumé, il semble que l'Europe ait périodiquement besoin de contacts réels avec l'Orient pour rester spirituellement vivante. Il est exact qu'il y a en Europe quelque chose qui s'oppose à l'esprit d'Orient, quelque chose de spécifiquement occidental. Mais ce quelque chose se trouve à l'état pur et à la deuxième puissance en Amérique et menace de nous dévorer.\par
La civilisation européenne est une combinaison de l'esprit d'Orient avec son contraire, combinaison dans laquelle l'esprit d'Orient doit entrer dans une proportion assez considérable. Cette proportion est loin d'être réalisée aujourd'hui. Nous avons besoin d'une injection d'esprit oriental.\par
L'Europe n'a peut-être pas d'autre moyen d'éviter d'être décomposée par l'influence américaine qu'un contact nouveau, véritable, profond avec l'Orient. Actuellement, si on met ensemble un Américain, un Anglais et un Hindou, l'Américain et l'Anglais fraterniseront extérieurement, tout en se regardant chacun comme très supérieur à l'autre, et laisseront l'Hindou seul. L'apparition progressive d'une atmosphère où les réflexes soient différents est peut-être spirituellement une question de vie ou de mort pour l'Europe.\par
Or la colonisation, loin d'être l'occasion de contacts avec des civilisations orientales, comme ce fut le cas pour les Croisades, empêche de tels contacts. Le milieu très restreint et très intéressant des arabisants français est peut-être la seule exception. Pour des Anglais vivant en Inde, pour les Français vivant en Indochine, le milieu humain est constitué par les blancs. Les indigènes font partie du décor.\par
Encore les Anglais ont-ils une position cohérente. Ils font des affaires et c'est tout. Les Français, qu'ils le veuillent ou nan, transportent partout les principes de 1789. Dès lors il ne peut arriver que deux choses. Ou les indigè­nes se sentent choqués dans leur attachement à leur propre tradition par cet apport étranger. Ou ils adoptent sincèrement ces principes et sont révoltés de n'en pas avoir le bénéfice. Si étrange que cela puisse paraître, ces deux réactions hostiles existent souvent chez les mêmes individus.\par
Il en serait tout autrement si les contacts des Européens avec l'Asie, l'Afrique, l'Océanie, se faisaient sur la base des échanges de culture. Nous avons senti ces dernières années jusqu'au fond de l'âme que la civilisation occidentale moderne, y compris notre conception de la démocratie, est insuf­fisante. L'Europe souffre de plusieurs maladies tellement graves qu'on ose à peine y penser. L'une est la poussée toujours croissante des campagnes vers les villes et des métiers manuels vers les occupations non manuelles, qui menace la base physique de l'existence sociale. Une autre est le chômage. Une autre est la destruction volontaire de produits de première nécessité, comme le blé. Une autre est l'agitation perpétuelle et le besoin constant de distractions. Une autre est la maladie périodique de la guerre totale. À tout cela s'ajoute aujourd'hui l'accoutumance croissante à une cruauté à la fois massive et raffinée, au maniement le plus brutal de la matière humaine. Avec tout cela, nous ne pouvons plus ni dire ni penser que nous ayons reçu d'en haut la mission d'apprendre à vivre à l'univers.\par
Malgré tout cela, nous avons sans doute certaines leçons à donner. Mais nous en avons beaucoup à recevoir de formes de vie qui, si imparfaites soient-elles, portent en tout cas dans leur passé millénaire la preuve de leur stabilité. On les accuse d'être immobiles. En réalité elles sont probablement toutes depuis longtemps décadentes. Mais elles tombent lentement.\par
Le malheur a suscité en nous, Français, une aspiration très vive vers notre propre passé. Ceux qui parlent de la tradition républicaine de la France ne pensent pas à la Troisième République, mais à 1789 et aux mouvements sociaux du début du siècle dernier. Ceux qui parlent de sa tradition chrétienne ne pensent pas à la monarchie, mais au moyen âge. Beaucoup parlent des deux, et le peuvent sans aucune contradiction. Ce passé est nôtre ; mais il a l'inconvénient d'être passé. Il est absent. Les civilisations millénaires d'Orient, malgré de très grandes différences, sont beaucoup plus proches de notre moyen âge que nous ne sommes nous-mêmes. En nous réchauffant au double rayonnement de notre passé et des choses présentes qui en constituent une image transposée, nous pouvons trouver la force de nous préparer un avenir.\par
Il y va du destin de l'espèce humaine. Car de même que l'hitlérisation de l'Europe préparerait sans doute l'hitlérisation du globe terrestre - accomplie soit par les Allemands, soit par leurs imitateurs japonais - de même une américanisation de l'Europe préparerait sans doute une américanisation du globe terrestre. Le second mal est moindre que le premier, mais il vient immé­diatement après. Dans les deux cas, l'humanité entière perdrait son passé. Or le passé est une chose qui, une fois tout à fait perdue, ne se retrouve jamais plus. L'homme par ses efforts fait en partie son propre avenir, mais il ne peut pas se fabriquer un passé. Il ne peut que le conserver.\par
Les Encyclopédistes croyaient que l'humanité n'a aucun intérêt à conserver son passé. Instruits par une expérience cruelle, nous sommes en train de revenir de cette croyance. Mais nous ne posons pas la question en termes assez clairs pour la trancher nettement.\par
Le fond de la question est simple. Si les facultés purement humaines de l'homme suffisent, il n'y a aucun inconvénient à faire table rase de tout le passé et à compter sur les ressources de la volonté et de l'intelligence pour vaincre toute espèce d'obstacle. C'est ce qu'on a cru, et c'est ce qu'au fond personne ne croit plus, excepté les Américains, parce qu'ils n'ont pas encore été étourdis par le choc du malheur.\par
Si l'homme a besoin d'un secours extérieur, et si l'on admet que ce secours est d'ordre spirituel, le passé est indispensable, parce qu'il est le dépôt de tous les trésors spirituels. Sans doute l'opération de la grâce, à la limite, met l'homme en contact direct avec un autre monde. Mais le rayonnement des trésors spirituels du passé peut seul mettre une âme dans l'état qui est la condition nécessaire pour que la grâce soit reçue. C'est pourquoi il n'y a pas de religion sans tradition religieuse, et cela est vrai même lorsqu'une religion nouvelle vient d'apparaître.\par
La perte du passé équivaut à la perte du surnaturel. Quoique ni l'une ni l'autre perte ne soit encore consommée en Europe, l'une et l'autre sont assez avancées pour que nous puissions constater expérimentalement cette corres­pondance.\par
Les Américains n'ont d'autre passé que le nôtre ; ils y tiennent, à travers nous, par des fils extrêmement ténus. Même malgré eux, leur influence va nous envahir et, si elle ne rencontre pas d'obstacle suffisant, leur ôtera leur peu de passé, si l'on peut s'exprimer ainsi, en même temps qu'elle nous privera du nôtre. De l'autre côté l'Orient s'est accroché obstinément à son passé jusqu'à ce que notre influence, moitié par le prestige de l'argent, moitié par celui des armes, soit venu le déraciner à moitié. Mais il ne l'est encore qu'à moitié. Pourtant l'exemple des japonais montre que quand des Orientaux se décident à adopter nos tares, en les ajoutant aux leurs propres, ils les portent à la deuxième puissance.\par
Nous, Européens, nous sommes au milieu. Nous sommes le pivot. Le destin du genre humain tout entier dépend sans doute de nous, pour un espace de temps probablement très bref. Si nous laissons échapper l'occasion, nous sombrerons probablement bientôt non seulement dans l'impuissance, mais dans le néant. Si, tout en gardant le regard tourné vers l'avenir, nous essayons de rentrer en communication avec notre propre passé millénaire ; si dans cet effort nous cherchons un stimulant dans une amitié réelle, fondée sur le respect, avec tout ce qui en Orient est encore enraciné, nous pourrions peut-être préserver d'un anéantissement presque total le passé, et en même temps la vocation spirituelle du genre humain.\par
L'aventure du Père de Foucauld, ramené à la piété, et par suite au Christ, par une espèce d'émulation devant le spectacle de la piété arabe, serait ainsi comme un symbole de notre prochaine renaissance.\par
Pour cela, il faut que les populations dites de couleur, même si elles sont primitives, cessent d'être des populations sujettes. Mais du point de vue esquissé ici, faire avec elles des nations à l'européenne, démocratiques ou non, ne vaudrait pas mieux ; ce serait d'ailleurs une folie, aussi bien dans les cas où c'est possible que dans ceux où c'est impossible. Il n'y a que trop de nations dans le monde.\par
Il n'y a qu'une seule solution, c'est de trouver pour le mot de protection une signification qui ne soit pas un mensonge. Jusqu'ici ce mot n'a été em­ployé que pour mentir. S'il est trop discrédité, on peut lui chercher un synony­me. L'essentiel est de trouver une combinaison par laquelle des populations non constituées en nations, et se trouvant à certains égards dans la dépendance de certains États organisés, soient suffisamment indépendantes à d'autres égards pour pouvoir se sentir libres. Car la liberté, comme le bonheur, se définit avant tout par le sentiment qu'on la possède. Ce sentiment ne peut être ni suggéré par la propagande ni imposé par l'autorité. On peut seulement, et très facilement, forcer les gens à l'exprimer sans l'éprouver. C'est ce qui rend la discrimination très difficile. Le critérium est une certaine intensité de vie morale qui est toujours liée à la liberté.\par
Il y a deux facteurs favorables pour la solution de ce problème. Le premier, c'est qu'il se posera aussi pour les populations faibles d'Europe. Cela peut faire espérer davantage qu'il sera étudié. Ce qu'on peut poser en principe dès maintenant, c'est que, par exemple, la patrie annamite et la patrie tchèque ou norvégienne méritent le même degré de respect.\par
L'autre facteur favorable, c'est que l'Amérique, n'ayant pas de colonies, et par suite pas de préjugés coloniaux, et appliquant naïvement ses critères démocratiques à tout ce qui ne la regarde pas elle-même, considère le système colonial sans sympathie. Elle est sans doute sur le point de secouer sérieuse­ment l'Europe engourdie dans sa routine. Or en prenant le parti des popula­tions soumises par nous, elle nous fournit, sans le comprendre, le meilleur secours pour résister dans l'avenir prochain à sa propre influence. Elle ne le comprend pas ; mais ce qui serait désastreux, ce serait que nous ne le comprenions pas non plus.\par
Tant que la guerre dure, tous les territoires du monde sont avant tout des terrains stratégiques et doivent être traités comme tels. Cela implique la double obligation de ne rien dire qui cause des bouleversements immédiats, et de ne pas non plus ôter toute espérance de changement à des millions d'êtres malheureux que le malheur peut jeter du côté de l'ennemi. C'est d'ailleurs ce double souci qui décide aussi de notre orientation à l'égard des problèmes sociaux en France.\par
Mais en mettant à part toute considération stratégique, du point de vue politique il serait désastreux de prendre publiquement une position qui cristal­lise le {\itshape statu quo ante}. Peut-être la défiance des Américains à notre égard, quand elle ne procède pas de mauvais motifs, vient-elle de cette crainte légiti­me d'une cristallisation qui, en empêchant les problèmes urgents de se poser, ôte tout espoir de les résoudre, jusqu'au moment où une nouvelle catastrophe mondiale les ouvrirait à nouveau.\par
En matière politique et sociale, notre position officielle consiste à être disponibles pour tout ce qui sera juste, possible et conforme à la volonté du peuple français. Cette position ne peut être tenue que si elle vaut pour tous les problèmes sans exception, avec cette différence que dans tous les problèmes concernant les relations avec des populations non françaises, quelles qu'elles soient, la volonté du peuple français doit être composée, en un compromis qui fasse équilibre, avec la volonté de ces populations et celle des grandes nations qui, après avoir remporté la victoire, auront plus ou moins la responsabilité de l'ordre dans le monde.\par
Jusqu'à une date récente, la France a été une grande nation. Elle ne l'est pas en ce moment. Elle le redeviendra rapidement si elle est capable de faire rapidement le nécessaire à cet effet. Il est naturel que nous en ayons tous l'espérance. Mais elle ne l'est pas de droit divin. Il n'y a pas plus de hiérarchie de droit divin en matière internationale qu'en matière politique. La reconnais­sance de cette vérité est compatible avec le patriotisme le plus intense.\par
La grandeur passée de la France est venue surtout de son rayonnement spirituel et de l'aptitude qu'elle semblait posséder à ouvrir des routes au genre humain.\par
Peut-être peut-elle retrouver quelque chose de cela, même avant d'avoir récupéré aucune puissance, même avant la libération du territoire. Prostrée, étendue à terre, encore à demi assommée, peut-être peut-elle quand même essayer de commencer de nouveau à penser le destin du monde. Non pas en décider, car elle n'a aucune autorité pour cela. Le penser, ce qui est tout à fait différent.\par
Ce serait peut-être là le meilleur stimulant, le meilleur chemin pour retrou­ver le respect de soi-même.\par
La première condition, c'est de se garder absolument de rien cristalliser d'avance en aucun domaine.\par

\backmatter \section[Appendice, (Ébauches et variantes)]{Appendice \\
(Ébauches et variantes)}\renewcommand{\leftmark}{Appendice \\
(Ébauches et variantes)}

\noindent \par
\par
\subsection[1. Un petit point d'histoire, (Lettre au Temps)  (1939)]{1. \\
Un petit point d'histoire \\
(Lettre au Temps) \protect\footnotemark {\itshape  }\\
(1939)}
\footnotetext{ Variante de : {\itshape Rome et l'Albanie.}}
\noindent \par
Monsieur le Directeur,\par
Dans un article du 9 novembre intitulé « La question d'Épire dans les rapports italo-grecs », votre collaborateur Albert Mousset écrit que les soldats italiens « ont repris la tradition de la Rome de Paul-Émile ».\par
L'évocation du souvenir de Paul-Émile en cette matière a une signification que, d'après le contexte, votre collaborateur ignore probablement. Car l’action de Paul-Émile en Épire - pays qui comprenait d'ailleurs à cette époque le sud de l'Albanie actuelle - mérite de ne pas être oubliée.\par
On lit dans un fragment de Polybe (cité par Strabon, VII, 322) : « Paul-Émile, après la défaite de la Macédoine et de Persée, détruisit soixante-dix villes d'Épire ; il y fit cent cinquante mille esclaves. »\par
Tite-Live, Plutarque, Appien ont raconté comment il s'y est pris. Je cite Appien ({\itshape Ill.} IX) :\par
« Paul-Émile, après avoir capturé Persée, obéissant à un sénatus-consulte secret, passa exprès, en revenant vers Rome, près de soixante-dix villes qui lui appartenaient. [Lui désigne le roi Gentius qui, ayant déclaré la guerre à Rome, avait été vaincu et pris captif en une campagne de vingt jours.] Les habitants prirent peur ; Paul-Émile promit qu'on leur pardonnerait le passé s'ils livraient tout ce qu'ils possédaient d'or et d'argent. Ils s'y engagèrent. Paul-Émile envoya dans chacune de ces villes un détachement de son armée ; il convint avec les chefs d'une date, la même pour tous, et leur enjoignit de faire proclamer à l'aube du jour fixé, dans chaque ville, l'ordre d'apporter l'argent sur la place publique dans un délai de trois heures ; puis, le délai écoulé, de livrer le reste au pillage. C'est ainsi que Paul-Émile fit le sac de soixante-dix villes en une seule heure. »\par
Un léger effort d'imagination permet de se représenter ces soixante-dix villes, peuplées en moyenne d'un peu plus de deux mille habitants chacune, dans une région maintenant si pauvre ; de se représenter dans ces villes des gens paisibles, qui croyaient avoir acheté, au prix du sacrifice de leur fortune, une pleine sécurité. Sans crainte, se fiant à la promesse solennelle d'un général romain, ils ont ouvert eux-mêmes leurs portes aux soldats ; et en un moment toutes les familles, des plus humbles aux plus honorées, sont transformées en un amas de corps destinés à être dispersés dans tous les coins de l'empire pour y exécuter les volontés d'un maître. Car c'est bien de cela qu'il s'agit. Les Grecs ont un mot pour exprimer la servitude politique d'un pays soumis par la conquête, et un mot différent pour exprimer la condition des esclaves vendus à l'encan. C'est de ce second mot que se sert Polybe. Cent cinquante mille hommes, femmes, enfants, possédant tous une position élevée ou basse dans la cité, ayant tous droit à des égards plus ou moins grands, se croyant tous assurés de conserver cette position et ces droits, furent transformés en un instant en bétail. Soixante-dix cités furent anéanties en un instant, et tout cela par le procédé si simple qui consiste à violer la parole donnée. Telle est la marque que Paul-Émile laissa dans l'Épire.\par
Espérant que vous regarderez comme moi ce petit point d'histoire comme susceptible d'intéresser vos lecteurs, je vous prie, M. le Directeur, de bien vouloir agréer l'assurance de ma considération distinguée.\par
Simone WEIL,\par
agrégée de l'Université.\par

\begin{center}
\end{center}
\subsection[2. Note sur les récents événements d’Allemagne (Variante) (25 novembre 1932)]{2. \\
Note sur les récents événements d’Allemagne (Variante) \\
(25 novembre 1932)}
\noindent \par
\textbf{Les enseignements \\
de la grève des transports à berlin}\par
La grève, décidée par 78 \% des ouvriers, malgré l'opposition des cadres syndicaux et sur l'appel des communistes et des hitlériens, faillit constituer un événement décisif. En effet, par la nature même du métier des grévistes, toute la population ouvrière était mise en mouvement par la grève. Les ouvriers, les ouvrières de Berlin, non seulement vinrent apporter à manger aux membres des piquets de grève, mais encore les aidèrent à empêcher le départ d'autobus ou de tramways conduits par des jaunes. Cela, avec un plein succès, et pendant plusieurs jours, dans un pays qui compte près de huit millions de chômeurs. Un tel mouvement pouvait, d'un moment à l'autre, prendre un caractère politique ; aussi la {\itshape Deutsche Allgemeine Zeitung} jeta-t-elle un cri d'alarme, et réclama-t-elle une action vigoureuse de la police, laquelle devait être « couverte par ses chefs même en cas d'agression ». Il faut remarquer qu'en même temps le {\itshape Vorwœrts} présentait la grève comme une provocation des communistes et des hitlériens, provocation qui risquait de permettre à von Papen d'ajourner les élections. Pour le {\itshape Vorwœrts}, les élections sont bien plus importantes que l'action du prolétariat.\par
Cette grève, qui durait encore le jour des élections, amena, pour le parti communiste, à Berlin, un succès foudroyant ; un gain de 138 596 voix, grâce auquel il dépasse les hitlériens de 141 000 voix et les social-démocrates de 214 300 voix, tandis qu'il arrivait seulement au troisième rang en juillet.\par
Les nazis, eux-mêmes, du fait qu'ils ont participé à la grève, n'ont perdu que 37 000 voix ; ainsi, étant donné les 100 000 électeurs, venus sans doute des nazis, qu'ont gagnés les nationaux-allemands, les hitlériens ont dû, à Berlin, gagner des voix ouvrières.\par
Mais, aussitôt les élections passées, le parti hitlérien donna, comme il était à prévoir, le mot d'ordre de reprise du travail, en même temps que les cadres syndicaux accentuaient leur pression ; et le travail reprit aussitôt.\par
Cela montre que même à Berlin, même dans les circonstances les plus favorables, même au moment précis où il remporte une victoire éblouissante sur le terrain électoral, la puissance du parti communiste allemand, quand il s'agit d'une action réelle, est, dès qu'il est réduit à ses propres forces, exactement nulle.\par
\par
On peut ainsi apprécier la sincérité ou le discernement de {\itshape l'Humanité}, selon laquelle les six millions de bulletins communistes représentent « six millions de combattants pour les luttes extra-parlementaires, six millions de futurs grévistes » .\par
\textbf{Les élections allemandes}\par
Les élections constituent un événement moins important que l'échec de la grève des transports.\par
Cependant, le peuple allemand a eu à répondre, en quelque sorte, à une question de confiance posée par le « gouvernement des barons » . Réponse écrasante. Plus de 98,8 \% des voix sont allées aux quatre partis d'opposition (communiste, social-démocrate, national-socialiste et centre). Plus de 70 \% des voix sont allées aux trois partis qui ont, cette fois, tous trois mis en tête de toute leur propagande le mot d'ordre : « Contre le gouvernement des barons, pour le socialisme. »\par
Les deux partis de gouvernement (nationaux-allemands et populistes), soutenus par tout l'appareil d'État, n'ont gagné qu'un million deux mille voix, évidemment détachées du courant grand-bourgeois qui se trouvait, à côté de bien d'autres courants, dans le mouvement hitlérien.En plus de cette perte prévue et normale, les hitlériens ont perdu près d'un million d'électeurs, qui, pour la plupart, se sont vraisemblablement abstenus. Ainsi la démagogie révolutionnaire du parti hitlérien n'a pu compenser le fait qu'on cesse de voir en lui une force souveraine. Néanmoins, il est bien loin de se désagréger ; il est encore de loin le parti le plus fort, et a encore presque deux fois plus de voix que le parti communiste.\par
Le « bloc marxiste », comme disent les hitlériens, n'a perdu que 17 300 voix ; ainsi son importance relative s'est accrue, vu le nombre des abstentions. Sa composition intérieure a changé. Comme en juillet, le parti communiste gagne, et la social-démocratie perd. Comme en juillet, les gains de l'un (604 511 voix) correspondent presque exactement aux pertes de l'autre (721 818 voix) ; et les chiffres de ces gains et de ces pertes atteignent presque ceux de juillet, ce qui montre combien le rythme s'accélère. Cependant, la social-démocratie, elle non plus, ne se désagrège pas ; elle dépasse encore le parti communiste de plus de un million deux cent mille voix.\par
L'échec des « barons », le succès des communistes, rendent le parti hitlé­rien indispensable à la grande bourgeoisie. La {\itshape Deutsche Allgemeine Zeitung} s'en aperçoit de plus en plus, et non sans angoisse. Les « barons » devront disparaître ou s'entendre avec les hitlériens. Si cette entente se produit, comment se produira-t-elle ? Par un « bloc des droites », du centre à Hitler ? Par un « gouvernement syndical » allant du chef syndical Leipart au national-socialiste Gregor Strasser ? La {\itshape Deutsche Allgemeine Zeitung} (c'est-à-dire l'industrie lourde), préférerait de beaucoup la première solution. De toutes manières, le danger fasciste, bien que peut-être non immédiat, est plus mena­çant que jamais. Tout mouvement avorté qui, comme la grève des transports, effraye la bourgeoisie sans l'affaiblir, le rend plus aigu.\par
({\itshape Libres Propos}, nouvelle série, 68 année, n° 11, 25 novembre 1932.)\par

\begin{center}
\end{center}
\subsection[3. La situation en Allemagne (note) (25 février 1933)]{3. \\
La situation en Allemagne (note) \\
(25 février 1933)}
\noindent \par
\subsubsection[Le camp ouvrier]{Le camp ouvrier}
\noindent La situation est inchangée. Le 22 janvier, les hitlériens ont accompli la provocation qu'ils avaient annoncée une semaine à l'avance, en défilant en plein quartier rouge de Berlin, devant la maison même du parti communiste. La contre-manifestation communiste était interdite ; le parti communiste empêcha, avec raison, ses militants d'aller se faire massacrer dans la rue ; il lança en revanche des appels à la grève ; mais, malgré la vive indignation des masses social-démocrates devant la provocation hitlérienne, ces appels, adressés « aux ouvriers » en général, et sans propositions concrètes aux organisations, ne rencontrèrent aucun écho.\par
Cette passivité laissait la voie libre pour Hitler ; et celui-ci fut effective­ment nommé chancelier quelques jours plus tard.\par
Il y eut de vives réactions de la classe ouvrière, au cours desquelles l'unité d'action fut réalisée à la base avec un plein succès (grève générale à Lübeck après le meurtre d'un bourgmestre social-démocrate, etc.) ; mais ces réactions se firent sans coordination, comme c'est toujours le cas pour les mouvements spontanés. Parti communiste et social-démocratie continuent à manœuvrer pour éviter le front unique en faisant semblant de le désirer. Le parti commu­niste a refusé brutalement une proposition de la social-démocratie (« pacte de non-agression » , c'est-à-dire arrêt des attaques haineuses entre les deux partis) qui ouvrait tout au moins la voie à des négociations. Si seulement des négociations étaient entamées, et si l'un des deux partis désirait réellement les faire aboutir à une unité d'action effective, il aurait les plus grandes chances d'y réussir, étant donné la poussée de la base. La situation est tellement claire qu'on doit supposer que l'Internationale communiste freine le mouvement révolutionnaire en Allemagne, aussi systématiquement que le fait la social-démocratie. Aucune « erreur » ne peut expliquer une telle politique. Quant aux propositions partielles de front unique (dans le cadre des organisations locales ou des fédérations syndicales), on ne peut y voir qu'une manœuvre destinée à donner le change au prolétariat, tant qu'elles ne seront pas accompagnées d'offres concrètes faites par le parti communiste lui-même au parti social-démocrate lui-même.\par
Si le prolétariat allemand ne réalise pas, avec ou contre les partis qui disent le défendre, son unité d'action, il demeurera en fait impuissant, en dépit de l'héroïsme déployé dans les luttes partielles.\par

\subsubsection[Le camp bourgeois]{Le camp bourgeois}
\noindent Quelle est la portée exacte de la nomination de Hitler comme chancelier ?\par
Tout d'abord, il est clair qu'il n'a pas le pouvoir. Le pouvoir militaire (la Reichswehr) et le pouvoir économique sont aux mains de la grande bour­geoisie. Ce gouvernement, comme les deux précédents, ceux de von Papen et de von Schleicher, n'est qu'une étape dans les efforts que fait la grande bourgeoisie allemande pour se servir, auprès des masses, de la démagogie hitlérienne, sans abandonner le pouvoir au mouvement hitlérien.\par
Il faut remarquer cependant le progrès continu accompli par le parti hitlérien, non pas quant à ses effectifs, ni quant à ses succès électoraux, mais eu égard au rapport des forces à l'intérieur de la bourgeoisie. von Papen essaya, en vain, de gouverner en écartant complètement Hitler du pouvoir ; von Schleicher essaya, tout aussi vainement, d'apprivoiser le mouvement hitlérien, en lui accordant un rôle de second ordre ; il fallut, malgré les vives répugnances de Hindenbourg et de tous les grands bourgeois, accorder le pouvoir à Hitler.\par
Ce progrès a été accompli au moyen du chantage. La grande bourgeoisie est résolue à accorder le moins possible à Hitler ; mais comme elle ne peut, sans compromettre jusqu'à son existence, se passer du mouvement hitlérien ni même le laisser s'affaiblir, ce minimum signifie en fait des concessions de plus en plus considérables. Le chantage continue à l'intérieur du gouverne­ment. Tant que Hitler gardera en main l'atout qui lui a permis d'obtenir le poste de chancelier, la bourgeoisie allemande devra lui accorder une part sans cesse croissante du pouvoir, et finalement, peut-être, le pouvoir total. Déjà il a remporté un succès, dit-on, en installant ses amis dans les principaux postes de l'administration et de la police en Prusse. Ce n'est pas que la possibilité d'une rupture entre la grande bourgeoisie et lui soit complètement exclue ; mais une telle rupture entraînerait aussitôt une situation si évidemment dangereuse, devant un prolétariat que les provocations fascistes ont fait jusqu'à un certain point sortir de sa passivité, qu'elle ne ferait peut-être que précipiter la capitu­lation de la grande bourgeoisie devant les exigences de Hitler. D'une manière générale, on peut dire que d'une part les fautes, les revirements, les divisions intérieures de la bourgeoisie, et d'autre part les succès partiels du prolétariat, renforcent la position de Hitler, du fait que Hitler s'appuie uniquement sur la crainte qu'éprouve en ce moment la bourgeoisie devant la menace d'un soulèvement ouvrier. Faute d'hommes audacieux, décidés, intelligents, à la tête du prolétariat allemand, celui-ci ne peut profiter assez rapidement de ses avantages momentanés et des faiblesses momentanées de l'adversaire ; dès lors la bourgeoisie, quand elle s'est mise dans une situation dangereuse, a le temps d'en prendre conscience et d'y parer en s'aidant de la démagogie et des bandes terroristes que le mouvement hitlérien met à sa disposition. Tous les événements qui effraient la bourgeoisie allemande, sans constituer pour elle une défaite effective, rapprochent le moment où le fascisme triomphera. Tel a été le cas de la grève des transports de Berlin.\par
Un seul événement, en dehors d'une insurrection victorieuse du prolétariat, pourrait, semble-t-il, compromettre la situation de Hitler ; ce serait un redressement rapide de l'économie allemande. Un tel redressement consoli­derait sans doute le fascisme, comme en Italie, s'il se produisait après la conquête du pouvoir effectif par Hitler ; mais s'il se produit avant que les partis de la grande bourgeoisie aient abandonné le pouvoir au national-socialisme, un retour de la prospérité pousserait sans doute la bourgeoisie à tenter d'adoucir la lutte des classes ; et la social-démocratie reviendrait peut-être en ce cas au premier plan de la scène politique.\par
En résumé, si l'on admet que l'économie tend vers une nouvelle période de prospérité, que la bourgeoisie allemande tend à s'unir derrière le parti hitlé­rien, que le prolétariat allemand tend à s'unir derrière les ouvriers révolution­naires, tout dépend du rythme selon lequel se développeront, les uns par rapport aux autres, ces trois facteurs décisifs.\par
({\itshape Libres Propos}, nouvelle série, 7\textsuperscript{e} année, n° 2, 25 février 1933)
\subsection[4. Quelques remarques, sur la réponse de la M.O.R. (Variante) (1939)]{4. \\
Quelques remarques \\
sur la réponse de la M.O.R. \\
(Variante) \\
(1939)}
\noindent \par
Les camarades M.O.R. de l'Hérault parlent, au sujet des articles publiés par l'E. E., d'analyse « bourrée d'erreurs et d'affirmations tendancieuses ». Ils se gardent de préciser. C'est qu'ils ne le peuvent pas, et qu'ils savent bien que tous les faits avancés sont malheureusement exacts.\par
Leur réponse, en revanche, ne contient presque que des inexactitudes, inexactitudes qui ne peuvent guère venir de l'ignorance, car, avant de se charger de la réponse aux articles incriminés, il faut supposer qu'ils se sont documentés.\par
Ils disent que le Parti Communiste a proposé à deux reprises à la social-démocratie une action commune, à savoir la grève générale. Il est exact que deux fois (contre von Papen et contre Hitler) le parti a lancé un appel en ce sens. Mais la tactique du front unique consiste-t-elle à lancer un appel à une organisation en vue d'une action qu'on détermine à l'avance, surtout quand il s'agit d'un mot d'ordre aussi sérieux que celui de la grève générale, et que ce mot d'ordre n'a pas été préparé ? Non. La tactique du front unique consiste à faire son possible pour constituer, à tous les degrés de la hiérarchie, des organisations de front unique, organisations au moyen desquelles chaque parti peut proposer sa solution à l'ensemble de la classe ouvrière organisée.\par
Il est exact que la social-démocratie allemande a repoussé ces deux appels. Mais, au début de février, elle a proposé au parti communiste ce qu'elle nommait un « pacte de non-agression », et celui-ci a répondu par un refus brutal, qui n'ouvrait pas la voie à d'autres négociations. Le 19 février, l'Inter­nationale Socialiste elle-même a proposé le front unique pour la lutte contre le fascisme et le capitalisme à l'Internationale Communiste. Sans réponse. Trois semaines après seulement, l'Internationale Communiste a engagé ses fractions nationales à faire aux partis socialistes correspondants des propositions de front unique, et même à accepter la fameuse « non-agression », c'est-à-dire l'interruption des attaques. À cette date, l'incendie du Reichstag avait déjà eu lieu ; les deux partis, communiste et social-démocrate, étaient déjà en train d'être réduits en fait à l'illégalité, et le front unique était devenu impassible en Allemagne. Il est difficile de concevoir pire incohérence.\par
Quant à la question des rapports avec Hitler, je ne veux pas répéter les faits que j'ai déjà cités, et auxquels les camarades de l'Hérault ne répondent pas. Je veux seulement noter l'analogie inquiétante que nous pouvons remar­quer entre la politique du parti communiste allemand et celle du parti français. Nous n'avons pas, heureusement, de mouvement national-socialiste ; mais les mouvements sur lesquels s'appuie toute politique fasciste existent chez nous. Ils sont aisés à reconnaître. Ce sont tous les mouvements de révolte qui ne se placent pas sur le terrain de la lutte des classes ; mouvement des anciens combattants, mouvement des contribuables, mouvement des petits commer­çants, mouvement agraire guidé par la Ligue Agraire. Le caractère fasciste de ces mouvements saute aux yeux. Notre parti communiste les appuie tous plus ou moins. {\itshape L'Humanité} a raconté avec approbation comment des petits commerçants et des paysans réunis par la...
\subsection[5. Réflexions pour déplaire, (Variante) (1936 ?)]{5. \\
Réflexions pour déplaire \\
(Variante) \\
(1936 ?)}
\noindent \par
Ce que je vais écrire déplaira, je le sais, à tous les camarades ou peu s'en faut. Mais quoi ! Nous n'avons pas ici à nous plaire mutuellement, nous avons à dire, chacun pour son compte, ce que nous pensons.\par
Nous suivons tous avec anxiété, avec angoisse, la lutte de nos camarades d'Espagne. Nous tâchons de les aider. Mais cela ne suffit pas. Il faut aussi tirer pour nous, honnêtement, les leçons de l'expérience qu'ils paient en ce moment de leur sang.\par
\par
Quand Lénine, après avoir tracé, dans ses écrits, l'esquisse d'un État sans armée, ni police, ni bureaucratie distinctes de la population, a commencé à construire la machine bureaucratique, militaire et policière la plus lourde qui ait jamais écrasé un malheureux peuple, on a pu interpréter cette volte-face de plusieurs manières. Lénine était le chef d'un parti politique ; il visait le pouvoir ; sa bonne foi pouvait être mise en question.\par
Mais on ne peut mettre en question la bonne foi libertaire de nos cama­rades d'Espagne. Cependant que voyons-nous ? Le conseil des milices, où ils ont toujours exercé l'influence dominante, vient de remettre en vigueur, à l'usage des miliciens du front d'Aragon, le code militaire de la République bourgeoise. La contrainte de la mobilisation vient se substituer à la pratique des engagements volontaires. Le conseil de la Généralité, où la C.N.T. détient les postes économiques, vient de prendre un décret selon lequel des heures supplémentaires non payées peuvent être imposées aux ouvriers sans aucune limite, et les ouvriers qui ne produiraient pas à la cadence jugée normale doivent être considérés comme factieux et traités comme tels; autrement dit, la peine de mort est appliquée à la production industrielle. La presse catalane, et notamment la {\itshape Soli} \footnote{Solidaridad obrera.}, organe de la C.N.T., exerce un bourrage de crâne qui dépasse peut-être celui des journaux français pendant la guerre. Quant à la police, on avoue publiquement que pendant les trois premiers mois de la guerre civile les comités d'investigation, les militants responsables et les individus irresponsables ont fusillé sans le moindre simulacre de jugement, donc sans aucune possibilité de contrôle. Au front, on fusille des enfants de seize ans, quand on les prend au cours d'un engagement les armes à la main. Quant aux paysans, ceux d'Aragon se plaignent que certaines colonnes - souhaitons que ce ne soient pas celles de la C.N.T. - exercent parfois une véritable dictature sur les villages occupés, et que par ailleurs on ne leur fournit pas les semences et outils nécessaires à leur travail, et qu'ils ne peuvent acheter faute d'argent.\par
\par

\begin{center}
\end{center}
\subsection[6. Quelques réflexions concernant l'honneur et la dignité nationale, (À propos de la dernière question d'Alain) (1936 ?)]{6. \\
Quelques réflexions concernant l'honneur et la dignité nationale \\
(À propos de la dernière question d'Alain) \\
(1936 ?)}
\noindent \par
Le sentiment de l'honneur est évidemment la source de toutes les guerres, considérées à l'échelle de l'individu. Toutes sortes d'intérêts économiques peuvent être en jeu entre deux peuples qui se battent ; mais ce ne sont certes pas ces intérêts qui fournissent à chaque combattant l'énergie nécessaire pour dominer la peur. On peut en dire autant des questions territoriales. Qu'impor­tait, somme toute, à un Marseillais que l'Alsace fût allemande ou française ? Qui d'entre nous, aujourd'hui, se sent malheureux parce que les Canadiens de race et de langue française font partie de l'Empire anglais ? Qui est-ce qui souffrirait si la Tunisie passait sous la domination italienne ou allemande ? Qui, surtout, est prêt à considérer ces questions comme des questions de vie ou de mort ? Elles sont à peu près aussi étrangères à chacun de nous que la présence d'Hélène à Sparte ou à Troie pouvait l'être aux guerriers grecs et troyens qui pendant dix ans moururent jour après jour auprès du rivage de la mer. Cet immortel poème de l'Iliade n'a pas vieilli. Au cours de tout combat, l'objet même du combat est toujours oublié. Il n'y a que deux mobiles assez forts pour pousser les hommes à tuer et à mourir, à savoir l'honneur et la puissance. Mais la puissance, on n'en parle guère, d'autant que beaucoup d'hommes accepteraient volontiers de vivre sans puissance ; il n'y a pas là un facteur assez fort d'unanimité. On parle d'honneur, car il est entendu qu'on ne peut pas vivre sans honneur. Dès que l'honneur entre en jeu, n'importe quoi peut devenir une question de vie ou de mort ; Diomède est alors tout prêt à mourir pour rendre Hélène à Ménélas ; les petits gars de Marseille et d'ailleurs versent leur sang pour restituer la nationalité française à des paysans alsaciens qui ne parlent qu'un dialecte germanique. La vie de toute une jeunesse est ainsi à la merci de quiconque possède les moyens de réveiller à telle ou telle occasion le sentiment de l'honneur offensé. Aujourd'hui, des souvenirs encore cuisants empêchent que l'on ose parler de se battre pour l'honneur ; en revanche, dans tous les milieux, on parle de maintenir « la paix dans l'honneur », « la paix dans la dignité ». Cette sinistre formule, qui, en 1914, sous la plume de Poincaré, a ouvert les hostilités, implique qu'on est résolu à subordonner le souci de la paix au souci de l'honneur. Il est urgent de regarder cette notion d'honneur ou de dignité en face une bonne fois. On s'abstient généralement de le faire, parce qu'on craint d'être soupçonné de lâcheté ; on se rend ainsi effectivement coupable d'une lâcheté intellectuelle qui dégrade la pensée.\par
………………………………………………………………………...\par
« La dignité vaut mieux que la vie », « l'honneur vaut mieux que la vie » , ces formules sont ambiguës. Elles peuvent vouloir dire qu'il vaut mieux mourir que de se mépriser. Certes tout homme digne de ce nom refusera toujours d'éviter la mort au prix du mépris de soi-même, et « pour vivre, de perdre les raisons de vivre ». Seulement la dignité ainsi comprise est chose intérieure à chaque être humain, qui jamais n'est engagée dans les affaires internationales. À qui, à quel homme déterminé est-ce qu'un conflit interna­tional peut imposer la nécessité de choisir entre la vie et l'estime de soi-même ? À proprement parler, aucun homme n'a le choix. Les non-combattants n'ont pas à aller mourir ; par suite ils ne risquent en aucun cas d'avoir à s'accuser eux-mêmes de lâcheté. Les combattants sont envoyés à la mort par contrainte, y compris même les engagés volontaires, puisque les engagements ne sont pas résiliables ; et la durée des hostilités est toujours déterminée par ceux qui n'y prennent aucune part. Il est exclusivement du ressort de chaque homme de décider à partir de quelles limites il lui devient impossible de préserver sa vie sans perdre sa propre estime ; nul ne peut confier même à un ami le soin de résoudre une pareille question.
\subsection[7. Réponse au questionnaire d'Alain, (variante) (1936 ?)]{7. \\
Réponse au questionnaire d'Alain \\
(variante) \\
(1936 ?)}
\noindent \par
Je n'ai le loisir de répondre qu'à la dernière question. Elle me paraît de beaucoup la plus importante. Depuis le 7 mars, dans tous les milieux, dans toutes les classes sociales, sans distinction de métier, de niveau de culture, d'opinion politique, on a pu observer un souci général de conserver la dignité de la France devant l'acte d'Hitler. Contrairement à ce que beaucoup d'entre nous ont longtemps espéré, ce souci semble aussi puissant dans le peuple que dans l'élite. La formule « la paix dans la dignité » , « la paix dans l'honneur » , formule qui subordonne le souci de la paix à celui de la dignité nationale, formule de sinistre mémoire qui, en 1914, a ouvert les hostilités, cette formule semble ne choquer presque personne. Mais il ne suffit pas de demander, comme Alain, si ceux qui parlent de sacrifier la vie à l'honneur sont aussi ceux qui sont menacés de mourir. La question est plus vaste.\par
La dignité est un attribut de l'être humain considéré comme individu. Certes elle comporte une certaine solidarité. Je peux considérer ma dignité comme blessée par un mauvais traitement infligé à d'autres êtres humains, ou à une collectivité dont je suis membre. Néanmoins ce sera moi et seulement moi qui serai en cause dès que je voudrai me placer sur le seul terrain de l'honneur. Dès qu'on parle de dignité ou d'honneur, il faut dissoudre la collectivité en individus. C'est ainsi seulement que la formule « La dignité vaut mieux que la vie » peut avoir un sens.\par
Encore faut-il savoir quel sens au juste. Le seul qu'on puisse raisonna­blement soutenir, c'est qu'il vaut mieux mourir que de se mépriser. Tout homme sent vivement que choisir le mépris de soi-même plutôt que la mort, c'est « pour vivre, perdre les raisons de vivre ». Il est facile dès lors de comprendre dans quel cas la dignité d'un homme peut être brisée par un outrage subi de la part d'autrui. C'est lorsqu'il est impossible de subir passivement l'outrage sans s'accuser soi-même de lâcheté. Ce qui délivre alors de la honte, ce n'est pas la vengeance, c'est le danger. C'est ce qui devient évident dès que l'on prend des exemples. Ainsi, lorsqu'un homme tout-puissant est insulté par un faible, la clémence ne le diminue pas, la vengeance ne l'honore pas, bien au contraire. De même tuer l'offenseur par ruse n'est pas un moyen de rétablir l'honneur. On peut en conclure que si, parmi ceux que la mobilisation n'appellerait pas au front, quelques-uns sentent leur dignité blessée par les actes d'un gouvernement étranger, aucune guerre ne peut leur servir de remède.\par
Mais la formule « La dignité vaut mieux que la vie » peut être interprétée autrement. On peut entendre par ces mots qu'il vaut mieux mourir et tuer que de supporter une humiliation, l'humiliation étant définie non par le mépris de soi, mais comme un rapport extérieur.\par

\begin{center}
\end{center}
\subsection[8. Progrès et production, (fragment) (Pouvoir des mots)  (1937 ?)]{8. \\
Progrès et production \\
(fragment) \\
(Pouvoir des mots) \protect\footnotemark {\itshape  }\\
(1937 ?)}
\footnotetext{ Peut-être une ébauche pour un texte destiné aux {\itshape Nouveaux Cahiers} qui avaient une rubrique : {\itshape Pouvoir des mots}. (Note de l'éditeur.)}
\noindent \par
Nous vivons dans un âge éclairé, qui a secoué les superstitions et les dieux. Il ne reste attaché qu'à quelques divinités qui réclament et obtiennent la plus haute considération intellectuelle, telles que Patrie, Production, Progrès, Science. Par malheur, ces divinités si épurées, si affinées, tout à fait abstraites comme il convient à une époque hautement civilisée, sont pour la plupart de l'espèce anthropophage. Elles aiment le sang. Il leur faut des sacrifices humains. Zeus était moins exigeant. Mais c'est qu'on n'aurait pas accordé à Zeus plus que quelques gouttes de vin et un peu de graisse de bœuf. Au lieu que le Progrès - que ne lui accorderait-on pas ? Aussi riait-on parfois de Zeus, tandis qu'on ne rit jamais du Progrès. Nous sommes une civilisation qui ne rit pas de ses dieux. Est-ce par hasard que depuis l'intronisation dans l'Olympe de ces dieux dont on ne rit pas, il n'y a presque plus de comédie ?\par
On peut tout accorder au Progrès, car on ignore tout à fait ce qu'il demande. Qui a jamais tenté de définir un progrès ? Si l'on proposait ce thème dans un concours, il serait sans doute instructif et amusant de comparer les formules. Je propose la définition que voici, la seule à mon avis pleinement satisfaisante et qui s'applique à tous les cas : on dit qu'il y a progrès toutes les fois que les statisticiens peuvent, après avoir dressé des statistiques compa­rées, en tirer une fonction qui croît avec le temps. S'il y a en France - simple supposition - deux fois plus d'hôpitaux qu'il y a vingt ans, trois fois plus qu'il y a quarante ans, il y a progrès. S'il y a deux, trois fois plus d'automobiles, il y a progrès. S'il y a deux, trois fois plus de canons, il y a progrès. S'il y a deux, trois fois plus de cas de tuberculose... mais non, cet exemple ne conviendra que le jour où on fabriquera de la tuberculose. Il convient d'ajouter à la définition ci-dessus que la fonction doit exprimer l'accroissement de choses fabriquées.\par

\subsection[9. Esquisse d'une apologie de la banqueroute  (1937 ?)]{9. \\
Esquisse d'une apologie de la banqueroute \protect\footnotemark  \\
(1937 ?)}
\footnotetext{Variante de Quelques méditations concernant l'économie. (Note de l'éditeur.)}
\noindent \par
Le mot de banqueroute est un de ceux qu'on emploie avec gêne, qui sonnent mal, comme adultère ou escroquerie. Quand on le prononce au sujet des finances de son propre pays, on parle volontiers d' « humiliante banque­route ». On peut chercher des excuses à une banqueroute, on peut trouver des raisons d'atténuer telle ou telle responsabilité, mais personne n'a même l'idée que la banqueroute ne procède pas de quelque manière d'un péché ; personne ne considère qu'elle puisse constituer un phénomène normal. Déjà le vieux Céphalès, pour faire comprendre à Socrate qu'il avait mené une vie irrépro­chable, lui disait : « Je n'ai trompé personne, et j'ai payé mes dettes. » Socrate, ce mauvais esprit, doutait que ce fût là une définition suffisante de la justice. Le Français moyen - et nous sommes tous la plupart du temps des Français moyens - applique volontiers à l'État le critérium de Céphalès, du moins en ce qui concerne le second point ; car quant au premier, personne ne demande à un gouvernement de ne pas mentir.\par
Proudhon, dans ce lumineux ouvrage de jeunesse intitulé Qu'est-ce que la propriété ? prouve par le raisonnement le plus simple et le plus évident que l'idéal de ce bon Céphalès est une absurdité. L'idée fondamentale de Proud­hon, dans ce petit livre trop méconnu, c'est que la propriété est non pas mauvaise, non pas injuste, mais impossible. Il entend par propriété non pas le droit de posséder un bien quelconque, mais le droit bien plus important de le prêter à intérêt, quelque forme que prenne d'ailleurs cet intérêt : loyer, fermage, rente, dividende.\par
La démonstration de Proudhon repose sur une loi mathématique fort claire. La fructification du capital implique une progression géométrique. Le capital ne rapporterait-il que 1 \%, il s'accroîtrait néanmoins selon une pro­gression géométrique à raison de 1+1/100. Toute progression géométrique engendre des grandeurs astronomiques avec une rapidité qui dépasse l'imagi­nation. Un calcul simple montre qu'un capital ne rapportant que l'intérêt dérisoire de 1 \% double en un siècle, se multiplie par sept en deux siècles ; et, avec l'intérêt encore modeste de 3 \%, il est centuplé dans le même espace de temps. Il est donc mathématiquement impossible que tous les hommes d'un pays soient vertueux à la manière de Céphalès pendant deux siècles ; car, bien qu'une portion relativement petite des biens meubles et immeubles soit louée ou placée à intérêt, il est mathématiquement impossible que la valeur de cette portion centuple en quelques générations. S'il est nécessaire à l'ordre social que les gens payent leurs dettes, il est plus nécessaire encore que les gens ne payent pas leurs dettes.\par
Depuis qu'existent la monnaie et le prêt à intérêt, l'humanité oscille entre ces deux nécessités contradictoires, et toujours avec une inconscience digne d'admiration. Si on s'amusait à reprendre toute l'histoire connue en la présentant comme l'histoire des dettes payées et non payées, on arriverait à rendre compte d'une bonne partie des grands événements passés. Chacun sait que, par exemple, la réforme de Solon à Athènes, la création des tribuns à Rome sont issues de troubles suscités par l'endettement excessif de la population. L'endettement de l'État a toujours constitué un phénomène non moins fertile en conséquences. Qu'il s'agisse de la population ou de l'État, il n'y a jamais eu d'autre remède à l'endettement que l'abolition des dettes, ouverte ou déguisée.\par

\begin{center}
…………………………………………………………………………….\end{center}
\noindent [Nous savons construire des mécanismes qui reviennent] à l'état initial dès qu'une certaine limite est dépassée ; mais nous ne savons pas construire de tels agencements automatiques pour la machine sociale. Les souffrances, le sang et les larmes des hommes en tiennent lieu.\par
Nous pouvons aujourd'hui nous livrer à des méditations amères sur le phénomène de l'endettement. L'État français a été engagé jusqu'à mi-corps par la guerre dans cet engrenage mathématique dont le pays ne semble pas pouvoir se libérer. Le sang a été consommé gratuitement et bientôt oublié, ou peu s'en faut ; mais les familles qui ont donné leurs fils ont prêté leur argent, et ce prêt vieux de plus de vingt ans nous étrangle tous les jours davantage. Machiavel disait que les hommes oublient plus facilement la mort de leur père que la perte de leur patrimoine. Il avait raison sans doute, mais la justesse de cette formule nous apparaîtrait aujourd'hui d'une manière plus éclatante si on disait fils au lieu de père. Aucun gouvernement n'a osé encore annoncer qu'il considérait comme nulles les charges financières léguées par la guerre. Malgré toutes les difficultés d'une telle opération il faudra pourtant un jour en arriver là, car il est impossible que ces charges continuent longtemps encore à faire boule de neige. C'est d'autant plus impossible que les charges qui procèdent de la guerre éventuelle font une autre boule de neige non moins redoutable. Car aujourd'hui encore les mêmes hommes qui donneraient sans hésiter leur sang ou celui des leurs et ne demanderaient rien en échange ont besoin de quatre pour cent et d'une garantie de change pour collaborer à la défense nationale.\par
La notion de contrat entre l'État et les particuliers est une absurdité à une époque comme la nôtre.\par

\begin{center}
\end{center}
\subsection[10. Méditation sur un cadavre, (variante) (1937)]{10. \\
Méditation sur un cadavre \\
(variante) \\
(1937)}
\noindent \par
Il n'y a pas de difficultés économiques. Il n'y a que des difficultés politiques.\par
Les hommes se résignent plus facilement à leurs souffrances s'ils les croient imposées par le pouvoir que s'ils croient que le pouvoir essaie de les en délivrer sans y parvenir. Car, par une singulière aberration, le pouvoir leur semble une force plus invincible que la nature des choses.\par
Le gouvernement de Front Populaire - le premier, celui qui restera sous ce nom - est mort ; il appartient désormais à l'histoire, il est aussi passé que le règne d'Antonin ou de Caligula ; et, déliés de toute obligation de partisans ou d'adversaires à son égard, nous pouvons, en simples spectateurs, rêver librement devant son cadavre.\par
Combien l'heure où il s'est formé est proche et lointaine ! Heure de cauchemar pour quelques-uns ; pour beaucoup, songe merveilleux, ivresse des Saturnales où l'âme, depuis longtemps contractée par la soumission et la contrainte, se dilate dans l'extase d'une délivrance à laquelle elle ne croit qu'à moitié. Pour tous, atmosphère de rêve, d'irréalité. C'était en effet comme un rêve. Car qu'y avait-il de changé, entre mars 1936 et juin de la même année, dans les forces dont l'équilibre constituait la société française ? Rien, sinon que ceux qui en mars ne parlaient que pour commander se trouvaient trop heureux, en juin, qu'on voulût bien encore leur laisser la parole pour émettre des avis ; ceux qui en hiver se croyaient parqués jusqu'à la mort dans le troupeau à qui on accorde seulement le droit de se taire imaginaient, au solstice d'été, que leurs cris pouvaient changer le cours des astres.\par
L'imagination collective, dont les soudains retournements feront toujours le désespoir de ceux qui désirent comprendre l'histoire, est un facteur réel de la vie sociale, et des plus importants. Dans une certaine mesure elle modèle la réalité à son image, en ce sens que tant que la multitude souffrante croit ne rien pouvoir, elle ne peut effectivement rien ; et quand elle croit tout pouvoir, elle peut effectivement quelque chose, jusqu'au jour où, sentant que ce quelque chose n'est pas tout, elle retombe dans son premier sentiment d'im­puissance. Retz analysait supérieurement ce jeu de pendule. On peut en dire autant d'ailleurs, toute proportion gardée, de tous les groupements sociaux susceptibles d'avoir part au pouvoir. Mais l'imagination collective est instable, et ses reflux ne laissent le plus souvent après eux rien qui ressemble aux images qu'elle portait dans son flux.\par
L'art politique consiste à prévoir plus ou moins ces oscillations mysté­rieuses, à les sentir quand elles se produisent, à utiliser à chaque moment dans sa plénitude cette force que constitue l'imagination collective, force aveugle et qu'un homme qui sait où il va peut diriger ; à en diminuer par des artifices les remous. Car il y a un art politique. Les dictateurs le savent ; rien de plus raffiné que la manière dont gouverne Mussolini, ou même Hitler, ou même Staline. Les démocrates l'ignorent, du moins en France, car il se peut que par exemple Roosevelt ne l'ignore pas. Nos hommes de gauche, et particulière­ment nos socialistes, gouvernent comme si le choix du moment, l'ordre de succession dans les mesures prises, la manière de présenter les mesures, et tant d'autres choses analogues n'importaient pas en politique. Un concerto peut être ruiné si le chef d'orchestre fait attaquer telle note quelques secondes trop tôt ; et la politique, tellement plus complexe, pourrait se travailler comme une chose placée hors du temps et de l'espace ? Il n'en est rien. On a porté, on portera sur l'équipe gouvernementale de juin 1936 bien des jugements injustes par trop d'hostilité ou de faveur ; peut-être l'appréciation la plus juste consisterait-elle à les ranger parmi ces architectes qui ne savent faire que des dessins fort agréables sur le papier, mais non conformes aux lois des matériaux de construction ; ou parmi ces poètes qui ne savent écrire que des projets de poèmes rédigés en prose ; ou parmi ces auteurs dramatiques dont les œuvres font de l'effet sous forme de livres, mais ne passent jamais la rampe ; bref parmi tous ceux dont les bonnes intentions pavent l'enfer. Les gens de ce caractère, quand ils se mêlent d'agir, sont souvent traités de purs théoriciens ; mais ils pèchent au contraire par insuffisance de théorie. Ils ont négligé de méditer sur la matière et les instruments propres à leur art.\par
La méditation sur la matière et les instruments de l'art politique est plus facile aux dictateurs ou aux politiciens ambitieux qu'à d'autres, parce qu'ils méprisent les hommes. Un dictateur peut aimer très vivement son peuple, mais un tel amour ressemble probablement beaucoup à celui d'un cavalier pour un beau cheval ; cela peut être fort tendre, mais cela s'accommode parfaitement d'une vue claire, froide et cynique de l'usage du mors et de la cravache. Les hommes d'esprit sincèrement civique, comme Léon Blum, ont au cœur un tout autre amour, sans aucun mélange de mépris ; cet amour-là, comme les larmes, brouille la vue. Ce n'est pas qu'à l'égard des hommes le mépris soit plus justifié que l'estime ; il y a chez la plupart des hommes, ou plutôt chez tous, assez de bassesse et assez de vertu sublime pour justifier pleinement l'un ou l'autre. D'ailleurs le mépris ou l'estime des hommes, cela définit moins un jugement qu'une résolution prise une fois pour toutes soit d'utiliser leurs faiblesses, soit de contribuer à maintenir vivantes leurs grandeurs.\par
C'est pour une autre raison que le mépris des hommes favorise une vue lucide de l'art politique ; c'est parce que cet art a les hommes pour matière, et considérer les hommes comme une simple matière aux mains de quelques techniciens, c'est déjà une vue méprisante. Cependant, sans une telle vue, il n'y a pas d'art de gouverner. Aussi les grands hommes d'État, au cours de l'histoire, ont-ils presque tous et peut-être tous exercé leur génie dans le sens d'une plus grande oppression. Les Gracques, avec leur grand cœur et leur grande intelligence, n'ont su que périr misérablement ; le plus noble sang du monde a ainsi coulé en vain, et l'histoire l'a presque oublié, alors qu'elle a immortalisé le nom d'Auguste.\par
On peut comprendre ainsi que ce gouvernement de juin 1936, dirigé par l'homme le plus intelligent de notre personnel politique, ait commis non seulement, comme il était inévitable, beaucoup de fautes, mais certaines fautes si grossières. Une intelligence ne peut être tout à fait vigoureuse sans un peu de cynisme, et le cynisme s'allie rarement à l'esprit civique. Mussolini a lu et médité Machiavel ; il l'a compris ; il n'a guère fait que l'appliquer. Léon Blum ne s'est certainement pas formé sur la lecture de Machiavel, ce physicien du pouvoir politique. Une telle formation l'aurait empêché de négliger quelques maximes lumineuses qui sont à l'exercice du pouvoir ce qu'est le solfège au chant.\par
L'une de ces maximes, c'est que celui qui s'empare du pouvoir doit prendre tout de suite toutes les mesures de rigueur qu'il estime nécessaires, et n'en plus prendre par la suite, ou en tout cas de moins en moins. Puisque les ministres socialistes croient à l'efficacité de certaines mesures fiscales et financières pour remplir les caisses de l'État, soutenir la monnaie et établir un peu de civisme en matière d'argent, ils devaient de toute évidence prendre ces mesures en juin 1936 et non un an plus tard. Il fallait à ce moment faire publiquement le bilan d'une situation déjà catastrophique et décider aussitôt toutes les mesures d'exception jugées désirables, y compris la dévaluation ; et tout ce qu'on a accusé par la suite le « mur d'argent » d'avoir empêché, il fallait le tenter alors. La droite, ou ce qu'on nomme ainsi, était alors résignée et préparée à recevoir des coups. Quand un pouvoir nouvellement institué commence par assener à ses adversaires les coups qu'il veut leur donner, puis les laisse à peu près tranquilles, ils lui savent gré de tout le mal qu'ils n'en souffrent pas ; quand il commence par ménager les adversaires, ils s'irritent ensuite de la moindre menace. Le pire de tout est de les ménager tout en laissant peser sans cesse sur eux des menaces vagues et jamais réalisées ; on s'attire alors à la fois l'hostilité et le mépris, et on se perd. C'est ce qui est arrivé.\par
Le principe fondamental du pouvoir et de toute action politique, c'est qu'il ne faut jamais présenter l'apparence de la faiblesse. La force se fait non seulement craindre, mais en même temps toujours un peu aimer, même par ceux qu'elle fait violemment plier sous elle ; la faiblesse non seulement n'est pas redoutée, mais inspire toujours un peu de mépris et de répulsion même à ceux qu'elle favorise. Il n'y a pas de vérité plus amère, et c'est pourquoi elle est généralement méconnue. Sylla, après son abdication, a vécu en parfaite sécurité dans cette Rome où il avait fait couler tant de sang ; les Gracques ont péri lâchement abandonnés par cette multitude à qui ils avaient voué leur vie On croit généralement que les hommes se déterminent d'après des raisonne­ments soit sur la justice, soit sur leur intérêt ; en réalité l'empire de la force façonne souverainement sentiments et pensées. Combien de gens, sans même s'en rendre compte, concevaient et sentaient la question sociale, dont les données n'avaient pourtant pas changé, tout autrement en juin 1936 qu'en avril de la même année ! Les guerres sont suivies de révolutions dans les pays vaincus, non dans les pays vainqueurs. Pourtant le peuple vainqueur a autant de raisons de se révolter et autant de puissance pour le faire que le peuple vaincu ; mais celui-ci a en face de lui un État qui s'est montré faible. Cette force qui règne jusque dans les consciences est toujours en grande partie imaginaire. Le lion rampe devant le dompteur qui présente l'apparence d'une force invincible, et lui lèche la main ; le même lion dévore le dompteur qui a laissé voir de la crainte ou de l'irrésolution. L'individu en face de la foule est toujours un peu comme le dompteur devant le lion ; c'est la situation de l'homme au pouvoir.\par

\begin{center}
\end{center}
\subsection[11. Les nouvelles données du problème colonial dans l'empire français, (variante) (1938)]{11. \\
Les nouvelles données du problème colonial dans l'empire français \\
(variante) \\
(1938)}
\noindent \par
La colonisation est essentiellement l'effet d'un rapport de forces. Elle commence presque toujours par l'exercice de la force sous sa forme pure, c'est-à-dire par la conquête. Le peuple conquis voit d'abord avec douleur, la portion du peuple attachée aux traditions voit fort longtemps avec douleur, ses coutumes, ses mœurs, sa culture en partie abolies et en partie dégradées par une domination étrangère. La puissance d'attraction de la civilisation victo­rieuse, parée de ses propres prestiges et de tous ceux de la victoire, peut fournir pourtant un principe de collaboration. Mais quand arrive à la maturité une jeunesse élevée dans la culture du vainqueur, cette jeunesse supporte mal d'être traitée en inférieure par des hommes à qui elle se sent semblable et égale. Les transformations économiques bouleversent d'abord des habitudes ; la technique occidentale finit sans doute par en imposer par sa puissance, mais si après la colonisation la misère de la masse se trouve accrue, ou même simplement maintenue, ou même si elle diminue légèrement, mais à un rythme qui ne correspond pas à la mise en valeur du pays, la technique apparaît comme un bien monopolisé par des étrangers et que le peuple conquis souhaite utiliser pour lui. Une collaboration entre colonisateurs et colonisés n'est possible que si d'abord des égards convenables sont accordés aux tradi­tions de la population, et surtout si chaque étape dans l'assimilation apparaît comme une étape vers l'indépendance politique et économique.\par
Au contraire, si le peuple autrefois soumis par les armes a le sentiment que le vainqueur compte prolonger indéfiniment le rapport de conquérant à conquis, il s'établit une paix qui diffère de la guerre uniquement par le fait que l'un des camps est privé d'armes. Il est naturel, par malheur, que toute colonisation tende par une sorte de phénomène d'inertie à établir une telle situation. Que la nation colonisatrice se réclame par tradition d'un idéal de liberté n'y change rien. Car le manque de liberté dans les colonies met les sujets dans l'impossibilité de se plaindre, et les citoyens de la métropole sont bien trop persuadés qu'ils sont humains et généreux pour s'enquérir avec soin des motifs légitimes de plainte qui peuvent exister sur des terres lointaines. La générosité va très rarement jusqu'à dépenser du temps et des soins pour rechercher des injustices lointaines et muettes dont on pourrait être indirec­tement responsable. Aussi les conflits de doctrines et de partis dans la nation colonisatrice n'ont-ils que peu de retentissement en bien ou en mal sur le régime des colonies. Les colonies n'ont guère à compter sur la générosité tant qu'elles n'ont pas de force. Mais c'est justement la force qui leur manque.
 


% at least one empty page at end (for booklet couv)
\ifbooklet
  \newpage\null\thispagestyle{empty}\newpage
\fi

\ifdev % autotext in dev mode
\fontname\font — \textsc{Les règles du jeu}\par
(\hyperref[utopie]{\underline{Lien}})\par
\noindent \initialiv{A}{lors là}\blindtext\par
\noindent \initialiv{À}{ la bonheur des dames}\blindtext\par
\noindent \initialiv{É}{tonnez-le}\blindtext\par
\noindent \initialiv{Q}{ualitativement}\blindtext\par
\noindent \initialiv{V}{aloriser}\blindtext\par
\Blindtext
\phantomsection
\label{utopie}
\Blinddocument
\fi
\end{document}
