%%%%%%%%%%%%%%%%%%%%%%%%%%%%%%%%%
% LaTeX model https://hurlus.fr %
%%%%%%%%%%%%%%%%%%%%%%%%%%%%%%%%%

% Needed before document class
\RequirePackage{pdftexcmds} % needed for tests expressions
\RequirePackage{fix-cm} % correct units

% Define mode
\def\mode{a4}

\newif\ifaiv % a4
\newif\ifav % a5
\newif\ifbooklet % booklet
\newif\ifcover % cover for booklet

\ifnum \strcmp{\mode}{cover}=0
  \covertrue
\else\ifnum \strcmp{\mode}{booklet}=0
  \booklettrue
\else\ifnum \strcmp{\mode}{a5}=0
  \avtrue
\else
  \aivtrue
\fi\fi\fi

\ifbooklet % do not enclose with {}
  \documentclass[french,twoside]{book} % ,notitlepage
  \usepackage[%
    papersize={105mm, 297mm},
    inner=12mm,
    outer=12mm,
    top=20mm,
    bottom=15mm,
    marginparsep=0pt,
  ]{geometry}
  \usepackage[fontsize=9.5pt]{scrextend} % for Roboto
\else\ifav
  \documentclass[french,twoside]{book} % ,notitlepage
  \usepackage[%
    a5paper,
    inner=25mm,
    outer=15mm,
    top=15mm,
    bottom=15mm,
    marginparsep=0pt,
  ]{geometry}
  \usepackage[fontsize=12pt]{scrextend}
\else% A4 2 cols
  \documentclass[twocolumn]{report}
  \usepackage[%
    a4paper,
    inner=15mm,
    outer=10mm,
    top=25mm,
    bottom=18mm,
    marginparsep=0pt,
  ]{geometry}
  \setlength{\columnsep}{20mm}
  \usepackage[fontsize=9.5pt]{scrextend}
\fi\fi

%%%%%%%%%%%%%%
% Alignments %
%%%%%%%%%%%%%%
% before teinte macros

\setlength{\arrayrulewidth}{0.2pt}
\setlength{\columnseprule}{\arrayrulewidth} % twocol
\setlength{\parskip}{0pt} % classical para with no margin
\setlength{\parindent}{1.5em}

%%%%%%%%%%
% Colors %
%%%%%%%%%%
% before Teinte macros

\usepackage[dvipsnames]{xcolor}
\definecolor{rubric}{HTML}{0c71c3} % the tonic
\def\columnseprulecolor{\color{rubric}}
\colorlet{borderline}{rubric!30!} % definecolor need exact code
\definecolor{shadecolor}{gray}{0.95}
\definecolor{bghi}{gray}{0.5}

%%%%%%%%%%%%%%%%%
% Teinte macros %
%%%%%%%%%%%%%%%%%
%%%%%%%%%%%%%%%%%%%%%%%%%%%%%%%%%%%%%%%%%%%%%%%%%%%
% <TEI> generic (LaTeX names generated by Teinte) %
%%%%%%%%%%%%%%%%%%%%%%%%%%%%%%%%%%%%%%%%%%%%%%%%%%%
% This template is inserted in a specific design
% It is XeLaTeX and otf fonts

\makeatletter % <@@@


\usepackage{blindtext} % generate text for testing
\usepackage{contour} % rounding words
\usepackage[nodayofweek]{datetime}
\usepackage{DejaVuSans} % font for symbols
\usepackage{enumitem} % <list>
\usepackage{etoolbox} % patch commands
\usepackage{fancyvrb}
\usepackage{fancyhdr}
\usepackage{fontspec} % XeLaTeX mandatory for fonts
\usepackage{footnote} % used to capture notes in minipage (ex: quote)
\usepackage{framed} % bordering correct with footnote hack
\usepackage{graphicx}
\usepackage{lettrine} % drop caps
\usepackage{lipsum} % generate text for testing
\usepackage[framemethod=tikz,]{mdframed} % maybe used for frame with footnotes inside
\usepackage{pdftexcmds} % needed for tests expressions
\usepackage{polyglossia} % non-break space french punct, bug Warning: "Failed to patch part"
\usepackage[%
  indentfirst=false,
  vskip=1em,
  noorphanfirst=true,
  noorphanafter=true,
  leftmargin=\parindent,
  rightmargin=0pt,
]{quoting}
\usepackage{ragged2e}
\usepackage{setspace}
\usepackage{tabularx} % <table>
\usepackage[explicit]{titlesec} % wear titles, !NO implicit
\usepackage{tikz} % ornaments
\usepackage{tocloft} % styling tocs
\usepackage[fit]{truncate} % used im runing titles
\usepackage{unicode-math}
\usepackage[normalem]{ulem} % breakable \uline, normalem is absolutely necessary to keep \emph
\usepackage{verse} % <l>
\usepackage{xcolor} % named colors
\usepackage{xparse} % @ifundefined
\XeTeXdefaultencoding "iso-8859-1" % bad encoding of xstring
\usepackage{xstring} % string tests
\XeTeXdefaultencoding "utf-8"
\PassOptionsToPackage{hyphens}{url} % before hyperref, which load url package
\usepackage{hyperref} % supposed to be the last one, :o) except for the ones to follow
\urlstyle{same} % after hyperref

% TOTEST
% \usepackage{hypcap} % links in caption ?
% \usepackage{marginnote}
% TESTED
% \usepackage{background} % doesn’t work with xetek
% \usepackage{bookmark} % prefers the hyperref hack \phantomsection
% \usepackage[color, leftbars]{changebar} % 2 cols doc, impossible to keep bar left
% \usepackage[utf8x]{inputenc} % inputenc package ignored with utf8 based engines
% \usepackage[sfdefault,medium]{inter} % no small caps
% \usepackage{firamath} % choose firasans instead, firamath unavailable in Ubuntu 21-04
% \usepackage{flushend} % bad for last notes, supposed flush end of columns
% \usepackage[stable]{footmisc} % BAD for complex notes https://texfaq.org/FAQ-ftnsect
% \usepackage{helvet} % not for XeLaTeX
% \usepackage{multicol} % not compatible with too much packages (longtable, framed, memoir…)
% \usepackage[default,oldstyle,scale=0.95]{opensans} % no small caps
% \usepackage{sectsty} % \chapterfont OBSOLETE
% \usepackage{soul} % \ul for underline, OBSOLETE with XeTeX
% \usepackage[breakable]{tcolorbox} % text styling gone, footnote hack not kept with breakable



% Metadata inserted by a program, from the TEI source, for title page and runing heads
\title{\textbf{ Oppression et liberté }}
\date{1934}
\author{Simone Weil}
\def\elbibl{Simone Weil. 1934. \emph{Oppression et liberté}}
\def\elsource{Simone Weil. \emph{Oppression et liberté} [Texte écrit en 1934.]}

% Default metas
\newcommand{\colorprovide}[2]{\@ifundefinedcolor{#1}{\colorlet{#1}{#2}}{}}
\colorprovide{rubric}{red}
\colorprovide{silver}{Gray}
\@ifundefined{syms}{\newfontfamily\syms{DejaVu Sans}}{}
\newif\ifdev
\@ifundefined{elbibl}{% No meta defined, maybe dev mode
  \newcommand{\elbibl}{Titre court ?}
  \newcommand{\elbook}{Titre du livre source ?}
  \newcommand{\elabstract}{Résumé\par}
  \newcommand{\elurl}{http://oeuvres.github.io/elbook/2}
  \author{Éric Lœchien}
  \title{Un titre de test assez long pour vérifier le comportement d’une maquette}
  \date{1566}
  \devtrue
}{}
\let\eltitle\@title
\let\elauthor\@author
\let\eldate\@date


\defaultfontfeatures{
  % Mapping=tex-text, % no effect seen
  Scale=MatchLowercase,
  Ligatures={TeX,Common},
}

\@ifundefined{\columnseprulecolor}{%
    \patchcmd\@outputdblcol{% find
      \normalcolor\vrule
    }{% and replace by
      \columnseprulecolor\vrule
    }{% success
    }{% failure
      \@latex@warning{Patching \string\@outputdblcol\space failed}%
    }
}{}

\hypersetup{
  % pdftex, % no effect
  pdftitle={\elbibl},
  % pdfauthor={Your name here},
  % pdfsubject={Your subject here},
  % pdfkeywords={keyword1, keyword2},
  bookmarksnumbered=true,
  bookmarksopen=true,
  bookmarksopenlevel=1,
  pdfstartview=Fit,
  breaklinks=true, % avoid long links
  pdfpagemode=UseOutlines,    % pdf toc
  hyperfootnotes=true,
  colorlinks=false,
  pdfborder=0 0 0,
  % pdfpagelayout=TwoPageRight,
  % linktocpage=true, % NO, toc, link only on page no
}


% generic typo commands
\newcommand{\astermono}{\medskip\centerline{\color{rubric}\large\selectfont{\syms ✻}}\medskip\par}%
\newcommand{\astertri}{\medskip\par\centerline{\color{rubric}\large\selectfont{\syms ✻\,✻\,✻}}\medskip\par}%
\newcommand{\asterism}{\bigskip\par\noindent\parbox{\linewidth}{\centering\color{rubric}\large{\syms ✻}\\{\syms ✻}\hskip 0.75em{\syms ✻}}\bigskip\par}%

% lists
\newlength{\listmod}
\setlength{\listmod}{\parindent}
\setlist{
  itemindent=!,
  listparindent=\listmod,
  labelsep=0.2\listmod,
  parsep=0pt,
  % topsep=0.2em, % default topsep is best
}
\setlist[itemize]{
  label=—,
  leftmargin=0pt,
  labelindent=1.2em,
  labelwidth=0pt,
}
\setlist[enumerate]{
  label={\bf\color{rubric}\arabic*.},
  labelindent=0.8\listmod,
  leftmargin=\listmod,
  labelwidth=0pt,
}
\newlist{listalpha}{enumerate}{1}
\setlist[listalpha]{
  label={\bf\color{rubric}\alph*.},
  leftmargin=0pt,
  labelindent=0.8\listmod,
  labelwidth=0pt,
}
\newcommand{\listhead}[1]{\hspace{-1\listmod}\emph{#1}}

\renewcommand{\hrulefill}{%
  \leavevmode\leaders\hrule height 0.2pt\hfill\kern\z@}

% General typo
\DeclareTextFontCommand{\textlarge}{\large}
\DeclareTextFontCommand{\textsmall}{\small}


% commands, inlines
\newcommand{\anchor}[1]{\Hy@raisedlink{\hypertarget{#1}{}}} % link to top of an anchor (not baseline)
\newcommand\abbr[1]{#1}
\newcommand{\autour}[1]{\tikz[baseline=(X.base)]\node [draw=rubric,thin,rectangle,inner sep=1.5pt, rounded corners=3pt] (X) {\color{rubric}#1};}
\newcommand\corr[1]{#1}
\newcommand{\ed}[1]{ {\color{silver}\sffamily\footnotesize (#1)} } % <milestone ed="1688"/>
\newcommand\expan[1]{#1}
\newcommand\foreign[1]{\emph{#1}}
\newcommand\gap[1]{#1}
\renewcommand{\LettrineFontHook}{\color{rubric}}
\newcommand{\initial}[2]{\lettrine[lines=2, loversize=0.3, lhang=0.3]{#1}{#2}}
\newcommand{\initialiv}[2]{%
  \let\oldLFH\LettrineFontHook
  % \renewcommand{\LettrineFontHook}{\color{rubric}\ttfamily}
  \IfSubStr{QJ’}{#1}{
    \lettrine[lines=4, lhang=0.2, loversize=-0.1, lraise=0.2]{\smash{#1}}{#2}
  }{\IfSubStr{É}{#1}{
    \lettrine[lines=4, lhang=0.2, loversize=-0, lraise=0]{\smash{#1}}{#2}
  }{\IfSubStr{ÀÂ}{#1}{
    \lettrine[lines=4, lhang=0.2, loversize=-0, lraise=0, slope=0.6em]{\smash{#1}}{#2}
  }{\IfSubStr{A}{#1}{
    \lettrine[lines=4, lhang=0.2, loversize=0.2, slope=0.6em]{\smash{#1}}{#2}
  }{\IfSubStr{V}{#1}{
    \lettrine[lines=4, lhang=0.2, loversize=0.2, slope=-0.5em]{\smash{#1}}{#2}
  }{
    \lettrine[lines=4, lhang=0.2, loversize=0.2]{\smash{#1}}{#2}
  }}}}}
  \let\LettrineFontHook\oldLFH
}
\newcommand{\labelchar}[1]{\textbf{\color{rubric} #1}}
\newcommand{\milestone}[1]{\autour{\footnotesize\color{rubric} #1}} % <milestone n="4"/>
\newcommand\name[1]{#1}
\newcommand\orig[1]{#1}
\newcommand\orgName[1]{#1}
\newcommand\persName[1]{#1}
\newcommand\placeName[1]{#1}
\newcommand{\pn}[1]{\IfSubStr{-—–¶}{#1}% <p n="3"/>
  {\noindent{\bfseries\color{rubric}   ¶  }}
  {{\footnotesize\autour{ #1}  }}}
\newcommand\reg{}
% \newcommand\ref{} % already defined
\newcommand\sic[1]{#1}
\newcommand\surname[1]{\textsc{#1}}
\newcommand\term[1]{\textbf{#1}}

\def\mednobreak{\ifdim\lastskip<\medskipamount
  \removelastskip\nopagebreak\medskip\fi}
\def\bignobreak{\ifdim\lastskip<\bigskipamount
  \removelastskip\nopagebreak\bigskip\fi}

% commands, blocks
\newcommand{\byline}[1]{\bigskip{\RaggedLeft{#1}\par}\bigskip}
\newcommand{\bibl}[1]{{\RaggedLeft{#1}\par\bigskip}}
\newcommand{\biblitem}[1]{{\noindent\hangindent=\parindent   #1\par}}
\newcommand{\dateline}[1]{\medskip{\RaggedLeft{#1}\par}\bigskip}
\newcommand{\labelblock}[1]{\medbreak{\noindent\color{rubric}\bfseries #1}\par\mednobreak}
\newcommand{\salute}[1]{\bigbreak{#1}\par\medbreak}
\newcommand{\signed}[1]{\bigbreak\filbreak{\raggedleft #1\par}\medskip}

% environments for blocks (some may become commands)
\newenvironment{borderbox}{}{} % framing content
\newenvironment{citbibl}{\ifvmode\hfill\fi}{\ifvmode\par\fi }
\newenvironment{docAuthor}{\ifvmode\vskip4pt\fontsize{16pt}{18pt}\selectfont\fi\itshape}{\ifvmode\par\fi }
\newenvironment{docDate}{}{\ifvmode\par\fi }
\newenvironment{docImprint}{\vskip6pt}{\ifvmode\par\fi }
\newenvironment{docTitle}{\vskip6pt\bfseries\fontsize{18pt}{22pt}\selectfont}{\par }
\newenvironment{msHead}{\vskip6pt}{\par}
\newenvironment{msItem}{\vskip6pt}{\par}
\newenvironment{titlePart}{}{\par }


% environments for block containers
\newenvironment{argument}{\itshape\parindent0pt}{\vskip1.5em}
\newenvironment{biblfree}{}{\ifvmode\par\fi }
\newenvironment{bibitemlist}[1]{%
  \list{\@biblabel{\@arabic\c@enumiv}}%
  {%
    \settowidth\labelwidth{\@biblabel{#1}}%
    \leftmargin\labelwidth
    \advance\leftmargin\labelsep
    \@openbib@code
    \usecounter{enumiv}%
    \let\p@enumiv\@empty
    \renewcommand\theenumiv{\@arabic\c@enumiv}%
  }
  \sloppy
  \clubpenalty4000
  \@clubpenalty \clubpenalty
  \widowpenalty4000%
  \sfcode`\.\@m
}%
{\def\@noitemerr
  {\@latex@warning{Empty `bibitemlist' environment}}%
\endlist}
\newenvironment{quoteblock}% may be used for ornaments
  {\begin{quoting}}
  {\end{quoting}}

% table () is preceded and finished by custom command
\newcommand{\tableopen}[1]{%
  \ifnum\strcmp{#1}{wide}=0{%
    \begin{center}
  }
  \else\ifnum\strcmp{#1}{long}=0{%
    \begin{center}
  }
  \else{%
    \begin{center}
  }
  \fi\fi
}
\newcommand{\tableclose}[1]{%
  \ifnum\strcmp{#1}{wide}=0{%
    \end{center}
  }
  \else\ifnum\strcmp{#1}{long}=0{%
    \end{center}
  }
  \else{%
    \end{center}
  }
  \fi\fi
}


% text structure
\newcommand\chapteropen{} % before chapter title
\newcommand\chaptercont{} % after title, argument, epigraph…
\newcommand\chapterclose{} % maybe useful for multicol settings
\setcounter{secnumdepth}{-2} % no counters for hierarchy titles
\setcounter{tocdepth}{5} % deep toc
\markright{\@title} % ???
\markboth{\@title}{\@author} % ???
\renewcommand\tableofcontents{\@starttoc{toc}}
% toclof format
% \renewcommand{\@tocrmarg}{0.1em} % Useless command?
% \renewcommand{\@pnumwidth}{0.5em} % {1.75em}
\renewcommand{\@cftmaketoctitle}{}
\setlength{\cftbeforesecskip}{\z@ \@plus.2\p@}
\renewcommand{\cftchapfont}{}
\renewcommand{\cftchapdotsep}{\cftdotsep}
\renewcommand{\cftchapleader}{\normalfont\cftdotfill{\cftchapdotsep}}
\renewcommand{\cftchappagefont}{\bfseries}
\setlength{\cftbeforechapskip}{0em \@plus\p@}
% \renewcommand{\cftsecfont}{\small\relax}
\renewcommand{\cftsecpagefont}{\normalfont}
% \renewcommand{\cftsubsecfont}{\small\relax}
\renewcommand{\cftsecdotsep}{\cftdotsep}
\renewcommand{\cftsecpagefont}{\normalfont}
\renewcommand{\cftsecleader}{\normalfont\cftdotfill{\cftsecdotsep}}
\setlength{\cftsecindent}{1em}
\setlength{\cftsubsecindent}{2em}
\setlength{\cftsubsubsecindent}{3em}
\setlength{\cftchapnumwidth}{1em}
\setlength{\cftsecnumwidth}{1em}
\setlength{\cftsubsecnumwidth}{1em}
\setlength{\cftsubsubsecnumwidth}{1em}

% footnotes
\newif\ifheading
\newcommand*{\fnmarkscale}{\ifheading 0.70 \else 1 \fi}
\renewcommand\footnoterule{\vspace*{0.3cm}\hrule height \arrayrulewidth width 3cm \vspace*{0.3cm}}
\setlength\footnotesep{1.5\footnotesep} % footnote separator
\renewcommand\@makefntext[1]{\parindent 1.5em \noindent \hb@xt@1.8em{\hss{\normalfont\@thefnmark . }}#1} % no superscipt in foot


% orphans and widows
\clubpenalty=9996
\widowpenalty=9999
\brokenpenalty=4991
\predisplaypenalty=10000
\postdisplaypenalty=1549
\displaywidowpenalty=1602
\hyphenpenalty=400
% Copied from Rahtz but not understood
\def\@pnumwidth{1.55em}
\def\@tocrmarg {2.55em}
\def\@dotsep{4.5}
\emergencystretch 3em
\hbadness=4000
\pretolerance=750
\tolerance=2000
\vbadness=4000
\def\Gin@extensions{.pdf,.png,.jpg,.mps,.tif}
% \renewcommand{\@cite}[1]{#1} % biblio

\makeatother % /@@@>
%%%%%%%%%%%%%%
% </TEI> end %
%%%%%%%%%%%%%%


%%%%%%%%%%%%%
% footnotes %
%%%%%%%%%%%%%
\renewcommand{\thefootnote}{\bfseries\textcolor{rubric}{\arabic{footnote}}} % color for footnote marks

%%%%%%%%%
% Fonts %
%%%%%%%%%
\usepackage[]{roboto} % SmallCaps, Regular is a bit bold
% \linespread{0.90} % too compact, keep font natural
\newfontfamily\fontrun[]{Roboto Condensed Light} % condensed runing heads
\ifav
  \setmainfont[
    ItalicFont={Roboto Light Italic},
  ]{Roboto}
\else\ifbooklet
  \setmainfont[
    ItalicFont={Roboto Light Italic},
  ]{Roboto}
\else
\setmainfont[
  ItalicFont={Roboto Italic},
]{Roboto Light}
\fi\fi
\renewcommand{\LettrineFontHook}{\bfseries\color{rubric}}
% \renewenvironment{labelblock}{\begin{center}\bfseries\color{rubric}}{\end{center}}

%%%%%%%%
% MISC %
%%%%%%%%

\setdefaultlanguage[frenchpart=false]{french} % bug on part


\newenvironment{quotebar}{%
    \def\FrameCommand{{\color{rubric!10!}\vrule width 0.5em} \hspace{0.9em}}%
    \def\OuterFrameSep{\itemsep} % séparateur vertical
    \MakeFramed {\advance\hsize-\width \FrameRestore}
  }%
  {%
    \endMakeFramed
  }
\renewenvironment{quoteblock}% may be used for ornaments
  {%
    \savenotes
    \setstretch{0.9}
    \normalfont
    \begin{quotebar}
  }
  {%
    \end{quotebar}
    \spewnotes
  }


\renewcommand{\headrulewidth}{\arrayrulewidth}
\renewcommand{\headrule}{{\color{rubric}\hrule}}

% delicate tuning, image has produce line-height problems in title on 2 lines
\titleformat{name=\chapter} % command
  [display] % shape
  {\vspace{1.5em}\centering} % format
  {} % label
  {0pt} % separator between n
  {}
[{\color{rubric}\huge\textbf{#1}}\bigskip] % after code
% \titlespacing{command}{left spacing}{before spacing}{after spacing}[right]
\titlespacing*{\chapter}{0pt}{-2em}{0pt}[0pt]

\titleformat{name=\section}
  [block]{}{}{}{}
  [\vbox{\color{rubric}\large\raggedleft\textbf{#1}}]
\titlespacing{\section}{0pt}{0pt plus 4pt minus 2pt}{\baselineskip}

\titleformat{name=\subsection}
  [block]
  {}
  {} % \thesection
  {} % separator \arrayrulewidth
  {}
[\vbox{\large\textbf{#1}}]
% \titlespacing{\subsection}{0pt}{0pt plus 4pt minus 2pt}{\baselineskip}

\ifaiv
  \fancypagestyle{main}{%
    \fancyhf{}
    \setlength{\headheight}{1.5em}
    \fancyhead{} % reset head
    \fancyfoot{} % reset foot
    \fancyhead[L]{\truncate{0.45\headwidth}{\fontrun\elbibl}} % book ref
    \fancyhead[R]{\truncate{0.45\headwidth}{ \fontrun\nouppercase\leftmark}} % Chapter title
    \fancyhead[C]{\thepage}
  }
  \fancypagestyle{plain}{% apply to chapter
    \fancyhf{}% clear all header and footer fields
    \setlength{\headheight}{1.5em}
    \fancyhead[L]{\truncate{0.9\headwidth}{\fontrun\elbibl}}
    \fancyhead[R]{\thepage}
  }
\else
  \fancypagestyle{main}{%
    \fancyhf{}
    \setlength{\headheight}{1.5em}
    \fancyhead{} % reset head
    \fancyfoot{} % reset foot
    \fancyhead[RE]{\truncate{0.9\headwidth}{\fontrun\elbibl}} % book ref
    \fancyhead[LO]{\truncate{0.9\headwidth}{\fontrun\nouppercase\leftmark}} % Chapter title, \nouppercase needed
    \fancyhead[RO,LE]{\thepage}
  }
  \fancypagestyle{plain}{% apply to chapter
    \fancyhf{}% clear all header and footer fields
    \setlength{\headheight}{1.5em}
    \fancyhead[L]{\truncate{0.9\headwidth}{\fontrun\elbibl}}
    \fancyhead[R]{\thepage}
  }
\fi

\ifav % a5 only
  \titleclass{\section}{top}
\fi

\newcommand\chapo{{%
  \vspace*{-3em}
  \centering % no vskip ()
  {\Large\addfontfeature{LetterSpace=25}\bfseries{\elauthor}}\par
  \smallskip
  {\large\eldate}\par
  \bigskip
  {\Large\selectfont{\eltitle}}\par
  \bigskip
  {\color{rubric}\hline\par}
  \bigskip
  {\Large LIVRE LIBRE À PRIX LIBRE, DEMANDEZ AU COMPTOIR\par}
  \centerline{\small\color{rubric} {hurlus.fr, tiré le \today}}\par
  \bigskip
}}


\begin{document}
\pagestyle{empty}
\ifbooklet{
  \thispagestyle{empty}
  \centering
  {\LARGE\bfseries{\elauthor}}\par
  \bigskip
  {\Large\eldate}\par
  \bigskip
  \bigskip
  {\LARGE\selectfont{\eltitle}}\par
  \vfill\null
  {\color{rubric}\setlength{\arrayrulewidth}{2pt}\hline\par}
  \vfill\null
  {\Large LIVRE LIBRE À PRIX LIBRE, DEMANDEZ AU COMPTOIR\par}
  \centerline{\small{hurlus.fr, tiré le \today}}\par
  \newpage\null\thispagestyle{empty}\newpage
  \addtocounter{page}{-2}
}\fi

\thispagestyle{empty}
\ifaiv
  \twocolumn[\chapo]
\else
  \chapo
\fi
{\it\elabstract}
\bigskip
\makeatletter\@starttoc{toc}\makeatother % toc without new page
\bigskip

\pagestyle{main} % after style

  \section[Perspectives. Allons-nous vers la révolution prolétarienne ?]{Perspectives. Allons-nous vers la révolution prolétarienne ?}\renewcommand{\leftmark}{Perspectives. Allons-nous vers la révolution prolétarienne ?}

« \emph{Je n'ai que mépris pour le mortel qui se réchauffe avec des espérances creuses.}{\citbibl SOPHOCLE.} »\noindent Le moment depuis longtemps prévu est arrivé, ou le capitalisme est sur le point de voir son développement arrête par des limites infranchissables. De quelque manière que l'on interprète le phénomène de l'accumulation, il est clair que capitalisme signifie essentiellement expansion économique et que l'expansion capitaliste n'est plus loin du moment où elle se heurtera aux limites mêmes de la surface terrestre. Et cependant jamais le socialisme n'a été annonce par moins de signes précurseurs. Nous sommes dans une période de transition ; mais transition vers quoi ? Nul n'en a la moindre idée. D'autant plus frappante est l'inconsciente sécurité avec laquelle on s'installe dans la transition comme dans un état définitif, au point que les considérations concernant la crise du régime sont passées un peu partout a l'état de lieu commun. Certes on peut toujours croire que le socialisme viendra après-demain, et faire de cette croyance un devoir ou une vertu ; tant que l'on entendra de jour en jour par après-demain le surlendemain du jour présent, on sera sur de n'être jamais démenti ; mais un tel état d'esprit se distingue mal de celui des braves gens qui croient, par exemple, au jugement dernier. Si nous voulons traverser virilement cette sombre époque, nous nous abstiendrons, comme l'Ajax de Sophocle, de nous réchauffer avec des espérances creuses.\par
Tout au long de l'histoire, des hommes ont lutté, ont souffert et sont morts pour émanciper des opprimés. Leurs efforts, quand ils ne sont pas demeurés vains, n'ont jamais abouti à autre chose qu'à remplacer un régime d'oppression par un autre. Marx, qui en avait fait la remarque, a cru pouvoir établir scientifiquement qu'il en est autrement de nos jours, et que la lutte des opprimés aboutirait à présent à une émancipation véritable, non à une oppression nouvelle. C'est cette idée, demeurée parmi nous comme un article de foi, qu'il serait nécessaire d'examiner à nouveau, à moins de vouloir fermer systématiquement les yeux sur les événements des vingt dernières années. Épargnons-nous les désillusions de ceux qui, ayant lutté pour Liberté, Égalité, Fraternité, se sont trouves un beau jour avoir obtenu, comme dit Marx, Infanterie, Cavalerie, Artillerie. Encore ceux-là ont-ils pu tirer quelque enseignement des surprises de l'histoire ; plus triste est le sort de ceux qui ont péri en 1792 ou 93, dans la rue ou aux frontières, dans la persuasion qu'ils payaient de leur vie la liberté du genre humain. Si nous devons périr dans les batailles futures, faisons de notre mieux pour nous préparer à périr avec une vue claire du monde que nous abandonnerons.\par
La Commune de Paris a donné un exemple, non seulement de la puissance créatrice des masses ouvrières en mouvement, mais aussi de l'incapacité radicale, d'un mouvement spontané quand il s'agit de lutter contre une force organisée de -répression. Août 1914 a marqué la faillite de l'organisation des masses prolétariennes, sur le terrain politique et syndical, dans les cadres du régime. Dès ce moment, il a fallu abandonner une fois pour toutes l'espérance placée dans ce mode d'organisation non seulement par les réformistes, mais par Engels. En revanche, Octobre 1917 vint ouvrir de nouvelles et radieuses perspectives. On avait enfin trouvé le moyen de lier l'action légale à l'action illégale, le travail systématique des militants disciplinés au bouillonnement spontané des masses. Partout dans le monde devaient se former des partis communistes auxquels le parti bolchevik communiquerait son savoir ; ils devaient remplacer la social-démocratie, qualifiée par Rosa Luxembourg, dès août 1914, de « cadavre puant », et qui n'allait pas tarder à disparaître de la scène de l'histoire ; ils devaient s'emparer du pouvoir à brève échéance. Le régime politique créé spontanément par les ouvriers de Paris en 1871, puis par ceux de Saint-Pétersbourg en 1905, devait s'installer solidement en Russie et couvrir bientôt la surface du monde civilisé. Certes l'écrasement de la Révolution russe par une intervention brutale de l'impérialisme étranger pouvait anéantir ces brillantes perspectives ; mais à moins d'un semblable écrasement, Lénine et Trotsky étaient surs d'introduire dans l'histoire précisément cette série de transformations et non pas une autre.\par
Quinze ans se sont écoulés. La Révolution russe n'a pas été écrasée. Ses ennemis extérieurs et intérieurs ont été vaincus. Cependant nulle part sur la surface du globe, y compris le territoire russe, il n'y a de soviets ; nulle part sur la surface du globe, y compris le territoire russe, il n'y a de parti communiste proprement dit. Le « cadavre puant » de la social-démocratie a continué quinze ans durant a corrompre l'atmosphère politique, ce qui n'est guère le fait d'un cadavre ; s'il a été finalement en grande partie balayé, ç'a été par le fascisme et non parla révolution. Le régime issu d'Octobre, et qui devait s'étendre ou périr, s'est fort bien adapté, quinze ans durant, aux limites des frontières nationales ; son rôle à l'extérieur consiste à présent, comme les événements d'Allemagne le montrent avec évidence, à étrangler la lutte révolutionnaire du prolétariat. La bourgeoisie réactionnaire a fini par s'apercevoir elle-même qu'il est bien près d'avoir perdu toute force d'expansion, et se demande si elle ne pourrait pas à présent l'utiliser en contractant avec lui, en vue des guerres futures, des alliances défensives et offensives (cf. la {\itshape Deutsche Allgemeine Zeitung} du {\itshape 27} mai). A vrai dire ce régime ressemble au régime que croyait instaurer. Lénine dans la mesure où il exclut presque entièrement la propriété capitaliste ; pour tout le reste, il en est très exactement le contre-pied. Au lieu d'une liberté effective de la presse, l'impossibilité d'exprimer un jugement libre sous forme de document imprimé, ou dactylographié, ou manuscrit, ou même par la simple parole, sans risquer la déportation ; au lieu du libre jeu des partis dans les cadres du système soviétique, « un parti au pouvoir, et tous les autres en prison » ; au lieu d'un parti communiste destiné à rassembler, en vue d'une libre coopération, les hommes qui posséderaient le plus haut degré de dévouement, de conscience, de culture, d'esprit critique, une simple machine administrative, instrument passif aux mains du Secrétariat, et qui, au dire de Trotsky lui-même, n'a d'un parti que le nom ; au lieu de soviets, de syndicats et de coopératives fonctionnant démocratiquement et dirigeant la vie économique et politique, des organismes portant a vrai dire les mêmes noms, mais réduits à de simples appareils administratifs ; au lieu du peuple armé et organisé en milices pour assurer à lui seul la défense à l'extérieur et l'ordre à l'intérieur, une armée permanente, une police non contrôlée et cent fois mieux armée que celle du tsar ; enfin et surtout, au lieu des fonctionnaires élus, sans cesse contrôlés, sans cesse révocables, qui devaient assurer le gouvernement en attendant le moment où « chaque cuisinière apprendrait à gouverner l'État », une bureaucratie permanente, irresponsable, recrutée par cooptation, et possédant, par la concentration entre ses mains de tous les pouvoirs économiques et politiques, une puissance jusqu'ici inconnue dans l'histoire.\par
La nouveauté même d'un semblable régime le rend difficile à analyser. Trotsky persiste à dire qu'il s'agit d'une « dictature du prolétariat », d'un « État ouvrier » bien qu'à « déformations bureaucratiques », et que, concernant la nécessité, pour un tel régime, de s'étendre ou de périr, Lénine et lui ne se sont trompés que sur les délais. Mais quand une erreur de quantité atteint de telles proportions, il est permis de croire qu'il s'agit d'une erreur portant sur la qualité, autrement dit sur la nature même du régime dont on veut définir les conditions d'existence. D'autre part, nommer un État « État ouvrier » quand on explique par ailleurs que chaque ouvrier y est placé, économiquement et politiquement, a l'entière discrétion d'une caste bureaucratique, cela ressemble a une mauvaise plaisanterie. Quant aux « déformations », ce terme, singulièrement mal à sa place concernant un État dont tous les caractères sont exactement l'opposé de ceux que comporte théoriquement un État ouvrier, semble impliquer que le régime stalinien serait une sorte d'anomalie ou de maladie de la Révolution russe. Mais la distinction entre le pathologique et le normal n'a pas de valeur théorique. Descartes disait qu'une horloge détraquée n'est pas une exception aux lois de l'horloge, mais un mécanisme différent obéissant a ses lois propres ; de même il faut considérer le régime stalinien, non comme un État ouvrier détraque, mais comme un mécanisme social différent, défini par les rouages qui le composent, et fonctionnant conformément à la nature de ces rouages. Et, alors que les rouages d'un État ouvrier seraient les organisations démocratiques de la classe ouvrière, les rouages du régime stalinien sont exclusivement les pièces d'une administration centralisée dont dépend entièrement toute la vie économique, politique et intellectuelle du pays. Pour un tel régime, le dilemme « s'étendre ou périr » non seulement n'est plus valable, mais n'a même plus de sens ; le régime stalinien, en tant que système d'oppression, est aussi peu contagieux que pouvait l'être l'Empire pour les pays voisins de la France. La vue selon laquelle le régime stalinien constituerait une simple transition, soit vers le socialisme, soit vers le capitalisme, apparaît également comme arbitraire. L'oppression des ouvriers n'est évidemment pas une étape vers le socialisme. La « machine bureaucratique et militaire » qui constituait, aux yeux de Marx, le véritable obstacle à la possibilité d'une marche continue vers le socialisme par la simple accumulation de réformes successives, n'a sans doute pas perdu cette propriété du fait que, contrairement aux prévisions, elle survit à l'économie capitaliste. Quant à la restauration du capitalisme, qui ne pourrait se produire que comme une sorte de colonisation, elle n'est nullement impossible, en raison de l'avidité propre à tous les impérialismes et de la faiblesse économique et militaire de l'U.R.S.S. ; cependant les rivalités qui opposent les divers impérialismes empêchent, jusqu'ici, que le rapport des forces soit écrasant pour la Russie. En tout cas, la bureaucratie soviétique ne s'oriente nullement vers une capitulation, de sorte que le terme de transition serait de toute manière impropre. Rien ne permet de dire que la bureaucratie d'État russe prépare le terrain pour une domination autre que la sienne propre, qu'il s'agisse de la domination du prolétariat ou de celle de la bourgeoisie. En réalité, toutes les explications embarrassées par lesquelles les militants formés par le bolchevisme essaient de se dispenser de reconnaître la fausseté radicale des perspectives posées en octobre 1917, reposent sur le même préjugé que ces perspectives elles-mêmes, à savoir sur l'affirmation, considérée comme un dogme, qu'il ne peut y avoir actuellement que deux types d'État, l'État capitaliste et l'État ouvrier. À ce dogme, le développement du régime issu d'Octobre apporte le plus brutal démenti. D'État ouvrier, il n'en a jamais existé sur la surface de la terre, sinon quelques semaines à Paris, en 1871, et quelques mois peut-être en Russie, en 1917 et 1918. En revanche règne sur un sixième du globe, depuis près de quinze ans, un État aussi oppressif que n'importe quel autre et qui n'est ni capitaliste ni ouvrier. Certes Marx n'avait rien prévu de semblable. Mais Marx non plus ne nous est pas aussi cher que la vérité.\par
L'autre phénomène capital de notre époque, je veux dire le fascisme, ne rentre pas plus aisément que l'État russe dans les schémas du marxisme classique. Là-dessus aussi, bien entendu, il existe des lieux communs propres à sauver de la pénible obligation de réfléchir. Comme l'U.R.S.S. est un « État ouvrier » plus ou moins « déformé », le fascisme est un mouvement des masses petites-bourgeoises, reposant sur la démagogie, et qui constitue « la dernière carte de la bourgeoisie avant le triomphe de la révolution ». Car la dégénérescence du mouvement ouvrier a amené les théoriciens à représenter la lutte des classes comme un duel, ou un jeu entre partenaires conscients, et chaque évènement social ou politique comme une manœuvre de l'un des partenaires ; conception qui n'a pas plus de rapports avec le matérialisme que la mythologie grecque. Il existe des cercles restreints de grands financiers, de grands industriels, de politiciens réactionnaires qui défendent consciemment ce qu'ils pensent être les intérêts politiques de l'oligarchie capitaliste ; mais ils sont bien incapables aussi bien d'empêcher que de susciter un mouvement de masses comme le fascisme, ou même de le diriger. En fait, ils l'ont tantôt aidé, tantôt combattu, ont tente vainement de s'en faire un instrument docile et ont fini par capituler eux-mêmes devant lui. Certes c'est la présence d'un prolétariat exaspère qui fait pour eux de cette capitulation un moindre mal. Néanmoins le fascisme est tout autre chose qu'une carte entre leurs mains. La brutalité avec laquelle Hitler a congédié Hugenberg comme un domestique, et cela malgré les protestations de Krupp, est significative à cet égard. Il ne faut pas non plus oublier que le fascisme met radicalement fin à ce jeu des partis né du régime bourgeois et qu'aucune dictature bourgeoise, même en temps de guerre, n'avait encore supprimé ; et 'qu'il à installé à la place un régime politique dont la structure est à peu près celle du régime russe tel que l'a défini Tomsky : « Un parti au pouvoir et tous les autres en prison. » Ajoutons que la subordination mécanique du parti au chef est la même dans les deux cas, et assurée, dans les deux cas, par la police. Mais la souveraineté politique n'est rien sans la souveraineté économique ; aussi le fascisme tend-il à se rapprocher du régime russe aussi sur le terrain économique, par la concentration de tous les pouvoirs, aussi bien économiques que politiques, entre les mains du chef de l'État. Mais sur ce terrain, le fascisme se heurte à la propriété capitaliste qu'il ne veut pas détruire. Il y a là une contradiction dont on voit mal à quoi elle peut mener. Mais, de même que le mécanisme de l'État russe ne peut être expliqué par de simples « déformations », de même cette contradiction essentielle du mouvement fasciste ne peut être expliquée par la simple démagogie. Ce qui est sûr, c'est que, si le fascisme italien n'a obtenu la concentration des pouvoirs politiques qu'après de longues années qui ont épuisé son élan, le national-socialisme au contraire, parvenu au même résultat en moins de six mois, renferme encore une immense énergie et tend a aller beaucoup plus loin. Comme le montre notamment un rapport d'une grande société anonyme allemande, que {\itshape l'Humanité} a cité sans en apercevoir la signification, la bourgeoisie s'inquiète devant la menace de l'emprise étatique. Et effectivement Hitler a créé des organismes ayant un pouvoir souverain pour condamner ouvriers ou patrons à dix ans de travaux forcés et confisquer les entreprises.\par
L'on essaie vainement, pour faire rentrer à tout prix le national-socialisme dans les cadres du marxisme, de trouver, à l'intérieur même du mouvement, une forme déguisée de la lutte des classes entre la base, instinctivement socialiste, et les chefs, qui représenteraient les intérêts du grand capital et auraient pour tâche de duper les masses par une savante démagogie. Tout d'abord rien ne permet d'affirmer avec certitude que Hitler et ses lieutenants, quels que soient leurs liens avec le capital monopolisateur, en sont de simples instruments. Et surtout l'orientation des masses hitlériennes, si elle est violemment anticapitaliste, n'est nullement socialiste, non plus que la propagande démagogique des chefs ; car il s'agit de remettre l'économie non pas entre les mains des producteurs groupes en organisations démocratiques, mais bien entre les mains de l'appareil d'État. Or, bien que l'influence des réformistes et des staliniens l'ait fait oublier depuis longtemps, le socialisme, c'est la souveraineté économique des travailleurs et non pas de la machine bureaucratique et militaire de l'État. Ce qu'on nomme l'aile « national-bolchévique » du mouvement hitlérien n'est donc nullement socialiste. Ainsi les deux phénomènes politiques qui dominent notre époque ne peuvent ni l'un ni l'autre être situés dans le tableau traditionnel de la lutte des classes.\par
Il en est de même pour toute une série de mouvements contemporains issus de l'après-guerre, et remarquables par leurs affinités aussi bien avec le stalinisme qu'avec le fascisme. Telle est, par exemple, la revue allemande {\itshape Die Tat}, qui groupe une pléiade de jeunes et brillants économistes, est extrêmement proche du national-socialisme et considère l'U.R.S.S. comme le modèle de l'État futur, à l'abolition de la propriété privée près ; elle préconise actuellement une alliance militaire entre la Russie et l'Allemagne hitlérienne. En France, nous avons quelques cercles, comme celui de la revue {\itshape Plans}, où se retrouve une semblable ambiguïté. Mais le mouvement le plus significatif à cet égard, c'est ce mouvement technocratique qui a, dit-on, en un court espace de temps, couvert la surface des États-Unis ; on sait qu'il préconise, dans les limites d'une économie nationale fermée, l'abolition de la concurrence et des marchés et une dictature économique exercée souverainement par les techniciens. Ce mouvement, qu'on a souvent rapproche du stalinisme et du fascisme, a d'autant plus de portée qu'il ne semble pas être sans influence sur le cercle d'intellectuels de Columbia qui sont en ce moment les conseillers de Roosevelt.\par
De pareils courants d'idées sont quelque chose d'absolument nouveau et qui donne à notre époque son caractère propre. Au reste, la période. actuelle, si confuse soit-elle et si riche en courants politiques de toutes sortes, anciens et nouveaux, ne semble guère manquer que du mouvement même qui, d'après les prévisions, devrait en constituer le caractère essentiel, à savoir la lutte pour l'émancipation économique et politique des travailleurs. Il y a bien, dispersés çà et là et désunis par d'obscures querelles, une poignée de vieux syndicalistes et de communistes sincères ; il y a même quelques petites organisations qui ont gardé à peu près intacts les mots d'ordre socialistes. Mais l'idéal d'une société régie, sur le terrain économique et politique, par la coopération des travailleurs ne conduit presque plus aucun mouvement des masses, soit spontané, soit organisé ; et cela au moment même où il n'est question, dans tous les milieux, que de la faillite du capitalisme.\par
Devant cet état de choses, l'on est contraint, si l'on veut regarder la réalité en face, de se demander si le successeur du régime capitaliste ne doit pas être, plutôt que la libre association des producteurs, un nouveau système d'oppression. je voudrais à ce sujet soumettre une idée, à titre de simple hypothèse, à l'examen des camarades. On peut dire en abrégeant que l'humanité a connu jusqu'ici deux formes principales d'oppression, l'une, esclavage ou servage, exercée au nom de la force armée, l'autre au nom de la richesse transformée ainsi en capital ; il s'agit de savoir s'il n'est pas en ce moment en train de leur succéder une oppression d'une espèce nouvelle, l'oppression exercée au nom de la fonction.\par

\begin{center}
* * *\end{center}
\noindent La lecture même de Marx montre avec évidence que déjà, il y a un demi-siècle, le capitalisme avait subi des modifications profondes et de nature à transformer le mécanisme même de l'oppression. Cette transformation n'a fait que s'accentuer depuis la mort de Marx jusqu'à nos jours, et à un rythme particulièrement accéléré durant la période d'après-guerre. Déjà dans Marx il apparaît que le phénomène qui définit le capitalisme, à savoir l'achat et la vente de la force de travail, est devenu, au cours du développement de la grande industrie, un facteur subordonné dans l'oppression des masses laborieuses ; l'instant décisif, quant à l'asservissement du travailleur, n'est plus celui où, sur le marché du travail, l'ouvrier vend son temps au patron, mais celui où, à peine le seuil de l'usine franchi, il est happé par l'entreprise. On connaît, à ce sujet, les terribles formules de Marx : « Dans l'artisanat et la manufacture, le travailleur se sert de l'outil ; dans la fabrique, il est au service de la machine. » « Dans la fabrique existe un mécanisme mort indépendant des ouvriers, et qui se les incorpore comme des rouages vivants. » « Le renversement (du rapport entre le travailleur et les conditions du travail) ne devient une réalité saisissable dans la technique elle-même qu'avec le machinisme. » « La séparation des forces spirituelles du procès de production d'avec le travail manuel, et leur transformation en forces d'oppression du capital sur le travail, s'accomplit pleinement... dans la grande industrie construite sur la base du machinisme. Le détail de la destinée individuelle... de l'ouvrier travaillant à la machine disparaît comme une mesquinerie devant la science, les formidables forces naturelles et le travail collectif qui sont cristallisés dans le système des machines et constituent la puissance du maître. » Si l'on néglige la manufacture, qui peut être regardée comme une simple transition, on peut dire que l'oppression des ouvriers salariés, d'abord fondée essentiellement sur les rapports de propriété et d'échange, au temps des ateliers, est devenue par le machinisme un simple aspect des rapports contenus dans la technique même de la production. À l'opposition créée par l'argent entre acheteurs et vendeurs de la force de travail s'est ajoutée une autre opposition, créée par le moyen même de la production, entre ceux qui disposent de la machine et ceux dont la machine dispose. L'expérience russe a montré que, contrairement à ce que Marx a trop hâtivement admis, la première de ces oppositions peut être supprimée sans que disparaisse la seconde. Dans, les pays capitalistes, ces deux oppositions coexistent, et cette coexistence crée une confusion considérable. Les mêmes hommes se vendent au capital et servent la machine ; au contraire, ce ne sont pas toujours les mêmes hommes qui disposent des capitaux et qui dirigent l'entreprise.\par
À vrai dire, il existait encore, il n'y a pas bien longtemps, une catégorie d'ouvriers qui, tout en étant salariés, n'étaient pas de simples rouages vivants au service des machines, mais exécutaient au contraire leur travail en utilisant les machines avec autant de liberté, d'initiative et d'intelligence que l'artisan qui manie son outil ; c'étaient les ouvriers qualifiés. Cette catégorie d'ouvriers qui, dans chaque {\itshape entreprise, constituait} le facteur essentiel de la production a été à peu près supprimée par la rationalisation ; à présent un régleur se charge de disposer une certaine quantité de machines selon les exigences du travail à exécuter et le travail est accompli sous ses ordres par des manoeuvres spécialisés, capables seulement de faire fonctionner un type de machine et un seul par des gestes toujours identiques et auxquels l'intelligence n'a aucune part. Ainsi l'usine est partagée, actuellement, en deux camps nettement délimités, ceux qui exécutent le travail sans y prendre à proprement parler aucune part active, et ceux qui dirigent le travail sans rien exécuter. Entre ces deux fractions de la population d'une entreprise, la machine elle-même constitue une barrière infranchissable. En même temps, le développement du système des sociétés anonymes a établi une barrière, à vrai dire moins nette, entre ceux qui dirigent l'entreprise et ceux qui la possèdent. Un homme comme Ford, à la fois capitaliste et chef d'entreprise, apparaît de nos jours comme une survivance du passé, ainsi que l'a remarqué l'économiste américain Pound. « Les entreprises », écrit Palewski dans un livre paru en 1928, « tendent de plus en plus à échapper des mains de ces capitaines d'industrie, chefs et possesseurs primitifs de l'affaire... L'ère des conquérants tend peu à peu à n'être que le passé. Nous arrivons à l'époque qu'on a pu appeler l'ère des techniciens de la direction, et ces techniciens sont aussi éloignés des ingénieurs et des capitalistes que les ouvriers. Le chef n'est plus un capitaliste maître de l'entreprise, il est remplacé par un conseil de techniciens. Nous vivons encore sur ce passé si proche et l'esprit a quelque peine à saisir cette évolution. »\par
Ici encore il s'agit d'un phénomène que Marx avait aperçu. Seulement, tandis qu'au temps de Marx le personnel administratif de l'entreprise n'était guère qu'une équipe d'employés au service des capitalistes, de nos jours, en face des petits actionnaires réduits au simple rôle de parasites et des grands capitalistes principalement occupés du jeu financier, les « techniciens de la direction » constituent une couche sociale distincte, dont l'importance tend a croître et qui absorbe par diverses voies une quantité considérable des profits. Laurat, analysant dans son livre sur l'U.R.S.S. le mécanisme de l'exploitation exercée par la bureaucratie, note que « la consommation personnelle des bureaucrates », consommation disproportionnée, dans l'ensemble, avec la valeur des services rendus par eux, « effectuée régulièrement et à titre de revenu fixe », s'opère quasi indépendamment des nécessites d'accumulation qui ne se matérialisent dans la rubrique « bénéfices » que lorsque les « frais d'administration », c'est-à-dire les besoins de la bureaucratie, sont couverts ; et il oppose à ce système le système capitaliste où « la nécessité de l'accumulation prime le versement du dividende ». Mais il oublie que,, si l'accumulation passe avant les dividendes, les « frais d'administration », dans les sociétés capitalistes tout comme en U.R.S.S., passent avant l'accumulation. Jamais ce phénomène n'a été si frappant qu'aujourd'hui, où des entreprises proches de la faillite, ayant renvoyé une foule d'ouvriers, travaillant au tiers ou au quart de leur capacité de production, conservent presque intact un personnel administratif composé de quelques directeurs grassement rétribués et d'employés mal payés, mais en quantité tout à fait disproportionnée avec le rythme de la production. Ainsi il y a, autour de l'entreprise, trois couches sociales bien distinctes : les ouvriers, instruments passifs de l'entreprise, les capitalistes dont la domination repose sur un système économique en voie de décomposition, et les administrateurs qui s'appuient au contraire sur une technique dont l'évolution ne fait qu'augmenter leur pouvoir.\par
Ce développement de la bureaucratie dans l'industrie n'est que l'aspect le plus caractéristique d'un phénomène tout à fait général. L'essentiel de ce phénomène consiste dans une spécialisation qui s'accentue de jour en jour. La transformation qui a eu lieu dans l'industrie, où les ouvriers qualifiés, capables de comprendre et de manier toutes sortes de machines, ont été remplacés par des manoeuvres spécialisés, automatiquement dressés à servir une seule espèce de machine, cette transformation est l'image d'une évolution qui s'est produite dans tous les domaines. Si les ouvriers sont de plus en plus dépourvus de connaissances techniques, les techniciens, non seulement sont souvent assez ignorants de la pratique du travail, mais encore leur compétence est en bien des cas limitée à un domaine tout à fait restreint ; en Amérique on s'est même mis a créer des ingénieurs spécialisés, comme de vulgaires manoeuvres, dans une catégorie déterminée de machines, et, chose significative, l'U.R.S.S. s'est empressée d'imiter l'Amérique sur ce point. Il va de soi, au reste, que les techniciens ignorent les fondements théoriques des connaissances qu'ils utilisent. Les savants, à leur tour, non seulement restent étrangers aux problèmes techniques, mais sont de plus entièrement privés de cette vue d'ensemble qui est l'essence même de la culture théorique. On pourrait compter sur les doigts, dans le monde entier, les savants qui ont un aperçu de l'histoire et du développement de leur propre science ; il n'en est point qui soit réellement compétent à l'égard des sciences autres que la sienne propre. Comme la science forme un tout indivisible, on peut dire qu'il n'y a plus a proprement parler de savants, mais seulement des manoeuvres du travail scientifique, rouages d'un ensemble que leur esprit n'embrasse point. On pourrait multiplier les exemples. Dans presque tous les domaines, l'individu, enfermé dans les limites d'une compétence restreinte, se trouve pris dans un ensemble qui le dépasse, sur lequel il doit régler toute son activité, et dont il ne peut comprendre le fonctionnement. Dans une telle situation, il est une fonction qui prend une importance primordiale, à savoir celle qui consiste simplement à coordonner ; on peut la nommer fonction administrative ou bureaucratique. La rapidité avec laquelle la bureaucratie a envahi presque toutes les branches de l'activité humaine est quelque chose de stupéfiant des qu'on y songe. L'usine rationalisée, où l'homme se trouve privé, au profit d'un mécanisme inerte, de tout ce qui est initiative, intelligence, savoir, méthode, est comme une image de la société actuelle. Car la machine bureaucratique, pour être formée dé chair, et de chair bien nourrie, n'en est pas moins aussi irresponsable et aussi inconsciente que les machines de fer et d'acier. Toute l'évolution de la société actuelle tend à développer les diverses formes d'oppression bureaucratique et à leur donner une sorte d'autonomie par rapport au capitalisme proprement dit. Aussi notre devoir est-il de définir ce nouveau facteur politique plus clairement que n'a pu le faire Marx.\par
À vrai dire, Marx avait bien aperçu la force d'oppression que constitue la bureaucratie. Il avait parfaitement vu que le véritable obstacle aux réformes émancipatrices n'est pas le système des échanges et de la propriété, mais « la machine bureaucratique et militaire » de l'État. Il avait bien compris que la tare la plus honteuse qu'ait à effacer le socialisme, ce n'est pas le salariat, mais « la dégradante division du travail manuel et du travail intellectuel » ou, selon une autre formule, « la séparation des forces spirituelles du travail d'avec le travail manuel ». Mais Marx ne s'est pas demandé s'il ne s'agit pas là d'un ordre de problèmes indépendant des problèmes que pose le jeu de l'économie capitaliste proprement dite. Bien qu'il ait assisté à la séparation de la propriété et de la fonction dans l'entreprise capitaliste, il ne s'est pas demandé si la fonction administrative, dans la mesure où elle est permanente, ne pourrait pas, indépendamment de tout monopole de la propriété, donner naissance à une nouvelle classe oppressive. Et cependant, si l'on voit très bien comment une révolution peut « exproprier les expropriateurs », on ne voit pas comment un mode de production fondé sur la subordination de ceux qui exécutent a ceux qui coordonnent pourrait ne pas produire automatiquement une structure sociale définie par la dictature d'une caste bureaucratique. Non pas qu'on ne puisse imaginer un contrôle et un système de roulement qui rétablirait l'égalité aussi bien dans l'État que dans le procès même de la production industrielle ; mais en fait, quand une couche sociale se trouve pourvue d'un monopole quelconque, elle le conserve jusqu'à ce que les bases mêmes en soient sapées par le développement historique. C'est ainsi que le féodalisme est tombé non pas sous la poussée des masses populaires s'emparant elles-mêmes de la force armée, mais par la substitution du commerce à la guerre comme moyen principal de domination. De même, la couche sociale définie par l'exercice des fonctions d'administration n'acceptera jamais, quel que soit le régime légal de la propriété, d'ouvrir l'accès de ces fonctions aux masses laborieuses, d'apprendre « à chaque cuisinière à gouverner l'État » ou à chaque manœuvre à diriger l'entreprise. Tout régime de domination d'une classe sur une autre répond en somme, dans l'histoire, à la distinction entre une fonction sociale dominante et une ou plusieurs fonctions subordonnées ; ainsi, au moyen âge, la production était quelque chose de subordonne par rapport à la défense des champs à main armée ; à l'étape suivante, la production, devenue essentiellement industrielle, s'est trouvée subordonnée à la circulation. Il y aura socialisme quand la fonction dominante sera le travail productif lui-même ; mais c'est ce qui ne peut avoir lieu tant que durera un système de production où le travail proprement dit se trouve subordonné, par l'intermédiaire de la machine, à la fonction consistant à coordonner les travaux. Aucune expropriation ne peut résoudre ce problème, contre lequel s'est brisé l'héroïsme des ouvriers russes. La suppression de la division des hommes en capitalistes et en prolétaires n'implique nullement que doive disparaître, même progressivement, « la séparation des forces spirituelles du travail d'avec le travail manuel ».\par
Les technocrates américains ont tracé un tableau enchanteur d'une société où, le marché étant supprimé, les techniciens se trouveraient tout-puissants, et useraient de leur puissance de manière à donner à tous le plus de loisir et de bien-être possible. Cette conception rappelle, par son caractère utopique, celle du despotisme éclairé chère à nos pères. Toute puissance exclusive et non contrôlée devient oppressive aux mains de ceux qui en détiennent le monopole. Et des à présent l'on voit fort bien comment se dessine, a l'intérieur même du système capitaliste, l'action oppressive de cette couche sociale nouvelle. Sur le terrain de la production, la bureaucratie, mécanique irresponsable, engendre, comme l'a noté Laurat à propos de l'U.R.S.S., d'une part un parasitisme sans limites, d'autre part une anarchie qui, en dépit de tous les « plans », équivaut pour le moins à l'anarchie causée par la concurrence capitaliste. Quant aux rapports entre la production et la consommation, il serait vain d'espérer qu'une caste bureaucratique, qu'elle soit russe ou américaine, les rétablisse en subordonnant la première à la seconde. Tout groupe humain qui exerce une puissance l'exerce, non pas de manière à rendre heureux ceux qui y sont soumis, mais de manière à accroître cette puissance ; c'est là une question de vie ou de mort pour n'importe quelle domination. Tant que la production en est restée a un stade primitif, la question de la puissance s'est résolue par les armes. Les transformations économiques l'ont transportée sur le plan de la production elle-même ; c'est ainsi qu'est né le régime capitaliste. L'évolution du régime a, par la suite, rétabli la guerre comme moyen essentiel de lutte pour le pouvoir, mais sous une autre forme ; la supériorité dans la lutte militaire suppose, de nos jours, la supériorité dans la production elle-même. Si la production a pour fin, aux mains des capitalistes, le jeu de la concurrence, elle aurait nécessairement pour fin, aux mains des techniciens organisés en une bureaucratie d'État, la préparation à la guerre. Au reste, comme Rousseau l'avait déjà compris, aucun système d'oppression n'a intérêt au bien-être des opprimés ; c'est sur la misère que l'oppression peut peser le plus aisément de tout son poids.\par
Quant à l'atmosphère morale que peut amener un régime de dictature bureaucratique, on peut dès à présent se rendre compte de ce qu'elle peut être. Le capitalisme n'est qu'un système d'exploitation du travail productif ; si l'on excepte les tentatives d'émancipation du prolétariat, il a donné un libre essor, dans tous les domaines, a l'initiative, au libre examen, à l'invention, au génie. Au contraire, la machine bureaucratique, qui exclut tout jugement et tout génie, tend, par sa structure même, à la totalité des pouvoirs. Elle menace donc l'existence même de tout ce qui est encore précieux pour nous dans le régime bourgeois., Au lieu du choc des opinions contraires, on aurait, sur toutes choses, une opinion officielle dont nul ne pourrait, s'écarter ; au lieu du cynisme propre au système capitaliste, qui dissout tous les liens d'homme à homme pour les remplacer par de purs rapports d'intérêts, un fanatisme soigneusement cultivé, propre à faire de la misère, aux yeux des masses, non plus un fardeau passivement supporté, mais un sacrifice librement consenti ; un mélange de dévouement mystique et de bestialité sans frein ; une religion de l'État qui étoufferait toutes les valeurs individuelles, c'est-à-dire toutes les valeurs vraies. Le système capitaliste et même le régime féodal, qui, par le désordre qu'il comportait, permettait çà et là a des individus et à des collectivités de se développer d'une manière indépendante, sans parler de ce bienheureux régime grec où les esclaves étaient du moins employés à nourrir des hommes libres, toutes ces formes d'oppression apparaissent comme des formes de vie libre et heureuse auprès d'un système qui anéantirait méthodiquement toute initiative, toute culture, toute pensée.\par
Sommes-nous réellement menaces d'être soumis à un tel régime ? Nous en sommes peut-être plus que menacés ; il semble que nous le voyions se développer sous nos yeux. La guerre, qui se continue sous forme de préparation à la guerre, a donné une fois pour toutes à l'appareil d'État un rôle important dans la production. Bien que, même en pleine lutte, les intérêts des capitalistes aient souvent passé avant l'intérêt de la défense nationale, comme le montre l'exemple de Briey, la préparation systématique à la guerre suppose pour chaque État une certaine réglementation de l'économie, une certaine tendance vers l'indépendance économique. D'autre part, dans tous les domaines, la bureaucratie s'est, depuis la guerre, monstrueusement développée. Certes la bureaucratie ne s'est pas encore constituée en un système d'oppression ; si elle s'est infiltrée partout, elle demeure cependant diffuse, dispersée en une foule d'appareils que le jeu même du régime capitaliste empêche de se cristalliser autour d'un noyau central, Fried, le principal théoricien de la revue {\itshape Die Tat}, disait en 1930 : « Nous sommes pratiquement sous la domination de la bureaucratie syndicale, de la bureaucratie industrielle et de la bureaucratie d'État, et ces trois bureaucraties se ressemblent tant qu'on pourrait mettre l'une à la place de l'autre. » Or, sous l'influence de la crise, ces trois bureaucraties tendent à se fondre en un appareil unique. C'est ce qu'on voit en Amérique, où Roosevelt, sous l'influence d'une pléiade de techniciens, essaie de fixer les prix et les salaires, en accord avec les unions d'industriels et d'ouvriers. C'est ce qu'on voit en Allemagne, où, avec une rapidité. foudroyante, l'appareil d'État s'est annexé l'appareil syndical et tend à mettre la main sur l'économie. Quant à la Russie, il y a longtemps que les trois bureaucraties de l'État, des entreprises et des organisations ouvrières n'y forment plus qu'un seul et même appareil.\par
La question des perspectives se pose ainsi de deux manières ; d'une part, pour la Russie où les masses travailleuses ont exproprié propriétaires et capitalistes, il s'agit de savoir si la bureaucratie peut effacer, sans guerre civile, jusqu'aux traces des conquêtes d'Octobre. Il semble bien que les faits nous contraignent, malgré Trotsky, à répondre par l'affirmative. Quant aux autres pays, il faut examiner si le capitalisme proprement dit peut y périr sans une semblable expropriation, par une simple transformation du sens de la propriété. Sur ce point, les faits sont beaucoup moins clairs. Certes l'on peut dire que dès maintenant le régime capitaliste n'existe plus a proprement parler. Il n'y a plus à proprement parler de marché du travail. La réglementation du salaire et de l'embauche, le service du travail semblent autant d'étapes dans la transformation du salariat en une forme d'exploitation nouvelle. Il semble aussi qu'en Allemagne les commissaires installés par Hitler dans les trusts et les grandes entreprises, exercent réellement -un pouvoir dictatorial. L'abandon systématique de la monnaie or dans le inonde est aussi un phénomène important. D'autre part il faut tenir compte de faits tels que la « clôture de la révolution nationale » en Allemagne et la constitution d'un conseil supérieur de l'économie qui comprend tous les magnats. Cependant le mouvement national-socialiste est loin d'avoir dit son dernier Mot. Les capitulations successives de la bourgeoisie devant ce mouvement montrent assez quel est le rapport des forces. La séparation de la propriété et de l'entreprise, qui a transformé la plupart des propriétaires de capital en simples parasites, permet des mots d'ordre tels que « la lutte contre l'esclavage de l'intérêt », qui sont anticapitalistes sans être prolétariens. Quant aux grands magnats du capital industriel et financier, leur participation à la dictature économique de l'État n'exclut pas nécessairement la suppression du rôle qu'ils ont joué jusqu'ici dans l'économie. Enfin, si les phénomènes politiques peuvent être considérés comme des signes de l'évolution économique, on ne peut négliger le fait que tous les courants politiques qui touchent les masses, qu'ils s'intitulent fascistes, socialistes ou communistes, tendent à la même forme de capitalisme d'État. Seuls s'opposent à ce grand courant quelques défenseurs du libéralisme économique, de plus en plus timides et de moins en moins écoutés. Bien rares sont ceux de nos camarades qui se souviennent qu'on pourrait y opposer aussi la démocratie ouvrière. En présence de tous ces faits, et de bien d'autres, nous sommes contraints de nous demander nettement vers quel régime nous mènera la crise actuelle, si elle se prolonge, ou, en cas d'un retour rapide de la bonne conjoncture, les crises ultérieures.\par
Devant une semblable évolution, la pire déchéance serait d'oublier nous-mêmes le but que nous poursuivons. Cette déchéance a déjà atteint plus ou moins gravement un grand nombre de nos camarades, et elle nous menace tous. N'oublions pas que nous voulons faire de l'individu et non de la collectivité la suprême valeur. Nous voulons faire des hommes complets en supprimant cette spécialisation qui nous mutile tous. Nous voulons donner au travail manuel la dignité a laquelle il a droit, en donnant à l'ouvrier la pleine intelligence de la technique au lieu d'un simple dressage ; et donner à l'intelligence son objet propre, en la mettant en contact avec le monde par le moyen du travail. Nous voulons mettre en pleine lumière les rapports véritables de l'homme et de la nature, ces rapports que déguise, dans toute société fondée sur l'exploitation, « la dégradante division du travail en travail intellectuel et travail manuel ». Nous voulons rendre à l'homme, c'est-à-dire à l'individu, la domination qu'il a pour fonction propre d'exercer sur la nature, sur les outils, sur la société elle-même ; rétablir la subordination des conditions matérielles du travail par rapport aux travailleurs ; et au lieu de supprimer la propriété individuelle, « faire de la propriété individuelle une vérité, en transformant les moyens de production... qui servent aujourd'hui surtout à asservir et exploiter le travail, en de simples instruments du travail libre et associé ».\par
C'est là la tâche propre de notre génération. Depuis plusieurs siècles, depuis la Renaissance, les hommes de pensée et d'action travaillent méthodiquement à rendre l'esprit humain maître des forces de la nature ; et le succès a dépassé les espérances. Mais au cours du siècle dernier l'on a compris que la société elle-même est une force de la nature, aussi aveugle que les autres, aussi dangereuse pour l'homme s'il ne parvient pas à la maîtriser. Actuellement cette force pèse sur nous plus cruellement que l'eau, la terre, l'air et le feu ; d'autant qu'elle a elle-même entre ses mains, par les progrès de la technique, le maniement de l'eau, de la terre, de l'air et du feu. L'individu s'est trouvé brutalement dépossédé des moyens de combat et de travail ; ni la guerre ni la production ne sont plus possibles sans une subordination totale de l'individu à l'outillage collectif. Or le mécanisme social, par son fonctionnement aveugle, est en train, comme le montre tout ce qui arrive depuis août 1914, de détruire toutes les conditions du bien-être matériel et moral de l'individu, toutes les conditions du développement intellectuel et de la culture. Maîtriser ce mécanisme est pour nous une question de vie ou de mort ; et le maîtriser, c'est le soumettre à l'esprit humain, c'est-à-dire à l'individu. La subordination de la société à l'individu, c'est la définition de la démocratie véritable, et c'est aussi celle du socialisme. Mais comment maîtriser cette puissance aveugle, alors qu'elle possède, comme Marx l'a montré en des formules saisissantes, toutes les forces intellectuelles et matérielles cristallisées en un monstrueux outillage ? Nous chercherions en vain dans la littérature marxiste une réponse à cette question.\par
Faut-il donc désespérer. ? Certes les raisons ne manqueraient pas. L'on voit mal où l'on pourrait placer son espérance. La capacité de juger librement se fait de plus en plus rare, en particulier dans les milieux intellectuels, par cette spécialisation qui force chacun, dans les questions fondamentales que pose chaque recherche théorique, à croire sans savoir. Ainsi, même dans le domaine de la théorie pure, le jugement individuel se trouve découronne devant les résultats acquis par l'effort collectif. Quant à la classe ouvrière, sa situation d'instrument passif de la production ne la prépare guère à prendre ses propres destinées en mains. Les générations actuelles ont été d'abord décimées et démoralisées par la guerre ; puis la paix et la prospérité, une fois revenues, ont amené d'une part un luxe et une fièvre de spéculation qui ont profondément corrompu toutes les couches de la population, d'autre part des modifications techniques qui ont enlevé à la classe ouvrière sa force principale. Car l'espoir du mouvement révolutionnaire reposait sur les ouvriers qualifiés, seuls à unir, dans le travail industriel, la réflexion et l'exécution, à prendre une part active et essentielle dans la marche de l'entreprise, seuls capables de se sentir prêts à assumer un jour la responsabilité de toute la vie économique et politique. En fait, ils formaient le noyau le plus solide des organisations révolutionnaires. Or la rationalisation a supprimé leur fonction et n'a guère laissé subsister que des manoeuvres spécialisés, complètement asservis à la machine. Ensuite est venu le chômage, qui s'est abattu sur la classe ouvrière ainsi mutilée sans provoquer de réaction. S'il a exterminé moins d'hommes que la guerre, il a produit un abattement autrement profond, en réduisant de larges masses ouvrières, et en particulier toute la jeunesse, à une situation de parasite qui, à force de se prolonger, a fini par sembler définitive à ceux qui la subissent. Les ouvriers qui sont demeurés dans les entreprises ont fini par considérer eux-mêmes le travail qu'ils accomplissent non plus comme une activité indispensable à la production, niais comme une faveur accordée par l'entreprise. Ainsi le chômage, là où il est le plus étendu, en arrive à réduire le prolétariat tout entier à un état d'esprit de parasite. Certes la prospérité peut revenir, mais aucune prospérité ne peut sauver les générations qui ont passé leur adolescence et leur jeunesse dans une oisiveté plus exténuante que le travail, ni préserver les générations suivantes d'une nouvelle crise ou d'une nouvelle guerre. Les organisations peuvent-elles donner au prolétariat la force qui lui manque ? La complexité même du régime capitaliste, et par suite des problèmes que pose la lutte à mener contre lui, transporte dans le sein même du mouvement ouvrier « la dégradante division du travail en travail manuel et travail intellectuel ». La lutte spontanée s'est toujours révélée impuissante, et l'action organisée sécrète en quelque sorte automatiquement un appareil de direction qui, tôt ou tard, devient oppressif. De nos jours cette oppression s'effectue sous la forme d'une liaison organique soit avec l'appareil d'État national, soit avec l'appareil d'État russe. Et ainsi nos efforts risquent, non seulement de rester vains, mais encore de se tourner contre nous, au profit de notre ennemi capital, le fascisme. Le travail d'agitation, en exaspérant la révolte, peut favoriser la démagogie fasciste, comme le montre l'exemple du parti communiste allemand. Le travail d'organisation, en développant la bureaucratie, peut favoriser également l'avènement du fascisme, comme le montre l'exemple de la social-démocratie. Les militants ne peuvent pas remplacer la classe ouvrière. L'émancipation des travailleurs sera l'œuvre des travailleurs eux-mêmes, ou elle ne sera pas. Or le fait le plus tragique de l'époque actuelle, c'est que la crise atteint le prolétariat plus profondément que la classe capitaliste, de sorte qu'elle apparaît comme n'étant pas simplement la crise d'un régime, mais de notre société elle-même.\par
Ces vues seront sans doute taxées de défaitisme, même par des camarades qui cherchent à voir clair. Il est douteux cependant que nous ayons avantage à employer dans nos rangs le vocabulaire de l'État-major. Le terme même de découragement ne saurait avoir de sens parmi nous. La seule question qui se pose est de savoir si nous devons ou non continuer à lutter ; dans le premier cas, nous lutterons avec autant d'ardeur que si la victoire était sûre. Il n'y a aucune difficulté, une fois qu'on a décidé d'agir, à garder intacte, sur le plan de l'action, l'espérance même qu'un examen cri-tique a montré être presque sans fondement ; c'est là l'essence même du courage. Or, étant donne qu'une défaite risquerait d'anéantir, pour une période indéfinie, tout ce qui fait a nos yeux la valeur de la vie humaine, il est clair que nous devons lutter par tous les moyens qui nous semblent avoir une chance quelconque d'être efficaces. Un homme que l'on jetterait à la mer en plein océan ne devrait pas se laisser couler, malgré le peu de chances qu'il aurait de trouver le salut, mais nager jusqu'à l'épuisement. Et nous ne sommes pas véritablement sans espoir. Le seul fait que nous existons, que nous concevons et voulons autre chose que ce qui existe, constitue pour nous une raison d'espérer. La classe ouvrière contient encore, dispersés çà et là, en grande partie hors des organisations, des ouvriers d'élite, animés de cette force d'âme et d'esprit que l'on ne trouve que dans le prolétariat, prêts, le cas échéant, à se consacrer tout entiers, avec la résolution et la conscience qu'un bon ouvrier met dans son travail, à l'édification d'une société raisonnable. Dans des circonstances favorables, un mouvement spontané des masses peut les porter au premier plan de la scène de l'histoire. En attendant, l'on ne peut que les aider à se former, à réfléchir, à prendre de l'influence dans les organisations ouvrières restées encore vivantes, c'est-à-dire, pour la France, dans les syndicats, enfin à se grouper pour mener, dans la rue ou dans les entreprises, les actions qui sont encore possibles malgré l'inertie actuelle des masses. Un effort tendant à grouper tout ce qui est resté sain au cœur même des entreprises, en évitant aussi bien l'excitation des sentiments élémentaires de révolte que la cristallisation d'un appareil, ce n'est pas encore grand'chose, mais il n'y a pas autre chose. Le seul espoir du socialisme réside dans ceux qui, dès à présent, ont réalisé en eux-mêmes, autant qu'il est possible dans la société d'aujourd'hui, cette union du travail manuel et du travail intellectuel qui définit la société que nous nous proposons.\par
Mais, à côté de cette tache, l'extrême faiblesse des armes dont nous disposons nous oblige à en entreprendre une autre. Si, comme ce n'est que trop possible, nous devons périr, faisons en sorte que nous ne périssions pas sans avoir existé. Les forces redoutables que nous avons à combattre s'apprêtent à nous écraser ; et certes elles peuvent nous empêcher d'exister pleinement, c'est-à-dire d'imprimer au monde la marque de notre volonté. Mais il est un domaine où elles Sont impuissantes. Elles ne peuvent nous empêcher de travailler à concevoir clairement l'objet de nos efforts, afin que, si nous ne pouvons accomplir ce que nous voulons, nous l'ayons du moins voulu, et non pas désiré aveuglément ; et d'autre part notre faiblesse peut à la vérité nous empêcher de vaincre, mais non pas de comprendre la force qui nous écrase. Rien au monde ne peut nous interdire d'être lucides. Il n'y a aucune contradiction entre cette tâche d'éclaircissement théorique et les tâches que pose la lutte effective ; il y a corrélation au contraire, puisqu'on ne peut agir sans savoir ce que Pori veut, et quels obstacles on a à vaincre. Néanmoins, le temps dont nous disposons étant de toutes manières limité, l'on est forcé de le répartir entre la réflexion et l'action, ou, pour parler plus modestement, la préparation à l'action. Cette répartition ne peut être déterminée par aucune règle, mais seulement par le tempérament, la tournure d'esprit, les dons naturels de chacun, les conjectures que chacun forme concernant l'avenir, le hasard des circonstances. En tout cas le plus grand malheur pour nous serait de périr impuissants à la fois à réussir et à comprendre.\par


\signed{{\itshape (Révolution Prolétarienne}, no 158, 25 août 1933.)}
\section[Réflexions concernant la technocratie, le national-socialisme, l’U.R.S.S. et quelques autres points]{Réflexions concernant la technocratie, le national-socialisme, l’U.R.S.S. et quelques autres points}\renewcommand{\leftmark}{Réflexions concernant la technocratie, le national-socialisme, l’U.R.S.S. et quelques autres points}

\noindent \par
Ce ne sont ici que quelques idées, peut-être hasardées, certainement hérétiques par rapport à toutes les orthodoxies, destinées avant tout à faire réfléchir les militants.\par

\begin{center}
* * *\end{center}
\noindent Nous vivons sur une doctrine élaborée par un grand homme certes, mais un grand homme mort il y a cinquante ans. Il a créé une méthode ; il l'a appliquée aux phénomènes de son temps ; il ne pouvait l'appliquer aux phénomènes du nôtre.\par
Les militants d'avant-guerre ont senti la nécessité d'appliquer la méthode marxiste à la forme nouvelle qu'avait prise le capitalisme de leur temps. La mince brochure de Lénine concernant l'impérialisme témoigne d'un tel souci, pour lequel les préoccupations quotidiennes des militants laissaient malheureusement peu de loisir.\par
Quant à nous, Marx représente pour nous, dans le meilleur des cas, une doctrine ; bien plus souvent un simple nom, que l'on jette à la tête de l'adversaire pour le pulvériser ; presque jamais une méthode. Le marxisme ne peut cependant rester vivant qu'à titre de méthode d'analyse, dont chaque génération se sert pour définir les phénomènes essentiels de sa propre époque. Or il semble que nos corps vivent seuls dans cette période prodigieusement nouvelle, qui dément toutes les prévisions antérieures ; et que nos esprits continuent à se mouvoir, sinon au temps de la première Internationale, du moins au temps d'avant-guerre, à l'époque de la C.G.T. révolutionnaire et du parti bolchévik russe. Nul n'essaie de définir la période actuelle. Trotsky a bien dit et même répété à maintes reprises que, depuis 1914, le capitalisme est entré dans une nouvelle période, celle de son déclin ; mais il n'a jamais eu le temps de dire ce qu'il entend par là au juste, ni sur quoi il se fonde. On ne saurait le lui reprocher, mais cela ôte toute valeur à sa formule. Et personne, que je sache, n'est allé plus loin.\par
Celui qui admet la formule de Lénine : « Sans théorie révolutionnaire pas de mouvement révolutionnaire » est forcé d'admettre aussi qu'il n'y a à peu près pas de mouvement révolutionnaire en ce moment.\par

\begin{center}
* * *\end{center}
\noindent Il y a un peu plus de deux ans paraissait en Allemagne un livre qui a fait un assez grand bruit, intitulé {\itshape La Fin du Capitalisme ;} l'auteur, Ferdinand Fried, appartenait à cette célèbre revue, {\itshape Die Tat}, qui a longtemps préconisé un capitalisme d'État, une économie dirigée et fermée, avec une dictature appuyée à la fois sur les organisations syndicales et sur le mouvement national-socialiste. Les révolutionnaires n'ont guère porté attention à l'ouvrage de Fried, et l'ont jugé médiocre ; c'est qu'ils ont eu le tort d'y chercher un système cohérent ; et la valeur du livre, considéré comme un simple document, leur a échappé. L'idée essentielle du livre, c'est celle du pouvoir de la Bureaucratie. Ce ne sont plus les possesseurs du capital, les propriétaires de l'outillage qui dirigent l'entreprise ; grâce aux actions, ces propriétaires sont fort nombreux, et les quelques gros actionnaires qui les dirigent se préoccupent surtout d'opérations financières. Ceux qui conduisent l'entreprise elle-même, administrateurs, ingénieurs, techniciens de toute espèce, ce sont, à part quelques exceptions, non des propriétaires, mais des salariés ; c'est une bureaucratie. Parallèlement le pouvoir d'État, dans tous les pays, s'est concentré de plus en plus entre les mains d'un appareil bureaucratique. Enfin le mouvement ouvrier est au pouvoir d'une bureaucratie syndicale. « Aujourd'hui nous sommes pratiquement sous la domination de la bureaucratie syndicale, de la bureaucratie industrielle et de la bureaucratie d'État, et ces trois bureaucraties se ressemblent tant que l'on pourrait mettre l'une à la place de l'autre. » La conclusion est qu'il faut organiser une économie fermée, dirigée par cette triple bureaucratie unie en un même appareil. C'est le programme même du fascisme, avec cette différence que le fascisme brise l'appareil syndical et crée des syndicats placés sous sa domination directe.\par
On a beaucoup parlé en Amérique, ces temps-ci, d'une théorie nouvelle qui avait nom « technocratie ». L'idée, comme le nom même l'indique, était celle d'une économie nouvelle, qui ne serait plus ballottée au hasard des concurrences, qui ne serait pas non plus, comme le veut le socialisme, aux mains des ouvriers, mais qui serait dirigée par les techniciens, investis d'une sorte de dictature. Les modalités de cette économie nouvelle, la méthode de répartition, la monnaie fondée sur « l'unité d'énergie », ce ne sont la que des détails.\par
L'essentiel était cette idée, qui a, nous dit-on, préoccupé pendant quelque temps tous les Américains, de substituer à la classe capitaliste une autre classe dirigeante, qui n'aurait été autre que cette bureaucratie industrielle signalée par Fried.\par
Ces courants de pensée absolument nouveaux, propres à l'après-guerre, et qui se sont développés avec la crise actuelle, doivent nous porter a examiner ce qu'est devenu, de nos jours, le procès de la production industrielle. Et nous devons reconnaître que les deux catégories économiques établies par Marx, capitalistes et prolétariat, ne suffisent plus à saisir la forme de la production. Les capitalistes se sont de plus en plus détachés de la production elle-même, pour se consacrer à la guerre économique. Le premier roi du pétrole, Rockefeller, a conquis sa suprématie par une trouvaille d'ordre industriel, les pipe-lines ; le second, Deterding, n'a été le concurrent heureux de Rockefeller que grâce à des coups de bourse et à des manoeuvres financières. Cette succession est symbolique.\par

\begin{center}
* * *\end{center}
\noindent Caste ou classe, la bureaucratie est un facteur nouveau dans la lutte sociale. Elle a transformé, en U.R.S.S., la dictature du prolétariat en une dictature exercée par elle-même, et dirige depuis lors les ouvriers révolutionnaires du monde entier. En Allemagne au contraire elle s'est alliée au capital financier pour l'extermination des meilleurs ouvriers. On peut dire que dans aucun des deux cas elle n'a joué un rôle indépendant ; mais, tant que la féodalité a duré, la bourgeoisie aussi a dû s'allier avec les classes opprimées contre elle, ou avec elle contre les classes opprimées. Ce qui est grave, c'est que nulle part les ouvriers ne sont organisés d'une manière indépendante. Les communistes obéissent à cette bureaucratie russe, aussi incapable à présent de jouer un rôle progressif dans le reste du monde que la bourgeoisie française après Thermidor, quand elle eut écrasé ces sans-culottes sur lesquels elle s'était appuyée. Les ouvriers réformistes sont aux mains de cette bureaucratie syndicale qui ressemble à la bureaucratie industrielle et à la bureaucratie d'État comme une goutte d'eau à deux autres, et s'agglutine mécaniquement à l'appareil d'État. Les anarchistes n'échappent à l'emprise de la bureaucratie que parce qu'ils ignorent l'action méthodiquement organisée. En face de cette situation, les polémiques des communistes oppositionnels, des syndicalistes révolutionnaires, etc., semblent pour le moins manquer singulièrement d'actualité.\par
Les communistes accusent les social-démocrates d'être les « fourriers du fascisme », et ils ont cent fois raison. Ils se vantent d'être, eux, un parti capable de lutter efficacement contre le fascisme, et ils ont malheureusement tort. Devant la menace fasciste, une question se pose aux militants. Est-il possible d'organiser les ouvriers d'un pays quelconque sans que cette organisation sécrète pour ainsi dire une bureaucratie qui subordonne aussitôt l'organisation a un appareil d'État, soit celui du pays lui-même, soit celui de l'U.R.S.S. ?\par
La sinistre comédie que jouent depuis déjà tant de mois, aux dépens du prolétariat allemand, la social-démocratie et l'Internationale Communiste \footnote{Les communistes les plus fanatiques devraient ouvrir les yeux devant l'appel lancé le 5 mars par l'Internationale Communiste. Depuis des mois et des mois, les oppositionnels sont injuriés parce qu'ils proclament l'urgence de propositions de front unique au sommet. Au début de février, le parti communiste allemand repousse fièrement, sans même offrir de négocier, le « pacte de non-agression » offert par la social-démocratie. Le 19 février, l'Internationale Socialiste propose le front unique sans conditions, et n'obtient d'autre réponse que le discours de Thorez au Comité Central contre tout front unique au sommet, contre toute suspension des attaques à l'égard de la social-démocratie. Survient l'incendie du Reichstag, l'arrestation de milliers de militants, la terreur qui rend illégaux aussi bien social-démocrates que communistes, qui pousse les chefs social-démocrates, affolés, dans les bras de Hitler (cf. la lettre de Well), qui rend tout travail de propagande et d'organisation presque impossible. Et alors, alors seulement, l'Internationale Communiste, le 5 mars, accepte, non seulement la proposition du 19 février, mais même le « pacte de non-agression » ! Ainsi aucun principe ne s'opposait à cette tactique ? Mais alors qu'est-ce qui empêchait de l'adopter dès février, ou même dès janvier, ou même auparavant, quand le prolétariat allemand pouvait encore prendre l'offensive et lutter avec des chances sérieuses de succès ? Ce retard n'est-il pas une trahison ?} montre que la question est urgente, et peut-être la seule qui importe présentement.\par

\begin{center}
\end{center}
\section[Sur le livre de Lénine « Matérialisme et empiriocriticisme »]{Sur le livre de Lénine « Matérialisme et empiriocriticisme »}\renewcommand{\leftmark}{Sur le livre de Lénine « Matérialisme et empiriocriticisme »}

\noindent \par
Cet ouvrage, le seul qu'ait publié Lénine sur des questions de pure philosophie, est dirigé contre Mach et contre les disciples, avoués ou non, qu'il avait en 1908 dans les rangs de la social-démocratie, et surtout de la social-démocratie russe ; le plus connu était Bogdanov. Lénine y examine en détail les doctrines de ses adversaires, doctrines qui tentaient toutes, avec plus ou moins de raffinements, de résoudre le problème de la connaissance en supprimant la notion d'un objet extérieur à la pensée ; il montre qu'elles se ramènent au fond, une fois dépouillées de leur phraséologie prétentieuse, à l'idéalisme de Berkeley, c'est-à-dire a la négation du monde extérieur ; il leur oppose le matérialisme de Marx et d'Engels. Dans cette polémique, qui l'écartait de ses préoccupations habituelles, Lénine a manifesté une fois de plus sa puissance de travail, son goût de la documentation sérieuse. L'intérêt de la discussion est facile à comprendre : on ne peut se réclamer du « socialisme scientifique » si l'on n'a pas une notion nette de ce qu'est la science, si par suite on n'a pas pose en termes clairs le problème de la connaissance, des rapports entre la pensée et son objet. Cependant l'ouvrage de Lénine est presque aussi ennuyeux et même presque aussi peu instructif que n'importe quel manuel de philosophie. Cela tient en partie à la médiocrité des adversaires auxquels Lénine s'attaque, mais surtout a la méthode même de Lénine.\par
Lénine a étudié la philosophie d'abord en 1899, étant en Sibérie, puis en 1908, lorsqu'il préparait le livre en question pour un but bien déterminé, à savoir pour réfuter les théoriciens du mouvement ouvrier qui voulaient s'écarter du matérialisme d’Engels. C'est une méthode bien caractéristique que celle qui consiste à réfléchir pour réfuter, la solution étant donnée avant la recherche. Et par quoi pouvait, donc être donnée cette solution ? Par le Parti, comme elle est donnée, pour le catholique, par l'Église. Car « la théorie de la connaissance, tout comme l'économie politique, est, dans notre société contemporaine, une science de Parti ». À vrai dire on ne peut nier qu'il n'y ait un rapport étroit entre la culture théorique et la division de la société en classes. Toute société oppressive donne naissance à une conception fausse des rapports de l'homme et de la nature, du seul fait que seuls y sont en contact direct avec la nature les exploités, c'est-à-dire ceux qui sont exclus de la culture théorique, privés du droit et de la possibilité de s'exprimer ; et inversement la conception fausse ainsi. formée tend à faire durer l'oppression, dans la mesure ou elle fait apparaître comme légitime cette séparation de la pensée et du travail. En ce sens on peut dire de tel système philosophique, de telle conception de la science qu'ils sont réactionnaires ou bourgeois. Mais ce n'est pas ainsi que semble l'entendre Lénine. Il ne dit pas : telle conception déforme le véritable rapport de l'homme avec le monde, donc elle est réactionnaire ; mais : telle conception s'écarte du matérialisme, mène a l'idéalisme, donne des arguments à la religion, elle est réactionnaire, donc fausse. Il ne s'agit pas du tout pour lui de voir clair dans sa propre pensée, mais uniquement de maintenir intactes les traditions philosophiques sur lesquelles vivait le Parti. Une telle méthode de pensée n'est pas celle d'un homme libre. Comment pourtant Lénine aurait-il pu réfléchir autrement ? Du moment qu'un parti se trouve cimenté non seulement par la coordination des actions, mais aussi par l'unité de-la doctrine, il devient impossible à un bon militant de penser autrement qu'en esclave. Il est facile dès lors de se représenter comment peut se conduire un tel parti, une fois au pouvoir. Le régime étouffant qui pèse en ce moment sur le peuple russe était déjà impliqué en germe dans l'attitude de Lénine vis-à-vis de sa propre pensée. Longtemps avant de ravir la liberté de pensée à la Russie tout entière, le parti bolchévik l'avait déjà enlevée a son propre chef.\par
Marx, heureusement, s'y prenait autrement pour réfléchir. Malgré bien des polémiques qui n'ajoutent rien à sa gloire, il cherchait plutôt à mettre de l'ordre dans sa propre pensée qu'à réduire en poudre ses adversaires ; et il avait appris de Hegel qu'au lieu de réfuter les conceptions incomplètes il vaut mieux les « surmonter en les conservant ». Aussi la pensée de Marx diffère-t-elle sensiblement de celle des marxistes, sans en excepter Engels, et nulle part autant que dans la solution du problème dont s'occupe ici Lénine, à savoir le problème de la connaissance et, plus généralement, des rapports de la pensée et du monde.\par
Pour expliquer comment il peut se faire que la pensée connaisse le monde, on peut ou représenter le monde comme une simple création de la pensée, ou représenter la pensée comme un des produits du monde, produit qui, par un hasard inexplicable, en constituerait aussi l'image ou le reflet. Lénine pose que toute philosophie doit se ramener, au fond, à l'une de ces deux conceptions, et opte, bien entendu, pour la seconde. Il cite la formule d'Engels selon laquelle la pensée et la conscience « sont des produits du cerveau humain, étant, en fin de compte, des produits de la nature » ; de sorte que « les produits du cerveau humain étant, en fin de compte, des produits de la nature, loin d'être en contradiction avec l'ensemble de la nature, y correspondent » ; et il répète à satiété que cette correspondance consiste en ce que les produits du cerveau humain sont, apparemment grâce à la Providence, les photographies, les images, les reflets de la nature. Comme si les Pensées d'un fou n'étaient pas, au même titre, des « produits de la nature » ! Or, les deux conceptions entre lesquelles Lénine veut nous contraindre a choisir procèdent toutes deux de la même méthode ; pour mieux résoudre le problème, elles en suppriment l'un des deux termes. L'une supprime le monde, objet de la connaissance, l'autre l'esprit, sujet de la connaissance ; toutes deux ôtent à la connaissance toute signification. Si l'on veut, non pas bâtir une théorie, mais se rendre compte de la condition où l'homme se trouve réellement Place, on ne se demandera pas comment il peut se faire que le monde soit connu, mais comment, en fait, l'homme connaît le monde ; et l'on devra reconnaître l'existence et d'un monde qui dépasse la pensée, et d'une pensée qui, loin de refléter passivement le monde, s'exerce sur lui a la fois pour le connaître et pour le transformer. C'est ainsi que pensait Descartes, dont il est significatif que Lénine, dans ce livre, ne mentionne même pas le nom ; c'est ainsi également, on ne peut en douter, que pensait Marx.\par
On objectera sans doute que Marx ne s'est jamais dit en désaccord avec la, doctrine exposée par Engels dans ses ouvrages philosophiques, qu'il a lu {\itshape l'Anti-Dühring en} manuscrit et l'a approuve ; mais cela signifie seulement que Marx n'a jamais pris le temps de réfléchir a ces problèmes assez pour prendre conscience de ce qui le séparait d'Engels. Toute l’œuvre de Marx est imprégnée d'un esprit incompatible avec le matérialisme grossier d'Engels et de Lénine. Jamais il ne considère l'homme comme étant une simple partie de la nature, mais toujours comme étant aussi, du fait qu'il exerce une activité libre, un terme antagoniste vis-à-vis de la nature. Dans une étude sur Spinoza, il reproche expressément à celui-ci de confondre l'homme avec la nature qui le contient, au lieu de les opposer. Dans ses {\itshape Thèses sur Feuerbach}, il écrit : « Le défaut principal de toutes les doctrines matérialistes qui ont été formées jusqu'à ce jour, y compris celle de Feuerbach, consiste en ce que le réel, le sensible, ne sont conçus que sous la forme de l'objet, de la contemplation, et non comme activité humaine sensible, comme praxis, d'une manière subjective. C'est pourquoi le cote actif a été développé, d'une manière abstraite, il est vrai, en opposition avec le matérialisme, par l'idéalisme - qui, bien entendu, ne connaît pas l'activité réelle, sensible, comme telle. » Bien que ces formules soient obscures, elles disent du moins clairement qu'il s'agit de faire une synthèse de l'idéalisme et du matérialisme, synthèse ou soit sauvegardée une opposition radicale entre la nature passive et l'activité humaine. À vrai dire Marx refuse de concevoir une pensée pure qui s'exercerait hors de toute prise de contact avec la nature ; mais il n'y a rien de commun entre une doctrine qui fait de l'homme tout entier un simple produit de la nature, de la pensée un simple reflet, et une conception qui montre la réalité apparaissant au 'contact de la pensée et du monde, dans l'acte par lequel l'homme pensant prend possession du monde. C'est selon cette conception qu'il faut interpréter le matérialisme historique, qui signifie, comme Marx l'explique longuement dans son {\itshape Idéologie allemande, que} les pensées formées par les hommes dans des conditions techniques, économiques et sociales déterminées répondent à la manière dont ils agissent sur la nature en produisant leurs propres conditions d'existence. C'est de cette conception enfin qu'il faut tirer la notion même de la révolution prolétarienne ; car, l'essence même du régime capitaliste consiste, comme l'a montre Marx avec force, en un « renversement du rapport entre le sujet et l'objet », renversement constitué par la subordination du sujet à l'objet, du « travailleur aux conditions matérielles du travail » ; et la révolution ne peut avoir d'autre sens que de restituer au sujet pensant le rapport qu'il doit avoir avec la matière, en lui rendant la domination qu'il a pour fonction d'exercer sur elle.\par
Il n'est nullement surprenant que le parti bolchévik, dont l'organisation même a toujours reposé sur la subordination de l'individu, et qui, une fois au pouvoir, devait asservir le travailleur à la machine tout autant que le capitalisme, ait adopté pour doctrine le matérialisme naïf d'Engels plutôt que la philosophie de Marx. Il n'est pas étonnant non plus que Lénine s'en soit tenu a une méthode purement polémique, et ait mieux aimé embarrasser ses adversaires dans toutes sortes de difficultés, plutôt que de montrer comment sa théorie matérialiste aurait évité des difficultés analogues. Une citation de {\itshape l'Anti-Dürhing} remplace pour lui toutes les analyses ; mais ce n'est pas en parlant avec mépris des « erreurs depuis longtemps réfutées de Kant » qu'il peut empêcher la {\itshape Critique de la Raison pure} de demeurer, malgré ses lacunes, bien autrement instructive que {\itshape l'Anti-Dühring} pour quiconque veut réfléchir sur le problème de la connaissance. Et l'on ne peut que rire lorsqu'on le voit, lui qui a constamment invoque le « matérialisme dialectique » comme une doctrine complète et susceptible de tout résoudre, avouer, dans un fragment concernant la dialectique, qu'on ne s'est occupe encore que de vulgariser la dialectique, et non d'en vérifier la justesse par l'histoire des sciences.\par
Un tel ouvrage est une marque bien affligeante de la carence du mouvement socialiste dans le domaine de la théorie pure. Et l'on ne peut s'en consoler en se disant que l'action sociale et politique importe plus que la philosophie ; la révolution doit être une révolution intellectuelle autant que sociale, et la spéculation purement théorique y a sa tâche, dont elle ne peut se dispenser sous peine de rendre tout le reste impossible. Tous les révolutionnaires authentiques ont compris que la révolution implique la diffusion des connaissances dans la population tout entière. Il y a la-dessus accord complet entre Blanqui, qui juge le communisme impossible avant qu'on n'ait partout répandu « les lumières », Bakounine, qui voulait voir la science, selon son admirable formule, « ne faire qu'un avec la vie réelle et immédiate de tous les individus », et Marx, pour qui le socialisme devait être avant tout l'abolition de la « dégradante division du travail en travail intellectuel et travail manuel ». Cependant l'on ne semble pas avoir compris quelles sont les conditions d'une telle transformation. Envoyer tous les citoyens au lycée et à l'université jusqu'à dix-huit ou vingt ans serait un remède faible, ou pour mieux dire nul, à l'état de choses dont nous souffrons. S'il s'agissait simplement de vulgariser la science telle que nos savants nous l'ont faite, ce serait chose facile ; mais de la science actuelle on ne peut rien vulgariser, si ce n'est les résultats, obligeant ainsi ceux que l'on a l'illusion d'instruire à croire sans savoir. Quant aux méthodes, qui constituent l'âme même de la science, elles sont par leur essence même impénétrables aux profanes, et par suite aussi aux savants eux-mêmes, dont la spécialisation fait toujours des profanes en dehors du domaine très restreint qui leur est propre. Ainsi, comme le travailleur, dans la production moderne, doit se subordonner aux conditions matérielles du travail, de même la pensée, dans l'investigation scientifique, doit de nos jours se subordonner aux résultats acquis de la science ; et la science, qui devait faire clairement comprendre toutes choses et dissiper tous les mystères, est devenue elle-même le mystère par excellence, au point que l'obscurité, voire même l'absurdité, apparaissent aujourd'hui, dans une théorie scientifique, comme un signe de profondeur. La science est devenue la forme la plus moderne de la conscience de l'homme qui ne s'est pas encore retrouvé ou s'est de nouveau perdu, selon la belle formule de Marx concernant la religion. Et sans doute la science actuelle est-elle bien propre à servir de théologie à notre société de plus en plus bureaucratique, s'il est vrai, comme l'écrivait Marx dans sa jeunesse, que « l'âme universelle de la bureaucratie est le secret, le mystère, à l'intérieur d'elle-même par la hiérarchie, vis-à-vis de l'extérieur par son caractère de corps fermé ». Plus généralement tout privilège, et par suite toute oppression, a pour condition l'existence d'un savoir essentiellement impénétrable aux masses travailleuses qui se trouvent ainsi obligées de croire comme elles sont contraintes d'obéir. La religion, de nos jours, ne suffit pas à remplir ce rôle, et la science lui a succédé. Aussi la belle formule de Marx concernant la critique de la religion comme condition de toute critique doit-elle être étendue aussi à la science moderne. Le socialisme ne sera même pas concevable tant que la science n'aura pas été dépouillée de son mystère.\par
Descartes avait cru autrefois avoir fondé une science sans mystère, c'est-à-dire une science où il y aurait assez d'unité et de simplicité dans la méthode pour que les parties les plus compliquées soient seulement plus longues et non pas plus difficiles à comprendre que les parties les plus simples ; où chacun pourrait par suite comprendre comment ont été trouvés les résultats mêmes auxquels il n'a pas eu le temps de parvenir ; ou chaque résultat serait donné avec la méthode qui a conduit à le découvrir, de manière que chaque écolier ait le sentiment d'inventer à nouveau la science. Le même Descartes avait formé le projet d'une École des Arts et Métiers où chaque artisan apprendrait à se rendre pleinement compte des fondements théoriques de son propre métier ; il se montrait ainsi plus socialiste, sur le terrain de la culture, que n'ont été tous les disciples de Marx. Cependant il n'a accompli ce qu'il voulait que dans une très faible mesure, et s'est même trahi lui-même, par vanité, en publiant une Géométrie volontairement obscure. Après lui, il ne s'est guère trouvé de savants pour chercher à saper leurs propres privilèges de caste. Quant aux intellectuels du mouvement ouvrier, ils n'ont pas songé à s'attaquer à une tâche aussi indispensable ; tâche écrasante, il est vrai, qui implique une révision critique de la science tout entière, et surtout de la mathématique, où la quintessence du mystère s'est réfugiée ; mais tâche clairement posée par la notion même du socialisme, et dont l'accomplissement, indépendant des conditions extérieures et de la situation du mouvement ouvrier, dépend seulement de ceux qui oseront l'entreprendre ; au reste si importante qu'un pas fait dans cette vole serait plus utile peut-être a l'humanité et au prolétariat que bien des victoires partielles dans le domaine de l'action. Mais les théoriciens du mouvement socialiste, quand ils quittent le domaine de l'action pratique ou cette agitation vaine au milieu des tendances, fractions et sous-fractions qui leur donne l'illusion d'agir, ne songent nullement à saper les privilèges de la caste intellectuelle ; loin de là, ils élaborent une doctrine compliquée et mystérieuse qui sert de soutien à l'oppression bureaucratique au sein du mouvement ouvrier. En ce sens la philosophie est bien, comme le dit Lénine, une affaire de parti.\par
{\itshape (Critique sociale}, novembre 1933.)\par

\section[Réflexions sur les causes de la liberté et de l’oppression sociale]{Réflexions sur les causes de la liberté et de l’oppression sociale}\renewcommand{\leftmark}{Réflexions sur les causes de la liberté et de l’oppression sociale}

« \emph{En ce qui concerne les choses humaines, ne pas rire, ne pas pleurer ; ne pas s'indigner, mais comprendre.}{\citbibl SPINOZA.} »« \emph{L'être doué de raison peut faire de tout obstacle une matière de son travail, et en tirer parti.}{\citbibl MARC-AURÈLE.} »\noindent La période présente est de celles où tout ce qui semble normalement constituer une raison de vivre s'évanouit, où l'on doit, sous peine de sombrer dans le désarroi ou l'inconscience, tout remettre en question. Que le triomphe des mouvements autoritaires et nationalistes ruine un peu partout l'espoir que de braves gens avaient mis dans la démocratie et dans le pacifisme, ce n'est qu'une partie du mal dont nous souffrons ; il est bien plus profond et bien plus étendu. On peut se demander s'il existe un domaine de la vie publique ou privée où les sources mêmes de l'activité et de l'espérance ne soient pas empoisonnées par les conditions dans lesquelles nous vivons. Le travail ne s'accomplit plus avec la conscience orgueilleuse qu'on est utile, mais avec le sentiment humiliant et angoissant de posséder un privilège octroyé par une passagère faveur du sort, un privilège dont on exclut plusieurs être humains du fait même qu'on en jouit, bref une place. Les chefs d'entreprise eux-mêmes ont perdu cette naïve croyance en un progrès économique illimité qui leur faisait imaginer qu'ils avaient une mission. Le progrès technique semble avoir fait faillite, puisque au lieu du bien-être il n'a apporté aux masses que la misère physique et morale où nous les voyons se débattre ; au reste les innovations techniques ne sont plus admises nulle part, ou peu s'en faut, sauf dans les industries de guerre. Quant au progrès scientifique, on voit mal à quoi il peut être utile d'empiler encore des connaissances sur un amas déjà bien trop vaste pour pouvoir être embrassé par la pensée même des spécialistes ; et l'expérience montre que nos aïeux se sont trompés en croyant à la diffusion des lumières, puisqu'on ne peut divulguer aux masses qu'une misérable caricature de la culture scientifique moderne, caricature qui, loin de former leur jugement, les habitue à la crédulité. L'art lui-même subit le contrecoup du désarroi général, qui le prive en partie de son public, et par là même porte atteinte a l'inspiration. Enfin la vie familiale n'est plus qu'anxiété depuis que la société s'est fermée aux jeunes. La génération même pour qui l'attente fiévreuse de l'avenir est la vie tout entière végète, dans le monde entier, avec la conscience qu'elle n'a aucun avenir, qu'il n'y a point de place pour elle dans notre univers. Au reste ce mal, s'il est plus aigu pour les jeunes, est commun à toute l'humanité d'aujourd'hui. Nous vivons une époque privée d'avenir. L'attente de ce qui viendra n'est plus espérance, mais angoisse.\par
Il est cependant, depuis 1789, un mot magique qui contient en lui tous les avenirs imaginables, et n'est jamais si riche d'espoir que dans les situations désespérées ; c'est le mot de révolution. Aussi le prononce-t-on souvent depuis quelque temps. Nous devrions être, semble-t-il, en pleine période révolutionnaire ; mais en fait tout se passe comme si le mouvement révolutionnaire tombait en décadence avec le régime même qu'il aspire à détruire. Depuis plus d'un siècle, chaque génération de révolutionnaires a espéré tour à tour en une révolution prochaine ; aujourd'hui, cette espérance a perdu tout ce qui pouvait lui servir de support. Ni dans le régime issu de la révolution d'Octobre, ni dans les deux Internationales, ni dans les partis socialistes ou communistes indépendants, ni dans les syndicats, ni dans les organisations anarchistes, ni dans les petits groupement de jeunes qui ont surgi en si grand nombre depuis quelque temps, on ne peut trouver quoi que ce soit de vigoureux, de sain ou de pur ; voici longtemps que la classe ouvrière n'a donné aucun signe de cette spontanéité sur laquelle comptait Rosa Luxembourg, et qui d'ailleurs ne s'est jamais manifestée que pour être aussitôt noyée dans le sang ; les classes moyennes ne sont séduites par la révolution que quand elle est évoquée, à des fins démagogiques, par des apprentis dictateurs. On répète souvent que la situation est objectivement révolutionnaire, et que le « facteur subjectif » fait seul défaut ; comme si la carence totale de la force même qui pourrait seule transformer le régime n'était pas un caractère objectif de la situation actuelle, et dont il faut chercher les racines dans la structure de notre société ! C'est pourquoi le premier devoir que nous impose la période présente est d'avoir assez de courage intellectuel pour nous demander si le terme de révolution est autre chose qu'un mot, s'il a un contenu précis, s'il n'est pas simplement un des nombreux mensonges qu'a suscités le régime capitaliste dans son essor et que la crise actuelle nous rend le service de dissiper. Cette question semble impie, à cause de tous les êtres nobles et purs qui ont tout sacrifié, y compris leur vie, à ce mot. Mais seuls des prêtres peuvent prétendre mesurer la valeur d'une idée à la quantité de sang qu'elle a fait répandre. Qui sait si les révolutionnaires n'ont pas versé leur sang aussi vainement que ces Grecs et ces Troyens du poète qui, dupés par une fausse apparence, se, battirent dix ans autour de l'ombre d'Hélène ?\par
\subsection[CRITIQUE DU MARXISME.]{CRITIQUE DU MARXISME.}
\noindent Jusqu'à ces temps-ci, tous ceux qui ont éprouvé le besoin d'étayer leurs sentiments révolutionnaires par des conceptions précises ont trouvé ou cru trouver ces conceptions dans Marx. Il est entendu une fois pour toutes que Marx, grâce à sa théorie générale de l'histoire et à son analyse de la société bourgeoise, a démontré la nécessité inéluctable d'un bouleversement proche où l'oppression que nous fait subir le régime capitaliste serait abolie ; et même, à force d'en être persuadé, on se dispense en général d'examiner de plus près la démonstration. Le « socialisme scientifique » est passé à l'état de dogme, exactement comme ont fait tous les résultats obtenus par la science moderne, résultats auxquels chacun pense qu'il a le devoir de croire, sans jamais songer à s'enquérir de la méthode. En ce qui concerne Marx, si l'on cherche à s'assimiler véritablement sa démonstration, on s'aperçoit aussitôt qu'elle comporte beaucoup plus de difficultés que les propagandistes du « socialisme scientifique » ne le laissent supposer.\par
À vrai dire, Marx rend admirablement compte du mécanisme de l'oppression capitaliste ; mais il en rend si bien compte qu’on a peine à se, représenter comment ce mécanisme pourrait cesser de fonctionner. D'ordinaire, on ne retient de cette oppression que l'aspect économique, a savoir l'extorsion de la plus-value ; et si l'on s'en tient à ce point de vue, il est certes facile d'expliquer aux masses que cette extorsion est liée à la concurrence, elle-même liée à la propriété privée, et que le jour où la propriété deviendra collective tout ira bien. Cependant, même dans les limites de ce raisonnement simple en apparence, mille difficultés surgissent pour un examen attentif. Car Marx a bien montre que la véritable raison de l'exploitation des travailleurs, ce n'est pas le désir qu'auraient les capitalistes de jouir et de consommer, mais la nécessité d'agrandir l'entreprise le plus rapidement possible afin de la rendre plus puissante que ses concurrentes. Or ce n'est pas seulement l'entreprise, mais toute espèce de collectivité travailleuse, quelle qu'elle soit, qui a besoin de restreindre au maximum la consommation de ses membres pour consacrer le plus possible de temps à se forger des armes contre les collectivités rivales ; de sorte qu'aussi longtemps qu'il y aura, sur la surface du globe, une lutte pour la puissance, et aussi longtemps que le facteur décisif de la victoire sera la production industrielle, les ouvriers seront exploités. À vrai dire, Marx supposait précisément, sans le prouver d'ailleurs, que toute espèce de lutte pour la puissance disparaîtra le jour où le socialisme sera établi dans tous les pays industriels ; le seul malheur est que, comme Marx l'avait reconnu lui-même, la révolution ne peut se faire partout à la fois ; et lorsqu'elle se f ait dans un pays, elle ne supprime pas pour ce pays, mais accentue au contraire la nécessité d'exploiter et d'opprimer les masses travailleuses, de peur d'être plus faible que les autres nations. C'est ce dont l'histoire de la révolution russe constitue une illustration douloureuse.\par
Si l'on considère d'autres aspects de l'oppression capitaliste, il apparaît d'autres difficultés plus redoutables encore, ou, pour mieux dire, la même difficulté, éclairée d'un jour plus cru. La force que possède la bourgeoisie pour exploiter et opprimer les ouvriers réside dans les fondements mêmes de notre vie sociale, et ne peut être anéantie par aucune transformation politique et juridique. Cette force, c'est d'abord et essentiellement le régime même de la production moderne, à savoir la grande industrie. À ce sujet, les formules vigoureuses abondent, dans Marx, concernant l'asservissement du travail vivant au travail mort, « le renversement du rapport entre l'objet et le sujet », « la subordination du travailleur aux conditions matérielles du travail ». « Dans la fabrique », écrit-il dans le {\itshape Capital}, « il existe un mécanisme indépendant des travailleurs, et qui se les incorpore comme des rouages vivants... La séparation entre les forces spirituelles qui interviennent dans la production et le travail manuel, et la transformation des premières en puissance du capital sur le travail, trouvent leur achèvement dans la grande industrie fondée sur le machinisme. Le détail de la destinée individuelle du manœuvre sur machine disparaît comme un néant devant la science, les formidables forces naturelles et le travail collectif qui sont incorporés dans l'ensemble des machines et constituent avec elles la puissance du maître ». Ainsi la complète subordination de l'ouvrier à l'entreprise et à ceux qui la dirigent repose sur la structure de l'usine et non sur le régime de la propriété. De même « la séparation entre les forces spirituelles qui interviennent dans la production et le travail manuel », ou, selon une autre formule, « la dégradante division du travail en travail manuel et travail intellectuel » est la base même de notre culture, qui est une culture de spécialistes. La science est un monopole, non pas à cause d'une mauvaise organisation de l'instruction publique, mais par sa nature même ; les profanes n'ont accès qu'aux résultats, non aux méthodes, c'est-à-dire qu'ils ne peuvent que croire et non assimiler.\par
Le « socialisme scientifique » lui-même est demeuré le monopole de quelques-uns, et les « intellectuels » ont malheureusement les mêmes privilèges dans le mouvement ouvrier que dans la société bourgeoise. Et il en est de même encore sur le plan politique. Marx avait clairement aperçu que l'oppression étatique repose sur l'existence d'appareils de gouvernement permanents et distincts de la population, à savoir les appareils bureaucratique, militaire et policier ; mais ces appareils permanents sont l'effet inévitable de la 'distinction radicale qui existe en fait entre les fonctions de direction et les fonctions d'exécution. Sur ce point encore, le mouvement ouvrier reproduit intégralement les vices de la société bourgeoise. Sur tous les plans, on se heurte au même obstacle. Toute notre civilisation est fondée sur la spécialisation, laquelle implique l'asservissement de ceux qui exécutent à ceux qui coordonnent ; et sur une telle base, on ne peut qu'organiser et perfectionner l'oppression, mais non pas l'alléger. Loin que la société capitaliste ait élaboré dans son sein les conditions matérielles d'un régime de liberté et d'égalité, l'instauration d'un tel régime suppose une transformation préalable de la production et de la culture.\par
Que Marx et ses disciples aient pu croire cependant à la possibilité d'une démocratie effective sur les bases de la civilisation actuelle, c'est ce qu'on peut comprendre seulement si l'on fait entrer en ligne de compte leur théorie du développement des forces productives. On sait qu'aux yeux de Marx, ce développement constitue, en dernière analyse, le véritable moteur de l'histoire, et qu'il est à peu près illimité. Chaque régime social, chaque classe dominante a pour « tâche », pour « mission historique », de porter les forces productives à un degré sans cesse plus élevé, jusqu'au jour où tout progrès ultérieur est arrêté par les cadres sociaux ; à ce moment les forces productives se révoltent, brisent ces cadres, et une classe nouvelle s'empare du pouvoir. Constater que le régime capitaliste écrase des millions d'hommes, cela ne permet que de le condamner moralement ; ce qui constitue la condamnation historique du régime, c'est le fait qu'après avoir rendu possible le progrès de la production il y fait à présent obstacle. La tâche des révolutions consiste essentiellement dans l'émancipation non pas des hommes mais des forces productives. À vrai dire il est clair que, dès que celles-ci ont atteint un développement suffisant pour que la production puisse s'accomplir au prix d'un faible effort, les deux tâches coïncident ; et Marx supposait que tel est le cas à notre époque. C'est cette supposition qui lui a permis d'établir un accord indispensable à sa tranquillité morale entre ses aspirations idéalistes et sa conception matérialiste de l'histoire. À ses yeux, la technique actuelle, une fois libérée des formes capitalistes de l'économie, peut donner aux hommes, dès maintenant, assez de loisir pour leur permettre un développement harmonieux de leurs facultés, et par suite faire disparaître dans une certaine mesure la spécialisation dégradante établie par le capitalisme ; et surtout le développement ultérieur de la technique doit alléger davantage de jour en jour le poids de la nécessité matérielle, et par une conséquence immédiate celui de la contrainte sociale, jusqu'à ce que l'humanité atteigne enfin un état à proprement parler paradisiaque, où la production la plus abondante coûterait un effort insignifiant, où l'antique malédiction du travail serait levée, bref où serait retrouvé le bonheur d'Adam et d'Eve avant leur faute. On comprend fort bien, à partir de cette conception, la position des bolcheviks, et pourquoi tous, y compris Trotsky, traitent les idées démocratiques avec un mépris souverain. Ils se sont trouvés impuissants à réaliser la démocratie ouvrière prévue par Marx ; mais ils ne se troublent pas pour si peu de chose, convaincus comme ils sont d'une part que toute tentative d'action sociale qui ne consiste pas à développer les forces productives est vouée d'avance à l'échec, d'autre part que tout progrès des forces productives fait avancer l'humanité sur la vole de la libération, même si c'est au prix d'une oppression provisoire. Avec une pareille sécurité morale, il n'est pas surprenant qu'ils aient étonné le monde par leur force.\par
Il est rare cependant que les croyances réconfortantes soient en même temps raisonnables. Avant même d'examiner la conception marxiste des forces productives, on est frappé par le caractère mythologique qu'elle présente dans toute la littérature socialiste, où elle est admise comme un postulat. Marx n'explique jamais pourquoi les forces productives tendraient à s'accroître ; en admettant sans preuve cette tendance mystérieuse, il s'apparente non pas à Darwin, comme il aimait à le croire, mais à Lamarck, qui fondait pareillement tout son système biologique sur une tendance inexplicable des êtres vivants à l'adaptation. De même pourquoi est-ce que, lorsque les institutions sociales s'opposent au développement des forces productives, la victoire devrait appartenir d'avance a celles-ci plutôt qu'à celles-là ? Marx ne suppose évidemment pas que les hommes transforment consciemment leur état social pour améliorer leur situation économique ; il sait fort bien que jusqu'à nos jours les transformations sociales n'ont jamais été accompagnées d'une conscience claire de leur portée réelle ; il admet donc implicitement que les forces productives possèdent une vertu secrète qui leur permet de surmonter les obstacles. Enfin pourquoi pose-t-il sans démonstration, et comme une vérité évidente, que les forces productives sont susceptibles d'un développement illimité? Toute cette doctrine, sur laquelle repose entièrement la conception marxiste de la révolution, est absolument dépourvue de tout caractère scientifique. Pour la comprendre, il faut se souvenir des origines hégéliennes de la pensée marxiste. Hegel croyait en un esprit caché à 1'oeuvre dans l'univers, et que l'histoire du monde est simplement l'histoire de cet esprit du monde, lequel, comme tout ce qui est spirituel, tend indéfiniment à la perfection. Marx a prétendu « remettre sur ses pieds » la dialectique hégélienne, qu'il accusait d'être « sens dessus dessous » ; il a substitué la matière à l'esprit comme moteur de l'histoire ; mais par un paradoxe extraordinaire, il a conçu l'histoire, à partir de cette rectification, comme s'il attribuait à la matière ce qui est l'essence même de l'esprit, une perpétuelle aspiration au mieux. Par là il s'accordait d'ailleurs profondément avec le courant général de la pensée capitaliste ; transférer le principe du progrès de l'esprit aux choses, c'est donner une expression philosophique à ce « renversement du rapport entre le sujet et l'objet » dans lequel Marx voyait l'essence même du capitalisme. L'essor de la grande industrie a fait des forces productives la divinité d'une sorte de religion dont Marx a subi malgré lui l'influence en élaborant sa conception de l'histoire. Le terme de religion peut surprendre quand il s'agit de Marx ; mais croire que notre volonté converge avec une volonté mystérieuse qui serait à l'œuvre dans le monde et nous aiderait a vaincre, c'est penser religieusement, c'est croire à la Providence. D'ailleurs le vocabulaire même de Marx en témoigne, puisqu'il contient des expressions quasi mystiques, telles que « la mission historique du prolétariat ». Cette religion des forces productives au nom de laquelle des générations de chefs d'entreprise ont écrase les masses travailleuses sans le moindre remords constitue également un facteur d'oppression a l'intérieur du mouvement socialiste ; toutes les religions font de l'homme un simple instrument de la Providence, et le socialisme lui aussi met les hommes au service du progrès historique, c'est-à-dire du progrès de la production. C'est pourquoi, quel que soit l'outrage infligé à la mémoire de Marx par le culte que lui vouent les oppresseurs de la Russie moderne, il n'est pas entièrement immérité. Marx, il est vrai, n'a jamais eu d'autre mobile qu'une aspiration généreuse à la liberté et a l'égalité ; seulement cette aspiration, séparée de la religion matérialiste avec laquelle elle se confondait dans son esprit, n'appartient plus qu'à ce que Marx nommait dédaigneusement le socialisme utopique. Si l'œuvre de Marx ne contenait rien de plus précieux, elle pourrait être oubliée sans inconvénient, à l'exception du moins des analyses économiques.\par
Mais ce n'est pas le cas ; on trouve chez Marx une autre conception que cet hégélianisme à rebours, à savoir un matérialisme qui n'a plus rien de religieux et constitue non pas une doctrine, mais une méthode de connaissance et d'action. Il n'est pas rare de voir ainsi chez d'assez grands esprits deux conceptions distinctes et même incompatibles se confondre à la faveur de l'imprécision inévitable du langage ; absorbés par l'élaboration d'idées nouvelles, le temps leur manque pour faire l'examen critique de ce qu'ils ont trouvé. La grande idée de Marx, c'est que dans la société aussi bien que dans la nature rien ne s'effectue autrement que par des transformations matérielles. « Les hommes font leur propre histoire, mais dans des conditions déterminées. » Désirer n'est rien, il faut connaître les conditions matérielles qui déterminent nos possibilités d'action ; et dans le domaine social, ces conditions sont définies par la manière dont l'homme obéit aux nécessités matérielles en subvenant à ses propres besoins, autrement dit par le mode de production. Une amélioration méthodique de l'organisation sociale suppose au préalable une étude approfondie du mode de production, pour chercher à savoir d'une part ce qu'on peut en attendre, dans l'avenir immédiat et lointain, du point de vue du rendement, d'autre part quelles formes d'organisation sociale et de culture sont compatibles avec lui, et enfin comment il peut être lui-même transformé. Seuls des êtres irresponsables peuvent négliger une telle étude et prétendre néanmoins a régenter la société ; et par malheur tel est le cas partout, aussi bien dans les milieux révolutionnaires que dans les milieux dirigeants. La méthode matérialiste, cet instrument que nous a légué Marx, est un instrument vierge ; aucun marxiste ne s'en est véritablement servi, à commencer par Marx lui-même. La seule idée vraiment précieuse qui se trouve dans l'œuvre de Marx est la seule aussi qui ait été complètement négligée. Il n'est pas étonnant que les mouvements sociaux issus de Marx aient fait faillite.\par
La première question à poser est celle du rendement du travail. A-t-on des raisons de supposer que la technique moderne, à son niveau actuel, soit capable, dans l'hypothèse d'une répartition équitable, d'assurer à tous assez de bien-être et de loisir pour que le développement de l'individu cesse d'être entravé par les conditions modernes du travail ? Il semble qu'il y ait a ce sujet beaucoup d'illusions, savamment entretenues par la démagogie. Ce ne sont pas les profits qu'il faut calculer ; ceux des profits qui sont réinvestis dans la production seraient dans l'ensemble ôtés aux travailleurs sous tous les régimes. Il faudrait pouvoir faire la somme de tous les travaux dont on pourrait se dispenser au prix d'une transformation du régime de la propriété. Encore la question ne serait-elle pas résolue par la ; il faut tenir compte des travaux qu'impliquerait la réorganisation complète de l'appareil de production, réorganisation nécessaire pour que la production soit adaptée à sa fin nouvelle, a savoir le bien-être des masses ; il ne faut pas oublier que la fabrication des armements ne serait pas abandonnée avant que le régime capitaliste ne soit détruit partout ; surtout il faut prévoir que la destruction du profit individuel, tout en faisant disparaître certaines formes de gaspillage, en susciterait nécessairement d'autres. Des calculs précis sont évidemment impossibles a établir ; mais ils ne sont pas indispensables pour apercevoir que la suppression de la propriété privée serait loin de suffire à empêcher que le labeur des mines et des usines continue à peser comme un esclavage sur ceux qui y sont assujettis.\par
Mais, si l'état actuel de la technique ne suffit pas à libérer les travailleurs, peut-on du moins raisonnablement espérer qu'elle soit destinée à un développement illimité, qui impliquerait un accroissement illimité du rendement du travail ? C'est ce que tout le monde admet, chez les capitalistes comme chez les socialistes, et sans la moindre étude préalable de la question ; il suffit que le rendement de l'effort humain ait augmenté d'une manière inouïe depuis trois siècles pour qu'on s'attende à ce que cet accroissement se poursuive au même rythme. Notre culture soi-disant scientifique nous a donné cette funeste habitude de généraliser, d'extrapoler arbitrairement, au lieu d'étudier les conditions d'un phénomène et les limites qu'elles impliquent ; et Marx, que sa méthode dialectique devait préserver d'une telle erreur, y est tombé sur ce point comme les autres.\par
Le Problème est capital, et de nature à déterminer toutes nos perspectives ; il faut le formuler avec la dernière précision. A cet effet, il importe de savoir tout d'abord en quoi consiste le progrès technique, quels facteurs y interviennent, et examiner séparément chaque facteur ; car on confond sous le nom de progrès technique des procédés entièrement différents, et qui offrent des possibilités de développement différentes. Le premier procédé qui s'offre à l'homme pour produire plus avec un effort moindre, c'est l'utilisation des sources naturelles d'énergie ; et il est vrai en un sens qu'on ne peut assigner aux bienfaits de ce procédé une limite précise, parce qu'on ignore quelles nouvelles énergies l'on pourra un jour utiliser ; mais ce n'est pas à dire qu'il puisse y avoir dans cette voie des perspectives de progrès indéfini, ni que le progrès y soit en général assuré. Car la nature ne nous donne pas cette énergie, sous quelque forme que celle-ci se présente, force animale, houille ou pétrole ; il faut la lui arracher et la transformer par notre travail pour l'adapter à nos fins propres. Or ce travail ne devient pas nécessairement moindre à mesure que le temps passe ; actuellement, c'est même le contraire qui se produit pour nous, puisque l'extraction de la houille et du pétrole devient sans cesse et automatiquement moins fructueuse et plus coûteuse. Bien plus, les gisements actuellement connus sont destinés à s'épuiser au bout d'un temps relativement court. On peut trouver de nouveaux gisements ; mais la recherche, l'installation d'exploitations nouvelles dont certaines sans doute échoueront, tout cela sera coûteux ; au reste nous ne savons pas combien il existe en général de gisements inconnus, et de toute manière la quantité n'en sera pas illimitée. On peut aussi, et on devra sans doute un jour, trouver des sources d'énergie nouvelles ; seulement rien ne garantit que l'utilisation en exigera moins de travail que l'utilisation de la houille ou des huiles lourdes ; le contraire est également possible. Il peut même arriver a la rigueur que l'utilisation d'une source d'énergie naturelle coûte un travail supérieur aux efforts humains que l'on cherche à remplacer. Sur ce terrain c'est le hasard qui décide ; car la découverte d'une source d'énergie nouvelle et facilement accessible ou d'un procédé économique de transformation pour une source d'énergie connue n'est pas de ces choses auxquelles on soit sûr d'arriver à condition de réfléchir avec méthode et d'y mettre le temps. On se fait illusion à ce sujet parce qu'on a l'habitude de considérer le développement de la science du dehors et en bloc ; on ne se rend pas compte que si certains résultats scientifiques dépendent uniquement du bon usage que fait le savant de sa raison, d'autres ont pour condition d'heureuses rencontres. C'est le cas en ce qui concerne l'utilisation des forces de la nature. Certes toute source d'énergie est transformable à coup sûr ; mais le savant n'est pas plus sûr de rencontrer au cours de ses recherches quelque chose d'économiquement avantageux que l'explorateur de parvenir à un territoire fertile. C'est de quoi on peut trouver un exemple instructif dans les fameuses expériences concernant l'énergie thermique des mers, autour desquelles on a fait tant de bruit, et si vainement. Or dès lors que le hasard entre en jeu, la notion de progrès continu n'est plus applicable. Ainsi espérer que le développement de la science amènera quelque jour, d'une manière en quelque sorte automatique, la découverte d'une source d'énergie qui serait utilisable d'une manière presque immédiate pour tous les besoins humains, c'est rêver. On ne peut démontrer que ce soit impossible ; et à vrai dire il est possible aussi qu'un beau jour quelque transformation soudaine de l'ordre astronomique octroie à de vastes étendues du globe terrestre le climat enchanteur qui permet, dit-on, à certaines peuplades primitives de vivre sans travail; mais les possibilités de cet ordre ne doivent jamais entrer en ligne de compte. Dans l'ensemble, il ne serait pas raisonnable de prétendre déterminer dès maintenant ce que l'avenir réserve au genre humain en te domaine.\par
Il n'existe par ailleurs qu'une autre ressource permettant de diminuer la somme de l'effort humain, à savoir ce que l'on peut nommer, en se servant d'une expression moderne, la rationalisation du travail. On y peut distinguer deux aspects, l'un qui concerne le rapport entre les efforts simultanés, l'autre le rapport entre les efforts successifs ; dans les deux cas le progrès consiste à augmenter le rendement des efforts par la manière dont on les combine. Il est clair que dans ce domaine on peut à la rigueur faire abstraction des hasards, et que la notion de progrès y a un sens ; la question est de savoir si ce progrès est illimité, et, dans le cas contraire, si nous sommes encore loin de la limite. En ce qui concerne ce qu'on peut nommer la rationalisation du travail dans l'espace, les facteurs d'économie sont la concentration, la division et la coordination des travaux. La concentration du travail implique la diminution de toutes sortes de dépenses qu'on peut englober sous le nom de frais généraux, parmi lesquelles les dépenses concernant le local, les transports, parfois l'outillage. La division du travail, elle, a des effets beaucoup plus étonnants. Tantôt elle permet d'obtenir une rapidité considérable dans l'exécution d'ouvrages que des travailleurs isolés pourraient accomplir aussi bien, mais beaucoup plus lentement, et cela parce que chacun devrait faire pour son compte l'effort de coordination que l'organisation du travail permet à un seul homme d'assumer pour le compte de beaucoup d'autres ; la célèbre analyse d'Adam Smith concernant la fabrication des épingles en fournit un exemple. Tantôt, et c'est ce qui importe le plus, la division et la coordination des efforts rend possibles des oeuvres colossales qui dépasseraient infiniment les possibilités d'un homme seul. Il faut tenir compte aussi des économies que permet en ce qui concerne les transports d'énergie et de matière première la spécialisation par régions, et sans doute encore de bien d'autres économies qu'il serait trop long de rechercher. Quoi qu'il en soit, des qu’on jette un regard sur le régime actuel de la production, il semble assez clair non seulement que ces facteurs d'économie comportent une limite au delà de laquelle ils deviennent facteurs de dépense, mais encore que cette limite est atteinte et dépassée. Depuis des années déjà l'agrandissement des entreprises s'accompagne non d'une diminution, mais d'un accroissement des frais généraux ; le fonctionnement de l'entreprise, devenu trop complexe pour permettre un contrôle efficace, laisse une marge de plus en plus grande au gaspillage et suscite une extension accélérée et sans doute dans une certaine mesure parasitaire du personnel affecté à la coordination des diverses parties de l'entreprise. L'extension des échanges, qui a autrefois joué un rôle formidable comme facteur de progrès économique, se met elle aussi à causer plus de frais qu'elle n'en évite, parce que les marchandises restent longtemps improductives, parce que le personnel affecté aux échanges s'accroît lui aussi à un rythme accéléré, et parce que les transports consomment une énergie sans cesse accrue en raison des innovations destinées à augmenter la vitesse, innovations nécessairement de plus en plus coûteuses et de moins en moins efficaces à mesure qu'elles se succèdent. Ainsi à tous ces égards le progrès se transforme aujourd'hui, d'une manière a proprement parler mathématique, en régression.\par
Le progrès dû, a la coordination des efforts dans le temps est sans doute le facteur le plus important du progrès technique ; il, est aussi le plus difficile à analyser. Depuis Marx, on a coutume de le désigner en parlant de la substitution du travail mort au travail vivant, formule d'une redoutable imprécision, en ce sens qu'elle évoque l'image d'une évolution continue vers une étape de la technique où, si l'on peut parler ainsi, tous les travaux à faire seraient déjà faits. Cette image est aussi chimérique que celle d'une source naturelle d'énergie qui serait aussi immédiatement accessible à l'homme que sa propre force vitale. La substitution d'ont il s'agit met simplement à la place des mouvements qui permettraient d'obtenir directement certains résultats d'autres mouvements qui produisent ce résultat indirectement grâce à la disposition assignée a des choses inertes ; c'est toujours confier à la matière ce qui semblait être le rôle de l'effort humain, mais au lieu d'utiliser l'énergie que fournissent certains phénomènes naturels, on utilise la résistance, la solidité, la dureté que possèdent certains matériaux. Dans un cas comme dans l'autre, les propriétés de la matière aveugle et indifférente ne peuvent être adaptées aux fins humaines que par le travail humain ; et dans un cas comme dans l'autre la raison interdit d'admettre à l’avance que ce travail d'adaptation doive nécessairement être inférieur à l'effort que devraient fournir les hommes Pour atteindre directement la fin qu'ils ont en vue. Mais alors que l'utilisation des sources naturelles d'énergie dépend pour une part considérable de rencontres imprévisibles, l'utilisation de matériaux inertes et résistants s'est effectuée dans l'ensemble selon une progression continue que l'on peut embrasser et prolonger par la pensée lorsqu'on en a une fois aperçu le principe. La première étape, vieille comme l'humanité, consiste à confier à des objets placés en des lieux convenables tous les efforts de résistance ayant pour but d'empêcher certains mouvements de la part de certaines choses. La deuxième étape définit le machinisme proprement dit ; le machinisme est devenu possible le jour où l'on s'est aperçu que l'on pouvait non seulement utiliser la matière inerte pour assurer l'immobilité là où il le fallait, mais encore la charger de conserver les rapports permanents des mouvements entre eux, rapports qui jusque-là devaient être à chaque fois établis par la pensée. À cette fin il faut et il suffit que l'on ait pu inscrire ces rapports, en les transposant, dans les formes imprimées à la matière solide. C'est ainsi qu'un des premiers progrès qui aient ouvert la voie au machinisme a consisté à dispenser le tisserand d'adapter le choix des -fils à tirer sur son métier au dessin de l'étoffe, et cela grâce à un carton percé de trous qui correspondent au dessin. Si l'on n'a pu obtenir les transpositions de cet ordre dans les diverses espèces de travail que peu à peu et grâce à des inventions apparemment dues à l'inspiration ou au hasard, c'est parce que le travail manuel combine les éléments permanents qu'il contient de manière à les dissimuler le plus souvent sous une apparence de variété ; c'est pourquoi le travail parcellaire des manufactures a dû précéder la grande industrie. Enfin la troisième et dernière étape correspond à la technique automatique, qui ne fait que commencer à apparaître ; le principe en réside dans la possibilité de confier à la machine non seulement une opération toujours identique à elle-même, mais encore un ensemble d'opérations variées. Cet ensemble peut être aussi vaste, aussi complexe qu'on voudra ; il est seulement nécessaire qu'il s'agisse d'une variété définie et limitée à l'avance. La technique automatique, qui se trouve encore à un état en quelque sorte primitif, peut donc théoriquement se développer indéfiniment ; et l'utilisation d'une telle technique pour satisfaire les besoins humains ne comporte d'autres limites que celles qu'impose la part de l'imprévu dans les conditions de l'existence humaine. Si l'on pouvait concevoir des conditions de vie ne comportant absolument aucun imprévu, le mythe américain du robot aurait un sens, et la suppression complète du travail humain par un aménagement systématique du monde serait possible. Il n'en est rien, et ce ne sont là que fictions ; encore ces fictions seraient-elles utiles à élaborer, à titre de limite idéale, si les hommes avaient du moins le pouvoir de diminuer progressivement par une méthode quelconque cette part d'imprévu dans leur vie. Mais ce n’est pas le cas non plus, et jamais aucune technique ne dispensera les hommes de renouveler et d'adapter continuellement, à la sueur de leur front, l'outillage dont ils se servent.\par
Dans ces conditions il est facile de concevoir qu'un certain degré d'automatisme puisse être plus coûteux en efforts humains qu'un degré moins élevé. Du moins est-ce facile à concevoir abstraitement ; il est presque impossible d'arriver en cette matière à une appréciation concrète a cause du grand nombre de facteurs qu'il faudrait faire entrer en ligne de compte. L'extraction des métaux dont les machines sont faites ne peut s'opérer qu'avec du travail humain ; et, comme il s'agit de mines, le travail devient de plus en plus pénible à mesure qu'il s'effectue, sans compter que les gisements connus risquent de s'épuiser d'une manière relativement rapide ; les hommes se reproduisent, non le fer. Il ne faut pas oublier non plus, bien que les bilans financiers, les statistiques, les ouvrages des économistes dédaignent de le noter, que le travail des mines est plus douloureux, plus épuisant, plus dangereux que la plupart des autres travaux; le fer, le charbon, la potasse, tous ces produits sont souillés de sang. Au reste les machines automatiques ne sont avantageuses qu'autant que l'on s'en sert pour produire en série et en quantités massives ; leur fonctionnement est donc lié au désordre et au gaspillage qu'entraîne une centralisation économique exagérée ; d'autre part elles créent la tentation de produire beaucoup plus qu'il n'est nécessaire pour satisfaire les besoins réels, ce qui amène à dépenser sans profit des trésors de force humaine et de matières premières. Il ne faut pas négliger non plus les dépenses qu'entraîne tout progrès technique, à cause des recherches préalables, de la nécessité d'adapter à ce progrès d'autres branches de la production, de l'abandon du vieux matériel qui souvent est rejeté alors qu'il aurait pu servir encore longtemps. Rien de tout cela n'est susceptible d'être même approximativement mesuré. Il est seulement clair, dans l'ensemble, que plus le niveau de la technique est élevé, plus les avantages que peuvent apporter des progrès nouveaux diminuent par rapport aux inconvénients. Nous n'avons cependant aucun moyen de nous rendre clairement compte si nous sommes près ou loin de la limite à partir de laquelle le progrès technique doit se transformer en facteur de régression économique. Nous pouvons seulement essayer de le deviner empiriquement, d'après la manière dont évolue l'économie actuelle.\par
Or ce que nous voyons, c'est que depuis quelques années, dans presque toutes les industries, les entreprises refusent systématiquement d'accueillir les innovations techniques. La presse socialiste et communiste tire de ce fait des déclamations éloquentes contre le capitalisme, mais elle omet d'expliquer par quel miracle des innovations actuellement dispendieuses deviendraient économiquement avantageuses en régime socialiste ou soi-disant tel. Il est plus raisonnable de supposer que dans ce domaine nous ne sommes pas loin de la limite du progrès utile ; et même, étant donné que la complication des rapports économiques actuels et l'extension formidable du crédit empêchent les chefs d'entreprise de s'apercevoir immédiatement qu'un facteur autrefois avantageux a cessé de l'être, on peut conclure, avec toutes les réserves qui conviennent concernant un problème aussi confus, que vraisemblablement cette limite est déjà dépassée.\par
Une étude sérieuse de la question devrait à vrai dire prendre en considération bien d'autres éléments. Les divers facteurs qui contribuent à accroître le rendement du travail ne se développent pas séparément, bien qu'il faille les séparer dans l'analyse ; ils se combinent, et ces combinaisons produisent des effets difficiles à prévoir. Au reste le progrès technique ne sert pas seulement à obtenir à peu de frais ce qu'on obtenait auparavant avec beaucoup d'efforts ; il rend aussi possibles des ouvrages qui auraient été sans lui presque inimaginables. Il y aurait lieu d'examiner la valeur de ces possibilités nouvelles, en tenant compte du fait qu'elles ne sont pas seulement possibilités de construction, mais aussi de destruction. Mais une telle étude devrait obligatoirement tenir compte des rapports économiques et sociaux qui sont nécessairement liés à une forme déterminée de la technique. Pour l'instant, il suffit d'avoir compris que la possibilité de progrès ultérieurs en ce qui concerne le rendement du travail n'est pas hors de doute ; que, selon toute apparence, on a présentement autant de raisons de s'attendre à le voir diminuer qu'augmenter ; et, ce qui est le plus important, qu'un accroissement continu et illimité de ce rendement est à proprement parler inconcevable. C'est uniquement l'ivresse produite par la rapidité du progrès technique qui a fait naître la folle idée que le travail pourrait un jour devenir superflu. Sur le plan de la science pure, cette idée s'est traduite par la recherche de la « machine à mouvement perpétuel », c'est-à-dire de la machine qui produirait indéfiniment du travail sans jamais en consommer ; et les savants en ont fait prompte justice en posant la loi de la conservation de l'énergie. Dans le domaine social, les divagations sont mieux accueillies. «L'étape supérieure du communisme » considérée par Marx comme le dernier terme de l'évolution sociale est, en somme, une utopie absolument analogue à celle du mouvement perpétuel. Et c'est au nom de cette utopie que les révolutionnaires ont versé leur sang. Pour mieux dire ils ont versé leur sang au nom ou de cette utopie ou de la croyance également utopique que le système de production actuel pourrait être mis par un simple décret au service d'une société d'hommes libres et égaux. Quoi d'étonnant si tout ce sang a coulé en vain ? L'histoire du mouvement ouvrier s'éclaire ainsi d'une lumière cruelle, mais particulièrement vive. On peut la résumer tout entière en remarquant que la classe ouvrière n'a jamais fait preuve de force qu'autant qu'elle a servi autre chose que la révolution ouvrière. Le mouvement ouvrier a pu donner l'illusion de la puissance aussi longtemps qu'il s'est agi pour lui de contribuer à liquider les vestiges de la féodalité, à aménager la domination capitaliste soit sous la forme du capitalisme prive, soit sous la forme du capitalisme d'Etat, comme ce fut le cas en Russie ; à présent que sur ce terrain son rôle est terminé, et que la crise pose devant lui le problème de la prise effective du pouvoir par les masses travailleuses, il s'effrite et se dissout avec une rapidité qui brise le courage de ceux qui avaient mis leur foi en lui. Sur ses ruines se déroulent des controverses interminables qui ne peuvent s'apaiser que par les formules les plus ambiguës ; car parmi tous les hommes qui s'obstinent encore à parler de révolution, il n'y en a peut-être pas deux qui attribuent à ce terme le même contenu. Et cela n'a rien d'étonnant. Le mot de révolution est un mot pour lequel on tue, pour lequel on meurt, pour lequel on envoie les masses populaires à la mort, mais qui n'a aucun contenu.\par
Peut-être cependant peut-on donner un sens à l'idéal révolutionnaire, sinon en tant que perspective possible, du moins en tant que limite théorique des transformations sociales réalisables. Ce que nous demanderions a la révolution, c'est l'abolition de l'oppression sociale ; mais pour que cette notion ait au moins des chances d'avoir une signification quelconque, il faut avoir soin de distinguer entre oppression et subordination des caprices individuels a un ordre social. Tant qu'il y aura une société, elle enfermera la vie des individus dans des limites fort étroites et leur imposera ses règles ; mais cette contrainte inévitable ne mérite d'être nommée oppression que dans la mesure ou, du fait qu’elle provoque une séparation entre ceux qui l'exercent et ceux qui la subissent, elle met les seconds à la discrétion des premiers et fait ainsi peser jusqu'à l'écrasement physique et moral la pression de ceux qui commandent sur ceux qui exécutent. Même après cette distinction, rien ne permet au premier abord de supposer que la suppression de l'oppression soit ou possible ou même seulement concevable à titre de limite. Marx a fait voir avec force, dans des analyses dont lui-même a méconnu la portée, que le régime actuel de la production, à savoir la grande industrie, réduit l'ouvrier à n'être qu’un rouage de la fabrique et un simple instrument aux mains de ceux qui le dirigent ; et il est vain d'espérer que le progrès technique puisse, par une diminution progressive et continue de l'effort de la production, alléger, jusqu'à le faire presque disparaître, le double poids sur l'homme de la nature et de la société. Le problème est donc bien clair ; il s'agit de savoir si l'on peut concevoir une organisation de la production qui, bien qu'impuissante à éliminer les nécessités naturelles et la contrainte sociale qui en résulte, leur permettrait du moins de s'exercer sans écraser sous l'oppression les esprits et les corps. À une époque comme la nôtre, avoir saisi clairement ce problème est peut-être une condition pour pouvoir vivre en paix avec soi. Si l'on arrive a concevoir concrètement les conditions de cette organisation libératrice, il ne reste qu'à exercer, pour se diriger vers elle, toute la puissance d'action, petite ou grande, dont on dispose ; et si l'on comprend clairement que la possibilité d'un tel mode de production n'est pas même concevable, on y gagne du moins de pouvoir légitimement se résigner à l'oppression, et cesser de s'en croire complice du fait qu'on ne fait rien d'efficace pour l'empêcher.
\subsection[ANALYSE DE L‘OPPRESSION.]{ANALYSE DE L‘OPPRESSION.}
\noindent Il s'agit en somme de connaître ce qui lie l'oppression en général et chaque forme d'oppression en particulier au régime de la production ; autrement dit d'arriver à saisir le mécanisme de l'oppression, à comprendre en vertu de quoi elle surgit, subsiste, se transforme, en vertu de quoi peut-être elle pourrait théoriquement disparaître. C'est là, ou peu s'en faut, une question neuve. Pendant des siècles, des âmes généreuses ont considéré la puissance des oppresseurs comme constituant une usurpation pure et simple, à laquelle il fallait tenter de s'opposer soit par la simple expression d'une réprobation radicale, soit par la force armée mise au service de la justice. Des deux manières, l'échec a toujours été complet ; et jamais il n'était plus significatif que quand il prenait un moment l'apparence de la victoire, comme ce fut le cas pour la Révolution française, et qu'après avoir effectivement réussi à faire disparaître une certaine forme d'oppression, on assistait, impuissant, à l'installation immédiate d'une oppression nouvelle.\par
La réflexion sur cet échec retentissant, qui était venu couronner tous les autres, amena enfin Marx à comprendre qu'on ne peut supprimer l'oppression tant que subsistent les causes qui la rendent inévitable, et que ces causes résident dans les conditions objectives, c'est-à-dire matérielles, de l'organisation sociale. Il élabora ainsi une conception de l'oppression tout à fait neuve, non plus en tant qu'usurpation d'un privilège, mais en tant qu'organe d'une fonction sociale. Cette fonction, c'est celle même qui consiste à développer les forces productives, dans la mesure où ce développement exige de durs efforts et de lourdes privations; et, entre ce développement et l'oppression sociale, Marx et Engels ont aperçu des rapports réciproques. Tout d'abord, selon eux, l'oppression s'établit seulement quand les progrès de la production ont suscité une division du travail assez poussée pour que l'échange, le commandement militaire et le gouvernement constituent des fonctions distinctes ; d'autre part l'oppression, une fois établie, provoque le développement ultérieur des forces productives, et change de forme à mesure que l'exige ce développement, jusqu'au jour où, devenue pour lui une entrave et non une aide, elle disparaît purement et simplement. Quelque brillantes que soient les analyses concrètes par lesquelles les marxistes ont illustré ce schéma, et bien qu'il constitue un progrès sur les naïves indignations qu'il a remplacées, on ne peut dire qu'il mette en lumière le mécanisme de l'oppression. Il n'en décrit que partiellement la naissance ; car pourquoi la division du travail se tournerait-elle nécessairement en oppression ? Il ne permet nullement d'en attendre raisonnablement la fin ; car, si Marx a cru montrer comment le régime capitaliste finit par entraver la production, il n'a même pas essayé de prouver que, de nos jours, tout autre régime oppressif l'entraverait pareillement ; et de plus on ignore pourquoi l'oppression ne pourrait pas réussir à se maintenir, même une fois devenue un facteur de régression économique. Surtout Marx omet d'expliquer pourquoi l'oppression est invincible aussi longtemps qu'elle est utile, pourquoi les opprimés en révolte n'ont jamais réussi à fonder une société non oppressive, soit sur la base des forces productives de leur époque, soit même au prix d'une régression économique qui pouvait difficilement accroître leur misère ; et enfin il laisse tout à fait dans l'ombre les principes généraux du mécanisme par lequel une forme déterminée d'oppression est remplacée par une autre.\par
Bien plus, non seulement les marxistes n'ont résolu aucun de ces problèmes, mais ils n'ont même pas cru devoir les formuler. Il leur a semblé avoir suffisamment rendu compte de l'oppression sociale en posant qu’elle correspond à une fonction dans la lutte contre la nature. Au reste ils n'ont vraiment mis cette correspondance en lumière que pour le régime capitaliste ; mais de toute manière, supposer qu'une telle correspondance constitue une explication du phénomène, c'est appliquer inconsciemment aux organismes sociaux le fameux principe de Lamarck, aussi inintelligible que commode, « la fonction crée l'organe ». La biologie n'a commencé d'être une science que le jour où Darwin a substitué à ce principe la notion des conditions d'existence. Le progrès consiste en ce que la fonction n'est plus considérée comme la cause, mais comme l'effet de l'organe, seul ordre intelligible ; le rôle de cause n'est dès lors attribué qu'à un mécanisme aveugle, celui de l'hérédité combiné avec les variations accidentelles. Par lui-même, à vrai dire, ce mécanisme aveugle ne peut que produire au hasard n'importe quoi ; l'adaptation de l'organe à la fonction rentre ici en jeu de manière à limiter le hasard en éliminant les structures non viables, non plus à titre de tendance mystérieuse, mais à titre de condition d'existence ; et cette condition se définit par le rapport de l'organisme considéré au milieu pour une part inerte et pour une part vivant qui l'entoure, et tout particulièrement aux organismes semblables qui lui font concurrence. L'adaptation est dès lors conçue par rapport aux êtres vivants comme une nécessite extérieure et non plus intérieure. Il est clair que cette méthode lumineuse n'est pas valable seulement en biologie, mais partout où l'on se trouve en présence de structures organisées qui n'ont été organisées par personne. Pour pouvoir se réclamer de la science en matière sociale, il faudrait avoir accompli par rapport au marxisme un progrès analogue à celui que Darwin a accompli par rapport à Lamarck. Les causes de l'évolution sociale ne doivent plus être cherchées ailleurs que dans les efforts quotidiens des hommes considérés comme individus. Ces efforts ne se dirigent certes pas n'importe où ; ils dépendent, pour chacun, du tempérament, de l'éducation, des routines, des coutumes, des préjugés, des besoins naturels ou acquis, de l'entourage, et surtout, d'une manière générale, de la nature humaine, terme qui, pour être malaisé à définir, n'est probablement pas vide de sens. Mais étant donné la diversité presque indéfinie des individus, étant donne surtout que la nature humaine comporte entre autres choses le pouvoir d'innover, de créer, de se dépasser soi-même, ce tissu d'efforts incohérents produirait n'importe quoi en fait d'organisation sociale, si le hasard ne se trouvait en ce domaine limité par les conditions d'existence auxquelles toute société doit se conformer sous peine d'être ou subjuguée ou anéantie. Ces conditions d'existence sont le plus souvent ignorées des hommes qui s'y soumettent ; elles agissent non pas en imposant aux efforts de chacun une direction déterminée, mais en condamnant à être inefficaces tous les efforts dirigés dans les voles qu'elles interdisent.\par
Ces conditions d'existence sont déterminées tout d'abord, comme pour les êtres vivants, d'une part par le milieu naturel, d'autre part par l'existence, par l'activité et particulièrement par la concurrence des autres organismes de même espèce, c'est-à-dire en l'occurrence des autres groupements sociaux. Mais un troisième facteur entre encore en jeu, à savoir l'aménagement du milieu naturel, l'outillage, l'armement, les procédés de travail et de combat ; et ce facteur occupe une place à part du fait que, s'il agit sur la forme de l'organisation sociale, il en subit à son tour la réaction. Au reste ce facteur est le seul sur lequel les membres d'une société puissent peut-être avoir quelque prise. Cet aperçu est trop abstrait pour pouvoir guider ; mais si l'on pouvait a partir de cette vue sommaire arriver à des analyses concrètes, il deviendrait enfin possible de poser le problème social. La bonne volonté éclairée des hommes agissant en tant qu'individus est l'unique principe possible du progrès social ; si les nécessités sociales, une fois clairement aperçues, se révélaient comme étant hors de la portée de cette bonne volonté au même titre que celles qui régissent les astres, chacun n'aurait plus qu'à regarder se dérouler l'histoire comme on regarde se dérouler les saisons, en faisant son possible pour éviter à lui-même et aux êtres aimés le malheur d'être soit un instrument soit une victime de l'oppression sociale. S'il en est autrement, il faudrait tout d'abord définir à titre de limite idéale les conditions objectives qui laisseraient place à une organisation sociale absolument pure d'oppression ; puis chercher par quels moyens et dans quelle mesure on peut transformer les conditions effectivement données de manière à les rapprocher de cet idéal ; trouver quelle est la forme la moins oppressive d'organisation sociale pour un ensemble de conditions objectives déterminées ; enfin définir dans ce domaine le pouvoir d'action et les responsabilités des individus considérés comme tels. À cette condition seulement l'action politique pourrait devenir quelque chose d'analogue à un travail, au lieu d'être, comme ce fut le cas jusqu'ici, soit un jeu, soit une branche de la magie.\par
Par malheur, pour en arriver là, il ne faut pas seulement des réflexions approfondies, rigoureuses, soumises, afin d'éviter toute erreur, au contrôle le plus serré ; il faut aussi des études historiques, techniques et scientifiques, d'une étendue et d'une précision inouïes, et menées d'un point de vue tout à fait nouveau. Cependant les événements n'attendent pas ; le temps ne s'arrêtera pas pour nous ménager des loisirs ; l'actualité s'impose à nous d'une manière urgente, et nous menace de catastrophes qui entraîneraient, parmi bien d'autres malheurs déchirants, l'impossibilité matérielle d'étudier et d'écrire autrement qu'au service des oppresseurs. Que faire ? Rien ne servirait de se laisser emporter dans la mêlée par un entraînement irréfléchi. Nul n’a la plus faible idée ni des buts ni des moyens de ce qu'on nomme encore par habitude l'action révolutionnaire. Quant au réformisme, le principe du moindre mal qui en constitue la base est certes éminemment raisonnable, si discrédité soit-il par la faute de ceux qui en ont fait usage jusqu'ici, seulement, s'il n'a encore servi que de prétexte à capituler, ce n'est pas dû à la lâcheté de quelques chefs, mais à une ignorance par malheur commune à tous ; car tant qu'on n'a pas défini le pire et le mieux en fonction d'un idéal clairement et concrètement conçu, puis déterminé Il marge exacte des possibilités, on ne sait pas quel est le moindre mal, et dès lors on est contraint d'accepter sous ce nom tout ce qu'imposent effectivement ceux qui ont en main la force, parce que n'importe quel mal réel est toujours moindre que les maux possibles que risque toujours d'amener une action non calculée. D'une manière générale, les aveugles que nous sommes actuellement n'ont guère le choix qu'entre la capitulation et l'aventure. L'on ne peut pourtant se dispenser de déterminer dès maintenant l'attitude à prendre par rapport à la situation présente. C'est pourquoi, en attendant d'avoir, si toutefois la chose est possible, démonté le mécanisme social, il est permis peut-être d'essayer d'en esquisser les principes ; pourvu qu'il soit bien entendu qu'une telle esquisse exclut toute espèce d'affirmation catégorique, et vise uniquement à soumettre quelques idées, à titre d'hypothèses, à l'examen critique des gens de bonne foi. Au reste on est loin d'être sans guide en la matière. Si le système de Marx, dans ses grandes lignes, est d'un faible secours, il en est autrement des analyses auxquelles il a été amené par l'étude concrète du capitalisme, et dans lesquelles, tout en croyant se borner à caractériser un régime, il a sans doute plus d'une fois saisi la nature cachée de l'oppression elle-même.\par
Parmi toutes les formes d'organisation sociale que nous présente l'histoire, fort rares sont celles qui apparaissent comme vraiment pures d'oppression ; encore sont-elles assez mal connues. Toutes correspondent à un niveau extrêmement bas de la production, si bas que la division du travail y est à peu près inconnue, sinon entre les sexes, et que chaque famille ne produit guère plus que ce qu'elle a besoin de consommer. Il est assez clair d'ailleurs qu'une pareille condition matérielle exclut forcement l'oppression, puisque chaque homme, contraint de se nourrir lui-même, est sans cesse aux prises avec la nature extérieure ; la guerre même, à ce stade, est guerre de pillage et d'extermination, non de conquête, parce que les moyens d'assurer la conquête et surtout d'en tirer parti font défaut. Ce qui est surprenant, ce n'est pas que l'oppression apparaisse seulement à partir des formes plus élevées de l'économie, c'est qu'elle les accompagne toujours. C'est donc qu'entre une économie tout à fait primitive et les formes économiques plus développées il n'y a pas seulement différence de degré, mais aussi de nature. Et en effet, si, du point de vue de la consommation, il n'y a que passage à un peu plus de bien-être, la production, qui est le facteur décisif, se transforme, elle, dans son essence même . Cette transformation consiste à première vue en un affranchissement progressif à l'égard de la nature. Dans les formes tout à fait primitives de la production, chasse, pêche, cueillette, l'effort humain apparaît comme une simple réaction à la pression inexorable continuellement exercée par la nature sur l'homme, et cela de deux manières ; tout d'abord il s'accomplit, ou peu s'en faut, sous la contrainte immédiate, sous l'aiguillon continuellement ressenti des besoins naturels ; et par une conséquence indirecte, l'action semble recevoir sa forme de la nature elle-même, à cause du rôle important qu'y jouent une intuition analogue a l'instinct animal et une patiente observation des phénomènes naturels les plus fréquents, a cause aussi de la répétition indéfinie des procédés qui ont souvent réussi sans qu'on sache pourquoi, et qui sont sans doute regardés comme étant accueillis par la nature avec une faveur particulière. À ce stade, chaque homme est nécessairement libre à l'égard des autres hommes, parce qu'il est en contact immédiat avec les conditions de sa propre existence, et que rien d'humain ne s'interpose entre elles et lui ; mais en revanche, et dans la même mesure, il est étroitement assujetti a la domination de la nature, et il le laisse bien voir en la divinisant. Aux étapes supérieures de la production, la contrainte de la nature continue certes à s'exercer, et toujours impitoyablement, mais d'une manière en apparence moins immédiate ; elle semble devenir de plus en plus large et laisser une marge croissante au libre choix de l'homme, à sa faculté d'initiative et de décision. L'action n'est plus collée d'instant en instant aux exigences de la nature ; on apprend à constituer des réserves, à longue échéance, pour des besoins non encore ressentis ; les efforts qui ne sont susceptibles que d'une utilité indirecte se font de plus en plus nombreux ; du même coup une coordination systématique dans le temps et dans l'espace devient possible et nécessaire, et l'importance s'en accroît continuellement. Bref l'homme semble passer par étapes, a l'égard de la nature, de l'esclavage à la domination. En même temps la nature perd graduellement son caractère divin, et la divinité revêt de plus en plus la forme humaine. Par malheur, cette émancipation n'est qu'une flatteuse apparence. En réalité, à ces étapes supérieures, l'action humaine continue, dans l'ensemble, à n'être que pure obéissance à l'aiguillon brutal d'une nécessité immédiate ; seulement, au lieu d'être harcelé par la nature, l'homme est désormais harcelé par l'homme. Au reste c'est bien toujours la pression de la nature qui continue à se faire sentir, quoique indirectement ; car l'oppression s'exerce par la force, et en fin de compte, toute force a sa source dans la nature.\par
La notion de force est loin d'être simple, et cependant elle est la première à élucider pour poser les problèmes sociaux. La force et l'oppression, cela fait deux ; mais ce qu'il faut comprendre avant tout, c'est que ce n'est pas la manière dont on use d'une force quelconque, mais sa nature même qui détermine si elle est ou non oppressive. C'est ce que Marx a clairement aperçu en ce qui concerne l'État ; il a compris que cette machine à broyer les hommes ne peut cesser de broyer tant qu'elle est en fonction, entre quelques mains qu'elle soit. Mais cette vue a une portée beaucoup plus générale. L'oppression procède exclusivement de conditions objectives. La première d'entre elles est l'existence de privilèges ; et ce ne sont pas les lois ou les décrets des hommes qui déterminent les privilèges, ni les titres de propriété ; c'est la nature même des choses. Certaines circonstances, qui correspondent à des étapes sans doute inévitables du développement humain, font surgir des forces qui s'interposent entre l'homme du commun et ses propres conditions d'existence, entre l'effort et le fruit de l'effort, et qui sont, par leur essence même, le monopole de quelques-uns, du fait qu'elles ne peuvent être réparties entre tous ; dès lors ces privilégiés, bien qu'ils dépendent, pour vivre, du travail d'autrui, disposent du sort de ceux même dont ils dépendent, et l'égalité périt. C'est ce qui se produit tout d'abord lorsque les rites religieux par lesquels l'homme croit se concilier la nature, devenus trop nombreux et trop compliqués pour être connus de tous, deviennent le secret et par suite le monopole de quelques prêtres; le prêtre dispose alors, bien que ce soit seulement par une fiction, de toutes les puissances de la nature, et c'est en leur nom qu'il commande. Rien d'essentiel n'est changé lorsque ce monopole est constitué non plus par des rites, mais par des procédés scientifiques, et que ceux qui le détiennent s'appellent, au lieu de prêtres, savants et techniciens. Les armes, elles aussi, donnent naissance a un privilège du jour où d'une part elles sont assez puissantes pour rendre impossible toute défense d'hommes désarmés contre des hommes armés, et où d'autre part leur maniement est devenu assez perfectionne et par suite assez difficile pour exiger un long apprentissage et une pratique continuelle. Car dès lors les travailleurs sont impuissants à se défendre, au lieu que les guerriers, tout en se trouvant dans l'impossibilité de produire, peuvent toujours s'emparer par les armes des fruits du travail d'autrui ; ainsi les travailleurs sont à la merci des guerriers, et non inversement. Il en est de même pour l'or, et plus généralement pour la monnaie, dès que la division du travail est assez poussée pour qu'aucun travailleur ne puisse vivre de ses produits sans en avoir échange au moins une partie avec ceux des autres ; l'organisation des échanges devient alors nécessairement le monopole de quelques spécialistes, et ceux-ci, ayant la monnaie en mains, peuvent à la fois se procurer, pour vivre, les fruits du travail d'autrui, et priver les producteurs de l'indispensable. Enfin partout où dans la lutte contre les hommes ou contre la nature les efforts ont besoin de s'ajouter et de se coordonner entre eux pour être efficaces, la coordination devient le monopole de quelques dirigeants dès qu'elle atteint un certain degré de complication, et la première loi de l'exécution est alors l'obéissance ; c'est le cas aussi bien pour l'administration des affaires publiques que pour celle des entreprises. Il peut y avoir d'autres sources de privilège, mais ce sont là les principales; au reste, sauf la monnaie qui apparaît à un moment déterminé de l'histoire, tous ces facteurs jouent sous tous les régimes oppressifs ; ce qui change, c'est la manière dont ils se répartissent et se combinent, c'est le degré de concentration du pouvoir, c'est aussi le caractère plus ou moins fermé et par suite plus ou moins mystérieux de chaque monopole. Cependant les privilèges, par eux-mêmes, ne suffisent pas à déterminer l'oppression. L'inégalité pourrait facilement être adoucie par la résistance des faibles et l'esprit de justice des forts ; elle ne ferait pas surgir une nécessite plus brutale encore que celle des besoins naturels eux-mêmes, s'il n'intervenait pas un autre facteur, à savoir la lutte pour la puissance.\par
Comme Marx l'a compris clairement pour le capitalisme, comme quelques moralistes l'ont aperçu d'une manière plus générale, la puissance enferme une espèce de fatalité qui pèse aussi impitoyablement sur ceux qui commandent que sur ceux qui obéissent ; bien plus, c'est dans la mesure où elle asservit les premiers que, par leur intermédiaire, elle écrase les seconds. La lutte contre la nature comporte des nécessités inéluctables et que rien ne peut faire fléchir, mais ces nécessités enferment leurs propres limites ; la nature résiste, mais elle ne se défend pas, et là où elle est seule en jeu, chaque situation pose des obstacles bien définis qui donnent sa mesure à l'effort humain, Il en est tout autrement dès que les rapports entre hommes se substituent au contact direct de l'homme avec la nature. Conserver la puissance est, pour les puissants, une nécessité vitale, puisque c'est leur puissance qui les nourrit ; or ils ont à la conserver à la fois contre leurs rivaux et contre leurs inférieurs, lesquels ne peuvent pas ne pas chercher à se débarrasser de maîtres dangereux ; car, par un cercle sans issue, le maître est redoutable à l'esclave du fait même qu'il le redoute, et réciproquement ; et il en est de même entre puissances rivales.\par
Bien plus, les deux luttes que doit mener chaque homme puissant, l'une contre ceux sur qui il règne et l'autre contre ses rivaux, se mêlent inextricablement et sans cesse chacune rallume l'autre. Un pouvoir, quel qu'il soit, doit toujours tendre à s'affermir à l'intérieur au moyen de succès remportés au-dehors, car ces succès lui donnent des moyens de contrainte plus puissants ; de plus, la lutte contre ses rivaux rallie à sa suite ses propres esclaves, qui ont l'illusion d'être intéressés à l'issue du combat. Mais, pour obtenir de la part des esclaves l'obéissance et les sacrifices indispensables à un combat victorieux, le pouvoir doit se faire plus oppressif ; pour être en mesure d'exercer cette oppression, il est encore plus impérieusement contraint de se tourner vers l'extérieur ; et ainsi de suite. On peut parcourir la même chaîne en partant d'un autre chaînon, montrer qu'un groupement social pour être en mesure de se défendre contre les puissances extérieures qui voudraient se l'annexer, doit lui-même se soumettre a une autorité oppressive ; que le pouvoir ainsi établi, pour se Maintenir en place, doit attiser les conflits avec les pouvoirs rivaux ; et ainsi de suite, encore une fois. C'est ainsi que le plus funeste des cercles vicieux entraîne la société tout entière à la suite de ses maîtres, dans une ronde insensée.\par
On ne peut briser le cercle que de deux manières, ou en supprimant l'inégalité, ou en établissant un pouvoir stable, un pouvoir tel qu'il y ait équilibre entre ceux qui commandent et ceux qui obéissent. Cette seconde solution est celle qu'ont recherchée tous ceux que l'on nomme partisans de l'ordre, ou du moins tous ceux d'entre eux qui n'ont été mus ni par la servilité ni par l'ambition ; ce fut sans doute le cas des écrivains latins qui louèrent « l'immense majesté de la paix romaine », de Dante, de l'école réactionnaire du début du XIXe siècle, de Balzac, et, aujourd'hui, des hommes de droite sincères et réfléchis. Mais cette stabilité du pouvoir, objectif de ceux qui se disent réalistes, apparaît comme une chimère, si l'on y regarde de près, au même titre que l'utopie anarchiste.\par
Entre l'homme et la matière, chaque action, heureuse ou non, établit un équilibre qui ne peut être rompu que du dehors ; car la matière est inerte, Une pierre déplacée accepte sa place nouvelle ; le vent accepte de conduire a destination le même bateau qu'il aurait détourné de sa route si voile et gouvernail n'avaient été bien disposés. Mais les hommes sont des êtres essentiellement actifs, et possèdent une faculté de se déterminer eux-mêmes qu'ils ne peuvent jamais abdiquer, même s'ils le désirent, sinon le jour où ils retombent par la mort à l'état de matière inerte ; de sorte que toute victoire sur les hommes renferme en elle-même le germe d'une défaite possible, à moins d'aller jusqu'à l'extermination. Mais l'extermination supprime la, puissance en en supprimant l'objet. Ainsi il y a, dans l'essence même de la puissance, une contradiction fondamentale, qui l'empêche de jamais exister à proprement parler ; ceux qu'on nomme les maîtres, sans cesse contraints de renforcer leur pouvoir sous peine de se le voir ravir, ne sont jamais qu'à la poursuite d'une domination essentiellement impossible à posséder, poursuite dont les supplices infernaux de la mythologie grecque offrent de belles images. Il en serait autrement si un homme pouvait posséder en lui-même une force supérieure à celle de beaucoup d'autres réunis ; mais ce n'est jamais le cas ; les instruments du pouvoir, armes, or, machines, secrets magiques ou techniques, existent toujours en dehors de celui qui en dispose, et peuvent être pris par d'autres. Ainsi tout pouvoir est instable.\par
D'une manière générale, entre êtres humains, les rapports de domination et de soumission n'étant jamais pleinement acceptables, constituent toujours un déséquilibre sans remède et qui s'aggrave perpétuellement lui-même ; il en est ainsi même dans le domaine de la vie privée, où l'amour, par exemple, détruit tout équilibre dans l'âme dès qu'il cherche à s'asservir son objet ou à s'y asservir. Mais là du moins rien d'extérieur ne s'oppose à ce que la raison revienne tout mettre en ordre en établissant la liberté et l'égalité ; au lieu que les rapports sociaux, dans la mesure où les procédés mêmes du travail et du combat excluent l'égalité, semblent faire peser la folie sur les hommes comme une fatalité extérieure. Car du fait qu'il n'y a jamais pouvoir, mais seulement course au pouvoir, et que cette course est sans terme, sans limite, sans mesure, il n'y a pas non plus de limite ni de mesure aux efforts qu'elle exige ; ceux qui s'y livrent, contraints de faire toujours plus que leurs rivaux, qui s'efforcent de leur coté de faire plus qu'eux, doivent sacrifier non seulement l'existence des esclaves, mais la leur propre et celle des êtres les plus chers ; c'est ainsi qu'Agamemnon immolant sa fille revit dans les capitalistes qui, pour maintenir leurs privilèges, acceptent d'un coeur léger des guerres susceptibles de leur ravir leurs fils.\par
Ainsi la course au pouvoir asservit tout le monde, les puissants comme les faibles. Marx l'a bien vu en ce qui concerne le régime capitaliste. Rosa Luxembourg protestait contre l'apparence de « carrousel dans le vide » que présente le tableau marxiste de l'accumulation capitaliste, ce tableau où la consommation apparaît comme un « mal nécessaire » à réduire au minimum, un simple moyen pour maintenir en vie ceux qui se consacrent soit comme chefs soit comme ouvriers au but suprême, but qui n'est autre que la fabrication de l'outillage, c'est-à-dire des moyens de la production. Et pourtant c'est la profonde absurdité de ce tableau qui en fait la profonde vérité ; vérité qui déborde singulièrement le cadre du régime capitaliste. Le seul caractère propre a ce régime, c'est que les instruments de la production industrielle y sont en même temps les armes principales dans la course au pouvoir ; mais toujours les procédés de la course au pouvoir, quels qu'ils soient,, se soumettent les hommes par le même vertige et s'imposent a eux à titre de fins absolues. C'est le reflet de ce vertige qui donne une grandeur épique à des œuvres comme la {\itshape Comédie humaine, ou} les {\itshape Histories} de Shakespeare, ou les chansons de geste, ou {\itshape l'Iliade.} Le véritable sujet de {\itshape l'Iliade}, c'est l'emprise de la guerre sur les guerriers, et, par leur intermédiaire, sur tous les humains ; nul ne sait pourquoi chacun se sacrifie, et sacrifie tous les siens à une guerre meurtrière et sans objet, et c'est pourquoi, tout au long du poème, c'est aux dieux qu'est attribuée l'influence mystérieuse qui fait échec aux pourparlers de paix, rallume sans cesse les hostilités, ramène les combattants qu'un éclair de raison pousse à abandonner la lutte.\par
Ainsi dans cet antique et merveilleux poème apparaît déjà le mal essentiel de l'humanité, la substitution des moyens aux fins. Tantôt la guerre apparaît au premier plan, tantôt la recherche de la richesse, tantôt la production ; mais le mal reste le même. Les moralistes vulgaires se plaignent que l'homme soit mené par son intérêt personnel ; plût au ciel qu'il en fût ainsi ! L'intérêt est un principe d'action égoïste, mais borné, raisonnable, qui ne peut engendrer des maux illimités. La loi de toutes les activités qui dominent l'existence sociale, c'est au contraire, exception faite pour les sociétés primitives, que chacun y sacrifie la vie humaine, en soi et en autrui, à des choses qui ne constituent que des moyens de mieux vivre. Ce sacrifice revêt des formes diverses, mais tout se résume dans la question du pouvoir. Le pouvoir, par définition, ne constitue qu'un moyen ; ou pour mieux dire posséder un pouvoir, cela consiste simplement à posséder des moyens d'action qui dépassent la force si restreinte dont un individu dispose par lui-même. Mais la recherche du pouvoir, du fait même qu'elle est essentiellement impuissante à se saisir de son objet, exclut toute considération de fin, et en arrive, par un renversement inévitable, à tenir lieu de toutes les fins. C'est ce renversement du rapport entre le moyen et la fin, c'est cette folie fondamentale qui rend compte de tout ce qu'il y a d'insensé et de sanglant tout au long de l'histoire. L'histoire humaine n'est que l'histoire de l'asservissement qui fait des hommes, aussi bien oppresseurs qu'opprimés, le simple jouet des instruments de domination qu'ils ont fabriqués eux-mêmes, et ravale ainsi l'humanité vivante à être la chose de choses inertes.\par
Aussi ce ne sont pas les hommes, mais les choses qui donnent à cette course vertigineuse au pouvoir sa limite et ses lois. Les désirs des hommes sont impuissants à la régler. Les maîtres peuvent bien rêver de modération, mais il leur est interdit de pratiquer cette vertu, sous peine de défaite, sinon dans une très faible mesure ; aussi, en dehors d'exceptions quasi miraculeuses, telles que Marc-Aurèle, deviennent-ils rapidement incapables même de la concevoir. Quant aux opprimés, leur révolte permanente, qui bouillonne toujours bien qu'elle n'éclate que par moments, peut jouer de manière à aggraver le mal aussi bien que de manière à le restreindre ; et elle constitue surtout dans l'ensemble un facteur aggravant, du fait qu'elle contraint les maîtres à faire peser leur pouvoir toujours plus lourdement de crainte de le perdre. De temps en temps, les opprimés arrivent à chasser une équipe d'oppresseurs et à la remplacer par une autre, et parfois même à changer la forme de l'oppression ; mais quant à supprimer l'oppression elle-même, il faudrait à cet effet en supprimer les sources, abolir tous les monopoles, les secrets magiques ou techniques qui donnent prise sur la nature, les armements, la monnaie, la coordination des travaux. Quand les opprimés seraient assez conscients pour s'y déterminer, ils ne pourraient y réussir. Ce serait se condamner à être aussitôt asservis par les groupements sociaux qui n'ont pas opéré la même transformation ; et quand même ce danger serait écarté par miracle, ce serait se condamner à mort, car, quand on a une fois oublié les -procédés de la production primitive et transformé le milieu naturel auquel ils correspondaient, on ne peut retrouver le contact immédiat avec la nature. Ainsi, malgré toutes les velléités de mettre fin à la folie et a l'oppression, la concentration du pouvoir et l'aggravation de son caractère tyrannique n'auraient point de bornes s'il ne s'en trouvait heureusement dans la nature des choses. Il importe de déterminer sommairement quelles peuvent être ces bornes ; et à cet effet il faut garder présent à l'esprit que, si l'oppression est une nécessité de la vie sociale, cette nécessite n'a rien de providentieI. Ce n'est pas parce qu'elle devient nuisible à la production que l'oppression peut prendre fin ; la « révolte des force productrices », si naïvement invoquée par Trotsky comme un facteur de l'histoire, est une pure fiction. On se tromperait de même en supposant que l'oppression cesse d'être inéluctable dès que les forces productives sont assez développées pour pouvoir assurer à tous le bien-être et le loisir. Aristote admettait qu'il n'y aurait plus aucun obstacle à la suppression de l'esclavage si l'on pouvait faire assumer les travaux indispensables par des « esclaves mécaniques », et Marx, quand il a tenté d'anticiper sur l'avenir de l’espèce humaine, n'a fait que reprendre et développer cette conception. Elle serait juste si les hommes étaient conduits par la considération du bien-être ; mais, depuis l'époque de {\itshape l'Iliade} jusqu'à nos jours, les exigences insensées de la lutte pour le pouvoir ôtent même le loisir de songer au bien-être. L'élévation du rendement de l'effort humain demeurera impuissante à alléger le poids de cet effort aussi longtemps que la structure sociale impliquera le renversement du rapport entre le moyen et la fin, autrement dit aussi longtemps que les procédés du travail et du combat donneront à quelques-uns un pouvoir discrétionnaire sur les masses ; car les fatigues et les privations devenues inutiles dans la lutte contre la nature se trouveront absorbées par la guerre menée entre les hommes pour la défense ou la conquête des privilèges. Dès lors que la société est divisée en hommes qui ordonnent et hommes qui exécutent, toute la vie sociale est commandée par la lutte pour le pouvoir, et la lutte pour la subsistance n'intervient guère que comme un facteur, à vrai dire indispensable, de la première. La vue marxiste selon laquelle l'existence sociale est déterminée par les rapports entre l'homme et la nature établis par la production reste bien la seule base solide pour toute étude historique ; seulement ces rapports doivent être considérés d'abord en fonction du problème du pouvoir, les moyens de subsistance constituant simplement une donnée de ce problème. Cet ordre semble absurde, mais il ne fait que refléter l'absurdité essentielle qui est au cœur même de la vie sociale. Une étude scientifique de l'histoire serait donc une étude des actions et des réactions qui se produisent perpétuellement entre l'organisation du pouvoir et les procédés de la production ; car si le pouvoir dépend des conditions matérielles de la vie, il ne cesse jamais de transformer ces conditions elles-mêmes. Une telle étude dépasse actuellement de très loin nos possibilités ; mais, avant d'aborder la complexité infinie des faits, il est bon d'élaborer un schéma abstrait de ce jeu d'actions et de réactions, a peu près comme les astronomes ont dû inventer une sphère céleste imaginaire pour s'y reconnaître dans les mouvements et les positions des astres.\par
Il faut tenter tout d'abord de dresser une liste des nécessités inéluctables qui bornent toute espèce de pouvoir. En premier lieu, un pouvoir quelconque s'appuie sur des instruments qui ont dans chaque situation une portée déterminée. Ainsi on ne commande pas de la même manière au moyen de soldats armés de flèches, de lances et d'épées qu'au moyen d'avions et de bombes incendiaires ; la puissance de l'or dépend du rôle joué par les échanges dans la vie économique ; celle des secrets techniques est mesurée par la différence entre ce qu'on peut accomplir par leur moyen et ce qu'on peut accomplir sans eux ; et ainsi de suite. À vrai dire, il faut toujours faire entrer en ligne de compte dans ce bilan les ruses grâce auxquelles les puissants obtiennent par persuasion ce qu'ils sont hors d'état d'obtenir par contrainte, soit en mettant les opprimés dans une situation telle qu'ils aient ou croient avoir un intérêt immédiat à faire ce qu'on leur demande, soit en leur inspirant un fanatisme propre à leur faire accepter tous les sacrifices. En second lieu, comme le pouvoir qu'exerce réellement un être humain ne s'étend qu'à ce qui se trouve effectivement soumis à son contrôle, le pouvoir se heurte toujours aux bornes mêmes de la faculté de contrôle, lesquelles sont fort étroites. Car aucun esprit ne peut embrasser une masse d'idées à la fois ; aucun homme ne peut se trouver a la fois en plusieurs lieux ; et pour le maître comme pour l'esclave la journée n'a jamais que vingt-quatre heures. La collaboration constitue en apparence un remède à cet inconvénient ; mais comme elle n'est jamais complètement pure de rivalité, il en résulte des complications infinies. Les facultés d'examiner, de comparer, de peser, de décider, de combiner sont essentiellement individuelles, et par suite il en est aussi de même du pouvoir, dont l'exercice est inséparable de ces facultés ; le pouvoir collectif est une fiction, du moins en dernière analyse. Quant à la quantité d'affaires qui peuvent tomber sous le contrôle d'un seul homme, elle dépend dans une très large mesure de facteurs individuels tels que l'étendue et la rapidité de l'intelligence, la capacité de travail, la fermeté du caractère ; mais elle dépend également des conditions objectives du contrôle, rapidité plus ou moins grande des transports et des informations, simplicité ou complication des rouages du pouvoir. Enfin l'exercice d'un pouvoir quelconque a pour condition un excédent dans la production des subsistances, et un excédent assez considérable pour que tous ceux qui se consacrent, soit en qualité de maîtres, soit en qualité d'esclaves, à la lutte pour le pouvoir, puissent vivre. Il est clair que la mesure de cet excédent dépend du mode de production, et par suite aussi de l’organisation sociale. Voilà donc trois facteurs qui permettent de concevoir le pouvoir politique et social comme constituant à chaque instant quelque chose d'analogue à une force mesurable. Cependant, pour compléter le tableau, il faut tenir compte du fait que les hommes qui se trouvent en rapport, soit à titre de maîtres soit à titre d'esclaves, avec le phénomène du pouvoir sont inconscients de cette analogie. Les puissants, qu'ils soient prêtres, chefs militaires, rois ou capitalistes, croient toujours commander en vertu d'un droit divin ; et ceux qui leur sont soumis se sentent écrases par une puissance qui leur paraît divine ou diabolique, mais de toutes manières surnaturelle. Toute société oppressive est cimentée par cette religion du pouvoir, qui fausse tous les rapports sociaux en permettant aux puissants d'ordonner au delà de ce qu'ils peuvent imposer ; il n'en est autrement que dans les moments d'effervescence populaire, moments où au contraire tous, esclaves révoltés et maîtres menaces, oublient combien les chaînes de l'oppression sont lourdes et solides.\par
Ainsi une étude scientifique de l'histoire devrait commencer par analyser les réactions exercées à chaque instant par le pouvoir sur les conditions qui lui assignent objectivement ses bornes ; et une esquisse hypothétique du jeu de ces réactions est indispensable pour guider une telle analyse, d'ailleurs beaucoup trop difficile eu égard à nos possibilités actuelles. Certaines de ces réactions sont conscientes et voulues. Tout pouvoir, s'efforce consciemment, dans la mesure de ses moyens, mesure déterminée par l'organisation sociale, d'améliorer dans son propre domaine la production et le contrôle ; l'histoire en fournit maint exemple, depuis les pharaons jusqu'à nos jours, et c'est là-dessus que s'appuie la notion de despotisme éclairé. En revanche tout pouvoir s'efforce aussi, et toujours consciemment, de détruire chez ses rivaux les moyens de produire et d'administrer, et est de leur part l'objet d'une tentative analogue. Ainsi la lutte pour le pouvoir est à la fois constructrice et destructrice, et amène ou un progrès ou une décadence économique selon que la construction ou la destruction l'emporte ; et il est clair que dans une civilisation déterminée la destruction s'opérera dans une mesure d'autant plus grande qu'il sera plus difficile à un pouvoir de s'étendre sans se heurter à des pouvoirs rivaux de force à peu près égale. Mais les conséquences indirectes de l'exercice du pouvoir ont beaucoup plus d'importance que les efforts conscients des puissants. Tout pouvoir, du fait même qu'il s'exerce, étend jusqu'à la limite du possible les rapports sociaux sur lesquels il repose ; ainsi le pouvoir militaire multiplie les guerres, le capital commercial multiplie les échanges. Or il arrive parfois, par une sorte de hasard providentiel, que cette extension fait surgir, par un mécanisme quelconque, des ressources nouvelles rendant possible une nouvelle extension, et ainsi de suite, à peu près comme la nourriture renforce les corps vivants en pleine croissance et leur permet ainsi de conquérir plus de nourriture encore de manière à acquérir de plus grandes forces. Tous les régimes offrent des exemples de ces hasards providentiels ; car sans de tels hasards, aucune forme de pouvoir ne pourrait durer, de sorte que les pouvoirs qui en bénéficient sont seuls à subsister. Ainsi la guerre permettait aux Romains de ravir des esclaves, c'est-à-dire des travailleurs dans la force de l'âge dont d'autres avaient eu a nourrir l'enfance ; le profit tiré du travail des esclaves permettait de renforcer l'armée, et l'armée plus forte entreprenait des guerres plus vastes qui lui valaient un butin d'esclaves nouveau et plus considérable. De même les routes que les Romains construisaient à des fins militaires facilitaient par la suite l'administration et l'exploitation des provinces et contribuaient par conséquent à entretenir des ressources pour les guerres nouvelles. Si l'on passe aux temps modernes, on voit par exemple que l'extension des échanges a provoqué une division plus grande du travail, laquelle à son tour a rendu indispensable une plus grande circulation des marchandises ; de plus la productivité accrue qui en est résultée a fourni des ressources nouvelles qui ont pu se transformer en capital commercial et industriel. En ce qui concerne la grande industrie, il est clair que chaque progrès important du machinisme a créé à la fois des ressources, des instruments et un stimulant pour un progrès nouveau. De même c'est la technique de la grande industrie qui s'est trouvée fournir les moyens de contrôle et d'information indispensables à l'économie centralisée à laquelle la grande industrie aboutit fatalement, tels que le télégraphe, le téléphone, la presse quotidienne. On peut en dire autant des moyens de transport. On pourrait trouver tout au cours de l'histoire une immense quantité d'exemples analogues, portant sur les plus grands et sur les Plus petits aspects de la vie sociale. On peut définir la croissance d'un régime par le fait qu'il lui suffit de fonctionner pour susciter de nouvelles ressources lui permettant de fonctionner sur une plus grande échelle.\par
Ce phénomène de développement automatique est si frappant qu'on serait tenté d'imaginer qu'un régime heureusement constitué, si l'on peut s'exprimer ainsi, subsisterait et progresserait sans fin. C'est exactement là ce que le XIXe siècle, socialistes compris, s'est figuré concernant le régime de la grande industrie. Mais s'il est facile d'imaginer d'une manière vague un régime oppressif qu, ne connaîtrait jamais de décadence, il n'en est plus de même si l'on veut concevoir clairement et concrètement l'extension indéfinie d'un pouvoir déterminé. S'il pouvait étendre sans fin ses moyens de contrôle, il s'approcherait indéfiniment d'une limite qui serait comme l'équivalent de l'ubiquité ; s'il pouvait étendre sans fin ses ressources, tout se passerait comme si la nature environnante évoluait graduellement vers cette générosité sans réserve dont Adam et Eve bénéficiaient au paradis terrestre ; et enfin s'il pouvait étendre indéfiniment la portée de ses propres instruments - qu'il s'agisse d'armes, d'or, de secrets techniques, de machines ou d'autre chose - il tendrait à abolir cette corrélation qui, en liant indissolublement la notion de maître à celle d'esclave, établit entre maître et esclave un rapport de dépendance réciproque. On ne peut prouver que tout cela soit impossible ; mais il faut admettre que c'est impossible, ou bien se résoudre à penser l'histoire humaine comme un conte de fées. D'une manière générale, on ne peut considérer le monde où nous vivons comme soumis a des lois que si l'on admet que tout phénomène y est limité ; et c'est le cas aussi pour le phénomène du pouvoir, comme l'avait compris Platon. Si l'on veut considérer le pouvoir comme un phénomène concevable, il faut penser qu'il peut étendre les bases sur lesquelles il repose jusqu'à un certain point seulement, après quoi il se heurte comme à un mur infranchissable. Mais néanmoins il ne lui est pas loisible de s'arrêter ; l'aiguillon de la rivalité le contraint à aller plus loin et toujours plus loin, c'est-à-dire à dépasser les limites à l'intérieur desquelles il peut effectivement s'exercer. Il s'étend au-delà de ce qu'il peut contrôler ; il commande au delà de ce qu'il peut imposer ; il dépense au delà de ses propres ressources. Telle est la contradiction interne que tout régime oppressif porte en lui comme un germe de mort ; elle est constituée par l'opposition entre le caractère nécessairement limité des bases matérielles du pouvoir et le caractère nécessairement illimité de la course au pouvoir en tant que rapport entre les hommes.\par
Car dès qu'un pouvoir dépasse les limites qui lui sont imposées par la nature des choses, il rétrécit les bases sur lesquelles il s'appuie, il rend ces limites mêmes de plus en plus étroites. En s'étendant au delà de ce qu'il peut contrôler, il engendre un parasitisme, un gaspillage, un désordre qui, une fois apparus, s'accroissent automatiquement. En essayant de commander là même où il n'est pas en état de contraindre, il provoque des réactions qu'il ne peut ni prévoir ni régler. Enfin, en voulant étendre l'exploitation des opprimés au-delà de ce que permettent les ressources objectives, il épuise ces ressources elles-mêmes ; c'est là sans doute ce que signifie le conte antique et populaire de la poule aux oeufs d'or. Quelles que soient les sources d'où les exploiteurs tirent les biens qu'ils s'approprient, un moment vient où tel procédé d'exploitation, qui était d'abord, à mesure qu'il s'étendait, de plus en plus productif, se fait au contraire ensuite de plus en plus coûteux. C'est ainsi que l'armée romaine, qui avait d'abord enrichi Rome, finit par la ruiner ; c'est ainsi que les chevaliers du moyen âge, dont les combats avaient d'abord donné une sécurité relative aux paysans qui se trouvaient quelque peu protégés contre le brigandage, finirent au cours de leurs guerres continuelles par dévaster les campagnes qui les nourrissaient ; et le capitalisme semble bien traverser une phase de ce genre. Encore une fois, on ne peut prouver qu'il doive toujours en être ainsi ; mais il faut l'admettre, à moins de supposer la possibilité de ressources inépuisables. Ainsi c'est la nature même des choses qui constitue cette divinité justicière que les Grecs adoraient sous le nom de Némésis, et qui châtie la démesure.\par
Quand une forme déterminée de domination se trouve ainsi arrêtée dans son essor et acculée à la décadence, il s'en faut qu'elle commence à disparaître peu à peu ; parfois c'est alors au contraire qu'elle se fait le plus durement oppressive, qu'elle écrase les êtres humains sous son poids, qu'elle broie sans pitié corps, coeurs et esprits. Seulement comme tous se mettent peu à peu à manquer des ressources qu'il faudrait aux uns pour vaincre, aux autres pour vivre, un moment vient où, de toutes parts, on cherche fiévreusement des expédients. Il n'y a aucune raison pour qu'une telle recherche ne demeure pas vaine ; et en ce cas le régime ne peut que finir par sombrer faute de ressources pour subsister, et céder la place non pas à un autre régime mieux organise, mais à un désordre, à une misère, à une vie primitive qui durent jusqu'à ce qu'une cause quelconque fasse surgir de nouveaux rapports de force. S'il en est autrement, si la recherche de ressources nouvelles est fructueuse, de nouvelles formes de vie sociale surgissent et un changement de régime se prépare lentement et comme souterrainement. Souterrainement, car ces formes nouvelles ne peuvent se développer que pour autant qu'elles sont compatibles avec l'ordre établi et qu'elles ne présentent, tout au moins en apparence, aucun danger pour les pouvoirs constitués ; sans quoi rien ne pourrait empêcher ces pouvoirs de les anéantir, aussi longtemps qu'ils sont les plus forts. Pour que les nouvelles formes sociales l'emportent sur les anciennes, il faut qu'au préalable ce développement continu les ait amenées à jouer effectivement un rôle plus important dans le fonctionnement de l'organisme social, autrement dit qu'elles aient suscité des forces supérieures à celles dont disposent les pouvoirs officiels. Ainsi il n'y a jamais véritablement rupture de continuité, non pas même quand la transformation du régime semble l'effet d'une lutte sanglante ; car la victoire ne fait alors que consacrer des forces oui, dès avant la lutte, constituaient le facteur décisif de la vie collective, des formes sociales qui avaient commencé depuis longtemps à se substituer progressivement à celles sur lesquelles reposait le régime en décadence. C'est ainsi que, dans l'Empire romain, les barbares s'étaient mis à occuper les postes les plus importants, l'armée se disloquait peu a peu en bandes menées par des aventuriers et l'institution du colonat substituait progressivement le servage à l'esclavage, tout cela longtemps avant les grandes invasions. De même la bourgeoisie française n'a pas, il s'en faut, attendu 1789 pour l'emporter sur la noblesse. La révolution russe a, il est vrai, grâce à un singulier concours de circonstances, paru faire surgir quelque chose d'entièrement nouveau ; mais la vérité est que les privilèges supprimés par elle n'avaient depuis longtemps aucune base sociale en dehors de la tradition ; que les institutions surgies au cours de l'insurrection n'ont peut-être pas été effectivement en fonction l'espace d'un matin ; et que les forces réelles, à savoir la grande industrie, la police, l'armée, la bureaucratie, loin d'avoir été brisées par la révolution, sont parvenues grâce à elle a une puissance inconnue dans les autres pays. D'une manière générale ce renversement soudain du rapport des forces qui est ce qu'on entend d'ordinaire par révolution n'est pas seulement un phénomène inconnu dans l'histoire, c'est encore, si l'on y regarde de près, quelque chose à proprement parler d'inconcevable, car ce serait une victoire de la faiblesse sur la force, l'équivalent d'une balance dont le plateau le moins lourd s'abaisserait. Ce que l'histoire nous présente, ce sont de lentes transformations de régimes où les événements sanglants que nous baptisons révolutions jouent un rôle fort secondaire, et d'où ils peuvent même être absents ; c'est le cas lorsque la couche sociale qui dominait au nom des anciens rapports de force arrive à conserver une partie du pouvoir à la faveur des rapports nouveaux, et l'histoire d'Angleterre en fournit un exemple. Mais quelques formes que prennent les transformations sociales, l'on n'aperçoit, si l'on essaie d'en mettre à nu le mécanisme, qu'un morne jeu de forces aveugles qui s'unissent ou se heurtent, qui progressent ou déclinent, qui se substituent les unes aux autres, sans jamais cesser de broyer sous elles les malheureux humains. Ce sinistre engrenage ne présente à première vue aucun défaut par où une tentative de délivrance puisse trouver passage. Mais ce n'est pas d'une esquisse aussi vague, aussi abstraite, aussi misérablement sommaire que l'on peut prétendre tirer une conclusion.\par
Il faut poser encore une fois le problème fondamental, à savoir en quoi consiste le lien qui semble jusqu'ici unir l'oppression sociale et le progrès dans les rapports de l'homme avec la nature. Si l'on considère en gros l'ensemble du développement humain jusqu'à nos jours, si surtout l'on oppose les peuplades primitives, organisées presque sans inégalité, a notre civilisation actuelle, il semble que l'homme ne puisse parvenir à alléger le joug des nécessités naturelles sans alourdir d'autant celui de l'oppression sociale, comme par le jeu d'un mystérieux équilibre. Et même, chose plus singulière encore, on dirait que, si la collectivité humaine s'est dans une large mesure affranchie du poids dont les forces démesurées de la nature accablent la faible humanité, elle a en revanche pris en quelque sorte la succession de la nature au point d'écraser l'individu d'une manière analogue.\par
En quoi l'homme primitif est-il esclave ? C'est qu'il ne dispose presque pas de sa propre activité ; il est le jouet du besoin, qui lui dicte chacun de ses gestes, ou peu s'en faut, et le harcèle de son aiguillon impitoyable ; et ses actions sont réglées non pas par sa propre pensée, mais par les coutumes et les caprices également incompréhensibles d'une nature qu'il ne peut qu'adorer avec une aveugle soumission. Si l'on ne considère que la collectivité, les hommes semblent s'être élevés de nos jours à une condition qui se trouve aux antipodes de cet état servile. Presque aucun de leurs travaux ne constitue une simple réponse à l'impérieuse impulsion du besoin; le travail s'accomplit de manière a prendre possession de la nature et à l'aménager en sorte que les besoins se trouvent satisfaits. L'humanité ne se croit plus en présence de divinités capricieuses dont il faille se concilier la faveur ; elle sait qu'elle a simplement à manier de la matière inerte, et s'acquitte de cette tâche en se réglant méthodiquement sur des lois clairement conçues. Enfin il semble que nous soyons parvenus à cette époque prédite par Descartes où les hommes emploieraient « la force et les actions du feu, de l'eau, de l'air, des astres et de tous les autres corps » en même façon que les métiers des artisans, et se rendraient ainsi maîtres de la nature. Mais, par un renversement étrange, cette domination collective se transforme en asservissement dès que l'on descend à l'échelle de l'individu, et en un asservissement assez proche de celui que comporte la vie primitive. Les efforts du travailleur moderne lui sont imposés par une contrainte aussi brutale, aussi impitoyable et qui le serre d'aussi près que la faim serre de près le chasseur primitif ; depuis ce chasseur primitif jusqu'à l'ouvrier de nos grandes fabriques, en passant par les travailleurs égyptiens menés à coups de fouet, par les esclaves antiques, par les serfs du moyen age que menaçait constamment l'épée des seigneurs, les hommes n'ont jamais cessé d'être poussés au travail par une force extérieure et sous peine de mort presque immédiate. Et quant à l'enchaînement des mouvements du travail, il est souvent, lui aussi, impose du dehors à nos ouvriers tout comme aux hommes primitifs, et aussi mystérieux aux premiers qu'aux seconds ; bien plus, dans ce domaine, la contrainte est en certains cas sans comparaison plus brutale aujourd'hui qu'elle n'a jamais été ; si livré que pût être un homme primitif à la routine et aux tâtonnements aveugles, il pouvait au moins tenter de réfléchir, de combiner et d'innover à ses risques et périls, liberté dont un travailleur à la chaîne est absolument privé. Enfin si l'humanité semble parvenue à disposer de ces forces de la nature qui pourtant, selon la parole de Spinoza, « dépassent infiniment celles de l'homme », et cela presque aussi souverainement qu'un cavalier dispose de son cheval, cette victoire n'appartient pas aux hommes pris un à un ; seules les plus vastes collectivités sont en état de manier « la force et les actions du feu, de l'eau, de l'air... et de tous les autres corps qui nous entourent » ; quant aux membres de ces collectivités, oppresseurs et opprimés y sont pareillement soumis aux exigences implacables de la lutte pour le pouvoir.\par
Ainsi, en dépit du progrès, l'homme n'est pas sorti de la condition servile dans laquelle il se trouvait quand il était livré faible et nu à toutes les forces aveugles qui composent l'univers ; simplement la puissance qui le maintient sur les genoux a été comme transférée de la matière inerte à la société qu'il forme lui-même avec ses semblables. Aussi est-ce cette société qui est imposée à son adoration à travers toutes les formes que prend tour à tour le sentiment religieux. Dès lors la question sociale se pose sous une forme assez claire ; il faut examiner le mécanisme de ce transfert ; chercher pourquoi l'homme a dû payer à ce prix sa puissance sur la nature ; concevoir en quoi peut consister pour lui la situation la moins malheureuse, c'est-à-dire celle où il serait le moins asservi à la double domination de la nature et de la société ; enfin apercevoir quels chemins peuvent rapprocher d'une telle situation, et quels instruments pourrait fournir aux hommes d'aujourd'hui la civilisation actuelle s'ils aspiraient à transformer leur vie en ce sens.\par
Nous acceptons trop facilement le progrès matériel comme un don du ciel, comme une chose qui va de soi ; il faut regarder en face les conditions au prix desquelles il s'accomplit. La vie primitive est quelque chose d'aisément compréhensible ; l'homme est piqué par la faim, ou tout au moins par la pensée elle-même lancinante qu'il sera bientôt saisi par la faim, et il part en quête de nourriture ; il frissonne sous l'emprise du froid, ou du moins sous l'emprise de la pensée qu'il aura bientôt froid, et il cherche des choses bonnes à créer ou à conserver la chaleur ; et ainsi de suite. Quant à la manière de s'y prendre, elle lui est donnée tout d'abord par le pli, pris dès l'enfance, d'imiter les anciens, et aussi par les habitudes qu'il s'est lui-même données, au cours de multiples tâtonnements, en répétant les procédés qui ont réussi ; lorsqu'il est pris au dépourvu, il tâtonne encore, pousse qu'il est a agir par un aiguillon qui ne lui laisse point de répit. En tout cela, l'homme n'a qu'à céder a sa propre nature, et non à la vaincre.\par
Au contraire, dès qu'on passe à un stade plus avancé de la civilisation, tout devient miraculeux. On voit alors les hommes mettre de côté des choses bonnes à consommer, désirables, et dont cependant ils se privent. On les voit abandonner dans une large mesure la recherche de la nourriture, de la chaleur et du reste, et consacrer le meilleur de leurs forces à des travaux en apparence stériles. A vrai dire ces travaux, pour la plupart, loin d'être stériles, sont infiniment plus productifs que les efforts de l'homme primitif, car ils ont pour effet un aménagement de la nature extérieure dans un sens favorable a la vie humaine ; mais cette efficacité est indirecte, et souvent séparée de l'effort par tant d'intermédiaires que l'esprit a peine a les parcourir ; elle est à longue échéance, souvent à si longue échéance que seules les générations futures en profiteront ; alors qu'au contraire la fatigue exténuante, les douleurs, les dangers liés à ces travaux se font immédiatement et perpétuellement ressentir. Or chacun sait bien par sa propre expérience combien il est rare que l‘idée abstraite d'une utilité lointaine l'emporte sur les douleurs, les besoins, les désirs présents. Il faut pourtant qu'elle l'emporte dans l'existence sociale, sous peine de retour à la vie primitive.\par
Mais ce qui est plus miraculeux encore, c'est la coordination des travaux. Tout niveau un peu élevé de la production suppose une coopération plus ou moins étendue ; et la coopération se définit par le fait que les efforts de chacun n'ont de sens et d'efficacité que par leur rapport et leur exacte correspondance avec les efforts de tous les autres, de manière que tous les efforts forment un seul travail collectif. Autrement dit, les mouvements de plusieurs hommes doivent se combiner à la manière dont se combinent les mouvements d'un seul homme. Mais comment cela se peut-il ? Une combinaison ne s'opère que si elle est pensée ; or un rapport ne se forme jamais qu'a l'intérieur d'un esprit. Le nombre {\itshape deux} pensé par un homme ne peut s'ajouter au nombre {\itshape deux} pensé par un autre homme pour former le nombre {\itshape quatre ;} de même la conception qu'un des coopérateurs se fait du travail partiel qu'il accomplit ne peut se combiner avec la conception que chacun des autres se fait de sa tâche respective pour former un travail cohérent. Plusieurs esprits humains ne s'unissent point en un esprit collectif, et les termes d'âme collective. de pensée collective, si couramment employés de nos jours, sont tout à fait vides de sens. Dès lors, pour que les efforts de plusieurs se combinent, il faut qu'ils soient tous dirigés par un seul et même esprit, comme l'exprime le célèbre vers de {\itshape Faust} : « Un esprit suffit pour mille bras. »\par
Dans l'organisation égalitaire des peuplades primitives, rien ne permet de résoudre aucun de ces problèmes, ni celui de la privation, ni celui du stimulant de l'effort, ni celui de la coordination des travaux ; en revanche l'oppression sociale fournit une solution immédiate, en créant, pour dire la chose en gros, deux catégories d'hommes, ceux qui commandent et ceux qui obéissent. Le chef coordonne sans peine les efforts des hommes qui sont subordonnés à ses ordres ; il n'a aucune tentation à vaincre pour les réduire au strict nécessaire ; et quant au stimulant de l'effort, une organisation oppressive est admirablement propre à faire galoper les hommes au-delà des limites de leurs forées, les uns étant fouettés par l'ambition, les autres, selon la parole d'Homère, « sous la pression d'une dure nécessité ».\par
Les résultats sont souvent prodigieux lorsque la division des catégories sociales est assez profonde pour que ceux qui décident les travaux ne soient jamais exposés à en ressentir ou même à en connaître ni les peines épuisantes, ni les douleurs, ni les dangers, cependant que ceux qui exécutent et souffrent n'ont pas le choix, étant perpétuellement sous le coup d'une menace de mort plus ou moins déguisée. C'est ainsi que l'homme n'échappe dans une certaine mesure aux caprices d'une nature aveugle qu'en se livrant aux caprices non moins aveugles de la lutte pour le pouvoir. Cela n'est jamais plus vrai que lorsque l'homme arrive, comme c'est le cas pour nous, a une technique assez avancée pour avoir la maîtrise des forces de la nature ; car, pour qu'il puisse en être ainsi, la coopération doit s'accomplir a une échelle tellement vaste que les dirigeants se trouvent avoir en main une masse d'affaires qui dépasse formidablement leur capacité de contrôle. L'humanité se trouve de ce fait le jouet des forces de la nature, sous la nouvelle forme que leur a donnée le progrès technique, autant qu'elle l'a jamais été dans les temps primitifs ; nous en avons fait, nous en faisons, nous en ferons J'amère expérience. Quant aux tentatives pour conserver la technique en secouant l'oppression, elles suscitent aussitôt une telle paresse et un tel désordre que ceux qui s'y sont livres se trouvent le plus souvent contraints de remettre presque aussitôt la tête sous le joug ; l'expérience en a été faite sur une petite échelle dans des coopératives de production, sur une vaste échelle lors de la révolution russe. Il semblerait que l'homme naisse esclave, et que la servitude soit sa condition propre.
\subsection[TABLEAU THÉORIQUE D’UNE SOCIÉTÉ LIBRE.]{TABLEAU THÉORIQUE D’UNE SOCIÉTÉ LIBRE.}
\noindent Et pourtant rien au monde ne peut empêcher l'homme de se sentir ne pour la liberté. Jamais, quoi qu'il advienne, il ne peut accepter la servitude ; car il pense. Il n'a jamais cesse de rêver une liberté sans limites, soit comme un bonheur passe dont un châtiment l'aurait privé, soit comme un bonheur a venir qui lui serait dû par une sorte de pacte. avec une providence mystérieuse. Le communisme imaginé par Marx est la forme la plus récente de ce rêve. Ce rêve est toujours demeuré vain, comme tous les rêves, ou, s'il a pu consoler, ce n'est que comme un opium ; il est temps de renoncer à rêver la liberté, et de se décider à la concevoir.\par
C'est la liberté parfaite qu'il faut s'efforcer de se représenter clairement, non pas dans l'espoir d'y atteindre, mais dans l'espoir d'atteindre une liberté moins imparfaite que n'est notre condition actuelle ; car le meilleur n'est concevable que par le parfait. On ne peut se diriger que vers un idéal. L'idéal est tout aussi irréalisable que le rêve, mais, à la différence du rêve, il a rapport à la réalité ; il permet, à titre de limite, de ranger des situations ou réelles ou réalisables dans l'ordre de la moindre a la plus haute valeur. La liberté parfaite ne peut pas être conçue comme consistant simplement dans la disparition de cette nécessité dont nous subissons perpétuellement la pression ; tant que l'homme vivra, c'est-à-dire tant qu'il constituera un infime fragment de cet univers impitoyable, la pression de la nécessité ne se relâchera jamais un seul instant. Un état de choses où l'homme aurait autant de jouissances et aussi peu de fatigues qu'il lui plairait ne peut pas trouver place, sinon par fiction, dans le monde où nous vivons. La nature est, il est vrai, plus clémente ou plus sévère aux besoins humains, selon les climats et peut-être selon les époques ; mais attendre l'invention miraculeuse qui la rendrait clémente partout et une fois pour toutes, c'est à peu près aussi raisonnable que les espérances attachées autrefois à la date de l'an mille. Au reste, si l'on examine cette fiction de près, il n'apparaît même pas qu'elle vaille un regret. I1 suffit de tenir compte de la faiblesse humaine pour comprendre qu'une vie d'où la notion même du travail aurait à peu près disparu serait livrée aux passions et peut-être à la folie ; il n'y a pas de maîtrise de soi sans discipline, et il n'y a pas d'autre source de discipline pour l'homme que l'effort demandé par les obstacles extérieurs. Un peuple d'oisifs pourrait bien s'amuser à se donner des obstacles, s'exercer aux sciences, aux arts, aux jeux ; mais les efforts qui procèdent de la seule fantaisie ne constituent pas pour l'homme un moyen de dominer ses propres fantaisies. Ce sont les obstacles auxquels on se heurte et qu'il faut surmonter qui fournissent l'occasion de se vaincre soi-même. Même les activités en apparence les plus libres, science, art, sport, n'ont de valeur qu'autant qu'elles imitent l'exactitude, la rigueur, le scrupule propres aux travaux, et même les exagèrent. Sans le modèle que leur fournissent sans le savoir le laboureur, le forgeron, le marin qui travaillent comme il faut, pour employer cette expression d'une ambiguïté admirable, elles sombreraient dans le pur arbitraire. La seule liberté qu'on puisse attribuer à l’âge d'or, c'est celle dont jouiraient les petits enfants si les parents ne leur imposaient pas des règles ; elle n'est en réalité qu'une soumission inconditionnée au caprice. Le corps humain ne peut en aucun cas cesser de dépendre du puissant univers dans lequel il est pris ; quand même l'homme cesserait d'être soumis aux choses et aux autres hommes par les besoins et les dangers, il ne leur serait que plus complètement livré par les émotions qui le saisiraient continuellement aux entrailles et dont aucune activité régulière ne le défendrait plus. Si l'on devait entendre par liberté la simple absence de toute nécessité, ce mot serait vide de toute signification concrète ; mais il ne représenterait pas alors pour nous ce dont la privation ôte à la vie sa valeur.\par
On peut entendre par liberté autre chose que la possibilité d'obtenir sans effort ce qui plait. Il existe une conception bien différente de la liberté, une conception héroïque qui est celle de la sagesse commune. La liberté véritable ne se définit pas par un rapport entre le désir et la satisfaction, mais par un rapport entre la pensée et l'action ; serait tout à fait libre l'homme dont toutes les actions procéderaient d'un jugement préalable concernant la fin qu'il se propose et l’enchaînement des moyens propres a amener cette fin. Peu importe que les actions en elles-mêmes soient aisées ou douloureuses, et peu importe même qu'elles soient couronnées de succès ; la douleur et l'échec peuvent rendre l'homme malheureux, mais ne peuvent pas l'humilier aussi longtemps que c'est lui-même qui dispose de sa propre faculté d'agir. Et disposer de ses propres actions ne signifie nullement agir arbitrairement ; les actions arbitraires ne procèdent d'aucun jugement, et ne peuvent à proprement parler être appelées libres. Tout jugement porte sur une situation objective, et par suite sur un tissu de nécessités. L'homme vivant ne peut en aucun cas cesser d'être enserré de toutes parts par une nécessité absolument inflexible ; mais comme il pense, il a le choix entre céder aveuglement à l'aiguillon par lequel elle le pousse de l'extérieur, ou bien se conformer à la représentation intérieure qu'il s'en forge ; et c'est en quoi consiste l'opposition entre servitude et liberté. Les deux termes de cette opposition ne sont au reste que des limites idéales entre lesquelles se meut la vie humaine sans pouvoir jamais en atteindre aucune, sous peine de n'être plus la vie. Un homme serait complètement esclave si tous ses gestes procédaient d'une autre source que sa pensée, à savoir ou bien les réactions irraisonnées du corps, ou bien la pensée d'autrui ; l'homme primitif affamé dont tous les bonds sont provoqués par les spasmes qui tordent ses entrailles, l'esclave romain perpétuellement tendu vers les ordres d'un surveillant armé d'un fouet, l'ouvrier moderne qui travaille à la chaîne, approchent de cette condition misérable. Quant à la liberté complète, on peut en trouver un modèle abstrait dans un problème d'arithmétique ou de géométrie bien résolu ; car dans un problème tous les éléments de la solution sont donnés, et l'homme ne peut attendre de secours que de son propre jugement, seul capable d'établir entre ces éléments le rapport qui constitue par lui-même la solution cherchée. Les efforts et les victoires de la mathématique ne dépassent pas le cadre de la feuille de papier, royaume des signes et des dessins ; une vie entièrement libre serait celle où toutes les difficultés réelles se présenteraient comme des sortes de problèmes, où toutes les victoires seraient comme des solutions mises en action. Tous les éléments du succès seraient alors donnés, c'est-à-dire connus et maniables comme sont les signes du mathématicien ; pour obtenir le résultat voulu, il suffirait de mettre ces éléments en rapport grâce à la direction méthodique qu'imprimerait la pensée non plus à de simples traits de plume, mais à des mouvements effectifs et qui laisseraient leur marque dans le monde. Pour mieux dire, l'accomplissement de n'importe quel ouvrage consisterait en une combinaison d'efforts aussi consciente et aussi méthodique que peut l'être la combinaison de chiffres par laquelle s'opère la solution d'un problème lorsqu'elle procède de la réflexion. L'homme aurait alors constamment son propre sort en mains ; il forgerait à chaque moment les conditions de sa propre existence par un acte de la pensée. Le simple désir, il est vrai, ne le mènerait à rien, Il ne recevrait rien gratuitement ; et même les possibilités d'effort efficace seraient pour lui étroitement limitées. Mais le fait même de ne pouvoir rien obtenir sans avoir mis en action, pour le conquérir, toutes les puissances de la pensée et du corps permettrait à l'homme de s'arracher sans retour à l'emprise aveugle des passions. Une vue claire du possible et de l'impossible, du facile et du difficile, des peines qui séparent le projet de l'accomplissement efface seule les désirs insatiables et les craintes vaines ; de là et non d'ailleurs procèdent la tempérance et le courage, vertus sans lesquelles la vie n'est qu'un honteux délire. Au reste toute espèce de vertu a sa source dans la rencontre qui heurte la pensée humaine à une matière sans indulgence et sans perfidie. On ne peut rien concevoir de plus grand pour l'homme qu'un sort qui le mette directement aux prises avec la nécessité nue, sans qu'il ait rien à attendre que de soi, et tel que sa vie soit une perpétuelle création de lui-même par lui-même. L'homme est un être borné à qui il n'est pas donné d'être, comme le Dieu des théologiens, l'auteur direct de sa propre existence ; mais l'homme posséderait l'équivalent humain de cette puissance divine si les conditions matérielles qui lui permettent d'exister étaient exclusivement l'œuvre de sa pensée dirigeant l'effort de ses muscles. Telle serait la liberté véritable.\par
Cette liberté n'est qu'un idéal, et ne peut pas plus se trouver dans une situation réelle que la droite parfaite ne peut être tracée par le crayon. Mais cet idéal sera utile à concevoir si nous pouvons apercevoir en même temps ce qui nous sépare de lui, et quelles circonstances peuvent nous en éloigner ou nous en approcher. L'obstacle qui apparaît le premier est constitué par la complexité et l'étendue de ce monde auquel nous avons affaire, complexité et étendue qui dépassent infiniment la portée de notre esprit. Les difficultés de la vie réelle ne constituent pas des problèmes à notre mesure ; elles sont comme des problèmes dont les données seraient en quantité innombrable, car la matière est doublement indéfinie, eu égard et à l'extension et à la divisibilité. Aussi est-il impossible à un esprit humain de tenir compte de tous les facteurs dont dépend le succès de l'action en apparence la plus simple ; n'importe quelle situation laisse place à des hasards sans nombre, et les choses échappent à notre pensée comme des fluides qu'on voudrait prendre entre les doigts. Dès lors il semblerait que la pensée ne puisse s'exercer que sur de vaines combinaisons de signes, et que l'action doive se réduire au tâtonnement le plus aveugle. Mais en fait il n'en est pas ainsi. Certes nous ne pouvons jamais agir à coup sur ; mais cela n'importe pas tant qu'on pourrait le croire. Nous pouvons aisément supporter que les conséquences de nos actions dépendent de hasards incontrôlables ; ce qu'il nous faut à tout prix soustraire au hasard, ce sont nos actions elles-mêmes, et cela de manière à les soumettre à la direction de la pensée. Il suffit pour cela que l'homme puisse concevoir une chaîne d'intermédiaires unissant les mouvements dont il est capable aux résultats qu'il veut obtenir ; et il le peut souvent, grâce à la stabilité relative qui persiste, à travers les aveugles remous de l'univers, à l'échelle de l'organisme humain, et qui seule permet à cet organisme de subsister. Certes cette chaîne d'intermédiaires ne constitue jamais qu'un schéma abstrait ; quand on passe à l'exécution, des accidents peuvent a chaque instant intervenir pour déjouer les plans les mieux établis ; mais si l'intelligence a su élaborer clairement le plan abstrait de l'action à exécuter, cela veut dire qu'elle est arrivée non certes à éliminer le hasard, mais a lui faire une part circonscrite et limitée, et, pour ainsi dire, à 1e filtrer, en classant par apport à ce plan la masse indéfinie des accidents possibles en quelques séries bien déterminées.\par
Ainsi l'esprit est impuissant à se reconnaître dans les remous innombrables que forment en pleine mer le vent et l'eau ; mais si on place au milieu de ces remous un bateau dont voiles et gouvernail soient disposés de telle ou telle manière, on peut faire la liste des actions qu'ils peuvent lui faire subir. Tous les outils sont ainsi, d'une manière plus ou moins parfaite, comme des instruments à définir les hasards. L'homme pourrait de la sorte éliminer le hasard sinon autour de lui, du moins en lui-même ; cependant cela même est un idéal inaccessible. Le monde est trop fertile en situations dont la complexité nous dépasse pour que l'instinct, la routine, le tâtonnement, l'improvisation puissent jamais cesser de jouer un rôle dans nos travaux ; l'homme ne peut que restreindre de plus en plus ce rôle grâce aux progrès de la science et de la technique. Ce qui importe, c'est que ce rôle soit subordonné et n'empêche pas la méthode de constituer l'âme même du travail. Il faut aussi qu'il apparaisse comme provisoire, et que routine et tâtonnement soient toujours considérés non pas comme des principes d'action, mais comme des pis-aller destinés à combler les lacunes de la pensée méthodique ; c'est à quoi les hypothèses scientifiques nous aident puissamment, en nous faisant concevoir les phénomènes à moitié connus comme régis par des lois analogues à celles qui déterminent les phénomènes les plus clairement compris. Et même là où nous ne savons rien, nous pouvons encore supposer que de semblables lois s'appliquent ; cela suffit pour éliminer, à défaut de l'ignorance, le sentiment du mystère, et nous faire comprendre que nous vivons dans un monde où l'homme ne doit attendre de miracles que de soi.\par
Il est cependant une source de mystère que nous ne pouvons éliminer, et qui n'est autre que notre propre corps. L'extrême complexité des phénomènes vitaux peut peut-être être progressivement débrouillée, tout au moins dans une certaine mesure ; mais une ombre impénétrable enveloppera toujours le rapport immédiat qui lie nos pensées à nos mouvements. Dans ce domaine nous ne pouvons concevoir aucune nécessité, du fait même que nous ne pouvons pas déterminer des chaînons intermédiaires ; au reste la notion de nécessité, telle que la pensée humaine la forme, n'est proprement applicable qu'à la matière. On ne peut même trouver dans les phénomènes en question, à défaut d'une nécessite clairement concevable, une régularité même approximative. Parfois les réactions du corps vivant sont complètement étrangères à la pensée ; parfois, mais rarement, elles en exécutent simplement les ordres ; plus souvent elles accomplissent ce que l'âme a désiré sans que celle-ci y prenne aucune part ; souvent aussi elles accompagnent les vœux formés par l'âme sans y correspondre d'aucune manière ; d'autres fois encore elles précèdent les pensées. Aucun classement n'est possible. C'est pourquoi, lorsque les mouvements du corps vivant jouent le premier rôle dans la lutte contre la nature, la notion même de nécessité peut difficilement se former ; en cas de succès la nature semble obéir ou complaire immédiatement aux désirs, et, en cas d'échec, les repousser. C'est ce qui a lieu dans les actions accomplies ou sans instruments ou avec des instruments si bien adaptes aux membres vivants qu'ils ne font guère qu'en prolonger les mouvements naturels. On peut comprendre ainsi que les primitifs, malgré leur habileté extrême à accomplir tout ce dont ils ont besoin pour subsister, se représentent le rapport entre l'homme et le monde sous l'aspect non du travail, mais de la magie. Entre eux et le réseau de nécessités qui constitue la nature et définit les conditions réelles de l'existence s'interposent des lors comme un écran toutes sortes de caprices mystérieux à la merci desquels ils croient se trouver ; et si peu oppressive que puisse être la société qu'ils forment, ils n'en sont pas moins esclaves par rapport à ces caprices imaginaires, souvent interprétés d'ailleurs par des prêtres et des sorciers en chair et en os. Ces croyances survivent sous forme de superstitions, et, contrairement à ce que nous aimons à penser, aucun homme n'en est complètement dégagé ; mais leur emprise perd sa force à mesure que, dans la lutte contre la nature, le corps vivant passe au second plan et les instruments inertes au premier. C'est le cas lorsque les instruments, cessant de se modeler sur la structure de l'organisme humain, le contraignent au contraire à adapter ses mouvements à leur forme. Dès lors il n'y a plus aucune correspondance entre les gestes à exécuter et les passions ; la pensée doit se soustraire au désir et à la crainte, et s'appliquer uniquement à établir un rapport exact entre les mouvements imprimés aux instruments et le but poursuivi. La docilité du corps en pareil cas est une espèce de miracle, mais un miracle dont la pensée n'a pas à tenir compte ; le corps, rendu comme fluide par l'habitude, selon la belle expression de Hegel, fait simplement passer dans les instruments les mouvements conçus par l'esprit. L'attention se porte exclusivement sur les combinaisons formées par les mouvements de la matière inerte, et la notion de nécessité apparaît dans sa pureté, sans aucun mélange de magie. Par exemple, sur terre et porté par les désirs et les craintes qui meuvent ses jambes pour lui, l'homme se trouve souvent avoir passé d'un lieu à un autre sans savoir comment ; sur mer au contraire, comme les désirs et les craintes n'ont pas prise sur le bateau, il faut perpétuellement ruser et combiner, disposer voiles et gouvernail, transformer la poussée du vent par un enchaînement d'artifices qui ne peut être l'œuvre que d'une pensée lucide. On ne peut pas réduire entièrement le corps humain à ce rôle d'intermédiaire docile entre la pensée et les instruments, mais on peut l'y réduire de plus en plus ; c'est à quoi contribue chaque progrès de la technique.\par
Mais, par malheur, quand même on arriverait à soumettre strictement et jusque dans le détail tous les travaux sans exception à la pensée méthodique, un nouvel obstacle à la liberté surgirait aussitôt, à cause de la profonde différence de nature qui sépare la spéculation théorique et l'action. En réalité, il n'y a rien e commun entre la résolution d'un problème et l'exécution d'un travail même parfaitement méthodique, entre l'enchaînement des notions et l'enchaînement des mouvements. Celui qui s'attaque a une difficulté d'ordre théorique procède en allant du simple au complexe, du clair à l'obscur ; les mouvements du travailleur ne sont pas, eux, plus simples ou plus clairs les uns que les autres, mais simplement ceux qui précèdent sont la condition de ceux qui suivent. Par ailleurs la pensée rassemble le plus souvent ce que l'exécution doit séparer, ou sépare ce que l'exécution doit unir. C'est pourquoi, lorsqu'un travail quelconque présente à la pensée des difficultés non immédiatement surmontables, il est impossible d'unir l'examen de ces difficultés et l'exécution du travail ; l'esprit doit d'abord résoudre le problème théorique par ses procédés propres, et ensuite la solution peut être appliquée à l'action. On ne peut dire en pareil cas que l'action soit à proprement parler méthodique ; elle est conforme à la méthode, ce qui est bien différent. La différence est capitale ; car celui 'qui applique la méthode n'a pas besoin de la concevoir au moment où il l'applique. Bien plus, s'il s'agit de choses compliquées, il ne le peut, quand il l'aurait élaborée lui-même ; car l'attention, toujours contrainte de se porter sur le moment présent de l'exécution, ne peut guère embrasser en même temps l'enchaînement de rapports dont dépend l'ensemble de l'exécution. Dès lors ce qui est exécuté, ce n'est pas une pensée, c'est un schéma abstrait indiquant une suite de mouvements, et aussi peu pénétrable à l’esprit, au moment de l'exécution, qu'une recette due à la simple routine ou un rite magique. Par ailleurs une seule et même conception est applicable, avec ou sans modifications de détail, un nombre indéfini de fois ; car bien que la pensée embrasse d'un coup la série des applications possibles d'une méthode, l'homme n'est pas dispensé pour autant de les réaliser une par une toutes les fois que c'est nécessaire. Ainsi à un seul éclair de pensée correspond une quantité illimitée d'actions aveugles. Il va de soi que ceux qui reproduisent indéfiniment l'application de telle ou telle méthode de travail ne se sont souvent jamais donné la peine de la comprendre ; il arrive au reste fréquemment que chacun d'eux ne soit charge que d'une partie de l'exécution, toujours la même. Cependant que ses compagnons font le reste. Dès lors on se trouve en présence d'une situation paradoxale ; à savoir qu'il y a de la méthode dans les mouvements du travail, mais non pas dans la pensée du travailleur. On dirait que la méthode a transféré son siège de l'esprit dans la matière. C'est ce dont les machines automatiques offrent la plus frappante image. Du moment que la pensée qui a élaboré une méthode d'action n'a pas besoin d'intervenir dans l'exécution, on peut confier cette exécution à des morceaux de métal aussi bien et mieux qu'a des membres vivants ; et on se trouve ainsi devant le spectacle étrange de machines où la méthode s'est si parfaitement cristallisée en métal qu'il semble que ce soit elles qui pensent, et les hommes attachés à leur service qui soient réduits à l'état d'automates. Au reste cette opposition entre l'application et l'intelligence de la méthode se retrouve, absolument identique, dans le cadre même de la pure théorie. Pour prendre un exemple simple, il est tout à fait impossible, au moment où l'on fait une division difficile, d'avoir la théorie de la division présente à l'esprit ; et cela non seulement parce que cette théorie, qui repose sur le rapport de la division à la multiplication, est d'une certaine complexité, mais surtout parce qu'en exécutant chacune des opérations partielles au bout desquelles la division est accomplie, on oublie que les chiffres représentent tantôt des unités, tantôt des dizaines, tantôt des centaines. Les signes se combinent selon les lois des choses qu'ils signifient ; mais, faute de pouvoir conserver le rapport de signe à signifié perpétuellement présent à l'esprit, on les manie comme s'ils se combinaient d'après leurs propres lois ; et de ce fait les combinaisons deviennent inintelligibles, ce qui veut dire qu'elles s'accomplissent automatiquement. Le caractère machinal des opérations arithmétiques est illustré par l'existence de machines a compter ; mais un comptable aussi n'est pas autre chose qu'une machine à compter imparfaite et malheureuse. La mathématique ne progresse qu'en travaillant sur les signes, en élargissant leur signification, en créant des signes de signes ; ainsi les lettres courantes de l'algèbre représentent des quantités quelconques, ou même des opérations virtuelles, comme c'est le cas pour les valeurs négatives ; d'autres lettres représentent des fonctions algébriques, et ainsi de suite. Comme à chaque étage, si l'on peut ainsi parler, on en arrive inévitablement à perdre de vue le rapport de signe à signifié, les combinaisons de signes, bien que toujours rigoureusement méthodiques, deviennent bien vite impénétrables à la pensée. Il n'existe pas de machine algébrique satisfaisante, bien que plusieurs tentatives aient été faites dans ce sens ; mais les calculs algébriques n'en sont pas moins le plus souvent aussi automatiques que le travail du comptable. Ou pour mieux dire ils le sont plus, en ce sens qu'ils le sont, en quelque sorte, essentiellement. Après avoir fait une division, on peut toujours réfléchir sur elle, en rendant aux signes leur signification, jusqu'à ce qu'on ait compris le pourquoi de chaque partie de l'opération ; mais il n'en est pas de même en algèbre, où les signes, à force d'être maniés et combinés entre eux en tant que tels, finissent par faire preuve d'une efficacité dont leur signification ne rend pas compte. Tels sont, par exemple, les signes {\itshape e} et {\itshape i} ; en les maniant convenablement, on aplanit merveilleusement toutes sortes de difficultés, et notamment si on les combine d'une certaine manière avec π, on arrive à l'affirmation que la quadrature du cercle est impossible ; cependant aucun esprit au monde ne conçoit quel rapport les quantités, si on peut les nommer ainsi, que désignent ces lettres peuvent avoir avec le problème de la quadrature du cercle. Le calcul met les signes en rapport sur le papier, sans que les objets signifiés soient en rapport dans l'esprit ; de sorte que la question même de la signification des signes finit par ne plus rien vouloir dire. On se trouve ainsi avoir résolu un problème par une sorte de magie, sans que l'esprit ait mis en rapport les données et la solution. Dès lors là aussi, comme dans le cas de la machine automatique, la méthode semble avoir pour domaine les choses au lieu de la pensée ; seulement, en l'occurrence, les choses ne sont pas des morceaux de métal, mais des traits sur du papier blanc. C'est ainsi qu'un savant a pu dire : « Mon crayon en sait plus que moi. » Il va de soi que les mathématiques supérieures ne sont pas un pur produit de l'automatisme, et que la pensée et même le génie ont eu part et ont part à leur élaboration ; il en résulte un extraordinaire mélange d'opérations aveugles avec des éclairs de pensée ; mais là où la pensée ne domine pas tout, elle joue nécessairement un rôle subordonné. Et plus le progrès de la science accumule les combinaisons toutes faites de signes, plus la pensée est écrasée, impuissante à faire l'inventaire des notions qu'elle manie. Bien entendu, le rapport des formules ainsi élaborées avec les applications pratiques dont elles sont susceptibles est, lui aussi, souvent tout à fait impénétrable à la pensée, et, de ce fait, apparaît comme aussi fortuit que l'efficacité d'une formule magique. Le travail se trouve en pareil cas automatique pour ainsi dire à la deuxième puissance ; ce n'est pas seulement l'exécution, c'est aussi l'élaboration de la méthode de travail qui s'accomplit sans être dirigée par la pensée. On pourrait concevoir, à titre de limite abstraite, une civilisation où toute activité humaine, dans le domaine du travail comme dans celui de la spéculation théorique, serait soumise jusque dans le détail à une rigueur toute mathématique, et cela sans qu'aucun être humain comprenne quoi que ce soit a ce qu'il ferait ; la notion de nécessité serait alors absente de tous les esprits, et cela d'une manière tout autrement radicale que chez les peuplades primitives dont nos sociologues affirment qu'elles ignorent la logique.\par
Par opposition, le seul mode de production pleinement libre serait celui où la pensée méthodique se trouverait à l'œuvre tout au cours du travail. Les difficultés à vaincre devraient être si variées que jamais il ne fût possible d'appliquer des règles toutes faites ; non certes que le rôle des connaissances acquises doive être nul ; mais il faut que le travailleur soit obligé de toujours garder présente à l'esprit la conception directrice du travail qu'il exécute, de manière à pouvoir l'appliquer intelligemment à des cas particuliers toujours nouveaux. Une telle présence d'esprit a naturellement pour condition que cette fluidité du corps que produisent l'habitude et l'habileté atteigne un degré fort élevé. Il faut aussi que toutes les notions utilisées au cours du travail soient assez lumineuses pour pouvoir être évoquées tout entières en un clin d'œil ; il dépend de la souplesse plus ou moins grande de l'intelligence, mais plus encore de la voie plus ou moins directe par laquelle une notion s'est formée dans l'esprit, que la mémoire puisse conserver la notion elle-même ou seulement la formule qui lui servait d'enveloppe. Par ailleurs il va de soi que le degré de complication des difficultés à résoudre ne doit jamais être trop élevé, sous peine d'établir une coupure entre la pensée et l'action. Bien entendu un tel idéal ne pourra jamais être pleinement réalisable ; on ne peut pas éviter, dans la vie pratique, d'accomplir des actions qu'il soit impossible de comprendre au moment où on les accomplit, soit qu'il faille se fier à des règles toutes faites ou bien à l'instinct, au tâtonnement, à la routine. Mais on peut du moins élargir peu à peu le domaine du travail lucide, et cela peut-être indéfiniment. Il suffirait à cette fin que l'homme visât non plus à étendre indéfiniment ses connaissances et son pouvoir, mais plutôt à établir, aussi bien dans l'étude que dans le travail, un certain équilibre entre l'esprit et l'objet auquel l'esprit s'applique.\par
Mais il existe encore un autre facteur de servitude ; c'est pour chacun l'existence des autres hommes. Et même, à y bien regarder, c'est a proprement parler le seul facteur de servitude ; l'homme seul peut asservir l'homme. Les primitifs mêmes ne seraient pas esclaves de la nature s'ils n'y logeaient des êtres imaginaires analogues à l'homme, et dont les volontés sont d'ailleurs interprétées par des hommes. En ce cas comme dans tous les autres, c'est le monde extérieur qui est la source de la puissance ; mais si derrière les forces infinies de la nature il n'y avait pas, soit par fiction, soit en réalité, des volontés divines ou humaines, la nature pourrait briser l'homme et non pas l'humilier. La matière peut démentir les prévisions et ruiner les efforts, elle n'en demeure pas moins inerte, faite pour être conçue et maniée du dehors ; mais on ne peut jamais ni pénétrer ni manier du dehors la pensée humaine. Dans la mesure où le sort d'un homme dépend d'autres hommes, sa propre vie échappe non seulement à ses mains, mais aussi à son intelligence ; le jugement et la résolution n'ont plus rien à quoi s'appliquer ; au lieu de combiner et d'agir, il faut s'abaisser à supplier ou a menacer ; et l'âme tombe dans des gouffres sans fond de désir et de crainte, car il n'y a pas de limites aux satisfactions et aux souffrances qu'un homme peut recevoir des autres hommes. Cette dépendance avilissante n'est pas le fait des opprimés seuls, mais au même titre quoique de manières différentes, des opprimés et des puissants. Comme l'homme puissant ne vit que de ses esclaves, l'existence d'un monde inflexible lui échappe presque entièrement ; ses ordres lui paraissent contenir en eux-mêmes une efficacité mystérieuse ; il n'est jamais capable à proprement parler de vouloir, mais est en proie à des désirs auxquels jamais la vue claire de la nécessité ne vient apporter une limite. Comme il ne conçoit pas d'autre méthode d'action que de commander, quand il lui arrive, comme cela est inévitable, de commander en vain, il passe tout d'un coup du sentiment d'une puissance absolue au sentiment d'une impuissance radicale, ainsi qu'il arrive souvent dans les rêves ; et les craintes sont alors d'autant plus accablantes qu'il sent continuellement sur lui la menace de ses rivaux. Quant aux esclaves, ils sont, eux, continuellement aux prises avec la matière ; seulement leur sort dépend non de cette matière qu'ils brassent, mais de maîtres aux caprices desquels on ne peut assigner ni lois ni limites.\par
Mais ce serait encore peu de chose de dépendre d'êtres qui, bien qu'étrangers, sont du moins réels, et qu'on peut sinon pénétrer, du moins voir, entendre, deviner par analogie avec soi-même. En réalité, dans toutes les sociétés oppressives, un homme quelconque, à quelque rang qu'il se trouve, dépend non seulement de ceux qui sont placés au-dessus ou au-dessous de lui, mais avant tout du jeu même de la vie collective, jeu aveugle qui détermine seul les hiérarchies sociales ; et peu importe à cet égard que la puissance laisse apparaître son origine essentiellement collective ou bien semble loger dans certains individus déterminés comme la vertu dormitive dans l'opium. Or s'il y a au monde quelque chose d'absolument abstrait, d'absolument mystérieux, d'inaccessible aux sens et à la pensée, c'est la collectivité ; l'individu qui en est membre ne peut, semble-t-il, l'atteindre ni la saisir par aucune ruse, peser sur elle par aucun levier ; il se sent vis-à-vis d'elle de l'ordre de l'infiniment petit. Si les caprices d'un individu apparaissent à tous les autres comme arbitraires, les secousses de la vie collective semblent l'être à la deuxième puissance. Ainsi entre l'homme et cet univers qui lui est assigné par le sort comme l'unique matière de sa pensée et de son action, les rapports d'oppression et de servitude placent d'une manière permanente l'écran impénétrable de l'arbitraire humain. Quoi d'étonnant si au lieu d'idées on ne rencontre guère que des opinions, au lieu d'action une agitation aveugle ? On ne pourrait se représenter la possibilité d'un progrès quelconque au seul vrai sens de ce mot, c'est-à-dire un progrès dans l'ordre des valeurs humaines, que si l'on pouvait concevoir à titre de limite idéale une société qui armerait l'homme contre le monde sans l'en séparer.\par
Pas plus que l'homme n'est fait pour être le jouet d'une nature aveugle, il n'est fait pour être le jouet des collectivités aveugles qu'il forme avec ses semblables ; mais pour cesser d'être livré à la société aussi passivement qu’une goutte d'eau à la mer, il faudrait qu'il puisse et la connaître et agir sur elle. Dans tous les domaines, il est vrai, les forces collectives dépassent infiniment les forces individuelles ; ainsi l'on ne peut pas plus facilement concevoir un individu disposant même d'une portion de la vie collective qu'une ligne s'allongeant par l'addition d'un point. C'est la du moins l'apparence ; mais en réalité il y a une exception et une seule, à savoir le domaine de la pensée. En ce qui concerne la pensée, le rapport est retourné ; là l'individu dépasse la collectivité autant que quelque chose dépasse rien, car la pensée ne se forme que dans un esprit, se trouvant seul en face de lui-même; les collectivités ne pensent point. Il est vrai que la pensée ne constitue nullement une force par elle-même. Archimède a été tué, dit-on, par un soldat ivre ; et si on l'avait mis à tourner une meule sous le fouet d'un surveillant d'esclaves, il aurait tourné exactement de la même manière que l'homme le plus épais. Dans la mesure où la pensée plane au-dessus de la mêlée sociale, elle peut juger, mais non pas transformer. Toutes les forces sont matérielles ; l'expression de force spirituelle est essentiellement contradictoire ; la pensée ne peut être une force que dans la mesure où elle est matériellement indispensable. Pour exprimer la même idée sous un autre aspect, l'homme n'a rien d'essentiellement individuel, n'a rien qui lui soit absolument propre, si ce n'est la faculté de penser ; et cette société dont il dépend étroitement à chaque instant de son existence dépend eh retour quelque peu de lui dès le moment où elle a besoin qu'il pense. Car tout le reste peut être imposé du dehors par la force, y compris les mouvements du corps, mais rien au monde ne peut contraindre un homme à exercer sa puissance de pensée, ni lui soustraire le contrôle de sa propre pensée. Si l'on a besoin qu'un esclave pense, il vaut mieux lâcher le fouet ; sinon l'on a bien peu de chances d'obtenir des résultats de bonne qualité. Ainsi, si l'on veut former, d'une manière purement théorique, la conception d'une société où la vie collective serait soumise aux hommes considérés en tant qu'individus au lieu de se les soumettre, il faut se représenter une forme de vie matérielle dans laquelle n'interviendraient que des efforts exclusivement dirigés par la pensée claire, ce qui impliquerait que chaque travailleur ait lui-même à contrôler, sans se référer à aucune règle extérieure, non seulement l'adaptation de ses efforts avec l'ouvrage à produire, mais encore leur coordination avec les efforts de tous les autres membres de la collectivité. La technique devrait être de nature à mettre perpétuellement à l'oeuvre la réflexion méthodique ; l'analogie entre les techniques des différents travaux devrait être assez étroite et la culture technique assez étendue pour que chaque travailleur se fasse une idée nette de toutes les spécialités ; la coordination devrait s'établir d'une manière assez simple pour que chacun en ait perpétuellement une connaissance précise, en ce qui concerne la coopération des travailleurs aussi bien que les échanges des produits ; les collectivités ne seraient jamais assez étendues pour dépasser la portée d'un esprit humain ; la communauté des intérêts serait assez évidente pour effacer les rivalités ; et comme chaque individu serait en état de contrôler l'ensemble de la vie collective, celle-ci serait toujours conforme à la volonté générale. Les privilèges fondés sur l'échange des produits, les secrets de la production ou la coordination des travaux se trouveraient automatiquement abolis. La fonction de coordonner n'impliquerait plus aucune puissance, puisqu'un contrôle continuel exercé par chacun rendrait toute décision arbitraire impossible. D'une manière générale la dépendance des hommes les uns vis-à-vis des autres n'impliquerait plus que leur sort se trouve livré a l'arbitraire, et elle cesserait d'introduire dans la vie humaine quoi que ce soit de mystérieux, puisque chacun serait en état de contrôler l'activité de tous les autres en faisant appel à sa seule raison. Il n'y a qu’une seule et même raison pour tous les hommes ; ils ne deviennent étrangers et impénétrables les uns aux autres que lorsqu'ils s'en écartent ; ainsi une société où toute la vie matérielle aurait pour condition nécessaire et suffisante que chacun exerce sa raison pourrait être tout à fait transparente pour chaque esprit. Quant au stimulant nécessaire pour surmonter les fatigues, les douleurs et les dangers, chacun le trouverait dans le désir d'obtenir l'estime de ses compagnons, mais plus encore en lui-même ; pour les travaux qui sont des créations de l'esprit, la contrainte extérieure, devenue inutile et nuisible, est remplacée par une sorte de contrainte intérieure ; le spectacle de l'ouvrage inachevé attire l'homme libre aussi puissamment que le fouet pousse l'esclave. Une telle société serait seule une société d'hommes libres, égaux et frères. Les hommes seraient à vrai dire pris dans des liens collectifs, mais exclusivement en leur qualité d'hommes ; ils ne seraient jamais traités les uns par les autres comme des choses. Chacun verrait en chaque compagnon de travail un autre soi-même placé à un autre poste, et l'aimerait comme le veut la maxime évangélique. Ainsi l'on posséderait en plus de la liberté un bien plus précieux encore ; car si rien n'est plus odieux que l'humiliation et l'avilissement de l'homme par l'homme, rien n'est si beau ni si doux que l'amitié.\par
Ce tableau, considère en lui-même, est si possible plus éloigné encore des conditions réelles de la vie humaine que la fiction de l'âge d'or. Mais à la différence de cette fiction il peut servir, en tant qu'idéal, de point de repère pour l'analyse et l'appréciation des formes sociales réelles. Le tableau dune vie sociale absolument oppressive et soumettant tous les individus au jeu d'un mécanisme aveugle était lui aussi purement théorique ; l'analyse qui situerait une société par rapport a ces deux tableaux serrerait déjà de plus près la réalité, tout en demeurant très abstraite. Il apparaît ainsi une nouvelle méthode d'analyse sociale qui n'est pas celle de Marx, bien qu'elle parte, comme le voulait Marx, des rapports de production ; mais au lieu que Marx, dont la conception reste d'ailleurs peu précise sur ce point, semble avoir voulu ranger les modes de production en fonction du rendement, on les analyserait en fonction des rapports entre la pensée et l'action. Il va de soi qu'un tel point de vue n'implique nullement que l'humanité ait évolué, au cours de l'histoire, des formes les moins conscientes aux formes les plus conscientes de la production ; la notion de progrès est indispensable à quiconque cherche à forger d'avance l'avenir, mais elle ne peut qu'égarer l'esprit quand on étudie le passé. Il faut alors y substituer la notion d'une échelle des valeurs conçue en dehors du temps ; mais il n'est pas non plus possible de disposer les diverses formes sociales en série sur une telle échelle. Ce que l'on peut faire, c'est rapporter à une semblable échelle tel ou tel aspect de la vie sociale prise à une époque déterminée. Il est assez clair que les travaux diffèrent réellement entre eux par quelque chose qui ne se rapporte ni au bien-être, ni au loisir, ni à la sécurité, et qui pourtant tient au coeur de tout homme ; un pêcheur qui lutte, contre les flots et le vent sur son petit bateau, bien qu'il souffre du froid, de la fatigue, du manque de loisir et même de sommeil, du danger, d'un niveau de vie primitif, a un sort plus enviable que l'ouvrier qui travaille à la chaîne, pourtant mieux partagé sur presque tous ces points. C'est que son travail ressemble beaucoup plus au travail d'un homme libre, quoique la routine et l'improvisation aveugle y aient une part parfois assez large. L'artisan du moyen âge occupe lui aussi, de ce point de vue, une place assez honorable, bien que le « tour de main » qui joue un si grand rôle dans tous les travaux faits à la main soit dans une large mesure quelque chose d'aveugle ; quant à l'ouvrier pleinement qualifié formé par la technique des temps modernes, il est peut-être ce qui ressemble le plus au travailleur parfait. On trouve des différences analogies dans l'action collective ; une équipe de travailleurs à la chaîne surveillés par un contremaître est un triste spectacle, au lieu qu'il est beau de voir une poignée d'ouvriers du bâtiment, arrêtés par une difficulté, réfléchir chacun de son côté, indiquer divers moyens d'action, et appliquer unanimement la méthode conçue par l'un d'eux, lequel peut indifféremment avoir ou ne pas avoir une autorité officielle sur les autres. Dans de pareils moments l'image d'une collectivité libre apparaît presque pure. Quant au rapport entre la nature du travail et la condition du travailleur, il est évident, lui aussi, dès qu'on jette un coup d'œil sur l'histoire ou sur la société actuelle ; même les esclaves antiques étaient traités avec égards lorsqu'on les employait comme médecins ou comme pédagogues. Mais toutes ces remarques ne portent encore que sur des détails. Une méthode qui permettrait d'aboutir à des vues d'ensemble concernant les diverses organisations sociales en fonction des notions de servitude et de liberté serait plus précieuse.\par
Il faudrait tout d'abord dresser quelque chose comme une carte de la vie sociale, carte dans laquelle seraient indiqués les points où il est indispensable que la pensée s'exerce, et par suite, si l'on peut ainsi parler, les zones d'influence de l'individu sur la société. On peut distinguer trois manières dont la pensée peut intervenir dans la vie sociale ; elle peut élaborer des spéculations purement théoriques, dont des techniciens appliqueront ensuite les résultats ; elle peut s'exercer dans l'exécution ; elle peut s'exercer dans le commandement et la direction. Dans tous les cas, il ne s'agit que d'un exercice partiel et pour ainsi dire mutilé da pensée, puisque jamais l'esprit n'y embrasse pleinement son objet ; mais c'est assez pour que ceux qui sont obligés de penser lorsqu'ils s'acquittent de leur fonction sociale conservent mieux que les autres la forme humaine. Cela n'est pas vrai seulement des opprimés, mais de tous les degrés de l'échelle sociale. Dans une société fondée sur l'oppression, ce ne sont pas seulement les faibles, mais aussi les plus puissants qui sont asservis aux exigences aveugles de la vie collective, et il y a amoindrissement du cœur et de l'esprit chez les uns comme chez les autres, bien que de manière différente ; or si l'on oppose deux couches sociales oppressives telles que, par exemple, les citoyens d'Athènes et la bureaucratie soviétique, on trouve une distance au moins aussi grande qu'entre un de nos ouvriers qualifiés et un esclave grec. Quant aux conditions selon lesquelles la pensée a plus ou moins part a l'exercice du pouvoir, il serait facile de les établir d'après le degré de complication et d'étendue des affaires, le caractère général des difficultés à résoudre et la répartition des fonctions. Ainsi les membres d'une société oppressive ne se distinguent pas seulement d'après le lieu plus élevé ou plus bas où ils se trouvent accrochés au mécanisme social, mais aussi par le caractère plus conscient ou plus passif - de leurs rapports avec lui, et cette seconde distinction, plus importante que la première, est sans lien direct avec elle. Quant à l'influence que les hommes chargés de fonctions sociales soumises à la direction de leur propre intelligence peuvent exercer sur la société dont ils font partie, elle dépend, bien entendu, de la nature et de l'importance de ces fonctions ; il serait fort intéressant de poursuivre l'analyse jusqu'au détail sur ce point, mais aussi fort difficile. Un autre facteur très important des relations entre l'oppression sociale et les individus est constitué par les facultés de contrôle plus ou moins étendues que peuvent exercer, sur les diverses fonctions qui consistent essentiellement à coordonner, les hommes qui n'en sont pas investis ; il est clair que plus ces fonctions échappent au contrôle, plus la vie collective est écrasante pour l'ensemble des individus. Il faut enfin tenir compte du caractère des liens qui maintiennent l'individu dans la dépendance matérielle de la société qui l'entoure ; ces liens sont tantôt plus lâches et tantôt plus étroits, et il peut s'y trouver des différences considérables, selon qu'un homme est plus ou moins contraint, à chaque moment de son existence, de se tourner vers autrui pour avoir les moyens de consommer, les moyens de produire, et se préserver des périls. Par exemple un ouvrier qui possède un jardin assez grand pour l'approvisionner en légumes est plus indépendant que ceux de ses camarades qui doivent demander toute leur nourriture aux marchands ; un artisan qui possède ses outils est plus indépendant qu'un ouvrier d'usine dont les mains deviennent inutiles lorsqu'il plaît au patron de lui retirer l'usage de sa machine. Quant à la défense contre les dangers, la situation de l'individu à cet égard dépend du mode de combat que pratique la société où il se trouve ; là où le combat est le monopole des membres d'une certaine couche sociale, la sécurité de tous les autres dépend de ces privilégiés ; là où la puissance des armements et le caractère collectif du combat donnent le monopole de la force militaire au pouvoir central, celui-ci dispose à son gré de la sécurité des citoyens. En résumé la société la moins mauvaise est celle où le commun des hommes se trouve le plus souvent dans l'obligation de penser en agissant, a les plus grandes possibilités de contrôle sur l'ensemble de la vie collective et possède le plus d'indépendance. Au reste les conditions nécessaires pour diminuer le poids oppressif du mécanisme social se contrarient les unes les autres dès que certaines limites sont dépassées ; ainsi il ne s'agit pas de s'avancer aussi loin que possible dans une direction déterminée, mais, ce qui est beaucoup plus difficile, de trouver un certain équilibre optimum.\par
La conception purement négative d'un affaiblissement de l'oppression sociale ne peut par elle-même donner un objectif aux gens de bonne volonté. Il est indispensable de se faire au moins une représentation vague de la civilisation à laquelle on souhaite que l'humanité parvienne ; et peu importe que cette représentation tienne plus de la simple rêverie que de la pensée véritable. Si les analyses précédentes sont correctes, la civilisation la plus pleinement humaine serait celle qui aurait le travail manuel pour centre, celle où le travail manuel constituerait la suprême valeur. Il ne s'agit de rien de pareil à la religion de la production qui régnait en Amérique pendant la période de prospérité, qui règne en Russie depuis le plan quinquennal; car cette religion a pour objet véritable les produits du travail et non le travailleur, les choses et non l'homme. Ce n'est pas par son rapport avec ce qu'il produit que le travail manuel doit devenir la valeur la plus haute, mais par son rapport avec l'homme qui l'exécute ; il ne doit pas être l'objet d'honneurs ou de récompenses, mais constituer pour chaque être humain ce dont il a besoin le plus essentiellement pour que sa vie prenne par elle-même un sens et une valeur à ses propres yeux. Même de nos jours, les activités qu'on nomme désintéressées, sport ou même art ou même pensée, n'arrivent peut-être pas à donner l’équivalent de ce que l'on éprouve à se mettre directement aux prises avec le monde par un travail non machinal. Rimbaud se plaignait que « nous ne sommes pas au monde » et que « la vraie vie est absente » ; en ces moments de joie et de plénitude incomparables on sait par éclairs que la vraie vie est là, on éprouve par tout son être que le monde existe et que l'on est au monde. Même la fatigue physique n'arrive pas à diminuer la puissance de ce sentiment, mais plutôt, tant qu'elle n'est pas excessive, elle l'augmente. S'il en peut être ainsi à notre époque, quelle merveilleuse plénitude de vie ne pourrait-on pas attendre d'une civilisation où le travail serait assez transformé pour exercer pleinement toutes les facultés, pour constituer l'acte humain par excellence ? Il devrait alors se trouver au centre même de la culture. Naguère la culture était considérée par beaucoup comme une fin en soi, et de nos jours ceux qui y voient plus qu'une simple distraction y cherchent d'ordinaire un moyen de s'évader de la vie réelle. Sa valeur véritable consisterait au contraire à préparer à la vie réelle, à armer l'homme pour qu'il puisse entretenir, avec cet univers qui est son partage et avec ses frères dont la condition est identique à la sienne, des rapports dignes de la grandeur humaine. La science est aujourd'hui regardée par les uns comme un simple catalogue de recettes techniques, par les autres comme un ensemble de pures spéculations de l'esprit qui se suffisent à elles-mêmes ; les premiers font trop peu de cas de l'esprit et les seconds du monde. La pensée est bien la suprême dignité de l'homme ; mais elle s'exerce à vide, et par suite ne s'exerce qu'en apparence, lorsqu'elle ne saisit pas son objet, lequel ne peut être que l'univers. Or ce qui procure aux spéculations abstraites des savants ce rapport avec l'univers qui seul leur donne une valeur concrète, c'est le fait qu'elles sont directement ou, indirectement applicables. De nos jours, il est vrai, leurs propres applications leur demeurent étrangères ; ceux qui élaborent ou étudient ces spéculations le font sans penser à leur valeur théorique. Du moins il en est le plus souvent ainsi. Le jour où il serait impossible de comprendre les notions scientifiques, même les plus abstraites, sans apercevoir clairement, du même coup, leur rapport avec des applications possibles, et également impossible d'appliquer même indirectement ces notions sans les connaître et les comprendre à fond, la science serait devenue concrète et le travail conscient ; et alors seulement l'une et l'autre auront leur pleine valeur jusque-là science et travail auront toujours quelque chose d'incomplet et d'inhumain. Ceux qui ont dit jusqu'ici que les applications sont le but de la science voulaient dire que la vérité ne vaut pas la peine d'être cherchée et que le succès seul importe ; mais on pourrait l'entendre autrement ; on peut concevoir une science qui se proposerait comme fin dernière de perfectionner la technique non pas en la rendant plus puissante, mais simplement plus consciente et plus méthodique. Au reste le rendement pourrait bien progresser en même temps que la lucidité ; « cherchez d'abord le royaume des cieux et tout le reste vous sera donné par surcroît ». Une telle science serait en somme une méthode pour maîtriser la nature, ou un catalogue des notions indispensables pour arriver à cette maîtrise, disposées selon un ordre qui les rende transparentes à l'esprit. C'est sans doute ainsi que Descartes a conçu la science. Quant à l'art d'une semblable civilisation, il cristalliserait dans des oeuvres l'expression de cet heureux équilibre entre l'esprit et le corps, entre l'homme et l'univers, qui ne peut exister en acte que dans les formes les plus nobles du travail physique ; au reste, même dans le passé, les œuvres d'art les plus pures ont toujours exprimé le sentiment, ou, pour parler d'une manière peut-être plus exacte, le pressentiment d'un tel équilibre. Le sport aurait pour fin essentielle de donner au corps humain cette souplesse et, comme dit Hegel, cette fluidité qui le rend pénétrable à la pensée et permet à celle-ci d'entrer directement en contact avec les choses. Les rapports sociaux seraient directement modelés sur l'organisation du travail ; les hommes se grouperaient en petites collectivités travailleuses, où la coopération serait la loi suprême, et où chacun pourrait clairement comprendre et contrôler le rapport des règles auxquelles sa vie serait soumise avec l'intérêt général. Au reste chaque moment de l'existence apporterait à chacun l'occasion de comprendre et d'éprouver combien tous les hommes sont profondément un, puisqu'ils ont tous a mettre aux prises une même raison avec des obstacles analogues ; et tous les rapports humains, depuis les plus superficiels jusqu'aux plus tendres,, auraient quelque chose de cette fraternité virile qui unit les compagnons de travail.\par
Sans doute c'est là une pure utopie. Mais décrire même sommairement un état de choses qui serait meilleur que ce qui est, c'est toujours bâtir une utopie ; pourtant rien n'est plus nécessaire à la vie que des descriptions semblables, pourvu qu'elles soient toujours dictées par la raison. Toute la pensée moderne depuis la Renaissance est d'ailleurs imprégnée d'aspirations plus ou moins vagues vers cette civilisation utopique ; on a même pu croire quelque temps que c'était cette civilisation qui se formait, et qu'on entrait dans l'époque où la géométrie grecque descendrait sur la terre. Descartes l'a certainement cru, ainsi que quelques-uns de ses contemporains. Au reste la notion du travail considéré comme une valeur humaine est sans doute l'unique conquête spirituelle qu'ait fait la pensée humaine depuis le miracle grec ; c'était peut-être là la seule lacune à l'idéal de vie humaine que la Grèce a élaboré et qu'elle a laissé après elle comme un héritage impérissable. Bacon est le premier qui ait fait apparaître cette notion. À l'antique et désespérante malédiction de la Genèse, qui faisait apparaître le monde comme un bagne et le travail comme la marque de l'esclavage et de l'abjection des hommes, il a substitué dans un éclair de génie la véritable charte des rapports de l'homme avec le monde : « L'homme commande à la nature en lui obéissant. » Cette formule si simple devrait constituer à elle seule la Bible de notre époque. Elle suffit pour définir le travail véritable, celui qui fait les hommes libres, et cela dans la mesure même où il est un acte de soumission consciente à la nécessité. Après Descartes, les savants ont progressivement glissé à considérer la science pure comme un but en soi ; mais l'idéal d'une vie consacrée à une forme libre de labeur physique a commencé en revanche à apparaître aux écrivains ; et même il domine l'oeuvre maîtresse du poète généralement considéré comme le plus aristocratique de tous, à savoir Goethe. Faust, symbole de l'âme humaine dans sa poursuite inlassable du bien, abandonne avec dégoût la recherche abstraite de la vérité, devenue à ses yeux un jeu vide et stérile ; l'amour ne le conduit qu'à détruire l'être aimé ; la puissance politique et militaire se révèle comme un pur jeu d'apparences ; la rencontre de la beauté le comble, mais seulement l'espace d'un éclair ; la situation de chef d'entreprise lui donne un pouvoir qu'il croit substantiel, mais qui le livre néanmoins à la tyrannie des passions. Il aspire enfin à être dépouillé de sa puissance magique, qu'on peut considérer comme le symbole de toute espèce de puissance ; il s'écrie : « Si je me tenais devant toi, Nature, seulement en ma qualité d'homme, cela vaudrait alors la peine d'être une créature humaine » ; et il finit par atteindre, au moment de mourir, le pressentiment du bonheur le plus plein, en se représentant une vie qui s'écoulerait librement parmi un peuple libre et qu'occuperait tout entière un labeur physique pénible et dangereux, mais accompli au milieu d'une fraternelle coopération. Il serait facile de citer encore d'autres noms illustres, parmi lesquels Rousseau, Shelley et surtout Tolstoï, qui a développé ce thème tout au long de son œuvre avec un accent incomparable. Quant au mouvement ouvrier, toutes les fois qu'il a su échapper à la démagogie, c'est sur la dignité du travail qu'il a fondé les revendications des travailleurs. Proudhon osait écrire : « Le génie du plus simple artisan l'emporte autant sur les matériaux qu'il exploite que l'esprit d'un Newton sur les sphères inertes dont il calcule les distances, les masses et les révolutions » ; Marx, dont l'œuvre enferme bien des contradictions, donnait comme caractéristique essentielle de l'homme, par opposition avec les animaux, le fait qu'il produit les conditions de sa propre existence et ainsi se produit indirectement lui-même. Les syndicalistes révolutionnaires, qui mettent au centre de la question sociale la dignité du producteur considéré comme tel, se rattachent au même courant. Dans l'ensemble, nous pouvons avoir la fierté d'appartenir a une civilisation qui a apporté avec elle le pressentiment d'un idéal nouveau.
\subsection[ESQUISSE DE LA VIE SOCIALE. CONTEMPORAINE]{ESQUISSE DE LA VIE SOCIALE \\
CONTEMPORAINE}
\noindent Il est impossible de concevoir quoi que ce soit de plus contraire à cet idéal que la forme qu'a prise de nos jours la civilisation moderne, au terme d'une évolution de plusieurs siècles. Jamais l'individu n'a été aussi complètement livré à une collectivité aveugle, et jamais les hommes n'ont été plus incapables non seulement de soumettre leurs actions à leurs pensées, mais même de penser. Les termes d'oppresseurs et d'opprimés, la notion de classes, tout cela est bien près de perdre toute signification, tant sont évidentes l'impuissance et l'angoisse de tous les hommes devant la machine sociale, devenue une machine à briser les cœurs, à écraser les esprits, une machine à fabriquer de l'inconscience, de la sottise, de la corruption, de la veulerie, et surtout du vertige. La cause de ce douloureux état de choses est bien claire. Nous vivons dans un monde où rien n'est à la mesure de l'homme ; il y a une disproportion monstrueuse entre le corps de l'homme, l'esprit de l'homme et les choses qui constituent actuellement les éléments de la vie humaine ; tout est déséquilibre. Il n'existe pas de catégorie, de groupe ou de classe d'hommes qui échappe tout à fait à ce déséquilibre dévorant, à l'exception peut-être de quelques îlots de vie plus primitive ; et les jeunes, qui y ont grandi, qui y grandissent, reflètent plus que les autres à l'intérieur d'eux-mêmes le chaos qui les entoure. Ce déséquilibre est essentiellement une affaire de quantité. La quantité se change en qualité, comme 1’a dit Hegel, et en particulier une simple différence de quantité suffit à transporter du domaine de l'humain au domaine de l'inhumain. Abstraitement les quantités sont indifférentes, puisqu'on peut changer arbitrairement l'unité de mesure ; mais concrètement certaines unités de mesure sont données et sont demeurées jusqu'ici invariables, par exemple le corps humain, la vie humaine, l'année, la journée, la rapidité moyenne de la pensée humaine. La vie actuelle n'est pas organisée à la mesure de toutes ces choses ; elle s'est transportée dans un tout autre ordre de grandeurs, comme si l'homme s'efforçait de l'élever au niveau des forces de la nature extérieure en négligeant de tenir compte de sa nature propre. Si l'on ajoute que, selon toute apparence, le régime économique a épuisé sa capacité de construction et commence à ne pouvoir fonctionner qu'en sapant peu à peu ses bases matérielles, on apercevra dans toute sa simplicité l'essence véritable de la misère sans fond qui constitue le lot des générations présentes. En apparence presque tout s'accomplit de nos jours méthodiquement ; la science est reine, le machinisme envahit peu à peu tout le domaine du travail, les statistiques prennent une importance croissante, et, sur un sixième du globe, le pouvoir central tente de régler l'ensemble de la vie sociale d'après des plans. Mais en réalité l'esprit méthodique disparaît progressivement, du fait que la pensée trouve de moins en moins où mordre. Les mathématiques constituent a elles seules un ensemble trop vaste et trop complexe pour pouvoir être embrassé par un esprit ; à plus forte raison le tout forme par les mathématiques et les sciences de la nature ; à plus forte raison le tout formé par la science et ses applications ; et d'autre part tout est trop étroitement lié pour que la pensée puisse véritablement saisir des notions partielles. Or tout ce que l'individu devient impuissant à dominer, la collectivité s'en empare. C'est ainsi que la science est depuis longtemps déjà et dans une mesure de plus en plus large une oeuvre collective. À vrai dire les résultats nouveaux sont toujours en fait l'oeuvre d'hommes déterminés ; mais, sauf peut-être de rares exceptions, la valeur d'un résultat quelconque dépend d'un ensemble si complexe de rapports avec les découvertes passées et avec les recherches possibles que l'esprit même de l'inventeur ne peut en faire le tour. Ainsi les clartés, en s'accumulant, font figure d'énigmes, à la manière d'un verre trop épais qui cesse d'être transparent. À plus forte raison la vie pratique prend un caractère de plus en plus collectif, et l'individu comme tel y est de plus en plus insignifiant. Les progrès de la technique et la production en série réduisent de plus en plus les ouvriers à un rôle passif ; ils en arrivent dans une proportion croissante et dans une mesure de plus en plus grande à une forme de travail qui leur permet d'accomplir les gestes nécessaires sans en concevoir le rapport avec le résultat final. D'autre part une entreprise est devenue quelque chose de trop vaste et de trop complexe pour qu'un homme puisse pleinement s'y reconnaître ; et d'ailleurs, dans tous les domaines, tous les hommes qui se trouvent aux postes importants de la vie sociale sont chargés d'affaires qui dépassent considérablement la portée d'un esprit humain. Quant à l'ensemble de la vie sociale, elle dépend de tant de facteurs dont chacun est impénétrablement obscur et qui se mêlent en des rapports inextricables que personne n'aurait même l'idée de chercher à en concevoir le mécanisme. Ainsi la fonction sociale la plus essentiellement attachée à l'individu, celle qui consiste à coordonner, à diriger, à décider, dépasse les capacités individuelles et devient dans une certaine mesure collective et comme anonyme.\par
Dans la mesure même où ce qu'il y a de systématique dans la vie contemporaine échappe à l'emprise de la pensée, la régularité y est établie par des choses qui Constituent l'équivalent de ce que serait la pensée collective si la collectivité pensait. La cohésion de la science est assurée par des signes ; à savoir d'une part par des mots ou des expressions toutes faites qu'on utilise au delà de ce que comporteraient les notions qui y étaient primitivement renfermées, d'autre part par les calculs algébriques. Dans le domaine du travail, les choses qui assument les fonctions essentielles sont les machines. La chose qui met en rapport production et consommation et qui règle l'échange des produits, c'est la monnaie. Enfin là où la fonction de coordonner et de diriger est trop lourde pour l'intelligence et la pensée d’un homme seul, elle est confiée à une machine étrange, dont les pièces sont des hommes, où les engrenages sont constitués par des règlements, des rapports et des statistiques, et qui se nomme organisation bureaucratique. Toutes ces choses aveugles imitent à s'y méprendre l'effort de la pensée. Le simple jeu du calcul algébrique est parvenu plus d'une fois à ce qu'on pourrait appeler une notion nouvelle, à cela près que ces simili-notions n'ont pas d'autre contenu que des rapports de signes ; et ce même calcul est souvent merveilleusement propre à transformer des séries de résultats expérimentaux en lois, avec une facilité déconcertante qui rappelle les transformations fantastiques que l'on voit dans les dessins animés. Les machines automatiques semblent présenter le modèle du travailleur intelligent, fidèle, docile et consciencieux. Quant à la monnaie, les économistes ont longtemps été persuadés qu'elle possède la vertu d'établir entre les diverses fonctions économiques des rapports harmonieux. Et les mécanismes bureaucratiques parviennent presque à remplacer des chefs. Ainsi dans tous les domaines la pensée, apanage de l'individu, est subordonnée à de vastes mécanismes qui cristallisent la vie collective, et cela au point qu'on a presque perdu le sens de ce qu'est la véritable pensée. Les efforts, les peines, les ingéniosités des êtres de chair et de sang que le temps amène par vagues successives à la vie sociale n'ont de valeur sociale et d'efficacité qu'à condition de venir à leur tour se cristalliser dans ces grands mécanismes. Le renversement du rapport entre moyens et fins, renversement qui est dans une certaine mesure la loi de toute société oppressive, devient ici total ou presque, et s'étend à presque tout, Le savant ne fait pas appel à la science afin d'arriver à voir plus clair dans sa propre pensée, mais aspire à trouver des résultats qui puissent venir s'ajouter à la science constituée. Les machines ne fonctionnent pas pour permettre aux hommes de vivre, mais on se résigne à nourrir les hommes afin qu'ils servent les machines. L'argent ne fournit pas un procédé commode pour échanger les produits, c'est l'écoulement des marchandises qui est un moyen pour faire circuler l'argent. Enfin l'organisation n'est pas un moyen pour exercer une activité collective, mais l'activité d'un groupe, quel qu'il puisse être, est un moyen pour renforcer l'organisation. Un autre aspect du même renversement consiste dans le fait que les signes, mots et formules algébriques dans le domaine de la connaissance, monnaie et symboles de crédit dans la vie économique, font fonction de réalités dont les choses réelles ne constitueraient que les ombres, exactement comme dans le conte d'Andersen où le savant et son ombre intervertissaient leurs rôles ; c'est que les signes sont la matière des rapports sociaux, au lieu que la perception de la réalité est chose individuelle. La dépossession de l'individu au profit de la collectivité n'est au reste pas totale, et elle ne peut l'être ; mais on conçoit mal comment elle pourrait aller beaucoup plus loin qu'aujourd'hui. La puissance et la concentration des armements mettent toutes les vies humaines à la merci du pouvoir central. En raison de l'extension formidable des échanges, la plupart des hommes ne peuvent atteindre la plupart des choses qu'ils consomment que par l'intermédiaire de la société et contre de l'argent ; les paysans eux-mêmes sont aujourd'hui soumis dans une large mesure à cette nécessité d'acheter. Et comme la grande industrie est un régime de production collective, bien des hommes sont contraints, pour que leurs mains puissent atteindre la matière du travail, de passer par une collectivité qui se les incorpore et les astreint à une tache plus ou moins servile ; lorsque la collectivité les repousse, la force et l'habileté de leurs mains restent vaines. Les paysans eux-mêmes, qui échappaient jusqu'ici à cette condition misérable, y ont été réduits récemment sur un sixième du globe. Un état de choses aussi étouffant suscite bien ça et là une réaction individualiste ; l'art, et notamment la littérature, en porte des traces ; mais comme en vertu des conditions objectives, cette réaction ne peut mordre ni sur le domaine de la pensée ni sur celui de l'action, elle demeure enfermée dans les jeux de la vie intérieure ou dans ceux de l'aventure et des actes gratuits, c'est-à-dire qu'elle ne sort pas du royaume des ombres ; et tout porte à croire que même cette ombre de réaction est vouée à disparaître presque complètement.\par
Quand l'homme est à ce point asservi, les jugements de valeur ne peuvent se fonder, en quelque domaine que ce soit, que sur un critérium purement extérieur ; il n'y a pas, dans le langage, de terme assez étranger à la pensée pour exprimer convenablement quelque chose d'aussi dépourvu de sens ; mais l'on peut dire que ce critérium se définit par l'efficacité, à condition d'entendre par là des succès remportés à vide. Même une notion scientifique n'est pas appréciée d'après son contenu, lequel peut être tout à fait inintelligible, mais d'après les facilités qu'elle procure pour coordonner, abréger, résumer. Dans le domaine économique, une entreprise est jugée non d'après l'utilité réelle des fonctions sociales qu'elle remplit, mais d'après l’extension qu'elle a prise et la rapidité avec laquelle elle se développe ; et ainsi pour tout. Ainsi le jugement des valeurs est en quelque sorte confié aux choses au lieu de l'être à la pensée. L'efficacité des efforts de toute espèce doit toujours, il est vrai, être contrôlée par la pensée, car, d'une manière générale, tout contrôle procède de l'esprit ; mais la pensée est réduite à un rôle si subalterne qu'on peut dire, pour simplifier, que la fonction de contrôler est passée de la pensée aux choses. Mais cette complication exorbitante de toutes les activités théoriques et pratiques qui a ainsi découronné la pensée en arrive, lorsqu'elle s'aggrave encore, à rendre ce contrôle exercé par les choses à son tour défectueux et presque impossible. Tout est alors aveugle. C'est ainsi que, dans le domaine de la science, l'accumulation démesurée des matériaux de toute espèce aboutit à un chaos tel que le moment semble proche où tout système apparaîtra comme arbitraire. Le chaos de la vie économique est encore bien plus évident. Dans l'exécution même du travail, la subordination d'esclaves irresponsables à des chefs débordés par la quantité des choses à surveiller, et d'ailleurs irresponsables eux aussi dans une large mesure, est cause de malfaçons et de négligences innombrables ; ce mal, d'abord limité aux grandes entreprises industrielles, s'est étendu aux champs là où les paysans sont asservis à la manière des ouvriers, c'est-à-dire en Russie soviétique. L'extension formidable du crédit empêche la monnaie de jouer son rôle régulateur en ce qui concerne les échanges et les rapports des diverses branches de la production ; et c'est bien en vain que l'on essaierait d'y remédier à coups de statistiques. L'extension parallèle de la spéculation aboutir à rendre la prospérité des entreprises indépendante, dans une large mesure, de leur bon fonctionnement ; du fait que les ressources apportées par la production même de chacune d'elles comptent de moins en moins à côté de l'apport perpétuel de capital nouveau. Bref, dans tous les domaines, le succès est devenu quelque chose de presque arbitraire ; il apparaît de plus en plus comme l'œuvre du pur hasard ; et comme il constituait la règle unique dans toutes les branches de l'activité humaine, notre civilisation est envahie par un désordre continuellement croissant, et ruinée par un gaspillage proportionnel au désordre. Cette transformation s’accomplit au moment même où les sources de profit d'où l'économie capitaliste a autrefois tiré son développement prodigieux se font de moins en moins abondantes, où les conditions techniques du travail imposent par elles-mêmes au progrès de l'équipement industriel un rythme rapidement décroissant.\par
Tant de changements profonds se sont opérés presque à notre insu, et pourtant nous vivons une période où l'axe même du système social est pour ainsi dire en train de se retourner. Tout au cours de l'essor du régime industriel la vie sociale s'est trouvée orientée dans le sens de la construction. L'équipement industriel de la planète était par excellence le terrain sur lequel se livrait la lutte pour le pouvoir. Faire grandir une entreprise plus vite que ses rivales, et cela par ses propres ressources, tel était en général le but de l'activité économique. L'épargne était la règle de la vie économique ; on restreignait au maximum la consommation non seulement des ouvriers, mais aussi des capitalistes, et, d'une manière générale, toutes les dépenses tendant à autre chose qu'à l'équipement industriel. Les gouvernements avaient avant tout pour mission de préserver la paix civile et internationale. Les bourgeois avaient le sentiment qu'il en serait indéfiniment ainsi, pour le plus grand bonheur de l'humanité ; mais il ne pouvait pas en être indéfiniment ainsi. De nos jours, la lutte pour le pouvoir, tout en gardant dans une certaine mesure l'apparence des mêmes formes, a complètement changé de nature. L'augmentation formidable de la part prise dans les entreprises par le capital matériel, si on la compare à celle du travail vivant, la diminution rapide du taux de profit qui en a résulté, la masse perpétuellement croissante des frais généraux, le gaspillage, le coulage, l'absence de tout élément régulateur permettant d'ajuster les diverses branches de la production, tout empêche que l’activité sociale puisse encore avoir pour pivot le développement de l'entreprise par la transformation du profit en capital. Il semble que la lutte économique ait cessé d'être une rivalité pour devenir une sorte de guerre. Il s'agit non plus tant de bien organiser le travail que d'arracher la plus grande part possible de capital disponible épars dans la société en écoulant des actions, et d'arracher ensuite la plus grande quantité possible de l'argent dispersé de toutes parts en écoulant des produits ; tout se joue dans le domaine de l'opinion et presque de la fiction, à coups de spéculation et de publicité. Le crédit étant à la clef de tout succès économique, l'épargne est remplacée par les dépenses les plus folles. Le terme de propriété est devenu presque vide de sens ; il ne s'agit plus pour l'ambitieux de faire prospérer une affaire dont il serait le propriétaire, mais de faire passer sous son contrôle le plus large secteur possible de l'activité économique. En un mot, pour caractériser d'une manière d'ailleurs vague et sommaire cette transformation d'une obscurité presque impénétrable, il s'agit à présent dans la lutte pour la puissance économique bien moins de construire que de conquérir ; et comme la conquête est destructrice, le système capitaliste, demeuré pourtant en apparence à peu près le même qu'il y a cinquante ans, s'oriente tout entier vers la destruction. Les moyens de la lutte économique, publicité, luxe, corruption, investissements formidables reposant presque entièrement sur le crédit, écoulement de produits inutiles par des procédés presque violents, spéculations destinées à ruiner les entreprises rivales, tendent tous à saper les bases de notre vie économique bien plutôt qu'à les élargir. Mais tout cela est peu de chose auprès de deux phénomènes connexes qui commencent à apparaître clairement et à faire peser sur la vie de chacun une menace tragique ; à savoir d'une part le fait que l'État tend de plus en plus, et avec une extraordinaire rapidité, à devenir le centre de la vie économique et sociale, et d'autre part la subordination de l'économique au militaire. Si l'on essaie d'analyser ces phénomènes dans le détail, on est arrêté par un enchevêtrement presque inextricable de causes et d'effets réciproques ; mais la tendance générale est assez claire. Il est assez naturel que le caractère de plus en plus bureaucratique de l'activité économique favorise les progrès de la puissance de l'État, lequel est l'organisation bureaucratique par excellence. La transformation profonde de la lutte économique joue dans le même sens ; l'État est incapable de construire, mais du fait qu'il concentre entre ses mains les moyens de contrainte les plus puissants, il est amené en quelque sorte par son poids même à devenir peu à peu l'élément central là où il s'agit de conquérir et de, détruire. Enfin, étant donné que l'extraordinaire complication des opérations d'échanges et de crédit empêche désormais que la monnaie puisse suffire à coordonner la vie économique il faut bien qu'un semblant de coordination bureaucratique y supplée ; et l'organisation bureaucratique centrale qui est l'appareil d'État, doit naturellement être amenée tôt ou tard à prendre la haute main dans cette coordination. Le pivot autour duquel tourne la vie sociale ainsi transformée n'est autre que la préparation à la guerre. Dès lors que la lutte pour la puissance s'opère par la conquête et la destruction, autrement dit par une guerre économique diffuse, il n'est pas étonnant que la guerre proprement dite vienne au premier plan. Et comme la guerre est la forme propre de la lutte pour la puissance lorsque les compétiteurs sont des États, tout progrès dans la mainmise de l'État sur la vie économique a pour effet d'orienter la vie industrielle dans une mesure encore un peu plus grande vers la préparation à la guerre ; cependant que réciproquement les exigences continuellement croissantes de la préparation à la guerre contribuent a soumettre de jour en jour davantage l'ensemble des activités économiques et sociales de chaque pays à l'autorité du pouvoir central. Il apparaît assez clairement que l'humanité contemporaine tend un peu partout à une forme totalitaire d'organisation sociale, pour employer le terme que les nationaux-socialistes ont mis à la mode, c'est-à-dire à un régime où le pouvoir d'État déciderait souverainement dans tous les domaines, même et surtout dans le domaine de la pensée. La Russie offre un exemple presque parfait d'un tel régime, pour le plus grand malheur du peuple russe ; les autres pays ne pourront que s'en approcher, à moins de bouleversements analogues à celui d'octobre 1917, mais il semble inévitable que tous s'en approchent plus ou moins au cours des années qui viennent. Cette évolution ne fera que donner au désordre une forme bureaucratique, et accroître encore l'incohérence, le gaspillage, la misère. Les guerres amèneront une consommation insensée de matières premières et d'outillage, une folle destruction des biens de toute espèce que nous ont légués les générations précédentes. Quand le chaos et la destruction auront atteint la limite à partir de laquelle le fonctionnement même de l'organisation économique et sociale sera devenu matériellement impossible, notre civilisation périra ; et l'humanité, revenue à un niveau de vie plus ou moins primitif et à une vie sociale dispersée en des collectivités beaucoup plus petites, repartira sur une voie nouvelle qu'il nous est absolument impossible de prévoir.\par
Se figurer que l'on peut aiguiller l'histoire dans une direction différente en transformant le régime à coups de réformes ou de révolutions, espérer le salut d'une action défensive ou offensive contre la tyrannie et le militarisme, c'est rêver tout éveillé. Il n'existe rien sur quoi appuyer même de simples tentatives. La formule de Marx selon laquelle le régime engendrerait ses propres fossoyeurs reçoit tous les jours de cruels démentis ; et l'on se demande d'ailleurs comment Marx a jamais pu croire que l'esclavage puisse former des hommes libres. Jamais encore dans l'histoire un régime d'esclavage n'est tombé sous les coups des esclaves. La vérité, c'est que, selon une formule célèbre, l'esclavage avilit l'homme jusqu'à s'en faire aimer ; que la liberté n'est précieuse qu’aux yeux de ceux qui la possèdent effectivement ; et qu'un régime entièrement inhumain, comme est le nôtre, loin de forger des êtres capables d'édifier une société humaine, modèle à son image tous ceux qui lui sont soumis, aussi bien opprimés qu'oppresseurs. Partout, à des degrés différents, l'impossibilité de mettre en rapport ce qu'on donne et ce qu'on reçoit a tué le sens du travail bien fait, le sentiment de la responsabilité, a suscité la passivité, l'abandon, l'habitude de tout attendre de l'extérieur, la croyance aux miracles. Même aux champs, le sentiment d'un lien profond entre la terre qui nourrit l'homme et l'homme qui travaille la terre s'est effacé dans une large mesure depuis que le goût de la spéculation, les variations imprévisibles des monnaies et des prix ont habitué les paysans à tourner leurs regards du côté de la ville. L'ouvrier n'a pas conscience de gagner sa vie en faisant acte de producteur ; simplement l'entreprise l'asservit chaque jour durant de longues heures, et lui octroie chaque semaine une somme d'argent qui lui donne le pouvoir magique de susciter en un instant des produits tout fabriqués, exactement comme font les riches. La présence de chômeurs innombrables, la cruelle nécessité de mendier une place font apparaître le salaire comme étant moins un salaire qu'une aumône. Quant aux chômeurs eux-mêmes, ils ont beau être des parasites involontaires et d'ailleurs misérables, ils n'en sont pas moins des parasites, D'une manière générale, le rapport entre le travail fourni et l'argent reçu est si difficilement saisissable qu'il apparaît comme presque contingent, de sorte que le travail apparaît comme un esclavage, l'argent comme une faveur. Les milieux que l'on nomme dirigeants sont atteints par la même passivité que tous les autres, du fait que, débordés comme ils sont par un océan de problèmes inextricables, ils ont depuis longtemps renoncé, à diriger. On chercherait en vain, du plus haut au plus bas de l'échelle sociale, un milieu d'hommes en qui puisse un jour germer l'idée qu'ils pourraient, le cas échéant, avoir à prendre en mains les destinées de la société ; les déclamations des fascistes pourraient seules faire illusion à ce sujet, mais elles sont creuses. Comme il arrive toujours, la confusion mentale et la passivité laissent libre cours à l'imagination. De toutes parts on est obsédé par une représentation de la vie sociale qui, tout en différant considérablement d'un milieu à l'autre, est toujours faite de mystères, de qualités occultes, de mythes, d'idoles, de monstres ; chacun croit que la puissance réside mystérieusement dans un des milieux où il n'a pas accès, parce que presque personne ne comprend qu'elle ne réside nulle part, de sorte que partout le sentiment dominant est cette peur vertigineuse que produit toujours la perte du contact avec la réalité. Chaque milieu apparaît du dehors comme un objet de cauchemar. Dans les milieux qui se rattachent au mouvement ouvrier, les rêves sont hantés par des monstres mythologiques qui ont nom Finance, Industrie, Bourse, Banque et autres ; les bourgeois rêvent d'autres monstres qu'ils nomment meneurs, agitateurs, démagogues; les politiciens considèrent les capitalistes comme des êtres surnaturels qui possèdent seuls la clef de la situation, et réciproquement ; chaque peuple regarde les peuples d'en face comme des monstres collectifs animés d'une perversité diabolique. On pourrait développer ce thème à l'infini. Dans une pareille situation, n'importe quel soliveau peut être regardé comme un roi et en tenir lieu dans une certaine mesure grâce a cette seule croyance ; et cela n'est pas vrai seulement en ce qui concerne les hommes, mais aussi en ce qui concerne les milieux dirigeants. Rien n'est plus facile non plus que de répandre un mythe quelconque à travers toute une population. Il ne faut pas s'étonner dès lors de l'apparition de régimes « totalitaires » sans précédent dans l'histoire. On dit souvent que la force est impuissante à dompter la pensée ; mais pour que ce soit vrai, il faut qu'il y ait pensée. Là où les, opinions irraisonnées tiennent lieu d'idées, la force peut tout. Il est bien injuste de dire par exemple que le fascisme anéantit la pensée libre ; en réalité c'est l'absence de pensée libre qui rend possible d'imposer par la force des doctrines officielles entièrement dépourvues de signification. À vrai dire un tel régime arrive encore à accroître considérablement l'abêtissement général, et il y a peu d'espoir pour les générations qui auront grandi dans les conditions qu'il suscite. De nos jours toute tentative pour abrutir les êtres humains trouvé à sa disposition des moyens puissants. En revanche une chose est impossible, quand même on disposerait de la meilleure des tribunes ; à savoir diffuser largement les idées claires, des raisonnements corrects, des aperçus raisonnables.\par
Il n'y a pas de secours a espérer des hommes ; et quand il en serait autrement, les hommes n'en seraient pas moins vaincus d'avance par la puissance des choses. La société actuelle ne fournit pas d'autres moyens d'action que des machines à écraser l'humanité ; quelles que puissent être les intentions de ceux qui les prennent en main, ces machines écrasent et écraseront aussi longtemps qu'elles existeront. Avec les bagnes industriels que constituent les grandes usines, on ne peut fabriquer que des esclaves, et non pas des travailleurs libres, encore moins des travailleurs qui constitueraient une classe dominante. Avec des canons, des avions, des bombes, on peut répandre la mort, la terreur, l'oppression, mais non pas la vie et la liberté. Avec les masques à gaz, les abris, les alertes, on peut forger de misérables troupeaux d'êtres affolés, prêts à céder aux terreurs les plus insensées et à accueillir avec reconnaissance les plus humiliantes tyrannies, mais non pas des citoyens. Avec la grande presse et la T.S.F., on peut faire avaler par tout un peuple, en même temps que le petit déjeuner ou le repas du soir, des opinions toutes faites et par là même absurdes, car même des vues raisonnables se déforment et deviennent fausses dans l'esprit qui les reçoit sans réflexion ; mais on ne peut avec ces choses susciter même un éclair de pensée. Et sans usines, sans armes, sans grande presse on ne peut rien contre ceux qui possèdent tout cela. Il en est ainsi pour tout. Les moyens puissants sont oppressifs, les moyens faibles sont inopérants. Toutes les fois que les opprimés ont voulu constituer des groupements capables d'exercer une influence réelle, ces groupements, qu'ils aient eu nom partis ou syndicats, ont intégralement reproduit dans leur sein toutes les tares du régime qu'ils prétendaient réformer ou abattre, à savoir l'organisation bureaucratique, le renversement du rapport entre les moyens et les fins, le mépris de l'individu, la séparation entre la pensée et l'action, le caractère machinal de la pensée elle-même, l'utilisation de l'abêtissement et du mensonge comme moyens de propagande, et ainsi de suite. L'unique possibilité de salut consisterait dans une coopération méthodique de tous, puissants et faibles, en vue d'une décentralisation progressive de la vie sociale ; mais l'absurdité d'une telle idée saute immédiatement aux yeux. Une telle coopération ne peut pas s ‘imaginer même en rêve dans une civilisation qui repose sur la rivalité, sur la lutte, sur la guerre. En dehors d'une telle coopération, il est impossible d'arrêter la tendance aveugle de la machine sociale vers une centralisation croissante, jusqu'à ce que la machine elle-même s'enraye brutalement et vole en éclats. Que peuvent peser les souhaits et les vœux de ceux qui ne sont pas aux postes de commande, alors que, réduits à l'impuissance la plus tragique, ils sont les simples jouets de forces aveugles et brutales ? Quant à ceux qui possèdent un pouvoir économique ou politique, harcelés qu'ils sont d'une manière continuelle par les ambitions rivales et les puissances hostiles, ils ne peuvent travailler à affaiblir leur propre pouvoir sans se condamner presque à coup sûr à en être dépossédés. Plus ils se sentiront animés de bonnes intentions, plus ils seront amenés même malgré eux à tenter d'étendre leur pouvoir pour étendre leur capacité de faire le bien ; ce qui revient à opprimer dans l'espoir de libérer, comme a fait Lénine. Il est de toute évidence impossible que la décentralisation parte du pouvoir Central ; dans la mesure même où le pouvoir central s'exerce, il se subordonne tout le reste. D'une manière générale l'idée du despotisme éclairé, qui a toujours eu un caractère utopique, est de nos jours tout à fait absurde. En présence de problèmes dont la variété et la complexité dépassent infiniment les grands comme les petits esprits, aucun despote au monde ne peut être éclairé. Si quelques hommes peuvent espérer, à force de réflexions honnêtes et méthodiques, apercevoir quelques lueurs dans cette obscurité impénétrable, ce n'est certes pas le cas pour ceux que les soucis et les responsabilités du pouvoir privent à la fois de loisir et de liberté d'esprit. Dans une pareille situation, que peuvent faire ceux qui s'obstinent encore, envers et contre tout, à respecter la dignité humaine en eux-mêmes et chez autrui ? Rien, sinon s'efforcer de mettre un peu de jeu dans les rouages de la machine qui nous broie ; saisir toutes les occasions de réveiller un peu la pensée partout où ils le peuvent ; favoriser tout ce qui est susceptible, dans le domaine de la politique, de l'économie ou de la technique, de laisser çà et là à l'individu une certaine liberté de mouvements à l'intérieur des liens dont l'entoure l'organisation sociale. C'est certes quelque chose, mais cela ne va pas loin. Dans l'ensemble, la situation ou nous sommes est assez semblable à celle de voyageurs tout à fait ignorants qui se trouveraient dans une automobile lancée a toute vitesse et sans conducteur à travers un pays accidenté. Quand se produira la cassure après laquelle il pourra être question de chercher à construire quelque chose de nouveau ? C'est peut-être une affaire de quelques dizaines d'années, peut-être aussi de siècles. Aucune donnée ne permet de déterminer un délai probable. Il semble cependant que les ressources matérielles de notre civilisation ne risquent pas d'être épuisées avant un temps assez long, même en tenant compte de guerres ; et d'autre part, comme la centralisation, en abolissant toute initiative individuelle et toute vie locale, détruit par son existence même tout ce qui pourrait servir de base a une organisation différente, on peut supposer que le système actuel subsistera jusqu'à l'extrême limite des possibilités. Somme toute il paraît raisonnable de penser que les générations qui seront en présence des difficultés suscitées par l'effondrement du régime actuel sont encore à naître. Quant aux générations actuellement vivantes, elles sont peut-être, de toutes celles qui se sont succédé 'au cours de l'histoire humaine, celles qui auront eu à supporter le plus de responsabilités imaginaires et le moins de responsabilités réelles. Cette situation, une fois pleinement comprise, laisse une liberté d'esprit merveilleuse.
\subsection[CONCLUSION]{CONCLUSION}
\noindent Qu'est-ce au juste qui périra et qu'est-ce qui subsistera de la civilisation actuelle ? Dans quelles conditions, en quel sens l'histoire se déroulera-t-elle par la suite ? Ces questions sont insolubles. Ce que nous savons d'avance, c'est que la vie sera d'autant moins inhumaine que la capacité individuelle de penser et d'agir sera plus grande. La civilisation actuelle, dont nos descendants recueilleront sans doute tout au moins des fragments en héritage, contient, nous ne le sentons que trop, de quoi écraser l'homme ; mais elle contient aussi, du moins en germe, de quoi le libérer. Il y a dans notre science, malgré toutes les obscurités qu'amène une sorte de nouvelle scolastique, des éclairs admirables, des parties limpides et lumineuses, des démarches parfaitement méthodiques de l'esprit. Dans notre technique aussi il y a des germes de libération du travail. Non pas sans doute, comme on le croit communément, du côté des machines automatiques ; celles-ci apparaissent bien comme étant propres, du point de vue purement technique, à décharger les hommes de ce que le travail peut contenir de machinal et d'inconscient, mais en revanche elles sont indissolublement liées a une organisation de l'économie centralisée à l'excès, et par suite très oppressive. Mais d'autres formes de la machine-outil ont produit, surtout avant la guerre, le plus beau type peut-être de travailleur conscient qui soit apparu dans l'histoire, à savoir l'ouvrier qualifié. Si, au cours des vingt dernières années, la machine-outil a pris des formes de plus en plus automatiques, si le travail accompli, même sur les machines de modèle relativement ancien, est devenu de plus en plus machinal, c'est la concentration croissante de l'économie qui en est cause. Qui sait si Une industrie dispersée en d'innombrables petites entreprises ne susciterait pas une évolution inverse de la machine-outil, et, parallèlement, des formes de travail demandant encore bien plus de conscience et d'ingéniosité que le travail le plus qualifié des usines modernes ? Il est d'autant moins défendu de l'espérer que l'électricité fournit la forme d'énergie qui conviendrait à une semblable organisation industrielle. Etant donné que notre impuissance presque complète à l'égard des maux présents nous dispense du moins, une fois clairement comprise, de nous soucier de l'actualité en dehors des moments où nous en subissons directement l'atteinte, quelle tâche plus noble pourrions-nous assumer que celle de préparer méthodiquement un tel avenir en travaillant à faire l'inventaire de la civilisation présente ? C'est à vrai dire une tâche qui dépasse de très loin les possibilités si restreintes d'une vie humaine ; et d'autre part s'orienter dans une pareille voie, c'est se condamner à coup sûr à la solitude morale, à l'incompréhension, à l'hostilité aussi bien des ennemis de l'ordre existant que de ses serviteurs ; quant aux générations futures, rien ne permet de supposer que le hasard leur fasse même parvenir, le cas échéant, à travers les catastrophes qui nous séparent d'elles, les fragments d'idées que pourraient élaborer de nos jours quelques esprits solitaires. Mais il serait fou de se plaindre d'une telle situation. Jamais aucun pacte avec la Providence n'a promis l'efficacité aux efforts même les plus généreux. Et quand on a résolu de ne faire confiance, en soi-même et autour de soi, qu'à des efforts ayant leur source et leur principe dans la pensée de celui même qui les accomplit, il serait ridicule de désirer qu'une opération magique permette d'obtenir de grands résultats avec les forces infimes dont disposent les individus isolés. Ce n'est jamais par de pareilles raisons qu'une âme ferme peut se laisser détourner, quand elle aperçoit clairement une chose à faire, et une seule. Il s'agirait donc de séparer, dans la civilisation actuelle, ce qui appartient de droit à l'homme considère comme individu et ce qui est de nature à fournir des armes contre lui a la collectivité, tout en cherchant les moyens de développer les premiers éléments au détriment des seconds. En ce qui concerne la science, il ne faut plus essayer d'ajouter à la masse déjà trop grande qu'elle constitue ; il faut en faire le bilan pour permettre à l'esprit d'y mettre en lumière ce qui lui appartient en propre, ce qui est constitué par des notions claires, et de mettre à part ce qui n'est que procédé automatique pour coordonner, unifier, résumer ou même découvrir ; il faut tenter de ramener ces procédés eux-mêmes à des démarches conscientes de l'esprit ; il faut d'une manière générale, partout où on le peut, concevoir et présenter les résultats comme un simple moment dans l'activité méthodique de la pensée. À cet effet une étude sérieuse de l'histoire des sciences est sans doute indispensable. Quant à la technique, il faudrait l'étudier d'une ma mère approfondie, dans son histoire, dans son état actuel, dans ses possibilités de développement, et cela d'un point de vue tout a fait nouveau, qui ne serait plus celui du rendement, mais celui du rapport du travailleur avec son travail. Enfin il faudrait mettre en pleine lumière l'analogie des démarches qu'accomplit la pensée humaine, d'une part dans la vie quotidienne et notamment dans le travail, d'autre part dans l'élaboration méthodique de la science. Quand même une suite de réflexions ainsi orientées devrait rester sans influence sur l'évolution ultérieure de l'organisation sociale, elle n'en perdrait pas pour cela sa valeur ; les destinées futures de l'humanité ne sont pas l'unique objet qui mérite considération. Seuls des fanatiques peuvent n'attacher de prix à leur propre existence que pour autant qu'elle sert une cause collective; réagir contre la subordination de l'individu à la collectivité implique qu'on commence par refuser de subordonner sa propre destinée au cours de l'histoire. Pour se déterminer à un pareil effort d'analyse critique, il suffit de comprendre qu'il permettrait à celui qui l'entreprendrait d'échapper à la contagion de la folie et du vertige collectif en renouant pour son compte, par dessus l'idole sociale, le pacte originel de l'esprit avec l'univers.\par

\section[Fragments, 1933-1938]{Fragments \\
1933-1938}\renewcommand{\leftmark}{Fragments \\
1933-1938}

\noindent \par
\subsection[I]{I}
\noindent La situation où nous nous trouvons est d'une gravité sans précédent. Le prolétariat le plus avancé, le mieux organisé du monde a non seulement été vaincu, mais a capitulé sans résistance. C'est la seconde fois en vingt ans. Pendant la guerre, nos aînés ont encore pu espérer que le prolétariat russe, par son magnifique soulèvement, allait réveiller les ouvriers européens. Pour nous, rien ne nous permet de former un espoir analogue ; aucun signe n'annonce nulle part une victoire susceptible de compenser l'écrasement sans combat des ouvriers allemands. Jamais peut-être, depuis qu'un mouvement ouvrier existe, le rapport des forces n'a été aussi défavorable au prolétariat qu'aujourd'hui, cinquante ans après la mort de Marx.\par
Que nous reste-t-il de Marx, cinquante ans après sa mort ? Sa doctrine n'est pas destructible ; chacun peut la chercher dans ses œuvres et se l'assimiler en la pensant à nouveau ; et bien que l'on répande de nos jours, sous le nom de marxisme, quelques formules desséchées et dépourvues de signification réelle, quelques militants remontent aux sources. Mais, bien que les analyses de Marx aient une valeur qui ne peut périr, l'objet de ces analyses, à savoir la société contemporaine de Marx, n'est plus. Le marxisme ne peut vivre qu'à une condition, c'est que le précieux outil que Constitue la méthode marxiste passe de génération en génération sans se rouiller, chaque génération s'en servant pour définir le monde où elle vit. C'est ce qu'a compris la génération d'avant-guerre, comme en témoignent la brochure de Lénine sur l'impérialisme et plusieurs ouvrages allemands. Tout cela est malheureusement bien sommaire. Mais nous, depuis la guerre, qu'avons-nous fait à cet égard ? On dirait, à lire la littérature du mouvement ouvrier, qu'il n'y a rien de nouveau depuis Marx et Lénine. Il y a cependant, « sur un sixième du globe », un régime économique tel qu'on n'en a jamais connu ni suppose un semblable ; dans le reste du monde, la monnaie-papier, l'inflation, la part croissante de l'État dans l'économie, la rationalisation et bien d'autres changements sont venus modifier, et peut-être transformer, les rapports économiques ; nous vivons, depuis plus de quatre ans, une crise telle qu'il n'y en a jamais eu de semblable. Que savons-nous sur tout cela ? Pour moi, je ne puis énumérer ces questions sans prendre conscience, avec un amer sentiment de honte, de ma propre ignorance ; et par malheur il n'y a, à ma connaissance, dans la littérature du mouvement ouvrier, rien qui permette de croire qu'il y a, actuellement, des marxistes capables de résoudre ou même de formuler clairement les questions essentielles que pose l'état présent de l'économie. Aussi ne faut-il pas s'étonner que, cinquante ans après la mort de Marx, les marxistes eux-mêmes traitent en fait la politique comme si c'était un domaine à part, à peu près séparé du domaine des faits économiques. Dans la presse communiste quotidienne, la division en classes, destinée chez Marx à expliquer les phénomènes politiques par les rapports de production, est devenue la source d'une mythologie nouvelle ; la bourgeoisie, notamment, y est une divinité mystérieuse et maléfique, qui suscite les phénomènes dont elle a besoin, et dont les désirs et les ruses expliquent à peu près tout ce qui se passe. La littérature communiste plus sérieuse n'échappe pas entièrement à ce ridicule, et cela même dans les groupements d'opposition, même dans certaines analyses de Trotsky. Et bien entendu, les conceptions politiques, n'étant pas appuyées sur l'économie et ne pouvant pas plus avancer dans le vide qu'un oiseau ne pourrait voler sans la résistance de l'air, sont celles que nous ont léguées l'avant-guerre et la guerre. Le réformisme reste ce qu'il a toujours été ; l'idéologie anarchiste aussi ; les syndicalistes révolutionnaires rêvent à la vieille C.G.T. ; les communistes orthodoxes et oppositionnels se disputent pour savoir qui imite le mieux le parti bolchévik d'avant-guerre. Tous traversent en inconscients cette période si neuve où nous sommes, période qu'aucune des analyses précédemment faites ne permet de définir, et où il semble que les corps soient seuls a vivre, alors que les esprits se meuvent encore dans le monde disparu de l'avant-guerre.
\subsection[II]{II}
\noindent La question de la structure sociale se ramène a la question des classes.\par
Jusqu'ici, dans l'histoire, l'on ne connaît que des sociétés divisées en classes, a l'exception des sociétés tout a fait primitives, où aucune différenciation ne s'est encore produite. Dès que la production est quelque peu développée, la société se divise en diverses catégories qui s'opposent les unes aux autres, et dont les intérêts diffèrent. L'opposition la plus marquée est celle qui existe entre les non-producteurs et les producteurs, autrement dit entre les exploiteurs et les exploites ; car les non--producteurs consomment nécessairement ce que d'autres produisent, et par suite les exploitent. Le mécanisme de l'exploitation définit la structure sociale de chaque époque. Au reste, il va sans dire qu'une théorie matérialiste ne peut jamais considérer les exploiteurs comme de simples parasites ; dans toute société divisée en classes, l'exploitation du travail d'autrui constitue une fonction sociale, rendue possible et nécessaire par le mécanisme de la production dans cette société. Et une société sans classes ne pourra être réalisée que si l'on obtient une forme de production qui exclue une telle fonction. Au reste, une société quelconque ne se divise jamais simplement, en exploiteurs et en exploités, mais en plusieurs classes, dont chacune se définit par son rapport au fait fondamental de l'exploitation.\par
On connaît, dans l'histoire, trois formes principales de société fondées sur l'exploitation : le régime de l'esclavage, le régime féodal et le régime capitaliste. On ne connaît qu'une forme de société sans exploiteurs, à savoir le communisme primitif, lié à une technique tout à fait arriérée. La question vitale qui se pose pour nous est de savoir si, a un niveau supérieur, avec une technique très développée, une production sans exploitation est de nouveau possible. Pour poser la question d'une manière correcte, il faut savoir étudier scientifiquement, non pas seulement les diverses structures sociales, mais surtout les transformations qui remplacent une structure par une autre.\par

\subsection[III]{III}
\noindent Ce qu'on nomme de nos jours, par un terme qui appellerait bien des éclaircissements, la lutte des classes est, de tous les conflits qui opposent des groupements humains, le plus concret, celui dont l'objectif est le plus sérieux. Pourtant la aussi interviennent parfois des entités purement imaginaires qui empêchent toute action dirigée, qui amènent presque tous les efforts à porter dans le vide, et qui presque seules suscitent le danger de haines inexpiables, de destructions inutiles, peut-être de tueries sans limites. La lutte de ceux qui obéissent contre ceux qui commandent, lorsque le mode de commandement entraîne l'écrasement de la dignité humaine chez ceux d'en bas, est ce qu'il y a au monde de plus légitime, de plus motivé, de plus authentique. Cette lutte a toujours existé, parce que ceux qui commandent tendent toujours, qu'ils le sachent ou non, à fouler aux pieds la dignité humaine au-dessous d'eux ; la fonction de commandement, pour autant qu’elle s'exerce, ne peut pas, sauf cas exceptionnels, respecter l'humanité dans la personne des agents d'exécution. Si elle s'exerce sans aucune résistance, elle en arrive inévitablement 1s'exercer comme si les hommes étaient des choses, et encore des choses exceptionnellement souples et maniables ; car l'homme soumis a la menace de mort, qui est en dernière analyse la sanction suprême de toute autorité, peut devenir beaucoup plus maniable que la matière inerte. Aussi longtemps qu'il y aura une hiérarchie sociale, quelle que puisse être d'ailleurs cette hiérarchie, ceux d'en bas devront lutter et lutteront pour ne pas perdre tous les droits d'un être humain. D'autre part, la résistance de ceux d'en haut aux efforts surgis d'en bas, si elle est naturellement moins sympathique, repose du moins sur des motifs concrets. D'abord, sauf le cas d'une générosité assez rare, les privilégiés préfèrent nécessairement garder intacts leurs privilèges matériels et moraux. Et surtout ceux qui sont investis des fonctions de commandement ont pour mission de défendre l'ordre indispensable à toute vie sociale, et le seul ordre possible à leurs yeux est celui qui existe. Ils ont raison dans une certaine mesure, car jusqu'à ce qu'un nouvel ordre soit établi en fait, personne ne peut affirmer qu'il sera possible ; c'est précisément pourquoi un progrès social petit ou grand n'est possible que si la pression d'en bas est assez forte pour imposer en fait des conditions nouvelles aux rapports sociaux. Il s'établit ainsi continuellement, entre la pression d'en bas et la résistance d'en haut, un équilibre instable qui définit à chaque instant la structure d'une société. Mais la rencontre de ces deux efforts opposés n'est pas une guerre, même s'il arrive que çà ou là il coule un peu de sang. Les colères y sont inévitables, mais non la haine. Elle peut d'un côté ou de l'autre, ou des deux côtés, tourner en extermination ; mais alors c'est qu'elle change de nature, et que les objectifs véritables de la lutte s'effacent de la pensée des hommes, soit qu’un désir aveugle de vengeance paralyse la pensée soit que l'intervention d'entités vides de sens donne l’illusion, toujours erronée, qu'un équilibre est impossible. Il y a alors catastrophe ; mais de telles catastrophes sont évitables. L'antiquité ne nous a pas seulement légué l'histoire des massacres interminables et inutiles autour de Troie, elle nous a légué aussi l'histoire de l'action énergique et pacifique par laquelle les plébéiens de Rome, sans verser une goutte de sang, sont sortis d'une situation qui touchait à l'esclavage et ont obtenu, comme garantie de leurs droits nouveaux, l'institution des tribuns. C'est exactement de la même manière que les ouvriers français, par l'occupation pacifique des usines, ont imposé les congés payes, les salaires garantis et les délégués ouvriers.\par
On ne peut pas énumérer toutes les abstractions vides qui faussent aujourd'hui la lutte sociale, et dont certaines risquent de la faire dégénérer en une guerre civile funeste pour les deux camps. Il y en a trop. On ne peut que prendre un exemple. Ainsi que peuvent avoir dans l'esprit ceux pour qui le mot « capitalisme» représente le mal absolu ? Nous vivons sous un régime qui comporte des formes de contraintes et d'oppression parfois écrasantes ; des inégalités très douloureuses ; des masses de souffrances inutiles. D'autre part, ce régime est économiquement caractérisé par un certain rapport entre la production et la circulation des marchandises, entre la circulation des marchandises et la monnaie. Dans quelle mesure exacte est-ce que ces deux rapports conditionnent les souffrances en question ? Dans quelle mesure ont-elles d'autres causes ? Dans quelle mesure l'établissement de tel ou tel autre système les allégerait-il ou les aggraverait-il ? Si on étudiait le problème ainsi posé, on pourrait peut-être apercevoir approximativement dans quelle mesure le capitalisme est un mal. Comme on reste dans l'ignorance, on rapporte toutes les souffrances qu'on subit ou qu'on constate autour de soi à quelques phénomènes économiques d'ailleurs perpétuellement changeants, et qu'on cristallise arbitrairement en une abstraction impossible à définir. De la même manière, un ouvrier rapporte arbitrairement au patron toutes les souffrances qu'il subit dans l'usine, sans se demander si dans tout autre système de propriété la direction de l'entreprise ne lui infligerait pas encore une partie de ces souffrances ou même n'en aggraverait pas certaines ; pour lui, la lutte « contre le patron » se confond avec la protestation irrépressible de l'être humain accablé par des conditions de vie trop dures. Dans l'autre camp, une ignorance identique fait assimiler à des fauteurs de désordre tous ceux qui envisagent la fin du capitalisme, parce qu'on ignore dans quelle mesure et à quelle condition les rapports économiques qui constituent actuellement le capitalisme peuvent être légitimement considérés comme nécessaires à l'ordre. Ainsi la lutte entre adversaires et défenseurs du capitalisme est une lutte d'aveugles ; les efforts des lutteurs, d'un côté comme de l'autre, n'embrassent que le vide ; et c'est pourquoi cette lutte risque de devenir impitoyable.\par
La chasse aux entités dans tous les domaines de la vie politique et sociale apparaît ainsi comme une oeuvre de salubrité publique. L'effort d'éclaircissement pour dégonfler les causes des conflits imaginaires n'a rien de commun avec celui des endormeurs qui tentent d'étouffer les conflits sérieux. C'est même exactement le contraire. Les beaux parleurs qui, en prêchant la paix internationale, comprennent par cette expression le maintien indéfini du statu quo au profit exclusif de l'État français, ceux qui, en recommandant la paix sociale, entendent conserver les privilèges intacts ou du moins subordonner toute modification à la bonne volonté des privilégiés, ceux-là sont les pires ennemis de la paix internationale et civile. Discriminer les oppositions imaginaires et les oppositions réelles, discréditer les abstractions vides et analyser les problèmes concrets, ce serait, si nos contemporains consentaient a un pareil effort intellectuel, diminuer les risques de guerre sans renoncer à la lutte, dont Héraclite disait qu'elle est la condition de la vie.
\subsection[IV]{IV}
\noindent Le marxisme est la plus haute expression spirituelle de la société bourgeoise. Par lui elle est arrivée à prendre conscience d'elle-même, en lui elle s'est niée elle-même. Mais cette négation à son tour ne pouvait être exprimée que sous une forme déterminée par l'ordre existant, sous une forme de pensée bourgeoise. C'est ainsi que chaque formule de la doctrine marxiste dévoile les caractéristiques de la société bourgeoise, mais en même temps les légitime. À force de développer la critique de l'économie capitaliste, le marxisme a fini par donner de larges fondements aux lois de cette même économie ; l'opposition contre la politique bourgeoise a abouti à revendiquer la possibilité d'accomplir le vieil idéal de la bourgeoisie, cet idéal qu'elle n'a réalisé que d'une manière ambiguë, formelle, purement juridique, mais de l'accomplir en luttant contre elle, d'une manière plus conséquente qu'elle et vraiment concrète ; la doctrine qui devait à l'origine servir à anéantir toutes les idéologies en démasquant les intérêts qu'elles recouvrent s'est transformée elle-même en une idéologie, dont on devait par la suite abuser pour diviniser les intérêts d'une certaine classe de la société bourgeoise.\par
Ainsi s'est répété le même phénomène qu'au temps où la jeune bourgeoisie avait commencé sa lutte contre la société féodale et ecclésiastique. Elle dût d'abord revêtir son opposition des formes religieuses de cette société elle-même, et, pour combattre l'Église, se réclamer du christianisme primitif. Au cours de sa lutte contre les deux autres ordres, la bourgeoisie prit conscience de former un ordre distinct, et montra ainsi que malgré son opposition contre le régime féodal, elle avait conscience d'en constituer une partie intégrante (exactement comme la conscience de classe du prolétariat d'aujourd'hui, qui s'est développée pour compenser une tendance non satisfaite à la propriété, manifeste seulement l'état d'esprit bourgeois des prolétaires ; car le fait de penser par classes est précisément propre à la société bourgeoise). La bourgeoisie ne pût se libérer de cette idéologie religieuse, ecclésiastique et féodale qu'à mesure que la société féodale tomba en décadence. Mais elle ne fit que purifier la représentation de Dieu des scories qui s'y attachaient depuis le temps de l'économie naturelle ; elle se fit un Dieu sublimé qui n'était plus qu'une Raison transcendante, devançant tous les événements et en déterminant l'orientation. Dans la philosophie de Hegel, Dieu apparaît encore, sous le nom d’« esprit du monde », comme moteur de l'histoire et législateur de la nature. Ce ne fut qu'après avoir accompli sa révolution que la bourgeoisie reconnut en ce Dieu une création de l'homme lui-même, et que l'histoire est l'œuvre propre de l'homme.\par
C'est Ludwig Feuerbach qui formula clairement cette idée ; mais comment « l'homme » arrive à faire l'histoire, c'est ce qu'il fut incapable d'expliquer. Car d'une juxtaposition d'hommes considérés seulement comme des êtres naturels ne peut sortir qu'un mélange d'actions, mais non pas un développement régulier et ascendant de l'humanité. La découverte première et décisive de Marx consista seulement en ce qu'il s'éleva au-dessus de l'homme abstrait de Feuerbach et commença à chercher l'explication du processus historique dans la coopération des individus, dans l'union et la lutte, dans les « rapports » multiples qui existent {\itshape entre} eux. Cependant ce progrès de la pensée est encore à présent acheté, d'un autre point de vue, au prix d'un recul inconscient. Karl Marx n'a pu surmonter l'« être humain » isolé de Feuerbach qu'en ramenant dans l'histoire, sous le nom de « société », le Dieu que Feuerbach en avait éliminé.\par
À vrai dire, Marx commence par nous présenter la nouvelle divinité sous une forme inoffensive, comme « ensemble des rapports sociaux », c'est-à-dire comme réunion de toutes les relations individuelles entre hommes concrets et actifs. Il souligne plus d'une fois que ces « rapports » sont bien entendu des produits empiriques de l'activité humaine, que leur « ensemble », si l'on tient absolument à donner un nom spécial aux relations changeantes qui unissent entre eux les hommes actifs, doit être regardé seulement comme un terme abrége désignant le {\itshape résultat} du processus historique. Mais plus Marx analyse profondément le cours de l'histoire et les lois économiques, plus il modifie son point de vue, jusqu'à ce que, d'une manière imprévue, la « collectivité »devienne une hypostase, la {\itshape condition} des actions individuelles, une « essence » qui « apparaît »dans l'action et la pensée des hommes et « se réalise » dans leur activité. Elle constitue, à côté du domaine « privé » de l'individualisme bourgeois, un domaine à part, celui du « général », et, en qualité de substance indépendante, est le fondement du premier ; par exemple la valeur d'un produit est déjà déterminée par elle, avant de se « réaliser » dans le prix concret, empirique du marché. Et en régime socialiste aussi, il y aura encore une certaine séparation entre les deux domaines. Qu'on réfléchisse seulement a la formule : « propriété individuelle sur la base d'une possession collective de la terre et des moyens de pro- duction », formule qui définit l'ordre économique futur dans un passage connu du {\itshape Capital.} La distinction d'une sphère générale et d'une sphère individuelle est ici expressément formulée ; mais la représentation d'une « possession collective » n'est possible que si l’on considère la « collectivité » comme une substance particulière, planant au-dessus des individus, et agissant à travers eux.\par
Si l'on conteste tout cela, qu'on examine de près la formule marxiste : l'existence sociale détermine la conscience. Elle contient plus de contradictions que de mots. Étant donné que ce qui est « social » ne peut trouver une existence que dans les esprits humains, « l'existence sociale » est par elle-même déjà conscience, elle ne peut déterminer en outre une conscience qu'il resterait d'ailleurs à définir. Poser ainsi une « existence sociale » comme un facteur de détermination particulier, sépare de notre conscience, caché on ne sait où, c'est en faire une hypostase ; et c'est en plus un bel exemple de l'inclination de Marx au dualisme. Mais si l'on veut considérer cette énigmatique « existence » comme un élément des rapports {\itshape entre} les hommes, et qui dépend de certaines institutions, telles que l'argent, on verra tout de suite clairement que cet élément ne joue que comme {\itshape résultat} d'actes conscients accomplis par des individus, et par suite dépend de la conscience, loin de la déterminer. De plus, si Marx, contrairement a tous les penseurs qui l'ont précédé, juge nécessaire de mettre à part une forme particulière de l'existence, qu'il nomme sociale, c'est donc qu'il l'oppose tacitement au reste de l'existence, à savoir la nature.\par

\begin{center}
\end{center}
\section[Examen critique des idées de révolution et de progrès ]{Examen critique des idées de révolution et de progrès \protect\footnotemark }\renewcommand{\leftmark}{Examen critique des idées de révolution et de progrès }

\footnotetext{ Ce texte constitue peut-être une nouvelle rédaction du début des {\itshape Réflexions sur les causes de la liberté et de l'oppression sociale.}}
\noindent \par
Un mot magique, aujourd'hui, semble capable de compenser toutes les souffrances, de satisfaire toutes les inquiétudes, de venger le passé, de remédier aux malheurs présents, de résumer toutes les possibilités d'avenir. C'est le mot de révolution. Il ne date pas d'hier. Il date de plus d'un siècle et demi. Un premier essai d'application, de 1789 à 1793, a donné quelque chose, mais non pas ce qu'on en attendait. Depuis, chaque génération de révolutionnaires se croit, dans sa jeunesse, désignée pour faire la vraie révolution, puis vieillit peu à peu et meurt en reportant ses espérances sur les générations suivantes ; elle ne risque pas d'en recevoir le démenti, puisqu'elle meurt. Ce mot a suscité des dévouements si purs, fait couler à plusieurs reprises un sang si généreux, constitué pour tant de malheureux la seule source du courage de vivre qu’il est presque sacrilège de l'examiner ; tout cela n'empêche pourtant pas que peut-être il ne soit vidé de sens. Les martyrs ne remplacent les preuves que pour les prêtres.\par
Si on considère le régime qu'il s'agirait d'abolir, le mot de révolution semble n'avoir jamais été si actuel, car, de toute évidence, ce régime est bien malade. Si on se retourne du côté des successeurs éventuels, on aperçoit une situation paradoxale. En ce moment, aucun mouvement organisé ne prend effectivement le mot de révolution comme un mot d'ordre déterminant l'orientation de l'action et de la propagande. Pourtant jamais on ne s'est tant réclamé de ce mot d'ordre ; et surtout il touche individuellement tous ceux que les conditions d'existence actuelles font souffrir dans leur chair ou dans leur âme, tous ceux qui sont des victimes ou qui simplement se croient des victimes, tous ceux aussi qui prennent généreusement à coeur le sort des victimes qui les entourent, bien d'autres encore. Ce mot renferme la solution de tous les problèmes insolubles. Les ravages de la guerre passée, la préparation d'une guerre éventuelle pèsent sur les peuples d'un poids de plus en plus écrasant ; chaque désordre dans la circulation de la monnaie et des produits, dans le crédit, dans les investissements, se répercute en atroces misères ; le progrès technique semble apporter au peuple plus de surmenage et d'insécurité que de bien-être ; tout cela s'évanouira a l'instant où sonnera l'heure de la révolution.\par
L'ouvrier qui, à l'usine, contraint à une obéissance passive, a un travail morne et monotone, « trouve le temps long », ou qui ne se croit pas fait pour le travail manuel, ou qui est persécuté par un chef, ou qui souffre, à la sortie, de ne pouvoir se procurer tel ou tel plaisir offert aux consommateurs bien munis d'argent, songe à la révolution. Le petit commerçant malheureux, le rentier ruine tournent les yeux vers la révolution. L'adolescent bourgeois en rébellion contre le milieu familial et la contrainte scolaire, l'intellectuel en mal d'aventures et qui s'ennuie, rêvent de révolution, L'ingénieur heurté à la fois dans sa raison et dans son amour-propre par la prédominance des considérations financières sur les considérations techniques, et qui voudrait voir la technique régir l'univers, aspire à la révolution. La plupart de ceux qui ont vivement à coeur la liberté, l'égalité, le bien-être général, qui souffrent de voir des misères et des injustices, attendent une révolution. Si on prenait un à un tous ceux à qui il est arrivé de prononcer avec espoir le mot de révolution, si on cherchait les mobiles réels qui ont orienté chacun d'eux dans ce sens, les changements précis, d'ordre général ou personnel, auxquels il aspire réellement, on verrait quelle extraordinaire diversité d'idées et de sentiments peut recouvrir un même mot. On s'apercevrait que la révolution d'un homme n'est pas toujours celle du voisin, il s’en faut, que même bien souvent elles sont incompatibles. On trouverait aussi qu'il n'y a souvent aucun rapport entre les aspirations de toute espèce que traduit ce mot dans la pensée des hommes qui le prononcent et les réalités auxquelles il est susceptible de correspondre au cas où l'avenir apporterait effectivement un bouleversement social.\par
Au fond on pense aujourd'hui à la révolution non comme à une solution des problèmes posés par l'actualité, mais comme à un miracle dispensant de résoudre les problèmes. La preuve qu'on la considère ainsi, c'est qu'on attend qu'elle tombe du ciel ; on attend qu'elle se fasse, on ne se demande pas qui la fera. Peu de gens sont assez naïfs pour compter à cet égard sur les grandes organisations, syndicales ou politiques, qui avec plus ou moins de conviction persistent à se réclamer d'elle. Dans leurs états-majors, quoique non totalement dépourvus d'hommes de valeur, le regard le plus optimiste ne pourrait apercevoir l'embryon d'une équipe capable de mener à bien une tâche de cette envergure. Les cadres de second plan, les jeunes, ne donnent aucune marque qu'ils puissent renfermer les éléments d'une telle équipe. D'ailleurs ces organisations reflètent une bonne part des tares qu'elles dénoncent dans la société où elles évoluent ; elles en renferment même d'autres plus graves, à cause de l'influence qu'exerce sur elles à distance un certain régime totalitaire pire que le régime capitaliste. Les petits groupements, d'allure extrémiste ou modérée, qui accusent les grandes organisations de ne rien faire et mettent une persévérance si touchante à annoncer la bonne nouvelle, seraient plus embarrassés encore pour désigner des hommes capables d'être les accoucheurs d'un ordre nouveau.\par
On se fie, il est vrai, ou du moins on le feint, à la spontanéité des masses. Juin 1936 a donné un exemple émouvant de cette spontanéité qu'on avait pu croire tuée, en France, dans le sang de la Commune. Un grand élan, sorti des entrailles de la masse, ingouvernable, a desserré soudain l'étau de la contrainte sociale, rendu l'atmosphère enfin respirable, changé les opinions dans tous les esprits, fait admettre comme évidentes des choses tenues six mois plus tôt pour scandaleuses. Grâce à l'incomparable puissance de persuasion que possède la force, des millions d'hommes ont fait apparaître, et d'abord à leurs propres yeux, qu'ils avaient part aux droits sacrés de l'humanité, ce que des intelligences même pénétrantes n'avaient pu apercevoir au temps où ils étaient faibles. Mais c'est tout. Sauf dans le sens d'un bouleversement plus profond, il ne pouvait y avoir autre chose. Les masses ne posent pas de problèmes, n'en résolvent pas ; donc elles n’organisent ni ne construisent, D'ailleurs elles aussi, profondément imprégnées des tares du régime où elles vivent, peinent et souffrent. Leurs aspirations portent la marque du régime. La société capitaliste ramène tout aux francs, aux sous, aux centimes ; les aspirations des masses aussi s'expriment principalement en francs, en sous, en centimes. Le régime repose sur l'inégalité ; les masses expriment des revendications inégales. Le régime repose sur la contrainte ; les masses, dès qu'elles ont droit à la parole, exercent dans leurs propres rangs une contrainte du même genre. On voit mal comment il pourrait surgir des masses, spontanément, le contraire du régime qui les a formées, ou plutôt déformées.\par
On se fait une étrange idée de la révolution, à examiner la chose de près. D'ailleurs, dire qu'on s'en fait une idée, c'est beaucoup dire. À quoi les révolutionnaires croient-ils pouvoir reconnaître le moment où il y aura révolution ? Aux barricades et aux fusillades dans les rues ? À l'installation au gouvernement d'une certaine équipe d'hommes ? À la violation de la légalité ? À certaines nationalisations ? À l'émigration massive des bourgeois ? À la promulgation d'un décret supprimant la propriété privée ? Tout cela n'est pas clair. Mais enfin il reste qu'on attend, sous le nom de révolution, un moment où les derniers seront les premiers, où les valeurs niées ou abaissées par le régime actuel surgiront au premier plan, où les esclaves, sans abandonner d'ailleurs leurs tâches, seront les seuls citoyens, où les fonctions sociales vouées aujourd'hui à la soumission, à l'obéissance et au silence auront les premières droit à la parole et à la délibération dans toutes les affaires d'intérêt public. Il ne s'agit pas là de prophéties religieuses. On présente un tel avenir comme correspondant au cours normal de l'histoire. C'est qu'on ne se fait aucune idée juste du cours normal de l'histoire. Même quand on l'a étudiée, on reste pénétré par le souvenir vague des manuels d'école primaire et des chronologies.\par
On se réclame de l'exemple de 1789. On nous dit que, ce que la bourgeoisie a fait par rapport à la noblesse en 1799, le prolétariat le fera par rapport à la bourgeoisie en une année non déterminée. On se figure qu'en cette année 1789, ou du moins de 1789 à 1793, une couche sociale jusque-là subalterne, la bourgeoisie, a chassé et remplacé ceux qui géraient la société, les rois et les nobles. De la même manière, on croit qu'à un certain moment qu'on désigne sous le nom de Grandes Invasions les barbares ont envahi l'Empire romain, ont brisé les cadres de l'Empire, réduit les Romains à un état très subalterne, et pris le commandement partout. Pourquoi les prolétaires n'en feraient-ils pas autant, à leur manière ? En effet, il en est ainsi dans les manuels. Dans les manuels, l'Empire romain dure jusqu'au moment où commencent les Grandes Invasions ; après quoi, c'est un nouveau chapitre. Dans les manuels, le roi, la noblesse et le clergé possèdent la France. Jusqu'au jour où on prend la Bastille ; ensuite, c'est le Tiers-État. Cette notion catastrophique de l'histoire, où les catastrophes sont marquées par les fins ou les débuts de chapitres, nous l'avons tous absorbée pendant des années ; nous ne nous en débarrassons pas, et nous réglons notre action sur elle. La division des manuels d'histoire en chapitres nous vaudra bien des erreurs désastreuses.\par
Cette division ne correspond à rien de ce qu'on sait concernant le passé. Il n'y a pas eu substitution violente des premières formes de la féodalité a l’Empire romain. Dans l'Empire lui-même, les barbares s'étaient mis à occuper les postes les plus importants, les Romains tombaient peu à peu à des places honorifiques ou subalternes, l'armée se disloquait en bandes menées par des aventuriers, le colonat remplaçait peu à peu l'esclavage, tout cela bien avant les grandes invasions. De même, en 1789, il y avait longtemps que la noblesse était réduite à une situation presque parasite. Un siècle plus tôt, Louis XIV, si fier envers les plus hauts personnages, devenait déférent devant un banquier. Les bourgeois occupaient les plus hautes fonctions de l'État, régnaient sous le nom du roi, exerçaient les magistratures, dirigeaient les entreprises industrielles et commerciales, s'illustraient dans les sciences et la littérature, et ne laissaient guère aux nobles qu'un monopole, celui des fonctions d'officiers supérieurs. On pourrait citer d'autres exemples.\par
Quand il semble qu'une lutte sanglante substitue un régime à un autre, cette lutte est en réalité la consécration d'une transformation déjà plus qu'à moitié accomplie, et amène au pouvoir une catégorie d'hommes qui le possédaient déjà plus qu'à moitié. Il y a la une nécessité. Comment pourrait-il y avoir rupture de continuité dans la vie sociale, puisqu'il faut manger, se vêtir, produire et échanger, commander et obéir tous les jours, et que tout cela ne peut se faire aujourd'hui que sous des formes sensiblement semblables à celles d'hier ? C'est sous un régime en apparence stable que s'opèrent lentement des transformations dans la structure des rapports sociaux, des changements dans les attributions des diverses catégories sociales. Les luttes violentes, quand elles se produisent, et elles ne se produisent pas toujours, ne jouent que le rôle de balances ; elles donnent le pouvoir à ceux qui l'ont déjà. C'est ainsi, pour s'en tenir a ces deux exemples, que les grandes invasions ont livré l'Empire romain aux barbares, qui s'en étaient déjà emparés du dedans, et que la prise de la Bastille, avec ce qui s'en est suivi, a consolidé l'État moderne, que les rois avaient constitué, et livré le pays aux bourgeois, qui y faisaient déjà à peu près tout. Si la révolution d'Octobre, en Russie, semble avoir crée de toutes pièces du nouveau, ce n'est qu'une apparence ; elle a seulement renforcé les pouvoirs qui déjà étaient les seuls réels sous le tsarisme, la bureaucratie, la police, l'armée. Ce genre d'événements abolit les privilèges qui ne correspondent à aucune fonction effective, mais ne bouleverse pas la répartition de ces fonctions et des pouvoirs qui y sont attachés. Aujourd'hui, il pourrait bien arriver que les financiers, les spéculateurs, les actionnaires, les collectionneurs de sièges d'administrateurs, les petits commerçants, les rentiers, tous ces, parasites petits et grands, soient un beau jour balayés. Cela pourrait bien aussi s'accompagner d'événements violents. Mais comment croire que ceux qui peinent en esclaves dans les usines et les mines deviendront, du coup, des citoyens dans une économie nouvelle ? D'autres qu'eux seront les bénéficiaires de l'opération.\par
Ceux qui prétendent appuyer de raisonnements, et même de raisonnements scientifiques, leur croyance en une révolution se réclament tous de Marx. Le socialisme dit scientifique crée par Marx est passé à l'état de dogme, comme d'ailleurs tous les résultats établis par la science moderne, et on accepte une fois pour toutes les conclusions sans jamais s'enquérir des méthodes et des démonstrations. On aime mieux croire que Marx a démontré la constitution future et prochaine d'une société socialiste, plutôt que de chercher dans ses oeuvres si on y peut trouver même la moindre tentative de démonstration. Marx, il est vrai, analyse et démonte avec une admirable clarté le mécanisme de l'oppression capitaliste ; mais il en rend si bien compte qu'on ne peut guère se représenter comment, avec les mêmes rouages, le mécanisme pourrait un beau jour se transformer -\_ au point que l'oppression s'évanouisse progressivement...\par

\begin{center}
\end{center}
\section[Méditation sur l’obéissance et la liberté]{Méditation sur l’obéissance et la liberté}\renewcommand{\leftmark}{Méditation sur l’obéissance et la liberté}

\noindent \par
La soumission du plus grand nombre au plus petit, ce fait fondamental de presque toute organisation sociale, n'a pas fini d'étonner tous ceux qui réfléchissent un peu. Nous voyons, dans la nature, les poids les plus lourds l'emporter sur les moins lourds, les races les plus prolifiques étouffer les autres. Chez les hommes, ces rapports si clairs semblent renverses. Nous savons, certes, par une expérience quotidienne, que l'homme n'est pas un simple fragment de la nature, que ce qu'il y a de plus élevé chez l'homme, la« volonté, l'intelligence, la foi, produit tous les jours des espèces de miracles. Mais ce n'est pas ce dont il s'agit ici. La nécessité impitoyable qui a maintenu et maintient sur les genoux les masses d'esclaves, les masses de pauvres, les masses de subordonnés, n'a rien de spirituel ; elle est analogue à tout ce qu'il y a de brutal dans la nature. Et pourtant elle s'exerce apparemment en vertu de lois contraires à celles de la nature. Comme si, dans la balance sociale, le gramme l'emportait sur le kilo.\par
Il y a près de quatre siècles, le jeune La Boétie, dans son {\itshape Contre-un}, posait la question. Il n'y répondait pas. De quelles illustrations émouvantes pourrions-nous appuyer son petit livre, nous qui voyons aujourd'hui, dans un pays qui couvre le sixième du globe, un seul homme saigner toute une génération ! C'est quand sévit la mort que le miracle de l'obéissance éclate aux yeux. Que beaucoup d'hommes se soumettent à un seul par crainte d'être tués par lui, c'est assez étonnant ; mais qu'ils restent soumis au point de mourir sur son ordre, comment le comprendre ? Lorsque l'obéissance comporte au moins autant de risques que la rébellion, comment se maintient-elle ?\par
La connaissance du monde matériel où nous vivons a pu se développer à partir du moment où Florence, après tant d'autres merveilles, a apporté à l'humanité, par l'intermédiaire de Galilée, la notion de force. C'est alors aussi seulement que l'aménagement du milieu matériel par l'industrie a pu être entrepris. Et nous, qui prétendons aménager le milieu social, nous n'en posséderons pas même la connaissance la plus grossière aussi longtemps que nous n'aurons pas clairement conçu la notion de force sociale. La société ne peut pas avoir ses ingénieurs aussi longtemps qu'elle n'aura pas eu son Galilée. Y a-t-il en ce moment, sur toute la surface de la terre, un esprit qui conçoive même vaguement comment il se peut qu'un homme, au Kremlin, ait la possibilité de faire tomber n'importe quelle tête dans les limites des frontières russes ?\par
Les marxistes n'ont pas facilité une vue claire du problème en choisissant l'économie comme clef de l'énigme sociale. Si l'on considère une société comme un être collectif, alors ce gros animal, comme tous les animaux, se définit principalement par la manière dont il s'assure la nourriture, le sommeil, la protection contre les - intempéries, bref la vie. Mais la société considérée dans son rapport avec l'individu ne peut pas se définir simplement par les modalités de la production. On a beau avoir recours a toutes sortes de subtilités pour faire de la guerre un phénomène essentiellement économique, il éclate aux yeux que la guerre est destruction et non production. L'obéissance et le commandement sont aussi des phénomènes dont les conditions de la production ne suffisent pas à rendre compte. Quand un vieil ouvrier sans travail et sans secours périt silencieusement dans la rue ou dans un taudis, cette soumission qui s'étend jusque dans la mort ne peut pas s'expliquer par le jeu des nécessités vitales. La destruction massive du blé, du café, pendant la crise est un exemple non moins clair. La notion de force et non la notion de besoin constitue la clef qui permet de lire les phénomènes sociaux.\par
Galilée n'a pas eu à se louer, personnellement, d'avoir mis tant de génie et tant de probité à déchiffrer la nature ; du moins ne se heurtait-il qu'à une poignée d'hommes puissants spécialisés dans l'interprétation des Écritures. L'étude du mécanisme social, elle, est entravée par des passions qui se retrouvent chez tous et chez chacun. Il n'est presque personne qui ne désire soit bouleverser, soit conserver les rapports actuels de commandement et de soumission. L'un et l'autre désir met un brouillard devant le regard de l'esprit, et empêche d'apercevoir les leçons de l'histoire, qui montre partout les masses sous le joug et quelques-uns levant le fouet.\par
Les uns, du côté qui fait appel aux masses, veulent montrer que cette situation est non seulement inique, mais aussi impossible, du moins pour l'avenir proche ou lointain. Les autres, du côté qui désire conserver l'ordre et les privilèges, veulent montrer que le joug pèse peu, ou même qu'il est consenti. Des deux côtés, on jette un voile sur l'absurdité radicale du mécanisme social, au lieu de regarder bien en face cette absurdité apparente et de l'analyser pour y trouver le secret de la machine. En quelque matière que ce soit, il n'y a pas d'autre méthode pour réfléchir. L'étonnement est le père de la sagesse, disait Platon.\par
Puisque le grand nombre obéit, et obéit jusqu'à se laisser imposer la souffrance et la mort, alors que le petit nombre commande, c'est qu'il n'est 'pas vrai que le nombre soit une force. Le nombre, quoi que l'imagination nous porte à croire, est une faiblesse. La faiblesse est du côté où on a faim, où on s'épuise, où on supplie, où on tremble, non du côté où on vit bien, où on accorde des grâces, où on menace. Le peuple n'est pas soumis bien qu'il soit le nombre, mais parce qu'il est le nombre. Si dans la rue un homme se bat contre vingt, il sera sans doute laissé pour mort sur le pavé. Mais sur un signe d'un homme blanc, vingt coolies annamites peuvent être frappés a coups de chicotte, l'un après l'autre, par un ou deux chefs d'équipe.\par
La contradiction n'est peut-être qu'apparente. Sans doute, en toute occasion, ceux qui ordonnent sont moins nombreux que ceux qui obéissent. Mais précisément parce qu'ils sont peu nombreux, ils forment un ensemble. Les autres, précisément parce qu'ils sont trop nombreux, sont un plus un plus un, et ainsi de suite. Ainsi la puissance d'une infime minorité repose malgré tout sur la force du nombre. Cette minorité l'emporte de beaucoup en nombre sur chacun de ceux qui composent le troupeau de la majorité. Il ne faut pas en conclure que l'organisation des masses renverserait le rapport ; car elle est impossible. On ne peut établir de cohésion qu'entre une petite quantité d'hommes. Au delà, il n'y a plus que juxtaposition d'individus, c'est-à-dire faiblesse.\par
Il y a cependant des moments où il n'en est pas ainsi. À certains moments de l'histoire, un grand souffle passe sur les masses ; leurs respirations, leurs paroles, leurs mouvements se confondent. Alors rien ne leur résiste. Les puissants connaissent à leur tour, enfin, ce que c'est que de se sentir seul et désarmé ; et ils tremblent. Tacite, dans quelques pages immortelles qui décrivent une sédition militaire, a su parfaitement analyser la chose. « Le principal signe d'un mouvement profond, impossible à apaiser, c'est qu'ils n'étaient pas disséminés ou manoeuvrés par quelques-uns, mais ensemble ils prenaient feu, ensemble ils se taisaient, avec une telle unanimité et une telle fermeté qu'on aurait cru qu'ils agissaient au commandement. » Nous avons assisté à un miracle de ce genre en juin 1936, et l'impression ne s'en est pas encore effacée.\par
De pareils moments ne durent pas, bien que les malheureux souhaitent ardemment les voir durer toujours. Ils ne peuvent pas durer, parce que cette unanimité, qui se produit dans le feu d'une émotion vive et générale, n'est compatible avec aucune action méthodique. Elle a toujours pour effet de suspendre toute action, et d'arrêter le cours quotidien de la vie. Ce temps d'arrêt ne peut se prolonger ; le cours de la vie quotidienne doit reprendre, les besognes de chaque jour s'accomplir. La masse se dissout de nouveau en individus, le souvenir de sa victoire s'estompe ; la situation primitive, ou une situation équivalente, se rétablit peu à peu ; et bien que dans l'intervalle les maîtres aient pu changer, ce sont toujours les mêmes qui obéissent.\par
Les puissants n'ont pas d'intérêt plus vital que d'empêcher cette cristallisation des foules soumises, ou du moins, car ils ne peuvent pas toujours l'empêcher, de la rendre le plus rare possible. Qu'une même émotion agite en même temps un grand nombre de malheureux, c'est ce qui arrive très souvent par le cours naturel des choses ; mais d'ordinaire cette émotion, à peine éveillée, est réprimée par le sentiment d'une impuissance irrémédiable. Entretenir ce sentiment d'impuissance, c'est le premier article d'une politique habile de la part des maîtres.\par
L'esprit humain est incroyablement flexible, prompt à imiter, prompt à plier sous les circonstances extérieures. Celui qui obéit, celui dont la parole d'autrui détermine les mouvements, les peines, les plaisirs, se sent inférieur non par accident, mais par nature. À l’autre bout de l'échelle, on se sent de même supérieur, et ces deux illusions se renforcent l'une l'autre. Il est impossible à l'esprit le plus héroïquement ferme de garder la conscience d'une valeur intérieure, quand cette conscience ne s'appuie sur rien d'extérieur. Le Christ lui-même, quand il s'est vu abandonné de tous, bafoué, méprisé, sa vie comptée pour rien, a perdu un moment le sentiment de sa mission ; que peut vouloir dire d'autre le cri : « Mon Dieu, pourquoi m'avez-vous abandonné ? » Il semble à ceux qui obéissent que quelque infériorité mystérieuse les a prédestinés de toute éternité à obéir ; et chaque marque de mépris, même infime, qu'ils souffrent de la part de leurs supérieurs ou de leurs égaux, chaque ordre qu'ils reçoivent, surtout chaque acte de soumission qu'ils accomplissent eux-mêmes les confirme dans ce sentiment.\par
Tout ce qui contribue à donner à ceux qui sont en bas de l'échelle sociale le sentiment qu'ils ont une valeur est dans une certaine mesure subversif. Le mythe de la Russie soviétique est subversif pour autant qu'il peut donner au manoeuvre d'usine communiste renvoyé par son contremaître le sentiment que malgré tout il a derrière lui l'armée rouge et Magnitogorsk, et lui permettre ainsi de conserver sa fierté. Le mythe de la révolution historiquement inéluctable joue le même rôle, quoique plus abstrait ; c'est quelque chose, quand on est misérable et seul, que d'avoir pour soi l'histoire. Le christianisme, dans ses débuts, était lui aussi dangereux pour l'ordre. Il n'inspirait pas aux pauvres, aux esclaves, la convoitise des biens et de la puissance, tout au contraire ; mais il leur donnait le sentiment d'une valeur intérieure qui les mettait sur le même plan ou plus haut que les riches, et c'était assez pour mettre la hiérarchie sociale en péril. Bien vite il s'est corrigé, a appris à mettre entre les mariages, les enterrements des riches et des pauvres la différence qui convient, et a reléguer les malheureux, dans les églises, aux dernières places.\par
La force sociale ne va pas sans mensonge. Aussi tout ce qu'il y a de plus haut dans la vie humaine, tout effort de pensée, tout effort d'amour est corrosif pour l'ordre. La pensée peut aussi bien, à aussi juste titre, être flétrie comme révolutionnaire d'un côté, comme contre-révolutionnaire de l'autre. Pour autant qu'elle construit sans cesse une échelle de valeurs « qui n'est pas de ce monde », elle est l'ennemie des forces qui dominent la société. Mais elle n'est pas plus favorable aux entreprises qui tendent à bouleverser ou à transformer la société, et qui, avant même d'avoir réussi, doivent nécessairement impliquer chez ceux qui s'y vouent la soumission du plus grand nombre au plus petit, le dédain des privilégiés pour la masse anonyme et le maniement du mensonge. Le génie, l'amour, la sainteté méritent pleinement le reproche qu'on leur fait bien. des fois de tendre à détruire ce qui est sans rien construire à la place. Quant à ceux qui veulent penser, aimer, et transposer en toute pureté dans l'action politique ce que leur inspire leur esprit et leur cœur, ils ne peuvent que périr égorgés, abandonnés même des leurs, flétris après leur mort par l'histoire, comme ont fait les Gracques.\par
Il résulte d'une telle situation, pour tout homme amoureux du bien public, un déchirement cruel et sans remède. Participer, même de loin, au jeu des forces qui meuvent l'histoire n'est guère possible sans se souiller ou sans se condamner d'avance a la défaite. Se réfugier dans l'indifférence ou dans une tour d'ivoire n'est guère possible non plus sans beaucoup d'inconscience. La formule du « moindre mal », si décriée par l'usage qu’en ont fait les social-démocrates, reste alors la seule applicable, à condition de l'appliquer avec la plus froide lucidité.\par
L'ordre social, quoique nécessaire, est essentiellement mauvais, quel qu'il soit. On ne peut reprocher à ceux qu'il écrase de le saper autant qu'ils peuvent ; quand ils se résignent, ce n'est pas par vertu, c'est au contraire sous l'effet d'une humiliation qui éteint chez eux les vertus viriles. On ne peut pas non plus reprocher à ceux qui l'organisent de le défendre, ni les représenter comme formant une conjuration contre le bien général. Les luttes entre concitoyens ne viennent pas d'un manque de compréhension ou de bonne volonté ; elles tiennent à la, nature des choses, et ne peuvent pas être apaisées, mais seulement étouffées par la contrainte. Pour quiconque aime la liberté, il n'est pas désirable qu'elles disparaissent, mais seulement qu'elles restent en deçà d'une certaine limite dé violence.\par

\begin{center}
\end{center}
\section[Sur les contradictions du marxisme]{Sur les contradictions du marxisme}\renewcommand{\leftmark}{Sur les contradictions du marxisme}

\noindent \par
À mes yeux, ce ne sont pas les événements qui imposent une révision du marxisme, c'est la doctrine de Marx qui, en raison des lacunes et des incohérences qu'elle renferme, est et a toujours été très au-dessous du rôle qu'on a voulu lui faire jouer ; ce qui ne signifie pas qu'il ait été élaboré alors ou depuis quelque chose de mieux. Ce qui me fait exprimer un jugement si catégorique, et si propre à déplaire, c'est le souvenir de mon expérience propre. Quand, étant encore dans l'adolescence, j'ai lu pour la première fois le Capital, certaines lacunes, certaines contradictions de première importance m'ont tout de suite sauté aux yeux. Leur évidence même, à ce moment, m'a empêchée de faire confiance à mon propre jugement ; je me disais que tant de grands esprits, qui ont adhéré au marxisme, avaient dû apercevoir aussi ces incohérences, ces lacunes si claires ; qu'elles avaient donc certainement été les unes comblées, les autres résolues, dans d'autres ouvrages de doctrine marxiste. À combien d'esprits jeunes n'arrive-t-il pas ainsi d'étouffer, par défiance d'eux-mêmes, leurs doutes les mieux fondés ? Pour moi, dans les années qui suivirent, l'étude des textes marxistes, des partis marxistes ou soi-disant tels, et des événements eux-mêmes n'a pu que confirmer le jugement de mon adolescence. Ce n'est donc pas par comparaison avec les faits, c'est en elle-même que j'estime la doctrine marxiste défectueuse ; ou plutôt, je pense que l'ensemble des écrits rédigés par Marx, Engels et ceux qui les ont pris comme guides ne forme pas une doctrine.\par
Il y a contradiction, contradiction évidente, éclatante, entre la méthode d'analyse de Marx et ses conclusions. Ce n'est pas étonnant : il a élaboré les conclusions avant la méthode. La prétention du marxisme à être une science est dès lors assez plaisante. Marx est devenu révolutionnaire dans sa jeunesse, sous l'emprise de sentiments généreux ; son idéal de cette époque était d'ailleurs humain, clair, conscient, raisonné, autant et même bien plus que par la suite de sa vie. Plus tard, il a tenté « d'élaborer une méthode pour l'étude des sociétés humaines. Sa force d'esprit ne lui permettait pas de fabriquer une simple caricature de méthode ; il a vu ou du moins entrevu une méthode véritable. Tels sont les deux apports faits par lui dans l'histoire de la pensée : il a aperçu, dans sa jeunesse, une formule neuve de l'idéal social, et, dans son âge mûr, la formule neuve ou partiellement neuve d'une méthode dans l'interprétation de l'histoire. Il a ainsi fait doublement preuve de génie. Par malheur, répugnant, comme tous les caractères forts, à laisser subsister en lui deux hommes, le révolutionnaire et le savant, répugnant aussi à l'espèce d'hypocrisie qu'implique l'adhésion à un idéal non accompagnée d'action, trop peu scrupuleux d'ailleurs à l'égard de sa propre pensée, il a tenu a faire de sa méthode un instrument pour prédire un avenir conforme a ses vœux. À cet effet, il lui a fallu donner un coup de pouce et à la méthode et à l'idéal, les déformer l'une et l'autre. Dans le relâchement de sa pensée qui a permis de telles déformations, il s'est laisse aller, lui, le non-conformiste, à une conformité inconsciente avec les superstitions les moins fondées de son époque, le culte de la production, le culte de la grande industrie, la croyance aveugle au progrès. Il a porté ainsi un tort grave, durable, peut-être irréparable, en tout cas difficile à réparer, à la fois à l'esprit scientifique et à l'esprit révolutionnaire. Je ne crois pas que le mouvement ouvrier redevienne dans notre pays quelque chose de vivant tant qu’il ne cherchera pas, je ne dis pas des doctrines, mais une source d'inspiration dans ce que Marx et les marxistes ont combattu et bien follement méprisé : dans Proudhon, dans les groupements ouvriers de 1848, dans la tradition syndicale, dans l'esprit anarchiste. Quant à une doctrine, l'avenir seul, au meilleur des cas, pourra peut-être en fournir une non le passé.\par
La conception que Marx se faisait des révolutions peut s'exprimer ainsi : une révolution se produit au moment où elle est déjà à peu près accomplie, c'est quand la structure d'une société a cessé de correspondre aux institutions que les institutions changent, et sont remplacées par d'autres qui reflètent la structure nouvelle. Notamment la partie de la société à qui la révolution donne le pouvoir est celle qui dès avant la révolution, quoique brimée par les institutions, jouait en fait le rôle le plus actif. D'une manière générale, le « matérialisme historique », si souvent mal compris, signifie que les institutions sont déterminées par le mécanisme effectif des rapports entre les hommes, lequel dépend lui-même de la forme que prennent à chaque moment les rapports entre l'homme et la nature, c'est-à-dire de la manière dont s'accomplit la production ; production des biens consommables, production des moyens de produire, et aussi - point important, bien que Marx le laisse dans l'ombre - production des moyens de combat. Les hommes ne sont pas des jouets impuissants du destin ; ce sont des êtres éminemment actifs ; mais leur activité; est à chaque instant limitée par la structure de la société qu'ils constituent entre eux, et ne modifie à son tour cette structure que par contrecoup, une fois qu'elle a modifié les rapports entre eux et la nature. La structure sociale ne peut jamais être modifiée qu'indirectement.\par
D'autre part l'analyse du régime actuel, analyse qui se trouve éparse dans plusieurs œuvres de Marx, place la source de l'oppression cruelle que souffrent les travailleurs non dans les hommes, non dans les institutions, mais dans le mécanisme même des rapports sociaux. Si les ouvriers sont épuisés de fatigue et de privations, c'est parce qu'ils ne sont rien et que le développement des entreprises est tout. Ils ne sont rien parce que le rôle de la plupart d'entre eux, dans la production, est un rôle de simples rouages, et ils sont dégradés à ce rôle de rouages parce que le travail intellectuel s'est séparé du travail manuel, et parce que le développement du machinisme a enlevé à l'homme le privilège de l'habileté pour le faire passer à la matière inerte. Le développement de l’entreprise est tout, parce que l'aiguillon de la concurrence contraint sans cesse les entreprises à s'agrandir pour subsister ; ainsi « le rapport entre la consommation et la production est renverse », « la consommation n'est qu'un mal nécessaire » ; et si les ouvriers ne touchent pas la valeur de leur travail, ce fait résulte simplement du « renversement du rapport entre le sujet et l'objet » qui sacrifie l'homme à l'outillage inerte, qui fait de la production des moyens de production le but suprême.\par
Le rôle de l'État donne lieu à une analyse semblable. Si l'État est oppressif, si ta démocratie est un leurre, c'est parce que l'État est composé de trois corps permanents, se recrutant par cooptation, distincts du peuple, à savoir l'armée, la police et la bureaucratie. Les intérêts de ces trois corps sont distincts des intérêts de la population, et par suite leur sont opposés. Ainsi la « machine de l'État » est oppressive par sa nature même, ses rouages ne peuvent fonctionner sans broyer les citoyens ; aucune bonne volonté ne peut en faire un instrument du bien public ; on ne peut l'empêcher d'opprimer qu'en la brisant. Au reste et, sur ce point, l'analyse de Marx est moins serrée - l'oppression exercée par la machine de l'État se confond avec l'oppression exercée par la grande industrie ; cette machine se trouve automatiquement au service de la principale force sociale, à savoir le capital, autrement dit l'outillage des entreprises industrielles. Ceux qui sont sacrifiés au développement de l'outillage industriel, c'est-à-dire les prolétaires, sont aussi ceux qui sont exposés à toute la brutalité de l'État, et l'État les maintient par force esclaves des entreprises.\par
Que conclure ? La conclusion s'impose à l'esprit c'est que rien de tout cela ne peut être aboli par une révolution ; au contraire, tout cela doit avoir disparu avant qu'une révolution puisse se produire ; ou, si elle se produit auparavant, ce ne sera qu'une révolution apparente, qui laissera l'oppression intacte ou même l'aggravera. Cependant Marx concluait exactement le contraire ; il concluait que la société était mûre pour une révolution libératrice. N'oublions pas qu'il y a près de cent ans il croyait déjà une telle révolution imminente. Sur ce point en tout cas, les faits lui ont infligé un démenti éclatant, éclatant en Europe et en Amérique, plus éclatant encore en Russie. Mais le démenti des faits était à peine utile ; dans la doctrine même de Marx, la contradiction était si éclatante qu'on peut s'étonner que ni lui, ni ses amis, ni ses disciples n'en aient pris conscience. Comment les facteurs d'oppression, si étroitement liés au mécanisme même de la vie sociale, devaient-ils soudain disparaître ? Comment est-ce que, la grande industrie, les machines et l'avilissement du travail manuel étant donnés, les ouvriers pouvaient être autre chose que de simples rouages dans les usines ? Comment, s'ils continuaient à être de simples rouages, pouvaient-ils en même temps devenir la « classe dominante » ? Comment, la technique du combat, celle de la surveillance, celle de l'administration étant données, les fonctions militaires, policières, administratives pouvaient-elles cesser d'être des spécialités, des professions, et par suite l'apanage de « corps permanents, distincts de la population » ? Ou bien faut-il admettre une transformation de l'industrie, de la machine, de la technique du travail manuel, de la technique de l'administration, de la technique de la guerre ? Mais de telles transformations sont lentes, progressives ; elles ne sont pas l'effet d'une révolution.\par
À de telles questions, qui découlent immédiatement des analyses de Marx, on peut affirmer que ni Marx, ni Engels, ni leurs disciples, n'ont apporté la moindre réponse. Ils les ont passées sous silence. Sur un seul point Marx et Engels ont signalé une transition possible du régime dit capitaliste vers une société meilleure ; ils ont cru voir que le développement même de la concurrence devait amener automatiquement, et dans un court délai, la disparition de la concurrence et en même temps celle de la propriété capitaliste. Effectivement la concentration des entreprises s'effectuait sous leurs yeux, comme elle s'effectue encore sous les nôtres. La concurrence étant ce qui, dans le régime capitaliste, fait du développement des. entreprises un but, et des hommes, considérés soit comme producteurs, soit comme consommateurs, un simple moyen, ils pouvaient considérer la disparition de la concurrence comme équivalente à la disparition du régime. Mais leur raisonnement péchait en un point ; du fait que la concurrence, qui fait manger les petits par les gros, diminue peu à peu le nombre des concurrents, on ne peut conclure que ce nombre doive un jour se réduire à l'unité. De plus, Marx et Engels, dans leur analyse, omettaient un facteur ; ce facteur, c'est la guerre. Jamais les marxistes n'ont analysé le phénomène de la guerre ni ses rapports avec le régime ; car je n'appelle pas analyse la simple affirmation que l'avidité des capitalistes est la cause des guerres. Quelle lacune ! Et quel crédit accorder à une théorie qui se dit scientifique, et qui est capable d'une pareille omission ? Or comme la production industrielle est de nos jours, non seulement le principal moyen d'enrichissement, mais aussi le principal moyen de combat militaire, il en résulte qu'elle est soumise non seulement à la concurrence entre entreprises, mais a une autre concurrence, plus pressante encore et plus impérieuse : la concurrence entre nations. Cette concurrence-là, comment l'abolir ? Doit-elle, comme l'autre, s'abolir par l'élimination progressive des concurrents ? Faut-il attendre, pour pouvoir espérer le socialisme, le jour où le monde se trouvera soumis à la « grande paix allemande » ou à la « grande paix japonaise » ? Ce jour n'est pas proche, à supposer qu'il doive jamais venir ; et les partis qui se réclament du socialisme font tout pour l'éloigner.\par
Les problèmes que le marxisme n'a pas résolus n'ont pas non plus été résolus par les faits ; ils sont de plus en plus aigus. Bien que les ouvriers vivent mieux qu'au temps de Marx - du moins dans les pays de race blanche, car il en est autrement, hélas, aux colonies ; et même la Russie doit peut-être être exceptée - les obstacles qui s'opposent à la libération des travailleurs sont plus durs qu'alors. Le système Taylor et ceux qui lui ont succède ont réduit les ouvriers bien plus encore qu'auparavant au rôle de simples rouages dans les usines ; à l'exception de quelques fonctions hautement qualifiées. Le travail manuel, dans la plupart des cas, est encore plus éloigné du travail de l'artisan, plus dénué d'intelligence et d'habileté, les machines sont encore plus oppressives. La course aux armements pousse plus impérieusement encore à sacrifier le peuple tout entier à la production industrielle. La machine de l'État se développe de jour en jour d'une manière plus monstrueuse, devient de jour en jour plus étrangère à l'ensemble de la population, plus aveugle, plus inhumaine. Un pays qui tenterait une révolution socialiste devrait très vite en arriver, pour se défendre contre les autres, à reproduire en les aggravant toutes les cruautés du régime qu'il aurait voulu abolir, sauf le cas où une révolution ferait tache d'huile, Sans doute peut-on espérer une pareille contagion, mais elle devrait être immédiate ou ne pas être, car une révolution dégénérée en tyrannie cesse d'être contagieuse ; et, entre autres obstacles, l'exaspération des nationalismes empêche qu'on puisse raisonnablement croire à l'extension immédiate d'une révolution dans plusieurs grands pays.\par
Ainsi la contradiction entre la méthode d'analyse élaborée par Marx et les espérances révolutionnaires qu'il a proclamées semble encore plus aiguë aujourd'hui qu'en son temps. Qu'en conclure ? Faut-il réviser le marxisme ? On ne révise pas ce qui n'existe pas, et il n'y a jamais eu de marxisme, mais plusieurs affirmations incompatibles, les unes fondées, les autres non ;par malheur, les mieux fondées sont les moins agréables. On nous demande encore si une telle révision doit être révolutionnaire. Mais qu'entend-on par révolutionnaire ? Ce mot souffre plusieurs interprétations. Etre révolutionnaire, est-ce attendre, dans un avenir prochain, une bienheureuse catastrophe, un bouleversement qui réalise sur cette terre une partie des promesses de l'Évangile, et nous donne enfin une société où les derniers seront les premiers ? Si c'est cela, je ne suis pas révolutionnaire, car un tel avenir, qui d'ailleurs me comblerait, est à mes yeux sinon impossible, au moins tout a fait improbable ; et je ne crois pas que quelqu'un puisse aujourd'hui avoir des raisons solides, sérieuses, d'être révolutionnaire en ce sens.\par
Ou bien, être révolutionnaire, est-ce appeler par ses voeux et aider par ses actes tout ce qui peut, directement et indirectement, alléger ou soulever le poids qui écrase la masse des hommes, les chaînes qui avilissent le travail, refuser les mensonges au moyen desquels on veut déguiser ou excuser l'humiliation systématique du plus grand nombre ? Dans ce cas il s'agit d'un idéal, d'un jugement de valeur, d'une volonté, et non pas d'une interprétation de l'histoire humaine et du mécanisme social. L'esprit révolutionnaire, pris en ce sens, est aussi ancien que l'oppression elle-même et durera autant qu'elle, plus longtemps même, car, si elle disparaît, il devra subsister pour l'empêcher de reparaître ; il est éternel ; il n'a pas à subir de révision, mais il peut s'enrichir, s'aiguiser, et il doit être purifié de tous les apports étrangers qui peuvent venir le déguiser et l'altérer. Cet éternel esprit de révolte qui animait les plébéiens de Rome, qui enflammait presque simultanément, vers la fin du XIVe e siècle, les ouvriers de la laine à Florence, les paysans anglais, les artisans de Gand, qu'a-t-il à prendre, pour se l'assimiler, dans l'œuvre de Marx ? Il a à y prendre ce qui a été précisément presque oublie par ce qu'on nomme le marxisme : la glorification du travail productif, conçu comme l'activité suprême de l'homme ; l'affirmation que seule une société où l'acte du travail mettrait en jeu toutes les facultés de l'homme, où l'homme qui travaille serait au premier rang, réaliserait la plénitude de la grandeur humaine. On trouve chez Marx, dans les écrits de jeunesse, des lignes d'accent lyrique concernant le travail ; on en trouve aussi chez Proudhon ; on en trouve aussi chez des poètes, chez Goethe, chez Verhaeren. Cette poésie nouvelle, propre à notre temps, et qui en fait peut-être la principale grandeur, ne doit pas se perdre. Les opprimés doivent y trouver l'évocation de leur patrie à eux, qui est une espérance.\par
Mais par ailleurs le marxisme a gravement altéré cet esprit de révolte qui, au siècle dernier, brillait d'un éclat si pur dans notre pays. Il y a mêlé à la fois des oripeaux faussement scientifiques, une éloquence messianique, un déchaînement d'appétits qui l'ont défiguré. Rien ne permet d'affirmer aux ouvriers que la science est avec eux. La science, c'est pour eux, comme d'ailleurs pour tous aujourd'hui, cette puissance mystérieuse qui, en un siècle, a transformé la face du monde au moyen de la technique industrielle ; quand on leur dit que la science est avec eux, ils croient aussitôt posséder une source illimitée de puissance. Il n'en est rien. On ne trouve pas, chez les communistes, socialistes ou syndicalistes de telle ou telle nuance, une connaissance plus claire ou plus précise de notre société et de son mécanisme que chez les bourgeois, les conservateurs ou les fascistes. Quand même les organisations ouvrières posséderaient une supériorité dans la connaissance qu'elles ne possèdent aucunement, elles n'auraient pas de ce fait entre les mains les moyens d'action indispensables ; la science n'est rien, pratiquement, sans les ressources de la technique, et elle ne les donne pas, elle permet seulement d'en user. Il serait plus faux encore de soutenir que la science permet de prévoir un triomphe prochain de la cause ouvrière ; cela n'est pas, et on ne peut même pas croire de bonne foi qu'il en soit ainsi si l'on ne ferme pas obstinément les yeux. Rien ne permet non plus d'affirmer aux ouvriers qu'ils ont une mission, une « tâche historique », comme disait Marx, qu'il leur incombe de sauver l'univers. Il n'y a aucune raison de leur supposer une pareille mission plutôt qu'aux esclaves de l'antiquité ou aux serfs du moyen âge. Comme les esclaves, comme les serfs, ils sont malheureux, injustement malheureux ; il est bon qu'ils se défendent, il serait beau qu'ils se libèrent ; il n'y a rien à en dire de plus. Ces illusions qu'on leur prodigue, dans un langage qui mélange déplorablement les lieux communs de la religion à ceux de la science, leur sont funestes. Car elles leur font croire que les choses vont être faciles, qu'ils sont poussés par derrière par un dieu moderne qu'on nomme Progrès, qu'une providence moderne, qu'on nomme l’Histoire, fait pour eux le plus gros de l'effort. Enfin rien ne permet de leur promettre, au terme de leur effort de libération, les jouissances et le pouvoir. Une ironie facile a fait beaucoup de mal en discréditant l'idéalisme élevé, l'esprit presque ascétique des groupes socialistes du début du XIXe siècle ; elle n'a abouti qu'à abaisser la classe ouvrière...\par

\begin{center}
\end{center}
\section[Fragments. Londres, 1943]{Fragments \\
Londres, 1943}\renewcommand{\leftmark}{Fragments \\
Londres, 1943}

\noindent \par
\subsection[I]{I}
\noindent L'image de la contradiction dans la matière, c'est le heurt entre forces opposées. Ce mouvement vers le bien, à travers les contradictions, que Platon a décrit comme étant celui de la créature pensante secourue par une grâce surnaturelle, Marx l'a purement et simplement attribué à la matière, mais a une certaine matière : à la matière sociale.\par
Il a été frappé par le fait que les groupes sociaux se fabriquent des morales à leur propre usage, morales par lesquelles chacun soustrait à l'atteinte du mal son activité spécifique. Il y a ainsi une morale de l'homme de guerre, une morale de l'homme d'affaires, et ainsi de suite, dont le premier article est de nier qu'on puisse commettre aucun mal quand on mène régulièrement la guerre, les affaires, et ainsi de suite. De plus, toutes les pensées qui circulent dans une société, quelle qu'elle soit, sont influencées par la morale particulière du groupe qui la domine. C'est là un fait qui n'a jamais été ignoré, et que Platon, par exemple, connaissait parfaitement.\par
Quand on l'a reconnu, on peut réagir de plusieurs manières, selon la profondeur de l'inquiétude morale. On peut le reconnaître pour les autres, mais l'ignorer pour soi. Cela signifie simplement qu'on admet comme absolue la morale particulière du milieu dont on se trouve être un membre. On est alors tranquille. Mais, du point de vue moral, on est mort. Le cas est extrêmement fréquent. Ou bien on peut se rendre compte de la misérable faiblesse de tout esprit humain. On est alors saisi par l'angoisse. Quelques-uns, pour échapper à cette angoisse, acceptent de laisser les mots « bien » et « mal » perdre toute signification. Ceux-là, au bout d'un temps plus ou moins long, se décomposent, tombent en pourriture. C'est peut-être ce qui serait arrivé à Montaigne sans l'influence de son ami stoïcien. D'autres cherchent anxieusement, désespérément, un chemin pour sortir du domaine des morales relatives et connaître le bien absolu. Parmi ceux-là on peut nommer des esprits de valeur très inégale, tels que Platon, Pascal, et, si étrange que cela puisse paraître, Marx.\par
Le vrai chemin existe. Platon et beaucoup d'autres l'ont parcouru. Mais il n'est ouvert qu'à ceux qui, se reconnaissant incapables de le trouver, ne le cherchent plus, et cependant ne cessent pas de le désirer à l'exclusion de toute autre chose. À ceux-là il est accordé de se nourrir d'un bien qui, étant situé hors de ce monde, n'est soumis à aucune influence sociale. C'est le pain transcendant dont il est question dans le texte original du {\itshape Pater}.\par
Marx a cherché autre chose, et il a cru trouver. Comme les mensonges en matière de morale émanent de groupes particuliers qui cherchent chacun à poser leur propre existence comme un bien absolu, il s'est dit que le jour où il n'y aurait plus de groupes particuliers les mensonges disparaîtraient. Il a admis, tout à fait arbitrairement, que le heurt des forces sociales amènerait un jour automatiquement cette destruction des groupes. Sentant irrésistiblement que la connaissance de la justice et de la vérité est en quelque sorte due à l'homme, dont le désir, en ce domaine, est trop profond pour admettre un refus ; ayant reconnu avec raison qu'aucun esprit humain, sans aucune exception, n'a la force de se soustraire aux facteurs de mensonge qui empoisonnent la vie sociale ; ignorant qu'il existe une source d'où cette force descend sur ceux qui la désirent avec une complète humilité, il a admis que la société, par un processus automatique de croissance, éliminera son propre poison. Il l'a admis sans aucune raison, sinon qu'il ne pouvait pas faire autrement.\par
C'est ainsi qu'il faut comprendre ce qui souvent apparaît chez lui comme la négation des notions mêmes de vérité, de justice, de valeur morale. La société étant encore empoisonnée, aucun esprit n'est capable d'accéder à la vérité et à la justice. Ceux qui prononcent ces mots mentent ou sont trompés par des menteurs. Celui qui veut servir la justice n'a qu'un moyen, c'est de hâter l'opération du mécanisme qui aboutira à une société sans poison. Peu importe de quels procédés il se sert à cet effet ; ils sont bons, s'ils sont efficaces. Ainsi Marx, exactement comme les hommes d'affaires de son temps ou les guerriers du moyen âge, aboutissait à une morale qui mettait au-dessus du péché la catégorie sociale dont il faisait partie, à savoir celle des révolutionnaires professionnels. Il retombait dans la faiblesse même qu'il avait fait tant d'efforts pour éviter, comme il arrive à tous ceux qui cherchent la force morale où elle n'est pas.\par
Quant à la nature de ce mécanisme producteur de paradis, il la déduisait d'un raisonnement presque puéril. Quand un groupe dominant cesse de dominer, il est remplacé par un groupe qui auparavant se trouvait naturellement plus bas. A force de répéter ce processus, la croissance sociale finit par amener en haut le groupe qui était tout en bas. Alors il n'y a plus de bas, plus d'oppression, plus d'intérêts de groupe contraires à l'intérêt général, plus de mensonge.\par
Autrement dit, à l'issue d'une évolution au cours de laquelle la force a changé de mains, un jour les faibles, demeurés tels, auront la force de leur côté. C'est là un exemple particulièrement absurde de la tendance à l'extrapolation qui était une des tares de la science et de toute la pensée du XIXe siècle, époque où, sauf les purs mathématiciens, on ignorait la notion de limite.\par
La force, en changeant de mains, demeure toujours une relation de plus fort à plus faible, une relation de domination. Elle peut changer de mains indéfiniment sans que jamais un terme de la relation soit éliminé. Au moment d'une transformation politique, ceux qui s'apprêtent à prendre le pouvoir possèdent déjà une force, c'est-à-dire une domination sur de plus faibles. S'ils n'en possèdent aucune, le pouvoir ne tombera pas entre leurs mains, à moins qu'il ne puisse intervenir un facteur efficace autre que la force ; ce que Marx n'admettait pas. Le matérialisme révolutionnaire de Marx consiste en somme a poser, d ‘une part que la force seule règle exclusivement les rapports sociaux, d'autre part qu'un jour les faibles, tout en demeurant les faibles, seraient quand même les plus forts. Il croyait au miracle sans croire au surnaturel. D'un point de vue Purement rationaliste, si l'on croit au miracle, il vaut mieux croire aussi à Dieu.\par
Ce qu'il y a au fond de la pensée de Marx, c'est une contradiction. Ce n'est pas à dire que la non-contradiction soit un critérium de vérité. Bien au contraire, la contradiction, comme Platon le savait, est l'unique instrument de la pensée qui s'élève. Mais il y a un usage légitime et un usage illégitime de la contradiction. L'usage illégitime consiste à combiner des affirmations incompatibles comme si elles étaient compatibles. L'usage légitime consiste, lorsque deux vérités incompatibles s'imposent à l'intelligence humaine, à les reconnaître comme telles, et à en faire pour ainsi dire les deux bras d'une pince, un instrument pour entrer indirectement en contact avec le domaine de la vérité transcendante inaccessible à notre intelligence. La contradiction ainsi maniée joue un rôle essentiel dans le dogme chrétien. Il serait facile de le montrer à propos d'un exemple comme celui de la Trinité. Elle joue un rôle analogue dans d'autres traditions. Il y a peut-être là un critérium pour discerner les traditions religieuses ou philosophiques authentiques.\par
La contradiction essentielle de la condition humaine, c'est que l'homme est soumis à la force, et désire la justice. Il est soumis à la nécessite, et désire le bien. Ce n'est pas son corps seul qui est ainsi soumis, mais aussi toutes ses pensées ; et pourtant l'être même de l'homme consiste à être tendu vers le bien. C'est pourquoi nous croyons tous qu'il y a une unité entre la nécessite et le bien. Certains croient que les pensées de l'homme concernant le bien possèdent ici-bas le plus haut degré de force. Ce sont ceux qu'on nomme les idéalistes. Ils se trompent doublement, d'abord en ce que ces pensées sont sans force, puis en ce qu'elles ne saisissent pas le bien. Elles sont influencées par la force ; de sorte que cette attitude est finalement une réplique moins énergique de l'attitude contraire. D'autres croient que la force est par elle-même orientée vers le bien. Ce sont des idolâtres. C'est là la croyance de tous les matérialistes qui ne tombent pas dans l'état d'indifférence. Ils se trompent aussi doublement ; d'abord la force est étrangère et indifférente au bien, puis elle n'est pas toujours et partout la plus forte. Seuls peuvent échapper à ces erreurs ceux qui ont recours à la pensée incompréhensible qu'il y a une unité entre la nécessité et le bien, autrement dit entre la réalité et le bien, hors de ce monde. Ceux-là croient aussi que quelque chose de cette unité se communique à ceux qui dirigent vers elle leur attention et leur désir. Pensée encore plus incompréhensible, mais expérimentalement vérifiée.\par
Marx était un idolâtre. Son idolâtrie avait pour objet la société future ; mais, comme tout idolâtre a besoin d'un objet présent, il la reportait sur la fraction de la société qu'il croyait sur le point d'opérer la transformation attendue, c'est-à-dire le prolétariat. Il se regardait comme étant son chef naturel, au moins pour la théorie et la stratégie générale ; mais en un autre sens il croyait recevoir de lui la lumière. Si on lui avait demandé pourquoi, toute pensée étant soumise aux fluctuations de la force, lui-même, Marx, ainsi qu’un grand nombre de ses contemporains, pensait continuellement à une société parfaitement juste, la réponse lui aurait été facile. À ses yeux, c'était là un effet mécanique de la transformation qui se préparait et qui, bien que non accomplie, était dans un état de germination assez avance pour se, refléter dans les pensées de quelques-uns. Il interprétait de même la soif de justice totale tellement ardente chez les ouvriers de cette époque.\par
Il avait raison en un sens. Presque tous les socialistes de ce temps, lui-même y compris, auraient sans doute été incapables de se mettre du côté des plus faibles si, à côté de la compassion causée par la faiblesse, il n'y avait eu le prestige lié à une apparence de force. Ce prestige venait non d'un avenir pressenti, mais d'un passé récent, de quelques scènes éclatantes et trompeuses de la Révolution française.\par
Les faits montrent que presque toujours les pensées des hommes sont façonnées, comme le pensait Marx, par les mensonges de la morale sociale. Presque toujours, mais non pas toujours. Cela aussi est certain. Il y a vingt-cinq siècles, certains philosophes grecs, dont les noms mêmes nous sont inconnus, affirmaient que l'esclavage est absolument contraire à la raison et à la nature. Autant les fluctuations de la morale selon les temps et les pays sont évidentes, autant aussi il est évident que la morale qui procède directement de la mystique est une, identique, inaltérable. On peut le vérifier en considérant l'Égypte, la Grèce, l'Inde, la Chine, le bouddhisme, la tradition musulmane, le christianisme, et le folklore de tous les pays. Cette morale est inaltérable parce qu'elle est un reflet du bien absolu qui est situé hors de ce monde. Il est vrai que toutes les religions, sans exception, ont fait des mélanges impurs de cette morale et de la morale sociale, avec des dosages variables. Elle n'en constitue pas moins la preuve expérimentale ici-bas que le bien pur et transcendant est réel ; en d'autres termes, la preuve expérimentale de l'existence de Dieu.
\subsection[II]{II}
\noindent L'oeuvre vraiment capitale de Marx, c'est l'application de sa méthode à l'étude de la société qui l'entourait. Il a défini avec une précision admirable les rapports de force dans cette société. Il a montré que le salariat est une forme d'oppression, que les travailleurs sont inévitablement asservis dans un système de production où, dépouilles de savoir et d'habileté, ils sont réduits presque au néant devant la prodigieuse combinaison de la science et des forces naturelles qui se trouve comme cristallisée dans la machine. Il a montré que l'État, étant constitue par des catégories d'hommes séparées de la population, bureaucratie, police, cadres de l'armée, forme lui-même une machine qui écrase automatiquement ceux qu'il prétend représenter. Il a aperçu que la vie économique allait devenir elle-même de plus en plus centralisée et bureaucratique, rapprochant ainsi les conducteurs de la production de ceux qui conduisent l'État.\par
Ces prémisses devaient le conduire à prévoir le phénomène moderne de l'État totalitaire et la nature des doctrines qui surgiraient autour de lui. Mais Marx voulait que ce sombre mécanisme apportât la justice. C'est pourquoi il n'a pas voulu prévoir. Il a admis l'absurdité la plus criante, la plus contraire à ses propres principes. Il a supposé que, tout étant réglé par la force, un prolétariat sans force allait néanmoins réussir un coup d'État politique, le faire suivre d'une mesure purement juridique, à savoir la suppression de la propriété individuelle, et se trouver de ce fait le maître dans tous les domaines de la vie sociale.\par
Il avait pourtant décrit lui-même ce prolétariat dépouillé de tout, sinon de ses faibles bras pour les besognes serviles et de sa soif brûlante de justice. Il avait montré comment les forces de la nature, canalisées par les machines, monopolisées par les maîtres des entreprises industrielles, réduisent presque à néant la simple force musculaire ; comment la culture moderne, mettant un abîme entre le travail manuel et le travail intellectuel, relègue l'esprit des ouvriers parmi les objets sans valeur ; comment l'habileté manuelle elle-même avait été enlevée aux hommes et transportée dans les machines Il avait fait voir avec la plus cruelle évidence que cette technique, cette culture, cette organisation du travail et de la vie sociale constituent les chaînes qui tiennent les travailleurs asservis. Et en même temps il a voulu croire que, tout cela demeurant intact, le prolétariat briserait la servitude et assumerait le commandement.\par
Cette croyance est également contraire aux préjugés matérialistes de Marx et à la partie solide, inaltérable de sa pensée. Il résulte immédiatement de ses analyses les plus profondes que la transformation de la production, de la culture intellectuelle, de l'organisation sociale doit dans l'ensemble précéder les bouleversements politiques et juridiques, comme ce fut le cas pour la révolution de 1789. Mais Marx n'a pas voulu voir Cette Conséquence tellement évidente, parce qu'elle était contraire à ses désirs. Ses disciples ne risquaient pas de la voir non plus, pour la même raison.\par
Quant à l'interprétation marxiste de l'histoire, on n'en peut rien dire, parce qu'il n'y eu a pas. Il n'y a eu aucune tentative d'expliquer l'évolution de la civilisation en fonction du développement des moyens de production. Bien plus, en posant que la lutte des classes est la clef de l'histoire, Marx n'a même pas cherché à établir que c'est là un principe d'explication matérialiste. Ce n'est nullement évident. L'aspiration de l'âme humaine vers la liberté, la convoitise de l'âme humaine à l'égard de la puissance, peuvent aussi bien s'analyser comme des faits d'ordre spirituel.\par
En mettant sur ces faits l'étiquette : lutte de classes, Marx a seulement simplifié d'une manière presque puérile. Il a oublié la guerre, facteur de l'histoire humaine aussi important que la lutte sociale. Aussi les marxistes se sont-ils toujours trouvés dans un désarroi ridicule devant tous les problèmes posés par la guerre. Au reste, cet oubli est caractéristique de tout le XIXe siècle ; en le commettant, Marx a donné une preuve de plus de servilité intellectuelle à l'égard des influences dominantes de son siècle. De même il a voulu oublier que les luttes des opprimés entre eux, des oppresseurs entre eux, sont aussi importantes que les luttes mutuelles des opprimes et des oppresseurs, et que d'ailleurs le plus souvent le même être humain est l'un et l'autre à la fois. Il a mis la notion d'oppression au centre de son oeuvre, mais n'a jamais cherché à l'analyser. Il ne s'est jamais demandé ce que c'est.\par
Ce qui a fait la prodigieuse fortune politique du marxisme, c'est avant tout cette juxtaposition de deux doctrines pauvres, sommaires et incompatibles entre elles. L'humanité a toujours fait reposer sur Dieu l'espoir d'assouvir sa soif de justice. Dès lors que Dieu était absent des âmes, il fallait perdre cet espoir ou le faire reposer sur la matière. L'homme ne peut supporter d'être seul à vouloir le bien. Il lui faut un allié tout-puissant. Si cet allié n'est pas esprit, il sera matière. Il s'agit simplement de deux expressions différentes de la même pensée fondamentale. Seulement la seconde expression est défectueuse. C'est une religion mal construite. Mais c'est une religion. Il n'est donc pas étonnant que le marxisme ait toujours eu un caractère religieux. Il a en commun avec les formes de vie religieuse les plus âprement combattues par Marx un grand nombre de choses, et notamment d'avoir été fréquemment utilisé, pour citer la formule de Marx, comme opium du peuple. Mais c'est une religion sans mystique, au vrai sens de ce mot.\par
Non seulement le matérialisme en général, mais l'espèce de matérialisme propre à Marx devait lui assurer une vaste influence. Le XIXe siècle a cru que la production industrielle était la clef du progrès humain. C'était la thèse des économistes, la pensée qui permettait aux industriels de faire mourir d'épuisement des générations d'enfants sans le moindre remords. Marx a simplement pris cette pensée et l'a transportée dans le camp révolutionnaire, préparant ainsi l'apparition d'une espèce très singulière de révolutionnaires bourgeois.\par
Mais il était réservé à notre époque d'utiliser les ouvrages de Marx au maximum. La doctrine idéaliste, utopique qui y est contenue est précieuse pour soulever les masses, leur faire porter un parti au pouvoir, maintenir la jeunesse dans l'état d'enthousiasme permanent nécessaire à tout régime totalitaire. En même temps l'autre doctrine, la doctrine matérialiste qui glace toutes les aspirations humaines sous le froid métallique de la force, fournit à un État totalitaire un grand nombre d'excellentes réponses devant les timides aspirations du peuple. D'une manière générale, la juxtaposition d'un idéalisme et d'un matérialisme également sommaires et grossiers est le caractère spirituel, si l'on ose employer ce mot, de notre époque.\par
Le vice d'une telle pensée n'est pas la combinaison du matérialisme et de l'idéalisme, car ils doivent être combinés. C'est de situer cette combinaison trop bas ; car leur unité réside en un lieu qui se trouve au-dessus du ciel, hors de ce monde.\par
Deux choses sont solides, indestructibles dans Marx. L'une, c'est la méthode qui fait de la société un objet d'étude scientifique en cherchant a'y définir des rapports de force ; l'autre, c'est l'analyse de la société capitaliste telle qu'elle existait au XIXe siècle. Le reste non seulement n'est pas vrai, mais est même trop inconsistant, trop vide, pour pouvoir être dit erroné.\par
En oubliant les facteurs spirituels, Marx ne risquait pas de se tromper beaucoup dans l'analyse d'une société qui ne leur laissait en somme aucune place. Au fond, le matérialisme de Marx exprimait seulement l’influence de cette société sur lui ; il a eu la faiblesse de devenir lui-même le meilleur exemple de sa thèse concernant la subordination de la pensée aux circonstances économiques. Mais à ses meilleurs moments il s'élevait au-dessus de cette faiblesse. Le matérialisme lui faisait alors horreur, et il le stigmatisait dans la société de son époque. Il a trouvé une formule impossible à surpasser quand il a dit que le capitalisme a pour essence la subordination du sujet à l'objet, de l'homme à la chose. L'analyse qu'il en a faite de ce point de vue est d'une vigueur, d'une profondeur incomparables ; aujourd'hui encore, aujourd'hui surtout, elle est infiniment précieuse à méditer.\par
Mais la méthode générale est bien plus précieuse encore. L'idée d'élaborer une mécanique des rapports sociaux a été pressentie par beaucoup d'esprits lucides. Ce fut sans doute la pensée de Machiavel. Comme dans la mécanique proprement dite, la notion fondamentale serait celle de force. La grande difficulté est de saisir cette notion.\par
Il n'y a rien dans une telle pensée qui soit incompatible avec la spiritualité la plus pure. Elle en est le complément. Platon comparait la société à un gigantesque animal que les hommes sont contraints de servir et qu'ils ont la faiblesse d'adorer. Le christianisme, si proche de Platon en tant de points, contient non seulement la même pensée, mais la même image ; la bête de l'Apocalypse est sœur du gros animal de Platon. Élaborer une mécanique sociale, c'est, au lieu d'adorer la bête, en étudier l'anatomie, la physiologie, les réflexes, et surtout chercher a comprendre le mécanisme de ses réflexes conditionnels, c'est-à-dire chercher une méthode pour la dresser.\par
La pensée fondamentale de Platon, qui est aussi celle du christianisme, mais qui a été bien oubliée, c'est que l'homme ne peut pas éviter d'être tout entier asservi à la bête, même jusqu'au centre le plus secret de son âme, excepté dans la mesure ou il est libéré par l'opération surnaturelle de la grâce. L'asservissement spirituel consiste dans la confusion du nécessaire et du bien ; car « on ignore quelle distance sépare l'essence du nécessaire et celle du bien ».\par
La bête a une doctrine, la doctrine de la force. Quelques Athéniens, cités par Thucydide, l'ont exprimée crûment, avec une netteté merveilleuse, quand ils ont dit a des malheureux qui les suppliaient : « Nous croyons au sujet des dieux d'après la tradition, et nous savons au sujet des hommes par une expérience certaine, que toujours chacun, par une nécessité de la nature, commande partout où il en a le pouvoir. » On voit bien que ces Athéniens étaient pour la bête des adorateurs de fraîche date, fils d'ancêtres étrangers à ce culte ; les vrais fidèles de ce culte n'en expriment guère la doctrine, si ce n'est par l'action. Pour justifier cette action, ils inventent des idolâtries.\par
L'opposé de cette doctrine, en ce qui concerne la divinité, c'est le dogme de l'Incarnation. « Étant égal à Dieu, il n'a pas regardé cette égalité comme un butin... Il s'est vidé... Il a pris la condition d'esclave... Il est devenu obéissant jusqu'à la mort. »\par
La bête est maîtresse ici-bas. Le diable a dit au Christ : « Je te donnerai cette puissance et la gloire qui y est attachée, car elles m'ont été abandonnées. » La description des sociétés humaines en fonction des seuls rapports de force rend compte de presque tout. Elle ne laisse de côté que le surnaturel.\par
La part du surnaturel ici-bas est secrète, silencieuse, presque invisible, infiniment petite. Mais elle est décisive. Proserpine ne croyait pas changer sa destinée en mangeant un seul grain de grenade ; et dès cet instant, pour toujours, l'autre monde a été sa patrie et son royaume.\par
Cette opération décisive de l'infiniment petit est un paradoxe que l'intelligence humaine a du mal à reconnaître. Par ce paradoxe s'accomplit la sage persuasion dont parle Platon, cette persuasion au moyen de laquelle la providence divine amène la nécessité à orienter la plupart des choses vers le bien.\par
La nature, qui est un miroir des vérités divines, présente partout une image de ce paradoxe. Ainsi les catalyseurs, les bactéries. Par rapport à un corps solide, un point est un infiniment petit. Pourtant, dans chaque corps, il est un point qui l'emporte sur la masse entière, car s'il est soutenu le corps ne tombe pas ; ce point est le centre de gravité.\par
Mais un point soutenu n'empêche une masse de tomber que si elle est disposée symétriquement autour de lui, ou si l'asymétrie comporte certaines proportions. Le levain ne fait lever la pâte que s'il lui est mélangé. Le catalyseur n'agit qu'au contact des éléments de la réaction. De même il existe des conditions matérielles pour l'opération surnaturelle du divin présent ici-bas sous forme d'infiniment petit.\par
La misère de notre condition soumet la nature humaine a une pesanteur morale qui la tire continuellement vers le bas, vers le mal, vers une soumission totale à la force. « Dieu vit que les pensées du coeur de l'homme tendaient toujours, constamment au mal. » Cette pesanteur est ce qui contraint l'homme, d'une part à perdre la moitié de son âme, selon un proverbe antique, le jour où il devient esclave, et d'autre part à toujours commander, selon le mot cité par Thucydide, partout où il en a le pouvoir. Comme la pesanteur proprement dite, elle a ses lois. Au moment où on les étudie, on ne saurait être trop froid, trop lucide, trop cynique. En ce sens, dans cette mesure, il faut être matérialiste.\par
Mais un architecte étudie, non pas seulement la chute des corps, mais aussi les conditions d'équilibre. La véritable connaissance de la mécanique sociale implique celle des conditions auxquelles l'opération surnaturelle d'une quantité infiniment petite de bien pur, placée au point convenable, peut neutraliser la pesanteur.\par
Ceux qui nient la réalité du surnaturel ressemblent vraiment à des aveugles. La lumière aussi ne heurte pas, ne pèse rien. Mais par elle les plantes et les arbres montent vers le ciel malgré la pesanteur. On ne la mange pas, mais les graines et les fruits que l'on mange ne mûriraient pas sans elle.\par
De même les vertus purement humaines ne germeraient pas hors de la nature animale de l'homme sans la lumière surnaturelle de la grâce. Quand l'homme se détourne de cette lumière, une décomposition lente, progressive, mais infaillible, le soumet finalement tout entier, jusqu'au fond de l'âme, à l'emprise de la force. Autant qu'il est possible à une créature pensante, il devient matière. De même une plante privée de lumière est changée peu a peu en quelque chose d'inerte.\par
Ceux qui croient que le surnaturel, par définition, opère d'une manière arbitraire et qui échappe à toute étude le méconnaissent comme ceux qui en nient la réalité. Les mystiques authentiques, comme saint Jean de la Croix, décrivent l'opération de la grâce sur l'âme avec une précision de chimiste ou de géologue. L'influence du surnaturel sur les sociétés humaines, quoique peut-être encore plus mystérieuse, peut sans doute aussi être étudiée.\par
Si l'on regarde de près non seulement le moyen âge chrétien, mais toutes les civilisations vraiment créatrices, on s'aperçoit que chacune, au moins pendant un temps, a eu au centre même une place vide réservée au surnaturel pur, à la réalité située hors de ce monde. Tout le reste était orienté vers ce vide.\par
Il n'y a pas deux méthodes d'architecture sociale. Il n'y en a jamais eu qu'une. Elle est éternelle. Mais c'est toujours l'éternel qui exige de l'esprit humain un véritable effort d'invention. Elle consiste à disposer les forces aveugles de la mécanique sociale autour du point qui sert aussi de centre aux forces aveugles de la mécanique céleste ; c'est-à-dire « l'Amour qui meut le soleil et les autres étoiles ».\par
Ce n'est certainement pas facile, ni à concevoir d'une manière plus précise. ni à accomplir. Mais en tout cas, pour se diriger dans ce sens, la première condition est d'y penser. Il ne s'agit pas d'une de ces choses qu'on peut obtenir par accident. Peut-être peut-on la recevoir au bout d'un long et persévérant désir.\par
L'imitation de l'ordre du monde fut la grande pensée de l'antiquité pré-romaine. Ce devait être aussi la grande pensée du christianisme, puisque le modèle parfait proposé a l'imitation de chaque homme était le même être que la Sagesse ordonnatrice de l'univers. Effectivement cette pensée a remué souterrainement tout le moyen âge.\par
Aujourd'hui, hébétés que nous sommes depuis plusieurs siècles par l'orgueil de la technique, nous avons oublié qu'il existe un ordre divin de l'univers. Nous ignorons que le travail, l'art, la science, sont seulement différentes manières d'entrer en contact avec lui.\par
Si l'humiliation du malheur nous réveillait, si nous retrouvions cette grande vérité, nous pourrions effacer ce qui est le scandale de la pensée moderne, l'hostilité entre la religion et la science.\par

\section[Y a-t-il une doctrine marxiste ?]{Y a-t-il une doctrine marxiste ?}\renewcommand{\leftmark}{Y a-t-il une doctrine marxiste ?}

\noindent \par
Beaucoup de gens se déclarent ou adversaires, ou partisans, ou partisans mitigés de la doctrine marxiste. On ne pense guère à se demander : Marx avait-il une doctrine ? On n'imagine pas qu'une chose qui a excité tant de controverses puisse ne pas exister. Pourtant le cas est fréquent. La question vaut la peine d'être posée et examinée. Après un examen attentif, il y a peut-être lieu de répondre négativement.\par
On est généralement d'accord pour dire que Marx est matérialiste. Il ne l'a pas toujours été. Dans sa jeunesse, il était parti pour élaborer une philosophie du travail dans un esprit très proche au fond de celui de Proudhon. Une philosophie du travail n'est pas matérialiste. Elle dispose tous les problèmes relatifs à l'homme autour d'un acte qui, constituant une prise directe et réelle sur la matière, enferme la relation de l'homme -avec le terme antagoniste. Le terme antagoniste, c'est la matière. L'homme n'y est pas ramené, il y est opposé.\par
Dans cette voie, le jeune Marx n'a même pas commencé l'ébauche d'une ébauche. Il n'a guère fourni que quelques indications. Proudhon, de son côté, a seulement jeté quelques éclairs parmi beaucoup de fumée. Une telle philosophie reste à faire. Elle est peut-être indispensable. Elle est peut-être plus particulièrement un besoin de cette époque-ci. Plusieurs signes montrent qu'au siècle dernier il s'en préparait un embryon. Mais il n'en est rien sorti. Peut-être est-ce une création réservée à notre siècle.\par
Marx a été arrêté jeune encore par un accident très fréquent au XIX e siècle ; il s'est pris au sérieux. Il a été saisi d'une sorte d'illusion messianique qui lui a fait croire qu'un rôle décisif lui était réservé pour le salut du genre humain. Dès lors il ne pouvait pas conserver la capacité de penser au sens complet du mot. La philosophie du travail qui germait en lui, il l'a abandonnée, quoiqu'il ait continué, mais de plus en plus rarement avec le temps, à mettre çà et là dans ses écrits des formules qui sen inspiraient. Étant hors d'état d'élaborer une doctrine, il a pris les deux croyances les plus courantes à son époque, l'une et l'autre pauvres, sommaires, médiocres, et de plus impossibles à penser ensemble. L'une est le scientisme, l'autre le socialisme utopique.\par
Pour les adopter ensemble, il leur a donné une unité fictive au moyen de formules qui, si on leur demande leur signification, n'en révèlent en fin de compte aucune, sinon un état sentimental. Mais quand un auteur choisit les mots habilement, le lecteur a rarement l'impolitesse de poser une telle question. Moins une formule a de signification, plus épais est le voile qui couvre les contradictions illégitimes d'une pensée.\par
Ce n'est pas, bien entendu, que Marx ait jamais eu l'intention de tromper le public. Le public qu'il avait besoin de tromper pour pouvoir vivre, c'était lui-même. C'est pourquoi il a entouré le fond de sa conception de nuages métaphysiques qui, lorsqu'on les regarde fixement pendant un certain temps, deviennent transparents, mais se révèlent vides.\par
Mais ces deux systèmes qu'il a pris tout faits, il ne leur a pas seulement fabrique une liaison fictive, il les a aussi repensés. Son esprit, d'une portée inférieure à ce qu'exige la mise au jour d'une doctrine, était capable d'idées de génie. Il y a dans son oeuvre des fragments compacts, inaltérables de vérité, qui ont naturellement leur place dans toute doctrine vraie. C'est ainsi qu'ils ne sont pas seulement compatibles avec le christianisme, mais infiniment précieux pour lui. Ils doivent être repris à Marx. C'est d'autant plus facile que ce qu'on nomme aujourd'hui le marxisme, c'est-à-dire le courant de pensée qui se réclame de Marx, n'en fait aucun usage. La vérité est trop dangereuse à toucher. C'est un explosif.\par
Le scientisme du XIXe siècle était la croyance que la science de l'époque, au moyen d'un simple développement dans des directions déjà définies par les résultats obtenus, fournirait une réponse certaine à tous les problèmes susceptibles de se poser aux hommes, sans exception. Ce qui s'est passé en fait, c'est qu'après avoir pris un peu d'expansion la science elle-même a craqué. Celle qui est en faveur aujourd'hui, bien qu'elle dérive de celle-là, est une autre science. Celle du XIXe siècle a été déposée respectueusement au musée avec l'étiquette : science classique.\par
Elle était bien construite, simple et homogène. La mécanique y était reine. La physique en était le centre. Comme c'était la branche qui avait obtenu de loin les résultats les plus brillants, elle influençait naturellement beaucoup toutes les autres études. L'idée d'étudier l'homme comme le physicien étudie la matière inerte devait dès lors s'imposer, et était effectivement très répandue. Mais on ne pensait guère à l'homme que comme individu. La matière était dès lors la chair ; ou bien on s'efforçait de définir un équivalent psychologique de l'atome. Ceux qui réagissaient contre cette obsession de l'individu étaient aussi en réaction contre le scientisme.\par
Marx le premier, et sauf erreur le seul - car on n'a pas continue ses recherches - a eu la double pensée de prendre la société comme fait humain fondamental et d'y étudier, comme le physicien dans la matière, les rapports de force.\par
C'est là une idée de génie, au sens complet du mot. Ce n'est pas une doctrine, C'est un instrument d'étude, de recherche, d’exploration et peut-être de construction pour toute doctrine qui ne risque pas de tomber en poussière au contact dune vérité.\par
Marx, ayant eu cette idée, s'est empressé de la rendre stérile, autant qu'il dépendait de lui, en plaquant dessus le misérable scientisme de son époque. Ou plutôt Engels, qui lui était, très inférieur et le savait, a fait pour lui cette opération ; mais Marx l'a couverte de son autorité. Il en est résulté un système d'après lequel les rapports de force qui définissent la structure sociale déterminent entièrement et le destin et les pensées des hommes. Un tel système est impitoyable. La force y est tout ; il ne laisse aucune espérance pour la justice. Il ne laisse même pas l'espérance de la concevoir dans sa vérité, puisque les pensées ne font que refléter les rapports de force.\par
Mais Marx était un cœur généreux. Le spectacle de l'injustice le faisait réellement, on peut dire charnellement souffrir. Cette souffrance était assez intense pour l'empêcher de vivre s'il n'avait eu l'espoir d'un règne prochain et terrestre de la justice intégrale. Pour lui comme pour beaucoup, le besoin était la première des évidences.\par
La plupart des êtres humains ne mettent pas en doute la vérité d'une pensée sans laquelle littéralement ils ne pourraient pas vivre. Arnolphe ne mettait pas en doute la fidélité d'Agnès. Le choix suprême pour toute âme est peut-être ce choix entre la vérité et la vie. Qui veut préserver sa vie la perdra. Cette sentence serait légère si elle touchait seulement ceux qui en aucune circonstance n'acceptent de mourir. Ils sont en somme assez rares. Elle devient terrible quand elle est appliquée à ceux qui refusent de perdre, fussent elles fausses, les pensées sans lesquelles ils se sentent hors d'état de vivre.\par
La conception courante de la justice au temps de Marx était celle du socialisme qu'il a lui-même nommé utopique. Elle était très pauvre en effort de pensée, mais comme sentiment elle était généreuse et humaine, voulant la liberté, la dignité, le bien-être, le bonheur et tous les biens possibles pour tous. Marx l'a adoptée. Il a seulement tenté de la rendre plus précise, et y a ajouté ainsi des idées intéressantes, mais rien qui soit vraiment de premier ordre.\par
Ce qu'il a changé, c'est le caractère de l'espérance. Une probabilité fondée sur le progrès humain ne pouvait lui suffire. À son angoisse il fallait une certitude. On ne fonde pas une certitude sur l'homme. Si le XVIIIe siècle a eu par moments cette illusion - et il ne l'a eue que par moments - les convulsions de la Révolution et de la guerre avaient été assez atroces pour y remédier.\par
Dans les siècles antérieurs, les gens qui avaient besoin d'une certitude l'appuyaient sur Dieu. La philosophie du XVIIIe siècle et les merveilles de la technique avaient semblé porter l'homme tellement haut que l'habitude s'en était perdue. Mais ensuite, l'insuffisance radicale de tout ce qui est humain étant redevenue sensible, on eut besoin de chercher un support. Dieu était démodé. On prit la matière. L'homme ne peut pas supporter plus d'un moment d'être seul à vouloir le bien. Il lui faut un allié tout-puissant. Si l'on ne croit pas à la toute-puissance lointaine, silencieuse, secrète d’un esprit, il ne reste que la toute-puissance évidente de la matière.\par
C'est là l'absurdité inévitable de tout matérialisme. Si le matérialiste pouvait écarter tout souci du bien, il serait parfaitement cohérent. Mais il ne peut pas. L'être même de l'homme n'est pas autre chose qu'un effort perpétuel vers un bien ignoré. Et le matérialiste est un homme. C'est pourquoi il ne peut pas s'empêcher de finir par regarder la matière comme une machine à fabriquer du bien.\par
La contradiction essentielle dans la vie humaine, c'est que l'homme ayant pour être même l'effort vers le bien est en même temps soumis dans son être tout entier, dans sa pensée comme dans sa chair, à une force aveugle, à une nécessité absolument indifférente au bien. C'est ainsi ; et c'est pourquoi aucune pensée humaine ne peut échapper a la contradiction, Loin que la contradiction soit toujours un critérium d'erreur, elle est quelquefois un signe de vérité. Platon le savait. Mais on peut distinguer les cas. Il y a un usage légitime et un usage illégitime de la contradiction.\par
L'usage illégitime consiste à accoupler des pensées incompatibles comme si elles étaient compatibles. L'usage légitime consiste, d'abord, quand deux pensées incompatibles se présentent à l'esprit, à épuiser toutes les ressources de l'intelligence pour essayer d'éliminer au moins l'une des deux. Si c'est impossible, si elles s'imposent l'une et l'autre, il faut alors reconnaître la contradiction comme un fait. Puis il faut s'en servir comme d'un outil à deux branches, comme d'une pince, pour entrer par elle en contact direct avec le domaine transcendant de la vérité inaccessible aux facultés humaines. Le contact est direct, quoiqu'il se fasse par un intermédiaire, de même que le sens du toucher est directement affecte par les rugosités d'une table sur laquelle on promène, non pas la main, mais un crayon. Ce contact est réel, quoique étant au nombre des choses qui par nature sont impossibles, car il s'agit d'un contact entre l'esprit et ce qui n'est pas pensable. Il est surnaturel, mais réel.\par
Cet usage légitime de la contradiction comme passage au transcendant a un équivalent, pour ainsi dire une image, très fréquent dans la mathématique. Il joue un rôle essentiel dans le dogme chrétien, comme on peut s'en rendre compte au sujet de la Trinité, de l'Incarnation ou de tout autre exemple. Il en est de même dans d'autres traditions. Il y a là peut-être un critérium pour discerner les traditions religieuses et philosophiques authentiques.\par
C'est surtout la contradiction essentielle, la contradiction entre le bien et la nécessité, ou celle équivalente entre la justice et la force, dont l'usage constitue un critérium. Le bien et la nécessité, comme l'a dit Platon, sont séparés par une distance infinie. Ils n'ont rien en commun. Ils sont totalement autres. Quoique nous soyons contraints de leur assigner une unité, cette unité est un mystère ; elle demeure pour nous un secret. La contemplation de cette unité inconnue est la vie religieuse authentique.\par
Fabriquer une équivalent fictif, erroné de cette unité, qui serait saisissable pour les facultés humaines, c'est le fond des formes inférieures de la vie religieuse. À toute forme authentique de la vie religieuse correspond une forme inférieure, qui s'appuie en apparence sur la même doctrine, mais ne la comprend pas. Mais la réciproque n'est pas vraie. Il y a des manières de penser qui ne sont compatibles qu'avec une vie religieuse de qualité inférieure.\par
À cet égard le matérialisme tout entier, en tant qu'il attribue à la matière la fabrication automatique du bien, est à classer parmi les formes inférieures de la vie religieuse. Cela se vérifie même pour les économistes bourgeois du me siècle, les apôtres du libéralisme, qui ont un accent véritablement religieux quand ils parlent de la production. Cela se vérifie bien plus encore pour le marxisme. Le marxisme est tout à fait une religion, au sens le plus impur de ce mot. Il a notamment en commun avec toutes les formes inférieures de la vie religieuse le fait d'avoir été continuellement utilisé, selon la parole si juste de Marx, comme un opium du peuple.\par
Au reste une spiritualité comme celle de Platon n'est séparée du matérialisme que par une nuance, un infiniment petit. Il dit, non pas que le bien est un produit automatique de la nécessité, mais que l'Esprit domine la nécessité par la persuasion ; il lui persuade de faire tourner vers le bien la plupart des choses qui se produisent ; et la nécessité est vaincue par cette sage persuasion. De même Eschyle disait : « Dieu ne s'arme d'aucune violence. Tout ce qui est divin est sans effort. Demeurant en haut, sa sagesse parvient de la néanmoins à opérer, de son siège pur. » La même pensée se trouve en Chine, en Inde, dans le christianisme. Elle est exprimée par la première ligne du Pater, qu'il vaudrait mieux traduire : « Notre Père, celui des cieux » ; et plus encore par la merveilleuse parole : « Votre Père qui est dans le secret. »\par
La part du surnaturel ici-bas, c'est le secret, le silence, l'infiniment petit. Mais l'opération de cet infiniment petit est décisive. Proserpine croyait ne s'engager à rien quand, moitié contrainte, moitié séduite, elle a consenti à manger un seul grain de grenade ; mais dès cet instant, pour toujours, l'autre monde a été son royaume et sa patrie. Une perle dans un champ n'est guère visible. Le grain de sénevé est la plus petite des graines...\par
L'opération décisive de l'infiniment petit est un paradoxe ; l'intelligence humaine a du mal à la reconnaître ; mais la nature, qui est un miroir des vérités divines, en présente partout des images. Ainsi les catalyseurs, les bactéries, les ferments. Par rapport à un corps solide, un point est un infiniment petit. Pourtant, dans chaque corps, il est un point qui l'emporte sur la masse entière, de sorte que si ce point est soutenu, le corps ne tombe pas. La clef de voûte porte d'en haut tout un édifice. Archimède disait : « Donne-moi un point d'appui et je soulèverai le monde. » La présence muette du surnaturel ici-bas est ce point d'appui. C'est pourquoi, dans les premiers siècles, on comparait la Croix à une balance.\par
Si une île tout à fait séparée n'avait jamais été peuplée que d'aveugles, la lumière serait pour eux ce qu'est pour nous le surnaturel. On est tenté de croire d'abord que pour eux elle ne serait rien, qu'en faisant à leur usage une physique d'où toute théorie de la lumière soit absente on leur donnerait une explication complète de leur monde. Car la lumière ne heurte pas, ne pousse pas, ne pèse pas, n'est pas mangée. Pour eux, elle est absente. Mais on ne peut pas la laisser hors du compte. Par elle seule les arbres et les plantes montent vers le ciel malgré la pesanteur. Par elle seule mûrissent les graines, les fruits et tout ce qu'on mange.\par
En assignant au bien et à la nécessité une unité transcendante, on donne au problème humain essentiel une solution incompréhensible, surtout lorsqu'on y ajoute, comme il est indispensable, la croyance plus incompréhensible encore qu'il se communique quelque chose de cette unité transcendante à ceux qui, sans la comprendre, sans pouvoir faire à son égard aucun usage ni de leur intelligence ni de leur volonté, la contemplent avec amour et désir.\par
Ce qui échappe aux facultés humaines ne peut être, par définition, ni vérifié ni réfuté. Mais il en procède des conséquences qui sont situées au niveau d'au-dessous, dans le domaine accessible à nos facultés ; ces conséquences peuvent être soumises à une vérification. En fait cette épreuve réussit. Une seconde vérification indirecte est constituée par le consentement universel. En apparence l'extrême variété des religions et des philosophies indiquerait que cette preuve n'existe pas ; cette considération a même conduit beaucoup d'esprits au scepticisme. Mais un examen plus attentif montre que, excepté dans les pays qui ont subordonné leur vie spirituelle à l'impérialisme, toute religion porte en son centre secret une doctrine mystique ; et quoique les doctrines mystiques diffèrent {\itshape entre elles}, elles sont non pas simplement semblables, mais absolument identiques en un certain nombre de points essentiels. Une troisième vérification indirecte, c'est l'expérience intérieure. C'est une preuve indirecte, même, pour ceux qui font l'expérience, en ce sens que c'est une expérience qui échappe a leurs facultés ; ils n'en saisissent que l'apparence extérieure et le savent. Pourtant ils en savent aussi la signification. Il y a, tout au long des siècles passés, un très petit nombre d'être humains, évidemment incapables, non seulement de mensonge, mais aussi d'autosuggestion, dont le témoignage en cette matière est décisif.\par
Ces trois preuves sont peut-être les seules possibles ; mais elles suffisent. On peut y ajouter l'équivalent d'une preuve par l'absurde en examinant les autres solutions, celles qui fabriquent pour le bien et la nécessite une unité fictive au niveau des facultés humaines. Elles ont des conséquences absurdes, et dont l'absurdité est vérifiable à la fois par le raisonnement et par l'expérience.\par
Parmi toutes ces solutions insuffisantes, les meilleures de loin, les plus utilisables, les seules peut-être qui contiennent des fragments de vérité pure sont les solutions matérialistes. Le matérialisme rend compte de tout, à l'exception du surnaturel. Ce n'est pas une petite lacune, car dans le surnaturel tout est contenu et infiniment dépasse. Mais si l'on ne tient pas compte du surnaturel, on a raison d'être matérialiste. Cet univers, avec le surnaturel en moins, n'est que matière. En le décrivant seulement comme matière, on saisit une parcelle de vérité. En le décrivant comme une combinaison de matière et de forces spécifiquement morales qui appartiendraient à ce monde, qui seraient au niveau de la nature, on fausse tout. C'est pourquoi, pour un chrétien, les écrits de Marx sont bien plus précieux que ceux, par exemple, de Voltaire et des Encyclopédistes, qui trouvaient moyen d'être athées sans être matérialistes. Ils étaient athées, non pas simplement en ce sens qu'ils excluaient plus ou moins nettement la notion d'un Dieu personnel, ce qui est le cas pour certaines sectes bouddhistes qui malgré cela se sont élevées jusqu'à la vie mystique, mais en ce sens qu'ils excluaient tout ce qui n'est pas de ce monde. Ils croyaient, les naïfs, que la justice est de ce monde. C'est là l'illusion extrêmement dangereuse enfermée dans ce qu'on nomme les principes de 1789, la foi laïque, et ainsi de suite.\par
Parmi toutes les formes de matérialisme, l'œuvre de Marx contient une indication extrêmement précieuse, quoiqu'il n'en ait guère fait un usage réel, et ses adhérents encore bien moins. C'est la notion de matière non physique. Marx, regardant avec raison la société comme étant en ce monde le fait humain primordial, n’a fait attention qu'à la matière sociale ; mais on peut considérer de même, en second lieu, la matière psychologique ; il y a plusieurs courants en ce sens dans la psychologie moderne, quoique, sauf erreur, la notion n’en ait pas été formulée. Un certain nombre de préjugés courants empêche qu'elle le soit.\par
L'idée est celle-ci ; elle est indispensable à toute doctrine solide ; elle est centrale. Il y a sous tous les phénomènes d'ordre moral, soit collectifs, soit individuels, quelque chose d’analogue à la matière proprement dite. Quelque chose d'analogue ; non pas la matière elle-même. C'est pourquoi les systèmes que Marx classait dans ce qu'il nommait le matérialisme mécanique, avec une nuance de mépris justifie, systèmes, qui cherchent à expliquer toute la pensée humaine par un mécanisme physiologique, ne sont que niaiserie. Les pensées sont soumises a un mécanisme qui leur est propre. Mais c'est un mécanisme. Quand nous pensons la matière, nous pensons un système mécanique de forces soumises à une aveugle et rigoureuse nécessité. Il en est de même pour cette matière non tangible qui est la substance de nos pensées. Seulement il est très difficile d'y saisir la notion de force et de concevoir les lois de cette nécessité.\par
Mais même avant d'y être parvenu, il est déjà extrêmement utile de savoir que cette nécessité spécifique existe. Cela permet d'éviter deux erreurs dans lesquelles on tombe sans cesse, car dès qu'on sort de l’une on tombe dans l'autre. L'une est de croire que les phénomènes moraux sont calqués sur les phénomènes matériels ; par exemple, que le bien-être moral résulte automatiquement et exclusivement du bien-être physique. L'autre est de croire que les phénomènes moraux sont arbitraires et qu'ils peuvent être provoqués par l'autosuggestion ou la suggestion extérieure, ou encore par un acte de volonté.\par
Ils ne sont pas soumis à la nécessité physique, mais ils sont soumis à la nécessité. Ils subissent la répercussion des phénomènes physiques, mais une répercussion spécifique, conforme aux lois propres de la nécessité a laquelle ils sont soumis. Tout ce qui est réel est soumis à la nécessité. Il n'y a rien de plus réel que l'imagination ; ce qui est imaginé n'est pas réel, mais l'état où se trouve l'imagination est un fait. Un certain état de l'imagination étant donné, il ne peut être modifié que si les causes susceptibles de produire un tel effet sont mises en jeu. Ces causes n'ont aucun rapport direct avec les choses imaginées ; mais d'un autre côté elles ne sont pas n'importe quoi. La relation de cause à effet est aussi rigoureusement déterminée dans ce domaine que dans celui de la pesanteur. Elle est seulement plus difficile à connaître.\par
Les erreurs sur ce point sont innombrables et sont causes de souffrances innombrables dans la vie quotidienne. Par exemple, si un enfant dit qu'il se sent malade, ne va pas à l'école, et trouve soudain des forces pour jouer avec de petits camarades, la famille indignée pense qu'il a menti. On lui dit : « Puisque tu avais la force de jouer, tu avais aussi celle de travailler. » Or l'enfant peut très bien avoir été sincère. Il a été retenu par un sentiment d'épuisement réel que la vue des petits camarades et l'attrait du jeu ont réellement fait disparaître, au lieu que l'étude ne contenait pas un stimulant suffisant pour produire cet effet. De même, il est naïf de notre part de nous étonner quand nous prenons fermement une résolution et ne la tenons pas. Quelque chose nous stimulait à prendre la résolution, mais ce quelque chose n'était pas assez fort pour nous pousser a l'exécution ; bien plus, l'acte même de prendre une résolution a pu épuiser le stimulant et empêcher ainsi même un commencement d'exécution. C'est ce qui se produit souvent quand il s'agit d'actions extrêmement difficiles. Le cas bien connu de saint Pierre en est gans doute un exemple.\par
Cette espèce d'ignorance intervient constamment, pour les vicier, dans les rapports entre les gouvernements et les peuples, entre les classes dominantes et les masses. Par exemple, les patrons ne conçoivent que deux manières de rendre leurs ouvriers heureux ; ou bien élever leur salaire, ou bien leur dire qu'ils sont heureux et chasser les méchants communistes qui leur assurent le contraire. Ils ne peuvent pas comprendre que d'une part, le bonheur d'un ouvrier consiste avant tout dans une certaine disposition d'esprit à l'égard de son travail ; et que d'autre part cette disposition d'esprit n'apparaît que si sont réalisées certaines conditions objectives, impossibles à connaître sans une étude sérieuse. Cette double vérité, convenablement transposée, est la clef de tous les problèmes pratiques de la vie humaine.\par
Dans le jeu de cette nécessité qui régit les pensées et les actes des hommes, les rapports de la société et de l'individu sont très complexes. Mais la primauté du social saute aux yeux. Marx a eu raison de commencer par poser la réalité d'une matière sociale, d'une nécessité sociale, dont il faut au moins entrevoir les lois avant d'oser penser aux destinées du genre humain.\par
Cette idée était originale par rapport à son temps ; mais absolument parlant elle ne l'est pas. D'ailleurs il est probable qu'aucune vérité n'est vraiment originale. Élaborer une mécanique des rapports sociaux a été très probablement la véritable intention de Machiavel, qui était un grand esprit. Mais bien plus anciennement Platon a eu constamment présente à la pensée la réalité de la nécessité sociale.\par
Platon sentait surtout très vivement que la matière sociale est un obstacle infiniment plus difficile à franchir que la chair proprement dite entre l'âme et le bien. C'est aussi la pensée chrétienne. Saint Paul dit qu'il n'y a pas a lutter contre la chair, mais contre le diable ; et le diable est chez lui dans la matière sociale, puisqu'il a pu dire au Christ, en lui montrant les royaumes de ce monde : « Je te donnerai toute cette puissance et la gloire qui lui est attachée, car elles m'ont été abandonnées. »Aussi est-il nommé le Prince de ce monde. Puisqu'il est le père du mensonge, c'est donc que la matière sociale est le milieu de culture et de prolifération par excellence pour le mensonge et l'erreur. Telle est bien la pensée de Platon. Il comparait la société a un gigantesque animal que les hommes sont contraints de servir et dont ils étudient les réflexes pour en tirer leurs convictions concernant le bien et le mal. Le christianisme a gardé cette image. La bête de l'Apocalypse est sœur de celle de Platon.\par
La pensée centrale, essentielle de Platon, qui est elle aussi une pensée chrétienne, c'est que tous les hommes sont absolument incapables d'avoir sur le bien et le mal d'autres opinions que celles dictées par les réflexes de l'animal, excepte les âmes prédestinées qu'une grâce surnaturelle tire vers Dieu.\par
Il n'a pas beaucoup développé cette pensée, quoi qu'elle soit présente derrière tout ce qu'il écrit, sans doute parce qu'il savait que l'animal est méchant et se venge. C'est un thème de réflexion presque inexploré. Il s'en faut de beaucoup qu'il y ait là une vérité évidente ; c'est une vérité très profondément cachée. Elle est cachée notamment par les conflits d'opinion. Si deux hommes sont en désaccord violent sur le bien et le mal, on peut difficilement croire que tous deux sont aveuglement soumis a l'opinion de la société qui les entoure. En particulier, celui qui réfléchit sur ces quelques lignes de Platon est très vivement tente d'expliquer par l'influence de l'animal les opinions des gens avec qui il discute, tout en expliquant les siennes propres par une vue exacte de la justice et du bien. Or on n'a compris la vérité formulée par Platon que lorsqu'on l'a reconnue vraie pour soi-même.\par
En réalité, à une époque donnée, dans un ensemble social donné, les divergences d'opinion sont beaucoup moindres qu'il ne paraît. Il y a beaucoup moins de divergences que de conflits. Les luttes les plus violentes opposent souvent des gens qui pensent exactement ou presque exactement la même chose. Notre époque est très féconde en paradoxes de ce genre. Le fonds commun aux différents courants d'opinion à une époque donnée est l'opinion du gros animal à cette époque. Par exemple, depuis dix ans, chaque tendance politique, y compris les plus petits groupuscules, accusait toutes les autres, sans exception, de fascisme, et subissait en retour la même accusation ; excepte, bien entendu, ceux qui regardaient cette épithète comme un éloge. Probablement l'épithète était toujours partiellement justifiée. Le gros animal européen du Xe siècle a un goût prononcé pour le fascisme. Un autre exemple amusant est le problème des populations de couleur. Chaque pays est très sentimental au sujet du malheur de celles qui dépendent d'autres pays, mais s'indigne si l'on met en doute le bonheur parfait dont jouissent les siennes. Il y a beaucoup de cas analogues, où la divergence apparente des attitudes est en réalité une identité.\par
D'autre part, l'animal étant gigantesque et les hommes tout petits, chacun est différemment situé par rapport a lui. En suivant l'image de Platon, on peut imaginer que parmi les gens chargés de l'étriller, un s'occupe d'un genou, un autre d'un ongle, un autre du cou, un autre du dos. Il peut aimer qu'on le chatouille sous le menton et qu'on lui tapote le dos. Un de ses serviteurs soutiendra en conséquence que c'est le chatouillement qui est le plus grand des biens ; un autre, que c'est le tapotement. Autrement dit, la société est faite de groupes qui s'entrecroisent de toutes sortes de manières, et la morale sociale varie de groupe en groupe. On ne pourrait pas trouver deux individus dont les milieux sociaux soient vraiment identiques ; le milieu de chacun est fait d'un enchevêtrement de groupes qui nulle part ailleurs ne se retrouve tel quel. Ainsi l'originalité apparente des individus ne contredit pas la thèse d'une subordination totale de la pensée à l'opinion sociale.\par
Cette thèse est celle même de Marx. La seule différence entre lui et Platon à ce sujet, c'est qu'il ignore la possibilité d'exceptions opérées par l'intervention surnaturelle de la grâce. Cette lacune laisse tout à fait intacte la vérité d'une partie de ses recherches, mais est cause que le reste est seulement du verbiage.\par
Marx a cherché a concevoir le mécanisme de l'opinion sociale. Le phénomène de la morale professionnelle lui en a fourni la clef. Chaque groupe professionnel se fabrique une morale en vertu de laquelle l'exercice de la profession, dès lors qu'il est soustrait aux règles, est hors de toute atteinte du mal. C'est là un besoin presque vital, car la tension du travail, quel qu'il soit, est par elle-même si grande qu'elle serait intolérable s'il s'y mêlait le souci harcelant du bien et du mal. Pour s'en protéger, il faut une armure. La morale à l'usage de la profession joue ce rôle.\par
Par exemple, un médecin à qui l'on donne a soigner un condamné à mort ne se posera généralement pas la question extrêmement angoissante de savoir s'il est bon de le guérir. Il est admis qu'un médecin essaie de guérir. Même pour les esclaves de Rome, il y avait une morale à leur usage, selon laquelle un esclave ne peut jamais mal faire s'il obéit à son maître ou agit dans ses intérêts. Bien entendu, cette morale était propagée par les maîtres ; mais elle était dans une large mesure adoptée par les esclaves, et c'est pourquoi les révoltes d'esclaves ont été rares, eu égard à leur nombre et à leur horrible malheur. Au temps où la guerre était une profession, les hommes de guerre avaient une morale selon laquelle tout acte de guerre, conforme aux coutumes de la guerre, et utile à la victoire, est légitime et bon ; y compris, par exemple, les viols de femmes ou les meurtres d'enfants au cours des sacs de villes, car la licence accordée aux soldats en ces occasions était indispensable au moral de l'armée. Au commerce correspond une morale où le vol est le crime par excellence, et où tout échange avantageux d'un objet contre de l'argent est légitime et bon. Le caractère commun à toutes ces morales, et à toute espèce de morale sociale, a été exprimé par Platon en une formule définitive : « Ils nomment justes et belles les choses nécessaires, car ils ignorent combien est grande en réalité la distance qui sépare l'essence du nécessaire et celle du bien. »\par
La conception de Marx, c'est que l'atmosphère morale d'une société donnée, atmosphère qui pénètre partout et se combine avec la morale particulière de chaque milieu, est elle-même composée par un mélange des morales de groupe, avec un dosage qui reflète exactement la quantité de puissance exercée par chaque groupe. Ainsi selon qu'une société est dominée par les propriétaires de vastes entreprises agricoles, ou par des militaires, ou par des commerçants, ou par des industriels, ou par des banquiers, ou par des bureaucrates, elle sera imprégnée tout entière par la conception du monde liée a la morale professionnelle des propriétaires, ou des militaires, et ainsi de suite. Cette conception du monde s'exprimera partout, dans la politique, dans les lois, même dans les spéculations abstraites et en apparence désintéressées des intellectuels. Chacun y sera soumis, mais personne n'en aura conscience, car chacun croira qu'il s'agit, non d'une conception particulière, mais d'une manière de penser inhérente à la nature humaine.\par
Tout cela est en grande partie vrai et facile à vérifier. Pour ne citer qu'un exemple, il est singulier de voir quelle place tient le vol dans le code pénal français. Avec certaines circonstances aggravantes, il est plus sévèrement puni que le viol des enfants. Pourtant les hommes qui ont fait ce code n'avaient pas seulement de l'argent, mais aussi des enfants que sans doute ils aimaient ; s'ils avaient eu a choisir entre perdre une partie de leur fortune et voir souiller leurs enfants, rien n'autorise à supposer qu'ils auraient préféré l'argent. Mais en rédigeant le code ils n'étaient à leur propre insu que les organes des réflexes sociaux ; et dans une société fondée sur le commerce, le vol est l'acte antisocial par excellence. Au lieu que la traite des femmes, par exemple, est une espèce de commerce ; aussi s'est-on difficilement et mollement décidé à la punir.\par
Tant de faits cependant semblent contredire la théorie qu'elle serait réfutée aussitôt qu'examinée, s'il ne fallait la nuancer par la considération du temps. L'homme est conservateur, et le passe à tendance à demeurer par son propre poids. Par exemple, une grande partie du code vient d'un temps où le commerce était bien plus important qu'aujourd'hui ; ainsi, d'une manière générale, l'atmosphère morale d'une société contient des éléments qui proviennent de classes autrefois dominantes, depuis lors disparues ou plus ou moins déchues. Mais l'inverse aussi est vrai, Comme un chef de l'opposition, destiné à devenir premier ministre, a déjà une clientèle, de même une classe plus ou moins faible, mais destinée à bientôt dominer, a déjà autour d'elle une ébauche du courant d'idées qui dominera avec et par elle. C'est ainsi que Marx expliquait le socialisme de son époque, y compris le phénomène Marx. Il se regardait comme étant l'hirondelle dont la simple présence annonce par elle-même l'imminence du printemps, c'est-à-dire de la révolution. Il était pour lui-même un présage.\par
La seconde démarche de sa tentative d'explication a consisté à chercher le mécanisme de la puissance sociale. Cette partie de sa pensée est extrêmement faible. Il a cru pouvoir affirmer que les rapports de puissance dans une société donnée, si l'on fait abstraction des traces du passé, dépendent entièrement des conditions techniques de la production. Ces conditions étant données, une société a la structure qui rend possible le maximum de production. En essayant de produire toujours davantage, elle améliore les conditions de la production. Ainsi ces conditions changent. Un moment vient où se produit une rupture de continuité, comme lorsque de l'eau, étant graduellement échauffée, se met soudain à bouillir. Les conditions nouvelles rendent nécessaire une nouvelle structure. Il se produit un changement effectif de puissance, suivi, après un certain intervalle et avec des circonstances plus ou moins violentes, du changement politique, juridique, idéologique correspondant. Quand les circonstances sont violentes, on appelle cela une révolution.\par
Il y a là une pensée juste, mais, par une ironie singulière, en contradiction absolue avec la position politique de Marx. C'est qu'une révolution visible ne se produit jamais que comme sanction d'une révolution invisible déjà consommée. Quand une couche sociale s'empare bruyamment du pouvoir, c’est qu'elle le possédait déjà silencieusement, au moins dans une très grande mesure ; autrement elle n'aurait pas la force nécessaire pour s'en emparer. C'est là une évidence, dès lors qu'on regarde la société comme étant régie par des rapports de force. Cela est pleinement vérifié par la Révolution française, qui, comme Marx lui-même l'a montré, a officiellement livré à la bourgeoisie le pouvoir qu'elle possédait déjà en fait au moins depuis Louis XIV. Cela est vérifié aussi par les révolutions récentes qui, dans plusieurs pays, ont mis la totalité de la vie nationale sous le pouvoir de l'État. Auparavant déjà, l'État était beaucoup et presque tout.\par
La conséquence évidente, semble-t-il, pour un partisan de la révolution ouvrière, c'est qu'avant de lancer les ouvriers dans l'aventure d'une révolution politique, il faut chercher s'il existe des méthodes susceptibles de les amener à s'emparer silencieusement, graduellement, presque invisiblement, d'une grande partie de la puissance sociale réelle ; et qu'il faut ou appliquer ces méthodes si elles existent, ou renoncer à la révolution ouvrière si elles n'existent pas. Mais si évidente que soit cette conséquence, Marx ne l'a pas vue, et cela parce qu'il ne pouvait pas la voir sans perdre ce qui était pour lui sa raison de vivre. Pour la même raison, ses disciples, soit réformistes, soit révolutionnaires, ne risquaient pas de la voir. C'est pourquoi on peut dire, sans crainte d'exagérer, que comme théorie de la révolution ouvrière le marxisme est un néant.\par
Le reste de sa théorie des transformations sociales s'appuie sur plusieurs niaiseries. La première consiste à adopter pour l'histoire humaine le principe d'explication de Lamarck, « la fonction crée l'organe » ; ce principe selon lequel la girafe aurait tellement essayé de manger des bananes que son cou se serait allongé. C'est le genre d'explication qui, sans contenir même un commencement d'indication pour la solution d'un problème, donne la fausse impression qu'il est résolu, et empêche ainsi de le poser. Le problème est de savoir comment les organes des animaux se trouvent être adaptés aux besoins ; en donnant comme réponse la supposition d'une tendance à l'adaptation inhérente à la vie animale, on tombe dans la faute que Molière a ridiculisée pour toujours à propos de la vertu dormitive de l'opium.\par
Darwin a nettoyé le problème par la notion simple et géniale de conditions d'existence. Il est étonnant qu'il y ait des animaux sur la terre. Mais dès lors qu'il y en a, il n'est pas étonnant qu'il y ait correspondance entre leurs organes et les nécessités de leur vie, car autrement ils ne vivraient pas. Il n'y a aucune chance qu'on découvre jamais dans un recoin du monde une espèce exclusivement mangeuse de bananes, mais qu'un défaut de conformation malencontreux empêcherait de manger des bananes.\par
Il y a là une de ces évidences trop évidentes et que personne ne voit, jusqu'à ce qu'une intuition géniale les rende manifestes. En fait, celle-là avait été reconnue par les Grecs, comme c'est le cas pour presque toutes nos idées ; mais elle avait été oubliée ensuite. Darwin était contemporain de Marx. Mais Marx, comme tous les scientistes, était très en retard en matière de science. Il a cru faire œuvre de savant en transportant purement et simplement les naïvetés de Lamarck dans le domaine social.\par
Il a même ajouté un degré d'arbitraire en plus en admettant que la fonction crée non seulement un organe capable de l'accomplir, mais encore, en gros, dans l'ensemble, l'organe capable de l'accomplir avec le plus haut degré d'efficacité. Sa sociologie est fondée sur des postulats qui, soumis à l'examen du raisonnement, se révèlent sans fondement, et qui, comparés aux faits, sont manifestement faux.\par
Il suppose d'abord que, les conditions techniques de la production étant données, la société. a la structure capable de les utiliser au maximum. Pourquoi ? En vertu de quelle nécessité les choses se passeraient-elles de manière que la capacité de production soit utilisée au maximum ? En fait personne n'a aucune idée de ce que peut être un tel maximum. Il est seulement visible qu'il y a toujours eu beaucoup de gaspillage dans toutes les sociétés. Mais cette idée de Marx s'appuie sur des notions tellement vagues qu'on ne peut même pas montrer qu'elle soit fausse, faute de pouvoir la saisir.\par
En second lieu, la société s'efforcerait continuellement d'améliorer la production. C'est le postulat des économistes libéraux, transféré de l'individu à la société. On peut l'admettre avec réserves ; mais en fait il y a eu beaucoup de sociétés où pendant des siècles les gens ne songeaient qu'à vivre comme vivaient leurs pères.\par
En troisième lieu, cet effort réagirait sur les conditions mêmes de la production, et cela toujours de manière à les améliorer. Si on raisonne sur cette affirmation, on voit qu'elle est arbitraire ; si on la compare aux faits, on voit qu'elle est fausse. Il n'y a aucune raison pour qu'en essayant de faire rendre davantage aux conditions de la production on les développe toujours. On peut aussi bien les épuiser. Cela se produit très souvent. C'est le cas par exemple pour une mine et pour un champ. Le même phénomène se produit, de période en période, à une grande échelle, et provoque de grandes crises. C'est l'histoire de la poule aux oeufs d'or. Esope en savait beaucoup plus long là-dessus que Marx.\par
En quatrième lieu, quand cette amélioration a dépassé un certain degré, la structure sociale, qui auparavant était la plus efficace possible du point de vue de la production, ne l'est plus ; et de ce seul fait, d'après Marx, il résulte nécessairement que la société abandonne cette structure et en adopte une autre qui soit la plus efficace possible.\par
Cela, c'est le comble de l'arbitraire. Cela ne résiste pas à une minute d'examen attentif. Certainement, de tous les hommes qui ont participé aux changements politiques, sociaux, économiques des siècles passes, aucun ne s'est jamais dit : « Je vais provoquer un changement de structure sociale afin que la capacité de production actuelle soit utilisée au maximum. » On ne voit pas non plus le moindre signe d'un mécanisme automatique qui résulterait des lois de la nécessité sociale et déclencherait une transformation lorsque la capacité de production ne serait pas pleinement utilisée. Ni Marx ni les marxistes n'ont jamais fourni la moindre indication en ce sens.\par
Faut-il donc supposer qu'il y a derrière l'histoire humaine un esprit tout-puissant, une sagesse qui veille au cours des événements et le dirige ? Marx alors admettrait sans le dire la vérité que connaissait Platon. Il n'y a pas d'autre manière de rendre compte de sa conception. Mais elle reste quand même bizarre. Pourquoi cet esprit caché veillerait-il aux intérêts de la production ? L'esprit est ce qui tend au bien. La production n'est pas le bien. Les industriels du XIXe siècle ont été seuls à faire la confusion. L'esprit caché qui dirige les destinées du genre humain n'est pourtant pas celui d'un industriel du XIX e siècle.\par
L'explication, c'est que le XIXe siècle a été obsédé par la production, et surtout par le progrès de la production, et que Marx a été servilement soumis à l'influence de son époque. Cette influence lui a fait oublier que la production n'est pas le bien. Il a oublié aussi qu'elle n'est pas la seule nécessité, ce qui est cause d'une autre niaiserie ; la croyance que la production est l'unique facteur des rapports de force. Marx oublie purement et simplement la guerre. Il en a été de même de la plupart de ses contemporains. Les gens du XIXe siècle, tout en se gorgeant de chansons de Béranger et d'images d'Epinal à la louange de Napoléon, avaient presque oublié l'existence de la guerre. Marx a pensé à indiquer une fois brièvement que les modalités de la guerre dépendent des conditions de la production ; mais il n'a pas vu la relation réciproque par laquelle les conditions de la production sont soumises aux modalités de la guerre. L'homme peut être menace de mort, ou par la nature, ou par son semblable, et la force en fin de compte se ramène à la menace de mort. En considérant les rapports de force, il faut toujours concevoir la force sous son double aspect, le besoin et les armes.\par
Cet oubli de la part de Marx a eu pour conséquence, dans les milieux marxistes, un désarroi ridicule devant la guerre et les problèmes relatifs à la guerre et à la paix. Il n'y a rigoureusement rien, dans ce qu'on nomme la doctrine marxiste, qui indique l'attitude que doit prendre un marxiste à l'égard de ces problèmes. Pour une époque comme la nôtre, c'est une lacune assez sérieuse.\par
La seule forme de guerre dont Marx tienne compte, c'est la guerre sociale, ouverte ou sourde, sous le nom de lutte des classes. Il en fait même l'unique principe d'explication historique. Comme d'autre part le développement de la production est aussi l'unique principe de développement historique, il faut supposer que ces deux phénomènes n'en font qu'un. Mais Marx ne dit pas comment ils se ramènent l'un à l'autre. Certainement les opprimés qui se révoltent ou les inférieurs qui veulent devenir supérieurs ne pensent jamais à augmenter la capacité de production de la société. La seule liaison qu'on puisse concevoir, c'est que la protestation permanente des hommes contre la hiérarchie sociale maintient la société dans l'état de fluidité nécessaire pour que les forces de production puissent la modeler à leur gré.\par
En ce cas la lutte des classes n'est pas un principe agissant, mais seulement une condition négative. Le principe agissant reste cet esprit mystérieux qui veille a maintenir la production au niveau maximum, et que les marxistes nomment parfois, au pluriel, les forces productives. Ils prennent cette mythologie très au sérieux. Trotsky a écrit que la guerre de 1914 était en réalité une révolte des forces productives contre les limitations du système capitaliste. On peut rêver longtemps devant une pareille formule et s'en demander la signification, jusqu'à ce qu'on soit forcé de s'avouer qu'elle ne veut rien dire.\par
Au reste Marx a eu raison de regarder l'amour de la liberté et l'amour de la domination comme les deux ressorts qui agitent perpétuellement la vie sociale. Seulement il a oublié de montrer qu'il y a là un principe d'explication matérialiste. Ce n'est pas évident. L'amour de la liberté et l'amour de la domination sont deux faits humains qu'on peut interpréter de plusieurs manières différentes.\par
De plus ces deux faits ont une portée bien plus étendue que le rapport d'opprimé à oppresseur qui a seul retenu l'attention de Marx. On ne peut pas faire usage de la notion d'oppression sans avoir fait un sérieux effort pour la définir, car elle n'est pas claire. Marx ne s'en est pas donné la peine. Les mêmes hommes sont opprimés à certains égards, oppresseurs à d'autres ; ou encore peuvent désirer le devenir, et ce désir peut l'emporter sur celui de la liberté ; et les oppresseurs, de leur côté, pensent bien moins souvent à maintenir leurs inférieurs dans l'obéissance qu'à l'emporter sur leurs semblables. Il y a ainsi non pas l'analogue d'une bataille où s'opposent deux camps, mais comme un enchevêtrement extraordinairement complexe de guérillas. Cet enchevêtrement est régi pourtant par des lois. Mais elles sont à découvrir.\par
La seule contribution réelle de Marx à la science sociale, c'est d'avoir posé qu'il en faut une. C'est beaucoup ; c'est immense ; mais nous en sommes toujours au même point. Il en faut toujours une. Marx ne s'est pas même préparé à commencer à la constituer. Ses disciples encore moins. Dans le terme de socialisme scientifique par lequel le marxisme s'est désigné lui-même, l'épithète scientifique ne correspond pas à autre chose qu'à une fiction. On serait tenté de dire plus crûment un mensonge ; mais Marx et la plupart de ses disciples n'ont pas voulu mentir. Si ces hommes n'avaient pas été d'abord leurs propres dupes, on devrait qualifier d'escroquerie l'opération par laquelle ils ont fait tourner à leur bénéfice exclusif le respect des hommes d'aujourd'hui pour la science.\par
Marx était incapable d'un véritable effort de pensée scientifique, parce que cela ne l'intéressait pas. Ce matérialiste ne s'intéressait qu'à la justice. Il en était obsédé. Sa vue si claire de la nécessité sociale était de nature à le désespérer, puisque c'est une nécessité assez puissante pour empêcher les hommes, non seulement d'obtenir, mais même de penser la justice. Il ne voulait pas du désespoir. Il sentait irrésistiblement en lui-même que le désir de justice de l'homme est trop profond pour admettre un refus. Il s'est réfugié dans un rêve où la matière sociale elle-même se charge des deux fonctions qu'elle interdit à l'homme, à savoir non seulement d'accomplir, mais de penser la justice.\par
Il a mis à ce rêve l'étiquette de matérialisme dialectique. C'était assez pour le couvrir d'un voile. Ces deux mots sont d'un vide presque impénétrable. Un jeu très amusant, mais un peu cruel, consiste à demander à un marxiste leur signification.\par
On leur trouve quand même une espèce de signification en cherchant beaucoup. Platon nommait dialectique le mouvement de l'âme qui, à chaque étape, pour monter au domaine supérieur, s'appuie sur les contradictions irréductibles du domaine dans lequel elle se trouve. Au terme de cette ascension, elle est au contact du bien absolu.\par
L'image de la contradiction dans la matière, c'est le heurt des forces de direction différente, Marx a purement et simplement attribué à la matière sociale ce mouvement vers le bien à travers les contradictions, que Platon a décrit comme étant celui de la créature pensante tirée en haut par l'opération surnaturelle de la grâce.\par
Il est facile de voir comment il y a été conduit. Tout d'abord, il a adopté sans réserves les deux croyances fausses auxquelles tenaient si fort les bourgeois de son temps. L'une est la confusion entre la production et le bien, et par suite entre le progrès de la production et le progrès vers le bien ; l'autre est la généralisation arbitraire par laquelle on fait du progrès de la production, si sensible au XIXe siècle, la loi permanente de l'histoire humaine.\par
Seulement, contrairement aux bourgeois, Marx n'était pas heureux. La pensée de la misère le bouleversait, comme quiconque n'est pas insensible. Il lui fallait, comme compensation, quelque chose de catastrophique, une revanche éclatante, un châtiment. Il ne pouvait pas se représenter le progrès comme un mouvement continu. Il le voyait comme une série de secousses violentes, explosives. Il est bien inutile de se demander qui, des bourgeois ou de lui, avait raison. Cette notion même de progrès en faveur au XIXe siècle n'a pas de sens.\par
Les Grecs employaient le mot dialectique quand ils pensaient a la vertu de la contradiction comme support de l'âme tirée en haut par la grâce. Comme Marx de son côté combinait l'image matérielle de la contradiction et l'image matérielle du salut de l'âme, à savoir les heurts entre forces et le progrès de la production, il a eu raison peut-être d'employer ce mot de dialectique. Mais d'un autre côté ce mot, accouplé à celui de matérialisme, révèle aussitôt l'absurdité. Si Marx ne l'a pas senti, c'est qu'il n'a pas emprunté le mot aux Grecs, mais à Hegel, qui déjà l'employait sans signification précise. Quant au public, il ne risquait pas d'être choqué ; la pensée grecque n'est plus assez vivante pour cela. Les mots étaient très bien choisis au contraire pour amener les gens a se dire : « Cela doit signifier quelque chose. » Quand des lecteurs ou des auditeurs ont été mis dans cet état, ils sont très accessibles à la suggestion.\par
Autrefois, dans les universités populaires, des ouvriers disaient parfois, avec une sorte d'avidité timide, à des intellectuels qui se disaient marxistes : « Nous voudrions bien savoir ce que c'est que le matérialisme dialectique. » Il est peu probable qu'ils aient jamais obtenu satisfaction.\par
Quant au mécanisme de la production automatique du bien absolu par les conflits sociaux, la conception que Marx en avait n'est pas difficile à saisir ; tout cela est très sommaire.\par
La source du mensonge social résidant dans les groupes en lutte pour la domination ou l'émancipation, la disparition de ces groupes abolirait le mensonge, et l'homme serait dans la justice et la vérité. Et par quel mécanisme ces groupes peuvent-ils disparaître ? C'est très simple. Toutes les fois qu'une transformation sociale se produit, le groupe dominant tombe, et un groupe relativement inférieur prend sa place. On n'a qu'a généraliser ; toute la science et même toute la pensée du me siècle avait cette coutume vicieuse de l'extrapolation sans contrôle ; excepté dans la mathématique, la notion de limite était presque ignorée. Si chaque fois un groupe place plus bas s'élève à la domination, un jour ce sera le plus bas de tous ; dès lors il n'y aura plus d'inférieurs, plus d'oppression, plus de structure sociale par groupes ennemis, plus de mensonge. Les hommes posséderont la justice, et parce qu'ils la posséderont, ils la connaîtront telle qu'elle est.\par
C'est ainsi qu'il faut comprendre les passages où Marx semble exclure complètement les notions mêmes de justice, de vérité ou de bien. Tant que la justice est absente, l'homme ne peut pas la penser, et à plus forte raison il ne peut pas se la procurer ; elle ne peut lui venir que du dehors. La société étant viciée, empoisonnée, et le poison social s'infiltrant dans toutes les pensées de tous les hommes, tout ce que les hommes imaginent sous le nom de justice est du mensonge. Quiconque parle de justice, de vérité, ou de n'importe quelle espèce de valeur morale, ment ou se laisse duper par des menteurs. Comment donc servir la justice, si on ne la connaît pas ? L'unique moyen, d'après Marx, est de hâter l'opération de ce mécanisme, inscrit dans la structure même de la matière sociale, qui apportera automatiquement la justice aux hommes.\par
Il est difficile de se rendre compte vraiment si Marx pensait que le rôle du prolétariat dans ce mécanisme, en le mettant plus près de la société future, lui communiquait, à lui et aux écrivains ou militants qui se rangeaient avec lui, comme une première lueur de la vérité ; ou s'il regardait le prolétariat seulement comme un instrument aveugle de cette entité qu'il nommait l'histoire. Sans doute sa pensée a-t-elle oscille sur ce point. Mais certainement il regardait le prolétariat, y compris ses alliés et chefs venus du dehors, avant tout comme un instrument.\par
Il regardait comme juste et bon, non pas ce qui apparaît tel à un des esprits fausses par le mensonge social, mais exclusivement ce qui pouvait hâter l'apparition d'une société sans mensonge ; en revanche, dans ce domaine, tout ce qui est efficace, sans aucune exception, est parfaitement juste et bon, non pas en soi, mais relativement au but final.\par
Ainsi Marx, finalement, retombait dans cette morale de groupe qui lui répugnait au point de lui faire haïr la société. Comme autrefois les féodaux, comme de son temps les gens d'affaires, il s'était fabriqué une morale qui mettait au-dessus du bien et du mal l'activité du groupe social dont il faisait partie, celui des révolutionnaires professionnels.\par
Il en est toujours ainsi. L'espèce de défaillance que l'on redoute et que l'on hait le plus, dont on a le plus horreur, est toujours celle où l'on tombe, quand on ne cherche pas la source du bien où elle est. C'est le piège perpétuellement tendu à tout homme, et contre lequel il n'est qu'une seule protection.\par
Ce mécanisme producteur de paradis que Marx imaginait est quelque chose d'évidemment puéril. La force est une relation ; ceux qui sont forts le sont par rapport à de plus faibles. Ceux qui sont faibles n'ont pas la possibilité de s'emparer du pouvoir social ; ceux qui s'emparent du pouvoir social par la force constituent toujours, même avant cette opération. un groupe auquel des masses humaines sont soumises. Le matérialisme révolutionnaire de Marx consiste à poser, d'une part que tout est règle exclusivement par la force, d'autre part qu'un jour viendra soudain où la force sera du côté des faibles. Non pas que certains qui étaient faibles deviendront forts, changement qui s'est toujours produit ; mais que la masse entière des faibles, demeurant la masse des faibles, aura la force de son côté.\par
Si l'absurdité ne saute pas aux yeux, c'est qu'on pense que le nombre est une force. Mais le nombre est une force aux mains de celui qui en dispose, non pas aux mains de ceux qui le constituent. Comme l'énergie enfermée dans le charbon est une force seulement après avoir passé par une machine à vapeur, de même l'énergie enfermée dans une masse humaine est une force seulement pour un groupe extérieur à la masse, beaucoup plus petit qu'elle, et ayant établi avec elle des relations qui, au prix d'une étude très attentive, pourraient peut-être être définies. Il en résulte que la force de la masse est utilisée pour des intérêts qui lui sont extérieurs, exactement comme la force d'un boeuf pour l'intérêt du laboureur, d'un cheval pour l'intérêt du cavalier. Quelqu'un peut pousser le cavalier à terre et se mettre en selle à sa place, puis être renversé à son tour ; cela peut se répéter cent et mille fois ; le cheval devra quand même courir sous l'éperon. Et s'il renverse lui-même le cavalier, un autre en prendra bientôt la place.\par
Marx savait très bien tout cela ; il l'a exposé brillamment à propos de l'État bourgeois ; mais il voulait l'oublier quand il s'agissait de la révolution. Il savait que la masse est faible et ne constitue une force qu'aux mains d'autrui ; car s'il en était autrement il n'y aurait jamais eu d'oppression. Il se laissait persuader uniquement par la généralisation, le passage à la limite de ce changement perpétuel qui met périodiquement ceux qui étaient moins forts à la place de ceux qui étaient plus forts. Le passage à la limite, quand il est appliqué à une relation dont il supprime un des termes, est par trop absurde. Mais ce raisonnement misérable suffisait à Marx, parce que tout suffit pour persuader celui qui sent que, s'il n'était pas persuadé, il ne pourrait pas vivre.\par
L'idée que la faiblesse comme telle, demeurant faible, peut constituer une force, n'est pas une idée nouvelle. C'est l'idée chrétienne elle-même, et la Croix en est l'illustration. Mais il s'agit d'une force d'une tout autre espèce que celle qui est maniée par les forts ; c'est une force qui n'est pas de ce monde, qui est surnaturelle. Elle opère à la manière du surnaturel, décisivement, mais secrètement, silencieusement, sous l'apparence de l'infiniment petit ; et si elle pénètre les masses par rayonnement, elle n'habite pas en elles, mais dans certaines âmes. Marx a admis cette contradiction d'une faiblesse forte, sans admettre le surnaturel qui seul rend la contradiction légitime.\par
De même, Marx a senti une vérité, une vérité essentielle, quand il a compris que l'homme ne conçoit la justice que s'il l'a...\par
(Ici s'arrête le manuscrit, écrit à Londres en 1943 et inachevé...)\par

\begin{center}
\end{center}
\backmatter \section[Appendice ]{Appendice \protect\footnotemark }\renewcommand{\leftmark}{Appendice }

\footnotetext{On trouvera ci-après des ébauches, des plans et des variantes qui se rapportent à certains textes de ce volume. Pour les ébauches et les variantes, nous indiquons les pages auxquelles elles correspondent approximativement.}
\noindent \par
\subsection[PERSPECTIVES]{PERSPECTIVES}

\begin{center}
ÉBAUCHES\end{center}
\noindent {\itshape Pages 15-20}. – [Il ne faut pas oublier qu'à plusieurs] reprises, et notamment lors de la grève des transports, hitlériens et communistes [ont combattu] ensemble la social-démocratie, accusée de part et d'autre de trahison à l'égard de la classe révolutionnaire ; ni qu'après les élections du 6 novembre les hitlériens ont dû démentir le bruit qui courait d'un gouvernement de coalition des partis national-socialiste et communiste. Tout ce qu'on trouve à dire à ce sujet, c'est que cette propagande est démagogique. Et il est bien évident qu'elle est démagogique ; mais c'est là une appréciation, non une explication. Expliquer un mouvement politique par la démagogie, c'est raisonner comme Voltaire, qui expliquait la religion comme une savante duperie inventée par les prêtres pour exploiter les fidèles. Un matérialiste doit chercher ce qui, dans les bases économiques du mouvement, rend possible cette orientation démagogique. De même le programme du mouvement national-socialiste, comportant une économie fermée, dirigée souverainement par l'État conformément à un plan, est quelque chose de nouveau, qui ne ressemble à aucun programme d'aucun mouvement politique d'avant-guerre. Ici encore, ce n'est pas expliquer ce programme que dire que c'est un simple programme d'agitation, impossible à réaliser. « Les idées ne tombent pas du ciel. » Les programmes non plus. L'orientation et le programme d'un mouvement politique sont déterminés, non par les ruses des chefs, mais par la composition sociale du mouvement.\par
Le régime dit soviétique qui existe en U.R.S.S. est le second fait nouveau qui donne son caractère à la politique de notre époque. Ici beaucoup de camarades protesteront. La plupart des communistes, même oppositionnels, considèrent l'U.R.S.S., non comme un phénomène nouveau par rapport aux analyses antérieures, mais comme une confirmation éclatante des prévisions de Marx et des marxistes russes. Et cependant, ce qui se produit en Russie, qui l'avait prévu ? Lénine et Trotsky prévoyaient ou le triomphe rapide de la révolution dans les pays industriels d'Occident, ou le retour rapide de la Russie au capitalisme. Aucun des deux ne s'est produit. Trotsky dit, il est vrai, ne s'être trompé que sur l'appréciation du délai ; mais, après quinze ans, on peut penser que « la quantité se change en qualité », et l'erreur concernant le délai en erreur fondamentale. Mais, question bien. plus importante, qui avait prévu le régime actuel de l'U.R.S.S. ? Lénine et Trotsky prévoyaient la dictature du prolétariat. Ce qu'est la dictature du prolétariat pour un marxiste, on peut le savoir en lisant {\itshape L’État et la Révolution} de Lénine. La dictature du prolétariat, c'est un régime politique où il n'y a ni bureaucratie permanente, dont les membres forment une caste distincte de la population, ni police permanente, ni armée permanente. Or, du vivant même de Lénine, les exigences de la lutte et de l'administration ont très vite donné naissance à un appareil d'État distinct et permanent, comprenant bureaucratie, armée et police. Depuis la mort de Lénine, le fossé qui sépare cet appareil de la population travailleuse, et notamment du prolétariat, s'est formidablement élargi. L'oppression exercée par cet appareil sur les ouvriers devient de plus en plus étouffante. Cette oppression, Trotsky ne cesse de la dénoncer avec force, et elle apparaît clairement dans les documents officiels de l'U.R.S.S., comme le décret établissant le passeport intérieur, et surtout les décrets punissant de renvoi l'ouvrier qui est resté absent de l'usine un jour sans excuse valable, et empêchant l'ouvrier renvoyé de continuer à manger, même une journée, à la coopérative de l'usine. Tout cela n'empêche pas non seulement les staliniens, mais même Trotsky, de dire qu'en U.R.S.S. il y a dictature du prolétariat. Trotsky dit que, dans l'appareil d'État russe, la dictature du prolétariat se reflète déformée par la bureaucratie, mais qu'elle s'y reflète tout de même, dé manière que l'U.R.S.S. demeure un État ouvrier. On se demande ce que peut signifier un tel langage chez un marxiste. Descartes disait qu'il ne faut pas étudier une horloge détraquée comme une horloge qui s'écarte des lois des horloges, mais comme un mécanisme obéissant à ses lois propres, composé de rouages dont il faut déterminer les rapports. De même on n'explique pas l‘État russe en disant que c'est un « État ouvrier déformé », mais en cherchant sur quelles couches sociales il s'appuie, et quels rapports de force existent, dans le domaine politique, et par suite aussi dans le domaine économique, entre ces couches sociales. En U.R.S.S., il est clair que ce n'est pas le prolétariat qui domine ; l’État n'est donc pas ouvrier. Trotsky, pour montrer à Urbahns que l’État russe est ouvrier, montrait que ce n'et pas un État bourgeois ; il s'appuyait donc sur le principe suivant : l’État est ouvrier ou bourgeois. C'est en partant de ce principe qu'il avait prédit, ainsi que Lénine, que la révolution d'Octobre s'étendrait ou périrait. Car, disaient-ils, il ne peut y avoir dictature du prolétariat dans les limites des frontières d'un pays arriéré comme la Russie ; et un pays ne peut vivre que sous la dictature du prolétariat ou sous celle de la bourgeoisie. Comme il n'est résulté, de ce principe, que des prévisions démenties par l'expérience, il semble raisonnable de se demander si le principe lui-même n'est pas inexact. Ce qui semble le prouver, c'est que l’État russe n'est évidemment pas bourgeois, et que cependant il n'est prolétarien ni par sa structure, ni par ses aspirations. Il aspire à une économie entièrement dirigée par lui et entièrement fermée. « C'est là, écrit Trotsky, l'idéal de Hitler, et non pas d'un marxiste . » Mais Trotsky ne donne à cette formule qu'une valeur polémique. Bien qu'il répète volontiers que « les idées ne tombent pas du ciel », il ne se demande pas par quel hasard l'idéal de Hitler a germé dans le pays de la révolution d'Octobre.\par
À côté de ces grands phénomènes, quelques autres, de portée à peu près nulle, mais très significatifs, se sont produits ces derniers temps. Tel est le mouvement qu'on a désigné en Amérique sous le nom de technocratie. Si on laisse de côté les détails, tels que le projet d'une monnaie basée sur l'unité d'énergie, il s'agissait essentiellement d'une dictature de techniciens, qui régleraient souverainement production et consommation. Comment cette dictature sera établie, les « technocrates » ne le disent pas, et ils disent même se moquer de la politique. Un autre mouvement intéressant est celui qui a eu pour organe la revue allemande {\itshape Die Tot} ; le manifeste de ce mouvement est un livre paru il y a deux ans, dont l'auteur est Ferdinand Fried, et le titre {\itshape Là Fin du Capitalisme}. Le programme est, à peu de choses près, le programme hitlérien, avec cette différence que les syndicats réformistes sont considérés comme devant constituer un appui pour la souveraineté de l’État en matière économique ; l'idée essentielle est que la concurrence, « l’âme du capitalisme », s'est transformée en son contraire, et que par suite l'économie doit être entre les mains, non plus des capitalistes, mais d'une bureaucratie.\par
Dans tous ces phénomènes se reflète une idéologie tout à fait nouvelle, et jusqu'à un certain point identique à elle-même sous ses formes diverses. On a souvent rapproché l'idéal technocratique du fascisme et surtout du communisme ; par ce terme, les bourgeois désignent, bien entendu, la structure et l'orientation du régime actuel de l'U.R.S.S. Et Ferdinand Fried, qui est tout proche de Hitler, considère l'U.R.S.S. comme un modèle du régime qu'il propose ; il pense seulement qu'on peut aboutir à ce régime sans avoir besoin de supprimer, comme ont fait les Russes, la propriété privée. Si à cette idéologie politique nouvelle ne correspondait pas quelque chose de nouveau dans le mode de production, ce serait à désespérer du matérialisme historique.\par
{\itshape Pages 21-22.} - ... Nous ne nous rendons pas assez clairement compte de la profondeur des transformations subies par le régime capitaliste. Déjà dans le {\itshape Capital}, il apparaît clairement que l'achat de la force de travail par le propriétaire de l'entreprise, achat par quoi se définit le système capitaliste, n'est plus, au moment de la grande industrie, qu'un facteur subordonné de l'oppression qui écrase le producteur. Sur le marché du travail, l'ouvrier se vend au patron ; mais, quand il a franchi le seuil de la fabrique, il devient l'esclave de l'entreprise. On connaît, là-dessus, les terribles formules de Marx : « Dans l'artisanat et la manufacture, l'ouvrier se servait de l'outil ; dans la fabrique il est au service de la machine. » Il est « un rouage vivant d'un mécanisme inerte ». « La séparation des forces spirituelles du procès de production et du travail manuel et leur transformation en puissance d'oppression du capital sur le travail s'accomplit... dans la grande industrie basée sur la machine. La destinée individuelle de l'ouvrier des fabriques disparaît... devant la science, les forces naturelles formidables et le travail collectif qui sont cristallisés dans la machine et constituent la force du maître. » Ainsi, dès le dix-neuvième siècle, l'opposition entre le prolétariat et le patronat a cessé d'être une question de propriété ; elle est devenue une question de fonction par rapport à la machine. Au temps de l'atelier et de la manufacture, c'était l'argent qui séparait en deux classes la population industrielle ; au temps de la grande industrie, c'est la machine elle-même qui sépare par une barrière infranchissable d'une part ceux qui la dirigent, d'autre part, ceux qui en forment les rouages vivants. Et cette nouvelle forme de la division des classes transforme entièrement la question sociale. Car l'on voit très bien comment une révolution peut supprimer l'argent ou du moins en changer la fonction ; mais on ne voit nullement, et Marx lui-même ne s'explique guère là-dessus, comment peuvent se transformer les propriétés sociales du machinisme. Les classes ne sont plus ceux qui vendent et ceux qui achètent la force de travail, mais ceux qui disposent de la machine et ceux dont la machine dispose.\par
{\itshape Pages 22-24.} - En fait, il y a quelque chose de nouveau dans la production. La transformation commençait à peine avant la guerre ; elle ne s'est traduite dans le domaine des idées qu'après la guerre, et surtout depuis la crise. Cette transformation a abouti d'une part à accroître le fossé qui sépare l'ouvrier de l'entreprise, et d'autre part à séparer de l'entreprise le capitaliste lui-même ; l'entreprise n'est plus représentée, comme du temps de Marx, par le Capitaliste, mais par une bureaucratie anonyme et irresponsable. C'est ce qu'exposent fort clairement divers ouvrages bourgeois, et notamment une {\itshape Histoire des chefs d'entreprise}, de Palewski, parue en 1928. « Les entreprises, peut-on y lire, tendent de plus en plus à échapper des mains de ces capitaines, chefs et possesseurs primitifs de l'affaire. Le contrôle financier des sociétés ne leur appartient plus que rarement... Nous arrivons à l'époque qu'on peut appeler l'ère des techniciens de la direction, et ces techniciens sont aussi éloignés des ingénieurs et des capitalistes que des ouvriers… Le chef n'est plus un capitaliste maître de l'entreprise, il est remplacé par un conseil de techniciens. Nous vivons encore sur ce passé si proche, et l'esprit a quelque peine à saisir cette évolution. » Au reste, dès 1914, l'Américain Jones écrivait. « La troisième période (il distingue trois périodes, dans l'industrie américaine, depuis 1840), qui s'ouvre actuellement, est caractérisée par l'organisation administrative des entreprises, » Mais, à cette époque, l'organisation est le plus souvent encore l’œuvre de « capitaines d'industrie»; de nos jours au contraire, un capitaine d'industrie, tel que Ford, apparaît, comme le dit l'économiste Pound, comme une survivance du passé. En même temps que l'entreprise échappe au capitaliste, le prolétariat perd le peu de prise qu'il avait sur elle, du fait de la diminution des ouvriers qualifiés. Ce second changement est plus récent encore que le premier. L'ouvrier qualifié, celui qui sait, non seulement servir une machine, mais régler et conduire toute une série de machines, celui sans la science de qui l'usine ne peut fonctionner, a presque disparu pendant les années de prospérité économique. Il a été remplacé par une équipe de manoeuvres spécialisés, qui ne peuvent que servir un type déterminé de machines, équipe conduire par un régleur, qui ne travaille pas lui-même et qui appartient au personnel de la maîtrise. Il n'y a plus guère d'ouvriers qualifiés qu'à l'outillage ; mais il y a des entreprises, et c'est notamment le cas pour Citroën, où il n'y a plus du tout d'ouvriers qualifiés. Ainsi trois catégories économiques se distinguent nettement vis-à-vis de l'entreprise ; les propriétaires du capital, qui possèdent seulement, sous le nom d'actions, des morceaux de papier donnant droit à une part des profits ; les manoeuvres, spécialisés ou non, qui sont seulement des instruments vivants, mais aveugles, de l'entreprise ; et la direction de l'entreprise, qui consiste en un appareil bureaucratique, rétribué en partie par un traitement fixe, en partie par des primes. Les ingénieurs, les contremaîtres, les régleurs, tous gens qui sont séparés du travail matériel de la production, et ne font que le diriger, sont sous les ordres de cet appareil bureaucratique. Les ouvriers qualifiés ne forment plus que de petits noyaux impuissants, de qui la marche de l'entreprise a cessé de dépendre. Le caractère essentiel que Marx a mis en lumière dans son étude du régime, à savoir la domination des hommes par les choses, n'est pleinement réalisé que de nos jours. La prise que la conscience humaine avait encore sur l'entreprise, d'une part par le capitaliste propriétaire et chef de l'entreprise, d'autre part par l'ouvrier qualifié, a été défaite par l'évolution du régime. L'ensemble des machines qui forme le squelette matériel de l'entreprise n'est plus mené que par cette machine humaine que constitue la bureaucratie. Cette bureaucratie constitue-t-elle une nouvelle classe ? Avant de répondre à cette question, il faudrait savoir ce qu'elle signifie au juste...\par
{\itshape Pages 25-26.} Le caractère essentiel de notre époque, c'est que chaque homme, intellectuel, technicien ou ouvrier, est pris dans un ensemble qui le dépasse, dont il ne peut comprendre le fonctionnement, et par suite agit sans pouvoir prendre conscience de ce qu'il fait. Dans une pareille situation, il est une fonction qui prend une importance exceptionnelle ; c'est celle qui consiste simplement à coordonner. On peut nommer cette fonction la fonction bureaucratique ; et on peut remarquer qu'il n'est pas, de nos jours, un seul mode d'activité humaine qui ne soit plus ou moins envahi par la bureaucratie. La machine bureaucratique, qui, pour être composée de chair bien nourrie, n'en est pas moins aussi aveugle que les machines de fer et d'acier, met à son service comme celles-ci toute l'activité humaine. Nulle part cette domination n'existe sous une forme plus pure que dans ce que l'on nomme encore, par habitude, l'Union des Républiques Socialistes Soviétiques. Mais, dans les pays capitalistes eux-mêmes, il y a, en dehors de l'oppression capitaliste, une sorte d'oppression bureaucratique diffuse.\par
{\itshape Pages 29-31.} - Jusqu'ici la bureaucratie, dans les pays capitalistes, reste pour ainsi dire diffuse et ne forme pas encore un système cohérent. Elle est divisée, dans la plupart des pays, en trois tronçons principaux, la bureaucratie d'État, la bureaucratie industrielle et la bureaucratie syndicale ; et Fried, pseudonyme collectif des meilleurs théoriciens de la revue {\itshape Die Tat}, remarquait en 1930 que ces trois bureaucraties se ressemblent tellement qu'on pourrait mettre n'importe laquelle des trois à la place de n'importe quelle autre. Ces trois bureaucraties soeurs tendent à s'unir en un système unique, et les fascistes de la revue {\itshape Die Tat} préconisent cette union. Ce n'est cependant qu'en U.R.S.S. qu'une seule et même bureaucratie a entre ses mains l'État, l'économie et les organisations ouvrières ; l'expropriation des capitalistes par le prolétariat soulevé a rendu cette concentration facile. Le fascisme au contraire, s'il est arrivé, lentement en Italie, avec une rapidité foudroyante en Allemagne, à fondre la bureaucratie d’État et la bureaucratie syndicale, voit en revanche la bureaucratie des entreprises et des cartels séparée de l'appareil d'État par le jeu de la propriété capitaliste. C'est par là, et non pas seulement par la simple démagogie, qu'on peut expliquer les velléités anticapitalistes du mouvement fasciste. Le fascisme n'est pas seulement un mouvement de révolte aveugle de toutes les couches sociales opprimées, mais aussi une expression du développement irrépressible de la bureaucratie au milieu du marasme général. Et les caractères politiques du fascisme s'expliquent aisément par là. Le capitalisme est seulement un système d'exploitation, qui ne tend qu'à l'expansion des entreprises ; au reste il s'accommode d'une pleine liberté dans tous les domaines, à la seule exclusion de la résistance ouvrière, et même, dans sa période progressive, constitue un puissant facteur d'émancipation. La machine bureaucratique, au contraire, tend, par sa structure propre, à la totalité du pouvoir, à la suppression de toute initiative et de toute pensée libre dans tous les domaines. L'exemple de l'U.R.S.S. le montre assez éloquemment ; et le fascisme est encore beaucoup plus significatif à cet égard. Le système d'oppression qui se dessine en ce moment dans le monde est propre à nous faire regretter amèrement l'heureux système grec, où le travail des esclaves nourrissait du moins des hommes libres ; la féodalité, où, à travers les violences et les guerres, parvenaient à s'insinuer et à prospérer des individus et des collectivités indépendantes ; le capitalisme, pendant le développement duquel l'invention et le génie ont pu se donner libre cours dans presque tous les domaines.
\subsection[SUR LE LIVRE DE LÉNINE « MATÉRIALISME ET EMPIRIOCRITICISME »]{SUR LE LIVRE DE LÉNINE « MATÉRIALISME ET EMPIRIOCRITICISME »}

\begin{center}
ÉBAUCHE.\end{center}
\noindent {\itshape Pages 51-52.} -... Le résultat le plus funeste de cette attitude d'esprit, et qui fait peser sur tous les marxistes une lourde responsabilité, a été d'empêcher d'accomplir, ou même d'esquisser, ou même d'entrevoir le seul travail de théorie pure qui puisse ou plutôt doive de toute nécessité avoir place dans la construction d'une société sans oppresseurs. Ce travail est pourtant impliqué dans toute conception vraiment socialiste. Si « la séparation de la théorie et de la pratique », « la dégradante division du travail en travail manuel et intellectuel », « la séparation des forces spirituelles du processus de la production d'avec le travail manuel » constituent la tare capitale de notre société, il faut être capables de supprimer cette tare. La critique de la religion constitue bien toujours, comme dit Marx, la condition de tout progrès ; mais ce que Marx et les marxistes n'ont pas clairement aperçu, c'est que l'esprit religieux, de nos jours, dans tout {\itshape ce} qu'il contient de rétrograde, s'est surtout réfugié dans la science elle-même. Une science telle que la nôtre, essentiellement fermée aux profanes, et par suite aussi aux savants eux-mêmes, qui sont profanes hors de leur étroite spécialité, est la théologie propre d'une société de plus en plus bureaucratique. « L'âme universelle de la bureaucratie est le {\itshape secret}, le mystère », écrivait Marx dans sa jeunesse ; et le mystère est fonde sur la spécialisation. Le mystère est la condition de tout privilège et par suite de toute oppression ; et c’est dans la science elle-même, cette briseuse d'idoles, cette destructrice de mystère, que le mystère a trouvé son dernier refuge...
\subsection[RÉFLEXIONS SUR LA LIBERTÉ ET L'OPPRESSION SOCIALE]{RÉFLEXIONS SUR LA LIBERTÉ ET L'OPPRESSION SOCIALE}
\subsubsection[A. PLANS]{A. PLANS}
\paragraph[I.- ESQUISSE D'UN BILAN.]{I.- ESQUISSE D'UN BILAN.}

\begin{center}
Réflexions concernant la liberté, l'oppression, la civilisation actuelle.\end{center}
\noindent Désarroi de notre temps.\par
Critique de l'idée de révolution.\par
Critique de la croyance à un progrès matériel illimité. Esquisse des fondements d'une science sociale.\par
Limite idéale du progrès humain.\par
Analyse de l'oppression sociale et de ses différentes formes.\par
Esquisse d'une méthode pour décrire les différentes structures sociales.\par
Tableau sommaire de la société actuelle.\par
Aspirations modernes vers une civilisation nouvelle. Valeur des formes d'activité actuellement possibles. Conclusion.
\paragraph[II]{II}
\noindent Désarroi de notre époque.\par
Examen critique de l'idée de révolution.\par
Marx - plus-value, etc. - progrès de la production - liaison arbitraire avec cours de l'histoire - (providentielle) - Hégélianisme - religion de la production - incertitude de l'axiome (progrès indéfini).\par
Problème du progrès de la production - f\textsuperscript{t} poser la question en elle-m. - forces de la nature - concentration sous ses divers aspects - substitution du travail passé au travail présent - exemples (sériés). ({\itshape À approfondir.)}\par
Principes d'une étude méthodique de la question sociale - néant de {\itshape ttes} les positions (révolutionnaires - réformistes - conservateurs) - fausse science - notions de limites, etc., etc. - conditions d'existence (Lamark-Darwin) [analogie Marx-Lamarck].\par
Limite idéale du progrès humain - Liberté : définition - Liberté : à l'égard de la nature - du corps - des autres hommes - de la société - égalité (fraternité).\par
Analyse de l'oppression sociale.\par
L'oppression sociale apparaît en proportion de l'allègement de l'oppression de la nature.\par
Nature de la force (privilège).\par
Contradiction de la notion de privilège. Fonction oppressive des privilèges.\par
Évolution des formes d'oppression (à {\itshape approfondir)}\par
Raison générale de l'oppression.\par
Ce tableau indique également une limite abstraite.\par
Idée d'une carte sociale (trouver une autre expression).\par
Tableau social de notre époque {\itshape (approfondir le rôle de la monnaie).}\par
Aspirations fécondes de notre époque (autre expression) - mais conclusion pessimiste : ça ne prouve rien.\par
Devoir. (Si jamais la vertu stoïcienne a été nécessaire pour avoir le courage de vivre, c'est aujourd'hui.)
\paragraph[III]{III}
\noindent Électricité.\par
Mécanique.\par
La résistance des matériaux.\par
Ordre théorique :\par
1º étude de la résistance des matériaux.\par
2º rapports de solides indéformables.\par
Mais la réalité pose des problèmes plus complexes, ex. chercher la cause d'une déformation {\itshape effective.} L'ingénieur fait-il alors des raisonnements mathématiques ?\par
Question : qu'est-ce que la culture mathématique apporte à la {\itshape perception} (au sens propre) des problèmes industriels ? Poser la question nettement dans la préface.\par
Dans mon introduction, poser le problème ainsi\par
1º une organisation dit {\itshape travail} moins oppressive avec le même rendement (i. e. : moralement plus de responsabilité, plus d'intelligence, une subordination moins militaire).\par
2º une organisation de l’{\itshape économie.}\par
3º une organisation {\itshape sociale.}\par
Idée fondamentale de mon article : à mesure qu'une vue de plus en plus claire des lois de la matière nous en a fait acquérir la maîtrise, nous subissons à un degré de plus en plus élevé le poids des rapports humains dont les lois nous sont inconnues. Un peu de courage intellectuel et de clairvoyance suffit pour nous faire apercevoir clairement le danger que le jeu aveugle de la machine sociale réduise l'humanité à l'excès de misère et d'abaissement dont les progrès techniques l'ont quelque peu tirée. D'autre part, d'un point de vue moral et intellectuel, cette ignorance totale à l'égard du facteur essentiel de notre existence rend vains tous les espoirs qu'ont mis les hommes les plus éminents des siècles précédents dans le progrès de lumières ; elle fait reparaître sous une autre forme l'équivalent des sottises, des superstitions, des folies provoquées autrefois par l'ignorance portant sur les phénomènes de la nature, et cela également à un degré de plus en plus élevé.\par
Ce double danger, matériel et moral, rabat au second plan toutes les tâches d'ordre théorique ou pratique qui absorbent actuellement l'ensemble des énergies humaines.\par
Il ne s'agit pas de former un groupement pour y parer, pas plus qu'il n'aurait été raisonnable au XVIe siècle de constituer une ligue contre les forces de la nature. C'est seulement dans l'homme pris comme individu que se trouvent la clairvoyance et la bonne volonté, uniques sources de l'action efficace. Mais les individus peuvent associer leurs efforts sans renoncer à leur indépendance.\par
Vanité des étiquettes politiques. On peut se lancer dans le mouvement révolutionnaire avec un esprit de chef désireux de brasser les masses et de jouer un grand rôle sur la scène de l'histoire, ou bien de soldat fanatique ; et il se trouve dans les rangs des conservateurs des hommes de bonne volonté simplement disposés à faire concourir les forces dont ils disposent au plus grand bien de tous. C'est par le caractère que les hommes sont frères ou étrangers entre eux. Ces quelques pages constituent un appel, et cet appel est adressé sans distinction à tous ceux qui d'une part sont épris de probité intellectuelle, ont besoin de voir clair dans toutes leurs pensées et de toucher la réalité du doigt, et qui d'autre part ont l'âme assez généreuse pour être fermement résolus à faire tout ce qui est en leur pouvoir pour combattre tout ce qui contribue à diminuer, à humilier, à écraser des êtres humains, et cela non par rapport à un avenir indéterminé, mais présentement.\par
Ce que je leur demande. Énumérer dans le plus grand détail.\par
Quant à la forme... Contacts personnels... Rien de trop public. Une revue à tirage restreint ? En tout cas complétée par des renseignements d'ordre privé.
\subsubsection[B. ÉBAUCHES]{B. ÉBAUCHES}
\noindent {\itshape Pages 60-61} - On a l'habitude de considérer surtout l'oppression capitaliste sous la forme qu'en reflète la comptabilité, C'est-à-dire l'extorsion de la plus-value. C'est d'ailleurs là l'effet d'une déformation de pensée qui nous vient du régime actuel, où la comptabilité prime tout. Si l'on s'en tient à ce point de vue, il est facile d'expliquer aux masses que l'extorsion de la plus-Value est liée à la concurrence, elle-même liée à la propriété privée, et que le jour où la propriété sera collective l'ensemble des travailleurs recevra l'équivalent de la valeur créée par le travail. Il est facile aussi de montrer que plus la propriété se concentre sous forme de monopoles, plus il est facile de la rendre collective ; et que plus le capitalisme se développe, plus il se heurte à des difficultés qui finissent par entraver le développement du régime. Cependant, si l'on y regarde de près, les choses ne sont pas si simples. Marx a très bien montré que l'exploitation des travailleurs a pour cause principale non pas un désir de luxe et de jouissance de la part des capitalistes, mais la nécessité pour chaque entreprise de dépasser ses concurrentes afin d'être plus forte qu'elles. Or ce n'est pas seulement une entreprise, mais n'importe quelle collectivité travailleuse qui a besoin de restreindre la consommation de ses membres pour consacrer le plus d'efforts possibles à se forger des armes contre les collectivités rivales. Du moins il en est ainsi dès qu'il y a rivalité. Aussi longtemps qu'il y aura sur terre une lutte pour la puissance, et que le facteur décisif de victoire sera la production industrielle, ceux qui mèneront cette lutte exploiteront le plus qu'ils pourront les travailleurs de l'industrie. Marx admettait, il est vrai, que toute espèce de rivalité et de lutte pour la puissance disparaîtrait le jour où le socialisme serait établi dans tous les pays industriels. Mais dès lors le socialisme ne se définit plus par l'opération relativement simple de supprimer la compétition entre entreprises capitalistes ; il s'agit de supprimer toute espèce de compétition entre collectivités quelles qu'elles soient, et c'est un tout autre problème. Le malheur, c'est que, comme Marx l'avait reconnu, la révolution ne peut pas éclater partout à la fois. Quand elle se fait dans un pays, elle ne supprime pas pour ce pays, mais au contraire accentue la nécessité d'opprimer et d'exploiter les masses travailleuses, pour ne pas être plus faible que les autres nations. C'est de quoi l'histoire de la révolution russe fournit un exemple douloureux. D'ailleurs quand même la révolution éclaterait à la fois dans beaucoup de pays, est-on bien sûr que ces pays cesseraient pour autant d'être adversaires et rivaux ?\par
Pages 60-61, - On a l'habitude de considérer surtout l'oppression capitaliste du point de vue de l'argent, point de vue qui, dans notre société, domine la pensée même des réfractaires. Si l'on se contente de dénoncer le profit capitaliste comme un vol de ce qui devrait revenir aux masses travailleuses, il est facile d'expliquer que le profit est lié à la concurrence, à la propriété privée, et que la propriété collective des moyens de production permettrait de restituer à l'ensemble des travailleurs la valeur totale de leur travail. C'est là un raisonnement qui persuade facilement les masses. Chaque ouvrier a l'impression qu'il travaille « pour le patron », et se représente sans peine un état de choses où lui et l'ensemble de ses camarades posséderaient l'ensemble des usines, et travailleraient pour eux-mêmes. Mais examinées plus attentivement, les choses ne sont pas si simples. Marx a très bien montré pourquoi les travailleurs sont sacrifiés au profit ; ce n'est pas parce que les patrons ont besoin de jouissances et de luxe, c'est parce que chaque entreprise, aiguillonnée par la concurrence, doit se développer le plus possible de peur d'être écrasée par ses rivales ; le profit est pour elle une arme, et tout ce que reçoivent les hommes qui la font fonctionner l'affaiblit. Or toute collectivité travailleuse, quelle qu'elle soit, qu'il s'agisse d'une entreprise, d'une nation ou de toute autre chose, a besoin, si elle est entourée de collectivités adversaires ou rivales, de réduire la consommation de ses membres pour consacrer le plus d'efforts possibles à se forger des armes. Ce n'est pas seulement la concurrence capitaliste entre entreprises industrielles ou commerciales qu'il faudrait abolir pour supprimer l'exploitation ; il faudrait qu'aucune concurrence économique ou militaire ne dresse plus les unes contre les autres des collectivités dont chacune craint d'être asservie.\par
{\itshape Pages 97-99}. - ... Aristote l'admettait quand il posait Pour Condition à l'émancipation de tous les hommes l'apparition d'« esclaves mécaniques » qui assumeraient les travaux indispensables ; et c'est, en somme, cette vue d'Aristote qui sert de base à la conception marxiste de la révolution. Cette vue serait juste si les hommes étaient conduits par le désir du bien-être, si les exigences insensées de la lutte pour le pouvoir laissaient seulement le loisir de songer au bien-être. L'élévation du rendement de l'effort humain ne peut servir à grand-chose tant que cet effort s'accomplira selon des procédés qui donnent à quelques hommes un pouvoir quasi discrétionnaire sur les masses ; car, comme ces privilégiés seront sans cesse dans la nécessité de tout mettre en oeuvre pour maintenir ou étendre leurs privilèges, les fatigues et les privations devenues inutiles dans la lutte contre la nature subsisteront du fait de la guerre entre les hommes. Dès que la société est divisée entre hommes qui commandent et hommes qui obéissent, le facteur décisif de l'existence sociale est la lutte pour le pouvoir, et la lutte pour la subsistance n'a d'importance qu'en tant qu'elle constitue un élément de la première. Il est vrai, comme l'a vu Marx, que les structures sociales sont principalement déterminées par les rapports entre l'homme et la nature établis par la production ; mais ce n'est pleinement vrai que si l’on considère ces rapports avant tout en fonction du problème du pouvoir, le problème de la subsistance étant rejeté au second plan. Cet ordre semble absurde ; mais il ne fait que refléter l'absurdité fondamentale de notre vie même.\par
La lutte pour le pouvoir est déterminée, quant à ses procédés et quant à ses perspectives, par le mode de production ; mais elle réagit à son tour sur le mode de production ; la transformation qu'elle y détermine la transforme à son tour ; et ainsi de suite indéfiniment. Le jeu de ces actions et de ces réactions est d'une complication extrême, et on ne peut guère espérer que l'analyse théorique arrive à le serrer de bien près ; il faut essayer d'en reconstituer les grandes lignes par un schéma abstrait, comme l'astronome tente de reconstituer nos cieux changeants en combinant des mouvements d'une régularité géométrique. Tout d'abord, des conditions objectives déterminées assignent à l'oppression des limites infranchissables au delà desquelles le pouvoir se détruit lui-même ; le chef succombe dès qu'il croit pouvoir étendre son autorité au delà de la force dont il dispose. Ainsi, dans les sociétés où la production est principalement agricole et où le principal mode d'exploitation est le pillage à main armée, l'exploitation est limitée d'abord par la nécessité de laisser aux producteurs tout au moins la possibilité de subsister.
\subsubsection[C. VARIANTE]{C. VARIANTE}
\noindent {\itshape Dans un des manuscrits des} Réflexions sur les causes de la liberté et de l'oppression sociale, le {\itshape paragraphe qui commence} par peut-être cependant peut-on donner un sens, {\itshape et qui s'achève par} cesser de s'en croire complice du fait qu'on ne fait rien d'efficace pour l'empêcher {\itshape (pages 79-80), est remplacé par celui-ci} :\par
Pour rendre moins illusoires les aspirations à la liberté, il t d'abord distinguer entre la contrainte sociale, borne inévitable posée par l'existence même de la société aux caprices et fantaisies de l'individu, et l'oppression, qui est un abus de domination faisant peser jusqu'à l'écrasement physique et moral la pression de ceux qui commandent sur ceux qui exécutent ; il faut d'autre part concevoir l'évanouissement de l'oppression comme une simple limite analogue à celles qui permettent l'application de la mathématique à la physique. Mais encore est-il douteux, tant qu'on ne l'a pas établi, que la suppression de l'oppression soit même concevable à titre de limite. Étant bien entendu qu'on ne peut espérer faire diminuer jusqu'à l'évanouissement le double poids sur l'homme de la nature et de la société, le problème qui se pose bien clairement est celui de savoir si l'on peut concevoir à titre de limite une organisation de la société, et particulièrement de la production, qui ne comporte pas l'écrasement physique et moral de ceux qui obéissent.
\subsection[FRAGMENT III]{FRAGMENT III}

\begin{center}
VARIANTE\end{center}
\noindent {\itshape Pages 172-173.} - Un ouvrier rapporte arbitrairement au patron toutes les souffrances qu'il subit dans l'usine, sans se demander si dans tout autre système de propriété la direction de l'entreprise ne lui infligerait pas encore une partie des mêmes souffrances, ou bien peut-être des souffrances identiques, ou peut-être même des souffrances accrues ; il ne se demande pas non plus quelle part de ces souffrances on pourrait supprimer, en en éliminant les causes, sans toucher au système de propriété actuel. Pour lui, la lutte « contre le patron » se confond avec la protestation irrépressible de l'être humain écrasé par une vie trop dure. Le patron, de son côté, se préoccupe avec raison de son autorité. Seulement le rôle de l'autorité patronale consiste exclusivement à indiquer les fabrications, à coordonner au mieux les travaux partiels, à contrôler, en recourant à une certaine contrainte, la bonne exécution du travail ; tout régime de l'entreprise, quel qu'il soit, où cette coordination et ce contrôle peuvent être convenablement assurés accorde suffisamment à l'autorité patronale. Pour le patron cependant le sentiment qu'il a de son autorité dépend surtout d'une certaine atmosphère de soumission et de respect qui a peu de rapport avec la bonne exécution du travail ; d'autre part il attribue toujours les velléités de révolte à quelques individus, alors que la révolte, qu'elle soit bruyante ou muette, refoulée ou exprimée, est inséparable de toutes conditions d'existence physiquement ou moralement accablantes. Si pour l'ouvrier la lutte « contre le patron » se confond avec le sentiment de la dignité, pour le patron la lutte contre les « meneurs », contre les « agitateurs » se confond avec le souci de sa fonction et la conscience professionnelle ; des deux côtés il s'agit d'efforts à vide, et qui par suite ne peuvent se renfermer dans aucune limite raisonnable.\par
Alors qu'on constate que les grèves qui se déroulent autour de revendications déterminées aboutissent sans trop de mal à un arrangement, on a vu des grèves qui ressemblaient à des guerres en ce sens que ni d'un côté ni de l'autre la lutte n'avait d'objectif ; des grèves où l'on ne pouvait apercevoir rien de réel ni de tangible, rien, sinon l'arrêt de la production, la détérioration des machines, la misère, la faim, les larmes des femmes, la sous-alimentation des enfants ; et l'acharnement de part et d'autre était tel qu'elles donnaient l'impression de ne jamais devoir finir. Dans de pareils événements, la guerre civile existe déjà en germe.\par
Somme toute, l'histoire humaine apparaît comme un tissu d'absurdités qui non seulement font mourir, mais, ce qui est infiniment plus grave, font oublier la valeur de la vie. Tout se passe comme si une fatalité mauvaise rendait les hommes fous. On se dit pourtant que le rôle joué par ces absurdités doit avoir une cause, et effectivement il a une cause. Il y a dans la vie humaine une absurdité radicale, essentielle, à laquelle on n'aperçoit aucun remède : c'est la nature du pouvoir. La nécessité d'un certain pouvoir est bien réelle, parce que l'ordre est indispensable à l'existence, mais l'attribution du pouvoir est à peu près arbitraire, parce que les hommes sont semblables ou peu s'en faut, et la stabilité du pouvoir repose ainsi essentiellement sur le prestige, autrement dit sur l'imagination. Si la raison est ce qui mesure, comme l'expliquait Platon, l'imagination, elle, est étrangère à toute mesure. Traduites dans le langage du pouvoir, toutes les absurdités énumérées ici semblent se transformer en vérités d'évidence. Il était bien malheureux que Pâris eût enlevé Hélène, mais du moment qu'il l'avait enlevée, les Grecs pouvaient-ils supporter cette injure sans donner aux Troyens l'impression qu'ils pouvaient tout se permettre en Grèce, sans les provoquer à venir ravager le pays ? Les Troyens, de leur côté, pouvaient-ils rendre Hélène sans inspirer aux Grecs l'envie de venir piller une ville qui donnait une telle preuve de faiblesse ?
 


% at least one empty page at end (for booklet couv)
\ifbooklet
  \newpage\null\thispagestyle{empty}\newpage
\fi

\ifdev % autotext in dev mode
\fontname\font — \textsc{Les règles du jeu}\par
(\hyperref[utopie]{\underline{Lien}})\par
\noindent \initialiv{A}{lors là}\blindtext\par
\noindent \initialiv{À}{ la bonheur des dames}\blindtext\par
\noindent \initialiv{É}{tonnez-le}\blindtext\par
\noindent \initialiv{Q}{ualitativement}\blindtext\par
\noindent \initialiv{V}{aloriser}\blindtext\par
\Blindtext
\phantomsection
\label{utopie}
\Blinddocument
\fi
\end{document}
