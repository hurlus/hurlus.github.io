%%%%%%%%%%%%%%%%%%%%%%%%%%%%%%%%%
% LaTeX model https://hurlus.fr %
%%%%%%%%%%%%%%%%%%%%%%%%%%%%%%%%%

% Needed before document class
\RequirePackage{pdftexcmds} % needed for tests expressions
\RequirePackage{fix-cm} % correct units

% Define mode
\def\mode{a4}

\newif\ifaiv % a4
\newif\ifav % a5
\newif\ifbooklet % booklet
\newif\ifcover % cover for booklet

\ifnum \strcmp{\mode}{cover}=0
  \covertrue
\else\ifnum \strcmp{\mode}{booklet}=0
  \booklettrue
\else\ifnum \strcmp{\mode}{a5}=0
  \avtrue
\else
  \aivtrue
\fi\fi\fi

\ifbooklet % do not enclose with {}
  \documentclass[french,twoside]{book} % ,notitlepage
  \usepackage[%
    papersize={105mm, 297mm},
    inner=12mm,
    outer=12mm,
    top=20mm,
    bottom=15mm,
    marginparsep=0pt,
  ]{geometry}
  \usepackage[fontsize=9.5pt]{scrextend} % for Roboto
\else\ifav
  \documentclass[french,twoside]{book} % ,notitlepage
  \usepackage[%
    a5paper,
    inner=25mm,
    outer=15mm,
    top=15mm,
    bottom=15mm,
    marginparsep=0pt,
  ]{geometry}
  \usepackage[fontsize=12pt]{scrextend}
\else% A4 2 cols
  \documentclass[twocolumn]{report}
  \usepackage[%
    a4paper,
    inner=15mm,
    outer=10mm,
    top=25mm,
    bottom=18mm,
    marginparsep=0pt,
  ]{geometry}
  \setlength{\columnsep}{20mm}
  \usepackage[fontsize=9.5pt]{scrextend}
\fi\fi

%%%%%%%%%%%%%%
% Alignments %
%%%%%%%%%%%%%%
% before teinte macros

\setlength{\arrayrulewidth}{0.2pt}
\setlength{\columnseprule}{\arrayrulewidth} % twocol
\setlength{\parskip}{0pt} % classical para with no margin
\setlength{\parindent}{1.5em}

%%%%%%%%%%
% Colors %
%%%%%%%%%%
% before Teinte macros

\usepackage[dvipsnames]{xcolor}
\definecolor{rubric}{HTML}{0c71c3} % the tonic
\def\columnseprulecolor{\color{rubric}}
\colorlet{borderline}{rubric!30!} % definecolor need exact code
\definecolor{shadecolor}{gray}{0.95}
\definecolor{bghi}{gray}{0.5}

%%%%%%%%%%%%%%%%%
% Teinte macros %
%%%%%%%%%%%%%%%%%
%%%%%%%%%%%%%%%%%%%%%%%%%%%%%%%%%%%%%%%%%%%%%%%%%%%
% <TEI> generic (LaTeX names generated by Teinte) %
%%%%%%%%%%%%%%%%%%%%%%%%%%%%%%%%%%%%%%%%%%%%%%%%%%%
% This template is inserted in a specific design
% It is XeLaTeX and otf fonts

\makeatletter % <@@@


\usepackage{blindtext} % generate text for testing
\usepackage{contour} % rounding words
\usepackage[nodayofweek]{datetime}
\usepackage{DejaVuSans} % font for symbols
\usepackage{enumitem} % <list>
\usepackage{etoolbox} % patch commands
\usepackage{fancyvrb}
\usepackage{fancyhdr}
\usepackage{fontspec} % XeLaTeX mandatory for fonts
\usepackage{footnote} % used to capture notes in minipage (ex: quote)
\usepackage{framed} % bordering correct with footnote hack
\usepackage{graphicx}
\usepackage{lettrine} % drop caps
\usepackage{lipsum} % generate text for testing
\usepackage[framemethod=tikz,]{mdframed} % maybe used for frame with footnotes inside
\usepackage{pdftexcmds} % needed for tests expressions
\usepackage{polyglossia} % non-break space french punct, bug Warning: "Failed to patch part"
\usepackage[%
  indentfirst=false,
  vskip=1em,
  noorphanfirst=true,
  noorphanafter=true,
  leftmargin=\parindent,
  rightmargin=0pt,
]{quoting}
\usepackage{ragged2e}
\usepackage{setspace}
\usepackage{tabularx} % <table>
\usepackage[explicit]{titlesec} % wear titles, !NO implicit
\usepackage{tikz} % ornaments
\usepackage{tocloft} % styling tocs
\usepackage[fit]{truncate} % used im runing titles
\usepackage{unicode-math}
\usepackage[normalem]{ulem} % breakable \uline, normalem is absolutely necessary to keep \emph
\usepackage{verse} % <l>
\usepackage{xcolor} % named colors
\usepackage{xparse} % @ifundefined
\XeTeXdefaultencoding "iso-8859-1" % bad encoding of xstring
\usepackage{xstring} % string tests
\XeTeXdefaultencoding "utf-8"
\PassOptionsToPackage{hyphens}{url} % before hyperref, which load url package
\usepackage{hyperref} % supposed to be the last one, :o) except for the ones to follow
\urlstyle{same} % after hyperref

% TOTEST
% \usepackage{hypcap} % links in caption ?
% \usepackage{marginnote}
% TESTED
% \usepackage{background} % doesn’t work with xetek
% \usepackage{bookmark} % prefers the hyperref hack \phantomsection
% \usepackage[color, leftbars]{changebar} % 2 cols doc, impossible to keep bar left
% \usepackage[utf8x]{inputenc} % inputenc package ignored with utf8 based engines
% \usepackage[sfdefault,medium]{inter} % no small caps
% \usepackage{firamath} % choose firasans instead, firamath unavailable in Ubuntu 21-04
% \usepackage{flushend} % bad for last notes, supposed flush end of columns
% \usepackage[stable]{footmisc} % BAD for complex notes https://texfaq.org/FAQ-ftnsect
% \usepackage{helvet} % not for XeLaTeX
% \usepackage{multicol} % not compatible with too much packages (longtable, framed, memoir…)
% \usepackage[default,oldstyle,scale=0.95]{opensans} % no small caps
% \usepackage{sectsty} % \chapterfont OBSOLETE
% \usepackage{soul} % \ul for underline, OBSOLETE with XeTeX
% \usepackage[breakable]{tcolorbox} % text styling gone, footnote hack not kept with breakable



% Metadata inserted by a program, from the TEI source, for title page and runing heads
\title{\textbf{ La condition ouvrière }}
\date{1934/1942}
\author{Simone Weil}
\def\elbibl{Simone Weil. 1934/1942. \emph{La condition ouvrière}}
\def\elsource{Simone Weil. \emph{La condition ouvrière} (Recueil de textes écrits entre 1934 et 1942).}

% Default metas
\newcommand{\colorprovide}[2]{\@ifundefinedcolor{#1}{\colorlet{#1}{#2}}{}}
\colorprovide{rubric}{red}
\colorprovide{silver}{Gray}
\@ifundefined{syms}{\newfontfamily\syms{DejaVu Sans}}{}
\newif\ifdev
\@ifundefined{elbibl}{% No meta defined, maybe dev mode
  \newcommand{\elbibl}{Titre court ?}
  \newcommand{\elbook}{Titre du livre source ?}
  \newcommand{\elabstract}{Résumé\par}
  \newcommand{\elurl}{http://oeuvres.github.io/elbook/2}
  \author{Éric Lœchien}
  \title{Un titre de test assez long pour vérifier le comportement d’une maquette}
  \date{1566}
  \devtrue
}{}
\let\eltitle\@title
\let\elauthor\@author
\let\eldate\@date


\defaultfontfeatures{
  % Mapping=tex-text, % no effect seen
  Scale=MatchLowercase,
  Ligatures={TeX,Common},
}

\@ifundefined{\columnseprulecolor}{%
    \patchcmd\@outputdblcol{% find
      \normalcolor\vrule
    }{% and replace by
      \columnseprulecolor\vrule
    }{% success
    }{% failure
      \@latex@warning{Patching \string\@outputdblcol\space failed}%
    }
}{}

\hypersetup{
  % pdftex, % no effect
  pdftitle={\elbibl},
  % pdfauthor={Your name here},
  % pdfsubject={Your subject here},
  % pdfkeywords={keyword1, keyword2},
  bookmarksnumbered=true,
  bookmarksopen=true,
  bookmarksopenlevel=1,
  pdfstartview=Fit,
  breaklinks=true, % avoid long links
  pdfpagemode=UseOutlines,    % pdf toc
  hyperfootnotes=true,
  colorlinks=false,
  pdfborder=0 0 0,
  % pdfpagelayout=TwoPageRight,
  % linktocpage=true, % NO, toc, link only on page no
}


% generic typo commands
\newcommand{\astermono}{\medskip\centerline{\color{rubric}\large\selectfont{\syms ✻}}\medskip\par}%
\newcommand{\astertri}{\medskip\par\centerline{\color{rubric}\large\selectfont{\syms ✻\,✻\,✻}}\medskip\par}%
\newcommand{\asterism}{\bigskip\par\noindent\parbox{\linewidth}{\centering\color{rubric}\large{\syms ✻}\\{\syms ✻}\hskip 0.75em{\syms ✻}}\bigskip\par}%

% lists
\newlength{\listmod}
\setlength{\listmod}{\parindent}
\setlist{
  itemindent=!,
  listparindent=\listmod,
  labelsep=0.2\listmod,
  parsep=0pt,
  % topsep=0.2em, % default topsep is best
}
\setlist[itemize]{
  label=—,
  leftmargin=0pt,
  labelindent=1.2em,
  labelwidth=0pt,
}
\setlist[enumerate]{
  label={\bf\color{rubric}\arabic*.},
  labelindent=0.8\listmod,
  leftmargin=\listmod,
  labelwidth=0pt,
}
\newlist{listalpha}{enumerate}{1}
\setlist[listalpha]{
  label={\bf\color{rubric}\alph*.},
  leftmargin=0pt,
  labelindent=0.8\listmod,
  labelwidth=0pt,
}
\newcommand{\listhead}[1]{\hspace{-1\listmod}\emph{#1}}

\renewcommand{\hrulefill}{%
  \leavevmode\leaders\hrule height 0.2pt\hfill\kern\z@}

% General typo
\DeclareTextFontCommand{\textlarge}{\large}
\DeclareTextFontCommand{\textsmall}{\small}


% commands, inlines
\newcommand{\anchor}[1]{\Hy@raisedlink{\hypertarget{#1}{}}} % link to top of an anchor (not baseline)
\newcommand\abbr[1]{#1}
\newcommand{\autour}[1]{\tikz[baseline=(X.base)]\node [draw=rubric,thin,rectangle,inner sep=1.5pt, rounded corners=3pt] (X) {\color{rubric}#1};}
\newcommand\corr[1]{#1}
\newcommand{\ed}[1]{ {\color{silver}\sffamily\footnotesize (#1)} } % <milestone ed="1688"/>
\newcommand\expan[1]{#1}
\newcommand\foreign[1]{\emph{#1}}
\newcommand\gap[1]{#1}
\renewcommand{\LettrineFontHook}{\color{rubric}}
\newcommand{\initial}[2]{\lettrine[lines=2, loversize=0.3, lhang=0.3]{#1}{#2}}
\newcommand{\initialiv}[2]{%
  \let\oldLFH\LettrineFontHook
  % \renewcommand{\LettrineFontHook}{\color{rubric}\ttfamily}
  \IfSubStr{QJ’}{#1}{
    \lettrine[lines=4, lhang=0.2, loversize=-0.1, lraise=0.2]{\smash{#1}}{#2}
  }{\IfSubStr{É}{#1}{
    \lettrine[lines=4, lhang=0.2, loversize=-0, lraise=0]{\smash{#1}}{#2}
  }{\IfSubStr{ÀÂ}{#1}{
    \lettrine[lines=4, lhang=0.2, loversize=-0, lraise=0, slope=0.6em]{\smash{#1}}{#2}
  }{\IfSubStr{A}{#1}{
    \lettrine[lines=4, lhang=0.2, loversize=0.2, slope=0.6em]{\smash{#1}}{#2}
  }{\IfSubStr{V}{#1}{
    \lettrine[lines=4, lhang=0.2, loversize=0.2, slope=-0.5em]{\smash{#1}}{#2}
  }{
    \lettrine[lines=4, lhang=0.2, loversize=0.2]{\smash{#1}}{#2}
  }}}}}
  \let\LettrineFontHook\oldLFH
}
\newcommand{\labelchar}[1]{\textbf{\color{rubric} #1}}
\newcommand{\milestone}[1]{\autour{\footnotesize\color{rubric} #1}} % <milestone n="4"/>
\newcommand\name[1]{#1}
\newcommand\orig[1]{#1}
\newcommand\orgName[1]{#1}
\newcommand\persName[1]{#1}
\newcommand\placeName[1]{#1}
\newcommand{\pn}[1]{\IfSubStr{-—–¶}{#1}% <p n="3"/>
  {\noindent{\bfseries\color{rubric}   ¶  }}
  {{\footnotesize\autour{ #1}  }}}
\newcommand\reg{}
% \newcommand\ref{} % already defined
\newcommand\sic[1]{#1}
\newcommand\surname[1]{\textsc{#1}}
\newcommand\term[1]{\textbf{#1}}

\def\mednobreak{\ifdim\lastskip<\medskipamount
  \removelastskip\nopagebreak\medskip\fi}
\def\bignobreak{\ifdim\lastskip<\bigskipamount
  \removelastskip\nopagebreak\bigskip\fi}

% commands, blocks
\newcommand{\byline}[1]{\bigskip{\RaggedLeft{#1}\par}\bigskip}
\newcommand{\bibl}[1]{{\RaggedLeft{#1}\par\bigskip}}
\newcommand{\biblitem}[1]{{\noindent\hangindent=\parindent   #1\par}}
\newcommand{\dateline}[1]{\medskip{\RaggedLeft{#1}\par}\bigskip}
\newcommand{\labelblock}[1]{\medbreak{\noindent\color{rubric}\bfseries #1}\par\mednobreak}
\newcommand{\salute}[1]{\bigbreak{#1}\par\medbreak}
\newcommand{\signed}[1]{\bigbreak\filbreak{\raggedleft #1\par}\medskip}

% environments for blocks (some may become commands)
\newenvironment{borderbox}{}{} % framing content
\newenvironment{citbibl}{\ifvmode\hfill\fi}{\ifvmode\par\fi }
\newenvironment{docAuthor}{\ifvmode\vskip4pt\fontsize{16pt}{18pt}\selectfont\fi\itshape}{\ifvmode\par\fi }
\newenvironment{docDate}{}{\ifvmode\par\fi }
\newenvironment{docImprint}{\vskip6pt}{\ifvmode\par\fi }
\newenvironment{docTitle}{\vskip6pt\bfseries\fontsize{18pt}{22pt}\selectfont}{\par }
\newenvironment{msHead}{\vskip6pt}{\par}
\newenvironment{msItem}{\vskip6pt}{\par}
\newenvironment{titlePart}{}{\par }


% environments for block containers
\newenvironment{argument}{\itshape\parindent0pt}{\vskip1.5em}
\newenvironment{biblfree}{}{\ifvmode\par\fi }
\newenvironment{bibitemlist}[1]{%
  \list{\@biblabel{\@arabic\c@enumiv}}%
  {%
    \settowidth\labelwidth{\@biblabel{#1}}%
    \leftmargin\labelwidth
    \advance\leftmargin\labelsep
    \@openbib@code
    \usecounter{enumiv}%
    \let\p@enumiv\@empty
    \renewcommand\theenumiv{\@arabic\c@enumiv}%
  }
  \sloppy
  \clubpenalty4000
  \@clubpenalty \clubpenalty
  \widowpenalty4000%
  \sfcode`\.\@m
}%
{\def\@noitemerr
  {\@latex@warning{Empty `bibitemlist' environment}}%
\endlist}
\newenvironment{quoteblock}% may be used for ornaments
  {\begin{quoting}}
  {\end{quoting}}

% table () is preceded and finished by custom command
\newcommand{\tableopen}[1]{%
  \ifnum\strcmp{#1}{wide}=0{%
    \begin{center}
  }
  \else\ifnum\strcmp{#1}{long}=0{%
    \begin{center}
  }
  \else{%
    \begin{center}
  }
  \fi\fi
}
\newcommand{\tableclose}[1]{%
  \ifnum\strcmp{#1}{wide}=0{%
    \end{center}
  }
  \else\ifnum\strcmp{#1}{long}=0{%
    \end{center}
  }
  \else{%
    \end{center}
  }
  \fi\fi
}


% text structure
\newcommand\chapteropen{} % before chapter title
\newcommand\chaptercont{} % after title, argument, epigraph…
\newcommand\chapterclose{} % maybe useful for multicol settings
\setcounter{secnumdepth}{-2} % no counters for hierarchy titles
\setcounter{tocdepth}{5} % deep toc
\markright{\@title} % ???
\markboth{\@title}{\@author} % ???
\renewcommand\tableofcontents{\@starttoc{toc}}
% toclof format
% \renewcommand{\@tocrmarg}{0.1em} % Useless command?
% \renewcommand{\@pnumwidth}{0.5em} % {1.75em}
\renewcommand{\@cftmaketoctitle}{}
\setlength{\cftbeforesecskip}{\z@ \@plus.2\p@}
\renewcommand{\cftchapfont}{}
\renewcommand{\cftchapdotsep}{\cftdotsep}
\renewcommand{\cftchapleader}{\normalfont\cftdotfill{\cftchapdotsep}}
\renewcommand{\cftchappagefont}{\bfseries}
\setlength{\cftbeforechapskip}{0em \@plus\p@}
% \renewcommand{\cftsecfont}{\small\relax}
\renewcommand{\cftsecpagefont}{\normalfont}
% \renewcommand{\cftsubsecfont}{\small\relax}
\renewcommand{\cftsecdotsep}{\cftdotsep}
\renewcommand{\cftsecpagefont}{\normalfont}
\renewcommand{\cftsecleader}{\normalfont\cftdotfill{\cftsecdotsep}}
\setlength{\cftsecindent}{1em}
\setlength{\cftsubsecindent}{2em}
\setlength{\cftsubsubsecindent}{3em}
\setlength{\cftchapnumwidth}{1em}
\setlength{\cftsecnumwidth}{1em}
\setlength{\cftsubsecnumwidth}{1em}
\setlength{\cftsubsubsecnumwidth}{1em}

% footnotes
\newif\ifheading
\newcommand*{\fnmarkscale}{\ifheading 0.70 \else 1 \fi}
\renewcommand\footnoterule{\vspace*{0.3cm}\hrule height \arrayrulewidth width 3cm \vspace*{0.3cm}}
\setlength\footnotesep{1.5\footnotesep} % footnote separator
\renewcommand\@makefntext[1]{\parindent 1.5em \noindent \hb@xt@1.8em{\hss{\normalfont\@thefnmark . }}#1} % no superscipt in foot


% orphans and widows
\clubpenalty=9996
\widowpenalty=9999
\brokenpenalty=4991
\predisplaypenalty=10000
\postdisplaypenalty=1549
\displaywidowpenalty=1602
\hyphenpenalty=400
% Copied from Rahtz but not understood
\def\@pnumwidth{1.55em}
\def\@tocrmarg {2.55em}
\def\@dotsep{4.5}
\emergencystretch 3em
\hbadness=4000
\pretolerance=750
\tolerance=2000
\vbadness=4000
\def\Gin@extensions{.pdf,.png,.jpg,.mps,.tif}
% \renewcommand{\@cite}[1]{#1} % biblio

\makeatother % /@@@>
%%%%%%%%%%%%%%
% </TEI> end %
%%%%%%%%%%%%%%


%%%%%%%%%%%%%
% footnotes %
%%%%%%%%%%%%%
\renewcommand{\thefootnote}{\bfseries\textcolor{rubric}{\arabic{footnote}}} % color for footnote marks

%%%%%%%%%
% Fonts %
%%%%%%%%%
\usepackage[]{roboto} % SmallCaps, Regular is a bit bold
% \linespread{0.90} % too compact, keep font natural
\newfontfamily\fontrun[]{Roboto Condensed Light} % condensed runing heads
\ifav
  \setmainfont[
    ItalicFont={Roboto Light Italic},
  ]{Roboto}
\else\ifbooklet
  \setmainfont[
    ItalicFont={Roboto Light Italic},
  ]{Roboto}
\else
\setmainfont[
  ItalicFont={Roboto Italic},
]{Roboto Light}
\fi\fi
\renewcommand{\LettrineFontHook}{\bfseries\color{rubric}}
% \renewenvironment{labelblock}{\begin{center}\bfseries\color{rubric}}{\end{center}}

%%%%%%%%
% MISC %
%%%%%%%%

\setdefaultlanguage[frenchpart=false]{french} % bug on part


\newenvironment{quotebar}{%
    \def\FrameCommand{{\color{rubric!10!}\vrule width 0.5em} \hspace{0.9em}}%
    \def\OuterFrameSep{\itemsep} % séparateur vertical
    \MakeFramed {\advance\hsize-\width \FrameRestore}
  }%
  {%
    \endMakeFramed
  }
\renewenvironment{quoteblock}% may be used for ornaments
  {%
    \savenotes
    \setstretch{0.9}
    \normalfont
    \begin{quotebar}
  }
  {%
    \end{quotebar}
    \spewnotes
  }


\renewcommand{\headrulewidth}{\arrayrulewidth}
\renewcommand{\headrule}{{\color{rubric}\hrule}}

% delicate tuning, image has produce line-height problems in title on 2 lines
\titleformat{name=\chapter} % command
  [display] % shape
  {\vspace{1.5em}\centering} % format
  {} % label
  {0pt} % separator between n
  {}
[{\color{rubric}\huge\textbf{#1}}\bigskip] % after code
% \titlespacing{command}{left spacing}{before spacing}{after spacing}[right]
\titlespacing*{\chapter}{0pt}{-2em}{0pt}[0pt]

\titleformat{name=\section}
  [block]{}{}{}{}
  [\vbox{\color{rubric}\large\raggedleft\textbf{#1}}]
\titlespacing{\section}{0pt}{0pt plus 4pt minus 2pt}{\baselineskip}

\titleformat{name=\subsection}
  [block]
  {}
  {} % \thesection
  {} % separator \arrayrulewidth
  {}
[\vbox{\large\textbf{#1}}]
% \titlespacing{\subsection}{0pt}{0pt plus 4pt minus 2pt}{\baselineskip}

\ifaiv
  \fancypagestyle{main}{%
    \fancyhf{}
    \setlength{\headheight}{1.5em}
    \fancyhead{} % reset head
    \fancyfoot{} % reset foot
    \fancyhead[L]{\truncate{0.45\headwidth}{\fontrun\elbibl}} % book ref
    \fancyhead[R]{\truncate{0.45\headwidth}{ \fontrun\nouppercase\leftmark}} % Chapter title
    \fancyhead[C]{\thepage}
  }
  \fancypagestyle{plain}{% apply to chapter
    \fancyhf{}% clear all header and footer fields
    \setlength{\headheight}{1.5em}
    \fancyhead[L]{\truncate{0.9\headwidth}{\fontrun\elbibl}}
    \fancyhead[R]{\thepage}
  }
\else
  \fancypagestyle{main}{%
    \fancyhf{}
    \setlength{\headheight}{1.5em}
    \fancyhead{} % reset head
    \fancyfoot{} % reset foot
    \fancyhead[RE]{\truncate{0.9\headwidth}{\fontrun\elbibl}} % book ref
    \fancyhead[LO]{\truncate{0.9\headwidth}{\fontrun\nouppercase\leftmark}} % Chapter title, \nouppercase needed
    \fancyhead[RO,LE]{\thepage}
  }
  \fancypagestyle{plain}{% apply to chapter
    \fancyhf{}% clear all header and footer fields
    \setlength{\headheight}{1.5em}
    \fancyhead[L]{\truncate{0.9\headwidth}{\fontrun\elbibl}}
    \fancyhead[R]{\thepage}
  }
\fi

\ifav % a5 only
  \titleclass{\section}{top}
\fi

\newcommand\chapo{{%
  \vspace*{-3em}
  \centering % no vskip ()
  {\Large\addfontfeature{LetterSpace=25}\bfseries{\elauthor}}\par
  \smallskip
  {\large\eldate}\par
  \bigskip
  {\Large\selectfont{\eltitle}}\par
  \bigskip
  {\color{rubric}\hline\par}
  \bigskip
  {\Large LIVRE LIBRE À PRIX LIBRE, DEMANDEZ AU COMPTOIR\par}
  \centerline{\small\color{rubric} {hurlus.fr, tiré le \today}}\par
  \bigskip
}}


\begin{document}
\pagestyle{empty}
\ifbooklet{
  \thispagestyle{empty}
  \centering
  {\LARGE\bfseries{\elauthor}}\par
  \bigskip
  {\Large\eldate}\par
  \bigskip
  \bigskip
  {\LARGE\selectfont{\eltitle}}\par
  \vfill\null
  {\color{rubric}\setlength{\arrayrulewidth}{2pt}\hline\par}
  \vfill\null
  {\Large LIVRE LIBRE À PRIX LIBRE, DEMANDEZ AU COMPTOIR\par}
  \centerline{\small{hurlus.fr, tiré le \today}}\par
  \newpage\null\thispagestyle{empty}\newpage
  \addtocounter{page}{-2}
}\fi

\thispagestyle{empty}
\ifaiv
  \twocolumn[\chapo]
\else
  \chapo
\fi
{\it\elabstract}
\bigskip
\makeatletter\@starttoc{toc}\makeatother % toc without new page
\bigskip

\pagestyle{main} % after style

  \section[Trois lettres, à Mme Albertine Thévenon, (1934-1935)]{Trois lettres \\
à M\textsuperscript{me} Albertine Thévenon \\
(1934-1935)}\renewcommand{\leftmark}{Trois lettres \\
à M\textsuperscript{me} Albertine Thévenon \\
(1934-1935)}

\subsection[I]{I}

\salute{Chère Albertine,}
\noindent Je profite des loisirs forcés que m'impose une légère maladie (début d'otite – ça n'est rien) pour causer un peu avec toi. Sans ça, les semaines de travail, chaque effort en plus de ceux qui me sont imposés me coûte. Mais ce n'est pas seulement ça qui me retient : c'est la multitude des choses à dire et l'impossibilité d'exprimer l'essentiel. Peut-être, plus tard, les mots justes me viendront-ils : maintenant, il me semble qu'il me faudrait pour traduire ce qui importe un autre langage. Cette expérience, qui correspond par bien des côtés à ce que j'attendais, en diffère quand même par un abîme : c'est la réalité, non plus l'imagination. Elle a changé pour moi non pas telle ou telle de mes idées (beaucoup ont été au contraire confirmées), mais infiniment plus, toute ma perspective sur les choses, le sentiment même que j'ai de la vie. Je connaîtrai encore la joie, mais il y a une certaine légèreté de cœur qui me restera, il me semble, toujours impossible. Mais assez là-dessus : on dégrade l'inexprimable à vouloir l'exprimer.\par
En ce qui concerne les choses exprimables, j'ai pas mal appris sur l'organisation d'une entreprise. C'est inhumain : travail parcellaire – à la tâche – organisation purement bureaucratique des rapports entre les divers éléments de l'entreprise, les différentes opérations du travail. L'attention, privée d'objets dignes d'elle, est par contre contrainte à se concentrer seconde par seconde sur un problème mesquin, toujours le même, avec des variantes : faire 50 pièces en 5 minutes au lieu de 6, ou quoi que ce soit de cet ordre. Grâce au ciel, il y a des tours de main à acquérir, ce qui donne de temps à autre de l'intérêt à cette recherche de la vitesse. Mais ce que je me demande, c'est comment tout cela peut devenir humain : car si le travail parcellaire n'était pas à la tâche, l'ennui qui s'en dégage annihilerait l'attention, occasionnerait une lenteur considérable et des tas de loupés. Et si le travail n'était pas parcellaire... Mais je n'ai pas le temps de développer tout cela par lettre. Seulement, quand je pense que les grands chefs bolcheviks prétendaient créer une classe ouvrière libre et qu'aucun d'eux – Trotsky sûrement pas, Lénine je ne crois pas non plus – n'avait sans doute mis le pied dans une usine et par suite n'avait la plus faible idée des conditions réelles qui déterminent la servitude ou la liberté pour les ouvriers – la politique m'apparaît comme une sinistre rigolade.\par
Je dois dire que tout cela concerne le travail non qualifié. Sur le travail qualifié, j'ai encore à peu près tout à apprendre. Ça va venir, j'espère.\par
Pour moi, cette vie est assez dure, à parler franchement. D'autant que les maux de tête n'ont pas eu la complaisance de me quitter pour faciliter l'expérience – et travailler à des machines avec des maux de tête, c'est pénible. C'est seulement le samedi après-midi et le dimanche que je respire, me retrouve moi-même, réacquiers la faculté de rouler dans mon esprit des morceaux d'idées. D'une manière générale, la tentation la plus difficile à repousser, dans une pareille vie, c'est celle de renoncer tout à fait à penser : on sent si bien que c'est l'unique moyen de ne plus souffrir ! D'abord de ne plus souffrir moralement. Car la situation même efface automatiquement les sentiments de révolte : faire son travail avec irritation, ce serait le faire mal, et se condamner à crever de faim ; et on n'a personne à qui s'attaquer en dehors du travail lui-même. Les chefs, on ne peut pas se permettre d'être insolent avec eux, et d'ailleurs bien souvent ils n'y donnent même pas lieu. Ainsi il ne reste pas d'autre sentiment possible à l'égard de son propre sort que la tristesse. Alors on est tenté de perdre purement et simplement conscience de tout ce qui n'est pas le train-train vulgaire et quotidien de la vie. Physiquement aussi, sombrer, en dehors des heures de travail, dans une demi-somnolence est une grande tentation. J'ai le plus grand respect pour les ouvriers qui arrivent à se donner une culture. Ils sont le plus souvent costauds, c'est vrai. Quand même, il faut qu'ils aient quelque chose dans le ventre. Aussi est-ce de plus en plus rare, avec les progrès de la rationalisation. Je me demande si cela se voit chez des manœuvres spécialisés.\par
Je tiens le coup, quand même. Et je ne regrette pas une minute de m'être lancée dans cette expérience. Bien au contraire, je m'en félicite infiniment toutes les fois que j'y pense. Mais, chose bizarre, j'y pense rarement. J'ai une faculté d'adaptation presque illimitée, qui me permet d'oublier que je suis un « professeur agrégé » en vadrouille dans la classe ouvrière, de vivre ma vie actuelle comme si j'y étais destinée depuis toujours (et, en un sens, c'est bien vrai) et que cela devait toujours durer, comme si elle m'était imposée par une nécessité inéluctable et non par mon libre choix.\par
Je te promets pourtant que quand je ne tiendrai plus le coup j'irai me reposer quelque part – peut-être chez vous.\par

\begin{center}
………………………………………………………………………………..\end{center}
\noindent Je m'aperçois que je n'ai rien dit des compagnons de travail. Ça sera pour une autre fois. Mais ça aussi, c'est difficile à exprimer... On est gentil, très gentil. Mais de vraie fraternité, je n'en ai presque pas senti. Une exception : le magasinier du magasin des outils, ouvrier qualifié, excellent ouvrier, et que j'appelle à mon secours toutes les fois que je suis réduite au désespoir par un travail que je n'arrive pas à bien faire, parce qu'il est cent fois plus gentil et plus intelligent que les régleurs (lesquels ne sont que des manœuvres spécialisés). Il y a pas mal de jalousie parmi les ouvrières, qui se font en fait concurrence, du fait de l'organisation de l'usine. Je n'en connais que 3 ou 4 pleinement sympathiques. Quant aux ouvriers, quelques-uns semblent très chics. Mais il y en a peu là où je suis, en dehors des régleurs, qui ne sont pas des vrais copains. J'espère changer d'atelier dans quelque temps, pour élargir mon champ d'expérience.\par

\begin{center}
………………………………………………………………………………..\end{center}

\salute{Allons, au revoir. Réponds-moi bientôt.}


\signed{S. W.}
\subsection[II]{II}

\salute{Ma chère Albertine,}
\noindent Je crois sentir que tu as mal interprété mon silence. Tu crois, semble-t-il, que je suis embarrassée pour m'exprimer franchement. Non, nullement ; c'est l'effort d'écrire, simplement, qui était trop lourd. Ce que ta grande lettre a remué en moi, c'est l'envie de te dire que je suis profondément avec toi, que c'est de ton côté que me porte tout mon instinct de fidélité à l'amitié.\par

\begin{center}
………………………………………………………………………………..\end{center}
\noindent Mais avec tout ça je comprends des choses que peut-être tu ne comprends pas, parce que tu es trop différente. Vois-tu, tu vis tellement dans l'instant – et je t'aime pour ça – que tu ne te représentes pas peut-être ce que c'est que de concevoir toute sa vie devant soi, et de prendre la résolution ferme et constante d'en faire quelque chose, de l'orienter d'un bout à l'autre par la volonté et le travail dans un sens déterminé. Quand on est comme ça – moi, je suis comme ça, alors je sais ce que c'est – ce qu'un être humain peut vous faire de pire au monde, c'est de vous infliger des souffrances qui brisent la vitalité et par conséquent la capacité de travail.\par

\begin{center}
………………………………………………………………………………..\end{center}
\noindent Je ne sais que trop (à cause de mes maux de tête) ce que c'est que de savourer ainsi la mort tout vivant ; de voir des années s'étendre devant soi, d'avoir mille fois de quoi les remplir, et de penser que la faiblesse physique forcera à les laisser vides, que les franchir simplement jour par jour sera une tâche écrasante.\par

\begin{center}
………………………………………………………………………………..\end{center}
\noindent J'aurais voulu te parler un peu de moi, je n'en ai plus le temps. J'ai beaucoup souffert de ces mois d'esclavage, mais je ne voudrais pour rien au monde ne pas les avoir traversés. Ils m'ont permis de m'éprouver moi-même et de toucher du doigt tout ce que je n'avais pu qu'imaginer. J'en suis sortie bien différente de ce que j'étais quand j'y suis entrée – physiquement épuisée, mais moralement endurcie (tu comprendras en quel sens je dis ça).\par
Écris-moi à Paris. Je suis nommée à Bourges. C'est loin. On n'aura guère la possibilité de se voir.\par

\begin{center}
………………………………………………………………………………..\end{center}

\salute{Je t'embrasse.}


\signed{SIMONE.}
\subsection[III]{III}

\salute{Chère Albertine,}
\noindent Ça me fait du bien de recevoir un mot de toi. Il y a des choses, il me semble, pour lesquelles on ne se comprend que toi et moi. Tu vis encore ; ça, tu ne peux pas savoir comme j'en suis heureuse…………. Tu méritais bien de te libérer. La vie les vend cher, les progrès qu'elle fait faire. Presque toujours au prix de douleurs intolérables.\par

\begin{center}
………………………………………………………………………………..\end{center}
\noindent Tu sais, j'ai une idée qui me vient juste à l'instant.\par
Je nous vois toutes les deux, pendant les vacances, avec quelques sous en poche, marchant le long des routes, des chemins et des champs, sac au dos. On coucherait des fois dans les granges. Des fois on donnerait un coup de main pour la moisson, en échange de la nourriture………………………….Qu'en dis-tu ?….………...\par

\begin{center}
………………………………………………………………………………..\end{center}
\noindent Ce que tu écris de l'usine m'est allé droit au cœur. C'est ça que je sentais, moi, depuis mon enfance. C'est pour ça qu'il a fallu que je finisse par y aller, et ça me faisait de la peine, avant, que tu ne comprennes pas. Mais une fois dedans, comme c'est autre chose ! Maintenant, c'est comme ceci que je sens la question sociale : une usine, cela doit être ce que tu as senti ce jour-là à Saint-Chamond, ce que j'ai senti si souvent, un endroit où on se heurte durement, douloureusement, mais quand même joyeusement à la vraie vie. Pas cet endroit morne où on ne fait qu'obéir, briser sous la contrainte tout ce qu'on a d'humain, se courber, se laisser abaisser au-dessous de la machine.\par
Une fois j'ai senti pleinement, dans l'usine, ce que j'avais pressenti, comme toi, du dehors. À ma première boîte. Imagine-moi, devant un grand four, qui crache au-dehors des flammes et des souffles embrasés que je reçois en plein visage. Le feu sort de cinq ou six trous qui sont dans le bas du four. Je me mets en plein devant pour enfourner une trentaine de grosses bobines de cuivre qu'une ouvrière italienne, au visage courageux et ouvert, fabrique à côté de moi ; c'est pour les trams et les métros, ces bobines. Je dois faire bien attention qu'aucune des bobines ne tombe dans un des trous, car elle y fondrait ; et pour ça, il faut que je me mette en plein en face du four, et que jamais la douleur des souffles enflammés sur mon visage et du feu sur mes bras (j'en porte encore la marque) ne me fasse faire un faux mouvement. Je baisse le tablier du four ; j'attends quelques minutes ; je relève le tablier et avec un crochet je retire les bobines passées au rouge, en les attirant à moi très vite (sans quoi les dernières retirées commenceraient à fondre), et en faisant bien plus attention encore qu'à aucun moment un faux mouvement n'en envoie une dans un des trous. Et puis ça recommence. En face de moi un soudeur, assis, avec des lunettes bleues et un visage grave, travaille minutieusement ; chaque fois que la douleur me contracte le visage, il m'envoie un sourire triste, plein de sympathie fraternelle, qui me fait un bien indicible. De l'autre côté, une équipe de chaudronniers travaille autour de grandes tables ; travail accompli en équipe, fraternellement, avec soin et sans hâte ; travail très qualifié, où il faut savoir calculer, lire des dessins très compliqués, appliquer des notions de géométrie descriptive. Plus loin, un gars costaud frappe avec une masse sur les barres de fer en faisant un bruit à fendre le crâne. Tout ça, dans un coin tout au bout de l'atelier, où on se sent chez soi, où le chef d'équipe et le chef d'atelier ne viennent pour ainsi dire jamais. J'ai passé là 2 ou 3 heures à 4 reprises (je m'y faisais de 7 à 8 F l'heure – et ça compte, ça, tu sais !). La première fois, au bout d'une heure et demie, la chaleur, la fatigue, la douleur m'ont fait perdre le contrôle de mes mouvements ; je ne pouvais plus descendre le tablier du four. Voyant ça, tout de suite un des chaudronniers (tous de chics types) s'est précipité pour le faire à ma place. J'y retournerais tout de suite, dans ce petit coin d'atelier, si je pouvais (ou du moins dès que j'aurais retrouvé des forces). Ces soirs-là, je sentais la joie de manger un pain qu'on a gagné.\par
Mais ça a été unique dans mon expérience de la vie d'usine. Pour moi, moi personnellement, voici ce que ça a voulu dire, travailler en usine. Ça a voulu dire que toutes les raisons extérieures (je les avais crues intérieures, auparavant) sur lesquelles s'appuyaient pour moi le sentiment de ma dignité, le respect de moi-même ont été en deux ou trois semaines radicalement brisées sous le coup d'une contrainte brutale et quotidienne. Et ne crois pas qu'il en soit résulté en moi des mouvements de révolte. Non, mais au contraire la chose au monde que j'attendais le moins de moi-même – la docilité. Une docilité de bête de somme résignée. Il me semblait que j'étais née pour attendre, pour recevoir, pour exécuter des ordres – que je n'avais jamais fait que ça – que je ne ferais jamais que ça. Je ne suis pas fière d'avouer ça. C'est le genre de souffrances dont aucun ouvrier ne parle : ça fait trop mal même d'y penser. Quand la maladie m'a contrainte à m'arrêter, j'ai pris pleinement conscience de l'abaissement où je tombais, je me suis juré de subir cette existence jusqu'au jour où je parviendrais, en dépit d'elle, à me ressaisir. Je me suis tenu parole. Lentement, dans la souffrance, j'ai reconquis à travers l'esclavage le sentiment de ma dignité d'être humain, un sentiment qui ne s'appuyait sur rien d'extérieur cette fois, et toujours accompagné de la conscience que je n'avais aucun droit à rien, que chaque instant libre de souffrances et d'humiliations devait être reçu comme une grâce, comme le simple effet de hasards favorables.\par
Il y a deux facteurs, dans cet esclavage : la vitesse et les ordres. La vitesse : pour « y arriver » il faut répéter mouvement après mouvement à une cadence qui, étant plus rapide que la pensée, interdit de laisser cours non seulement à la réflexion, mais même à la rêverie. Il faut, en se mettant devant sa machine, tuer son âme pour 8 heures par jour, sa pensée, ses sentiments, tout. Est-on irrité, triste ou dégoûté, il faut ravaler, refouler tout au fond de soi, irritation, tristesse ou dégoût : ils ralentiraient la cadence. Et la joie de même. Les ordres : depuis qu'on pointe en entrant jusqu'à ce qu'on pointe en sortant, on peut à chaque moment recevoir n'importe quel ordre. Et toujours il faut se taire et obéir. L'ordre peut être pénible ou dangereux à exécuter, ou même inexécutable ; ou bien deux chefs donner des ordres contradictoires ; ça ne fait rien : se taire et plier. Adresser la parole à un chef – même pour une chose indispensable – c'est toujours, même si c'est un brave type (même les braves types ont des moments d'humeur) s'exposer à se faire rabrouer ; et quand ça arrive, il faut encore se taire. Quant à ses propres accès d'énervement et de mauvaise humeur, il faut les ravaler ; ils ne peuvent se traduire ni en paroles ni en gestes, car les gestes sont à chaque instant déterminés par le travail. Cette situation fait que la pensée se recroqueville, se rétracte, comme la chair se rétracte devant un bistouri. On ne {\itshape peut pas} être « conscient ».\par
Tout ça, c'est pour le travail non qualifié, bien entendu. (Surtout celui des femmes.)\par
Et à travers tout ça un sourire, une parole de bonté, un instant de contact humain ont plus de valeur que les amitiés les plus dévouées parmi les privilégiés grands ou petits. Là seulement on sait ce que c'est que la fraternité humaine. Mais il y en a peu, très peu. Le plus souvent, les rapports même entre camarades reflètent la dureté qui domine tout là-dedans.\par
Allons, assez bavardé. J'écrirais des volumes sur tout ça.\par


\signed{S. W.}
\noindent Je voulais te dire aussi : le passage de cette vie si dure à ma vie actuelle, je sens que ça me corrompt. Je comprends ce que c'est qu'un ouvrier qui devient « permanent », maintenant. Je réagis tant que je peux. Si je me laissais aller, j'oublierais tout, je m'installerais dans mes privilèges sans vouloir penser que ce sont des privilèges. Sois tranquille, je ne me laisse pas aller. À part ça, j'y ai laissé ma gaieté, dans cette existence ; j'en garde au cœur une amertume ineffaçable. Et quand même, je suis heureuse d'avoir vécu ça.\par
Garde cette lettre – je te la redemanderai peut-être si, un jour, je veux rassembler tous mes souvenirs de cette vie d'ouvrière. Pas pour publier quelque chose là-dessus (du moins je ne pense pas), mais pour me défendre moi-même de l'oubli. C'est difficile de ne pas oublier, quand on change si radicalement de manière, de vivre.
\section[Lettre, à une élève, (1934)]{Lettre \\
à une élève \\
(1934)}\renewcommand{\leftmark}{Lettre \\
à une élève \\
(1934)}


\salute{Chère petite,}
\noindent Il y a longtemps que je veux vous écrire, mais le travail d'usine n'incite guère à la correspondance. Comment avez-vous su ce que je faisais ? Par les sœurs Dérieu, sans doute ? Peu importe, d'ailleurs, car je voulais vous le dire. Vous, du moins, n'en parlez pas, même pas à Marinette, si ce n'est déjà fait. C'est ça le « contact avec la vie réelle » dont je vous parlais. Je n'y suis arrivée que par faveur ; un de mes meilleurs copains connaît l'administrateur délégué de la Compagnie, et lui a expliqué mon désir ; l'autre a compris, ce qui dénote une largeur d'esprit tout à fait exceptionnelle chez cette espèce de gens. De nos jours, il est presque impossible d'entrer dans une usine sans certificat de travail – surtout quand on est, comme moi, lent, maladroit et pas très costaud.\par
Je vous dis tout de suite – pour le cas où vous auriez l'idée d'orienter votre vie dans une direction semblable – que, quel que soit mon bonheur d'être arrivée à travailler en usine, je ne suis pas moins heureuse de n'être pas enchaînée à ce travail. J'ai simplement pris une année de congé « pour études personnelles ». Un homme, s'il est très adroit, très intelligent et très costaud, peut à la rigueur espérer, dans l'état actuel de l'industrie française, arriver dans l'usine à un poste où il lui soit permis de travailler d'une manière intéressante et humaine ; et encore les possibilités de cet ordre diminuent de jour en jour avec les progrès de la rationalisation. Les femmes, elles, sont parquées dans un travail tout à fait machinal, où on ne demande que de la rapidité. Quand je dis machinal, ne croyez pas qu'on puisse rêver à autre chose en le faisant, encore moins réfléchir. Non, le tragique de cette situation, c'est que le travail est trop machinal pour offrir matière à la pensée, et que néanmoins il interdit toute autre pensée. Penser, c'est aller moins vite ; or il y a des normes de vitesse, établies par des bureaucrates impitoyables, et qu'il faut réaliser, à la fois pour ne pas être renvoyé et pour gagner suffisamment (le salaire étant aux pièces). Moi, je n'arrive pas encore à les réaliser, pour bien des raisons : le manque d'habitude, ma maladresse naturelle, qui est considérable, une certaine lenteur naturelle dans les mouvements, les maux de tête, et une certaine manie de penser dont je n'arrive pas à me débarrasser... Aussi je crois qu'on me mettrait à la porte sans une protection d'en haut. Quant aux heures de loisir, théoriquement on en a pas mal, avec la journée de 8 heures ; pratiquement elles sont absorbées par une fatigue qui va souvent jusqu'à l'abrutissement. Ajoutez, pour compléter le tableau, qu'on vit à l'usine dans une subordination perpétuelle et humiliante, toujours aux ordres des chefs. Bien entendu, tout cela fait plus ou moins souffrir, selon le caractère, la force physique, etc. ; il faudrait des nuances ; mais enfin, en gros, c'est ça.\par
Ça n'empêche pas que – tout en souffrant de tout cela – je suis plus heureuse que je ne puis dire d'être là où je suis. Je le désirais depuis je ne sais combien d'années, mais je ne regrette pas de n'y être arrivée que maintenant, parce que c'est maintenant seulement que je suis en état de tirer de cette expérience tout le profit qu'elle comporte pour moi. J'ai le sentiment, surtout, de m'être échappée d'un monde d'abstractions et de me trouver parmi des hommes réels – bons ou mauvais, mais d'une bonté ou d'une méchanceté véritable. La bonté surtout, dans une usine, est quelque chose de réel quand elle existe ; car le moindre acte de bienveillance, depuis un simple sourire jusqu'à un service rendu, exige qu'on triomphe de la fatigue, de l'obsession du salaire, de tout ce qui accable et incite à se replier sur soi. De même la pensée demande un effort presque miraculeux pour s'élever au-dessus des conditions dans lesquelles on vit. Car ce n'est pas là comme à l'université, où on est payé pour penser ou du moins pour faire semblant ; là, la tendance serait plutôt de payer pour ne pas penser ; alors, quand on aperçoit un éclair d'intelligence, on est sûr qu'il ne trompe pas. En dehors de tout cela, les machines par elles-mêmes m'attirent et m'intéressent vivement. J'ajoute que je suis en usine principalement pour me renseigner sur un certain nombre de questions fort précises qui me préoccupent, et que je ne puis vous énumérer.\par
Assez parlé de moi. Parlons de vous. Votre lettre m'a effrayée. Si vous persistez à avoir pour principal objectif de connaître toutes les sensations possibles – car, comme état d'esprit passager, c'est normal à votre âge -vous n'irez pas loin. J'aimais bien mieux quand vous disiez aspirer à prendre contact avec la vie réelle. Vous croyez peut-être que c'est la même chose ; en fait, c'est juste le contraire. Il y a des gens qui n'ont vécu que de sensations et pour les sensations ; André Gide en est un exemple. Ils sont en réalité les dupes de la vie, et, comme ils le sentent confusément, ils tombent toujours dans une profonde tristesse où il ne leur reste d'autre ressource que de s'étourdir en se mentant misérablement à eux-mêmes. Car la réalité de la vie, ce n'est pas la sensation, c'est l'activité – j'entends l'activité et dans la pensée et dans l'action. Ceux qui vivent de sensations ne sont, matériellement et moralement, que des parasites par rapport aux hommes travailleurs et créateurs, qui seuls sont des hommes. J'ajoute que ces derniers, qui ne recherchent pas les sensations, en reçoivent néanmoins de bien plus vives, plus profondes, moins artificielles et plus vraies que ceux qui les recherchent. Enfin la recherche de la sensation implique un égoïsme qui me fait horreur, en ce qui me concerne. Elle n'empêche évidemment pas d'aimer, mais elle amène à considérer les êtres aimés comme de simples occasions de jouir ou de souffrir, et à oublier complètement qu'ils existent par eux-mêmes. On vit au milieu de fantômes. On rêve au lieu de vivre.\par
En ce qui concerne l'amour, je n'ai pas de conseils à vous donner, mais au moins des avertissements. L'amour est quelque chose de grave où l'on risque souvent d'engager à jamais et sa propre vie et celle d'un autre être humain. On le risque même toujours, à moins que l'un des deux ne fasse de l'autre son jouet ; mais en ce dernier cas, qui est fort fréquent, l'amour est quelque chose d'odieux. Voyez-vous, l'essentiel de l'amour, cela consiste en somme en ceci qu'un être humain se trouve avoir un besoin vital d'un autre être – besoin réciproque ou non, durable ou non, selon les cas. Dès lors le problème est de concilier un pareil besoin avec la liberté, et les hommes se sont débattus dans ce problème depuis des temps immémoriaux. C'est pourquoi l'idée de rechercher l'amour pour voir ce que c'est, pour mettre un peu d'animation dans une vie trop morne, etc., me paraît dangereuse et surtout puérile. Je peux vous dire que quand j'avais votre âge, et plus tard aussi, et que la tentation de chercher à connaître l'amour m'est venue, je l'ai écartée en me disant qu'il valait mieux pour moi ne pas risquer d'engager toute ma vie dans un sens impossible à prévoir avant d'avoir atteint un degré de maturité qui me permette de savoir au juste ce que je demande en général à la vie, ce que j'attends d'elle. Je ne vous donne pas cela comme un exemple ; chaque vie se déroule selon ses propres lois. Mais vous pouvez y trouver matière à réflexion. J'ajoute que l'amour me paraît comporter un risque plus effrayant encore que celui d'engager aveuglément sa propre existence ; c'est le risque de devenir l'arbitre d'une autre existence humaine, au cas où on est profondément aimé. Ma conclusion (que je vous donne seulement à titre d'indication) n'est pas qu'il faut fuir l'amour, mais qu'il ne faut pas le rechercher, et surtout quand on est très jeune. Il vaut bien mieux alors ne pas le rencontrer, je crois.\par
Il me semble que vous devriez pouvoir réagir contre l'ambiance. Vous avez le royaume illimité des livres ; c'est loin d'être tout, mais c'est beaucoup, surtout à titre de préparation à une vie plus concrète. Je voudrais aussi vous voir vous intéresser à votre travail de classe, où vous pouvez apprendre beaucoup plus que vous ne croyez. D'abord à travailler : tant qu'on est incapable de travail suivi, on n'est bon à rien dans aucun domaine. Et puis vous former l'esprit. Je ne vous recommence pas l'éloge de la géométrie. Quant à la physique, vous ai-je suggéré l'exercice suivant ? C'est de faire la critique de votre manuel et de votre cours en essayant de discerner ce qui est bien raisonné de ce qui ne l'est pas. Vous trouverez ainsi une quantité surprenante de faux raisonnements. Tout en s'amusant à ce jeu, extrêmement instructif, la leçon se fixe souvent dans la mémoire sans qu'on y pense. Pour l'histoire et la géographie, vous n'avez guère à ce sujet que des choses fausses à force d'être schématiques; mais si vous les apprenez bien, vous vous donnerez une base solide pour acquérir ensuite par vous-même des notions réelles sur la société humaine dans le temps et dans l'espace, chose indispensable à quiconque se préoccupe de la question sociale. Je ne vous parle pas du français, je suis sûre que votre style se forme.\par
J'ai été très heureuse quand vous m'avez dit que vous étiez décidée à préparer l'école normale ; cela m'a libérée d'une préoccupation angoissante. Je regrette d'autant plus vivement que cela semble être sorti de votre esprit..\par
Je crois que vous avez un caractère qui vous condamne à souffrir beaucoup toute votre vie. J'en suis même sûre. Vous avez trop d'ardeur et trop d'impétuosité pour pouvoir jamais vous adapter à la vie sociale de notre époque. Vous n'êtes pas seule ainsi. Mais souffrir, cela n'a pas d'importance, d'autant que vous éprouverez aussi de vives joies. Ce qui importe, c'est de ne pas rater sa vie. Or, pour ça, il faut se discipliner.\par
Je regrette beaucoup que vous ne puissiez pas faire de sport : c'est cela qu'il vous faudrait. Faites encore effort pour persuader vos parents. J'espère, au moins, que les vagabondages joyeux à travers les montagnes ne vous sont pas interdits. Saluez vos montagnes pour moi.\par
Je me suis aperçue, à l'usine, combien il est paralysant et humiliant de manquer de vigueur, d'adresse, de sûreté dans le coup d'œil. À cet égard, rien ne peut suppléer, malheureusement pour moi, à ce qu'on n'a pas acquis avant 20 ans. Je ne saurais trop vous recommander d'exercer le plus que vous pouvez vos muscles, vos mains, vos yeux. Sans un pareil exercice, on se sent singulièrement incomplet.\par
Écrivez-moi, mais n'attendez de réponse que de loin en loin. Écrire me coûte un effort excessivement pénible. Écrivez-moi 228, rue Lecourbe, Paris, XV\textsuperscript{e}. J'ai pris une petite chambre tout près de mon usine.\par
Jouissez du printemps, humez l'air et le soleil (s'il y en a), lisez de belles choses.\par
[i] (en grec dans le texte)\par


\signed{S. WEIL.}
\section[Lettre, à Boris Souvarine, (1935)]{Lettre \\
à Boris Souvarine \\
(1935)}\renewcommand{\leftmark}{Lettre \\
à Boris Souvarine \\
(1935)}

\noindent \par
Vendredi.\par
Cher Boris, je me contrains à vous écrire quelques lignes, parce que sans cela je n'aurais pas le courage de laisser une trace écrite des premières impressions de ma nouvelle expérience. La soi-disant petite boîte sympathique s'est avérée être, au contact, d'abord une assez grande boîte, et puis surtout une sale, une très sale boîte. Dans cette sale boîte, il y a un atelier particulièrement dégoûtant : c'est le mien. Je me hâte de vous dire, pour vous rassurer, que j'en ai été tirée à la fin de la matinée, et mise dans un petit coin tranquille où j'ai des chances de rester toute la semaine prochaine, et où je ne suis pas sur une machine.\par
Hier, j'ai fait le même boulot toute la journée (emboutissage à une presse). Jusqu'à 4 h j'ai travaillé au rythme de 400 pièces à l'heure (j'étais à l'heure, remarquez, avec salaire de 3 F), avec le sentiment que je travaillais dur. À 4 h, le contremaître est venu me dire que si je n'en faisais pas 800 il me renverrait : « Si, à partir de maintenant vous en faites 800, je consentirai peut-être à vous garder. » Vous comprenez, on nous fait une grâce en nous permettant de nous crever, et il faut dire merci. J'ai tendu toutes mes forces, et suis arrivée à 600 à l'heure. On m'a quand même laissée revenir ce matin (ils manquent d'ouvrières, parce que la boîte est trop mauvaise pour que le personnel y soit stable, et qu'il y a des commandes urgentes pour les armements). J'ai fait ce boulot 1 h encore, en me tendant encore un peu plus, et ai fait un peu plus de 650. On m'a fait faire diverses autres choses, toujours avec la même consigne, à savoir y aller à toute allure. Pendant 9 h par jour (car on rentre à 1 h, non 1 h 1/4 comme je vous l'avais dit) les ouvrières travaillent ainsi, littéralement sans une minute de répit. Si on change de boulot, si on cherche une boîte, etc., c'est toujours en courant. Il y a une chaîne (c'est la première fois que j'en vois une, et cela m'a fait mal) où on a, m'a dit une ouvrière, doublé le rythme depuis 4 ans ; et aujourd'hui encore le contremaître a remplacé une ouvrière de la chaîne à sa machine et a travaillé 10 mn à toute allure (ce qui est bien facile quand on se repose après) pour lui prouver qu'elle devait aller encore plus vite. Hier soir, en sortant, j'étais dans un état que vous pouvez imaginer (heureusement les maux de tête du moins me laissaient du répit) ; au vestiaire, j'ai été étonnée de voir que les ouvrières étaient encore capables de babiller, et ne semblaient pas avoir au cœur la rage concentrée qui m'avait envahie. Quelques-unes pourtant (2 ou 3) m'ont exprimé des sentiments de cet ordre. Ce sont celles qui sont malades, et ne peuvent pas se reposer. Vous savez que le pédalage exigé par les presses est quelque chose de très mauvais pour les femmes ; une ouvrière m'a dit qu'ayant eu une salpingite, elle n'a pas pu obtenir d'être mise ailleurs que sur les presses. Maintenant, elle est enfin ailleurs qu'aux machines, mais la santé définitivement démolie.\par
En revanche, une ouvrière qui est à la chaîne, et avec qui je suis rentrée en tram, m'a dit qu'au bout de quelques années, ou même d'un an, on arrive à ne plus souffrir, bien qu'on continue à se sentir abrutie. C'est à ce qu'il me semble le dernier degré de l'avilissement. Elle m'a expliqué comment elle et ses camarades étaient arrivées à se laisser réduire à cet esclavage (je le savais bien, d'ailleurs). Il y a 5 ou 6 ans, m'a-t-elle dit, on se faisait 70 F par jour, et « pour 70 F on aurait accepté n'importe quoi, on se serait crevé ». Maintenant encore certaines qui n'en ont pas absolument besoin sont heureuses d'avoir, à la chaîne, 4 F l'heure et des primes. Qui donc, dans le mouvement ouvrier ou soi-disant tel, a eu le courage de penser et de dire, pendant la période des hauts salaires, qu'on était en train d'avilir et de corrompre la classe ouvrière ? Il est certain que les ouvriers ont mérité leur sort : seulement la responsabilité est collective, et la souffrance est individuelle. Un être qui a le cœur bien placé doit pleurer des larmes de sang s'il se trouve pris dans cet engrenage.\par
Quant à moi, vous devez vous demander ce qui me permet de résister à la tentation de m'évader, puisque aucune nécessité ne me soumet à ces souffrances. Je vais vous l'expliquer : c'est que même aux moments où véritablement je n'en peux plus, je n'éprouve à peu près pas de pareille tentation. Car ces souffrances, je ne les ressens pas comme miennes, je les ressens en tant que souffrances des ouvriers, et que moi, personnellement, je les subisse ou non, cela m'apparaît comme un détail presque indifférent. Ainsi le désir de connaître et de comprendre n'a pas de peine à l'emporter.\par
Cependant, je n'aurais peut-être pas tenu le coup si on m'avait laissé dans cet atelier infernal. Dans le coin où je suis maintenant, je suis avec des ouvriers qui ne s'en font pas. Je n'aurais jamais cru que d'un coin à l'autre d'une même boîte il puisse y avoir de pareilles différences.\par
Allons, assez pour aujourd'hui. Je regrette presque de vous avoir écrit. Vous êtes assez malheureux sans que j'aille encore vous entretenir de choses tristes.\par

\salute{Affectueusement.}


\signed{S. W.}
\section[Fragment de lettre, à X, (1933-1934 ?)]{Fragment de lettre \\
à X \\
(1933-1934 ?)}\renewcommand{\leftmark}{Fragment de lettre \\
à X \\
(1933-1934 ?)}


\salute{Monsieur,}
\noindent J'ai tardé à vous répondre, parce que le rendez-vous s'arrange mal. Je ne pourrais être à Moulins qu'assez tard dans l'après-midi du lundi (vers 4 h), et je repartirais à 9 h. Si vos occupations là-bas vous permettent de me consacrer quelques heures dans cet intervalle, je viendrai. Vous n'auriez qu'à me fixer en ce cas un rendez-vous précis en tenant compte que je ne connais pas Moulins. J'espère que cela s'arrangera. Je crois que nous aurons avantage à causer plutôt qu'à écrire.\par
C'est pourquoi je préfère réserver pour notre prochaine rencontre ce qui m'est venu à l'esprit à la lecture de vos lettres. Je veux seulement signaler une incertitude qui m'avait déjà inquiétée en écoutant votre conférence.\par
Vous dites : Tout homme est opérateur de séries et animateur de suites.\par
Tout d'abord il faudrait, ce me semble, distinguer diverses espèces de rapports entre l'homme et les suites qui interviennent dans son existence, selon qu'il joue un rôle plus ou moins actif à leur égard. Un homme peut créer des suites (inventer... ) – il peut en recréer par la pensée – il peut en exécuter sans les penser – il peut servir d'occasion à des suites pensées, exécutées par d'autres – etc. Mais c'est là quelque chose d'évident.\par
Voici ce qui m'inquiète un peu. Quand vous dites que, par exemple, le manœuvre spécialisé, une fois sorti de l'usine, cesse d'être emprisonné dans le domaine de la série, vous avez évidemment raison. Mais qu'en concluez-vous ? Si vous en concluez que tout homme, si opprimé soit-il, conserve encore quotidiennement l'occasion de faire acte d'homme, et donc ne dépouille jamais tout à fait sa qualité d'homme, très bien. Mais si vous en concluez que la vie d'un manœuvre spécialisé de chez Renault ou Citroën est une vie acceptable pour un homme désireux de conserver la dignité humaine, je ne puis vous suivre. Je ne crois d'ailleurs pas que ce soit là votre pensée – je suis même convaincue du contraire – mais j'aimerais le maximum de précision sur ce point.\par
« La quantité se change en qualité », disent les marxistes après Hegel. Les séries et les suites ont place dans chaque vie humaine, c'est entendu, mais il y a une question de proportion, et on peut dire en gros qu'il y a une limite à la place que peut tenir la série dans une vie d'homme sans le dégrader.\par
Au reste je pense que nous sommes d'accord là-dessus.\par

\begin{center}
……………………………………………………………...\end{center}
\noindent (en grec dans le texte) \footnote{Bien malgré toi, sous la pression d'une dure nécessité.}
\section[Journal d'usine]{Journal d'usine}\renewcommand{\leftmark}{Journal d'usine}

\noindent \par

\begin{quoteblock}
 \noindent Non seulement que l'homme sache ce qu'il fait – mais si possible qu'il en {\itshape perçoive l'usage} – qu'il perçoive la nature modifiée par lui.\par
 Que pour chacun son propre travail soit un {\itshape objet de contemplation}.
 \end{quoteblock}

\subsection[Première semaine]{Première semaine}
\noindent \par
Entrée le mardi 4 décembre 1934.\par
{\itshape Mardi}. – 3 h de travail dans la journée : début de la matinée, 1 h de {\itshape perçage} (Catsous).\par
Fin de la matinée, 1 h de {\itshape presse} avec Jacquot (c'est là que j'ai fait connaissance avec le magasinier). Fin de l'après-midi : 3/4 h à tourner une {\itshape manivelle} pour aider à faire cartons (avec Dubois).\par
{\itshape Mercredi matin}. – {\itshape Balancier} toute la matinée, avec des arrêts. Fait sans me presser, par suite sans fatigue. Coulé !\par
De 3 à 4, {\itshape travail facile à presse} ; 0,70 \% Coulé néanmoins.\par
À 4 h 3/4 : {\itshape machine à boutons}.\par
{\itshape Jeudi matin}. – {\itshape Machine à boutons} ; 0,56 \%, (devait être 0,72). 1.160 dans toute la matinée – très difficile.\par
Après-midi. – Panne d'électricité. Attente de 1 h 1/4 à 3 h. Sortie à 3 h.\par
{\itshape Vendredi}. – Pièces à angle droit, à la {\itshape presse} (outil devant seulement accentuer l'angle droit). {\itshape 100 pièces loupées} (écrasées, {\itshape la vis s'étant desserrée}).\par
À partir de 11 h, {\itshape travail à la main} : ôter les cartons dans un montage qu'on voulait refaire (circuits magnétiques fixes – remplacer carton par plaquettes de cuivre). Outils : maillet, tuyau à air comprimé, lame de scie, boîte à lumière, très fatigante pour les yeux.\par
Tour à l'outillage, mais pas le temps d'y voir grand-chose. Engueulée pour y être allée.\par
Samedi. – Cartons.\par
Pas un seul bon non coulé.\par
Ouvrières\par
M\textsuperscript{me} Forestier.\par
Mimi.\par
Admiratr. de Tolstoï (Eugénie).\par
Ma coéquip. des barres de fer (Louisette).\par
Sœur de Mimi.\par
Chat.\par
Blonde de l'usine de guerre.\par
Rouquine (Joséphine).\par
Divorcée.\par
Mère du gosse brûlé.\par
Celle qui m'a donné un petit pain.\par
Italienne.\par
Dubois.\par
Personnages :\par
Mouquet.\par
Chastel.\par
{\itshape Magasinier} (Pommera).\par
Régleurs :\par
Ilion.\par
Léon.\par
Catsous (Michel).\par
« Jacquot » (redevenu ouvrier).\par
Robert.\par
« Biol » (fond).\par
(ou V... ?)\par
« …. » (four).\par
Ouvriers :\par
violoniste\par
blond avantageux\par
vieux à lunettes (lecteur de l'Auto)\par
chanteur au four\par
ouvr. à lunettes du perçage (« on y va voir »... très gentil)\par
gars au maillet (boit – le seul)\par
son coéquipier\par
mon « fiancé »\par
son frangin (?)\par
jeune Ital. blond\par
soudeur\par
chaudronnier
\subsection[Deuxième semaine]{Deuxième semaine}
\noindent \par
{\itshape Lundi, mardi, mercredi}. – Chef du personnel me fait appeler à 10 h pour me dire qu'on met mon taux d'affûtage à 2 F (en fait, ce sera 1,80 F). {\itshape Ôter les cartons.} Mardi violent mal de tête, travail très lent et mauvais (mercredi je suis arrivée à le faire vite et bien, en tapant fort et juste avec le maillet – mais un mal aux yeux terrible).\par
{\itshape Jeudi}. – De 10 h (ou plus tôt ?) à 2 h environ, planage avec le grand balancier. Travail recommencé, une fois achevé entièrement, sur l'ordre du chef d'atelier, et recommencé de manière {\itshape pénible} et dangereuse.\par
Ordre de recommencer justifié, ou brimade ? En tout cas, Mouquet me l'a fait recommencer d'une manière épuisante et dangereuse (il fallait se baisser à chaque fois sous peine de recevoir le lourd contrepoids en plein sur la tête). Pitié et indignation muettes des voisins. Moi, en fureur contre moi-même (sans raison, car personne ne m'avait dit que je ne frappais pas assez fort), avais le sentiment stupide que ça ne valait pas la peine de faire attention à me protéger. Pas d'accident néanmoins. Régleur (Léon) très irrité, sans doute contre Mouquet, mais non explicitement.\par
À 11 h 3/4, regard...\par
Après-midi : arrêt jusqu'à 4 h.\par
De 4 h à 5 h 3/4...\par
Vendredi.\par
Presse – {\itshape rondelles} auxquelles l'outil donnait un trou et une forme. (O) Travaillé toute la journée. Bon non coulé, malgré nécessité de remettre un ressort, {\itshape le ressort s'étant cassé.} Première fois que j'ai travaillé toute la journée à la même machine : grande fatigue, bien que je n'aie pas donné toute ma vitesse. Erreur sur le compte, rectifiée à ma demande par l'ouvrière qui m'a suivie (très chic !).\par
{\itshape Samedi}. – 1 h pour pratiquer un trou dans des bouts de laiton, placés contre une butée très basse que je ne voyais pas, ce qui m'en a fait louper 6 ou 7 (travail fait la veille avec succès par une nouvelle qui n'avait jamais travaillé, au dire du régleur Léon, qui gueule tant qu'il peut). Coulé – mais pas de réprimande pour les pièces loupées, parce que le compte y est.\par
3/4 h pour couper de petites barres de laiton avec Léon.\par
Facile – pas de bêtise.\par
Arrêt, nettoyage de machines.\par
1 bon non coulé (de 25,50 F).\par
Ouvrière renvoyée – tuberculeuse – avait plusieurs fois loupé des centaines de pièces (mais combien ?). Une fois, juste avant de tomber très malade ; aussi on lui avait pardonné. Cette fois, 500. Mais en équipe du soir (2 h 1/2 à 10 h 1/2), quand toutes les lumières sont éteintes, sauf les baladeuses (lesquelles n'éclairent rien du tout). Le drame se complique du fait que la responsabilité du monteur (Jacquot) est automatiquement engagée. Les ouvrières avec lesquelles je suis (Chat et autres, à l'arrêt – dont adm. de Tolstoï ?) pour Jacquot. Une d'elles : « Il faut être plus consciencieux, {\itshape quand on a sa vie à gagner.} »\par
Il paraît que cette ouvrière avait refusé la commande en question (sans doute délicate et mal payée), « travail trop dur », dit-on. Le chef d'atelier lui avait dit : « Si ce n'est pas fait demain matin... ». On en a conclu, sans doute, qu'elle avait loupé par mauvaise volonté. Pas un mot de sympathie des ouvrières, qui connaissent pourtant cet écœurement devant une besogne où l'on s'épuise en sachant qu'on gagnera 2 F ou moins et qu'on sera engueulé pour avoir coulé le bon – écœurement que la maladie doit décupler. Ce manque de sympathie s'explique du fait qu'un «  mauvais » boulot, s'il est épargné à une, est fait par une autre... Commentaire d'une ouvrière (M\textsuperscript{me} Forestier ?) « Elle n'aurait pas dû répondre... quand on a sa vie à gagner, il faut ce qu'il faut... (répété plusieurs fois)... Elle aurait pu alors aller dire au sous-directeur : J'ai eu tort, oui, mais ce n'est pas tout à fait de ma faute quand même : on n'y voit pas bien clair, etc. Je ne le ferai plus, etc. »\par
« Quand on a sa vie à gagner » : cette expression a en partie pour cause le fait que certaines ouvrières, mariées, travaillent non pour vivre, mais pour avoir un peu plus de bien-être. (Celle-là avait un mari, mais chômeur.) Inégalité très considérable entre les ouvrières..\par
Système de salaire. Bon coulé au-dessous de 3 F. On règle les bons coulés, à la fin de la quinzaine, en petit comité (Mouquet, le chrono... Le chrono est impitoyable ; M., sans doute, défend un peu les ouvrières), à un prix arbitraire – des fois 4 F, des fois 3 F, des fois au taux d'affûtage (2,40 F pour les autres). Des fois on ne paie que le prix effectivement réalisé, en déduisant du boni la différence avec le taux d'affûtage. Quand une ouvrière se juge victime d'une injustice, elle va se plaindre. Mais c'est humiliant, vu qu'elle n'a aucun droit et se trouve à la merci du bon vouloir des chefs, lesquels décident d'après la valeur de l'ouvrière, et dans une large mesure d'après leur fantaisie.\par
Le temps perdu entre les tâches ou doit être marqué sur les bons (mais alors on risque de les couler, surtout pour les petites commandes) ou est déduit de la paye. On compte alors moins de 96 h pour la quinzaine.\par
C'est un mode de contrôle ; sans cela on marquerait toujours des temps plus courts que ceux effectivement employés.\par
Système des heures d'avance.\par
Histoire racontée. Mouquet : sœur de Mimi va le trouver pour se plaindre du prix d'un bon ; il la renvoie brutalement à son travail. Elle s'en va en rouspétant. 10 mn – il va la trouver : « Qu'est-ce qu'il y a ? » et arrange l'affaire.\par
« Il n'y en a pas beaucoup qui osent couler les bons. »
\subsection[Troisième semaine]{Troisième semaine}
\noindent \par
Tâches :\par
Lundi 17, matin. – Au petit balancier.\par
{\itshape Planage} toute la matinée – fatigant – coulé.\par
Le souvenir de mon aventure au grand balancier me fait craindre de ne pas frapper assez fort. D'autre part il ne faut pas, paraît-il, frapper trop fort. Et le bon comporterait une vitesse qui me semble fantastique...\par
Fin de la matinée : rondelles dans barres de métal avec presse lourde de Robert.\par
Après-midi – {\itshape presse}: pièces fort difficiles à placer, à 0,56 \% (600 de 2 h 1/2 à 5 h 1/4 ; une 1/2 h pour remonter la machine, qui s'était déréglée parce que j'avais laissé une pièce dans l'outil. Fatiguée et écœurée. Sentiment d'avoir été un être libre 24 h (le dimanche), et de devoir me réadapter à une condition servile. Dégoût, à cause de ces 56 centimes, contraignant à se tendre et à s'épuiser avec la certitude d'une engueulade ou pour lenteur ou pour loupage... Augmenté par le fait que je dîne chez mes parents – Sentiment d'esclavage –\par
Vertige de la vitesse. (Surtout quand pour s'y jeter il faut vaincre fatigue, maux de tête, écœurement.)\par
Mimi à côté de moi –\par
Mouquet : ne pas mettre les doigts. « Vous ne mangez pas avec vos doigts... »\par
{\itshape Mardi 18}. – Mêmes pièces – 500 de 7 h à 8 h 3/4, {\itshape toutes loupées.}\par
De 9 h à 5 h, travail à deux, payé à l'heure : barres de fer de 3 m de long, lourdes de 30 à 50 kg. Fort pénible, mais non énervant. Une certaine joie de l'effort musculaire... mais le soir épuisement. Les autres regardent avec pitié, notamment Robert.\par
{\itshape Mercredi 19}. – 7 h à 11 h, arrêt.\par
11 h à 5, {\itshape lourde presse pour faire des rondelles} dans une barre de tôle avec Robert. Bon coulé (2 F l'heure ; 2,28 F pour mille rondelles). Mal de tête très violent, travail accompli en pleurant presque sans arrêt. (En rentrant, crise de sanglots interminable.) Pas de bêtises cependant, sauf 3 ou 4 pièces loupées.\par
Conseils du magasinier, lumineux. Ne pédaler qu'avec la jambe, pas avec tout le corps ; pousser la bande avec une main, la maintenir avec l'autre, au lieu de tirer et maintenir avec la même. Rapport du travail avec l'athlétisme.\par
Robert assez dur quand il voit que j'ai loupé deux pièces.\par
{\itshape Jeudi 20, vendredi 21.} – Presse légère pour marquer les rivets – 0,62 \% – réalisé 2,40 F l'heure (plus).\par
(Avertissement aimable du chef d'équipe : si vous les loupez, on vous fout à la porte.) 3.000 – gagné 18,60 F. Bon coulé néanmoins : minimum 3 F. Pas de bêtises, mais retardée par des scrupules irraisonnés.\par
Rivetage : travail de combinaison. Seule difficulté, faire les opérations dans l'ordre. Ici, par exemple, deux loupés parce que j'avais riveté avant d'avoir tout assemblé, par distraction.\par
Le jeudi, paye : 241,60 F.\par
{\itshape Samedi 22}. – Rivetage avec Ilion. Travail assez agréable – 0,028 la pièce. {\itshape Bon non coulé}, mais cela en donnant toute ma vitesse. Effort constant – non sans un certain plaisir, parce que je réussis.\par
Salaire probable : 48 h à 1,80 F = 86,25 F. Boni pour le mardi, si on a travaillé à 4 F l'heure, 17,60 F pour le mercredi 1,20 F, pour jeudi et vendredi 0,60 x 15 (environ) = 9 F ; pour samedi 1,20 X 3,5 = 4,20 F. Donc :\par
17,60 F + 1,60 F + 9 F + 4,20 F = 32,40 F. Cela ferait 86,25 F + 32,40 F = 118,65 F. Là-dessus peu-être une retenue correspondant à la tâche où j'ai loupé 500 pièces.\par
En fait j'ai eu un boni de 36,75 F (mais 3/4 d'h déduits, soit 1,20 F). Donc 4,35 F de plus que je n'avais cru. Sans doute un bon arrangé – probablement planage de lundi matin.\par
Un bon non coulé (de 12 F).
\subsection[Quatrième semaine]{Quatrième semaine}
\noindent \par
Mise à pied (semaine Noël-jour de l'an). Prends froid – ai de la fièvre au cours de la semaine (fort peu) et des maux de tête terribles ; quand vient la fin des fêtes et le moment de reprendre le travail, je suis encore enrhumée, et, surtout, brisée de fatigue.\par
Jeune chômeur rencontré le jour de Noël...
\subsection[Cinquième semaine]{Cinquième semaine}
\noindent \par
{\itshape Mercredi 2}. – 7 h 1/4 à 8 h 3/4: {\itshape découpage dans longue bande métal}, à grosse presse avec Robert. 677 pièces à 0,319 \%. Marqué 1 h 10. Accroc au début par manque d'huile. Difficulté à couper la bande. À la tirer. Retiré pièces trop souvent. Gagné 1,85 F ; au taux d'affûtage on doit me payer 2,10 F. {\itshape Différence de 0,25 F.}\par
8 h 50 à 11 h 3/4 : {\itshape trous pour connexions}, avec le petit balancier (nom ?). Lenteur au début, parce que trop enfoncé outil, trop longuement placé pièce – et regardé à côté. 830 pièces à 0,84 \% . Gagné 7 F ; coulé, mais de peu. Effect. 2,30 F, marqué pour 2,80 F.\par
Pour la matinée : 1 h à regagner.\par
1 h 1/4 à 2 h 1/2 : arrêt (1 h seulement marquée).\par
2 h 1/2 à 4 h : {\itshape presse}. {\itshape Cambré pièces} découpées le matin : 600. 0,54 \% ; gagné donc 3,24 F. Marqué 1 h 20 (1/4 h de plus que si pas coulé).\par
4 h 1/2 à 5 h 1/4 : {\itshape four.} Travail très pénible : non seulement chaleur intolérable, mais les flammes vont jusqu'à vous lécher les mains et les bras. Il faut dompter les réflexes, sous peine de louper... (une loupée !). Il y a 500 pièces (le reste jeudi matin), payées 4,80 F les 100. Donc 24 F le tout.\par
Je dispose de 8 heures.\par
En dehors de ça, j'ai dans la journée 3 h 40 + 1 h 1/4 + 1 h 20 = 6 h 1/4. 2 h 3/4 à regagner. En tenir compte. Demain je ne ferai sans doute pas plus de 3 1/2 ou 4 h...\par
Four. Le premier soir, vers 5 h, la douleur de la brûlure, l'épuisement et les maux de tête me font perdre tout à fait la maîtrise de mes mouvements. Je n'arrive pas à baisser le tablier du four. Un chaudronnier se précipite et le baisse pour moi. Quelle reconnaissance, à des moments pareils ! Aussi quand le petit gars qui m'a allumé le four m'a montré comment baisser le tablier avec un crochet, avec bien moins de peine. En revanche, quand Mouquet me suggère de mettre les pièces à ma droite pour passer moins souvent devant le four, j'ai surtout du dépit de n'y avoir pas songé moi-même. Toutes les fois que je me suis brûlée, le soudeur m'a adressé un sourire de sympathie.\par
3 bons non coulés (four 2, 1 rivetage) pour 24,60 F + 9,20 F + 29,40 F = 63,20 F.\par
{\itshape Jeudi 3.} – 7 h-9 h 1/4: {\itshape four}. Nettement moins pénible que la veille, malgré un mal de tête violent dès le réveil. Ai appris à ne pas tellement m'exposer à la flamme, et à courir peu de risques de louper. Très dur néanmoins. Bruit terrible des coups de maillet, à quelques mètres.\par
Gagné 24,60 F au four. Marqué 6 h. Mis 3 h (donc 8,20 F l'h).\par
9 h 1/4-11 h 1/4 (ou 1/2 ?) : passé journée au perçage. {\itshape Rivetage} amusant : passer rivets dans piles de feuilles métalliques trouées. Mais bon inévitablement coulé. Marqué combien ? Sans doute 1 h 1/4 ? ou 1/2 ? ou 3/4 ? En tout cas au-dessous de mon tx d'affûtage (d{\itshape iff. de plus de 1 h, sans doute}).\par
11 h 1/2-3 h : déjeuné au rest. russe. {\itshape Rivetage} amusant et facile. 400 pièces à 0,023 = 9,20 F. Marqué 2 h 1/2 (de 3,70 F de l'h). À la rentrée de 1 h 1/4, souffrant d'un mal de tête accablant, j'ai loupé 5 pièces en les posant à l'envers avant de river. Heureusement le jeune chef d'équipe du perçage est venu voir...\par
Fait pour plus de 3 F l'h.\par
3 h 1/4-5 h 1/4 : {\itshape four} beaucoup moins pénible que la veille au soir et le matin – fait 300 pièces (rythme 7,35 F).\par
{\itshape Vendredi 4.} – 7 h-8 h 1/2 : {\itshape découpage de bandes dans laiton} à grosse presse. Pris mon temps, me sachant de l'avance. Médité sur un mystère exaspérant : la dernière pièce découpée dans la bande était échancrée ; or celle qui tombait échancrée était la 7\textsuperscript{e}. Explication simple donnée par le régleur (Robert) : il en restait toujours 6 dans la matrice. M. 1 h 1/4. 578 pour 0,224\par
Gagné 1,30 F ! Diff. avec {\itshape tx d'affûtage} = {\itshape 0,95 F.}\par
\par
8 h 3/4-1 h 1/2 (debout) : {\itshape polissage}. Une petite commande, marquée 10 mn, puis 300 pièces à 0,023. Gagné 6,90 F. Marqué 2 h 3/4 (ou 2 h 1/2 ?). 2,40 F ou 2,70 F l'h. Travail au tapis à polir, délicat. Fait lentement et, apparemment, {\itshape mal} (tour de main non attrapé) ; néanmoins pièces pas loupées. Mais M...t m'a fait arrêter, et fait faire à une autre les 200 pièces restantes.\par
Four. Coin tout différent, bien qu'à côté de notre atelier. Les chefs n'y vont jamais. Atmosphère libre et fraternelle, sans plus rien de servile ni de mesquin. Le chic petit gars qui sert de régleur... Le soudeur... Le jeune Italien aux cheveux blonds... mon « fiancé »... son frangin... l'Italienne... le gars costaud au maillet...\par
Enfin un atelier joyeux. Travail en équipe. Chaudronnerie, instruments : surtout le maillet ; on pratique les coudages avec une petite machine à main, puis on les arrange au maillet ; donc tour de main indispensable. Nombreux calculs, pour mesure – on met les boîtes ensemble, etc. Travail à deux le plus souvent, ou même plus.\par

\tableopen{}
\begin{tabularx}{\linewidth}
{|l|X|X|}
\hline \\
Mercredi, allée à réunion de XV\textsuperscript{e} sect. soc. et comm. concernant Citroën. Confidentiel. Pas d'ouvriers de chez Citroën, semble-t-il. \\
Réaction faible, à l'usine, là-dessus. 2 ouvrières – « On est des fois révolutionné, mais il y a de quoi. » C'est tout. Magasinier : «  C'est comme ça... » \\
À la chaudronnerie, un ouvrier avait sur sa table le tract distribué à la réunion de la veille.  &  \\
Heures mn \\
–– \\
1 h ¼ \\
2 h ½ \\
1 h \\
1 h ¼ ± ¼ ? \\
6 h \\
1 h ½ \\
6 h ½ \\
1 h ¼ \\
2 h ¾ \\
5 mn [¼] 10mn \\
5 mn     1 h ½ /25 mn \\
–      1 h ¼ \\
7 h ¾ \\
 ¾ \\
31 h (½) 20mn \\
(1h d’avance peut-être 1 h 25)  &  \\
Sous \\
–– \\
1,85 F \\
7 F \\
1,80 F \\
3,25 F \\
24,60 F \\
? 1 F \\
9,20 F \\
1,30 F \\
6,90 F \\
? \\
2,45 F \\
1,30 F \\
29,40 F \\
2,10 F \\
92,15 F \\
Tx d’af. \\
1,80 F en \\
30 h ½ \\
=54,60 F ; \\
Boni : \\
37,75 F ; \\
cela fait \\
un peu plus \\
de 3 F l’h. \\
(0,65 de plus).  \\
\hline
\end{tabularx}
\tableclose{}

\noindent 1 h 1/2-3 h 5 (debout) : {\itshape avec le régleur du fond} ({\itshape Biol} ?). {\itshape Grosses pièces}. Placer en enfonçant ; serrer avec une barre mobile ; pédale ; desserrer la barre ; taper sur un levier pour dégager la pièce ; la retirer avec vigueur... 1 F \% ! Marqué 1 h 25 mn. – 244 pièces : gagné 2,44 F. Régleur rude et très sympathique. Je l'avais déjà aidé à découper des tôles, avec grand plaisir. Bon coulé, mais par faute du chrono.\par
Diff. avec tx d'aff. : 0,25 F.\par
3 h 1/4-4 h 50 (environ) : {\itshape boîtes de tôle} : badigeonner à l'huile, passer autour d'une tige, frapper ; l'outil les forme. Mettre la soudure du bon côté. Épuisée d'avoir passé la journée et la veille debout ; mouvements lents. Grand plaisir à penser que cette boîte avait été faite par les copains de l'équipe de chaudronnerie, soudée... Pendant ce travail, quête pour une ouvrière malade. Donné 1 F. Marqué 1 h 1/4. Gagné ? Fait 137 pièces, 0,92 \% – gagné 1,30 F environ. Pourtant le chef d'équipe n'a rien dit. {\itshape Diff. av. tx d'aff. : 0,90 F.}\par
{\itshape Samedi 5}. – 7 h-10 h : {\itshape four.} À peine pénible : pas de maux de tête, fait à loisir 300 pièces. Pour les 600, gagné 29,40 F. Marqué 7 h 3/4. Travaillé rythme de 4,90 F.\par
10 h-11 h {\itshape cartons} (à continuer). Facile. Une seule bêtise à faire : bourrer. L'ai faite ! Engueulade de Léon, 50 c. \% Fait 425. Gagné 2,12 F. Marqué 3/4 h. Paye à 10 h : 115 F ; boni : 36,75 F.\par
Total des différences avec taux d'affûtage : 0,25 F + 1 F + 0,95 + 0,25 + 0,90 = {\itshape 2,50 F} (ne ruinera pas l'usine...).
\subsection[Sixième semaine]{Sixième semaine}
\noindent \par
\par
Lundi 7. – 7 h-9 h ½ : continué les {\itshape cartons}. En ai fait 865 de 7 h à 8 h 3/4 ( l h 3/4 à 50 c. \%) ; j'aurais dû en faire 1.050. Puis suis allée cisailler les trop gros, ce pourquoi Bret m'a marqué 1/2 h (effectivement).\par
À 9 h 1/4 suis allée les découper, jusqu'à 9 h 1/2. Marqué sur le 1\textsuperscript{er} bon 1/2 h (donc 1 h 1/4 pour 680 pièces), soit pour 3,40 F ; donc 2,72 F l'h : {\itshape coulé}. Marqué sur 2\textsuperscript{e} bon 1 h 10 ; pour un peu plus de 700 pièces ; NON COULÉ. Total : 1 h 10 mn + 1/2 h + 1/2 h = {\itshape 2 h 10 mn.}\par
9 h 1/2-10 h 20 : 1 h travail à l'heure (découpé extrémités de longues bandes déjà découpées, pour Bret).\par
10 h 20-2 h 40 : {\itshape planage} à la presse (avec chic régleur du fond) des grosses pièces où découpé des languettes vendredi de 1 h 1/2 à 3 h (une autre les avait cambrées dans l'intervalle). 0,80 \% ! Fait 516 en 2 h 50 mn. Marqué {\itshape 2 h 1/2.} Gagné 4,15 F, soit offic. 1,65 F de l'h. Diff. av. tx d'aff. pour 2 h 1/2 : {\itshape 0,37 F.}\par
2 h 45 à 5 h 1/4 : {\itshape presse pour ovaliser} petites pièces destinées à être soudées. 0,90 \%. Très facile. (Le chrono est sûrement fou !) En ai fait 1.400 ; donc gagné 1.400 x 0,90 = 14 x 90 = 12,60 F. Rythme réel : 5,05 F ! Marqué 1/2 h + 3/4 h + 2 h 1/4 [3 commandes] = 3 h 1/2 ; là, rythme : 3,60 F (à continuer).\par
Total des heures: 2 h 10 mn + 1 h + 2 h 1/2 + 3 h ½ = {\itshape 9 h 10 mn} ; soit {\itshape 25 mn d'avance} (soit 1 h 25 ou 1 h 50).\par
Total des prix : 3,40 F + 4,15 F + 12,60 F = 20,15 F ; y ajouter 1 h 1/2 payée à l'heure (entre 4,50 F et 6 F). (La journée à 3 F l'h serait de 26,25 F ; mais pour le bon coulé de planage on me doit plus que sur 1,80 F.) Disons 25 F en 8 h 3/4. Exactement 2,88 F l'h.\par
{\itshape Mardi 8 matin}. – 7 h 1/2-11 h 1/4 .{\itshape 1.181 pièces planées à la presse.} Accident à 7 h 1/4 : une pièce collée à l'outil le cale. Calme et patience du régleur (Ilion). 25 loupées seulement. Pas de ma faute ; mais prendre garde désormais à cette machine. 2 h 3/4. 5,30 F (0,45 \%).\par
Coulé. (Pendant qu'on la réparait, passé 1 h 1/4 à tourner manivelle pour découper cartons. L'ouvrière levait la manivelle trop tôt et m'accusait de tourner trop vite... 515.645. Trav. à l'heure.)\par
11 h 1/4-3 h 40 : {\itshape grande presse} avec Robert : ôter bavures – facile. C 280-804 – mis {\itshape 2 h 1/2} (juste {\itshape non coulé} ; n'ai eu le bon qu'à la fin). Robert, auparavant un peu sec, devenu très gentil, patient, attentif à me faire comprendre mon travail. Le magasinier a dû lui parler. Robert est sympathique décidément. Importance des qualités humaines d'un régleur.\par
3 h 45-5 h 1/4 et  \footnote{Phrase inachevée dans le texte original.}\par
{\itshape Mercredi 9}. – 7 h-1 h 1/2 {\itshape cambrage à la machine à boutons}. L'outil grippait – huiler chaque pièce – (à ce propos, le chef d'équipe m'a parlé sur un ton de gentillesse peu habituel) – long – 62 \% ; mais le tarif ne compte pas sans doute. Fait 833 – marqué en tout {\itshape 6 h.} – Travail pas trop ennuyeux, grâce au sentiment de responsabilité (j'étudiais la manière d'éviter le grippage).\par
1 h 1/2-3 h 1/2 {\itshape trous percés à la presse} (pièces comme celles que j'ai planées quand le chef m'a fait recommencer). La butée était d'abord mal mise. Ilion ne s'en fait pas pour autant – la rectifie à loisir – chante par bribes. Travaillé lentement à cause du souci de vérifier (je craignais de mal mettre à la butée). {\itshape H.} ? – marqué 1 h 1/4 – {\itshape coulé.}\par
3 h 3/4-5 h 1/4 {\itshape rivetage avec Léon} : capots d'acier enveloppés dans papier. Facile : faire attention seulement à bien mettre les rondelles (fraisure en haut).\par
Ai travaillé avec le rythme voulu, i. e. ininterrompu. Mais lenteurs au début (à restreindre à l'avenir).\par
6 bons, dont 4 non coulés. Travaillé en moyenne au rythme de 2,88 F.\par
Journée sans incidents. Pas trop pénible. Fraternité silencieuse avec le régleur bourru du fond (le seul). Parlé à personne. Rien de fort instructif.\par
Je me sens bien mieux à l'usine depuis que j'ai été dans l'atelier du fond, même quand je n'y suis plus.\par
Une ouvrière du perçage a eu toute une touffe de cheveux arrachée par sa machine, en dépit de son filet ; on voit une grande plaque nue sur sa tête. Cela s'est passé à la fin d'une matinée. Elle n'en est pas moins venue l'après-midi travailler, bien qu'elle ait eu très mal et encore plus peur.\par
{\itshape Très} froid, cette semaine. Grande inégalité de température selon les endroits de l'usine ; il y en a où je suis transie à ma machine au point d'en être nettement ralentie dans le travail. On passe d'une machine placée devant une bouche à air chaud, ou même d'un four, à une machine exposée aux courants d'air. Les vestiaires ne sont pas chauffés du tout ; on y est glacé pendant les 5 mn qu'on prend pour se laver les mains et s'habiller. L'une de nous a une bronchite chronique, au point qu'elle doit se faire mettre des ventouses tous les deux jours...\par
{\itshape Jeudi 10}. – (Éveillée à 3 h 1/2 du matin par une vive douleur à l'oreille, avec frissons, sentiment de fièvre ...)\par
7 h-10 h 40 : continué – rythme rapide, malgré malaise. Effort, mais aussi après quelque temps sorte de bonheur machinal, plutôt avilissant – une pièce loupée (pas d'engueulade). Vers la fin, incident bureaucratique : 10 rondelles manquantes.\par
L'incident bureaucratique est fort drôle. Je parle du manque de 10 rondelles à Léon qui, pas content (tout comme s'il y avait de ma faute), me renvoie au chef d'équipe. Celui-ci m'envoie sèchement à M\textsuperscript{me} Blay, au cagibi de verre. Elle m'emmène au magasin de Bretonnet, qui n'y est pas, ne trouve pas de rondelles, en conclut qu'il n'y en a pas, rentre au cagibi, téléphone au bureau dont elle suppose que vient la commande ; on l'adresse à M. X. Elle téléphone à son bureau, où on lui dit qu'il est allé faire un tour au bureau de M. Y, et on refuse d'aller le chercher. Elle raccroche, rit et peste (mais toujours de bonne humeur) pendant quelques minutes, et téléphone au bureau de M. Y, où on lui passe M. X, qui dit qu'il n'a rien à voir avec cette commande. Elle raconte en riant ses tribulations à Mouquet, et conclut qu'il n'y a qu'à passer à la quantité. Mouquet approuve tranquillement, ajoutant qu'ils ne sont pas outillés pour faire des rondelles. Je vais le dire au chef d'équipe, puis à Léon (qui m'engueule !). Pendant que je fais mon bon, on a apparemment fait de nouvelles recherches chez Bretonnet ; Léon m'apporte une quinzaine de rondelles (en m'engueulant encore !) et je vais faire les 10 pièces qui restent. Bien entendu, toutes ces tractations bureaucratiques représentent pour moi autant de temps non payé...\par
Intervalle – chef d'équipe et Léon s'accrochent légèrement au sujet d'une machine à me trouver.\par
\par
10 h 45 à 11 h 25 {\itshape recuit} dans four à Léon – 25 pièces – obligée de rester constamment devant le four (d'ailleurs petit) pour surveiller, Chaleur mal tolérable. Marqué 35 mn – 0,036 la pièce ; travaillé pour 0,90 F.\par
11 h 1/2 à 5 h {\itshape trous dans gros et lourd écran} (0,56 \% ; prix fantaisiste). C. 12190, B55 – 213 pièces – marqué 4h.\par
{\itshape Drame} – légère lâcheté de Léon (« Je ne veux pas être responsable des bêtises d'autrui. ») Il va avec ma pièce la plus mal faite au chef d'équipe (sa violence – ) – Le chef d'équipe – contrairement à son habitude plutôt gentil – vient voir et trouve que les butées sont insuffisantes. Il les fait modifier. Léon met une butée continue derrière. Je fais encore une mauvaise pièce, trompée par l'ancienne butée. Léon tempête et va au chef d'équipe. Heureusement j'en fais ensuite une bonne. Je continue, mais en tremblant. En désespoir de cause je vais chercher le magasinier, qui m'explique gentiment et d'une manière lumineuse (au lieu d'empoigner la pièce, soutenir par en dessous, et pousser constamment en avant avec les pouces ; la faire glisser le long de la butée pour m'assurer qu'elle y est). Mimi, venue à mon secours auparavant, n'avait pas su m'aider, sauf en me recommandant de moins m'en faire.\par
Formidable distance entre le magasinier et les régleurs – surtout Léon, le plus médiocre.\par
Je dis à Mimi, lui indiquant le tarif : « Tant pis, je n'ai qu'à couler le bon. » Elle répond : « Oui, puisqu'ils ne veulent pas nous payer les pièces mal faites, il n'y a rien d'autre à faire » (!).\par
{\itshape Vendredi 11}. – 7 h-8 h 5 : {\itshape id}. fait 601 pièces, soit 5,04 F. Marqué 1 h 1/2. {\itshape Non coulé}. Travaillé à près de 4 F l'h, offic. pour 3,40 F.\par
8 h 1/4-10 h 1/4 – {\itshape contacts} : petites barres de cuivre à percer en les plaçant à la butée ; pas de difficulté ; je demande à Ilion à quoi ça sert, il me répond par une blague. Au contraire Robert m'explique toujours quand je lui demande, et me montre le dessin ; mais le magasinier a dû lui parler. Quant à Léon, quand je regarde ses commandes, il m'engueule. Pourquoi ? hiérarchie ? Non : il croit que je veux m'arranger pour avoir les meilleures. En tout cas ce n'est pas de la camaraderie.\par
9 C 412087, B 2, 600 à 0,64 \% = 3,84 F. Marqué 1 h 3/4. Coulé. À la fin, léger incident avec le cisailleur (refuse de refaire des pièces, ce qui s'avère d'ailleurs inutile).\par
10 h 3/4 à 11 h 1/2 – {\itshape grosse presse à Robert}.\par
11 h 3/4 à 5 h 3/4 – {\itshape bandes de cuivre à découper et percer} (avec Léon). Second drame. – Au bout de 250 pièces, Léon s'aperçoit que les trous ne sont pas au milieu (je n'en avais rien vu). – Nouveaux cris. Mouquet survient, voit mon air désolé, et est très gentil. Du coup Léon – qui s'en fout dès lors que sa responsabilité est dégagée – ne dira plus rien. Moi, au lieu de comprendre que l'exactitude de ces trous est apparemment sans grande importance, je m'arrête à chaque pièce pour voir si c'est à la butée, je compare tout le temps au modèle. Léon m'engueule encore, dans de bonnes intentions cette fois, ne pouvant évidemment comprendre qu'on soit consciencieux aux dépens de son porte-monnaie. J'accélère un peu, mais à 5 h 3/4 n'ai fait que 1.845 pièces. Payé 0,45 \%; donc gagné 4,50 F + 3,60 F + 20 cent. = 8,30 F, soit à peine 2 F de l'heure. Aurais à rattraper plus de 1 h 1/2. Il y a 10.000 pièces.\par
Léon me donne ce travail comme une grande faveur. Effectivement c'est une grosse commande. Néanmoins même le dernier jour, déjà faite à ce travail, et donnant toute ma vitesse parce qu'anxieuse de rattraper mon retard, je fais à peine les 3 F réglementaires. Je suis un peu malade, c'est vrai. Mais le travail n'en est pas moins très mal payé.\par
{\itshape Samedi 12.} – {\itshape Id.} force à fond. Trouve procédés : d'abord poser les bandes droites (Léon avait mal arrangé les supports). Puis faire glisser la bande le long de la butée par un mouvement continu. Réalise d'abord 800 pièces en 1 h et quelque, puis ralentis sous l'effet de la fatigue. {\itshape Très} pénible. – Dos cassé qui me fait penser à l'arrachage des patates – bras droit constamment tendu – pédale un peu dure. Grâce au ciel, c'est samedi !\par
N'arrive pas à me rattraper. En fais 2.600, soit 9 F + 2,70 F = 11,70 F en 4 h. Loin de me rattraper, c'est encore de 30 c. (soit 60 pièces) au-dessous de la vitesse, réglementaire. Et j'y ai mis toute mon énergie... Me suis endormie trop tard, c'est vrai.\par
Ai fait en tout : 4.400.\par
Après-midi et dimanche pénibles : maux de tête – mal dormi, mon unique nuit [inquiétudes ... ].\par

\subsection[Septième semaine]{Septième semaine}
\noindent \par
{\itshape Lundi 14}. – {\itshape Id}. force encore plus – acquiers continuité plus grande dans coups de pédale. En ai fait 10.150 à la fin, soit dans la journée 5.050, ou\par
22,50 F + 3,75 F = 26,25 F eu 8 h 3/4.\par
À peine 3 F l'h (s'en ft de 60 cent.).\par
Je suis épuisée. Avec cela je ne me suis pas rattrapée car j'aurais dû faire les 10.000 pièces (45 F) en 15 h, et j'y ai mis 16 h 3/4.\par
À 5 h 3/4, arrête ma machine dans l'état d'âme morne et sans espoir qui accompagne l'épuisement complet. Cependant il me suffit de me heurter au gars chanteur du four qui a un bon sourire – de rencontrer le magasinier – d'entendre au vestiaire un échange de plaisanteries plus joyeux qu'à l'ordinaire – ce peu de fraternité me met l'âme en joie au point que pendant quelque temps je ne sens plus la fatigue. Mais chez moi, maux de tête...\par
{\itshape Mardi 15}. – 7 h-7 h ½ : {\itshape id.} – finis (restait 200 environ). Marque en tout 17 h 1/2. Bon coulé, mais qui reste au-dessus de 2,50 F.\par
Erre, un peu, vainement.\par
8 h... : {\itshape colliers} avec Biol. Très grosse presse (emboutisseuse) – pièces très lourdes (1 kg. ?). Il y en aura à faire 250. Payé 3,50 F \%. Faut graisser chaque pièce, et l'outil à chaque fois. Travail très dur : debout, pièces lourdes. Suis mal en point : mal aux oreilles, à la tête...\par
Incident avec la courroie, Mouquet-Biol.\par
Premier incident, le matin : Biol et Mouquet. On a arrangé la courroie de la machine avant que j'y travaille, mais mal, il faut croire ; car elle s'en va sur le côté. Mouquet la fait arrêter (Biol était fautif dans une certaine mesure, il aurait dû l'arrêter avant), et dit à Biol : « C'est la poulie qui s'est déplacée, c'est pour ça que la courroie s'en va. » Biol, regardant pensivement la courroie, commence une phrase : « Non... » et Mouq. l'interrompt – « Ce n'est pas non que je dis, moi, c'est oui. Quand même !... » Biol, sans répliquer un mot, va chercher le type chargé de réparer. Pour moi, forte envie de gifler Mouquet pour sa réaction d'officier et son ton humiliant d'autorité. (Par la suite j'apprends que Biol est universellement regardé comme une sorte de {\itshape minus habens}.)\par
2\textsuperscript{e} L'après-midi, tout d'un coup, l'outil emporte une pièce, et je n'arrive pas à la déplacer. Une petite tige empêchant de tomber la barre qui est au-dessus de l'outil avait glissé hors de son trou, et je ne l'avais pas vue ; l'outil s'était ainsi enfoncé dans la pièce. Biol me parle comme si c'était de ma faute.\par
Mardi à 1 h, distribution de tracts du syndicat unitaire. Pris, avec un sentiment de plaisir visible (et que je partage) par presque tous les hommes et pas mal de femmes. Sourire de l'Italienne. Le gars chanteur... On le tient à la main avec ostentation, plusieurs le lisent en entrant dans l'usine. Contenu idiot.\par
Histoire entendue : un ouvrier a fait des bobines avec le crochet trop court d'un centimètre. Le chef d'atelier (Mouq.) lui dit : « Si elles sont foutues, vous êtes foutu. » Mais par hasard une autre commande comportait juste de telles bobines, et l'ouvrier est gardé...\par
L'épuisement finit par me faire oublier les raisons véritables de mon séjour en usine, rend presque invincible pour moi la tentation la plus forte que comporte cette vie : celle de ne plus penser, seul et unique moyen de ne pas en souffrir. C'est seulement le samedi après-midi et le dimanche que me reviennent des souvenirs, des lambeaux d'idées, que je me souviens que je suis aussi un être pensant. Effroi qui me saisit en constatant la dépendance où je me trouve à l'égard des circonstances extérieures : il suffirait qu'elles me contraignent un jour à un travail sans repos hebdomadaire – ce qui après tout est toujours possible – et je deviendrais une bête de somme, docile et résignée (au moins pour moi). Seul le sentiment de la fraternité, l'indignation devant les injustices infligées à autrui subsistent intacts – mais jusqu'à quel point tout cela résisterait-il à la longue ? – Je ne suis pas loin de conclure que le salut de l'âme d'un ouvrier dépend d'abord de sa constitution physique. Je ne vois pas comment ceux qui ne sont pas costauds peuvent éviter de tomber dans une forme quelconque de désespoir – soûlerie, ou vagabondage, ou crime, ou débauche, ou simplement, et bien plus souvent, abrutissement – (et la religion ?).\par
La révolte est impossible, sauf par éclairs (je veux dire même à titre de sentiment). D'abord, contre quoi ? On est seul avec son travail, on ne pourrait se révolter que contre lui – or travailler avec irritation, ce serait mal travailler, donc crever de faim. Cf. l'ouvrière tuberculeuse renvoyée pour avoir loupé une commande. On est comme les chevaux qui se blessent eux-mêmes dès qu'ils tirent sur le mors – et on se courbe. On perd même conscience de cette situation, on la subit, c'est tout. Tout réveil de la pensée est alors douloureux.\par
La jalousie entre ouvriers. La conversation entre le grand blond avantageux et Mimi, accusée de s'être dépêchée afin d'arriver à point pour la « bonne commande ». – Mimi à moi : « Vous n'êtes pas jalouse, vous avez tort. » Elle dit pourtant ne pas l'être – mais peut-être l'est-elle quand même.\par
Cf. incident avec la rouquine, mardi soir. Réclame un travail qu'Ilion est en train de me donner, comme s'étant arrêtée avant moi (mais elle a une commande en train, seulement interrompue ; elle ne le dit à Ilion que quand je me suis éloignée...). Le boulot est mauvais (0,56 \%, pièces à mettre à une butée si plate qu'il est presque impossible de voir si elle y est bien) ; cependant je dois faire un effort sur moi-même pour le lui céder, car j'ai entre une heure et trois heures de retard. Mais sûrement, quand elle a vu que le boulot était mauvais, elle a pensé que c'était là la raison pour laquelle je le lui avais cédé.\par
La même rouquine, au temps des mises à pied, ne tenait pas du tout à ce qu'on en exempte celles seules et avec gosses.\par
Je ne trouve rien d'autre. Robert me refuse un travail parce que, dit-il, je louperais la moitié. Je vais donc simplement causer avec le magasinier, bien contente en un sens, car je suis à bout.\par
Le mardi soir de la 7\textsuperscript{e} semaine (15 janvier) Baldenweek me diagnostique une otite. Je me transporte jeudi rue Auguste-Comte où je reste la 8\textsuperscript{e} et la 9\textsuperscript{e} semaines. 10\textsuperscript{e}, 11\textsuperscript{e}, 12\textsuperscript{e} j. jusqu'à vendredi à Montana, en Suisse, où je vois le frère de A. L. et Fehling. Je rentre rue Lecourbe samedi soir (23 février). Rentre à l'usine le 25. Absence : un mois et 10 jours. Avais demandé permission de 15 jours la veille du 1\textsuperscript{er} février. Pris 10 jours de plus : 25 jours. À la date du 24 février, ai travaillé en tout 5 semaines (en comptant seulement les jours de travail effectif)\par
Repos de 6 semaines.
\subsection[Treizième semaine]{Treizième semaine}
\noindent (Sem. de 40 h : sortie à 4 h 1/2, repos le s.).\par
{\itshape Lundi 25}. – 7 h-8 h 1/4 (env.) : arrêt avec Mimi-Eugénie – la copine de Louisette, etc.\par
Ap. 8 h 1/4 : {\itshape marquer rivets} à presse légère. M\textsuperscript{e} travail que jeudi et vendredi de la 3\textsuperscript{e} semaine, sauf qu'il n'y a qu'un côté qu'on puisse mettre à la butée, ce qui oblige à regarder chaque pièce, et retarde. Je n'arrive pas à aller vite : je fais en tout 2.625 pièces, soit à peu près 400 à l'heure (compte tenu du fait que j'ai perdu 10 mn à toucher ma paye, le matin à 11 h). La 1\textsuperscript{re} heure, je n'arrive pas à travailler ; ma main tremble d'énervement. Après, ça va, sauf la lenteur. Mais je travaille sans fatigue. Au reste je n'ai pas le bon.\par
Si je pouvais être tous les jours aussi peu énervée et fatiguée, je ne serais pas malheureuse à l'usine.\par
{\itshape Mardi.} – Encore rivets. J'ai le bon : 0,62 \%, comme l'autre fois (où cependant les deux côtés allaient à la butée). Je fais le reste à 500 à l'heure environ, soit 3 F, mais ne rattrape pas le retard de la veille. À midi, rentre chez moi en proie à un épuisement extrême ; ne mange guère, arrive à peine à me traîner à l'usine, Mais, le travail une fois repris, la fatigue disparaît, remplacée par une sorte d'allégresse, et je sors sans fatigue. Finis les pas de vis entre 3 h 1/2 et 4 h (com. 406367, b. 3). Il y en a 6.011. J'en ai donc fait 3.375 en plus de 7 h (ce n'est quand même pas 500 à l'heure), soit 21 F, En tout 37,20 F. Marque 13 h 3/4.\par
De 4 h à 4 h ½ : rondelles, tj. avec Jacquot, à presse à main. Faut les soutenir avec la main pour les enfiler dans la matrice. Mouquet veut faire faire un montage plus commode : Jacquot n'y arrive pas, faute de blocs exactement à la hauteur voulue, et me fait seulement perdre du temps. 110 rondelles.\par
{\itshape Mercredi.} – Fini 8 h 10. {\itshape 560 rondelles} en tout, à 0,468 \% : gagné 2,60 F ! Mimi me suit (je la retarde un peu), se plaint amèrement de son bon, d'un ton un peu harassé [c. 406246, b. I].\par
Marq. 1 h 1/4.\par
{\itshape Clinquants}. Je crois d'abord que je n'y arriverai pas, mais j'y arrive très bien. Jacquot, très doux, m'avait dit de lui dire si je n'y arrivais pas. Erreur sur le prix : 2,80 \%, mais c'est pour 100 paquets de 6, soit le montant de la commande ! C'est du moins ce que dit Mimi. Je ne m'étais jamais pressée avant. Fini à 10 h, gagné exactement 2,80 F ! Marqué 2 h – com. 425512, b. 2.\par
Conversations à l'arrêt. La copine de Louisette a eu un abcès à la gorge – s'est arrêtée 5 jours – est revenue : « Les gosses, ça ne demande pas si on est malade » ; a travaillé 2 jours, s'est arrêtée encore ; est revenue après que l'abcès a percé. Elle est toujours gaie. Elle devient nerveuse, dit-elle, ne peut plus supporter que ses gosses se donnent du mouvement en jouant, etc.\par
Mouquet – il lui a dit : « Vous avez les cheveux aussi longs que le corps. » Elle était vexée, vexée. Aurait voulu répondre grossièrement. « On ne peut pas répondre. » La sœur de Mimi, elle, répond. Une fois, elle va le trouver pour réclamer pour un bon ; il la renvoie brutalement à son boulot ; elle y va en rouspétant. 1/4 d'heure après il va la trouver et arrange le bon... « Quand le travail ne va pas, il vaut mieux s'adresser à lui qu'à un régleur ou à Chastel ; et il est alors très gentil. » Mais parfois colère ; et il manque de tact. On cite de ses mots vexants : « Vous n'avez jamais été à la chasse ? » à la sœur de Mimi. – Eugénie interrompt son travail pour venir me raconter joyeusement qu'elle a vu les animaux d'un cirque, à la porte de Versailles (2 F d'entrée) ; qu'elle a caressé le léopard...\par
Doléances du petit manœuvre : il a fait 2 ans de latin, 1 an de grec, de l'anglais (il se vante de tout ça naïvement), est de son métier employé de bureau (il en est très fier) et on l'a mis manœuvre ! « Il faut obéir à des c... qui ne savent même pas signer leur nom ! » Et on se fait engueuler par eux, encore. « Si c'est ça, la camaraderie ouvrière !… » Après ça, on échange des sourires quand il passe. Il a peut-être 17 ans. Assez prétentieux.\par
\par
Léon n'est pas là (s'est blessé le bras). Soulagement indescriptible. Jacquot le remplace, détendu et tout à fait charmant.\par
{\itshape Rivetage, au grand balancier}. Difficile – les pièces ne vont pas toutes. Une pièce loupée, qui donne à Jacquot un air grave. Le compte n'y est pas ; passé à la quantité ! (108 pièces, je crois, au lieu de 125). Payé 0,034 pièce, soit 3,65 F en tout (1 h perdue). Et j'ai fini à 2 h 3/4 ! Marqué 3 h. Ensuite 3/4 d'heure arrêt chez Bretonnet (couper les déchets) ; enfin des {\itshape cartons} que je finis juste à 4 h 1/2 avec Jacquot, presse à main ou pied à volonté. Jacquot tj. gentil (m'arrange une caisse, etc.). Le petit manœuvre vient me déranger. Pas marqué prix, mais bon coulé.\par
Gagné ces 3 jours 37,20 F + 3,60 F + 2,60 F + 2,80 F + 3,65 F + (mettons !) 2,50 F = 52,35 F  ! ! ! soit 17,43 F par journée de 8 h, soit une moyenne de 2,20 F l'heure ! Au-dessous du taux d'affûtage officiel !\par
Le soir, à mes cartons, maux de tête. Mais en même temps sentiment de ressources physiques. Les bruits de l'usine, dont certains à présent significatifs (les coups de maillet des chaudronniers, la masse ... ), me causent en même temps une profonde joie morale et une douleur physique. Impression fort curieuse.\par
En rentrant, maux de tête accrus, vomissements, ne mange pas, ne dors guère ; à 4 h 1/2, décide de rester à la maison ; à 5 h me lève... Compresses d'eau chaude, cachet. Jeudi matin, ça va.\par
{\itshape Jeudi}. – « Plaquettes d'entrefer. » Com. c 421346, b. 1. 0,56 \%. 1.068 pièces, soit 6 F. Fini à 9 h 5(?), marq. 2 h, bon {\itshape non coulé} (le seul).\par
« {\itshape Déflecteur du doigt mobile} » avec Robert – pièces que je crois d'abord difficiles à placer ; mais je reconnais ensuite que l'outil les met en place en tombant, et ça va plus vite. 510 pièces, 0,71 \%, soit 3,50 F. Finis à 10 h 3/4, marqué 1 h 1/2 [soit 2,30 F l'heure]. Com. 421329, b. 1.\par
{\itshape Arrêt} (déchets). Bretonnet marque 1/2 h.\par
{\itshape Plaques de serrage à la cisaille} (avec Jacquot) (debout, un pied sur la pédale, à la presse où j'avais fait avec Louisette les grosses barres de 40 kg). Com. 421322, b. 1. 0,43 \%, marque 350 (j'apprends le lendemain qu'il y en avait plus, je n'avais pas compté). 1,50 F. Marque 35 mn. Fini à 11 h 3/4; gagné ce matin – 6 F + 3,50 F + 0,90 F + 1,50 F = 11,90 F, en 4 h 3/4, soit exact. 2,50 F l'heure.\par
Après-midi : découpé cartons à l'heure avec la sœur de Mimi ; tourné la manivelle. Très agréable, sans à coups comme les fois d'avant. Marqué 1 h 1/4.\par
À 2 h 1/2, mise par Jacquot à {\itshape Cosses} (pièces pour moteurs électriques, dit le magasinier) C. 421337, b. 1 – 0,616 \%, travail à la pièce.\par
La difficulté était de mettre les pièces à la butée de manière que le 2\textsuperscript{e} angle droit se fasse. Si elles n'étaient pas juste à la butée, la pièce était loupée.\par
Jacquot me l'explique gentiment. Je m'applique, sûre de moi. Je réussis plusieurs. Une, trop large, n'entre pas dans le creux de la matrice, et, n'étant pas maintenue, recule. Chatel, juste derrière moi, me dit pas trop brutalement de les mettre mieux à la butée. J'en réussis d'autres, puis en loupe encore. Non seulement certaines pièces sont trop larges, mais d'autres trop étroites, et la butée, arrondie par l'usure, les fait glisser. Je montre à Jacquot : il dit de mettre les larges de côté. Je l'appelle encore ; il parle à Chatel, me dit de continuer, et, si ça ne va pas, de le dire à Chatel. J'essaie encore, puis vais chez Chatel, une pièce loupée à la main. Il me dit : elle est morte, celle-là. Faut les mettre à la butée. J'essaie d'expliquer. Il dit, sans se déranger : Allez-y, et tâchez de ne pas continuer comme ça. J'appelle aussitôt le magasinier, qui dit : Ça ne va pas bien, évidemment, quoique moi je les réussirais toutes. Il essaye en les mettant avec le doigt et en les maintenant quand l'outil tombe... et en loupe pas mal aussi ! Il étudie ça longtemps, appelle un type de l'outillage qui lui dit que la butée est usée (je l'avais vu tout de suite !), enlève la matrice, va limer la butée, remonte la machine. Je continue au doigt (dangereux !). Ça va mieux, mais pas encore bien. Je vais le retrouver ; il est avec Mouquet qui vient voir, donne ordre d'élargir un peu la matrice et mettre l'outil plus bas pour que ma main ne risque pas de passer dessous. Ça va jusque 4 h 1/2... Il y a un peu plus de 100 pièces faites, et une 40taine loupées.\par
Pour ces 4 jours, je suis payée 66,55 F (4 F de retenue pour A. S.). Mais les 2 derniers sont payés au taux d'affûtage : 14,40 F par jour, pour moi (1,80 F l'heure). J'ai 12,95 F de boni pour les 2 premiers.\par
28,80 F + 12,95 F = 41,75 F. Où diable l'ont-ils pris ? Il y a arrêt (1 h 1/4 soit 3,25 F ?) et puis ?\par
{\itshape Vendredi 1\textsuperscript{er} mars}. – Fais mes cosses. Finis à 10 h ½ ; en ai fait en tout 2.131, soit 2.030 environ ce matin-là en 3 h 1/2 (soit 580 à l'heure, à 0,616 \% !). Gagné en tout 13 F. Explique à Chatel que j'ai perdu 2 h la veille ; il grommelle : « 2 h ! », et met sur le bon : temps perdu..., mais ne met pas {\itshape combien} ! Marqué 2 h et 3 h 1/2.\par
Arrêt jusqu'à 11 h 3/4.\par
Dispute à l'arrêt entre Dubois et Eugénie et la rouquine.\par
{\itshape Recuis} au petit four, à la rentrée ; ça va, c'est-à-dire que je ne perds pas mon sang-froid en ôtant les pièces. Pénible, parce que je suis perpétuellement devant le four (pas comme au grand). On m'interrompt à 2 h parce que... les pièces sont pour le laminage à froid ! ! ! Je ne marque que mon temps sur le bon. Marqué 3/4 h.\par
Attends Robert pendant bien 20 mn. Une autre aussi...\par
Vais, sur le conseil du magas., demander à Delouche l'autorisation de rester jusqu'à 5 h 1/4. Accordé. Vais le soir même à l'outillage. Le contremaÎtre ne me voit pas.\par
« {\itshape Poignées} » {\itshape à la cisaille} c. 918452, b. 31. Avec Robert. 300 à faire, à 0,616 \%, soit 1,85 F en tout. Je ne pense pas au prix, à la vitesse obligée, et je les fais tout à mon aise, prenant bien soin, à chaque fois, de placer la pièce au bout arrondi bien à la butée. Certaines barres sont tordues et rendent difficile de maintenir à la butée. Beaucoup trop long : fini à 3 h 25 (mais commencé tard). Marqué 1 h.\par
{\itshape Cosses}. Les mêmes. Toujours 0,616 \% – dernière opération : les mettre en V. À la pince-boutons. Retardée souvent par la difficulté de détacher la pièce de l'outil, au reste, facile à placer.\par
La pièce, pendant que l'outil la met en V, se plie légèrement. Je le montre à Jacquot (qui pourtant m'avait dit que ce n'était pas la peine de regarder les pièces), il le montre à Chatel ; tous deux discutent gravement, puis Chatel dit qu'on planera (mais comment ?) et me fait continuer. Je continue tout à mon aise, bien trop lentement. Fais 281 pièces seulement ! Reste 1.850, à faire au plus en 3 h 1/4, c'est-à-dire, en tenant compte des pertes de temps, au rythme de 600 à l'heure. Indispensable !\par
\par
Si je marque 1 h vendr. pour les cosses, j'ai 1/2 h perdue. Mais plutôt perdre 1 heure que de couler mon bon, si possible. 1/4 d'heure perdu (si nettoyage compte 1/4 h).\par
Mais non : en réalité 0,72 (boutons), de 15,30 F-5 h. Reste 4 h, soit 460 à l'heure, pour rattraper. Aurais dû faire ds 1 h : 425. Si je ne fais lundi que 425 à l'heure, pour pas couler le bon, perdre encore vendredi 20 mn.\par
Mais non, d'ailleurs : il y a 1/4 h pour le nettoyage des machines. Ne dois donc compter que 3/4 h vendredi, et n'ai rien à rattraper que 5 mn, négligeables. Ai donc encore 4 h 1/4. Dois finir à 11 h 1/4.\par
Beaucoup moins fatiguée que je ne le craignais. Moments d'euphorie, même, à mes machines, comme je n'en ai pas eu à Montana même (effet à retardement !). Mais la question de la nourriture reste aussi angoissante.
\subsection[Quatorzième semaine]{Quatorzième semaine}
\noindent \par
{\itshape Lundi 4.} – Maux de tête vifs, lundi, en me levant. Par malchance toute la journée on fait marcher à côté de moi la chose tournante au bruit infernal. À midi, à peine capable de manger. Mais cela n'empêche pas la vitesse, et sans cachets.\par
{\itshape Cosses}. – Finis seulement à 11 h 3/4, mais non par ma faute : plus d'1/2 h, sûrement, est perdue dans la matinée (beaucoup plus, même) à cause de la machine. Avec les boutons, dit Jacquot, ça ne va jamais. Je le persuade de mettre la pédale, bien que ce soit plus dangereux. Ça ne va pas non plus ; je dois l'appeler encore. Sur ordre de Mouquet, il remet les boutons. Va toujours pas. Le petit Jacquot s'impatiente... À 11 h 10, se met à démonter la machine – ressort cassé. Mais, quand il la remonte, rien ne va plus. Il devient nerveux, nerveux... Le chef d'équipe, quand je lui remets mon bon (car je renonce à finir les pièces, vu que ce qui est fait est plus que le compte), est sarcastique pour J.\par
Après-m. : 1/2 h arrêt. Puis 2 commandes de {\itshape plaquettes}, 520 chacune, à 0,71 \% (c. 421275, b. 4). Je perds du temps au début : pour retirer les pièces, pour les compter – aussi pour les placer, car j'y prends des précautions inutiles – et pédale mal (pas à fond ; pédale dure). 1\textsuperscript{re} comm. finie à 3 h 1/4. 2\textsuperscript{e} commencée à 3 h 25 (je perds 5 mn à attendre, ne m'apercevant pas que Jacquot a préparé la machine), faite à un train d'enfer, mon maximum, finie juste à 4 h 1/2 : là, j'ai fait du 3,60 F de l'heure. Marqué 1 h 20 chaque. 4 h 1/2 + 1/2 + 2 h 40 = 7,40 F. Gagné vendredi et lundi : 12,30 F + 1,35 F + 1,85 F + 14,40 F + 0,90 F + 7,80 F = {\itshape 39,60 F.} Là-dessus, 21,20 F pour lundi. Mis 1 h pour vendredi et 4 h 1/2 pour lundi.\par
Vendredi, j'ai vu la lourde machine de Biol en préparation (pas prête). Le magas. me dit : prends pas ça, c'est trop dur. Je trouve autre chose. Lundi, je vois Eugénie qui le fait toute la journée. Suis bourrelée de remords. Si j'avais voulu m'arranger pour le prendre, je l'aurais pu sans doute. Et je sais combien c'est pénible : c'est ce que j'avais fait la dernière après-midi lors de l'otite, ou quelque chose d'équivalent. À 4 h 1/2, elle est visiblement épuisée.\par

\tableopen{}
\begin{tabularx}{\linewidth}
{|l|X|}
\hline \\
Jacquot et la machine. \\
Le magasinier, le dessinateur, et l' « outil universel ». \\
L'outillage et son contremaître.  & Que s'était-il passé avec la machine ? (idiote, de n'a – voir pas observé avec plus d'attention). – Quand j'appuyais sur les boutons, l'outil tombait parfois 2 fois ; le chef d'équipe, voyant ça, dit : «  Ça ne doit pas faire ça » (c'est tout ! »). Plus tard, ça refait la même chose, seulement la 2\textsuperscript{e} fois ça reste ! Jacquot la relève et je continue... jusqu'à ce que ça recommence. Il finit par me faire arrêter. Ilion, qui passe, lui dit que 1e «  doigt » (le ressort) de la grande roue est cassé. C'est vrai. Mais il y avait, paraît-il, encore autre chose. On voit que, pour le petit Jacquot, la machine est une drôle de bête... \\
\hline
\end{tabularx}
\tableclose{}

\noindent {\itshape Mardi matin. –} 3 commandes analogues à lundi soir.\par
1] 600 à 0,56 \%, petites pièces difficiles à ôter, marque 1 h 1/4.\par
2] 550 à 0,71 \%, m. 1 h 20.\par
3] 550 à 0,71 \%, m. 1 h 20.\par
{\itshape Très} fatigant à la longue, car la pédale est très dure (mal au ventre). Jacquot toujours charmant.\par
Après, tombe sur Biol (nostalgie des pièces lourdes qui m'ont donné des remords !), il me met au « piano », où je passe aussi toute l'après-midi, exception faite pour arrêt de 2 h 3/4 à 3 h 3/4. Les 2 commandes payées 0,50 \%, l'une 630, l'autre 315.\par
Temps m. 2 h, puis 3 h 1/4.\par
Total : 1 h l/4, l h 20, l h 20, 2 h 3/4= 6h 40 mn, il me faudrait 1 h 20 d'arrêt, je crois que j'en ai 1 h. ce qui ferait 20 mn perdues.\par
À 4 h 1/2, très fatiguée, au point que je pars tout de suite. Le soir, vifs maux de tête.\par
Au « piano » d'abord ai beaucoup de peine à cause de ma crainte de mal buter ; à la fin de l'après-midi, ça va un peu mieux. Mais bouts des doigts sanglants.\par
{\itshape Mercredi matin}. – Encore piano (630 pièces), ça va encore mieux, sauf le mal aux doigts – néanmoins prends plus de 1 h 1/2. Marqué 1 h 20. Robert, aussitôt après, me fait faire une commande de 50 pièces (c. 421146 27) (Payé ?) Assez gentil pour me donner un autre bon de commande de 50 mêmes, qu'il a faites parce que c'était pressé, pour y mettre du temps. Difficultés : certaines n'entrent pas dans la butée. Il me les fait mettre de côté pour les faire lui-même. Retardée par une lourde fatigue et des maux de tête, passé 1/2 h que je partage entre les deux bons. Après, encore « piano » : les mêmes 630, à refaire autrement. J'essaye de faire de la vitesse et manque en louper ; néanmoins je ne me laisse plus trop retenir par la crainte de louper (bien qu'il ne faille pas en perdre une seule, m'a dit Biol, car le compte n'y est peut-être pas, ou juste). Je recompte en les refaisant. Avais d'abord trouvé 610. Trouve là 620, à quelques unités près. L'ouvrière qui les avait faites avant a dit en avoir trouvé 630 : la 2\textsuperscript{e} fois, je dis qu'il y a le compte, pour en avoir fini. Comment veut-on qu'on compte convenablement, au taux de 0,50 \% ? Marqué 1 h 20. Après, Robert me reprend. 2 commandes marquées 25 mn chacune (quoi ?).\par
Fini tout cela (y compris confection des bons) à 11 h 1/4. Je dis au chef avoir fini à 11 h 5, il me marque un bon d'arrêt à 11 h, moyennant quoi je n'ai pas pris de retard ce matin. Il me reproche d'avoir marqué tous mes bons à la fois.\par
Après-midi, arrêt j. à 2 h. Puis {\itshape calottes} : 200 à 1,45 \% ! je devrais donc y mettre moins d'une heure. Or, elles sont lourdes, il faut les prendre d'une caisse, et on donne 4 coups de pédale pour chacune, et 2 opérations.\par

\tableopen{}
\begin{tabularx}{\linewidth}
{|l|X|}
\hlineOn les met d’abord Ainsi \noindent\includegraphics[]{}  &  Puis on les retourne. À la 2\textsuperscript{e} opération ainsi : \includegraphics[]{}  \\
\hline
\end{tabularx}
\tableclose{}

\noindent puis on les retourne. Donc, avec 1\textsuperscript{er} montage, on les fait toutes, 2 coups de pédale à chacune, puis 2\textsuperscript{e} montage {\itshape id}. – il faut donc donner 800 coups de pédale. Or elles ne sont pas si faciles à placer : on doit passer les trous dans les vis, etc. Je n'ai le bon qu'une fois la 1\textsuperscript{re} opération finie. J'ai le sentiment souvent de ne pas donner toute ma vitesse. Néanmoins je m'épuise. Le soir, je me sens, pour la première fois, vraiment écrasée de fatigue, comme avant de partir pour Montana ; sentiment de commencer à nouveau à glisser à l'état de bête de somme. Reste néanmoins : conversation avec le magasinier, visite à l'outillage.\par
{\itshape Jeudi}. – Continue mêmes pièces jusqu'à 8 h. Marque 3 h 1/2 : la vérité (oublie noter commande). Après, c. 421360, b. 230 plaquettes serrage à 1,28 F \%. Fini à 9 h 3/4. Marqué 1 h 10 m (y a-t-il eu 1/2 h arrêt entre- temps ? je ne sais plus). Fait avec Jacquot, à la petite presse à main. Jacquot a toujours de charmants sourires.\par
Après, arrêt jusqu'à 11 h. À l'arrêt, sens tout le poids de la fatigue, attends du travail qu'on me donnera avec un sentiment de malaise. Les ouvrières s'irritent de perdre souvent leur tour d'arrêt pour des commandes de 100 pièces (notamment la sœur de Mimi). Jacquot vient, apportant une commande de 5.000 pièces ; c'est mon tour. Ce sont des rondelles à découper dans des bandes, avec pédalage continu. Prix 0,224 \%, (à peu près). Je voudrais bien ne pas couler le bon. Je me mets au travail sans arrière-pensée. Jacquot me fait une seule recommandation : ne pas laisser bourrer, de peur de casser l'outil. La fatigue et le désir d'aller vite m'énervent un peu. Je mets une bande, en commençant, pas assez loin, ce qui m'oblige à recommencer le 1\textsuperscript{er} coup de pédale et loupe une pièce (1 loupée sur 5.000, c'est peu, mais si cela se produisait à toutes les bandes, cela ferait beaucoup). Cela m'arrive plusieurs fois. Enfin, énervée, je remets alors la bande trop loin, elle passe par-dessus la butée et au lieu d'une rondelle il tombe un cône. Au lieu d'appeler aussitôt Jacquot, je retourne la bande, mais, n'ayant pas conscience de la faute commise, je passe encore par-dessus la butée (du moins c'est vraisemblable), et c'est encore un cône qui tombe, et, aussitôt après, le « grenadier » de l'outil (?). L'outil est cassé. Ce qui me peine le plus, c'est le ton sec et dur que prend ce cher petit Jacquot. La commande était pressée, le montage, peut-être difficile, était à refaire, tout le monde était énervé par des accidents similaires arrivés les jours précédents (et peut-être le jour même ?). Le chef d'équipe, bien entendu, m'engueule comme un adjudant qu'il est, mais collectivement, en quelque sorte (« c'est malheureux d'avoir des ouvrières qui... »). Mimi, qui me voit désolée, me réconforte gentiment. Il est 11 h 3/4.\par
\par
Après-midi (vifs maux de tête). Arrêt j. 3 h 1/2. 500 pièces, encore des ronds à couper dans des bandes (quelle malchance !), mais à petite presse à main. Je suis horriblement énervée par la crainte de recommencer. Effectivement, je passe plus d'une fois la bande un peu au-dessus de la butée au 1\textsuperscript{er} coup de pédale, mais il n'en résulte rien ; à chaque fois je tremble... Jacquot a retrouvé ses sourires (je dois faire appel à lui pour quelques caprices de la machine, qui refuse de se mettre en marche, ou bien fonctionne n. fois de suite pour un coup de pédale), mais je n'ai plus le cœur d'y répondre.\par
Incident entre Joséphine (la rouquine) et Chatel. On lui a donné, paraît-il, un boulot très peu rémunérateur (à la presse à côté de la mienne, qui est celle à boutons en face le bureau du chef). Elle rouspète. Chatel l'engueule comme du poisson pourri, très grossièrement, il me semble (mais je ne discerne pas bien les paroles). Elle ne réplique rien, se mord les lèvres, dévore son humiliation, réprime visiblement une envie de pleurer et, sans doute, une envie plus forte encore de répondre avec violence. 3 ou 4 ouvrières assistent à la scène, en silence, ne retenant qu'à moitié un sourire (Eugénie parmi elles). Car si Joséphine n'avait pas ce mauvais boulot, l'une d'elles l'aurait ; elles sont donc bien contentes que Joséphine se fasse engueuler, et le disent ouvertement, plus tard, à l'arrêt – mais non pas en sa présence. Inversement Joséphine n'aurait vu aucun inconvénient à ce qu'on refile le mauvais boulot à une autre.\par
Conversations à l'arrêt (je devrais les noter toutes). Sur les maisons de banlieue (sœur de Mimi et Joséphine). Quand Nénette est là, il n'y a le plus souvent que des plaisanteries et des confidences à faire rougir tout un régiment de hussards. (Cf. celle dont l'« ami » est peintre [mais elle vit seule], et qui se vante de coucher avec lui 3 fois par jour, matin, midi et soir ; qui explique la différence entre la « technique » dudit et celle d'un autre – qui se fait aider pécuniairement par lui, et « ne se prive de rien » ; autant que j'ai compris, le temps qu'elle ne passe pas à faire l'amour, elle le passe à se faire la cuisine et à manger.)\par
Mais chez Nénette, il y a autre chose que ça encore – quand elle parle de ses gosses (garçon de 13 ans, fille de 6) – de leurs études – du goût de son fils pour la lecture (elle en parle avec respect). Les derniers jours de cette semaine, semaine où elle a tout le temps été à l'arrêt, elle a une gravité inaccoutumée ; elle se demande évidemment comment elle fera pour payer la pension des gosses \footnote{La rencontre au métro, alors que je suis chez Renault. Raconte que huit jours plus tôt elle a été malade, n'a pas pu prévenir et n'ose plus retourner à l'Alsthom, – (qu'est-ce qu'elle risque ? Mais...). Sans doute coup de tête... Air de compassion peinée, quand je dis que je suis chez Renault.}.\par
Incident au sujet de M\textsuperscript{me} Forestier. Il est question de quête pour elle. Eugénie déclare qu'elle ne donnera rien. Joséphine aussi (mais celle-là ne doit pas donner souvent), et ajoute que M\textsuperscript{me} Forestier est passée à l'usine dire bonjour à tout le monde (le jour même où je suis rentrée) à cause de la quête. Nénette et l'Italienne, autrefois ses grandes copines, ne donneront rien non plus. Elle a, paraît-il, fait du mal, non à elles, mais à plusieurs autres (?).\par
L'Italienne est malade. Ma 2\textsuperscript{e} semaine, elle avait demandé à « aller à la pêche » et Mouquet a refusé ; or il n'y en avait que 2, et il n'y a eu que de l'arrêt. Elle a 2 gosses ; son mari est briquetier (manœuvre) et gagne 2,75 F l'heure. Elle ne peut donc pas se soigner. Elle a le foie malade, et des maux de tête que les bruits de l'usine rendent intolérables (je connais ça !).\par
{\itshape Vendredi}. – Arrêt. Je ne le passe pas, comme je l'aurais fait quelques semaines plus tôt en pareille circonstance, à trembler à l'idée des bêtises que je ferai peut-être. Preuve que je suis un peu plus sûre de moi que naguère.\par
Ilion m'appelle (à quelle heure ?) pour échancrer des couvercles pour métros. Il y a un côté : je crains vivement de me tromper par distraction. 149 couvercles (bon de 150) à 1,35 F \%. Je ne cherche guère à aller vite, craignant trop de louper : car une seule pièce « morte » serait, là, d'une grosse importance. Une alerte : l'outil manque de pénétration, l'échancrure ne s'en va pas. Beaucoup de temps perdu pour la manutention : il y a 3 chariots. J'en trouve 147 ; émoi du chef d'équipe, qui me fait passer 1/4 d'h à recompter (mais ce 1/4 d'h ne sera pas mis dans le bon, mais dans l'arrêt) com. 421211, b. 3. Fini à 9 h. Arrêt jusqu'à 10 h : fatiguée, inquiète, je voudrais rester à l'arrêt toute la journée. À 10 h on m'appelle pour ôter cartons de circuits magnétiques (ce que j'avais fait fin de la 1\textsuperscript{re} semaine). Je vois qu'il y en a pour jusqu'au soir. Soulagement considérable. J'emploie la technique découverte le dernier jour que je l'avais fait (beaucoup de petits coups de maillet) et travaille bien et assez vite (plus de 30 pièces à l'heure ; or, les p\textsuperscript{ers} jours, j'en avais fait 15, et Mouquet avait estimé la valeur de mon travail à 1,80 F l'h, puisqu'il m'avait dit qu'en 5 h j'avais fait à peine pour 9 F de travail). Pas de crainte de faire de bêtises, d'où détente. Néanmoins (et bien qu'ayant mangé à midi au restaurant) je me sens prise vers le milieu de l'après-midi d'une très grande fatigue et fais bon accueil à l'annonce que je suis à pied.
\subsection[Quinzième semaine]{Quinzième semaine}
\noindent \par
Mise à pied (du 8 mars au 18 mars). – Maux de tête samedi et dimanche – prostration presque totale jusqu'à mercredi à midi ; l'après-midi, par un magnifique temps de printemps, vais chez Gibert de 3 à 7 heures. Lendemain, vais chez Martinet, achète manuel de dessin industriel. Après-midi, vendredi, prostration. La nuit, ne dors pas (mal de tête) ; dors jusqu'à midi. Samedi vois Guihéneuf de 2 h (à sa boîte) à 10 h 1/2. Dimanche quelconque.
\subsection[Seizième semaine]{Seizième semaine}
\noindent \par
{\itshape Lundi 18}. – {\itshape Rondelles dans bande} j. 7 h 50 (?). Avec Léon, revenu (mon cher petit Jacquot de nouveau ouvrier). 0,336 \%, 336 pièces. Encore la frousse. Bêtise commise 2 fois, heureusement passée inaperçue ; je n'en prends conscience qu'après la 2\textsuperscript{e} : je retourne la bande après le 1\textsuperscript{er} coup de pédale ; or le trou qu'il a pratiqué n'est pas au milieu de la bande, parce qu'on appuie en arrière. Il en résulte quelques pièces tordues que je cache, et l'outil ne s'en porte probablement pas mieux. Travail très lent, aucun souci de vitesse. Marqué 40 mn.\par
{\itshape Planage au petit balancier} de ces mêmes rondelles, ce qui me permet de supprimer une pièce loupée qui m'avait échappé. Comm. 907405, b. 34, 0,28 \%. Fini 8 h 1/2 h, marque 1/2 h (donc perdu 20 mn en tout), gagné 0,95 F ! Mon taux d'affûtage... Je n'ai guère cherché la vitesse.\par
{\itshape Planage au petit balancier de shunts} c. 420500. 796 pièces jusqu'à 2 h 1/4. Marqué 4 h 1/4. Payé 1,12 F \% ; gagné 8,90 F (guère plus de 2 F l'h). Chatel me fait frapper 4 à 5 coups par pièce (2 sur une extrémité, 2 ou 3 sur l'autre). Je lui dis, en lui remettant le bon, que dans ces conditions je n'ai pas pu ne pas le couler. Il me répond du ton le plus insolent : coulé à 1,12 F ! Ça ne m'impressionne pas, vu son incapacité. J'ignore s'il a mis quelque chose s. le bon ; sûrement non. J'aurais dû frapper moins de coups... J'ai essayé d'aller vite, mais je me surprenais sans cesse à retomber dans la rêverie. Contrôle de la vitesse difficile, car je ne comptais pas. Fatiguée, notamment à la sortie de 11 h 3/4 (mange au « Prisunic » ; détente ; délicieux instants avant la rentrée : fortifs, ouvriers... Me retrouve esclave devant ma machine).\par
Vu, dans l'entrée, par séries, des shunts semblables, liés d'un côté à des doigts de contact, de l'autre à des bobines métalliques.\par
Arrêt – théor. de 2 h à 3 h.\par
{\itshape Douilles à poinçonner dans drageoi}r avec Robert. C. 406426. 580 pièces 0,50 \%, : donc 2,90 F. Marqué 1 h 10, rythme 2,45 F l'h. Fait en réalité de (2 h 30 à 4 h 10), soit 1 h 40 mn. Mais perdu du temps en essayant des pinces les 100 premières, et, à la fin, en ramassant les pièces. Là encore, n'atteins rythme ininterrompu que par moments, et retombe dans la rêverie. Compteur pour contrôler : après avoir fait 40 ou 45 pièces en 5 min, j'en fais 20 les 5 mn suivantes, où je me suis laissée aller à rêver.\par
Arrêt de 4 h 1/4 à 4 h 1/2.\par
Total : 40 mn + 1/2 h + 4 h 1/4 + 1 h + 1 h 10 mn + 1/4 h = 8 h juste.\par
Rentre (à 5 h 1/2) fraîche et dispose. La tête pleine d'idées tout le soir – cependant j'ai souffert – surtout au balancier – bien plus que le lundi après Montana.\par
Le repas au « Prisunic » est-il pour quelque chose dans mon bien-être du soir ?\par
{\itshape Mardi. – Arrêt} j. 8 h 1/4.\par
{\itshape Rivetage de doigts de contact} avec Léon, jusqu'au soir. 500 à 4 F 12 \%, c. 414754, b. 1. Pour interrupteurs. Équipement de trains. D'abord très lent Chatel m'a fait peur, je crains de faire quelque bêtise il ne s'agit pas de louper des pièces, et j'en ai loupé la 1\textsuperscript{re}. Il y a 4 pièces à assembler : contact et 2 plaquettes, et paquet de 10 clinquants (mais certains paquets n'en ont que 9). Il fallait faire attention aux 2 trous, inégaux, de la grde plaquette – mettre la petite, la bavure au-dessus, et dans le sens du cisaillage. Fais les premiers 70 en 2 h, je crois... Après, ne cesse de rêver. N'arrive au rythme ininterrompu que l'après-midi (réconfortée par le déjeuner et la flânerie), mais en me répétant continuellement la liste des opérations (fil fer – grand trou – bavure – sens – fil fer ... ), moins encore pour me préserver d'une étourderie que pour m'empêcher de penser, condition de la vitesse.\par
Sens profondément l'humiliation de ce vide imposé à la pensée. J'arrive enfin à aller un peu vite (à la fin, je fais plus que 3 F l'h), mais l'amertume au cœur.\par
{\itshape Mercredi. – Id.} j. 8 h 1/2, m. 7 h 3/4, gagné 20,60 F (en 8 h 1/4, soit 2,50 F de l'h).\par
Je ne retrouve pas le « rythme ininterrompu »; j'aurais dû finir à 8 h.\par
{\itshape Polissage} des mêmes j. à 3 h 3/4, m. 5 h 1/4, gagné 13,50 F. C. 414754, b. 4. 0,027 p. C'est ce que j'avais fait la semaine du four ; Mouquet m'avait ôté ce travail, comme mal fait, et effectivement, je m'en tirais fort mal. Je commence donc avec appréhension. Je vais très, très lentement d'abord. Catsous m'abandonne à moi-même. Je fais une 1\textsuperscript{re} découverte concernant le sens dans lequel on doit tourner la pièce : dans celui où l'entraînerait le ruban, mais en la tirant en sens contraire du ruban. Ainsi la pièce et le ruban restent en contact (du moins je me figure que c'est là la raison). La 2\textsuperscript{e} (faite depuis longtemps, mais je l'applique là) est qu'une main ne doit faire qu'une opération à la fois. J'appuie donc de la main gauche, je tire de la main droite ; quant à tourner, je n'ai pas à le faire, le ruban s'en charge. Quant au rythme, je vais d'abord à mon aise ; puis, constatant mon extrême lenteur, je m'efforce vers le « rythme ininterrompu », mais avec répugnance et ennui ; aussi le plaisir d'avoir conquis un tour de main m'est-il tout à fait insensible. à midi, je déjeune en vitesse à « Prisunic », puis vais m'asseoir au soleil en face de chez les aviateurs; j'y demeure dans une telle inertie que j'arrive à l'usine dans une sorte de demi-rêve, sans me presser le moins du monde, à 1 h 13 ou 14... On fermait la porte !\par
4 h-4 h 1/2 à rivetage, voir le lendemain.\par
Paye – 125 F (dont 4 F avancés). La précédente, 70 F. Soit 192 F pour 32 + 48 = 80 heures... donc 2,40 F l'heure exactement...\par
Conversation avec Pommier – Connaît tous les outils.\par
Soir, maux de tête, et fatigue fort amère au cœur. Je ne mange pas, sinon un peu de pain et de miel. Prends un tub pour me faire dormir, mais le mal de tête me tient éveillée à peu près toute la nuit. À 4 h 1/2 du matin, je suis prise d'un grand besoin de sommeil. Mais il faut se lever. Je repousse la tentation de prendre 1/2 journée.\par
{\itshape Jeudi.} – Tte la journée : rivetage des armatures – ai atteint 700 à 4 h 1/2 (en 8 h 3/4) – entrain en sortant à midi – épuisement après le repas. Soir : trop fatiguée pour manger, reste étendue sur le lit ; peu à peu lassitude très douce – sommeil délicieux.\par
C. 421121, b. 3 – 0,056 pièce – 800 pièces. Marqué 14 h 1/4.\par
Pensée vide, toute la journée, sans artifice comme pour le rivetage, par un effort de volonté soutenu sans trop de peine. Pourtant je m'étais levée avec un mal de tête qui a failli me faire rester.\par
Encouragée par le fait que c'est du « bon boulot », quoique dur. Et aussi – surtout – par une sorte d'esprit sportif. Travail réellement {\itshape ininterrompu}.\par
Outillage (Mouquet y vient...)\par
Italienne et Mouquet.\par
« 4 sous... de l'heure, ça ne vous suffit pas dans cette période de chômage ? »\par
Réflexions d'Ilion :\par
« Le patron sera toujours assez riche... Ça va toujours trop vite, c'est pour ça qu'il n'y a pas de travail... »\par
Sur un « J. P. » qui passe : « et ce sont les mieux vus encore » –\par
\par
{\itshape Vendredi}. – Finis rivetage. Mais il manque des rivets (à vrai dire, il y en avait dans les rainures de la machine). 8 1/4 à 8 3/4, 50 rallonges à 0,54 \% c. ? (sûrement 413910), marque 1/4 h. Rondelles carton, non chronométrées, bon de travail n° 1747, c\textsuperscript{de} 1415, marque 2 h (mis 2 h 1/4) – calottes. C. 412105, b. 1, 0,72 \% (boutons), 400 pièces. Marqué 3 h 1/2 (je ne les ai pas finies en partant, mais Chatel les finit). Perdu 1 h ; la veille j'en avais pris (retard rattrapé) 3, reste 2.\par
Machine démolie par Ilion (au cours d'un montage, il a cassé q. chose).\par
Le magasinier : « Les régleurs ne savent pas se servir des freins. » « Ils ne savent pas mettre les boutons. C'est toujours trop court, de sorte que le clapet... (?). »\par
Lundi. –\par
J. 8 h fini {\itshape circuits magnétiques}, commande 20154 – il n'en reste que 25 environ. Je travaille facilement, sans me presser, sans lenteur néanmoins. Marque 1 h. J'ai 6 h en tout (le bon n'était pas passé).\par
« {\itshape Rallonges} » (boîtes à 4 côtés à mettre en forme). Prix dérisoire (0,923 \%), 50 pièces ! C. 413910, b. 1. Je marque 1/2 h. Finis à 9 h 3/4. On n'en met pas deux à la fois, me dit Mimi. On met de l'huile à toutes. – Alors ?\par
J. 10 h 3/4, {\itshape clinquants} av. Léon, à côté d'Eugénie qui pose des rivets. C. 425537, b. 2 – 200 paquets de 6 – 2,80 F \%. Je vais vite (le mercredi après Montana, j'avais mis 2 h pour 100 paquets !). Donc gagné 5,60 F. Je marque 1 h 50 mn (non coulé). J'ai réalisé à peu près, là encore, le rythme ininterrompu.\par
{\itshape Débiter pièces dans barres métalliques} à la presse où j'ai passé un mercredi avec Louisette. Bien mettre à la butée, bien tenir parallèle... je ne vais pas vite. Cela dure jusqu'à 1 h 50 mn. Je marque apparemment par erreur trop de temps : 1 h 40 mn. C. 4009194, b. 97346 pièces à 0,88 \% ! (Je crois en faire 360, mais vendredi Catsous m'apprendra qu'il n'y en avait que 330 !). Travaille sans tendre aucunement à la vitesse : fatiguée et découragée par le prix, ayant aussi un alibi dans les difficultés à faire tomber les pièces.\par
{\itshape Déchets} de 1 h 3/4 à 3 h 1/2 (donc 1 h 3/4).\par
{\itshape Mêmes pièces} à mettre en triangle, sur {\itshape même} bon, Profond dégoût, qui me fait ralentir.\par
Fini à 4 h 1/2 – marqué 3 h 1/4 en tout.\par
Mardi. –\par
1/4 h déchets.\par
Conversation aux déchets : Souchal grossier. Joséphine l'a sommé un jour de venir la... \footnote{Mot illisible dans le texte.}, l'y a fait contraindre par Mouquet. Celui-ci est juste, mais capricieux. Règle les bons coulés tantôt à... tantôt à... pas selon la dureté du travail !\par
Pièces délicates à placer à la butée : matrice presque sans relief (« {\itshape bilames} »), avec Léon. C. 421227, 2.100 pièces avec les boutons, donc à 0,72. Marqué 6 h 1/4. Même machine où j'avais fait les cosses la 2\textsuperscript{e} fois, et que Jacquot n'avait pu arranger.\par
1/2 h déchets (perdu 40 mn ces 2 jours).\par
Pommera (Jacquot et la machine aux cosses). Régleurs et machines.\par
{\itshape Mercredi.} – 1/2 h déchets.\par
Piano de 7 h 1/2 à 8 h 1/4. C. 15682, b. 11, puis c. 15682, b. 8, les deux à 0,495 \%, 180 pièces pour la 1\textsuperscript{re}, 460 pour la 2\textsuperscript{e}. Marque 25 mn, puis 1 h 1/4. Lenteur lamentable. Celle dont l'ami est peintre vient me \footnote{Phrase inachevée dans le texte.}.\par
Rivetage, « support inférieur ensem. ». C. 24280, b. 45, 200 pièces à 0,10 F (autrefois 0,028 !) (prix provisoire pour la commande Souchal) à p. 9 h 3/4 j. jeudi matin. Marqué 6 h 1/4 en tout. Fait le matin 75 pièces, soit 7,50 F encore. Mal de tête TRES violent, ce jour-là, sans quoi j'aurais pu aller plus vite. Je me suis couchée bien la veille, mais réveillée à 2 h. Le matin, envie de rester chez moi. À l'usine, chaque mouvement fait mal. Louisette, de sa machine, voit que ça ne va pas.\par
Une ouvrière du perçage : son gosse de 9 ans au vestiaire. Il vient travailler ? « Je voudrais bien qu'il soit assez grand pour ça », dit la mère, Elle raconte que son mari vient de lui être renvoyé de l'hôpital où on ne peut à peu près rien pour lui (pleurésie et grave maladie du cœur). Il y a encore une fille de 10 mois...\par
{\itshape Jeudi.} –\par
3/4 h déchets.\par
C. 428195, b. 1, marq. 2 h. C. 23273, b. 21, 198 pièces (ttes comptées) à 1,008 F \%, (temps ? 2 h je crois). Rondelles : 10.000 à 7,50 F, marqué 1 h 1/2 pour ce jour-là ; perdu 1 h 3/4.\par
Machine à bras. Deux leviers, dont un de sécurité, empêche l'autre de s'abaisser ; je ne comprenais pas à quoi il servait, le magas. me l'explique [cf. Descartes et Tantale !].\par
{\itshape Vendredi.} – Finis rondelles à la hâte. En les passant au tamis, m'aperçois que beaucoup sont loupées. J'en « fous en l'air » le plus possible ; ai néanmoins fort peur. Je marque 10.000, bien qu'il en manquât déjà en dehors de celles « foutues en l'air », et 2 h 1/2, ce qui fait un bon non coulé.\par
8-9 h déchets.\par
9 h-10 h 1/2, pièces faciles à faire. C. 421324, bon de 500 ; il n'y en a que 464 ; Robert me fait passer le bon. Payé 0,61 \%. Marque 1 h (bon coulé), car je crois avoir perdu plus d'1/2 h à regarder Robert se débattre avec une machine. Le clapet ne s'écartait plus (Pommera est venu après ; une pièce manquait, un coin). Il y était quand je suis arrivée ; ne s'est pas interrompu pour moi. Ça a recommencé plusieurs fois. L'ouvrière (pour une fois) semblait un petit peu intéressée (je ne la connais pas ; brune aux cheveux un peu fous, à l'air sympathique).\par
10 h 1/2 à 4 h 1/2, déchets (chance, car c'est un repos inexprimable ; dans l'après-midi, je finis même par m'asseoir) – avec seulement 200 pièces à recuire, au four de Léon, jusqu'à 2 h – Marqué 50 mn, 0,021 pièce ; donc gagné 4,20 F (mais est-ce bien recuit ?). Je n'ose marquer plus de 50 mn, et ne prends pas le temps de calculer. Cela fait, hélas, 5 F l'heure. Baissera-t-on le bon à cause de moi ? J'aurais mieux fait d'attendre et de marquer au moins 1 h. En tout cas perdu en tout 25 mn.\par

\tableopen{}
\begin{tabularx}{\linewidth}
{|l|X|X|X|X|X|X|X|X|X|}
\hline\multicolumn{11}{l}{quinzaine} \\
\hline
 &  & \multicolumn{4}{l}{Bons non coulés} & \multicolumn{5}{l}{Bons coulés} \\
\hline
 &  & \multicolumn{4}{l}{} & \multicolumn{5}{l}{} \\
\hline
\multicolumn{2}{l}{ Déchets \\
– } &  {\itshape N\textsuperscript{os}} \\
–  &  Prix \\
–  & \multicolumn{2}{l}{Temps} &  N.c. \\
–  &  Prix \\
–  & \multicolumn{3}{l}{Temps} \\
\hline
1 h &  &  421121 \\
(Armatures R)  & 44,80 F & 14 h & 15 mn & 907.405 rondelles L &  1 F 12 \\
[coulé pour raison  &  & \multicolumn{2}{l}{40 mn} \\
\hline
 & 15 mn &  24280 \\
(support R)  & 20 F & 6 h & 15 mn & {\itshape Id.} plan L & 0,95 F &  & \multicolumn{2}{l}{30 mn} \\
\hline
1 h & 15 mn & ? & 7,50 F & 2 h & 30 mn & 420.500 pl. shunts L & 8,90 F & 4 h & \multicolumn{2}{l}{15 mn} \\
\hline
1 h & 45 mn & (Rondelles I) &  &  &  &  &  &  & \multicolumn{2}{l}{} \\
\hline
 & 15 mn &  408294 \\
(four L)  & 4,20 F \\
76,50 F  &  –– \\
22 h  &  50 mn \\
110 mn  &  406.426 douilles R \\
414.754 doigts L  &  2,90 F \\
20,60 F  &  1 h \\
7 h  & \multicolumn{2}{l}{ 10 mn \\
45 mn } \\
\hline
 & 30 mn &  &  &  &  & {\itshape Id} – (pol.) Q & 13,50 F & 5 h & \multicolumn{2}{l}{15 mn} \\
\hline
 & 30 mn &  &  &  &  & 413.910 cal. I & 0,27 F &  & \multicolumn{2}{l}{15 mn} \\
\hline
 & 45 mn &  &  & 23 h & 50 mn & 412.105 cal. I & 2,88 F & 3 h & \multicolumn{2}{l}{30 mn} \\
\hline
1 h &  & oublié &  &  &  & 413.910 cal. I & 0.46 F &  & \multicolumn{2}{l}{30 mn} \\
\hline
1 h & 15 mn &  &  &  &  & ******** &  &  & \multicolumn{2}{l}{} \\
\hline
2 h & 30 mn & 425537 & 5,60 F & 1 h & 50 mn &  &  &  & \multicolumn{2}{l}{} \\
\hline
 & 240 mn &  & 32,10 F & 25 h & 40 mn & 4.009.194 perc. L & 2,90 F & 3 h & \multicolumn{2}{l}{15 mn} \\
\hline
\multicolumn{2}{l}{7 h (60 x 4)} & \multicolumn{4}{l}{ Quels bons {\itshape auraient dû} ne pas être coulés ? Ceux de planage (mais...) – les douilles, les doigts (si j'avais pris le bon système tout de suite...) le polissage, si cela n'avait pas été seulement la 2\textsuperscript{e} fois, le piano – là, la responsabilité revient aux maux de tête). Les pièces △ (démoralisée par l'annonce de renvoi). \\
Dorénavant : chercher d'abord le {\itshape système} pour obtenir avec sécurité la plus grande rapidité. Après, viser au {\itshape rythme ininterrompu.} } & 421.227 bil. L & 15,12 F & 6 h & \multicolumn{2}{l}{15 mn} \\
\hline
\multicolumn{2}{l}{7h + 4h = 11h} &  15.682 pian B \\
Id.    B \\
428.195 mcbr. L \\
23.173 – R \\
421.342 – R  &  0,89 F \\
2.30 F \\
2,80 F ? \\
2,14 F \\
2,83 F  &  1 h \\
2 h \\
2 h \\
1 h  & \multicolumn{2}{l}{ 25 mn \\
15 mn \\
( ? ) } \\
\hline
\multicolumn{2}{l}{ 12 h } &  à ajouter : \\
1.415  &  90,55 F \\
?  &  35 h \\
40 h \\
2 h  & \multicolumn{2}{l}{ 300 mn \\
5 h } \\
\hline
\multicolumn{2}{l}{} & \multicolumn{4}{l}{} &  & \multicolumn{2}{l}{(b, tr, 1.747)} & \multicolumn{2}{l}{} \\
\hline
\multicolumn{6}{l}{ Manque 20 mn \\
Mais si on ajoute 3 F pour les circuits (?) et 5,50 F pour les calottes, et peut-être 1,50 F quelque part ailleurs, soit 10 F, j'aurai 167 F pour 65 h), soit 2,55 F l'heure environ... \\
Si j'ai pour ces 65 h 170 F, et pour les 11 h de déchets et les 2 h de cartons 32,50 F, et pour les 5 h de circuits en retard 15 F, cela me ferait en tout 217,50 F, moins la retenue des assurances sociales ! \\
Ajouter aux 167 F, 6 F pour les clinquants, soit 173 F. En tout peut-être 223 F, dont 209 F seraient pour cette quinzaine-ci. \\
Dans l'ensemble, je n'ai pas fait de progrès qui soient appréciables dans le cadre du salaire... } & \multicolumn{2}{l}{ 80,50 –––––– 82,10 } &  & \multicolumn{2}{l}{} \\
\hline
\multicolumn{5}{l}{162,60 F pour 65 h ¾ de travail} \\
\hline
\multicolumn{2}{l}{} & \multicolumn{2}{l}{} &  \\
\hline
\multicolumn{5}{l}{ 157 |  64 290 |  2,45 340 |––––––––––––––– 20 | 163 | | |–––––– | 310 | 66 | 460 | 2,4766 | 440 | | 24 | } \\
\hline
\multicolumn{5}{l}{} \\
\hline
\end{tabularx}
\tableclose{}

\noindent Chatel charmant – on me laisse une liberté totale – on me traite en condamnée à mort...\par
Nénette soudain grave. « Tu vas chercher du boulot ? Pauvre Simone ! » Elle-même à pied la semaine prochaine. « C'est impossible d'y arriver. » Je dis à Louisette ce que j'en pense ; elle répond que Mouquet a refusé à Nénette d'être exempte de la mise à pied. Mme Forestier l'avait été il y a 2 ans, mais par ordre d'en haut.\par
{\itshape Lundi. – Recuit} j. 9 h 10 plaques (arrêts de bobine), 200 à 0,021 [421263, b. 21].\par
Tiges : 180 à 0,022 [928494, b. 48], marq. 1 h 1/4 et 1 h.\par
{\itshape Balancier}, calibrage de corps de carrure (comme 2\textsuperscript{e} jour ?) [22616, b. 17, 2 bons], 116 p. à 0,022 \%, – chaque opération, l'une difficile, l'autre facile ! mis 50 mn (fini à 11 h 1/2).\par
Petites pièces : 421446, 150 pièces sur 400 à 0,62 \%, soit 0,90 F en tout – marqué 1/4 h. Mal au ventre violent – infirmerie. M'en vais à 2 h 1/2 après avoir vainement essayé de tenir le coup. Prostration j. 6 h environ, après, pas fatiguée.\par
{\itshape Mardi.} – Bornes, 240 à 0,53 \%, [409134, 409332].\par
Satisfaction profonde que le travail aille mal... Mouquet.\par
{\itshape Rondelles} : 421437, b. 1, 0,56 \%, 865 p., m. 1 h 1/4, presse à cosses 2.\par
{\itshape Guides d'enclenche} [12270, b. 68] : 1,42 F \%, 150 p., presse à Robert (mais il est à la pêche, c'est avec Biol) – ce sont des barres dans lesquelles on coupe en 2 coups de pédale successifs, parce que l'outil n'a pas la longueur voulue. Elles ne sont pas plates. Si on les entre ainsi faciles à entrer, presque impossibles à sortir. Ainsi, très dures à entrer, sortables. Biol recommande la 1\textsuperscript{re} manière, Pommera (très dédaigneux pour lui) la 2\textsuperscript{e} – Mouquet vient – me fait prendre la 1\textsuperscript{re} man., mais me donne une clef pour sortir (Pom. l'apportait, Mouquet a dit : « Je vais lui montrer »). Je la manie d'abord maladroitement. Il doit me rappeler le principe du levier...\par
Pour la 1\textsuperscript{re} fois pt-être, je rentre à 1 h 1/4 avec plaisir – dû, aussi, à la manière dont Mouquet m'a parlé.\par
Je jouis de faire un travail dur, qui « ne va pas ». À 1 h 1/4, je dis à Pommera que le travail qui ne va pas est bien moins embêtant. Il dit – « C'est vrai. » Je m'écorche les mains (une mauvaise coupure). Question du rythme inexistante, puisque le bon ne compte pas. Je remarque que devant Mouquet je prends sans effort le « rythme ininterrompu ». Lui une fois parti, non... Ce n'est pas parce que c'est le chef – c'est que quelqu'un me regarde et attend après moi.\par
DÉCHETS : 2 h 1/2 à 3 h 1/4.\par
{\itshape Piano} : 344 tôles à 0,56 \%, [508907 b. 10], m. 50 mn.\par
Guides (?) : 40009195, 1 h.\par
Soir, pas fatiguée. Vais à Puteaux par un beau soleil, un vent frais – (métro, taxi col.). Vient par autobus j. rue d'Orléans. Délicieux – monte chez B. Mais me couche tard.
\subsection[Le mystère de l'usine]{Le mystère de l'usine}
\noindent \par
\subsubsection[I. Le mystère de la machine.]{I. Le mystère de la machine.}
\noindent \par
\par
Guihéneuf : faute d'avoir fait des mathématiques, la machine est un mystère pour l'ouvrier. Il n'y voit pas un équilibre de forces. Aussi manque-t-il de sécurité à son égard. Ex. : le tourneur qui, par tâtonnement, a trouvé un outil permettant de cylindrer à la fois l'acier et le nickel, au lieu de changer d'outil pour passer d'un métal à l'autre. Pour Guihéneuf, c'est une coupe, simplement ; il y va carrément. L'autre, avec un respect superstitieux. De même une machine qui ne va pas. L'ouvrier verra qu'il faut y mettre telle ou telle chose... mais souvent y fait une réparation qui, tout en lui permettant de marcher, la voue à une usure plus rapide, ou à un nouvel accroc. L'ingénieur, jamais. Même s'il ne se sert jamais du calcul différentiel, les formules différentielles appliquées à l'étude de la résistance des matériaux lui permettent de se faire une idée précise d'une machine en tant que jeu déterminé de forces.\par
La presse qui ne marchait pas et Jacquot. Il est clair que, pour Jacquot, cette presse était un mystère, et de même la cause qui l'empêchait de marcher. Non pas simplement en tant que facteur inconnu, mais en soi, en quelque sorte. Ça ne marche pas... Comme un refus de la machine.\par
Ce que je ne comprends pas dans les presses : Jacquot et la presse qui frappait 10 coups de suite.
\subsubsection[II. Le mystère de la fabrication.]{II. Le mystère de la fabrication.}
\noindent \par
Bien entendu, l'ouvrier ignore l'usage de chaque pièce, 1] la manière dont elle se combine avec les autres, 2] la succession des opérations accomplies sur elle, 3] l'usage ultime de l'ensemble.\par
Mais il y a plus : le rapport des causes et des effets dans le travail même n'est pas saisi.\par
Rien n'est {\itshape moins} instructif qu'une machine...\par

\subsubsection[III. Le mystère du « tour de main ».]{III. Le mystère du « tour de main ».}
\noindent \par
Circuits d'où j'ai dû ôter les cartons. Au début je ne savais pas les séparer à coups de maillet. J'ai fait alors des raisonnements sur le principe du levier qui ne m'ont guère servi.. Après quoi, j'ai su très bien, sans jamais m'être rendu compte ni comment j'ai appris ni comment je procède.\par
Principe essentiel de l'habileté manuelle dans le travail à la machine (et ailleurs ?) mal exprimé. Que chaque main ne fasse qu'{\itshape une} opération simple. Ex. travail sur bandes métalliques : une main pousse, une autre appuie à la butée. Plaques de tôle : ne pas tenir avec la main ; laisser reposer sur la main, appuyer vers la butée avec le pouce. Ruban à polir : appuyer avec une main, tirer avec une autre, laisser le ruban tourner la pièce, etc.
\subsection[Transformations souhaitables.]{Transformations souhaitables.}
\noindent \par
Des machines-outils diverses se côtoyant dans un même atelier. Le montage à côté. La {\itshape disposition} de l'usine visant à donner à chaque travailleur une vue d'ensemble (cela suppose évidemment la suppression du système des régleurs).\par
Spécialisations dégradantes :\par
De l'ouvrier – de la machine – des parties d'usines [des ingénieurs ?]
\subsection[Organisation de l'usine.]{Organisation de l'usine.}
\noindent \par
Manque de tabourets, de caisses, de pots d'huile.\par
Chronométrage fantaisiste. Et ce sont les tâches misérablement payées pour lesquelles on se fatigue le plus, parce qu'on tend toutes ses forces, jusqu'à l'extrême limite, pour ne pas couler le bon. (Cf. conv. avec Mimi, mardi 7\textsuperscript{e} sem.) On s'épuise, on se crève pour 2 F l'heure. Et non parce qu'on fait une tâche qui exige qu'on s'y crève ; non, seulement à. cause du caprice et de la négligence du chrono. On se crève sans qu'aucun résultat, soit subjectif (salaire), soit objectif (œuvre accomplie), corresponde à la peine. Là, on se sent vraiment esclave, humilié jusqu'au plus profond de soi.\par
Pommera, lui, estime le chronométreur (Souchal) ; l'excuse en disant que son métier est impossible, pris comme il est entre la direction et les ouvrières. D'abord, dit-il, quand Souchal est derrière les ouvrières, elles en mettent un coup. Il y a aussi la question des temps faux : un bon non coulé ne peut jamais être rectifié par la suite.\par
Pour chaque tâche, il y a une quantité limitée – et faible – de fautes possibles, susceptibles les unes de casser l'outil, les autres de louper la pièce. En ce qui concerne l'outil, il n'y a même que quelques fautes possibles par catégorie de tâches. Il serait facile aux régleurs de signaler ces possibilités aux ouvrières, pour qu'elles aient quelque sécurité.\par
À remarquer si les presses sont {\itshape spécialisées} ? Tenter une nomenclature – Presse à planer – Emboutisseuse de Biol. –\par
Chefs et bureaucrates\par
G...\par
X. Vient du Génie maritime.\par
« Un directeur c'est une machine à prendre des responsabilités », « pas de métier plus stupide que celui de directeur ». « Un bon directeur doit avant tout ne pas être un bon technicien. En savoir assez seulement pour qu'on ne lui fasse pas avaler de bourdes. »\par
D....\par
X. Ponts et chaussées.\par
D'abord directeur et administrateur délégué. À présent, a formé un directeur pour s'épargner du travail.\par
Est arrivé à la tête de l'entreprise ignorant tout de la technique de la fabrication. S'est senti perdu pendant 1 an.\par

\tableopen{}
\begin{tabularx}{\linewidth}
{|l|X|X|}
\hlineMouquet (chef d'atelier). Chrono (Souchal pet. brun). Mme Biay (?). M. Chanes. \}  & Cagibi de verre & Le plus intéressant est évidemment Mouquet [Chrono : type odieux, grossier, paraît-il, avec les ouvrières – fait tendre toujours vers le plus bas – chronomètre à peu près au hasard – je ne lui ai jamais parlé. Pommera n'en pense aucun mal. \\
\hline
\end{tabularx}
\tableclose{}

\noindent Le plus intéressant est évidemment Mouquet [Chrono : type odieux, grossier, paraît-il, avec les ouvrières – fait tendre toujours vers le plus bas – chronomètre à peu près au hasard – je ne lui ai jamais parlé. Pommera n'en pense aucun mal.\par
........chef d'équipe des presses.\par
Catsous –    perceuses.\par
Mouquet et les pièces sur lesquelles j'ai passé au début 5 jours à retirer les cartons.\par
Mouquet – tête sculpturale, tourmentée – q, eh. de monastique – toujours tendu – « j'y penserai cette nuit ». L'ai vu une seule fois allègre.\par
Régleurs :\par
Ilion (chef) – Léon – Catsous – Jacquot (redevenu ouvrier) Robert – Biol.\par
Ouvrières :\par
\par
M\textsuperscript{me} Forestier – Mimi – sœur de Mimi – Admiratrice de Tolstoï – Eugénie – Louisette, sa copine (jeune veuve avec 2 gosses) – Nénette – rouquine (Joséphine) – Chat – blonde aux 2 gosses – séparée de son mari – mère du gosse brûlé – celle qui m'a donné un petit pain – celle qui est atteinte de bronchite chronique – celle qui a perdu un gosse et est heureuse de n'en point avoir, et a perdu « heureusement » son 1\textsuperscript{er} mari tuberculeux depuis 8 ans (c'est Eugénie !) – Italienne (la plus symp. de beaucoup) – Alice (la plus symp. de beaucoup) – Dubois (Oh, ma mère ! si tu me voyais !) – celle qui est malade, vit seule (qui m'a donné l'adresse de Puteaux) – décolleteuse qui chante – décolleteuse aux 2 gosses et au mari malade.\par
Mimi – 26 ans – mariée depuis 8 ans à un gars du bâtiment (connu à Angers), qui a fait 2 ans chez Citroën et est à présent chômeur, quoique bon ouvrier. Travaillait à Angers dans un tissage (11 F par jour !). Chez A. depuis 6 ans. A pris 6 mois à acquérir un rythme assez rapide pour « gagner sa vie » – au cours desquels elle a pleuré bien souvent, croyant qu'elle n'y arriverait jamais. A travaillé encore 1 an 1/2, quoique vite et bien, dans un état de nervosité perpétuelle (peur de mal faire). Au bout de 2 ans seulement est devenue assez sûre d'elle pour « ne pas s'en faire ».\par
Une de ses premières réflexions (je lui disais être exaspérée par l'ignorance de ce que je fais) : « On nous prend pour des machines... d'autres sont là pour penser pour nous... » (exactement le mot de Taylor, mais avec amertume).\par
Pas d'amour-propre professionnel. Cf. sa réponse le jeudi de la 6\textsuperscript{e} semaine.\par
Incomparablement moins vulgaire que la moyenne.\par
Nénette (M\textsuperscript{me} A., 35 ans environ (?). Fils de 13 ans, fille de 6 ans 1/2. Veuve. Plaisanteries et confidences à faire rougir un corps de garde forment presque toute sa conversation. Vivacité et vitalité extraordinaire. Bonne ouvrière – se fait presque tj. plus de 4 F. Dans la boîte depuis 2 ans.\par
{\itshape Mais} – respect immense pour l'instruction (parle de son fils « tj. en train de lire »).\par
Sa gaieté assez vulgaire disparaît la semaine où elle est presque tout le temps à l'arrêt. « Il faut compter sou par sou. »\par
\par
Dit de son fils : « L'idée de l'envoyer à l'atelier, je ne sais pas ce que ça me fait » (pourtant un observateur superficiel pourrait croire qu'elle est heureuse à l'atelier).\par
Joséphine.\par
Eugénie\par
Ouvriers :\par
{\itshape Le magasinier} (Pommera).\par
{\itshape Histoire} : né à la campagne – famille de 12 enfants – gardait les vaches à 9 ans – a attrapé son certificat d'études à 12 ans. N'a jamais travaillé en usine avant la guerre – travaillait dans des garages – n'a jamais fait d'apprentissage, ni eu d'autre culture technique ou générale que celle qu'il s'est donnée dans les cours du soir. A fait la guerre (déjà marié) dans les chasseurs alpins, comme chef de section (?). A perdu à ce moment les quelques sous qu'il avait ramassés, et a dû en conséquence travailler en usine en rentrant. J'ignore ce qu'il a fait les 4 premières années. Mais, après, il a été 6 ans régleur aux presses, dans une autre boîte. Et les 6 dernières années, magasinier du magasin des outils à l’Alst. Partout, dit-il, il a été bien tranquille. Néanmoins, il ne me souhaite pas de rester dans les machines aussi longtemps que lui.\par
{\itshape Travail} :\par
Donne les outils marqués sur la commande (ça, n'importe qui pourrait le faire).\par
Modifie parfois la commande, en indiquant d'autres outils permettant de remplacer, par exemple, 3 opérations par 2, d'où économie pour la maison. Ça lui est arrivé à plusieurs reprises. (Il faut être rudement sûr de soi !) Aussi a-t-il la sécurité que comporte la conscience d'être un homme précieux, et que personne n'oserait embêter.\par
{\itshape Culture} :\par
Technique : connaît le tour – la fraiseuse – l'ajustage. Explique merveilleusement bien comment il faut s'y prendre (à la différence des régleurs).\par
Générale ? S'exprime fort bien. Mais quoi d'autre ?\par
Violoniste – grand blond – gars du four – lecteur de l'Auto – gentil type du perçage – petit gars qui m'a mise au four – jeune Italien – mon « fiancé » – type en gris de la cisaille – jeune cisailleur. Bretonnet – nouveau manœuvre – gars du transport aérien – équipe de 2 à la réparation des machines (...) [machine à Biol, machine à Ilion].\par

\tableopen{}
\begin{tabularx}{\linewidth}
{|l|X|X|}
\hline Solidarité ouvrière ? pas de solidarité anonyme (ex. Louisette...). \\
Leur donner le sentiment qu'ils ont quelque chose d'eux-mêmes à donner. \\
Délégués ouvriers, sécurité contre menace de renvois. \\
Attributions ? \\
Sécurité. \\
Organisation du chômage partiel. \\
Revendications.  & \{ &  Contrôle ouvrier sur la comptabilité ? \\
Journal avec comptes ? \\
Innovations techniques et d'organisation ? \\
Conférences ? \\
Primes contre gaspillage ?  \\
\hline
\end{tabularx}
\tableclose{}

\noindent Éloges.\par
Ces soucis supplémentaires comment ?...\par

\tableopen{}
\begin{tabularx}{\linewidth}
{|l|X|X|X|X|}
\hline 2 boîtes de suggestion. \\
Vulgarisation, préparer...  &  &  1 pour le {\itshape bien de la maison.} \\
1 pour le {\itshape bien des ouvriers.}  &  &  Innovations \\
Techniques. Gaspillage.  \\
\hline
\end{tabularx}
\tableclose{}

\noindent Raconter l'incident bureaucratique \footnote{Voir Journal d'Usine, p. 62.} … Liaison.\par
« Piège à capitalistes » : renouvellement de l'outillage. L'un renouvelle un outillage amorti ; les autres doivent en faire autant, quoique non amorti (parce qu'on calcule le prix de revient particulier, non général). La fois suivante, le premier pâtit à son tour...\par
Naïveté d'un homme qui n'a jamais souffert...
\subsection[À la recherche de l’embauche]{À la recherche de l’embauche}
\noindent \par
{\itshape Lundi.} – Seule. À Issy – Malakoff. Ennuyeux – rien à signaler.\par
{\itshape Mardi} – (sous la pluie) – avec une ouvrière (me parle de son garçon de 13 ans qu'elle laisse à l'école. « Sans ça qu'est-ce qu'il peut devenir ? Un martyr comme nous autres »).\par
{\itshape Mercredi} – (temps divin) avec 2 ajusteurs. Un de 18 ans. Un de 58. Très intéressant, mais fort réservé. Un homme, selon toute apparence. Vivant seul (sa femme l'a plaqué). Un « violon d'Ingres », la photo. « On a tué le cinéma en le rendant parlant, au lieu de le laisser ce qu'il est véritablement, la plus belle application de la photographie. » Souvenirs de guerre, sur un ton singulier, comme d'une vie pareille à une autre, un boulot seulement plus dur et plus dangereux (artilleur, il est vrai). « Celui qui dit qu'il n'a jamais eu peur ment. » Mais lui ne semble pas avoir subi la peur au point d'en avoir été intérieurement humilié. Sur le travail – « on demande de plus en plus aux professionnels, depuis quelque temps ; il faudrait presque des connaissances d'ingénieur ». Me parle des « développés ». Il faut trouver les dimensions de la tôle plate dont on fera ensuite une pièce pleine de courbes et de lignes brisées.\par
[{\itshape Tâcher de savoir de la manière la plus précise ce que c'est qu'un développé}.]\par
Une fois, il a raté un essai, autant que j'ai compris parce qu'il avait oublié de multiplier le diamètre par π.\par
À son âge, dit-il, on a le dégoût du travail (ce travail auquel, étant jeune, il s'intéressait avec passion). Mais il ne s’agit pas du travail même, il s'agit de la subordination. La tôle... « Il faudrait pouvoir travailler pour soi. » « Je voudrais faire autre chose. » Travaillait (aux « Mureaux »), mais s'attend à moitié à être viré, Pour avoir coulé des bons (il est au temps). Se plaint des bureaux des temps. « Ils ne peuvent pas se rendre compte. » Discussion avec le contremaître, pour des pièces à faire en 7 mn ; il en met 14 ; le contremaître, pour lui montrer, en fait une en 7, mais, dit-il, mauvaise – (c'est donc de l'ajustage en série ?).\par
Parle de ses boulots passés. Des planques. A été mécanicien dans un tissage. « Ça, c'est le rêve. » Passait son temps à « faire de la perruque ». N'a même pas perçu, de toute évidence, le sort misérable des esclaves. Affecte un certain cynisme. Pourtant, de toute évidence, homme de cœur.\par
Toute la matinée, conversation à 3 extraordinairement libre, aisée, d'un niveau supérieur aux misères de l'existence qui sont la préoccupation dominante des esclaves, surtout des femmes. Après l'Alsthom, quel soulagement !\par
Le petit est intéressant aussi. En longeant Saint-Cloud, il dit : « Si j'étais en forme (il ne l'est pas, hélas, parce qu'il a faim...) je dessinerais. » « Tout le monde a quelque chose à quoi il s'intéresse. » « Moi, dit l'autre, c'est la photo. » Le petit me demande : « Et vous, quelle est votre passion ? » Embarrassée, je réponds : « La lecture. » Et lui – « Oui, je vois ça. Pas des romans. Plutôt philosophique, n'est-ce pas ? » On parle alors de Zola, de Jack London.\par
Tous deux, de toute évidence, ont des tendances révolutionnaires (mot très impropre – non, plutôt ils ont une conscience de classe, et un esprit d'hommes libres). Mais quand il s'agit de défense nationale, on ne s'entend plus. D'ailleurs je n'insiste pas.\par
Camaraderie totale. Pour la 1\textsuperscript{re} fois de ma vie, en somme. Aucune barrière, ni dans la différence des classes (puisqu'elle est supprimée), ni dans la différence des sexes. Miraculeux.
\subsection[Dimanche de Pâques]{Dimanche de Pâques}
\noindent \par
En revenant de l'église où j'avais espéré (sottement) entendre du chant grégorien, je tombe sur une petite exposition où on aperçoit un métier de Jacquard {\itshape en march}e. Moi qui l'avais si passionnément, si vainement contemplé au Cons. des Arts et Métiers, je m'empresse de descendre. Explications de l'ouvrier, qui voit que je m'intéresse (en sortant, 2 tournées Claquesin... je l'intrigue beaucoup !). Il fait tout : carton (d'après dessin {\itshape du carton}, non de l'étoffe – il saurait, dit-il, trouver lui-même le dessin du carton (?) et, aussi, lire sur le carton le dessin de l'étoffe (?) ; cependant, quand je lui demande s'il saurait lire sur le carton des lettres à tisser dans l'étoffe, il dit – et encore avec hésitation – que oui, mais pas couramment). Montage de la machine (ce qui signifie disposer tous les fils, sans erreur – travail excessivement minutieux) – et tissage, accompli en lançant la navette et en pédalant ; pédale lourde à cause de toutes les aiguilles et tous les fils soulevés, mais il dit n'être jamais fatigué. J'ai enfin compris – à peu près – le rapport du carton, des aiguilles et du fil. Il y a, dit-il, un métier Jacquard dans chaque tissage, pour les échantillons ; mais il pense que ça ,va disparaître. Excessivement fier de son savoir...
\subsection[Deuxième boîte, du jeudi 11 avril au mardi 7 mai, Carnaud, Forges de Basse-Indre, rue du Vieux-Pont de Sèvres, Boulogne-Billancourt]{Deuxième boîte, du jeudi 11 avril au mardi 7 mai, Carnaud, Forges de Basse-Indre, \\
rue du Vieux-Pont de Sèvres, Boulogne-Billancourt}
\noindent \par
1\textsuperscript{re} JOURNÉE. – Atelier de Gautier : bidons d'huile [après, masques à gaz] (ateliers strictement spécialisés). Des chaînes et quelques presses. On me met à une presse.\par
Pièces    \begin{figure}[htbp]
\noindent\noindent\includegraphics[]{}\end{figure}
    à emboutir pour en faire. \begin{figure}[htbp]
\noindent\noindent\includegraphics[]{}\end{figure}
\par
Le point sert à déterminer le sens – petite presse, pédale douce ; c'est ce point qui me gêne. Il faut compter (ignorant quel est le contrôle, je compte consciencieusement ; à tort). Je les range dans l'ordre et les compte par 50, puis les fais en vitesse. Je force, quoique non au maximum, et fais 400 à l'h. Je travaille plus dur qu'en général à l'Alsthom. L'après-midi, fatigue, augmentée par l'atmosphère étouffante, chargée d'odeurs de couleurs, vernis, etc. Je me demande si je pourrai maintenir la cadence. Mais à 4 h Martin, contremaître (un beau gars à l'air et à la voix affables), vient me dire bien poliment : « Si vous n'en faites pas 800, je ne vous garderai pas. Si vous en faites 800 les 2 h qui restent, je {\itshape consentirai peut-être} à vous garder. Il y en a qui en font 1.200. » Je force, la rage au cœur, et j'arrive à 600 l'h (en trichant un peu sur le compte et le sens des pièces). À 5 h 1/2 Martin vient prendre le compte et dit : « Ce n'est pas assez. » Puis il me met à ranger les pièces d'une autre, laquelle n'a pas un mot ni un sourire d'accueil. À 6 h, en proie à une rage concentrée et froide, je vais dans le bureau du chef d'atelier, et, demande carrément – « Est-ce que je dois revenir demain matin ? » Il dit, assez étonné : « Revenez toujours, on verra ; mais il faut aller plus vite. » Je réponds : « Je tâcherai », et pars. Au vestiaire, étonnement d'entendre les autres caqueter, jacasser, sans paraître avoir au cœur la même rage que moi. Au reste, le départ de l'usine se fait en vitesse ; jusqu'à la sonnerie, on travaille comme si on en avait encore pour des heures ; la sonnerie n’a pas encore commencé à retentir que toutes se lèvent comme mues par un ressort, courent pointer, courent au vestiaire, enfilent leurs affaires en échangeant quelques mots, courent chez elles. Moi, malgré ma fatigue, j'ai tellement besoin d'air frais que je vais à pied jusqu'à la Seine ; là je m'assieds au bord, sur une pierre, morne, épuisée et le cœur serré par la rage impuissante, me sentant vidée de toute ma substance vitale ; je me demande si, au cas où je serais condamnée à cette vie, j'arriverais à traverser tous les jours la Seine sans me jeter une fois dedans.\par
Le lendemain matin, de nouveau sur ma machine. 630 à l'h, en bandant désespérément toutes mes forces. Tout d'un coup Martin, qui s'approche suivi de Gautier, me dit : « Arrêtez. » Je m'arrête, mais reste assise devant ma machine sans comprendre ce qu'on me veut. Ce qui me vaut une engueulade, car, quand un chef dit : « Arrêtez », il faut, paraît-il, être immédiatement debout à ses ordres, prête à bondir [sur] le nouveau travail qu'il va vous indiquer. « On ne dort pas ici. » (Effectivement, pas une seconde, dans cet atelier, en 9 h par jour, qui ne soit pas une seconde de travail. Je n'ai pas vu une fois une ouvrière lever les yeux de sur son travail, ou deux ouvrières échanger quelques mots. Inutile d'ajouter que dans cette boîte les secondes de la vie des ouvrières sont la seule chose qu'on économise aussi précieusement ; par ailleurs gaspillage, coulage à revendre. Aucun chef que j'aie vu analogue à Mouquet. Chez Gautier, leur travail semble consister surtout à pousser les ouvrières.) On me met à une machine où il s'agit seulement d'enfiler de minces bandes métalliques flexibles, dorées dessous, argentées dessus, en faisant attention de ne pas en mettre 2 à la fois, et « à toute allure ». Mais souvent elles sont collées. La 1\textsuperscript{re} fois que j'en mets 2 (ce qui arrête la machine), le régleur vient l'arranger. La 2\textsuperscript{e} fois j'avertis Martin, qui me remet à ma 1\textsuperscript{re} machine pendant qu'on arrange l'autre. 640 à l'h à peu près... À 11 h, une femme vient m'emmener avec un gentil sourire dans un autre atelier ; on me met dans une grande salle claire, à côté de l'atelier, où un ouvrier montre à un autre comment vernir au pistolet pneumatique...\par
[J'ai oublié de noter mon impression le 1\textsuperscript{er} jour, à 8 h, en arrivant au bureau d'embauche. Moi malgré mes craintes – je suis heureuse, reconnaissante à la boîte, comme une chômeuse enfin casée. Je trouve 5 ou 6 ouvrières qui m'étonnent par leur air morne. J'interroge, on ne dit pas grand-chose ; je comprends enfin que cette boite est un bagne (rythme forcené, doigts coupés à profusion, débauchage sans scrupules) et que la plupart d'entre elles y ont travaillé – soit qu'elles aient été jetées sur le pavé à l'automne, soit qu'elles aient voulu s'évader – et reviennent la rage au cœur, rongeant leur frein.]\par
La porte ouvre 10 mn avant l'heure. Mais c'est une façon de parler. Avant, une petite porte est ouverte dans le portail. À la 1\textsuperscript{re} sonnerie (il y en a 3 à 5 mn d'int.), la petite porte se ferme et la moitié du portail s'ouvre. Les jours de pluie battante, spectacle singulier de voir le troupeau des femmes arrivées avant que ça « ouvre » rester debout sous la pluie à côté de cette petite porte ouverte en attendant la sonnerie (cause, les vols ; cf. réfectoire). Aucune protestation, aucune réaction.\par
Une belle fille, forte, fraîche et saine dit un jour au vestiaire, après une journée de 10 h : On en a marre de la journée. Vivement le 14 juillet qu'on danse. Moi : Vous pouvez penser à danser après 10 h de boulot ? Elle : Bien sûr ! Je danserais toute la nuit, etc. (en riant). Puis, sérieusement – ça fait 5 ans que je n'ai pas dansé. On a envie de danser, et puis on danse devant la lessive.\par
Deux ou trois, mélancoliques, à sourire triste, ne sont pas de la même espèce vulgaire que les autres. Une me demande comment ça va. Je lui dis que je suis dans un coin tranquille. Elle, avec un sourire doux et mélancolique : Tant mieux ! Espérons que ça durera. Et de même encore une ou deux fois.\par
Ceux qui souffrent ne peuvent pas se plaindre, dans cette vie-là. Seraient incompris des autres, moqués peut-être de ceux qui ne souffrent pas, considérés comme des ennuyeux par ceux qui, souffrant, ont bien assez de leur propre souffrance. Partout la même dureté que de la part des chefs, à de rares exceptions près.\par
Au vernissage. Observé 5 ouvriers. Le charpentier – mon copain camionneur – le « type d'en bas » (étamage), à moitié chef d'équipe. L'électricien, ancien inscrit maritime (dont le passage était pour moi et mon copain comme un souffle du large). Le mécano (hélas, à peine aperçu).\par
[Remarque : séparation des sexes, mépris des hommes pour les femmes, réserve des femmes à l'égard des hommes (malgré les échanges de plaisanteries obscènes) bien plus prononcés chez les ouvriers qu'ailleurs.]\par
Ouvrières : L'ancienne découpeuse qui, il y a 7 ans (28 : pleine prospérité) a eu une salpingite, et n'a pu obtenir d'être retirée des presses qu'au bout de plusieurs années – le ventre dès lors complètement et définitivement démoli. Parle avec beaucoup d'amertume. Seulement, elle n'a pas eu l'idée de changer de boîte – alors qu'elle le pouvait facilement !
\subsection[Pour la 2e fois, à la recherche du boulot]{Pour la 2\textsuperscript{e} fois, à la recherche du boulot}
\noindent \par
Débauchée le mardi 7 mai. Mercredi, jeudi, vendredi passés dans la prostration sinistre que donnent les maux de tête. Vendredi matin, juste courage de me lever à temps pour téléphoner à Det. Samedi, dimanche, repos.\par
{\itshape Lundi} 13. – Devant chez Renault. Conversation entendue entre 3, que je prends d'abord pour des professionnels. Un, qui écoute d'un air malin (physionomie fine) sera pris, donc pas revu. – Vieil ouvrier, man. spécialisé sur presses, figure tannée de travailleur – mais intelligence dégradée par l'esclavage. Communiste vieux modèle. Ce sont les patrons qui font les syndicats confédérés. Ils choisissent les chefs. Ceux-ci, quand ça ne va pas, vont dire au patron : « Je ne pourrai plus les retenir si... Il y en a un qui me l'a dit lui-même ! » Et après disent aux ouvriers : « Les grèves ne réussissent pas lorsqu'il y a du chômage, vous allez souffrir », etc.. Bref, ressasse toutes les bêtises inventées par des bonzes bien planqués.\par
Le 3\textsuperscript{e}, gars du bâtiment, tendances syndicalistes (a travaillé à Lyon), un chic type.\par
{\itshape Mardi 14}. – Matin : inertie. Après-midi, Saint-Ouen (Luchaire). La place est prise...\par
{\itshape Mercredi 15}. – Vais porte de Saint-Cloud, mais le temps de téléphoner à Det., il est trop tard pour aller chez Renault ou chez Salmson. Vais voir chez Caudron. Devant la porte, une demi-douzaine de professionnels, tous avec références aviation : menuisiers d'aviation, ajusteurs... De nouveau le même refrain : « Des professionnels comme ils en demandent, ils n'en trouveront pas. On n'en fait plus... » Il s'agit encore de la même chose : les développés. Autant que je peux comprendre, il y a 2 types d'essai, la « queue d'arrondi » : \begin{figure}[htbp]
\noindent\noindent\includegraphics[]{}\includegraphics[]{}\end{figure}
 (à peu près), qui doit s'emboîter {\itshape exactement} dans une tôle qu'on défend de limer, et les développés. Il y a quelque chose d'artiste, semble- t-il, chez les ajusteurs.\par
Celui avec lequel je me suis liée. En apparence, brute épaisse. Certificats mirifiques. Une lettre de recommandation du Conservatoire des Arts et Métiers (où il a été apprenti jusqu'à 19 ans) : « Mécanicien qui fait honneur à son métier. » Habite Bagnolet (une bicoque à lui ??), ce qui complique pour lui la recherche du boulot ; explique ainsi son refus de faire plus de 8 h, mais je crois que ce n'est pas seulement ça. On fait 10 h chez Renault. Trop pour lui. Avec le train, etc., « le dimanche on peut rester couché pour se reposer » (donc les sous, ça lui est égal). Ajoute : « 5 h, c'est assez pour moi. » À été contremaître plus d'une fois (certificats à l'appui). Mais, me dit-il, je suis trop révolutionnaire, je n’ai jamais pu embêter les ouvriers. » Son erreur d’interprétation à mon égard, son attitude après. En me quittant : « Vous ne m'en voulez pas ? » Doit venir me voir chez moi. Mais pas devant Renault le lendemain matin... Le jour d'après, on frappe. Je suis couchée, n'ouvre pas. Était-ce lui ? Je n'entendrai plus jamais parler de lui...\par
Autre jour, devant Gévelot – le type à cheveux blancs, qui avant la guerre se destinait à la musique. Se dit comptable (mais se trompe dans calculs élémentaires) – cherche place de manœuvre. Pitoyable raté... On attend de 7 h 1/4 à 7 h 3/4 sous un peu de pluie, après quoi « pas d'embauche ». Chez Renault, embauche finie. Une heure d'attente devant Salmson.\par
Autre fois, chez Gévelot. On fait entrer les femmes. Grossièreté, dureté du type de l'embauche (chef du personnel ?) lequel engueule d'ailleurs aussi un contremaître qui répond bien humblement (plaisir de voir ça). Nous parcourt du regard comme des chevaux. « Celle-là la plus costaude. » Sa manière d'interroger la gosse de 20 ans qui 3 ans plus tôt a quitté parce qu'enceinte... Avec moi, poli. Prend mon adresse.\par
Celle qui, mère de 2 enfants, disait vouloir travailler parce qu'elle « s'ennuyait à la maison » et dont le mari travaillait 15 h par jour et ne voulait pas qu'elle travaille ! Indignation d'une autre, mère de 2 enfants aussi, bien malheureuse de devoir travailler (devant Salmson).\par
Autre fois (?) rencontre la petite qui dit : « La baisse du franc, ça sera la famine, on l'a dit à la T. S. F. », etc.\par
Autre fois course à Ivry. « Pas de femmes. » Maux de tête...\par
Autre fois, devant Langlois (petite boîte), à Ménilmontant, à 7 h (annonce) – Attends j. 8 h 1/2. Puis, à Saint-Denis, mais c'est trop tard.\par
Retourne à Saint-Denis. Pénible de marcher ainsi quand on ne mange pas...\par
De nouveau chez Luchaire à Saint-Ouen avant 7 h 1/2 (c'est le jour même où, l'après-midi, je serai embauchée chez Renault).\par
La dernière semaine je décide de ne dépenser que 3,50 F p. jour, communications comprises. La faim devient un sentiment permanent. Est-ce plus ou moins pénible que de travailler et de manger ? Question non résolue... Si, plus pénible somme toute.
\subsection[Renault]{Renault}
\noindent \par
Fraiseuse.\par
{\itshape Mercredi 5.} – Jour de l'embauche, de 1 h 1/2 à 5 h. Les visages autour de moi ; le jeune et bel ouvrier ; le gars du bâtiment ; sa femme.\par
Émotions terribles, le jour de l'embauche, et le lendemain en allant affronter l'inconnu ; dans ce métro matinal (j'arrive à 6 h 3/4), l'appréhension est forte jusqu'au malaise physique. Je vois qu'on me regarde ; je dois être fort pâle. Si jamais j'ai connu la peur, c'est ce jour-là. J'ai dans l'esprit un atelier de presses, et 10 h par jour, et des chefs brutaux, et des doigts coupés, et la chaleur, et les maux de tête, et... L'ancienne ouvrière sur presses avec qui j’ai causé au bureau d'embauche n'a pas contribué à m'encourager. En arrivant à l'atelier 21, je sens ma volonté défaillir. Mais du moins ce ne sont pas des presses – quelle chance !\par
Quand j'avais, 3 mois plus tôt, entendu raconter l'histoire de la fraise qui avait traversé la main d'une ouvrière, je m'étais dit qu'avec une image pareille dans la mémoire il ne me serait pas facile de travailler jamais sur une fraiseuse. Cependant, à cet égard, je n'ai eu de peur à surmonter à aucun moment.\par
{\itshape Jeudi 6}. – De 8 h à 12 h, regardé \footnote{Simone Weil faisait alors partie de l'équipe qui travaillait de 14 h 1/2 à 22 h.} – de 2 h 1/2 à 10 h, travaillé. 400 les 2 premières h. En tout 2.050, perdu 1 h 1/2 ou plus par la faute du régleur. Épuisée en sortant.\par
L'inconvénient d'une situation d'esclave, c'est qu'on est tenté de considérer comme réellement existants des êtres humains qui sont de pâles ombres dans la caverne. Ex. : mon régleur, ce jeune salaud. Réaction nécessaire là-dessus. [Ça m'a passé, après des semaines.]\par
Idée de Dickmann. Mais si les ouvriers se font d'autres ressources, et par un travail {\itshape libre}, se résigneront-ils à ces vitesses d'esclaves ? (Sinon, tant mieux !)\par
Ceux qui me disent de ne pas me crever. C'est (je l'apprends plus tard) le contremaître d'une autre équipe, tout au bout de l'atelier. Très gentil celui-là, d'une bonté positive (alors que celle de Leclerc, mon chef à moi, provient plutôt du je-m'en-foutisme). Depuis, aux rares occasions que j'ai de lui parler, toujours particulièrement gentil avec moi. Un jour, il me regarde en passant, alors que je transvase misérablement des gros boulons dans une caisse vide, avec les mains... Ne jamais oublier cet homme.\par
\par
Le contremaître et la manivelle. Il me dit : « Essayez comme ça », alors qu'il est évident qu'elle va s'en aller.\par
{\itshape Vendredi 7.} – 2.500 juste, épuisée, plus encore que la veille (surtout après 7 h 1/2 !). Philippe me regarde en rigolant... À 7 h n'en ai fait que 1.600.\par
La petite ds le métro « pas de courage ». – Moi non plus...\par
{\itshape Samedi 8}. – 2.400, nettoyage. Fatiguée, mais moins que la veille (2.400 en 8 h, soit 300 l'h seulement).\par
{\itshape Mardi 11}. – 2.250 dont 900 après 7 h – pas trop forcé – à peine fatiguée en sortant. Fini à – 10.\par
{\itshape Mercredi 12.} – Panne de courant (bonheur !).\par
{\itshape Jeudi 13.} – 2.240, fini à 9 h 1/2 (plus de pièces) – là-dessus, 1.400 avant 7 h, 840 après (dont 330 seulement à 4 h). Violents maux de tête. À plat en sortant. Mais plus de courbatures...\par
{\itshape Vendredi 14.} – 1.350 et 300 autres. Pas fatiguée.\par
{\itshape Samedi 15}. – 2.000, fini à 8 h 40, nettoyage : à peine le temps de finir. Pas trop fatiguée [cette 1\textsuperscript{re} semaine, question de caisse pas trop angoissante grâce à la gentillesse des autres]\par
[{\itshape Dimanche.} – Maux de tête, nuit de dimanche à lundi pas dormi.]\par
{\itshape Lundi 17}. – 2.450 (1.950 à 8 h 35) – fatiguée en sortant, mais non épuisée.\par
{\itshape Mardi 18}. – 2.300 (2.000 à 8 h 3/4) – pas forcé – pas fatiguée en sortant – mal à la tête tte la journée.\par
{\itshape Mercredi 19}. – 2.400 (2.000 à 8 h 35), très fatiguée. Le petit salaud de régleur me dit qu'il en faut plus de 3.000.\par
{\itshape Jeudi 20}. – Vais à la boîte avec un sentiment excessivement pénible ; chaque pas me coûte (moralement ; au retour, c'est physiquement). Suis dans cet état de demi-égarement où je suis une victime désignée pour n'importe quel coup dur... De 2 h 1/2 à 3 h 35, 400 pièces. De 3 h 35 à 4 h 1/4, temps perdu par le monteur à casquette – (il me refait mes loups) – Grosses pièces – lent et très dur à cause de la nouvelle disposition de la manivelle de l'étau. Ai recours au chef – Discussion – Reprends – Me fraise le bout du pouce (le voilà, le coup dur) – Infirmerie – Finis les 500 à 6 h 1/4 – Plus de pièces pour moi (je suis si fatiguée que j'en suis soulagée !). Mais on m'en promet. En fin de compte, je n'en ai qu'à 7 h 1/2 et seulement 500 (pour finir les 1.000). [Le type blond a bien peur que je ne me plaigne au contremaître.] À 8 h, 245. Fais les 500 gros, en souffrant beaucoup, en 1 h 1/2. 10 mn pour le montage. C'est une autre partie de la fraise qui fonctionne : ça va ! je fais 240 petits en 1/2 h exactement. Libre à 9 h 40. Mais gagné 16,45 F !!! (non, grosses pièces un peu plus payées). M'en vais fatiguée...\par
1\textsuperscript{er} repas avec les ouvrières (le casse-croûte).\par
Le monteur à casquette : « S'il touche à votre machine, envoyez-le promener... Il démolit tout ce qu'il touche... »\par
Il me donne ordre de transporter une caisse de 2.000 pièces. Je lui dis : « Je ne peux pas la bouger seule. – Débrouillez-vous. Ça n'est pas mon boulot. »\par
À propos des pièces qu'on me fait attendre, la commençante : « Le contremaître a dit que si on attendait, on devait prendre en compensation sur le salaire de celle qui vous fait attendre. »\par
{\itshape Vendredi 21}. – Levée très tard – prête seulement juste à temps. Vais à la boîte avec peine – mais, contrairement à ce qui était le cas les fois d'avant, peine bien plus physique que morale. Je crains pourtant de ne pouvoir en faire assez. De nouveau ce sentiment « tenons toujours le coup aujourd'hui... », comme à l'A. Il y a eu la veille 15 jours que je suis là ; et je me dis que sans doute je ne puis pas tenir plus de 15 jours...\par
Une fois là, j'ai 450 pièces à finir, puis 2.000 : ça va, rien à compter. Commence à 2 h 35 ; fais les 450 à 3 h 40. Puis continue, au rythme ininterrompu, en fixant mon attention sur la pièce, en maintenant en moi l'idée fixe « il faut... » Je crois qu'il y a trop peu d'eau ; perds beaucoup de temps à chercher le seau (qui était à sa place !). Puis je verse trop d'eau ; ça déborde ; faut en ôter, chercher sciure, balayer... Le type des tours automatiques m'aide gentiment. À 7 h 20, perds beaucoup de temps (1/4 h à 20 mn) à chercher boîte. J'en trouve enfin une, pleine de copeaux ; je vais la vider ; le régleur me donne l'ordre de la remettre. J'obéis. [Le lendemain, une du perçage m'apprend qu'elle était à sa femme, et dit: « Moi, je ne l'aurais pas remise. » Sympathique, le perçage ; groupe à part.] Tout au fond de l'atelier (21 B) j'en trouve une ; une ouvrière s'oppose à ce que je la prenne ; je cède encore (à tort !). Je renonce. Je continue, et, quand je n'en ai plus que 500 environ, les vide partie sur la machine, partie dans une sorte de panier pris dans la machine derrière moi, et mets les 1.500 pièces faites dans la caisse ainsi vidée : manutention assez longue et très pénible, sans aide. Enfin fini à 9 h 35. Vais en vitesse en chercher 75, histoire de battre un peu mon record. Donc : 2.525. Rentre r. Aug. C. Dors en métro. Acte de volonté distinct pour chaque pas. Une fois rentrée, très gaie. Couchée, lis j. 2 h du matin. Réveil à 7 h 1/4 (dents).\par
{\itshape Samedi 22}. – Temps magnifique. Matinée joyeuse. Ne pense à la boîte qu'en y allant ; alors, sentiment pénible (mais moins impression d'esclavage). L'autre n'était pas venue. Prends caisse de 2.000 (– les 75) [lourd !]. Commence à 2 h 3/4. À 3 h 3/4, en ai fait peut-être 425 (ce qui ferait 500). On me change de machine. Boulot facile et bien payé (3,20 F \%), mais fraise plus dangereuse. En fais 350 (soit 4,20 F). Fini vers 5 h 5. Perds 10 mn. Reviens à ma machine ; recommence à 5 h 1/4. Vitesse qui va de soi, sans obsession artificielle, sans forcer, en maintenant seulement le « rythme ininterrompu » ; ai fait 1.850 à 8 h 1/2 (soit 1.350 en 3 h, ou 450 l’h !). 1 en 8 sec. Repas gai (la « grosse » pourtant manque). Impression de détente : samedi soir, pas de chefs, laisser-aller... Tout le monde (sauf moi) s'attarde j. 10 h 25.\par
Retour – m'attarde devant la musique. Air frais, délicieux. Éveillée ds métro, encore du ressort pour marcher. Fatiguée-pourtant. Mais en somme heureuse...\par
{\itshape Lundi 24.} – Mal dormi (démangeaisons). Matin, pas d'appétit, maux de tête assez violents. Sentiment de souffrance et d'angoisse au départ.\par
En arrivant, catastrophe : ma coéquipière n'étant pas venue, on a volé la caisse où tombent les pièces. Je perds 1 h à en trouver une autre (il faut une percée). Me mets au boulot ; fraise usée. Un nouveau régleur (en gris), dans l'atelier depuis 1 semaine, me la remplace (de lui-même !). À cette occasion, il s'aperçoit qu'il y a du jeu un peu partout. Notamment la bague qui maintient la fraise « a foutu le camp depuis au moins dix ans ». Il s'étonne que « les deux copains (!) » ne l'aient pas remise. Ma machine est un « vieux clou », dit-il. Il semble connaître un peu son affaire. Mais, total, je me mets au boulot à 4 h 1/2. Découragée, à plat (maux de tête). Fais 1.850 pièces en tout (en 5 h, soit pas 400 l'h). Le soir, je perds encore du temps à chercher une caisse puis, n'en trouvant pas, à transvaser pièces ds panier pris à machine d'à côté. Et combien lourde à manier, la caisse où sont tombées près de 16.000 pièces, qu'il faut verser dans une autre. Rentre (A. C.) fatiguée, mais pas trop. Dégoûtée surtout d'avoir fait si peu. Et mourant de soif.\par
{\itshape Mardi 25}. – Éveillée à 7 h. Séance fatigante et longue chez dentiste – mal aux dents tte la matinée. Presque en retard. Chaud. Ai peine à monter l'escalier en arrivant... Trouve ma nouvelle coéquipière (Alsacienne). Encore caisse à chercher... En prends une près d'une machine. La propriétaire arrive, furieuse. Prends à la place celle où étaient les pièces à faire, en la vidant (restaient 200). On n'est donc pas plus avancé ! En trouve une autre. Vais la remplir, par pelletées, au tour. La ramène (lourd !). Puis (à 2 h 55) vais à l'infirmerie – (copeau provoquant début d'abcès). Au retour, trouve mes 2.000 pièces vidées près de ma machine (la caisse reprise par la 1\textsuperscript{re} propriétaire en mon absence). Nouvelles recherches. M'adresse au chef en face l'ascenseur. Me dit : « Je vais vous en faire donner une. » J'attends. M'engueule parce que j'attends. Retourne à ma machine.\par
Mon voisin me donne une caisse. À ce moment survient mon chef (Leclerc). Commence à m'engueuler. Je lui dis qu'on a versé mes pièces en mon absence. Va s'expliquer avec mon voisin. Je ramasse les pièces. Changer la fraise. Total : me mets au boulot à 4 h 5 ! Avec un dégoût que je refoule afin d'aller vite. Je voudrais faire quand même 2.500. Mais j'ai peine à soutenir la vitesse. Les 200 restant de l'autre carton filent vite (en 20 ou 25 mn), Après, ça ralentit.\par
Effets de ce système à plusieurs régleurs : vers 6 h 1/2, coupe mal. Le régleur en gris déplace la fraise – la manipule – la redéplace – et, je crois bien, la remet à sa place primitive... À 7 h, j'ai dû en faire 1.300, pas plus. Après pause, encore recherche de caisse, manipulations, faute de caisse. À 9 h 35 ou 40 ai fini carton (donc 2.200). En fais encore 50... Je l'avais fait déplacer à 9 h 1/4 par le jeune régleur (Philippe) ; il m'avait bien fait attendre 1/4 h. Et, déjà, je l'avais appelé trop tard. 2.250, par conséquent. Médiocre... En rentrant, dois forcer pour marcher, mais pourtant non pas par pas.\par
N'ai pas maintenu le « rythme ininterrompu. » Gênée par mon doigt. Aussi trop grande confiance.\par
Faut absolument stabiliser la question des caisses. Et d'abord proposer à l'ouvrière du tour de nous en donner une 1 fois sur 2 ? On ne lui en donne jamais, dit-elle. Mais à nous non plus. Quand on en cherchait par 500, c'était différent. Maintenant que c'est par 2.000...\par
{\itshape Mercredi 26}. – Fatigue, le matin – courage pr pas beaucoup plus que la journée... accablement sourd – mx de tête – découragement – peur, ou plutôt angoisse (devant travail, ma caisse, vitesse, etc.) – lourd temps d'orage.\par
Vais à l'infirmerie. « On vous l'ouvrira quand il faudra, et sans vous demander votre avis. » Travail. Souffre du bras, de l'épuisement, des maux de tête. (Un peu de fièvre ? Pas le soir en tout cas.) Mais j'arrive à force de vitesse à ne pas souffrir pendant des espaces de temps successifs de 10 mn (à) 1/4 d'h. À 5 h paye. Après, j'en ai marre. Je compte mes pièces, essuie ma machine et demande à m'en aller. Vais trouver Leclerc (contremaître) dans le bureau du chef d'atelier, qui me propose l'assurance.\par
Attends 1/2 h devant ce bureau, par la faute de la pointeau. Vois les complications des livraisons. La camaraderie entre les contremaîtres...\par
En sortant de chez dentiste (mardi matin, je crois – ou plutôt jeudi matin), et en montant dans le W, réaction bizarre. Comment, moi, l'esclave, je peux donc monter dans cet autobus, en user pour mes 12 sous au même titre que n'importe qui ? Quelle faveur extraordinaire ! Si on m'en faisait brutalement redescendre en me disant que des modes de locomotion si commodes ne sont pas pour moi, que je n'ai qu'à aller à pied, je crois que ça me semblerait tout naturel. L'esclavage m'a fait perdre tout à fait le sentiment d'avoir des droits. Cela me paraît une faveur d'avoir des moments où je n'ai rien à supporter en fait de brutalité humaine. Ces moments, c'est comme les sourires du ciel, un don du hasard. Espérons que je garderai cet état d'esprit, si raisonnable.\par
Mes camarades n'ont pas, je crois, cet état d'esprit au même degré : ils n'ont pas pleinement compris qu'ils sont des esclaves. Les mots de juste et d'injuste ont sans doute conservé jusqu'à un certain point un sens pour eux – dans cette situation où tout est injustice.\par
{\itshape Jeudi 4 juillet}, – Retourne pas à ma fraiseuse, grâce ciel ! (Occupée par une autre qui a l'air d'en faire, d'en faire...) Petite machine à ébavurer les trous percés ds pas de vis. 2 espèces de pièces (1, 2\textsuperscript{e} : des clous). 1.300 de la première (1,50 F \%), 950 (?) de la 2\textsuperscript{e} (0,60 F Puis 260 pièces polies au ruban à polir (1 F \%).\par
{\itshape Vendredi 5 juillet}. – Le lendemain, congé : quel bonheur ! Mal dormi (dents). Matin, séance chez dentiste. Maux de tête, épuisement [inquiétude aussi, ce qui n'arrange pas les choses...]. Plus que 3 semaines ! Oui, mais 3 semaines, c'est n fois 1 jour ! Or, plus de courage que pour 1 jour, 1 seul. Et encore, en serrant les dents avec le courage du désespoir. La veille, le petit Italien m'a dit : « Vous maigrissez (il me l'avait dit 10 jours avant), vous y allez trop souvent (!). » Tout ça, c'est mes sentiments avant d'aller au boulot.\par
À bout de forces, en voyant mes voisines (machine à fendre les têtes...) se préparer à laver la machine, et à leur instigation, je vais demander à Leclerc si je pars à 7 h. Il me répond sèchement : « Vous n'allez pas venir pour faire 2 h, tout de même ! » Philippe le soir me fait attendre je ne sais combien de temps, pour m'embêter. Mais moi, saisie par le dégoût...\par
On dirait que, par convention, la fatigue n'existe pas... Comme le danger à la guerre, sans doute.\par
Semaine suivante : lundi 8 à vendredi 12.\par
{\itshape Lundi, mardi}. – Commencé à 7 h carton de 3.500 pièces (laiton ?).\par
{\itshape Mercredi}. – 8.000 pièces ou à peu près dans ma journée : fini le carton de la veille (à 10 h 45). Fais carton de 5.000 (recommencé à 11 h 45). Termine à 6 b [dîner avec A]. Épuisée. C'étaient des pièces faciles (je ne sais plus au juste lesquelles ; laiton, puis acier, je crois). « Rythme ininterrompu. »\par
{\itshape Jeudi}. – Brisée, crevée par l'effort de la veille, vais très lentement.\par
{\itshape Vendredi}. – Filet. Femme de l'Italien.\par
Soir : réunion R. P. – Louzon ne me reconnaît pas. Dit que j'ai changé de figure. « Air plus costaud » –\par

\begin{center}
Incidents notables\end{center}
\noindent \par
Le régleur en gris (Michel) et son mépris pour les 2 autres, surtout le « ballot ».\par
\par
Le mauvais montage qui casse la fraise ; incidents. Le monteur ballot avait mis un montage qui n'allait qu'à moitié. En appuyant sur la fraise, il arrive plusieurs fois que la fraise s'arrête. Une fois déjà ça m'était arrivé, et on avait dit : « Elle n'est pas assez serrée. » Je vais donc chercher le régleur, et lui demande de serrer plus. D'abord, il ne veut pas venir. Il me dit que c'est moi qui appuie trop. Enfin il vient. Dit : « Ce n'est pas là (en montrant les blocs de serrage de la fraise) mais là (en montrant la poulie de l'arbre porte-fraise et la courroie) que ça fatigue » (???). S'en va. Je continue. Ça ne va pas. Enfin une pièce se bloque dans le montage, casse 3 dents... Il va chercher Leclerc pour me faire engueuler. Leclerc l'engueule, lui, à cause du choix du montage, et dit que la fraise peut encore aller. 1/2 h après (ou 1/4) L. revient. Je lui dis : « Des fois la fraise s'arrête. » Il m'explique (sur un ton désagréable) que la machine n'est pas une forte machine, et que probablement j'appuie trop. Il me montre comment travailler – sans s'apercevoir qu'il va tout au plus au rythme de 600 à l'h, et encore ! (i. e. 2,70 F] (je ne peux pas le chronométrer...). Mais même de cette manière, la fraise ralentit au moment d'appuyer. Je le lui signale. Il dit que ça ne fait rien. Un moment vient où la fraise s'arrête tout à fait, et ne repart plus. J'appelle le régleur, qui s'apprête déjà à gueuler. Ma voisine dit : « C'est trop serré. » Une autre fois ça se produit encore ; la machine, en tournant, serre automatiquement l'arbre si un certain boulon qui doit le fixer n'est, lui, pas assez serré.\par
On desserre en tournant en sens inverse de la fraise.\par
Difficulté (pour moi) de penser la machine soit devant elle, soit loin d'elle...\par
Quelles peuvent être les causes pour lesquelles la fraise s'arrête ? (l'arbre aussi s'était-il arrêté ? Oublié de remarquer). Du jeu, ou dans la fraise, ou dans la pièce. (C'était le cas.) Une trop grande résistance, si on demande à la machine plus de travail qu'elle ne peut fournir (était-ce là ce que voulait dire le ballot ?) [mais qu'est-ce qui détermine cette puissance ?].\par
À étudier : notion de la puissance d'une machine.\par
Lettre de Chartier. Scie et rabot. Peut-être que pour la machine il en est autrement...\par
Chercher comment les machines tirent leur puissance du moteur unique. Si elles sont ordonnées par rangées en fortes et faibles ?\par
\par
{\itshape Mercredi 17}. – Rentrée – temps frais – moins de souffrance (morale) que je n'aurais pu craindre. Je me retrouve facile au joug...\par
Pas de boulot. Vais du côté de tours automatiques (Cuttat), que j'avais étudiés pendant les 4 jours de vacances.\par
j. 8 h 1/2, attends l'huile.\par
Vis 4 X 10 acier 7010105 | 041916 | fr. 1.\par
5.000 à 4,50 F, soit 23,50 F.\par
? petite série donnée par Leclerc, que Michel en ¾ h n'est pas encore arrivé à monter.\par
Il vient quand Michel peine sur le montage depuis 3/4 h. « Qui vous a donné ces pièces à faire ? » Je réponds : « Vous ! » Il est gentil. Me fait changer de pièces ; 3/4 h perdus, non payés ! Michel dit qu'on aurait pu le faire... Il les montera sur une autre machine (celle de la petite que je chine à son sujet). À ce propos, conversation avec lui sur Leclerc. S'y connaît-il, aux machines ? – certaines, d'autres non. Michel me raconte qu'il a été chef d'équipe 2 mois, débarqué parce que trop bon garçon ! – Mais celui-là n'est pas méchant » (que je dis). – Michel croit qu'il ne restera pas. Mais il y était quand la petite Espagnole est arrivée, il y a 1 an 1/2.\par
Vis C 4 X 8 acier (7010103) 043408 | fr. 1.\par
5.000 à 4,50 F, m. 1 F, soit 23,50 F.|\par
Je ne les finis pas.\par
{\itshape Jeudi 18}. – Finis les C 4 X 8.\par
Vis laiton (740657 {\itshape bis} | | 1417 (!), gde scie spéciale : 127 | 2).\par
100 (!!!) à 0,0045, soit 1,45 F.\par
Ajustage laiton | 6005346 | 027947, 1 fr. 1,5 (?).\par
600 à 0,045, soit 2,25 F + 0,45 F = 2,70 F (fraise à l'envers !).\par
Nouveau régl. (manœuvre spécialisé ? à vérifier). Il demande « à quoi ça sert », me fait chercher le dessin, ce qui dure longtemps et n'aide guère...\par
Gagné ces 2 jours (18 h) 23,50 F + 23,50 F + 1,45 F + 2,70 F = 51,15 F.\par
Pas 3 F ! 2,85 F ! Je toucherai ça et la semaine d'avant le 19, et le jeudi et vendredi d'avant (en tout 7 + 7 + 9 + 10 + 9 + 10 + 9 + 18 h = 79 h).\par
Les vis acier C 4 X 8, j'en fais d'abord un paquet de 1.000. Vais chez Goncher pour la suite : pas prêt. C'est tout juste s'il ne m'engueule pas (alors que je serais, moi, en droit de me plaindre). J'y retourne l'après-midi pour les 4.000 restants, mais les prends en 4 ou 5 fois et à chaque fois attends longtemps. Cela me donne l'occasion d'admirer les Cuttat... le jeune régleur a, je crois, fini par remarquer que je ne hais pas d'attendre ainsi.
\subsection[Incidents]{Incidents}
\noindent \par
Changement de régleur. Le gros incapable est parti mardi après-midi. (Savoir ce qu'il est devenu ?) Remplacé par un qui, paraît-il, vient d'une autre partie de l'atelier. Pas je m'en foutiste, celui-là. Nerveux, gestes fébriles, saccadés. Ses mains tremblent. Il me fait pitié. Il met 1 h à me faire un montage (pour 600 pièces !), et encore met la fraise à l'envers. (Ça marche quand même : cuivre, heureusement),\par
Essaye de faire montage moi-même – ignore le côté des bagues. (Elles sont composées de 2 cylindres creux de diamètre différent.) Je l'observerais facilement au prochain démontage... La vraie difficulté, c'est la faiblesse musculaire : je n'arrive pas à desserrer.\par
Conversation avec Michel. Compétence technique de Leclerc ? « Pour certaines machines, pas pour d'autres. » Pas ouvrier. Pas méchant – « sera viré » –\par
Il m'avait donné des pièces qui vont mal sur cette machine.\par
{\itshape Vendredi 19 juillet}. – Vis fixation gâche, acier, 7051634 | 054641 | 1 fr. 1,5.\par
1.000 à 5 F, soit 6 F (montage difficile à trouver, et encore peu satisfaisant).\par
7 bouchons (petits | 7050846 | 041784 | fr. 1,5.\par
3.000 à 5 F, soit 16 F | j'essaye de faire passer 3 montages, mais...).\par
Vis 5 x 22 (?) | 7051551 | 039660 | fr. 1,2.\par
550 (!) à 0,0045, soit 2,25 F + 0,235 F + 1 F = 3,50 F (à peu près).\par
Vis fixant couronne | 7050253 | 45759 | fr. 1.\par
500 (!) à 0,005, m. 1,75 F, soit 3,75 F.\par
6 F + 16 F + 3,50 F + 3,75 F = 29,25 F.\par
En 9 h, soit 3,25 F l'h (27 F + 2,25 F ; juste !). Mais en réalité 8 h (heure de nettoyage), ce qui fait plus de 3,50 F ! exactement 3,65 F. Mais il est vrai que les vis d'acier, j'en avais fait une bonne partie la veille...\par
{\itshape Samedi.} – Maux de tête violents, état de détresse, après-midi mieux (mais pleure chez B...\par
{\itshape Dimanche.} – Art italien.\par
{\itshape Lundi 22}. – Fini pièces de vendredi (10 mn à 1/4 h). Monte moi-même pour la 1\textsuperscript{re} fois (sauf mise au milieu, pas arrivée tout à fait, ai dû appeler et attendre régleur [béret]. Puis change montage, pas fraise ; mais appelle régleur – lunettes pour mettre au milieu (ce qu'il ne fait pas), mais passe un temps infini à régler la profondeur de la fente. A fini à 10 h 1/2 ; j'ai fait alors un carton de 1.000 pièces (gagné 5,70 F en 3 h...). Nouveau carton de 1.000. Les petites au « côté bombé » en cuivre rouge. Certaines tiennent pas dans le montage ; je casse 2 dents... À 12 h, ai à peine commencé nouveau carton de 2.000 (laiton). Ai gagné 1 F + 3,70 F + 1 F + 5 F + 1 F = 11,70 F. Quand j'aurai fini le carton, aurai 20,70 F. Faut que je fasse encore 2 000 en plus...\par
Bouchon conduit circulaire – cuivre rouge, 6002400.\par
1.000 à 3,70 F + 1 F, soit 4,70 F.\par
{\itshape id}., plus petit, 1.000 à 5 F + 1 F, soit 6 F.\par
Après-midi.\par
Vis laiton, 705700 | 0 | 079658 (fr. 0,8).\par
2.000 à 4 F + 1 F, soit 9 F.\par
Bouchon (grands) 6002400 | 071844.\par
1.000 à 3,70 F, soit 4,70 F.\par
{\itshape id}., 071848.\par
1.000 à 3,70 F, soit 4,70 F.\par
Vis laiton 70500 | 379652 | fr. 0,8.\par
Commencé seulement.\par
Gagné :4,70F+ 6F +4,70F+ 4,70F + 1 F + 9 F = 30,10 F.\par
Leclerc me fait appeler qd j'ai fini 071841 et viens de commencer 848. Commence par m'engueuler parce que je fais ces pièces sans lui en parler. Demande le n°. Je lui apporte mon carnet ! Le regarde et devient gentil, gentil.\par
Mardi.\par
Fais les vis, 2.000 à 4 F.\par
Puis vis C 4 X 8 acier | 7010103 | 043409 | fraise 1\par
5.000 à 4,50 F = m. 1 F, soit 23,50 F.\par
23,50 F + 8 F 31,50 F (en 2 jours 61,60 F, soit 2 fois 30,80 F, soit 3,08 F l'h).\par
Gagné en 3 jours 29,25 F + 30,10 F + 31,50 F, soit 90,85 F, cela en :\par
28 [29] heures\par
28 x 3 = 84\par
[29]     [87]\par
28 X 0,50 = 14\par
28 X 0,25 = 7\par
84 + 7 = 91. Donc j'ai fait une moyenne de 3,25 F...\par
C. 4. 8. Commencées à 11 h – 5. Fraise mauvaise après 3/4 h (fume). Néanmoins, ce n'est qu'après 2 h 1/2 qu'elle est changée par Michel. [C'est de ma faute : pourquoi ne pas la changer plus vite ? Peur de me faire engueuler...] Michel dit qu'elle a servi à l'envers (?). La 2\textsuperscript{e}, quoique mise par lui, ne tient pas le coup (ce sont des scies nouvelles, trop grandes, dit ma voisine, pour des 1 [1]).\par
– Engueulade pour fraise cassée (récits Esp. s. émotions de débutante). Changée à 4 h. Après, je tiens jusqu'à 6 h (ai cassé 2 dents). Pénible, de travailler avec une mauvaise scie. Aussi, excuse envers moi-même pour ne pas me transformer en automate...\par
{\itshape Mercredi}. – Malheurs de la jeune Espagnole (ses pièces – sa fraise – le nouveau régleur – Leclerc).\par
La veille au soir, la scie – mise à 6 h par le monteur à béret – s'était desserrée à 7 h 1/4. Je le lui avais dit en passant. Je la retrouve desserrée. Je l'appelle. Il se fait attendre – m'engueule. J'appuyais trop, paraît-il. Je suis à peu près sûre que non (car la scie cassée m'avait fichu la frousse). Je le lui dis. Persiste à gueuler (soit dit par métaphore, car il n'élève pas la voix). Cet incident me fait froid au cœur pour q. temps, car je n'aurais demandé qu'à le considérer comme un camarade... À 10 h, nouvelle scie, mise par lui aussi. Ça lui prend 20 nm environ. Tout d'un coup, le moteur du fond s'arrête. On attend j. pas loin de 11 h. [J'avais fini les 5.000 de la veille (et trouvé une caisse pour eux) à 8 h 1/2.] J'apprends que la paye est aujourd'hui, non demain comme je croyais, ce qui me met la joie au cœur, car je n'aurai pas à me priver de manger... Aussi à midi, je ne recule devant rien (paquet de cigarettes – compote...).\par
À 3 h incident désastreux : je casse une dent de ma scie. Je sais comment ça s'est passé... Épuisée, je songe à mes fatigues de la M*. À Adrien – À sa femme – À ce que m'a dit Jeannine, que Michel la force à se crever – À ce que devrait en ressentir Pierre – À la jeunesse de Trotsky (« Quelle honte »...) et, de là, à son choix entre populisme et marxisme – À ce moment précis, je mets une pièce qui ne s'enfonce pas dans le montage (copeau, ou bavure), je l'appuie quand même sur la fraise... (mot grec) ! Je n'ose pas la changer, bien sûr – L'Espagnole me conseille d'avoir recours à Michel ; je lui parle, mais il ne viendra pas de la soirée. Je garde la même fraise j. 7 h. Par chance elle tient le coup – mais il faut dire que je la traite avec ménagements ! Vers 5 h, elle se desserre encore. Je n'ose appeler personne, bien entendu ! Je la serre, et fais 200 ou 300 pièces (ou un peu plus ?) pas au milieu du tout. Puis je prends une grande résolution et arrive à la mettre au milieu moi-même ! (mais en m'aidant d'une pièce déjà faite).\par
Paye 255 F (je craignais de n'avoir pas 200...) pour 81 h.\par
Nuit pas dormi.\par
{\itshape Jeudi}. – Encore une 1/2 h-3/4 h avec la scie. Puis Michel me la change, en même temps que celle de la machine qu'il règle. Je monte moi-même, mais n'arrive pas à mettre au milieu. En désespoir de cause, finis par avoir recours au régleur à lunettes. C'est fini à 9 h. – Matinée pénible – Les jambes me font mal – J'en ai marre, marre... (Ces pièces C 4 X 8 m'exaspèrent, avec le danger permanent de casser la fraise, la nécessité de conserver une vacance mentale intégrale...) 3 fausses alertes, et à 11 h – un mouvement, une parole avaient attiré mon attention – catastrophe : dent cassée. Heureusement, ce que j'ai à faire après demande une fraise 1,2. Pourvu qu'ensuite...\par
À midi, une pièce qui saute desserre la fraise.\par
Je reprends conscience de la nécessité de réagir, moralement, si je ne veux pas finir avec une mauvaise conscience. Et je me reprends en main.\par
À 1 h 1/2 je serre la fr. et la remets au milieu moi-même [ce que je n'avais pu faire la veille] grâce à la résolution, prise à déjeuner, d'y aller doucement [je me sers d'une pièce faite]. Le régl. à béret regarde gentiment et, quand c'est fait, achève de serrer. Fini à 2 h. Le même me monte les nouvelles pièces. Fait à 2 h 1/2.\par
2 h 1/2-4 h 1/2, ça ne va pas – Michel – son explication, conversation avec lui. Régl. à béret arrange.\par
4 h 1/2-6 h 1/2, je fais le reste des 2.000 (j'en avais fait 200 peut-être).\par
Vais chercher boulot. Leclerc gentil, gentil... Suis d'autant plus embêtée pour ma fraise, d'autant que ce boulot est à faire avec la fraise 1 – des C 4 x 10 acier. Vais à 7 h 3 mn changer à la fois 0,8, 1,5 et la 1 à dent cassée. Ça réussit. Me voici donc avec une belle fraise neuve... Mais j'ai 5.000 de ces saletés de pièces à faire (pas tout à fait les mêmes cependant). Gare à moi !\par
sens dans lequel il faut tourner les boulons pour les faire aller vers la poulie ; aussi sens dans lequel tournent la poulie et le reste.\par

\begin{center}
 \noindent\includegraphics[]{} \end{center}
\noindent La fraise se déporte dans le sens indiqué par la flèche ; étant montée sur un cône, il en résulte que la rainure non seulement cesse d'être au milieu, mais encore est de moins en moins profonde, ou même cesse de se produire.\par
Cause : serrage insuffisant au bout – ou usure de la fraise – ou effort trop grand de l'ouvrier qui appuie.\par
Effort trop grand : la fraise allant plus lentement que la poulie et l'arbre, tout se passe comme si on la faisait tourner en sens contraire (?).\par
Autres phénomènes de déréglage :\par
La fraise s'arrête parce que les bagues autour sont desserrées (ou parce qu'elles n'ont pas été assez serrées, ou parce qu'on appuie trop).\par
La fraise s'arrête (avec l'arbre et la poulie) parce que l'arbre est trop serré au bout ({\itshape b} se serre automatiquement parce que {\itshape a} n'est pas assez serré) [toujours défaut de réglage].\par
Ce jour-là, je crois qu'une des causes était le serrage insuffisant du montage, que la fraise devait enfoncer tout en travaillant : d'où effort trop grand.\par
À midi une joie. L'avis NON, selon lequel messieurs les ouvriers, etc... qu'on se repose samedi.\par
Nuit : ré-attaque offensive de l'eczéma qui me laissait en paix depuis une huitaine.\par
Gagné en ces 2 jours 45 F + 2 F + 12 F = 59 F...\par
(ou 58 F). Pas 3 F de l'h...\par
{\itshape Vendredi.} – Me fais monter pièces cherchées la veille par béret. Pendant ce temps, pèse – 250 en plus. Leclerc dit de les faire – commence à 8 h 1/4. En ai fait 200 à 10 h 1/2 à peu près. Fais changer la fraise. Faut attendre... recommence à 11 h 1/4. En ai fait moins de 3.000 dans ma matinée (soit 14 F, ou moins – pas plus de 3 F l'h !). Travail {\itshape très} pénible. Mais ne me laisse pas accabler moralement comme la veille. Physiquement cependant je suis plus mal. Après déjeuner (mangé pour 5,50 F dans l'espoir de me réconforter) c'est bien pire. Vertiges, éblouissement – travail inconscient. Heureusement ces pièces ne sautent pas comme les C 4 X 8... Je crois vraiment, pendant 2 h ou 2 h 1/2, que je vais m'évanouir. À la fin, je me résous à ralentir, et ça va mieux. Fini après 4 h (4 h ¼ ou 4 h 1/2). Leclerc me dit de ne marquer nulle part les 250 de rab, qu'on ne me paierait pas (elles manquent sûrement ailleurs, dit-il...). Me donne du « bon boulot » (les longues vis de laiton à 4 F. Le temps de les monter, 5 h. À 5 h 1/2, arrête pour laver la machine (on part à 6 h 1/2). Heure, en somme, relativement agréable, sauf les premiers moments de hâte et d'angoisse.\par
\par
Conversations avec régleur à béret qui, dirait-on, se met à s'intéresser à moi...\par
Vis C 4 x 10 acier | 7010105 | 041918 | fr. 1 | 5.000 à 4,50 F.\par
Arrêt des comptes lundi.\par
N. d'h total : 8 h + 10 + 10 + 10 + 10 + 9 + 10 = 67 h.\par
Gagné jusqu'ici 90,75 F + 47 F + 12 F + 23,50 F = 173,25 F, tout de même 3 F de l'h...\par
Il s'agirait de gagner 4,50 F de l'h lundi... Les 4.000 à 4 F feront 18 F (2 cartons). Restera 27 F... Il faudrait faire ces 4.000 en 3 h au plus. Et après faire encore 5.500... guère possible !\par
{\itshape Dimanche soir}. – Rentre à 11 h 40. Me couche. Ne dormant pas, m'aperçois vers minuit 1/2 que j'ai oublié mon tablier ! Dès lors dors encore moins. Me lève à 5 h 1/4 ; à 5 h 3/4, téléphone chez moi ; prends le métro j. à Trocadéro et reviens (40 mn en tout, dans la foule). Aussi, fatiguée et maux de tête.\par
{\itshape Lundi}. – C'est ce soir ou demain que je dois m'échapper. J'ai mal à la tête. Finis les 4.000 à midi seulement... (et même j'y passe encore 1/4 h de 1 h 1/2 à 1 h 3/4).\par
De nouveau la machine s'est déréglée, comme jeudi. La fraise pourtant toute neuve. Lucien (r. béret) me dit encore (plus doucement) que j'appuie trop. Mais je suis sûre qu'en fait il n'a pas assez serré. Quoi qu'il en soit, comme la fraise s'était déjà déréglée vendredi soir sans que je m'en sois aperçue, au point qu'un certain nombre de pièces n'ont même pas été touchées par la fraise, je dois perdre du temps à trier et à refaire. Je perds aussi un bon quart d'heure (au moins) à accompagner l'Espagnole qui cherche un plein seau de savon lubrifiant pour sa nouvelle machine, trop lourd pour être porté par elle seule, et que le manœuvre chargé d'en donner fait poser. Et après – quant à la vitesse – je suis malgré tout démoralisée par les reproches de Lucien. Je sais que si ça se reproduit, les choses iront mal. Et comme toujours quand je ne bande pas toutes mes forces, sans arrière-pensée, vers la cadence rapide, je ralentis. Quoi qu'il en soit, cela fait tout de même 4 X 4 = 16 F + 2 F (?) de montage (2 cartons).\par
Vis laiton (7050010 | 4.000 à 4 F |. Ensuite 400 pièces (sur 1.000, l'Esp. fait les 600 autres) dont je n'aurai le carton que mercredi.\par
Vis blocage acier 1774815 | 000987 | 400 à 0,50 F \% |m. 1,25 F | fr. 1,2. Je les fais sur la petite machine de l'Espagnole, placée ailleurs. Le régl. à lunettes fait le montage pendant que je finis mes vis de laiton. (Peu avant midi, alors que je ne savais pas encore qu'il préparait ça pour moi, il m'a donné ordre de changer la fraise et chercher les pièces sur un ton d'autorité sans réplique auquel j'ai obéi sans rien dire, mais qui a suffi pour faire monter en moi, à la sortie, le flot de colère et d'amertume qu'au cours d'une pareille existence on a constamment au fond de soi, toujours prêt à refluer sur le cœur. Je me suis reprise cependant. C'est un incapable (manœuvre spécialisé, dit l'Espagnole ?), il faut bien qu'il parle en maître.\par
Je les commence à 1 h 3/4. La machine m'est nouvelle. J'y passe, je crois bien, près d'1 h (l'Espagnole, elle, fera les 600 en 20 mn !). Après, je vais demander le carton. Ça perd du temps. (Il n'y en a pas.) Un jeune homme vient prendre les 400 pièces. Je vais dire à Leclerc qu'il n'y a pas de carton. Quelqu'un que je ne connais pas (blouse grise) lui parle familièrement, et, autant que je comprends, d'une engueulade qu'il risque, lui, Leclerc. Il semble mécontent de me voir là (ça se comprend), et son mécontentement me fait oublier de lui demander des pièces. Après, il se balade dans l'atelier ; je ne veux pas, en allant vers lui, risquer de me faire rabrouer comme l'autre fois; et je perds plus de 3/4 h (aussi à aller à la recherche du régleur sur tours qui m'a donné les 400 pièces, pour savoir s'il y en a d'autres, je ne le retrouve pas).\par
Leclerc me donne enfin des C 4 X 16.\par
Vis C 4 X 16 acier | 7010111 | 013259 1 5.000 à 4,50 F | m. 1 h | fraise 1.\par
Par compensation, j'ai enfin Michel pour me monter ma machine. Il est 3 h 1/2, je ne peux plus passer le carton. (C'est l'arrêt des comptes, et on ne les passe que jusqu'à 3 h.) Le retard que j'ai, donc, au lieu de le rattraper (et c'est surtout pour ça que j'avais tenu à venir aujourd'hui) je l'augmente. Cette pensée me démoralise, eu égard à la vitesse. Car ce que je fais à partir de maintenant compte dans une quinzaine que je ne ferai pas entière ; que m'importe donc ma moyenne horaire ? Je suis déprimée par les maux de tête, et vais – sans m'en apercevoir – très, très lentement. Ces pièces, je ne les aurai finies que le lendemain à midi (et même pas tout à fait), ce qui fait pour 15 h de travail (et m. plus). 18 F + 3,25 F + 23,50 F = 44,75 F. Or pour faire 3 F de l'h, je devrais avoir gagné 45 F en ces 15 h.\par
Arrêt des comptes à 3 h.\par
\par
{\itshape Mardi.} – Fini les C 4 X 16.\par
Vis M. P. R., chez Gorger (tours automatiques).\par
Vis M. P. R. à grosse tête hexagonale ⎔\par
Il faut les placer de manière que le fraisage soit perpendiculaire à 2 côtés parallèles :  \{deux demi hexagones côte à côte\}   \par
Sans quoi pièce loupée. Acier fort dur. En les plaçant, on risque de les tourner. Dans toute l'après-midi (et le lendemain 3/4 h) je n'en fais qu'un carton de 1.400 (5 F les 1.000 + 1 F m., soit 8 F), avec une interruption pour 1.000 grosses vis en laiton sur la machine à côté, dont je n'ai pas le carton, mais qui sûrement ne sont pas payées plus de 4,50 F au maximum. Soit en 6 h 1/4 (ou plus ?) gagné 8 F + 4,50 F = 12,50 F. Ça, c'est du joli ! 2 F l'heure ! Heureusement que je me porte malade le mercredi matin.\par
Quête pour une ouvrière enceinte. On donne 1 F, 1,50 F (moi 2 F). Discussion au vestiaire (ça s'était déjà produit il y a 1 an, pour la même). « Alors, tous les ans ! – C'est un grand malheur, et puis c'est tout. Ça peut arriver à n'importe qui. – Quand on ne sait pas, on n'a qu'à ne pas... » L'Espagnole : « Je trouve que ça n'est pas une raison pour quêter, et toi ? » Je dis « Si » avec conviction, et elle n'insiste pas.\par
Quittant le lundi soir avec l'intention de me déclarer malade le lendemain matin, je me garde de manger plus qu'un sandwich acheté à 7 h, avec un verre de cidre. Me réveille à 5 h 1/2 (exprès). Mange un petit pain le mardi matin. {\itshape Id}. seulement à midi, 3 petits pains le soir, et vais à pied porte de Saint-Cloud, avec un café express pour me faire dormir. Or tout ce régime a l'effet de me mettre en état d'euphorie !... Seulement une lenteur extrême dans le boulot.\par
{\itshape Mercredi matin.} – Finis le carton de 1.400 M. P. R., j'en fais 200 sur le nouveau carton – 5 F ou 5,80 F ? Vais très, très lentement, mais me sens, par un sacré esprit de contradiction, singulièrement joyeuse et en forme.\par
Leclerc et Gorger [chef d'équipe des tours auto], les cartons des 1.000 pièces de laiton. Leclerc : « Si vous voulez arrêter, arrêtez-vous. »\par
Gagné 27,50 F + 1 F + 1 F + 4 F (?) + 1 F + 7,50 F (?) = 37 F ou 40,60 F | théoriquement en 11 h 1/2 [34,50 F...]\par
\par
{\itshape Lundi-mardi}. – Vis C 4 X | 6 acier | 5.000 à 4,50 F + m. 1 F | fr. 1. 7010 III | 013252, fixant brides.\par
Vis M. P. R. acier, 4.000 à 5,80 F + m. 2 F 1,5 | 747327 | 046-543.\par
Ergot d'arrêt acier 2.000 à 4,50 F + 2 F (?) | 7050129 | 099937 | fr. 1.\par
23,50 F + 23,20 F + 2 F + 9 F + 2 F = 59,70 F.\par
37 F + 59,70 F = 96,70 F en 11 h 1/2 + 20 h 1/2 = 32 h. 32 x 3 = 96.\par
Donc à supposer minimum de 3 F, je suis à jour, mais juste... et il y aurait 12 F à rattraper sur l'autre quinzaine !\par
Épisodes : Gorger...\par
Michel...\par
Malice cousue de fil blanc de Juliette...\par
Lundi, mal en point. Rentrée infiniment plus pénible que je n'aurais cru. Les jours me paraissent une éternité. Chaleur... Maux de tête... Ces vis C 4 x 16 me répugnent. C'est du « bon boulot » ; il faudrait le faire vite, je n'y arrive pas. À peine fini, je crois, à 3 h 1/2. Accablement, amertume du travail abrutissant, dégoût. Peur aussi, toujours, de desserrer la fraise. Ça m'arrive cependant. Attente, pour faire changer les fraises. J'arrive pour la 1\textsuperscript{re} fois à changer une fraise moi-même, sans aucune aide, et Philippe dit que c'est bien au milieu. Victoire, meilleure que la vitesse. J'apprends aussi, après une nouvelle mauvaise expérience, à régler moi-même le serrage de la vis et de la manivelle du bout. Lucien oublie parfois complètement de la serrer... Les M. P. R. Michel me met en garde. Il ne les règle pas, mais le « lunettes ». Je les fais un peu plus vite que la fois d'avant, mais encore très, très lentement.\par
{\itshape Mercredi.} – Arrêtoir acier, fr. 1,5.\par

\tableopen{}
\begin{tabularx}{\linewidth}
{|l|X|X|}
\hline & 009182    1.000 &  \\
\hline
C 001268 & 097384     – & à 4,50 F (2 mont.) \\
\hline
 & 097385     – &  \\
\hline
\end{tabularx}
\tableclose{}

\noindent Bouchon conduit circulaire cuivre rouge 10 C. V., fr. 1,5.\par

\tableopen{}
\begin{tabularx}{\linewidth}
{|l|X|X|}
\hline & 071853    1.000 &  \\
\hline
C 002400 &  50    – & 3,70 \\
\hline
 &  47    – &  \\
\hline
\end{tabularx}
\tableclose{}

\noindent 4,50 F x 3 + 3,70    F x 3 + 3 F...\par
13,50 F + 11,10 F + 3 F = 27,60 F, travaillé 10 h 1/2. Manque donc 4 F.\par
{\itshape Jeudi.} – Boulon serrage acier 8 C. V., fr. 1.\par
737887 | 084097, 3.000 à 4,50 F m. 1 F.\par
Bouchons conduit circulaire cuivre rouge, fr. 1,5.\par
13,50 F + 3,70 F + 5 F + 3,80 F + 4 F = 30 F. Manque 1,50 F.\par
Donc manque 5,50 F en tout. Peut-être compensé par la semaine d'avant.\par
Épisode des « arrêtoirs ». Michel, jeudi matin.\par
Lourdeur, mercredi et jeudi. Délices de la fraîcheur jeudi soir. Bonne...\par
Les arrêtoirs avaient été commencés la veille à 5 h. Ce mardi où j'ai cru m'évanouir, tant il faisait lourd, tant je me sentais tout le corps en feu, tant j'avais mal à la tête... Juliette me dit : « Fraise 1,5. » Je démonte ma fraise de 1, je vais changer les 2 et j'en tends une à Philippe en disant simplement : « C'est celle de 1. »\par
Chez Renault.\par
{\itshape Lange} : chef d'atelier – ancien régleur -maniaque pour l'ordre et la propreté, à part ça... Sourcils froncé, etc. ; attitude respectueuse des chefs d'équipe. Avec moi, assez gentil.\par
{\itshape Roger} (rempl. Leclerc) : régleur des perceuses.\par
{\itshape Philippe} : brute, régl. des tours.\par
Gros yeux... : grand blond, autre régl. des tours.\par
Lunettes...\par
Ouvriers : Arménien, fraiseur manœuvre à côté de 1\textsuperscript{re} machine, ouvrier gentil et doux qui blague sur « les femmes qui iront à la guerre ». Italien, celui qui le remplace (sympath.).\par
Ouvrières : Bertrand – autre voisine (Juliette) – commençante – celle qui flirte avec Michel -La grande brune à 2 gosses – vieille des tours – femme d'Italien – perçage...\par
Chefs éq. :\par
{\itshape Fortin} : quel chic type...\par
{\itshape Gorcher} : tours auto, rigolo, sympathique.\par
Leclerc.\par
Chef en face ascenseur – ton de supériorité intolérable.\par
Michel.\par
Lucien.\par
Gagné à cette expérience ? Le sentiment que je ne possède aucun droit, quel qu'il soit, à quoi que ce soit (attention de ne pas le perdre). La capacité de me suffire moralement à moi-même, de vivre dans cet état d'humiliation latente perpétuelle sans me sentir humiliée à mes propres yeux ; de goûter intensément chaque instant de liberté ou de camaraderie, comme s'il devait être éternel. Un contact direct avec la vie...\par
J’ai failli être brisée. Je l'ai presque été – mon courage, le sentiment de ma dignité ont été à peu près brisés pendant une période dont le souvenir m'humilierait, si ce n'était que je n'en ai à proprement parler pas conservé le souvenir. Je me levais avec angoisse, j'allais à l'usine avec crainte : je travaillais comme une esclave ; la pause de midi était un déchirement ; rentrée à 5 h 3/4, préoccupée aussitôt de dormir assez (ce que je ne faisais pas) et de me réveiller assez tôt. Le temps était un poids intolérable. La crainte – la peur – de ce qui allait suivre ne cessait d'étreindre le cœur que le samedi après-midi et le dimanche matin. Et l'objet de la crainte, c'étaient les {\itshape ordres.}\par
Le sentiment de la dignité personnelle tel qu'il a été fabriqué par la société est brisé. Il faut s'en forger un autre (bien que l'épuisement éteigne la conscience de sa propre faculté de penser !). M'efforcer de conserver cet autre.\par
On se rend compte enfin de sa propre importance.\par
\par
La classe de ceux qui ne comptent pas – dans aucune situation – aux yeux de personne... et qui ne compteront pas, jamais, quoi qu'il arrive (en dépit du dernier vers de la 1\textsuperscript{re} strophe de l'{\itshape Internationale}).\par
Question de Det. (solidarité ouvrière).\par
Problème : conditions objectives telles que 1° Les hommes soient des chics types et 2° produisent.\par
On a toujours besoin pour soi-même de signes {\itshape extérieurs} de sa propre valeur.\par
Le fait capital n'est pas la souffrance, mais l'humiliation.\par
Là-dessus, peut-être, que Hitler base sa force (au lieu que le stupide « matérialisme »...).\par
[Si le syndicalisme donnait un sentiment de responsabilité dans la vie quotidienne...]\par
Ne jamais oublier cette observation : j'ai toujours trouvé, chez ces êtres frustes, la générosité du cœur et l'aptitude aux idées générales en fonction directe l'une de l'autre.\par
Une oppression évidemment inexorable et invincible n'engendre pas comme réaction immédiate la révolte, mais la soumission.\par
À l'Alsthom, je ne me révoltais guère que le dimanche...\par
Chez Renault, j'étais arrivée à une attitude plus stoïcienne. Substituer l'acceptation à la soumission.
\section[Fragments ]{Fragments \protect\footnotemark }\renewcommand{\leftmark}{Fragments }

\footnotetext{Pages écrites pendant le séjour à l'usine (1934-1935) et dans l'année suivante.}
\noindent \par
Organisation bureaucratique de l'usine : – les bureaux, organes de coordination, sont l'âme de l'usine. Les procédés de fabrications (y compris les secrets) y résident. C'est pourquoi on y diminue moins le personnel que dans les ateliers, où, sauf chefs d'ateliers, contremaîtres, magasiniers, etc., tout est interchangeable. Les manœuvres principalement, bien entendu ; mais même les ouvriers qualifiés. Un tourneur de chez Alsthom pourrait être remplacé par un de chez Citroën que personne ne s'en apercevrait. (Si un ouvrier qualifié est attaché à l'entreprise, c'est uniquement par l'intermédiaire de la machine, surtout dans le cas du fraiseur.)\par
Chez les ouvrières (manœuvres), aucun attachement à l'entreprise.\par
Régleurs : ce sont des camarades, avec une nuance de fraternité protectrice. (Une vieille ouvrière trouve tout naturel qu'un régleur de 25 ans ait à la guider... La participation des femmes à la production industrielle a sûrement facilité la différenciation des catégories.) Mais leur caractère change sans doute avec celui de la production. Ici, il y a tout le temps des machines à monter (surtout en ce moment, période de toutes petites commandes que l'entreprise refuserait sans doute en période plus prospère). Là où il y a peu de machines à monter et beaucoup de surveillance, ils tiennent peut-être plus du chef.\par
Concurrence entre les ouvrières.\par
Quand on a l'occasion d'échanger un regard avec un ouvrier – qu'on le rencontre au passage, qu'on lui demande quelque chose, qu'on le regarde à sa machine – sa première réaction est toujours de sourire. Tout à fait charmant. Ce n'est ainsi que dans une usine.\par
Le directeur est comme le roi de France. Il délègue les parties peu aimables de l'autorité à ses subordonnés, et en garde pour lui le côté gracieux.\par
Sentiment d'être livré à une grande machine qu'on ignore. On ne sait pas à quoi répond le travail que l'on fait. On ne sait pas ce qu'on fera demain. Ni si les salaires seront diminués. Ni si on débauchera.\par
Caractère {\itshape peu adaptable} de toute grande usine. Formidable quantité d'outils ; spécialisation des machines. Tout se passe comme s'il y avait trop peu de machines, alors qu'il y en a trop.\par
Le caractère de la technique et de l'organisation des grandes usines modernes n'est pas lié seulement à la production en série, mais aussi à la {\itshape précision des formes.} Quel ouvrier ferait des pièces aussi justes que fait un outil ? Or, un outil {\itshape spécialisé} est très coûteux sans une production massive.\par
Partie d'artisanat dans le travail de l'ouvrier. À étudier.\par
Exemple : un monteur de presse doit savoir serrer la vis de manière que l'outil obtienne juste la transformation désirée, mais pas plus (exemple mes 100 pièces loupées). Il le fait au jugé, en essayant. Mais il faut, bien entendu, qu'il sente la chose au bout des doigts.\par
En somme, que doit savoir un régleur de presses ?\par
On lui indique l'outil sur la feuille. [Néanmoins, en certains cas, il faut vérifier l'efficacité de l'outil en fonction du dessin : angles, etc.] Le magasinier le lui remet ou, au besoin, un autre plus apte. Il doit : 1° Savoir à quelles machines l'outil peut s'adapter. Un outil peut convenir à plusieurs machines, mais non pas à toutes. Cela dépend 1] de la structure (mais je crois que pour la structure la plupart sont équivalentes), 2] de la force. La force nécessaire n'est pas, je crois, indiquée sur le papier (à vérifier). Comme on fait toujours à peu près les mêmes opérations, l'expérience décide. {\itshape Ce point est à étudier de plus près}. 2° Savoir adapter l'outil à la machine au moyen d'un montage approprié (comment ? à étudier). 3° Monter le support de manière qu'il soit sous l'outil (il faut du coup d'œil), et, le cas échéant, de manière qu'il permette de prendre une position commode au cours du travail. 4° Serrer la vis. Je crois que c'est tout...\par
Remarquer qu'un régleur de presses serait perdu devant un tour ou une fraiseuse – et réciproquement. Du point de vue de la sécurité dans l'entreprise, c'est, d'un certain côté, un avantage : on ne les remplacera pas par des gens venus d'ailleurs. D'un autre côté, c'est un inconvénient : s'il y en a trop aux presses, on n'en prendra pas un pour le loger ailleurs. L'inconvénient l'emporte. Car on peut toujours les remplacer par des manœuvres spécialisés.\par
Questions à étudier : les {\itshape outils} : leur forme et leur efficacité.\par
L'étudier d'abord sur les machines où je travaille.\par
À étudier les rôles de :\par
manœuvre sur machine (moi...)\par
manœuvre\par
spécialisé ouvrier qualifié de la fabrication (y en a-t-il ?)\par
ouvrier qualifié de l'outillage\par
régleur\par
magasinier\par
chef d'équipe\par
chef d'atelier\par
dessinateur\par
ingénieur\par
sous-directeur\par
directeur.\par
Transposition et correspondance : la forme d'un outil et son action.\par
Peut-on {\itshape lire} l'action de l'outil en le voyant ?\par
M’y exercer.\par
Interroger le magasinier.\par
Il n'y a d'ailleurs pas que les presses...\par
À noter : jusqu'ici je n'ai vu que deux types heureux de leur travail :\par
\par
l'ouvrier qui est au four et chante tout le temps (m'informer un peu à son sujet).\par
le magasinier.\par
Savoir d'où sort le chef d'équipe ?\par
L'observer plus constamment pour savoir ce qu'il fait (y penser une journée). Surtout des paperasses, il me semble. Il ne surveille à peu près pas le travail (observations aux ouvriers au cours de leur travail, très rares). Il est fort rare qu'on le voie après une machine.\par
Savoir d'où sort le chef d'atelier ? Ce qu'il fait ?\par
Travail beaucoup plus concret, il me semble – observer combien de temps il passe dans son bureau.\par
Remarques sur le genre d'attention que réclame le travail manuel (mais en tenant compte 1° du caractère spécial du travail que je fais, 2° de mon tempérament).\par
< Quand tu seras à l'arrêt, débrouille-toi pour sortir de temps à autre... >\par
< Il te faut une discipline de l'attention toute nouvelle pour toi : savoir passer de l'attention attachée à l'attention libre de la réflexion, et inversement. Sans quoi ou tu t'abrutiras, ou tu bousilleras le travail – c'est une discipline. >\par
Manœuvres spécialisés : tous des hommes (cependant le magasinier m'a dit qu'il y avait des découpeuses spécialisées – mais je n'ai jamais vu une femme toucher à une machine autrement que pour la conduire). Montent leurs propres machines (conseillés au besoin par le régleur). Doivent savoir lire les dessins, etc. Comment ont-ils appris à monter une machine ? À élucider.\par
« Manœuvres sur machines. » Femmes. Leur seul contact avec les machines consiste, semble-t-il, à savoir les pièges de chacune, i. e. les dangers de pièces loupées que chacune comporte. Elles arrivent à percevoir que quelque chose ne va pas à telle ou telle machine dont elles sont familières. Cela pour celles qui ont des années d'usine.\par
Le chef d'atelier n'aime pas que les ouvrières momentanément sans travail soient réunies à beaucoup pour causer. Sans doute craint-il qu'il ne s'engendre ainsi un mauvais esprit... Les ouvrières ne songent pas à s'étonner de choses de ce genre, et ne se demandent pas le pourquoi. Leur commentaire : « Les chefs, c'est fait pour commander. »\par
\footnote{ Voir page 49. [Dans l’édition numérique, voir \hyperref[condition_ouvriËre_V_sem_II]{\dotuline{Journal d’usine : deuxième semaine}}.]} Drame à l'usine aujourd'hui (jeudi). On a renvoyé une ouvrière qui avait loupé 400 pièces. Tuberculeuse, avec un mari chômeur la moitié du temps, et des gosses (d'un autre, je crois), élevés par la famille du père. Sentiment des autres ouvrières, mélange de pitié et du « c'est bien fait pour elle » des petites filles en classe. Elle était, paraît-il, mauvaise camarade et mauvaise ouvrière. Commentaires. Elle avait allégué l'obscurité (après 6 h 1/2, on éteint toutes les lampes). « Et moi, j'ai bien fait telles et telles choses sans lumière. » « Elle n'aurait pas dû répondre au chef (elle avait refusé de faire le travail), elle aurait dû aller dire au sous-directeur : J'ai eu tort, mais, etc. » « Quand on doit gagner sa vie, il faut faire ce qu'il faut. » « Quand on a sa vie à gagner, il faut être plus consciencieux (!). »\par
Quelques ouvrières :\par
La vieille qui est allée en Russie en 1905 – qui ne « s'ennuyait jamais quand elle vivait seule, parce qu'elle lisait le soir » – qui a une Schwärmerei pour Tolstoï (Résurrection : « sublime », « cet homme comprenait l'amour »).\par
Celle qui a un port de reine et dont le mari travaille chez Citroën.\par
Celle de trente-six ans qui vit chez ses parents.\par
L'Alsacienne.\par
Quelques ouvriers :\par
Le magasinier.\par
L'ancien ajusteur et professeur de violon.\par
Le blond à l'air conquérant, manœuvre spécialisé.\par
Jacquot.\par
Le régleur en chef.\par
Le gros gars du Nord, régleur,\par
Le charmant type à lunettes (régleur ou chef d'équipe ?).\par
Celui au four qui chante tout le temps.\par

\begin{center}
*\end{center}
\noindent L'ignorance totale de ce à quoi on travaille est excessivement démoralisante. On n'a pas le sentiment qu'un produit résulte des efforts qu'on fournit. On ne se sent nullement au nombre des producteurs. On n'a pas le sentiment, non plus, du rapport entre le travail et le salaire. L'activité semble arbitrairement imposée et arbitrairement rétribuée. On a l'impression d'être un peu comme des gosses à qui la mère, pour les faire tenir tranquilles, donne des perles à enfiler en leur promettant des bonbons.\par
Savoir si un ouvrier qualifié ?...\par
{\itshape Question à poser au magasinier}. Est-ce qu'on invente parfois des outils ?\par
Question : Quelles répercussions ont eues sur le développement de l'industrie le {\itshape Traité de mécanique} de d'Alembert et la {\itshape Mécanique analytique} de Lagrange ?\par
Principe des machines-outils. Les outils sont des transformations de mouvements. Inutile donc que le mouvement à transformer soit imprimé par la main.\par
Question : Peut-on créer des {\itshape machines automatiques souples} ? Pourquoi pas ?\par
Idéal : 1° qu'il n'y ait {\itshape autorité} que de l'{\itshape homme sur la chose} et non de l'{\itshape homme sur l'homme}.\par
2° Que tout ce qui, dans le travail, ne constitue pas la traduction d'une pensée en acte soit confié à la chose.\par
(Que le {\itshape parcellaire} soit le fait de la machine...) avec une idée universelle des transformations de mouvements...\par
Que toutes les notions physiques expriment directement des réalités techniques (mais {\itshape sous forme de rapport}) ; exemple : puissance.\par

\begin{center}
*\end{center}
\noindent Puissance que, peut fournir une machine mue par une courroie de transmission (calculée à l'avance d'après la robustesse de la machine), dépend de :\par

\tableopen{}
\begin{tabularx}{\linewidth}
{|l|X|}
\hlineN. de tours par sec. de l'arbre principal qui lui fournit son mouvement \{figure illisible\} Rayon de la poulie montée sur cet arbre à laquelle elle est reliée {\itshape d}/2. & Vitesse linéaire de la courroie. \\
\hline
 Coefficient de frottement (tg ω) [qui augmente quand le glissement varie en augmentant ?] \\
Pression (fonction de la tension du brin conduit {\itshape t}). \\
Arc enveloppé sur l'une et l'autre poulie (α).  & Effort tangentiel.  \\
\hline
\end{tabularx}
\tableclose{}


\begin{center}
n/60, πd, (e\textasciicircumfα -I) ; {\itshape e} étant la base des logarithmes népériens.\end{center}
\noindent Différence entre filetage, cylindrage, détalonnage.\par

\begin{center}
*\end{center}
\noindent Visite aux Arts et Métiers :\par
Engrenages, transformations de mouvements...\par
Recommencer. Quitter Renault pas trop tard...\par
Fraiseuse :\par
Rythme ininterrompu (avoir toujours fait 2.000 et qq. centaines à 7 h).\par
Serrer l'étau.\par
Mettre à part les loupés.\par
Faire tomber les pièces dans la caisse (coup sec, mais pas trop fort) :\par
Bien ramasser les pièces tombées dans la sciure.\par
Ôter de la sciure tous les jours.\par
Compter.\par
M'arrêter à 6 h 1/2.\par
Apprendre à faire plus vite le découpage des bandes métalliques (mvt plus continu).\par
Faire plus vite planage (placer plus vite...\par
Se rendre compte clairement avant chaque travail (ou, pour les travaux tout à fait nouveaux, au bout de quelque temps) des difficultés possibles, notamment comment la machine peut se dérégler, de la liste complète des fautes à éviter. De temps à autre se la répéter mentalement. Ne pas se laisser ralentir par le souci de difficultés imaginaires.\par
Prendre rythme défini surtout par un {\itshape mouvement continu} de la pièce finie à la pièce nouvelle, de la pièce placée au coup de pédale.\par
M'efforcer systématiquement d'attraper le tour de main pour placer et retirer la pièce, notamment {\itshape tour de main pour placer à la butée} ({\itshape très important}). [Supporter de la main et pousser d'un doigt s/ la butée ; ne jamais saisir la pièce avec la main.]\par
Ne pas oublier que le SOMMEIL est ce qu'il y a de plus nécessaire au travail.\par
Bêtises commises à éviter dorénavant (relire cette liste 2 fois par jour) :\par
1. BOURRER LA MACHINE [cartons] {\itshape peut causer accidents graves.}\par
2. NE PAS REGARDER DE {\itshape TRÈS} PRÈS UNE PIÈCE TOUTES LES... (500 loupées).\par
3. NE PAS CONSERVER MODÈLES.\par
4. METTRE PIÈCES À L'ENVERS (rivetage ; commis 2 fois ; failli le faire plusieurs autres fois).\par
5. {\itshape Pédaler avec tout le corps}.\par
6. {\itshape Garder le pied appuyé sur la pédale}.\par
7. LAISSER UNE PIÈCE DANS L'OUTIL (on risque d'abîmer l'outil – fait encore au planage).\par
8. MAL PLACER LA PIÈCE (pas à la butée).\par
9. {\itshape Ne pas mettre d'huile quand il faut.}\par
10. METTRE DEUX PIÈCES DE SUITE.\par
11. {\itshape Ne pas observer la position des mains du régleur.}\par
12. {\itshape Ne pas remarquer quand il arrive quelque chose à la machine} (colliers avec Biol).\par
13.    PLACER BANDE MÉTALLIQUE AU-DELÀ DE LA BUTÉE (cassé outil jeudi 6 mars).\par
14. {\itshape Pédaler avant que la pièce soit placée.}\par
15. {\itshape Retourner une bande métallique commencée.}\par
16. Laisser des pièces non usinées.\par
\subsection[Usines de R. (Monsieur B)]{Usines de R. (Monsieur B)}
\noindent 1 fois sur 2 un bon ouvrier fait un mauvais chef d'équipe [lui raconter l'histoire de Morillon].\par
Talent de l'organisation : se demande d'{\itshape où vient}... (quelque chose qui ne colle pas).\par
Lui et ingénieur en chef ont pratiquement le même domaine.\par
14-18, adaptation de l'outillage à la production de guerre. Méthode cartésienne (division des difficultés).\par
Journée occupée par des détails {\itshape à propos desquels} on soulève les problèmes essentiels d'organisation.\par
Règle les détails : 1° ou qui sont hors du domaine de la {\itshape responsabilité} de celui qui s'adresse à lui, 2° ou qui sont trop difficiles\par
à éclaircir.\par
< Cf. Detœuf – un subordonné – vient lui exposer une difficulté, et ce qu'il fait – 9 fois sur 10, il approuve. La 10\textsuperscript{e}, fait une suggestion brillante. L'autre est content dans tous les cas... Cf. Tolstoï. >\par
Les diagrammes, etc. Un {\itshape chef} doit imaginer tout ça sans aucun effort ; ça va de soi. Trouve plutôt des idées en regardant les statistiques qu'en regardant les choses [remarquable...].\par
\par
< Fait aussi travail d'ingénieur ; recherches de nouveaux modèles. >\par
Formation d'esprit : analyses chimiques.\par
Travail principal : concordance des opérations, rythme...\par
9/10 de manœuvres.\par
Fonte de la fonte dans des chaudières.\par
Coulage de la fonte dans des moules en sable dur.\par
Presses à main – hydrauliques pour presser le sable. 4 machines (inventées en 1927 par un ingénieur sorti des Arts et Métiers).\par
Le sable est automatiquement passé, etc. – puis passe sous rouleaux – puis convoyeur sur lequel on coule la fonte. La 1\textsuperscript{re} a coûté 400.000 F.\par
Atelier de perçage, polissage, ébarbage à la meule, 1 femme sur une presse.\par
Quelques femmes debout, dont l'une à une machine (?) où il faut soulever des poids fort lourds.\par
Atelier de montage.\par
Chaque ouvrier entre 2 étagères où sont toutes les pièces {\itshape ds l'ordre}. Hommes et femmes, certaines pièces assez lourdes...\par
Atelier d'émaillage.\par
At. de mécanique (quelques tourneurs, fraiseurs, ajusteurs) (il devait y en avoir un autre qu'on n'a pas vu ?).\par
{\itshape M. B.} : directeur technique, d'abord simple chimiste (sans diplôme ? est-ce possible ? demander encore des détails).\par
Accidents : sur 1 journée de trav. de l'usine, en moy. 1 h de perdue...\par
Diminution verticale, ces derniers temps.\par
Fondeurs – lunettes en verre triplex. Souvent ne les mettent pas. Pourquoi ? B. dit q. ce n'est pas à cause de la cadence, mais à cause de l'incommodité (?).\par
Émailleurs – cages de verre où aspiration, pour éviter intoxication au plomb. Certains mettent la tête ds la cage.\par
Renvois pour infractions aux règlements de sécurité.\par
Les Polonais ont besoin de {\itshape recevoir des ordres}.\par
Commission de sécurité av. ingénieurs, dessinateurs, chefs du personnel, ouvriers nommés par B. (les + intelligents et les « mauvaises têtes » –).\par
A à résoudre ts les problèmes insolubles – surtout détails – beaucoup d'imprévu... On vient le trouver... réunit les ingénieurs une fois par semaine.\par
Moyenne des salaires : hommes : trentaine de fr. (32...) ; femmes : 20 F. 21 F...\par
M. jeune, 27 ans – sorti de Centrale depuis 3 ans – a grandi dans l'usine..., fils aîné.\par
Math. sup. : gymnastique de l'esprit – irremplaçable à son avis –\par
Son attitude avec l'automobiliste en panne – réaction de sa mère et de l'horrible bourgeoise [« son moteur ne marche pas avec du vin », « ne parlez pas au conducteur » (!!!)].\par
M\textsuperscript{me} M.\par
L'horrible bourgeoise...\par
Faut-il être {\itshape dur} pour conserver la clarté et la précision d'esprit, la décision ?\par
Les math. supérieures ne seraient-elles pas elles aussi (cf. Chartier) un moyen de « former l'attention en tuant la réflexion » ?\par
Quel rôle, chez ces gens-là, joue la question d'argent ?\par

\begin{center}
*\end{center}
\noindent Demander à D.\par
Qui détermine l'outillage ? L'achat des machines (tj. D. lui-même), etc. ? et selon quelles règles ?\par
Au tourneur.\par
A-t-il des calculs à faire ?\par
Guihéneuf. « C'est l'expérience... » Mais cependant D. ?\par

\begin{center}
*\end{center}
\noindent Rythme ininterrompu. Le travail à la main le comporte-t-il jamais ? La machine dispense la pensée d'intervenir {\itshape si peu que ce soit}, même par la simple conscience des opérations accomplies ; le rythme le lui interdit.\par
(Guihéneuf et ses manettes...)\par

\begin{center}
*\end{center}
\noindent Visite à G.\par
Biographie : menuisier, 3 ans à l'école professionnelle, où a subi l'influence d'un professeur socialiste. A subi l'influence de la tradition du compagnonnage, par de vieux ouvriers. A fait son « tour de France » en allant dans chaque ville au siège de son syndicat (a tout de suite été syndicaliste, non socialiste), a suivi des cours du soir, s'est instruit de tout ce qui concernait la fabrication du bois. Mobilisé milieu 17, a été mis dans l'aviation et envoyé dans une école. À l'armistice, toujours mobilisé, envoyé à Paris, dans un ministère. Libéré en 20, a travaillé dans des usines d'aviation (?). Part pour la Russie (23), y a travaillé comme ouvrier dans des usines d'avions. Envoyé en Sibérie comme inspecteur d'une grosse entreprise de bois, passe ensuite directeur d'une usine ; y {\itshape double} la production, sans changer l'outillage. Puis passe directement du trust (tj. membre du Parti, où entré en France en 21, à la suite de Monatte). Dégoûté du régime, à la réflexion, demande à étudier. Reçoit bourse. Avale en quelques mois toute la mathématique du 2\textsuperscript{e} degré, passe l'examen d'entrée. Étudie 3 ans. Ingénieur 6 mois dans une usine d'avions (moteurs). Revient en France en janvier 34. Sans travail, cherche en vain une place d'ingénieur, de correcteur, etc. Finit par entrer comme tourneur n'ayant jamais travaillé sur un tour) dans une petite boîte dont il connaît le contremaître (homme vaniteux et brutal), travaille aux pièces. Tour non automatique (du même genre qu'à l'outillage). Au bout de 2 jours, réalise les normes. Il y a presque un an qu'il y est, n'a jamais eu de coup dur. Mais fatigué et abruti.\par
Renseignements :\par

\tableopen{}
\begin{tabularx}{\linewidth}
{|l|X|X|}
\hlineSur la Russie & Sur le travail d’ouvrier & Sur la technique \\
\hline
–– & –– & –– \\
\hline
Spécialistes du Gosplan acquièrent du doigté, de l'intuition…, seraient difficilement remplaça-bles – seront irrem-plaçables dans 10 ans. & On ne peut pas penser à autre chose, on ne pense à {\itshape rien.} &  Rôle des math. \\
Avantage à en avoir appris. \\
Tech. très sup. qui {\itshape lisent} la math. comme un langage à travers lequel ils aperçoivent directement les réalités. \\
Ex. : ils comprennent mieux un ouvrage technique dans une langue étrangère qu'ils ignorent que s'ils connaissaient la langue sans comprendre les formules (???).  \\
\hline
\end{tabularx}
\tableclose{}


\begin{center}
*\end{center}
\noindent Le {\itshape Racine} de Tal. – Une idée : la mort partout présente dans ses tragédies, des héros qui tous, dès le début, courent à la mort. La mort est en eux (Iphigénie...). Au contraire dans Homère, Sophocle : le drame est que ce sont de pauvres gens (en grec) qui voudraient vivre, qui sont eux, écrasés par un destin extérieur, mais qui les broie jusqu'au fond d'eux-mêmes (Ajax, Œdipe, Électre).\par
Humanité commune. La tragédie de Racine est bien une tragédie de cour. Le pouvoir seul peut créer un tel désert dans les âmes. Poète inhumain, car si telle était la « condition humaine », comme le dit T., tout le monde serait déjà mort...\par
C'est toujours l'orgueil, qui est humilié dans Racine. (Avec quelle insolence et quelle cruauté... Tu pleures, malheureuse... Et d'un cruel refus...) C'est la fierté dans Homère, Sophocle.\par
Comparer :\par

\begin{quoteblock}
 \noindent Andromaque, sans vous, \\
{\itshape n'aurait jamais d'un maître embrassé les genoux}
 \end{quoteblock}

\noindent (ça, c'est l'esclavage du courtisan, la servitude non physique ; il est clair que l'Andromaque de Racine ne porte pas d'eau, ne tisse pas la laine. C'est d'une manière bien différente qu'on est humilié par un contremaître...)\par
Fragments    163\par
et : ... (en grec)……………………………\par
……………………………………………\par
……………………………………………… \footnote{\noindent {\itshape Iliade}, VI, 456-458 :\par
... tu tisseras la toile pour une autre\par
Et tu porteras J'eau de la Messéis ou l'Hypérée,\par
Bien malgré toi, sous la pression d'une dure nécessité.
}\par
Le pouvoir. Ses espèces, ses degrés, la profonde transformation qu'il opère dans les âmes. Capitaine et matelot (Peisson). Chef d'atelier (Mouquet) et ouvrier...\par
Autre chose : dans Homère, Achille sait courir, etc. Hector, dompteur de chevaux. Ulysse. Ds Sophocle : Philoctète, etc. Aux héros de Racine, il ne reste que le pouvoir {\itshape pur}, sans aucun savoir-faire. (Hippolyte, personnage sacrifié, car lui justement ne court pas à la mort.) Pas étonnant que Racine ait eu la vie privée la plus paisible. Ses tragédies sont en somme froides, elles n'ont rien de douloureux. Seul est douloureux le sort de l'homme de cœur qui veut vivre et ne peut y arriver (Ajax).\par
(Les personnages de Racine sont précisément des abstractions en ce sens qu'ils sont déjà morts.) [Qui donc disait : Quand Racine écrit le mot : mort, il ne pense pas à la mort ? Rien de plus vrai. Cf. sa peur extrême de mourir. Au lieu que pour ses héros, comme Tal. l'a bien vu, la mort est une détente. Il faut n'avoir que 25 ans pour croire que ça, c'est un poète humain...]\par
{\itshape Questions à me poser} :\par
Part du « tour de main » dans le travail à la machine. Caractère plus ou moins conscient de ce tour de main.\par
< Cf. magasinier, et, au contraire, régleurs, notamment cette brute épaisse de Léon. >\par

\begin{center}
*\end{center}
\noindent Idée universelle du travail mécanique : combinaison de mouvements, ex. : fraisage, faire apparaître la pure idée dans ces exemples bien ordonnés...\par
< Chartier n'a qu'une vue superficielle et primaire du machinisme. >\par
Analogie entre travail et géométrie...\par
La physique serait à diviser en 2 parties :\par
1° les phénomènes naturels qui sont des objets de contemplation (astronomie) ;\par
2° les phénomènes naturels qui sont matière et obstacle du travail.\par
Il faudrait ne pas séparer géométrie, physique et mécanique [pratique]...\par
{\itshape Nouvelle méthode de raisonner} qui soit absolument {\itshape pure} – et à la fois intuitive et concrète.\par
Descartes est encore trop peu dégagé du syllogisme.\par
Re-méditer sur la « connaissance du 3\textsuperscript{e} genre » – à lier au théorème « plus le corps est apte... plus l'âme aime Dieu ».\par

\begin{center}
*\end{center}
\noindent Savoir s'il y a dans l'entreprise des problèmes – des difficultés – des complications ou dépenses évitables – dont {\itshape personne} ne s'occupe, parce que personne n’en a la responsabilité. Mais comment savoir ? Interroger Det. ? Difficile, puisque par définition il ignorerait ces choses.\par

\begin{center}
*\end{center}
\noindent Le travail peut être pénible (même très pénible) de deux manières. La peine peut être ressentie comme celle d'une lutte victorieuse sur la matière et sur soi ({\itshape four}), ou comme celle d'une servitude dégradante (les 1.000 pièces de cuivre à 0,45 \%, de la 6\textsuperscript{e} et 7\textsuperscript{e} semaine, etc.). [Il y a des intermédiaires, il me semble.] À quoi tient la différence ? Le salaire y est, je crois, pour quelque chose. Mais le facteur essentiel, c'est certainement la nature de la peine. Ce serait à étudier de près afin de discriminer nettement, et, si possible, classer.\par

\begin{center}
*\end{center}
\noindent Une {\itshape critique} de la mathématique serait relativement facile. Il faudrait la faire sous un angle tout à fait matérialiste : les {\itshape instruments} (signes) ont trahi les grands esprits que furent Descartes, Lagrange, Gallois, et tant d'autres. Descartes, dans les {\itshape Regulae}, a aperçu que la question des signes était l'essentielle, et non pas seulement leur exactitude et leur précision, mais des qualités en apparence secondaires telles que la maniabilité, la facilité, etc., qui semblent ne comporter que des différences de degré ; mais en réalité il en est tout autrement, et là plus qu'ailleurs « la quantité se change en qualité ». Mais Descartes s'est arrêté à mi-chemin, et sa {\itshape Géométrie} est presque d'un mathématicien vulgaire (quoique de 1\textsuperscript{er} ordre). Une critique minutieuse des signes serait facile et utile. Mais un aperçu positif, là est la grande affaire.\par
< Signes et bureaucratie. >\par
Chercher les conditions {\itshape matérielles} de la pensée claire.\par
Combien il serait facile (et difficile !) de trouver de la joie dans {\itshape tous} les contacts avec le monde...\par

\begin{center}
*\end{center}
\noindent En quoi consiste la difficulté de l'exercice de l'entendement ? En ce qu'on ne peut véritablement réfléchir que sur le particulier, alors que l'objet de la réflexion est par essence l'universel. On ignore comment les Grecs ont résolu cette difficulté. Les modernes l'ont résolue par des signes {\itshape représentant ce qui est commun à plusieurs choses} ; or cette solution n'est pas bonne. La mienne est...\par
[Descartes aurait vu le décalage formidable entre les {\itshape Regulae} et la {\itshape Géométrie} sans la faute [impardonnable d'avoir rédigé celle-ci en mathématicien vulgaire.]\par
Des 2 manières de comprendre une démonstration...\par

\begin{center}
*\end{center}
\noindent Dans TOUTE opération mathématique, il y a deux choses à distinguer :\par
1° Des {\itshape signes} étant donnés, avec des lois conventionnelles, que peut-on savoir de leurs rapports mutuels ? Il faudrait arriver à une conception assez claire des combinaisons de signes pour former une théorie universelle de {\itshape toutes} les combinaisons de signes prises {\itshape comme telles} (théorie des groupes ?).\par
2° Rapport entre les combinaisons de signes et les problèmes réels que pose la nature (ce rapport consistant {\itshape toujours} en une {\itshape analogie}).\par
\par
En ce qui concerne les combinaisons de signes prises comme telles, il faudrait un catalogue complet des difficultés – en tenant compte de celles qui concernent le temps et l'espace.\par
Quant à l'application, une étude clairvoyante laisserait sans doute apercevoir qu'elle repose non point sur la propriété de représenter les choses qui seraient contenues dans les signes (qualité occulte), mais sur une {\itshape analogie des opérations}.\par
{\itshape Il faudrait une liste des applications de la mathématique}.\par
Il n'existe pas de conception générale de la science...\par
Mouvement ascendant et descendant perpétuel des choses aux symboles [aux symboles de plus en plus abstraits] et des symboles aux choses. Ex. : géométrie et théorie des groupes (invariants....) [continu – discontinu....].\par
Faire une liste des difficultés que comportent les travaux ? – difficile.\par
Et une série des travaux ? La mécanique ayant {\itshape le plus de rapports} avec la mathématique.\par
Aussi {\itshape série des signes} dans l'effort perpétuel de ceux qui les créent pour en rendre les combinaisons de plus en plus analogues aux conditions {\itshape réelles} du travail humain.\par

\begin{center}
*\end{center}
\noindent Maître et serviteur. Aujourd'hui, serviteurs {\itshape absolument} serviteurs, sans le retournement hégélien.\par
C'est à cause de la maîtrise des forces de la nature...\par

\begin{center}
*\end{center}
\noindent Dans toutes les autres formes d'esclavage, l'esclavage est dans les circonstances. Là seulement il est transporté dans le travail lui-même.\par
Effets de l'esclavage sur l'âme.\par

\begin{center}
*\end{center}
\noindent Ce qui compte dans une vie humaine, ce ne sont pas les événements qui y dominent le cours des années – ou même des mois – ou même des jours. C'est la manière dont s'enchaîne une minute à la suivante, et ce qu'il en coûte à chacun dans son corps, dans son cœur, dans son âme – et par-dessus tout dans l'exercice de sa faculté d'attention – pour effectuer minute par minute cet enchaînement.\par
Si j'écrivais un roman, je ferais quelque chose d'entièrement nouveau.\par
Conrad : union entre le vrai marin (chef, évidemment...) et son bateau, telle que chaque ordre doit venir par inspiration, sans hésitation ni incertitude. Cela suppose {\itshape un régime de l'attention} très différent et de la réflexion et du travail asservi.\par
Questions :\par
1° Y a-t-il parfois une pareille union entre un ouvrier et se machine ? (Difficile à savoir.)\par
2° Quelles sont les conditions d'une telle union\par
1] Dans la structure de la machine.\par
2] Dans la culture technique de l'ouvrier.\par
3] Dans la nature des travaux.\par
Cette union est évidemment la condition d'un bonheur plein. Elle seule fait du travail un équivalent de l'art.
\section[Lettres, à un ingénieur Directeur d'usine ]{Lettres \\
à un ingénieur Directeur d'usine \protect\footnotemark }\renewcommand{\leftmark}{Lettres \\
à un ingénieur Directeur d'usine }

\footnotetext{ Cet ingénieur avait fondé une petite revue ouvrière, {\itshape Entre Nous}.}
\noindent \par
\noindent (Bourges, janvier-juin 1936)\par
\noindent Bourges, le 13 janvier 1936.\par
Monsieur,\par
Je ne peux pas dire que votre réponse m'ait étonnée. J'en espérais une autre, mais sans trop y compter.\par
Je n'essaierai pas de défendre le texte \footnote{Voir le texte à la suite de la lettre.} que vous avez refusé. Si vous étiez catholique, je ne résisterais pas à la tentation de vous montrer que l'esprit qui inspirait mon article, et qui vous a choqué, n'est pas autre chose que l'esprit chrétien pur et simple ; je crois que cela ne me serait pas difficile. Mais je n'ai pas lieu d'user de tels arguments avec vous. D'ailleurs je ne veux pas discuter. Vous êtes le chef, et n'avez pas à rendre compte de vos décisions.\par
Je veux seulement vous dire que la « tendance » qui vous a semblé inadmissible avait été développée par moi à dessein et de propos délibéré. Vous m'avez dit – je répète vos propres termes – qu'il est très difficile d'élever les ouvriers. Le premier des principes pédagogiques, c'est que pour élever quelqu'un, enfant ou adulte, il faut d'abord l'élever à ses propres yeux. C'est cent fois plus vrai encore quand le principal obstacle au développement réside dans des conditions de vie humiliantes.\par
Ce fait constitue pour moi le point de départ de toute tentative efficace d'action auprès des masses populaires, et surtout des ouvriers d'usine. Et, je le comprends bien, c'est précisément ce point de départ que vous n'admettez pas. Dans l'espoir de vous le faire admettre, et parce que le sort de huit cents ouvriers est entre vos mains, je m'étais fait violence pour vous dire sans réserves ce que mon expérience m'avait laissé sur le cœur. J'ai dû faire un pénible effort sur moi-même pour vous dire de ces choses qu'il est à peine supportable de confier à ses égaux, dont il est intolérable de parler devant un chef. Il m'avait semblé vous avoir touché. Mais j'avais sans doute tort d'espérer qu'une heure d'entretien puisse l'emporter sur la pression des occupations quotidiennes. Commander ne rend pas facile de se mettre à la place de ceux qui obéissent.\par
À mes yeux, la raison d'être essentielle de ma collaboration à votre journal résidait dans le fait que mon expérience de l'an passé me permet peut-être d'écrire de manière à alléger un peu le poids des humiliations que la vie impose jour par jour aux ouvriers de R., comme à tous les ouvriers des usines modernes. Ce n'est pas là le seul but, mais c'est, j'en suis convaincue, la condition essentielle pour élargir leur horizon. Rien ne paralyse plus la pensée que le sentiment d'infériorité nécessairement imposé par les atteintes quotidiennes de la pauvreté, de la subordination, de la dépendance. La première chose à faire pour eux, c'est de les aider à retrouver ou à conserver, selon le cas, le sentiment de leur dignité. Je ne sais que trop combien il est difficile, dans une pareille situation, de conserver ce sentiment, et combien tout appui moral peut être alors précieux. J'espérais de tout mon cœur pouvoir, par ma collaboration à votre journal, apporter un petit peu un tel appui aux ouvriers de R.\par
Je ne crois pas que vous vous fassiez une idée exacte de ce qu'est au juste l'esprit de classe. À mon avis, il ne peut guère être excité par de simples paroles prononcées ou écrites. Il est déterminé par les conditions de vie effectives. Les humiliations, les souffrances imposées, la subordination le suscitent ; la pression inexorable et quotidienne de la nécessité ne cesse pas de le réprimer, et souvent au point de le tourner en servilité chez les caractères les plus faibles. En dehors de moments exceptionnels qu'on ne peut, je crois, ni amener ni éviter, ni même prévoir, la pression de la nécessité est toujours largement assez puissante pour maintenir l'ordre ; car le rapport des forces n'est que trop clair. Mais si l'on pense à la santé morale des ouvriers, le refoulement perpétuel d'un esprit de classe qui couve toujours sourdement à un degré quelconque va presque partout beaucoup plus loin qu'il ne serait souhaitable. Donner parfois expression à cet esprit – sans démagogie, bien entendu – ce ne serait pas l'exciter, mais au contraire en adoucir l'amertume. Pour les malheureux, leur infériorité sociale est infiniment plus lourde à porter du fait qu'ils la trouvent présentée partout comme quelque chose qui va de soi.\par
Surtout je ne vois pas comment des articles comme le mien pourraient avoir mauvais effet étant publiés dans votre journal. Dans tout autre journal, ils pourraient à la rigueur sembler tendre à dresser les pauvres contre les riches, les subordonnés contre les chefs ; mais paraissant dans un journal contrôlé par vous, un tel article peut seulement donner aux ouvriers le sentiment qu'on fait un pas vers eux, qu'on fait effort pour les comprendre. Je pense qu'ils vous en sauraient gré. Je suis convaincue que si les ouvriers de R. pouvaient trouver dans votre journal des articles vraiment faits pour eux, où soient soigneusement ménagées toutes leurs susceptibilités – car la susceptibilité des malheureux est vive, quoique muette –, où soit développé tout ce qui peut les élever à leurs propres yeux, il n'en résulterait que du bien à tous les points de vue.\par
Ce qui peut au contraire aviver l'esprit de classe, ce sont les phrases malheureuses qui, par l'effet d'une cruauté inconsciente, mettent indirectement l'accent sur l'infériorité sociale des lecteurs. Ces phrases malheureuses sont nombreuses dans la collection de votre journal. Je vous les signalerai à la prochaine occasion, si vous le désirez. Peut-être est-il impossible d'avoir du tact vis-à-vis de ces gens-là quand on se trouve depuis trop longtemps dans une situation trop différente de la leur.\par
Par ailleurs, il se peut que les raisons que vous me donnez pour écarter mes deux suggestions soient tout à fait justes. La question est d'ailleurs relativement secondaire.\par
\par
Je vous remercie de m'avoir envoyé les derniers numéros du journal.\par
Je m'abstiendrai de venir vous voir à R., pour la raison que je vous ai donnée, si vous restez disposé à m'y prendre comme ouvrière. Mais j'ai lieu de croire que vos dispositions à mon égard sont changées. Un tel projet, pour réussir, exige un degré fort élevé de confiance et de compréhension mutuelle.\par
Si vous n'êtes plus disposé à m'embaucher, ou si M. M*** \footnote{Le propriétaire de l'usine.} s'y oppose, je viendrai certainement à R., comme vous voulez bien m'y autoriser, dès que j'en trouverai le temps. Je vous préviendrai à l'avance.\par
Veuillez recevoir, Monsieur, l'assurance de mes sentiments distingués.\par
S. WEIL.\par

\begin{center}
Un appel aux ouvriers de R. \footnote{Voir lettre précédente.}\end{center}
\noindent Chers amis inconnus qui peinez dans les ateliers de R., je viens faire appel à vous. Je viens vous demander votre collaboration pour {\itshape Entre Nous}.\par
On n'a pas besoin de boulot supplémentaire, penserez-vous. On en a assez comme ça.\par
Vous avez bien raison. Et pourtant je viens vous demander de bien vouloir prendre une plume et du papier, et parler un peu de votre travail.\par
Ne vous récriez pas. Je sais bien : quand on a fait ses huit heures, on en a marre, on en a jusque-là, pour employer des expressions qui ont le mérite de bien dire ce qu'elles veulent dire. On ne demande qu'une chose, c'est de ne plus penser à l'usine jusqu'au lendemain matin. C'est un état d'esprit tout à fait naturel, auquel il est bon de se laisser aller. Quand on est dans cet état d'esprit, on n'a rien de mieux à faire qu'à se détendre : causer avec des copains, lire des choses distrayantes, prendre l'apéro, faire une partie de cartes, jouer avec ses gosses.\par
Mais est-ce qu'il n'y a pas aussi certains jours où cela vous pèse de ne jamais pouvoir vous exprimer, de toujours devoir garder pour vous ce que vous avez sur le cœur ? C'est à ceux qui connaissent cette souffrance-là que je m'adresse. Peut-être que quelques-uns d'entre vous ne l'ont jamais éprouvée. Mais quand on l'éprouve, c'est une vraie souffrance.\par
À l'usine, vous êtes là seulement pour exécuter des consignes, livrer des pièces conformes aux ordres reçus, et recevoir, les jours de paye, la quantité d'argent déterminée par le nombre de pièces et les tarifs. À part ça, vous êtes des hommes – vous peinez, vous souffrez, vous avez des moments de joie aussi, peut-être des heures agréables ; parfois vous pouvez vous laisser un peu aller, parfois vous êtes contraints à de terribles efforts sur vous-mêmes ; certaines choses vous intéressent, d'autres vous ennuient. Mais tout ça, personne autour de vous ne peut s'en occuper. Vous-mêmes vous êtes forcés de ne pas vous en occuper. On ne vous demande que des pièces, on ne vous donne que des sous.\par
Cette situation pèse parfois sur le cœur, n'est-il pas vrai ? Elle donne parfois le sentiment d'être une simple machine à produire.\par
Ce sont là les conditions du travail industriel. Ce n'est la faute de personne. Peut-être bien que quelques-uns, parmi vous, s'en accommodent sans peine. C'est une question de tempérament. Mais il y a des caractères qui sont sensibles à ces choses-là. Pour les hommes de ce caractère, cet état de choses est quand même trop dur.\par
Je voudrais faire servir {\itshape Entre Nous} à y remédier un peu, si vous voulez bien m'y aider.\par
Voici ce que je vous demande. Si un soir, ou bien un dimanche, ça vous fait tout d'un coup mal de devoir toujours renfermer en vous-mêmes ce que vous avez sur le cœur, prenez du papier et une plume. Ne cherchez pas des phrases bien tournées. Employez les premiers mots qui vous viendront à l'esprit. Et dites ce que c'est pour vous que votre travail.\par
Dites si le travail vous fait souffrir. Racontez ces souffrances, aussi bien les souffrances morales que les souffrances physiques. Dites s'il y a des moments où vous n'en pouvez plus ; si parfois la monotonie du travail vous écœure ; si vous souffrez d'être toujours préoccupés par la nécessité d'aller vite ; si vous souffrez d'être toujours sous les ordres des chefs.\par
Dites aussi si vous éprouvez parfois la joie du travail, la fierté de l'effort accompli. S'il vous arrive de vous intéresser à votre tâche. Si certains jours vous avez plaisir à sentir que ça va vite, et que par suite vous gagnez bien. Si quelquefois vous pouvez passer des heures à travailler machinalement, presque sans vous en apercevoir, en pensant à autre chose, en vous perdant dans des rêveries agréables. Si vous êtes parfois contents de n'avoir qu'à exécuter les tâches qu'on vous donne sans avoir besoin de vous casser la tête.\par
Dites, d'une manière générale, si vous trouvez le temps long à l'usine, ou si vous le trouvez court. Peut-être bien que ça dépend des jours. Cherchez alors à vous rendre compte de quoi ça dépend au juste.\par
Dites si vous êtes pleins d'entrain quand vous allez au travail, ou si tous les matins vous pensez : « Vivement la sortie ! » Dites si vous sortez gaiement le soir, ou bien épuisés, vidés, assommés par la journée de travail.\par
Dites enfin si, à l'usine, vous vous sentez soutenus par le sentiment réconfortant de vous trouver au milieu de copains, ou si au contraire vous vous sentez seuls.\par
Surtout dites tout ce qui vous viendra à l'esprit, tout ce qui vous pèse sur le cœur.\par
Et quand vous aurez fini d'écrire, il sera tout à fait inutile de signer. Vous tâcherez même de vous arranger pour qu'on ne puisse pas deviner qui vous êtes.\par
Même, comme cette précaution risque de ne pas suffire, vous en prendrez une autre, si vous le voulez bien. Au lieu d'envoyer ce que vous aurez écrit à {\itshape Entre Nous}, vous me l'enverrez à moi. Je recopierai vos articles pour {\itshape Entre Nous}, mais en les arrangeant de manière que personne ne puisse s'y reconnaître. Je couperai un même article en plusieurs morceaux, je mettrai parfois ensemble des morceaux d'articles différents. Les phrases imprudentes, je m'arrangerai pour qu'on ne puisse même pas savoir de quel atelier elles viennent. S'il y a des phrases qu'il me semble dangereux pour vous de publier même avec ces précautions, je les supprimerai. Soyez sûrs que je ferai bien attention. Je sais ce que c'est que la situation d'un ouvrier dans une usine. Je ne voudrais pour rien au monde que par ma faute il arrive un coup dur à l'un de vous.\par
De cette manière vous pourrez vous exprimer librement, sans aucune préoccupation de prudence. Vous ne me connaissez pas. Mais vous sentez bien, n'est-ce pas, que je désire seulement vous servir, que pour rien au monde je ne voudrais vous nuire ? Je n'ai aucune responsabilité dans la fabrication des cuisinières. Ce qui m'intéresse, c'est seulement le bien-être physique et moral de ceux qui les fabriquent.\par
Exprimez-vous bien sincèrement. N'atténuez rien, n'exagérez rien, ni en bien ni en mal. Je pense que cela vous soulagera un peu de dire la vérité sans réserves.\par
Vos camarades vous liront. S'ils sentent comme vous, ils seront bien contents de voir imprimées des choses qui peut-être remuaient au fond de leur cœur sans pouvoir se traduire par des mots ; ou peut-être des choses qu'ils auraient su exprimer, mais taisaient par force. S'ils sentent autrement, ils prendront la plume à leur tour pour s'expliquer. De toutes manières vous vous comprendrez mieux les uns les autres. La camaraderie ne pourra qu'y gagner, et ce sera déjà un grand bien.\par
Vos chefs aussi vous liront. Ce qu'ils liront ne leur fera peut-être pas toujours plaisir. Ça n'a pas d'importance, Ça ne leur fera pas de mal d'entendre des vérités désagréables.\par
Ils vous comprendront bien mieux après vous avoir lus. Bien souvent des chefs qui au fond sont des hommes bons se montrent durs, simplement parce qu'ils ne comprennent pas. La nature humaine est faite comme ça. Les hommes ne savent jamais se mettre à la place les uns des autres.\par
Peut-être qu'ils trouveront moyen de remédier au moins en partie à certaines des souffrances que vous aurez signalées. Ils montrent beaucoup d'ingéniosité dans la fabrication des cuisinières, vos chefs. Qui sait s'ils ne pourraient pas faire aussi preuve d'ingéniosité dans l'organisation de conditions de travail plus humaines ? La bonne volonté ne leur manque sûrement pas. La meilleure preuve, c'est que ces lignes paraissent dans {\itshape Entre Nous.}\par
Malheureusement leur bonne volonté ne suffit pas. Les difficultés sont immenses. Tout d'abord l'impitoyable loi du rendement pèse sur vos chefs comme sur vous ; elle pèse d'un poids inhumain sur toute la vie industrielle. On ne peut pas passer outre. Il faut s'y plier, aussi longtemps qu'elle existe. Tout ce qu'on peut faire provisoirement, c'est d'essayer de tourner les obstacles à force d'ingéniosité; c'est chercher l'organisation la plus humaine compatible avec un rendement donné.\par
Seulement voici ce qui complique tout. Vous êtes ceux qui supportez le poids du régime industriel ; et ce n'est pas vous qui pouvez résoudre ou même poser les problèmes d'organisation. Ce sont vos chefs qui ont la responsabilité de l'organisation. Or vos chefs, comme tous les hommes, jugent les choses de leur point de vue et non du vôtre. Ils ne se rendent pas bien compte de la manière dont vous vivez. Ils ignorent ce que vous pensez. Même ceux qui ont été ouvriers ont oublié bien des choses.\par
Ce que je vous propose vous permettrait peut-être de leur faire comprendre ce qu'ils ne comprennent pas, et cela sans danger et sans humiliation pour vous. De leur côté, peut-être qu'en réponse ils se serviront à leur tour d'{\itshape Entre Nous}. Peut-être vous feront-ils part des obstacles que leur imposent les nécessités de l'organisation industrielle.\par
La grande industrie est ce qu'elle est. Le moins qu'on puisse en dire, c'est qu'elle impose de dures conditions d'existence. Mais il ne dépend ni de vous ni des patrons de la transformer dans un avenir prochain.\par
Dans une pareille situation, voici, il me semble, quel serait l'idéal. Il faudrait que les chefs comprennent quel est au juste le sort des hommes qu'ils utilisent comme main-d'œuvre. Et il faudrait que leur préoccupation dominante soit non d'augmenter toujours le rendement au maximum, mais d'organiser les conditions de travail les plus humaines compatibles avec le rendement indispensable à l'existence de l'usine.\par
Il faudrait d'autre part que les ouvriers connaissent et comprennent les nécessités auxquelles la vie de l'usine est soumise. Ils pourraient ainsi contrôler et apprécier la bonne volonté des chefs. Ils perdraient le sentiment d'être soumis à des ordres arbitraires, et les souffrances inévitables deviendraient peut-être moins amères à supporter.\par
Bien sûr, cet idéal n'est pas réalisable. Les préoccupations quotidiennes pèsent beaucoup trop sur les uns et sur les autres. D'ailleurs la relation de chef à subordonné n'est pas de celles qui facilitent la compréhension mutuelle. On ne comprend jamais tout à fait ceux à qui on donne des ordres. On ne comprend jamais tout à fait non plus ceux de qui on reçoit des ordres.\par
Mais cet idéal, on peut peut-être un peu s'en approcher. Il dépend maintenant de vous d'essayer. Même si vos petits articles n'ont pas pour résultat de sérieuses améliorations pratiques, vous aurez toujours la satisfaction d'avoir une fois exprimé votre point de vue à vous.\par
Ainsi c'est entendu, n'est-ce pas ? Je compte bien recevoir bientôt beaucoup d'articles.\par
Je ne veux pas terminer sans remercier de tout cœur M. B. pour avoir bien voulu publier cet appel.\par

\begin{center}
*\end{center}
\noindent \par
\noindent Bourges, 31 janvier 1936.\par
\noindent Monsieur,\par
Votre lettre supprime toutes les raisons qui me détournaient d'aller à R. J'irai donc vous voir, sauf avis contraire de votre part, le vendredi 14 février après déjeuner.\par
Vous jugez la manière dont je me représente les conditions morales de vie des ouvriers trop poussée au noir. Que vous répondre, sinon vous répéter – si pénible que soit un pareil aveu – que j'ai eu, moi, tout le mal du monde à conserver le sentiment de ma dignité ? À parler plus franc, je l'ai à peu près perdu sous le premier choc d'un si brutal changement de vie, et il m'a fallu péniblement le retrouver. Un jour je me suis rendu compte que quelques semaines de cette existence avaient presque suffi à me transformer en bête de somme docile, et que le dimanche seulement je reprenais un peu conscience de moi-même. Je me suis alors demandé avec effroi ce que je deviendrais si jamais les hasards de la vie me mettaient dans le cas de travailler de la sorte sans repos hebdomadaire. Je me suis juré de ne pas sortir de cette condition d'ouvrière avant d'avoir appris à la supporter de manière à y conserver intact le sentiment de ma dignité d'être humain. Je me suis tenu parole. Mais j'ai éprouvé jusqu'au dernier jour que ce sentiment était toujours à reconquérir, parce que toujours les conditions d'existence l'effaçaient et tendaient à me ravaler à la bête de somme.\par
Il me serait facile et agréable de me mentir un peu à moi-même, d'oublier tout cela. Il m'aurait été facile de ne pas l'éprouver, si seulement j’avais fait cette expérience comme une sorte de jeu, à la manière d'un explorateur qui va vivre au milieu de peuplades lointaines, mais sans jamais oublier qu'il leur est étranger. Bien au contraire j'écartais systématiquement tout ce qui pouvait me rappeler que cette expérience était une simple expérience.\par
Vous pouvez mettre en question la légitimité d'une généralisation. Je l'ai fait moi-même. Je me suis dit que peut-être ce n'étaient pas les conditions de vie qui étaient trop dures, mais la force de caractère qui me manquait. Pourtant elle ne me manquait pas tout à fait, puisque j'ai su tenir jusqu'à la date que je m'étais d'avance assignée.\par
J'étais, il est vrai, très inférieure en résistance physique à la plupart de mes camarades – heureusement pour eux. Et la vie d'usine est tout autrement opprimante quand elle pèse sur le corps vingt-quatre heures sur vingt-quatre, ce qui était assez souvent mon cas, que quand elle pèse seulement huit heures, ce qui est le cas des plus costauds. Mais d'autres circonstances compensaient dans une large mesure cette inégalité.\par
Au reste plus d'une confidence ou demi-confidence d'ouvrier est venue confirmer mes impressions.\par
Reste la question de la différence entre R. et les usines que j'ai connues. En quoi peut consister cette différence ? Je mets à part la proximité de la campagne. Dans les dimensions ? Mais ma première usine était une usine de 300 ouvriers, et où le directeur avait l'impression de bien connaître son personnel. Dans les œuvres sociales ? Quelle qu'en puisse être l'utilité matérielle, moralement elles ne font, je le crains, qu'accroître la dépendance. Dans les fréquents contacts entre supérieurs et inférieurs ? Je me représente mal qu'ils puissent constituer un réconfort moral pour les inférieurs. Y a-t-il encore autre chose ? Je ne demande qu'à me rendre compte.\par
Ce que vous m'avez raconté du silence observé par tous ceux qui assistaient à la dernière assemblée générale de la coopérative ne confirme que trop, il me semble, mes suppositions. Vous n'y êtes pas allé, de peur de leur ôter le courage de parler – et néanmoins personne n'a rien osé dire. Les résultats constants des élections municipales me paraissent eux aussi significatifs. Et enfin je ne puis oublier les regards des mouleurs, quand je passais parmi eux aux côtés du fils du patron.\par
Votre argument le plus puissant pour moi, quoiqu'il soit absolument sans rapport avec la question, c'est l'impossibilité où vous seriez de me croire sans perdre du même coup presque tout stimulant pour le travail. Effectivement, je ne me verrais guère, moi, à la tête d'une usine, à supposer même que je possède les capacités nécessaires. Cette considération ne change rien à ma manière de voir, mais m'ôte dans une large mesure le désir de vous la faire partager. Ce n'est pas de gaieté de cœur, croyez-le bien, que je me détermine à dire des choses démoralisantes. Mais devrais-je, sur une pareille question, vous cacher ce que je pense être la vérité ?\par
Il faut me pardonner si je prononce le mot de chef avec un peu trop d'amertume. Il est bien difficile qu'il en soit autrement quand on a subi une subordination totale, et qu'on n'oublie pas. Il est tout à fait exact que vous aviez pris soin de me donner toutes vos raisons concernant mon article, et que je n'avais pas le droit de m'exprimer comme je l'ai fait à ce sujet.\par
Vous exagérez un peu en supposant que je mets à votre compte un passif écrasant et un actif nul. Ce que je mets au passif, je le mets au passif de la fonction plutôt que de l'homme. Et à l'actif je sais tout au moins qu'il y a à mettre des intentions. J'admets volontiers qu'il y a aussi des réalisations ; je suis seulement convaincue qu'il y en a beaucoup moins et d'une portée beaucoup moindre qu'on ne se trouve amené à croire quand on voit les choses d'en haut. On est très mal placé en haut pour se rendre compte et en bas pour agir. Je pense que c'est là, d'une manière générale, une des causes essentielles des malheurs humains. C'est pourquoi j'ai tenu à aller moi-même tout en bas, et y retournerai peut-être. C'est pourquoi aussi je voudrais tant pouvoir, dans quelque entreprise, collaborer d'en bas avec celui qui la dirige. Mais c'est sans doute une chimère.\par
Je pense que je ne conserverai de nos relations aucune amertume personnelle, au contraire. Pour moi qui ai choisi délibérément et presque sans espérance de me placer au point de vue de ceux d'en bas, il est réconfortant de pouvoir m'entretenir à cœur ouvert avec un homme tel que vous. Cela aide à ne pas désespérer des hommes, à défaut des institutions. L'amertume que j'éprouve concerne uniquement mes camarades inconnus des ateliers de R., pour qui je dois renoncer à tenter quoi que ce soit. Mais je n'ai à m'en prendre qu'à moi-même de m'être laissée aller à des espérances déraisonnables.\par
Quant à vous, je ne peux que vous remercier de bien vouloir vous prêter à des entretiens dont j'ignore si vous pouvez tirer quelque profit, mais qui sont précieux pour moi.\par
Veuillez agréer l'assurance de mes sentiments distingués.\par
S. WEIL.\par

\begin{center}
*\end{center}
\noindent Bourges, le 3 mars 1936.\par
\noindent Monsieur,\par
\par
Je crois qu'il y a avantage entre nous à faire alterner les échanges de vues écrits et oraux ; d'autant plus que j'ai l'impression de n'avoir pas su me faire bien comprendre, lors de notre dernière entrevue.\par
Je n'ai pu vous citer aucun cas concret de mauvais accueil de la part d'un chef à une plainte légitime d'ouvrier. Comment aurais-je pu risquer d'en faire l'expérience ? Si j'avais rencontré un pareil accueil, le subir en silence – comme j'aurais probablement fait – aurait été une humiliation bien autrement douloureuse que la chose même dont j'aurais pu avoir à me plaindre. Répliquer sous l'empire de la colère aurait probablement signifié devoir aussitôt chercher une nouvelle place. Bien sûr, on ne sait pas d'avance qu'on sera mal accueilli, mais on sait que c'est possible, et cette possibilité suffit. C'est possible parce qu'un chef, comme tout homme, a ses moments d'humeur. Et puis on a le sentiment qu'il n'est pas normal, dans une usine, de prétendre à une considération quelconque. Je vous ai raconté comment un chef, en me contraignant à risquer, deux heures durant, de me faire assommer par un balancier, m'a fait sentir pour la première fois pour combien au juste je comptais : à savoir zéro. Par la suite, toutes sortes de petites choses m'ont rafraîchi la mémoire à ce sujet. Exemple : dans une autre usine, on ne pouvait entrer qu'au signal d'une sonnerie, dix minutes avant l'heure ; mais avant la sonnerie, une petite porte pratiquée dans le grand portail était ouverte. Les chefs arrivés en avance passaient par là ; les ouvrières – moi-même plus d'une fois parmi elles – attendaient bien patiemment dehors, devant cette porte ouverte, même sous une pluie battante. {\itshape Etcetera}...\par
Sans doute on peut prendre le parti de se défendre fermement, au risque de changer de place ; mais celui qui prend ce parti, il y a bien des chances pour qu'il ne le tienne pas longtemps, et dès lors mieux vaut commencer par ne pas le prendre. Actuellement, dans l'industrie, pour qui n'a pas de certificats de chef ou de bon professionnel, chercher une place – errer de boîte en boîte en se livrant à des calculs avant d'oser acheter un billet de métro, stationner indéfiniment devant les bureaux d'embauche, être repoussé et revenir jour après jour – c'est une expérience où on laisse une bonne partie de sa fierté. Du moins je l'ai observé autour de moi et d'abord sur moi-même. Je reconnais qu'on peut en conclure purement et simplement que je n'ai rien dans le ventre ; je me le suis dit à moi-même plus d'une fois.\par
En tout cas ces souvenirs me font trouver tout à fait normale la réponse de votre ouvrier communiste. Je dois vous l'avouer, ce que vous m'avez dit à ce sujet m'est resté sur le cœur. Que vous ayez, vous, autrefois, fait preuve de plus de courage envers des chefs, cela ne vous donne pas le droit de le juger. Non seulement les difficultés économiques n'étaient pas comparables, mais encore votre situation morale était tout autre, si du moins, comme j'ai cru le comprendre, vous occupiez à ces moments des postes plus ou moins responsables. Pour moi, à risques égaux ou même plus grands, je résisterai, je pense, le cas échéant, à mes chefs universitaires (s'il survient quelque gouvernement autoritaire) avec une tout autre fermeté que je ne ferais dans une usine devant le contremaître ou le directeur. Pourquoi ? Sans doute pour une raison analogue à celle qui rendait le courage plus facile pendant la guerre à un gradé qu'à un soldat – fait bien connu des anciens combattants, et que j'ai entendu signaler plus d'une fois. Dans l'Université, j'ai des droits, une dignité et une responsabilité à défendre. Qu'ai-je à défendre comme ouvrière d'usine, alors que je dois chaque jour renoncer à toute espèce de droits à l'instant même où je pointe à la pendule ? Je n'ai à y défendre que ma vie. S'il fallait à la fois subir la subordination de l'esclave et courir les dangers de l'homme libre, ce serait trop. Forcer un homme qui se trouve dans une telle situation à choisir entre se mettre en danger et se défiler, comme vous dites, c'est lui infliger une humiliation qu'il serait plus humain de lui épargner.\par
Ce que vous m'avez raconté au sujet de la réunion de la coopérative, quand vous me disiez –avec une nuance de dédain, me semblait-il – que personne n'avait osé y parler, m'avait inspiré des réflexions analogues. N'y a-t-il pas là une situation pitoyable ? On se trouve, sans aucun recours, sous le coup d'une force complètement hors de proportion avec celle qu'on possède, force sur laquelle on ne peut rien, par laquelle en risque constamment d'être écrasé – et quand, l'amertume au cœur, on se résigne à se soumettre et à plier, on se fait mépriser pour manque de courage par ceux mêmes qui manient cette force.\par
Je ne puis parler de ces choses sans amertume, mais croyez bien qu'elle n'est pas dirigée contre vous ; il y a là une situation de fait dans laquelle, somme toute, il ne serait sans doute pas juste de vous assigner une plus grande part de responsabilité qu'à moi-même ou à n'importe qui.\par
Pour revenir à la question des rapports avec les chefs, j'avais, pour moi, une règle de conduite bien ferme. Je ne conçois les rapports humains que sur le plan de l'égalité ; dès lors que quelqu'un s'est mis à me traiter en inférieure, il n'y a plus à mes yeux de rapports humains possibles entre lui et moi, et je le traite à mon tour en supérieur, c'est-à-dire que je subis son pouvoir comme je subirais le froid ou la pluie. Un aussi mauvais caractère est peut-être exceptionnel ; cependant soit fierté, soit timidité, soit mélange des deux, j'ai toujours vu que le silence est à l'usine un phénomène général. J'en sais des exemples bien frappants.\par
Si je vous ai proposé d'établir une boîte à suggestions concernant non plus la production, mais le bien-être des ouvriers, c'est que cette idée m'était venue à l'usine. Un pareil procédé éviterait tout risque d'humiliation – vous me direz que vous recevez toujours bien les ouvriers, mais savez-vous vous-même si vous n'avez pas vous aussi des moments d'humeur ou des ironies déplacées ? – il constituerait une invitation formelle de la part de la direction, et puis, rien qu'à voir la boîte dans l'atelier, on aurait un peu moins l'impression de compter pour rien.\par
J'ai tiré en somme deux leçons de mon expérience. La première, la plus amère et la plus imprévue, c'est que l'oppression, à partir d'un certain degré d'intensité, engendre non une tendance à la révolte, mais une tendance presque irrésistible à la plus complète soumission. Je l'ai constaté sur moi-même, moi qui pourtant, vous l'avez deviné, n'ai pas un caractère docile ; c'est d'autant plus concluant. La seconde, c'est que l'humanité se divise en deux catégories, les gens qui comptent pour quelque chose et les gens qui comptent pour rien. Quand on est dans la seconde, on en arrive à trouver naturel de compter pour rien – ce qui ne veut certes pas dire qu'on ne souffre pas. Moi, je le trouvais naturel. Tout comme, malgré moi, j'en arrive à trouver à présent presque naturel de compter pour quelque chose. (Je dis malgré moi, car je m'efforce de réagir ; tant j'ai honte de compter pour quelque chose, dans une organisation sociale qui foule aux pieds l'humanité.) La question, pour l'instant, est de savoir si, dans les conditions actuelles, on peut arriver dans le cadre d'une usine à ce que les ouvriers comptent et aient conscience de compter pour quelque chose. Il ne suffit pas à cet effet qu'un chef s'efforce d'être bon pour eux ; il faut bien autre chose.\par
À mon sens, il faudrait d'abord à cet effet qu'il soit bien entendu entre le chef et les ouvriers que cet état de choses, dans lequel eux et tant d'autres comptent pour rien, ne peut être considéré comme normal ; que les choses ne sont pas acceptables telles qu'elles sont. Certes, au fond, chacun le sait bien ; mais de part et d'autre personne n'ose y faire la moindre allusion – et, soit dit tout à fait en passant, quand un article y fait allusion, il n'est pas inséré dans le journal... Il faudrait qu'il soit bien entendu aussi que cet état de choses est dû à des nécessités objectives, et essayer de les tirer un peu au clair. L'enquête que j'imaginais devait avoir pour complément dans mon esprit (je ne sais si je l'ai marqué dans le papier que vous avez entre les mains) des exposés de vous concernant les obstacles aux améliorations demandées (organisation, rendement, etc.). Dans certains cas, des exposés d'ordre plus général seraient à y joindre. La règle de ces échanges de vues devrait être une égalité totale entre les interlocuteurs, une franchise et une clarté complètes de part et d'autre. Si on pouvait en arriver là, ce serait déjà à mes yeux un résultat. Il me semble que n'importe quelle souffrance est moins accablante, risque moins de dégrader, quand on conçoit le mécanisme des nécessités qui la causent ; et que c'est une consolation de la sentir comprise et dans une certaine mesure partagée par ceux qui ne la subissent pas. De plus, on peut peut-être obtenir des améliorations.\par
Je suis convaincue aussi que de ce côté seulement on peut trouver un stimulant intellectuel pour les ouvriers. Il faut toucher pour intéresser. À quel sentiment faire appel pour toucher des hommes dont la sensibilité est quotidiennement heurtée et comprimée par l'asservissement social ? Il faut, je crois, passer par le sentiment même qu'ils ont de cet asservissement. Je peux me tromper, à vrai dire. Mais ce qui me confirme dans cette opinion, c'est qu'on ne trouve en général que deux espèces d'ouvriers qui s'instruisent tout seuls : ou des hommes désireux de monter en grade, ou des révoltés. J'espère que cette remarque ne vous fera pas peur.\par
Si, par exemple, au cours de ces échanges de vues, l'ignorance des ouvriers arrivait à être reconnue d'un commun accord comme constituant l'un des obstacles à une organisation plus humaine, ne serait-ce pas là la seule introduction possible à une série d'articles de véritable vulgarisation ? La recherche d'une véritable méthode de vulgarisation chose complètement inconnue jusqu'à nos jours est une de mes préoccupations dominantes, et à cet égard la tentative que je vous propose me serait peut-être infiniment précieuse.\par
Bien sûr, tout cela comporte un risque. Retz disait que le Parlement de Paris avait provoqué la Fronde en levant le voile qui doit recouvrir les rapports entre les droits des rois et ceux des peuples, « droits qui ne s'accordent jamais si bien que dans le silence ». Cette formule peut s'étendre à toute espèce de domination. Si vous ne réussissiez qu'à demi dans une telle tentative, il en résulterait que les ouvriers continueraient à compter pour rien, tout en cessant de le trouver naturel ; ce qui serait un mal pour tout le monde. Courir ce risque, ce serait sans aucun doute pour vous assumer une grosse responsabilité. Mais vous refuser à le courir, ce serait aussi assumer une grosse responsabilité. Tel est l'inconvénient de la puissance.\par
À mon avis d'ailleurs vous vous exagérez ce risque. Vous semblez craindre de modifier le rapport de forces qui soumet les ouvriers à votre domination. Mais cela me paraît impossible. Deux choses seulement peuvent le modifier : ou le retour d'une prospérité économique assez grande pour que la main-d'œuvre manque, ou un mouvement révolutionnaire. Les deux sont tout à fait improbables dans un avenir prochain. Et, s'il se produisait un mouvement révolutionnaire, ce serait un souffle surgi soudain des grands centres et qui balayerait tout ; ce que vous pouvez faire ou ne pas faire à R. n'a aucune prise sur les phénomènes de cette envergure. Mais dans la mesure où on peut prédire en cette matière, il ne se produira rien de pareil, à moins peut-être d'une guerre malheureuse. Pour moi, je connais quelque peu de l'intérieur, d'une part le mouvement ouvrier français, d'autre part les masses ouvrières de la région parisienne ; et j'ai acquis la conviction, fort triste pour moi, que non seulement la capacité révolutionnaire, mais plus généralement la capacité d'action de la classe ouvrière française est à peu près nulle. Je crois que les bourgeois seuls peuvent se faire illusion à ce sujet. Nous en reparlerons, si vous voulez.\par
La tentative que je vous propose se ferait étape par étape ; vous seriez maître, à n'importe quel moment, de tout retirer et de serrer la vis. Les ouvriers n'auraient qu'à se soumettre, avec seulement plus d'amertume au cœur. Que voulez-vous qu'ils fassent d'autre ? Mais je reconnais que ce risque est encore suffisamment sérieux.\par
À vous de savoir si le risque vaut la peine d'être couru. Moi-même il me paraîtrait ridicule de se lancer à l'aveugle. Il faudrait au préalable tâter le terrain par une série de coups de sonde. Dans mon esprit, l'article que vous avez refusé devait constituer l'un de ces coups de sonde. Il serait trop long de vous exposer par écrit comment.\par
À propos du journal, j'ai le sentiment de vous avoir très mal expliqué ce qu'il y a de mauvais dans les passages que je vous ai reprochés (récits de repas confortables, etc.).\par
Je vais me servir d'une comparaison. Les murs d'une chambre, même pauvre et nue, n'ont rien de pénible à regarder; mais si la chambre est une cellule de prison, chaque regard sur le mur est une souffrance. Il en est exactement de même pour la pauvreté, quand elle est liée à une subordination et à une dépendance complètes. Comme l'esclavage et la liberté sont de simples idées, et que ce sont les choses qui font souffrir, chaque détail de la vie quotidienne où se reflète la pauvreté à laquelle on est condamné fait mal ; non pas à cause de la pauvreté, mais à cause de l'esclavage. À peu près, j'imagine, comme le bruit des chaînes pour les forçats d'autrefois. C'est ainsi aussi que font mal toutes les images du bien-être dont on est privé, quand elles se présentent de manière à rappeler qu'on en est privé ; parce que ce bien-être implique aussi la liberté. L'idée d'un bon repas dans un cadre agréable était pour moi, l'an dernier, quelque chose de poignant comme l'idée des mers et des plaines pour un prisonnier, et pour les mêmes raisons. J'avais des aspirations au luxe que je n'ai éprouvées ni avant ni depuis. Vous pouvez supposer que c'est parce que maintenant je les satisfais dans une certaine mesure. Mais non ; entre nous soit dit, je n'ai pas beaucoup changé ma manière de vivre depuis l'an dernier. Il m'a paru tout à fait inutile de perdre des habitudes que je me trouverai presque sûrement un jour ou l'autre dans le cas de devoir reprendre, soit volontairement, soit par contrainte, et que je puis conserver sans grand effort. L'an dernier, la privation la plus insignifiante par elle-même me rappelait toujours un peu que je ne comptais pas, que je n'avais droit de cité nulle part, que j'étais au monde pour me soumettre et obéir. Voilà pourquoi il n'est pas vrai que le rapport entre votre niveau de vie et celui des ouvriers soit analogue au rapport entre le vôtre et celui d'un millionnaire ; dans un cas il y a différence de degré, dans l'autre de nature. Et voilà pourquoi, quand vous avez l'occasion de faire un « gueuleton », il faut en jouir et vous taire.\par
Il est vrai que, quand on est pauvre et dépendant, on a toujours comme ressource, si l'on a l'âme forte, le courage et l'indifférence aux souffrances et aux privations. C'était la ressource des esclaves stoïciens. Mais cette ressource est interdite aux esclaves de l'industrie moderne. Car ils vivent d'un travail pour lequel, étant donné la succession machinale des mouvements et la rapidité de la cadence, il ne peut y avoir d'autre stimulant que la peur et l'appât des sous. Supprimer en soi ces deux sentiments à force de stoïcisme, c'est se mettre hors d'état de travailler à la cadence exigée. Le plus simple alors, pour souffrir le moins possible, est de rabaisser toute son âme au niveau de ces deux sentiments ; mais c'est se dégrader. Si l'on veut conserver sa dignité à ses propres yeux, on doit se condamner à des luttes quotidiennes avec soi-même, à un déchirement perpétuel, à un perpétuel sentiment d'humiliation, à des souffrances morales épuisantes ; car sans cesse on doit s'abaisser pour satisfaire aux exigences de la production industrielle, se relever pour ne pas perdre sa propre estime, et ainsi de suite. Voilà ce qu'il y a d'horrible dans la forme moderne de l'oppression sociale ; et la bonté ou la brutalité d'un chef ne peut pas y changer grand-chose. Vous apercevez clairement, je pense, que ce que je viens de dire est applicable à tout être humain, quel qu'il soit, placé dans cette situation.\par
Que faire, direz-vous encore ? Encore une fois, je crois que faire sentir à ces hommes qu'on les comprend serait déjà pour les meilleurs d'entre eux un réconfort. La question est de savoir si en fait, parmi les ouvriers travaillant actuellement à R., il y en a qui aient assez d'élévation de cœur et d'esprit pour qu'on puisse les toucher de la manière que j'imagine. Au cours de vos rapports de chef à subordonnés avec eux, vous n'avez aucun moyen de vous en rendre compte. Je crois que moi, je le pourrais, par les coups de sonde dont je vous parlais. Mais à cet effet, il faudrait que le journal ne me soit pas fermé...\par
Je vous ai dit, je crois, tout ce que j'ai à vous dire. À vous de réfléchir. Le pouvoir et la décision sont entièrement entre vos mains. Je ne puis que me mettre à votre disposition, le cas échéant ; et remarquez que je m'y mets tout entière, puisque je suis prête à me soumettre de nouveau corps et âme, pour un espace de temps indéterminé, au monstrueux engrenage de la production industrielle. Je mettrais en somme autant que vous en jeu dans l'affaire ; ce doit être pour vous une garantie de sérieux.\par
Je n'ai qu'une chose à ajouter. Croyez bien que, si vous refusez catégoriquement de vous engager dans la voie que je vous suggère, je le comprendrai très bien, et n'en resterai pas moins complètement convaincue de votre bonne volonté. Et je vous saurai toujours un gré infini d'avoir bien voulu vous entretenir avec moi à cœur ouvert comme vous avez fait.\par
Je n'ose vous parler de nouvelle entrevue, car je crains d'abuser ; pourtant j'aurais encore, pour ma propre instruction, des questions à vous poser (notamment sur vos premières études de chimie, et sur votre travail d'adaptation de l'outillage industriel pendant la guerre). Au reste, j'hésite de nouveau, pour les mêmes raisons qu'auparavant, à vous voir à l'usine. Je vous laisse le soin de régler la question.\par
\par
Croyez à mes sentiments les meilleurs.\par
S. WEIL.\par
{\itshape P.-S.} – Je n'ai plus aucun droit à vous demander de me faire le service d'{\itshape Entre Nous}, mais ça me ferait quand même bien plaisir.\par

\begin{center}
*\end{center}
\noindent Bourges, 16 mars 1936.\par
\noindent Monsieur,\par
Il faut que je m'excuse de vous accabler ainsi de mes lettres : vous devez me trouver, je le crains, de plus en plus empoisonnante... Mais votre usine m'obsède, et je voudrais en finir avec cette préoccupation.\par
Je me dis que peut-être bien ma position, entre vous et les organisations ouvrières, ne vous paraît pas nette ; que si, au cours de nos entretiens, vous avez confiance en moi (je le sens bien), vous me soupçonnez peut-être plus ou moins, après coup, de toutes sortes d'arrière-pensées. S'il en était ainsi, vous auriez tort de ne pas me le dire brutalement, et de ne pas me questionner. Il n'y a pas de véritable confiance, de véritable cordialité possible sans une franchise un peu brutale. De toutes manières, je vous dois compte de ma position en matière sociale et politique.\par
Je souhaite de tout mon cœur une transformation aussi radicale que possible du régime actuel dans le sens d'une plus grande égalité dans le rapport des forces.\par
Je ne crois pas du tout que ce qu'on nomme de nos jours révolution puisse y mener. Après comme avant une révolution soi-disant ouvrière, les ouvriers de R. continueront à obéir passivement, aussi longtemps que la production sera fondée sur l'obéissance passive. Que le directeur de R. soit sous les ordres d'un administrateur délégué représentant quelques capitalistes, ou sous les ordres d'un « trust d'État » soi-disant socialiste, la seule différence sera que dans le premier cas l'usine d'une part, la police, l'armée, les prisons, etc., de l'autre sont entre des mains différentes, et dans le second cas entre les mêmes mains. L'inégalité dans le rapport des forces n'est donc pas diminuée, mais accentuée.\par
\par
Cette considération ne me porte pourtant pas à être {\itshape contre} les partis dits révolutionnaires. Car aujourd'hui tous les groupements politiques qui comptent tendent également et à l'accentuation de l'oppression, et à la mainmise de l'État sur tous les instruments de puissance ; les uns appellent ça révolution ouvrière, d'autres fascisme, d'autres organisation de la défense nationale. Quelle que soit l'étiquette, deux facteurs priment tout : d'une part la subordination et la dépendance impliquées par les formes modernes de la technique et de l'organisation économique, d'autre part la guerre. Tous ceux qui veulent une « rationalisation » croissante d'une part, la préparation à la guerre d'autre part se valent à mes yeux, et c'est le cas pour tous.\par
En ce qui concerne les usines, la question que je me pose, tout à fait indépendante du régime politique, est celle d'un passage progressif de la subordination totale à un certain mélange de subordination et de collaboration, l'idéal étant la coopération pure.\par
En me renvoyant mon article, vous me reprochiez d'exciter un certain esprit de classe, par opposition à l'esprit de collaboration que vous voulez voir régner dans la communauté de R. Par esprit de classe, vous entendez, je suppose, esprit de révolte. Or je ne désire exciter rien de pareil. Entendons-nous bien : quand les victimes de l'oppression sociale se révoltent en fait, toutes mes sympathies vont vers eux, quoique non mêlées d'espérance ; quand un mouvement de révolte aboutit à un succès partiel, je me réjouis. Mais je ne désire pourtant absolument pas susciter l'esprit de révolte, et cela moins dans l'intérêt de l'ordre que dans l'intérêt moral des opprimés. Je sais trop bien que lorsqu'on est sous les chaînes d'une nécessité trop dure, si on se révolte un moment, on tombe sur les genoux le moment d'après. L'acceptation des souffrances physiques et morales inévitables, dans la mesure précise où elles sont inévitables, c'est le seul moyen de conserver sa dignité. Mais acceptation et soumission sont deux choses bien différentes.\par
L'esprit que je désire susciter, c'est précisément cet esprit de collaboration que vous m'opposiez. Mais un esprit de collaboration suppose une collaboration effective. Je n'aperçois à présent rien de tel à R., mais au contraire une subordination totale. C'est pourquoi j'avais rédigé cet article – qui devait, dans ma pensée, être le début d'une série – d'une manière qui pouvait vous donner l'impression d'un encouragement déguisé à la révolte ; car pour faire passer des hommes d'une subordination totale à un degré quelconque de collaboration, il faut bien, il me semble, commencer par leur faire relever la tête.\par
Je me demande si vous vous rendez compte de la puissance que vous exercez. C'est une puissance de dieu plutôt que d'homme. Avez-vous jamais pensé à ce que cela signifie, pour un de vos ouvriers, d'être renvoyé par vous ? Le plus souvent, je suppose, il faut qu'il quitte la commune pour chercher du travail. Il passe donc dans des communes où il n'a aucun droit à aucun secours. Si une malchance – trop probable dans les circonstances actuelles – prolonge vainement sa course errante de bureau d'embauche en bureau d'embauche, il descend, degré par degré, abandonné de Dieu et des hommes, absolument privé de toute espèce de recours, une pente qui, si quelque entreprise ne lui fait pas enfin l'aumône d'une place, le mènera en fin de compte non seulement à la mort lente, mais tout d'abord à une déchéance sans fond ; et cela sans qu'aucune fierté, aucun courage, aucune intelligence puisse l'en défendre. Vous savez bien, n'est-ce pas, que je n'exagère pas ? Tel est le prix dont on risque d'être contraint de payer, pour peu que la malchance s'en mêle, le malheur d'avoir été jugé par vous, pour une raison ou pour une autre, indésirable à R.\par
Quant à ceux qui demeurent à R., ce sont presque tous des manœuvres ; à l'usine, ils n'ont donc pas à collaborer, mais seulement à obéir, obéir encore et toujours, depuis le moment où ils pointent pour entrer jusqu'au moment où ils pointent pour sortir. Hors de l'usine, ils se trouvent au milieu de choses qui toutes sont faites pour eux, mais qui toutes sont faites par vous. Même leur propre coopérative, en fait ils ne la contrôlent pas.\par
Loin de moi l'idée de vous reprocher cette puissance. Elle a été mise entre vos mains. Vous l'exercez, j'en suis persuadée, avec la plus grande générosité possible – du moins étant donné d'une part l'obsession du rendement, d'autre part le degré inévitable d'incompréhension. Il n'en reste pas moins vrai qu'il n'y a, toujours et partout, que subordination.\par
Tout ce que vous faites pour les ouvriers, vous le faites gratuitement, généreusement, et ils sont perpétuellement vos obligés. Eux ne font rien qui ne soit fait ou par contrainte ou par l'appât du gain. Tous leurs gestes sont dictés ; le seul domaine où ils puissent mettre du leur, c'est la quantité, et à leurs efforts dans ce domaine correspond seulement une quantité supplémentaire de sous. Jamais ils n'ont droit à une récompense morale de la part d'autrui ou d'eux-mêmes : remerciement, éloge, ou simplement satisfaction de soi. C'est là un des pires facteurs de dépression morale dans l'industrie moderne ; je l'éprouvais tous les jours, et beaucoup, j'en suis sûre, sont comme moi. (J'ajouterai d'ailleurs ce point à mon petit questionnaire, si vous l'utilisez.)\par
Vous pouvez vous demander quelles formes concrètes de collaboration j'imagine. Je n'ai encore que quelques ébauches d'idées à ce sujet ; mais j'ai quelque confiance qu'on pourrait concevoir quelque chose de plus complet en étudiant concrètement la question.\par
Je n'ai plus qu'à vous laisser à vos propres réflexions. Vous avez un temps pour ainsi dire illimité pour décider – si toutefois quelque guerre ou quelque dictature « totalitaire » ne vient pas un de ces jours ôter à tous presque tout pouvoir de décision en tout domaine...\par
Je ne suis pas sans remords à votre sujet. Au cas, après tout probable, où nos échanges de vues resteraient sans effet, je n'aurais rien fait d'autre que vous communiquer des préoccupations douloureuses. Cette pensée me fait de la peine. Vous êtes relativement heureux, et le bonheur est pour moi quelque chose de précieux et digne de respect. Je ne désire pas communiquer inutilement autour de moi l'amertume ineffaçable que mon expérience m'a laissée.\par
Veuillez croire à mes sentiments les meilleurs.\par
S. WEIL.\par
{\itshape P.-S.} – Il y a un point que je m'en veux d'avoir oublié à notre dernière entrevue ; je le note seulement pour me garantir, le cas échéant, d'un nouvel oubli. J'ai cru comprendre, d'après une histoire racontée par vous, qu'à l'usine il est interdit de causer sous peine d'amende. En est-il bien ainsi ? Si c'est le cas, j'aurais bien des choses à vous dire sur la dure contrainte que constitue pour un ouvrier un tel règlement, et, plus généralement, sur le principe que, dans une journée de travail, il ne faut pas gaspiller une minute.\par

\begin{center}
*\end{center}
\noindent Mardi 30 mars.\par
\noindent Monsieur,\par
\par
Merci de votre invitation. Malheureusement il faut reculer l'entrevue de 3 semaines. Cette semaine, impossible d'aller vous voir ; je suis, physiquement, complètement à plat, et j'ai à peine la force de faire la classe. Ensuite, quinze jours de vacances, que je ne passerai pas à Bourges. À la rentrée, j'espère être relativement en forme. Voulez-vous convenir, pour fixer les idées, et sauf avis contraire de part et d'autre, que j'irai vous voir le lundi 20 avril ?\par
En somme, il me semble que le seul obstacle sérieux ce que vous me preniez comme ouvrière, c'est un certain manque de confiance. Les obstacles matériels dont vous m'avez parlé sont des difficultés surmontables. Voici ce que je veux dire – Vous pensez bien que je ne considère pas les ouvriers de R. comme un terrain d'expérience ; je serais tout aussi malheureuse que vous qu'une tentative pour alléger leur sort aboutisse à l'aggraver. Si donc, en travaillant à R., j'y sentais, pour employer vos termes, une certaine sérénité que l'exécution de mes projets serait susceptible de troubler, j'y renoncerais la première. Là-dessus nous sommes d'accord. Le point délicat, c'est l'appréciation de la situation morale des ouvriers.\par
Sur ce point, vous ne vous fieriez pas à moi. C'est très légitime et je le comprends. Je me rends compte d'ailleurs que je suis moi-même cause dans une certaine mesure de ce manque de confiance, du fait que je vous ai écrit avec une extrême maladresse, en exprimant toutes les idées sous leur forme la plus brutale. Mais c'était consciemment. Je suis incapable d'user d'adresse, pour quelque intérêt que ce soit, envers les gens auxquels je tiens.\par
Si vous passez à Paris, ne manquez pas de voir le nouveau film de Charlot. Voilà enfin quelqu'un qui a exprimé une partie de ce que j'ai ressenti.\par
Ne croyez pas que les préoccupations sociales me fassent perdre toute joie de vivre. À cette époque de l'année surtout, je n'oublie jamais que « Christ est ressuscité ». (Je parle par métaphore, bien entendu.) J'espère qu'il en sera de même pour tous les habitants de R.\par
Bien cordialement,\par
S. WEIL.\par
Comme on ne doit pas se voir d'ici quelque temps, je veux vous dire d'un mot que les anecdotes et réflexions sur la vie d'usine contenues dans mes lettres vous ont donné de moi, à en juger par la réponse, une opinion plus mauvaise que je ne mérite. Il m'est apparemment impossible de me faire comprendre.\par
Peut-être le film de Charlot y réussirait-il mieux que ce que je puis vous dire.\par
Si moi, qui suis vaguement censée avoir appris à m'exprimer, je n'arrive pas à me faire comprendre de vous, malgré votre bonne volonté, on se demande quels procédés pourraient amener de la compréhension entre la moyenne des ouvriers et des patrons.\par
Un mot encore, concernant l'approbation que vous accordez à la division du travail qui assigne à l'un le soin de pousser la varlope, à l'autre celui de penser l'assemblage. C'est là, je pense, la question fondamentale, et le seul point qui nous sépare essentiellement. J'ai remarqué, parmi les êtres frustes parmi lesquels j'ai vécu, que toujours (je n'ai trouvé aucune exception, je crois) l'élévation de la pensée (la faculté de comprendre et de former les idées générales) allait de pair avec la générosité de cœur. Autrement dit, ce qui abaisse l'intelligence dégrade tout l'homme.\par
Autre remarque, que je mets par écrit pour que vous puissiez la méditer. En tant qu'ouvrière, j'étais dans une situation doublement inférieure, exposée à sentir ma dignité blessée non seulement par les chefs, mais aussi par les ouvriers, du fait que je suis une femme, (Notez bien que je n'avais aucune sotte susceptibilité à l'égard du genre de plaisanteries traditionnel à l'usine.) J'ai constaté, non pas tant à l'usine qu'au cours de mes courses errantes de chômeuse, pendant lesquelles je me faisais une loi de ne jamais repousser une occasion d'entrer en conversation, qu'à peu près constamment les ouvriers capables de parler à une femme sans la blesser sont des professionnels, et ceux qui ont tendance à la traiter comme un jouet des manœuvres spécialisés. À vous de tirer les conclusions.\par
À mon avis le travail doit tendre, dans toute la mesure des possibilités matérielles, à constituer une éducation. Et que penser d'une classe où l'on établirait des exercices de nature radicalement différente pour les mauvais élèves et pour les bons ?\par
Il y a des inégalités naturelles. À mon avis, l'organisation sociale – en se plaçant du point de vue moral – est bonne pour autant qu'elle tend à les atténuer (en élevant, non en abaissant, bien entendu), mauvaise pour autant qu'elle tend à les aggraver, odieuse quand elle crée des cloisons étanches.\par

\begin{center}
*\end{center}
\noindent Monsieur \footnote{Non datée (avril 1936 ?).},\par
J'ai encore réfléchi à ce que vous m'avez dit. Voici mes conclusions. Vous allez croire que j'ai un caractère bien irrésolu, mais j'ai simplement l'esprit lent. Je m'excuse de n'être pas arrivée immédiatement à une décision définitive, comme j'aurais dû.\par
Voici. Étant donné les possibilités immédiates et fort étendues de connaître votre usine que vous voulez bien m’accorder, il ne serait pas raisonnable de ma part de les sacrifier à un projet peut-être irréalisable. Car je ne pourrais travailler chez vous dans des conditions acceptables que s'il y avait une place libre et aucune demande à R., chose peu vraisemblable dans un avenir prochain. Autrement, même si vous m'inscriviez sur une liste et me faisiez passer à mon tour, les ouvriers trouveraient anormal qu'on m'embauche alors que des femmes de R. demanderaient à l'être ; ils devineraient que vous me connaissez ; je ne pourrais fournir aucune explication claire ; et des rapports de camaraderie confiante deviendraient extrêmement difficiles à établir. Ainsi, sans écarter complètement mon projet primitif, qui se trouve pourtant renvoyé à un avenir indéterminé, j'accepte votre proposition de consacrer une journée à l'usine. Je vous proposerai une date ultérieurement.\par
Quant à M. M. \footnote{Propriétaire de l’usine.}, je vous laisse le soin de décider s'il vaut mieux lui demander immédiatement d'accorder ou refuser une autorisation de principe, tout en lui faisant remarquer que mon projet est soumis à des conditions qui rendent l'exécution peu probable, en tout cas prochainement ; ou s'il vaut mieux ne rien dire jusqu'au jour où une possibilité concrète de travailler chez vous se présenterait pour moi. L'avantage qu'il y aurait pour moi à savoir tout de suite à quoi m'en tenir serait que, s'il dit non, je ne serai retenue dans mes investigations à R. par aucune arrière-pensée ; dans le cas contraire, je tâcherai à tout hasard de ne pas trop me faire remarquer au cours de mes visites à l'usine. D'un autre côté, un projet si vague ne vaut peut-être pas la peine qu'on en parle. À vous de faire ce qu'il vous plaira. Encore une fois, je m'excuse d'avoir varié comme j'ai fait.\par
Permettez-moi de vous rappeler que je vous demande en tout cas de ne pas parler à M. M. de mon expérience dans les usines parisiennes, ni d'ailleurs à personne.\par
J'ai pensé à ce que vous m'avez dit sur la manière dont s'opère le choix des ouvriers à renvoyer, en cas de réduction du personnel. Je sais bien que votre méthode est la seule défendable du point de vue de l'entreprise. Mais placez-vous un moment à l'autre point de vue – celui d'en bas. Quelle puissance donne à vos chefs de service cette responsabilité de désigner, parmi les ouvriers polonais, ceux qui sont à renvoyer comme étant les moins utiles ! Je ne les connais pas, j'ignore comment ils usent d'une telle puissance. Mais je peux me représenter la situation de ces ouvriers polonais, qui, je pense, se doutent qu'un jour ou l'autre vous pourrez être de nouveau contraint de renvoyer quelques-uns d'entre eux devant le chef de service qui serait chargé ce jour-là de vous désigner tel ou tel comme étant moins utile que ses camarades. Combien ils doivent trembler devant lui et redouter de lui déplaire ! Me jugerez-vous encore ultra-sensible si je vous dis que j'imagine cela très bien et que cela me fait mal ? Supposez-vous placé dans une telle situation, avec femme et enfants à votre charge, et demandez-vous dans quelle mesure il vous serait possible de conserver votre dignité.\par
N'y aurait-il pas moyen d'établir –en le faisant savoir, bien entendu – n'importe quel autre critérium non soumis à l'arbitraire : charges de famille, ancienneté, tirage au sort, ou combinaison des trois ? Cela comporterait peut-être de graves inconvénients, je n'en sais rien ; mais je vous supplie de considérer quels avantages moraux en résulteraient pour ces malheureux, placés dans une si douloureuse insécurité par la faute du gouvernement français.\par
Voyez-vous, ce n'est pas la subordination en elle-même qui me choque, mais certaines formes de subordination comportant des conséquences moralement intolérables. Par exemple, quand les circonstances sont telles que la subordination implique non seulement la nécessité d'obéir, mais aussi le souci constant de ne pas déplaire, cela me paraît dur à supporter. – D'un autre côté, je ne puis accepter les formes de subordination où l'intelligence, l'ingéniosité, la volonté, la conscience professionnelle n'ont à intervenir que dans l'élaboration des ordres par le chef, et où l'exécution exige seulement une soumission passive dans laquelle ni l'esprit ni le cœur n'ont part ; de sorte que le subordonné joue presque le rôle d'une chose maniée par l'intelligence d'autrui. Telle était ma situation comme ouvrière.\par
Au contraire quand les ordres confèrent une responsabilité à celui qui les exécute, exigent de sa part les vertus de courage, de volonté, de conscience et d'intelligence qui définissent la valeur humaine, impliquent une certaine confiance mutuelle entre le chef et le subordonné, et ne comportent que dans une faible mesure un pouvoir arbitraire entre les mains du chef, la subordination est une chose belle et honorable.\par
Soit dit en passant, j'aurais été reconnaissante à un chef qui aurait bien voulu m'assigner un jour quelque tâche, même pénible, malpropre, dangereuse et mal rétribuée, mais qui aurait impliqué de sa part une certaine confiance en moi ; et j’aurais obéi, ce jour-là, de tout mon cœur. Et je suis sûre que beaucoup d'ouvriers sont comme moi. Il y a là une ressource morale qu'on n'utilise pas.\par
Mais assez là-dessus. Je vous écrirai le plus tôt que je pourrai quelle journée je compte passer à R. Il m'est impossible de vous dire combien je vous sais gré des facilités que vous me procurez pour comprendre ce que c'est qu'une usine.\par
Bien cordialement.\par
S. WEIL.\par
{\itshape P.-S.} –Pourriez-vous me faire envoyer les numéros de votre journal parus depuis le n° 30 ? Ma collection s'arrête là. Mais je serais malheureuse si quelqu'un subissait une engueulade à cause de moi...\par

\begin{center}
*\end{center}
\noindent Monsieur \footnote{Non datée (avril 1936 ?).},\par
J'aurais voulu vous répondre plus tôt. Je n'ai pas eu jusqu'ici la possibilité de fixer une date. Vous convient-il que j'aille vous voir jeudi 30 avril, à l'heure habituelle ? Si oui, inutile de répondre. La proposition que vous me faites, de passer une journée entière à R. pour tout voir de plus près, est celle qui pouvait me faire le plus de plaisir ; seulement je pense qu'une entrevue préalable est nécessaire pour fixer le programme. Je vous remercie infiniment de me fournir ainsi le moyen de mieux me rendre compte. Je ne demande certes qu'à mettre en tout domaine mes idées à l'épreuve du contact avec les faits ; et croyez bien que la probité intellectuelle est toujours à mes yeux le premier des devoirs.\par
Je voudrais, pour abréger les explications orales, vous savoir persuadé que vous avez mal interprété certaines de mes réactions. L'hostilité systématique envers les supérieurs, l'envie à l'égard des plus favorisés, la haine de la discipline, le mécontentement perpétuel, tous ces sentiments mesquins sont absolument étrangers à mon caractère. J'ai au plus haut point le respect de la discipline dans le travail, et je méprise quiconque ne sait pas obéir. Je sais très bien aussi que toute organisation implique des ordres donnés et reçus. Mais il y a ordres et ordres. Moi, j'ai subi comme ouvrière une subordination qui m'a été intolérable, encore que j'aie toujours (ou presque) strictement obéi, et que je sois parvenue péniblement à une espèce de résignation. Je n'ai pas à me justifier (pour employer votre expression) d'avoir éprouvé dans cette situation une souffrance intolérable, j'ai seulement à essayer d'en déterminer exactement les causes ; tout ce qu'on pourrait avoir à me reprocher à ce sujet serait de me tromper dans cette détermination, ce qui est peut-être le cas. D'autre part jamais, en aucun cas, je ne consentirai à juger convenable pour un de mes semblables, quel qu'il soit, ce que je juge moralement intolérable pour moi-même ; si différents que soient les hommes, mon sentiment de la dignité humaine reste identiquement le même, qu'il s'agisse de moi ou de n'importe quel homme, même si entre lui et moi on peut établir à d'autres égards des rapports de supériorité ou d'infériorité. Sur ce point, jamais rien au monde ne me fera varier, du moins je l'espère. Pour tout le reste, je ne demande qu'à me débarrasser de toutes les idées préconçues susceptibles de fausser mon jugement.\par
Une de vos phrases m'a fait longtemps rêver ; c'est celle où vous parlez de contacts plus intimes entre l'usine et moi qui pourraient peut-être être organisés un jour. Avez-vous quelque chose de concret dans l'esprit, en vous exprimant de la sorte ? Si oui, j'espère que vous m'en ferez part. Je me demande si vous désirez seulement, par pure générosité à mon égard, me donner des moyens de m'instruire, de compléter, préciser et rectifier mes vues trop sommaires et sans doute partiellement fausses sur l'organisation industrielle ; ou bien si vous pensez que je pourrais être éventuellement capable de me rendre utile, autrement que de la manière que je vous avais suggérée. Pour moi, je n'ai jusqu'à ce jour aucune raison d'avoir confiance dans mes propres capacités ; mais si vous aviez dans l'esprit une manière quelconque de les mettre à l'épreuve, dans l'intérêt de la population ouvrière, sur la base de quelques idées sur lesquelles, en dépit des divergences, nous serions au préalable arrivés à nous mettre d'accord, cela mériterait réflexion de ma part.\par
Nous parlerons de tout cela et de bien d'autres choses jeudi, si vous le voulez bien. Si vendredi vous convient mieux, vous n'avez qu'à m'en avertir et je m'y conformerai.\par
Bien cordialement.\par
S. WEIL.\par

\begin{center}
*\end{center}
\noindent Monsieur \footnote{Non datée (avril-mai 1936 ?).},\par
Il ne m'est pas encore possible de vous fixer une date. Mais, en attendant, j'ai été si touchée de la générosité dont vous faites preuve à mon égard – en me recevant, en répondant à mes questions, en m'ouvrant votre usine comme vous faites, que j'ai résolu de vous faire de la copie, de manière à vous faire regagner au moins une partie du temps que je vous coûte.\par
Cependant je me demandais avec inquiétude comment j'arriverais à prendre sur moi d'écrire en me soumettant à des limites imposées, car il s'agit évidemment de vous faire de la prose bien sage, autant que j'en suis capable... Heureusement il m'est revenu à la mémoire un vieux projet qui me tient vivement à cœur, celui de rendre les chefs-d'œuvre de la poésie grecque (que j'aime passionnément) accessibles aux masses populaires. J'ai senti, l'an dernier, que la grande poésie grecque serait cent fois plus proche du peuple, s'il pouvait la connaître, que la littérature française classique et moderne.\par
J'ai commencé par {\itshape Antigone}. Si j'ai réussi dans mon dessein, cela doit pouvoir intéresser et toucher tout le monde –depuis le directeur jusqu'au dernier manœuvre ; et celui-ci doit pouvoir pénétrer là-dedans presque de plain-pied, et cependant sans avoir jamais l'impression d'aucune condescendance, d'aucun effort accompli pour se mettre à sa portée. C'est ainsi que je comprends la vulgarisation. Mais j'ignore si j'ai réussi.\par
{\itshape Antigone} n'a rien d'une histoire morale pour enfants sages ; il espère cependant que vous n'irez pas jusqu'à trouver Sophocle subversif...\par
Si cet article plaît –et s'il ne plaît pas, c'est que je ne sais pas écrire – je pourrai vous en faire encore toute une série, d'après d'autres tragédies de Sophocle, et d'après l'{\itshape Iliade}. Homère et Sophocle fourmillent de choses poignantes, profondément humaines, qu'il s'agit seulement d'exprimer et de représenter de manière à les rendre accessibles à tous.\par
Je pense avec une certaine satisfaction que si je fais ces articles, et si on les lit, les manœuvres les plus illettrés de R. en sauront plus sur la littérature grecque que 99 \% des bacheliers –et encore !...\par
Au reste, c'est aux approches de l'été seulement que j'aurai assez de loisir pour ce travail.\par
À bientôt, j'espère, et bien cordialement.\par
S. WEIL.\par
J'espère que vous pourrez vous arranger pour passer ce papier en une seule fois.\par

\begin{center}
*\end{center}

\begin{center}
Fragment de lettre \footnote{Non daté (avril-mai 1936 ?).}\end{center}
\noindent Monsieur,\par
En principe, je pense venir dans 15 jours. J'écrirai pour confirmer.\par
Vous pouvez mettre, comme pseudonyme au papier sur {\itshape Antigone}, « Cléanthe » (c'est le nom d'un Grec qui combinait l'étude de la philosophie stoïcienne avec le métier de porteur d'eau). Je signerais, sans la question de l'embauchage éventuel.\par
Si vous pensez que cela m'a coûté de présenter {\itshape Antigone} comme j'ai fait, vous avez tort de m'en remercier : on ne remercie pas les gens des contraintes qu'on leur impose. Mais en fait ce n'est pas le cas, ou à peu près pas. Je trouve plus beau d'exposer le drame dans sa nudité. Peut-être m'arrivera-t-il pour d'autres textes d'esquisser en quelques mots des applications possibles à la vie contemporaine ; j'espère toutefois qu'elles ne vous paraîtront pas inacceptables.\par
Ce qui, en revanche, m'a été pénible, c'est le fait même d'écrire en ayant présente à l'esprit la question : est-ce que ceci peut passer ? Cela ne m'était jamais arrivé, et il y a bien peu de considérations capables de m'amener à m'y résoudre. La plume se refuse à ce genre de contrainte, quand on a appris à la manier comme il convient. Mais je continuerai néanmoins, bien entendu.\par
J'ai une grande ambition, mais à laquelle j'ose à peine penser, tant elle est difficile à réaliser : ce serait, après cette série de papiers, d'en faire une autre – mais compréhensible et intéressante pour n'importe quel manœuvre – sur la création de la science moderne par les Grecs ; histoire merveilleuse, et généralement ignorée, même des gens cultivés.\par
Vous ne m'avez pas comprise en ce qui concerne les licenciements. Ce n'est pas l'arbitraire même que je voudrais voir limiter. Lorsqu'il s'agit d'une mesure aussi cruelle (ce n'est pas à vous que ce reproche s'adresse) le choix en lui-même me paraît dans une certaine mesure indifférent. Ce que je trouve incompatible avec la dignité humaine, c'est la crainte de déplaire engendrée chez les subordonnés par la croyance en un choix susceptible d'être arbitraire. La règle la plus absurde en elle-même, mais fixe, serait un progrès à cet égard, ou encore, l'organisation d'un procédé de contrôle quelconque permettant aux ouvriers de se rendre compte que le choix n'est pas arbitraire. Bien sûr, vous êtes seul juge des possibilités. En tout cas, comment ne considérerais-je pas les hommes placés dans cette situation morale comme des opprimés ? Ce qui n'implique pas nécessairement que vous soyez un oppresseur.\par

\begin{center}
*\end{center}
\noindent Monsieur \footnote{Non datée (avril-mai 1936). Voir notes pp. 207, 209, 211.},\par
J'ai attendu de jour en jour, pour vous écrire, de pouvoir vous fixer une date. Je n'ai pas eu jusqu'ici la possibilité de le faire, parce que je n'ai pas été bien du tout tous ces temps-ci. Or, passer toute une journée à visiter une usine, c'est fatigant ; et ce ne peut être profitable que si on est capable de conserver jusqu'au soir sa lucidité et sa présence d'esprit.\par
Je viendrai, sauf avis contraire, le vendredi 12 juin, à 7 h 40 comme convenu.\par
Je vous apporterai un nouveau papier sur une autre tragédie de Sophocle. Mais je ne vous le laisserai que si vous pouvez trouver des dispositions typographiques satisfaisantes. Car pour {\itshape Antigone}, j'ai quelques reproches assez sérieux à vous faire concernant la disposition typographique.\par
Toute réflexion faite, je ne visiterai pas de logement ouvrier. Je ne peux pas croire qu'une visite de ce genre ne risque pas de blesser ; et il faudrait des considérations bien puissantes pour m'amener à risquer de blesser des gens qui, lorsqu'on les blesse, doivent se taire et même sourire.\par
D'ailleurs, quand je dis qu'il y a risque de blesser, au fond je suis convaincue que les ouvriers sont effectivement blessés par des choses de ce genre, pour peu qu'ils aient pu garder quelque fierté. Supposez qu'un visiteur particulièrement curieux désire connaître les conditions de vie non seulement des ouvriers, mais aussi du directeur, et que M. M., à cet effet, lui fasse visiter votre maison. J'ai peine à croire que vous trouveriez cela tout naturel. Je ne vois aucune différence entre les deux cas.\par
\par
J'ai vu avec plaisir qu'il semble y avoir eu collaboration ouvrière dans votre journal, à propos de la question des croissants. L'article de l'ouvrière qui en demande la suppression m'a particulièrement frappée. Vous me donnerez, j'espère, quelques renseignements sur elle.\par
Bien cordialement.\par
S. WEIL.\par
{\itshape P.-S.} –J'ai été très intéressée aussi par la réponse de celle qui demande des articles sur l'organisation de l'usine.\par

\begin{center}
*\end{center}
\noindent {\itshape Mercredi} (10 juin 1936).\par
\noindent Monsieur,\par
Je me trouve dans la nécessité d'aller à Paris demain et après-demain, pour y voir des amis de passage. Il faut donc encore remettre cette visite.\par
Au reste, cela vaut mieux ainsi : en ce moment, je serais incapable de me trouver parmi vos ouvriers sans aller à eux pour les féliciter chaleureusement.\par
Vous ne doutez pas, je pense, des sentiments de joie et de délivrance indicible que m'a apportés ce beau mouvement gréviste. Les suites seront ce qu'elles pourront être. Mais elles ne peuvent effacer la valeur de ces belles journées joyeuses et fraternelles, ni le soulagement qu'ont éprouvé les ouvriers à voir ceux qui les dominent plier une fois devant eux.\par
Je vous écris ainsi pour ne pas laisser d'équivoque entre nous. Si j'apportais à vos ouvriers mes félicitations pour leur victoire, vous trouveriez sans doute que j'abuse de votre hospitalité. Il vaut mieux attendre que les choses se tassent. Si toutefois, après ces quelques lignes, vous consentez encore à me recevoir...\par
Bien cordialement.\par
S. WEIL.\par

\begin{center}
Réponse de M. B.\end{center}
\noindent 13.6.36.\par
\noindent Mademoiselle,\par
Si, par hypothèse, les événements qui vous réjouissent avaient évolué à l'inverse, je ne crois pas, mes réactions n'étant pas à sens unique, que j'eusse éprouvé des « sentiments de joie et de délivrance indicibles » à voir les ouvriers plier devant les patrons.\par
Au moins, je suis tout à fait sûr qu'il m'aurait été impossible de vous en adresser le témoignage.\par
Je vous prie, Mademoiselle, d'agréer mes regrets de ne pouvoir, sans mensonge, vous exprimer que des sentiments de courtoisie.\par

\begin{center}
*\end{center}
\noindent Monsieur \footnote{Non datée, juin 1936.},\par
Vous m'écrivez exactement comme si j'avais manqué d'élégance morale au point de triompher de vaincus et d'opprimés. Bien sûr, si vous étiez en prison, ou sur le pavé, ou exilé, ou quoi que ce soit de ce genre, je m'abstiendrais d'exprimer de la joie à ce sujet ou même d'en éprouver. Mais, jusqu'à nouvel ordre, vous êtes directeur à R., n'est-ce pas ? Les ouvriers continuent à travailler sous vos ordres ? Même avec les nouveaux salaires, vous continuez à gagner un peu plus qu'un mouleur, j'imagine ? En dernière analyse, rien d'essentiel n'a changé. Quant à l'avenir, personne ne sait ce qu'il apportera, ni si la victoire ouvrière actuelle aura constitué en fin de compte une étape vers un régime totalitaire communiste, ou vers un régime totalitaire fasciste, ou (ce que j'espère, hélas, sans y croire) vers un régime non totalitaire.\par
Croyez-moi – et surtout, n'imaginez pas que je parle ironiquement – si ce mouvement gréviste a provoqué en moi une joie pure (joie assez vite remplacée, d'ailleurs, par l'angoisse qui ne me quitte pas depuis l'époque déjà lointaine où j'ai compris vers quelles catastrophes nous allons), c'est non seulement dans l'intérêt des ouvriers, mais aussi dans l'intérêt des patrons. Je ne pense pas en ce moment à l'intérêt matériel – peut-être les conséquences de cette grève seront-elles en fin de compte néfastes pour l'intérêt matériel des uns et des autres, on ne sait pas – mais à l'intérêt moral, au salut de l'âme. Je pense qu'il est bon pour les opprimés d'avoir pu pendant quelques jours affirmer leur existence, relever la tête, imposer leur volonté, obtenir des avantages dus à autre chose qu'à une générosité condescendante. Et je pense qu'il est également bon pour les chefs – pour le salut de leur âme –d'avoir dû à leur tour, une fois dans leur vie, plier devant la force et subir une humiliation. J'en suis heureuse pour eux.\par
Qu'est-ce que j'aurais dû faire ? Ne pas éprouver cette joie ? Mais je la juge légitime. Je n'ai eu à aucun moment d'illusion sur les conséquences possibles du mouvement, je n'ai rien fait pour le susciter ni le prolonger ; du moins pouvais-je partager la joie pure et profonde qui animait mes camarades d'esclavage. Ne pas vous exprimer cette joie ? Mais comprenez donc notre situation respective. Des relations cordiales entre vous et moi impliqueraient de ma part la pire hypocrisie si je vous laissais croire un instant qu'elles comportent la moindre nuance de bienveillance à l'égard de la force oppressive que vous représentez et que vous maniez dans votre sphère, comme subordonné immédiat du patron. Il serait facile et avantageux pour moi de vous laisser dans l'erreur à ce sujet. En m'exprimant avec une franchise brutale qui ne peut avoir, pratiquement, que de mauvaises conséquences, je vous donne un témoignage d'estime.\par
Bref, il dépend de vous de renouer ou non les relations qui existaient entre nous avant les événements actuels. Dans l'un et l'autre cas, je n'oublierai pas que je vous dois, sur le plan intellectuel, une vue un peu plus claire concernant certains des problèmes qui me préoccupent\par
S. WEIL.\par
{\itshape P.-S.} –J'ai encore un service à vous demander, que, j'espère, vous voudrez bien me rendre dans tous les cas. Je vais probablement me décider, en fin de compte, à écrire quelque chose concernant le travail industriel. Voudriez-vous avoir l'obligeance de me renvoyer toutes les lettres où je vous ai parlé de la condition ouvrière ? J'y ai noté des faits, des impressions et des idées dont certains ne me reviendraient peut-être pas à l'esprit. Merci d'avance.\par
J'espère, d'autre part, qu'aucun changement dans vos sentiments à mon égard ne vous fera oublier que vous m'avez promis un secret absolu concernant mon expérience dans les usines.
\section[La vie et la grève des ouvrières métallos  (sur le tas) (10 juin 1936)]{La vie et la grève des ouvrières métallos \protect\footnotemark  (sur le tas) \\
(10 juin 1936)}\renewcommand{\leftmark}{La vie et la grève des ouvrières métallos  (sur le tas) \\
(10 juin 1936)}

\footnotetext{ Article paru sous le pseudonyme de S. Galois dans la Révolution prolétarienne du 10 juin 1936 et dans les {\itshape Cahiers de} « {\itshape Terre Libre} » du 15 juillet 1936.}
\noindent \par
Enfin, on respire ! C'est la grève chez les métallos. Le public qui voit tout ça de loin ne comprend guère. Qu'est-ce que c'est ? Un mouvement révolutionnaire ? Mais tout est calme. Un mouvement revendicatif ? Mais pourquoi si profond, si général, si fort, et si soudain ?\par
Quand on a certaines images enfoncées dans l'esprit, dans le cœur, dans la chair elle-même, on comprend. On comprend tout de suite. Je n'ai qu'à laisser affluer les souvenirs.\par
Un atelier, quelque part dans la banlieue, un jour de printemps, pendant ces premières chaleurs si accablantes pour ceux qui peinent. L'air est lourd d'odeurs de peinture et de vernis. C'est ma première journée dans cette usine. Elle m'avait parue accueillante, la veille : au bout de toute une journée passée à arpenter les rues, à présenter des certificats inutiles, enfin ce bureau d'embauche avait bien voulu de moi. Comment se défendre, au premier instant, d'un sentiment de reconnaissance ? Me voici sur une machine. Compter cinquante pièces... les placer une à une sur la machine, d'un côté, pas de l'autre... manier à chaque fois un levier... ôter la pièce... en mettre une autre…encore une autre… compter encore... je ne vais pas assez vite. La fatigue se fait déjà sentir. Il faut forcer, empêcher qu'un instant d'arrêt sépare un mouvement du mouvement suivant. Plus vite, encore plus vite ! Allons bon ! Voilà une pièce que j'ai mise du mauvais côté. Qui sait si c'est la première ? Il faut faire attention. Cette pièce est bien placée. Celle-là aussi, Combien est-ce que j'en ai fait les dernières dix minutes ? Je ne vais pas assez vite. Je force encore, peu à peu, la monotonie de la tâche m'entraîne à rêver. Pendant quelques instants, je pense à bien des choses. Réveil brusque : combien est-ce que j'en fais ? Ça ne doit pas être assez. Ne pas rêver. Forcer encore. Si seulement je savais combien il faut en faire ! Je regarde autour de moi ! Personne ne lève la tête, jamais. Personne ne sourit. Personne ne dit un mot. Comme on est seul ! je fais 400 pièces à l'heure. Savoir si c'est assez ? Pourvu que je tienne à cette cadence, au moins... La sonnerie de midi, enfin. Tout le monde se précipite à la pendule de pointage, au vestiaire, hors de l'usine. Il faut aller manger. J'ai encore un peu d'argent, heureusement. Mais il faut faire attention. Qui sait si on va me garder ici ? Si je ne chômerai pas encore des jours et des jours ? Il faut aller dans un de ces restaurants sordides qui entourent les usines. Ils sont chers, d'ailleurs. Certains plats semblent assez tentants, mais ce sont d'autres qu’il faut choisir, les meilleur marché. Manger coûte un effort encore. Ce repas n est pas une détente. Quelle heure est-il ? Il reste quelques moments pour flâner. Mais sans s'écarter trop : pointer une minute en retard, c'est travailler une heure sans salaire. L'heure avance. Il faut rentrer. Voici ma machine. Voici mes pièces. Il faut recommencer. Aller vite... Je me sens défaillir de fatigue et d'écœurement. Quelle heure est-il ? Encore deux heures avant la sortie. Comment est-ce que je vais pouvoir tenir ? Voilà que le contremaître s'approche. « Combien en faites-vous ? 400 à l'heure ? Il en faut 800. Sans quoi je ne vous garderai pas. Si à partir de maintenant vous en faites 800, je consentirai peut-être à vous garder. » Il parle sans élever la voix. Pourquoi élèverait-il la voix, quand d'un mot il peut provoquer tant d'angoisse ? Que répondre ? « Je tâcherai. » Forcer. Forcer encore. Vaincre à chaque seconde ce dégoût, cet écœurement qui paralysent. Plus vite. Il s'agit de doubler la cadence. Combien en ai-je fait, au bout d'une heure ? 600. Plus vite. Combien, au bout de cette dernière heure ? 650. La sonnerie. Pointer, s'habiller, sortir de l'usine, le corps vidé de toute énergie vitale, l'esprit vide de pensée, le cœur submergé de dégoût, de rage muette, et par-dessus tout cela d'un sentiment d'impuissance et de soumission. Car le seul espoir pour le lendemain, c'est qu'on veuille bien me laisser passer encore une pareille journée. Quant aux jours qui suivront, c'est trop loin. L'imagination se refuse à parcourir un si grand nombre de minutes mornes.\par
Le lendemain, on veut bien me laisser me remettre à ma machine, quoique je n'aie pas fait la veille les 800 pièces exigées. Mais il va falloir les faire ce matin. Plus vite. Voilà le contremaître. Qu'est-ce qu'il va me dire ? « Arrêtez. » J'arrête. Qu'est-ce qu'on me veut ? Me renvoyer ? J'attends un ordre. Au lieu d'un ordre, il vient une sèche réprimande, toujours sur le même ton bref. « Dès qu'on vous dit d'arrêter, il faut être debout pour aller sur une autre machine On ne dort pas ici. » Que faire ? Me taire. Obéir immédiatement. Aller immédiatement à la machine qu’on me désigne. Exécuter docilement les gestes qu'on m’indique. Pas un mouvement d’impatience : tout mouvement d’impatience se traduit par de la lenteur ou de la maladresse. L'irritation, c'est bon pour ceux qui commandent, C'est défendu à ceux qui obéissent. Une pièce. Encore une pièce. Est-ce que j'en fais assez ? Vite. Voilà que j'ai failli louper une pièce. Attention ! Voilà que je ralentis. Vite. Plus vite...\par
Quels souvenirs encore ? il n'en vient que trop pêle-mêle. Des femmes qui attendent devant une porte d'usine. On ne peut entrer que dix minutes avant l'heure, et quand on habite loin, il faut bien venir une vingtaine de minutes en avance, pour ne pas risquer une minute de retard. Un portillon est ouvert, mais officiellement « ce n'est pas ouvert ». Il pleut à torrents. Les femmes sont dehors sous la pluie, devant cette porte ouverte. Quoi de plus naturel que de s'abriter quand il pleut et que la porte d'une maison est ouverte ? Mais ce mouvement si naturel, en ne pense même pas à le faire devant cette usine, parce que c est défendu. Aucune maison étrangère n’est si étrangère que cette usine où on dépense quotidiennement ses forces pendant huit heures.\par
Une scène de renvoi. On me renvoie d'une usine où j'ai travaillé un mois, sans qu’on m'ait jamais fait aucune observation. Et pourtant on embauche tous les jours. Qu'est-ce qu'on a contre moi ? On n'a pas daigné me le dire. Je reviens à l'heure de la sortie. Voilà le chef d'atelier. Je lui demande bien poliment une explication.\par
Je reçois comme réponse : « Je n'ai pas de comptes à vous rendre », et aussitôt il s'en va. Que faire ? Un scandale ? Je risquerais de ne trouver d'embauche nulle part. Non, m'en aller bien sagement, recommencer à arpenter les rues, à stationner devant les bureaux d'embauche, et, à mesure que les semaines s'écoulent, sentir croître, au creux de l'estomac, une sensation qui s'installe en permanence et dont il est impossible de dire dans quelle mesure c'est de l'angoisse et dans quelle mesure de la faim.\par
Quoi encore ? Un vestiaire d'usine, au cours d'une semaine rigoureuse d'hiver. Le vestiaire n'est pas chauffé. On entre là-dedans, quelquefois juste après avoir travaillé devant un four. On a un mouvement de recul, comme devant un bain froid. Mais il faut entrer. Il faut passer là dix minutes. Il faut mettre dans l'eau glacée des mains couvertes de coupures, où la chair est à vif, il faut les frotter vigoureusement avec de la sciure de bois pour ôter un peu l'huile et la poussière noire. Deux fois par jour. Bien sûr, on supporterait des souffrances encore plus pénibles, mais celles-là sont si inutiles ! Se plaindre à la direction ? Personne n'y songe un seul instant. « Ils se foutent bien de nous. » C'est vrai ou ce n'est pas vrai – mais en tout cas c'est bien l'impression qu'ils nous donnent. On ne veut pas risquer de se faire rembarrer. Plutôt souffrir tout cela en silence. C'est encore moins douloureux.\par
Des conversations, à l'usine. Un jour, une ouvrière amène au vestiaire un gosse de neuf ans. Les plaisanteries fusent. « Tu l'amènes travailler ? » Elle répond : « Je voudrais bien qu'il puisse travailler. » Elle a deux gosses et un mari malade à sa charge. Elle gagne bien de 3 à 4 francs de l'heure. Elle aspire au moment où enfin ce gosse pourra être enfermé à longueur de journée dans une usine pour rapporter quelques sous. Une autre, bonne camarade et affectueuse, qu'on interroge sur sa famille. « Vous avez des gosses ? » – Non, heureusement. C'est-à-dire, j'en avais un, mais il est mort. » Elle parle d'un mari malade qu'elle a eu huit ans à sa charge. « Il est mort, heureusement... » C'est beau, les sentiments, mais la vie est trop dure...\par
Des scènes de paie. On défile comme un troupeau, devant le guichet, sous l'œil des contremaîtres. On ne sait pas ce qu'on touchera : il y aurait toujours à faire des calculs tellement compliqués que personne ne s'en sort, et il y a souvent de l'arbitraire. Impossible de se défendre du sentiment que ce peu d'argent qu'on vous passe à travers le guichet est une aumône.\par
La faim. Quand on gagne 3 francs de l'heure, ou même 4 francs, ou même un peu plus, il suffit d'un coup dur, une interruption de travail, une blessure, pour devoir pendant une semaine ou plus travailler en subissant la faim. Pas la sous-alimentation, qui peut, elle, se produire en permanence, même sans coup dur – la faim. La faim jointe à un dur travail physique, c'est une sensation poignante. Il faut travailler aussi vite que d'habitude, sans quoi on ne mangera pas encore assez la semaine suivante. Et par-dessus le marché, on risque de se faire engueuler pour production insuffisante. Peut-être renvoyer. Ce ne sera pas une excuse de dire qu'on a faim. On a faim, mais il faut quand même satisfaire les exigences de ces gens par qui on peut en un instant être condamné à avoir encore plus faim. Quand on n'en peut plus, on n'a qu'à forcer. Toujours forcer. En sortant de l'usine, rentrer aussitôt chez soi pour éviter la tentation de dîner, et attendre l'heure du sommeil, qui d'ailleurs sera troublé parce que même la nuit on a faim. Le lendemain, forcer encore. Tous ces efforts, ils auront leur contrepartie : les quelques billets, les quelques pièces qu'on recevra au travers d'un guichet. Que demander d'autre ? On n'a droit à rien d'autre. On est là pour obéir et se taire. On est au monde pour obéir et se taire.\par
Compter sous par sous. Pendant huit heures de travail, on compte sous par sous. Combien de sous rapporteront ces pièces ? Qu'est-ce que j'ai gagné cette heure-ci ? Et l'heure suivante ? En sortant de l'usine, on compte encore sous par sous. On a un tel besoin de détente que toutes les boutiques attirent. Est-ce que je peux prendre un café ? Mais ça coûte dix sous. J'en ai déjà pris un hier. Il me reste tant de sous pour la quinzaine. Et ces cerises ? Elles coûtent tant de sous. On fait son marché : combien coûtent les pommes de terre, ici ? Deux cents mètres plus loin, elles coûtent deux sous de moins. Il faut imposer ces deux cents mètres à un corps qui se refuse à marcher. Les sous deviennent une obsession. Jamais, à cause d'eux, on ne peut oublier la contrainte de l'usine. Jamais on ne se détend. Ou, si on fait une folie – une folie à l'échelle de quelques francs – on subira la faim. Il ne faut pas que ça arrive souvent : on finirait par travailler moins vite, et par un cercle impitoyable la faim engendrerait encore plus de faim. Il ne faut pas se faire prendre par ce cercle. Il mène à l'épuisement, à la maladie, à la mort. Car quand on ne peut plus produire assez vite, on n'a plus droit à vivre. Ne voit-on pas les hommes de 40 ans refusés partout, à tous les bureaux d'embauche, quels que soient leurs certificats. ? À 40 ans, on est compté comme un incapable. Malheur aux incapables.\par
La fatigue. La fatigue, accablante, amère, par moments douloureuse au point qu'on souhaiterait la mort. Tout le monde, dans toutes les situations, sait ce que c'est que d'être fatigué, mais pour cette fatigue-là, il faudrait un nom à part. Des hommes vigoureux, dans la force de l'âge, s'endorment de fatigue sur la banquette du métro. Pas après un coup dur, après une journée de travail normale. Une journée comme il y en aura une encore le lendemain, le surlendemain, toujours. En descendant dans la rame de métro, au sortir de l'usine, une angoisse occupe toute la pensée : est-ce que je trouverai une place assise ? Ce serait trop dur de devoir rester debout. Mais bien souvent il faut rester debout. Attention qu'alors l'excès de fatigue n'empêche pas de dormir ! Le lendemain il faudrait forcer encore un peu plus.\par
La peur. Rares sont les moments de la journée où le cœur n'est pas un peu comprimé par une angoisse quelconque. Le matin, l'angoisse de la journée à traverser. Dans les rames de métro qui mènent à Billancourt, vers 6 h 1/2 du matin, on voit la plupart des visages contractés par cette angoisse. Si on n'est pas en avance, la peur de la pendule de pointage. Au travail, la peur de ne pas aller assez vite, pour tous ceux qui ont du mal à y arriver. La peur de louper des pièces en forçant sur la cadence, parce que la vitesse produit une espèce d'ivresse qui annule l'attention. La peur de tous les menus accidents qui peuvent amener des loupés ou un outil cassé. D'une manière générale, la peur des engueulades. On s'exposerait à bien des souffrances rien que pour éviter une engueulade. La moindre réprimande est une dure humiliation, parce qu'on n'ose pas répondre. Et combien de choses peuvent amener une réprimande ! La machine a été mal réglée par le régleur ; un outil est en mauvais acier ; des pièces sont impossibles à bien placer : on se fait engueuler. On va chercher le chef à travers l'atelier pour avoir du boulot, on se fait rembarrer. Si on l'avait attendu à son bureau, on aurait risqué une engueulade aussi. On se plaint d'un travail trop dur ou d'une cadence impossible à suivre, on s'entend brutalement rappeler qu'on occupe une place que des centaines de chômeurs prendraient volontiers. Mais pour oser se plaindre, il faut véritablement qu'on n'en puisse plus. Et c'est çà la pire angoisse, l'angoisse de sentir qu'on s'épuise ou qu'on vieillit, que bientôt en n'en pourra plus. Demander un poste moins dur ? Il faudrait avouer qu'on ne peut plus occuper celui où on est. On risquerait d'être jeté à la porte. Il faut serrer les dents. Tenir. Comme un nageur sur l'eau. Seulement avec la perspective de nager toujours, jusqu'à la mort. Pas de barque par laquelle on puisse être recueilli. Si on s'enfonce lentement, si on coule, personne au monde ne s'en apercevra seulement. Qu'est-ce qu'on est ? Une unité dans les effectifs du travail. On ne compte pas. À peine si on existe.\par
La contrainte. Ne jamais rien faire, même dans le détail, qui constitue une initiative. Chaque geste est simplement l'exécution d'un ordre. En tout cas pour les manœuvres spécialisés. Sur une machine, pour une série de pièces, cinq ou six mouvements simples sont indiqués, qu'il faut seulement répéter à toute allure. Jusqu'à quand ? Jusqu'à ce qu'on reçoive l'ordre de faire autre chose. Combien durera cette série de pièces ? Jusqu'à ce que le chef donne une autre série. Combien de temps restera-t-on sur cette machine ? Jusqu'à ce que le chef donne ordre d'aller sur une autre. On est à tout instant dans le cas de recevoir un ordre. On est une chose livrée à la volonté d'autrui. Comme ce n'est pas naturel à un homme de devenir une chose, et comme il n'y a pas de contrainte tangible, pas de fouet, pas de chaînes, il faut se plier soi-même à cette passivité. Comme on aimerait pouvoir laisser son âme dans la case où on met le carton de pointage, et la reprendre à la sortie ! Mais on ne peut pas. Son âme, on l'emporte à l'atelier. Il faut tout le temps la faire taire. À la sortie, souvent on ne l'a plus, parce qu'on est trop fatigué. Ou si on l'a encore, quelle douleur, le soir venu, de se rendre compte de ce qu'on a été huit heures durant ce jour-là, et de ce qu'on sera huit heures encore le lendemain, et le lendemain du lendemain...\par
Quoi encore ? L'importance extraordinaire que prend la bienveillance ou l'hostilité des supérieurs immédiats, régleurs, chef d'équipe, contremaître, ceux qui donnent à leur gré le « bon » ou le « mauvais » boulot, qui peuvent à leur gré aider ou engueuler dans les coups durs. La nécessité perpétuelle de ne pas déplaire. La nécessité de répondre aux paroles brutales sans aucune nuance de mauvaise humeur, et même avec déférence, s'il s'agit d'un contremaître. Quoi encore ? Le « mauvais boulot », mal chronométré, sur lequel on se crève pour ne pas « couler » le bon, parce qu'on risquerait de se faire engueuler pour vitesse insuffisante ; ce n'est jamais le chronométreur qui a tort. Et si ça se produisait trop souvent, on risquerait le renvoi. Et tout en se crevant, on ne gagne à peu près rien, justement parce que c'est du « mauvais boulot ». Quoi encore ? Mais ça suffit. Ça suffit pour montrer ce qu'est une vie pareille, et que si on s'y soumet, c'est, comme dit Homère an sujet des esclaves, « bien malgré soi, sous la pression d'une dure nécessité ».\par

\begin{center}
*\end{center}
\noindent Dès qu'on a senti la pression s'affaiblir, immédiatement les souffrances, les humiliations, les rancœurs, les amertumes silencieusement amassées pendant des années ont constitué une force suffisante pour desserrer l'étreinte. C'est toute l'histoire de la grève. Il n'y a rien d'autre.\par
Des bourgeois intelligents ont cru que la grève avait été provoquée par les communistes pour gêner le nouveau gouvernement. J'ai entendu moi-même un ouvrier intelligent dire qu'au début la grève avait sans doute été provoquée par les patrons pour gêner ce même gouvernement. Cette rencontre est drôle. Mais aucune provocation n'était nécessaire. On pliait sous le joug. Dès que le joug s'est desserré, on a relevé la tête. Un point c'est tout.\par
Comment est-ce que ça s'est passé ? Oh ! bien simplement. L'unité syndicale n'a pas constitué un facteur décisif. Bien sûr, c'est un gros atout, mais qui joue dans d'autres corporations beaucoup plus que pour les métallos de la région parisienne, parmi lesquels on ne comptait, il y a un an, que quelques milliers de syndiqués. Le facteur décisif, il faut le dire, c'est le gouvernement du Front populaire. D'abord, on peut enfin – enfin ! – faire une grève sans police, sans gardes mobiles. Mais ça, ça joue pour toutes les corporations. Ce qui compte surtout, c'est que les usines de mécanique travaillent presque toutes pour l'État, et dépendent de lui pour boucler le budget. Cela, chaque ouvrier le sait. Chaque ouvrier, en voyant arriver au pouvoir le parti socialiste, a eu le sentiment que, devant le patron, il n'était plus le plus faible. La réaction a été immédiate.\par
Pourquoi les ouvriers n'ont-ils pas attendu la formation du nouveau gouvernement ? Il ne faut pas, à mon avis, chercher là-dessous des manœuvres machiavéliques. Nous ne devons pas non plus, nous autres, nous hâter de conclure que la classe ouvrière se méfie des partis ou du pouvoir d'État. Nous aurions, par la suite, de sérieuses désillusions. Bien sûr, il est réconfortant de constater que les ouvriers aiment encore mieux faire leurs propres affaires que de les confier au gouvernement. Mais ce n'est pas, je crois, cet état d'esprit qui a déterminé la grève. Non. En premier lieu on n'a pas eu la force d'attendre. Tous ceux qui ont souffert savent que lorsqu'on croit qu'on va être délivré d'une souffrance trop longue et trop dure, les derniers jours d'attente sont intolérables. Mais le facteur essentiel est ailleurs. Le public, et les patrons, et Léon Blum lui-même, et tous ceux qui sont étrangers à cette vie d'esclave sont incapables de comprendre ce qui a été décisif dans cette affaire. C'est que dans ce mouvement il s'agit de bien autre chose que de telle ou telle revendication particulière, si importante soit-elle. Si le gouvernement avait pu obtenir pleine et entière satisfaction par de simples pourparlers, on aurait été bien moins content. Il s'agit, après avoir toujours plié, tout subi, tout encaissé en silence pendant des mois et des années, d'oser enfin se redresser. Se tenir debout. Prendre la parole à son tour. Se sentir des hommes, pendant quelques jours. Indépendamment des revendications, cette grève est en elle-même une joie. Une joie pure. Une joie sans mélange.\par
Oui, une joie. J'ai été voir les copains dans une usine où j'ai travaillé il y a quelques mois. J'ai passé quelques heures avec eux. Joie de pénétrer dans l'usine avec l'autorisation souriante d'un ouvrier qui garde la porte. Joie de trouver tant de sourires, tant de paroles d'accueil fraternel. Comme on se sent entre camarades dans ces ateliers où, quand j'y travaillais, chacun se sentait tellement seul sur sa machine ! Joie de parcourir librement ces ateliers où on était rivé sur sa machine, de former des groupes, de causer, de casser la croûte. Joie d'entendre, au lieu du fracas impitoyable des machines, symbole si frappant de la dure nécessité sous laquelle on pliait, de la musique, des chants et des rires. On se promène parmi ces machines auxquelles on a donné pendant tant et tant d'heures le meilleur de sa substance vitale, et elles se taisent, elles ne coupent plus de doigts, elles ne font plus de mal. Joie de passer devant les chefs la tête haute. On cesse enfin d'avoir besoin de lutter à tout instant, pour conserver sa dignité à ses propres yeux, contre une tendance presque invincible à se soumettre corps et âme. Joie de voir les chefs se faire familiers par force, serrer des mains, renoncer complètement à donner des ordres. Joie de les voir attendre docilement leur tour pour avoir le bon de sortie que le comité de grève consent à leur accorder. Joie de dire ce qu'on a sur le cœur à tout le monde, chefs et camarades, sur ces lieux où deux ouvriers pouvaient travailler des mois côte à côte sans qu'aucun des deux sache ce que pensait le voisin. Joie de vivre, parmi ces machines muettes, au rythme de la vie humaine – le rythme qui correspond à la respiration, aux battements du cœur, aux mouvements naturels de l'organisme humain – et non à la cadence imposée par le chronométreur, Bien sûr, cette vie si dure recommencera dans quelques jours. Mais on n'y pense pas, on est comme les soldats en permission pendant la guerre. Et puis, quoi qu'il puisse arriver par la suite, on aura toujours eu ça. Enfin, pour la première fois, et pour toujours, il flottera autour de ces lourdes machines d'autres souvenirs que le silence, la contrainte, la soumission. Des souvenirs qui mettront un peu de fierté au cœur, qui laisseront un peu de chaleur humaine sur tout ce métal.\par
On se détend complètement. On n'a pas cette énergie farouchement tendue, cette résolution mêlée d'angoisse si souvent observée dans les grèves. On est résolu, bien sûr, mais sans angoisse. On est heureux. On chante, mais pas l'Internationale, pas la Jeune Garde ; on chante des chansons, tout simplement, et c'est très bien. Quelques-uns font des plaisanteries, dont on rit pour le plaisir de s'entendre rire. On n'est pas méchant. Bien sûr, on est heureux de faire sentir aux chefs qu'ils ne sont pas les plus forts. C'est bien leur tour. Ça leur fait du bien. Mais on n'est pas cruel. On est bien trop content. On est sûr que les patrons céderont. On croit qu'il y aura un nouveau coup dur au bout de quelques mois, mais on est prêt. On se dit que si certains patrons ferment leurs usines, l'État les reprendra. On ne se demande pas un instant s'il pourra les faire fonctionner aux conditions désirées. Pour tout Français, l'État est une source de richesse inépuisable. L'idée de négocier avec les patrons, d'obtenir des compromis, ne vient à personne. On veut avoir ce qu'on demande. On veut l'avoir parce que les choses qu'on demande, on les désire, mais surtout parce qu'après avoir si longtemps plié, pour une fois qu'on relève la tête, on ne veut pas céder. On ne veut pas se laisser rouler, être pris pour des imbéciles. Après avoir passivement exécuté tant et tant d'ordres, c'est trop bon de pouvoir enfin pour une fois en donner à ceux mêmes de qui on les recevait. Mais le meilleur de tout, c'est de se sentir tellement des frères...\par
Et les revendications, que faut-il en penser ? Il faut noter d'abord un fait bien compréhensible, mais très grave. Les ouvriers font la grève, mais laissent aux militants le soin d'étudier le détail des revendications. Le pli de la passivité contracté quotidiennement pendant des années et des années ne se perd pas en quelques jours, même quelques jours si beaux. Et puis ce n'est pas au moment où pour quelques jours on s'est évadé de l'esclavage qu'on peut trouver en soi le courage d'étudier les conditions de la contrainte sous laquelle on a plié jour après jour, sous laquelle on pliera encore. On ne peut pas penser à ça tout le temps. Il y a des limites aux forces humaines. On se contente de jouir, pleinement, sans arrière-pensée, du sentiment qu'enfin on compte pour quelque chose ; qu'on va moins souffrir ; qu'on aura des congés payés – cela, on en parle avec des yeux brillants, c'est une revendication qu'on n'arrachera plus du cœur de la classe ouvrière –, qu'on aura de meilleurs salaires et quelque chose à dire dans l'usine, et que tout cela, on ne l'aura pas simplement obtenu, mais imposé. On se laisse, pour une fois, bercer par ces douces pensées, on n'y regarde pas de plus près.\par
Or, ce mouvement pose de graves problèmes. Le problème central, à mes yeux, c'est le rapport entre les revendications matérielles et les revendications morales. Il faut regarder les choses en face. Est-ce que les salaires réclamés dépassent les possibilités des entreprises dans le cadre du régime ? Et si oui, que faut-il en penser ? Il ne s'agit pas simplement de la métallurgie, puisqu'à juste titre le mouvement revendicatif est devenu général. Alors ? Assisterons-nous à une nationalisation progressive de l'économie sous la poussée des revendications ouvrières, à une évolution vers l'économie d'État et le pouvoir totalitaire ? Ou à une recrudescence du chômage ? Ou à une reculade des ouvriers obligés de baisser la tête une fois de plus sous la contrainte des nécessités économiques ? Dans chacun de ces cas, ce beau mouvement aurait une triste issue.\par
J'aperçois, pour moi, une autre possibilité. Il est à vrai dire délicat d'en parler publiquement dans un moment pareil. En plein mouvement revendicatif, on ose difficilement suggérer de limiter volontairement les revendications. Tant pis. Chacun doit prendre ses responsabilités. Je pense, pour moi, que le moment serait favorable, si on savait l'utiliser, pour constituer le premier embryon d'un contrôle ouvrier. Les patrons ne peuvent pas accorder des satisfactions illimitées c'est entendu ; que du moins ils ne soient plus seuls juges de ce qu'ils peuvent ou disent pouvoir. Que partout où les patrons invoquent comme motif de résistance la nécessité de boucler le budget, les ouvriers établissent une commission de contrôle des comptes constituée par quelques-uns d'entre eux, un représentant du syndicat, un technicien membre d'une organisation ouvrière. Pourquoi, là où l'écart entre leurs revendications et les offres du patronat est grand, n'accepteraient-ils pas de réduire considérablement leurs prétentions jusqu'à ce que la situation de l'entreprise s'améliore, et sous la condition d'un contrôle syndical permanent ? Pourquoi même ne pas prévoir dans le contrat collectif, pour les entreprises qui seraient au bord de la faillite, une dérogation possible aux clauses qui concernent les salaires, sous la même condition ? Il y aurait alors enfin et pour la première fois, à la suite d'un mouvement ouvrier, une transformation durable dans le rapport des forces. Ce point vaut la peine d'être sérieusement médité par les militants responsables.\par
Un autre problème, qui concerne plus particulièrement les bagnes de la mécanique, est lui aussi à considérer. C'est la répercussion des nouvelles conditions de salaires sur la vie quotidienne à l'atelier. Tout d'abord, l'inégalité entre les catégories sera-t-elle intégralement maintenue ou diminuée ? Il serait déplorable de la maintenir. L'effacer serait un soulagement, un progrès prodigieux quant à l'amélioration des rapports entre ouvriers. Si on se sent seul dans une usine, et on s'y sent très seul, c'est en grande partie à cause de l'obstacle qu'apportent aux rapports de camaraderie de petites inégalités, grandes par rapport à ces maigres salaires. Celui qui gagne un peu moins jalouse celui qui gagne un peu plus. Celui qui gagne un peu plus méprise celui qui gagne un peu moins. C'est ainsi. Ce n'est pas ainsi pour tous, mais c'est ainsi pour beaucoup. On ne peut pas sans doute encore établir l'égalité, mais du moins on peut diminuer considérablement les différences. Il faut le faire. Mais ce qui me paraît le plus grave, le voici. On aura, pour chaque catégorie, un salaire minimum. Mais le travail aux pièces est maintenu. Que se passera-t-il alors en cas de « bons coulés », c'est-à-dire au cas où le salaire calculé en fonction des pièces exécutées est inférieur au salaire minimum ? Le patron réglera la différence, c'est entendu. La fatigue, le manque de vivacité, la malchance de tomber sur du « mauvais boulot » ou de travailler sur une machine détraquée ne seront plus automatiquement punis par un abaissement presque illimité des salaires. On ne verra plus une ouvrière gagner douze francs dans une journée parce qu'elle aura dû attendre quatre ou cinq heures qu'on ait fini de réparer sa machine. Très bien. Mais il y a à craindre alors qu'à cette injuste punition d'un salaire dérisoire se substitue une punition plus impitoyable, le renvoi. Le chef saura de quels ouvriers il a dû relever le salaire pour observer la clause du contrat, il saura quels ouvriers sont restés le plus souvent au-dessous du minimum. Pourra-t-on l'empêcher de les mettre à la porte pour rendement insuffisant ? Les pouvoirs du délégué d'atelier peuvent-ils s'étendre jusque-là ? Cela me paraît presque impossible, quelles que soient les clauses du contrat collectif. Dès lors, il est à craindre qu'à l'amélioration des salaires corresponde une nouvelle aggravation des conditions morales du travail, une terreur accrue dans la vie quotidienne de l'atelier, une aggravation de cette cadence du travail qui déjà brise le corps, le cœur et la pensée. Une loi impitoyable, depuis une vingtaine d'années, semble faire tout servir à l'aggravation de la cadence.\par
Je m'en voudrais de terminer sur une note triste. Les militants ont, en ces jours, une terrible responsabilité. Nul ne sait comment les choses tourneront. Plusieurs catastrophes sont à craindre. Mais aucune crainte n'efface la joie de voir ceux qui toujours, par définition, courbent la tête, la redresser. Ils n'ont pas, quoi qu'on suppose du dehors, des espérances illimitées. Il ne serait même pas exact de parler en général d'espérances. Ils savent bien qu'en dépit des améliorations conquises le poids de l'oppression sociale, un instant écarté, va retomber sur eux. Ils savent qu'ils vont se retrouver sous une domination dure, sèche et sans égards. Mais ce qui est illimité, c'est le bonheur présent. Ils se sont enfin affirmés. Ils ont enfin fait sentir à leurs maîtres qu'ils existent. Se soumettre par force, c'est dur ; laisser croire qu'on veut bien se soumettre, c'est trop. Aujourd'hui, nul ne peut ignorer que ceux à qui on a assigné pour seul rôle sur cette terre de plier, de se soumettre et de se taire plient, se soumettent et se taisent seulement dans la mesure précise où ils ne peuvent pas faire autrement. Y aura-t-il autre chose ? Allons-nous enfin assister à une amélioration effective et durable des conditions du travail industriel ? L'avenir le dira ; mais cet avenir, il ne faut pas l'attendre, il faut le faire.
\section[Lettre ouverte, à un syndiqué, (Après juin 1936)]{Lettre ouverte \\
à un syndiqué \\
(Après juin 1936)}\renewcommand{\leftmark}{Lettre ouverte \\
à un syndiqué \\
(Après juin 1936)}

\noindent Camarade, tu es l'un des quatre millions qui sont venus rejoindre notre organisation syndicale. Le mois de juin 1936 est une date dans ta vie. Te rappelles-tu, avant ? C'est loin, déjà. Ça fait mal de s'en souvenir. Mais il ne faut pas oublier. Te rappelles-tu ? On n'avait qu'un droit : le droit de se taire. Quelquefois, pendant qu'on était à son boulot, sur sa machine, le dégoût, l'épuisement, la révolte, gonflaient le cœur ; à un mètre de soi, un camarade subissait les mêmes douleurs, éprouvait la même rancœur, la même amertume ; mais on n'osait pas échanger les paroles qui auraient pu soulager, parce qu'on avait peur.\par
Est-ce que tu te rappelles bien, maintenant, comme on avait peur, comme on avait honte, comme on souffrait ? Il y en avait qui n'osaient pas avouer leurs salaires, tellement ils avaient honte de gagner si peu. Ceux qui, trop faibles ou trop vieux, ne pouvaient pas suivre la cadence du travail n'osaient pas l'avouer non plus. Est-ce que tu te rappelles comme on était obsédé par la cadence du travail ? On n'en faisait jamais assez ; il fallait toujours être tendu pour faire encore quelques pièces de plus, gagner encore quelques sous de plus. Quand, en forçant, en s'épuisant, on était arrivé à aller plus vite, le chronométreur augmentait les normes. Alors on forçait encore, on essayait de dépasser les camarades, on se jalousait, on se crevait toujours plus.\par
Ces sorties, le soir, tu te rappelles ? Les jours où on avait eu du « mauvais boulot ». On sortait, le regard éteint, vidé, crevé. On usait ses dernières forces pour se précipiter dans le métro, pour chercher avec angoisse s'il restait une place assise. S’il en restait, on somnolait sur la banquette. S'il n'en restait pas, on se raidissait pour arriver à rester debout. On n'avait plus de force pour se promener, pour causer, pour lire, pour jouer avec ses gosses, pour vivre. On était tout juste bon pour aller au lit. On n'avait pas gagné grand-chose, en se crevant sur du « mauvais boulot » ; on se disait que si ça continuait, la quinzaine ne serait pas grosse, qu'on devrait encore se priver, compter les sous, se refuser tout ce qui pourrait détendre un peu, faire oublier.\par
Tu te rappelles les chefs, comment ceux qui avaient un caractère brutal pouvaient se permettre toutes les insolences ? Te rappelles-tu qu'on n'osait presque jamais répondre, qu'on en arrivait à trouver presque naturel d'être traité comme du bétail ? Combien de douleurs un cœur humain doit dévorer en silence avant d'en arriver là, les riches ne le comprendront jamais. Quand tu osais élever la voix parce qu'on t'imposait un boulot par trop dur, ou trop mal payé, ou trop d'heures supplémentaires, te rappelles-tu avec quelle brutalité on te disait : « C'est ça ou la porte. » Et, bien souvent, tu te taisais, tu encaissais, tu te soumettais, parce que tu savais que c'était vrai, que c'était ça ou la porte. Tu savais bien que rien ne pouvait les empêcher de te mettre sur le pavé comme on met un outil usé au rancart. Et tu avais beau te soumettre, souvent on te jetait quand même sur le pavé. Personne ne disait rien. C'était normal. Il ne te restait qu'à souffrir de la faim en silence, à courir de boîte en boîte, à attendre debout, par le froid, sous la pluie, devant les portes des bureaux d'embauche. Tu te rappelles tout cela ? Tu te rappelles toutes les petites humiliations qui imprégnaient ta vie, qui faisaient froid au cœur, comme l'humidité imprègne le corps quand on n'a pas de feu ?\par
Si les choses ont changé quelque peu, n'oublie pourtant pas le passé. C'est dans tous ces souvenirs, dans toute cette amertume que tu dois puiser ta force, ton idéal, ta raison de vivre. Les riches et les puissants trouvent le plus souvent leur raison de vivre dans leur orgueil, les opprimés doivent trouver leur raison de vivre dans leurs hontes. Leur part est encore la meilleure, parce que leur cause est celle de la justice. En se défendant, ils défendent la dignité humaine foulée aux pieds. N'oublie jamais, rappelle-toi tous les jours que tu as ta carte syndicale dans ta poche parce qu'à l'usine tu n'étais pas traité comme un homme doit l'être, et que tu en as eu assez.\par
Rappelle-toi surtout, pendant ces années de souffrances trop dures, de quoi tu souffrais le plus. Tu ne t'en rendais peut-être pas bien compte, mais si tu réfléchis un moment, tu sentiras que c'est vrai. Tu souffrais surtout parce que lorsqu'on t'infligeait une humiliation, une injustice, tu étais seul, désarmé, il n'y avait rien pour te défendre. Quand un chef te brimait ou t'engueulait injustement, quand on te donnait un boulot qui dépassait tes forces, quand on t'imposait une cadence impossible à suivre, quand on te payait misérablement, quand on te jetait sur le pavé, quand on refusait de t'embaucher parce que tu n'avais pas les certificats qu'il fallait ou parce que tu avais plus de quarante ans, quand on te rayait des secours de chômage, tu ne pouvais rien faire, tu ne pouvais même pas te plaindre. Ça n'intéressait personne, tout le monde trouvait ça tout naturel. Tes camarades n'osaient pas te soutenir, ils avaient peur de se compromettre s'ils protestaient. Quand on t'avait mis à la porte d'une boîte, ton meilleur copain était quelquefois gêné d'être vu avec toi devant la porte de l'usine. Les camarades se taisaient, ils te plaignaient à peine, Ils étaient trop absorbés par leurs propres soucis, leurs propres souffrances.\par
Comme on se sentait seul ! Tu te rappelles ? Tellement seul qu'on en avait froid au cœur. Seul, désarmé, sans recours, abandonné. À la merci des chefs, des patrons, des gens riches et puissants qui pouvaient tout se permettre. Sans droits, alors qu'eux avaient tous les droits. L'opinion publique était indifférente. On trouvait naturel qu'un patron soit maître absolu dans son usine. Maître des machines d'acier qui ne souffrent pas ; maître aussi des machines de chair, qui souffraient, mais devaient taire leurs souffrances sous peine de souffrir encore plus. Tu étais une de ces machines de chair. Tu constatais tous les jours que seuls ceux qui avaient de l'argent dans leurs poches pouvaient, dans la société capitaliste, faire figure d'hommes, réclamer des égards. Toi, on aurait ri si tu avais demandé à être traité avec égards. Même entre camarades, on se traitait souvent aussi durement, aussi brutalement qu'on était traité par les chefs. Citoyen d'une grande ville, ouvrier d'une grande usine, tu étais aussi seul, aussi impuissant, aussi peu soutenu qu'un homme dans le désert, livré aux forces de la nature. La société était aussi indifférente aux hommes sans argent que le vent, le sable, le soleil sont indifférents. Tu étais plutôt une chose qu'un homme, dans la vie sociale. Et tu en arrivais quelquefois, quand c'était trop dur, à oublier toi-même que tu étais un homme.\par
C'est cela qui a changé, depuis juin. On n'a pas supprimé la misère ni l'injustice. Mais tu n'es plus seul. Tu ne peux pas toujours faire respecter tes droits ; mais il y a une grande organisation qui les reconnaît, qui les proclame, qui peut élever la voix et qui se fait entendre. Depuis juin, il n'y a pas un seul Français qui ignore que les ouvriers ne sont pas satisfaits, qu'ils se sentent opprimés, qu'ils n'acceptent pas leur sort. Certains te donnent tort, d'autres te donnent raison ; mais tout le monde se préoccupe de ton sort, pense à toi, craint ou souhaite ta révolte. Une injustice commise envers toi peut, dans certaines circonstances, ébranler la vie sociale. Tu as acquis une importance. Mais n'oublie pas d'où te vient cette importance. Même si, dans ton usine, le syndicat s'est imposé, même si tu peux à présent te permettre beaucoup de choses, ne te figure pas que « c'est arrivé ». Reprends la juste fierté à laquelle tout homme a droit, mais ne tire de tes droits nouveaux aucun orgueil. Ta force ne réside pas en toi-même. Si la grande organisation syndicale qui te protège venait à décliner, tu recommencerais à subir les mêmes humiliations qu'auparavant, tu serais contraint à la même soumission, au même silence, tu en arriverais de nouveau à toujours plier, à tout supporter, à ne jamais oser élever la voix. Si tu commences à être traité en homme, tu le dois au syndicat. Dans l'avenir, tu ne mériteras d'être traité comme un homme que si tu sais être un bon syndiqué.\par
Être un bon syndiqué, qu'est-ce que cela veut dire ? C'est beaucoup plus peut-être que tu ne te l'imagines. Prendre la carte, les timbres, ce n'est encore rien. Exécuter fidèlement les décisions du syndicat, lutter quand il y a lutte, souffrir quand il le faut, ce n'est pas encore assez. Ne crois pas que le syndicat soit simplement une association d'intérêts. Les syndicats patronaux sont des associations d'intérêts ; les syndicats ouvriers, c'est autre chose. Le syndicalisme, c'est un idéal auquel il faut penser tous les jours, sur lequel il faut toujours avoir les yeux fixés. Être syndicaliste, c'est une manière de vivre, cela veut dire se conformer dans tout ce qu'on fait à l'idéal syndicaliste. L'ouvrier syndicaliste doit se conduire pendant toutes les minutes qu'il passe à l'usine autrement que l'ouvrier non syndiqué. Au temps où tu n'avais aucun droit, tu pouvais ne te reconnaître aucun devoir. Maintenant tu es quelqu'un, tu possèdes une force, tu as reçu des avantages ; mais en revanche tu as acquis des responsabilités. Ces responsabilités, rien dans ta vie de misère ne t'a prépare à y faire face. Tu dois à présent travailler à te rendre capable de les assumer ; sans cela les avantages nouvellement acquis s'évanouiront un beau jour comme un rêve. On ne conserve ses droits que si en est capable de les exercer comme il faut.
\section[Lettres, à Auguste Detœuf, (1936-1937)]{Lettres \\
à Auguste Detœuf \\
(1936-1937)}\renewcommand{\leftmark}{Lettres \\
à Auguste Detœuf \\
(1936-1937)}

\noindent \par
Cher Monsieur,\par
Je m'en veux beaucoup de ne pas arriver à me faire pleinement comprendre de vous, car c'est certainement de ma faute. Si mon projet doit se réaliser un jour – le projet de rentrer chez vous comme ouvrière pour une durée indéterminée, afin de collaborer avec vous de cette place à des tentatives de réformes – il faudra qu'une pleine compréhension se soit établie auparavant.\par
J'ai été frappée de ce que vous m'avez dit l'autre jour, que la dignité est quelque chose d'intérieur qui ne dépend pas des gestes extérieurs. Il est tout à fait vrai qu'on peut supporter en silence et sans réagir beaucoup d'injustices, d'outrages, d'ordres arbitraires sans que la dignité disparaisse, au contraire. Il suffit d'avoir l'âme forte. De sorte que si je vous dis, par exemple, que le premier choc de cette vie d'ouvrière a fait de moi pendant un certain temps une espèce de bête de somme, que j'ai retrouvé peu à peu le sentiment de ma dignité seulement au prix d'efforts quotidiens et de souffrances morales épuisantes, vous êtes en droit de conclure que c'est moi qui manque de fermeté. D'autre part, si je me taisais – ce que j'aimerais bien mieux – à quoi servirait que j’aie fait cette expérience ?\par
De même je ne pourrai pas me faire comprendre tant que vous m'attribuerez, comme vous le faites évidemment, une certaine répugnance soit à l'égard du travail manuel en lui-même, soit à l'égard de la discipline et de l'obéissance en elles-mêmes. J'ai toujours eu au contraire un vif penchant pour le travail manuel (quoique je ne sois pas douée à cet égard, c'est vrai) et notamment pour les tâches les plus pénibles. Longtemps avant de travailler en usine, j'avais appris à connaître le travail des champs : foins – moisson – battage – arrachage des pommes de terre (de 7 h du matin à 10 h du soir...), et malgré des fatigues accablantes j'y avais trouvé des joies pures et profondes. Croyez bien aussi que je suis capable de me soumettre avec joie et avec le maximum de bonne volonté à toute discipline nécessaire à l'efficacité du travail, pourvu que ce soit une discipline humaine.\par
J'appelle humaine toute discipline qui fait appel dans une large mesure à la bonne volonté, à l'énergie et à l'intelligence de celui qui obéit. Je suis entrée à l'usine avec une bonne volonté ridicule, et je me suis aperçue assez vite que rien n'était plus déplacé. On ne faisait appel en moi qu’a ce qu’on pouvait obtenir par la contrainte la plus brutale.\par
L'obéissance telle que je l'ai pratiquée se définit par les caractères que voici. D'abord elle réduit le temps à la dimension de quelques secondes. Ce qui définit chez tout être humain le rapport entre le corps et l'esprit, à savoir que le corps vit dans l'instant présent, et que l'esprit domine, parcourt et oriente le temps, c'est cela qui a défini à cette époque le rapport entre moi et mes chefs. Je devais limiter constamment mon attention au geste que j'étais en train de faire. Je n'avais pas à le coordonner avec d'autres mais seulement à le répéter jusqu'à la minute où un ordre viendrait m'en imposer un autre. C'est un fait bien connu que lorsque le sentiment du temps se borne à l'attente d'un avenir sur lequel on ne peut rien, le courage s'efface. En second lieu, l'obéissance engage l'être humain tout entier ; dans votre sphère un ordre oriente l'activité, pour moi un ordre pouvait bouleverser de fond en comble le corps et l'âme, parce que j'étais – comme plusieurs autres – presque continuellement à la limite de mes forces. Un ordre pouvait tomber sur moi dans un moment d'épuisement, et me contraindre à forcer – à forcer jusqu'au désespoir. Un chef peut imposer soit des méthodes de travail, soit des outils défectueux, soit une cadence, qui ôtent toute espèce d'intérêt aux heures passées hors de l'usine, par l'excès de la fatigue. De légères différences de salaires peuvent aussi, dans certaines situations, affecter la vie elle-même. Dans ces conditions, on dépend tellement des chefs qu'on ne peut pas ne pas les craindre, et – encore un aveu pénible – il faut un effort perpétuel pour ne pas tomber dans la servilité. En troisième lieu, cette discipline ne fait appel, en fait de mobiles, qu'à l'intérêt sous sa forme la plus sordide – à l'échelle des sous – et à la crainte. Si on accorde une place importante en soi-même à ces mobiles, on s'avilit. Si on les supprime, si en se rend indifférent aux sous et aux engueulades, on se rend du même coup inapte à obéir avec la complète passivité requise et à répéter les gestes du travail à la cadence imposée ; inaptitude promptement punie par la faim. J'ai parfois pensé qu'il vaudrait mieux être plié à une semblable obéissance du dehors, par exemple à coups de fouet, que de devoir ainsi s'y plier soi-même en refoulant ce qu'on a de meilleur en soi.\par
Dans cette situation, la grandeur d'âme qui permet de mépriser les injustices et les humiliations est presque impossible à exercer. Au contraire, bien des choses en apparence insignifiantes – le pointage, la nécessité de présenter une carte d'identité à l'entrée de l'usine (chez Renault), la manière dont s'effectue la paie, de légères réprimandes – humilient profondément, parce qu'elles rappellent et rendent sensible la situation où on se trouve. De même pour les privations et pour la faim.\par
La seule ressource pour ne pas souffrir, c'est de sombrer dans l'inconscience. C'est une tentation à laquelle beaucoup succombent, sous une forme quelconque, et à laquelle j'ai souvent succombé. Conserver la lucidité, la conscience, la dignité qui conviennent à un être humain, c'est possible, mais c'est se condamner à devoir surmonter quotidiennement le désespoir. Du moins c'est ce que j'ai éprouvé.\par
Le mouvement actuel est à base de désespoir. C'est pourquoi il ne peut être raisonnable. Malgré vos bonnes intentions, vous n'avez rien tenté jusqu'ici pour délivrer de ce désespoir ceux qui vous sont subordonnés ; aussi n'est-ce pas à vous à blâmer ce qu'il y a de déraisonnable dans ce mouvement. C'est pour cela que, l'autre jour, je me suis un peu échauffée dans la discussion – ce que j'ai regretté par la suite – quoique je sois entièrement d'accord avec vous sur la gravité des dangers à craindre. Pour moi aussi, c'est au fond le désespoir qui fait que j'éprouve une joie sans mélange à voir enfin mes camarades relever une bonne fois la tête, sans aucune considération des conséquences possibles.\par
Cependant je crois que si les choses tournent bien, c'est-à-dire si les ouvriers reprennent le travail dans un délai assez court, et avec le sentiment d'avoir remporté une victoire, la situation sera favorable dans quelque temps pour tenter des réformes dans vos usines. Il faudra d'abord leur laisser le temps de perdre le sentiment de leur force passagère, de perdre l'idée qu'on peut les craindre, de reprendre l'habitude de la soumission et du silence. Après quoi vous pourrez peut-être établir directement entre eux et vous les rapports de confiance indispensables à toute action, en leur faisant sentir que vous les comprenez – si toutefois j'arrive à vous les faire comprendre, ce qui suppose évidemment d'abord que je ne me trompe pas en croyant les avoir compris moi-même.\par
En ce qui concerne la situation actuelle, si les ouvriers reprennent le travail avec des salaires peu supérieurs à ceux qu'ils avaient, cela ne peut se produire que de deux manières. Ou ils auront le sentiment de céder à la force, et se remettront au travail avec humiliation et désespoir. Ou on leur accordera des compensations morales, et il n'y en a qu'une possible : la faculté de contrôler que les bas salaires résultent d'une nécessité, et non pas d'une mauvaise volonté du patron. C'est presque impossible, je le sais bien. En tout cas les patrons, s'ils étaient sages, devraient tout faire pour que les satisfactions qu'ils accorderont donnent aux ouvriers l'impression d'une victoire. Dans leur état d'esprit actuel, ils ne supporteraient pas le sentiment de la défaite.\par
Je reviendrai sans doute à Paris mercredi soir. Je passerai volontiers chez vous jeudi ou vendredi matin avant 9 heures, si toutefois je ne vous dérange pas et s'il vous paraît utile que nous causions. Je me connais ; je sais qu'une fois cette période d'effervescence passée je n'oserai plus aller ainsi chez vous, de peur de vous importuner, et, de votre côté, vous serez peut-être de nouveau entraîné par le courant des occupations quotidiennes à ajourner certains problèmes.\par
Si je risque de vous déranger le moins du monde, vous n'aurez qu'à me le faire savoir, ou bien simplement ne pas me recevoir. Je sais très bien que vous avez bien autre chose à faire qu'à causer.\par
Croyez à toute ma sympathie.\par
S. WEIL.\par
{\itshape P.-S.} –Vous avez vu les {\itshape Temps modernes} \footnote{Film de Charlie Chaplin.}, je suppose ? La machine à manger, voilà le plus beau et le plus vrai symbole de la situation des ouvriers dans l'usine.\par
\noindent Vendredi.\par
\noindent Cher Monsieur,\par
Ce matin, j'ai réussi à pénétrer par fraude chez Renault, malgré la sévérité du service d'ordre. J'ai pensé qu'il pouvait être utile de vous communiquer mes impressions.\par
1° Les ouvriers ne savent rien des pourparlers. – On ne les met au courant de rien. Ils croient que Renault refuse d'accepter le contrat collectif. Une ouvrière m'a dit : il paraît que pour les salaires, c'est arrangé, mais il ne veut pas admettre le contrat collectif. Un ouvrier m'a dit : pour nous je crois que ça se serait arrangé il y a 3 jours, mais comme les gens de la maîtrise nous ont soutenus, nous les soutenons à notre tour. Etc. – Ils trouvent, hélas, naturel de ne rien savoir. Ils ont tellement l'habitude...\par
2° On commence nettement à en avoir marre. Certains, quoique ardents, l'avouent ouvertement.\par
3° Il règne une atmosphère extraordinaire de défiance, de suspicion. Un cérémonial singulier : ceux qui sortent et ne rentrent pas, qui s'absentent sans autorisation, on les voue à l'infamie en écrivant leurs noms sur un tableau dans un atelier (coutume russe), en les pendant en effigie et en organisant en leur honneur un enterrement burlesque. Presque sûrement, à la reprise du travail, on exigera leur renvoi. Par ailleurs, peu de camaraderie dans l'atmosphère. Silence général.\par
4° Il y a 3 jours (je crois) un syndicat « professionnel » des agents de maîtrise (à partir des régleurs inclusivement !) a été constitué, sur l'initiative des Croix de Feu à ce qu'on dit. Les ouvriers disent qu'il a été dissous dès le lendemain, et que 97 \% des agents de maîtrise et techniciens ont adhéré à la C. G. T.\par
Seulement la caisse d'assurances de Renault – qui occupe un local de Renault, et fait partie de l'entreprise – est en grève, mais sans drapeaux à la porte, et affiche deux exemplaires d'un papier démentant la dissolution du syndicat, annonçant qu'il compte 3.500 adhérents, qu'il en a été constitué d'autres semblables chez Citroën, Fiat, etc., et qu'il va immédiatement se mettre à recruter parmi les ouvriers. Cela à quelques mètres des bâtiments où flottent les drapeaux rouges. Nul ne semble se soucier de lacérer ces papiers ou même de les démentir.\par
Conclusion : il est certain à présent qu'il y a manœuvre. Mais de qui ? Maurice Thorez a fait un discours invitant clairement à mettre fin à la grève.\par
J'en arrive à me demander si les cadres subalternes du parti communiste n'ont pas échappé à la direction du parti pour tomber aux mains d'on ne sait qui. Car il est assez clair que tout se fait encore au nom du parti communiste ({\itshape Internationale}, banderoles, faucilles et marteaux, etc., à profusion), quoiqu'il coure le bruit d'une mauvaise réception faite à Costes.\par
J'en reste toujours à mon idée, peut-être utopique, mais la seule issue, il me semble, autre que l'État totalitaire. Si la classe ouvrière impose aussi brutalement sa force, il faut qu'elle assume des responsabilités correspondantes. Il est inadmissible et en dernière analyse impossible qu'une catégorie sociale irresponsable impose ses désirs par la force et que les chefs, seuls responsables, soient contraints de céder. Il faut ou un certain partage des responsabilités, ou un rétablissement brutal de la hiérarchie, lequel n'irait sans doute pas, de quelque manière qu'il se fasse, sans effusion de sang.\par
J'imagine très bien un chef d'entreprise disant en substance à ses ouvriers, une fois le travail repris (si les choses s'arrangent tant bien que mal, provisoirement) : on entre de votre fait dans une ère nouvelle. Vous avez voulu mettre fin aux souffrances que vous imposaient depuis des années les nécessités de la production industrielle. Vous avez voulu manifester votre force. Fort bien. Mais il en résulte une situation entendez faire peser la force de vos revendications sur les entreprises industrielles, vous devez pouvoir faire face aux responsabilités des conditions nouvelles que vous avez suscitées. Nous sommes désireux de faciliter l'adaptation de l'entreprise à ce nouveau rapport de forces. À cet effet, nous favoriserons l'organisation de cercles d'études techniques, économiques et sociales dans l'usine. Nous donnerons des locaux à ces cercles, nous les autoriserons à faire appel, pour des conférences, d'une part aux techniciens de l'usine, d'autre part à des techniciens et économistes membres des organisations syndicales ; nous organiserons pour eux des visites de l'usine avec explications techniques, nous favoriserons la création de bulletins de vulgarisation ; tout cela pour permettre aux ouvriers, et plus particulièrement aux délégués ouvriers, de comprendre ce qu'est l'organisation et la gestion d'une entreprise industrielle.\par
C'est une idée hardie, sans doute, et peut-être dangereuse. Mais qu'est-ce qui n'est pas dangereux en ce moment ? L'élan dont sont animés les ouvriers la rendrait peut-être praticable. En tout cas je vous demande instamment de la prendre en considération.\par
Je conçois ainsi la question de l'autorité, sur le plan de la pure théorie : d'une part les chefs doivent commander, bien sûr, et les subordonnés obéir ; d'autre part les subordonnés ne doivent pas se sentir livrés corps et âme à une domination arbitraire, et à cet effet ils doivent non certes collaborer à l'élaboration des ordres, mais pouvoir se rendre compte dans quelle mesure les ordres correspondent à une nécessité.\par
Mais tout ça, c'est l'avenir. La situation présente se résume ainsi :\par
1° Les patrons ont accordé des concessions incontestablement satisfaisantes, d'autant que vos ouvriers se sont trouvés satisfaits à moins.\par
2° Le parti communiste a pris officiellement position (quoique avec des périphrases) pour la reprise du travail, et par ailleurs je sais de source sûre que dans certains syndicats les militants communistes ont effectivement travaillé à empêcher la grève (services publics).\par
3° Les ouvriers de chez Renault et sans doute des autres usines ignorent tout des pourparlers en cours ; ce ne sont donc pas eux qui agissent pour empêcher l'accord.\par
J'ai écrit à Roy (qui aujourd'hui est absent de Paris) pour lui donner ces renseignements, et je les ai également transmis à un militant responsable de l'Union des syndicats de la Seine, un camarade sérieux et qui leur a accordé l'attention convenable.\par
Tout ce que je vous dis là se rapporte à la situation présente ; car le refus de la convention conclue entre les patrons et la C. G. T. (15 à 7 \%) semble avoir été au contraire tout à fait spontané.\par
Bien sympathiquement.\par
S. WEIL.\par
Je reviendrai sans doute à Paris demain soir pour 24 h. Il est extrêmement pénible et angoissant de devoir rester en province dans une pareille situation.\par
\subsection[a) Lettre de Simone Weil ]{a) Lettre de Simone Weil \protect\footnotemark }
\footnotetext{{\itshape Nouveaux Cahiers}, 15 décembre 1937. – Correspondance entre S. Weil et A. Detœuf.}
\noindent Cher ami,\par
Dans le train, j'ai entendu causer deux patrons, moyens patrons apparemment (voyageant en seconde, ruban rouge), l'un, semblait-il, provincial, et l'autre faisant la navette entre la province et la région parisienne, le premier dans le textile, le second dans le textile et la métallurgie ; cheveux blancs, un peu corpulents, air très respectable ; le second jouant un certain rôle dans le syndicalisme patronal de la métallurgie parisienne. Leurs propos m'ont semblé si remarquables que je les ai notés en arrivant chez moi. Je vous les transcris (en les mêlant de quelques commentaires)\par
…………………………………………………………………….\par
« Voilà qu'on reparle du contrôle de l'embauche et de la débauche. Dans les mines, on met des commissions paritaires, oui, avec les représentants ouvriers à côté du patron. Vous vous rendez compte ? On ne va plus pouvoir prendre et renvoyer qui on veut ? – Oh ! c'est incontestablement une violation de la liberté. –C'est la fin de tout ! –Oui, vous avez raison ; comme vous disiez tout à l'heure, ils font si bien qu'on est complètement dégoûté, {\itshape si dégoûté qu'on ne prend plus les commandes, même si on en a}. – Parfaitement. – Nous, nous avons voté à la presque unanimité une résolution pour dire qu'on ne veut pas du contrôle, qu'on fermerait plutôt les usines. Si on en faisait autant partout, ils devraient céder. – Oh ! si la loi passait, on n'aurait plus qu'à fermer tous. – Oui, quoi, on n'a plus rien à perdre... »\par
Parenthèse : Il est étrange que des hommes qui sont bien nourris, bien vêtus, bien chauffés, qui voyagent confortablement en seconde, croient n'avoir rien à perdre. Si leur tactique, qui était celle des patrons russes en 1917, amenait un bouleversement social qui les chasse, errants, sans ressources, sans passeport, sans carte de travail, en pays étranger, ils s'apercevraient alors qu'ils avaient beaucoup à perdre. Dès maintenant, ils pourraient se documenter auprès de ceux qui, ayant occupé en Russie des situations équivalentes aux leurs, sont encore aujourd'hui à peiner misérablement comme manœuvres chez Renault.\par
« ...Oui, quoi, on n'a plus rien à perdre ! – Rien. – Et puis enfin, on serait comme un capitaine de navire qui n'a plus rien à dire, qui n'a plus qu'à s'enfermer dans sa cabine, pendant que l'équipage est sur la passerelle. »\par
…………………………………………………………………….\par
« ...Le patron est l’être le plus détesté. Détesté de tout le monde. Et c'est lui pourtant qui fait vivre tout le monde. Comme c'est étrange, cette injustice. Oui, détesté de tous, – Autrefois, au moins, il y avait des égards. Je me souviens, dans ma jeunesse... – C'est fini, ça. – Oui, même là où la maîtrise est bonne... – Oh ! les salopards ont fait tout ce qu'il fallait pour nous amener là. {\itshape Mais ils le paieront}. »\par
Cette dernière parole sur un ton de haine concentrée. Sans vouloir être alarmiste, de pareilles conversations, il faut le reconnaître, ne peuvent avoir lieu que dans une atmosphère qui n'est pas celle de la paix civile.\par
« ... On ne s'en rend pas du tout compte, mais le fleuve de la vie sociale dérive de la caisse des patrons. S'ils fermaient tous en même temps, qui est-ce qui pourrait faire quoi que ce soit ? On sera forcé d'en venir là, alors les gens comprendront. Les patrons ont eu le tort d'avoir peur. Ils n'avaient qu'à dire : les leviers de commande, c'est nous qui les avons. Et ils auraient imposé leur volonté. »\par
On les aurait bien étonnés en leur disant que leur plan n'est que l'équivalent patronal de la grève générale, à l'égard de laquelle, sans doute, ils n'ont pas assez de mots pour exprimer leur réprobation. Si les patrons peuvent légitimement faire une telle grève pour avoir le droit de prendre ou renvoyer qui bon leur semble, pourquoi les ouvriers ne pourraient-ils pas faire la grève générale pour avoir le droit de n'être pas refusés ou renvoyés par caprice ? Eux, dans les sombres années 1934-35, n'avaient vraiment plus grand-chose à perdre.\par
Par ailleurs, ces deux braves messieurs n'ont même pas l'air d'imaginer que si les patrons bouclaient tous ensemble, on rouvrirait les usines sans leur demander la clef et on les ferait tourner sans eux. L'exemple de la Russie tend à faire penser que les années qui suivraient ne seraient agréables pour personne mais elles ne le seraient surtout pas pour eux.\par
\par
« ...Oui, après tout, on n'a plus rien à perdre. Oh ! non, plus rien du tout; autant crever. – Oui, s'il faut crever, en tout cas, il vaut mieux crever en beauté. – J'ai bien l'impression que cela va être maintenant la bataille de la Marne des patrons. Ils sont complètement acculés, et maintenant... »\par
Ici l'arrêt du train a mis fin à la conversation. L'évocation de la bataille de la Marne, elle aussi, fait plutôt songer à la guerre civile qu'à de simples conflits sociaux. Ces souvenirs militaires, ces termes de « crever » et « on n'a plus rien à perdre », répétés à satiété, sonnaient d'une manière assez comique de la part de ces messieurs corrects, bedonnants, bien nourris, ayant au plus haut point cet aspect confortable, pacifique et rassurant qui est celui du Français moyen.\par
Ce n'est là qu'une conversation particulière. Mais je pense qu'une conversation, dans un lieu presque public, entre deux personnes – et c'était évidemment le cas – dont l'originalité n'est pas la principale qualité, ne peut avoir lieu que si une atmosphère assez générale la rend possible ; de sorte qu'une seule conversation est concluante. Celle-là est, je crois, bonne à mettre au dossier qu'on pourrait constituer à la suite de l'article de Detœuf : Sabotage patronal et sabotage ouvrier. J'avais donné raison, en gros, à Detœuf ; je crois encore qu'il a eu raison, mais plus pour une période à présent écoulée que pour le moment présent. Ou plutôt, pour ne pas exagérer, je pense que la situation se développe de manière à lui donner un peu moins raison tous les jours. En tout cas, ce qu'on doit constater, c'est que des pensées de sabotage circulent ; que chez certains le dégoût a provoqué l'équivalent patronal d'une grève perlée. Du moins c'est ce que j'ai entendu affirmer en propres termes ; je vous garantis l'exactitude des phrases que je vous rapporte.\par
Vous pouvez publier cette lettre dans les {\itshape Nouveaux Cahiers}, (C'est même pour cela que je vous l'écris.)\par
Bien amicalement.\par
S. WEIL.\par
{\itshape P.-S.} – Voici ce que la situation présente a de plus paradoxal. Les patrons, parce qu'ils croient qu'ils n'ont plus rien à perdre, prennent le vocabulaire et l'attitude révolutionnaire. Les ouvriers, parce qu'ils croient qu'ils ont quelque chose d'assez important à perdre, prennent le vocabulaire et l'attitude conservatrice.\par
\par
*
\subsection[b) Réponse de A. Detœuf]{b) Réponse de A. Detœuf}
\noindent Ma chère amie,\par
La conversation que vous rapportez est des plus intéressantes ; sans généraliser au point où vous le faites, je crois qu'elle reflète un état d'esprit des plus fréquents. Mais elle ne m'inspire pas les mêmes réflexions qu’à vous. Vous raisonnez avec votre âme qui s'identifie, par tendresse et esprit de justice, avec l'âme ouvrière, alors qu'il s'agit de comprendre des patrons, qui sont peut-être d'anciens ouvriers, mais qui sont certainement depuis longtemps des patrons.\par
Voulez-vous que nous laissions de côté ce qu'il y a d'un peu grotesque, et aussi d'un peu odieux dans le fait d'être bedonnant, bien nourri ? C'est un malheur que les deux industriels que vous avez rencontrés et moi-même partageons avec des représentants de la classe ouvrière, et même avec des ouvriers, qui ne jugent pas pour cela que tout soit pour le mieux dans le meilleur des mondes. Si j'insiste sur ce point, secondaire assurément dans votre esprit, c'est qu'à la vérité, dans l'exposé objectif de la conversation que vous avez entendue, et dans les commentaires d'une logique impitoyable qui l'accompagnent, ce seul caractère pittoresque, physique, parle à l'imagination et écarte ainsi, me semble-t-il, de la sérénité nécessaire.\par
Oublions donc, si vous le voulez bien, l'aspect physique de vos deux patrons. Que résulte-t-il de leur conversation ? Incontestablement qu'ils sont exaspérés, qu'ils croient n'avoir plus rien à perdre, qu'ils sont disposés à fermer leurs usines pour résister à une loi sur l'embauchage qui les priverait de certaines prérogatives qu'ils jugent indispensables à leur gestion, et qu'une grève générale des patrons leur paraîtrait une insurrection patriotique.\par
Vous leur dites qu'ils ont beaucoup plus à perdre qu'ils ne le croient, qu'ils envisagent d'user d'un moyen d'action qu'ils réprouvent chez leurs employés, que leurs usines fonctionneront bien sans eux ; et vous concluez que la tendance au sabotage patronal s'accroît.\par
Et, dans tout cela, il y a une part de vérité, mais, à mon sens, cette part de la vérité qui ne peut conduire, dans l'immédiat, à rien de pratique, à rien de meilleur.\par
Mettez-vous un peu à la place de vos deux patrons. Ces hommes ont cru être tout-puissants dans leur entreprise ; ils y ont risqué ce qu'ils avaient d'argent ; ils ont probablement peiné longtemps et durement, avec de graves soucis ; ils se sont débattus pendant des années contre tout le monde : leurs concurrents, leurs fournisseurs, leurs clients, leur personnel. Ils ont été formés à regarder le monde comme composé d'ennemis, à ne pouvoir compter sur personne, que sur quelques employés exceptionnels, dont, la plupart du temps, ils trouvaient le dévouement naturel. Ils ont l'impression de n'avoir jamais rien demandé à personne, de n'avoir jamais désiré qu'une chose, c'est qu'on leur fiche la paix ; qu'on les laisse se débrouiller. Se débrouiller, en roulant quelquefois celui-ci, en écrasant quelquefois celui-là, il est vrai. Mais sans remords, sans l'ombre d'un souci, puisqu'ils appliquent la règle commune ; puisqu'ils jouent le jeu ; puisque personne ne leur a appris qu'il y a une solidarité sociale ; puisque personne autour d'eux ne la pratique. Ils sont assurés d'avoir fait leur devoir, en essayant de gagner de l'argent ; et ils accueillent volontiers cette idée supplémentaire qu'en défendant leur peau, ce qui est leur principale raison d'agir, ils enrichissent la collectivité et rendent service à la nation. Ils en sont d'autant plus convaincus, qu'ils ont vu, à côté d'eux, des gens gagner plus d'argent qu'eux en se bornant à jouer des rôles de commissionnaires, d'intermédiaires, à spéculer et quelquefois à escroquer l'épargne, sans être punis.\par
Ajoutez à cela que les dernières années de ce régime les ont persuadés que, seules, la menace et la violence réussissent ; qu'en criant assez fort, qu'en se montrant assez indisciplinés vis-à-vis de l'État, qu'en affirmant qu'on entend se soustraire aux lois, on est assuré (à la condition d'être assez nombreux) non seulement de l'impunité, mais encore du succès. Et vous voudriez que, seuls, ils conservent le souci de ne pas créer de difficultés au Gouvernement, à un Gouvernement appuyé par un parti qui envisage leur totale dépossession !\par
Je ne vous dis pas ici que leurs raisons soient valables, que leur sentiment soit juste ; je vous demande seulement de constater qu'à moins d'être au-dessus de l’humanité, ils ne peuvent guère penser autrement.\par
Lorsqu'ils parlent de « crever », lorsqu'ils disent « qu'ils n'ont plus rien à perdre », pour une part, ils exagèrent ; ils cherchent à la fois, à trouver chez le confrère cet appui qui leur a toujours manqué, et à le convaincre qu'ils ont plus d'énergie et d'esprit collectif qu'ils n'en ont réellement. Mais ils le croient vraiment. Et ici, il faut bien que vous fassiez un effort d'imagination pour vous rendre compte que ces hommes n'ont pas tant d'imagination que vous leur en prêtez. N'avoir plus rien à perdre, pour eux, c'est abandonner leur entreprise, leur raison d'être, leur milieu social, tout ce qui est, pour eux, l'existence. Ils ne connaissent pas la faim, ils ne peuvent pas imaginer la faim ; ils ne connaissent pas l'exil, ils ne peuvent pas imaginer l'exil ; mais ils connaissent l'exemple de la faillite, de la ruine, du déclassement, des enfants qu'on ne peut pas établir comme il était de toute éternité entendu qu'on les établirait. Et la destruction des conditions habituelles de leur existence, c'est, pour eux, la destruction de leur existence. Supposez qu'on vous dise : vous continuerez à bien manger, à avoir chaud ; on s'occupera de vous, mais vous serez idiote et considérée par tous comme une épave. Ne diriez-vous pas : « Je n'aurai plus rien à perdre » ? Ce qu'est pour vous l'activité de votre esprit – ce que sont pour vous vos émotions sociales, morales, esthétiques, pour eux, tout cela est accroché à leur usine, à une usine qui a toujours fonctionné d'une certaine façon et qu'ils n'imaginent pas fonctionnant autrement. Je laisse de côté exprès tout ce qu'il peut y avoir chez eux de beau, de noble, de désintéressé. Car il y a tout de même de tout cela ; mais pour le découvrir, il faudrait avoir exercé vis-à-vis d'eux sa sympathie depuis longtemps.\par
Accordez-moi donc que vos deux patrons ne peuvent guère penser autrement qu'ils ne font, et passons à un second point. Sont-ils inutiles, et, comme vous le dites, se passera-t-on d'eux ? Je ne crois ni l'un ni l'autre. S'il est relativement aisé de remplacer le dirigeant d'une grande entreprise par un fonctionnaire, le petit patron ne peut être remplacé que par un patron. Fonctionnarisée, son entreprise s'arrêterait très vite. Toute son activité, tout son débrouillage, toute son adaptation quotidienne à une situation sans cesse changeante, toute cette action qui exige des décisions, des risques, des responsabilités ininterrompues est tout le contraire de l'action du salarié, surtout du salarié d'une collectivité. De toutes les difficultés qu'a rencontrées l'économie communiste russe, celles qui viennent de la suppression du petit commerce, de la petite industrie, de l'artisanat, sont les plus graves, celles qu'elle n'a pas surmontées et qu'elle ne surmontera pas. Quelle que soit l'Économie nouvelle qu'on envisage, le patronat petit et moyen demeurera. Vous trouvez qu'il comprend mal la situation ; il ne la comprendra pas du jour au lendemain ; mais il peut apprendre à la comprendre. Il a déjà, depuis dix-huit mois, appris beaucoup plus qu'on ne croit.\par
Ne faites donc pas la même erreur que lui. Il veut faire des choses que vous jugez absurdes, et vous avez besoin de lui. Si vous voulez qu'il ne les fasse pas, il faut tâcher de le calmer. Certaines précautions sont nécessaires pour l'embauchage et le débauchage : il faut les prendre, mais en les réduisant au {\itshape strict minimum indispensable} ; notamment, est-ce bien sur les petits patrons que doit s'exercer l'effort de réglementation pour la protection de la masse ouvrière ? Je ne le crois pas. Si les embauchages sont faits correctement dans la grande industrie, ne croyez-vous pas que le jeu naturel de l'offre et de la demande conduira à l'embauchage correct dans la petite industrie ? Si vous voulez réglementer un trop grand nombre d'entreprises, vous créez un fonctionnariat excessif, un contrôle impraticable, des frictions constantes. Ce n'est pas par une action directe, c'est par une action indirecte que vous devez arriver à faire l'éducation du patronat petit et moyen. Celui-ci a l'habitude de s'adapter à ce qui est la force des choses ; s'il proteste aujourd'hui, c'est parce qu'il a devant lui la force des hommes, d'hommes qu'il n'a pas choisis, d'hommes qu'il estime tyranniques.\par
N'essayez pas de lui imposer votre volonté par des règlements qu'il ne comprend pas ; vous n'y arriveriez pas. D'une part, vous ne pourrez le remplacer, non seulement parce que l'État échouera lamentablement dans cet essai, mais parce qu'il n'osera jamais l'entreprendre. Les masses ouvrières sont concentrées, il est vrai, mais elles ne représentent qu'un quart de ce pays ; elles ne peuvent lui imposer leur volonté. Pour avoir, faute d'expérience, manqué de mesure dans leurs revendications de salaires, voici qu'une grande partie du pays les désavoue, sinon en paroles, du moins du fond du cœur. Ce n'est point en France qu'une exploitation d'État des petites entreprises sera jamais envisagée. Et, d'autre part, si vous renoncez à l'exploitation directe, soyez assurée que vos règlements multiples, divers et nécessairement inhumains, seront rapidement tournés, moqués, et tomberont en désuétude.\par
Vos patrons sont exaspérés ; pas au point, soyez-en assurée, d'oublier leur intérêt personnel, qui, pour une grande part, se confond avec l'intérêt général. Une grève générale contre des menaces de législation étroite de l'embauchage, je ne la considère pas comme exclue ; car il s'agit de mesures qui atteignent chacun directement dans ce qu'il croit être ses œuvres vives. Mais ce n'est qu'une manifestation. Ce qui est redoutable, ce n'est pas cela ; c'est l'état d'esprit avec lequel sera appliquée une législation peut-être bureaucratique, peut-être tatillonne, peut-être anti-économique, peut-être même anti-sociale ; une législation qui ne sera pas comprise par une partie de ceux à qui elle s'appliquera. Il faut une législation qui soit comprise, et pour cela qui ne transforme pas du tout au tout le régime actuel ; qui empêche les abus sans prétendre régler l'exercice courant de l'autorité patronale. Et elle est possible. Mais il faut la vouloir et ne pas se laisser entraîner à jeter le désordre, sous le prétexte d'établir un peu d'ordre ; à exaspérer une partie, et la plus active peut-être de l'Économie, sous le prétexte d'établir la paix sociale ; à promulguer, avec un gouvernement aussi faible que celui que nous avons, des lois que ce gouvernement sera, dès l'origine, incapable d'appliquer.\par
Il faut accepter qu'il y ait des hommes bedonnants et qui ne raisonnent pas toujours très juste, pour qu'au lieu de quelques chômeurs à peu près secourus, il n'y ait pas un peuple entier crevant de faim et exposé à toutes les aventures.\par
A. DETŒUF,
\section[Remarques sur les enseignements à tirer des conflits du Nord  (1936-1937 ?)]{Remarques sur les enseignements à tirer des conflits du Nord \protect\footnotemark  \\
(1936-1937 ?)}\renewcommand{\leftmark}{Remarques sur les enseignements à tirer des conflits du Nord  \\
(1936-1937 ?)}

\footnotetext{Rapport à la C. G. T. au retour d'une mission (1936-1937 ?).}
\noindent \par
\subsection[QUESTION DE LA DISCIPLINE, DE LA QUALITÉ ET DU RENDEMENT]{QUESTION DE LA DISCIPLINE, DE LA QUALITÉ ET DU RENDEMENT}
\noindent Il y a d'autant plus intérêt à examiner sérieusement cette question qu'elle se pose plus ou moins pour toute l'industrie française. Dans le Nord, elle est devenue rapidement l'objectif essentiel des conflits. Les patrons ont lutté pour les sanctions avec le sentiment de défendre la cause de l'autorité dans la France entière ; les ouvriers avec le sentiment de défendre les conquêtes morales de juin pour toute une classe ouvrière de France. Il serait absurde de considérer, comme on l'a fait jusqu'ici dans des déclarations officielles, que les plaintes des patrons sont entièrement mensongères ; car elles ne le sont pas. Elles sont certes exagérées, mais elles contiennent une part de vérité incontestable.\par
Il est facile de comprendre les données du problème. Avant juin, les usines vivaient sous le régime de la terreur. Cette terreur amenait fatalement les patrons, même les meilleurs, aux solutions de facilité. Le choix des chefs était devenu presque indifférent ; ils n'avaient pas besoin de se faire respecter parce qu'ils avaient le pouvoir de tout faire plier devant eux ; ils n'avaient même pas besoin, le plus souvent, de compétence technique parce qu'on poursuivait la baisse du prix de revient par l'aggravation de la cadence et la baisse du salaire. Toute l'organisation du travail était comprise de manière à faire appel, chez les ouvriers, aux mobiles les plus bas, la peur, le désir de se faire bien voir, l'obsession des sous, la jalousie entre camarades. Le mois de juin a apporté à la classe ouvrière une transformation morale qui a supprimé toutes les conditions sur lesquelles se fondait l'organisation des usines. Il aurait fallu procéder à une réorganisation. Les patrons ne l'ont pas fait.\par
Le mouvement de juin a été avant tout une réaction de détente, et cette détente dure encore. La crainte, la jalousie, la course aux primes ont disparu dans une assez large mesure, alors que la conscience professionnelle et l'amour du travail avaient été considérablement affaiblis chez les ouvriers, au cours des années qui ont précédé juin, par la disqualification progressive du travail et par une oppression inhumaine qui implantait au cœur des ouvriers la haine de l'usine. Devant cette détente générale, les patrons se sont sentis paralysés parce qu'ils n'ont pas compris. Ils ont continué à faire tourner les usines en profitant des habitudes acquises ; leur seule innovation, purement négative et provoquée par la crainte, a consisté à supprimer pratiquement les sanctions, dans une plus ou moins grande mesure selon les cas, et parfois totalement. Dès lors il devenait inévitable qu'il y ait du jeu dans les rouages de transmission de l'autorité patronale, et un certain flottement dans la production.\par
Il s'est ainsi produit depuis juin une transformation psychologique du côté ouvrier comme du côté patronal. C'est là un fait d'une importance capitale. La lutte des classes n'est pas simplement fonction des intérêts, la manière dont elle se déroule dépend en grande partie de l'état d'esprit qui règne dans tel ou tel milieu social.\par
Du côté ouvrier, la nature même du travail semble avoir changé, dans une mesure plus ou moins grande selon les usines. Sur le papier le travail aux pièces est maintenu, mais les choses se passent dans une certaine mesure comme s'il n'existait plus ; en tout cas la cadence du travail a perdu son caractère obsédant, les ouvriers ont tendance à revenir au rythme naturel du travail. Du point de vue syndicaliste qui est le nôtre, il y a là incontestablement un progrès moral, d'autant plus que l'accroissement de la camaraderie a contribué à ce changement en supprimant, chez les ouvriers, le désir de se dépasser les uns les autres. Mais en même temps, à la faveur du relâchement de la discipline, la mentalité bien connue de l'ouvrier qui a trouvé une «  planque » a pu se développer chez certains. Et ce qui, du point de vue syndicaliste, est plus grave que la diminution de la cadence, c'est qu'il y a eu incontestablement dans certaines usines diminution de la qualité du travail, du fait que les contrôleurs et vérificateurs, ne subissant plus au même degré la pression patronale et devenus sensibles à celle de leurs camarades, sont devenus plus larges pour les pièces loupées. Quant à la discipline, les ouvriers se sont sentis le pouvoir de désobéir et en ont profité de temps à autre. Ils ont tendance, notamment, à refuser l'obéissance aux contremaîtres qui n'adhèrent pas à la C. G. T. Dans certains endroits, particulièrement à Maubeuge, des contremaîtres ont presque perdu le pouvoir de déplacer les ouvriers. Il y a eu plusieurs cas de refus d'obéissance devant lesquels la maîtrise a dû s'incliner ; il y a eu aussi des cas fréquents de réunions pendant les heures de travail, à quelques-uns, ou par équipes, ou par ateliers, et de débrayage pour des motifs insignifiants.\par
Les contremaîtres, habitués à commander brutalement et qui avant juin n'avaient presque jamais eu besoin de persuader, se sont trouvés tout à fait désorientés ; placés entre les ouvriers et la direction devant laquelle ils étaient responsables, mais qui ne les soutenait pas, leur situation est devenue moralement très difficile. Aussi sont-ils passés peu à peu pour la plupart, surtout à Lille, dans le camp anti-ouvrier, et cela même lorsqu'ils gardaient la carte de la C. G. T. À Lille, on a remarqué que vers le mois d'octobre, ils commençaient à revenir à leurs manières autoritaires d'autrefois. Quant aux directeurs et aux patrons, ils ont presque tout laissé faire, presque tout supporté passivement et sans rien dire ; mais les griefs et les rancœurs se sont accumulés dans leur esprit, et le jour où pour couronner tout le reste une grève apparemment sans objectif a éclaté, ils se sont trouvés décidés à briser le syndicat au prix de n'importe quels sacrifices. Dès lors le conflit a eu pour objectif les conquêtes mêmes de juin qu'il s'agissait d'un côté de conserver, de l'autre de détruire, alors que jusque-là ces conquêtes n'étaient même pas mises en question. Et les patrons, en voyant la misère accabler peu à peu les grévistes, ont pu se rendre compte de leur pouvoir, dont ils avaient perdu conscience depuis juin.\par
La désaffection des techniciens à l'égard du mouvement ouvrier est au reste une des principales causes qui ont amené le patronat à reprendre confiance dans sa propre force. Cette désaffection progressive, que l'on pouvait prévoir dès le mois de juin, qu'il était impossible d'éviter entièrement, a pris des proportions désastreuses pour le mouvement syndical. Les patrons n'ont plus peur, comme en juin, que l'usine tourne sans eux. L'expérience a été faite à Lille. Dans une usine de 450 ouvriers, le patron, ayant décidé le lock-out parce que les ouvriers ne voulaient pas permettre le renvoi du délégué principal, a abandonné l'usine ; les techniciens et employés, tous syndiqués à la C. G. T., l'ont tous suivi, et les ouvriers, après avoir essayé pendant deux jours de faire marcher l'usine seuls, ont dû renoncer. Une telle expérience change d'une manière décisive le rapport des forces.
\subsection[RÔLE DES DÉLÉGUÉS OUVRIERS]{RÔLE DES DÉLÉGUÉS OUVRIERS}
\noindent Les délégués ouvriers ont joué un rôle de premier plan dans cette évolution. Élus pour veiller à l'application des lois sociales, ils sont bientôt devenus un pouvoir dans les usines et se sont considérablement écartés de leur mission théorique. La cause doit en être cherchée d'une part dans la panique qui a saisi les patrons après juin et les a parfois amenés à une attitude voisine de l'abdication, d'autre part dans le cumul des fonctions propres des délégués et de fonctions syndicales qui n'ont jamais été prévues par aucun texte. Les délégués sont peu à peu apparus aux ouvriers comme une émanation de l'autorité syndicale, et les ouvriers, habitués depuis des années à l'obéissance passive, peu entraînés à la pratique de la démocratie syndicale, se sont accoutumés à recevoir leurs ordres.\par
L'assemblée des délégués d'une usine ou d'une localité remplace ainsi en fait dans une certaine mesure l'assemblée générale d'une part, d'autre part les organismes proprement syndicaux. C'est ainsi qu'à Maubeuge les délégués d'une usine, s'étant réunis pour examiner les moyens d'imposer aux patrons la conclusion du contrat collectif, ont envisagé de proposer à l'assemblée des délégués de Maubeuge un ralentissement général de la production ; et le lendemain un des délégués de cette usine a pris sur lui d'ordonner à son équipe de diminuer la cadence du travail. À Lille, quand le bureau du syndicat a décidé la généralisation de la grève, il a convoqué les délégués pour leur transmettre le mot d'ordre. Un délégué qui ordonne un débrayage au secteur qu'il représente est immédiatement obéi. Ainsi les délégués ont un pouvoir double ; un pouvoir vis-à-vis des patrons, parce qu'ils peuvent appuyer toutes les réclamations, même les plus infimes ou les plus absurdes, par la menace du débrayage ; vis-à-vis des ouvriers, parce qu'ils peuvent à leur choix appuyer ou non la demande de tel ou tel ouvrier, interdire ou non qu'on lui impose une sanction, parfois même demander son renvoi.\par
Quelques faits précis survenus à Maubeuge peuvent donner une idée des abus auxquels on arrive. Dans une usine, les délégués font sortir un syndiqué chrétien ; le directeur le fait revenir à sa place ordinaire, et les délégués, pour se venger du directeur, viennent interdire à telle ou telle équipe l'exécution d'un travail urgent. Aucune sanction n'a été prise. Ailleurs, une équipe ayant chanté l'{\itshape Internationale} sur le passage de visiteurs, le délégué, appelé au bureau pour donner des explications, fait débrayer avant de s'y rendre. Aucune sanction. Ailleurs, les délégués ordonnent une grève perlée sans consulter le syndicat. Ailleurs, les délégués font débrayer pour obtenir le renvoi des syndiqués chrétiens. Ailleurs, plusieurs délégués amènent les ouvriers assiéger un atelier, pendant les heures de travail, pour sortir de l'usine un autre délégué, adhérant à la C. G. T., qu'ils accusent d'être vendu à la direction. Les délégués décident aussi de la cadence du travail, tantôt la font descendre au-dessous de ce que comporte un travail normal, tantôt la font monter au point que les ouvriers ne peuvent pas suivre.\par
Même là où les abus ne vont pas si loin, les délégués ont souvent tendance à accroître l'importance de leur rôle au-delà de ce qui est utile. Ils recueillent presque indistinctement les réclamations légitimes ou absurdes, importantes ou infimes, ils harcèlent la maîtrise et la direction, souvent avec la menace du débrayage à la bouche, et créent chez les chefs, sur qui pèsent déjà lourdement les préoccupations purement techniques, un état nerveux intolérable. Il y a lieu d'ailleurs de se demander s'il s'agit seulement de maladresse, ou s'il n'y a pas là quelquefois une tactique consciente, comme semblerait l'indiquer une phrase prononcée un jour par un délégué ouvrier d'une autre région, qui se vantait de harceler son chef d'atelier tous les jours, sans répit, pour ne jamais lui laisser le loisir de reprendre le dessus. D'autre part, le pouvoir que possèdent les délégués a dès à présent créé une certaine séparation entre eux et les ouvriers du rang ; de leur part la camaraderie est mêlée d'une nuance très nette de condescendance, et souvent les ouvriers les traitent un peu comme des supérieurs hiérarchiques. Cette séparation est d'autant plus accentuée que les délégués négligent souvent de rendre compte de leurs démarches. Enfin, comme ils sont pratiquement irresponsables, du fait qu'ils sont élus pour un an, et comme ils usurpent en fait des fonctions proprement syndicales, ils en arrivent tout naturellement à dominer le syndicat. Ils ont la possibilité d'exercer sur les ouvriers syndiqués ou non une pression considérable, et c'est eux qui déterminent en fait l'action syndicale, du fait qu'ils peuvent à volonté provoquer des heurts, des conflits, des débrayages et presque des grèves.
\subsection[CONCLUSION]{CONCLUSION}
\noindent Toutes ces remarques concernent le Nord, mais il y a à coup sûr un état de choses plus ou moins général, qui se reproduit à des degrés différents un peu dans tous les coins de la France. Il importe donc d'en tirer quelques conclusions pratiques pour l'action syndicale.\par
1° L'état d'exaspération contenue et silencieuse dans lequel se trouvent un peu partout un certain nombre de chefs, de directeurs d'usines, de patrons, {\itshape rend toute grève extrêmement dangereuse dans la période actuelle}. Là où les chefs et patrons sont encore décidés à supporter bien des choses pour éviter la grève, il pourra se faire que la grève une fois déclenchée les amène brusquement à la résolution farouche de briser le syndicat, même au risque de couler leur usine. Or quand un patron en est arrivé là, il a toujours le pouvoir de briser le syndicat en infligeant aux ouvriers les souffrances de la faim. Il ne peut être retenu que par la crainte d'être exproprié ; mais cette crainte, qu'on éprouvait en juin, n'existe plus, d'une part parce qu'on sait que le gouvernement ne réquisitionne pas les usines, d'autre part parce que les patrons réussissent de mieux en mieux à séparer les techniciens des ouvriers. Même une grève en apparence victorieuse, si elle est longue, peut être funeste au syndicat, comme on l'a vu chez Sautter-Harlé, et comme on risque de le voir dans le Nord ; car le patron, après la reprise du travail, peut toujours procéder à des licenciements massifs, sans que les ouvriers, épuisés par la grève, aient la force de réagir.\par
Tous ces dangers sont encore bien plus grands lorsqu'il s'agit de grèves sans objectif précis, comme c'était le cas à Lille, à Pompey et à Maubeuge, grèves qui donnent aux patrons et au public l'impression d'une agitation aveugle dont on peut tout craindre et qu'il faut briser à tout prix.\par
La loi sur l'arbitrage obligatoire est donc dans les conditions actuelles une ressource précieuse pour la classe ouvrière, et l'action syndicale doit en ce moment tendre essentiellement à l'utiliser.\par
2° Rétablir {\itshape la subordination normale des délégués à l'égard du syndicat est presque devenu une question de vie ou de mort pour notre mouvement syndical}. Divers moyens peuvent être préconisés à cet effet ; il semble nécessaire de les employer tous, y compris les plus énergiques.\par
Le plus efficace consisterait à instituer {\itshape des sanctions syndicales}. La C. G. T. pourrait décréter publiquement que lorsqu'un délégué demandera le renvoi d'un ouvrier, ou donnera des ordres concernant le travail, ou ordonnera un débrayage ou une grève perlée sans décision préalable et régulièrement prise du syndicat, elle réclamera automatiquement à ce délégué sa démission. On pourrait aussi obliger les délégués à faire un rapport mensuel au syndicat énumérant brièvement toutes leurs démarches auprès de la direction, et donner à tous les syndiqués la faculté de lire ce rapport. On pourrait d'une part diffuser très largement parmi les délégués et parmi tous les ouvriers des textes indiquant nettement et énergiquement les limites du rôle et du pouvoir des délégués ; d'autre part porter à la connaissance des patrons que les délégués sont subordonnés à la C. G. T. et qu'à ce titre l'organisation syndicale, à ses divers échelons, est l'arbitre naturel de tous les différends entre patrons et délégués ouvriers. Enfin la séparation morale qui tend à se créer entre délégués et ouvriers du rang semble indiquer la nécessité impérieuse de décider {\itshape la non-rééligibilité des délégués au bout d'un an}.\par
3° La C. G. T. ne peut pas ignorer le problème de la discipline du travail et du rendement. Nous n'avons pas lieu d'hésiter à reconnaître que le problème se pose ; ce n'est pas à nous qu'on peut reprocher le fait qu'il se pose. La classe ouvrière, au cours des années passées, n'a pas été formée par le mouvement syndical, dont on entravait l'influence par tous les moyens ; elle a reçu l'empreinte que lui a imprimée le patronat par le régime et les mœurs implantés dans les usines. S'il a plu aux patrons d'instituer dans les usines un régime de travail tel que tout progrès moral de la classe ouvrière devait inévitablement troubler la production, ils en portent l'entière responsabilité ; et c'est même là la meilleure marque du mal qu'ils ont fait au temps où ils étaient les maîtres.\par
Cependant la C. G. T., si elle n'est pas responsable du passé, est responsable de l'avenir en raison de la puissance qu'elle a acquise. Il se pose devant l'industrie française un problème qui n'est pas particulier à un département, à une corporation, mais qui se retrouve partout à des degrés différents. Ce problème, les patrons sont incapables de le résoudre, parce qu'ils ne sont même pas parvenus à en comprendre les données. La C. G. T. a là une occasion unique de montrer sa capacité en s'attaquant à ce problème dans son ensemble, sur le plan national ; il y a même probablement nécessité vitale pour notre mouvement ouvrier à parvenir à une solution.\par
Avant juin, il y avait dans les usines un certain ordre, une certaine discipline, qui étaient fondés sur l'esclavage. L'esclavage a disparu dans une large mesure ; l'ordre lié à l'esclavage a disparu du même coup. Nous ne pouvons que nous en féliciter. Mais l'industrie ne peut pas vivre sans ordre. La question se pose donc d'un ordre nouveau, compatible avec les libertés nouvellement acquises, avec le sentiment renouvelé de la dignité ouvrière et de la camaraderie. La situation actuelle, qui reproduit exactement l'organisation périmée du travail avec les sanctions en moins, est instable et par suite grosse de conflits possibles. D'un côté les patrons, se sentant privés d'action sur leurs propres usines du fait qu'ils n'osent plus guère prendre de sanctions, cherchent par tous les moyens à reprendre des morceaux de l'autorité perdue et s'exaspèrent s'ils n'y arrivent pas ; d'autre part les ouvriers sont maintenus par ces tentatives dans une alerte continuelle et une sourde effervescence. Au reste l'absence de sanctions ne peut pas se perpétuer sans un danger grave et réel pour la production ; et il n'est même pas de l'intérêt moral de la classe ouvrière que les ouvriers se sentent irresponsables dans l'accomplissement du travail. Il faut donc obtenir une discipline, un ordre, des sanctions qui ne rétablissent pas l'arbitraire patronal d'avant juin.\par
La C. G. T. peut s'appuyer d'une part sur l'autorité morale qu'elle possède auprès des ouvriers, d'autre part sur le fait que dans les circonstances actuelles il y a dans une certaine mesure coïncidence entre l'intérêt des patrons et celui du mouvement ouvrier. La stabilisation des conquêtes de juin est un moindre mal pour les patrons soucieux de l'intérêt immédiat de leurs entreprises, par rapport au désordre et aux menaces vagues qu'ils sentent peser sur eux ; pour nous, cette stabilisation est pour la période actuelle une nécessité vitale.\par
Dans ces conditions, n'y aurait-il pas un intérêt capital pour la C. G. T. à prendre les mesures suivantes :\par
1° Mettre à l'étude dans les syndicats, dans les Fédérations, au Bureau confédéral la question d'un ordre nouveau, d'une discipline nouvelle dans les entreprises industrielles.\par
2° Inviter d'une part toutes les sections syndicales, d'autre part tous les patrons à transmettre au Bureau confédéral des rapports sur toutes les difficultés qui concernent les questions d'ordre, de discipline, de rendement, de qualité du travail, ces rapports étant destinés d'une part à fournir les éléments d'une étude d'ensemble, d'autre part à donner au Bureau confédéral la possibilité de fournir le cas échéant un avis motivé.\par
3° Inviter la Confédération générale de la production française à étudier en commun avec la C. G. T., toujours dans le même domaine, d'une part le problème dans son ensemble, d'autre part tous les cas particuliers présentant un certain caractère de gravité.
\section[Principes d'un projet pour un régime intérieur nouveau dans les entreprises industrielles, (1936-1937 ?)]{Principes d'un projet pour un régime intérieur nouveau dans les entreprises industrielles \\
 (1936-1937 ?)}\renewcommand{\leftmark}{Principes d'un projet pour un régime intérieur nouveau dans les entreprises industrielles \\
 (1936-1937 ?)}

\noindent \par
Nous nous trouvons en ce moment dans un état d'équilibre social instable qu'il y a lieu de transformer, si possible, pour une certaine période, en un équilibre stable. Malgré l'opposition qui existe entre les objectifs et les aspirations des deux classes en présence, cette transformation est en ce moment conforme à l'intérêt des deux parties. La classe ouvrière a un intérêt vital à assimiler ses conquêtes récentes, à les fortifier, à les implanter solidement dans les mœurs. Seuls quelques fanatiques irresponsables, d'ailleurs sans influence, peuvent désirer dans la période actuelle précipiter sa marche en avant. Les patrons soucieux de l'avenir prochain de leurs entreprises ont eux aussi intérêt à cette consolidation. Ils ne pourraient revenir à l'état de choses d'il y a un an qu'au prix d'une lutte acharnée qui causerait beaucoup de dégâts, qui ruinerait beaucoup d’entreprises, qui tournerait peut-être à la guerre civile, et qui aurait cinquante pour cent de chances d'aboutir à la dépossession définitive du patronat. D'autre part un ordre nouveau, même s'il comporte de leur part certaines concessions importantes, serait de beaucoup préférable pour les patrons au désordre qui, {\itshape s'il faut les croire}, règne actuellement dans un certain nombre d'entreprises, et à l'incertitude qui les exaspère. Dans ces limites précises et sur cette base on peut concevoir pour une certaine période une collaboration constructive entre les éléments sérieux et responsables de la classe ouvrière et du patronat.\par
L'élaboration d'un nouveau régime intérieur des entreprises pose un problème dont les données sont déterminées en partie par le régime actuel, mais qui, dans son essence, est lié à l'existence de la grande industrie, indépendamment du régime social. Il consiste à établir un certain équilibre, dans le cadre de chaque entreprise, entre les droits que peuvent légitimement revendiquer les travailleurs en tant qu'êtres humains et l'intérêt matériel de la production. Un tel équilibre ne s'établirait automatiquement que s'il pouvait y avoir coïncidence parfaite entre les mesures à prendre en vue de ces deux objectifs ; coïncidence qui n’est concevable dans aucune hypothèse. En fait, cet équilibre ne peut jamais être fondé que sur un compromis. L'existence actuelle du régime capitaliste n'intervient dans les données du problème que pour donner un sens déterminé à la notion de l'intérêt de la production ; cet intérêt, dans le régime actuel, se mesure dans chaque entreprise par l'argent et se définit d'après les lois de l'économie capitaliste. Les patrons, en raison des avantages personnels qu'ils poursuivent, mais bien plus encore en raison de leur fonction, représentent nécessairement l'intérêt de la production ainsi défini. Ils tendent tout naturellement à faire de cet intérêt la règle unique de l'organisation des entreprises. Ils y ont à peu près complètement réussi, à la naturellement à faire entrer leurs droits et leur dignité d'hommes en ligne de compte. Ils ont accompli de sérieux progrès dans ce sens en juin dernier.\par
Il s'agit à présent de cristalliser ces progrès en un régime nouveau, qui serve la production dans toute la mesure compatible avec l'état d'esprit actuel des ouvriers, avec le sentiment renouvelé de la dignité et de la fraternité ouvrière, avec les avantages moraux acquis. Le sens dans lequel doit s'accomplir cette tentative est indiqué par la nature même du problème. Le patronat, dans sa mission de défendre la production de l'entreprise, a vu s'affaiblir entre ses mains les armes dont il disposait à l'égard des ouvriers : la terreur, l'excitation des petites jalousies, l'appel à l'intérêt personnel le plus sordide. Ce qui a été perdu de ce côté, il faut essayer de le regagner du côté des mobiles élevés auxquels le patronat s'adressait si rarement : l'amour-propre professionnel, l'amour du travail, l'intérêt pris dans la tâche bien accomplie, le sentiment de la responsabilité.\par
\par
Il faut en second lieu que les ouvriers se sentent liés à la production par autre chose que par la préoccupation obsédante de gagner quelques sous de plus en gagnant quelques minutes sur les temps alloués. Il faut qu'ils puissent mettre en jeu les facultés qu'aucun être humain normal ne peut laisser étouffer en lui-même sans souffrir et sans se dégrader, l'initiative, la recherche, le choix des procédés les plus efficaces, la responsabilité, la compréhension de l'œuvre à accomplir et des méthodes à employer. Ce ne sera possible que si la première condition est réalisée. Le sentiment d'infériorité n'est pas favorable au développement des facultés humaines.\par
C'est à cette double préoccupation que répondent les indications suivantes :\par
\subsection[DISCIPLINE DU TRAVAIL]{DISCIPLINE DU TRAVAIL}
\noindent La discipline du travail ne doit plus être unilatérale, mais reposer sur la notion d'obligations réciproques. À cette condition seulement elle peut être acceptée, et non plus simplement subie. La direction d'une entreprise a la responsabilité du matériel et de la production : à ce titre son autorité doit jouer sans aucune entrave, dans certaines limites bien définies. Mais ce n'est pas à la direction que doit être confiée la responsabilité de la partie vivante d'une entreprise ; cette responsabilité doit revenir à la section syndicale, et celle-ci doit posséder un pouvoir, également dans des limites bien définies, pour la sauvegarde des êtres humains engagés dans la production. La discipline d'une entreprise doit reposer sur la coexistence de ces deux pouvoirs.\par
La section syndicale doit imposer le respect de la vie et de la santé des ouvriers. Tout ouvrier doit pouvoir en appeler à elle s'il reçoit un ordre qui mette en péril sa santé ou sa vie ; soit qu'on lui impose un travail malsain, ou trop dur pour ses forces physiques, ou une cadence impliquant des risques d'accident grave, ou une méthode de travail dangereuse ; elle doit pouvoir en pareil cas, dans les circonstances graves, couvrir de son autorité un refus d'obéissance sérieusement motivé ; elle doit enfin pouvoir faire appliquer les dispositifs de sécurité et les mesures d'hygiène qu'elle juge nécessaires et empêcher d'une manière générale la cadence du travail d'atteindre une vitesse dangereuse ou épuisante. Au cas où la direction contesterait la justesse de ses décisions, elle doit être dans l'obligation de produire l'avis motivé d'hommes qualifiés choisis selon la circonstance (médecins ou techniciens).\par
La direction doit avoir pleine autorité, dans les limites déterminées par les droits de la section syndicale, pour veiller au respect du matériel, à la qualité et à la quantité du travail, à l'exécution des ordres. Elle doit avoir le pouvoir absolu de déplacer les ouvriers dans l'entreprise, sous la seule réserve qu'il lui serait interdit, lorsqu'un ouvrier déplacé subit de ce fait un déclassement, de mettre à sa place primitive un autre ouvrier embauché au-dehors ou pris dans une catégorie inférieure.\par
Ces deux autorités doivent s'appuyer l'une et l'autre, le cas échéant, par des sanctions. La direction peut prendre des sanctions pour négligence, faute professionnelle, mauvais travail ou refus d'obéir. La section syndicale à son tour doit pouvoir prendre des sanctions, soit contre la direction, soit contre les agents de maîtrise, dans le cas où ses décisions, prises dans le cadre indiqué plus haut et régulièrement motivées, n'auraient pas été exécutées et où il en serait résulté un dommage effectif ou un danger sérieux.\par
Le mode d'application des sanctions pourrait être déterminé comme suit. La personne menacée de sanction pourrait toujours en appeler devant une commission tripartite (ouvriers, techniciens, patrons) fonctionnant pour un groupe d'entreprises ; et au cas où cette commission ne serait pas unanime, en appeler de nouveau devant un expert nommé d'une manière permanente par les fédérations ouvrière ou patronale, ou, à leur défaut, par le gouvernement. Toute sanction confirmée serait automatiquement aggravée d'une manière considérable, toute sanction non confirmée vaudrait une amende à la partie qui l'aurait proposée.\par
Les sanctions seraient d'une part, pour l'ensemble du personnel salarié, le déclassement temporaire ou définitif, la mise à pied, le renvoi ; d'autre part pour la maîtrise et la direction le blâme, des amendes, et en cas de faute très grave, notamment de faute très grave ayant entraîné une mort, l'interdiction définitive d'exercer un commandement industriel.\par
En aucun cas des actes commis au cours d'une grève ne peuvent être l'objet de sanctions, non plus que la grève elle-même. Si des violences se sont produites pendant une grève, elles relèvent de la correctionnelle, mais les condamnations en correctionnelle ne doivent pas rompre le contrat de travail, sauf le cas de longues peines de prison sans sursis.
\subsection[LICENCIEMENTS]{LICENCIEMENTS}
\noindent Les conditions actuelles du fonctionnement des entreprises ne permettent pas d'ôter aux patrons la possibilité de licencier des ouvriers soit pour réorganisation technique de l'entreprise, soit pour manque de travail. Mais il faut admettre aussi que le respect de la vie humaine doit limiter le pouvoir de prendre une mesure aussi grave, qui risque de briser une existence.\par
On peut admettre le compromis suivant. Le patron qui licencie un ouvrier a le devoir de lui chercher au préalable une place dans une autre entreprise. Il pourra prendre des mesures de licenciement sans rendre de comptes à personne sauf les trois cas suivants :\par
1° Si l'ouvrier licencié est un responsable syndical.\par
2° Si le patron qui le licencie lui fournit une place inacceptable pour des raisons graves.\par
3° Si le patron le licencie sans pouvoir lui indiquer une autre place.\par
Dans chacun de ces trois cas, l'ouvrier licencié pourra obliger le patron à soumettre la mesure de licenciement au contrôle d'experts nommés par le gouvernement et la C. G. T. Ceux-ci examinent notamment si le licenciement n'aurait pas pu être évité par la répartition des heures de travail. S'ils tombent d'accord pour juger que le licenciement n'est pas justifié, le patron devra, après avoir reçu leur avis motivé, reprendre le ou les ouvriers en cause.\par
Lorsqu'un patron aura licencié un ouvrier, il ne pourra plus embaucher personne, soit dans la même profession, soit comme manœuvre, sans avoir fait d'abord appel à lui. La section syndicale doit avoir les pouvoirs nécessaires pour contrôler l'application de cette règle.\par
La formation professionnelle des ouvriers a été complètement négligée par le patronat toutes ces dernières années. Il en est résulté la situation où nous nous trouvons présentement. La valeur professionnelle de la classe ouvrière française a été amoindrie par cette négligence. La C. G. T. est prête à étudier avec le C. G. P. F. et le gouvernement la question de la formation professionnelle des jeunes et des adultes et de la rééducation professionnelle des chômeurs.
\subsection[RÉGIME DU TRAVAIL]{RÉGIME DU TRAVAIL}
\noindent Parallèlement à l'organisation générale de la formation professionnelle, il faut prendre progressivement, dans les entreprises, les mesures propres à intéresser les ouvriers à leur travail autrement que par l'appât du gain.\par
Les ouvriers ne doivent plus ignorer ce qu'ils fabriquent, usiner une pièce sans savoir où elle ira ; il faut leur donner le sentiment de collaborer à une œuvre, leur donner la notion de la coordination des travaux. Le meilleur moyen serait peut-être d'organiser le samedi des visites de l'entreprise, par équipes, avec autorisation, pour les ouvriers, d'emmener leur famille, et sous la conduite du technicien qualifié capable de faire un exposé simple et intéressant. Il serait bon également de rendre compte aux ouvriers de toutes les innovations, fabrications nouvelles, changements de méthode, perfectionnements techniques. Il faut leur donner le sentiment que l'entreprise vit, et qu'ils participent à cette vie. La direction et la section syndicale doivent collaborer d'une manière permanente à cet effet.\par
Il faut aussi chercher d'autres moyens de stimuler les suggestions que les primes classiques. De suggestions qui comportent pour l'usine un avantage permanent, il semble normal que les ouvriers tirent aussi une récompense permanente. On peut imaginer toutes sortes de modalités. Par exemple des diminutions de la cadence ou des améliorations dans les mesures d'hygiène pour les ateliers qui auraient fourni des suggestions intéressantes ; la suppression totale du travail aux pièces, remplacé par le travail à l'heure au taux horaire moyen, pour les ateliers qui feraient preuve dans ce domaine d'une activité intellectuelle constante, etc. Dans la recherche des modes de travail et de rétribution propres à stimuler chez les ouvriers les mobiles les plus élevés sans nuire au rendement global, et à leur donner le maximum de liberté sans nuire à l'ordre, la direction et la section syndicale doivent aussi collaborer d'une manière permanente. Sur ce terrain, l'expérience seule décide, et les initiatives les plus hardies sont les meilleures. La section syndicale d'une entreprise doit toujours pouvoir réclamer la mise à l'essai de toute méthode ayant fait ses preuves dans une entreprise analogue.
\section[La rationalisation  (23 février 1937)]{La rationalisation \protect\footnotemark  \\
(23 février 1937)}\renewcommand{\leftmark}{La rationalisation  \\
(23 février 1937)}

\footnotetext{Simone Weil a fait le 23 février 1937, devant un auditoire ouvrier une conférence dont nous n'avons pas le manuscrit original mais seulement ce texte partiel recueilli par un auditeur.}
\noindent \par
Le mot de « rationalisation » est assez vague. Il désigne certaines méthodes d'organisation industrielle, plus ou moins rationnelles d'ailleurs, qui règnent actuellement dans les usines, sous diverges formes. Il y a, en effet, plusieurs méthodes de rationalisation, et chaque chef d'entreprise les applique à sa manière. Mais elles ont toutes des points communs et se réclament toutes de la science, en ce sens que les méthodes de rationalisation sont présentées comme des méthodes d'organisation scientifique du travail.\par
La science n'a été, au début, que l'étude des lois de la nature. Elle est intervenue ensuite dans la production par l'invention et la mise au point des machines et par la découverte de procédés permettant d'utiliser les forces naturelles. Enfin, à notre époque, vers la fin du siècle dernier, on a songé à appliquer la science, non plus seulement à l'utilisation des forces de la nature, mais à l'utilisation de la force humaine de travail.\par
C'est quelque chose de tout à fait nouveau, dont nous commençons à apercevoir les effets.\par
\par
On parle souvent de la révolution industrielle pour désigner justement la transformation qui s'est produite dans l'industrie lorsque la science s'est appliquée à la production et qu'est apparue la grosse industrie. Mais on peut dire qu'il y a eu une deuxième révolution industrielle. La première se définit par l'utilisation scientifique de la matière inerte et des forces de la nature. La deuxième se définit par l'utilisation scientifique de la matière vivante, c'est-à-dire des hommes.\par
La rationalisation apparaît comme un perfectionnement de la production. Mais si on considère la rationalisation du seul point de vue de la production, elle se range parmi les innovations successives dont est fait le progrès industriel ; tandis que si on se place du point de vue ouvrier, l'étude de la rationalisation fait partie d'un très grand problème, le problème d'un régime acceptable dans les entreprises industrielles. Acceptable pour les travailleurs, bien entendu ; et c'est surtout sous ce dernier aspect que nous devons envisager la rationalisation, car si l'esprit du syndicalisme se différencie de l'esprit qui anime les milieux dirigeants de notre société, c'est surtout parce que le mouvement syndical s'intéresse encore plus au producteur qu'à la production, contrairement à la société bourgeoise qui s'intéresse surtout à la production plutôt qu'au producteur.\par
Le problème du régime le plus désirable dans les entreprises industrielles est un des plus importants, peut-être même le plus important, pour le mouvement ouvrier. Il est d'autant plus étonnant qu'il n'ait jamais été posé. À ma connaissance, il n'a pas été étudié par égard. Les théoriciens étaient peut-être mal placés pour traiter ce sujet, faute d'avoir été eux-mêmes au nombre des rouages d'une usine.\par
Le mouvement ouvrier lui-même, qu'il s'agisse du syndicalisme ou des organisations ouvrières qui ont précédé les syndicats, n'a pas songé non plus à traiter largement les différents aspects de ce problème. Bien des raisons peuvent l'expliquer, notamment les préoccupations immédiates, urgentes, quotidiennes qui s'imposent souvent d'une manière trop impérieuse aux travailleurs pour leur laisser le loisir de réfléchir aux grands problèmes. D'ailleurs, ceux qui, parmi les militants ouvriers, restent soumis à la discipline industrielle, n'ont guère la possibilité ni le goût d'analyser théoriquement la contrainte qu'ils subissent chaque jour : ils ont besoin de s'évader ; et ceux qui sont investis de fonctions permanentes ont souvent tendance à oublier, au milieu de leur activité quotidienne, qu'il y a là une question urgente et douloureuse.\par
De plus, il faut bien le dire, nous subissons tous une certaine déformation qui vient de ce que nous vivons dans l'atmosphère de la société bourgeoise, et même nos aspirations vers une société meilleure s'en ressentent. La société bourgeoise est atteinte d'une monomanie : la monomanie de la comptabilité. Pour elle, rien n'a de valeur que ce qui peut se chiffrer en francs et en centimes. Elle n'hésite jamais à sacrifier des vies humaines à des chiffres qui font bien sur le papier, chiffres de budget national ou de bilans industriels. Nous subissons tous un peu la contagion de cette idée fixe, nous nous laissons également hypnotiser par les chiffres. C'est pourquoi, dans les reproches que nous adressons au régime économique, l'idée de l'exploitation, de l'argent extorqué pour grossir les profits, est presque la seule que l'on exprime nettement. C'est une déformation d'esprit d'autant plus compréhensible que les chiffres sont quelque chose de clair, qu'on saisit du premier coup, tandis que les choses qu'on ne peut pas traduire en chiffres demandent un plus grand effort d'attention. Il est plus facile de réclamer au sujet du chiffre marqué sur une feuille de paie que d'analyser les souffrances subies au cours d'une journée de travail. C'est pourquoi la question des salaires fait souvent oublier d'autres revendications vitales. Et on arrive même à considérer la transformation du régime comme définie par la suppression de la propriété capitaliste et du profit capitaliste comme si cela était équivalent à l'instauration du socialisme.\par
Eh bien, c'est là une lacune extrêmement grave pour le mouvement ouvrier, car il y a bien autre chose que la question des profits et de la propriété dans toutes les souffrances subies par la classe ouvrière du fait de la société capitaliste.\par
L'ouvrier ne souffre pas seulement de l'insuffisance de la paie. Il souffre parce qu'il est relégué par la société actuelle à un rang inférieur, parce qu'il est réduit à une espèce de servitude. L'insuffisance des salaires n'est qu'une conséquence de cette infériorité et de cette servitude. La classe ouvrière souffre d'être soumise à la volonté arbitraire des cadres dirigeants de la société, qui lui imposent, hors de l'usine, son niveau d'existence, et, dans l'usine, ses conditions de travail.\par
Les souffrances subies dans l'usine du fait de l'arbitraire patronal pèsent autant sur la vie d'un ouvrier que les privations subies hors de l'usine du fait de l'insuffisance de ses salaires.\par
\par
Les droits que peuvent conquérir les travailleurs sur le lieu du travail ne dépendent pas directement de la propriété ou du profit, mais des rapports entre l'ouvrier et la machine, entre l'ouvrier et les chefs, et de la puissance plus ou moins grande de la direction. Les ouvriers peuvent obliger la direction d'une usine à leur reconnaître des droits sans priver les propriétaires de l'usine ni de leur titre de propriété ni de leurs profits ; et réciproquement, ils peuvent être tout à fait privés de droits dans une usine qui serait une propriété collective. Les aspirations des ouvriers à avoir des droits dans l'usine les amènent à se heurter non pas avec le propriétaire mais avec le directeur. C'est quelquefois le même homme, mais peu importe.\par
Il y a donc deux questions à distinguer : l'exploitation de la classe ouvrière qui se définit par le profit capitaliste, et l'oppression de la classe ouvrière sur le lieu du travail qui se traduit par des souffrances prolongées, selon le cas, 48 heures ou 40 heures par semaine, mais qui peuvent se prolonger encore au-delà de l'usine sur les 24 heures de la journée.\par
La question du régime des entreprises, considérée du point de vue des travailleurs, se pose avec des données qui tiennent à la structure même de la grande industrie. Une usine est essentiellement faite pour produire. Les hommes sont là pour aider les machines à sortir tous les jours le plus grand nombre possible de produits bien faits et bon marché. Mais d'un autre côté, ces hommes sont des hommes ; ils ont des besoins, des aspirations à satisfaire, et qui ne coïncident pas nécessairement avec les nécessités de la production, et même en fait n'y coïncident pas du tout le plus souvent. C'est une contradiction que le changement de régime n'éliminerait pas. Mais nous ne pouvons pas admettre que la vie des hommes soit sacrifiée à la fabrication des produits.\par
Si demain on chasse les patrons, si on collectivise les usines, cela ne changera en rien ce problème fondamental qui fait que ce qui est nécessaire pour sortir le plus grand nombre de produits possible, ce n'est pas nécessairement ce qui peut satisfaire les hommes qui travaillent dans l'usine.\par
Concilier les exigences de la fabrication et les aspirations des hommes qui fabriquent est un problème que les capitalistes résolvent facilement en supprimant l'un de ses termes : ils font comme si ces hommes n'existaient pas. À l'inverse, certaines conceptions anarchistes suppriment l'autre terme : les nécessités de la fabrication. Mais comme on peut les oublier sur le papier, non les éliminer en fait, ce n'est pas là une solution. La solution idéale, ce serait une organisation du travail telle qu'il sorte chaque soir des usines à la fois le plus grand nombre possible de produits bien faits et des travailleurs heureux. Si, par un hasard providentiel, on pouvait trouver une telle méthode de travail, assez parfaite pour rendre le travail joyeux, la question ne se poserait plus. Mais cette méthode n'existe pas, et c'est même tout le contraire qui se passe. Et si une telle solution n'est pas pratiquement réalisable, c'est justement parce que les besoins de la production et les besoins des producteurs ne coïncident pas forcément. Ce serait trop beau si les procédés de travail les plus productifs étaient en même temps les plus agréables. Mais on peut tout au moins s'approcher d'une telle solution en cherchant des méthodes qui concilient le plus possible les intérêts de l'entreprise et les droits des travailleurs. On peut poser en principe qu'on peut résoudre leur contradiction par un compromis en trouvant un moyen terme, tel que ne soient pas entièrement sacrifiés ni les uns ni les autres ; ni les intérêts de la production ni ceux des producteurs. Une usine doit être organisée de manière que la matière première qu'elle utilise ressorte en produits qui ne soient ni trop rares, ni trop coûteux, ni défectueux, et qu'en même temps les hommes qui y entrent un matin n'en sortent pas diminués physiquement ni moralement le soir, au bout d'un jour, d'un an ou de vingt ans.\par
C'est là le véritable problème, le problème le plus grave qui se pose à la classe ouvrière : trouver une méthode d'organisation du travail qui soit acceptable à la fois pour la production, pour le travail et pour la consommation.\par
Ce problème, on n'a même pas commencé à le résoudre, puisqu'il n'a pas été posé ; de sorte que si demain nous nous emparions des usines, nous ne saurions quoi en faire et nous serions forcés de les organiser comme elles le sont actuellement, après un temps de flottement plus ou moins long.\par
Je n'ai pas moi-même de solution à vous présenter. Ce n'est pas là quelque chose qu'on puisse improviser de toutes pièces sur le papier. C'est dans les usines seulement qu'on peut arriver peu à peu à imaginer un système de ce genre et à le mettre à l'épreuve, exactement comme les patrons et les chefs d'entreprises, les techniciens, sont arrivés peu à peu à concevoir et à mettre au point le système actuel. Pour comprendre comment se pose le problème, il faut avoir étudié le système qui existe, l'avoir analysé, en avoir fait la critique, avoir apprécié en quoi il est bon ou mauvais, et pourquoi. Il faut partir du régime actuel pour en concevoir un meilleur.\par
Je vais donc essayer d'analyser ce régime (que vous connaissez mieux que qui que ce soit) en me référant à la fois à. son histoire, aux ouvrages de ceux qui ont contribué à l'élaborer, et à la vie quotidienne des usines dans la période qui a précédé le mouvement de juin 1936.\par
Pour caractériser le régime actuel de l'industrie et les changements introduits dans l'organisation du travail, on parle à peu près indifféremment de rationalisation ou de taylorisation. Le mot de rationalisation a plus de prestige auprès du public parce qu'il semble indiquer que l'organisation actuelle du travail est celle qui satisfait toutes les exigences de la raison, une organisation rationnelle du travail devant nécessairement répondre à l'intérêt de l'ouvrier, du patron et du consommateur. Il semble vraiment que personne ne puisse s'élever là contre. Le pouvoir des mots est très grand, et on s'est beaucoup servi de celui-là ; de même que de l'expression « organisation scientifique du travail » parce que le mot « scientifique » a encore plus de prestige que le mot « rationnel ».\par
Quand on parle de taylorisation, on indique l'origine du système parce que c'est Taylor qui en a trouvé l'essentiel, qui a donné l'impulsion et marqué l'orientation de cette méthode de travail. De sorte que pour en connaître l'esprit, il faut nécessairement se référer à Taylor. C'est facile puisqu'il a écrit lui-même un certain nombre d'ouvrages sur ce sujet en faisant sa propre biographie.\par
L'histoire des recherches de Taylor est très curieuse et très instructive. Elle permet de voir de quelle manière s'est orienté ce système à son début. Elle permet même, mieux que toute autre chose, de comprendre ce qu'est, au fond, la rationalisation elle-même.\par
Quoique Taylor ait baptisé son système « Organisation scientifique du travail », ce n'était pas un savant. Sa culture correspondait peut-être au baccalauréat, et encore ce n'est pas sûr. Il n'avait jamais fait d'études d'ingénieur. Ce n'était pas non plus un ouvrier à proprement parler, quoiqu'il ait travaillé en usine. Comment donc le définir ? C'était un contremaître, mais non pas de l'espèce de ceux qui sont venus de la classe ouvrière et qui en ont gardé le souvenir. C'était un contremaître du genre de ceux dont on trouve des types actuellement dans les syndicats professionnels de maîtrise et qui se croient nés pour servir de chiens de garde au patronat. Ce n'est ni par curiosité d'esprit, ni par besoin de logique qu'il a entrepris ses recherches. C'est son expérience de contremaître chien de garde qui l'a orienté dans toutes ses études et qui lui a servi d'inspiratrice pendant trente-cinq années de recherches patientes. C'est ainsi qu'il a donné à l'industrie, outre son idée fondamentale d'une nouvelle organisation des usines, une étude admirable sur le travail des tours à dégrossir.\par
Taylor était né dans une famille relativement riche et aurait pu vivre sans travailler, n’étaient les principes puritains de sa famille et de lui-même, qui ne lui permettaient pas de rester oisif. Il fit ses études dans un lycée, mais une maladie des yeux les lui fit interrompre à 18 ans. Une singulière fantaisie le poussa alors à entrer dans une usine où il fit un apprentissage d'ouvrier mécanicien. Mais le contact quotidien avec la classe ouvrière ne lui donna à aucun degré l'esprit ouvrier. Au contraire, il semble qu'il y ait pris conscience d'une manière plus aiguë de l'opposition de classe qui existait entre ses compagnons de travail et lui-même, jeune bourgeois, qui ne travaillait pas pour vivre, qui ne vivait pas de son salaire, et qui, connu de la direction, était traité en conséquence.\par
Après son apprentissage, à l'âge de 22 ans, il s'embaucha comme tourneur dans une petite usine de mécanique, et dès le premier jour il entra tout de suite en conflit avec ses camarades d'atelier qui lui firent comprendre qu'on lui casserait la figure s'il ne se conformait pas à la cadence générale du travail ; car à cette époque régnait le système du travail aux pièces organisé de telle manière que, dès que la cadence augmentait, on diminuait les tarifs. Les ouvriers avaient compris qu'il ne fallait pas augmenter la cadence pour que les tarifs ne diminuent pas ; de sorte que chaque fois qu'il entrait un nouvel ouvrier, on le prévenait d'avoir à ralentir sa cadence sous peine d'avoir la vie intenable.\par
Au bout de deux mois, Taylor est arrivé à devenir contremaître. En racontant cette histoire, il explique que le patron avait confiance en lui parce qu'il appartenait à une famille bourgeoise. Il ne dit pas comment le patron l'avait distingué si rapidement, puisque ses camarades l'empêchaient de travailler plus vite qu'eux, et on peut se demander s'il n'avait pas gagné sa confiance en lui racontant ce qui s'était dit entre ouvriers.\par
Quand il est devenu contremaître, les ouvriers lui ont dit : « On est bien content de t'avoir comme contremaître, puisque tu nous connais et que tu sais que si tu essaies de diminuer les tarifs on te rendra la vie impossible. » À quoi Taylor répondit en substance : « Je suis maintenant de l'autre côté de la barricade, je ferai ce que je dois faire. » Et en fait, ce jeune contremaître fit preuve d'une aptitude exceptionnelle pour faire augmenter la cadence et renvoyer les plus indociles.\par
Cette aptitude particulière le fit monter encore en grade jusqu'à devenir directeur de l'usine. Il avait alors vingt-quatre ans.\par
Une fois directeur, il a continué à être obsédé par cette unique préoccupation de pousser toujours davantage la cadence des ouvriers. Évidemment, ceux-ci se défendaient, et il en résultait que ses conflits avec les ouvriers allaient en s'aggravant. Il ne pouvait exploiter les ouvriers à sa guise parce qu'ils connaissaient mieux que lui les meilleures méthodes de travail. Il s'aperçut alors qu'il était gêné par deux obstacles : d'un côté il ignorait quel temps était indispensable pour réaliser chaque opération d'usinage et quels procédés étaient susceptibles de donner les meilleurs temps ; d'un autre côté, l'organisation de l'usine ne lui donnait pas le moyen de combattre efficacement la résistance passive des ouvriers. Il demanda alors à l'administrateur de l'entreprise l'autorisation d'installer un petit laboratoire pour faire des expériences sur les méthodes d'usinage. Ce fut l'origine d'un travail qui dura vingt-six ans et amena Taylor à la découverte des aciers rapides, de l'arrosage de l'outil, de nouvelles formes d'outil à dégrossir, et surtout il a découvert, aidé d'une équipe d'ingénieurs, des formules mathématiques donnant les rapports les plus économiques entre la profondeur de la passe, l'avance et la vitesse des tours ; et pour l'application de ces formules dans les ateliers, il a établi des règles à calcul permettant de trouver ces rapports dans tous les cas particuliers qui pouvaient se présenter.\par
Ces découvertes étaient les plus importantes à ses yeux parce qu'elles avaient un retentissement immédiat sur l'organisation des usines. Elles étaient toutes inspirées par son désir d'augmenter la cadence des ouvriers et par sa mauvaise humeur devant leur résistance. Son grand souci était d'éviter toute perte de temps dans le travail. Cela montre tout de suite quel était l’esprit du système. Et pendant vingt-six ans il a travaillé avec cette unique préoccupation. Il a conçu et organisé progressivement le bureau des méthodes avec les fiches de fabrication, le bureau des temps pour l'établissement du temps qu'il fallait pour chaque opération, la division du travail entre les chefs techniques et un système particulier de travail aux pièces avec prime.\par
Cet aperçu permet de comprendre en quoi a consisté l’originalité de Taylor et quels sont les fondements de la rationalisation. Jusqu'à lui, on n'avait guère fait de recherches de laboratoire que pour découvrir des dispositifs mécaniques nouveaux, pour trouver de nouvelles machines, tandis que lui a eu l’idée d’étudier scientifiquement les meilleurs procédés pour utiliser les machines existantes. Il n’a pas fait, à proprement parler, de découvertes, sauf celle des aciers rapides. Il a cherché simplement les procédés les plus scientifiques pour utiliser au mieux les machines qui existaient déjà ; et non seulement les machines mais aussi les hommes. C'était son obsession. Il a fait son laboratoire pour pouvoir dire aux ouvriers : Vous avez eu tort de faire tel travail en une heure, il fallait le faire en une demi-heure. Son but était d'ôter aux travailleurs la possibilité de déterminer eux-mêmes les procédés et le rythme de leur travail, et de remettre entre les mains de la direction le choix des mouvements à exécuter au cours de la production. Tel était l'esprit de ses recherches. Il ne s'agissait pas pour Taylor de soumettre les méthodes de production à l'examen de la raison, ou du moins ce souci ne venait qu'en deuxième lieu ; son souci primordial était de trouver les moyens de forcer les ouvriers à donner à l'usine le maximum de leur capacité de travail. Le laboratoire était pour lui un moyen de recherche, mais avant tout un moyen de contrainte.\par
Cela résulte explicitement de ses propres ouvrages. La méthode de Taylor consiste essentiellement en ceci : d'abord, on étudie scientifiquement les meilleurs procédés à employer pour n'importe quel travail, même le travail de manœuvres (je ne parle pas de manœuvres spécialisés, mais de manœuvres proprement dits), même la manutention ou les travaux de ce genre ; ensuite, on étudie les temps par la décomposition de chaque travail en mouvements élémentaires qui se reproduisent dans des travaux très différents, d'après des combinaisons diverses ; et une fois mesuré le temps nécessaire à chaque mouvement élémentaire, on obtient facilement le temps nécessaire à des opérations très variées. Vous savez que la méthode de mesure des temps, c'est le chronométrage. Il est inutile d'insister là-dessus. Enfin, intervient la division du travail entre les chefs techniques. Avant Taylor, un contremaître faisait tout ; il s'occupait de tout. Actuellement, dans les usines, il y a plusieurs chefs pour un même atelier : il y a le contrôleur, il y a le contremaître, etc.\par
Le système particulier de travail aux pièces avec prime consistait à mesurer les temps par unité en se basant sur le maximum de travail que pouvait produire le meilleur ouvrier pendant une heure par exemple, et pour tous ceux qui produiront ce maximum, chaque pièce sera payée tel prix, tandis qu'elle sera payée à un prix plus bas pour ceux qui produiront moins ; ceux qui produiront nettement moins que ce maximum toucheront moins que le salaire vital. Autrement dit, il s'agit d'un procédé pour éliminer tous ceux qui ne sont pas des ouvriers de premier ordre capables d'atteindre ce maximum de production.\par
Somme toute, ce système contient l'essentiel de ce que l'on appelle aujourd'hui la rationalisation. Les contremaîtres égyptiens avaient des fouets pour pousser les ouvriers à produire ; Taylor a remplacé le fouet par les bureaux et les laboratoires, sous le couvert de la science.\par
L'idée de Taylor était que chaque homme est capable de produire un maximum de travail déterminé. Mais c'est tout à fait arbitraire, et inapplicable pour un grand nombre d'usines. Dans une seule usine, cela a pour résultat que les ouvriers costauds, les plus résistants, restent dans l'usine, tandis que les autres s'en vont ; il est impossible d'avoir suffisamment d'ouvriers costauds pour toutes les machines de toute une ville et d'arriver à une telle sélection sur une grande échelle. Supposez qu'il y ait un certain pourcentage de travaux nécessitant une grande force physique : il n'est pas prouvé qu'il y aura le même pourcentage d'hommes remplissant cette condition.\par
Les recherches de Taylor ont commencé en 1880. La mécanique commençait alors seulement à devenir une industrie. Pendant toute la première moitié du XIX\textsuperscript{e} siècle, la grande industrie avait presque été limitée au textile. C'est seulement vers 1850 qu'on s'était mis à construire des tours en bâti métallique. Quand Taylor était enfant, la plupart des mécaniciens étaient encore des artisans travaillant dans leurs propres ateliers. C'est au moment même où Taylor commençait ses travaux que naquit la Fédération américaine du travail, formée de quelques syndicats qui venaient de se constituer, et notamment le Syndicat des métallurgistes. Une des méthodes de l'action syndicale consistait, vers cette époque, à limiter la production pour empêcher le chômage et la réduction des tarifs aux pièces. Dans l'esprit de Taylor, comme dans celui des industriels auxquels il communiquait progressivement les résultats de ses études, le premier avantage de la nouvelle organisation du travail devait être de briser l'influence des syndicats. Dès son origine, la rationalisation a été essentiellement une méthode pour faire travailler plus, plutôt qu'une méthode pour travailler mieux.\par
Après Taylor, il n'y a pas eu beaucoup d'innovations sensationnelles dans le sens de la rationalisation.\par
\par
Il y a eu d'abord le travail à la chaîne, inventé par Ford, qui a supprimé dans une certaine mesure le travail aux pièces et à la prime, même dans ses usines. La chaîne, originellement, c'est simplement un procédé de manutention mécanique. Pratiquement, c'est devenu une méthode perfectionnée pour extraire des travailleurs le maximum de travail dans un temps déterminé.\par
Le système des montages à la chaîne a permis de remplacer des ouvriers qualifiés par des manœuvres spécialisés dans les travaux en série, où, au lieu d'accomplir un travail qualifié, il n'y a plus qu'à exécuter un certain nombre de gestes mécaniques qui se répètent constamment. C'est un perfectionnement du système de Taylor qui aboutit à ôter à l'ouvrier le choix de sa méthode et l'intelligence de son travail, et à renvoyer cela au bureau d'études. Ce système des montages fait aussi disparaître l'habileté manuelle nécessaire à l'ouvrier qualifié.\par
L'esprit d'un tel système apparaît suffisamment par la manière dont il a été élaboré, et on peut voir tout de suite que le mot de rationalisation lui a été appliqué à tort.\par
Taylor ne recherchait pas une méthode de rationaliser le travail, mais un moyen de contrôle vis-à-vis des ouvriers, et s'il a trouvé en même temps le moyen de simplifier le travail, ce sont deux choses tout à fait différentes. Pour illustrer la différence entre le travail rationnel et le moyen de contrôle, je vais prendre un exemple de véritable rationalisation, c'est-à-dire de progrès technique qui ne pèse pas sur les ouvriers et ne constitue pas une exploitation plus grande de leur force de travail.\par
Supposez un tourneur travaillant sur des tours automatiques. Il en a quatre à surveiller. Si un jour on découvre un acier rapide permettant de doubler la production de ces quatre tours et si on embauche un autre tourneur de sorte que chacun d'eux n'ait que deux tours, chacun a alors le même travail à faire et néanmoins la production est meilleur marché.\par
Il peut donc y avoir des améliorations techniques qui améliorent la production sans peser le moins du monde sur les travailleurs.\par
Mais la rationalisation de Ford consiste non pas à travailler mieux, mais à faire travailler plus. En somme, le patronat a fait cette découverte qu'il y a une meilleure manière d'exploiter la force ouvrière que d'allonger la journée de travail.\par
\par
En effet, il y a une limite à la journée de travail, non seulement parce que la journée proprement dite n'est que de vingt-quatre heures, sur lesquelles il faut prendre aussi le temps de manger et de dormir, mais aussi parce que, au bout d'un certain nombre d'heures de travail, la production ne progresse plus. Par exemple, un ouvrier ne produit pas plus en dix-sept heures qu'en quinze heures, parce que son organisme est plus fatigué et qu'automatiquement il va moins vite.\par
Il y a donc une limite de la production qu'on atteint assez facilement par l'augmentation de la journée de travail, tandis qu'on ne l'atteint pas en augmentant son intensité.\par
C'est une découverte sensationnelle du patronat. Les ouvriers ne l'ont peut-être pas encore bien comprise, les patrons n'en ont peut-être pas absolument conscience ; mais ils se conduisent comme s'ils la comprenaient très bien.\par
C'est une chose qui ne vient pas immédiatement à l'esprit, parce que l'intensité du travail n'est pas mesurable comme sa durée.\par
Au mois de juin, les paysans ont pensé que les ouvriers étaient des paresseux parce qu'ils ne voulaient travailler que quarante heures par semaine  ; parce qu'on a l'habitude de mesurer le travail par la quantité d'heures et que cela se chiffre, tandis que le reste ne se chiffre pas.\par
Mais l'intensité du travail peut varier. Pensez, par exemple, à la course à pied et rappelez-vous le coureur de Marathon tombé mort en arrivant au but pour avoir couru trop vite. On peut considérer cela comme une intensité-limite de l'effort. Il en est de même dans le travail. La mort, évidemment, c'est l'extrême limite à ne pas atteindre, mais tant qu'on n'est pas mort au bout d'une heure de travail, c'est, aux yeux des patrons, qu'on pouvait travailler encore plus. C'est ainsi également qu'on bat tous les jours de nouveaux records sans que personne ait l'idée que la limite soit encore atteinte. On attend toujours le coureur qui battra le dernier record. Mais si on inventait une méthode de travail qui fasse mourir les ouvriers au bout de cinq ans, par exemple, les patrons manqueraient très vite de main-d'œuvre et cela irait contre leurs intérêts. Ils ne s'en apercevraient pas tout de suite, parce qu'il n'existe aucun moyen scientifique de mesurer l'usure de l'organisme humain par le travail ; mais peut-être qu'à la génération suivante, ils s'en apercevraient et réviseraient leurs méthodes, exactement comme on s'est rendu compte des milliers de morts prématurées provoquées par le travail des enfants dans les usines.\par
Il peut arriver la même chose pour les adultes avec l'intensité du travail. Il y a seulement un an, dans les usines de mécanique de la région parisienne, un homme de quarante ans ne pouvait plus trouver d'embauche, parce qu'on le considérait comme déjà usé, vidé, et impropre pour la production à la cadence actuelle.\par
Il n'y a donc aucune limite à l'augmentation de la production en intensité. Taylor raconte avec orgueil qu'il est arrivé à doubler et même tripler la production dans certaines usines simplement par le système des primes, la surveillance des ouvriers et le renvoi impitoyable de ceux qui ne voulaient pas ou ne pouvaient pas suivre la cadence. Il explique qu'il est parvenu à trouver le moyen idéal pour supprimer la lutte des classes, parce que son système repose sur un intérêt commun de l'ouvrier et du patron, tous les deux gagnant davantage avec ce système, et le consommateur lui-même se trouvant satisfait parce que les produits sont meilleur marché. Il se vantait de résoudre ainsi tous les conflits sociaux et d'avoir créé l'harmonie sociale.\par
Mais prenons l'exemple d'une usine dont Taylor ait doublé la production sans changer les méthodes de fabrication, simplement en organisant cette police des ateliers. Imaginons, d'autre part, une usine où l'on travaillerait sept heures par jour pour trente francs, et où le patron déciderait un beau jour de faire travailler quatorze heures par jour pour quarante francs. Les ouvriers ne considéreraient pas qu'ils y gagnent, et certainement se mettraient immédiatement en grève. Pourtant, cela revient exactement au système Taylor. En travaillant quatorze heures par jour au lieu de sept, on se fatiguerait au moins deux fois plus. Je suis même convaincue qu'à partir d'une certaine limite, il est beaucoup plus grave pour l'organisme humain d'augmenter la cadence comme Taylor que d'augmenter la durée du travail.\par
Quand Taylor a instauré son système, il y a eu certaines réactions de la part des ouvriers. En France, les syndicats ont vivement réagi au début de l'introduction de ce système dans les usines françaises. Il y a eu des articles de Pouget, de Merrheim, comparant la rationalisation à un nouvel esclavage. En Amérique, il y a eu des grèves. Finalement, ce système a tout de même triomphé et a été pour beaucoup dans le développement des industries de guerre ; ce qui a fait penser que la guerre était pour beaucoup dans ce triomphe de la rationalisation.\par
\par
Le grand argument de Taylor, c'est que ce système sert les intérêts du public, c'est-à-dire des consommateurs. Évidemment, l'augmentation de la production peut leur être favorable quand il s'agit de denrées alimentaires, du pain, du lait, de la viande, du beurre, du vin, de l'huile, etc. Mais ce n'est pas cette production qui augmente avec le système Taylor ; d'une manière générale, ce n'est pas ce qui sert à satisfaire les principaux besoins de l'existence. Ce qui a été rationalisé, c'est la mécanique, le caoutchouc, le textile, c'est-à-dire essentiellement ce qui produit le moins d'objets consommables. La rationalisation a surtout servi à la fabrication des objets de luxe et à cette industrie doublement de luxe qu'est l'industrie de guerre, qui non seulement ne bâtit pas, mais détruit. Elle a servi à accroître considérablement le poids des travailleurs inutiles, de ceux qui fabriquent des choses inutiles ou de ceux qui ne fabriquent rien et qui sont employés dans les services de publicité et autres entreprises de ce genre, plus ou moins parasitaires. Elle a accru formidablement le poids des industries de guerre, qui, à elles seules, dépassent toutes les autres par leur importance et leurs inconvénients. La taylorisation a servi essentiellement à augmenter tout ce poids et à faire peser, somme toute, l'augmentation de la production globale sur un nombre toujours plus réduit de travailleurs.\par
Du point de vue de l'effet moral sur les ouvriers, la taylorisation a sans aucun doute provoqué la disqualification des ouvriers. Ceci a été contesté par les apologistes de la rationalisation, notamment par Dubreuilh dans {\itshape Standards.} Mais Taylor a été le premier à s'en vanter, en arrivant à ne faire entrer que 75 \% d'ouvriers qualifiés dans la production, contre 125 \% d'ouvriers non qualifiés pour le finissage. Chez Ford, il n'y a que 1 \% d'ouvriers qui aient besoin d'un apprentissage de plus d'un jour.\par
Ce système a aussi réduit les ouvriers à l'état de molécules, pour ainsi dire, en en faisant une espèce de structure atomique dans les usines. Il a amené l'isolement des travailleurs. C'est une des formules essentielles de Taylor qu'il faut s'adresser à l'ouvrier individuellement ; considérer en lui l'individu. Ce et de la concurrence. C'est cela qui produit cette solitude qui est peut-être le caractère le plus frappant des usines organisées selon le système actuel, solitude morale qui a été certainement diminuée par les événements de juin. Ford dit ingénument qu'il est excellent d'avoir des ouvriers qui s'entendent bien, mais qu'il ne faut pas qu'ils s'entendent trop bien, parce que cela diminue l'esprit de concurrence et d'émulation indispensable à la production.\par
\par
La division de la classe ouvrière est donc à la base de cette méthode. Le développement de la concurrence entre les ouvriers en fait partie intégrante ; comme l'appel aux sentiments les plus bas. Le salaire en est l'unique mobile. Quand le salaire ne suffit pas, c'est le renvoi brutal. À chaque instant du travail, le salaire est déterminé par une prime. À tout instant, il faut que l'ouvrier calcule pour savoir ce qu'il a gagné. Ce que je dis est d'autant plus vrai qu'il s'agit de travail moins qualifié.\par
Ce système a produit la monotonie du travail. Dubreuilh et Ford disent que le travail monotone n'est pas pénible pour la classe ouvrière. Ford dit bien qu'il ne pourrait pas passer une journée entière à un seul travail de l'usine, mais qu'il faut croire que ses ouvriers sont autrement faits que lui, parce qu'ils refusent un travail plus varié. C'est lui qui le dit. Si vraiment il arrive que par un tel système la monotonie soit supportable pour les ouvriers, c'est peut-être ce que l'on peut dire de pire d'un tel système ; car il est certain que la monotonie du travail commence toujours par être une souffrance. Si on arrive à s’y accoutumer, c'est au prix d'une diminution morale.\par
En fait, on ne s'y accoutume pas, sauf si l'on peut travailler en pensant à autre chose. Mais alors il faut travailler à un rythme ne réclamant pas trop d'assiduité dans l'attention nécessitée par la cadence du travail. Mais si on fait un travail auquel ou doive penser tout le temps, on ne peut pas penser à autre chose, et il est faux de dire que l'ouvrier puisse s'accommoder de la monotonie de ce travail. Les ouvriers de Ford n'avaient pas le droit de parler. Ils ne cherchaient pas à avoir un travail varié parce que, au bout d'un certain temps de travail monotone, ils sont incapables de faire autre chose.\par
La discipline dans les usines, la contrainte, est une autre caractéristique du système. C'est même son caractère essentiel ; et c'est le but pour lequel il a été inventé, puisque Taylor a fait ses recherches exclusivement pour briser la résistance de ses ouvriers. En leur imposant tels ou tels mouvements en tant de secondes, ou tels autres en tant de minutes, il est évident qu'il ne reste à l'ouvrier aucun pouvoir de résistance. C'est de cela que Taylor était le plus fier, et c'est cela qu'il développait le plus volontiers en ajoutant que son système permettait de briser la puissance des syndicats dans les usines.\par
Au cours d'une enquête faite en Amérique sur le système Taylor, un ouvrier interrogé par Henri de Man lui a dit : « Les patrons ne comprennent pas que nous ne voulions pas nous laisser chronométrer ; pourtant, que diraient nos patrons si nous leur demandions de nous montrer leurs livres de comptabilité et si nous leur disions : Sur tant de bénéfices que vous faites, nous jugeons que telle part doit vous rester et telle autre part nous revenir sous forme de salaires ? La connaissance des temps de travail est pour nous exactement l'équivalent de ce qu'est pour eux le secret industriel et commercial. »\par
Cet ouvrier avait admirablement compris la situation. Le patron a non seulement la propriété de l'usine, des machines, le monopole des procédés de fabrication et des connaissances financières et commerciales concernant son usine, il prétend encore au monopole du travail et des temps de travail. Que reste-t-il aux ouvriers ? Il leur reste l'énergie qui permet de faire un mouvement, l'équivalent de la force électrique ; et on l'utilise exactement comme on utilise l'électricité.\par
Par les moyens les plus grossiers, en employant comme stimulant à la fois la contrainte et l'appât du gain, en somme par une méthode de dressage qui ne fait appel à rien de ce qui est proprement humain, on dresse l'ouvrier comme on dresse un chien, en combinant le fouet et les morceaux de sucre. Heureusement qu'on n'en arrive pas là tout à fait, parce que la rationalisation n'est jamais parfaite et que, grâce au ciel, le chef d'atelier ne connaît jamais tout. Il reste des moyens de se débrouiller, même pour un ouvrier non qualifié. Mais si le système était strictement appliqué, ce serait exactement cela.\par
Il a encore un certain nombre d'avantages pour la direction et d'inconvénients pour les ouvriers. Tandis que la direction a le monopole de toutes les connaissances concernant le travail, elle n'a pas la responsabilité des coups durs à cause du travail aux pièces et à la prime. Avant juin, on était arrivé à ce miracle que tout ce qui était bien était porté au bénéfice des patrons, mais tous les coups durs étaient à la charge des ouvriers qui perdaient leur salaire si une machine était déréglée, qui devaient se débrouiller si quelque chose ne collait pas, si un ordre était inapplicable ou si deux ordres étaient contradictoires (car théoriquement ça colle toujours ; l'acier des outils est toujours bon, et si l'outil se casse, c'est toujours la faute de l'ouvrier), etc. Et comme le travail est aux pièces, les chefs font encore une faveur quand ils veulent bien aider à réparer des coups durs. De sorte que véritablement ce système est idéal pour les patrons, puisqu'il comporte tous les avantages pour eux, tandis qu'il réduit les ouvriers à l'état d'esclaves et leur impose tout de même des initiatives toutes les fois que ça ne colle pas. C'est un raffinement d'où résulte de la souffrance dans les deux cas, parce que dans tous les cas c'est l'ouvrier qui a tort.\par
On ne peut appeler scientifique un tel système qu'en partant du principe que les hommes ne sont pas des hommes, et en faisant jouer à la science le rôle rabaissé d'instrument de contrainte. Mais le rôle véritable de la science en matière d'organisation du travail est de trouver de meilleures techniques. En règle générale, le fait qu'il est si facile d'exploiter toujours plus la force ouvrière crée une sorte de paresse chez les chefs, et on a vu dans beaucoup d'usines une négligence incroyable de leur part vis-à-vis des problèmes techniques et des problèmes d'organisation, parce qu'ils savaient qu'ils pouvaient toujours faire réparer leurs fautes par les ouvriers en augmentant un peu plus la cadence.\par
Taylor a toujours soutenu que le système était admirable parce qu'on pouvait trouver scientifiquement non seulement les meilleurs procédés de travail et les temps nécessaires pour chaque opération, mais encore la limite de la fatigue au-delà de laquelle il ne fallait pas faire aller un travailleur.\par
Depuis Taylor, une branche spéciale de la science s'est développée en ce sens : c'est ce qu'on appelle la psychotechnique, qui permet de définir les meilleures conditions psychologiques possibles pour tel ou tel travail, de mesurer la fatigue, etc.\par
Alors les industriels, grâce à la psychotechnique, peuvent dire qu'ils ont la preuve qu'ils ne font pas souffrir leurs ouvriers. Il leur suffit d'invoquer l'autorité de savants.\par
Mais la psychotechnique est encore imparfaite. Elle vient d'être créée. Et même serait-elle parfaite, elle n'atteindrait jamais les facteurs moraux ; car la souffrance à l'usine consiste surtout à trouver le temps long ; mais elle ne s'arrête pas là. Et jamais d'ailleurs aucun psychotechnicien n'arrivera à préciser dans quelle mesure un ouvrier trouve le temps long. C'est l'ouvrier lui-même qui peut le dire.\par
Ce qui est encore plus grave, c'est ça : il faut se méfier des savants, parce que la plupart du temps ils ne sont pas sincères. Rien n'est plus facile pour un industriel que d'acheter un savant, et lorsque le patron est l'État rien n'est plus facile pour lui que d'imposer telle ou telle règle scientifique. On le voit en ce moment en Allemagne où l'on découvre subitement que les graisses ne sont pas aussi nécessaires qu'on le pensait à l'alimentation humaine. On pourrait de même découvrir qu'il est plus facile à un ouvrier de faire deux mille pièces que mille. Les travailleurs ne doivent donc pas avoir confiance dans les savants, les intellectuels ou les techniciens pour régler ce qui est pour eux d'une importance vitale. Ils peuvent bien entendu prendre leurs conseils, mais ils ne doivent compter que sur eux-mêmes, et s'ils s'aident de la science ça devra être en l'assimilant eux-mêmes\par

\begin{center}
\_\_\_\_\_\_\_\end{center}
\noindent Ici s'arrête le texte qui a pu être recueilli.
\section[La condition ouvrière, (30 septembre 1937)]{La condition ouvrière \\
(30 septembre 1937)}\renewcommand{\leftmark}{La condition ouvrière \\
(30 septembre 1937)}

\noindent \par
Les études précédemment parues concernant la condition ouvrière dans divers pays indiquent assez, quand on les compare, quelle distance sépare des hommes qui portent tous le même nom d'ouvriers. Encore péchaient-elles gravement par abstraction ; car d'une profession à une autre, d'une ville à une autre, et même d'un coin à l'autre d'une même usine, que de différences ! À plus forte raison d'un pays à un autre. Tous les ouvriers travaillent soumis à des ordres, assujettis à un salaire ; pourtant y a-t-il plus que le nom de commun entre un ouvrier japonais ou indochinois et un ouvrier suédois ou un ouvrier français d'après juin 1936 ? Je dis d'après juin 1936, car au cours des sombres années qui ont précédé, la condition matérielle et morale des ouvriers français tendait cruellement à se rapprocher des pires formes du salariat.\par
L'examen de ces différences suggère qu'elles pourraient sans doute aller plus loin encore. Des hommes pourraient aller plus loin dans la misère et l'esclavage, plus loin dans le bien-être et l'indépendance que ne vont les plus malheureux et les moins malheureux des devrait de tous côtés faire plus attention. Les uns, qui méprisent les réformes comme une forme d'action lâche et peu efficace, réfléchiraient qu'il vaut mieux changer les choses que les mots, et que les grands bouleversements changent surtout les mots. Les autres, qui haïssent les réformes comme utopiques et dangereuses, s'apercevraient qu'ils croient à des fatalités illusoires, et que les larmes, l'épuisement, le désespoir ne sont peut-être pas aussi indispensables à l'ordre social qu'ils se l'imaginent.\par
Il est vrai pourtant qu'il y a, dans les formes les plus élevées de la condition ouvrière, quelque chose de singulièrement instable ; elles comportent peu de sécurité. Autour d'elles les flots de la misère générale agissent comme une mer qui ronge des îlots. Les pays où les travailleurs sont misérables exercent par leur seule existence une pression perpétuelle sur les pays de progrès social pour y atténuer les progrès ; et sans doute la pression inverse s'exerce aussi, mais apparemment beaucoup plus faible, car la première pression a pour mécanisme le jeu des échanges économiques, et la seconde la contagion sociale. Au reste quand le progrès social a pris la forme d'un bouleversement révolutionnaire, il en est encore exactement de même ; ou plutôt le peuple d'un État révolutionnaire semble être à l'égard de ce phénomène encore plus vulnérable et plus désarmé que tout autre. Il y a là un obstacle considérable à l'amélioration du sort des travailleurs, Beaucoup, trompés par des espérances enivrantes, ont le tort de l'oublier. D'autres, mus par des espérances moins généreuses, ont le tort de confondre cet obstacle avec ceux qui tiennent à la nature des choses.\par
Cette dernière erreur est entretenue par une certaine confusion de langage. On parle sans cesse, actuellement, de la production. Pour consommer, il faut d'abord produire, et pour produire il faut travailler. Voilà ce que, depuis juin 1936, on entend répéter partout, du {\itshape Temps} jusqu'aux organes de la C. G. T., et ce qu'on n'entend, bien entendu, contester nulle part, sinon par ceux que font rêver les formes modernes du mythe du mouvement perpétuel. C'est là, en effet, un obstacle au développement général du bien-être et des loisirs et qui tient à la nature des choses. Mais par lui-même il n'est pas aussi grand qu'on l'imagine d'ordinaire. Car seul est nécessaire à produire ce qu'il est nécessaire de consommer ; ajoutons-y encore l'utile et l'agréable, à condition qu'il s'agisse de véritable utilité et de plaisirs purs. À vrai dire, la justice ne trouve pas son compte dans le spectacle de milliers d'hommes peinant pour procurer à quelques privilégiés des jouissances délicates ; mais que dire des travaux qui accablent une multitude de malheureux sans même procurer aux privilégiés grands et petits de vraie satisfaction ? Et combien ces travaux ne tiennent-ils pas de place dans notre production totale, si l'on osait faire le compte ?\par
Pourtant de tels travaux sont, eux aussi, nécessaires, d'une nécessité qui tient non à la nature des choses, mais aux rapports humains ; inutiles à tous, ils sont nécessaires en chaque endroit du fait qu'on s'y livre partout ailleurs. La discrimination entre ces deux espèces de nécessités, la véritable et la fausse, n'est pas toujours aisée ; mais il existe pour elle un critérium sûr. Il est des produits dont la disette dans un pays est d'autant plus grave qu'elle s'étend aussi au reste du globe ; pour d'autres, la disette présente d'autant moins d'inconvénients qu'elle est plus générale. On peut ainsi distinguer grossièrement deux classes de travaux.\par
Si la récolte du blé diminuait en France de moitié, par suite de quelque fléau, les Français devraient mettre tout leur espoir dans une surabondance de blé au Canada ou ailleurs ; leur détresse deviendrait irrémédiable si la récolte avait en même temps diminué de moitié dans le monde entier. Au contraire, que le rendement des usines de guerre françaises diminue un beau jour de moitié, il n'en résultera pour la France aucun dommage, pourvu que pareille diminution ait lieu dans toutes les usines de guerre du monde. Le blé d'une part, la production de guerre de l'autre, constituent des exemples parfaits pour l'opposition qu'il s'agit d'illustrer. Mais la plupart des produits participent, à des degrés différents, de l'une et de l'autre catégorie. Ils servent pour une part à être consommés, et pour une part, soit à la guerre, soit à cette lutte analogue à la guerre qu'on appelle concurrence. Si l'on pouvait tracer un schéma figurant la production actuelle et illustrant cette division, on mesurerait exactement, au jour le jour, combien de sueur et de larmes les hommes ajoutent à la malédiction originelle.\par
Prenons l'exemple de l'automobile. Dans l'état actuel des échanges, l'automobile est un instrument de transport qui ne pourrait être supprimé sans graves désordres ; mais la quantité d'automobiles qui sort tous les jours des usines dépasse de beaucoup celle au-dessous de laquelle ces désordres se produiraient. Pourtant, une diminution considérable du rendement du travail dans ces usines aurait des effets désastreux, car les automobiles anglaises, italiennes, américaines, plus abondantes et moins chères, envahiraient le marché et provoqueraient faillite et chômage. C'est qu'une automobile ne sert pas seulement à rouler sur une route, elle est aussi une arme dans la guerre permanente que mènent entre elles la production française et celle des autres pays. Les barrières douanières, on le sait trop, sont des moyens de défense peu efficaces et dangereux.\par
Imaginons à présent la semaine de trente heures établie dans toutes les usines d'automobiles du monde, ainsi qu'une cadence du travail moins rapide. Quelles catastrophes en résultera-t-il ? Pas un enfant n'aura moins de lait, pas une famille n'aura plus froid, et même, vraisemblablement, pas un patron d'usine d'automobiles n'aura une vie moins large. Les villes deviendront moins bruyantes, les routes retrouveront quelquefois le bienfait du silence. À vrai dire, dans de telles conditions, beaucoup de gens seraient privés du plaisir de voir défiler les paysages à une cadence de cent kilomètres à l'heure ; en revanche, des milliers, des milliers et des milliers d'ouvriers pourraient enfin respirer, jouir du soleil, se mouvoir au rythme de la respiration, faire d'autres gestes que ceux imposés par des ordres ; tous ces hommes, qui mourront, connaîtraient de la vie, avant de mourir, autre chose que la hâte vertigineuse et monotone des heures de travail, l'accablement des repos trop brefs, la misère insondable des jours de chômage et des années de vieillesse. Il est vrai que les statisticiens, en comptant les autos, trouveraient qu'on a reculé dans la voie du progrès.\par
La rivalité militaire et économique est aujourd'hui et restera vraisemblablement un fait qu'on ne peut éliminer que dans la composition d'idylles ; il n'est pas question de supprimer la concurrence dans ce pays, à plus forte raison dans le monde. Ce qui apparaît comme éminemment souhaitable, ce serait d'ajouter au jeu de la concurrence quelques règles. La résistance de la tôle au découpage ou à l'emboutissage est à peu près la même dans toutes les usines de mécanique du monde ; si on pouvait en dire autant de la résistance ouvrière à l'oppression, aucun des effets heureux de la concurrence ne disparaîtrait, et que de difficultés évanouies !\par
Dans le mouvement ouvrier, cette nécessité d'étendre au monde entier les conquêtes ouvrières de chaque pays socialement avancé est passée depuis longtemps au rang de lieu commun. Après la guerre, la lutte de tendances roulait essentiellement sur la question de savoir s'il fallait chercher à assurer cette extension au moyen de la révolution mondiale ou au moyen du Bureau International du Travail. On ne sait pas ce qu'aurait donné la révolution mondiale, mais le B. I. T., il faut le reconnaître, n'a pas réussi brillamment.\par
À première vue, on pourrait supposer que lorsqu'un pays a réalisé des progrès sociaux qui le compromettent dans la lutte économique, toutes les classes sociales de ce pays doivent, ne serait-ce que par intérêt, unir leurs efforts pour donner aux réformes accomplies la plus grande extension possible en dehors des frontières. Il n'en est pourtant pas ainsi. Les feuilles les plus respectables de chez nous, généralement considérées comme les porte-parole de la haute bourgeoisie, répètent à satiété que la réforme des quarante heures sera admirable si elle devient internationale, ruineuse si elle reste seulement française ; cela n'a pas empêché, sauf erreur, certains de nos représentants patronaux à Genève de voter contre les quarante heures.\par
Pareilles choses n'auraient pas lieu si les hommes n'étaient menés que par l'intérêt ; mais à côté de l'intérêt, il y a l'orgueil. Il est doux d'avoir des inférieurs ; il est pénible de voir des inférieurs acquérir des droits, même limités, qui établissent entre eux et leurs supérieurs, à certains égards, une certaine égalité. On aimerait mieux leur accorder les mêmes avantages, mais à titre de faveur ; on aimerait mieux, surtout, parler de les accorder. S'ils ont enfin acquis des droits, on préfère que la pression économique de l'étranger vienne les miner, non sans dégâts de toutes sortes, plutôt que d'en obtenir l'extension hors des frontières. Le souci le plus pressant de beaucoup d'hommes situés plus ou moins haut sur l'échelle sociale est de maintenir leurs inférieurs « à leur place ». Non sans raison après tout ; car s'ils quittent une fois « leur place », qui sait jusqu'où ils iront ?\par
L'internationalisme ouvrier devrait être plus efficace ; malheureusement on ne se tromperait pas de beaucoup en le comparant à la jument de Roland, qui avait toutes les qualités sauf celle d'exister. Même l'Internationale socialiste d'avant guerre était surtout une façade, et la guerre l'a bien montré. À plus forte raison n'y a-t-il jamais eu, dans l'Internationale syndicale, si cruellement mutilée aujourd'hui du fait des États dictatoriaux, ni action concertée, ni même contact permanent entre les différents mouvements nationaux. Sans doute, dans les grands moments, l'enthousiasme déborde les frontières ; on a pu le constater en ce mois épique de juin 1936, et on a vu l'occupation des usines non seulement s'essayer en Belgique, mais encore enjamber l'océan et trouver aux États-Unis une extension inattendue. Sans doute aussi on a vu parfois une grande lutte ouvrière partiellement alimentée par des souscriptions venues de l'étranger. Néanmoins il n'y a pas de stratégie concertée, les états-majors n'unissent pas leurs armes et ne mettent pas d'unité dans leurs revendications ; on constate souvent même une ignorance surprenante à l'égard de ce qui se passe hors du territoire national. L'internationalisme ouvrier est jusqu'ici plus verbal que pratique.\par
Quant au gouvernement, son action serait décisive en cette matière, s'il agissait. Car un certain nivellement dans les conditions d'existence des ouvriers des différents pays – nivellement vers le haut, si l'on peut ainsi parler – ne peut guère être conçu que comme un élément dans ce fameux règlement général des problèmes économiques mondiaux que chacun reconnaît comme indispensable à la paix et à la prospérité, mais qu'on n'aborde jamais. Réciproquement, l'action ouvrière sera, par un triste paradoxe, et malgré les doctrines internationales, un obstacle à la détente des rapports internationaux aussi longtemps qu'on se laissera vivre dans la déplorable incurie actuelle.\par
C'est ainsi que les ouvriers français redouteront toujours de voir pénétrer en France les travailleurs des pays surpeuplés aussi longtemps que les étrangers y seront légalement abaissés à une situation de parias, privés de toute espèce de droits, impuissants à participer à la moindre action syndicale sans risquer la mort lente par la misère, expulsables à merci. Le progrès social dans un pays a comme conséquence paradoxale la tendance à fermer les frontières aux produits et aux hommes. Si les pays de dictature se replient sur eux-mêmes par obsession guerrière, et si les pays les plus démocratiques les imitent, non seulement parce qu'ils sont contaminés par cette obsession, mais aussi du fait même des progrès accomplis par eux, que pouvons-nous espérer ?\par
Toutes les considérations d'ordre national et international, économique et politique, technique et humanitaire, se joignent pour conseiller de chercher à agir. D'autant que les réformes accomplies en juin 1936, et qui, s'il faut en croire certains, mettent notre économie en péril, ne sont qu'une petite partie des réformes immédiatement souhaitables. Car la France n'est pas seulement une nation ; elle est un Empire ; et une multitude de misérables, nés par malheur pour eux avec une peau d'une couleur différente de la nôtre, avaient mis de telles espérances dans le gouvernement de mai 1936 qu'une si longue attente, si elle reste déçue, risque de nous amener un de ces jours des difficultés graves et sanglantes.
\section[Expérience de la vie d'usine  (Marseille, 1941-1942)]{Expérience de la vie d'usine \protect\footnotemark  \\
(Marseille, 1941-1942)}\renewcommand{\leftmark}{Expérience de la vie d'usine  \\
(Marseille, 1941-1942)}

\footnotetext{ Article écrit à Marseille en 1941, publié postérieurement en partie sous le pseudonyme d'Émile Novis dans {\itshape Économie et Humanisme.}}
\noindent \par
Les lignes qui suivent se rapportent à une expérience de la vie d'usine qui date d'avant 1936. Elles peuvent surprendre beaucoup de gens qui n'ont été en contact direct avec des ouvriers que par l'effet du Front Populaire. La condition ouvrière change continuellement ; elle est parfois autre d'une année à la suivante. Les années qui ont précédé 1936, très dures et très brutales en raison de la crise économique, reflètent mieux pourtant la condition prolétarienne que la période semblable à un rêve qui a suivi.\par
Des déclarations officielles ont fait savoir que désormais l'État français chercherait à mettre fin à la condition prolétarienne, c'est-à-dire à ce qu'il y a de dégradant dans la vie faite aux ouvriers, soit dans l'usine, soit hors de l'usine. La première difficulté à vaincre est l'ignorance. Au cours des dernières années on a bien senti qu'en fait les ouvriers d'usine sont en quelque sorte déracinés, exilés sur la terre de leur propre pays.\par
Mais on ne sait pas pourquoi. Se promener dans les faubourgs, apercevoir les chambres tristes et sombres, les maisons, les rues, n'aide pas beaucoup à comprendre quelle vie on y mène. Le malheur de l'ouvrier à l'usine est encore plus mystérieux. Les ouvriers eux-mêmes peuvent très difficilement écrire, parler ou même réfléchir à ce sujet, car le premier effet du malheur est que la pensée veut s'évader ; elle ne veut pas considérer le malheur qui la blesse. Aussi les ouvriers, quand ils parlent de leur propre sort, répètent-ils le plus souvent des mots de propagande faits par des gens qui ne sont pas ouvriers. La difficulté est au moins aussi grande pour un ancien ouvrier ; il lui est facile de parler de sa condition première, mais très difficile d'y penser réellement, car rien n'est plus vite recouvert par l'oubli que le malheur passé. Un homme de talent peut, grâce à des récits et par l'exercice de l'imagination, deviner et décrire dans une certaine mesure du dehors ; ainsi Jules Romains a consacré à la vie d'usine un chapitre des {\itshape Hommes de bonne volonté}. Mais cela ne va pas très loin.\par
Comment abolir un mal sans avoir aperçu clairement en quoi il consiste ? Les lignes qui suivent peuvent peut-être quelque peu aider à poser au moins le problème, du fait qu'elles sont le fruit d'un contact direct avec la vie d'usine.\par
L'usine pourrait combler l'âme par le puissant sentiment de vie collective – on pourrait dire unanime – que donne la participation au travail d'une grande usine. Tous les bruits ont un sens, tous sont rythmés, ils se fondent dans une espèce de grande respiration du travail en commun à laquelle il est enivrant d'avoir part. C'est d'autant plus enivrant que le sentiment de solitude n'en est pas altéré. Il n'y a que des bruits métalliques, des roues qui tournent, des morsures sur le métal ; des bruits qui ne parlent pas de nature ni de vie, mais de l'activité sérieuse, soutenue, ininterrompue de l'homme sur les choses. On est perdu dans cette grande rumeur, mais en même temps on la domine, parce que sur cette basse soutenue, permanente et toujours changeante, ce qui ressort, tout en s'y fondant, c'est le bruit de la machine qu'on manie soi-même. On ne se sent pas petit comme dans une foule, on se sent indispensable. Les courroies de transmission, là où il y en a, permettent de boire par les yeux cette unité de rythme que tout le corps ressent par les bruits et par la légère vibration de toutes choses. Aux heures sombres des matinées et des soirées d'hiver, quand ne brille que la lumière électrique, tous les sens participent à un univers où rien ne rappelle la nature, où rien n'est gratuit, où tout est heurt, heurt dur et en même temps conquérant, de l'homme avec la matière. Les lampes, les courroies, les bruits, la dure et froide ferraille, tout concourt à la transmutation de l'homme en ouvrier.\par
Si c'était cela, la vie d'usine, ce serait trop beau. Mais ce n'est pas cela. Ces joies sont des joies d'hommes libres ; ceux qui peuplent les usines ne les sentent pas, sinon en de courts et rares instants, parce qu'ils ne sont pas des hommes libres. Ils ne peuvent les sentir que lorsqu'ils oublient qu'ils ne sont pas libres ; mais ils peuvent rarement l'oublier, car l'étau de la subordination leur est rendu sensible à travers les sens, le corps, les mille petits détails qui remplissent les minutes dont est constituée une vie.\par
Le premier détail qui, dans la journée, rend la servitude sensible, c'est la pendule de pointage. Le chemin de chez soi à l'usine est dominé par le fait qu'il faut être arrivé avant une seconde mécaniquement déterminée. On a beau être de cinq ou dix minutes en avance ; l'écoulement du temps apparaît de ce fait comme quelque chose d'impitoyable, qui ne laisse aucun jeu au hasard. C'est, dans une journée d'ouvrier, la première atteinte d'une règle dont la brutalité domine toute la partie de la vie passée parmi les machines ; le hasard n'a pas droit de cité à l'usine. Il y existe, bien entendu, comme partout ailleurs, mais il n'y est pas reconnu. Ce qui est admis, souvent au grand détriment de la production, c'est le principe de la caserne : « Je ne veux pas le savoir. » Les fictions sont très puissantes à l'usine. Il y a des règles qui ne sont jamais observées, mais qui sont perpétuellement en vigueur. Les ordres contradictoires ne le sont pas selon la logique de l'usine. À travers tout cela il faut que le travail se fasse. À l'ouvrier de se débrouiller, sous peine de renvoi. Et il se débrouille.\par
Les grandes et petites misères continuellement imposées dans l'usine à l'organisme humain ou, comme dit Jules Romains, « cet assortiment de menues détresses physiques que la besogne n'exige pas et dont elle est loin de bénéficier », ne contribuent pas moins à rendre la servitude sensible. Non pas les souffrances liées aux nécessités du travail ; celles-là, on peut être fier de les supporter ; mais celles qui sont inutiles. Elles blessent l'âme parce que généralement on ne songe pas à aller s'en plaindre ; et on sait qu'on n'y songe pas. On est certain d'avance qu'on serait rabroué et qu'on encaisserait sans mot dire. Parler serait chercher une humiliation. Souvent, s'il y a quelque chose qu'un ouvrier ne puisse pas supporter, il aimera mieux se taire et demander son compte. De telles souffrances sont souvent par elles-mêmes très légères ; si elles sont amères, c'est que toutes les fois qu'on les ressent, et on les ressent sans cesse, le fait qu'on voudrait tant oublier, le fait qu'on n'est pas chez soi à l'usine, qu'on n'y a pas droit de cité, qu'on y est un étranger admis comme simple intermédiaire entre les machines et les pièces usinées, ce fait vient atteindre le corps et l'âme ; sous cette atteinte, la chair et la pensée se rétractent. Comme si quelqu'un répétait à l'oreille de minute en minute, sans qu'on puisse rien répondre : « Tu n'es rien ici. Tu ne comptes pas. Tu es là pour plier, tout subir et te taire. » Une telle répétition est presque irrésistible. On en arrive à admettre, au plus profond de soi, qu'on compte pour rien. Tous les ouvriers d'usine ou presque, et même les plus indépendants d'allure, ont quelque chose de presque imperceptible dans les mouvements, dans le regard, et surtout au pli des lèvres, qui exprime qu'on les a contraints de se compter pour rien.\par
Ce qui les y contraint surtout, c'est la manière dont ils subissent les ordres. On nie souvent que les ouvriers souffrent de la monotonie du travail, parce qu'on a remarqué que souvent un changement de fabrication est pour eux une contrariété. Pourtant le dégoût envahit l'âme, au cours d'une longue période de travail monotone. Le changement produit du soulagement et de la contrariété à la fois ; contrariété vive parfois dans le cas du travail aux pièces, à cause de la diminution de gain, et parce que c'est une habitude et presque une convention d'attacher insaisissables, inexprimables qui s'emparent de l'âme pendant le travail. Mais même si le travail est payé à l'heure, il y a contrariété, irritation, à cause de la manière dont le changement est ordonné. Le travail nouveau est imposé tout d'un coup, sans préparation, sous la forme d'un ordre auquel il faut obéir immédiatement et sans réplique. Celui qui obéit ainsi ressent alors brutalement que son temps est sans cesse à la disposition d'autrui. Le petit artisan qui possède un atelier de mécanique, et qui sait qu'il devra fournir dans une quinzaine tant de vilebrequins, tant de robinets, tant de bielles, ne dispose pas non plus arbitrairement de son temps ; mais du moins, la commande une fois admise, c'est lui qui détermine d'avance l'emploi de ses heures et de ses journées. Si même le chef disait à l'ouvrier, une semaine ou deux à l'avance : pendant deux jours vous me ferez des bielles, puis des vilebrequins, et ainsi de suite, il faudrait obéir, mais il serait possible d'embrasser par la pensée l'avenir prochain, de le dessiner d'avance, de le posséder. Il n'en est pas ainsi dans l'usine. Depuis le moment où on pointe pour entrer jusqu'à celui où on pointe pour sortir, on est à chaque instant dans le cas de subir un ordre. Comme un objet inerte que chacun peut à tout moment changer de place. Si on travaille sur une pièce qui doit prendre encore deux heures, on ne peut pas penser à ce qu'on fera dans trois heures sans que la pensée ait à faire un détour qui la contraint de passer par le chef, sans qu'on soit forcé de se redire qu'on est soumis à des ordres ; si on fait dix pièces par minute, il en est déjà de même pour les cinq minutes suivantes. Si l'on suppose que peut-être aucun ordre ne surviendra, comme les ordres sont le seul facteur de variété, les éliminer par la pensée, c'est se condamner à imaginer une répétition ininterrompue de pièces toujours identiques, des régions mornes et désertiques que la pensée ne peut pas parcourir. En fait, il est vrai, mille menus incidents peupleront ce désert, mais, s'ils comptent dans l'heure qui s'écoule, ils n'entrent pas en ligne de compte quand on se représente l'avenir. Si la pensée veut éviter cette monotonie, imaginer du changement, donc un ordre soudain, elle ne peut pas voyager du moment présent à un moment à venir sans passer par une humiliation. Ainsi la pensée se rétracte. Ce repliement sur le présent produit une sorte de stupeur. Le seul avenir supportable pour la pensée, et au-delà duquel elle n'a pas la force de s'étendre, c'est celui qui, lorsqu'on est en plein travail, sépare l'instant où on se trouve de l'achèvement de la pièce en cours, si l'on a la chance qu'elle soit un peu longue à achever. À certains moments, le travail est assez absorbant pour que la pensée se maintienne d'elle-même dans ces limites. Alors on ne souffre pas. Mais le soir, une fois sorti, et surtout le matin, quand on se dirige vers le lieu du travail et la pendule de pointage, il est dur de penser à la journée qu'il faudra parcourir. Et le dimanche soir, quand ce qui se présente à l'esprit, ce n'est pas une journée, mais toute une semaine, l'avenir est quelque chose de trop morne, de trop accablant, sous quoi la pensée plie.\par
La monotonie d'une journée à l'usine, même si aucun changement de travail ne vient la rompre, est mélangée de mille petits incidents qui peuplent chaque journée et en font une histoire neuve ; mais, comme pour le changement de travail, ces incidents blessent plus souvent qu'ils ne réconfortent. Ils correspondent toujours à une diminution de salaire dans le cas du travail aux pièces, de sorte qu'on ne peut les souhaiter. Mais souvent ils blessent aussi par eux-mêmes. L'angoisse répandue diffuse sur tous les moments du travail, s'y concentre, l'angoisse de ne pas aller assez vite, et quand, comme c'est souvent le cas, on a besoin d'autrui pour pouvoir continuer, d'un contremaître, d'un magasinier, d'un régleur, le sentiment de la dépendance, de l'impuissance, et de compter pour rien aux yeux de qui on dépend, peut devenir douloureux au point d'arracher des larmes aux hommes comme aux femmes. La possibilité continuelle de tels incidents, machine arrêtée, caisse introuvable, et ainsi de suite, loin de diminuer le poids de la monotonie, lui ôte le remède qu'en général elle porte en elle-même, le pouvoir d'assoupir et de bercer les pensées de manière à cesser, dans une certaine mesure, d'être sensible ; une légère angoisse empêche cet effet d'assoupissement et force à avoir conscience de la monotonie, bien qu'il soit intolérable d'en avoir conscience. Rien n'est pire que le mélange de la monotonie et du hasard ; ils s'aggravent l'un l'autre, du moins quand le hasard est angoissant. Il est angoissant dans l'usine, du fait qu'il n'est pas reconnu ; théoriquement, bien que tout le monde sache qu'il n'en est rien, les caisses où mettre les pièces usinées ne manquent jamais, les régleurs ne font jamais attendre, et tout ralentissement dans la production est une faute de l'ouvrier. La pensée doit constamment être prête à la fois à suivre le cours monotone de gestes indéfiniment répétés et à trouver en elle-même des ressources pour remédier à l'imprévu. Obligation contradictoire, impossible, épuisante. Le corps est parfois épuisé, le soir, au sortir de l'usine, mais la pensée l'est toujours, et elle l'est davantage. Quiconque a éprouvé cet épuisement et ne l'a pas oublié peut le lire dans les yeux de presque tous les ouvriers qui défilent le soir hors d'une usine. Combien on aimerait pouvoir déposer son âme, en entrant, avec sa carte de pointage, et la reprendre intacte à la sortie ! Mais le contraire se produit. On l'emporte avec soi dans l'usine, où elle souffre ; le soir, cet épuisement l'a comme anéantie, et les heures de loisir sont vaines.\par
Certains incidents, au cours du travail, procurent, il est vrai, de la joie, même s'ils diminuent le salaire. D’abord les cas, qui sont rares, où on reçoit d'un autre à cette occasion un précieux témoignage de camaraderie ; puis tous ceux où l'on peut se tirer d'affaire soi-même. Pendant qu'on s'ingénie, qu'on fait effort, qu'on ruse avec l'obstacle, l'âme est occupée d'un avenir qui ne dépend que de soi-même.\par
Plus un travail est susceptible d'amener de pareilles difficultés, plus il élève le cœur. Mais cette joie est incomplète par le défaut d'hommes, de camarades ou chefs, qui jugent et apprécient la valeur de ce qu'on a réussi. Presque toujours aussi bien les chefs que les camarades chargés d'autres opérations sur les mêmes pièces se préoccupent exclusivement des pièces et non des difficultés vaincues. Cette indifférence prive de la chaleur humaine dont on a toujours un peu besoin. Même l'homme le moins désireux de satisfactions d'amour-propre se sent trop seul dans un endroit où il est entendu qu'on s'intéresse exclusivement à ce qu'il a fait, jamais à la manière dont il s'y est pris pour le faire ; par là les joies du travail se trouvent reléguées au rang des impressions informulées, fugitives, disparues aussitôt que nées ; la camaraderie des travailleurs, ne parvenant pas à se nouer, reste une velléité informe, et les chefs ne sont pas des hommes qui guident et surveillent d'autres hommes, mais les organes d'une subordination impersonnelle, brutale et froide comme le fer. Il est vrai, dans ce rapport de subordination, la personne du chef intervient, mais c'est par le caprice ; la brutalité impersonnelle et le caprice, loin de se tempérer, s'aggravent réciproquement, comme la monotonie et le hasard.\par
\par
De nos jours, ce n'est pas seulement dans les magasins, les marchés, les échanges, que les produits du travail entrent seuls en ligne de compte, et non les travaux qui les ont suscités. Dans les usines modernes il en est de même, du moins au niveau de l'ouvrier. La coopération, la compréhension, l'appréciation mutuelle dans le travail y sont le monopole des sphères supérieures. Au niveau de l'ouvrier, les rapports établis entre les différents postes, les différentes fonctions, sont des rapports entre les choses et non entre les hommes. Les pièces circulent avec leurs fiches, l'indication du nom, de la forme, de la matière première ; on pourrait presque croire que ce sont elles qui sont les personnes, et les ouvriers qui sont des pièces interchangeables. Elles ont un état civil ; et quand il faut, comme c'est le cas dans quelques grandes usines, montrer en entrant une carte d'identité où l'on se trouve photographié avec un numéro sur la poitrine, comme un forçat, le contraste est un symbole poignant et qui fait mal.\par
Les choses jouent le rôle des hommes, les hommes jouent le rôle des choses ; c'est la racine du mal. Il y a beaucoup de situations différentes dans une usine ; l'ajusteur qui, dans un atelier d'outillage, fabrique, par exemple, des matrices de presses, merveilles d'ingéniosité, longues à façonner, toujours différentes, celui-là ne perd rien en entrant dans l'usine ; mais ce cas est rare. Nombreux au contraire dans les grandes usines et même dans beaucoup de petites sont ceux ou celles qui exécutent à toute allure, par ordre, cinq ou six gestes simples indéfiniment répétés, un par seconde environ, sans autre répit que quelques courses anxieuses pour chercher une caisse, un régleur, d'autres pièces, jusqu'à la seconde précise où un chef vient en quelque sorte les prendre comme des objets pour les mettre devant une autre machine ; ils y resteront jusqu'à ce qu'on les mette ailleurs. Ceux-là sont des choses autant qu'un être humain peut l'être, mais des choses qui n'ont pas licence de perdre conscience, puisqu'il faut toujours pouvoir faire face à l'imprévu. La succession de leurs gestes n'est pas désignée, dans le langage de l'usine, par le mot de rythme, mais par celui de cadence, et c'est juste, car cette succession est le contraire d'un rythme. Toutes les suites de mouvements qui participent au beau et s'accomplissent sans dégrader enferment des instants d'arrêt, brefs comme l'éclair, qui constituent le secret du rythme et donnent au spectateur, à travers même l'extrême rapidité, l'impression de la lenteur. Le coureur à pied, au moment qu'il dépasse un record mondial, semble glisser lentement, tandis qu'on voit les coureurs médiocres se hâter loin derrière lui ; plus un paysan fauche vite et bien, plus ceux qui le regardent sentent que, comme on dit si justement, il prend tout son temps. Au contraire, le spectacle de manœuvres sur machines est presque toujours celui d'une précipitation misérable d'où toute grâce et toute dignité sont absentes. Il est naturel à l'homme et il lui convient de s'arrêter quand il a fait quelque chose, fût-ce l'espace d'un éclair, pour en prendre conscience, comme Dieu dans la Genèse ; cet éclair de pensée, d'immobilité et d'équilibre, c'est ce qu'il faut apprendre à supprimer entièrement dans l'usine, quand on y travaille. Les manœuvres sur machines n'atteignent la cadence exigée que si les gestes d'une seconde se succèdent d'une manière ininterrompue et presque comme le tic-tac d'une horloge, sans rien qui marque jamais que quelque chose est fini et qu'autre chose commence. Ce tic-tac dont on ne peut supporter d'écouter longtemps la morne monotonie, eux doivent presque le reproduire avec leur corps. Cet enchaînement ininterrompu tend à plonger dans une espèce de sommeil, mais il faut le supporter sans dormir. Ce n'est pas seulement un supplice ; s'il n'en résultait que de la souffrance, le mal serait moindre qu'il n'est. Toute action humaine exige un mobile qui fournisse l'énergie nécessaire pour l'accomplir, et elle est bonne ou mauvaise selon que le mobile est élevé ou bas. Pour se plier à la passivité épuisante qu'exige l'usine, il faut chercher des mobiles en soi-même, car il n'y a pas de fouets, pas de chaînes ; des fouets, des chaînes rendraient peut-être la transformation plus facile. Les conditions mêmes du travail empêchent que puissent intervenir d'autres mobiles que la crainte des réprimandes et du renvoi, le désir avide d'accumuler des sous, et, dans une certaine mesure, le goût des records de vitesse. Tout concourt pour rappeler ces mobiles à la pensée et les transformer en obsessions ; il n'est jamais fait appel à rien de plus élevé ; d'ailleurs ils doivent devenir obsédants pour être assez efficaces. En même temps que ces mobiles occupent l'âme, la pensée se rétracte sur un point du temps pour éviter la souffrance, et la conscience s'éteint autant que les nécessités du travail le permettent. Une force presque irrésistible, comparable à la pesanteur, empêche alors de sentir la présence d'autres êtres humains qui peinent eux aussi tout près ; il est presque impossible de ne pas devenir indifférent et brutal comme le système dans lequel on est pris ; et réciproquement la brutalité du système est reflétée et rendue sensible par les gestes, les regards, les paroles de ceux qu'on a autour de soi. Après une journée ainsi passée, un ouvrier n'a qu'une plainte, plainte qui ne parvient pas aux oreilles des hommes étrangers à cette condition et ne leur dirait rien si elle y parvenait ; il a trouvé le temps long.\par
Le temps lui a été long et il a vécu dans l'exil. Il a passé sa journée dans un lieu où il n'était pas chez lui ; les machines et les pièces à usiner y sont chez elles, et il n'y est admis que, pour approcher les pièces des machines. On ne s'occupe que d'elles, pas de lui ; d'autres fois on s'occupe trop de lui et pas assez d'elles, car il n'est pas rare de voir un atelier où les chefs sont occupés à harceler ouvriers et ouvrières, veillant à ce qu'ils ne lèvent pas la tête même le temps d'échanger un regard, pendant que des monceaux de ferraille sont livrés à la rouille dans la cour. Rien n'est plus amer. Mais que l'usine se défende bien ou mal contre le coulage, en tout cas l'ouvrier sent qu'il n'y est pas chez lui. Il y reste étranger. Rien n'est si puissant chez l'homme que le besoin de s'approprier, non pas juridiquement, mais par la pensée, les lieux et les objets parmi lesquels il passe sa vie et dépense la vie qu'il a en lui ; une cuisinière dit « ma cuisine », un jardinier dit « ma pelouse », et c'est bien ainsi. La propriété juridique n'est qu'un des moyens qui procurent un tel sentiment, et l'organisation sociale parfaite serait celle qui par l'usage de ce moyen et des autres moyens donnerait ce sentiment à tous les êtres humains. Un ouvrier, sauf quelques cas trop rares, ne peut rien s'approprier par la pensée dans l'usine. Les machines ne sont pas à lui ; il sert l'une ou l'autre selon qu'il en reçoit l'ordre. Il les sert, il ne s'en sert pas ; elles ne sont pas pour lui un moyen d'amener un morceau de métal à prendre une certaine forme, il est pour elles un moyen de leur amener des pièces en vue d'une opération dont il ignore le rapport avec celles qui précèdent et celles qui suivent.\par
Les pièces ont leur histoire ; elles passent d'un stade de fabrication à un autre ; lui n'est pour rien dans cette histoire, il n'y laisse pas sa marque, il n'en connaît rien. S'il était curieux, sa curiosité ne serait pas encouragée, et d'ailleurs la même douleur sourde et permanente qui empêche la pensée de voyager dans le temps l'empêche aussi de voyager à travers l'usine et la cloue en un point de l'espace, comme au moment présent. L'ouvrier ne sait pas ce qu'il produit, et par suite il n'a pas le sentiment d'avoir produit, mais de s'être épuisé à vide. Il dépense à l'usine, parfois jusqu'à l'extrême limite, ce qu'il a de meilleur en lui, sa faculté de penser, de sentir, de se mouvoir ; il les dépense, puisqu'il en est vidé quand il sort ; et pourtant il n'a rien mis de lui-même dans son travail, ni pensée, ni sentiment, ni même, sinon dans une faible mesure, mouvements déterminés par lui, ordonnés par lui en vue d'une fin. Sa vie même sort de lui sans laisser aucune marque autour de lui. L'usine crée des objets utiles, mais non pas lui, et la paie qu'on attend chaque quinzaine par files, comme un troupeau, paie impossible à calculer d'avance, dans le cas du travail aux pièces, par suite de l'arbitraire et de la complication des comptes, semble plutôt une aumône que le prix d'un effort. L'ouvrier, quoique indispensable à la fabrication, n'y compte presque pour rien, et c'est pourquoi chaque souffrance physique inutilement imposée, chaque manque d'égard, chaque brutalité, chaque humiliation même légère semble un rappel qu'on ne compte pas et qu'on n'est pas chez soi. On peut voir des femmes attendre dix minutes devant une usine sous des torrents de pluie, en face d'une porte ouverte par où passent des chefs, tant que l'heure n'a pas sonné ; ce sont des ouvrières ; cette porte leur est plus étrangère que celle de n'importe quelle maison inconnue où elles entreraient tout naturellement pour se réfugier. Aucune intimité ne lie les ouvriers aux lieux et aux objets parmi lesquels leur vie s'épuise, et l'usine fait d'eux, dans leur propre pays, des étrangers, des exilés, des déracinés. Les revendications ont eu moins de part dans l'occupation des usines que le besoin de s'y sentir au moins une fois chez soi. Il faut que la vie sociale soit corrompue jusqu'en son centre lorsque les ouvriers se sentent chez eux dans l'usine quand ils font grève, étrangers quand ils travaillent. Le contraire devrait être vrai. Les ouvriers ne se sentiront vraiment chez eux dans leur pays, membres responsables du pays, que lorsqu'ils se sentiront chez eux dans l'usine pendant qu'ils y travaillent.\par
Il est difficile d'être cru quand on ne décrit que des impressions. Pourtant on ne peut décrire autrement le malheur d'une condition humaine. Le malheur n'est fait que d'impressions. Les circonstances matérielles de la vie, aussi longtemps qu'il est à la rigueur possible d'y vivre, ne rendent pas à elles seules compte du malheur, car des circonstances équivalentes, attachées à d'autres sentiments, rendraient heureux. Ce sont les sentiments attachés aux circonstances d'une vie qui rendent heureux ou malheureux, mais ces sentiments ne sont pas arbitraires, ils ne sont pas imposés ou effacés par suggestion, ils ne peuvent être changés que par une transformation radicale des circonstances elles-mêmes. Pour les changer, il faut d'abord les connaître. Rien n'est plus difficile à connaître que le malheur ; il est toujours un mystère. Il est muet, comme disait un proverbe grec. Il faut être particulièrement préparé à l'analyse intérieure pour en saisir les vraies nuances et leurs causes, et ce n'est pas généralement le cas des malheureux. Même si on est préparé, le malheur même empêche cette activité de la pensée, et l'humiliation a toujours pour effet de créer des zones interdites où la pensée ne s'aventure pas et qui sont couvertes soit de silence soit de mensonge. Quand les malheureux se plaignent, ils se plaignent presque toujours à faux, sans évoquer leur véritable malheur ; et d'ailleurs, dans le cas du malheur profond et permanent, une très forte pudeur arrête les plaintes. Ainsi chaque condition malheureuse parmi les hommes crée une zone de silence où les êtres humains se trouvent enfermés comme dans une île. Qui sort de l'île ne tourne pas la tête. Les exceptions, presque toujours, sont seulement apparentes. Par exemple, la même distance, la plupart du temps, malgré l'apparence contraire, sépare des ouvriers l'ouvrier devenu patron et l'ouvrier devenu, dans les syndicats, militant professionnel.\par
Si quelqu'un, venu du dehors, pénètre dans une de ces îles et se soumet volontairement au malheur, pour un temps limité, mais assez long pour s'en pénétrer, et s'il raconte ensuite ce qu’on éprouve, on pourra facilement contester la valeur de son témoignage. On dira qu'il a éprouvé autre chose que ceux qui sont là d'une manière permanente. On aura raison s'il s'est livré seulement à l'introspection ; de même s'il a seulement observé. Mais si, étant parvenu à oublier qu'il vient d'ailleurs, retournera ailleurs, et se trouve là seulement pour un voyage, il compare continuellement ce qu'il éprouve pour lui-même à ce qu'il lit sur les visages, dans les yeux, les gestes, les attitudes, les paroles, dans les événements petits et grands, il se crée en lui un sentiment de certitude, malheureusement difficile à communiquer. Les visages contractés par l'angoisse de la journée à traverser et les yeux douloureux dans le métro du matin ; la fatigue profonde, essentielle, la fatigue d'âme encore plus que de corps, qui marque les attitudes, les regards et le pli des lèvres, le soir, à la sortie ; les regards et les attitudes de bêtes en cage, quand une usine, après la fermeture annuelle de dix jours, vient de rouvrir pour une interminable année ; la brutalité diffuse et qu'on rencontre presque partout ; l'importance attachée par presque tous à des détails petits par eux-mêmes, mais douloureux par leur signification symbolique, tels que l'obligation de présenter une carte d'identité en entrant ; les vantardises pitoyables échangées parmi les troupeaux massés devant la porte des bureaux d'embauche, et qui, par opposition, évoquent tant d'humiliations réelles ; les paroles incroyablement douloureuses qui s'échappent parfois, comme par inadvertance, des lèvres d'hommes et de femmes semblables à tous les autres ; la haine et le dégoût de l'usine, du lieu du travail, que les paroles et les actes font si souvent apparaître, qui jette son ombre sur la camaraderie et pousse ouvriers et ouvrières, dès qu'ils sortent, à se hâter chacun chez soi presque sans échanger une parole ; la joie, pendant l'occupation des usines, de posséder l'usine par la pensée, d'en parcourir les parties, la fierté toute nouvelle de la montrer aux siens et de leur expliquer où on travaille, joie et fierté fugitives qui exprimaient par contraste d'une manière si poignante les douleurs permanentes de la pensée clouée ; tous les remous de la classe ouvrière, si mystérieux aux spectateurs, en réalité si aisés à comprendre ; comment ne pas se fier à tous ces signes, lorsqu'en même temps qu’on les lit autour de soi on éprouve en soi-même tous les sentiments correspondants ?\par
L'usine devrait être un lieu de joie, un lieu où, même s'il est inévitable que le corps et l'âme souffrent, l'âme puisse aussi pourtant goûter des joies, se nourrir de joies. Il faudrait pour cela y changer, en un sens peu de choses, en un sens beaucoup. Tous les systèmes de réforme ou de transformation sociale portent à faux ; s'ils étaient réalisés, ils laisseraient le mal intact ; ils visent à changer trop et trop peu, trop peu ce qui est la cause du mal, trop les circonstances qui y sont étrangères. Certains annoncent une diminution, d'ailleurs ridiculement exagérée, de la durée du travail ; mais faire du peuple une masse d'oisifs qui seraient esclaves deux heures par jour n'est ni souhaitable, quand ce serait possible, ni moralement possible, quand ce serait possible matériellement. Nul n'accepterait d'être esclave deux heures ; l'esclavage, pour être accepté, doit durer assez chaque jour pour briser quelque chose dans l'homme. S'il y a un remède possible, il est d'un autre ordre et plus difficile à concevoir. Il exige un effort d'invention. Il faut changer la nature des stimulants du travail, diminuer ou abolir les causes de dégoût, transformer le rapport de chaque ouvrier avec le fonctionnement de l'ensemble de l'usine, le rapport de l'ouvrier avec la machine, et la manière dont le temps s'écoule dans le travail.\par
Il n'est pas bon, ni que le chômage soit comme un cauchemar sans issue, ni que le travail soit récompensé par un flot de faux luxe à bon marché qui excite les désirs sans satisfaire les besoins. Ces deux points ne sont guère contestés. Mais il s'ensuit que la peur du renvoi et la convoitise des sous doivent cesser d'être les stimulants essentiels qui occupent sans cesse le premier plan dans l'âme des ouvriers, pour agir désormais à leur rang naturel comme stimulants secondaires. D'autres stimulants doivent être au premier plan.\par
Un des plus puissants, dans tout travail, est le sentiment qu'il y a quelque chose à faire et qu'un effort doit être accompli. Ce stimulant, dans une usine, et surtout pour le manœuvre sur machines, manque bien souvent complètement. Lorsqu'il met mille fois une pièce en contact avec l'outil d'une machine, il se trouve, avec la fatigue en plus, dans la situation d'un enfant à qui on a ordonné d'enfiler des perles pour le faire tenir tranquille ; l'enfant obéit parce qu'il craint un châtiment et espère un bonbon, mais son action n'a pas de sens pour lui, sinon la conformité avec l'ordre donné par la personne qui a pouvoir sur lui. Il en serait autrement si l'ouvrier savait clairement, chaque jour, chaque instant, quelle part ce qu'il est en train de faire a dans la fabrication de l'usine, et quelle place l'usine où il se trouve tient dans la vie sociale. Si un ouvrier fait tomber l'outil d'une presse sur un morceau de laiton qui doive faire partie d'un dispositif destiné au métro, il faudrait qu'il le sache, et que de plus il se représente quelles seront la place et la fonction de ce morceau de laiton dans une rame de métro, quelles opérations il a déjà subies et doit encore subir avant d'être mis en place. Il n'est pas question, bien entendu, de faire une conférence à chaque ouvrier avant chaque travail. Ce qui est possible, c'est de faire parcourir de temps à autre l'usine par chaque équipe d'ouvriers à tour de rôle, pendant quelques heures qui seraient payées au tarif ordinaire, et en accompagnant la visite d'explications techniques. Permettre aux ouvriers d'amener leurs familles pour ces visites serait mieux encore ; est-il naturel qu'une femme ne puisse jamais voir l'endroit où son mari dépense le meilleur de lui-même tous les jours et pendant toute la journée ? Tout ouvrier serait heureux et fier de montrer l'endroit où il travaille à sa femme et à ses enfants. Il serait bon aussi que chaque ouvrier voie de temps à autre, achevée, la chose à la fabrication de laquelle il a eu une part, si minime soit-elle, et qu'on lui fasse saisir quelle part exactement il y a prise. Bien entendu, le problème se pose différemment pour chaque usine, chaque fabrication, et on peut selon les cas particuliers, des méthodes infiniment variées pour stimuler et satisfaire la curiosité des travailleurs à l'égard de leur travail. Il n'y faut pas un grand effort d'imagination, à condition de concevoir clairement le but, qui est de déchirer le voile que met l'argent entre le travailleur et le travail. Les ouvriers croient, de cette espèce de croyance qui ne s'exprime pas en paroles, qui serait absurde ainsi exprimée, mais qui imprègne tous les sentiments, que leur peine se transforme en argent dont une petite part leur revient et dont une grosse part va au patron. Il faut leur faire comprendre, non pas avec cette partie superficielle de l'intelligence que nous appliquons aux vérités évidentes – car de cette manière ils le comprennent déjà – mais avec toute l'âme et pour ainsi dire avec le corps lui-même, dans tous les moments de leur peine, qu'ils fabriquent des objets qui sont appelés par des besoins sociaux, et qu'ils ont un droit limité, mais réel, à en être fiers.\par
Il est vrai qu'ils ne fabriquent pas véritablement des objets tant qu'ils se bornent à répéter longtemps une combinaison de cinq ou six gestes simples toujours identique à elle-même. Cela ne doit plus être. Tant qu'il en sera ainsi, et quoi qu'on fasse, il y aura toujours au cœur de la vie sociale un prolétariat avili et haineux. Il est vrai que certains êtres humains, mentalement arriérés, sont naturellement aptes à ce genre de travail ; mais il n'est pas vrai que leur nombre soit égal à celui des êtres humains qui en fait travaillent ainsi, et il s'en faut de très loin. La preuve en est que sur cent enfants nés dans des familles bourgeoises la proportion de ceux qui, une fois hommes, ne font que des tâches machinales est bien moindre que pour cent enfants d'ouvriers, quoique la répartition des aptitudes soit en moyenne vraisemblablement la même. Le remède n'est pas difficile à trouver, du moins dans une période normale où le métal ne manque pas. Toutes les fois qu'une fabrication exige la répétition d'une combinaison d'un petit nombre de mouvements simples, ces mouvements peuvent être accomplis par une machine automatique, et cela sans aucune exception. On emploie de préférence un homme parce que l'homme est une machine qui obéit à la voix et qu'il suffit à un homme de recevoir un ordre pour substituer en un moment telle combinaison de mouvements à telle autre. Mais il existe des machines automatiques à usages multiples qu'on peut également faire passer d'une fabrication à une autre en remplaçant une came par une autre. Cette espèce de machines est encore récente et peu développée ; nul ne peut prévoir jusqu'à quel point on pourra la développer si l'on s'en donne la peine. Il pourra alors apparaître des choses que l'on nommerait machines, mais qui, du point de vue de l'homme qui travaille, seraient exactement l'opposé de la plupart des machines actuellement en usage ; il n'est pas rare que le même mot recouvre des réalités contraires. Un manœuvre spécialisé n'a en partage que la répétition automatique des mouvements, pendant que la machine qu'il sert enferme, imprimée et cristallisée dans le métal, toute la part de combinaison et d'intelligence que comporte la fabrication en cours. Un tel renversement est contre nature ; c'est un crime. Mais si un homme a pour tâche de régler une machine automatique et de fabriquer les cames correspondant chaque fois aux pièces à usiner, il assume d'une part une partie de l'effort de réflexion et de combinaison, d'autre part un effort manuel comportant, comme celui des artisans, une véritable habileté. Un tel rapport entre la machine et l'homme est pleinement satisfaisant.\par
Le temps et le rythme sont le facteur le plus important du problème ouvrier. Certes le travail n'est pas le jeu ; il est à la fois inévitable et convenable qu'il y ait dans le travail de la monotonie et de l'ennui, et d'ailleurs il n'est rien de grand sur cette terre, dans aucun domaine, sans une part de monotonie et d'ennui. Il y a plus de monotonie dans une messe en chant grégorien ou dans un concerto de Bach que dans une opérette. Ce monde où nous sommes tombés existe réellement ; nous sommes réellement chair ; nous avons été jetés hors de l'éternité ; et nous devons réellement traverser le temps, avec peine, minute après minute. Cette peine est notre partage, et la monotonie du travail en est seulement une forme. Mais il n'est pas moins vrai que notre pensée est faite pour dominer le temps, et que cette vocation doit être préservée intacte en tout être humain. La succession absolument uniforme en même temps que variée et continuellement surprenante des jours, des mois, des saisons et des années convient exactement à notre peine et à notre grandeur. Tout ce qui parmi les choses humaines est à quelque degré beau et bon reproduit à quelque degré ce mélange d'uniformité et de variété ; tout ce qui en diffère est mauvais et dégradant. Le travail du paysan obéit par nécessité à ce rythme du monde ; le travail de l'ouvrier, par sa nature même, en est dans une large mesure indépendant, mais il pourrait l'imiter. C'est le contraire qui se produit dans les usines. L'uniformité et la variété s'y mélangent aussi, mais ce mélange est l'opposé de celui que procurent le soleil et les astres ; le soleil et les astres emplissent d'avance le temps de cadres faits d'une variété limitée et ordonnée en retours réguliers, cadres destinés à loger une variété infinie d'événements absolument imprévisibles et partiellement privés d'ordre ; au contraire, l'avenir de celui qui travaille dans une usine est vide à cause de l'impossibilité de prévoir, et plus mort que du passé à cause de l'identité des instants qui se succèdent comme les tic-tac d'une horloge. Une uniformité qui imite les mouvements des horloges et non pas ceux des constellations, une variété qui exclut toute règle et par suite toute prévision, cela fait un temps inhabitable à l'homme, irrespirable.\par
La transformation des machines peut seule empêcher le temps des ouvriers de ressembler à celui des horloges ; mais cela ne suffit pas ; il faut que l'avenir s'ouvre devant l'ouvrier par une certaine possibilité de prévision, afin qu'il ait le sentiment d'avancer dans le temps, d'aller à chaque effort vers un certain achèvement. Actuellement l'effort qu'il est en train d'accomplir ne le mène nulle part, sinon à l'heure de la sortie, mais comme un jour de travail succède toujours à un autre l'achèvement dont il s'agit n'est pas autre chose que la mort ; il ne peut s'en représenter un autre que sous forme de salaire, dans le cas du travail aux pièces, ce qui le contraint à l'obsession des sous. Ouvrir un avenir aux ouvriers dans la représentation du travail futur, c'est un problème qui se pose autrement pour chaque cas particulier. D'une manière générale, la solution de ce problème implique, outre une certaine connaissance du fonctionnement d'ensemble de l'usine accordée à chaque ouvrier, une organisation de l'usine comportant une certaine autonomie des ateliers par rapport à l'établissement et de chaque ouvrier par rapport à son atelier. À l'égard de l'avenir prochain, chaque ouvrier devrait autant que possible savoir à peu près ce qu'il aura à faire les huit ou quinze jours qui suivront, et même avoir un certain choix quant à l'ordre de succession des différentes tâches. À l'égard de l'avenir lointain, il devrait être en mesure d'y projeter quelques jalons, d'une manière certes moins étendue et moins précise que le patron et le directeur, mais pourtant en quelque manière analogue. De cette manière, sans que ses droits effectifs aient été le moins du monde accrus, il éprouvera ce sentiment de propriété dont le cœur de l'homme a soif, et qui, sans diminuer la peine, abolit le dégoût.\par
De telles réformes sont difficiles, et certaines des circonstances de la période présente en augmentent la difficulté. En revanche le malheur était indispensable pour faire sentir qu'on doit changer quelque chose. Les principaux obstacles sont dans les âmes. Il est difficile de vaincre la peur et le mépris. Les ouvriers, ou du moins beaucoup d'entre eux, ont acquis après mille blessures une amertume presque inguérissable qui fait qu'ils commencent par regarder comme un piège tout ce qui leur vient d'en haut, surtout des patrons ; cette méfiance maladive, qui rendrait stérile n'importe quel effort d'amélioration, ne peut être vaincue sans patience, sans persévérance. Beaucoup de patrons craignent qu'une tentative de réforme, quelle qu'elle soit, si inoffensive soit-elle, apporte des ressources nouvelles aux meneurs, à qui ils attribuent tous les maux sans exception en matière sociale, et qu'ils se représentent en quelque sorte comme des monstres mythologiques. Ils ont du mal aussi à admettre qu'il y ait chez les ouvriers certaines parties supérieures de l'âme qui s'exerceraient dans le sens de l'ordre social si l'on y appliquait les stimulants convenables. Et quand même ils seraient convaincus de l'utilité des réformes indiquées, ils seraient retenus par un souci exagéré du secret industriel ; pourtant l'expérience leur a appris que l'amertume et l'hostilité sourde enfoncées au cœur des ouvriers enferment de bien plus grands dangers pour eux que la curiosité des concurrents. Au reste l'effort à accomplir n'incombe pas seulement aux patrons et aux ouvriers, mais à toute la société ; notamment l'école devrait être conçue d'une manière toute nouvelle, afin de former des hommes capables de comprendre l'ensemble du travail auquel ils ont part. Non que le niveau des études théoriques doive être abaissé ; c'est plutôt le contraire ; on devrait faire bien plus pour provoquer l'éveil de l'intelligence ; mais en même temps l'enseignement devrait devenir beaucoup plus concret.\par
Le mal qu'il s'agit de guérir intéresse aussi toute la société. Nulle société ne peut être stable quand toute une catégorie de travailleurs travaille tous les jours, toute la journée, avec dégoût. Ce dégoût dans le travail altère chez les ouvriers toute la conception de la vie, toute la vie. L'humiliation dégradante qui accompagne chacun de leurs efforts cherche une compensation dans une sorte d'impérialisme ouvrier entretenu par les propagandes issues du marxisme ; si un homme qui fabrique des boulons éprouvait, à fabriquer des boulons, une fierté légitime et limitée, il ne provoquerait pas artificiellement en lui-même un orgueil illimité par la pensée que sa classe est destinée à faire l'histoire et à dominer tout. Il en est de même pour la conception de la vie privée, et notamment de la famille et des rapports entre sexes ; le morne épuisement du travail d'usine laisse un vide qui demande à être comblé et ne peut l'être que par des jouissances rapides et brutales, et la corruption qui en résulte est contagieuse pour toutes les classes de la société. La corrélation n'est pas évidente à première vue, mais pourtant il y a corrélation ; la famille ne sera véritablement respectée chez le peuple de ce pays tant qu'une partie de ce peuple travaillera continuellement avec dégoût.\par
Il est venu beaucoup de mal des usines, et il faut corriger ce mal dans les usines. C'est difficile, ce n'est peut-être pas impossible. Il faudrait d'abord que les spécialistes, ingénieurs et autres, aient suffisamment à cœur non seulement de construire des objets, mais de ne pas détruire des hommes. Non pas de les rendre dociles, ni même de les rendre heureux, mais simplement de ne contraindre aucun d'eux à s'avilir.
\section[Condition première d'un travail non servile ]{Condition première d'un travail non servile \protect\footnotemark }\renewcommand{\leftmark}{Condition première d'un travail non servile }

\footnotetext{ Écrit à Marseille en 1941, a paru partiellement dans le n° 4 du {\itshape Cheval de Troie} en 1947.}
\noindent \par
Il y a dans le travail des mains et en général dans le travail d'exécution, qui est le travail proprement dit, un élément irréductible de servitude que même une parfaite équité sociale n'effacerait pas. C'est le fait qu'il est gouverné par la nécessité, non par la finalité. On l'exécute à cause d'un besoin, non en vue d'un bien ; « parce qu'on a besoin de gagner sa vie », comme disent ceux qui y passent leur existence. On fournit un effort au terme duquel, à tous égards, on n'aura pas autre chose que ce qu'on a. Sans cet effort, on perdrait ce qu'on a.\par
Mais, dans la nature humaine, il n'y a pas pour l'effort d'autre source d'énergie que le désir. Et il n'appartient pas à l'homme de désirer ce qu'il a. Le désir est une orientation, un commencement de mouvement vers quelque chose. Le mouvement est vers un point où on n'est pas. Si le mouvement à peine commencé se boucle sur le point de départ, on tourne comme un écureuil dans une cage, comme un condamné dans une cellule. Tourner toujours produit vite l'écœurement.\par
L'écœurement, la lassitude, le dégoût, c'est la grande tentation de ceux qui travaillent, surtout s'ils sont dans des conditions inhumaines, et même autrement. Parfois cette tentation mord davantage les meilleurs.\par
Exister n'est pas une fin pour l'homme, c'est seulement le support de tous les biens, vrais ou faux. Les biens s'ajoutent à l'existence. Quand ils disparaissent, quand l'existence n'est plus ornée d'aucun bien, quand elle est nue, elle n'a plus aucun rapport au bien. Elle est même un mal. Et c'est à ce moment même qu'elle se substitue à tous les biens absents, qu'elle devient en elle-même l'unique fin, l'unique objet du désir. Le désir de l'âme se trouve attaché à un mal nu et sans voile. L'âme est alors dans l'horreur.\par
Cette horreur est celle du moment où une violence imminente va infliger la mort. Ce moment d'horreur se prolongeait autrefois toute la vie pour celui qui, désarmé sous l'épée du vainqueur, était épargné. En échange de la vie qu'on lui laissait, il devait dans l'esclavage épuiser son énergie en efforts, tout le long du jour, tous les jours, sans rien pouvoir espérer, sinon de n'être pas tué ou fouetté. Il ne pouvait plus poursuivre aucun bien sinon d'exister. Les anciens disaient que le jour qui l'avait fait esclave lui avait enlevé la moitié de son âme.\par
Mais toute condition où l'on se trouve nécessairement dans la même situation au dernier jour d'une période d'un mois, d'un an, de vingt ans d'efforts qu'au premier jour a une ressemblance avec l'esclavage. La ressemblance est l'impossibilité de désirer une chose autre que celle qu'on possède, d'orienter l'effort vers l'acquisition d'un bien. On fait effort seulement pour vivre.\par
L'unité de temps est alors la journée. Dans cet espace on tourne en rond. On y oscille entre le travail et le repos comme une balle qui serait renvoyée d'un mur à l'autre. On travaille seulement parce qu'on a besoin de manger. Mais on mange pour pouvoir continuer à travailler. Et de nouveau on travaille pour manger.\par
Tout est intermédiaire dans cette existence, tout est moyen, la finalité ne s'y accroche nulle part. La chose fabriquée est un moyen ; elle sera vendue. Qui peut mettre en elle son bien ? La matière, l'outil, le corps du travailleur, son âme elle-même, sont des moyens pour la fabrication. La nécessité est partout, le bien nulle part.\par
Il ne faut pas chercher de causes à la démoralisation du peuple. La cause est là ; elle est permanente ; elle est essentielle à la condition du travail. Il faut chercher les causes qui, dans des périodes antérieures, ont empêché la démoralisation de se produire.\par
Une grande inertie morale, une grande force physique qui rend l'effort presque insensible permettent de supporter ce vide. Autrement il faut des compensations. L'ambition d'une autre condition sociale pour soi-même ou pour ses enfants en est une. Les plaisirs faciles et violents en sont une autre, qui est de même nature ; c'est le rêve au lieu de l'ambition. Le dimanche est le jour où l'on veut oublier qu'il existe une nécessité du travail. Pour cela il faut dépenser. Il faut être habillé comme si on ne travaillait pas. Il faut des satisfactions de vanité et des illusions de puissance que la licence procure très facilement. La débauche a exactement la fonction d'un stupéfiant, et l'usage des stupéfiants est toujours une tentation pour ceux qui souffrent. Enfin la révolution est encore une compensation de même nature ; c'est l'ambition transportée dans le collectif, la folle ambition d'une ascension de tous les travailleurs hors de la condition de travailleurs.\par
Le sentiment révolutionnaire est d'abord chez la plupart une révolte contre l'injustice, mais il devient rapidement chez beaucoup, comme il est devenu historiquement, un impérialisme ouvrier tout à fait analogue à l'impérialisme national. Il a pour objet la domination tout à fait illimitée d'une certaine collectivité sur l'humanité tout entière et sur tous les aspects de la vie humaine. L'absurdité est que, dans ce rêve, la domination serait aux mains de ceux qui exécutent et qui par suite ne peuvent pas dominer.\par
En tant que révolte contre l'injustice sociale l'idée révolutionnaire est bonne et saine. En tant que révolte contre le malheur essentiel à la condition même des travailleurs, elle est un mensonge. Car aucune révolution n'abolira ce malheur. Mais ce mensonge est ce qui a la plus grande emprise, car ce malheur essentiel est ressenti plus vivement, plus profondément, plus douloureusement que l'injustice elle-même. D'ordinaire d'ailleurs on les confond. Le nom d'opium du peuple que Marx appliquait à la religion a pu lui convenir quand elle se trahissait elle-même, mais il convient essentiellement à la révolution. L'espoir de la révolution est toujours un stupéfiant.\par
La révolution satisfait en même temps ce besoin de l'aventure, comme étant la chose la plus opposée à la nécessité, qui est encore une réaction contre le même malheur. Le goût des romans et des films policiers, la tendance à la criminalité qui apparaît chez les adolescents correspond aussi à ce besoin.\par
\par
Les bourgeois ont été très naïfs de croire que la bonne recette consistait à transmettre au peuple la fin qui gouverne leur propre vie, c'est-à-dire l'acquisition de l'argent. Ils y sont parvenus dans la limite du possible par le travail aux pièces et l'extension des échanges entre les villes et les campagnes. Mais ils n'ont fait ainsi que porter l'insatisfaction à un degré d'exaspération dangereux. La cause en est simple. L'argent en tant que but des désirs et des efforts ne peut pas avoir dans son domaine les conditions à l'intérieur desquelles il est impossible de s'enrichir. Un petit industriel, un petit commerçant peuvent s'enrichir et devenir un grand industriel, un grand commerçant. Un professeur, un écrivain, un ministre sont indifféremment riches ou pauvres. Mais un ouvrier qui devient très riche cesse d'être un ouvrier, et il en est presque toujours de même d'un paysan. Un ouvrier ne peut pas être mordu par le désir de l'argent sans désirer sortir, seul ou avec tous ses camarades, de la condition ouvrière.\par
L'univers où vivent les travailleurs refuse la finalité. Il est impossible qu'il y pénètre des fins, sinon pour de très brèves périodes qui correspondent à des situations exceptionnelles. L'équipement rapide de pays neufs, tels qu'ont été l'Amérique et la Russie, produit changements sur changements à un rythme si allègre qu'il propose à tous, presque de jour en jour, des choses nouvelles à attendre, à désirer, à espérer ; cette fièvre de construction a été le grand instrument de séduction du communisme russe, par l'effet d'une coïncidence, car elle tenait à l'état économique du pays et non à la révolution ni à la doctrine marxiste. Quand on élabore des métaphysiques d'après ces situations exceptionnelles, passagères et brèves, comme l'ont fait les Américains et les Russes, ces métaphysiques sont des mensonges.\par
La famille procure des fins sous forme d'enfants à élever. Mais à moins qu'on n'espère pour eux une autre condition – et par la nature des choses de telles ascensions sociales sont nécessairement exceptionnelles – le spectacle d'enfants condamnés à la même existence n'empêche pas de sentir douloureusement le vide et le poids de cette existence.\par
Ce vide pesant fait beaucoup souffrir. Il est sensible même à beaucoup de ceux dont la culture est nulle et l'intelligence faible. Ceux qui, par leur condition, ne savent pas ce que c'est ne peuvent pas juger équitablement les actions de ceux qui le supportent toute leur vie. Il ne fait pas mourir, mais il est peut-être aussi douloureux que la faim. Peut-être davantage. Peut-être il serait littéralement vrai de dire que le pain est moins nécessaire que le remède à cette douleur.\par
Il n'y a pas le choix des remèdes. Il n'y en a qu'un seul. Une seule chose rend supportable la monotonie, c'est une lumière d'éternité ; c'est la beauté.\par
Il y a un seul cas où la nature humaine supporte que le désir de l'âme se porte non pas vers ce qui pourrait être ou ce qui sera, mais vers ce qui existe. Ce cas, c'est la beauté. Tout ce qui est beau est objet de désir, mais on ne désire pas que cela soit autre, on ne désire rien y changer, on désire cela même qui est. On regarde avec désir le ciel étoilé d'une nuit claire, et ce qu'on désire, c'est uniquement le spectacle qu'on possède.\par
Puisque le peuple est contraint de porter tout son désir sur ce qu'il possède déjà, la beauté est faite pour lui et il est fait pour la beauté. La poésie est un luxe pour les autres conditions sociales. Le peuple a besoin de poésie comme de pain. Non pas la poésie enfermée dans les mots ; celle-là, par elle-même, ne peut lui être d'aucun usage. Il a besoin que la substance quotidienne de sa vie soit elle-même poésie.\par
Une telle poésie ne peut avoir qu'une source. Cette source est Dieu. Cette poésie ne peut être que religion. Par aucune ruse, aucun procédé, aucune réforme, aucun bouleversement la finalité ne peut pénétrer dans l’univers où les travailleurs sont placés par leur condition même. Mais cet univers peut être tout entier suspendu à la seule fin qui soit vraie. Il peut être accroché à Dieu. La condition des travailleurs est celle où la faim de finalité qui constitue l'être même de tout homme ne peut pas être rassasiée, sinon par Dieu.\par
C'est là leur privilège. Ils sont seuls à le posséder. Dans toutes les autres conditions, sans exception, des fins particulières se proposent à l'activité. Il n'est pas de fin particulière, quand ce serait le salut d'une âme ou de plusieurs, qui ne puisse faire écran et cacher Dieu. Il faut par le détachement percer à travers l'écran. Pour les travailleurs il n'y a pas d'écran. Rien ne les sépare de Dieu. Ils n'ont qu'à lever la tête.\par
Le difficile pour eux est de lever la tête. Ils n'ont pas, comme c'est le cas de tous les autres hommes, quelque chose en trop dont il leur faille se débarrasser avec effort. Ils ont quelque chose en trop peu. Il leur manque des intermédiaires. Quand on leur a conseillé de penser à Dieu et de lui faire offrande de leurs peines et de leurs souffrances, on n'a encore rien fait pour eux.\par
Les gens vont dans les églises exprès pour prier ; et pourtant on sait qu'ils ne le pourront pas si on ne fournit pas à leur attention des intermédiaires pour en soutenir l'orientation vers Dieu. L'architecture même de l'église, les images dont elle est pleine, les mots de la liturgie et des prières, les gestes rituels du prêtre sont ces intermédiaires. En y fixant l'attention, elle se trouve orientée vers Dieu. Combien plus grande encore la nécessité de tels intermédiaires sur le lieu du travail, où l'on va seulement pour gagner sa vie ! Là tout accroche la pensée à la terre.\par
Or on ne peut pas y mettre des images religieuses et proposer à ceux qui travaillent de les regarder. On ne peut leur suggérer non plus de réciter des prières en travaillant. Les seuls objets sensibles où ils puissent porter leur attention, c'est la matière, les instruments, les gestes de leur travail. Si ces objets mêmes ne se transforment pas en miroirs de la lumière, il est impossible que pendant le travail l'attention soit orientée vers la source de toute lumière. Il n'est pas de nécessité plus pressante que cette transformation.\par
Elle n'est possible que s'il se trouve dans la matière, telle qu'elle s'offre au travail des hommes, une propriété réfléchissante. Car il ne s'agit pas de fabriquer des fictions ou des symboles arbitraires. La fiction, l'imagination, la rêverie ne sont nulle part moins à leur place que dans ce qui concerne la vérité. Mais par bonheur pour nous il y a une propriété réfléchissante dans la matière. Elle est un miroir terni par notre haleine. Il faut seulement nettoyer le miroir et lire les symboles qui sont écrits dans la matière de toute éternité.\par
L'Évangile en contient quelques-uns. Dans une chambre, on a besoin pour penser à la nécessité de la mort morale en vue d'une nouvelle et véritable naissance, de lire ou de se répéter les mots qui concernent le grain que la mort seule rend fécond. Mais celui qui est en train de semer peut s'il le veut porter son attention sur cette vérité sans l'aide d'aucun mot, à travers son propre geste et le spectacle du grain qui s'enfouit. S'il ne raisonne pas autour d'elle, s'il la regarde seulement, l'attention qu'il porte à l'accomplissement de sa tâche n'en est pas entravée, mais portée au degré le plus haut d'intensité. Ce n'est pas vainement qu'on nomme attention religieuse la plénitude de l'attention. La plénitude de l'attention n'est pas autre chose que la prière.\par
Il en est de même pour la séparation de l'âme et du Christ qui dessèche l'âme comme se dessèche le sarment coupé du cep. La taille de la vigne dure des jours et des jours, dans les grands domaines. Mais aussi il y a là une vérité qu'on peut regarder des jours et des jours sans l'épuiser.\par
Il serait facile de découvrir, inscrits de toute éternité dans la nature des choses, beaucoup d'autres symboles capables de transfigurer non pas seulement le travail en général, mais chaque tâche dans sa singularité. Le Christ est le serpent d'airain qu'il suffit de regarder pour échapper à la mort. Mais il faut pouvoir le regarder d'une manière tout à fait ininterrompue. Pour cela il faut que les choses sur lesquelles les besoins et les obligations de la vie contraignent à porter le regard reflètent ce qu'elles nous empêchent de regarder directement. Il serait bien étonnant qu'une église construite de main d'homme fût pleine de symboles et que l'univers n'en fût pas infiniment plein. Il en est infiniment plein. Il faut les lire.\par
L'image de la Croix comparée à une balance, dans l'hymne du vendredi saint, pourrait être une inspiration inépuisable pour ceux qui portent des fardeaux, manient des leviers, sont fatigués le soir par la pesanteur des choses. Dans une balance un poids considérable et proche du point d'appui peut être soulevé par un poids très faible placé à une très grande distance. Le corps du Christ était un poids bien faible, mais par la distance entre la terre et le ciel il a fait contrepoids à l'univers. D'une manière infiniment différente, mais assez analogue pour servir d'image, quiconque travaille, soulève des fardeaux, manie des leviers doit aussi de son faible corps faire contrepoids à l'univers. Cela est trop lourd, et souvent l'univers fait plier le corps et l'âme sous la lassitude. Mais celui qui s'accroche au ciel fera facilement contrepoids. Celui qui a une fois aperçu cette pensée ne peut pas en être distrait par la fatigue, l'ennui et le dégoût. Il ne peut qu'y être ramené.\par
Le soleil et la sève végétale parlent continuellement, dans les champs, de ce qu'il y a de plus grand au monde. Nous ne vivons pas d'autre chose que d'énergie solaire ; nous, la mangeons, et c'est elle qui nous maintient debout, qui fait mouvoir nos muscles, qui corporellement opère en nous tous nos actes. Elle est peut-être, sous des formes diverses, la seule chose dans l'univers qui constitue une force antagoniste à la pesanteur ; c'est elle qui monte dans les arbres, qui par nos bras soulève des fardeaux, qui meut nos moteurs. Elle procède d'une source inaccessible et dont nous ne pouvons pas nous rapprocher même d'un pas. Elle descend continuellement sur nous. Mais quoiqu'elle nous baigne perpétuellement nous ne pouvons pas la capter. Seul le principe végétal de la chlorophylle peut la capter pour nous et en faire notre nourriture. Il faut seulement que la terre soit convenablement aménagée par nos efforts ; alors, par la chlorophylle, l'énergie solaire devient chose solide et entre en nous comme pain, comme vin, comme huile, comme fruits. Tout le travail du paysan consiste à soigner et à servir cette vertu végétale qui est une parfaite image du Christ.\par
Les lois de la mécanique, qui dérivent de la géométrie et qui commandent à nos machines, contiennent des vérités surnaturelles. L'oscillation du mouvement alternatif est l'image de la condition terrestre. Tout ce qui appartient aux créatures est limité, excepté le désir en nous qui est la marque de notre origine ; et nos convoitises, qui nous font chercher l'illimité ici-bas, sont par là pour nous l'unique source d'erreur et de crime. Les biens que contiennent les choses sont finis, les maux aussi, et d'une manière générale une cause ne produit un effet déterminé que jusqu'à un certain point, au-delà duquel, si elle continue à agir, l'effet se retourne. C'est Dieu qui impose à toute chose une limite et par qui la mer est enchaînée. En Dieu il n'y a qu'un acte éternel et sans changement qui se boucle sur soi et n'a d'autre objet que soi. Dans les créatures il n'y a que des mouvements dirigés vers le dehors, mais qui par la limite sont contraints d'osciller ; cette oscillation est un reflet dégradé de l'orientation vers soi-même qui est exclusivement divine. Cette liaison a pour image dans nos machines la liaison du mouvement circulaire et du mouvement alternatif. Le cercle est aussi le lieu des moyennes proportionnelles ; pour trouver d'une manière parfaitement rigoureuse la moyenne proportionnelle entre l'unité et un nombre qui n'est pas un carré, il n'y a pas d'autre méthode que de tracer un cercle. Les nombres pour lesquels il n'existe aucune médiation qui les relie naturellement à l'unité sont des images de notre misère ; et le cercle qui vient du dehors, d'une manière transcendante par rapport au domaine des nombres, apporter une médiation est l'image de l'unique remède à cette misère. Ces vérités et beaucoup d'autres sont écrites dans le simple spectacle d'une poulie qui détermine un mouvement oscillant ; celles-là peuvent être lues au moyen de connaissances géométriques très élémentaires ; le rythme même du travail, qui correspond à l'oscillation, les rend sensibles au corps ; une vie humaine est un délai bien court pour les contempler.\par
On pourrait trouver bien d'autres symboles, quelques-uns plus intimement unis au comportement même de celui qui travaille. Parfois il suffirait au travailleur d'étendre à toutes les choses sans exception son attitude à l'égard du travail pour posséder la plénitude de la vertu. Il y a aussi des symboles à trouver pour ceux qui ont des besognes d'exécution autres que le travail physique. On peut en trouver pour les comptables dans les opérations élémentaires de l'arithmétique, pour les caissiers dans l'institution de la monnaie, et ainsi de suite. Le réservoir est inépuisable.\par
À partir de là on pourrait faire beaucoup. Transmettre aux adolescents ces grandes images, liées à des notions de science élémentaire et de culture générale, dans des cercles d'études. Les proposer comme thèmes pour leurs fêtes, pour leurs tentatives théâtrales. Instituer autour d'elles des fêtes nouvelles, par exemple la veille du grand jour où un petit paysan de quatorze ans laboure seul pour la première fois. Faire par leur moyen que les hommes et les femmes du peuple vivent perpétuellement baignés dans une atmosphère de poésie surnaturelle ; comme au moyen âge ; plus qu'au moyen âge ; car pourquoi se limiter dans l'ambition du bien ?\par
On leur éviterait ainsi le sentiment d'infériorité intellectuelle si fréquent et parfois si douloureux, et aussi l'assurance orgueilleuse qui s'y substitue quelquefois après un léger contact avec les choses de l'esprit. Les intellectuels, de leur côté, pourraient ainsi éviter à la fois le dédain injuste et l'espèce de déférence non moins injuste que la démagogie avait mise à la mode, il y a quelques années, dans certains milieux. Les uns et les autres se rejoindraient, sans aucune inégalité, au point le plus haut, celui de la plénitude de l'attention, qui est la plénitude de la prière. Du moins ceux qui le pourraient. Les autres sauraient du moins que ce point existe et se représenteraient la diversité des chemins ascendants, laquelle, tout en produisant une séparation aux niveaux inférieurs, comme fait l'épaisseur d'une montagne, n'empêche pas l'égalité.\par
Les exercices scolaires n'ont pas d'autre destination sérieuse que la formation de l'attention. L'attention est la seule faculté de l'âme qui donne accès à Dieu. La gymnastique scolaire exerce une attention inférieure discursive, celle qui raisonne ; mais, menée avec une méthode convenable, elle peut préparer l'apparition dans l'âme d'une autre attention, celle qui est la plus haute, l'attention intuitive. L'attention intuitive dans sa pureté est l'unique source de l'art parfaitement beau, des découvertes scientifiques vraiment lumineuses et neuves, de la philosophie qui va vraiment vers la sagesse, de l'amour du prochain vraiment secourable ; et c'est elle qui, tournée directement vers Dieu, constitue la vraie prière.\par
De même qu'une symbolique permettrait de bêcher et de faucher en pensant à Dieu, de même une méthode transformant les exercices scolaires en préparation pour cette espèce supérieure d'attention permettrait seule à un adolescent de penser à Dieu pendant qu'il s'applique à un problème de géométrie ou à une version latine. Faute de quoi le travail intellectuel, sous un masque de liberté, est lui aussi un travail servile.\par
Ceux qui ont des loisirs ont besoin, pour parvenir à l'attention intuitive, d'exercer jusqu'à la limite de leur capacité les facultés de l'intelligence discursive ; autrement elles font obstacle. Surtout pour ceux que leur fonction sociale oblige à faire jouer ces facultés, il n'est pas sans doute d'autre chemin. Mais l'obstacle est faible et l'exercice peut se réduire à peu de chose pour ceux chez qui la fatigue d'un long travail quotidien paralyse presque entièrement ces facultés. Pour eux le travail même qui produit cette paralysie, pourvu qu'il soit transformé en poésie, est le chemin qui mène à l'attention intuitive.\par
Dans notre société la différence d'instruction produit, plus que la différence de richesse, l'illusion de l'inégalité sociale. Marx, qui est presque toujours très fort quand il décrit simplement le mal, a légitimement flétri comme une dégradation la séparation du travail manuel et du travail intellectuel. Mais il ne savait pas qu'en tout domaine les contraires ont leur unité dans un plan transcendant par rapport à l'un et à l'autre. Le point d'unité du travail intellectuel et du travail manuel, c'est la contemplation, qui n'est pas un travail. Dans aucune société celui qui manie une machine ne peut exercer la même espèce d'attention que celui qui résout un problème. Mais l'un et l'autre peuvent, également s'ils le désirent et s'ils ont une méthode, en exerçant chacun l'espèce d'attention qui constitue son lot propre dans la société, favoriser l'apparition et le développement d'une autre attention située au-dessus de toute obligation sociale, et qui constitue un lien direct avec Dieu.\par
Si les étudiants, les jeunes paysans, les jeunes ouvriers se représentaient d'une manière tout à fait précise, aussi précise que les rouages d'un mécanisme clairement compris, les différentes fonctions sociales comme constituant des préparations également efficaces pour l'apparition dans l'âme d'une même faculté transcendante, qui a seule une valeur, l'égalité deviendrait une chose concrète, Elle serait alors à la fois un principe de justice et d'ordre.\par
La représentation tout à fait précise de la destination surnaturelle de chaque fonction sociale fournit seule une norme à la volonté de réforme. Elle permet seule de définir l'injustice. Autrement il est inévitable qu'on se trompe soit en regardant comme des injustices des souffrances inscrites dans la nature des choses, soit en attribuant à la condition humaine des souffrances qui sont des effets de nos crimes et tombent sur ceux qui ne les méritent pas.\par
Une certaine subordination et une certaine uniformité sont des souffrances inscrites dans l'essence même du travail et inséparables de la vocation surnaturelle qui y correspond. Elles ne dégradent pas. Tout ce qui s'y ajoute est injuste et dégrade. Tout ce qui empêche pas de retrouver la source perdue d'une telle poésie, il faut encore que les circonstances mêmes du travail lui permettent d'exister. Si elles sont mauvaises, elles la tuent.\par
Tout ce qui est indissolublement lié au désir ou à la crainte d'un changement, à l'orientation de la pensée vers l'avenir, serait à exclure d'une existence essentiellement, uniforme et qui doit être acceptée comme telle. En premier lieu la souffrance physique, hors celle qui est rendue manifestement inévitable par les nécessités du travail. Car il est impossible de souffrir sans aspirer au soulagement. Les privations seraient mieux à leur place dans toute autre condition sociale que dans celle-là. La nourriture, le logement, le repos et le loisir doivent être tels qu'une journée de travail prise en elle-même soit normalement vide de souffrance physique. D'autre part le superflu non plus n'est pas à sa place dans cette vie ; car le désir du superflu est par lui-même illimité et implique celui d'un changement de condition. Toute la publicité, toute la propagande, si variée dans ses formes, qui cherche à exciter le désir du superflu dans les campagnes et parmi les ouvriers doit être regardée comme un crime. Un individu peut toujours sortir de la condition ouvrière ou paysanne, soit par manque radical d'aptitude professionnelle, soit par la possession d'aptitudes différentes ; mais pour ceux qui y sont, il ne devrait y avoir de changement possible que d'un bien-être étroitement borné à un bien-être large ; il ne devrait y avoir aucune occasion pour eux de craindre tomber à moins ou d'espérer parvenir à plus. La sécurité devrait être plus grande dans cette condition sociale que dans toute autre. Il ne faut donc pas que les hasards de l'offre et de la demande en soient maîtres.\par
L'arbitraire humain contraint l'âme, sans qu'elle puisse s'en défendre, à craindre et à espérer. Il faut donc qu'il soit exclu du travail autant qu'il est possible. L'autorité ne doit y être présente que là où il est tout à fait impossible qu'elle soit absente. Ainsi la petite propriété paysanne vaut mieux que la grande. Dès lors, partout où la petite est possible, la grande est un mal. De même la fabrication de pièces usinées dans un petit atelier d'artisan vaut mieux que celle qui se fait sous les ordres d'un contremaître. Job loue la mort de ce que l'esclave n'y entend plus la voix de son maître. Toutes les fois que la voix qui commande se fait entendre alors qu'un arrangement praticable pourrait y substituer le silence, c'est un mal.\par
Mais le pire attentat, celui qui mériterait peut-être d'être assimilé au crime contre l'Esprit, qui est sans pardon, s'il n’était probablement commis par des inconscients, c'est l'attentat contre l'attention des travailleurs. Il tue dans l'âme la faculté qui y constitue la racine même de toute vocation surnaturelle. La basse espèce d'attention exigée par le travail taylorisé n'est compatible avec aucune autre, parce qu'elle vide l'âme de tout ce qui n'est pas le souci de la vitesse. Ce genre de travail ne peut pas être transfiguré, il faut le supprimer.\par
Tous les problèmes de la technique et de l'économie doivent être formulés en fonction d'une conception de la meilleure condition possible du travail. Une telle conception est la première des normes ; toute la société doit être constituée d'abord de telle manière que le travail ne tire pas vers en bas ceux qui l'exécutent.\par
Il ne suffit pas de vouloir leur éviter des souffrances, il faudrait vouloir leur joie. Non pas des plaisirs qui se paient, mais des joies gratuites qui ne portent pas atteinte à l'esprit de pauvreté. La poésie surnaturelle qui devrait baigner toute leur vie devrait aussi être concentrée à l'état pur, de temps à autre, dans des fêtes éclatantes. Les fêtes sont aussi indispensables à cette existence que les bornes kilométriques au réconfort du marcheur. Des voyages gratuits et laborieux, semblables au Tour de France d'autrefois, devraient dans leur jeunesse rassasier leur faim de voir et d'apprendre. Tout devrait être disposé pour que rien d'essentiel ne leur manque. Les meilleurs d'entre eux doivent pouvoir posséder dans leur vie elle-même la plénitude que les artistes cherchent indirectement par l'intermédiaire de leur art. Si la vocation de l'homme est d'atteindre la joie pure à travers la souffrance, ils sont placés mieux que tous les autres pour l'accomplir de la manière la plus réelle.
 


% at least one empty page at end (for booklet couv)
\ifbooklet
  \newpage\null\thispagestyle{empty}\newpage
\fi

\ifdev % autotext in dev mode
\fontname\font — \textsc{Les règles du jeu}\par
(\hyperref[utopie]{\underline{Lien}})\par
\noindent \initialiv{A}{lors là}\blindtext\par
\noindent \initialiv{À}{ la bonheur des dames}\blindtext\par
\noindent \initialiv{É}{tonnez-le}\blindtext\par
\noindent \initialiv{Q}{ualitativement}\blindtext\par
\noindent \initialiv{V}{aloriser}\blindtext\par
\Blindtext
\phantomsection
\label{utopie}
\Blinddocument
\fi
\end{document}
