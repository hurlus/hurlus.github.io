%%%%%%%%%%%%%%%%%%%%%%%%%%%%%%%%%
% LaTeX model https://hurlus.fr %
%%%%%%%%%%%%%%%%%%%%%%%%%%%%%%%%%

% Needed before document class
\RequirePackage{pdftexcmds} % needed for tests expressions
\RequirePackage{fix-cm} % correct units

% Define mode
\def\mode{a4}

\newif\ifaiv % a4
\newif\ifav % a5
\newif\ifbooklet % booklet
\newif\ifcover % cover for booklet

\ifnum \strcmp{\mode}{cover}=0
  \covertrue
\else\ifnum \strcmp{\mode}{booklet}=0
  \booklettrue
\else\ifnum \strcmp{\mode}{a5}=0
  \avtrue
\else
  \aivtrue
\fi\fi\fi

\ifbooklet % do not enclose with {}
  \documentclass[french,twoside]{book} % ,notitlepage
  \usepackage[%
    papersize={105mm, 297mm},
    inner=12mm,
    outer=12mm,
    top=20mm,
    bottom=15mm,
    marginparsep=0pt,
  ]{geometry}
  \usepackage[fontsize=9.5pt]{scrextend} % for Roboto
\else\ifav
  \documentclass[french,twoside]{book} % ,notitlepage
  \usepackage[%
    a5paper,
    inner=25mm,
    outer=15mm,
    top=15mm,
    bottom=15mm,
    marginparsep=0pt,
  ]{geometry}
  \usepackage[fontsize=12pt]{scrextend}
\else% A4 2 cols
  \documentclass[twocolumn]{report}
  \usepackage[%
    a4paper,
    inner=15mm,
    outer=10mm,
    top=25mm,
    bottom=18mm,
    marginparsep=0pt,
  ]{geometry}
  \setlength{\columnsep}{20mm}
  \usepackage[fontsize=9.5pt]{scrextend}
\fi\fi

%%%%%%%%%%%%%%
% Alignments %
%%%%%%%%%%%%%%
% before teinte macros

\setlength{\arrayrulewidth}{0.2pt}
\setlength{\columnseprule}{\arrayrulewidth} % twocol
\setlength{\parskip}{0pt} % classical para with no margin
\setlength{\parindent}{1.5em}

%%%%%%%%%%
% Colors %
%%%%%%%%%%
% before Teinte macros

\usepackage[dvipsnames]{xcolor}
\definecolor{rubric}{HTML}{800000} % the tonic 0c71c3
\def\columnseprulecolor{\color{rubric}}
\colorlet{borderline}{rubric!30!} % definecolor need exact code
\definecolor{shadecolor}{gray}{0.95}
\definecolor{bghi}{gray}{0.5}

%%%%%%%%%%%%%%%%%
% Teinte macros %
%%%%%%%%%%%%%%%%%
%%%%%%%%%%%%%%%%%%%%%%%%%%%%%%%%%%%%%%%%%%%%%%%%%%%
% <TEI> generic (LaTeX names generated by Teinte) %
%%%%%%%%%%%%%%%%%%%%%%%%%%%%%%%%%%%%%%%%%%%%%%%%%%%
% This template is inserted in a specific design
% It is XeLaTeX and otf fonts

\makeatletter % <@@@


\usepackage{blindtext} % generate text for testing
\usepackage[strict]{changepage} % for modulo 4
\usepackage{contour} % rounding words
\usepackage[nodayofweek]{datetime}
% \usepackage{DejaVuSans} % seems buggy for sffont font for symbols
\usepackage{enumitem} % <list>
\usepackage{etoolbox} % patch commands
\usepackage{fancyvrb}
\usepackage{fancyhdr}
\usepackage{float}
\usepackage{fontspec} % XeLaTeX mandatory for fonts
\usepackage{footnote} % used to capture notes in minipage (ex: quote)
\usepackage{framed} % bordering correct with footnote hack
\usepackage{graphicx}
\usepackage{lettrine} % drop caps
\usepackage{lipsum} % generate text for testing
\usepackage[framemethod=tikz,]{mdframed} % maybe used for frame with footnotes inside
\usepackage{pdftexcmds} % needed for tests expressions
\usepackage{polyglossia} % non-break space french punct, bug Warning: "Failed to patch part"
\usepackage[%
  indentfirst=false,
  vskip=1em,
  noorphanfirst=true,
  noorphanafter=true,
  leftmargin=\parindent,
  rightmargin=0pt,
]{quoting}
\usepackage{ragged2e}
\usepackage{setspace} % \setstretch for <quote>
\usepackage{tabularx} % <table>
\usepackage[explicit]{titlesec} % wear titles, !NO implicit
\usepackage{tikz} % ornaments
\usepackage{tocloft} % styling tocs
\usepackage[fit]{truncate} % used im runing titles
\usepackage{unicode-math}
\usepackage[normalem]{ulem} % breakable \uline, normalem is absolutely necessary to keep \emph
\usepackage{verse} % <l>
\usepackage{xcolor} % named colors
\usepackage{xparse} % @ifundefined
\XeTeXdefaultencoding "iso-8859-1" % bad encoding of xstring
\usepackage{xstring} % string tests
\XeTeXdefaultencoding "utf-8"
\PassOptionsToPackage{hyphens}{url} % before hyperref, which load url package

% TOTEST
% \usepackage{hypcap} % links in caption ?
% \usepackage{marginnote}
% TESTED
% \usepackage{background} % doesn’t work with xetek
% \usepackage{bookmark} % prefers the hyperref hack \phantomsection
% \usepackage[color, leftbars]{changebar} % 2 cols doc, impossible to keep bar left
% \usepackage[utf8x]{inputenc} % inputenc package ignored with utf8 based engines
% \usepackage[sfdefault,medium]{inter} % no small caps
% \usepackage{firamath} % choose firasans instead, firamath unavailable in Ubuntu 21-04
% \usepackage{flushend} % bad for last notes, supposed flush end of columns
% \usepackage[stable]{footmisc} % BAD for complex notes https://texfaq.org/FAQ-ftnsect
% \usepackage{helvet} % not for XeLaTeX
% \usepackage{multicol} % not compatible with too much packages (longtable, framed, memoir…)
% \usepackage[default,oldstyle,scale=0.95]{opensans} % no small caps
% \usepackage{sectsty} % \chapterfont OBSOLETE
% \usepackage{soul} % \ul for underline, OBSOLETE with XeTeX
% \usepackage[breakable]{tcolorbox} % text styling gone, footnote hack not kept with breakable


% Metadata inserted by a program, from the TEI source, for title page and runing heads
\title{\textbf{ Instructions pour une prise d’armes }}
\date{1866}
\author{Blanqui}
\def\elbibl{Blanqui. 1866. \emph{Instructions pour une prise d’armes}}
\def\elsource{ \href{https://www.marxists.org/francais/blanqui/1866/instructions.htm}{\dotuline{marxists.org}}\footnote{\href{https://www.marxists.org/francais/blanqui/1866/instructions.htm}{\url{https://www.marxists.org/francais/blanqui/1866/instructions.htm}}} \emph{Auguste Blanqui. Instructions pour une prise d’armes. L'Éternité par les astres, hypothèse astronomique et autres textes}, Société encyclopédique française, Editions de la Tête de Feuilles. 1972. Transcrit par Andy Blunden.}

% Default metas
\newcommand{\colorprovide}[2]{\@ifundefinedcolor{#1}{\colorlet{#1}{#2}}{}}
\colorprovide{rubric}{red}
\colorprovide{silver}{lightgray}
\@ifundefined{syms}{\newfontfamily\syms{DejaVu Sans}}{}
\newif\ifdev
\@ifundefined{elbibl}{% No meta defined, maybe dev mode
  \newcommand{\elbibl}{Titre court ?}
  \newcommand{\elbook}{Titre du livre source ?}
  \newcommand{\elabstract}{Résumé\par}
  \newcommand{\elurl}{http://oeuvres.github.io/elbook/2}
  \author{Éric Lœchien}
  \title{Un titre de test assez long pour vérifier le comportement d’une maquette}
  \date{1566}
  \devtrue
}{}
\let\eltitle\@title
\let\elauthor\@author
\let\eldate\@date


\defaultfontfeatures{
  % Mapping=tex-text, % no effect seen
  Scale=MatchLowercase,
  Ligatures={TeX,Common},
}


% generic typo commands
\newcommand{\astermono}{\medskip\centerline{\color{rubric}\large\selectfont{\syms ✻}}\medskip\par}%
\newcommand{\astertri}{\medskip\par\centerline{\color{rubric}\large\selectfont{\syms ✻\,✻\,✻}}\medskip\par}%
\newcommand{\asterism}{\bigskip\par\noindent\parbox{\linewidth}{\centering\color{rubric}\large{\syms ✻}\\{\syms ✻}\hskip 0.75em{\syms ✻}}\bigskip\par}%

% lists
\newlength{\listmod}
\setlength{\listmod}{\parindent}
\setlist{
  itemindent=!,
  listparindent=\listmod,
  labelsep=0.2\listmod,
  parsep=0pt,
  % topsep=0.2em, % default topsep is best
}
\setlist[itemize]{
  label=—,
  leftmargin=0pt,
  labelindent=1.2em,
  labelwidth=0pt,
}
\setlist[enumerate]{
  label={\bf\color{rubric}\arabic*.},
  labelindent=0.8\listmod,
  leftmargin=\listmod,
  labelwidth=0pt,
}
\newlist{listalpha}{enumerate}{1}
\setlist[listalpha]{
  label={\bf\color{rubric}\alph*.},
  leftmargin=0pt,
  labelindent=0.8\listmod,
  labelwidth=0pt,
}
\newcommand{\listhead}[1]{\hspace{-1\listmod}\emph{#1}}

\renewcommand{\hrulefill}{%
  \leavevmode\leaders\hrule height 0.2pt\hfill\kern\z@}

% General typo
\DeclareTextFontCommand{\textlarge}{\large}
\DeclareTextFontCommand{\textsmall}{\small}

% commands, inlines
\newcommand{\anchor}[1]{\Hy@raisedlink{\hypertarget{#1}{}}} % link to top of an anchor (not baseline)
\newcommand\abbr[1]{#1}
\newcommand{\autour}[1]{\tikz[baseline=(X.base)]\node [draw=rubric,thin,rectangle,inner sep=1.5pt, rounded corners=3pt] (X) {\color{rubric}#1};}
\newcommand\corr[1]{#1}
\newcommand{\ed}[1]{ {\color{silver}\sffamily\footnotesize (#1)} } % <milestone ed="1688"/>
\newcommand\expan[1]{#1}
\newcommand\foreign[1]{\emph{#1}}
\newcommand\gap[1]{#1}
\renewcommand{\LettrineFontHook}{\color{rubric}}
\newcommand{\initial}[2]{\lettrine[lines=2, loversize=0.3, lhang=0.3]{#1}{#2}}
\newcommand{\initialiv}[2]{%
  \let\oldLFH\LettrineFontHook
  % \renewcommand{\LettrineFontHook}{\color{rubric}\ttfamily}
  \IfSubStr{QJ’}{#1}{
    \lettrine[lines=4, lhang=0.2, loversize=-0.1, lraise=0.2]{\smash{#1}}{#2}
  }{\IfSubStr{É}{#1}{
    \lettrine[lines=4, lhang=0.2, loversize=-0, lraise=0]{\smash{#1}}{#2}
  }{\IfSubStr{ÀÂ}{#1}{
    \lettrine[lines=4, lhang=0.2, loversize=-0, lraise=0, slope=0.6em]{\smash{#1}}{#2}
  }{\IfSubStr{A}{#1}{
    \lettrine[lines=4, lhang=0.2, loversize=0.2, slope=0.6em]{\smash{#1}}{#2}
  }{\IfSubStr{V}{#1}{
    \lettrine[lines=4, lhang=0.2, loversize=0.2, slope=-0.5em]{\smash{#1}}{#2}
  }{
    \lettrine[lines=4, lhang=0.2, loversize=0.2]{\smash{#1}}{#2}
  }}}}}
  \let\LettrineFontHook\oldLFH
}
\newcommand{\labelchar}[1]{\textbf{\color{rubric} #1}}
\newcommand{\milestone}[1]{\autour{\footnotesize\color{rubric} #1}} % <milestone n="4"/>
\newcommand\name[1]{#1}
\newcommand\orig[1]{#1}
\newcommand\orgName[1]{#1}
\newcommand\persName[1]{#1}
\newcommand\placeName[1]{#1}
\newcommand{\pn}[1]{\IfSubStr{-—–¶}{#1}% <p n="3"/>
  {\noindent{\bfseries\color{rubric}   ¶  }}
  {{\footnotesize\autour{ #1}  }}}
\newcommand\reg{}
% \newcommand\ref{} % already defined
\newcommand\sic[1]{#1}
\newcommand\surname[1]{\textsc{#1}}
\newcommand\term[1]{\textbf{#1}}

\def\mednobreak{\ifdim\lastskip<\medskipamount
  \removelastskip\nopagebreak\medskip\fi}
\def\bignobreak{\ifdim\lastskip<\bigskipamount
  \removelastskip\nopagebreak\bigskip\fi}

% commands, blocks
\newcommand{\byline}[1]{\bigskip{\RaggedLeft{#1}\par}\bigskip}
\newcommand{\bibl}[1]{{\RaggedLeft{#1}\par\bigskip}}
\newcommand{\biblitem}[1]{{\noindent\hangindent=\parindent   #1\par}}
\newcommand{\dateline}[1]{\medskip{\RaggedLeft{#1}\par}\bigskip}
\newcommand{\labelblock}[1]{\medbreak{\noindent\color{rubric}\bfseries #1}\par\mednobreak}
\newcommand{\salute}[1]{\bigbreak{#1}\par\medbreak}
\newcommand{\signed}[1]{\bigbreak\filbreak{\raggedleft #1\par}\medskip}

% environments for blocks (some may become commands)
\newenvironment{borderbox}{}{} % framing content
\newenvironment{citbibl}{\ifvmode\hfill\fi}{\ifvmode\par\fi }
\newenvironment{docAuthor}{\ifvmode\vskip4pt\fontsize{16pt}{18pt}\selectfont\fi\itshape}{\ifvmode\par\fi }
\newenvironment{docDate}{}{\ifvmode\par\fi }
\newenvironment{docImprint}{\vskip6pt}{\ifvmode\par\fi }
\newenvironment{docTitle}{\vskip6pt\bfseries\fontsize{18pt}{22pt}\selectfont}{\par }
\newenvironment{msHead}{\vskip6pt}{\par}
\newenvironment{msItem}{\vskip6pt}{\par}
\newenvironment{titlePart}{}{\par }


% environments for block containers
\newenvironment{argument}{\itshape\parindent0pt}{\vskip1.5em}
\newenvironment{biblfree}{}{\ifvmode\par\fi }
\newenvironment{bibitemlist}[1]{%
  \list{\@biblabel{\@arabic\c@enumiv}}%
  {%
    \settowidth\labelwidth{\@biblabel{#1}}%
    \leftmargin\labelwidth
    \advance\leftmargin\labelsep
    \@openbib@code
    \usecounter{enumiv}%
    \let\p@enumiv\@empty
    \renewcommand\theenumiv{\@arabic\c@enumiv}%
  }
  \sloppy
  \clubpenalty4000
  \@clubpenalty \clubpenalty
  \widowpenalty4000%
  \sfcode`\.\@m
}%
{\def\@noitemerr
  {\@latex@warning{Empty `bibitemlist' environment}}%
\endlist}
\newenvironment{quoteblock}% may be used for ornaments
  {\begin{quoting}}
  {\end{quoting}}

% table () is preceded and finished by custom command
\newcommand{\tableopen}[1]{%
  \ifnum\strcmp{#1}{wide}=0{%
    \begin{center}
  }
  \else\ifnum\strcmp{#1}{long}=0{%
    \begin{center}
  }
  \else{%
    \begin{center}
  }
  \fi\fi
}
\newcommand{\tableclose}[1]{%
  \ifnum\strcmp{#1}{wide}=0{%
    \end{center}
  }
  \else\ifnum\strcmp{#1}{long}=0{%
    \end{center}
  }
  \else{%
    \end{center}
  }
  \fi\fi
}


% text structure
\newcommand\chapteropen{} % before chapter title
\newcommand\chaptercont{} % after title, argument, epigraph…
\newcommand\chapterclose{} % maybe useful for multicol settings
\setcounter{secnumdepth}{-2} % no counters for hierarchy titles
\setcounter{tocdepth}{5} % deep toc
\markright{\@title} % ???
\markboth{\@title}{\@author} % ???
\renewcommand\tableofcontents{\@starttoc{toc}}
% toclof format
% \renewcommand{\@tocrmarg}{0.1em} % Useless command?
% \renewcommand{\@pnumwidth}{0.5em} % {1.75em}
\renewcommand{\@cftmaketoctitle}{}
\setlength{\cftbeforesecskip}{\z@ \@plus.2\p@}
\renewcommand{\cftchapfont}{}
\renewcommand{\cftchapdotsep}{\cftdotsep}
\renewcommand{\cftchapleader}{\normalfont\cftdotfill{\cftchapdotsep}}
\renewcommand{\cftchappagefont}{\bfseries}
\setlength{\cftbeforechapskip}{0em \@plus\p@}
% \renewcommand{\cftsecfont}{\small\relax}
\renewcommand{\cftsecpagefont}{\normalfont}
% \renewcommand{\cftsubsecfont}{\small\relax}
\renewcommand{\cftsecdotsep}{\cftdotsep}
\renewcommand{\cftsecpagefont}{\normalfont}
\renewcommand{\cftsecleader}{\normalfont\cftdotfill{\cftsecdotsep}}
\setlength{\cftsecindent}{1em}
\setlength{\cftsubsecindent}{2em}
\setlength{\cftsubsubsecindent}{3em}
\setlength{\cftchapnumwidth}{1em}
\setlength{\cftsecnumwidth}{1em}
\setlength{\cftsubsecnumwidth}{1em}
\setlength{\cftsubsubsecnumwidth}{1em}

% footnotes
\newif\ifheading
\newcommand*{\fnmarkscale}{\ifheading 0.70 \else 1 \fi}
\renewcommand\footnoterule{\vspace*{0.3cm}\hrule height \arrayrulewidth width 3cm \vspace*{0.3cm}}
\setlength\footnotesep{1.5\footnotesep} % footnote separator
\renewcommand\@makefntext[1]{\parindent 1.5em \noindent \hb@xt@1.8em{\hss{\normalfont\@thefnmark . }}#1} % no superscipt in foot
\patchcmd{\@footnotetext}{\footnotesize}{\footnotesize\sffamily}{}{} % before scrextend, hyperref


%   see https://tex.stackexchange.com/a/34449/5049
\def\truncdiv#1#2{((#1-(#2-1)/2)/#2)}
\def\moduloop#1#2{(#1-\truncdiv{#1}{#2}*#2)}
\def\modulo#1#2{\number\numexpr\moduloop{#1}{#2}\relax}

% orphans and widows
\clubpenalty=9996
\widowpenalty=9999
\brokenpenalty=4991
\predisplaypenalty=10000
\postdisplaypenalty=1549
\displaywidowpenalty=1602
\hyphenpenalty=400
% Copied from Rahtz but not understood
\def\@pnumwidth{1.55em}
\def\@tocrmarg {2.55em}
\def\@dotsep{4.5}
\emergencystretch 3em
\hbadness=4000
\pretolerance=750
\tolerance=2000
\vbadness=4000
\def\Gin@extensions{.pdf,.png,.jpg,.mps,.tif}
% \renewcommand{\@cite}[1]{#1} % biblio

\usepackage{hyperref} % supposed to be the last one, :o) except for the ones to follow
\urlstyle{same} % after hyperref
\hypersetup{
  % pdftex, % no effect
  pdftitle={\elbibl},
  % pdfauthor={Your name here},
  % pdfsubject={Your subject here},
  % pdfkeywords={keyword1, keyword2},
  bookmarksnumbered=true,
  bookmarksopen=true,
  bookmarksopenlevel=1,
  pdfstartview=Fit,
  breaklinks=true, % avoid long links
  pdfpagemode=UseOutlines,    % pdf toc
  hyperfootnotes=true,
  colorlinks=false,
  pdfborder=0 0 0,
  % pdfpagelayout=TwoPageRight,
  % linktocpage=true, % NO, toc, link only on page no
}

\makeatother % /@@@>
%%%%%%%%%%%%%%
% </TEI> end %
%%%%%%%%%%%%%%


%%%%%%%%%%%%%
% footnotes %
%%%%%%%%%%%%%
\renewcommand{\thefootnote}{\bfseries\textcolor{rubric}{\arabic{footnote}}} % color for footnote marks

%%%%%%%%%
% Fonts %
%%%%%%%%%
\usepackage[]{roboto} % SmallCaps, Regular is a bit bold
% \linespread{0.90} % too compact, keep font natural
\newfontfamily\fontrun[]{Roboto Condensed Light} % condensed runing heads
\ifav
  \setmainfont[
    ItalicFont={Roboto Light Italic},
  ]{Roboto}
\else\ifbooklet
  \setmainfont[
    ItalicFont={Roboto Light Italic},
  ]{Roboto}
\else
\setmainfont[
  ItalicFont={Roboto Italic},
]{Roboto Light}
\fi\fi
\renewcommand{\LettrineFontHook}{\bfseries\color{rubric}}
% \renewenvironment{labelblock}{\begin{center}\bfseries\color{rubric}}{\end{center}}

%%%%%%%%
% MISC %
%%%%%%%%

\setdefaultlanguage[frenchpart=false]{french} % bug on part


\newenvironment{quotebar}{%
    \def\FrameCommand{{\color{rubric!10!}\vrule width 0.5em} \hspace{0.9em}}%
    \def\OuterFrameSep{\itemsep} % séparateur vertical
    \MakeFramed {\advance\hsize-\width \FrameRestore}
  }%
  {%
    \endMakeFramed
  }
\renewenvironment{quoteblock}% may be used for ornaments
  {%
    \savenotes
    \setstretch{0.9}
    \normalfont
    \begin{quotebar}
  }
  {%
    \end{quotebar}
    \spewnotes
  }


\renewcommand{\headrulewidth}{\arrayrulewidth}
\renewcommand{\headrule}{{\color{rubric}\hrule}}

% delicate tuning, image has produce line-height problems in title on 2 lines
\titleformat{name=\chapter} % command
  [display] % shape
  {\vspace{1.5em}\centering} % format
  {} % label
  {0pt} % separator between n
  {}
[{\color{rubric}\huge\textbf{#1}}\bigskip] % after code
% \titlespacing{command}{left spacing}{before spacing}{after spacing}[right]
\titlespacing*{\chapter}{0pt}{-2em}{0pt}[0pt]

\titleformat{name=\section}
  [block]{}{}{}{}
  [\vbox{\color{rubric}\large\raggedleft\textbf{#1}}]
\titlespacing{\section}{0pt}{0pt plus 4pt minus 2pt}{\baselineskip}

\titleformat{name=\subsection}
  [block]
  {}
  {} % \thesection
  {} % separator \arrayrulewidth
  {}
[\vbox{\large\textbf{#1}}]
% \titlespacing{\subsection}{0pt}{0pt plus 4pt minus 2pt}{\baselineskip}

\ifaiv
  \fancypagestyle{main}{%
    \fancyhf{}
    \setlength{\headheight}{1.5em}
    \fancyhead{} % reset head
    \fancyfoot{} % reset foot
    \fancyhead[L]{\truncate{0.45\headwidth}{\fontrun\elbibl}} % book ref
    \fancyhead[R]{\truncate{0.45\headwidth}{ \fontrun\nouppercase\leftmark}} % Chapter title
    \fancyhead[C]{\thepage}
  }
  \fancypagestyle{plain}{% apply to chapter
    \fancyhf{}% clear all header and footer fields
    \setlength{\headheight}{1.5em}
    \fancyhead[L]{\truncate{0.9\headwidth}{\fontrun\elbibl}}
    \fancyhead[R]{\thepage}
  }
\else
  \fancypagestyle{main}{%
    \fancyhf{}
    \setlength{\headheight}{1.5em}
    \fancyhead{} % reset head
    \fancyfoot{} % reset foot
    \fancyhead[RE]{\truncate{0.9\headwidth}{\fontrun\elbibl}} % book ref
    \fancyhead[LO]{\truncate{0.9\headwidth}{\fontrun\nouppercase\leftmark}} % Chapter title, \nouppercase needed
    \fancyhead[RO,LE]{\thepage}
  }
  \fancypagestyle{plain}{% apply to chapter
    \fancyhf{}% clear all header and footer fields
    \setlength{\headheight}{1.5em}
    \fancyhead[L]{\truncate{0.9\headwidth}{\fontrun\elbibl}}
    \fancyhead[R]{\thepage}
  }
\fi

\ifav % a5 only
  \titleclass{\section}{top}
\fi

\newcommand\chapo{{%
  \vspace*{-3em}
  \centering % no vskip ()
  {\Large\addfontfeature{LetterSpace=25}\bfseries{\elauthor}}\par
  \smallskip
  {\large\eldate}\par
  \bigskip
  {\Large\selectfont{\eltitle}}\par
  \bigskip
  {\color{rubric}\hline\par}
  \bigskip
  {\Large TEXTE LIBRE À PARTICPATION LIBRE\par}
  \centerline{\small\color{rubric} {hurlus.fr, tiré le \today}}\par
  \bigskip
}}

\newcommand\cover{{%
  \thispagestyle{empty}
  \centering
  {\LARGE\bfseries{\elauthor}}\par
  \bigskip
  {\Large\eldate}\par
  \bigskip
  \bigskip
  {\LARGE\selectfont{\eltitle}}\par
  \vfill\null
  {\color{rubric}\setlength{\arrayrulewidth}{2pt}\hline\par}
  \vfill\null
  {\Large TEXTE LIBRE À PARTICPATION LIBRE\par}
  \centerline{{\href{https://hurlus.fr}{\dotuline{hurlus.fr}}, tiré le \today}}\par
}}

\begin{document}
\pagestyle{empty}
\ifbooklet{
  \cover\newpage
  \thispagestyle{empty}\hbox{}\newpage
  \cover\newpage\noindent Les voyages de la brochure\par
  \bigskip
  \begin{tabularx}{\textwidth}{l|X|X}
    \textbf{Date} & \textbf{Lieu}& \textbf{Nom/pseudo} \\ \hline
    \rule{0pt}{25cm} &  &   \\
  \end{tabularx}
  \newpage
  \addtocounter{page}{-4}
}\fi

\thispagestyle{empty}
\ifaiv
  \twocolumn[\chapo]
\else
  \chapo
\fi
{\it\elabstract}
\bigskip
\makeatletter\@starttoc{toc}\makeatother % toc without new page
\bigskip

\pagestyle{main} % after style

  \section[{Instructions pour une prise d’armes}]{Instructions pour une prise d’armes}\renewcommand{\leftmark}{Instructions pour une prise d’armes}

\noindent Ce programme est purement militaire et laisse entièrement de côté la question politique et sociale, dont ce n’est point ici la place : il va sans dire d’ailleurs, que la révolution doit se faire au profit du travail contre la tyrannie du capital, et reconstituer la société sur la base de la justice.\par
Une insurrection parisienne, d’après les vieux errements, n’a plus aujourd’hui aucune chance de succès.\par
En 1830, le seul élan populaire a pu suffire à jeter bas un pouvoir surpris et terrifié par une prise d’armes, événement inouï, qui était à mille lieux de ses prévisions.\par
Cela était bon une fois. La leçon a profité au gouvernement, resté monarchique et contre-révolutionnaire, bien que sorti d’une Révolution. Il s’est mis à étudier la guerre des rues, et il y a repris bientôt la supériorité naturelle de l’art et de la discipline sur l’inexpérience et la confusion.\par
Cependant, dira-t-on, le peuple en 1848, a vaincu par la méthode de 1830. Soit. Mais point d’illusions ! La victoire de février n’est qu’un raccroc. Si {\scshape Louis-Philippe} s’était sérieusement défendu, force serait restée aux uniformes.\par
A preuve les journées de juin. C'est là qu’on a pu voir combien est funeste la tactique, ou plutôt l’absence de tactique de l’insurrection. Jamais elle n’avait eu la partie aussi belle : dix chances contre une.\par
D'un côté, le Gouvernement en pleine anarchie, les troupes demoralisées : de l’autre, tous les travailleurs debout et presque certains du succès. Comment ont-ils succombé ? Par défaut d’organisation. Pour se rendre compte de leur défaite, il suffit d’analyser leur stratégie.\par
Le soulèvement éclate. Aussitôt, dans les quartiers du travail, les barricades s’élèvent ça et là, à l’aventure, sur une multitude de points.\par
Cinq, dix, vingt, trente, cinquante hommes, réunis par hasard, la plupart sans armes, commencent à renverser des voitures, lèvent et entassent des pavés pour barrer la voie publique, tantôt au milieu des rues, plus souvent à leur intersection. Quantité de ces barrages seraient à peine un obstacle au passage de la cavalerie.\par
Parfois, après une grossière ébauche de retranchement, les constructeurs s’éloignent pour aller à la recherche de fusils et de munitions.\par
En juin, on a compté plus de six cents barricades, une trentaine au plus ont fait à elles seules tous les frais de la bataille. Les autres, dix-neuf sur vingt, n’ont pas brûlé une amorce. De là, ces glorieux bulletins qui racontaient avec fracas l’enlèvement de cinquante barricades, où il ne se trouvait pas une âme.\par
Tandis qu’on dépave ainsi les rues, d’autres petites bandes vont désarmer les corps de garde ou saisir la poudre et les armes chez les arquebusiers. Tout cela se fait, sans concert ni direction, au gré de la fantaisie individuelle.\par
Peu à peu, cependant, un certain nombre de barricades, plus hautes, plus fortes, mieux construites, attirent de préférence les défenseurs qui s’y concentrent. Ce n’est point le calcul, mais le hasard qui détermine l’emplacement de ces fortifications principales. Quelques-unes seulement, par une sorte d’inspiration militaire assez concevable, occupent les grands débouchés.\par
Durant cette première période de l’insurrection, les troupes, de leur côté, se sont réunies. Les généraux reçoivent et étudient les rapports de police. Ils se gardent bien d’aventurer leurs détachements sans données certaines, au risque d’un échec qui démoraliserait le soldat. Dès qu’ils connaissent bien les positions des insurgés, ils massent les régiments sur divers points qui constitueront désormais la base des opérations.\par
Les armées sont en présence. Voyons leurs manœuvres. Ici va se montrer à nu le vice de la tactique populaire, cause certaine des désastres.\par
Point de direction ni de commandement général, pas même de concert entre les combattants. Chaque barricade a son groupe particulier, plus ou moins nombreux, mais toujours isolé. Qu'il compte dix ou cent hommes, il n’entretient aucune communication avec les autres postes. Souvent il n’y a pas même un chef pour diriger la défense, et s’il y en a, son influence est à peu près nulle. Les soldats n’en font qu’à leur tête. Ils restent, ils partent, ils reviennent, suivant leur bon plaisir. Le soir, ils vont se coucher.\par
Par suite de ces allées et venues continuelles, on voit le nombre des citoyens présents varier rapidement, du tiers, de moitié, quelquefois des trois quarts. Personne ne peut compter sur personne. De là défiance du succès et découragement.\par
De ce qui se passe ailleurs on ne sait rien et on ne s’embarrasse pas davantage. Les canards circulent, tantôt noirs, tantôt roses. On écoute paisiblement le canon et la fusillade, en buvant sur le comptoir du marchand de vins. Quant à porter secours aux positions assaillies, on n’en a pas même l’idée. « Que chacun défende son poste, et tout ira bien », disent les plus solides. Ce singulier raisonnement tient à ce que la plupart des insurgés se battent dans leur propre quartier, faute capitale qui a des conséquences désastreuses, notamment les dénonciations des voisins, après la défaite.\par
Car, avec un pareil système, la défaite ne peut manquer. Elle arrive à la fin dans la personne de deux ou trois régiments qui tombent sur la barricade et en écrasent les quelques défenseurs. Toute la bataille n’est que la répétition monotone de cette manœuvre invariable. Tandis que les insurgés fument leur pipe derrière les tas de pavés, l’ennemi porte successivement toutes ses forces sur un point, puis sur un second, un troisième, un quatrième, et il extermine ainsi en détail l’insurrection.\par
Le populaire n’a garde de contrarier cette commode besogne. Chaque groupe attend philosophiquement son tour et ne s’aviserait pas de courir à l’aide du voisin en danger. Non « il défend son poste, il ne peut pas abandonner son poste. »\par
Et voilà comme on périt par l’absurde !\par
Lorsque, grâce à une si lourde faute, la grande révolte Parisienne de 1848 a été brisée comme verre par le plus pitoyable des gouvernements, quelle catastrophe n’aurait-on pas à redouter, si on recommençait la même sottise devant un militarisme farouche, qui a maintenant à son service les récentes conquêtes de la science et de l’art, les chemins de fer, le télégraphe électrique, les canons rayés, le fusil Chassepot ?\par
Par exemple, ce qu’il ne faut pas compter comme un des nouveaux avantages de l’ennemi, ce sont les voies stratégiques qui sillonnent maintenant la ville dans tous les sens. On les craint, on a tort. Il n’y a pas à s’en inquiéter. Loin d’avoir créé un danger de plus à l’insurrection, comme on se l’imagine, elles offrent au contraire un mélange d’inconvénients et d’avantages pour les deux partis. Si la troupe y circule avec plus d’aisance, par contre elle y est exposée fort à découvert.\par
De telles rues sont impraticables sous la fusillade. En outre, les balcons, bastions en miniature, fournissent des feux de flanc que ne comportent point les fenêtres ordinaires. Enfin, ces longues avenues en ligne droite méritent parfaitement le nom de boulevards qu’on leur a donné. Ce sont en effet de véritables boulevards qui constituent des fronts naturels d’une très grande force.\par
L'arme par excellence dans la guerre des rues, c’est le fusil. Le canon fait plus de bruit que de besogne. L'artillerie ne pourrait agir sérieusement que par l’incendie. Mais une telle atrocité, employée en grand et comme système, tournerait bientôt contre ses auteurs et ferait leur perte.\par
La grenade, qu’on a pris la mauvaise habitude d’appeler bombe, est un moyen secondaire, sujet d’ailleurs à une foule d’inconvénients; elle consomme beaucoup de poudre pour peu d’effet, est d’un maniement très dangereux, n’a aucune portée et ne peut agir que des fenêtres. Les pavés font presque autant de mal et ne coûtent pas si cher. Les ouvriers n’ont pas d’argent à perdre.\par
Pour l’intérieur des maisons, le revolver et l’arme blanche, baïonnette, épée, sabre et poignard. Dans un abordage la pique ou la pertuisane de huit pieds triompherait de la baïonnette.\par
L'armée n’a sur le peuple que deux grands avantages, le fusil Chassepot et l’organisation. Ce dernier surtout est immense, irrésistible. Heureusement on peut le lui ôter, et dans ce cas, l’ascendant passe du côté de l’insurrection.\par
Dans les luttes civiles, les soldats sauf de rares exceptions, ne marchent qu’avec répugnance, par contrainte et par eau-de-vie. Ils voudraient bien être ailleurs et regardent plus volontiers derrière que devant eux. Mais une main de fer les retient esclaves et victimes d’une discipline impitoyable ; sans affection pour le pouvoir, ils n’obéissent qu’à la crainte et sont incapables de la moindre initiative. Un détachement coupé est un détachement perdu. Les chefs ne l’ignorent pas, s’inquiètent avant tout de maintenir les communications entre tous leurs corps. Cette nécessité annule une partie de leur effectif.\par
Dans les rangs populaires, rien de semblable. Là on se bat pour une idée. Là on ne trouve que des volontaires, et leur mobile est l’enthousiasme, non la peur. Supérieurs à l’adversaire par le dévouement, ils le sont bien plus encore par l’intelligence. Ils l’emportent sur lui dans l’ordre moral et même physique, par la conviction, la vigueur, la fertilité des ressources, la vivacité de corps et d’esprit, ils ont la tête et le cœur. Nulle troupe au monde n’égale ces hommes d’élite.\par
Que leur manque-t-il donc pour vaincre ? Il leur manque l’unité et l’ensemble qui fécondent, en les faisant concourir au même but, toutes ces qualités que l’isolement frappe d’impuissance. Il leur manque l’organisation. Sans elle, aucune chance. L'organisation, c’est la victoire; l’éparpillement, c’est la mort.\par
Juin 1848 a mis cette vérité hors de conteste. Que serait-ce donc aujourd’hui ? Avec les vieux procédés, le peuple tout entier succomberait si la troupe voulait tenir, et elle tiendra tant qu’elle ne verra devant elle que des forces irrégulières, sans direction. Au contraire, l’aspect d’une armée parisienne en bon ordre manœuvrant selon les règles de la tactique frappera les soldats de stupeur et fera tomber leur résistance.\par
Une organisation militaire, surtout quand il faut l’improviser sur le champ de bataille, n’est pas une petite affaire pour notre parti. Elle suppose un commandement en chef et, jusqu’à un certain point, la série habituelle des officiers de tous grades. Où prendre ce personnel ? Les bourgeois révolutionnaires et socialistes sont rares et le peu qu’il y en a ne fait que la guerre de plume. Ces messieurs s’imaginent bouleverser le monde avec leurs livres et leurs journaux, et depuis seize ans ils barbouillent du papier à perte de vue, sans se fatiguer de leurs déboires, ils souffrent avec une patience chevaline le mors, la selle, la cravache, et ne lâcheraient pas une ruade. Fi donc ! Rendre les coups ! C'est bon pour des goujats.\par
Ces héros de l’écritoire professent pour l’épée le même dédain que l’épauletier pour leurs tartines. Ils ne semblent pas se douter que la force est la seule garantie de la liberté, qu’un pays est esclave où les citoyens ignorent le métier des armes et en abandonnent le privilège à une caste ou a une corporation.\par
Dans les républiques de l’antiquité, chez les Grecs et les Romains, tout le monde savait et pratiquait l’art de la guerre. Le militaire de profession était une espèce inconnue. {\scshape Cicéron} était général, {\scshape César} avocat. En quittant la toge pour l’uniforme, le premier venu se trouvait colonel ou capitaine et ferré à glace sur l’article. Tant qu’il n’en sera pas de même en France, nous resterons les Pékins taillés à merci par les traîneurs de sabre.\par
Des milliers de jeunes gens instruits, ouvriers et bourgeois, frémissent sous un joug abhorré. Pour le briser, songent-ils à prendre l’épée ? Non ! la plume, toujours la plume, rien que la plume. Pourquoi donc pas l’une et l’autre, comme l’exige le devoir d’un Républicain ? En temps de tyrannie, écrire est bien, combattre est mieux, quand la plume esclave demeure impuissante. Eh bien ! point ! On fait un journal, on va en prison, et nul ne songe à ouvrir un livre de manœuvres, pour y apprendre en vingt-quatre heures le métier qui fait toute la force de nos oppresseurs, et qui nous mettrait dans la main notre revanche et leur châtiment.\par
Mais à quoi bon ces plaintes ? C'est la sotte habitude de notre temps de se lamenter au lieu de réagir. La mode est aux jérémiades. Jérémie pose dans toutes les attitudes, il pleure, flagelle, il dogmatise, il régente, il tonne, fléau lui-même entre tous les fléaux. Laissons ces bobèches de l’élégie, fossoyeurs de la liberté ! Le devoir d’un révolutionnaire, c’est la lutte toujours, la lutte quand même, la lutte jusqu’à extinction.\par
Les cadres manquent pour former une armée ? Eh bien ! Il faut en improviser sur le terrain même, pendant l’action. Le peuple de Paris fournira les éléments, anciens soldats, ex-gardes nationaux. Leur rareté obligera de réduire à un minimum le chiffre des officiers et sous-officiers. Il n’importe. Le zèle, l’ardeur, l’intelligence des volontaires, compenseront ce déficit.\par
L'essentiel, c’est de s’organiser. Plus de ces soulèvements tumultueux, à dix mille têtes isolées, agissant au hasard, en désordre, sans nulle pensée d’ensemble, chacun dans son coin et selon sa fantaisie ! Plus de ces barricades à tort et à travers, qui gaspillent le temps, encombrent les rues, et entravent la circulation, nécessaire à un parti comme à l’autre. Le Républicain doit avoir la liberté de ses mouvements aussi bien que les troupes.\par
Point de courses inutiles, de tohu-bohu, de clameurs ! Les minutes et les pas sont également précieux. Surtout ne pas se claquemurer dans son quartier ainsi que les insurgés n’ont jamais manqué de le faire, à leur grande dommage. Cette manie, après avoir causé la défaite, a facilité les proscriptions. Il faut s’en guérir, sous peine de catastrophe.\par
Ces préliminaires posés, indiquons le mode d’organisation.\par
L'unité principale est le bataillon. Il se compose de huit compagnies ou pelotons.\par
Chaque compagnie compte un lieutenant, quatre sergents, cinquantesix soldats; en tout soixante et un hommes.\par
Deux compagnies forment une division commandée par un capitaine. Le bataillon présente par conséquent treize officiers, savoir: un commandant, quatre capitaines, huit lieutenants, plus 32 sergents, 448 soldats et le porte-drapeau, total: 494 hommes. Les tambours sont en sus, si on en trouve.\par
La rareté prévue de l’élément qui forme les cadres, oblige de supprimer dans chaque compagnie deux officiers, le capitaine et le sous-lieutenant, deux sous-officiers, le sergent-major et le fourrier, enfin les huit caporaux. L'état-major de la compagnie se trouve ainsi réduit de seize à cinq individus. Il est vrai qu’elle est moins nombreuse que dans l’armée, où elle compte 90 hommes sur pied de guerre. Proportion gardée, c’est une différence d’état-major de cinq à onze.\par
Le chiffre de la compagnie est faible, afin de faciliter les manœuvres tant du peloton que du bataillon.\par
Le capitaine, au lieu de commander un peloton comme dans la troupe, en commande deux, c’est-à-dire une division. Cependant les manœuvres par division n’auront presque jamais lieu. A peu près impraticables dans Paris, elles ne peuvent servir qu’à plier le bataillon en masse Par divisions, sur une place ou une grande voie. Mais il importe de donner un chef spécial à la division, soit qu’elle occupe une, deux ou quatre barricades. Dans le premier cas, la barricade est importante par le nombre de ses défenseurs. Dans les deux autres, il est essentiel de ne pas laisser dans une direction supérieure les deux ou quatre petits postes.
\section[{Organisation du peloton}]{Organisation du peloton}\renewcommand{\leftmark}{Organisation du peloton}

\noindent Le peloton se divise en deux sections, chacune de 28 soldats et de deux sous-officiers.\par
La section se subdivise en deux demi-sections, chacune de 14 soldats et un sous-officier.\par
Place des officiers et sous-officiers dans le peloton en bataille.\par
Le lieutenant à la droite de son peloton, au premier rang.\par
Le premier sergent derrière le lieutenant, au second rang.\par
Le deuxième sergent, à la gauche de la section de droite, au premier rang.\par
Le troisième sergent, derrière le deuxième, à la droite de la section de gauche, au second rang.\par
Le quatrième sergent, à la gauche de la section de gauche et du peloton, au premier rang.
\section[{Des guides}]{Des guides}\renewcommand{\leftmark}{Des guides}

\noindent Le premier sergent est guide de droit du peloton et de la section de droite. Il est guide de droite et de gauche de la première demi-section de droite.\par
Le deuxième sergent est guide de gauche de la section de droite. Il est guide de droite et de gauche de la seconde demi-section de droite. Il est porte-fanion du peloton.\par
Le troisième sergent est guide de droite de la section de gauche. Il est guide de droite et de gauche de la première demi-section de gauche.\par
Le quatrième sergent est guide de gauche du peloton et de la section de gauche. Il est guide de droite et de gauche de la seconde demi-section de gauche.\par
Placer les officiers et sous-officiers, quand le bataillon est en colonne, la droite ou la gauche en tête.\par

\begin{enumerate}[itemsep=0pt,]
\item En colonne, par pelotons, le lieutenant se tient à droite du peloton. Le premier, deuxième et quatrième sergents, au premier rang, le troisième au second rang, derrière le deuxième ;
\item En colonne par sections, le lieutenant se tient à droite de la section de tête. Les quatre sergents à droite et à gauche de leurs sections respectives au premier rang ;
\item En colonne par demi-sections le lieutenant se tient à la droite de la demi-section de tête. Les quatre sergents, étant guides de droite et de gauche de leurs demi-sections, sont tantôt à droite, tantôt à gauche, selon le commandement, toujours au premier rang.
\end{enumerate}

\noindent Les deux sergents qui se trouvent aux extrémités du bataillon en bataille, en sont guides de droite et de gauche et se tiennent au premier rang. Le lieutenant du peloton de droite, s’écarte à droite, pour faire place au guide.\par
Place des capitaines, en bataille et en colonne :\par
{\itshape Le bataillon étant en bataille, les capitaines se tiennent à quelques pas en arrière du centre de leurs divisions respectives. Le bataillon étant en colonne, chaque capitaine se tient sur le flanc gauche de sa division.}\par
Le chef de bataillon n’a point de place fixe.\par
\emph{Nota}. - Les quatre sous-officiers restent constamment dans les rangs qu’ils encadrent. Ils ne sont jamais en serre-file comme dans la troupe. Les ouvriers Parisiens, volontaires au service de la liberté, n’ont pas besoin de sergents pousseculs.\par
Place du porte-drapeau, en bataille et en colonne\par

\begin{enumerate}[itemsep=0pt,]
\item \emph{en bataille, le porte-drapeau est à la gauche du quatrième peloton, au premier rang} ;
\item en colonne, par divisions, le porte-drapeau est au centre, à égale distance entre la seconde et la troisième divisions ;
\item en colonne, par pelotons, le porte-drapeau est à gauche, dans l’alignement des guides, à égale distance entre le quatrième et le cinquième peloton ;
\item en colonne par sections, ou par demi-sections, le porte-drapeau est au centre, à. égale distance entre le quatrième et le cinquième peloton.
\end{enumerate}

\noindent Le drapeau est rouge, \emph{- chaque compagnie a son fanion ou guidon de couleur particulière} :\par

\begin{enumerate}[itemsep=0pt,]
\item \emph{peloton-fanion rouge} ;
\item  \emph{peloton-fanion violet;} 
\item  \emph{peloton-fanion verd (\emph{sic});} 
\item  \emph{peloton-fanion jaune;} 
\item  \emph{peloton-fanion bleu;} 
\item  \emph{peloton-fanion rose;} 
\item  \emph{peloton-fanion orange;} 
\item  \emph{peloton-fanion noir.} 
\end{enumerate}

\noindent Les officiers et sous-officiers porteront, comme insignes, un ruban de couleur du guidon de leur compagnie, les lieutenants au bras gauche, entre l’épaule et le coude, les sergents au poignet gauche. Le ruban de la 8e compagnie sera noir à double lisere rouge.\par
Les capitaines porteront entre l’épaule et le coude un ruban de la couleur de chacune des deux compagnies formant leur division, au bras droit du peloton impair, au bras gauche celui du peloton pair. Le ruban noir du 4e capitaine aura double liseré rouge.\par
Le chef de bataillon porte au bras gauche, entre l’épaule et le coude, un large ruban rouge, à frange pendante.\par
Le numéro de chaque. bataillon sera inscrit au haut de la hampe du fanion de ses huit compagnies.\par
Les diverses couleurs tant des fanions que des officiers et sous-officiers, ont pour but de faire reconnaître a première vue dans la mêlée les différentes compagnies et d’opérer un prompt ralliement.\par
Chaque homme occupant deux pieds dans le rang, la demi-section a cinq mètres de front, la section dix, le peloton, vingt, la division, quarante, le bataillon, cent soixante.\par
Il faut toujours manœuvrer avec 70 ou 75 centimètres de distance entre les deux rangs, afin que le second rang ne soit pas obligé d’emboîter le pas, chose très incommode pour des novices. Si on doit faire feu, le deuxième rang serre le premier, afin de passer les fusils entre les têtes des hommes du premier rang.
\section[{Des manœuvres}]{Des manœuvres}\renewcommand{\leftmark}{Des manœuvres}

\noindent Tous les officiers doivent connaître parfaitement l’école de peloton et l’école de bataillon. Pour savoir le moins, il est bon de savoir le plus. Néanmoins, il est évident qu’il n’y aura lieu d’employer qu’un petit nombre des mouvements décrits dans l’une et l’autre école. Il est donc essentiel d’étudier ceux-là de préférence. Ils ont surtout pour but de régulariser la formation en bataille.\par
Voici les principaux\par

\begin{enumerate}[itemsep=0pt,]
\item Le bataillon étant en bataille rompre à droite ou à gauche soit par pelotons, soit par sections, soit par demi-sections ;
\item Le bataillon étant en bataille, rompre en arrière à droite ou à gauche, soit par pelotons, soit par sections, soit par demi-sections. \\
\emph{Nota-bene.} - Dans ce dernier mouvement, faire par le flanc sans dédoubler. - Du reste, l’autre manière de rompre est préférable ;
\item Le bataillon marchant en colonne par pelotons, rompre les pelotons;
\item Le bataillon marchant en colonne par sections, rompre les sections. \\
\emph{Nota-bene}. - Ces deux derniers mouvements doivent s’exécuter au pas de gymnastique, afin de ne pas perdre de temps ni de terrain;
\item Le bataillon marchant en colonne par demi-sections, former les sections;
\item Le bataillon marchant en colonne par sections, former les pelotons. \\
\emph{Nota-bene.} - Les pelotons ayant vingt mètres de front, le bataillon ne pourra marcher en colonne par pelotons que sur les plus larges chaussées. \\
La marche la plus habituelle sera en colonne par sections qui n’occupe que onze de front. \\
On rompra les sections, avant d’entrer dans une rue ayant moins de douze mètres de large;
\item Le bataillon marchant en colonne par pelotons, ou par sections, ou par demi-sections, le former à droite ou à gauche en bataille.\emph{ \\
Nota-bene.} - Cette formation en bataille étant la plus prompte, est la meilleure. Mais elle présente des difficultés. On ne peut former régulièrement la colonne à droite ou à gauche en bataille, que si les pelotons, ou les sections ou les demi-sections ont exactement conservé leurs distances, c’est-à-dire si la distance qui les sépare est égale à leur front. Si elle est plus grande, il reste des vides dans le bataillon formé en bataille. Si, au contraire, la distance est moindre que le front, les fractions du bataillon, en arrivant à l’alignement, se heurtent et s’entassent les unes sur les autres, faute de place ;
\item La colonne étant en marche par pelotons, par sections ou par demi-sections, la former sur la droite ou sur la gauche en bataille. \\
\emph{Nota-bene.} - Ce mouvement n’a pas les inconvénients du précédent, et devant l’ennemi, il a l’avantage d’ouvrir le feu dès le début de la formation. Mais, pour mettre simplement la colonne en bataille, il est d’une extrême lenteur.
\end{enumerate}

\noindent Le mouvement de flanc, par dédoublement, a le très grand avantage de former instantanément le bataillon en colonne, s’il est en bataille, ou en bataille, s’il est en colonne. Mais il a cet inconvénient qu’il est impossible de serrer la colonne. En outre, les deux mouvements: faire par le flanc, et faire front, sont difficiles pour des hommes qui n’ont jamais été exercés. Néanmoins il sera utile d’enseigner cette manœuvre au bataillon, aussitôt qu’il sera organisé. L'intelligence des ouvriers Parisiens leur en fera comprendre le mécanisme en quelques minutes.\par
Lorsqu’un bataillon en marche doit faire tête de colonne à droite ou à gauche, pour entrer dans une rue latérale, il faut employer le mouvement « tournez à droite », ou « tournez à gauche », préférable à la conversion régulière qui est plus lente et plus difficile.\par
Tous les changements de direction de la colonne doivent se faire par ce même mouvement « tournez à droite ou à gauche ».\par
Le bataillon devra toujours marcher et manœuvrer au pas de route, c’est-à-dire les deux rangs à distance de 70 ou 75 centimètres, afin que le second rang ne soit pas obligé d’emboîter le pas, et marche en liberté.\par
Tous les mouvements devront être exécutés avec rapidité, sans se piquer de précision ni d’élégance. La promptitude avant tout.\par
Le port d’armes en sous-officier, le fusil dans la main droite, le bras allongé le long de la cuisse, la sous-garde tournée en avant.\par
Il faudra faire appel aux hommes qui savent battre la caisse. Les tambours sont de première nécessité pour les commandements.
\section[{Manœuvres par divisions}]{Manœuvres par divisions}\renewcommand{\leftmark}{Manœuvres par divisions}

\noindent Les manœuvres par divisions ne peuvent être que fort rares dans Paris. Il importe néanmoins d’étudier les suivantes:\par

\begin{enumerate}[itemsep=0pt,]
\item Le bataillon étant en colonne par pelotons, serrés en masse, ou à demi-distance ou à distance entière, former les divisions ;
\item Le bataillon étant en bataille, le ployer en colonne serrée par division sur l’une quelconque des quatre divisions, la droite ou la gauche en tête ;
\item Le bataillon étant en colonne serrée par divisions, en marche ou de pied ferme, le déployer sur l’une quelconque des quatre divisions. La colonne par peloton.
\end{enumerate}

\noindent \emph{Esquisse de la marche à suivre dans une prise d’armes à Paris.}\par
Les hommes qui prennent l’initiative du mouvement, ont choisi d’avance un commandant en chef et un certain nombre d’officiers, dont les fonctions commencent avec l’insurrection elle-même.
\section[{Manière d’organiser}]{Manière d’organiser}\renewcommand{\leftmark}{Manière d’organiser}

\noindent Aussitôt que des citoyens accourent, à la vue du soulèvement, les faire mettre en bataille sur deux rangs.\par
Les engager au silence et au calme, leur adresser une brève allocution. Leur annoncer ensuite que tout citoyen marchant sous le drapeau de la République, recevra des vivres et cinq francs par jour, en indemnité de salaire, pendant la durée de la lutte.\par
Inviter tous ceux qui ont servi dans l’armée ou fait partie de la garde nationale, à sortir des rangs et à se présenter sur le front de la ligne.\par
Les classer en officiers, sous-officiers et simples soldats. Mettre en réserve les premiers comme officiers supérieurs, choisir les sous-officiers pour lieutenants, chefs de pelotons, les simples soldats pour sergents.\par
Distribuer aux lieutenants et aux sergents un imprimé qui leur explique l’organisation de l’armée populaire et les diverses mesures à prendre.\par
Les caser à leurs places respectives comme-officiers et sous-officiers et encadrer entre eux les soldats de chaque peloton, et former ainsi les compagnies jusqu’à épuisement du personnel présent.\par
S'il n’y a pas assez d’hommes pour compléter un bataillon, ranger a la suite des pelotons constitués, les cadres de ceux qui restent à former, cadres prêts à recevoir les volontaires nouveaux.\par
Si, au contraire, c’est le personnel des cadres qui est insuffisant, faire appel aux hommes qui se sentent assez d’intelligence pour commander, leur assigner les fonctions de lieutenant et de sergents, et leur donner l’imprimé qui les mettra au courant de l’organisation.\par
Le nombre des pelotons ainsi formés restant inférieur à huit, déclarer néanmoins le bataillon constitué.\par
S'il est supérieur à huit, constituer avec l’excédent un deuxième bataillon, qui se complétera par l’adjonction de nouveaux volontaires.\par
Distribuer aux lieutenants et aux sergents les rubans de diverses couleurs qu’ils doivent porter comme insignes; déployer le drapeau du bataillon, ainsi que les fanions des compagnies qui seront confiés aux deuxièmes sergents.\par
Aussitôt le drapeau déployé, faire prêter aux officiers, sous-officiers et soldats le serment ci-après :\par
« Je jure de combattre jusqu’à la mort pour la République, d’obéir aux ordres des chefs, et de ne pas m’écarter un seul instant du drapeau, ni de jour ni de nuit, avant que la bataille soit terminée. »\par
Distribuer les armes disponibles aux compagnons et aux bataillons, dans l’ordre chronologique de leur formation; premiers organisés, premiers armés.\par
S'il n’existe que quelques fusils, les donner aux sergents porte-fanions.\par
Les officiers et sous-officiers feront constamment aux soldats les recommandations suivantes:\par
« Ne jamais perdre une seconde - rester en ordre - observer le silence (sauf le cri de Vive la République poussé seulement à un signal donné) - marcher d’un pas rapide. Dans le cas d’un engagement, n’agir que d’après le commandement. Si on a le dessous, se rallier vite et sans tumulte au drapeau et aux fanions. - Si on a le dessus, garder les rangs, sans bruit, ni cri, prêts à marcher. - Exécuter tous les ordres avec rapidité et si on doit s’éloigner du drapeau pour les remplir, le rallier vivement, aussitôt l’ordre accompli. »\par
Le cri de Vive la République ne doit être poussé qu’au signal des chefs, parce qu’une marche silencieuse est souvent de la plus impérieuse nécessité.\par
Qu'on soit en marche ou en halte, organiser aussitôt tous les ouvriers qui se rencontreront sur le passage de la colonne.\par
S'il y a des cadres en excédent, ils marcheront à la queue de la colonne, dans l’ordre des numéros de leurs compagnies, incorporant en chemin, sans s’arrêter, tous les hommes de bonne volonté trouvés sur la route.\par
Les officiers et sous-officiers des pelotons ainsi formés pendant la marche, demandent immédiatement aux citoyens incorporés s’ils ont servi dans l’armée ou appartenu à la garde nationale; et ils font sortir sur le flanc de la colonne ceux qui se trouvent dans ce cas.\par
Des officiers d’état-major accompagnent la colonne afin de constituer avec ces nouveaux éléments des cadres de compagnies et de bataillons, en assignant les grades d’après la règle indiquée plus haut. Ils distribuent les rubans servant d’insignes, font déployer les fanions et les drapeaux des nouveaux corps qui se mettent à la suite.\par
L'organisation des nouveaux bataillons continuera ainsi sans înterruption, pendant la durée de la lutte. Toute colonne en marche ralliera les ouvriers rencontrés sur son chemin et les formera en compagnies et en bataillons d’après les procédés ci-dessus.\par
Aussitôt que le nombre des bataillons dépassera neuf, ils pourront être réunis par régiments et par brigades.\par
Dès le début de l’insurrection, des citoyens dévoués seront chargés de couper les fils télégraphiques et de détruire les communications du gouvernement avec la province.
\section[{Mesures insurrectionnelles}]{Mesures insurrectionnelles}\renewcommand{\leftmark}{Mesures insurrectionnelles}

\noindent Aussitôt que la chose sera possible, le commandant en chef établira des commissions d’armement, de vivres et de sûreté publique.
\section[{Commission d’armement}]{Commission d’armement}\renewcommand{\leftmark}{Commission d’armement}

\noindent La commission d’armement fera rechercher, soit dans les magasins et fabriques d’arquebuserie, soit chez les particuliers, toutes les armes disponibles, fusils de guerre et de chasse, pistolets, revolvers, sabres et épées, ainsi que les poudres entreposées chez les débitants ou réunies en dépôt, notamment chez les artificiers.\par
Elle requerra le plomb en existence chez les plombiers, les moules à balles de tous calibres chez les quincailliers, fera fabriquer des mandrins par les tourneurs, des mesures à poudre, installera des ateliers où les femmes et les enfants seront employés moyennant salaire à la fonte des balles et à la confection des cartouches.\par
Elle fera préparer des fanions, des drapeaux et des rubans pour insignes.\par
Elle requerra chez les fabricants de produits chimiques, les matières qui entrent dans les diverses sortes de poudres notamment l’acide sulfurique et l’acide nitrique anhydres ou concentrés, éléments du fulmicoton. On mettra en réquisition pour ces travaux les élèves en pharmacie.
\section[{Commission des vivres}]{Commission des vivres}\renewcommand{\leftmark}{Commission des vivres}

\noindent La commission des vivres requerra chez les boulangers, bouchers et dans les entrepôts de liquides, le pain, la viande, les vins et liqueurs nécessaires à la consommation de l’armée Républicaine, Elle mettra en réquisition les traiteurs, restaurateurs et autres établissements de ce genre pour la préparation des vivres.\par
Il y aura, par chaque bataillon, un commissaire des vivres chargé de veiller à la distribution et de faire connaître à la commission les besoins du bataillon.
\section[{Commision de sûreté publique}]{Commision de sûreté publique}\renewcommand{\leftmark}{Commision de sûreté publique}

\noindent La commission de sûreté publique a pour mission de, déjouer les trames de la police et les manœuvres des contre-révolutionnaires, de faire imprimer, distribuer et afficher les proclamations ou arrêtés du Commandant en chef, de surveiller les télégraphes, les chemins de fer, les établissements impériaux, en un mot, de dissoudre les moyens d’action de l’ennemi, d’organiser et d’assurer ceux de la République.\par
Les fonds nécessaires pour le service de ces trois commissions et pour le paiement de l’indemnité quotidienne de cinq francs, allouée aux citoyens présents sous les drapeaux, seront prélevées sur les caisses publiques.\par
Il sera délivré aux marchands et industriels, récépissé régulier des livraisons de marchandises quelconques par eux fournies, sur réquisition. Ces fournitures seront soldées par le gouvernement républicain.\par
Les trois commissions rendront compte de leurs travaux, d’heure en heure au commandant en chef et exécuteront ses ordres.\par
Il sera formé un service spécial pour les ambulances.
\section[{Des barricades}]{Des barricades}\renewcommand{\leftmark}{Des barricades}

\noindent Aucun mouvement militaire ne devant avoir lieu que d’après l’ordre du commandant en chef, il ne sera élevé de barricades que sur les emplacements désignés par lui.\par
Sous peine d’une prompte débâcle, les barricades ne peuvent plus être aujourd’hui une œuvre comme en 1830 et 1848, confuse et désordonnée. Elles doivent faire partie d’un plan d’opération, arrêté d’avance.\par
Dans ce système, chaque retranchement est occupe par une garnison qu’on abandonne point à elle-même, qui reste en communication suivie avec les réserves et en reçoit constamment des renforts proportionnés aux dangers de l’attaque.\par
Le tohu-bohu et l’éparpillement ne constituaient pas le seul vice des anciennes barricades. Leur construction n’était pas moins défectueuse.\par
Amas informe de pavés, entremêlés de voitures sur le flanc, de poutres et de planches, ce mauvais barrage n’était pas un obstacle pour l’infanterie qui l’enlevait au pas de course. Quelques gros retranchements peut-être, faisaient exception. Encore pas un seul n’était à l’abri de l’escalade. Ils servaient eux-mêmes d’échelle.\par
Arrêter les troupes, les contraindre à un siège, résister même assez longtemps au canon, telle est. la destination d’une barricade. Il faut donc la construire d’après ces données, pour qu’elle atteigne son triple but. Jusqu’ici, elle n’y a pas satisfait le moins du monde.
\section[{Croquis de barricade}]{Croquis de barricade}\renewcommand{\leftmark}{Croquis de barricade}


\begin{figure}[h]
  \centering
  \includegraphics[width=\linewidth,]{blanqui1866_prise-armes/barricade.png}
\caption{Profil de la barricade complète, rempart et contre-garde avec glacis. Le rempart et le mur interne de la contre-garde sont maçonnés en plâtre.}
\end{figure}

\noindent Dans l’état actuel de Paris, malgré l’invasion du macadam, le pavé reste toujours le véritable élément de la fortification passagère, à condition toutefois d’en faire un usage plus sérieux que par le passé. C'est une affaire de bon sens et de calcul.\par
L'ancien pavé, qui tapisse encore la majeure partie de la voie publique est un cube de 25 centimètres de côté. On peut, dès lors, supputer par avance le nombre de ces blocs qui sera mis en œuvre pour bâtir un mur, dont les trois dimensions, longueur, largeur et hauteur sont déterminées.
\section[{Barricade régulière}]{Barricade régulière}\renewcommand{\leftmark}{Barricade régulière}

\noindent La barricade complète consiste dans un rempart et sa contre-garde ou couvre-face.\par
Le rempart est en pavés maçonnés au plâtre, large d’un mètre, haut de trois, encastré par des extrémités dans les murs de façade des maisons.\par
La contre-garde, placée à six mètres en avant du rempart se compose de deux parties attenantes l’une à l’autre, savoir : un mur interne de mêmes dimensions et constructions que le rempart, et un glacis en pavés secs amoncelés s’étendant sur une longueur de quatre mètres jusqu’à l’entrée de la rue.\par
Un mètre cube contient 64 pavés de 25 centimètres de côté. Le rempart ainsi que le mur interne de la contre-garde ont toujours deux facteurs fixes, la hauteur 3 mètres, la largeur ou épaisseur un mètre. La longueur seule varie. Elle dépend de la largeur de la rue.\par
En supposant ici la rue de 12 mètres, et par conséquent, le chiffre 12, facteur commun pour le rempart, le mur interne maçonné du glacis, et le glacis lui-même, on aura:\par

\tableopen{}
\begin{tabularx}{\linewidth}
{|l|X|X|}
\hlineLe rempart & = 3 × 1 × 12 & = 36 \\
\hline
Le mur interne du glacis & = 3 × 1 × 12 & = 36 \\
\hline
Le glacis & = 3 × 4 × 12 & = 72 \\
\hline
 & 2 &  \\
\hline
\end{tabularx}
\tableclose{}

\noindent Le cube total de la barricade et de sa contre-garde sera de 144 mètres qui, à 64 pavés par mètre cube, donnent 9186 pavés, représentant 191 rangées à 4 × 12 ou 48 par rangées. Ces 192 rangées occupent 48 mètres de long. Ainsi la rue serait dépavée dans une longueur de 48 mètres, pour fournir les matériaux du retranchement complet.\par
Le calcul n’ayant pas tenu compte de la place occupée par le plâtre dans le rempart et le mur interne de la contre-garde le nombre de pavés serait diminué d’autant. Il serait moindre encore dans le glacis, par suite des vides existant entre les pavés entassés en désordre.\par
Les petits pavés rectangulaires qui ont remplacé en partie le macadam des grandes voies, pourraient servir également à l’érection des barricades. Mais le travail des parties maçonnées serait plus long et consommerait plus de plâtre.\par
Dans tous les cas, il est bien évident qu’un pareil retranchement ne serait pas bâclé dans une heure. Or, il importe de se mettre en défense le plus promptement possible. On peut parer à cette difficulté.\par
Le détachement chargé de construire et d’occuper la barricade doit se rendre sur le terrain avec une voiture de sacs de plâtre, plus des brouettes, des voitures à bras, des leviers, des pics, des pelles, des pioches, des marteaux, des ciseaux à froid, des truelles, des seaux, des auges. Les réquisitions de tous ces objets seront faites chez les marchands respectifs dont les adresses se trouvent dans l’Almanach du Commerce. On choisira les plus voisins du point de départ.\par
Une fois sur le terrain, le chef du poste fait commencer le rempart à 15 mètres environ du débouché de la rue, et au lieu de trois mètres de hauteur, ne lui en donne que la moitié.\par
Ce mur de quatre pieds et demi a précisément la hauteur normale pour le tir d’un fantassin debout. On peut l’escalader sans doute, mais l’opération n’est pas commode. C'est déjà un obstacle respectable. Or, ce massif n’a que 18 mètres cubes ou 1152 pavés, qui représente 24 rangées ou 6 mètres de longueur à dépaver. Cela peut se faire assez rapidement.\par
On achève ensuite le rempart jusqu’à trois mètres à mi-hauteur (1 m 1/2), c’est-à-dire à un mètre et demi, on laisse, de distance en distance, des trous destinés à recevoir des solives, sur lesquelles on posera des planches formant banquette pour le tir.\par
Le dessus du mur interne de la contre-garde doit être plan, sans inclinaison ni en dedans ni en dehors afin de ne pas donner prise au boulet qui écrêterait la partie la plus haute amincie.\par
Le dessus du rempart peut être incliné légèrement, afin de ménager au tir une certaine plongée. Il sera crépi et lissé à la truelle, ainsi que la paroi faisant face à la contre-garde.\par
Les trous pratiqués à mi-hauteur pour l’échafaudage de construction, tant au mur de la contre-garde qu’au rempart seront bouchés avec soin. Les parois du rempart et de la contre-garde qui se font face, devront être lissées à la truelle, et n’offrir aucune aspérité favorisant l’escalade.\par
Les rangées de pavés de chaque assise de deux murs seront posées en échiquier, ainsi que les assises elles-mêmes, par rapport l’une à l’autre.\par
Si le rempart dépassait en hauteur le mur de la contre-garde, les boulets démoliraient la partie saillante. Dans le cas cependant où du rempart on voudrait tirer au loin sur l’ennemi, il suffirait d’y placer des sacs à plâtre remplis de terre. Les combattants se hausseraient eux-mêmes au moyen de pavés.\par
Du reste, le retranchement est plutôt une barrière qu’un champ d’action. C'est aux fenêtres que se trouve le véritable poste de combat. De là, des centaines de tirailleurs peuvent diriger dans tous les sens un feu meurtrier.\par
L'officier chargé de défendre le débouché d’une rue, fait occuper, en arrivant, les maisons des deux angles par le tiers de son monde, les hommes les mieux armés, détache en avant quelques vedettes pour éclairer les rues et prévenir une surprise, et commence les travaux du retranchement avec les précautions et dans l’ordre indiqués plus haut.\par
Si une attaque survient avant l’achèvement du mur simple, d’un mètre et demi de haut, l’officier se retire avec tout son monde dans les maisons des deux angles, après avoir mis en sûreté dans une cour intérieure, voiture, chevaux, matériel de toute espèce. Il se défend par les feux des fenêtres et les pavés lancés des étages supérieurs. Les petits pavés rectangulaires des grandes voies macadamisées sont excellents pour cet usage.\par
L'attaque repoussée, il reprend et presse sans relâche la construction de la barricade en dépit des interruptions. Au besoin des renforts arrivent.\par
Cette besogne terminée, on se met en communication avec les deux barricades latérales, en perçant les gros murs qui séparent les maisons situées sur le front de défense. La même opération s’exécute simultanément, dans les maisons des deux côtés de la rue barricadée jusqu’à son extrémité, puis en retour à droite et à gauche, le long de la rue parallèle au front de défense, en arrière.\par
Les ouvertures sont pratiquées au premier et au dernier étage, afin d’avoir deux routes; le travail se poursuit à la fois dans quatre directions.\par
Tous les îlots ou pâtés de maisons appartenant aux rues barricadées, doivent être percés dans leur pourtour, de manière que les combattants puissent entrer et sortir par la rue parallèle de derrière, hors de la vue et de la portée de l’ennemi.\par
Dans ce travail, la garnison de chaque barricade doit se rencontrer à mi-chemin, tant sur le front de défense que dans la rue de derrière avec les deux garnisons des deux barricades voisines, de droite et de gauche.\par

\begin{figure}[h]
  \centering
  \includegraphics[width=\linewidth,]{blanqui1866_prise-armes/phalanstere.png}
\caption{Plan d’un phalanstère en grande échelle}
\end{figure}

\noindent Exemple de barricades sur un front de défense, reliées entre elles par le percement des maisons des îlots adjacents.\par
Le boulevard Sébastopol étant supposé front de défense, on a pris sur ce front une étendue d’environ 140 mètres, qui comprend les débouchés de trois rues et un peu au-delà, savoir les rues Aubry-le-Boucher, de la Reynie, et des Lombards.\par
Les trois rues sont fermées à leur issue sur le boulevard, par des barricades avec contre-gardes. Les dimensions et les distances sont rigoureusement exactes sur le plan.\par
La garnison du retranchement La Reynie, après avoir complété les constructions de la rue et simultanément même percé des maisons le long du boulevard, vers la rue Aubry-le-Boucher, à droite, et vers la rue des Lombards, à gauche.\par
Elle fait la même opération des deux côtés de la rue de la Reynie, en gagnant la rue des Cinq-Diamants, et parvenue à l’extrémité, tourne à gauche, vers la rue Aubry-le-Boucher, à droite vers la rue des Lombards, en continuant son travail.\par
De leur côté, les garnisons des barricades Aubry-le-Boucher et Lombards vont à la rencontre des travailleurs La Reynie, d’après la même méthode, et la jonction s’opère à mi-chemin.\par
Les maisons ont été indiquées au hasard sur le boulevard Sébastopol, mais dans les rues de La Reynie, Aubry-le-Boucher, des Lombards et des Cinq-Diamants, le nombre des maisons ou plutôt des gros murs qui les séparent a été relevé avec exactitude sur un ancien plan très détaillé.\par
La Garnison La Reynie aurait donc à percer, entre la moitié des maisons du boulevard, entre les deux rues latérales, douze murs dans la rue de La Reynie, cinq d’un côté, sept de l’autre, plus sept autres dans la rue des Cinq-Diamants, cinq à droite, deux à gauche.\par
En admettant dix maisons sur le front Sébastopol, ce qui ne donne à chacune que neuf mètres de façade, il y aurait donc en tout 24 murs à percer, six pour chaque escouade de travailleurs, puisqu’on procéderait dans quatre directions à la fois.\par
Du reste, si on est en nombre, on peut percer en même temps toutes les maisons de la rue barricadée et de la rue de derrière, puisqu’on a ses communications libres, en arrière du retranchement.\par
L'intérieur des îlots consiste généralement en cours et jardins. On pourrait ouvrir des communications à travers ces espaces, séparés d’ordinaire par de faibles murs. La chose sera même indispensable sur les points que leur importance ou leur situation spéciale exposent aux attaques les plus sérieuses.\par
Il sera donc utile d’organiser des compagnies d’ouvriers non-combattants, maçons, charpentiers, etc., pour exécuter les travaux conjointement avec l’infanterie.\par
Lorsque sur le front de défense, une maison est plus particulièrement menacée, on démolit l’escalier du rez-de-chaussée, et l’on pratique des ouvertures dans les planchers des diverses chambres du premier étage, afin de tirer sur les soldats qui envahiraient le rez-de-chaussée pour y attacher des pétards. L'eau bouillante jouerait aussi un rôle utile dans cette circonstance.\par
Si l’attaque embrasse une grande étendue de front, on coupe les escaliers, et on perce les planchers dans toutes les maisons exposes. En règle générale, lorsque le temps et les autres travaux de défense plus urgents le permettent, il faut détruire l’escalier du rez-de-chaussée dans toutes les maisons de l’îlot, sauf une, à l’endroit de la rue derrière le moins exposé.\par
La troupe enlève toujours assez facilement les barricades, à cause du petit nombre de leurs défenseurs, de l’isolement où on les abandonne, et du défaut de confiance mutuelle dû à l’absence d’organisation et de commandement. Les choses prendraient une toute autre face, avec une direction énergique et l’envoi successif de puissants renforts.\par
Jusqu’ici dans les luttes parisiennes, les insurgés sont toujours demeurés inactifs derrière leur semblant de barricades, oisiveté fatale chez des combattants très mal armés, sans artillerie, presque sans munitions. La bravoure seule ne suffit pas à compenser tous les désavantages matériels.\par
Les ouvriers Parisiens semblent ignorer leur principale force, la supériorité de l’intelligence et de l’adresse. Inépuisables en ressources, ingenieux, tenaces, initiés à toutes les puissances de l’industrie, il leur serait facile d’improviser en peu d’heures tout un matériel de guerre. Charpentiers, menuisiers, mécaniciens, fondeurs, tourneurs, maçons, ils peuvent suffire à tout, et opposer à l’ennemi cent sapeurs du génie pour un.\par
Mais il faut pour cela une activité incessante. Pas un seul homme ne doit rester inoccupé. Quand une besogne est finie, on en commence une autre, il y a toujours quelque chose à faire. En voici quelques-unes qui ont leur importance : Emmancher droites sur des hampes de sept pieds des lames de faux dont on a redressé au feu le crochet de la base et coupé le bourrelet formant dos, on fait tourner les hampes chez le tourneur le plus proche. Les lames de faux se trouvent en quantité chez les quincailliers.\par
Enlever les portes des appartements ou prendre des planches dans les magasins, les percer d’étroites meurtrières, longues de dix centimètres, les doubler d’épaisses feuilles des tôles percées de la même façon, et garnir de ces volets mobiles l’ouverture des fenêtres, le devant et les côtés des balcons pour diriger des feux de flanc dans la longueur des rues.\par
Amonceler des pavés à tous les étages, les plus petits au quatrième, au cinquième, aux mansardes, les plus gros au second et au troisième. En munir surtout les chambres situées au-dessus du retranchement.\par
Tout chef de barricade fera prendre chez les marchands les plus proches, les matériaux ou engins utiles à la défense, il mettra en réquisition les industriels, tels que tourneurs, menuisiers, serruriers, etc..., pour le confectionnement des objets que les soldats de la garnison ne seraient pas en mesure de fabriquer eux-mêmes. Il délivrera en échange des récépissés réguliers, valant factures.\par
Les commandants de barricades ne retiendront pas auprès d’eux les recrues qui viendraient les rejoindre. Ils les adresseront à leur supérieur immédiat, les lieutenants au capitaine, les capitaines au chef de bataillon, afin que ces hommes soient dirigés sur la réserve où s’opèrent l’organisation des nouveaux corps.\par
Cette règle est dictée par des motifs impérieux: 1° l’indemnité ne peut être allouée aux volontaires que sur constatation officielle de leur présence sous le drapeau, avec date précise; 2° le commandant en chef doit toujours connaître le chiffre exact des forces de chaque retranchement; 3° le bon ordre exige que l’effectif des compagnies et des bataillons demeure à peu près uniforme.\par
Les commandants de barricade adresseront des rapports fréquents à leurs supérieurs qui les feront tenir au quartier général.
\section[{Défense des barricades}]{Défense des barricades}\renewcommand{\leftmark}{Défense des barricades}

\noindent En supposant que l’armée tienne pied et s’acharne à la lutte, il est aisé de pressentir sa méthode d’attaque contre les positions républicaines.\par
D'abord, des détachements plus ou moins nombreux tirant aux fenêtres pendant leur marche, s’avanceront pour enlever une barricade. S'ils sont repoussés, et peut-être même sans avoir couru cette chance, ils perceront les maisons des îlots qui font face aux insurgés, et arriveront ainsi par l’intérieur sur le front de défense.\par
Les deux parties n’étant plus alors séparées que par la largeur de la rue, les soldats dirigeront un feu violent sur les fenêtres en face, pour chasser les défenseurs. Il faut s’attendre aussi que la troupe, en cas de résistance un peu longue, amènera du canon à travers l’îlot quelle occupe.\par
Elle le mettra en batterie sous une porte cochère, vis-à-vis une des maisons du front de défense, puis ouvrant soudainement la porte, canonnera les murs à bout portant, pour jeter bas l’édifice. Il ne tombera pas aux premiers coups, il faut un certain temps.\par
Dès que le canon sera démasqué, les Républicains tireront sur les artilleurs par les ouvertures du rez-de-chaussée, soupiraux, portes et balcons ayant vue sur l’allée de la porte cochère. On percera rapidement des meurtrières vis-à-vis, afin de multiplier les feux.\par
Règle générale : il est inutile de riposter aux soldats qui fusillent des fenêtres. C'est perdre sa poudre. L'ennemi en a de reste. Elle est rare chez les insurgés. Il est donc indispensable de la ménager. On se garantira des balles au moyen des volets doublés de tôle qui garnissent les fenêtres des balcons.\par
La garnison, dédaignant le feu des croisées, surveillera la rue pour empêcher l’ennemi de la traverser. Dès qu’il tentera le passage, il faut le fusiller à outrance, l’accabler de pierres et de pavés, du haut des maisons. En même temps, on se tiendra prêt à la fusiller, à l’arroser d’eau bouillante par le plancher du premier étage, s’il pénétrait dans le rez-de-chaussée, malgré le barricadement des portes et des fenêtres. Durant le combat, veiller avec soin à ce qu’il ne puisse attacher des pétards. Ne pas ménager les pavés, les bouteilles pleines d’eau, même les meubles, à défaut d’autres projectiles. Oter les volets en tôle des hauts étages, pour lancer les pierres, en évitant les balles d’en face.\par
Quant au retranchement, il ne sera pas facile d’en avoir raison. Le boulet ne pourrait atteindre le rempart que par le tir à ricochet, et le faible intervalle de six mètres, qui le sépare de la contre-garde, rendrait ce tir inefficace.\par
L'obus sera également impuissant. il viendra faire explosion en avant ou en arrière ou dans l’intervalle des deux ouvrages, et ses éclats écorcheront le plâtre des murailles, rien de plus. Car il ne trouvera là personne. La barricade sera défendue par les fenêtres.\par

\begin{figure}[h]
  \centering
  \includegraphics[width=\linewidth,]{blanqui1866_prise-armes/secteur.png}
\caption{Plan du secteur envisagé}
\end{figure}

\noindent L'assaut serait très meurtrier pour les assaillants. Il faudrait essuyer la fusillade jusqu’au pied du glacis, et à partir de ce point, braver un péril plus redoutable encore, il ne serait possible de descendre du mur interne, puis de franchir le rempart qu’avec des échelles de huit pieds, bagage incommode, et sous une grêle de pavés et de balles.\par
Si, en construisant la barricade, on a pu enfermer une ou deux portes cochères, dans l’intervalle de six mètres, entre le rempart et sa contregarde, des pelotons de faucheurs massés derrière les battants de la porte qui s’ouvrira tout à coup, se jetteront sur les soldats qui seraient descendus de la contre-garde et les mettront en pièces dans cette souricière car leurs baïonnettes ne seront pas de longueur contre leurs pertuisanes.\par
S'il n’existe point de porte cochère, les faucheurs se masseront au rez-de-chaussée afin de s’élancer par les portes d’allées ainsi que par les fenêtres basses. Au préalable, le commandant aura fait cesser la pluie de balles et de pavés, ce que la troupe pourra prendre pour un signe de défaite, méprise qui lui deviendrait fatale.\par
Si l’ennemi est rebuté par la longue résistance d’une ou de plusieurs barricades, il recourra peut-être à l’incendie des maisons par les obus. Eteindre le feu sera difficile. Si on n’y réussit pas, la retraite deviendra inévitable. Il faudra se replier de maison en maison sur une deuxième ligne de défense. Les troupes ne joueraient pas longtemps ce jeu-là. On ne fera pas de Paris une seconde Saragosse.\par
La lutte des barricades fournira au commandant en chef l’occasion de prendre à son tour l’offensive et de jeter des colonnes d’attaque sur les flancs et les derrières des assaillants.\par
Les blessés seront évacués sur les ambulances, désignées aux chefs de corps; les morts seront transportés aux hôpitaux.
\section[{Des mines}]{Des mines}\renewcommand{\leftmark}{Des mines}

\noindent Les troupes pourraient avoir recours à la mine pour forcer un front de défense trop tenace. C'est un moyen puissant, mais assez peu probable. L'ennemi n’en usera certainement pas au début. Ce moyen est long et dénote d’ailleurs une certaine timidité, qui ébranlerait l’esprit du soldat en lui montrant l’insurrection très redoutable.\par
Cependant il se peut que la nécessité fasse passer par-dessus cet inconvénient. Dans ce cas, le système d’égouts prend une grande importance. Dans toutes les rues où il s’en trouve, ils deviendraient le point de départ des galeries de mines.\par
L'ennemi a un plan détaillé des égouts de Paris. Ils sont de plusieurs dimensions. La carte des plus grands, dit égouts collecteurs, est connue de tout le monde. On la trouve dans le second volume de Paris-Guide. Mais ceux-là ne forment que le très petit nombre. La masse des canaux moyens et des rigoles demeure inconnue. Il serait utile de s’en enquérir auprès des ouvriers égoutiers.\par
Pendant le combat, il sera indispensable de faire reconnaître ces voies souterraines, par de nombreux détachements, auxquels on tracera un itinéraire. Ils seront munis d’échelles pour remonter à volonté par tous les regards.\par
On barricadera les embranchements qui aboutissent aux collecteurs eux-mêmes, d’après un plan réglé sur celui des opérations à ciel ouvert.\par
Toute rue servant de défense peut être traversée par une galerie de mine, il faudra donc s’assurer si elle recouvre un égout, et dans ce cas, occuper l’égout par des barricades, lorsque le front de défense sera attaqué avec vigueur par l’ennemi.\par
Des sentinelles le parcourront à pas de loup, posant l’oreille contre la paroi du côté des troupes, afin d’entendre le bruit de la sape et avertir aussitôt. Du reste, l’ennemi ne tenterait de pénétrer dans l’égout par la sape que s’il ne pouvait y arriver par la voie naturelle des embranchements, sa rencontre dans ses détours souterrains serait donc l’indice de ses projets de mine. Ces rencontres viendraient accroître les difficultés de l’opération et la rendre moins probable.\par
Dans les rues sans égouts, s’il en existe, la galerie serait creusée directement, à partir d’une cave, pour traverser la rue jusqu’à la maison vis-à-vis. Ce travail serait plus difficile à découvrir et à surprendre que celui des égouts. Des sentinelles devront coller l’oreille au mur de la cave bordant la rue, afin d’écouter le bruit des mineurs. La garnison, prévenue, les attendrait à l’issue pour leur faire un mauvais parti.\par
Somme toute, la guerre de mine est peu probable; celle d’égout l’est davantage.
\section[{Des habitants des maisons occupées}]{Des habitants des maisons occupées}\renewcommand{\leftmark}{Des habitants des maisons occupées}

\noindent Les habitants des maisons occupées par les républicains seront invités dans leur propre intérêt, à se retirer avec leur numéraire, leurs valeurs quelconques et leur argenterie, après avoir fermé tous les meubles. On leur rappellera, d’après l’exemple du 2 décembre, que les soldats de Bonaparte, en pénétrant dans toute maison d’où il est parti un coup de feu, égorgent sans distinction hommes et femmes, vieillards au lit et enfants à la mamelle.\par
Si les vieillards, les femmes et les enfants se retirent, les hommes devront les suivre. On ne les laissera pas demeurer seuls au logis.\par
Lorsqu’on aura percé les murs de toutes les maisons d’un îlot, on pourra faire retirer les familles qui habitent le front de défense, dans la partie de derrière de l’îlot.\par
Dans le cas où, par suite de communications interceptées, les vivres viendraient à leur manquer, les Républicains leur en donneront, en prévenant du fait les commissaires de bataillon pour qu’ils approvisionnent en conséquence.\par
Il faut encore le répéter: la condition \emph{sine qua non} de la victoire, c’est l’organisation, l’ensemble, l’ordre et la discipline. Il est douteux que les troupes résistent longtemps à une insurrection organisée et agissant avec tout l’appareil d’une force gouvernementale. L'hésitation les gagnera, puis le trouble, puis le découragement, enfin la débâcle.
 


% at least one empty page at end (for booklet couv)
\ifbooklet
  \pagestyle{empty}
  \clearpage
  % 2 empty pages maybe needed for 4e cover
  \ifnum\modulo{\value{page}}{4}=0 \hbox{}\newpage\hbox{}\newpage\fi
  \ifnum\modulo{\value{page}}{4}=1 \hbox{}\newpage\hbox{}\newpage\fi


  \hbox{}\newpage
  \ifodd\value{page}\hbox{}\newpage\fi
  {\centering\color{rubric}\bfseries\noindent\large
    Hurlus ? Qu’est-ce.\par
    \bigskip
  }
  \noindent Des bouquinistes électroniques, pour du texte libre à participation libre,
  téléchargeable gratuitement sur \href{https://hurlus.fr}{\dotuline{hurlus.fr}}.\par
  \bigskip
  \noindent Cette brochure a été produite par des éditeurs bénévoles.
  Elle n’est pas faîte pour être possédée, mais pour être lue, et puis donnée.
  Que circule le texte !
  En page de garde, on peut ajouter une date, un lieu, un nom ; pour suivre le voyage des idées.
  \par

  Ce texte a été choisi parce qu’une personne l’a aimé,
  ou haï, elle a en tous cas pensé qu’il partipait à la formation de notre présent ;
  sans le souci de plaire, vendre, ou militer pour une cause.
  \par

  L’édition électronique est soigneuse, tant sur la technique
  que sur l’établissement du texte ; mais sans aucune prétention scolaire, au contraire.
  Le but est de s’adresser à tous, sans distinction de science ou de diplôme.
  Au plus direct ! (possible)
  \par

  Cet exemplaire en papier a été tiré sur une imprimante personnelle
   ou une photocopieuse. Tout le monde peut le faire.
  Il suffit de
  télécharger un fichier sur \href{https://hurlus.fr}{\dotuline{hurlus.fr}},
  d’imprimer, et agrafer ; puis de lire et donner.\par

  \bigskip

  \noindent PS : Les hurlus furent aussi des rebelles protestants qui cassaient les statues dans les églises catholiques. En 1566 démarra la révolte des gueux dans le pays de Lille. L’insurrection enflamma la région jusqu’à Anvers où les gueux de mer bloquèrent les bateaux espagnols.
  Ce fut une rare guerre de libération dont naquit un pays toujours libre : les Pays-Bas.
  En plat pays francophone, par contre, restèrent des bandes de huguenots, les hurlus, progressivement réprimés par la très catholique Espagne.
  Cette mémoire d’une défaite est éteinte, rallumons-la. Sortons les livres du culte universitaire, cherchons les idoles de l’époque, pour les briser.
\fi

\ifdev % autotext in dev mode
\fontname\font — \textsc{Les règles du jeu}\par
(\hyperref[utopie]{\underline{Lien}})\par
\noindent \initialiv{A}{lors là}\blindtext\par
\noindent \initialiv{À}{ la bonheur des dames}\blindtext\par
\noindent \initialiv{É}{tonnez-le}\blindtext\par
\noindent \initialiv{Q}{ualitativement}\blindtext\par
\noindent \initialiv{V}{aloriser}\blindtext\par
\Blindtext
\phantomsection
\label{utopie}
\Blinddocument
\fi
\end{document}
